\title{Numerical Methods in Physics \\ Phonons}
\author{Jesper Vesterberg (jeve0010@student.umu.se)}
\date{\today}

\begin{titlepage}
  \maketitle
  \thispagestyle{fancy}
  \lhead{
    Department of Physics \\
    Umeå Universitet
  }
  \rhead{\today}
  \begin{abstract}
    In this report we look at two different theories for calculating the heat capacity for rare gas solids. In the high temperature limit the theory developed in thermal physics seem to hold for calculating the heat capacity, but it seem to break apart at lower temperature. Thus we have devolped software that calculates the heat capacity for rare gas solids using solid state theory for Neon, Xeon, Argon and Krypton. We have then gone on and compared this with the thermal physics expression and seen that these theories approaches eachother in the high temperature limit. Beyond that we also looked at phonons frequencies in the rare gas solids and its periodic behavior. Lastly we will take a look at how these phonon frequencies depends on the volume for the solids volume. 
  \end{abstract}
  \cfoot{
    Numerical Methods in Physics \\
    Supervisor: Peter Olsson
  }
\end{titlepage}
