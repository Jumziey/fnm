\documentclass[11pt]{article}
\usepackage{report}

\begin{document}
\section{Introduction}
Placeholder
\section{Theory}
\subsection{Potential in a Crystal}
The structure of crystals is typically described by using a lattice $\mathbf{l} = (l_{x1}, l_{x2}, l_{x3})$, which is a vector of integers that describes the position of every atom in a crystal structure in terms of a lattice constant $a$. If one then go on to describe an atoms displacement in this structure as $\mathbf{u}^l = (u^l_{x1},u^l_{x2},u^l_{x3})$, we can begin to describe the potential energy in the system. By using a simple Taylor expansion around the potential energy in the crystal and expanding it to the second term and noting that the first order term is vanishing we can get the equations of motion\footnote{By no means a straightforward thing to do} 
\begin{equation}
	m \ddot{u}^l_{\alpha} = - \sum_{\mathbf{l}'\beta} D^{\mathbf{l}\mathbf{l}'}_{\alpha \beta} u^{\mathbf{l}'}_{\beta}
	\label{eq:motion}
\end{equation}
where $\alpha$ and $\beta$ can have the integer value of one, two or three\footnote{Representing the spacial dimensions}, $m$ is the mass of the atoms in the crystal\footnote{we are only gonna look at monoatomic gases so $m$ should be the same for all the atoms}, $\mathbf{l}$ and $\mathbf{l}'$ are only different site positions in the lattice. Lastly we have term $D^{\mathbf{l}\mathbf{l}'}_{\alpha \beta}$ which actually is written
\begin{equation}
	D^{\mathbf{l}\mathbf{l}'}_{\alpha \beta} = \delta_{\mathbf{l} \mathbf{l}'} \sum_{\mathbf{l}''} \frac{\partial^2 \phi (\mathbf{l}-\mathbf{l}'')}{\partial x_{\alpha} \partial x_{\beta}}
	-
	\frac{\partial^2 \phi (\mathbf{l}-\mathbf{l}')}{\partial x_{\alpha} \partial x_{\beta}}
\end{equation}
where $\phi$ is the potential force felt by every individual site in the lattice, $\delta_{\mathbf{l}\mathbf{l}'}$ is the kronecker delta and $\mathbf{l}''$ is yet another seperate index for the different sites in the lattice.
\subsection{Model for a Rare Gas Crystal}
Looking at equation \ref{eq:motion} we can see that we have some harmonic oscillators to solve thus we can start to see that we will be able to form an eigenvalue problem to solve for the different frequencies that can propagete through a crystal. But in order to simplify things we will only look at a couple of rare gases: Ne, Ar, Kr and Xe. One beutiful thing with these gases is that the crystal structure is so arranged that all the lattice sites are a $\sqrt{2}a$ distance away from its closest neighbour. We can then also use the Mie-Lennard-Jones potential to simulate the individual potential forces at each site, this gives a potential of
\begin{equation}
	\phi(r) = 2 \epsilon \big[\frac{1}{2} \big(\frac{\sigma}{r}\big)^{12} - \big(\frac{\sigma}{r}\big)^6\big]
	\label{eq:pot}
\end{equation}
where $\sigma$ is the distance between two sites in the lattice where the potential is zero, $\epsilon$ is the depth of the lennard-jones potential well\cite{bib:wiki:mlj} and $r$ is the actually distance between two sites. Using this information we can set up the dynamic matrix and solve for the frequency. To start with we write out the whole system described by equation \ref{eq:motion} and \ref{eq:pot}. After that we do the ansatz that 
\begin{equation}
	u^l_{\alpha} = \frac{\epsilon_{\alpha}}{\sqrt{m}}e^{i(\mathbf{k}\cdot\mathbf{r}_l-\omega t)}
\end{equation}
and we end up with 
\begin{align}
	\omega^2\varepsilon_x = &\big( \frac{1}{2m} [A+B][8-4\cos{q_x\pi}\cos{q_y\pi}-4\cos{q_x\pi}\cos{q_z\pi}] \\
	&+ \frac{B}{m}[4-4\cos{q_y\pi}\cos{q_z\pi}]\big)\varepsilon_x \\
	&+\frac{1}{2m}[A-B][4\sin{q_x\pi}\sin{q_y\pi}]\varepsilon_y \\
	&+\frac{1}{2m}[A-B][4\sin{q_x\pi}\sin{q_z\pi}]\varepsilon_z
	\label{eq:omega}
\end{align}
If one were to look at the crystal structure for these gases we can see that they are symetric in x,y and z-direction. Thus with some simple permutations we get
\begin{align}
	\omega^2\varepsilon_y = &\big( \frac{1}{2m} [A+B][8-4\cos{q_y\pi}\cos{q_z\pi}-4\cos{q_y\pi}\cos{q_x\pi}] \\
	&+ \frac{B}{m}[4-4\cos{q_z\pi}\cos{q_x\pi}]\big)\varepsilon_y \\
	&+\frac{1}{2m}[A-B][4\sin{q_y\pi}\sin{q_z\pi}]\varepsilon_z \\
	&+\frac{1}{2m}[A-B][4\sin{q_y\pi}\sin{q_x\pi}]\varepsilon_x \\
	\omega^2\varepsilon_z = &\big( \frac{1}{2m} [A+B][8-4\cos{q_z\pi}\cos{q_x\pi}-4\cos{q_z\pi}\cos{q_y\pi}] \\
	&+ \frac{B}{m}[4-4\cos{q_x\pi}\cos{q_y\pi}]\big)\varepsilon_z \\
	&+\frac{1}{2m}[A-B][4\sin{q_z\pi}\sin{q_x\pi}]\varepsilon_x \\
	&+\frac{1}{2m}[A-B][4\sin{q_z\pi}\sin{q_y\pi}]\varepsilon_y \\
	\label{eq:omegaRest}
\end{align}
where
\begin{align}
	A &\equiv \frac{1}{r_{nn}} \phi'(r_{nn}) \\
	B &\equiv \phi''(r_{nn}).
	\label{eq:AB}
\end{align}
With this we have eigenvalue problems in every direction to solve so that we can get the frequency of the phonons that can move through the crystals. 
\subsection{Volume Dependency of Phonon frequencies}
We want to go on and look how the dependency of the volume of the lattice changes the frequencies of the phonons that can propegate through the crystal. By doing a simple approximation we say that $\sigma$ and $\epsilon$ in equation \ref{eq:pot} is not affected by this change in volume. This will make the later implementation simpler. We can then calculate the volume dependence as 
\begin{equation}
	\gamma_j(\mathbf{q}) = - \frac{\partial \ln{ \omega(\mathbf{q},j)}}{\partial \ln{V}}
\end{equation}
where $V$ is the volume of the crystal structure in the lattice. Why we have some $\ln$ terms is basically due to the fact that everyone else are doing it. If one expands the derivate abit we end up with the  expression
\begin{equation}
	\gamma_j(\mathbf{q}) =  -\frac{V}{\omega(\mathbf{q},j)} \frac{\partial \omega(\mathbf{q},j)}{\partial V}
\end{equation}
Which then can be solved using a finite difference approximation scheme for the derivate for example
\begin{equation}
	\frac{df(x)}{dx} \approx \frac{f(x+h)-f(x-h)}{2h}
\end{equation}
\subsection{Heat Capacity}
One can derive an expression for the heat capacity as follows\cite{bib:solid}
\begin{equation}
	C_V = k_B \sum_{\mathbf{q}, j} \Big[\frac{\hbar \omega(\mathbf{q},j)}{k_B T}\Big]^2
	\frac{\exp{\Big[\frac{\hbar \omega(\mathbf{q},j)}{k_B T}\Big]}}{\Big(\exp{\Big[\frac{\hbar \omega(\mathbf{q},j)}{k_B T}\Big]}-1\Big)^2}
	\label{eq:CV}
\end{equation}
where $k_B$ is the Boltzmann constant and $j$ just represent each direction possible. 

Where not quite happy yet though, the most relevant information would be heat capacity per unit volume. To do this we look divide with the volume we have calculated the heat capacity for. This end up being for one brillouin zone. Here we take great care to not overdoing the summing of values at every site, thus we divide with the appropriate "sphere volume'' and avoid summing the sphere volume at every site that is outside the brilliouin zone. Thus we get some weight at every site described by $\mathbf{q}$\footnote{$\mathbf{q}$ is a simple rewrite of the familiar $\mathbf{k}$ with some constants}. The final expression becomes
\begin{equation}
	\frac{C_V}{V} = \frac{1}{(2\pi)^3} \frac{4}{1000} \big(\frac{\pi}{a})^3 \sum_j \sum_{\mathbf{q}} W(\mathbf{q})f_j(\mathbf{q}) 
\end{equation}
where $f_j$ is the expression within the sum sign in equation \ref{eq:CV} multiplied with boltzmann constant, $W(\mathbf{q})$ is the weight for a certain $mathbf{q}$, to compensate for the volume added that is outside the brillouin zone. 
\section{Code}
In to obtain usuable values for the volume dependency of phonon frequencies in one of the rare gases and the heat capacity using solid state physics theory, we have developed some code that calculate this. We will go through both the algorithm design and the code in more detail. 
\subsection{Algorithm design}
The code is written in $C$ and has been divided into two files: \verb+phonons.c+ and \verb+frequencies.c+. In \verb+frequencies.c+ we calculate, via a declared frequency function, the frequencies possible in different gases at different sites given certain properties. This is done using the eigenvalue problem described in equation \ref{eq:omega} and \ref{eq:omegaRest}. The declaration of the frequencies function  is as follows
\begin{lstlisting}
 void frequencies(double A, double B, double m, double *q, 
				double *omega, double *eps);
\end{lstlisting}
where A and B are the constants described in equation \ref{eq:AB}, m is the mass of the atoms at the sites in the lattice, q is the modified k vector values, omega and eps are the return values where omega is the frequencies in the tree different directions and eps is the vector correponding to each frequency describing its direction. This is a sperate function since it is essential in not only calculating the frequency but the volume dependency of the frequency aswell as the heat capacity of the rare gas. 

In \verb+phonons.c+ the rest of the program resides. Its main function is to use the frequency function to calculate volume dependency of the phonon frequencies aswell as heat capacity in the gases. Outside of that the code is responsible for user input/output. Making sure that the input is correctly interepred and that the output follows a proper output formatting that was described in the program specification.
\subsection{Detailed Code Description}
\subsection{System description}

\section{Results}

\section{Conclusions}
\begin{thebibliography}{99}
\bibitem{bib:wiki:mlj} [2015-03-15] \url{http://en.wikipedia.org/wiki/Lennard-Jones_potential}
\bibitem{bib:solid} Neil W. Ashcroft , N. David Mermin, \textit{Solid State Physics} (Cengage Learning 1976)
\bibitem{bib:ph} Carl Nordling, Jonny Östermann, \textit{Physics handbook}
\end{thebibliography}

\end{document}
