\documentclass[11pt]{article}
\usepackage{report}
\newif\iftikz
\tikztrue

\begin{document}
\section{Exercise 1}
By substituting $\Phi$ with $\varphi/r$ in equation 15 from the lab instruction, substituting in the expression for $\rho(r)$ and simplifying we get an expression
\begin{equation}
\frac{\partial^2 \varphi(r)}{\partial r^2} = -\frac{1}{2} e^{-r}r
\end{equation}
Putting this into the Numerov scheme we get
\begin{equation}
	\varphi_{j+1} = 2 \varphi_j - \varphi_{j-1} + \frac{h^2}{12} (S_{j+1} + 10S_j +S_{j-1})
\end{equation}
where $S_j$ is 
\begin{equation}
	S_j = \frac{1}{2}e^{-r}r
\end{equation}
Which is slightly confusing since $S$ from the Numerov's method should be $-\frac{1}{2}e^{-r}r$. But due to it being simpler to write we defined $S$ in this slightly obfuscated way. 


\section{Exercise 2}
The following three tables shows the values requested in the lab instructions.
\begin{table}[h]
\center
\caption{Table with values using $\varphi(1)$ as the analytical value}
\begin{tabular}{c|c|c}
r (cm) & $\varphi(r)$ (statvolt/cm) & error \\ \hline
2 & 0.72933 & -7.529e-07 \\ \hline 
4 & 0.945055 & -1.91455e-06 \\ \hline 
6 & 0.990088 & -3.08937e-06 \\ \hline 
8 & 0.998327 & -4.26028e-06 \\ \hline 
10 & 0.999733 & -5.42988e-06 \\ \hline 
12 & 0.999964 & -6.5992e-06 \\ \hline 
14 & 1 & -7.76847e-06 \\ \hline 
16 & 1.00001 & -8.93773e-06 \\ \hline 
18 & 1.00001 & -1.0107e-05 \\ \hline 
20 & 1.00001 & -1.12762e-05 \\ 
\end{tabular}
\end{table}

\begin{table}[h]
\center
\caption{Table with values using $\varphi(1)$ as the numerical value of $\Phi(0)$ multiplied with the step length}
\begin{tabular}{c|c|c}
r (cm) & $\varphi(r)$ (statvolt/cm) & error \\ \hline
2 & 0.730916 & -0.00158653 \\ \hline 
4 & 0.948227 & -0.00317347 \\ \hline 
6 & 0.994845 & -0.00476043 \\ \hline 
8 & 1.00467 & -0.00634738 \\ \hline 
10 & 1.00766 & -0.00793432 \\ \hline 
12 & 1.00948 & -0.00952127 \\ \hline 
14 & 1.0111 & -0.0111082 \\ \hline 
16 & 1.01269 & -0.0126952 \\ \hline 
18 & 1.01428 & -0.0142821 \\ \hline 
20 & 1.01587 & -0.0158691 \\ 
\end{tabular}
\end{table}

\begin{table}[h]
\center
\caption{Table with values using $\varphi(1)$ as 95\% of the analytical value}
\begin{tabular}{c|c|c}
r (cm) & $\varphi(r)$ (statvolt/cm) & error \\ \hline
2 & 0.679409 & 0.04992 \\ \hline 
4 & 0.845214 & 0.0998395 \\ \hline 
6 & 0.840326 & 0.149759 \\ \hline 
8 & 0.798644 & 0.199679 \\ \hline 
10 & 0.750129 & 0.249598 \\ \hline 
12 & 0.700439 & 0.299518 \\ \hline 
14 & 0.650556 & 0.349437 \\ \hline 
16 & 0.600642 & 0.399357 \\ \hline 
18 & 0.550724 & 0.449276 \\ \hline 
20 & 0.500804 & 0.499196 \\
\end{tabular}
\end{table}

\newpage
\section{Exercise 3}
In figure \ref{fig:phi3} and \ref{fig:phi3reg} we can see the difference between only using the  $\varphi(h)$ value that is off by 5\% and using the value compensated according how the lab instruction specified. The error has also been plotted for illustration. Lastly we have table with values for the compensated plots as done in the previous exercise.
\iftikz
\begin{figure}[H]
	\centering
	\newlength\figureheight 
	\newlength\figurewidth
	\setlength\figureheight{7cm} 
	\setlength\figurewidth{14cm}
	% This file was created by matlab2tikz.
% Minimal pgfplots version: 1.3
%
%The latest updates can be retrieved from
%  http://www.mathworks.com/matlabcentral/fileexchange/22022-matlab2tikz
%where you can also make suggestions and rate matlab2tikz.
%
\definecolor{mycolor1}{rgb}{0.00000,0.44700,0.74100}%
\definecolor{mycolor2}{rgb}{0.85000,0.32500,0.09800}%
%
\begin{tikzpicture}

\begin{axis}[%
width=0.95092\figurewidth,
height=\figureheight,
at={(0\figurewidth,0\figureheight)},
scale only axis,
xmin=0,
xmax=20,
xlabel={r (cm)},
ymin=0,
ymax=0.9,
ylabel={$\varphi(r)$ (statvolt/cm)},
title style={font=\bfseries},
title={$\varphi(r)$ using 95\% of analytical for $\varphi_1$},
legend style={legend cell align=left,align=left,draw=white!15!black},
title style={font=\small},ticklabel style={font=\tiny}
]
\addplot [color=mycolor1,solid]
  table[row sep=crcr]{%
0	0\\
0.1	0.0474247\\
0.2	0.0944041\\
0.3	0.140571\\
0.4	0.185632\\
0.5	0.229357\\
0.6	0.271569\\
0.7	0.312138\\
0.8	0.350971\\
0.9	0.38801\\
1	0.423221\\
1.1	0.456594\\
1.2	0.488137\\
1.3	0.517874\\
1.4	0.545841\\
1.5	0.572082\\
1.6	0.59665\\
1.7	0.619603\\
1.8	0.641004\\
1.9	0.660917\\
2	0.679409\\
2.1	0.696548\\
2.2	0.712401\\
2.3	0.727036\\
2.4	0.740517\\
2.5	0.752909\\
2.6	0.764275\\
2.7	0.774675\\
2.8	0.784168\\
2.9	0.792809\\
3	0.800653\\
3.1	0.807749\\
3.2	0.814147\\
3.3	0.819892\\
3.4	0.825029\\
3.5	0.829598\\
3.6	0.833638\\
3.7	0.837186\\
3.8	0.840277\\
3.9	0.842943\\
4	0.845214\\
4.1	0.847118\\
4.2	0.848682\\
4.3	0.849932\\
4.4	0.850889\\
4.5	0.851576\\
4.6	0.852014\\
4.7	0.852219\\
4.8	0.852212\\
4.9	0.852006\\
5	0.851618\\
5.1	0.851061\\
5.2	0.850349\\
5.3	0.849493\\
5.4	0.848505\\
5.5	0.847395\\
5.6	0.846173\\
5.7	0.844847\\
5.8	0.843425\\
5.9	0.841916\\
6	0.840326\\
6.1	0.838661\\
6.2	0.836928\\
6.3	0.835132\\
6.4	0.833279\\
6.5	0.831371\\
6.6	0.829416\\
6.7	0.827415\\
6.8	0.825373\\
6.9	0.823293\\
7	0.821178\\
7.1	0.819031\\
7.2	0.816855\\
7.3	0.814652\\
7.4	0.812424\\
7.5	0.810174\\
7.6	0.807903\\
7.7	0.805613\\
7.8	0.803306\\
7.9	0.800982\\
8	0.798644\\
8.1	0.796293\\
8.2	0.793929\\
8.3	0.791554\\
8.4	0.789168\\
8.5	0.786773\\
8.6	0.78437\\
8.7	0.781958\\
8.8	0.77954\\
8.9	0.777114\\
9	0.774683\\
9.1	0.772246\\
9.2	0.769804\\
9.3	0.767357\\
9.4	0.764906\\
9.5	0.762451\\
9.6	0.759993\\
9.7	0.757531\\
9.8	0.755067\\
9.9	0.752599\\
10	0.750129\\
10.1	0.747657\\
10.2	0.745183\\
10.3	0.742707\\
10.4	0.740229\\
10.5	0.73775\\
10.6	0.735269\\
10.7	0.732787\\
10.8	0.730304\\
10.9	0.727819\\
11	0.725334\\
11.1	0.722847\\
11.2	0.72036\\
11.3	0.717872\\
11.4	0.715383\\
11.5	0.712894\\
11.6	0.710404\\
11.7	0.707913\\
11.8	0.705423\\
11.9	0.702931\\
12	0.700439\\
12.1	0.697947\\
12.2	0.695455\\
12.3	0.692962\\
12.4	0.690469\\
12.5	0.687975\\
12.6	0.685482\\
12.7	0.682988\\
12.8	0.680494\\
12.9	0.678\\
13	0.675506\\
13.1	0.673011\\
13.2	0.670517\\
13.3	0.668022\\
13.4	0.665527\\
13.5	0.663032\\
13.6	0.660537\\
13.7	0.658042\\
13.8	0.655547\\
13.9	0.653051\\
14	0.650556\\
14.1	0.648061\\
14.2	0.645565\\
14.3	0.64307\\
14.4	0.640574\\
14.5	0.638079\\
14.6	0.635583\\
14.7	0.633088\\
14.8	0.630592\\
14.9	0.628096\\
15	0.6256\\
15.1	0.623105\\
15.2	0.620609\\
15.3	0.618113\\
15.4	0.615617\\
15.5	0.613122\\
15.6	0.610626\\
15.7	0.60813\\
15.8	0.605634\\
15.9	0.603138\\
16	0.600642\\
16.1	0.598146\\
16.2	0.59565\\
16.3	0.593155\\
16.4	0.590659\\
16.5	0.588163\\
16.6	0.585667\\
16.7	0.583171\\
16.8	0.580675\\
16.9	0.578179\\
17	0.575683\\
17.1	0.573187\\
17.2	0.570691\\
17.3	0.568195\\
17.4	0.565699\\
17.5	0.563203\\
17.6	0.560707\\
17.7	0.558211\\
17.8	0.555715\\
17.9	0.55322\\
18	0.550724\\
18.1	0.548228\\
18.2	0.545732\\
18.3	0.543236\\
18.4	0.54074\\
18.5	0.538244\\
18.6	0.535748\\
18.7	0.533252\\
18.8	0.530756\\
18.9	0.52826\\
19	0.525764\\
19.1	0.523268\\
19.2	0.520772\\
19.3	0.518276\\
19.4	0.51578\\
19.5	0.513284\\
19.6	0.510788\\
19.7	0.508292\\
19.8	0.505796\\
19.9	0.5033\\
20	0.500804\\
};
\addlegendentry{$\varphi(r)$};

\addplot [color=mycolor2,solid]
  table[row sep=crcr]{%
0	0\\
0.1	0.00249604\\
0.2	0.00499206\\
0.3	0.00748809\\
0.4	0.0099841\\
0.5	0.0124801\\
0.6	0.0149761\\
0.7	0.0174721\\
0.8	0.0199681\\
0.9	0.0224641\\
1	0.0249601\\
1.1	0.0274561\\
1.2	0.0299521\\
1.3	0.0324481\\
1.4	0.0349441\\
1.5	0.03744\\
1.6	0.039936\\
1.7	0.042432\\
1.8	0.044928\\
1.9	0.047424\\
2	0.04992\\
2.1	0.0524159\\
2.2	0.0549119\\
2.3	0.0574079\\
2.4	0.0599039\\
2.5	0.0623999\\
2.6	0.0648958\\
2.7	0.0673918\\
2.8	0.0698878\\
2.9	0.0723838\\
3	0.0748797\\
3.1	0.0773757\\
3.2	0.0798717\\
3.3	0.0823677\\
3.4	0.0848636\\
3.5	0.0873596\\
3.6	0.0898556\\
3.7	0.0923516\\
3.8	0.0948476\\
3.9	0.0973435\\
4	0.0998395\\
4.1	0.102335\\
4.2	0.104831\\
4.3	0.107327\\
4.4	0.109823\\
4.5	0.112319\\
4.6	0.114815\\
4.7	0.117311\\
4.8	0.119807\\
4.9	0.122303\\
5	0.124799\\
5.1	0.127295\\
5.2	0.129791\\
5.3	0.132287\\
5.4	0.134783\\
5.5	0.137279\\
5.6	0.139775\\
5.7	0.142271\\
5.8	0.144767\\
5.9	0.147263\\
6	0.149759\\
6.1	0.152255\\
6.2	0.154751\\
6.3	0.157247\\
6.4	0.159743\\
6.5	0.162239\\
6.6	0.164735\\
6.7	0.167231\\
6.8	0.169727\\
6.9	0.172223\\
7	0.174719\\
7.1	0.177215\\
7.2	0.179711\\
7.3	0.182207\\
7.4	0.184703\\
7.5	0.187199\\
7.6	0.189695\\
7.7	0.192191\\
7.8	0.194687\\
7.9	0.197183\\
8	0.199679\\
8.1	0.202175\\
8.2	0.204671\\
8.3	0.207167\\
8.4	0.209662\\
8.5	0.212158\\
8.6	0.214654\\
8.7	0.21715\\
8.8	0.219646\\
8.9	0.222142\\
9	0.224638\\
9.1	0.227134\\
9.2	0.22963\\
9.3	0.232126\\
9.4	0.234622\\
9.5	0.237118\\
9.6	0.239614\\
9.7	0.24211\\
9.8	0.244606\\
9.9	0.247102\\
10	0.249598\\
10.1	0.252094\\
10.2	0.25459\\
10.3	0.257086\\
10.4	0.259582\\
10.5	0.262078\\
10.6	0.264574\\
10.7	0.26707\\
10.8	0.269566\\
10.9	0.272062\\
11	0.274558\\
11.1	0.277054\\
11.2	0.27955\\
11.3	0.282046\\
11.4	0.284542\\
11.5	0.287038\\
11.6	0.289534\\
11.7	0.29203\\
11.8	0.294526\\
11.9	0.297022\\
12	0.299518\\
12.1	0.302014\\
12.2	0.30451\\
12.3	0.307006\\
12.4	0.309502\\
12.5	0.311998\\
12.6	0.314494\\
12.7	0.31699\\
12.8	0.319485\\
12.9	0.321981\\
13	0.324477\\
13.1	0.326973\\
13.2	0.329469\\
13.3	0.331965\\
13.4	0.334461\\
13.5	0.336957\\
13.6	0.339453\\
13.7	0.341949\\
13.8	0.344445\\
13.9	0.346941\\
14	0.349437\\
14.1	0.351933\\
14.2	0.354429\\
14.3	0.356925\\
14.4	0.359421\\
14.5	0.361917\\
14.6	0.364413\\
14.7	0.366909\\
14.8	0.369405\\
14.9	0.371901\\
15	0.374397\\
15.1	0.376893\\
15.2	0.379389\\
15.3	0.381885\\
15.4	0.384381\\
15.5	0.386877\\
15.6	0.389373\\
15.7	0.391869\\
15.8	0.394365\\
15.9	0.396861\\
16	0.399357\\
16.1	0.401853\\
16.2	0.404349\\
16.3	0.406845\\
16.4	0.409341\\
16.5	0.411837\\
16.6	0.414333\\
16.7	0.416829\\
16.8	0.419325\\
16.9	0.421821\\
17	0.424317\\
17.1	0.426812\\
17.2	0.429308\\
17.3	0.431804\\
17.4	0.4343\\
17.5	0.436796\\
17.6	0.439292\\
17.7	0.441788\\
17.8	0.444284\\
17.9	0.44678\\
18	0.449276\\
18.1	0.451772\\
18.2	0.454268\\
18.3	0.456764\\
18.4	0.45926\\
18.5	0.461756\\
18.6	0.464252\\
18.7	0.466748\\
18.8	0.469244\\
18.9	0.47174\\
19	0.474236\\
19.1	0.476732\\
19.2	0.479228\\
19.3	0.481724\\
19.4	0.48422\\
19.5	0.486716\\
19.6	0.489212\\
19.7	0.491708\\
19.8	0.494204\\
19.9	0.4967\\
20	0.499196\\
};
\addlegendentry{error};

\end{axis}
\end{tikzpicture}%
	\caption{Showing the value for $\varphi$ when $\varphi(h)$ was off by 5\%.}
	\label{fig:phi3}
\end{figure}
\fi
\iftikz
\begin{figure}[H]
	\centering
	\setlength\figureheight{7cm} 
	\setlength\figurewidth{14cm}
	% This file was created by matlab2tikz.
% Minimal pgfplots version: 1.3
%
%The latest updates can be retrieved from
%  http://www.mathworks.com/matlabcentral/fileexchange/22022-matlab2tikz
%where you can also make suggestions and rate matlab2tikz.
%
\definecolor{mycolor1}{rgb}{0.00000,0.44700,0.74100}%
\definecolor{mycolor2}{rgb}{0.85000,0.32500,0.09800}%
%
\begin{tikzpicture}

\begin{axis}[%
width=0.95092\figurewidth,
height=\figureheight,
at={(0\figurewidth,0\figureheight)},
scale only axis,
xmin=0,
xmax=20,
xlabel={r (cm)},
ymin=-0.2,
ymax=1.2,
ylabel={$\varphi$ (statvolt/cm)},
title style={font=\bfseries},
title={Fixed $\varphi(r)$ with linear regression},
legend style={legend cell align=left,align=left,draw=white!15!black},
title style={font=\small},ticklabel style={font=\tiny}
]
\addplot [color=mycolor1,solid]
  table[row sep=crcr]{%
0	0\\
0.1	0.0499207\\
0.2	0.0993961\\
0.3	0.148059\\
0.4	0.195616\\
0.5	0.241837\\
0.6	0.286545\\
0.7	0.32961\\
0.8	0.370939\\
0.9	0.410474\\
1	0.448181\\
1.1	0.48405\\
1.2	0.518089\\
1.3	0.550322\\
1.4	0.580785\\
1.5	0.609522\\
1.6	0.636586\\
1.7	0.662035\\
1.8	0.685932\\
1.9	0.708341\\
2	0.729329\\
2.1	0.748964\\
2.2	0.767313\\
2.3	0.784444\\
2.4	0.800421\\
2.5	0.815309\\
2.6	0.829171\\
2.7	0.842067\\
2.8	0.854056\\
2.9	0.865193\\
3	0.875533\\
3.1	0.885125\\
3.2	0.894019\\
3.3	0.90226\\
3.4	0.909893000000001\\
3.5	0.916958\\
3.6	0.923494\\
3.7	0.929538000000001\\
3.8	0.935125000000001\\
3.9	0.940287000000001\\
4	0.945054000000001\\
4.1	0.949454000000001\\
4.2	0.953514000000001\\
4.3	0.957260000000001\\
4.4	0.960713000000001\\
4.5	0.963896000000001\\
4.6	0.966830000000001\\
4.7	0.969531000000001\\
4.8	0.972020000000001\\
4.9	0.974310000000001\\
5	0.976418000000001\\
5.1	0.978357000000001\\
5.2	0.980141000000001\\
5.3	0.981781000000001\\
5.4	0.983289000000001\\
5.5	0.984675000000001\\
5.6	0.985949000000001\\
5.7	0.987119000000001\\
5.8	0.988193000000001\\
5.9	0.989180000000001\\
6	0.990086000000001\\
6.1	0.990917000000001\\
6.2	0.991680000000001\\
6.3	0.992380000000001\\
6.4	0.993023000000001\\
6.5	0.993611000000001\\
6.6	0.994152000000001\\
6.7	0.994647000000001\\
6.8	0.995101000000001\\
6.9	0.995517000000001\\
7	0.995898000000001\\
7.1	0.996247000000001\\
7.2	0.996567000000001\\
7.3	0.996860000000001\\
7.4	0.997128000000001\\
7.5	0.997374000000001\\
7.6	0.997599000000001\\
7.7	0.997805000000001\\
7.8	0.997994000000001\\
7.9	0.998166000000001\\
8	0.998324000000001\\
8.1	0.998469000000001\\
8.2	0.998601000000001\\
8.3	0.998722000000001\\
8.4	0.998832000000001\\
8.5	0.998933000000001\\
8.6	0.999026000000001\\
8.7	0.999110000000001\\
8.8	0.999188000000001\\
8.9	0.999258000000001\\
9	0.999323000000001\\
9.1	0.999382000000001\\
9.2	0.999436000000001\\
9.3	0.999485000000001\\
9.4	0.999530000000001\\
9.5	0.999571000000001\\
9.6	0.999609000000001\\
9.7	0.999643000000001\\
9.8	0.999675000000002\\
9.9	0.999703000000002\\
10	0.999729000000002\\
10.1	0.999753000000001\\
10.2	0.999775000000002\\
10.3	0.999795000000002\\
10.4	0.999813000000002\\
10.5	0.999830000000002\\
10.6	0.999845000000002\\
10.7	0.999859000000002\\
10.8	0.999872000000002\\
10.9	0.999883000000002\\
11	0.999894000000002\\
11.1	0.999903000000002\\
11.2	0.999912000000002\\
11.3	0.999920000000002\\
11.4	0.999927000000002\\
11.5	0.999934000000002\\
11.6	0.999940000000002\\
11.7	0.999945000000002\\
11.8	0.999951000000002\\
11.9	0.999955000000002\\
12	0.999959000000002\\
12.1	0.999963000000002\\
12.2	0.999967000000002\\
12.3	0.999970000000002\\
12.4	0.999973000000002\\
12.5	0.999975000000002\\
12.6	0.999978000000002\\
12.7	0.999980000000002\\
12.8	0.999982000000002\\
12.9	0.999984000000002\\
13	0.999986000000002\\
13.1	0.999987000000002\\
13.2	0.999989000000002\\
13.3	0.999990000000002\\
13.4	0.999991000000002\\
13.5	0.999992000000002\\
13.6	0.999993000000002\\
13.7	0.999994000000002\\
13.8	0.999995000000002\\
13.9	0.999995000000002\\
14	0.999996000000002\\
14.1	0.999997000000002\\
14.2	0.999997000000002\\
14.3	0.999998000000002\\
14.4	0.999998000000002\\
14.5	0.999999000000002\\
14.6	0.999999000000002\\
14.7	1\\
14.8	1\\
14.9	1\\
15	1\\
15.1	1.000001\\
15.2	1.000001\\
15.3	1.000001\\
15.4	1.000001\\
15.5	1.000002\\
15.6	1.000002\\
15.7	1.000002\\
15.8	1.000002\\
15.9	1.000002\\
16	1.000002\\
16.1	1.000002\\
16.2	1.000002\\
16.3	1.000003\\
16.4	1.000003\\
16.5	1.000003\\
16.6	1.000003\\
16.7	1.000003\\
16.8	1.000003\\
16.9	1.000003\\
17	1.000003\\
17.1	1.000003\\
17.2	1.000003\\
17.3	1.000003\\
17.4	1.000003\\
17.5	1.000003\\
17.6	1.000003\\
17.7	1.000003\\
17.8	1.000003\\
17.9	1.000004\\
18	1.000004\\
18.1	1.000004\\
18.2	1.000004\\
18.3	1.000004\\
18.4	1.000004\\
18.5	1.000004\\
18.6	1.000004\\
18.7	1.000004\\
18.8	1.000004\\
18.9	1.000004\\
19	1.000004\\
19.1	1.000004\\
19.2	1.000004\\
19.3	1.000004\\
19.4	1.000004\\
19.5	1.000004\\
19.6	1.000004\\
19.7	1.000004\\
19.8	1.000004\\
19.9	1.000004\\
20	1.000004\\
};
\addlegendentry{$\varphi(r)$};

\addplot [color=mycolor2,solid]
  table[row sep=crcr]{%
0	0\\
0.1	1.10622424831508e-08\\
0.2	7.16142198459169e-08\\
0.3	4.62160244518728e-08\\
0.4	-5.52427671662858e-08\\
0.5	-3.24640791804498e-07\\
0.6	-1.26922234555682e-07\\
0.7	-1.60118403036424e-07\\
0.8	4.5023588973736e-07\\
0.9	-6.62386884231481e-09\\
1	-1.61757163663623e-07\\
1.1	-1.79732023430912e-07\\
1.2	2.60940476448646e-07\\
1.3	5.41493879180344e-07\\
1.4	1.61299268719439e-07\\
1.5	2.19740247664468e-07\\
1.6	2.67609620130393e-07\\
1.7	4.80502440614039e-07\\
1.8	1.12378985339667e-07\\
1.9	1.92515861363596e-07\\
2	4.33526774257231e-07\\
2.1	3.22081386783246e-07\\
2.2	3.67439098591937e-07\\
2.3	-5.14004028451254e-07\\
2.4	-4.97236707919235e-07\\
2.5	-2.46903772649354e-07\\
2.6	-2.29892968328649e-07\\
2.7	4.5061587572448e-08\\
2.8	-1.50300523471536e-07\\
2.9	1.10861801849715e-07\\
3	-6.70919660161751e-07\\
3.1	-4.66103572849264e-07\\
3.2	-7.30343752586293e-07\\
3.3	-3.93613286409078e-07\\
3.4	-8.2889288088861e-07\\
3.5	-8.04411376398306e-07\\
3.6	-4.228524196348e-07\\
3.7	-5.04404678025594e-08\\
3.8	-2.38382880879406e-07\\
3.9	-6.38765123461837e-07\\
4	-9.16666203165661e-07\\
4.1	-6.59975372463428e-07\\
4.2	-2.88143481519398e-07\\
4.3	-9.60888433598761e-07\\
4.4	-4.87689819750337e-07\\
4.5	-2.3874928811729e-07\\
4.6	-1.05795729155034e-06\\
4.7	-1.78290681662752e-07\\
4.8	-1.1399666687284e-06\\
4.9	-7.11594689728479e-07\\
5	-8.14496799783626e-07\\
5.1	-4.50307581223441e-07\\
5.2	-6.31914739646078e-07\\
5.3	-3.17760223245145e-07\\
5.4	-3.49487667583581e-07\\
5.5	-3.92894241141306e-07\\
5.6	-8.82122635936788e-07\\
5.7	-9.67011265284867e-07\\
5.8	-4.63506966585925e-07\\
5.9	-8.07034135852369e-07\\
6	-1.00870666630559e-06\\
6.1	-6.14263918397029e-07\\
6.2	-6.65608813399388e-07\\
6.3	-6.64824670870701e-07\\
6.4	-1.54054733148001e-06\\
6.5	-6.16570155598239e-07\\
6.6	-1.58256145699198e-06\\
6.7	-1.46677663059869e-06\\
6.8	-1.6106505181801e-06\\
6.9	-1.64515926692754e-06\\
7	-1.46884499641065e-06\\
7.1	-1.22740086083706e-06\\
7.2	-1.29471853382856e-06\\
7.3	-1.25530465255874e-06\\
7.4	-8.87977310126686e-07\\
7.5	-1.15075820328325e-06\\
7.6	-1.16688051610137e-06\\
7.7	-1.21183700207972e-06\\
7.8	-1.70139700206295e-06\\
7.9	-1.18052527375845e-06\\
8	-1.31313951379219e-06\\
8.1	-1.87264729945014e-06\\
8.2	-1.73320685914469e-06\\
8.3	-1.86165960713858e-06\\
8.4	-1.31008573123204e-06\\
8.5	-1.20893730704008e-06\\
8.6	-1.76070643942428e-06\\
8.7	-1.23408878516251e-06\\
8.8	-1.95860551688298e-06\\
8.9	-1.31964932825746e-06\\
9	-1.75392247803163e-06\\
9.1	-1.74523712148655e-06\\
9.2	-1.82065028919709e-06\\
9.3	-1.54690785292289e-06\\
9.4	-1.52717367418731e-06\\
9.5	-1.39802185572524e-06\\
9.6	-1.82667164838612e-06\\
9.7	-1.50844606272571e-06\\
9.8	-2.16443665135291e-06\\
9.9	-1.53935823576923e-06\\
10	-1.39957857647754e-06\\
10.1	-1.53130911451083e-06\\
10.2	-1.73894397470509e-06\\
10.3	-1.84353539367521e-06\\
10.4	-1.68139465372974e-06\\
10.5	-2.10280843748745e-06\\
10.6	-1.97086131004109e-06\\
10.7	-2.16035575051343e-06\\
10.8	-2.55682183314399e-06\\
10.9	-2.05560927446502e-06\\
11	-2.56105513818383e-06\\
11.1	-1.98572102172445e-06\\
11.2	-2.24969403517239e-06\\
11.3	-2.27994634260931e-06\\
11.4	-2.00974844699608e-06\\
11.5	-2.37813179249091e-06\\
11.6	-2.32939660826936e-06\\
11.7	-1.81266125298674e-06\\
11.8	-2.78144961585181e-06\\
11.9	-2.19331341300144e-06\\
12	-2.00948647510746e-06\\
12.1	-2.19456835537368e-06\\
12.2	-2.71623481229444e-06\\
12.3	-2.5449729128546e-06\\
12.4	-2.65383869602065e-06\\
12.5	-2.01823549939029e-06\\
12.6	-2.61571121118465e-06\\
12.7	-2.425772853476e-06\\
12.8	-2.42971703501293e-06\\
12.9	-2.61047492966515e-06\\
13	-2.95247055437375e-06\\
13.1	-2.4414912170867e-06\\
13.2	-3.06456910359465e-06\\
13.3	-2.8098730541215e-06\\
13.4	-2.66660966541377e-06\\
13.5	-2.62493292146893e-06\\
13.6	-2.67586162583378e-06\\
13.7	-2.81120396916634e-06\\
13.8	-3.02348862302182e-06\\
13.9	-2.30590179750578e-06\\
14	-2.65222975492918e-06\\
14.1	-3.05680631074168e-06\\
14.2	-2.51446489074603e-06\\
14.3	-3.02049475509047e-06\\
14.4	-2.57060103014961e-06\\
14.5	-3.1608682182771e-06\\
14.6	-2.78772688755424e-06\\
14.7	-3.44792326445464e-06\\
14.8	-3.13849148136924e-06\\
14.9	-2.85672824607008e-06\\
15	-2.60016972652988e-06\\
15.1	-3.36657045540978e-06\\
15.2	-3.15388408234085e-06\\
15.3	-2.96024581292187e-06\\
15.4	-2.78395638309004e-06\\
15.5	-3.62346744453568e-06\\
15.6	-3.47736822869926e-06\\
15.7	-3.34437338356253e-06\\
15.8	-3.22331188029601e-06\\
15.9	-3.11311689338289e-06\\
16	-3.01281657499874e-06\\
16.1	-2.92152563674541e-06\\
16.2	-2.83843767845404e-06\\
16.3	-3.76281818970092e-06\\
16.4	-3.69399817057303e-06\\
16.5	-3.631368314716e-06\\
16.6	-3.57437370679392e-06\\
16.7	-3.52250898760786e-06\\
16.8	-3.47531394939704e-06\\
16.9	-3.43236951616532e-06\\
17	-3.39329408571842e-06\\
17.1	-3.35774019077917e-06\\
17.2	-3.32539145775357e-06\\
17.3	-3.29595983761255e-06\\
17.4	-3.26918307713697e-06\\
17.5	-3.24482242031188e-06\\
17.6	-3.22266050989395e-06\\
17.7	-3.2024994772728e-06\\
17.8	-3.18415920141923e-06\\
17.9	-4.16747572318243e-06\\
18	-4.15229980021969e-06\\
18.1	-4.13849559099955e-06\\
18.2	-4.12593945586792e-06\\
18.3	-4.11451886472847e-06\\
18.4	-4.1041314021717e-06\\
18.5	-4.09468386186607e-06\\
18.6	-4.0860914207741e-06\\
18.7	-4.07827688841955e-06\\
18.8	-4.07117002143575e-06\\
18.9	-4.06470690106353e-06\\
19	-4.05882936538315e-06\\
19.1	-4.05348449228349e-06\\
19.2	-4.04862412928253e-06\\
19.3	-4.04420446453724e-06\\
19.4	-4.04018563715525e-06\\
19.5	-4.03653138181248e-06\\
19.6	-4.03320870567825e-06\\
19.7	-4.03018759287388e-06\\
19.8	-4.0274407390184e-06\\
19.9	-4.02494330387082e-06\\
20	-4.02267269283829e-06\\
};
\addlegendentry{error};

\end{axis}
\end{tikzpicture}%
	\caption{Showing the compensated $\varphi$.}
	\label{fig:phi3reg}
\end{figure}
\fi

\begin{table}[h]
\center
\caption{Table with values using $\varphi(1)$ as 95\% of the analytical value, but compensated for the error.}
\begin{tabular}{c|c|c}
r (cm) & $\varphi(r)$ (statvolt/cm) & error \\ \hline
2 & 0.729329 & 4.33527e-07 \\ \hline
4 & 0.945054 & -9.16666e-07 \\ \hline
6 & 0.990086 & -1.00871e-06 \\ \hline
8 & 0.998324 & -1.31314e-06 \\ \hline
10 & 0.999729 & -1.39958e-06 \\ \hline
12 & 0.999959 & -2.00949e-06 \\ \hline
14 & 0.999996 & -2.65223e-06 \\ \hline
16 & 1 & -3.01282e-06 \\ \hline
18 & 1 & -4.1523e-06 \\ \hline
20 & 1 & -4.02267e-06 \\ 
\end{tabular}
\end{table}
\newpage
\section{How to Run and Find Code}
The \verb+c+ code developed for this assignment can be found in the directory
\begin{verbatim}
~jeve0010/Documents/jesper/fnm/lab4/src
\end{verbatim}
To compile the code simply use the \verb+make+ command. To run the \verb+c+ code execute the command \verb+./num+. This program generate the plot and table values in the subdirectory \verb+plots/data+. In order to plot this data and generate the table for the compensated $\varphi$ values when using a bad start value simple execute the matlab script \verb+phiPlot.m+ that resides in the subdirectory \verb+plots/code+. 
\end{document}
