\documentclass[11pt]{article}
\usepackage{report}
\newlength\figureheight 
\newlength\figurewidth
\newif\iftikz
\tikzfalse


\begin{document}
\section{Free Evolution}
\subsection{Selection of Time Step}
In order to find the biggest possible time steps we could use we simply started with reducing increasing the timestep until it we saw significant error. But we went so far that the the time step got so large that it went through the solution space ($x=[0,1]$) after just a few steps. Thus the smallest time step we used was  $3.45\cdot 10^{-4}$. The smallest spacial step we used before the solution started to get jaggy was $0.01$. Since the solution took such a small time to solve for we will use significantly smaller time steps and spacial time steps for the rest of the exercises.

\iftikz
\begin{figure}[H]
	\centering
	\begin{subfigure}{.9\linewidth}
		\setlength\figureheight{.5\linewidth} 
		\setlength\figurewidth{.9\linewidth}
		% This file was created by matlab2tikz.
% Minimal pgfplots version: 1.3
%
%The latest updates can be retrieved from
%  http://www.mathworks.com/matlabcentral/fileexchange/22022-matlab2tikz
%where you can also make suggestions and rate matlab2tikz.
%
\definecolor{mycolor1}{rgb}{0.00000,0.44700,0.74100}%
\definecolor{mycolor2}{rgb}{0.85000,0.32500,0.09800}%
%
\begin{tikzpicture}

\begin{axis}[%
width=0.95092\figurewidth,
height=\figureheight,
at={(0\figurewidth,0\figureheight)},
scale only axis,
xmin=0,
xmax=1,
xlabel={Position},
ymin=-3,
ymax=7,
title style={font=\bfseries},
title={Analytical Solution over Numerical Solution},
legend style={at={(0.03,0.97)},anchor=north west,legend cell align=left,align=left,draw=white!15!black},
title style={font=\small},ticklabel style={font=\tiny}
]
\addplot [color=mycolor1,solid]
  table[row sep=crcr]{%
0	4.983509e-05\\
0.01010101	4.097621e-05\\
0.02020202	3.436718e-05\\
0.03030303	2.931579e-05\\
0.04040404	2.537406e-05\\
0.05050505	2.224358e-05\\
0.06060606	1.971999e-05\\
0.07070707	1.76598e-05\\
0.08080808	1.59599e-05\\
0.09090909	1.454472e-05\\
0.1010101	1.335777e-05\\
0.1111111	1.235629e-05\\
0.1212121	1.150733e-05\\
0.1313131	1.078516e-05\\
0.1414141	1.017007e-05\\
0.1515152	9.644921e-06\\
0.1616162	9.199423e-06\\
0.1717172	8.818608e-06\\
0.1818182	8.502767e-06\\
0.1919192	8.230499e-06\\
0.2020202	8.021692e-06\\
0.2121212	7.828931e-06\\
0.2222222	7.724801e-06\\
0.2323232	7.568721e-06\\
0.2424242	7.619103e-06\\
0.2525253	7.374119e-06\\
0.2626263	7.800733e-06\\
0.2727273	7.072009e-06\\
0.2828283	8.51058e-06\\
0.2929293	6.524174e-06\\
0.3030303	9.695808e-06\\
0.3131313	7.235223e-06\\
0.3232323	8.796613e-06\\
0.3333333	1.725645e-05\\
0.3434343	1.457853e-06\\
0.3535354	5.023403e-05\\
0.3636364	2.094688e-05\\
0.3737374	9.912553e-05\\
0.3838384	0.0002442566\\
0.3939394	0.000261745\\
0.4040404	0.0009475293\\
0.4141414	0.001539987\\
0.4242424	0.002714287\\
0.4343434	0.006251136\\
0.4444444	0.01006635\\
0.4545455	0.0178519\\
0.4646465	0.03377363\\
0.4747475	0.05364224\\
0.4848485	0.08722348\\
0.4949495	0.1466257\\
0.5050505	0.2232999\\
0.5151515	0.3307566\\
0.5252525	0.5013224\\
0.5353535	0.7229525\\
0.5454545	0.9865734\\
0.5555556	1.344122\\
0.5656566	1.809075\\
0.5757576	2.315648\\
0.5858586	2.856428\\
0.5959596	3.495926\\
0.6060606	4.199714\\
0.6161616	4.833282\\
0.6262626	5.356597\\
0.6363636	5.841853\\
0.6464646	6.28096\\
0.6565657	6.539182\\
0.6666667	6.532512\\
0.6767677	6.327717\\
0.6868687	6.028471\\
0.6969697	5.649344\\
0.7070707	5.145246\\
0.7171717	4.512473\\
0.7272727	3.816756\\
0.7373737	3.140915\\
0.7474747	2.534426\\
0.7575758	2.008535\\
0.7676768	1.557409\\
0.7777778	1.174872\\
0.7878788	0.8581664\\
0.7979798	0.6051625\\
0.8080808	0.411551\\
0.8181818	0.2700715\\
0.8282828	0.171357\\
0.8383838	0.1054464\\
0.8484848	0.06317496\\
0.8585859	0.03701439\\
0.8686869	0.02131438\\
0.8787879	0.0121304\\
0.8888889	0.006865881\\
0.8989899	0.003892006\\
0.9090909	0.002226652\\
0.9191919	0.001296303\\
0.9292929	0.000774351\\
0.9393939	0.000478232\\
0.9494949	0.0003071622\\
0.959596	0.0002058649\\
0.969697	0.0001440401\\
0.979798	0.000104991\\
0.989899	7.941556e-05\\
1	6.204587e-05\\
};
\addlegendentry{$\text{|}\psi{}_\text{n}\text{(x,t)|}^\text{2}$};

\addplot [color=mycolor2,solid]
  table[row sep=crcr]{%
0	0.006382258\\
0.01010101	-0.005693388\\
0.02020202	0.005127128\\
0.03030303	-0.004655344\\
0.04040404	0.004257565\\
0.05050505	-0.003918664\\
0.06060606	0.00362728\\
0.07070707	-0.003374744\\
0.08080808	0.003154344\\
0.09090909	-0.002960805\\
0.1010101	0.002789925\\
0.1111111	-0.002638334\\
0.1212121	0.002503263\\
0.1313131	-0.002382463\\
0.1414141	0.002274064\\
0.1515152	-0.002176413\\
0.1616162	0.002088442\\
0.1717172	-0.002008374\\
0.1818182	0.001936676\\
0.1919192	-0.001869571\\
0.2020202	0.001812087\\
0.2121212	-0.00175266\\
0.2222222	0.001711034\\
0.2323232	-0.001649596\\
0.2424242	0.001634318\\
0.2525253	-0.001549213\\
0.2626263	0.001586159\\
0.2727273	-0.001446619\\
0.2828283	0.001541094\\
0.2929293	-0.001421135\\
0.3030303	0.001297681\\
0.3131313	-0.001836702\\
0.3232323	0.0003524648\\
0.3333333	-0.003030176\\
0.3434343	-0.0007223738\\
0.3535354	-0.002324622\\
0.3636364	0.003607669\\
0.3737374	0.006917565\\
0.3838384	0.01415357\\
0.3939394	0.008440161\\
0.4040404	-0.007736087\\
0.4141414	-0.0377529\\
0.4242424	-0.04553784\\
0.4343434	-0.00416396\\
0.4444444	0.08495478\\
0.4545455	0.1210858\\
0.4646465	0.0009741574\\
0.4747475	-0.2130927\\
0.4848485	-0.2134305\\
0.4949495	0.1629556\\
0.5050505	0.4654812\\
0.5151515	0.04552584\\
0.5252525	-0.6806516\\
0.5353535	-0.3426896\\
0.5454545	0.8629723\\
0.5555556	0.6414282\\
0.5656566	-1.112356\\
0.5757576	-0.8042046\\
0.5858586	1.497562\\
0.5959596	0.6698328\\
0.6060606	-2.009756\\
0.6161616	0.04957755\\
0.6262626	2.224081\\
0.6363636	-1.319297\\
0.6464646	-1.510677\\
0.6565657	2.478664\\
0.6666667	-0.4948481\\
0.6767677	-1.961102\\
0.6868687	2.288436\\
0.6969697	-0.4743964\\
0.7070707	-1.541135\\
0.7171717	2.118017\\
0.7272727	-1.160003\\
0.7373737	-0.3219549\\
0.7474747	1.30327\\
0.7575758	-1.407246\\
0.7676768	0.8689363\\
0.7777778	-0.1489086\\
0.7878788	-0.3996136\\
0.7979798	0.6473737\\
0.8080808	-0.6386003\\
0.8181818	0.4874338\\
0.8282828	-0.2991896\\
0.8383838	0.1377752\\
0.8484848	-0.02674833\\
0.8585859	-0.03589961\\
0.8686869	0.06289834\\
0.8787879	-0.06813551\\
0.8888889	0.06241376\\
0.8989899	-0.05262796\\
0.9090909	0.04249126\\
0.9191919	-0.03362744\\
0.9292929	0.0264944\\
0.9393939	-0.02100171\\
0.9494949	0.01686189\\
0.959596	-0.01376213\\
0.969697	0.01143267\\
0.979798	-0.009663701\\
0.989899	0.008300678\\
1	-0.007232945\\
};
\addlegendentry{$\text{imag(}\psi{}_\text{n}\text{(x,t))}$};

\addplot [color=red,mark size=0.7pt,only marks,mark=*,mark options={solid}]
  table[row sep=crcr]{%
0	1.65412229097909e-25\\
0.0202020202020202	5.65418478863692e-24\\
0.0404040404040404	1.73294987241482e-22\\
0.0606060606060606	4.76228762688986e-21\\
0.0808080808080808	1.17343452559411e-19\\
0.101010101010101	2.59248183105009e-18\\
0.121212121212121	5.13554098478904e-17\\
0.141414141414141	9.12158472755633e-16\\
0.161616161616162	1.45267349434318e-14\\
0.181818181818182	2.07433715889589e-13\\
0.202020202020202	2.65585463580452e-12\\
0.222222222222222	3.04889784599049e-11\\
0.242424242424242	3.13830439450614e-10\\
0.262626262626263	2.89641558271376e-09\\
0.282828282828283	2.39684695781992e-08\\
0.303030303030303	1.77841598272928e-07\\
0.323232323232323	1.18315057680938e-06\\
0.343434343434343	7.05765306368423e-06\\
0.363636363636364	3.77480229633893e-05\\
0.383838383838384	0.000181026324739673\\
0.404040404040404	0.000778400089026891\\
0.424242424242424	0.00300108045506635\\
0.444444444444444	0.0103744723815011\\
0.464646464646465	0.0321564441201201\\
0.484848484848485	0.0893683287447141\\
0.505050505050505	0.222696234357961\\
0.525252525252525	0.497571766572944\\
0.545454545454545	0.996809582644575\\
0.565656565656566	1.79053287028574\\
0.585858585858586	2.88380579888758\\
0.606060606060606	4.16450456383034\\
0.626262626262626	5.39230306301861\\
0.646464646464647	6.26035357491163\\
0.666666666666667	6.51683987985372\\
0.686868686868687	6.08259439701959\\
0.707070707070707	5.09042781848946\\
0.727272727272727	3.81973583040548\\
0.747474747474748	2.56995780171507\\
0.767676767676768	1.55035900118948\\
0.787878787878788	0.838594822382505\\
0.808080808080808	0.406710792647341\\
0.828282828282828	0.176861309631007\\
0.848484848484849	0.0689594191703558\\
0.868686868686869	0.024108376879539\\
0.888888888888889	0.00755711390395282\\
0.909090909090909	0.00212401501788712\\
0.929292929292929	0.000535269893495758\\
0.94949494949495	0.000120948829490799\\
0.96969696969697	2.45044038529882e-05\\
0.98989898989899	4.45143700656034e-06\\
};
\addlegendentry{$\text{|}\psi{}_\text{a}\text{(x,t)|}^\text{2}$};

\addplot [color=black,mark size=0.7pt,only marks,mark=*,mark options={solid}]
  table[row sep=crcr]{%
0	-3.48584777067594e-13\\
0.0202020202020202	-7.1155057653272e-13\\
0.0404040404040404	1.27637485171471e-11\\
0.0606060606060606	2.50408208561786e-11\\
0.0808080808080808	-2.67201294681666e-10\\
0.101010101010101	-1.47714312353884e-09\\
0.121212121212121	-1.82056037794223e-09\\
0.141414141414141	1.36004186618439e-08\\
0.161616161616162	1.02092343642842e-07\\
0.181818181818182	4.4656986824771e-07\\
0.202020202020202	1.6294741826555e-06\\
0.222222222222222	5.52109095603838e-06\\
0.242424242424242	1.73825991111427e-05\\
0.262626262626263	4.57615567101897e-05\\
0.282828282828283	7.09019433146301e-05\\
0.303030303030303	-0.000102621858997162\\
0.323232323232323	-0.000991959412866287\\
0.343434343434343	-0.00209849871392245\\
0.363636363636364	0.0021236122116585\\
0.383838383838384	0.0131112358923495\\
0.404040404040404	-0.00772567269742116\\
0.424242424242424	-0.0476648290951618\\
0.444444444444444	0.081708570418146\\
0.464646464646465	0.00781345708718012\\
0.484848484848485	-0.222100221350392\\
0.505050505050505	0.468504947044059\\
0.525252525252525	-0.671256889562341\\
0.545454545454545	0.843688882284122\\
0.565656565656566	-1.07704854861936\\
0.585858585858586	1.4688629312828\\
0.606060606060606	-1.9844012936074\\
0.626262626262626	2.25804866969184\\
0.646464646464647	-1.58044703493243\\
0.666666666666667	-0.378031090159871\\
0.686868686868687	2.2627132548072\\
0.707070707070707	-1.59674874009463\\
0.727272727272727	-1.08430724770132\\
0.747474747474748	1.34758253004581\\
0.767676767676768	0.837146683904619\\
0.787878787878788	-0.42918635583009\\
0.808080808080808	-0.637264980186851\\
0.828282828282828	-0.285298700105815\\
0.848484848484849	-0.0164008145131732\\
0.868686868686869	0.0661206863804965\\
0.888888888888889	0.0599900094185233\\
0.909090909090909	0.0361761540616706\\
0.929292929292929	0.0175778117298425\\
0.94949494949495	0.00659058768232577\\
0.96969696969697	0.00118409228734464\\
0.98989898989899	-0.000703987334150138\\
};
\addlegendentry{$\text{imag(}\psi{}_\text{a}\text{(x,t))}$};

\end{axis}
\end{tikzpicture}%
		\caption{The numerical solution ($\psi_n$) plotted against the analytical solution ($\psi_a$).}
		\label{fig:biggestStepPlot}
	\end{subfigure}
	\begin{subfigure}{.9\linewidth}
		\setlength\figureheight{.5\linewidth} 
		\setlength\figurewidth{.9\linewidth}
		% This file was created by matlab2tikz.
% Minimal pgfplots version: 1.3
%
%The latest updates can be retrieved from
%  http://www.mathworks.com/matlabcentral/fileexchange/22022-matlab2tikz
%where you can also make suggestions and rate matlab2tikz.
%
\definecolor{mycolor1}{rgb}{0.00000,0.44700,0.74100}%
%
\begin{tikzpicture}

\begin{axis}[%
width=0.95092\figurewidth,
height=\figureheight,
at={(0\figurewidth,0\figureheight)},
scale only axis,
xmin=0,
xmax=1,
xlabel={Position},
ymin=-0.08,
ymax=0.06,
ylabel={Magnitude},
title style={font=\bfseries},
title={$\text{Error of |}\psi{}_\text{n}\text{(x,t)|}^\text{2}\text{, dt = 0.000345, dx = 0.01}$},
title style={font=\small},ticklabel style={font=\tiny}
]
\addplot [color=mycolor1,solid,forget plot]
  table[row sep=crcr]{%
0	-4.983509e-05\\
0.01010101	-4.097621e-05\\
0.02020202	-3.436718e-05\\
0.03030303	-2.931579e-05\\
0.04040404	-2.537406e-05\\
0.05050505	-2.224358e-05\\
0.06060606	-1.971999e-05\\
0.07070707	-1.76598e-05\\
0.08080808	-1.59598999999999e-05\\
0.09090909	-1.45447199999994e-05\\
0.1010101	-1.33577699999974e-05\\
0.1111111	-1.23562899999883e-05\\
0.1212121	-1.15073299999486e-05\\
0.1313131	-1.07851599997806e-05\\
0.1414141	-1.01700699990878e-05\\
0.1515152	-9.64492099630984e-06\\
0.1616162	-9.19942298547319e-06\\
0.1717172	-8.81860794435212e-06\\
0.1818182	-8.5027667925658e-06\\
0.1919192	-8.23049824757093e-06\\
0.2020202	-8.02168934414603e-06\\
0.2121212	-7.82892187787017e-06\\
0.2222222	-7.72477051110156e-06\\
0.2323232	-7.56862183905732e-06\\
0.2424242	-7.61878917106114e-06\\
0.2525253	-7.3731524962667e-06\\
0.2626263	-7.79783657280128e-06\\
0.2727273	-7.06356253228115e-06\\
0.2828283	-8.48661148847926e-06\\
0.2929293	-6.45798886928311e-06\\
0.3030303	-9.51796645373502e-06\\
0.3131313	-6.77021604460534e-06\\
0.3232323	-7.61346492739771e-06\\
0.3333333	-1.43270938204716e-05\\
0.3434343	5.59977379190962e-06\\
0.3535354	-3.36876790803057e-05\\
0.3636364	1.68012531906737e-05\\
0.3737374	-1.53258895645637e-05\\
0.3838384	-6.32300561236675e-05\\
0.3939394	0.000118791038999799\\
0.4040404	-0.000169129429554457\\
0.4141414	9.41154957046354e-06\\
0.4242424	0.000286788791637682\\
0.4343434	-0.000594682639952643\\
0.4444444	0.000308095316563018\\
0.4545455	0.000663854829978203\\
0.4646465	-0.0016171252893417\\
0.4747475	0.000701519525555902\\
0.4848485	0.00214491359919021\\
0.4949495	-0.00361397283377782\\
0.5050505	-0.000603713437475339\\
0.5151515	0.00669177199486548\\
0.5252525	-0.00375109951202501\\
0.5353535	-0.00902068404955403\\
0.5454545	0.0102347466465921\\
0.5555556	0.0101974277804118\\
0.5656566	-0.0185405129334668\\
0.5757576	-0.0120976968631394\\
0.5858586	0.0273786508486924\\
0.5959596	0.0171537286738053\\
0.6060606	-0.0352098271302488\\
0.6161616	-0.0294033806247169\\
0.6262626	0.035704634243606\\
0.6363636	0.04807191540828\\
0.6464646	-0.0206077887800227\\
0.6565657	-0.0641593577166102\\
0.6666667	-0.015672275016124\\
0.6767677	0.0547153051813796\\
0.6868687	0.0541229086844499\\
0.6969697	-0.0084900335490099\\
0.7070707	-0.0548177670542502\\
0.7171717	-0.0423732992488821\\
0.7272727	0.0029815926172132\\
0.7373737	0.0352516146071236\\
0.7474747	0.0355345245637482\\
0.7575758	0.0149592303173445\\
0.7676768	-0.00705138782896197\\
0.7777778	-0.0189853976601937\\
0.7878788	-0.0195719142644879\\
0.7979798	-0.0131345600566658\\
0.8080808	-0.00484008075481551\\
0.8181818	0.00181202377254269\\
0.8282828	0.00550452932981824\\
0.8383838	0.006507000771152\\
0.8484848	0.00578462407786548\\
0.8585859	0.0043192017286811\\
0.8686869	0.00279395556872581\\
0.8787879	0.00155270749253078\\
0.8888889	0.000691227855469169\\
0.8989899	0.000169434036944505\\
0.9090909	-0.000102635716875982\\
0.9191919	-0.000215394165758254\\
0.9292929	-0.000239079994410615\\
0.9393939	-0.000220296095025064\\
0.9494949	-0.000186212913587908\\
0.959596	-0.000150676879557937\\
0.969697	-0.00011953575683468\\
0.979798	-9.44034678147457e-05\\
0.989899	-7.49641269111064e-05\\
1	-6.0224666885653e-05\\
};
\end{axis}
\end{tikzpicture}%
		\caption{The error is defined as the difference between the analytical solution and numerical solution.}
		\label{fig:biggestStepError}
	\end{subfigure}
	\label{fig:freeEvo}
	\caption{The numerical solution at the largest tested time and spacial step. One can clearly see in figure \ref{fig:biggestStepPlot} that the solution is starting to get jaggy from the spacial step. The larger time step never gets to the point where the solution seem to go bad before just running off our solution space $x=[0,1]$ in just a few steps. The error is not particularly large compared to the solution values, as seen in figure \ref{fig:biggestStepError}}
\end{figure}
\fi
\subsection{Behavior at the end of the Solution Space}
looking at figure \ref{fig:periodicInf} we can clearly see the periodic behavior of the numerical solution. This is in a stark contrast against the numerical solution. This probably lies in the fast fourier transform package used. This because the discrete fourier transform can only be done on periodic problems. Thus as soon we are outside of our spacial solutions space of $x=[0,1]$ we will get significant error. This is highlighted in figure \ref{fig:periodicInfError}.
\iftikz
\begin{figure}[H]
	\centering
	\begin{subfigure}{.9\linewidth}
		\setlength\figureheight{.5\linewidth} 
		\setlength\figurewidth{.9\linewidth}
		% This file was created by matlab2tikz.
% Minimal pgfplots version: 1.3
%
%The latest updates can be retrieved from
%  http://www.mathworks.com/matlabcentral/fileexchange/22022-matlab2tikz
%where you can also make suggestions and rate matlab2tikz.
%
\definecolor{mycolor1}{rgb}{0.00000,0.44700,0.74100}%
\definecolor{mycolor2}{rgb}{0.85000,0.32500,0.09800}%
%
\begin{tikzpicture}

\begin{axis}[%
width=0.95092\figurewidth,
height=\figureheight,
at={(0\figurewidth,0\figureheight)},
scale only axis,
xmin=0,
xmax=1,
xlabel={Position},
ymin=-3,
ymax=6,
title style={font=\bfseries},
title={Analytical Solution over Numerical Solution},
legend style={at={(0.03,0.97)},anchor=north west,legend cell align=left,align=left,draw=white!15!black},
title style={font=\small},ticklabel style={font=\tiny}
]
\addplot [color=mycolor1,solid,forget plot]
  table[row sep=crcr]{%
0	0.03279891\\
0.00010001	0.0326747\\
0.00020002	0.03186465\\
0.00030003	0.03087951\\
0.00040004	0.03040553\\
0.00050005	0.0300395\\
0.00060006	0.02995223\\
0.00070007	0.03017902\\
0.00080008	0.03050235\\
0.00090009	0.03055689\\
0.0010001	0.03044676\\
0.00110011	0.03006607\\
0.00120012	0.02965036\\
0.00130013	0.02982389\\
0.00140014	0.02958991\\
0.00150015	0.03005099\\
0.00160016	0.03027945\\
0.00170017	0.03061407\\
0.00180018	0.02989003\\
0.00190019	0.0295733\\
0.0020002	0.02899653\\
0.00210021	0.02903299\\
0.00220022	0.02924911\\
0.00230023	0.02902951\\
0.00240024	0.02921216\\
0.00250025	0.02825344\\
0.00260026	0.02756002\\
0.00270027	0.02747151\\
0.00280028	0.02778646\\
0.00290029	0.02774148\\
0.0030003	0.02800545\\
0.00310031	0.02795833\\
0.00320032	0.02792743\\
0.00330033	0.02714662\\
0.00340034	0.02759028\\
0.00350035	0.02813572\\
0.00360036	0.02849558\\
0.00370037	0.02847096\\
0.00380038	0.02843458\\
0.00390039	0.02867554\\
0.0040004	0.02826214\\
0.00410041	0.0277668\\
0.00420042	0.02778857\\
0.00430043	0.02731028\\
0.00440044	0.02749018\\
0.00450045	0.02714522\\
0.00460046	0.02627582\\
0.00470047	0.0257735\\
0.00480048	0.02510986\\
0.00490049	0.02574393\\
0.0050005	0.02594624\\
0.00510051	0.0259037\\
0.00520052	0.02527817\\
0.00530053	0.02531767\\
0.00540054	0.02539622\\
0.00550055	0.02551154\\
0.00560056	0.0258043\\
0.00570057	0.02536188\\
0.00580058	0.02516997\\
0.00590059	0.0246524\\
0.0060006	0.02464822\\
0.00610061	0.02497073\\
0.00620062	0.02504789\\
0.00630063	0.02473015\\
0.00640064	0.02398102\\
0.00650065	0.02297684\\
0.00660066	0.02354314\\
0.00670067	0.02356506\\
0.00680068	0.02328437\\
0.00690069	0.02316501\\
0.0070007	0.02315845\\
0.00710071	0.02345513\\
0.00720072	0.02380649\\
0.00730073	0.02427267\\
0.00740074	0.02428627\\
0.00750075	0.02451394\\
0.00760076	0.02453857\\
0.00770077	0.02509476\\
0.00780078	0.02548144\\
0.00790079	0.02510888\\
0.0080008	0.02451986\\
0.00810081	0.02378216\\
0.00820082	0.02357047\\
0.00830083	0.02378276\\
0.00840084	0.02306014\\
0.00850085	0.02206686\\
0.00860086	0.02150218\\
0.00870087	0.02125111\\
0.00880088	0.02139687\\
0.00890089	0.02140333\\
0.0090009	0.0217792\\
0.00910091	0.02191427\\
0.00920092	0.02128772\\
0.00930093	0.02224904\\
0.00940094	0.02238147\\
0.00950095	0.02224286\\
0.00960096	0.02157667\\
0.00970097	0.02127288\\
0.00980098	0.02116126\\
0.00990099	0.02105036\\
0.010001	0.02104549\\
0.01010101	0.02040838\\
0.01020102	0.01995105\\
0.01030103	0.020191\\
0.01040104	0.02085803\\
0.01050105	0.020893\\
0.01060106	0.02099419\\
0.01070107	0.02101218\\
0.01080108	0.02129707\\
0.01090109	0.02145562\\
0.0110011	0.02115158\\
0.01110111	0.02039286\\
0.01120112	0.02025078\\
0.01130113	0.02008614\\
0.01140114	0.02007754\\
0.01150115	0.01990756\\
0.01160116	0.01932657\\
0.01170117	0.01901076\\
0.01180118	0.01903419\\
0.01190119	0.01933833\\
0.0120012	0.01993611\\
0.01210121	0.01956339\\
0.01220122	0.01920894\\
0.01230123	0.02004775\\
0.01240124	0.02065795\\
0.01250125	0.01988511\\
0.01260126	0.01951036\\
0.01270127	0.0195763\\
0.01280128	0.02023035\\
0.01290129	0.01972612\\
0.0130013	0.01912858\\
0.01310131	0.0188694\\
0.01320132	0.01928717\\
0.01330133	0.01890234\\
0.01340134	0.01838385\\
0.01350135	0.01793528\\
0.01360136	0.01816563\\
0.01370137	0.01884729\\
0.01380138	0.01809102\\
0.01390139	0.01761618\\
0.0140014	0.01759306\\
0.01410141	0.01772421\\
0.01420142	0.01692827\\
0.01430143	0.01640072\\
0.01440144	0.016409\\
0.01450145	0.0168541\\
0.01460146	0.01687614\\
0.01470147	0.01680702\\
0.01480148	0.01678405\\
0.01490149	0.01751277\\
0.0150015	0.01820367\\
0.01510151	0.01763165\\
0.01520152	0.01732458\\
0.01530153	0.01755883\\
0.01540154	0.01785804\\
0.01550155	0.01744876\\
0.01560156	0.01674341\\
0.01570157	0.01720526\\
0.01580158	0.01725453\\
0.01590159	0.01647132\\
0.0160016	0.01651581\\
0.01610161	0.01654435\\
0.01620162	0.01658319\\
0.01630163	0.01675631\\
0.01640164	0.01696304\\
0.01650165	0.01727014\\
0.01660166	0.01696546\\
0.01670167	0.01617824\\
0.01680168	0.01558916\\
0.01690169	0.01619746\\
0.0170017	0.01634499\\
0.01710171	0.01612091\\
0.01720172	0.0159139\\
0.01730173	0.01637823\\
0.01740174	0.01614508\\
0.01750175	0.01561616\\
0.01760176	0.01534785\\
0.01770177	0.01555142\\
0.01780178	0.01525876\\
0.01790179	0.01470636\\
0.0180018	0.01507191\\
0.01810181	0.01440759\\
0.01820182	0.01389743\\
0.01830183	0.01355486\\
0.01840184	0.01385405\\
0.01850185	0.01459324\\
0.01860186	0.01479545\\
0.01870187	0.01467257\\
0.01880188	0.01500455\\
0.01890189	0.01525513\\
0.0190019	0.01491749\\
0.01910191	0.01474385\\
0.01920192	0.01487441\\
0.01930193	0.01485295\\
0.01940194	0.01488229\\
0.01950195	0.01498013\\
0.01960196	0.01461288\\
0.01970197	0.01426432\\
0.01980198	0.01434327\\
0.01990199	0.01435977\\
0.020002	0.01385353\\
0.02010201	0.01386864\\
0.02020202	0.01359846\\
0.02030203	0.01345535\\
0.02040204	0.01316322\\
0.02050205	0.01263971\\
0.02060206	0.01340157\\
0.02070207	0.01379289\\
0.02080208	0.01391897\\
0.02090209	0.01377972\\
0.0210021	0.01396197\\
0.02110211	0.01418684\\
0.02120212	0.01433694\\
0.02130213	0.01448349\\
0.02140214	0.01378539\\
0.02150215	0.01319255\\
0.02160216	0.01313312\\
0.02170217	0.01385992\\
0.02180218	0.01321704\\
0.02190219	0.01310436\\
0.0220022	0.01329737\\
0.02210221	0.01265199\\
0.02220222	0.01197205\\
0.02230223	0.0118698\\
0.02240224	0.01231787\\
0.02250225	0.01173885\\
0.02260226	0.01169175\\
0.02270227	0.01185941\\
0.02280228	0.0117783\\
0.02290229	0.01201141\\
0.0230023	0.01230409\\
0.02310231	0.01238037\\
0.02320232	0.01184208\\
0.02330233	0.01185691\\
0.02340234	0.0120895\\
0.02350235	0.01176416\\
0.02360236	0.01193758\\
0.02370237	0.01185876\\
0.02380238	0.01128476\\
0.02390239	0.01200747\\
0.0240024	0.01211972\\
0.02410241	0.01217124\\
0.02420242	0.01226636\\
0.02430243	0.01224921\\
0.02440244	0.01191926\\
0.02450245	0.01190428\\
0.02460246	0.01233165\\
0.02470247	0.01245038\\
0.02480248	0.01196744\\
0.02490249	0.01184733\\
0.0250025	0.01208723\\
0.02510251	0.01159301\\
0.02520252	0.01147872\\
0.02530253	0.01118573\\
0.02540254	0.01096636\\
0.02550255	0.01047533\\
0.02560256	0.01038058\\
0.02570257	0.01037632\\
0.02580258	0.01036965\\
0.02590259	0.01047896\\
0.0260026	0.01027745\\
0.02610261	0.01090067\\
0.02620262	0.01101785\\
0.02630263	0.01068623\\
0.02640264	0.01057648\\
0.02650265	0.01076076\\
0.02660266	0.01076977\\
0.02670267	0.01055801\\
0.02680268	0.01072351\\
0.02690269	0.01018356\\
0.0270027	0.01020656\\
0.02710271	0.01000544\\
0.02720272	0.009601827\\
0.02730273	0.009584042\\
0.02740274	0.009581976\\
0.02750275	0.009736833\\
0.02760276	0.01033626\\
0.02770277	0.009997812\\
0.02780278	0.009611788\\
0.02790279	0.01011936\\
0.0280028	0.01068211\\
0.02810281	0.01017663\\
0.02820282	0.01002172\\
0.02830283	0.009922288\\
0.02840284	0.009839875\\
0.02850285	0.01002707\\
0.02860286	0.009809572\\
0.02870287	0.009605796\\
0.02880288	0.009941631\\
0.02890289	0.009701868\\
0.0290029	0.009577669\\
0.02910291	0.00954319\\
0.02920292	0.009149161\\
0.02930293	0.009879691\\
0.02940294	0.009581105\\
0.02950295	0.0093875\\
0.02960296	0.009314919\\
0.02970297	0.009365687\\
0.02980298	0.008981108\\
0.02990299	0.009002584\\
0.030003	0.008668719\\
0.03010301	0.008782852\\
0.03020302	0.008990538\\
0.03030303	0.009209409\\
0.03040304	0.009155987\\
0.03050305	0.009201231\\
0.03060306	0.008981072\\
0.03070307	0.009134416\\
0.03080308	0.009125746\\
0.03090309	0.008506506\\
0.0310031	0.008647017\\
0.03110311	0.008573618\\
0.03120312	0.00841567\\
0.03130313	0.008184082\\
0.03140314	0.00796471\\
0.03150315	0.00782538\\
0.03160316	0.007801551\\
0.03170317	0.007257149\\
0.03180318	0.007601144\\
0.03190319	0.007692192\\
0.0320032	0.007754905\\
0.03210321	0.008490003\\
0.03220322	0.008390323\\
0.03230323	0.008426989\\
0.03240324	0.008650309\\
0.03250325	0.008150308\\
0.03260326	0.00862822\\
0.03270327	0.008631255\\
0.03280328	0.0088332\\
0.03290329	0.008684659\\
0.0330033	0.008555928\\
0.03310331	0.008735619\\
0.03320332	0.008732209\\
0.03330333	0.008046671\\
0.03340334	0.007917161\\
0.03350335	0.007824257\\
0.03360336	0.007737371\\
0.03370337	0.008019023\\
0.03380338	0.007685393\\
0.03390339	0.007447921\\
0.0340034	0.007173883\\
0.03410341	0.006986045\\
0.03420342	0.006991961\\
0.03430343	0.00691781\\
0.03440344	0.007182163\\
0.03450345	0.007127342\\
0.03460346	0.007268524\\
0.03470347	0.007188024\\
0.03480348	0.007007627\\
0.03490349	0.007305441\\
0.0350035	0.00744545\\
0.03510351	0.007233016\\
0.03520352	0.007477751\\
0.03530353	0.00738384\\
0.03540354	0.007183854\\
0.03550355	0.006832617\\
0.03560356	0.006798891\\
0.03570357	0.007083925\\
0.03580358	0.00652053\\
0.03590359	0.006520855\\
0.0360036	0.006728152\\
0.03610361	0.006783296\\
0.03620362	0.006660594\\
0.03630363	0.00680416\\
0.03640364	0.006986361\\
0.03650365	0.007069497\\
0.03660366	0.006912253\\
0.03670367	0.007204746\\
0.03680368	0.006947075\\
0.03690369	0.00677217\\
0.0370037	0.006517618\\
0.03710371	0.006341969\\
0.03720372	0.006378411\\
0.03730373	0.006344911\\
0.03740374	0.006591903\\
0.03750375	0.006772859\\
0.03760376	0.00686254\\
0.03770377	0.006703892\\
0.03780378	0.006414949\\
0.03790379	0.00635618\\
0.0380038	0.00631953\\
0.03810381	0.006518432\\
0.03820382	0.006450342\\
0.03830383	0.006369636\\
0.03840384	0.006477444\\
0.03850385	0.006308992\\
0.03860386	0.006166603\\
0.03870387	0.006092823\\
0.03880388	0.006379064\\
0.03890389	0.006104841\\
0.0390039	0.00614191\\
0.03910391	0.005830091\\
0.03920392	0.005623284\\
0.03930393	0.005418884\\
0.03940394	0.005247497\\
0.03950395	0.005443117\\
0.03960396	0.005417847\\
0.03970397	0.00585313\\
0.03980398	0.005511527\\
0.03990399	0.005508006\\
0.040004	0.005774232\\
0.04010401	0.005645905\\
0.04020402	0.005851713\\
0.04030403	0.005834255\\
0.04040404	0.005926942\\
0.04050405	0.00570357\\
0.04060406	0.005489222\\
0.04070407	0.005346348\\
0.04080408	0.005455909\\
0.04090409	0.00538873\\
0.0410041	0.005298873\\
0.04110411	0.005276357\\
0.04120412	0.005433797\\
0.04130413	0.005808833\\
0.04140414	0.005355826\\
0.04150415	0.005776454\\
0.04160416	0.005629201\\
0.04170417	0.005896409\\
0.04180418	0.005840726\\
0.04190419	0.005834513\\
0.0420042	0.005834301\\
0.04210421	0.005499234\\
0.04220422	0.005812607\\
0.04230423	0.00554965\\
0.04240424	0.005600339\\
0.04250425	0.005243153\\
0.04260426	0.00487147\\
0.04270427	0.00456888\\
0.04280428	0.004692562\\
0.04290429	0.004165969\\
0.0430043	0.004243096\\
0.04310431	0.004443942\\
0.04320432	0.00436417\\
0.04330433	0.004573209\\
0.04340434	0.004757453\\
0.04350435	0.005045549\\
0.04360436	0.004795178\\
0.04370437	0.005276419\\
0.04380438	0.004908096\\
0.04390439	0.004719769\\
0.0440044	0.004553008\\
0.04410441	0.004461041\\
0.04420442	0.004707908\\
0.04430443	0.005034878\\
0.04440444	0.004942825\\
0.04450445	0.005094678\\
0.04460446	0.004783427\\
0.04470447	0.004703993\\
0.04480448	0.004687545\\
0.04490449	0.004632402\\
0.0450045	0.004646111\\
0.04510451	0.004729542\\
0.04520452	0.004456228\\
0.04530453	0.004382034\\
0.04540454	0.004240763\\
0.04550455	0.004239705\\
0.04560456	0.004193251\\
0.04570457	0.004158962\\
0.04580458	0.004232914\\
0.04590459	0.004123133\\
0.0460046	0.004463883\\
0.04610461	0.004266157\\
0.04620462	0.004517729\\
0.04630463	0.004678357\\
0.04640464	0.004698267\\
0.04650465	0.004877839\\
0.04660466	0.004923176\\
0.04670467	0.004709238\\
0.04680468	0.004600252\\
0.04690469	0.00440147\\
0.0470047	0.004230036\\
0.04710471	0.004059681\\
0.04720472	0.003808869\\
0.04730473	0.003552785\\
0.04740474	0.003389714\\
0.04750475	0.003472435\\
0.04760476	0.003835628\\
0.04770477	0.003966962\\
0.04780478	0.00393837\\
0.04790479	0.003950855\\
0.0480048	0.003899902\\
0.04810481	0.003840653\\
0.04820482	0.003751005\\
0.04830483	0.003753817\\
0.04840484	0.003703659\\
0.04850485	0.003787015\\
0.04860486	0.003843523\\
0.04870487	0.003756728\\
0.04880488	0.003628575\\
0.04890489	0.003634349\\
0.0490049	0.003552841\\
0.04910491	0.003679132\\
0.04920492	0.003855274\\
0.04930493	0.003780428\\
0.04940494	0.003806095\\
0.04950495	0.003832593\\
0.04960496	0.003773936\\
0.04970497	0.003708721\\
0.04980498	0.003824191\\
0.04990499	0.004005917\\
0.050005	0.003876358\\
0.05010501	0.003728915\\
0.05020502	0.003794168\\
0.05030503	0.003770764\\
0.05040504	0.003668214\\
0.05050505	0.003640178\\
0.05060506	0.003782546\\
0.05070507	0.003707704\\
0.05080508	0.003566656\\
0.05090509	0.003621068\\
0.0510051	0.003439766\\
0.05110511	0.003342905\\
0.05120512	0.003102974\\
0.05130513	0.00291753\\
0.05140514	0.002845985\\
0.05150515	0.003000869\\
0.05160516	0.003121632\\
0.05170517	0.003363007\\
0.05180518	0.003297123\\
0.05190519	0.003224094\\
0.0520052	0.003052827\\
0.05210521	0.003070491\\
0.05220522	0.003157289\\
0.05230523	0.003177816\\
0.05240524	0.003123892\\
0.05250525	0.003002546\\
0.05260526	0.003105801\\
0.05270527	0.003168233\\
0.05280528	0.003116044\\
0.05290529	0.00310376\\
0.0530053	0.002859377\\
0.05310531	0.003134774\\
0.05320532	0.003103138\\
0.05330533	0.003021663\\
0.05340534	0.003211921\\
0.05350535	0.003228424\\
0.05360536	0.003255859\\
0.05370537	0.003219102\\
0.05380538	0.003427985\\
0.05390539	0.003480507\\
0.0540054	0.003281313\\
0.05410541	0.003014299\\
0.05420542	0.002632718\\
0.05430543	0.002662939\\
0.05440544	0.002635283\\
0.05450545	0.002733913\\
0.05460546	0.002704099\\
0.05470547	0.002768946\\
0.05480548	0.00273391\\
0.05490549	0.002644037\\
0.0550055	0.002843187\\
0.05510551	0.002810305\\
0.05520552	0.002834158\\
0.05530553	0.002965283\\
0.05540554	0.002952573\\
0.05550555	0.003231828\\
0.05560556	0.003229889\\
0.05570557	0.003046842\\
0.05580558	0.002803214\\
0.05590559	0.002886716\\
0.0560056	0.002699803\\
0.05610561	0.00259251\\
0.05620562	0.002497497\\
0.05630563	0.00236598\\
0.05640564	0.00234427\\
0.05650565	0.002274305\\
0.05660566	0.00224567\\
0.05670567	0.002264785\\
0.05680568	0.002259639\\
0.05690569	0.002094986\\
0.0570057	0.002118752\\
0.05710571	0.002234975\\
0.05720572	0.002504049\\
0.05730573	0.002664596\\
0.05740574	0.002505782\\
0.05750575	0.00250284\\
0.05760576	0.002670157\\
0.05770577	0.002863322\\
0.05780578	0.002758817\\
0.05790579	0.002728493\\
0.0580058	0.002748634\\
0.05810581	0.002815257\\
0.05820582	0.002671019\\
0.05830583	0.002550152\\
0.05840584	0.002488231\\
0.05850585	0.002283651\\
0.05860586	0.002256381\\
0.05870587	0.002389128\\
0.05880588	0.0023761\\
0.05890589	0.002368781\\
0.0590059	0.002382495\\
0.05910591	0.002372575\\
0.05920592	0.002410086\\
0.05930593	0.002426543\\
0.05940594	0.002248589\\
0.05950595	0.002155983\\
0.05960596	0.001984698\\
0.05970597	0.001998551\\
0.05980598	0.002206848\\
0.05990599	0.002199414\\
0.060006	0.002093164\\
0.06010601	0.002218783\\
0.06020602	0.002242471\\
0.06030603	0.002223206\\
0.06040604	0.002018011\\
0.06050605	0.00193467\\
0.06060606	0.001796589\\
0.06070607	0.001996639\\
0.06080608	0.002041163\\
0.06090609	0.001991367\\
0.0610061	0.001959344\\
0.06110611	0.001932947\\
0.06120612	0.002032873\\
0.06130613	0.001997857\\
0.06140614	0.001978261\\
0.06150615	0.002059095\\
0.06160616	0.001963452\\
0.06170617	0.00207171\\
0.06180618	0.002361791\\
0.06190619	0.002163765\\
0.0620062	0.002104117\\
0.06210621	0.002106156\\
0.06220622	0.002053985\\
0.06230623	0.002076208\\
0.06240624	0.002171138\\
0.06250625	0.002168706\\
0.06260626	0.002083409\\
0.06270627	0.001992857\\
0.06280628	0.001830893\\
0.06290629	0.001701871\\
0.0630063	0.001644819\\
0.06310631	0.001581419\\
0.06320632	0.001784408\\
0.06330633	0.001803032\\
0.06340634	0.001932399\\
0.06350635	0.001920145\\
0.06360636	0.00196842\\
0.06370637	0.002053729\\
0.06380638	0.002240904\\
0.06390639	0.002041307\\
0.0640064	0.001892587\\
0.06410641	0.001637451\\
0.06420642	0.001718805\\
0.06430643	0.001781795\\
0.06440644	0.001669173\\
0.06450645	0.001589179\\
0.06460646	0.00156879\\
0.06470647	0.00158998\\
0.06480648	0.001616001\\
0.06490649	0.001575271\\
0.0650065	0.001712378\\
0.06510651	0.001759014\\
0.06520652	0.001742739\\
0.06530653	0.001651579\\
0.06540654	0.001536079\\
0.06550655	0.001455519\\
0.06560656	0.001484478\\
0.06570657	0.001442868\\
0.06580658	0.001561539\\
0.06590659	0.001443403\\
0.0660066	0.001502358\\
0.06610661	0.001558735\\
0.06620662	0.001633075\\
0.06630663	0.00157995\\
0.06640664	0.001534228\\
0.06650665	0.00167251\\
0.06660666	0.001778973\\
0.06670667	0.001838002\\
0.06680668	0.001932532\\
0.06690669	0.001795086\\
0.0670067	0.001806466\\
0.06710671	0.001764944\\
0.06720672	0.001772846\\
0.06730673	0.001680197\\
0.06740674	0.001709618\\
0.06750675	0.001740619\\
0.06760676	0.001569594\\
0.06770677	0.001394041\\
0.06780678	0.001328509\\
0.06790679	0.001496255\\
0.0680068	0.0014177\\
0.06810681	0.001411976\\
0.06820682	0.001284749\\
0.06830683	0.001248329\\
0.06840684	0.001201505\\
0.06850685	0.001197113\\
0.06860686	0.001067194\\
0.06870687	0.0009809874\\
0.06880688	0.001180799\\
0.06890689	0.001501283\\
0.0690069	0.001503094\\
0.06910691	0.00130136\\
0.06920692	0.001515732\\
0.06930693	0.001614412\\
0.06940694	0.001653411\\
0.06950695	0.001532822\\
0.06960696	0.00156897\\
0.06970697	0.001594978\\
0.06980698	0.001327413\\
0.06990699	0.001117064\\
0.070007	0.0010896\\
0.07010701	0.001151534\\
0.07020702	0.00109173\\
0.07030703	0.001236969\\
0.07040704	0.001220059\\
0.07050705	0.001164747\\
0.07060706	0.001343049\\
0.07070707	0.001316647\\
0.07080708	0.001350395\\
0.07090709	0.001350037\\
0.0710071	0.001304153\\
0.07110711	0.00140101\\
0.07120712	0.001492532\\
0.07130713	0.001541976\\
0.07140714	0.001369302\\
0.07150715	0.001307207\\
0.07160716	0.001307626\\
0.07170717	0.001320234\\
0.07180718	0.001130304\\
0.07190719	0.00115291\\
0.0720072	0.001065816\\
0.07210721	0.001136576\\
0.07220722	0.001127326\\
0.07230723	0.001103261\\
0.07240724	0.001045486\\
0.07250725	0.001078626\\
0.07260726	0.001156773\\
0.07270727	0.001209118\\
0.07280728	0.001127894\\
0.07290729	0.001183338\\
0.0730073	0.001161001\\
0.07310731	0.001101548\\
0.07320732	0.001119629\\
0.07330733	0.001125708\\
0.07340734	0.001116986\\
0.07350735	0.001023456\\
0.07360736	0.001008934\\
0.07370737	0.0009649225\\
0.07380738	0.0009787418\\
0.07390739	0.0009369257\\
0.0740074	0.000954921\\
0.07410741	0.0009730922\\
0.07420742	0.001053994\\
0.07430743	0.001048891\\
0.07440744	0.0009349881\\
0.07450745	0.0009462137\\
0.07460746	0.001014707\\
0.07470747	0.001047297\\
0.07480748	0.001070986\\
0.07490749	0.001058692\\
0.0750075	0.001104565\\
0.07510751	0.001143096\\
0.07520752	0.001010357\\
0.07530753	0.0009884484\\
0.07540754	0.001141689\\
0.07550755	0.001244522\\
0.07560756	0.001184422\\
0.07570757	0.001071182\\
0.07580758	0.001222946\\
0.07590759	0.001316661\\
0.0760076	0.001161674\\
0.07610761	0.0008768073\\
0.07620762	0.0008953024\\
0.07630763	0.001093369\\
0.07640764	0.001124688\\
0.07650765	0.0009000488\\
0.07660766	0.0008052823\\
0.07670767	0.0007670848\\
0.07680768	0.0006888846\\
0.07690769	0.0006612707\\
0.0770077	0.0007361883\\
0.07710771	0.0007919448\\
0.07720772	0.0007025787\\
0.07730773	0.000780165\\
0.07740774	0.0008421752\\
0.07750775	0.000884049\\
0.07760776	0.0009612682\\
0.07770777	0.0009393991\\
0.07780778	0.000860494\\
0.07790779	0.0008838822\\
0.0780078	0.000909071\\
0.07810781	0.0008710655\\
0.07820782	0.0008558885\\
0.07830783	0.0008842407\\
0.07840784	0.0008439743\\
0.07850785	0.0009203671\\
0.07860786	0.0007930433\\
0.07870787	0.0007780339\\
0.07880788	0.0008299308\\
0.07890789	0.0008496115\\
0.0790079	0.000914926\\
0.07910791	0.0009977589\\
0.07920792	0.0009973923\\
0.07930793	0.0008498898\\
0.07940794	0.000767312\\
0.07950795	0.0008104733\\
0.07960796	0.0007855899\\
0.07970797	0.0008058593\\
0.07980798	0.00070093\\
0.07990799	0.000731751\\
0.080008	0.0007555885\\
0.08010801	0.0006497741\\
0.08020802	0.0006434299\\
0.08030803	0.0007946998\\
0.08040804	0.000849674\\
0.08050805	0.0008899058\\
0.08060806	0.0008955598\\
0.08070807	0.0008998896\\
0.08080808	0.000951929\\
0.08090809	0.0008426006\\
0.0810081	0.0007626101\\
0.08110811	0.00076264\\
0.08120812	0.0007941963\\
0.08130813	0.0007449969\\
0.08140814	0.0007650779\\
0.08150815	0.0006685281\\
0.08160816	0.0006538237\\
0.08170817	0.0006036514\\
0.08180818	0.000519616\\
0.08190819	0.0005027574\\
0.0820082	0.0004996426\\
0.08210821	0.0004505498\\
0.08220822	0.000545713\\
0.08230823	0.0006691671\\
0.08240824	0.0006869525\\
0.08250825	0.0007145338\\
0.08260826	0.0007585865\\
0.08270827	0.000780003\\
0.08280828	0.0007659221\\
0.08290829	0.0007639468\\
0.0830083	0.0006509054\\
0.08310831	0.000688231\\
0.08320832	0.0007541235\\
0.08330833	0.0005745899\\
0.08340834	0.0006132521\\
0.08350835	0.0006459514\\
0.08360836	0.0007151584\\
0.08370837	0.0006645439\\
0.08380838	0.0006943674\\
0.08390839	0.0007576039\\
0.0840084	0.0006941542\\
0.08410841	0.0006549954\\
0.08420842	0.000645115\\
0.08430843	0.00066934\\
0.08440844	0.0006277432\\
0.08450845	0.000604214\\
0.08460846	0.000658198\\
0.08470847	0.0007231711\\
0.08480848	0.000632125\\
0.08490849	0.0006047122\\
0.0850085	0.000683354\\
0.08510851	0.0008013122\\
0.08520852	0.0007461035\\
0.08530853	0.0006150955\\
0.08540854	0.0005491165\\
0.08550855	0.00057412\\
0.08560856	0.0003892717\\
0.08570857	0.0003200231\\
0.08580858	0.0003467168\\
0.08590859	0.0003345067\\
0.0860086	0.0004127892\\
0.08610861	0.000473253\\
0.08620862	0.0005146369\\
0.08630863	0.000528923\\
0.08640864	0.0005773587\\
0.08650865	0.0005668871\\
0.08660866	0.000595173\\
0.08670867	0.0005497511\\
0.08680868	0.0005783996\\
0.08690869	0.0007476953\\
0.0870087	0.0007845195\\
0.08710871	0.000610536\\
0.08720872	0.0005402758\\
0.08730873	0.0006052737\\
0.08740874	0.0005603943\\
0.08750875	0.0004601692\\
0.08760876	0.0004876186\\
0.08770877	0.0005179415\\
0.08780878	0.0005749436\\
0.08790879	0.0004721204\\
0.0880088	0.0005120626\\
0.08810881	0.0005044295\\
0.08820882	0.0004846912\\
0.08830883	0.0004321573\\
0.08840884	0.0005069835\\
0.08850885	0.0004854225\\
0.08860886	0.0005593259\\
0.08870887	0.0005067298\\
0.08880888	0.0004042348\\
0.08890889	0.0004340468\\
0.0890089	0.0004644995\\
0.08910891	0.000501016\\
0.08920892	0.0005613647\\
0.08930893	0.000552089\\
0.08940894	0.0005083832\\
0.08950895	0.00058583\\
0.08960896	0.0005243337\\
0.08970897	0.0004772152\\
0.08980898	0.0004888977\\
0.08990899	0.0005427072\\
0.090009	0.0004315433\\
0.09010901	0.0003588812\\
0.09020902	0.0003575113\\
0.09030903	0.0003906272\\
0.09040904	0.0003271767\\
0.09050905	0.0003514294\\
0.09060906	0.0004022206\\
0.09070907	0.0004200896\\
0.09080908	0.0004031773\\
0.09090909	0.0003930643\\
0.0910091	0.0003162622\\
0.09110911	0.0003284846\\
0.09120912	0.0003383815\\
0.09130913	0.0004171842\\
0.09140914	0.000437808\\
0.09150915	0.0003382735\\
0.09160916	0.0004053413\\
0.09170917	0.0004455895\\
0.09180918	0.000471399\\
0.09190919	0.0004616603\\
0.0920092	0.0004927528\\
0.09210921	0.0005897982\\
0.09220922	0.0005516271\\
0.09230923	0.0004770465\\
0.09240924	0.0004845091\\
0.09250925	0.000453494\\
0.09260926	0.0003859922\\
0.09270927	0.0003930486\\
0.09280928	0.0004008152\\
0.09290929	0.0003913126\\
0.0930093	0.0004719218\\
0.09310931	0.0005049672\\
0.09320932	0.0005113087\\
0.09330933	0.0004558629\\
0.09340934	0.0003526419\\
0.09350935	0.000304072\\
0.09360936	0.0002491351\\
0.09370937	0.0002541888\\
0.09380938	0.0002787425\\
0.09390939	0.0002920951\\
0.0940094	0.0003171202\\
0.09410941	0.0003003937\\
0.09420942	0.0003240092\\
0.09430943	0.0003420471\\
0.09440944	0.0002867048\\
0.09450945	0.0003247349\\
0.09460946	0.0003307265\\
0.09470947	0.0003597022\\
0.09480948	0.0003689019\\
0.09490949	0.0003701658\\
0.0950095	0.0003974529\\
0.09510951	0.0003596663\\
0.09520952	0.0002874466\\
0.09530953	0.0003004486\\
0.09540954	0.0003457463\\
0.09550955	0.0002943489\\
0.09560956	0.0002649512\\
0.09570957	0.0002193849\\
0.09580958	0.0002428233\\
0.09590959	0.0003007657\\
0.0960096	0.000371183\\
0.09610961	0.0004130805\\
0.09620962	0.0004870237\\
0.09630963	0.0004630716\\
0.09640964	0.0003672031\\
0.09650965	0.0004552271\\
0.09660966	0.0005675541\\
0.09670967	0.0004266081\\
0.09680968	0.0003030398\\
0.09690969	0.0002970065\\
0.0970097	0.0002712729\\
0.09710971	0.0002318649\\
0.09720972	0.0002499221\\
0.09730973	0.0002834431\\
0.09740974	0.000226265\\
0.09750975	0.000208654\\
0.09760976	0.0002658437\\
0.09770977	0.0002430822\\
0.09780978	0.0002474214\\
0.09790979	0.0002856024\\
0.0980098	0.0003257631\\
0.09810981	0.0003562947\\
0.09820982	0.0003894195\\
0.09830983	0.0003869254\\
0.09840984	0.0003333395\\
0.09850985	0.0003349105\\
0.09860986	0.0003331662\\
0.09870987	0.0002950022\\
0.09880988	0.0002528589\\
0.09890989	0.0003055484\\
0.0990099	0.0002917841\\
0.09910991	0.0001645628\\
0.09920992	0.0001441605\\
0.09930993	0.0002159751\\
0.09940994	0.0002246502\\
0.09950995	0.0002375859\\
0.09960996	0.000174235\\
0.09970997	0.000147128\\
0.09980998	0.0001516428\\
0.09990999	0.0001829799\\
0.10001	0.0002457809\\
0.10011	0.0002410658\\
0.10021	0.000244388\\
0.10031	0.0003404709\\
0.10041	0.0003870653\\
0.1005101	0.0003757715\\
0.1006101	0.000372733\\
0.1007101	0.0003688378\\
0.1008101	0.0003236815\\
0.1009101	0.0003047345\\
0.1010101	0.0003089025\\
0.1011101	0.0003046025\\
0.1012101	0.0002900064\\
0.1013101	0.0003127349\\
0.1014101	0.0002770593\\
0.1015102	0.0002479533\\
0.1016102	0.0002974073\\
0.1017102	0.0002183094\\
0.1018102	0.0001744761\\
0.1019102	0.0002340673\\
0.1020102	0.0001869042\\
0.1021102	0.0002259741\\
0.1022102	0.0002424325\\
0.1023102	0.0001787233\\
0.1024102	0.0001723447\\
0.1025103	0.000171769\\
0.1026103	0.0002151309\\
0.1027103	0.0002688395\\
0.1028103	0.0001997164\\
0.1029103	0.0001614063\\
0.1030103	0.0001263092\\
0.1031103	0.0001148933\\
0.1032103	0.0001331078\\
0.1033103	0.0001764134\\
0.1034103	0.0001783901\\
0.1035104	0.0002371345\\
0.1036104	0.0002469402\\
0.1037104	0.0002341161\\
0.1038104	0.0002808939\\
0.1039104	0.0002624409\\
0.1040104	0.0002633967\\
0.1041104	0.000211065\\
0.1042104	0.0002693361\\
0.1043104	0.0002474548\\
0.1044104	0.0002382144\\
0.1045105	0.0001972028\\
0.1046105	0.0001547556\\
0.1047105	0.0001339477\\
0.1048105	0.0001143501\\
0.1049105	0.0001287976\\
0.1050105	0.0002049354\\
0.1051105	0.0002325712\\
0.1052105	0.0002594732\\
0.1053105	0.0003049342\\
0.1054105	0.000285484\\
0.1055106	0.0002407369\\
0.1056106	0.0002237744\\
0.1057106	0.0002477049\\
0.1058106	0.0001611484\\
0.1059106	0.0001452653\\
0.1060106	0.0001312221\\
0.1061106	0.0001513431\\
0.1062106	0.0001528723\\
0.1063106	0.0001794423\\
0.1064106	0.0002238068\\
0.1065107	0.0002239951\\
0.1066107	0.0002190533\\
0.1067107	0.0002126014\\
0.1068107	0.0001513278\\
0.1069107	0.0001447735\\
0.1070107	0.0001748897\\
0.1071107	0.0001640082\\
0.1072107	0.0001572347\\
0.1073107	0.0001151167\\
0.1074107	0.0001155071\\
0.1075108	0.0001165962\\
0.1076108	0.0001812576\\
0.1077108	0.0001322886\\
0.1078108	0.0001378733\\
0.1079108	0.0001527227\\
0.1080108	0.0001502729\\
0.1081108	0.0001745085\\
0.1082108	0.0001634341\\
0.1083108	0.0001288394\\
0.1084108	0.0001389737\\
0.1085109	0.0001674897\\
0.1086109	0.0001853646\\
0.1087109	0.0001211291\\
0.1088109	0.0001553647\\
0.1089109	0.0001551714\\
0.1090109	0.0001110637\\
0.1091109	0.0001226049\\
0.1092109	0.0001416696\\
0.1093109	0.0001455569\\
0.1094109	0.0001339651\\
0.109511	0.0002172473\\
0.109611	0.000257557\\
0.109711	0.0002368339\\
0.109811	0.0002453466\\
0.109911	0.0002098713\\
0.110011	0.000179587\\
0.110111	0.0002318867\\
0.110211	0.0002158151\\
0.110311	0.0002470337\\
0.110411	0.0001963563\\
0.1105111	0.0001830456\\
0.1106111	0.0001388239\\
0.1107111	0.0001590598\\
0.1108111	0.0001179628\\
0.1109111	7.628033e-05\\
0.1110111	6.835693e-05\\
0.1111111	6.969415e-05\\
0.1112111	7.308747e-05\\
0.1113111	7.369802e-05\\
0.1114111	3.498144e-05\\
0.1115112	4.994931e-05\\
0.1116112	8.533238e-05\\
0.1117112	9.913957e-05\\
0.1118112	0.000146959\\
0.1119112	0.0001839695\\
0.1120112	0.0001839826\\
0.1121112	0.0002011051\\
0.1122112	0.0002586874\\
0.1123112	0.0002249501\\
0.1124112	0.0002364518\\
0.1125113	0.0001596547\\
0.1126113	0.0001154637\\
0.1127113	0.0001244012\\
0.1128113	0.0001057602\\
0.1129113	8.181185e-05\\
0.1130113	8.125224e-05\\
0.1131113	9.815883e-05\\
0.1132113	0.0001001123\\
0.1133113	0.0001167727\\
0.1134113	0.0001362699\\
0.1135114	0.0001308674\\
0.1136114	0.0001536051\\
0.1137114	0.0001731659\\
0.1138114	0.0001316843\\
0.1139114	9.506546e-05\\
0.1140114	0.0001383467\\
0.1141114	0.0001411878\\
0.1142114	0.0001286728\\
0.1143114	0.000103764\\
0.1144114	8.336658e-05\\
0.1145115	0.0001245693\\
0.1146115	0.0001599251\\
0.1147115	0.0001056959\\
0.1148115	0.0001220673\\
0.1149115	0.0001002202\\
0.1150115	0.000100592\\
0.1151115	0.000115213\\
0.1152115	0.0001488677\\
0.1153115	0.0001320547\\
0.1154115	9.312994e-05\\
0.1155116	8.785683e-05\\
0.1156116	7.798844e-05\\
0.1157116	0.0001396018\\
0.1158116	0.0001068272\\
0.1159116	7.270833e-05\\
0.1160116	5.638938e-05\\
0.1161116	5.253479e-05\\
0.1162116	6.344179e-05\\
0.1163116	7.768159e-05\\
0.1164116	6.663263e-05\\
0.1165117	8.907191e-05\\
0.1166117	0.0001247516\\
0.1167117	0.0001215134\\
0.1168117	9.00876e-05\\
0.1169117	9.271846e-05\\
0.1170117	8.012573e-05\\
0.1171117	5.778744e-05\\
0.1172117	5.851585e-05\\
0.1173117	9.079327e-05\\
0.1174117	9.291804e-05\\
0.1175118	8.914647e-05\\
0.1176118	9.156941e-05\\
0.1177118	0.0001548325\\
0.1178118	0.0001484153\\
0.1179118	0.0001245496\\
0.1180118	0.0001641983\\
0.1181118	0.0001177192\\
0.1182118	0.0001050112\\
0.1183118	0.000130316\\
0.1184118	0.000126778\\
0.1185119	0.0001192886\\
0.1186119	0.0001226804\\
0.1187119	7.962574e-05\\
0.1188119	5.610059e-05\\
0.1189119	6.028737e-05\\
0.1190119	9.540916e-05\\
0.1191119	0.0001174459\\
0.1192119	0.0001210442\\
0.1193119	8.892503e-05\\
0.1194119	0.0001003403\\
0.119512	8.125999e-05\\
0.119612	6.45152e-05\\
0.119712	6.217849e-05\\
0.119812	5.534132e-05\\
0.119912	5.651687e-05\\
0.120012	7.249881e-05\\
0.120112	5.670191e-05\\
0.120212	4.006544e-05\\
0.120312	3.405915e-05\\
0.120412	2.788833e-05\\
0.1205121	4.385601e-05\\
0.1206121	6.537113e-05\\
0.1207121	5.264492e-05\\
0.1208121	6.859674e-05\\
0.1209121	9.778951e-05\\
0.1210121	8.88784e-05\\
0.1211121	0.0001410804\\
0.1212121	0.0001227997\\
0.1213121	0.0001294422\\
0.1214121	0.0001401431\\
0.1215122	0.0001190495\\
0.1216122	8.747754e-05\\
0.1217122	0.0001038017\\
0.1218122	7.43547e-05\\
0.1219122	7.824951e-05\\
0.1220122	8.587082e-05\\
0.1221122	9.539535e-05\\
0.1222122	8.090865e-05\\
0.1223122	4.355359e-05\\
0.1224122	5.532059e-05\\
0.1225123	4.476742e-05\\
0.1226123	4.707219e-05\\
0.1227123	9.119312e-05\\
0.1228123	8.753455e-05\\
0.1229123	7.375932e-05\\
0.1230123	6.568932e-05\\
0.1231123	5.544791e-05\\
0.1232123	4.748162e-05\\
0.1233123	5.441328e-05\\
0.1234123	3.711952e-05\\
0.1235124	2.24949e-05\\
0.1236124	7.374288e-05\\
0.1237124	0.0001222923\\
0.1238124	0.0001089178\\
0.1239124	0.0001506718\\
0.1240124	0.000103669\\
0.1241124	0.0001051823\\
0.1242124	0.000114503\\
0.1243124	0.0001162666\\
0.1244124	8.389533e-05\\
0.1245125	4.86197e-05\\
0.1246125	3.231958e-05\\
0.1247125	4.005187e-05\\
0.1248125	1.81724e-05\\
0.1249125	1.362834e-06\\
0.1250125	1.017259e-05\\
0.1251125	1.959949e-05\\
0.1252125	6.052862e-05\\
0.1253125	6.036962e-05\\
0.1254125	6.473057e-05\\
0.1255126	8.041232e-05\\
0.1256126	8.230674e-05\\
0.1257126	7.834958e-05\\
0.1258126	8.303722e-05\\
0.1259126	9.38621e-05\\
0.1260126	0.0001152791\\
0.1261126	0.0001040963\\
0.1262126	5.559311e-05\\
0.1263126	6.424034e-05\\
0.1264126	7.404275e-05\\
0.1265127	6.495282e-05\\
0.1266127	7.154888e-05\\
0.1267127	4.664892e-05\\
0.1268127	6.482145e-05\\
0.1269127	8.540007e-05\\
0.1270127	6.515724e-05\\
0.1271127	3.745181e-05\\
0.1272127	3.984255e-05\\
0.1273127	5.848135e-05\\
0.1274127	6.787543e-05\\
0.1275128	6.271341e-05\\
0.1276128	4.955301e-05\\
0.1277128	5.493338e-05\\
0.1278128	8.120271e-05\\
0.1279128	8.386168e-05\\
0.1280128	8.247244e-05\\
0.1281128	5.94698e-05\\
0.1282128	4.926985e-05\\
0.1283128	3.041778e-05\\
0.1284128	1.805968e-05\\
0.1285129	2.609003e-05\\
0.1286129	1.634611e-05\\
0.1287129	1.730418e-05\\
0.1288129	1.88455e-05\\
0.1289129	8.540364e-06\\
0.1290129	2.123154e-05\\
0.1291129	4.45542e-05\\
0.1292129	3.987005e-05\\
0.1293129	5.48769e-05\\
0.1294129	6.89635e-05\\
0.129513	9.104843e-05\\
0.129613	9.901879e-05\\
0.129713	2.944111e-05\\
0.129813	5.659922e-05\\
0.129913	4.893707e-05\\
0.130013	5.568408e-05\\
0.130113	5.126538e-05\\
0.130213	3.882629e-05\\
0.130313	6.27159e-05\\
0.130413	4.485289e-05\\
0.1305131	3.430638e-05\\
0.1306131	4.632272e-05\\
0.1307131	3.640517e-05\\
0.1308131	5.277837e-05\\
0.1309131	7.660333e-05\\
0.1310131	6.008988e-05\\
0.1311131	5.16342e-05\\
0.1312131	4.946807e-05\\
0.1313131	4.287452e-05\\
0.1314131	7.109099e-05\\
0.1315132	8.263312e-05\\
0.1316132	2.999292e-05\\
0.1317132	2.479314e-05\\
0.1318132	1.552334e-05\\
0.1319132	1.24898e-05\\
0.1320132	7.672551e-06\\
0.1321132	2.246115e-05\\
0.1322132	3.730875e-05\\
0.1323132	6.983097e-05\\
0.1324132	6.665789e-05\\
0.1325133	4.504008e-05\\
0.1326133	6.305406e-05\\
0.1327133	6.580732e-05\\
0.1328133	4.497887e-05\\
0.1329133	3.908706e-05\\
0.1330133	3.282982e-05\\
0.1331133	2.633638e-05\\
0.1332133	2.923598e-05\\
0.1333133	2.281597e-05\\
0.1334133	2.932609e-05\\
0.1335134	5.277032e-05\\
0.1336134	4.724905e-05\\
0.1337134	1.665515e-05\\
0.1338134	9.325223e-06\\
0.1339134	1.658138e-05\\
0.1340134	2.146583e-05\\
0.1341134	9.566872e-06\\
0.1342134	1.502397e-05\\
0.1343134	2.19449e-05\\
0.1344134	3.839058e-05\\
0.1345135	1.59577e-05\\
0.1346135	3.060848e-05\\
0.1347135	2.184014e-05\\
0.1348135	6.438725e-05\\
0.1349135	0.000112057\\
0.1350135	9.806113e-05\\
0.1351135	7.300279e-05\\
0.1352135	9.672767e-05\\
0.1353135	8.994574e-05\\
0.1354135	5.978721e-05\\
0.1355136	5.981295e-05\\
0.1356136	6.7338e-05\\
0.1357136	9.157563e-05\\
0.1358136	8.239082e-05\\
0.1359136	4.157133e-05\\
0.1360136	3.1021e-05\\
0.1361136	1.784793e-05\\
0.1362136	1.248765e-05\\
0.1363136	8.552034e-06\\
0.1364136	1.687492e-05\\
0.1365137	2.337505e-05\\
0.1366137	8.355474e-06\\
0.1367137	1.539424e-05\\
0.1368137	9.070945e-06\\
0.1369137	1.938996e-05\\
0.1370137	3.368342e-05\\
0.1371137	1.28399e-05\\
0.1372137	3.958045e-05\\
0.1373137	1.719606e-05\\
0.1374137	1.703235e-05\\
0.1375138	4.373212e-05\\
0.1376138	1.978962e-05\\
0.1377138	2.468422e-05\\
0.1378138	1.85639e-05\\
0.1379138	2.834662e-05\\
0.1380138	2.679653e-05\\
0.1381138	4.157191e-05\\
0.1382138	4.69349e-05\\
0.1383138	2.154599e-05\\
0.1384138	1.454827e-05\\
0.1385139	2.932877e-05\\
0.1386139	2.51049e-05\\
0.1387139	2.202689e-05\\
0.1388139	1.16651e-05\\
0.1389139	3.081799e-05\\
0.1390139	4.444628e-05\\
0.1391139	1.578595e-05\\
0.1392139	3.031491e-05\\
0.1393139	4.273242e-05\\
0.1394139	6.084221e-05\\
0.139514	7.16272e-05\\
0.139614	4.370832e-05\\
0.139714	5.765393e-05\\
0.139814	4.436943e-05\\
0.139914	4.233289e-05\\
0.140014	3.955066e-05\\
0.140114	1.66152e-05\\
0.140214	3.263687e-05\\
0.140314	1.867677e-05\\
0.140414	1.608403e-05\\
0.1405141	1.88219e-05\\
0.1406141	1.050177e-05\\
0.1407141	9.319622e-06\\
0.1408141	1.435135e-05\\
0.1409141	1.749842e-05\\
0.1410141	3.232712e-05\\
0.1411141	3.461699e-05\\
0.1412141	2.838136e-05\\
0.1413141	1.967699e-05\\
0.1414141	3.932322e-05\\
0.1415142	4.249414e-05\\
0.1416142	5.373531e-05\\
0.1417142	5.542926e-05\\
0.1418142	4.414993e-05\\
0.1419142	4.761814e-05\\
0.1420142	1.920378e-05\\
0.1421142	4.293381e-06\\
0.1422142	2.081403e-06\\
0.1423142	2.035077e-06\\
0.1424142	1.042924e-06\\
0.1425143	6.099915e-06\\
0.1426143	1.683691e-05\\
0.1427143	3.770773e-05\\
0.1428143	4.583501e-05\\
0.1429143	2.959687e-05\\
0.1430143	3.192156e-05\\
0.1431143	1.906834e-05\\
0.1432143	2.689203e-05\\
0.1433143	2.232485e-05\\
0.1434143	2.132136e-05\\
0.1435144	2.245204e-05\\
0.1436144	2.492182e-05\\
0.1437144	1.710094e-05\\
0.1438144	1.573705e-05\\
0.1439144	4.353425e-05\\
0.1440144	7.013507e-05\\
0.1441144	5.017187e-05\\
0.1442144	2.123202e-05\\
0.1443144	2.701175e-05\\
0.1444144	3.679467e-05\\
0.1445145	6.519417e-05\\
0.1446145	4.111604e-05\\
0.1447145	2.247069e-05\\
0.1448145	1.870848e-05\\
0.1449145	2.658247e-05\\
0.1450145	1.642048e-05\\
0.1451145	2.2773e-05\\
0.1452145	1.371928e-05\\
0.1453145	1.757634e-05\\
0.1454145	5.742714e-06\\
0.1455146	3.411322e-06\\
0.1456146	1.435115e-05\\
0.1457146	2.160121e-06\\
0.1458146	1.540189e-06\\
0.1459146	2.043444e-06\\
0.1460146	8.49611e-06\\
0.1461146	8.747587e-06\\
0.1462146	1.199001e-05\\
0.1463146	2.69062e-05\\
0.1464146	1.26386e-05\\
0.1465147	1.211818e-05\\
0.1466147	3.472231e-05\\
0.1467147	4.695762e-05\\
0.1468147	4.31591e-05\\
0.1469147	3.364523e-05\\
0.1470147	3.278253e-05\\
0.1471147	7.256825e-05\\
0.1472147	4.644113e-05\\
0.1473147	2.989724e-05\\
0.1474147	2.264943e-05\\
0.1475148	5.689271e-06\\
0.1476148	7.209976e-06\\
0.1477148	3.00246e-06\\
0.1478148	9.810733e-06\\
0.1479148	9.927061e-06\\
0.1480148	1.994639e-05\\
0.1481148	1.329523e-05\\
0.1482148	1.2801e-05\\
0.1483148	3.754819e-05\\
0.1484148	3.235224e-05\\
0.1485149	2.086617e-05\\
0.1486149	5.505694e-06\\
0.1487149	4.988223e-06\\
0.1488149	1.697511e-05\\
0.1489149	1.989736e-05\\
0.1490149	4.270563e-05\\
0.1491149	2.901529e-05\\
0.1492149	2.910145e-05\\
0.1493149	2.075372e-05\\
0.1494149	1.85306e-05\\
0.149515	8.780812e-06\\
0.149615	7.337665e-06\\
0.149715	1.283733e-05\\
0.149815	1.33132e-05\\
0.149915	2.528943e-05\\
0.150015	9.976535e-06\\
0.150115	8.469798e-06\\
0.150215	1.845968e-05\\
0.150315	1.077445e-06\\
0.150415	7.376223e-06\\
0.1505151	1.587964e-05\\
0.1506151	1.242777e-05\\
0.1507151	9.714349e-06\\
0.1508151	1.039811e-05\\
0.1509151	1.41667e-05\\
0.1510151	2.372321e-05\\
0.1511151	1.772001e-05\\
0.1512151	6.548972e-06\\
0.1513151	6.784766e-06\\
0.1514151	9.926895e-06\\
0.1515152	1.734637e-05\\
0.1516152	8.704619e-06\\
0.1517152	1.640694e-05\\
0.1518152	1.141154e-05\\
0.1519152	8.651711e-06\\
0.1520152	4.916692e-06\\
0.1521152	1.568304e-05\\
0.1522152	1.351931e-05\\
0.1523152	6.115417e-06\\
0.1524152	1.386184e-05\\
0.1525153	2.733501e-05\\
0.1526153	3.273933e-05\\
0.1527153	3.460528e-05\\
0.1528153	3.138748e-05\\
0.1529153	2.570238e-05\\
0.1530153	3.023515e-05\\
0.1531153	2.147599e-05\\
0.1532153	3.787033e-05\\
0.1533153	3.099707e-05\\
0.1534153	3.218713e-05\\
0.1535154	1.462253e-05\\
0.1536154	6.705979e-06\\
0.1537154	6.437613e-06\\
0.1538154	1.405958e-05\\
0.1539154	1.09025e-05\\
0.1540154	2.082857e-05\\
0.1541154	8.780554e-06\\
0.1542154	1.20217e-06\\
0.1543154	4.879859e-06\\
0.1544154	7.863813e-07\\
0.1545155	5.299166e-06\\
0.1546155	1.113452e-05\\
0.1547155	1.187893e-06\\
0.1548155	1.258288e-05\\
0.1549155	1.136017e-05\\
0.1550155	1.639878e-05\\
0.1551155	3.376625e-05\\
0.1552155	4.807668e-05\\
0.1553155	4.311208e-05\\
0.1554155	3.821409e-05\\
0.1555156	4.211487e-05\\
0.1556156	2.401741e-05\\
0.1557156	3.948814e-06\\
0.1558156	9.216019e-06\\
0.1559156	1.284061e-05\\
0.1560156	1.121274e-05\\
0.1561156	9.274215e-06\\
0.1562156	1.080507e-05\\
0.1563156	9.344274e-06\\
0.1564156	2.239928e-06\\
0.1565157	7.790583e-06\\
0.1566157	1.358575e-05\\
0.1567157	1.16639e-05\\
0.1568157	5.695764e-06\\
0.1569157	1.982656e-05\\
0.1570157	2.21456e-05\\
0.1571157	7.905044e-06\\
0.1572157	8.331455e-06\\
0.1573157	7.351566e-06\\
0.1574157	2.46273e-05\\
0.1575158	2.747052e-05\\
0.1576158	2.052715e-05\\
0.1577158	1.11136e-05\\
0.1578158	3.432038e-06\\
0.1579158	1.052208e-06\\
0.1580158	7.26592e-06\\
0.1581158	4.355595e-06\\
0.1582158	7.817031e-06\\
0.1583158	6.643731e-06\\
0.1584158	1.557985e-05\\
0.1585159	3.052433e-05\\
0.1586159	4.36234e-05\\
0.1587159	2.379989e-05\\
0.1588159	2.155913e-05\\
0.1589159	2.15254e-05\\
0.1590159	1.763843e-05\\
0.1591159	1.551033e-05\\
0.1592159	6.38167e-06\\
0.1593159	1.385997e-07\\
0.1594159	3.822373e-06\\
0.159516	7.936648e-07\\
0.159616	2.684902e-06\\
0.159716	5.501016e-06\\
0.159816	3.600809e-06\\
0.159916	1.235101e-05\\
0.160016	1.730321e-05\\
0.160116	1.255093e-05\\
0.160216	2.232275e-05\\
0.160316	2.933421e-05\\
0.160416	2.290341e-05\\
0.1605161	1.564367e-05\\
0.1606161	3.386553e-05\\
0.1607161	2.534242e-05\\
0.1608161	2.498682e-05\\
0.1609161	7.426672e-06\\
0.1610161	1.698255e-05\\
0.1611161	2.194162e-05\\
0.1612161	4.430575e-06\\
0.1613161	8.331721e-06\\
0.1614161	9.460835e-06\\
0.1615162	1.816892e-05\\
0.1616162	2.585135e-06\\
0.1617162	8.414561e-06\\
0.1618162	5.455315e-06\\
0.1619162	8.397832e-06\\
0.1620162	1.54125e-05\\
0.1621162	2.211571e-05\\
0.1622162	1.186808e-05\\
0.1623162	1.086876e-06\\
0.1624162	3.787196e-06\\
0.1625163	5.550525e-06\\
0.1626163	1.038254e-05\\
0.1627163	5.576548e-06\\
0.1628163	4.284876e-06\\
0.1629163	1.633014e-05\\
0.1630163	1.38048e-05\\
0.1631163	5.811688e-06\\
0.1632163	2.606711e-06\\
0.1633163	4.931207e-07\\
0.1634163	5.932372e-06\\
0.1635164	3.976849e-07\\
0.1636164	1.669281e-06\\
0.1637164	1.16776e-07\\
0.1638164	3.098302e-06\\
0.1639164	9.569829e-06\\
0.1640164	2.464476e-05\\
0.1641164	1.811825e-05\\
0.1642164	2.466827e-05\\
0.1643164	2.288473e-05\\
0.1644164	5.915264e-06\\
0.1645165	5.324474e-06\\
0.1646165	1.988886e-05\\
0.1647165	3.205995e-05\\
0.1648165	1.705214e-05\\
0.1649165	2.607231e-05\\
0.1650165	8.331408e-06\\
0.1651165	1.285723e-05\\
0.1652165	6.433162e-06\\
0.1653165	4.428372e-06\\
0.1654165	2.482237e-06\\
0.1655166	5.75223e-06\\
0.1656166	5.029594e-06\\
0.1657166	8.39169e-06\\
0.1658166	1.104396e-05\\
0.1659166	1.397306e-06\\
0.1660166	2.673587e-06\\
0.1661166	1.523598e-06\\
0.1662166	7.035948e-07\\
0.1663166	1.536085e-05\\
0.1664166	4.774271e-06\\
0.1665167	9.552619e-06\\
0.1666167	1.005925e-05\\
0.1667167	1.720085e-05\\
0.1668167	1.84123e-05\\
0.1669167	1.345842e-05\\
0.1670167	2.346116e-05\\
0.1671167	1.432271e-05\\
0.1672167	1.841779e-05\\
0.1673167	1.118895e-05\\
0.1674167	8.900175e-06\\
0.1675168	9.108163e-06\\
0.1676168	1.323253e-05\\
0.1677168	1.47583e-05\\
0.1678168	4.350143e-06\\
0.1679168	8.888029e-07\\
0.1680168	4.803618e-07\\
0.1681168	7.211359e-08\\
0.1682168	2.263754e-06\\
0.1683168	2.048983e-06\\
0.1684168	2.094972e-06\\
0.1685169	5.315963e-06\\
0.1686169	1.81908e-05\\
0.1687169	7.819015e-06\\
0.1688169	3.378089e-06\\
0.1689169	1.652272e-06\\
0.1690169	6.048121e-06\\
0.1691169	4.138479e-06\\
0.1692169	2.47376e-06\\
0.1693169	3.350669e-06\\
0.1694169	1.196443e-05\\
0.169517	2.054327e-05\\
0.169617	1.654585e-05\\
0.169717	1.747949e-05\\
0.169817	3.216791e-05\\
0.169917	4.086344e-05\\
0.170017	1.198455e-05\\
0.170117	6.704065e-06\\
0.170217	7.087142e-06\\
0.170317	1.03868e-05\\
0.170417	4.149349e-06\\
0.1705171	9.964637e-06\\
0.1706171	8.589264e-06\\
0.1707171	3.825977e-06\\
0.1708171	5.316156e-07\\
0.1709171	7.56572e-07\\
0.1710171	1.178239e-05\\
0.1711171	1.070953e-05\\
0.1712171	5.519318e-06\\
0.1713171	8.84371e-06\\
0.1714171	2.542415e-06\\
0.1715172	3.140655e-06\\
0.1716172	9.900116e-07\\
0.1717172	3.143819e-06\\
0.1718172	1.005432e-06\\
0.1719172	3.560003e-07\\
0.1720172	4.110607e-07\\
0.1721172	1.09009e-06\\
0.1722172	5.666376e-06\\
0.1723172	3.786221e-06\\
0.1724172	5.006286e-06\\
0.1725173	1.580674e-05\\
0.1726173	2.050725e-05\\
0.1727173	1.408461e-05\\
0.1728173	1.283266e-05\\
0.1729173	3.358197e-06\\
0.1730173	1.006396e-05\\
0.1731173	4.431904e-06\\
0.1732173	6.161579e-06\\
0.1733173	8.682991e-06\\
0.1734173	1.065308e-06\\
0.1735174	2.816104e-07\\
0.1736174	1.225619e-07\\
0.1737174	5.575618e-07\\
0.1738174	1.909088e-06\\
0.1739174	1.702396e-05\\
0.1740174	1.569191e-05\\
0.1741174	1.330527e-05\\
0.1742174	4.688563e-05\\
0.1743174	3.814458e-05\\
0.1744174	3.417294e-05\\
0.1745175	1.561334e-05\\
0.1746175	8.131716e-06\\
0.1747175	8.93369e-06\\
0.1748175	5.521464e-06\\
0.1749175	6.290771e-06\\
0.1750175	8.15337e-06\\
0.1751175	5.631445e-06\\
0.1752175	8.06377e-08\\
0.1753175	1.777784e-06\\
0.1754175	2.315996e-07\\
0.1755176	6.105012e-06\\
0.1756176	8.695867e-06\\
0.1757176	1.036725e-05\\
0.1758176	9.725974e-06\\
0.1759176	1.179795e-05\\
0.1760176	8.46075e-06\\
0.1761176	1.447601e-05\\
0.1762176	1.26757e-05\\
0.1763176	9.254071e-06\\
0.1764176	2.090151e-05\\
0.1765177	8.262169e-06\\
0.1766177	3.424462e-06\\
0.1767177	1.021545e-06\\
0.1768177	2.249777e-06\\
0.1769177	2.794265e-06\\
0.1770177	4.389538e-06\\
0.1771177	3.708379e-06\\
0.1772177	2.569222e-06\\
0.1773177	2.679994e-06\\
0.1774177	1.114319e-06\\
0.1775178	8.535435e-06\\
0.1776178	1.725048e-05\\
0.1777178	9.159333e-06\\
0.1778178	1.478969e-05\\
0.1779178	2.443531e-05\\
0.1780178	3.055679e-05\\
0.1781178	1.394861e-05\\
0.1782178	9.276628e-06\\
0.1783178	3.332739e-06\\
0.1784178	8.451432e-07\\
0.1785179	6.853856e-06\\
0.1786179	1.4915e-05\\
0.1787179	2.155806e-05\\
0.1788179	1.884345e-05\\
0.1789179	2.327863e-05\\
0.1790179	4.617765e-05\\
0.1791179	2.942195e-05\\
0.1792179	1.850922e-05\\
0.1793179	9.422829e-06\\
0.1794179	1.566313e-05\\
0.179518	7.214541e-06\\
0.179618	4.064771e-07\\
0.179718	4.779524e-06\\
0.179818	8.868335e-07\\
0.179918	3.435782e-06\\
0.180018	1.308478e-06\\
0.180118	1.844332e-06\\
0.180218	4.631207e-07\\
0.180318	3.602013e-06\\
0.180418	2.916341e-06\\
0.1805181	9.749575e-07\\
0.1806181	1.316953e-06\\
0.1807181	2.191622e-07\\
0.1808181	1.08734e-06\\
0.1809181	9.3268e-06\\
0.1810181	6.364453e-06\\
0.1811181	7.621833e-07\\
0.1812181	4.860571e-06\\
0.1813181	8.125827e-06\\
0.1814181	1.298023e-05\\
0.1815182	5.705956e-06\\
0.1816182	6.282706e-06\\
0.1817182	6.084519e-07\\
0.1818182	3.813629e-08\\
0.1819182	1.155164e-06\\
0.1820182	7.0554e-07\\
0.1821182	4.887872e-07\\
0.1822182	6.888411e-06\\
0.1823182	1.215969e-05\\
0.1824182	3.425963e-06\\
0.1825183	3.811831e-06\\
0.1826183	1.049274e-05\\
0.1827183	4.784078e-06\\
0.1828183	5.279986e-06\\
0.1829183	3.486717e-06\\
0.1830183	1.301618e-06\\
0.1831183	3.585037e-06\\
0.1832183	1.212756e-05\\
0.1833183	8.88634e-06\\
0.1834183	2.491571e-06\\
0.1835184	9.360353e-07\\
0.1836184	1.033303e-06\\
0.1837184	2.251307e-07\\
0.1838184	5.085932e-06\\
0.1839184	1.186681e-06\\
0.1840184	1.196791e-06\\
0.1841184	1.249869e-07\\
0.1842184	4.524102e-07\\
0.1843184	1.791586e-06\\
0.1844184	7.759115e-06\\
0.1845185	7.914658e-06\\
0.1846185	2.960667e-06\\
0.1847185	3.597528e-06\\
0.1848185	5.479353e-06\\
0.1849185	1.652528e-06\\
0.1850185	3.040732e-06\\
0.1851185	1.403253e-05\\
0.1852185	3.601744e-06\\
0.1853185	2.215309e-06\\
0.1854185	3.829541e-06\\
0.1855186	7.341721e-06\\
0.1856186	2.940946e-06\\
0.1857186	2.589446e-07\\
0.1858186	3.70062e-07\\
0.1859186	1.117833e-05\\
0.1860186	3.943868e-06\\
0.1861186	3.358025e-06\\
0.1862186	4.517641e-06\\
0.1863186	3.579101e-06\\
0.1864186	2.388391e-06\\
0.1865187	1.335671e-06\\
0.1866187	8.499277e-06\\
0.1867187	4.53704e-06\\
0.1868187	2.88894e-06\\
0.1869187	3.227653e-06\\
0.1870187	1.533678e-06\\
0.1871187	2.362583e-06\\
0.1872187	5.795187e-06\\
0.1873187	2.268301e-05\\
0.1874187	1.34145e-05\\
0.1875188	7.578667e-06\\
0.1876188	7.467323e-07\\
0.1877188	7.53943e-07\\
0.1878188	4.351943e-06\\
0.1879188	6.405753e-06\\
0.1880188	7.604861e-06\\
0.1881188	1.200139e-05\\
0.1882188	8.323163e-06\\
0.1883188	1.405075e-05\\
0.1884188	1.129206e-05\\
0.1885189	1.132475e-05\\
0.1886189	1.291124e-05\\
0.1887189	2.12876e-05\\
0.1888189	1.192625e-05\\
0.1889189	6.730274e-06\\
0.1890189	1.24942e-05\\
0.1891189	3.170435e-06\\
0.1892189	9.531027e-06\\
0.1893189	1.073712e-05\\
0.1894189	1.442159e-05\\
0.189519	1.069829e-05\\
0.189619	9.828158e-06\\
0.189719	2.015212e-05\\
0.189819	1.758618e-05\\
0.189919	3.132878e-05\\
0.190019	2.882406e-05\\
0.190119	3.909712e-05\\
0.190219	2.767994e-05\\
0.190319	2.896743e-05\\
0.190419	1.31292e-05\\
0.1905191	8.261747e-06\\
0.1906191	4.981766e-06\\
0.1907191	8.168648e-06\\
0.1908191	7.327697e-06\\
0.1909191	5.145339e-06\\
0.1910191	5.067084e-06\\
0.1911191	2.344899e-06\\
0.1912191	3.637105e-06\\
0.1913191	4.222966e-06\\
0.1914191	1.597356e-06\\
0.1915192	5.234222e-06\\
0.1916192	1.001706e-05\\
0.1917192	8.650248e-06\\
0.1918192	9.781435e-07\\
0.1919192	3.035706e-06\\
0.1920192	4.310069e-06\\
0.1921192	1.813474e-06\\
0.1922192	5.270448e-06\\
0.1923192	1.442873e-05\\
0.1924192	8.187811e-06\\
0.1925193	4.422589e-06\\
0.1926193	2.937396e-06\\
0.1927193	4.574662e-07\\
0.1928193	6.358802e-06\\
0.1929193	1.211302e-05\\
0.1930193	3.316397e-06\\
0.1931193	8.45876e-06\\
0.1932193	1.603781e-05\\
0.1933193	9.95332e-07\\
0.1934193	1.876652e-07\\
0.1935194	6.502469e-06\\
0.1936194	3.080863e-06\\
0.1937194	3.486383e-06\\
0.1938194	5.875669e-06\\
0.1939194	6.484535e-06\\
0.1940194	1.613868e-06\\
0.1941194	2.341985e-06\\
0.1942194	5.344825e-06\\
0.1943194	1.390745e-05\\
0.1944194	9.53606e-06\\
0.1945195	5.639633e-06\\
0.1946195	4.20508e-06\\
0.1947195	3.095016e-06\\
0.1948195	7.031893e-06\\
0.1949195	1.509816e-06\\
0.1950195	1.276271e-07\\
0.1951195	7.438058e-07\\
0.1952195	4.10561e-06\\
0.1953195	1.180015e-06\\
0.1954195	2.660389e-06\\
0.1955196	7.551806e-06\\
0.1956196	8.068705e-06\\
0.1957196	1.438735e-05\\
0.1958196	2.074933e-06\\
0.1959196	3.31482e-06\\
0.1960196	2.116074e-06\\
0.1961196	7.875546e-07\\
0.1962196	5.9174e-06\\
0.1963196	3.566788e-06\\
0.1964196	1.480189e-05\\
0.1965197	1.190881e-05\\
0.1966197	3.505452e-06\\
0.1967197	5.160484e-06\\
0.1968197	2.939068e-06\\
0.1969197	8.876951e-06\\
0.1970197	7.987406e-06\\
0.1971197	2.599605e-06\\
0.1972197	9.228209e-06\\
0.1973197	6.618509e-06\\
0.1974197	5.761864e-06\\
0.1975198	5.614772e-06\\
0.1976198	2.136761e-06\\
0.1977198	4.245531e-07\\
0.1978198	1.390202e-06\\
0.1979198	4.986043e-06\\
0.1980198	5.219279e-08\\
0.1981198	2.435841e-06\\
0.1982198	1.752929e-06\\
0.1983198	4.587296e-06\\
0.1984198	9.538239e-08\\
0.1985199	4.070501e-07\\
0.1986199	1.976398e-07\\
0.1987199	1.349076e-06\\
0.1988199	9.235402e-06\\
0.1989199	7.336296e-06\\
0.1990199	1.549714e-05\\
0.1991199	3.263894e-06\\
0.1992199	1.238386e-06\\
0.1993199	2.68553e-07\\
0.1994199	2.987399e-06\\
0.19952	5.111251e-06\\
0.19962	9.476188e-06\\
0.19972	8.596943e-06\\
0.19982	1.733185e-06\\
0.19992	2.054313e-06\\
0.20002	1.106923e-07\\
0.20012	1.107316e-06\\
0.20022	4.025679e-06\\
0.20032	1.390377e-06\\
0.20042	7.469073e-06\\
0.2005201	4.000079e-06\\
0.2006201	9.734987e-07\\
0.2007201	7.668334e-07\\
0.2008201	5.27048e-06\\
0.2009201	7.825876e-06\\
0.2010201	1.418847e-07\\
0.2011201	3.557703e-08\\
0.2012201	5.437593e-06\\
0.2013201	2.5786e-06\\
0.2014201	6.500111e-06\\
0.2015202	3.121182e-07\\
0.2016202	1.166438e-05\\
0.2017202	1.506948e-05\\
0.2018202	9.963464e-06\\
0.2019202	1.019536e-05\\
0.2020202	6.148639e-06\\
0.2021202	6.948643e-06\\
0.2022202	1.343164e-05\\
0.2023202	2.758789e-05\\
0.2024202	1.095375e-05\\
0.2025203	3.183205e-06\\
0.2026203	6.873993e-06\\
0.2027203	9.652304e-06\\
0.2028203	8.942227e-06\\
0.2029203	1.116267e-05\\
0.2030203	3.511813e-06\\
0.2031203	2.042714e-08\\
0.2032203	7.541768e-07\\
0.2033203	1.037813e-05\\
0.2034203	9.613053e-06\\
0.2035204	1.378426e-05\\
0.2036204	7.720107e-06\\
0.2037204	8.051921e-06\\
0.2038204	2.468577e-05\\
0.2039204	2.063761e-05\\
0.2040204	9.36254e-06\\
0.2041204	6.21978e-06\\
0.2042204	1.643227e-05\\
0.2043204	1.993456e-05\\
0.2044204	1.124442e-05\\
0.2045205	5.58429e-06\\
0.2046205	1.456453e-06\\
0.2047205	1.330527e-06\\
0.2048205	2.353447e-06\\
0.2049205	2.211219e-06\\
0.2050205	2.475559e-07\\
0.2051205	1.131977e-05\\
0.2052205	8.405898e-06\\
0.2053205	1.136602e-06\\
0.2054205	7.015147e-07\\
0.2055206	2.464857e-06\\
0.2056206	3.490198e-06\\
0.2057206	3.1855e-06\\
0.2058206	5.119643e-06\\
0.2059206	1.50642e-05\\
0.2060206	8.972126e-06\\
0.2061206	1.286434e-05\\
0.2062206	3.435027e-06\\
0.2063206	6.484751e-06\\
0.2064206	6.02114e-06\\
0.2065207	4.77219e-06\\
0.2066207	1.288929e-05\\
0.2067207	9.7965e-06\\
0.2068207	6.190453e-06\\
0.2069207	1.256863e-05\\
0.2070207	1.601976e-05\\
0.2071207	7.114035e-06\\
0.2072207	7.783265e-06\\
0.2073207	2.334068e-05\\
0.2074207	2.785251e-05\\
0.2075208	1.193156e-05\\
0.2076208	5.319181e-06\\
0.2077208	2.861846e-06\\
0.2078208	2.944022e-06\\
0.2079208	2.851519e-06\\
0.2080208	7.288677e-06\\
0.2081208	4.760574e-07\\
0.2082208	1.511943e-06\\
0.2083208	8.87692e-07\\
0.2084208	6.287835e-07\\
0.2085209	1.147271e-06\\
0.2086209	1.416715e-05\\
0.2087209	1.067684e-05\\
0.2088209	1.228774e-05\\
0.2089209	1.9369e-07\\
0.2090209	4.852476e-06\\
0.2091209	4.442133e-06\\
0.2092209	7.892893e-06\\
0.2093209	8.447814e-07\\
0.2094209	1.200311e-06\\
0.209521	6.567255e-07\\
0.209621	3.152603e-06\\
0.209721	4.83703e-06\\
0.209821	8.268516e-07\\
0.209921	8.561238e-07\\
0.210021	3.575275e-06\\
0.210121	2.559247e-06\\
0.210221	3.492935e-06\\
0.210321	1.00891e-05\\
0.210421	6.276525e-06\\
0.2105211	6.32355e-06\\
0.2106211	4.73633e-06\\
0.2107211	2.603728e-06\\
0.2108211	3.307138e-06\\
0.2109211	3.763468e-06\\
0.2110211	1.38886e-06\\
0.2111211	8.003403e-06\\
0.2112211	5.857904e-06\\
0.2113211	8.674787e-06\\
0.2114211	5.743025e-06\\
0.2115212	8.173603e-07\\
0.2116212	3.481103e-06\\
0.2117212	6.867822e-06\\
0.2118212	1.108206e-05\\
0.2119212	1.08751e-05\\
0.2120212	5.083675e-06\\
0.2121212	1.0682e-08\\
0.2122212	4.115457e-06\\
0.2123212	2.911209e-06\\
0.2124212	8.138734e-07\\
0.2125213	5.812703e-06\\
0.2126213	8.562461e-06\\
0.2127213	3.123421e-06\\
0.2128213	1.44385e-06\\
0.2129213	5.592149e-06\\
0.2130213	9.091747e-06\\
0.2131213	2.403359e-06\\
0.2132213	6.398814e-06\\
0.2133213	5.855542e-06\\
0.2134213	6.28473e-06\\
0.2135214	1.206731e-05\\
0.2136214	1.292267e-05\\
0.2137214	4.433104e-06\\
0.2138214	7.660194e-06\\
0.2139214	2.211755e-05\\
0.2140214	1.243477e-05\\
0.2141214	3.553959e-06\\
0.2142214	8.063316e-06\\
0.2143214	6.816349e-06\\
0.2144214	6.280086e-06\\
0.2145215	1.23713e-05\\
0.2146215	3.640055e-06\\
0.2147215	3.698356e-06\\
0.2148215	1.201262e-05\\
0.2149215	8.685438e-06\\
0.2150215	2.490302e-06\\
0.2151215	1.670314e-06\\
0.2152215	6.18199e-06\\
0.2153215	4.403971e-06\\
0.2154215	1.850473e-06\\
0.2155216	2.696518e-06\\
0.2156216	1.96812e-06\\
0.2157216	4.249454e-07\\
0.2158216	3.024324e-06\\
0.2159216	1.009457e-06\\
0.2160216	2.203129e-06\\
0.2161216	2.662354e-06\\
0.2162216	1.170349e-05\\
0.2163216	6.038316e-06\\
0.2164216	4.285271e-06\\
0.2165217	6.746007e-07\\
0.2166217	1.658637e-06\\
0.2167217	3.591395e-06\\
0.2168217	9.048614e-06\\
0.2169217	9.679503e-07\\
0.2170217	1.20031e-06\\
0.2171217	4.142171e-06\\
0.2172217	7.852978e-06\\
0.2173217	3.1925e-06\\
0.2174217	3.772245e-06\\
0.2175218	9.813409e-07\\
0.2176218	1.141903e-05\\
0.2177218	1.421176e-05\\
0.2178218	7.44241e-06\\
0.2179218	7.267193e-06\\
0.2180218	2.953589e-06\\
0.2181218	2.489828e-07\\
0.2182218	2.723712e-06\\
0.2183218	6.780412e-06\\
0.2184218	3.219634e-06\\
0.2185219	2.579299e-06\\
0.2186219	4.565792e-06\\
0.2187219	1.102439e-07\\
0.2188219	3.241872e-06\\
0.2189219	1.063185e-06\\
0.2190219	5.373094e-06\\
0.2191219	1.539045e-06\\
0.2192219	1.382629e-06\\
0.2193219	2.382225e-06\\
0.2194219	5.077725e-07\\
0.219522	1.304895e-06\\
0.219622	1.064217e-05\\
0.219722	6.999812e-06\\
0.219822	7.747248e-06\\
0.219922	2.601447e-06\\
0.220022	5.153962e-06\\
0.220122	9.175537e-06\\
0.220222	1.47712e-05\\
0.220322	1.458055e-05\\
0.220422	1.050968e-05\\
0.2205221	1.646083e-05\\
0.2206221	3.805095e-06\\
0.2207221	1.934997e-06\\
0.2208221	4.917977e-07\\
0.2209221	9.20482e-07\\
0.2210221	9.534655e-07\\
0.2211221	1.636288e-06\\
0.2212221	8.807642e-07\\
0.2213221	8.398882e-07\\
0.2214221	9.518651e-06\\
0.2215222	1.716913e-05\\
0.2216222	1.196533e-05\\
0.2217222	1.085197e-05\\
0.2218222	4.880131e-06\\
0.2219222	6.698028e-07\\
0.2220222	1.050118e-05\\
0.2221222	8.197142e-06\\
0.2222222	2.183155e-07\\
0.2223222	8.360846e-06\\
0.2224222	1.31949e-05\\
0.2225223	1.010737e-05\\
0.2226223	3.246258e-06\\
0.2227223	2.890044e-06\\
0.2228223	5.026479e-06\\
0.2229223	2.862911e-07\\
0.2230223	1.030568e-06\\
0.2231223	2.65451e-06\\
0.2232223	1.773542e-06\\
0.2233223	5.525713e-06\\
0.2234223	3.818831e-06\\
0.2235224	4.453706e-06\\
0.2236224	1.263245e-06\\
0.2237224	5.170618e-06\\
0.2238224	9.050143e-06\\
0.2239224	5.509791e-06\\
0.2240224	5.062719e-06\\
0.2241224	1.724778e-05\\
0.2242224	4.523545e-06\\
0.2243224	6.718252e-07\\
0.2244224	5.829849e-06\\
0.2245225	1.101296e-05\\
0.2246225	1.173054e-05\\
0.2247225	1.269675e-05\\
0.2248225	1.507902e-05\\
0.2249225	2.6622e-06\\
0.2250225	9.241489e-06\\
0.2251225	1.16552e-06\\
0.2252225	3.088142e-07\\
0.2253225	1.560693e-06\\
0.2254225	1.958214e-06\\
0.2255226	1.772264e-06\\
0.2256226	5.963656e-06\\
0.2257226	1.350632e-06\\
0.2258226	6.779741e-07\\
0.2259226	2.268636e-06\\
0.2260226	2.614414e-06\\
0.2261226	1.11749e-06\\
0.2262226	1.997672e-06\\
0.2263226	4.509409e-06\\
0.2264226	1.749235e-06\\
0.2265227	3.029845e-06\\
0.2266227	1.049129e-06\\
0.2267227	7.87398e-07\\
0.2268227	1.444778e-07\\
0.2269227	1.973034e-06\\
0.2270227	1.270488e-06\\
0.2271227	2.664876e-07\\
0.2272227	3.860468e-06\\
0.2273227	1.277203e-07\\
0.2274227	2.171966e-06\\
0.2275228	3.773601e-06\\
0.2276228	1.261194e-05\\
0.2277228	9.120117e-06\\
0.2278228	2.078003e-05\\
0.2279228	1.177846e-05\\
0.2280228	2.864316e-05\\
0.2281228	2.05832e-05\\
0.2282228	1.955726e-05\\
0.2283228	6.859123e-06\\
0.2284228	6.368072e-06\\
0.2285229	7.075242e-06\\
0.2286229	9.509198e-06\\
0.2287229	7.025859e-06\\
0.2288229	1.201601e-05\\
0.2289229	1.419947e-05\\
0.2290229	3.49981e-06\\
0.2291229	3.449904e-06\\
0.2292229	7.600965e-06\\
0.2293229	1.840179e-06\\
0.2294229	2.72715e-06\\
0.229523	4.326745e-07\\
0.229623	5.918834e-07\\
0.229723	8.921898e-07\\
0.229823	4.420639e-06\\
0.229923	1.99085e-07\\
0.230023	5.067211e-07\\
0.230123	1.699172e-07\\
0.230223	1.653477e-06\\
0.230323	5.955693e-07\\
0.230423	5.396388e-07\\
0.2305231	8.287224e-07\\
0.2306231	3.857933e-06\\
0.2307231	1.021099e-05\\
0.2308231	1.951588e-08\\
0.2309231	4.065905e-06\\
0.2310231	6.277478e-06\\
0.2311231	1.195807e-05\\
0.2312231	1.253073e-05\\
0.2313231	2.434275e-05\\
0.2314231	3.15698e-05\\
0.2315232	8.664198e-06\\
0.2316232	6.886218e-06\\
0.2317232	3.827142e-07\\
0.2318232	3.681241e-06\\
0.2319232	6.13438e-06\\
0.2320232	1.11485e-05\\
0.2321232	1.302192e-05\\
0.2322232	5.538685e-06\\
0.2323232	1.019238e-05\\
0.2324232	1.326982e-05\\
0.2325233	4.188681e-06\\
0.2326233	5.331731e-06\\
0.2327233	6.725352e-06\\
0.2328233	4.612039e-06\\
0.2329233	1.697385e-05\\
0.2330233	1.905465e-05\\
0.2331233	2.15373e-06\\
0.2332233	6.505924e-06\\
0.2333233	2.852029e-06\\
0.2334233	3.901199e-06\\
0.2335234	6.049072e-06\\
0.2336234	9.75441e-06\\
0.2337234	5.680873e-06\\
0.2338234	5.12299e-06\\
0.2339234	3.355831e-06\\
0.2340234	4.317703e-06\\
0.2341234	8.985439e-06\\
0.2342234	3.000799e-06\\
0.2343234	6.028079e-06\\
0.2344234	4.542752e-06\\
0.2345235	7.68341e-08\\
0.2346235	5.016554e-07\\
0.2347235	2.51341e-06\\
0.2348235	1.59998e-07\\
0.2349235	2.435975e-06\\
0.2350235	1.14857e-06\\
0.2351235	2.637721e-07\\
0.2352235	1.297367e-07\\
0.2353235	2.094469e-06\\
0.2354235	6.123659e-06\\
0.2355236	1.871664e-06\\
0.2356236	5.167416e-06\\
0.2357236	7.004855e-07\\
0.2358236	8.770495e-06\\
0.2359236	9.863253e-06\\
0.2360236	1.209733e-06\\
0.2361236	3.375315e-06\\
0.2362236	3.175421e-06\\
0.2363236	4.251982e-06\\
0.2364236	2.169516e-06\\
0.2365237	1.013528e-06\\
0.2366237	4.766752e-06\\
0.2367237	7.451753e-09\\
0.2368237	1.207412e-07\\
0.2369237	2.667672e-06\\
0.2370237	1.343474e-05\\
0.2371237	9.058274e-06\\
0.2372237	4.151555e-06\\
0.2373237	2.364204e-06\\
0.2374237	7.820094e-06\\
0.2375238	2.880765e-06\\
0.2376238	3.836834e-06\\
0.2377238	5.101259e-06\\
0.2378238	7.704372e-06\\
0.2379238	1.058564e-05\\
0.2380238	1.915327e-05\\
0.2381238	7.291493e-06\\
0.2382238	9.19826e-06\\
0.2383238	3.672957e-06\\
0.2384238	7.254546e-07\\
0.2385239	7.716687e-07\\
0.2386239	5.514039e-06\\
0.2387239	3.70319e-06\\
0.2388239	1.254125e-05\\
0.2389239	1.312378e-06\\
0.2390239	4.049556e-06\\
0.2391239	1.009851e-05\\
0.2392239	3.498813e-06\\
0.2393239	1.029251e-05\\
0.2394239	1.157892e-05\\
0.239524	5.307354e-06\\
0.239624	3.159482e-07\\
0.239724	3.734287e-06\\
0.239824	7.699801e-07\\
0.239924	8.474659e-08\\
0.240024	1.912529e-06\\
0.240124	9.413616e-06\\
0.240224	2.949438e-06\\
0.240324	2.100703e-06\\
0.240424	8.034682e-08\\
0.2405241	4.769463e-06\\
0.2406241	5.110121e-06\\
0.2407241	1.428221e-06\\
0.2408241	1.157794e-06\\
0.2409241	4.778208e-06\\
0.2410241	1.592342e-06\\
0.2411241	3.427028e-07\\
0.2412241	1.320531e-06\\
0.2413241	2.176557e-06\\
0.2414241	3.25376e-06\\
0.2415242	1.481107e-05\\
0.2416242	6.58704e-06\\
0.2417242	1.605375e-05\\
0.2418242	2.790897e-06\\
0.2419242	4.956479e-06\\
0.2420242	3.227341e-06\\
0.2421242	1.05625e-05\\
0.2422242	1.376907e-06\\
0.2423242	5.342279e-07\\
0.2424242	2.106689e-06\\
0.2425243	1.990915e-06\\
0.2426243	7.14129e-07\\
0.2427243	4.038234e-06\\
0.2428243	3.063229e-07\\
0.2429243	1.725832e-07\\
0.2430243	1.65363e-06\\
0.2431243	3.094185e-07\\
0.2432243	6.09125e-08\\
0.2433243	1.99227e-06\\
0.2434243	9.081146e-06\\
0.2435244	1.36617e-05\\
0.2436244	2.140972e-05\\
0.2437244	1.433469e-05\\
0.2438244	1.055389e-05\\
0.2439244	9.360795e-06\\
0.2440244	8.407475e-06\\
0.2441244	1.480516e-06\\
0.2442244	5.139067e-06\\
0.2443244	6.673191e-06\\
0.2444244	9.786361e-06\\
0.2445245	5.19498e-06\\
0.2446245	1.955474e-06\\
0.2447245	7.134944e-06\\
0.2448245	6.052282e-06\\
0.2449245	1.425498e-05\\
0.2450245	2.171946e-05\\
0.2451245	1.406123e-05\\
0.2452245	5.805926e-06\\
0.2453245	1.577771e-05\\
0.2454245	1.206904e-05\\
0.2455246	1.736404e-05\\
0.2456246	1.074458e-05\\
0.2457246	6.196303e-06\\
0.2458246	6.406558e-06\\
0.2459246	1.230364e-05\\
0.2460246	8.911506e-06\\
0.2461246	1.439385e-05\\
0.2462246	1.043926e-05\\
0.2463246	9.209947e-06\\
0.2464246	9.413018e-06\\
0.2465247	9.517808e-06\\
0.2466247	7.286069e-06\\
0.2467247	6.351319e-06\\
0.2468247	4.047492e-06\\
0.2469247	8.6962e-06\\
0.2470247	1.429729e-05\\
0.2471247	8.738925e-07\\
0.2472247	2.409782e-06\\
0.2473247	3.124676e-06\\
0.2474247	7.452986e-06\\
0.2475248	5.024198e-06\\
0.2476248	3.938986e-06\\
0.2477248	5.417e-06\\
0.2478248	1.083967e-07\\
0.2479248	6.891005e-06\\
0.2480248	7.91424e-06\\
0.2481248	9.486057e-06\\
0.2482248	8.413044e-06\\
0.2483248	1.859628e-06\\
0.2484248	1.902221e-06\\
0.2485249	9.188872e-06\\
0.2486249	4.95116e-06\\
0.2487249	5.662373e-07\\
0.2488249	2.361242e-06\\
0.2489249	3.006344e-06\\
0.2490249	8.547229e-07\\
0.2491249	5.1555e-07\\
0.2492249	9.91668e-07\\
0.2493249	7.033594e-07\\
0.2494249	1.001666e-06\\
0.249525	4.68542e-06\\
0.249625	3.306061e-06\\
0.249725	5.161157e-06\\
0.249825	5.547617e-06\\
0.249925	1.537688e-06\\
0.250025	1.321642e-05\\
0.250125	5.259172e-06\\
0.250225	5.987893e-06\\
0.250325	4.279877e-07\\
0.250425	3.748018e-06\\
0.2505251	7.897702e-06\\
0.2506251	6.218001e-06\\
0.2507251	6.847021e-06\\
0.2508251	2.487278e-06\\
0.2509251	1.050512e-06\\
0.2510251	2.845994e-07\\
0.2511251	4.788909e-06\\
0.2512251	4.214003e-06\\
0.2513251	6.19013e-06\\
0.2514251	3.044298e-06\\
0.2515252	3.190924e-06\\
0.2516252	5.133442e-06\\
0.2517252	2.208649e-06\\
0.2518252	1.156703e-06\\
0.2519252	4.71755e-06\\
0.2520252	2.149723e-06\\
0.2521252	3.253682e-06\\
0.2522252	8.73538e-07\\
0.2523252	1.734159e-07\\
0.2524252	4.270506e-06\\
0.2525253	2.840233e-06\\
0.2526253	2.351095e-06\\
0.2527253	5.728175e-06\\
0.2528253	6.730344e-06\\
0.2529253	5.191527e-08\\
0.2530253	2.426357e-06\\
0.2531253	7.137553e-06\\
0.2532253	7.536372e-06\\
0.2533253	4.568205e-06\\
0.2534253	4.18612e-06\\
0.2535254	8.386637e-06\\
0.2536254	4.410232e-06\\
0.2537254	1.178113e-05\\
0.2538254	3.130821e-06\\
0.2539254	2.249506e-06\\
0.2540254	6.380815e-06\\
0.2541254	2.93957e-06\\
0.2542254	4.832619e-06\\
0.2543254	1.966121e-06\\
0.2544254	9.545657e-07\\
0.2545255	4.477898e-06\\
0.2546255	6.394152e-06\\
0.2547255	3.918539e-06\\
0.2548255	2.323097e-07\\
0.2549255	5.877669e-06\\
0.2550255	1.661534e-07\\
0.2551255	8.804337e-07\\
0.2552255	9.661041e-06\\
0.2553255	5.53488e-06\\
0.2554255	1.191317e-06\\
0.2555256	1.664065e-06\\
0.2556256	2.538292e-06\\
0.2557256	1.580291e-05\\
0.2558256	1.217369e-05\\
0.2559256	3.413119e-06\\
0.2560256	8.818793e-06\\
0.2561256	9.598016e-06\\
0.2562256	5.894388e-06\\
0.2563256	6.081449e-06\\
0.2564256	4.481558e-06\\
0.2565257	1.02268e-05\\
0.2566257	1.09338e-05\\
0.2567257	1.505128e-06\\
0.2568257	5.118368e-06\\
0.2569257	1.324831e-06\\
0.2570257	9.321259e-06\\
0.2571257	6.18159e-06\\
0.2572257	7.979237e-06\\
0.2573257	1.571813e-05\\
0.2574257	1.52636e-05\\
0.2575258	5.045183e-06\\
0.2576258	4.953265e-06\\
0.2577258	1.052601e-05\\
0.2578258	6.64529e-06\\
0.2579258	5.463153e-07\\
0.2580258	2.870147e-06\\
0.2581258	3.837445e-06\\
0.2582258	8.447618e-06\\
0.2583258	1.717944e-05\\
0.2584258	2.669597e-06\\
0.2585259	4.797368e-06\\
0.2586259	3.154255e-06\\
0.2587259	1.856617e-06\\
0.2588259	3.849895e-06\\
0.2589259	6.153606e-06\\
0.2590259	5.612928e-06\\
0.2591259	5.022377e-06\\
0.2592259	4.290164e-06\\
0.2593259	4.099166e-06\\
0.2594259	5.527955e-06\\
0.259526	8.670231e-06\\
0.259626	7.330376e-06\\
0.259726	5.079889e-07\\
0.259826	2.446553e-06\\
0.259926	1.122693e-06\\
0.260026	2.384297e-06\\
0.260126	1.149879e-06\\
0.260226	7.379685e-07\\
0.260326	5.788829e-06\\
0.260426	4.950272e-06\\
0.2605261	2.996912e-06\\
0.2606261	6.407764e-06\\
0.2607261	1.607627e-06\\
0.2608261	1.083425e-05\\
0.2609261	1.129584e-05\\
0.2610261	6.983832e-06\\
0.2611261	6.403558e-06\\
0.2612261	4.610291e-06\\
0.2613261	1.28713e-05\\
0.2614261	7.321934e-06\\
0.2615262	2.088265e-06\\
0.2616262	1.518459e-06\\
0.2617262	2.101871e-06\\
0.2618262	8.289366e-07\\
0.2619262	5.602461e-07\\
0.2620262	1.40402e-06\\
0.2621262	1.654375e-06\\
0.2622262	4.29423e-06\\
0.2623262	4.530978e-06\\
0.2624262	5.324021e-06\\
0.2625263	7.740339e-06\\
0.2626263	3.022187e-06\\
0.2627263	3.714214e-06\\
0.2628263	1.227461e-05\\
0.2629263	2.395567e-06\\
0.2630263	1.00267e-05\\
0.2631263	1.461514e-05\\
0.2632263	5.887549e-06\\
0.2633263	7.140528e-06\\
0.2634263	3.059967e-06\\
0.2635264	5.068145e-06\\
0.2636264	4.1875e-06\\
0.2637264	8.306508e-06\\
0.2638264	4.589326e-06\\
0.2639264	1.5024e-06\\
0.2640264	7.051176e-06\\
0.2641264	4.658274e-06\\
0.2642264	1.662507e-07\\
0.2643264	5.061596e-06\\
0.2644264	1.994942e-06\\
0.2645265	4.234645e-07\\
0.2646265	7.799586e-07\\
0.2647265	5.519613e-07\\
0.2648265	1.088096e-06\\
0.2649265	2.213113e-06\\
0.2650265	1.391709e-05\\
0.2651265	1.33829e-05\\
0.2652265	8.226762e-06\\
0.2653265	4.698872e-06\\
0.2654265	4.053221e-06\\
0.2655266	5.90354e-06\\
0.2656266	1.06026e-06\\
0.2657266	1.573245e-06\\
0.2658266	1.704861e-06\\
0.2659266	4.220734e-07\\
0.2660266	1.091701e-06\\
0.2661266	8.198112e-07\\
0.2662266	3.824531e-06\\
0.2663266	6.434559e-06\\
0.2664266	1.05575e-05\\
0.2665267	2.687351e-06\\
0.2666267	5.729369e-08\\
0.2667267	2.687215e-06\\
0.2668267	1.093954e-05\\
0.2669267	1.269955e-05\\
0.2670267	1.426562e-06\\
0.2671267	8.703938e-06\\
0.2672267	4.616653e-06\\
0.2673267	8.615017e-07\\
0.2674267	2.676444e-06\\
0.2675268	1.476049e-06\\
0.2676268	1.114845e-05\\
0.2677268	8.569e-06\\
0.2678268	1.425189e-05\\
0.2679268	1.42753e-05\\
0.2680268	6.524726e-06\\
0.2681268	9.259967e-06\\
0.2682268	1.938415e-05\\
0.2683268	1.666689e-05\\
0.2684268	5.78195e-06\\
0.2685269	6.656919e-07\\
0.2686269	5.016402e-06\\
0.2687269	1.279349e-06\\
0.2688269	4.585178e-06\\
0.2689269	9.052259e-06\\
0.2690269	3.323777e-06\\
0.2691269	3.046057e-06\\
0.2692269	2.299719e-06\\
0.2693269	4.709114e-06\\
0.2694269	1.356649e-05\\
0.269527	7.424596e-07\\
0.269627	4.206735e-06\\
0.269727	8.584251e-07\\
0.269827	2.53886e-06\\
0.269927	2.991265e-06\\
0.270027	2.139537e-07\\
0.270127	1.01951e-06\\
0.270227	2.042318e-06\\
0.270327	3.626735e-07\\
0.270427	1.509177e-06\\
0.2705271	2.220547e-07\\
0.2706271	4.718468e-06\\
0.2707271	1.56049e-05\\
0.2708271	1.31008e-05\\
0.2709271	3.485396e-06\\
0.2710271	8.704097e-07\\
0.2711271	3.232601e-06\\
0.2712271	4.254758e-06\\
0.2713271	3.456841e-06\\
0.2714271	3.181814e-06\\
0.2715272	1.383805e-06\\
0.2716272	2.64178e-06\\
0.2717272	7.388236e-06\\
0.2718272	7.121533e-06\\
0.2719272	1.369279e-05\\
0.2720272	2.632036e-06\\
0.2721272	3.633857e-06\\
0.2722272	4.241511e-07\\
0.2723272	1.666504e-08\\
0.2724272	3.38811e-06\\
0.2725273	6.393726e-06\\
0.2726273	3.164707e-06\\
0.2727273	5.915992e-06\\
0.2728273	1.236344e-05\\
0.2729273	1.375484e-05\\
0.2730273	4.067821e-06\\
0.2731273	6.711692e-06\\
0.2732273	2.913061e-07\\
0.2733273	1.850981e-06\\
0.2734273	3.357209e-07\\
0.2735274	1.718019e-06\\
0.2736274	2.112767e-06\\
0.2737274	3.524563e-06\\
0.2738274	1.198395e-06\\
0.2739274	9.68727e-07\\
0.2740274	3.675301e-06\\
0.2741274	1.870674e-05\\
0.2742274	8.946314e-07\\
0.2743274	1.906417e-06\\
0.2744274	4.031908e-07\\
0.2745275	6.969911e-07\\
0.2746275	5.701679e-06\\
0.2747275	1.819044e-06\\
0.2748275	6.020947e-06\\
0.2749275	3.414397e-06\\
0.2750275	2.105363e-06\\
0.2751275	5.397756e-06\\
0.2752275	2.291351e-06\\
0.2753275	1.417389e-05\\
0.2754275	3.949932e-06\\
0.2755276	4.796449e-06\\
0.2756276	8.159476e-06\\
0.2757276	3.083578e-06\\
0.2758276	4.510272e-06\\
0.2759276	3.584893e-07\\
0.2760276	2.486669e-06\\
0.2761276	1.116781e-06\\
0.2762276	3.443842e-06\\
0.2763276	1.369548e-05\\
0.2764276	1.020195e-05\\
0.2765277	2.537841e-06\\
0.2766277	6.361919e-06\\
0.2767277	1.187567e-05\\
0.2768277	1.612669e-05\\
0.2769277	1.168898e-05\\
0.2770277	4.69712e-06\\
0.2771277	9.671143e-06\\
0.2772277	5.392988e-06\\
0.2773277	1.785093e-05\\
0.2774277	4.842723e-06\\
0.2775278	2.898423e-07\\
0.2776278	3.105087e-06\\
0.2777278	1.748444e-07\\
0.2778278	8.660323e-07\\
0.2779278	3.788462e-06\\
0.2780278	2.965076e-06\\
0.2781278	1.366406e-06\\
0.2782278	1.762276e-06\\
0.2783278	6.041998e-06\\
0.2784278	1.170138e-06\\
0.2785279	1.264052e-06\\
0.2786279	4.956513e-06\\
0.2787279	3.577609e-06\\
0.2788279	1.141072e-06\\
0.2789279	2.928421e-06\\
0.2790279	3.459209e-06\\
0.2791279	4.430438e-06\\
0.2792279	1.035917e-06\\
0.2793279	3.926728e-06\\
0.2794279	7.359265e-07\\
0.279528	6.838806e-06\\
0.279628	1.099077e-05\\
0.279728	1.180294e-05\\
0.279828	1.103144e-05\\
0.279928	8.677058e-06\\
0.280028	5.601428e-06\\
0.280128	7.170073e-06\\
0.280228	8.090143e-07\\
0.280328	6.309424e-06\\
0.280428	8.305637e-06\\
0.2805281	1.603224e-05\\
0.2806281	2.352668e-05\\
0.2807281	3.185491e-05\\
0.2808281	3.091616e-06\\
0.2809281	4.400822e-07\\
0.2810281	1.649992e-06\\
0.2811281	1.630279e-06\\
0.2812281	1.015363e-05\\
0.2813281	7.873987e-06\\
0.2814281	9.016701e-06\\
0.2815282	3.645643e-07\\
0.2816282	4.475162e-06\\
0.2817282	1.831706e-06\\
0.2818282	3.124792e-06\\
0.2819282	6.356691e-06\\
0.2820282	5.225011e-06\\
0.2821282	5.659368e-06\\
0.2822282	8.16739e-06\\
0.2823282	5.114854e-06\\
0.2824282	8.614794e-06\\
0.2825283	1.18971e-05\\
0.2826283	2.092518e-07\\
0.2827283	2.117567e-06\\
0.2828283	6.827441e-06\\
0.2829283	7.894746e-08\\
0.2830283	8.979657e-06\\
0.2831283	1.009511e-05\\
0.2832283	9.196206e-07\\
0.2833283	9.006065e-07\\
0.2834283	1.221937e-05\\
0.2835284	7.677948e-06\\
0.2836284	3.1241e-06\\
0.2837284	3.666417e-06\\
0.2838284	4.340138e-07\\
0.2839284	1.760666e-08\\
0.2840284	2.9274e-06\\
0.2841284	3.689647e-06\\
0.2842284	5.684673e-06\\
0.2843284	1.069669e-06\\
0.2844284	1.729179e-07\\
0.2845285	6.413626e-06\\
0.2846285	9.911472e-06\\
0.2847285	5.67666e-06\\
0.2848285	5.383834e-06\\
0.2849285	2.183604e-06\\
0.2850285	3.195416e-06\\
0.2851285	2.638332e-06\\
0.2852285	2.345953e-06\\
0.2853285	3.398391e-06\\
0.2854285	5.206285e-06\\
0.2855286	2.008702e-06\\
0.2856286	3.196683e-07\\
0.2857286	5.092216e-06\\
0.2858286	1.106565e-06\\
0.2859286	8.331178e-06\\
0.2860286	5.63698e-06\\
0.2861286	1.466319e-07\\
0.2862286	2.051947e-06\\
0.2863286	8.488801e-06\\
0.2864286	1.028671e-06\\
0.2865287	2.8077e-07\\
0.2866287	2.048599e-06\\
0.2867287	1.118652e-06\\
0.2868287	2.325867e-06\\
0.2869287	2.155397e-06\\
0.2870287	6.557822e-07\\
0.2871287	2.454406e-06\\
0.2872287	9.100617e-06\\
0.2873287	7.225012e-06\\
0.2874287	2.219937e-06\\
0.2875288	6.123568e-07\\
0.2876288	1.021921e-05\\
0.2877288	5.899093e-06\\
0.2878288	2.199982e-06\\
0.2879288	4.204805e-07\\
0.2880288	2.407983e-06\\
0.2881288	3.575172e-07\\
0.2882288	2.520136e-06\\
0.2883288	3.250416e-09\\
0.2884288	2.641036e-06\\
0.2885289	5.345846e-06\\
0.2886289	3.296914e-06\\
0.2887289	4.925621e-06\\
0.2888289	2.549942e-06\\
0.2889289	2.752249e-06\\
0.2890289	7.616087e-07\\
0.2891289	1.88618e-06\\
0.2892289	2.491293e-06\\
0.2893289	6.071481e-06\\
0.2894289	5.885455e-06\\
0.289529	1.148524e-05\\
0.289629	2.551656e-05\\
0.289729	2.177975e-05\\
0.289829	5.529257e-06\\
0.289929	7.440268e-06\\
0.290029	5.128538e-07\\
0.290129	1.451572e-06\\
0.290229	8.73392e-07\\
0.290329	5.815855e-06\\
0.290429	1.05832e-08\\
0.2905291	1.498713e-06\\
0.2906291	2.183061e-06\\
0.2907291	2.858184e-06\\
0.2908291	2.448265e-06\\
0.2909291	3.640765e-06\\
0.2910291	9.681144e-06\\
0.2911291	3.692435e-07\\
0.2912291	2.929481e-07\\
0.2913291	4.725468e-06\\
0.2914291	3.65971e-06\\
0.2915292	2.868921e-07\\
0.2916292	2.090792e-06\\
0.2917292	2.352292e-07\\
0.2918292	4.593397e-07\\
0.2919292	3.715533e-06\\
0.2920292	6.398614e-07\\
0.2921292	2.141196e-06\\
0.2922292	6.491662e-06\\
0.2923292	6.689698e-06\\
0.2924292	9.907856e-06\\
0.2925293	7.443613e-06\\
0.2926293	8.275318e-06\\
0.2927293	1.036775e-05\\
0.2928293	2.59436e-05\\
0.2929293	2.811635e-05\\
0.2930293	2.696972e-05\\
0.2931293	1.438424e-05\\
0.2932293	9.111948e-06\\
0.2933293	1.712131e-06\\
0.2934293	6.948406e-07\\
0.2935294	6.413234e-06\\
0.2936294	2.503264e-06\\
0.2937294	1.637069e-06\\
0.2938294	3.306768e-06\\
0.2939294	2.58896e-06\\
0.2940294	2.84827e-07\\
0.2941294	4.813127e-06\\
0.2942294	1.824941e-06\\
0.2943294	4.22111e-06\\
0.2944294	6.684243e-07\\
0.2945295	1.107442e-05\\
0.2946295	2.113424e-06\\
0.2947295	2.62778e-06\\
0.2948295	1.102188e-05\\
0.2949295	8.273387e-06\\
0.2950295	1.298535e-07\\
0.2951295	1.456329e-06\\
0.2952295	6.859594e-06\\
0.2953295	7.76837e-06\\
0.2954295	3.478474e-06\\
0.2955296	8.475912e-07\\
0.2956296	3.28493e-06\\
0.2957296	5.543062e-06\\
0.2958296	2.565215e-06\\
0.2959296	5.694028e-06\\
0.2960296	6.952348e-06\\
0.2961296	6.214635e-06\\
0.2962296	4.9442e-06\\
0.2963296	4.7845e-06\\
0.2964296	5.806301e-06\\
0.2965297	7.224973e-06\\
0.2966297	5.499842e-06\\
0.2967297	1.345901e-06\\
0.2968297	4.008788e-07\\
0.2969297	3.625813e-06\\
0.2970297	4.572387e-06\\
0.2971297	6.685736e-06\\
0.2972297	3.693823e-06\\
0.2973297	8.80024e-07\\
0.2974297	2.254324e-06\\
0.2975298	2.130528e-06\\
0.2976298	6.594941e-06\\
0.2977298	2.343125e-06\\
0.2978298	1.279626e-06\\
0.2979298	3.219571e-07\\
0.2980298	6.175995e-06\\
0.2981298	3.676694e-06\\
0.2982298	1.948531e-06\\
0.2983298	1.119743e-06\\
0.2984298	1.902996e-07\\
0.2985299	6.722459e-07\\
0.2986299	8.969251e-07\\
0.2987299	3.466734e-07\\
0.2988299	3.349369e-06\\
0.2989299	1.363734e-05\\
0.2990299	1.571789e-06\\
0.2991299	1.967149e-06\\
0.2992299	1.027197e-05\\
0.2993299	1.089971e-05\\
0.2994299	7.370242e-06\\
0.29953	1.16593e-05\\
0.29963	4.752617e-06\\
0.29973	9.782695e-07\\
0.29983	4.163553e-06\\
0.29993	1.520213e-05\\
0.30003	1.184176e-05\\
0.30013	1.286871e-05\\
0.30023	7.947793e-06\\
0.30033	2.589351e-06\\
0.30043	1.536168e-06\\
0.3005301	1.141965e-06\\
0.3006301	1.737113e-06\\
0.3007301	1.079735e-05\\
0.3008301	1.339768e-05\\
0.3009301	1.235557e-05\\
0.3010301	9.790661e-06\\
0.3011301	1.012526e-06\\
0.3012301	3.583552e-07\\
0.3013301	5.588525e-07\\
0.3014301	9.501715e-07\\
0.3015302	1.244732e-06\\
0.3016302	1.534694e-06\\
0.3017302	2.07037e-08\\
0.3018302	5.351932e-06\\
0.3019302	1.076628e-06\\
0.3020302	4.424719e-07\\
0.3021302	2.727001e-06\\
0.3022302	1.372147e-06\\
0.3023302	7.136708e-06\\
0.3024302	3.676984e-06\\
0.3025303	6.792512e-06\\
0.3026303	2.146766e-06\\
0.3027303	6.040303e-06\\
0.3028303	8.302018e-06\\
0.3029303	4.47284e-06\\
0.3030303	1.820253e-05\\
0.3031303	1.74644e-05\\
0.3032303	1.133387e-05\\
0.3033303	1.558262e-06\\
0.3034303	1.043456e-06\\
0.3035304	2.371441e-06\\
0.3036304	4.126913e-06\\
0.3037304	4.890952e-07\\
0.3038304	1.112299e-06\\
0.3039304	3.944559e-07\\
0.3040304	4.046264e-06\\
0.3041304	1.445351e-06\\
0.3042304	2.45075e-06\\
0.3043304	2.112007e-06\\
0.3044304	6.770189e-08\\
0.3045305	1.034495e-06\\
0.3046305	1.570493e-07\\
0.3047305	3.019752e-06\\
0.3048305	9.916169e-07\\
0.3049305	3.778147e-06\\
0.3050305	1.006028e-07\\
0.3051305	1.173146e-06\\
0.3052305	2.766651e-07\\
0.3053305	1.402851e-06\\
0.3054305	1.128012e-06\\
0.3055306	8.758792e-06\\
0.3056306	5.308228e-06\\
0.3057306	1.581758e-05\\
0.3058306	9.351406e-06\\
0.3059306	1.326012e-06\\
0.3060306	4.375601e-06\\
0.3061306	7.361679e-06\\
0.3062306	8.743955e-06\\
0.3063306	1.737371e-06\\
0.3064306	5.987129e-06\\
0.3065307	4.118072e-06\\
0.3066307	1.020903e-05\\
0.3067307	9.050025e-06\\
0.3068307	1.061084e-05\\
0.3069307	6.791458e-06\\
0.3070307	3.165242e-06\\
0.3071307	4.226566e-08\\
0.3072307	1.443689e-06\\
0.3073307	9.767403e-06\\
0.3074307	1.637966e-05\\
0.3075308	7.584306e-06\\
0.3076308	3.082833e-06\\
0.3077308	1.572244e-06\\
0.3078308	7.575462e-08\\
0.3079308	1.053261e-06\\
0.3080308	2.596434e-07\\
0.3081308	6.071621e-07\\
0.3082308	3.856969e-06\\
0.3083308	4.684216e-06\\
0.3084308	1.018474e-06\\
0.3085309	2.239924e-06\\
0.3086309	7.654207e-06\\
0.3087309	3.791559e-06\\
0.3088309	5.280689e-06\\
0.3089309	1.961528e-06\\
0.3090309	1.186224e-05\\
0.3091309	8.98444e-06\\
0.3092309	3.102528e-06\\
0.3093309	7.127465e-09\\
0.3094309	4.900871e-06\\
0.309531	1.612705e-06\\
0.309631	3.077993e-06\\
0.309731	9.34914e-06\\
0.309831	3.762614e-06\\
0.309931	7.261407e-06\\
0.310031	1.35598e-05\\
0.310131	8.220592e-06\\
0.310231	3.081698e-06\\
0.310331	1.409994e-05\\
0.310431	8.546875e-06\\
0.3105311	1.196128e-05\\
0.3106311	1.338109e-05\\
0.3107311	2.004609e-05\\
0.3108311	7.457152e-06\\
0.3109311	4.968035e-06\\
0.3110311	1.703063e-06\\
0.3111311	3.465614e-06\\
0.3112311	6.562247e-06\\
0.3113311	3.765079e-06\\
0.3114311	7.221519e-06\\
0.3115312	9.337605e-06\\
0.3116312	6.750666e-06\\
0.3117312	3.62537e-07\\
0.3118312	1.07188e-06\\
0.3119312	1.466694e-06\\
0.3120312	4.460911e-06\\
0.3121312	2.531476e-06\\
0.3122312	1.496129e-06\\
0.3123312	5.325095e-07\\
0.3124312	5.886155e-06\\
0.3125313	1.280197e-06\\
0.3126313	2.33296e-06\\
0.3127313	2.371108e-06\\
0.3128313	3.507823e-06\\
0.3129313	1.770987e-06\\
0.3130313	5.665654e-07\\
0.3131313	1.956372e-06\\
0.3132313	2.780902e-06\\
0.3133313	3.144105e-06\\
0.3134313	8.141882e-06\\
0.3135314	6.991717e-06\\
0.3136314	3.90121e-06\\
0.3137314	5.818464e-06\\
0.3138314	6.852348e-07\\
0.3139314	1.284528e-06\\
0.3140314	4.866648e-07\\
0.3141314	3.663732e-06\\
0.3142314	2.401102e-06\\
0.3143314	6.313064e-06\\
0.3144314	7.581986e-06\\
0.3145315	1.531984e-05\\
0.3146315	3.577392e-06\\
0.3147315	1.614882e-06\\
0.3148315	3.988977e-07\\
0.3149315	5.298195e-07\\
0.3150315	2.961774e-06\\
0.3151315	3.90816e-06\\
0.3152315	2.410519e-06\\
0.3153315	9.344411e-06\\
0.3154315	1.993899e-05\\
0.3155316	2.878766e-05\\
0.3156316	1.042068e-05\\
0.3157316	1.24218e-06\\
0.3158316	1.167003e-06\\
0.3159316	4.936881e-08\\
0.3160316	5.029463e-07\\
0.3161316	1.616894e-06\\
0.3162316	4.923216e-07\\
0.3163316	1.219589e-06\\
0.3164316	1.367693e-07\\
0.3165317	4.729192e-07\\
0.3166317	2.961752e-07\\
0.3167317	1.532415e-06\\
0.3168317	6.70291e-07\\
0.3169317	3.187178e-06\\
0.3170317	2.445474e-06\\
0.3171317	5.19766e-06\\
0.3172317	9.433083e-06\\
0.3173317	5.51639e-06\\
0.3174317	4.726241e-06\\
0.3175318	5.585313e-06\\
0.3176318	1.397972e-05\\
0.3177318	4.050914e-06\\
0.3178318	6.211873e-06\\
0.3179318	4.927442e-06\\
0.3180318	1.381161e-06\\
0.3181318	5.184651e-06\\
0.3182318	2.85256e-06\\
0.3183318	3.922846e-07\\
0.3184318	1.505876e-06\\
0.3185319	3.626385e-06\\
0.3186319	5.402479e-06\\
0.3187319	4.354905e-07\\
0.3188319	2.104624e-06\\
0.3189319	5.840167e-06\\
0.3190319	3.50873e-06\\
0.3191319	3.425769e-06\\
0.3192319	9.248891e-08\\
0.3193319	7.497222e-07\\
0.3194319	3.802647e-06\\
0.319532	1.841772e-06\\
0.319632	1.821802e-06\\
0.319732	8.317228e-07\\
0.319832	3.801442e-06\\
0.319932	2.890151e-06\\
0.320032	8.088203e-07\\
0.320132	2.230702e-06\\
0.320232	8.091834e-07\\
0.320332	9.166526e-07\\
0.320432	1.2227e-06\\
0.3205321	1.509335e-06\\
0.3206321	5.251959e-07\\
0.3207321	2.853676e-06\\
0.3208321	6.90359e-10\\
0.3209321	6.426159e-07\\
0.3210321	3.839709e-06\\
0.3211321	1.083001e-06\\
0.3212321	2.273131e-06\\
0.3213321	1.412626e-06\\
0.3214321	4.41118e-06\\
0.3215322	1.139282e-06\\
0.3216322	2.537226e-06\\
0.3217322	7.666019e-07\\
0.3218322	3.948928e-07\\
0.3219322	5.716197e-06\\
0.3220322	5.571471e-06\\
0.3221322	4.067474e-06\\
0.3222322	4.070962e-06\\
0.3223322	1.504866e-06\\
0.3224322	1.863292e-06\\
0.3225323	4.966993e-06\\
0.3226323	1.258714e-05\\
0.3227323	1.3647e-05\\
0.3228323	2.755402e-06\\
0.3229323	1.164893e-06\\
0.3230323	5.31239e-06\\
0.3231323	2.760456e-05\\
0.3232323	4.413991e-05\\
0.3233323	3.677175e-05\\
0.3234323	2.515344e-05\\
0.3235324	2.874293e-06\\
0.3236324	2.307241e-06\\
0.3237324	1.179959e-06\\
0.3238324	5.638863e-06\\
0.3239324	7.300939e-06\\
0.3240324	7.394196e-06\\
0.3241324	3.614553e-06\\
0.3242324	1.735253e-06\\
0.3243324	1.133085e-05\\
0.3244324	8.51716e-06\\
0.3245325	1.105247e-05\\
0.3246325	1.432022e-05\\
0.3247325	1.814519e-05\\
0.3248325	5.57656e-06\\
0.3249325	1.019247e-05\\
0.3250325	9.936105e-07\\
0.3251325	8.600257e-06\\
0.3252325	4.151411e-06\\
0.3253325	1.793777e-06\\
0.3254325	1.059197e-05\\
0.3255326	7.866098e-06\\
0.3256326	3.255716e-06\\
0.3257326	7.513754e-06\\
0.3258326	7.888529e-06\\
0.3259326	8.940162e-06\\
0.3260326	5.996323e-07\\
0.3261326	5.847016e-06\\
0.3262326	1.677525e-06\\
0.3263326	1.909418e-06\\
0.3264326	2.641612e-06\\
0.3265327	2.558798e-06\\
0.3266327	4.818166e-06\\
0.3267327	5.48492e-06\\
0.3268327	3.060885e-06\\
0.3269327	8.89605e-06\\
0.3270327	2.750963e-06\\
0.3271327	1.462234e-06\\
0.3272327	3.803715e-06\\
0.3273327	6.514493e-06\\
0.3274327	3.00871e-06\\
0.3275328	1.129905e-06\\
0.3276328	1.360086e-06\\
0.3277328	4.355917e-06\\
0.3278328	2.482573e-06\\
0.3279328	6.043297e-06\\
0.3280328	8.637432e-06\\
0.3281328	4.110987e-06\\
0.3282328	9.522861e-06\\
0.3283328	5.957075e-06\\
0.3284328	2.815048e-06\\
0.3285329	9.10571e-07\\
0.3286329	7.954371e-07\\
0.3287329	1.144881e-06\\
0.3288329	4.479675e-06\\
0.3289329	2.124952e-06\\
0.3290329	7.176767e-06\\
0.3291329	5.083172e-07\\
0.3292329	1.487932e-06\\
0.3293329	3.961513e-06\\
0.3294329	7.294059e-06\\
0.329533	2.883074e-06\\
0.329633	6.425036e-07\\
0.329733	6.280349e-06\\
0.329833	3.911327e-06\\
0.329933	2.218925e-06\\
0.330033	5.681682e-07\\
0.330133	3.757142e-07\\
0.330233	1.242855e-06\\
0.330333	1.063353e-06\\
0.330433	4.666359e-06\\
0.3305331	6.495618e-07\\
0.3306331	6.058597e-08\\
0.3307331	7.405695e-07\\
0.3308331	2.373196e-06\\
0.3309331	7.92916e-07\\
0.3310331	1.58522e-07\\
0.3311331	1.377898e-07\\
0.3312331	2.572104e-06\\
0.3313331	2.96523e-06\\
0.3314331	1.91675e-06\\
0.3315332	1.386988e-06\\
0.3316332	2.165182e-07\\
0.3317332	2.39306e-06\\
0.3318332	4.704112e-07\\
0.3319332	5.27121e-06\\
0.3320332	4.348868e-06\\
0.3321332	3.73682e-07\\
0.3322332	2.710325e-06\\
0.3323332	3.466594e-06\\
0.3324332	6.04325e-06\\
0.3325333	1.211103e-06\\
0.3326333	1.449788e-05\\
0.3327333	3.518685e-06\\
0.3328333	1.915587e-07\\
0.3329333	2.029479e-06\\
0.3330333	3.610083e-06\\
0.3331333	1.438486e-06\\
0.3332333	2.642716e-06\\
0.3333333	2.04289e-06\\
0.3334333	2.236181e-06\\
0.3335334	1.096821e-06\\
0.3336334	2.272244e-06\\
0.3337334	3.933896e-06\\
0.3338334	4.515216e-06\\
0.3339334	5.829385e-06\\
0.3340334	2.22854e-06\\
0.3341334	4.255665e-06\\
0.3342334	2.78883e-06\\
0.3343334	2.001091e-06\\
0.3344334	7.043375e-06\\
0.3345335	1.425424e-05\\
0.3346335	6.230253e-06\\
0.3347335	7.104208e-06\\
0.3348335	4.125647e-07\\
0.3349335	3.004372e-06\\
0.3350335	6.541803e-06\\
0.3351335	6.439512e-06\\
0.3352335	6.821933e-06\\
0.3353335	1.670337e-06\\
0.3354335	1.419979e-06\\
0.3355336	1.019557e-05\\
0.3356336	1.46752e-05\\
0.3357336	1.9885e-05\\
0.3358336	6.151439e-06\\
0.3359336	3.547077e-06\\
0.3360336	2.918163e-06\\
0.3361336	4.337327e-06\\
0.3362336	3.070737e-06\\
0.3363336	6.040499e-06\\
0.3364336	3.505052e-06\\
0.3365337	1.218648e-06\\
0.3366337	4.320334e-06\\
0.3367337	5.846966e-07\\
0.3368337	3.023884e-06\\
0.3369337	5.141862e-06\\
0.3370337	1.926083e-06\\
0.3371337	1.935794e-07\\
0.3372337	7.319978e-06\\
0.3373337	2.643505e-06\\
0.3374337	1.566987e-08\\
0.3375338	4.186334e-06\\
0.3376338	2.88183e-06\\
0.3377338	4.324271e-06\\
0.3378338	1.216271e-05\\
0.3379338	2.225534e-05\\
0.3380338	2.013959e-05\\
0.3381338	4.517722e-06\\
0.3382338	1.316019e-06\\
0.3383338	1.470598e-06\\
0.3384338	4.648166e-07\\
0.3385339	5.886641e-06\\
0.3386339	1.395417e-05\\
0.3387339	1.21027e-05\\
0.3388339	1.64045e-05\\
0.3389339	1.063973e-05\\
0.3390339	3.060446e-05\\
0.3391339	1.856509e-05\\
0.3392339	1.252188e-05\\
0.3393339	3.167695e-06\\
0.3394339	3.60673e-06\\
0.339534	3.574949e-06\\
0.339634	2.222813e-06\\
0.339734	8.011689e-07\\
0.339834	2.422642e-06\\
0.339934	1.89564e-06\\
0.340034	4.492043e-06\\
0.340134	3.847188e-06\\
0.340234	1.800008e-06\\
0.340334	5.173319e-07\\
0.340434	3.308498e-06\\
0.3405341	3.87527e-06\\
0.3406341	1.101353e-05\\
0.3407341	1.827278e-05\\
0.3408341	4.11076e-06\\
0.3409341	1.177898e-06\\
0.3410341	9.859756e-06\\
0.3411341	8.107579e-07\\
0.3412341	1.405224e-06\\
0.3413341	1.664875e-06\\
0.3414341	7.206221e-06\\
0.3415342	1.219652e-06\\
0.3416342	1.164392e-05\\
0.3417342	3.163761e-06\\
0.3418342	7.748003e-06\\
0.3419342	1.017586e-06\\
0.3420342	6.82515e-06\\
0.3421342	4.425583e-06\\
0.3422342	5.919441e-06\\
0.3423342	6.460657e-06\\
0.3424342	4.842688e-06\\
0.3425343	1.323332e-05\\
0.3426343	1.505614e-06\\
0.3427343	3.16132e-06\\
0.3428343	9.648545e-07\\
0.3429343	1.439435e-06\\
0.3430343	2.756558e-07\\
0.3431343	1.942571e-06\\
0.3432343	9.210779e-06\\
0.3433343	6.377639e-06\\
0.3434343	6.152053e-06\\
0.3435344	1.003666e-05\\
0.3436344	1.127723e-05\\
0.3437344	2.443839e-06\\
0.3438344	8.051848e-06\\
0.3439344	8.255457e-06\\
0.3440344	4.717314e-06\\
0.3441344	5.484562e-06\\
0.3442344	4.288029e-07\\
0.3443344	2.628615e-07\\
0.3444344	8.169631e-07\\
0.3445345	1.855867e-06\\
0.3446345	3.360871e-06\\
0.3447345	7.71724e-07\\
0.3448345	1.992124e-06\\
0.3449345	5.064084e-06\\
0.3450345	1.821836e-06\\
0.3451345	3.491287e-06\\
0.3452345	4.86021e-06\\
0.3453345	2.108115e-06\\
0.3454345	2.903937e-07\\
0.3455346	5.49264e-06\\
0.3456346	1.456138e-05\\
0.3457346	2.590228e-05\\
0.3458346	1.404903e-05\\
0.3459346	2.259082e-06\\
0.3460346	4.512713e-06\\
0.3461346	9.45466e-07\\
0.3462346	4.520375e-07\\
0.3463346	5.119666e-06\\
0.3464346	4.851676e-06\\
0.3465347	4.39869e-06\\
0.3466347	6.914256e-06\\
0.3467347	8.593321e-08\\
0.3468347	1.152803e-06\\
0.3469347	3.836221e-06\\
0.3470347	4.366783e-07\\
0.3471347	7.784628e-08\\
0.3472347	3.72793e-07\\
0.3473347	5.156907e-07\\
0.3474347	5.889934e-06\\
0.3475348	5.970453e-07\\
0.3476348	6.042411e-07\\
0.3477348	1.729356e-06\\
0.3478348	7.633947e-06\\
0.3479348	1.256115e-05\\
0.3480348	1.401841e-05\\
0.3481348	7.655384e-06\\
0.3482348	1.27012e-06\\
0.3483348	5.079631e-06\\
0.3484348	1.946308e-05\\
0.3485349	1.584941e-05\\
0.3486349	4.532317e-06\\
0.3487349	2.917127e-07\\
0.3488349	4.621705e-06\\
0.3489349	1.67907e-06\\
0.3490349	4.737563e-06\\
0.3491349	1.670818e-06\\
0.3492349	2.353809e-06\\
0.3493349	9.533778e-06\\
0.3494349	7.38181e-06\\
0.349535	8.141703e-06\\
0.349635	3.239187e-06\\
0.349735	1.615355e-06\\
0.349835	6.678595e-06\\
0.349935	8.159658e-06\\
0.350035	1.644851e-05\\
0.350135	2.149528e-05\\
0.350235	5.742906e-06\\
0.350335	1.206259e-05\\
0.350435	2.93981e-06\\
0.3505351	7.980228e-06\\
0.3506351	4.185563e-06\\
0.3507351	5.267397e-06\\
0.3508351	3.026555e-06\\
0.3509351	4.494977e-08\\
0.3510351	4.184565e-06\\
0.3511351	8.552922e-06\\
0.3512351	4.701318e-07\\
0.3513351	4.00011e-06\\
0.3514351	1.757973e-06\\
0.3515352	1.590798e-06\\
0.3516352	1.209359e-06\\
0.3517352	1.093584e-06\\
0.3518352	2.087646e-06\\
0.3519352	1.218324e-06\\
0.3520352	1.637874e-06\\
0.3521352	8.855962e-07\\
0.3522352	2.737295e-06\\
0.3523352	5.878507e-06\\
0.3524352	4.722947e-06\\
0.3525353	4.61249e-06\\
0.3526353	3.119598e-07\\
0.3527353	4.96136e-06\\
0.3528353	7.922507e-08\\
0.3529353	2.857691e-06\\
0.3530353	5.771321e-06\\
0.3531353	4.943163e-06\\
0.3532353	3.288471e-06\\
0.3533353	9.936377e-06\\
0.3534353	5.069263e-06\\
0.3535354	1.910885e-06\\
0.3536354	1.224545e-06\\
0.3537354	1.778403e-05\\
0.3538354	1.285393e-05\\
0.3539354	6.135752e-06\\
0.3540354	1.447031e-05\\
0.3541354	1.097755e-05\\
0.3542354	1.340482e-05\\
0.3543354	1.592139e-05\\
0.3544354	6.784259e-06\\
0.3545355	3.67507e-06\\
0.3546355	8.2982e-06\\
0.3547355	8.288564e-06\\
0.3548355	6.359908e-06\\
0.3549355	6.531601e-06\\
0.3550355	1.258746e-05\\
0.3551355	1.035155e-05\\
0.3552355	5.635059e-06\\
0.3553355	5.411421e-06\\
0.3554355	6.930542e-07\\
0.3555356	1.2711e-06\\
0.3556356	6.402248e-06\\
0.3557356	3.708227e-06\\
0.3558356	2.233907e-07\\
0.3559356	4.12563e-06\\
0.3560356	5.645516e-06\\
0.3561356	1.153405e-06\\
0.3562356	6.563576e-07\\
0.3563356	1.404923e-06\\
0.3564356	3.70179e-06\\
0.3565357	5.761652e-06\\
0.3566357	5.199811e-06\\
0.3567357	7.316024e-06\\
0.3568357	7.267144e-06\\
0.3569357	4.512235e-06\\
0.3570357	1.152842e-06\\
0.3571357	1.892495e-07\\
0.3572357	6.593825e-06\\
0.3573357	5.314716e-06\\
0.3574357	8.321338e-06\\
0.3575358	6.316066e-06\\
0.3576358	5.091265e-06\\
0.3577358	8.375772e-07\\
0.3578358	1.300137e-05\\
0.3579358	1.807973e-05\\
0.3580358	1.46039e-05\\
0.3581358	2.07295e-05\\
0.3582358	3.350984e-05\\
0.3583358	2.717782e-05\\
0.3584358	5.492461e-06\\
0.3585359	2.488971e-06\\
0.3586359	3.774781e-06\\
0.3587359	2.91188e-06\\
0.3588359	7.918208e-06\\
0.3589359	7.524091e-06\\
0.3590359	4.375943e-06\\
0.3591359	4.359389e-06\\
0.3592359	2.699412e-06\\
0.3593359	3.854755e-06\\
0.3594359	8.767061e-07\\
0.359536	6.22318e-06\\
0.359636	1.050882e-07\\
0.359736	7.428666e-06\\
0.359836	3.729092e-06\\
0.359936	1.31636e-06\\
0.360036	5.642119e-06\\
0.360136	7.178168e-07\\
0.360236	4.398949e-06\\
0.360336	7.592734e-06\\
0.360436	8.702426e-06\\
0.3605361	6.975274e-06\\
0.3606361	1.004405e-05\\
0.3607361	1.155541e-05\\
0.3608361	2.897589e-05\\
0.3609361	1.046063e-05\\
0.3610361	9.76199e-06\\
0.3611361	7.502536e-06\\
0.3612361	8.792865e-06\\
0.3613361	4.514764e-07\\
0.3614361	1.517547e-06\\
0.3615362	5.054286e-06\\
0.3616362	3.974258e-06\\
0.3617362	3.239679e-06\\
0.3618362	1.50975e-06\\
0.3619362	2.806863e-06\\
0.3620362	6.062448e-06\\
0.3621362	4.515776e-06\\
0.3622362	3.992236e-06\\
0.3623362	1.924356e-06\\
0.3624362	1.374178e-05\\
0.3625363	1.03866e-05\\
0.3626363	1.289651e-05\\
0.3627363	2.072062e-06\\
0.3628363	3.283209e-06\\
0.3629363	3.365236e-06\\
0.3630363	1.425781e-06\\
0.3631363	1.484113e-06\\
0.3632363	2.472332e-06\\
0.3633363	4.474568e-07\\
0.3634363	9.697893e-07\\
0.3635364	1.196806e-06\\
0.3636364	3.532893e-06\\
0.3637364	7.850757e-06\\
0.3638364	7.207896e-07\\
0.3639364	2.737147e-08\\
0.3640364	1.391385e-06\\
0.3641364	8.295449e-06\\
0.3642364	1.430028e-05\\
0.3643364	1.336239e-05\\
0.3644364	3.032211e-06\\
0.3645365	2.064244e-06\\
0.3646365	1.412246e-05\\
0.3647365	2.302787e-05\\
0.3648365	1.4954e-05\\
0.3649365	8.086588e-06\\
0.3650365	5.378667e-06\\
0.3651365	3.501446e-06\\
0.3652365	5.108396e-06\\
0.3653365	4.315942e-06\\
0.3654365	1.166825e-06\\
0.3655366	3.150866e-06\\
0.3656366	3.017646e-06\\
0.3657366	2.341452e-06\\
0.3658366	2.52194e-05\\
0.3659366	1.517183e-05\\
0.3660366	1.680341e-05\\
0.3661366	7.504146e-06\\
0.3662366	1.461607e-05\\
0.3663366	1.720599e-05\\
0.3664366	1.349952e-05\\
0.3665367	4.425692e-06\\
0.3666367	2.859136e-06\\
0.3667367	1.529326e-07\\
0.3668367	1.068017e-06\\
0.3669367	2.938839e-07\\
0.3670367	2.596393e-06\\
0.3671367	2.13339e-06\\
0.3672367	5.306084e-07\\
0.3673367	3.192186e-06\\
0.3674367	1.092154e-06\\
0.3675368	5.709498e-06\\
0.3676368	1.052167e-05\\
0.3677368	6.233912e-06\\
0.3678368	3.458321e-06\\
0.3679368	3.704228e-06\\
0.3680368	1.972457e-05\\
0.3681368	3.479016e-05\\
0.3682368	1.397851e-05\\
0.3683368	1.352394e-05\\
0.3684368	1.601289e-05\\
0.3685369	4.245213e-06\\
0.3686369	1.359786e-05\\
0.3687369	1.751288e-05\\
0.3688369	1.585933e-05\\
0.3689369	5.291549e-06\\
0.3690369	3.38636e-06\\
0.3691369	5.945864e-06\\
0.3692369	1.243968e-06\\
0.3693369	7.637345e-07\\
0.3694369	4.132435e-06\\
0.369537	4.210006e-06\\
0.369637	2.141464e-06\\
0.369737	8.050282e-06\\
0.369837	4.174741e-06\\
0.369937	2.405132e-06\\
0.370037	1.468356e-05\\
0.370137	7.9016e-06\\
0.370237	5.41885e-06\\
0.370337	5.953235e-06\\
0.370437	4.060861e-06\\
0.3705371	3.307262e-06\\
0.3706371	4.744996e-06\\
0.3707371	3.856933e-07\\
0.3708371	2.165416e-06\\
0.3709371	1.641465e-05\\
0.3710371	1.339357e-05\\
0.3711371	1.011826e-05\\
0.3712371	5.181644e-06\\
0.3713371	1.189007e-05\\
0.3714371	2.007958e-05\\
0.3715372	2.097293e-05\\
0.3716372	9.871643e-06\\
0.3717372	1.507216e-05\\
0.3718372	2.154855e-05\\
0.3719372	4.632793e-06\\
0.3720372	5.260435e-06\\
0.3721372	1.118474e-05\\
0.3722372	5.770416e-07\\
0.3723372	2.101307e-06\\
0.3724372	5.750951e-06\\
0.3725373	2.637363e-05\\
0.3726373	1.948868e-05\\
0.3727373	1.005105e-05\\
0.3728373	3.01779e-06\\
0.3729373	1.185029e-05\\
0.3730373	1.335478e-05\\
0.3731373	1.579278e-05\\
0.3732373	1.414989e-05\\
0.3733373	1.557777e-05\\
0.3734373	1.111694e-05\\
0.3735374	1.154023e-05\\
0.3736374	5.325839e-06\\
0.3737374	1.838202e-06\\
0.3738374	1.417434e-06\\
0.3739374	2.751937e-07\\
0.3740374	1.564859e-05\\
0.3741374	1.379988e-05\\
0.3742374	3.731015e-06\\
0.3743374	3.358619e-06\\
0.3744374	4.722925e-06\\
0.3745375	1.64291e-06\\
0.3746375	9.670807e-06\\
0.3747375	1.906522e-05\\
0.3748375	2.245585e-05\\
0.3749375	2.200184e-05\\
0.3750375	1.837291e-05\\
0.3751375	6.399793e-06\\
0.3752375	1.511178e-05\\
0.3753375	1.393129e-05\\
0.3754375	7.496206e-06\\
0.3755376	1.14299e-05\\
0.3756376	1.905388e-05\\
0.3757376	5.603168e-06\\
0.3758376	1.948203e-06\\
0.3759376	8.740539e-06\\
0.3760376	1.044279e-05\\
0.3761376	1.977486e-05\\
0.3762376	8.237066e-06\\
0.3763376	1.481043e-05\\
0.3764376	1.21183e-05\\
0.3765377	1.015893e-05\\
0.3766377	1.914594e-05\\
0.3767377	1.121348e-05\\
0.3768377	7.82534e-06\\
0.3769377	1.401772e-05\\
0.3770377	1.526034e-05\\
0.3771377	7.556411e-06\\
0.3772377	2.406602e-06\\
0.3773377	1.853071e-06\\
0.3774377	1.182498e-05\\
0.3775378	1.588295e-05\\
0.3776378	5.895196e-06\\
0.3777378	5.019683e-06\\
0.3778378	1.602938e-06\\
0.3779378	3.648535e-06\\
0.3780378	3.972845e-06\\
0.3781378	8.575784e-06\\
0.3782378	4.932599e-06\\
0.3783378	2.194003e-05\\
0.3784378	2.432484e-05\\
0.3785379	1.760935e-05\\
0.3786379	2.743317e-05\\
0.3787379	3.06197e-05\\
0.3788379	2.776468e-05\\
0.3789379	8.947282e-06\\
0.3790379	1.974488e-05\\
0.3791379	1.774024e-05\\
0.3792379	1.901912e-05\\
0.3793379	1.024834e-05\\
0.3794379	1.35809e-05\\
0.379538	1.311369e-05\\
0.379638	1.016431e-05\\
0.379738	2.70923e-06\\
0.379838	3.254723e-06\\
0.379938	9.982678e-06\\
0.380038	2.48724e-06\\
0.380138	1.1304e-05\\
0.380238	1.106948e-05\\
0.380338	5.758106e-06\\
0.380438	2.018072e-06\\
0.3805381	6.088979e-06\\
0.3806381	1.040869e-05\\
0.3807381	3.514543e-05\\
0.3808381	2.687312e-05\\
0.3809381	2.059008e-05\\
0.3810381	2.398144e-05\\
0.3811381	1.133465e-05\\
0.3812381	1.994681e-05\\
0.3813381	7.38635e-06\\
0.3814381	1.20716e-06\\
0.3815382	5.090481e-06\\
0.3816382	5.0081e-07\\
0.3817382	8.139504e-06\\
0.3818382	2.456164e-05\\
0.3819382	1.865418e-05\\
0.3820382	1.782767e-05\\
0.3821382	1.674867e-05\\
0.3822382	2.004949e-05\\
0.3823382	2.285026e-05\\
0.3824382	2.400762e-05\\
0.3825383	5.045205e-06\\
0.3826383	5.064586e-06\\
0.3827383	1.255833e-05\\
0.3828383	9.237218e-06\\
0.3829383	5.098617e-06\\
0.3830383	1.06395e-05\\
0.3831383	2.009682e-05\\
0.3832383	1.193817e-05\\
0.3833383	3.102735e-06\\
0.3834383	1.194597e-05\\
0.3835384	1.610413e-05\\
0.3836384	2.656376e-05\\
0.3837384	2.750846e-05\\
0.3838384	2.223611e-05\\
0.3839384	4.260587e-05\\
0.3840384	2.5816e-05\\
0.3841384	1.636669e-05\\
0.3842384	6.218724e-06\\
0.3843384	6.607299e-06\\
0.3844384	1.61298e-05\\
0.3845385	2.047635e-05\\
0.3846385	1.222534e-05\\
0.3847385	8.664142e-06\\
0.3848385	4.565199e-06\\
0.3849385	6.835623e-06\\
0.3850385	4.461899e-06\\
0.3851385	4.052834e-06\\
0.3852385	1.112456e-05\\
0.3853385	1.394867e-05\\
0.3854385	1.748107e-05\\
0.3855386	2.23301e-05\\
0.3856386	2.425184e-05\\
0.3857386	9.573725e-06\\
0.3858386	2.546403e-05\\
0.3859386	1.842598e-05\\
0.3860386	2.04014e-05\\
0.3861386	3.395045e-05\\
0.3862386	1.73217e-05\\
0.3863386	1.295973e-05\\
0.3864386	1.635203e-05\\
0.3865387	1.806516e-05\\
0.3866387	1.589279e-05\\
0.3867387	1.692318e-05\\
0.3868387	2.012286e-05\\
0.3869387	1.656656e-05\\
0.3870387	6.353234e-06\\
0.3871387	1.083777e-05\\
0.3872387	1.300178e-05\\
0.3873387	1.298382e-05\\
0.3874387	3.245156e-05\\
0.3875388	4.155999e-05\\
0.3876388	3.326238e-05\\
0.3877388	2.856397e-05\\
0.3878388	2.461159e-05\\
0.3879388	2.295844e-05\\
0.3880388	1.079064e-05\\
0.3881388	5.327506e-06\\
0.3882388	9.094202e-07\\
0.3883388	1.270762e-06\\
0.3884388	5.176975e-06\\
0.3885389	1.310261e-05\\
0.3886389	2.2121e-05\\
0.3887389	4.61564e-05\\
0.3888389	2.967204e-05\\
0.3889389	2.648641e-05\\
0.3890389	2.368163e-05\\
0.3891389	4.554897e-05\\
0.3892389	3.342395e-05\\
0.3893389	2.82053e-05\\
0.3894389	2.888815e-05\\
0.389539	2.99263e-05\\
0.389639	2.522146e-05\\
0.389739	1.364485e-05\\
0.389839	1.46935e-05\\
0.389939	7.14355e-06\\
0.390039	1.268873e-05\\
0.390139	5.125605e-06\\
0.390239	1.876562e-05\\
0.390339	1.20295e-05\\
0.390439	9.632021e-06\\
0.3905391	2.800457e-05\\
0.3906391	1.623085e-05\\
0.3907391	2.625478e-05\\
0.3908391	4.757157e-05\\
0.3909391	4.177455e-05\\
0.3910391	1.79051e-05\\
0.3911391	1.831168e-05\\
0.3912391	9.607161e-06\\
0.3913391	1.736503e-05\\
0.3914391	2.503219e-05\\
0.3915392	1.823781e-05\\
0.3916392	9.971385e-06\\
0.3917392	1.852352e-05\\
0.3918392	2.647955e-05\\
0.3919392	3.836495e-05\\
0.3920392	4.320162e-05\\
0.3921392	3.90736e-05\\
0.3922392	4.392545e-05\\
0.3923392	2.580695e-05\\
0.3924392	3.181323e-05\\
0.3925393	1.161491e-05\\
0.3926393	1.997493e-06\\
0.3927393	1.083796e-05\\
0.3928393	2.212301e-05\\
0.3929393	1.693559e-05\\
0.3930393	6.674722e-06\\
0.3931393	1.427603e-05\\
0.3932393	1.382518e-05\\
0.3933393	2.578854e-05\\
0.3934393	4.619637e-05\\
0.3935394	5.348185e-05\\
0.3936394	6.960047e-05\\
0.3937394	5.81893e-05\\
0.3938394	4.565119e-05\\
0.3939394	3.703492e-05\\
0.3940394	2.97544e-05\\
0.3941394	2.812792e-05\\
0.3942394	1.40618e-05\\
0.3943394	1.1889e-05\\
0.3944394	1.830585e-05\\
0.3945395	3.199532e-05\\
0.3946395	3.42452e-05\\
0.3947395	2.403513e-05\\
0.3948395	3.143241e-05\\
0.3949395	3.342721e-05\\
0.3950395	1.59672e-05\\
0.3951395	6.084851e-06\\
0.3952395	3.266574e-05\\
0.3953395	3.417694e-05\\
0.3954395	3.834814e-05\\
0.3955396	1.068026e-05\\
0.3956396	1.450297e-05\\
0.3957396	9.095168e-06\\
0.3958396	4.159859e-06\\
0.3959396	7.909361e-06\\
0.3960396	2.176395e-05\\
0.3961396	3.776765e-05\\
0.3962396	4.958427e-05\\
0.3963396	6.178421e-05\\
0.3964396	6.142913e-05\\
0.3965397	5.048266e-05\\
0.3966397	5.897825e-05\\
0.3967397	3.038512e-05\\
0.3968397	2.296173e-05\\
0.3969397	1.360562e-05\\
0.3970397	1.689241e-05\\
0.3971397	2.640192e-05\\
0.3972397	3.228302e-05\\
0.3973397	3.819321e-05\\
0.3974397	3.940809e-05\\
0.3975398	3.984588e-05\\
0.3976398	1.917402e-05\\
0.3977398	3.071838e-05\\
0.3978398	3.308229e-05\\
0.3979398	2.391534e-05\\
0.3980398	3.658944e-05\\
0.3981398	3.430638e-05\\
0.3982398	2.468104e-05\\
0.3983398	2.942267e-05\\
0.3984398	2.827941e-05\\
0.3985399	4.775442e-05\\
0.3986399	4.042897e-05\\
0.3987399	2.90193e-05\\
0.3988399	4.658316e-05\\
0.3989399	3.293135e-05\\
0.3990399	2.040647e-05\\
0.3991399	2.326088e-05\\
0.3992399	6.407716e-06\\
0.3993399	1.421637e-05\\
0.3994399	4.293336e-05\\
0.39954	4.263155e-05\\
0.39964	4.82489e-05\\
0.39974	5.088337e-05\\
0.39984	3.429025e-05\\
0.39994	3.87348e-05\\
0.40004	3.131967e-05\\
};
\addplot [color=mycolor1,solid,forget plot]
  table[row sep=crcr]{%
0.40004	3.131967e-05\\
0.40014	2.028094e-05\\
0.40024	4.105511e-05\\
0.40034	6.541444e-05\\
0.40044	4.077597e-05\\
0.4005401	2.471025e-05\\
0.4006401	2.739853e-05\\
0.4007401	4.002566e-05\\
0.4008401	3.19007e-05\\
0.4009401	1.251966e-05\\
0.4010401	1.38584e-05\\
0.4011401	3.131108e-05\\
0.4012401	4.851751e-05\\
0.4013401	7.767788e-05\\
0.4014401	4.700643e-05\\
0.4015402	3.832384e-05\\
0.4016402	8.038985e-05\\
0.4017402	5.858949e-05\\
0.4018402	5.920385e-05\\
0.4019402	6.330061e-05\\
0.4020402	6.725136e-05\\
0.4021402	3.458492e-05\\
0.4022402	4.304391e-05\\
0.4023402	4.670745e-05\\
0.4024402	3.187074e-05\\
0.4025403	1.410997e-05\\
0.4026403	3.403731e-05\\
0.4027403	4.859369e-05\\
0.4028403	2.57822e-05\\
0.4029403	1.045571e-05\\
0.4030403	2.733252e-05\\
0.4031403	1.663096e-05\\
0.4032403	2.559506e-05\\
0.4033403	2.424068e-05\\
0.4034403	4.113302e-05\\
0.4035404	5.475949e-05\\
0.4036404	5.882046e-05\\
0.4037404	5.711921e-05\\
0.4038404	5.105233e-05\\
0.4039404	8.46406e-05\\
0.4040404	9.677193e-05\\
0.4041404	7.606394e-05\\
0.4042404	4.248153e-05\\
0.4043404	8.705622e-05\\
0.4044404	4.794799e-05\\
0.4045405	4.585662e-05\\
0.4046405	5.929561e-05\\
0.4047405	6.128144e-05\\
0.4048405	6.305063e-05\\
0.4049405	6.775536e-05\\
0.4050405	1.669385e-05\\
0.4051405	1.356519e-05\\
0.4052405	3.48901e-05\\
0.4053405	5.519978e-05\\
0.4054405	4.732745e-05\\
0.4055406	3.018507e-05\\
0.4056406	5.360759e-05\\
0.4057406	4.752957e-05\\
0.4058406	3.867393e-05\\
0.4059406	3.439483e-05\\
0.4060406	6.64244e-05\\
0.4061406	6.163865e-05\\
0.4062406	6.636252e-05\\
0.4063406	5.810888e-05\\
0.4064406	3.801697e-05\\
0.4065407	4.51024e-05\\
0.4066407	8.656727e-05\\
0.4067407	6.805117e-05\\
0.4068407	6.365775e-05\\
0.4069407	6.097403e-05\\
0.4070407	4.745311e-05\\
0.4071407	6.006939e-05\\
0.4072407	3.716086e-05\\
0.4073407	1.225095e-05\\
0.4074407	1.028097e-05\\
0.4075408	4.529431e-05\\
0.4076408	6.169749e-05\\
0.4077408	7.665785e-05\\
0.4078408	8.567033e-05\\
0.4079408	8.642397e-05\\
0.4080408	0.0001020176\\
0.4081408	9.130902e-05\\
0.4082408	8.141145e-05\\
0.4083408	7.444698e-05\\
0.4084408	6.752938e-05\\
0.4085409	5.292067e-05\\
0.4086409	3.90453e-05\\
0.4087409	3.309972e-05\\
0.4088409	4.314512e-05\\
0.4089409	6.872543e-05\\
0.4090409	3.27797e-05\\
0.4091409	5.952393e-05\\
0.4092409	9.378124e-05\\
0.4093409	8.478761e-05\\
0.4094409	6.85763e-05\\
0.409541	8.293683e-05\\
0.409641	6.08038e-05\\
0.409741	6.20212e-05\\
0.409841	0.0001008692\\
0.409941	9.308643e-05\\
0.410041	3.633727e-05\\
0.410141	2.317689e-05\\
0.410241	4.184159e-05\\
0.410341	3.749634e-05\\
0.410441	4.111339e-05\\
0.4105411	7.196789e-05\\
0.4106411	7.477749e-05\\
0.4107411	4.736652e-05\\
0.4108411	5.100037e-05\\
0.4109411	6.396115e-05\\
0.4110411	6.838777e-05\\
0.4111411	9.686667e-05\\
0.4112411	0.0001105041\\
0.4113411	8.605228e-05\\
0.4114411	0.0001001049\\
0.4115412	8.655205e-05\\
0.4116412	8.176533e-05\\
0.4117412	5.597321e-05\\
0.4118412	5.263962e-05\\
0.4119412	4.341789e-05\\
0.4120412	0.000127119\\
0.4121412	0.0001572886\\
0.4122412	9.034743e-05\\
0.4123412	8.987995e-05\\
0.4124412	9.438336e-05\\
0.4125413	0.000130819\\
0.4126413	7.780268e-05\\
0.4127413	6.727317e-05\\
0.4128413	7.577315e-05\\
0.4129413	5.561761e-05\\
0.4130413	5.243261e-05\\
0.4131413	7.052224e-05\\
0.4132413	7.353629e-05\\
0.4133413	4.552421e-05\\
0.4134413	5.657172e-05\\
0.4135414	6.387636e-05\\
0.4136414	7.613158e-05\\
0.4137414	7.807244e-05\\
0.4138414	6.635914e-05\\
0.4139414	6.514304e-05\\
0.4140414	0.0001056208\\
0.4141414	6.161237e-05\\
0.4142414	8.212646e-05\\
0.4143414	0.0001030951\\
0.4144414	0.0001079414\\
0.4145415	0.0001170389\\
0.4146415	0.0001099119\\
0.4147415	0.0001456714\\
0.4148415	0.0001349051\\
0.4149415	0.000104813\\
0.4150415	9.09941e-05\\
0.4151415	8.568923e-05\\
0.4152415	7.606441e-05\\
0.4153415	7.161205e-05\\
0.4154415	7.948421e-05\\
0.4155416	9.1566e-05\\
0.4156416	4.808736e-05\\
0.4157416	6.982834e-05\\
0.4158416	6.586169e-05\\
0.4159416	7.81606e-05\\
0.4160416	0.0001127454\\
0.4161416	0.0001042485\\
0.4162416	8.721969e-05\\
0.4163416	0.0001111197\\
0.4164416	0.0001819319\\
0.4165417	0.0001575713\\
0.4166417	0.0001299887\\
0.4167417	0.0001321209\\
0.4168417	0.0001031439\\
0.4169417	7.404489e-05\\
0.4170417	6.097214e-05\\
0.4171417	8.136796e-05\\
0.4172417	7.48921e-05\\
0.4173417	6.738947e-05\\
0.4174417	7.054593e-05\\
0.4175418	8.441393e-05\\
0.4176418	7.557922e-05\\
0.4177418	0.0001162125\\
0.4178418	0.000110381\\
0.4179418	7.57243e-05\\
0.4180418	0.0001063965\\
0.4181418	0.0001418003\\
0.4182418	0.0001162269\\
0.4183418	0.0001224724\\
0.4184418	0.000116293\\
0.4185419	0.0001639271\\
0.4186419	0.0001591714\\
0.4187419	0.0001122492\\
0.4188419	0.0001364733\\
0.4189419	0.0001334527\\
0.4190419	9.83388e-05\\
0.4191419	9.933334e-05\\
0.4192419	9.655659e-05\\
0.4193419	0.0001090223\\
0.4194419	0.0001010533\\
0.419542	8.217646e-05\\
0.419642	0.0001124769\\
0.419742	0.0001323988\\
0.419842	0.0001733654\\
0.419942	0.0001512082\\
0.420042	0.0001306081\\
0.420142	9.360846e-05\\
0.420242	0.0001131844\\
0.420342	0.0001461146\\
0.420442	0.0001550217\\
0.4205421	0.0001608097\\
0.4206421	0.0001136806\\
0.4207421	0.0001016064\\
0.4208421	0.0001121864\\
0.4209421	6.534996e-05\\
0.4210421	7.287681e-05\\
0.4211421	8.493885e-05\\
0.4212421	9.414303e-05\\
0.4213421	0.0001122449\\
0.4214421	0.0001301322\\
0.4215422	0.0001693256\\
0.4216422	0.0001480065\\
0.4217422	0.0001590606\\
0.4218422	0.0001770962\\
0.4219422	0.0001853826\\
0.4220422	0.0002027289\\
0.4221422	0.0001257514\\
0.4222422	0.0001085975\\
0.4223422	0.0001271135\\
0.4224422	0.0001465654\\
0.4225423	0.0001455057\\
0.4226423	0.0001521568\\
0.4227423	0.0001396901\\
0.4228423	0.0001562638\\
0.4229423	0.000187086\\
0.4230423	0.0001869857\\
0.4231423	0.0001169029\\
0.4232423	0.0001481795\\
0.4233423	0.0001524867\\
0.4234423	0.0001341171\\
0.4235424	0.0001271116\\
0.4236424	0.000105025\\
0.4237424	0.0001387782\\
0.4238424	0.000137201\\
0.4239424	0.0001348238\\
0.4240424	0.000121418\\
0.4241424	0.0001431222\\
0.4242424	0.0001750371\\
0.4243424	0.0001879828\\
0.4244424	0.0002035286\\
0.4245425	0.0001607081\\
0.4246425	0.0001520758\\
0.4247425	0.0001445843\\
0.4248425	0.0001241596\\
0.4249425	0.0001600565\\
0.4250425	0.0001452469\\
0.4251425	0.0001558197\\
0.4252425	0.0002183655\\
0.4253425	0.0001634633\\
0.4254425	0.0001694825\\
0.4255426	0.0001460209\\
0.4256426	0.0001717704\\
0.4257426	0.0001836489\\
0.4258426	0.000202515\\
0.4259426	0.0002174973\\
0.4260426	0.0001949627\\
0.4261426	0.000174986\\
0.4262426	0.000115011\\
0.4263426	0.0001153076\\
0.4264426	0.0002308622\\
0.4265427	0.0001996975\\
0.4266427	0.0001426886\\
0.4267427	0.0001421759\\
0.4268427	0.0001605817\\
0.4269427	0.0001599778\\
0.4270427	0.0002052454\\
0.4271427	0.0002061709\\
0.4272427	0.0002231178\\
0.4273427	0.000219343\\
0.4274427	0.0002866179\\
0.4275428	0.0002778465\\
0.4276428	0.0002247264\\
0.4277428	0.0002006\\
0.4278428	0.0001819519\\
0.4279428	0.0001813429\\
0.4280428	0.0001373801\\
0.4281428	0.0001307764\\
0.4282428	0.0001371318\\
0.4283428	0.0001413833\\
0.4284428	0.0001929383\\
0.4285429	0.0002291109\\
0.4286429	0.0001942777\\
0.4287429	0.0002464958\\
0.4288429	0.000173451\\
0.4289429	0.0002181441\\
0.4290429	0.0002071617\\
0.4291429	0.0002431859\\
0.4292429	0.0002309239\\
0.4293429	0.0002307707\\
0.4294429	0.0002345976\\
0.429543	0.0002683273\\
0.429643	0.0002026161\\
0.429743	0.0001992324\\
0.429843	0.0001818055\\
0.429943	0.0001789123\\
0.430043	0.000156547\\
0.430143	0.0001920751\\
0.430243	0.0002612251\\
0.430343	0.0002528677\\
0.430443	0.0002646316\\
0.4305431	0.000253647\\
0.4306431	0.0002331243\\
0.4307431	0.0002997227\\
0.4308431	0.0003122449\\
0.4309431	0.0002578775\\
0.4310431	0.0002135602\\
0.4311431	0.0002241311\\
0.4312431	0.0002129952\\
0.4313431	0.0001448875\\
0.4314431	0.0001590912\\
0.4315432	0.0001948305\\
0.4316432	0.0001778752\\
0.4317432	0.0002434969\\
0.4318432	0.0002607518\\
0.4319432	0.0002810819\\
0.4320432	0.0002675996\\
0.4321432	0.0002480022\\
0.4322432	0.0002928123\\
0.4323432	0.0002411509\\
0.4324432	0.0002337602\\
0.4325433	0.0002733543\\
0.4326433	0.0002914701\\
0.4327433	0.0002908906\\
0.4328433	0.000232506\\
0.4329433	0.0002757916\\
0.4330433	0.0003092529\\
0.4331433	0.000288732\\
0.4332433	0.0002868845\\
0.4333433	0.0002501643\\
0.4334433	0.0002555161\\
0.4335434	0.0002160778\\
0.4336434	0.0002047698\\
0.4337434	0.0002490298\\
0.4338434	0.0002448451\\
0.4339434	0.0002523559\\
0.4340434	0.0002887985\\
0.4341434	0.0003188139\\
0.4342434	0.0003543973\\
0.4343434	0.0002662361\\
0.4344434	0.0003126929\\
0.4345435	0.000320316\\
0.4346435	0.000309461\\
0.4347435	0.0002277933\\
0.4348435	0.0002942063\\
0.4349435	0.0002559989\\
0.4350435	0.0002773601\\
0.4351435	0.0002339129\\
0.4352435	0.0002971298\\
0.4353435	0.0003385144\\
0.4354435	0.0003578592\\
0.4355436	0.0003645208\\
0.4356436	0.0003568999\\
0.4357436	0.0003147016\\
0.4358436	0.0003252526\\
0.4359436	0.0003168429\\
0.4360436	0.0002739342\\
0.4361436	0.0002452517\\
0.4362436	0.0002628727\\
0.4363436	0.0002984438\\
0.4364436	0.0003394116\\
0.4365437	0.000286753\\
0.4366437	0.0002671459\\
0.4367437	0.0002626827\\
0.4368437	0.0002295019\\
0.4369437	0.0002746783\\
0.4370437	0.0003403751\\
0.4371437	0.0003658576\\
0.4372437	0.0003772206\\
0.4373437	0.0003835039\\
0.4374437	0.0004584115\\
0.4375438	0.0004687518\\
0.4376438	0.0003958275\\
0.4377438	0.0003778426\\
0.4378438	0.0003377219\\
0.4379438	0.0003452941\\
0.4380438	0.000332046\\
0.4381438	0.0003148791\\
0.4382438	0.0003995332\\
0.4383438	0.0004126719\\
0.4384438	0.0003554655\\
0.4385439	0.0003488391\\
0.4386439	0.0003050942\\
0.4387439	0.0003123767\\
0.4388439	0.0003239561\\
0.4389439	0.0002862217\\
0.4390439	0.0003114564\\
0.4391439	0.0003396829\\
0.4392439	0.0003956918\\
0.4393439	0.0004120828\\
0.4394439	0.0003763826\\
0.439544	0.0003794131\\
0.439644	0.0003197307\\
0.439744	0.0003590246\\
0.439844	0.000385561\\
0.439944	0.0004045989\\
0.440044	0.0004436284\\
0.440144	0.0003723254\\
0.440244	0.0004678223\\
0.440344	0.0004692408\\
0.440444	0.0004386266\\
0.4405441	0.0004264706\\
0.4406441	0.0004244792\\
0.4407441	0.0004662796\\
0.4408441	0.0003930091\\
0.4409441	0.0003461962\\
0.4410441	0.0003905332\\
0.4411441	0.0003277753\\
0.4412441	0.0003761963\\
0.4413441	0.000367432\\
0.4414441	0.0003909189\\
0.4415442	0.0004017145\\
0.4416442	0.0003673552\\
0.4417442	0.0004482216\\
0.4418442	0.0005905026\\
0.4419442	0.0006169264\\
0.4420442	0.0005836475\\
0.4421442	0.0004783782\\
0.4422442	0.0004517613\\
0.4423442	0.0004300426\\
0.4424442	0.0003891855\\
0.4425443	0.0004038751\\
0.4426443	0.0003661319\\
0.4427443	0.0004403943\\
0.4428443	0.00047024\\
0.4429443	0.0003750957\\
0.4430443	0.0003826572\\
0.4431443	0.0004373761\\
0.4432443	0.0004962367\\
0.4433443	0.00048704\\
0.4434443	0.0004973388\\
0.4435444	0.0006106633\\
0.4436444	0.0005604564\\
0.4437444	0.0005152899\\
0.4438444	0.0004410146\\
0.4439444	0.0004783178\\
0.4440444	0.000457152\\
0.4441444	0.0004446465\\
0.4442444	0.0004662558\\
0.4443444	0.0004624159\\
0.4444444	0.0003761841\\
0.4445445	0.0004346662\\
0.4446445	0.000452662\\
0.4447445	0.0005444553\\
0.4448445	0.0006346161\\
0.4449445	0.0005933329\\
0.4450445	0.0006184321\\
0.4451445	0.0006581002\\
0.4452445	0.0006654218\\
0.4453445	0.0005898409\\
0.4454445	0.0004946493\\
0.4455446	0.0004518563\\
0.4456446	0.0005129356\\
0.4457446	0.000568644\\
0.4458446	0.0005880607\\
0.4459446	0.0004839092\\
0.4460446	0.000626661\\
0.4461446	0.0005641228\\
0.4462446	0.0005748751\\
0.4463446	0.0005415761\\
0.4464446	0.0005570739\\
0.4465447	0.0005167148\\
0.4466447	0.0004711142\\
0.4467447	0.0004172915\\
0.4468447	0.0004630028\\
0.4469447	0.0005461979\\
0.4470447	0.0005150078\\
0.4471447	0.0005207002\\
0.4472447	0.0005255276\\
0.4473447	0.0006337718\\
0.4474447	0.0005914419\\
0.4475448	0.0005907083\\
0.4476448	0.0007117677\\
0.4477448	0.0008072882\\
0.4478448	0.0008129181\\
0.4479448	0.0006922719\\
0.4480448	0.000632294\\
0.4481448	0.0006547323\\
0.4482448	0.0007167224\\
0.4483448	0.0006421963\\
0.4484448	0.0005367892\\
0.4485449	0.0005917691\\
0.4486449	0.0006268746\\
0.4487449	0.0005642764\\
0.4488449	0.0005503103\\
0.4489449	0.0006109882\\
0.4490449	0.0006354881\\
0.4491449	0.0007237502\\
0.4492449	0.0007091287\\
0.4493449	0.0006717591\\
0.4494449	0.0006715045\\
0.449545	0.0005964735\\
0.449645	0.0006464543\\
0.449745	0.0006217391\\
0.449845	0.0005936714\\
0.449945	0.0006945775\\
0.450045	0.0007764023\\
0.450145	0.0008040657\\
0.450245	0.0007225745\\
0.450345	0.0006927711\\
0.450445	0.0007370154\\
0.4505451	0.0007101099\\
0.4506451	0.0007143342\\
0.4507451	0.0006002464\\
0.4508451	0.0006785076\\
0.4509451	0.000695396\\
0.4510451	0.0007029022\\
0.4511451	0.0007001306\\
0.4512451	0.0006813439\\
0.4513451	0.0007214285\\
0.4514451	0.0007497251\\
0.4515452	0.0007352803\\
0.4516452	0.0007560172\\
0.4517452	0.000831881\\
0.4518452	0.0008502925\\
0.4519452	0.000733151\\
0.4520452	0.0007318036\\
0.4521452	0.0008743856\\
0.4522452	0.0008204526\\
0.4523452	0.0008584035\\
0.4524452	0.0007704377\\
0.4525453	0.0008182677\\
0.4526453	0.0008784933\\
0.4527453	0.0008552697\\
0.4528453	0.000804406\\
0.4529453	0.000797859\\
0.4530453	0.0008057772\\
0.4531453	0.0007772179\\
0.4532453	0.000774957\\
0.4533453	0.000898939\\
0.4534453	0.0008160864\\
0.4535454	0.0007552686\\
0.4536454	0.0007681903\\
0.4537454	0.0007902487\\
0.4538454	0.0007796803\\
0.4539454	0.0006893397\\
0.4540454	0.0007802532\\
0.4541454	0.0008589218\\
0.4542454	0.0008816979\\
0.4543454	0.0009145115\\
0.4544454	0.0009857992\\
0.4545455	0.001067395\\
0.4546455	0.000929715\\
0.4547455	0.0008402581\\
0.4548455	0.000859173\\
0.4549455	0.0008907547\\
0.4550455	0.0009180331\\
0.4551455	0.0009453097\\
0.4552455	0.0008076035\\
0.4553455	0.0008032152\\
0.4554455	0.0009201626\\
0.4555456	0.001012241\\
0.4556456	0.001036373\\
0.4557456	0.001044863\\
0.4558456	0.001057086\\
0.4559456	0.0009963704\\
0.4560456	0.0009455024\\
0.4561456	0.001018381\\
0.4562456	0.001036\\
0.4563456	0.0009363568\\
0.4564456	0.001013312\\
0.4565457	0.00108555\\
0.4566457	0.001046963\\
0.4567457	0.0009157915\\
0.4568457	0.0008605273\\
0.4569457	0.0009135827\\
0.4570457	0.0009431873\\
0.4571457	0.0009428514\\
0.4572457	0.0009807576\\
0.4573457	0.0008929126\\
0.4574457	0.0009420069\\
0.4575458	0.0009113309\\
0.4576458	0.001018963\\
0.4577458	0.001103489\\
0.4578458	0.001138698\\
0.4579458	0.001144901\\
0.4580458	0.001058214\\
0.4581458	0.001152456\\
0.4582458	0.001214004\\
0.4583458	0.001309887\\
0.4584458	0.001237799\\
0.4585459	0.001142444\\
0.4586459	0.001224605\\
0.4587459	0.001019092\\
0.4588459	0.001023559\\
0.4589459	0.001096297\\
0.4590459	0.001008728\\
0.4591459	0.0009565091\\
0.4592459	0.001016484\\
0.4593459	0.001036063\\
0.4594459	0.00121625\\
0.459546	0.001251675\\
0.459646	0.001159686\\
0.459746	0.001221762\\
0.459846	0.001319924\\
0.459946	0.001321511\\
0.460046	0.001321867\\
0.460146	0.001325401\\
0.460246	0.001220789\\
0.460346	0.001244795\\
0.460446	0.001161452\\
0.4605461	0.001110332\\
0.4606461	0.001095727\\
0.4607461	0.001031134\\
0.4608461	0.001121119\\
0.4609461	0.001265289\\
0.4610461	0.001311196\\
0.4611461	0.001264952\\
0.4612461	0.001125335\\
0.4613461	0.00123239\\
0.4614461	0.001391695\\
0.4615462	0.001482227\\
0.4616462	0.00138382\\
0.4617462	0.001403407\\
0.4618462	0.001383451\\
0.4619462	0.001280761\\
0.4620462	0.0011695\\
0.4621462	0.001230501\\
0.4622462	0.001390749\\
0.4623462	0.001433982\\
0.4624462	0.001330002\\
0.4625463	0.001360847\\
0.4626463	0.00144205\\
0.4627463	0.001342016\\
0.4628463	0.001268581\\
0.4629463	0.001351874\\
0.4630463	0.00147618\\
0.4631463	0.001464812\\
0.4632463	0.00152462\\
0.4633463	0.001429852\\
0.4634463	0.001483188\\
0.4635464	0.001579004\\
0.4636464	0.001457732\\
0.4637464	0.001510466\\
0.4638464	0.001500177\\
0.4639464	0.001348764\\
0.4640464	0.001409332\\
0.4641464	0.001480013\\
0.4642464	0.001326139\\
0.4643464	0.001401954\\
0.4644464	0.00151298\\
0.4645465	0.001625706\\
0.4646465	0.001468001\\
0.4647465	0.001378039\\
0.4648465	0.001569204\\
0.4649465	0.001573752\\
0.4650465	0.001599366\\
0.4651465	0.001642667\\
0.4652465	0.001676057\\
0.4653465	0.001510481\\
0.4654465	0.001705917\\
0.4655466	0.001619958\\
0.4656466	0.001404942\\
0.4657466	0.001501645\\
0.4658466	0.001599989\\
0.4659466	0.001645867\\
0.4660466	0.001606798\\
0.4661466	0.001809332\\
0.4662466	0.001825436\\
0.4663466	0.001794714\\
0.4664466	0.00169463\\
0.4665467	0.001635106\\
0.4666467	0.00166118\\
0.4667467	0.001612485\\
0.4668467	0.001689047\\
0.4669467	0.001865283\\
0.4670467	0.001887834\\
0.4671467	0.001815502\\
0.4672467	0.001818017\\
0.4673467	0.001671338\\
0.4674467	0.00165663\\
0.4675468	0.001675017\\
0.4676468	0.001705223\\
0.4677468	0.001669977\\
0.4678468	0.001663086\\
0.4679468	0.001769618\\
0.4680468	0.00176661\\
0.4681468	0.00188842\\
0.4682468	0.001954134\\
0.4683468	0.001976946\\
0.4684468	0.001905504\\
0.4685469	0.001895701\\
0.4686469	0.001883657\\
0.4687469	0.001997104\\
0.4688469	0.001914612\\
0.4689469	0.00195957\\
0.4690469	0.00189386\\
0.4691469	0.002021815\\
0.4692469	0.002028163\\
0.4693469	0.001930715\\
0.4694469	0.001879601\\
0.469547	0.001939787\\
0.469647	0.001927207\\
0.469747	0.001855183\\
0.469847	0.001948519\\
0.469947	0.002007359\\
0.470047	0.002069558\\
0.470147	0.002097246\\
0.470247	0.002038345\\
0.470347	0.001969212\\
0.470447	0.002074747\\
0.4705471	0.00195105\\
0.4706471	0.002126826\\
0.4707471	0.002085472\\
0.4708471	0.002072218\\
0.4709471	0.002117018\\
0.4710471	0.002302198\\
0.4711471	0.002374832\\
0.4712471	0.002307948\\
0.4713471	0.002356754\\
0.4714471	0.002327915\\
0.4715472	0.002324915\\
0.4716472	0.002302721\\
0.4717472	0.002071711\\
0.4718472	0.00216029\\
0.4719472	0.002032002\\
0.4720472	0.002150715\\
0.4721472	0.002138388\\
0.4722472	0.002097976\\
0.4723472	0.002313401\\
0.4724472	0.002229894\\
0.4725473	0.002275767\\
0.4726473	0.002306186\\
0.4727473	0.002349951\\
0.4728473	0.002369865\\
0.4729473	0.002452967\\
0.4730473	0.00238007\\
0.4731473	0.002418152\\
0.4732473	0.002458201\\
0.4733473	0.002409498\\
0.4734473	0.002228443\\
0.4735474	0.002250249\\
0.4736474	0.002327477\\
0.4737474	0.002557853\\
0.4738474	0.002597855\\
0.4739474	0.002710832\\
0.4740474	0.00255883\\
0.4741474	0.002675243\\
0.4742474	0.00287694\\
0.4743474	0.00267003\\
0.4744474	0.002595671\\
0.4745475	0.002349103\\
0.4746475	0.002362446\\
0.4747475	0.002506457\\
0.4748475	0.002544954\\
0.4749475	0.002579694\\
0.4750475	0.002690201\\
0.4751475	0.002548166\\
0.4752475	0.002504585\\
0.4753475	0.002650438\\
0.4754475	0.0026175\\
0.4755476	0.002775252\\
0.4756476	0.002654069\\
0.4757476	0.002699388\\
0.4758476	0.002851199\\
0.4759476	0.002870905\\
0.4760476	0.002939042\\
0.4761476	0.002913343\\
0.4762476	0.002928902\\
0.4763476	0.002891518\\
0.4764476	0.002988216\\
0.4765477	0.002750796\\
0.4766477	0.002676456\\
0.4767477	0.002419684\\
0.4768477	0.002495846\\
0.4769477	0.002712797\\
0.4770477	0.002884711\\
0.4771477	0.003151713\\
0.4772477	0.002991782\\
0.4773477	0.003213448\\
0.4774477	0.003300293\\
0.4775478	0.003112174\\
0.4776478	0.003001227\\
0.4777478	0.003082949\\
0.4778478	0.003106303\\
0.4779478	0.003057572\\
0.4780478	0.002916781\\
0.4781478	0.002680711\\
0.4782478	0.00300466\\
0.4783478	0.003228918\\
0.4784478	0.003034591\\
0.4785479	0.002975456\\
0.4786479	0.003273843\\
0.4787479	0.00362176\\
0.4788479	0.003393063\\
0.4789479	0.003321667\\
0.4790479	0.003351317\\
0.4791479	0.003388673\\
0.4792479	0.003306118\\
0.4793479	0.003450913\\
0.4794479	0.003327208\\
0.479548	0.002993746\\
0.479648	0.003155349\\
0.479748	0.003290257\\
0.479848	0.003165505\\
0.479948	0.003350047\\
0.480048	0.003320663\\
0.480148	0.003112665\\
0.480248	0.003054759\\
0.480348	0.00335564\\
0.480448	0.003668548\\
0.4805481	0.003748392\\
0.4806481	0.003493825\\
0.4807481	0.003454105\\
0.4808481	0.003589585\\
0.4809481	0.003527292\\
0.4810481	0.003302478\\
0.4811481	0.003352957\\
0.4812481	0.00370258\\
0.4813481	0.003846912\\
0.4814481	0.003827059\\
0.4815482	0.004035007\\
0.4816482	0.00422027\\
0.4817482	0.004018706\\
0.4818482	0.003783053\\
0.4819482	0.003912253\\
0.4820482	0.003952913\\
0.4821482	0.003675173\\
0.4822482	0.003748043\\
0.4823482	0.003657418\\
0.4824482	0.003580676\\
0.4825483	0.003401949\\
0.4826483	0.003610704\\
0.4827483	0.003600634\\
0.4828483	0.003793001\\
0.4829483	0.003929747\\
0.4830483	0.003969883\\
0.4831483	0.004008457\\
0.4832483	0.003948796\\
0.4833483	0.004047823\\
0.4834483	0.004032749\\
0.4835484	0.004133497\\
0.4836484	0.004277509\\
0.4837484	0.004189791\\
0.4838484	0.004221586\\
0.4839484	0.004383014\\
0.4840484	0.004082533\\
0.4841484	0.00407683\\
0.4842484	0.004258952\\
0.4843484	0.004322573\\
0.4844484	0.003988606\\
0.4845485	0.004200265\\
0.4846485	0.004295494\\
0.4847485	0.004082913\\
0.4848485	0.004207849\\
0.4849485	0.004362834\\
0.4850485	0.004248953\\
0.4851485	0.004375019\\
0.4852485	0.004319571\\
0.4853485	0.004373037\\
0.4854485	0.004580765\\
0.4855486	0.004733413\\
0.4856486	0.004954649\\
0.4857486	0.004727256\\
0.4858486	0.004948613\\
0.4859486	0.004704855\\
0.4860486	0.0046089\\
0.4861486	0.004569641\\
0.4862486	0.004605763\\
0.4863486	0.004437644\\
0.4864486	0.004341497\\
0.4865487	0.004519451\\
0.4866487	0.004439916\\
0.4867487	0.004398793\\
0.4868487	0.004591341\\
0.4869487	0.004896996\\
0.4870487	0.004877218\\
0.4871487	0.004955971\\
0.4872487	0.00533243\\
0.4873487	0.005105121\\
0.4874487	0.004817865\\
0.4875488	0.004750349\\
0.4876488	0.004934386\\
0.4877488	0.005021655\\
0.4878488	0.004828932\\
0.4879488	0.004774604\\
0.4880488	0.005035163\\
0.4881488	0.005195652\\
0.4882488	0.00519769\\
0.4883488	0.005222244\\
0.4884488	0.005150517\\
0.4885489	0.005170999\\
0.4886489	0.005099494\\
0.4887489	0.005413285\\
0.4888489	0.005566906\\
0.4889489	0.005444772\\
0.4890489	0.005258478\\
0.4891489	0.005534228\\
0.4892489	0.005622213\\
0.4893489	0.005536298\\
0.4894489	0.005722483\\
0.489549	0.00547655\\
0.489649	0.005067017\\
0.489749	0.005507256\\
0.489849	0.005716474\\
0.489949	0.005550532\\
0.490049	0.005109614\\
0.490149	0.005404712\\
0.490249	0.005711987\\
0.490349	0.005619499\\
0.490449	0.005529244\\
0.4905491	0.005400029\\
0.4906491	0.005608097\\
0.4907491	0.005758263\\
0.4908491	0.005899573\\
0.4909491	0.005763354\\
0.4910491	0.005922383\\
0.4911491	0.006081179\\
0.4912491	0.00589583\\
0.4913491	0.00596544\\
0.4914491	0.006490471\\
0.4915492	0.006166909\\
0.4916492	0.00591222\\
0.4917492	0.005846905\\
0.4918492	0.005827432\\
0.4919492	0.005959681\\
0.4920492	0.006129385\\
0.4921492	0.006493798\\
0.4922492	0.00653285\\
0.4923492	0.006778223\\
0.4924492	0.006659844\\
0.4925493	0.006641735\\
0.4926493	0.006192009\\
0.4927493	0.006020197\\
0.4928493	0.006253567\\
0.4929493	0.006429185\\
0.4930493	0.006354228\\
0.4931493	0.00637863\\
0.4932493	0.00634865\\
0.4933493	0.006277455\\
0.4934493	0.006429837\\
0.4935494	0.006551081\\
0.4936494	0.006736832\\
0.4937494	0.006712473\\
0.4938494	0.006724685\\
0.4939494	0.006608187\\
0.4940494	0.007099258\\
0.4941494	0.007234683\\
0.4942494	0.006889992\\
0.4943494	0.006856018\\
0.4944494	0.006898273\\
0.4945495	0.007241447\\
0.4946495	0.007023417\\
0.4947495	0.006664043\\
0.4948495	0.006851049\\
0.4949495	0.006727496\\
0.4950495	0.007031112\\
0.4951495	0.007103277\\
0.4952495	0.007148256\\
0.4953495	0.007077475\\
0.4954495	0.007445103\\
0.4955496	0.007511633\\
0.4956496	0.007653357\\
0.4957496	0.007496475\\
0.4958496	0.007397402\\
0.4959496	0.007540961\\
0.4960496	0.007786501\\
0.4961496	0.007516847\\
0.4962496	0.00762105\\
0.4963496	0.007568184\\
0.4964496	0.007188054\\
0.4965497	0.007492354\\
0.4966497	0.007610497\\
0.4967497	0.007986836\\
0.4968497	0.007969096\\
0.4969497	0.007698221\\
0.4970497	0.00774832\\
0.4971497	0.007625749\\
0.4972497	0.007832936\\
0.4973497	0.00815561\\
0.4974497	0.007861228\\
0.4975498	0.00793742\\
0.4976498	0.008174138\\
0.4977498	0.008195592\\
0.4978498	0.007933528\\
0.4979498	0.008352071\\
0.4980498	0.008248278\\
0.4981498	0.00859916\\
0.4982498	0.008670628\\
0.4983498	0.008389313\\
0.4984498	0.008330265\\
0.4985499	0.008094371\\
0.4986499	0.007872119\\
0.4987499	0.008135042\\
0.4988499	0.008432257\\
0.4989499	0.008499153\\
0.4990499	0.008650618\\
0.4991499	0.008943719\\
0.4992499	0.008850925\\
0.4993499	0.008737606\\
0.4994499	0.008918299\\
0.49955	0.009033562\\
0.49965	0.009021799\\
0.49975	0.008949471\\
0.49985	0.009009688\\
0.49995	0.008947399\\
0.50005	0.009155313\\
0.50015	0.009213411\\
0.50025	0.009345245\\
0.50035	0.009236517\\
0.50045	0.009087522\\
0.5005501	0.00911854\\
0.5006501	0.009149282\\
0.5007501	0.009230069\\
0.5008501	0.00913978\\
0.5009501	0.009225558\\
0.5010501	0.009317322\\
0.5011501	0.009595349\\
0.5012501	0.009630063\\
0.5013501	0.009777305\\
0.5014501	0.009900513\\
0.5015502	0.009505482\\
0.5016502	0.009624662\\
0.5017502	0.009660942\\
0.5018502	0.009760637\\
0.5019502	0.009822771\\
0.5020502	0.01007069\\
0.5021502	0.01014866\\
0.5022502	0.01019404\\
0.5023502	0.009888185\\
0.5024502	0.01037665\\
0.5025503	0.01046227\\
0.5026503	0.01014418\\
0.5027503	0.009849868\\
0.5028503	0.01043199\\
0.5029503	0.01067255\\
0.5030503	0.01014014\\
0.5031503	0.01036622\\
0.5032503	0.01048609\\
0.5033503	0.01064202\\
0.5034503	0.01047516\\
0.5035504	0.01065908\\
0.5036504	0.01072488\\
0.5037504	0.01098054\\
0.5038504	0.0109544\\
0.5039504	0.01044764\\
0.5040504	0.0105382\\
0.5041504	0.01074175\\
0.5042504	0.01114183\\
0.5043504	0.0108639\\
0.5044504	0.01097883\\
0.5045505	0.01129112\\
0.5046505	0.01099391\\
0.5047505	0.01135359\\
0.5048505	0.01163166\\
0.5049505	0.01186458\\
0.5050505	0.01172482\\
0.5051505	0.01193807\\
0.5052505	0.01171734\\
0.5053505	0.01197837\\
0.5054505	0.01208118\\
0.5055506	0.01157008\\
0.5056506	0.01143548\\
0.5057506	0.01161354\\
0.5058506	0.01172633\\
0.5059506	0.01131697\\
0.5060506	0.01091779\\
0.5061506	0.0113937\\
0.5062506	0.01183263\\
0.5063506	0.0116188\\
0.5064506	0.01197831\\
0.5065507	0.01223936\\
0.5066507	0.01246206\\
0.5067507	0.01297372\\
0.5068507	0.01331844\\
0.5069507	0.01303273\\
0.5070507	0.01310068\\
0.5071507	0.0126983\\
0.5072507	0.01265269\\
0.5073507	0.01235784\\
0.5074507	0.01218552\\
0.5075508	0.0126348\\
0.5076508	0.01288284\\
0.5077508	0.01322251\\
0.5078508	0.01308554\\
0.5079508	0.01318154\\
0.5080508	0.01324204\\
0.5081508	0.01340212\\
0.5082508	0.01324636\\
0.5083508	0.01334435\\
0.5084508	0.01316432\\
0.5085509	0.01347983\\
0.5086509	0.01361146\\
0.5087509	0.01335804\\
0.5088509	0.01333189\\
0.5089509	0.01317265\\
0.5090509	0.01351155\\
0.5091509	0.01366815\\
0.5092509	0.01396391\\
0.5093509	0.01387317\\
0.5094509	0.01423777\\
0.509551	0.01392\\
0.509651	0.01392731\\
0.509751	0.01423306\\
0.509851	0.01425157\\
0.509951	0.01477771\\
0.510051	0.0144282\\
0.510151	0.01435922\\
0.510251	0.01464864\\
0.510351	0.01463726\\
0.510451	0.01467309\\
0.5105511	0.01421916\\
0.5106511	0.01395411\\
0.5107511	0.01443417\\
0.5108511	0.01512011\\
0.5109511	0.01541926\\
0.5110511	0.01528367\\
0.5111511	0.015353\\
0.5112511	0.01541803\\
0.5113511	0.01538943\\
0.5114511	0.01541966\\
0.5115512	0.01560315\\
0.5116512	0.01569458\\
0.5117512	0.01534265\\
0.5118512	0.01521469\\
0.5119512	0.01541291\\
0.5120512	0.01593843\\
0.5121512	0.01553578\\
0.5122512	0.0158755\\
0.5123512	0.01602742\\
0.5124512	0.01562088\\
0.5125513	0.01580826\\
0.5126513	0.01628183\\
0.5127513	0.01646148\\
0.5128513	0.01643719\\
0.5129513	0.01669327\\
0.5130513	0.01723998\\
0.5131513	0.01688239\\
0.5132513	0.01640141\\
0.5133513	0.0163313\\
0.5134513	0.01649363\\
0.5135514	0.01698633\\
0.5136514	0.01685629\\
0.5137514	0.01613646\\
0.5138514	0.01586846\\
0.5139514	0.01606489\\
0.5140514	0.01667178\\
0.5141514	0.0172548\\
0.5142514	0.01751283\\
0.5143514	0.01794344\\
0.5144514	0.01848066\\
0.5145515	0.018161\\
0.5146515	0.01795214\\
0.5147515	0.01811335\\
0.5148515	0.01805408\\
0.5149515	0.01830009\\
0.5150515	0.0180589\\
0.5151515	0.01863615\\
0.5152515	0.01844424\\
0.5153515	0.01822081\\
0.5154515	0.0184488\\
0.5155516	0.01825643\\
0.5156516	0.01827963\\
0.5157516	0.01844559\\
0.5158516	0.01850983\\
0.5159516	0.01884974\\
0.5160516	0.01823624\\
0.5161516	0.01861567\\
0.5162516	0.018617\\
0.5163516	0.01888072\\
0.5164516	0.01891015\\
0.5165517	0.01847776\\
0.5166517	0.01888476\\
0.5167517	0.01944783\\
0.5168517	0.01953094\\
0.5169517	0.01945501\\
0.5170517	0.01974139\\
0.5171517	0.02019705\\
0.5172517	0.02020451\\
0.5173517	0.02009462\\
0.5174517	0.02023852\\
0.5175518	0.02020433\\
0.5176518	0.02075489\\
0.5177518	0.02005284\\
0.5178518	0.02000179\\
0.5179518	0.02050629\\
0.5180518	0.02055078\\
0.5181518	0.02051199\\
0.5182518	0.02052591\\
0.5183518	0.02087579\\
0.5184518	0.02148867\\
0.5185519	0.02171671\\
0.5186519	0.0211073\\
0.5187519	0.02099006\\
0.5188519	0.02162311\\
0.5189519	0.02099753\\
0.5190519	0.02099223\\
0.5191519	0.02119316\\
0.5192519	0.02109024\\
0.5193519	0.02151253\\
0.5194519	0.02239425\\
0.519552	0.02263682\\
0.519652	0.0220723\\
0.519752	0.02187216\\
0.519852	0.02246897\\
0.519952	0.0230352\\
0.520052	0.02253634\\
0.520152	0.02222494\\
0.520252	0.02260038\\
0.520352	0.02252044\\
0.520452	0.02241908\\
0.5205521	0.02226085\\
0.5206521	0.02204503\\
0.5207521	0.02301164\\
0.5208521	0.02330849\\
0.5209521	0.0232292\\
0.5210521	0.02306282\\
0.5211521	0.02363738\\
0.5212521	0.02422842\\
0.5213521	0.02423699\\
0.5214521	0.02422433\\
0.5215522	0.02397115\\
0.5216522	0.02465751\\
0.5217522	0.02465823\\
0.5218522	0.02414963\\
0.5219522	0.02423277\\
0.5220522	0.02496511\\
0.5221522	0.02460083\\
0.5222522	0.02459528\\
0.5223522	0.02434254\\
0.5224522	0.02490581\\
0.5225523	0.024481\\
0.5226523	0.02453407\\
0.5227523	0.02499808\\
0.5228523	0.02556827\\
0.5229523	0.02615646\\
0.5230523	0.02535862\\
0.5231523	0.02543725\\
0.5232523	0.02553768\\
0.5233523	0.02614129\\
0.5234523	0.02652944\\
0.5235524	0.02661781\\
0.5236524	0.02649856\\
0.5237524	0.02640007\\
0.5238524	0.02619562\\
0.5239524	0.02653575\\
0.5240524	0.02626371\\
0.5241524	0.02629248\\
0.5242524	0.02609728\\
0.5243524	0.02657674\\
0.5244524	0.02694489\\
0.5245525	0.02706737\\
0.5246525	0.02657979\\
0.5247525	0.02707142\\
0.5248525	0.02787296\\
0.5249525	0.02843184\\
0.5250525	0.02872942\\
0.5251525	0.02909788\\
0.5252525	0.02855504\\
0.5253525	0.0289149\\
0.5254525	0.02893257\\
0.5255526	0.02833285\\
0.5256526	0.02861007\\
0.5257526	0.02776159\\
0.5258526	0.02811165\\
0.5259526	0.02827584\\
0.5260526	0.0292082\\
0.5261526	0.02908715\\
0.5262526	0.0290326\\
0.5263526	0.02962143\\
0.5264526	0.03016191\\
0.5265527	0.03022669\\
0.5266527	0.02966134\\
0.5267527	0.02973049\\
0.5268527	0.03051553\\
0.5269527	0.02966349\\
0.5270527	0.03008452\\
0.5271527	0.03050294\\
0.5272527	0.03001557\\
0.5273527	0.03091019\\
0.5274527	0.03050069\\
0.5275528	0.03077923\\
0.5276528	0.0313766\\
0.5277528	0.03119438\\
0.5278528	0.03104723\\
0.5279528	0.03141678\\
0.5280528	0.03181603\\
0.5281528	0.03207626\\
0.5282528	0.0321546\\
0.5283528	0.03235945\\
0.5284528	0.03184807\\
0.5285529	0.03215235\\
0.5286529	0.0323189\\
0.5287529	0.0327411\\
0.5288529	0.0334414\\
0.5289529	0.03313407\\
0.5290529	0.03297651\\
0.5291529	0.03347636\\
0.5292529	0.03327595\\
0.5293529	0.03304634\\
0.5294529	0.03320006\\
0.529553	0.03408912\\
0.529653	0.03365764\\
0.529753	0.03368131\\
0.529853	0.03405197\\
0.529953	0.03355483\\
0.530053	0.03435926\\
0.530153	0.03441717\\
0.530253	0.03429002\\
0.530353	0.0346018\\
0.530453	0.03500629\\
0.5305531	0.03521775\\
0.5306531	0.03586315\\
0.5307531	0.03544989\\
0.5308531	0.03610359\\
0.5309531	0.036477\\
0.5310531	0.03646354\\
0.5311531	0.03654321\\
0.5312531	0.03626331\\
0.5313531	0.03686383\\
0.5314531	0.03630447\\
0.5315532	0.03540812\\
0.5316532	0.03599641\\
0.5317532	0.03622338\\
0.5318532	0.03729852\\
0.5319532	0.03779091\\
0.5320532	0.0372871\\
0.5321532	0.03739053\\
0.5322532	0.03735344\\
0.5323532	0.03768818\\
0.5324532	0.0383501\\
0.5325533	0.03830032\\
0.5326533	0.03851459\\
0.5327533	0.03865531\\
0.5328533	0.03888769\\
0.5329533	0.03862226\\
0.5330533	0.03890874\\
0.5331533	0.03905725\\
0.5332533	0.03908864\\
0.5333533	0.03957341\\
0.5334533	0.03931285\\
0.5335534	0.03995847\\
0.5336534	0.03936091\\
0.5337534	0.04015043\\
0.5338534	0.04078902\\
0.5339534	0.04102766\\
0.5340534	0.04137291\\
0.5341534	0.04194618\\
0.5342534	0.04145153\\
0.5343534	0.04112573\\
0.5344534	0.04096386\\
0.5345535	0.04102397\\
0.5346535	0.04147078\\
0.5347535	0.041489\\
0.5348535	0.0414124\\
0.5349535	0.04217747\\
0.5350535	0.04224598\\
0.5351535	0.04242245\\
0.5352535	0.04273739\\
0.5353535	0.04300402\\
0.5354535	0.04255735\\
0.5355536	0.04277807\\
0.5356536	0.04333525\\
0.5357536	0.04318971\\
0.5358536	0.04372023\\
0.5359536	0.04375226\\
0.5360536	0.04394735\\
0.5361536	0.04441322\\
0.5362536	0.04498625\\
0.5363536	0.04596818\\
0.5364536	0.04593455\\
0.5365537	0.04600166\\
0.5366537	0.04536353\\
0.5367537	0.04494585\\
0.5368537	0.04551404\\
0.5369537	0.04491146\\
0.5370537	0.04503648\\
0.5371537	0.0452876\\
0.5372537	0.04617911\\
0.5373537	0.04565487\\
0.5374537	0.04620235\\
0.5375538	0.04743583\\
0.5376538	0.04787723\\
0.5377538	0.04740744\\
0.5378538	0.04777313\\
0.5379538	0.04776306\\
0.5380538	0.04827374\\
0.5381538	0.04900488\\
0.5382538	0.04811281\\
0.5383538	0.04861375\\
0.5384538	0.04924641\\
0.5385539	0.0488927\\
0.5386539	0.04825308\\
0.5387539	0.04871488\\
0.5388539	0.04961255\\
0.5389539	0.05076925\\
0.5390539	0.05059654\\
0.5391539	0.05016249\\
0.5392539	0.05053003\\
0.5393539	0.05074199\\
0.5394539	0.05021331\\
0.539554	0.04965676\\
0.539654	0.05010974\\
0.539754	0.04989783\\
0.539854	0.05053769\\
0.539954	0.05084937\\
0.540054	0.05081262\\
0.540154	0.05159508\\
0.540254	0.05221044\\
0.540354	0.05228407\\
0.540454	0.05339094\\
0.5405541	0.05451845\\
0.5406541	0.05493616\\
0.5407541	0.05451722\\
0.5408541	0.05478492\\
0.5409541	0.05524422\\
0.5410541	0.05437599\\
0.5411541	0.05381077\\
0.5412541	0.05350453\\
0.5413541	0.05486058\\
0.5414541	0.05497088\\
0.5415542	0.05467826\\
0.5416542	0.05494415\\
0.5417542	0.05535285\\
0.5418542	0.05492369\\
0.5419542	0.0552103\\
0.5420542	0.05597699\\
0.5421542	0.05704333\\
0.5422542	0.0575701\\
0.5423542	0.05801181\\
0.5424542	0.05730855\\
0.5425543	0.05639691\\
0.5426543	0.05655114\\
0.5427543	0.05758821\\
0.5428543	0.0575716\\
0.5429543	0.05790873\\
0.5430543	0.05770322\\
0.5431543	0.05824508\\
0.5432543	0.05864743\\
0.5433543	0.05919396\\
0.5434543	0.05963498\\
0.5435544	0.0601568\\
0.5436544	0.061047\\
0.5437544	0.060664\\
0.5438544	0.06079703\\
0.5439544	0.06060492\\
0.5440544	0.06020558\\
0.5441544	0.06089257\\
0.5442544	0.06179448\\
0.5443544	0.06220159\\
0.5444544	0.06187543\\
0.5445545	0.06158383\\
0.5446545	0.06174941\\
0.5447545	0.06338605\\
0.5448545	0.06395691\\
0.5449545	0.0637278\\
0.5450545	0.06306281\\
0.5451545	0.06291205\\
0.5452545	0.06372639\\
0.5453545	0.06309095\\
0.5454545	0.0639614\\
0.5455546	0.06482499\\
0.5456546	0.0652087\\
0.5457546	0.06541917\\
0.5458546	0.06574604\\
0.5459546	0.06574014\\
0.5460546	0.06542345\\
0.5461546	0.0658914\\
0.5462546	0.06633405\\
0.5463546	0.0654029\\
0.5464546	0.06561435\\
0.5465547	0.06726101\\
0.5466547	0.06780156\\
0.5467547	0.06805312\\
0.5468547	0.06722037\\
0.5469547	0.06816018\\
0.5470547	0.06879405\\
0.5471547	0.0690133\\
0.5472547	0.06931064\\
0.5473547	0.07038943\\
0.5474547	0.07002029\\
0.5475548	0.06952341\\
0.5476548	0.06962056\\
0.5477548	0.06996142\\
0.5478548	0.06985196\\
0.5479548	0.07026357\\
0.5480548	0.07074329\\
0.5481548	0.07203345\\
0.5482548	0.07308447\\
0.5483548	0.071936\\
0.5484548	0.07211707\\
0.5485549	0.07206892\\
0.5486549	0.07303562\\
0.5487549	0.07434574\\
0.5488549	0.07518176\\
0.5489549	0.0742338\\
0.5490549	0.07418737\\
0.5491549	0.07412587\\
0.5492549	0.07479269\\
0.5493549	0.07463342\\
0.5494549	0.07451053\\
0.549555	0.0752754\\
0.549655	0.07521869\\
0.549755	0.07423333\\
0.549855	0.07553655\\
0.549955	0.07618892\\
0.550055	0.07634626\\
0.550155	0.07670144\\
0.550255	0.07634559\\
0.550355	0.07755124\\
0.550455	0.07875283\\
0.5505551	0.07880328\\
0.5506551	0.07852245\\
0.5507551	0.08007283\\
0.5508551	0.08026126\\
0.5509551	0.08036652\\
0.5510551	0.0807324\\
0.5511551	0.08066351\\
0.5512551	0.08059113\\
0.5513551	0.08192279\\
0.5514551	0.08115475\\
0.5515552	0.08114911\\
0.5516552	0.08093646\\
0.5517552	0.08149602\\
0.5518552	0.08151354\\
0.5519552	0.0811291\\
0.5520552	0.08292534\\
0.5521552	0.08424127\\
0.5522552	0.08363539\\
0.5523552	0.08371631\\
0.5524552	0.08461016\\
0.5525553	0.08565333\\
0.5526553	0.08549148\\
0.5527553	0.08681234\\
0.5528553	0.08623708\\
0.5529553	0.08524189\\
0.5530553	0.08632289\\
0.5531553	0.08654447\\
0.5532553	0.08618201\\
0.5533553	0.08617581\\
0.5534553	0.08658772\\
0.5535554	0.08705782\\
0.5536554	0.08704395\\
0.5537554	0.08761731\\
0.5538554	0.08772897\\
0.5539554	0.08916262\\
0.5540554	0.08989124\\
0.5541554	0.09117192\\
0.5542554	0.09200531\\
0.5543554	0.09115909\\
0.5544554	0.09234336\\
0.5545555	0.09215821\\
0.5546555	0.09111886\\
0.5547555	0.09268684\\
0.5548555	0.0923941\\
0.5549555	0.09246142\\
0.5550555	0.0926575\\
0.5551555	0.09292763\\
0.5552555	0.09272624\\
0.5553555	0.09368442\\
0.5554555	0.09338437\\
0.5555556	0.09453743\\
0.5556556	0.09411319\\
0.5557556	0.09542694\\
0.5558556	0.09668171\\
0.5559556	0.0975318\\
0.5560556	0.09817547\\
0.5561556	0.09740064\\
0.5562556	0.09728414\\
0.5563556	0.09691834\\
0.5564556	0.09782678\\
0.5565557	0.09752466\\
0.5566557	0.09740687\\
0.5567557	0.09726453\\
0.5568557	0.09888513\\
0.5569557	0.09972605\\
0.5570557	0.1009841\\
0.5571557	0.1012669\\
0.5572557	0.1024722\\
0.5573557	0.1022976\\
0.5574557	0.1034994\\
0.5575558	0.1025663\\
0.5576558	0.1018113\\
0.5577558	0.1021107\\
0.5578558	0.1037056\\
0.5579558	0.1043135\\
0.5580558	0.1036781\\
0.5581558	0.1033437\\
0.5582558	0.1039405\\
0.5583558	0.1042098\\
0.5584558	0.1055726\\
0.5585559	0.1060113\\
0.5586559	0.1063903\\
0.5587559	0.1062332\\
0.5588559	0.1075576\\
0.5589559	0.1059791\\
0.5590559	0.1080851\\
0.5591559	0.1089028\\
0.5592559	0.1086841\\
0.5593559	0.1094435\\
0.5594559	0.1100637\\
0.559556	0.1099473\\
0.559656	0.1096137\\
0.559756	0.1104971\\
0.559856	0.1114514\\
0.559956	0.111861\\
0.560056	0.112649\\
0.560156	0.1134491\\
0.560256	0.1134351\\
0.560356	0.1142428\\
0.560456	0.1138487\\
0.5605561	0.114838\\
0.5606561	0.1147383\\
0.5607561	0.1145901\\
0.5608561	0.1148575\\
0.5609561	0.1138967\\
0.5610561	0.1149231\\
0.5611561	0.1151924\\
0.5612561	0.115143\\
0.5613561	0.1164775\\
0.5614561	0.1180263\\
0.5615562	0.1185599\\
0.5616562	0.1191123\\
0.5617562	0.1205811\\
0.5618562	0.1206808\\
0.5619562	0.1209244\\
0.5620562	0.1213511\\
0.5621562	0.1225572\\
0.5622562	0.1223189\\
0.5623562	0.1213518\\
0.5624562	0.1208145\\
0.5625563	0.1216788\\
0.5626563	0.1230796\\
0.5627563	0.123381\\
0.5628563	0.1228234\\
0.5629563	0.1235664\\
0.5630563	0.1254965\\
0.5631563	0.1251457\\
0.5632563	0.1245532\\
0.5633563	0.1252312\\
0.5634563	0.1282127\\
0.5635564	0.1273523\\
0.5636564	0.1272093\\
0.5637564	0.1281119\\
0.5638564	0.1286566\\
0.5639564	0.1287013\\
0.5640564	0.1305847\\
0.5641564	0.1301183\\
0.5642564	0.1316286\\
0.5643564	0.1328785\\
0.5644564	0.1324086\\
0.5645565	0.1310121\\
0.5646565	0.1318967\\
0.5647565	0.1335866\\
0.5648565	0.1339935\\
0.5649565	0.1335733\\
0.5650565	0.1329967\\
0.5651565	0.1337947\\
0.5652565	0.1348455\\
0.5653565	0.1364488\\
0.5654565	0.1354507\\
0.5655566	0.1367133\\
0.5656566	0.1374627\\
0.5657566	0.1363492\\
0.5658566	0.1356025\\
0.5659566	0.1375468\\
0.5660566	0.1393449\\
0.5661566	0.1390579\\
0.5662566	0.1392686\\
0.5663566	0.1419216\\
0.5664566	0.1419115\\
0.5665567	0.1421505\\
0.5666567	0.1426233\\
0.5667567	0.143483\\
0.5668567	0.1424732\\
0.5669567	0.1435651\\
0.5670567	0.1455433\\
0.5671567	0.1451184\\
0.5672567	0.1445946\\
0.5673567	0.1445388\\
0.5674567	0.1454138\\
0.5675568	0.1479981\\
0.5676568	0.1495813\\
0.5677568	0.1492108\\
0.5678568	0.1485162\\
0.5679568	0.1494212\\
0.5680568	0.1506293\\
0.5681568	0.1503922\\
0.5682568	0.1496197\\
0.5683568	0.1499733\\
0.5684568	0.150413\\
0.5685569	0.1501095\\
0.5686569	0.1525013\\
0.5687569	0.1517104\\
0.5688569	0.1526974\\
0.5689569	0.1536476\\
0.5690569	0.153637\\
0.5691569	0.153862\\
0.5692569	0.1552856\\
0.5693569	0.1565065\\
0.5694569	0.1574385\\
0.569557	0.1580395\\
0.569657	0.1582386\\
0.569757	0.1596585\\
0.569857	0.1607801\\
0.569957	0.1599455\\
0.570057	0.1596824\\
0.570157	0.1610542\\
0.570257	0.1616709\\
0.570357	0.1619416\\
0.570457	0.1627871\\
0.5705571	0.1642432\\
0.5706571	0.1638593\\
0.5707571	0.163538\\
0.5708571	0.1643256\\
0.5709571	0.1660461\\
0.5710571	0.1663257\\
0.5711571	0.1658307\\
0.5712571	0.1659683\\
0.5713571	0.1674275\\
0.5714571	0.1671291\\
0.5715572	0.1691046\\
0.5716572	0.1701615\\
0.5717572	0.1695906\\
0.5718572	0.1703494\\
0.5719572	0.1705388\\
0.5720572	0.1746531\\
0.5721572	0.1743864\\
0.5722572	0.1723615\\
0.5723572	0.1715534\\
0.5724572	0.173629\\
0.5725573	0.1744867\\
0.5726573	0.1751784\\
0.5727573	0.1751185\\
0.5728573	0.1759763\\
0.5729573	0.1781377\\
0.5730573	0.1781088\\
0.5731573	0.1774865\\
0.5732573	0.177556\\
0.5733573	0.1787612\\
0.5734573	0.1801669\\
0.5735574	0.1822179\\
0.5736574	0.1823576\\
0.5737574	0.180641\\
0.5738574	0.1818122\\
0.5739574	0.1825895\\
0.5740574	0.1833948\\
0.5741574	0.1855868\\
0.5742574	0.1877752\\
0.5743574	0.1881674\\
0.5744574	0.1872467\\
0.5745575	0.1877518\\
0.5746575	0.1888443\\
0.5747575	0.188575\\
0.5748575	0.1898625\\
0.5749575	0.1906252\\
0.5750575	0.1905325\\
0.5751575	0.1902596\\
0.5752575	0.1905926\\
0.5753575	0.1918063\\
0.5754575	0.1913457\\
0.5755576	0.191294\\
0.5756576	0.1929805\\
0.5757576	0.1956429\\
0.5758576	0.1950421\\
0.5759576	0.1948967\\
0.5760576	0.1972625\\
0.5761576	0.1991386\\
0.5762576	0.1997398\\
0.5763576	0.1987182\\
0.5764576	0.1990969\\
0.5765577	0.201016\\
0.5766577	0.2018206\\
0.5767577	0.2022144\\
0.5768577	0.201361\\
0.5769577	0.2030776\\
0.5770577	0.204893\\
0.5771577	0.2046224\\
0.5772577	0.203766\\
0.5773577	0.2063394\\
0.5774577	0.2063614\\
0.5775578	0.2058704\\
0.5776578	0.2068466\\
0.5777578	0.2081561\\
0.5778578	0.2091303\\
0.5779578	0.2112712\\
0.5780578	0.2118104\\
0.5781578	0.2121002\\
0.5782578	0.2133904\\
0.5783578	0.213041\\
0.5784578	0.2153462\\
0.5785579	0.2148442\\
0.5786579	0.2155064\\
0.5787579	0.2171157\\
0.5788579	0.2159746\\
0.5789579	0.2168084\\
0.5790579	0.216295\\
0.5791579	0.2163727\\
0.5792579	0.2165855\\
0.5793579	0.2195705\\
0.5794579	0.2200933\\
0.579558	0.2210532\\
0.579658	0.2207702\\
0.579758	0.2221621\\
0.579858	0.2242105\\
0.579958	0.2248977\\
0.580058	0.2276462\\
0.580158	0.2292992\\
0.580258	0.2290455\\
0.580358	0.2286418\\
0.580458	0.2297913\\
0.5805581	0.2278179\\
0.5806581	0.2273688\\
0.5807581	0.228679\\
0.5808581	0.2290667\\
0.5809581	0.2307986\\
0.5810581	0.2337605\\
0.5811581	0.2336956\\
0.5812581	0.2352793\\
0.5813581	0.2369933\\
0.5814581	0.2379383\\
0.5815582	0.2373818\\
0.5816582	0.2379362\\
0.5817582	0.238653\\
0.5818582	0.2391624\\
0.5819582	0.2388814\\
0.5820582	0.2402373\\
0.5821582	0.2407369\\
0.5822582	0.2402041\\
0.5823582	0.2427387\\
0.5824582	0.2430765\\
0.5825583	0.2434032\\
0.5826583	0.2464865\\
0.5827583	0.2452025\\
0.5828583	0.2453242\\
0.5829583	0.2466889\\
0.5830583	0.2484468\\
0.5831583	0.2508555\\
0.5832583	0.2524104\\
0.5833583	0.2524078\\
0.5834583	0.2538954\\
0.5835584	0.2555852\\
0.5836584	0.2550206\\
0.5837584	0.2551593\\
0.5838584	0.254306\\
0.5839584	0.2561555\\
0.5840584	0.2579639\\
0.5841584	0.2581653\\
0.5842584	0.2569595\\
0.5843584	0.257754\\
0.5844584	0.2571758\\
0.5845585	0.258361\\
0.5846585	0.259889\\
0.5847585	0.2608869\\
0.5848585	0.2648419\\
0.5849585	0.2671879\\
0.5850585	0.266305\\
0.5851585	0.2668187\\
0.5852585	0.2659576\\
0.5853585	0.2668599\\
0.5854585	0.2698013\\
0.5855586	0.2705262\\
0.5856586	0.2716926\\
0.5857586	0.2727884\\
0.5858586	0.2731246\\
0.5859586	0.2729226\\
0.5860586	0.2735001\\
0.5861586	0.2744205\\
0.5862586	0.2771499\\
0.5863586	0.277226\\
0.5864586	0.2772458\\
0.5865587	0.2786016\\
0.5866587	0.2786156\\
0.5867587	0.2802042\\
0.5868587	0.2806559\\
0.5869587	0.2825191\\
0.5870587	0.282393\\
0.5871587	0.2836942\\
0.5872587	0.2854755\\
0.5873587	0.2856299\\
0.5874587	0.2872603\\
0.5875588	0.286528\\
0.5876588	0.2885035\\
0.5877588	0.288996\\
0.5878588	0.2907477\\
0.5879588	0.2898741\\
0.5880588	0.2914996\\
0.5881588	0.2933489\\
0.5882588	0.2947927\\
0.5883588	0.2974796\\
0.5884588	0.2959604\\
0.5885589	0.2986561\\
0.5886589	0.2989971\\
0.5887589	0.2972409\\
0.5888589	0.2972618\\
0.5889589	0.2983813\\
0.5890589	0.3015588\\
0.5891589	0.3019917\\
0.5892589	0.3016552\\
0.5893589	0.3024059\\
0.5894589	0.3055678\\
0.589559	0.3064204\\
0.589659	0.3085692\\
0.589759	0.3104791\\
0.589859	0.3118574\\
0.589959	0.3122745\\
0.590059	0.311997\\
0.590159	0.3124906\\
0.590259	0.3137746\\
0.590359	0.3144304\\
0.590459	0.3139771\\
0.5905591	0.3154026\\
0.5906591	0.3154327\\
0.5907591	0.3167094\\
0.5908591	0.3185626\\
0.5909591	0.3209891\\
0.5910591	0.320813\\
0.5911591	0.3224142\\
0.5912591	0.3234849\\
0.5913591	0.3271834\\
0.5914591	0.3296277\\
0.5915592	0.3282395\\
0.5916592	0.3280893\\
0.5917592	0.3282906\\
0.5918592	0.3272597\\
0.5919592	0.3275074\\
0.5920592	0.3284011\\
0.5921592	0.3288412\\
0.5922592	0.3299991\\
0.5923592	0.3304157\\
0.5924592	0.3333503\\
0.5925593	0.3359863\\
0.5926593	0.3388965\\
0.5927593	0.3410386\\
0.5928593	0.3408568\\
0.5929593	0.3433127\\
0.5930593	0.3446573\\
0.5931593	0.3447421\\
0.5932593	0.3425977\\
0.5933593	0.3451932\\
0.5934593	0.3487139\\
0.5935594	0.3501281\\
0.5936594	0.34979\\
0.5937594	0.3492887\\
0.5938594	0.349792\\
0.5939594	0.3523429\\
0.5940594	0.3530272\\
0.5941594	0.3537861\\
0.5942594	0.3549828\\
0.5943594	0.3565438\\
0.5944594	0.3537373\\
0.5945595	0.3547111\\
0.5946595	0.3556834\\
0.5947595	0.3574686\\
0.5948595	0.3613071\\
0.5949595	0.3619623\\
0.5950595	0.3627666\\
0.5951595	0.3640136\\
0.5952595	0.3667419\\
0.5953595	0.3679337\\
0.5954595	0.3699607\\
0.5955596	0.3681684\\
0.5956596	0.371383\\
0.5957596	0.3731426\\
0.5958596	0.3750261\\
0.5959596	0.3728945\\
0.5960596	0.3733728\\
0.5961596	0.3735964\\
0.5962596	0.3741319\\
0.5963596	0.3783981\\
0.5964596	0.3794404\\
0.5965597	0.3815361\\
0.5966597	0.3815495\\
0.5967597	0.3827075\\
0.5968597	0.3860374\\
0.5969597	0.3876954\\
0.5970597	0.388413\\
0.5971597	0.3863478\\
0.5972597	0.3875884\\
0.5973597	0.3905752\\
0.5974597	0.3922799\\
0.5975598	0.3916368\\
0.5976598	0.3904366\\
0.5977598	0.3940628\\
0.5978598	0.3954587\\
0.5979598	0.3970848\\
0.5980598	0.3993251\\
0.5981598	0.4002932\\
0.5982598	0.4010213\\
0.5983598	0.4023211\\
0.5984598	0.4013846\\
0.5985599	0.4041111\\
0.5986599	0.4063708\\
0.5987599	0.4051943\\
0.5988599	0.4068858\\
0.5989599	0.4071614\\
0.5990599	0.4078322\\
0.5991599	0.4091903\\
0.5992599	0.4130353\\
0.5993599	0.4135898\\
0.5994599	0.4149827\\
0.59956	0.4182396\\
0.59966	0.4191609\\
0.59976	0.4190599\\
0.59986	0.4198402\\
0.59996	0.4214513\\
0.60006	0.4218214\\
0.60016	0.4251204\\
0.60026	0.4294779\\
0.60036	0.4273294\\
0.60046	0.4284806\\
0.6005601	0.4287131\\
0.6006601	0.4310598\\
0.6007601	0.4310312\\
0.6008601	0.4309644\\
0.6009601	0.4336937\\
0.6010601	0.4361545\\
0.6011601	0.4352562\\
0.6012601	0.4360881\\
0.6013601	0.4394345\\
0.6014601	0.4383113\\
0.6015602	0.4405269\\
0.6016602	0.4410148\\
0.6017602	0.4453187\\
0.6018602	0.4470446\\
0.6019602	0.4490782\\
0.6020602	0.4503316\\
0.6021602	0.4525213\\
0.6022602	0.4517762\\
0.6023602	0.4538125\\
0.6024602	0.4542551\\
0.6025603	0.4505587\\
0.6026603	0.4536188\\
0.6027603	0.4537202\\
0.6028603	0.455771\\
0.6029603	0.4608379\\
0.6030603	0.461986\\
0.6031603	0.4653235\\
0.6032603	0.4658273\\
0.6033603	0.4674059\\
0.6034603	0.4680029\\
0.6035604	0.470178\\
0.6036604	0.4698062\\
0.6037604	0.470336\\
0.6038604	0.4722769\\
0.6039604	0.4743479\\
0.6040604	0.4763194\\
0.6041604	0.4745732\\
0.6042604	0.478651\\
0.6043604	0.4783162\\
0.6044604	0.480164\\
0.6045605	0.4829299\\
0.6046605	0.484632\\
0.6047605	0.4849673\\
0.6048605	0.486489\\
0.6049605	0.4882522\\
0.6050605	0.488936\\
0.6051605	0.4906078\\
0.6052605	0.4916274\\
0.6053605	0.4946293\\
0.6054605	0.4957843\\
0.6055606	0.4967548\\
0.6056606	0.4973392\\
0.6057606	0.4976786\\
0.6058606	0.4984673\\
0.6059606	0.4982986\\
0.6060606	0.501913\\
0.6061606	0.5049065\\
0.6062606	0.5070805\\
0.6063606	0.5079544\\
0.6064606	0.5106219\\
0.6065607	0.5122527\\
0.6066607	0.5157345\\
0.6067607	0.5155011\\
0.6068607	0.5150648\\
0.6069607	0.5164189\\
0.6070607	0.5189478\\
0.6071607	0.5211866\\
0.6072607	0.5196428\\
0.6073607	0.5218598\\
0.6074607	0.5242468\\
0.6075608	0.5261172\\
0.6076608	0.5258614\\
0.6077608	0.5273029\\
0.6078608	0.5297922\\
0.6079608	0.5321719\\
0.6080608	0.5338505\\
0.6081608	0.5344538\\
0.6082608	0.5359706\\
0.6083608	0.5350363\\
0.6084608	0.5361131\\
0.6085609	0.5378841\\
0.6086609	0.5399574\\
0.6087609	0.5428884\\
0.6088609	0.5403341\\
0.6089609	0.5464615\\
0.6090609	0.5513723\\
0.6091609	0.5538692\\
0.6092609	0.555533\\
0.6093609	0.5562239\\
0.6094609	0.5585045\\
0.609561	0.5608679\\
0.609661	0.5599002\\
0.609761	0.5593806\\
0.609861	0.5599392\\
0.609961	0.5617993\\
0.610061	0.5607576\\
0.610161	0.5602498\\
0.610261	0.5649728\\
0.610361	0.566638\\
0.610461	0.5680881\\
0.6105611	0.5704398\\
0.6106611	0.5733313\\
0.6107611	0.5765167\\
0.6108611	0.5794731\\
0.6109611	0.5774257\\
0.6110611	0.5810686\\
0.6111611	0.582909\\
0.6112611	0.5840909\\
0.6113611	0.5863308\\
0.6114611	0.5840756\\
0.6115612	0.5836447\\
0.6116612	0.5841973\\
0.6117612	0.5890423\\
0.6118612	0.5919241\\
0.6119612	0.5953416\\
0.6120612	0.5990706\\
0.6121612	0.5991379\\
0.6122612	0.6007072\\
0.6123612	0.6030655\\
0.6124612	0.6079114\\
0.6125613	0.6077384\\
0.6126613	0.6061135\\
0.6127613	0.6055741\\
0.6128613	0.6091056\\
0.6129613	0.612098\\
0.6130613	0.612399\\
0.6131613	0.6125746\\
0.6132613	0.6126209\\
0.6133613	0.6164725\\
0.6134613	0.620524\\
0.6135614	0.6214916\\
0.6136614	0.6260904\\
0.6137614	0.6269057\\
0.6138614	0.626878\\
0.6139614	0.6254797\\
0.6140614	0.6263956\\
0.6141614	0.6303621\\
0.6142614	0.6311912\\
0.6143614	0.6325075\\
0.6144614	0.6365033\\
0.6145615	0.638142\\
0.6146615	0.6397098\\
0.6147615	0.64546\\
0.6148615	0.6474711\\
0.6149615	0.6455811\\
0.6150615	0.6454254\\
0.6151615	0.648525\\
0.6152615	0.6512508\\
0.6153615	0.6514203\\
0.6154615	0.6527776\\
0.6155616	0.656374\\
0.6156616	0.6612039\\
0.6157616	0.6587885\\
0.6158616	0.6581154\\
0.6159616	0.6611223\\
0.6160616	0.6600678\\
0.6161616	0.6642052\\
0.6162616	0.6719789\\
0.6163616	0.6732919\\
0.6164616	0.6723338\\
0.6165617	0.6733966\\
0.6166617	0.6762648\\
0.6167617	0.6788671\\
0.6168617	0.6799859\\
0.6169617	0.684654\\
0.6170617	0.6873329\\
0.6171617	0.6879287\\
0.6172617	0.6864046\\
0.6173617	0.6877236\\
0.6174617	0.6914454\\
0.6175618	0.6927663\\
0.6176618	0.6925591\\
0.6177618	0.6934544\\
0.6178618	0.6937142\\
0.6179618	0.6966676\\
0.6180618	0.6957091\\
0.6181618	0.699505\\
0.6182618	0.6991675\\
0.6183618	0.7023573\\
0.6184618	0.7090754\\
0.6185619	0.7101581\\
0.6186619	0.7151859\\
0.6187619	0.7171848\\
0.6188619	0.7221992\\
0.6189619	0.7231637\\
0.6190619	0.7232251\\
0.6191619	0.7238962\\
0.6192619	0.7251656\\
0.6193619	0.7266781\\
0.6194619	0.7253797\\
0.619562	0.7303269\\
0.619662	0.7336881\\
0.619762	0.7335224\\
0.619862	0.7318327\\
0.619962	0.734333\\
0.620062	0.7371969\\
0.620162	0.7437995\\
0.620262	0.7448231\\
0.620362	0.7445305\\
0.620462	0.7470534\\
0.6205621	0.7498991\\
0.6206621	0.7523076\\
0.6207621	0.7536368\\
0.6208621	0.754964\\
0.6209621	0.7537855\\
0.6210621	0.7579336\\
0.6211621	0.7614312\\
0.6212621	0.7618663\\
0.6213621	0.7641744\\
0.6214621	0.7675847\\
0.6215622	0.769038\\
0.6216622	0.7710721\\
0.6217622	0.7724514\\
0.6218622	0.7726785\\
0.6219622	0.7731904\\
0.6220622	0.7767659\\
0.6221622	0.7786539\\
0.6222622	0.783858\\
0.6223622	0.7849474\\
0.6224622	0.7877789\\
0.6225623	0.7890131\\
0.6226623	0.793223\\
0.6227623	0.80058\\
0.6228623	0.7984403\\
0.6229623	0.7959668\\
0.6230623	0.799836\\
0.6231623	0.8019043\\
0.6232623	0.804494\\
0.6233623	0.8036682\\
0.6234623	0.8052843\\
0.6235624	0.8087257\\
0.6236624	0.8098889\\
0.6237624	0.8118305\\
0.6238624	0.8152902\\
0.6239624	0.8169234\\
0.6240624	0.8202058\\
0.6241624	0.8243645\\
0.6242624	0.8256853\\
0.6243624	0.8277529\\
0.6244624	0.8258771\\
0.6245625	0.8290799\\
0.6246625	0.8347456\\
0.6247625	0.834713\\
0.6248625	0.8340118\\
0.6249625	0.8402933\\
0.6250625	0.837793\\
0.6251625	0.8400968\\
0.6252625	0.8423771\\
0.6253625	0.8460131\\
0.6254625	0.8500135\\
0.6255626	0.8524916\\
0.6256626	0.8538036\\
0.6257626	0.8557226\\
0.6258626	0.8609498\\
0.6259626	0.8625332\\
0.6260626	0.8652237\\
0.6261626	0.8692786\\
0.6262626	0.8726172\\
0.6263626	0.8746104\\
0.6264626	0.8726782\\
0.6265627	0.8702722\\
0.6266627	0.8719888\\
0.6267627	0.8726023\\
0.6268627	0.874941\\
0.6269627	0.8751574\\
0.6270627	0.8815938\\
0.6271627	0.8844385\\
0.6272627	0.8877513\\
0.6273627	0.8958192\\
0.6274627	0.8953512\\
0.6275628	0.8987687\\
0.6276628	0.9025187\\
0.6277628	0.8994506\\
0.6278628	0.9025819\\
0.6279628	0.903953\\
0.6280628	0.9058501\\
0.6281628	0.908712\\
0.6282628	0.9084736\\
0.6283628	0.9151278\\
0.6284628	0.9154739\\
0.6285629	0.9170567\\
0.6286629	0.9205277\\
0.6287629	0.9223621\\
0.6288629	0.9250273\\
0.6289629	0.9263578\\
0.6290629	0.9320212\\
0.6291629	0.9362934\\
0.6292629	0.9388703\\
0.6293629	0.9397292\\
0.6294629	0.935361\\
0.629563	0.935172\\
0.629663	0.9385315\\
0.629763	0.9395702\\
0.629863	0.9450162\\
0.629963	0.9499597\\
0.630063	0.9533573\\
0.630163	0.9557829\\
0.630263	0.9573282\\
0.630363	0.9628072\\
0.630463	0.9647518\\
0.6305631	0.9657557\\
0.6306631	0.9668938\\
0.6307631	0.9706516\\
0.6308631	0.9752255\\
0.6309631	0.9735948\\
0.6310631	0.9754078\\
0.6311631	0.9786246\\
0.6312631	0.9796448\\
0.6313631	0.979448\\
0.6314631	0.9791316\\
0.6315632	0.9852423\\
0.6316632	0.9911837\\
0.6317632	0.9948244\\
0.6318632	0.9951822\\
0.6319632	0.9967861\\
0.6320632	1.000088\\
0.6321632	1.000586\\
0.6322632	1.004606\\
0.6323632	1.008058\\
0.6324632	1.01195\\
0.6325633	1.013178\\
0.6326633	1.019457\\
0.6327633	1.022575\\
0.6328633	1.018634\\
0.6329633	1.019984\\
0.6330633	1.026487\\
0.6331633	1.026842\\
0.6332633	1.026293\\
0.6333633	1.033008\\
0.6334633	1.034582\\
0.6335634	1.035462\\
0.6336634	1.034742\\
0.6337634	1.037243\\
0.6338634	1.039317\\
0.6339634	1.043643\\
0.6340634	1.05222\\
0.6341634	1.053199\\
0.6342634	1.056887\\
0.6343634	1.058425\\
0.6344634	1.057911\\
0.6345635	1.061687\\
0.6346635	1.063208\\
0.6347635	1.065102\\
0.6348635	1.062727\\
0.6349635	1.069718\\
0.6350635	1.07316\\
0.6351635	1.074262\\
0.6352635	1.081174\\
0.6353635	1.086771\\
0.6354635	1.084299\\
0.6355636	1.088533\\
0.6356636	1.094989\\
0.6357636	1.098835\\
0.6358636	1.097082\\
0.6359636	1.096745\\
0.6360636	1.101489\\
0.6361636	1.103749\\
0.6362636	1.106193\\
0.6363636	1.105162\\
0.6364636	1.104685\\
0.6365637	1.110702\\
0.6366637	1.116568\\
0.6367637	1.118275\\
0.6368637	1.11721\\
0.6369637	1.120525\\
0.6370637	1.122456\\
0.6371637	1.122251\\
0.6372637	1.129046\\
0.6373637	1.132749\\
0.6374637	1.134383\\
0.6375638	1.139913\\
0.6376638	1.141652\\
0.6377638	1.143597\\
0.6378638	1.14735\\
0.6379638	1.148988\\
0.6380638	1.15198\\
0.6381638	1.155487\\
0.6382638	1.160705\\
0.6383638	1.162596\\
0.6384638	1.168545\\
0.6385639	1.166433\\
0.6386639	1.167008\\
0.6387639	1.170744\\
0.6388639	1.17497\\
0.6389639	1.177276\\
0.6390639	1.1808\\
0.6391639	1.178002\\
0.6392639	1.183601\\
0.6393639	1.184971\\
0.6394639	1.186734\\
0.639564	1.196045\\
0.639664	1.195893\\
0.639764	1.196899\\
0.639864	1.201681\\
0.639964	1.20048\\
0.640064	1.20044\\
0.640164	1.208793\\
0.640264	1.212931\\
0.640364	1.210939\\
0.640464	1.21225\\
0.6405641	1.223043\\
0.6406641	1.226087\\
0.6407641	1.226842\\
0.6408641	1.228641\\
0.6409641	1.234942\\
0.6410641	1.235418\\
0.6411641	1.234064\\
0.6412641	1.237757\\
0.6413641	1.240643\\
0.6414641	1.239245\\
0.6415642	1.244326\\
0.6416642	1.246598\\
0.6417642	1.252473\\
0.6418642	1.258797\\
0.6419642	1.255644\\
0.6420642	1.260094\\
0.6421642	1.264388\\
0.6422642	1.270926\\
0.6423642	1.271047\\
0.6424642	1.278677\\
0.6425643	1.285134\\
0.6426643	1.282331\\
0.6427643	1.283689\\
0.6428643	1.287034\\
0.6429643	1.287151\\
0.6430643	1.28906\\
0.6431643	1.287289\\
0.6432643	1.290873\\
0.6433643	1.298517\\
0.6434643	1.300011\\
0.6435644	1.304974\\
0.6436644	1.300294\\
0.6437644	1.304234\\
0.6438644	1.308172\\
0.6439644	1.314979\\
0.6440644	1.318475\\
0.6441644	1.321846\\
0.6442644	1.327304\\
0.6443644	1.331628\\
0.6444644	1.334404\\
0.6445645	1.338273\\
0.6446645	1.339431\\
0.6447645	1.3422\\
0.6448645	1.343273\\
0.6449645	1.345215\\
0.6450645	1.346887\\
0.6451645	1.341214\\
0.6452645	1.345695\\
0.6453645	1.352811\\
0.6454645	1.3595\\
0.6455646	1.367978\\
0.6456646	1.370368\\
0.6457646	1.373031\\
0.6458646	1.37664\\
0.6459646	1.37345\\
0.6460646	1.377417\\
0.6461646	1.381115\\
0.6462646	1.386168\\
0.6463646	1.384899\\
0.6464646	1.388601\\
0.6465647	1.390667\\
0.6466647	1.393583\\
0.6467647	1.398833\\
0.6468647	1.40202\\
0.6469647	1.402547\\
0.6470647	1.407902\\
0.6471647	1.410765\\
0.6472647	1.414605\\
0.6473647	1.414938\\
0.6474647	1.418843\\
0.6475648	1.423106\\
0.6476648	1.425492\\
0.6477648	1.432763\\
0.6478648	1.430713\\
0.6479648	1.434619\\
0.6480648	1.438657\\
0.6481648	1.441811\\
0.6482648	1.445582\\
0.6483648	1.445658\\
0.6484648	1.447431\\
0.6485649	1.454751\\
0.6486649	1.456227\\
0.6487649	1.462616\\
0.6488649	1.463274\\
0.6489649	1.462136\\
0.6490649	1.468558\\
0.6491649	1.470416\\
0.6492649	1.472145\\
0.6493649	1.481467\\
0.6494649	1.488677\\
0.649565	1.488904\\
0.649665	1.491429\\
0.649765	1.491644\\
0.649865	1.49484\\
0.649965	1.495623\\
0.650065	1.496674\\
0.650165	1.502989\\
0.650265	1.50491\\
0.650365	1.509529\\
0.650465	1.50866\\
0.6505651	1.512396\\
0.6506651	1.51686\\
0.6507651	1.516598\\
0.6508651	1.52191\\
0.6509651	1.527272\\
0.6510651	1.533833\\
0.6511651	1.537956\\
0.6512651	1.541497\\
0.6513651	1.543357\\
0.6514651	1.547509\\
0.6515652	1.549977\\
0.6516652	1.553699\\
0.6517652	1.555376\\
0.6518652	1.559895\\
0.6519652	1.563897\\
0.6520652	1.565165\\
0.6521652	1.569538\\
0.6522652	1.575191\\
0.6523652	1.575186\\
0.6524652	1.575791\\
0.6525653	1.582912\\
0.6526653	1.585984\\
0.6527653	1.580459\\
0.6528653	1.578614\\
0.6529653	1.584906\\
0.6530653	1.589976\\
0.6531653	1.599257\\
0.6532653	1.600574\\
0.6533653	1.603179\\
0.6534653	1.608515\\
0.6535654	1.612705\\
0.6536654	1.615802\\
0.6537654	1.620579\\
0.6538654	1.630872\\
0.6539654	1.631306\\
0.6540654	1.632317\\
0.6541654	1.635899\\
0.6542654	1.638569\\
0.6543654	1.639168\\
0.6544654	1.642523\\
0.6545655	1.647704\\
0.6546655	1.64901\\
0.6547655	1.649724\\
0.6548655	1.652813\\
0.6549655	1.656356\\
0.6550655	1.66187\\
0.6551655	1.672949\\
0.6552655	1.672319\\
0.6553655	1.66972\\
0.6554655	1.67298\\
0.6555656	1.67995\\
0.6556656	1.683521\\
0.6557656	1.683562\\
0.6558656	1.689276\\
0.6559656	1.689101\\
0.6560656	1.694634\\
0.6561656	1.702828\\
0.6562656	1.707206\\
0.6563656	1.706408\\
0.6564656	1.707976\\
0.6565657	1.710064\\
0.6566657	1.71428\\
0.6567657	1.714973\\
0.6568657	1.721083\\
0.6569657	1.725746\\
0.6570657	1.728244\\
0.6571657	1.730933\\
0.6572657	1.740104\\
0.6573657	1.743045\\
0.6574657	1.74504\\
0.6575658	1.749122\\
0.6576658	1.751874\\
0.6577658	1.757228\\
0.6578658	1.761521\\
0.6579658	1.758988\\
0.6580658	1.765457\\
0.6581658	1.769562\\
0.6582658	1.770213\\
0.6583658	1.777383\\
0.6584658	1.777269\\
0.6585659	1.781571\\
0.6586659	1.780762\\
0.6587659	1.784657\\
0.6588659	1.790778\\
0.6589659	1.80006\\
0.6590659	1.804042\\
0.6591659	1.801639\\
0.6592659	1.799848\\
0.6593659	1.806958\\
0.6594659	1.816728\\
0.659566	1.822249\\
0.659666	1.824168\\
0.659766	1.826836\\
0.659866	1.827339\\
0.659966	1.828437\\
0.660066	1.828643\\
0.660166	1.833633\\
0.660266	1.839423\\
0.660366	1.841745\\
0.660466	1.84623\\
0.6605661	1.848856\\
0.6606661	1.851892\\
0.6607661	1.853728\\
0.6608661	1.853184\\
0.6609661	1.86884\\
0.6610661	1.877789\\
0.6611661	1.877076\\
0.6612661	1.878475\\
0.6613661	1.8851\\
0.6614661	1.889467\\
0.6615662	1.893014\\
0.6616662	1.895536\\
0.6617662	1.900991\\
0.6618662	1.904628\\
0.6619662	1.901039\\
0.6620662	1.902126\\
0.6621662	1.902501\\
0.6622662	1.905255\\
0.6623662	1.912712\\
0.6624662	1.919324\\
0.6625663	1.918649\\
0.6626663	1.926437\\
0.6627663	1.932236\\
0.6628663	1.937382\\
0.6629663	1.939085\\
0.6630663	1.939045\\
0.6631663	1.945383\\
0.6632663	1.944583\\
0.6633663	1.949056\\
0.6634663	1.958791\\
0.6635664	1.957946\\
0.6636664	1.96331\\
0.6637664	1.96822\\
0.6638664	1.969858\\
0.6639664	1.973898\\
0.6640664	1.980267\\
0.6641664	1.985336\\
0.6642664	1.984362\\
0.6643664	1.986946\\
0.6644664	1.993923\\
0.6645665	1.996484\\
0.6646665	2.000357\\
0.6647665	2.001524\\
0.6648665	2.007344\\
0.6649665	2.009899\\
0.6650665	2.01887\\
0.6651665	2.020106\\
0.6652665	2.02136\\
0.6653665	2.02939\\
0.6654665	2.033696\\
0.6655666	2.039059\\
0.6656666	2.03807\\
0.6657666	2.036687\\
0.6658666	2.040247\\
0.6659666	2.042481\\
0.6660666	2.046845\\
0.6661666	2.05494\\
0.6662666	2.061395\\
0.6663666	2.063807\\
0.6664666	2.067463\\
0.6665667	2.076831\\
0.6666667	2.074505\\
0.6667667	2.076645\\
0.6668667	2.077201\\
0.6669667	2.082242\\
0.6670667	2.086164\\
0.6671667	2.088942\\
0.6672667	2.092037\\
0.6673667	2.101193\\
0.6674667	2.104867\\
0.6675668	2.115604\\
0.6676668	2.117581\\
0.6677668	2.119068\\
0.6678668	2.120738\\
0.6679668	2.124418\\
0.6680668	2.12631\\
0.6681668	2.133432\\
0.6682668	2.135118\\
0.6683668	2.135081\\
0.6684668	2.141765\\
0.6685669	2.146648\\
0.6686669	2.155283\\
0.6687669	2.161018\\
0.6688669	2.15758\\
0.6689669	2.157092\\
0.6690669	2.16136\\
0.6691669	2.164641\\
0.6692669	2.16969\\
0.6693669	2.171842\\
0.6694669	2.181819\\
0.669567	2.183877\\
0.669667	2.182413\\
0.669767	2.188059\\
0.669867	2.195389\\
0.669967	2.198457\\
0.670067	2.200783\\
0.670167	2.209688\\
0.670267	2.214192\\
0.670367	2.218177\\
0.670467	2.218768\\
0.6705671	2.224895\\
0.6706671	2.23409\\
0.6707671	2.240277\\
0.6708671	2.240062\\
0.6709671	2.240895\\
0.6710671	2.238976\\
0.6711671	2.245131\\
0.6712671	2.246148\\
0.6713671	2.246777\\
0.6714671	2.253584\\
0.6715672	2.263597\\
0.6716672	2.26732\\
0.6717672	2.273847\\
0.6718672	2.272113\\
0.6719672	2.276611\\
0.6720672	2.278507\\
0.6721672	2.280277\\
0.6722672	2.283259\\
0.6723672	2.290077\\
0.6724672	2.290877\\
0.6725673	2.298823\\
0.6726673	2.302283\\
0.6727673	2.307291\\
0.6728673	2.315466\\
0.6729673	2.321266\\
0.6730673	2.323796\\
0.6731673	2.330912\\
0.6732673	2.330431\\
0.6733673	2.330921\\
0.6734673	2.340774\\
0.6735674	2.34784\\
0.6736674	2.3496\\
0.6737674	2.350858\\
0.6738674	2.343806\\
0.6739674	2.351457\\
0.6740674	2.350734\\
0.6741674	2.356764\\
0.6742674	2.363358\\
0.6743674	2.371544\\
0.6744674	2.37553\\
0.6745675	2.380849\\
0.6746675	2.381688\\
0.6747675	2.385592\\
0.6748675	2.396957\\
0.6749675	2.400322\\
0.6750675	2.405665\\
0.6751675	2.410777\\
0.6752675	2.409811\\
0.6753675	2.418024\\
0.6754675	2.417071\\
0.6755676	2.409673\\
0.6756676	2.41895\\
0.6757676	2.423863\\
0.6758676	2.431255\\
0.6759676	2.435264\\
0.6760676	2.432903\\
0.6761676	2.435523\\
0.6762676	2.444578\\
0.6763676	2.447783\\
0.6764676	2.449317\\
0.6765677	2.459021\\
0.6766677	2.468129\\
0.6767677	2.472667\\
0.6768677	2.471701\\
0.6769677	2.47934\\
0.6770677	2.48493\\
0.6771677	2.483097\\
0.6772677	2.488039\\
0.6773677	2.494644\\
0.6774677	2.496537\\
0.6775678	2.498168\\
0.6776678	2.504278\\
0.6777678	2.506609\\
0.6778678	2.513317\\
0.6779678	2.521652\\
0.6780678	2.52687\\
0.6781678	2.527346\\
0.6782678	2.527178\\
0.6783678	2.534853\\
0.6784678	2.543041\\
0.6785679	2.537853\\
0.6786679	2.530013\\
0.6787679	2.5382\\
0.6788679	2.550283\\
0.6789679	2.553325\\
0.6790679	2.555038\\
0.6791679	2.559589\\
0.6792679	2.565366\\
0.6793679	2.572564\\
0.6794679	2.582902\\
0.679568	2.583856\\
0.679668	2.588046\\
0.679768	2.589456\\
0.679868	2.592637\\
0.679968	2.596194\\
0.680068	2.603992\\
0.680168	2.60548\\
0.680268	2.61046\\
0.680368	2.61266\\
0.680468	2.620923\\
0.6805681	2.621931\\
0.6806681	2.626404\\
0.6807681	2.624971\\
0.6808681	2.636161\\
0.6809681	2.637828\\
0.6810681	2.653733\\
0.6811681	2.656183\\
0.6812681	2.649654\\
0.6813681	2.651221\\
0.6814681	2.653848\\
0.6815682	2.661036\\
0.6816682	2.66996\\
0.6817682	2.672774\\
0.6818682	2.671829\\
0.6819682	2.680144\\
0.6820682	2.679733\\
0.6821682	2.6842\\
0.6822682	2.692285\\
0.6823682	2.689899\\
0.6824682	2.693128\\
0.6825683	2.70655\\
0.6826683	2.707038\\
0.6827683	2.712803\\
0.6828683	2.715039\\
0.6829683	2.717013\\
0.6830683	2.723013\\
0.6831683	2.731616\\
0.6832683	2.737173\\
0.6833683	2.740617\\
0.6834683	2.74217\\
0.6835684	2.749237\\
0.6836684	2.755103\\
0.6837684	2.758606\\
0.6838684	2.758841\\
0.6839684	2.759951\\
0.6840684	2.759749\\
0.6841684	2.763635\\
0.6842684	2.769985\\
0.6843684	2.782369\\
0.6844684	2.785989\\
0.6845685	2.78609\\
0.6846685	2.794155\\
0.6847685	2.800867\\
0.6848685	2.806485\\
0.6849685	2.805845\\
0.6850685	2.810578\\
0.6851685	2.812041\\
0.6852685	2.812924\\
0.6853685	2.815509\\
0.6854685	2.824055\\
0.6855686	2.821785\\
0.6856686	2.827612\\
0.6857686	2.836015\\
0.6858686	2.84248\\
0.6859686	2.842104\\
0.6860686	2.843209\\
0.6861686	2.846554\\
0.6862686	2.859653\\
0.6863686	2.867037\\
0.6864686	2.866262\\
0.6865687	2.867134\\
0.6866687	2.876346\\
0.6867687	2.881284\\
0.6868687	2.886249\\
0.6869687	2.898284\\
0.6870687	2.904524\\
0.6871687	2.906835\\
0.6872687	2.907161\\
0.6873687	2.907331\\
0.6874687	2.908005\\
0.6875688	2.906205\\
0.6876688	2.901333\\
0.6877688	2.908952\\
0.6878688	2.913638\\
0.6879688	2.923922\\
0.6880688	2.931362\\
0.6881688	2.938255\\
0.6882688	2.941324\\
0.6883688	2.949123\\
0.6884688	2.956614\\
0.6885689	2.957874\\
0.6886689	2.963809\\
0.6887689	2.968467\\
0.6888689	2.969688\\
0.6889689	2.96991\\
0.6890689	2.974808\\
0.6891689	2.978068\\
0.6892689	2.978948\\
0.6893689	2.98086\\
0.6894689	2.991175\\
0.689569	3.001309\\
0.689669	2.999749\\
0.689769	3.002584\\
0.689869	3.010331\\
0.689969	3.016021\\
0.690069	3.023769\\
0.690169	3.024128\\
0.690269	3.027758\\
0.690369	3.032932\\
0.690469	3.040417\\
0.6905691	3.040564\\
0.6906691	3.040836\\
0.6907691	3.052204\\
0.6908691	3.049577\\
0.6909691	3.045079\\
0.6910691	3.045301\\
0.6911691	3.053076\\
0.6912691	3.062563\\
0.6913691	3.079737\\
0.6914691	3.08656\\
0.6915692	3.090356\\
0.6916692	3.090879\\
0.6917692	3.091283\\
0.6918692	3.094169\\
0.6919692	3.095306\\
0.6920692	3.096065\\
0.6921692	3.103683\\
0.6922692	3.111749\\
0.6923692	3.114857\\
0.6924692	3.120224\\
0.6925693	3.121105\\
0.6926693	3.128245\\
0.6927693	3.126937\\
0.6928693	3.131771\\
0.6929693	3.139133\\
0.6930693	3.148425\\
0.6931693	3.147954\\
0.6932693	3.153951\\
0.6933693	3.15971\\
0.6934693	3.161608\\
0.6935694	3.1683\\
0.6936694	3.173391\\
0.6937694	3.17883\\
0.6938694	3.176694\\
0.6939694	3.184094\\
0.6940694	3.188822\\
0.6941694	3.191282\\
0.6942694	3.184089\\
0.6943694	3.196098\\
0.6944694	3.211739\\
0.6945695	3.217495\\
0.6946695	3.216897\\
0.6947695	3.215685\\
0.6948695	3.223894\\
0.6949695	3.222284\\
0.6950695	3.226393\\
0.6951695	3.227938\\
0.6952695	3.234047\\
0.6953695	3.232894\\
0.6954695	3.233627\\
0.6955696	3.23804\\
0.6956696	3.242919\\
0.6957696	3.255584\\
0.6958696	3.268663\\
0.6959696	3.269932\\
0.6960696	3.271973\\
0.6961696	3.275804\\
0.6962696	3.288854\\
0.6963696	3.292365\\
0.6964696	3.303931\\
0.6965697	3.308692\\
0.6966697	3.304201\\
0.6967697	3.307856\\
0.6968697	3.303636\\
0.6969697	3.308256\\
0.6970697	3.310052\\
0.6971697	3.31498\\
0.6972697	3.318448\\
0.6973697	3.323922\\
0.6974697	3.324397\\
0.6975698	3.322485\\
0.6976698	3.328662\\
0.6977698	3.337795\\
0.6978698	3.351184\\
0.6979698	3.354814\\
0.6980698	3.355443\\
0.6981698	3.360294\\
0.6982698	3.366853\\
0.6983698	3.360445\\
0.6984698	3.372694\\
0.6985699	3.377407\\
0.6986699	3.383752\\
0.6987699	3.390297\\
0.6988699	3.392205\\
0.6989699	3.40345\\
0.6990699	3.402461\\
0.6991699	3.400261\\
0.6992699	3.400546\\
0.6993699	3.405567\\
0.6994699	3.405961\\
0.69957	3.411483\\
0.69967	3.419622\\
0.69977	3.428069\\
0.69987	3.433974\\
0.69997	3.440389\\
0.70007	3.444596\\
0.70017	3.455974\\
0.70027	3.455529\\
0.70037	3.458542\\
0.70047	3.466673\\
0.7005701	3.463773\\
0.7006701	3.463033\\
0.7007701	3.462667\\
0.7008701	3.467144\\
0.7009701	3.473838\\
0.7010701	3.48108\\
0.7011701	3.481827\\
0.7012701	3.484372\\
0.7013701	3.498422\\
0.7014701	3.491743\\
0.7015702	3.495262\\
0.7016702	3.504402\\
0.7017702	3.513732\\
0.7018702	3.516454\\
0.7019702	3.526755\\
0.7020702	3.526243\\
0.7021702	3.527153\\
0.7022702	3.535931\\
0.7023702	3.538981\\
0.7024702	3.538102\\
0.7025703	3.54054\\
0.7026703	3.55024\\
0.7027703	3.548614\\
0.7028703	3.556334\\
0.7029703	3.56425\\
0.7030703	3.56864\\
0.7031703	3.573059\\
0.7032703	3.572138\\
0.7033703	3.581633\\
0.7034703	3.584104\\
0.7035704	3.588953\\
0.7036704	3.596533\\
0.7037704	3.601529\\
0.7038704	3.602485\\
0.7039704	3.603796\\
0.7040704	3.615602\\
0.7041704	3.611599\\
0.7042704	3.613122\\
0.7043704	3.609038\\
0.7044704	3.61325\\
0.7045705	3.614637\\
0.7046705	3.620535\\
0.7047705	3.629636\\
0.7048705	3.635141\\
0.7049705	3.644469\\
0.7050705	3.657455\\
0.7051705	3.65999\\
0.7052705	3.662017\\
0.7053705	3.66801\\
0.7054705	3.671457\\
0.7055706	3.681969\\
0.7056706	3.680869\\
0.7057706	3.679225\\
0.7058706	3.678027\\
0.7059706	3.680509\\
0.7060706	3.688058\\
0.7061706	3.688783\\
0.7062706	3.694125\\
0.7063706	3.701732\\
0.7064706	3.706145\\
0.7065707	3.709727\\
0.7066707	3.716241\\
0.7067707	3.724209\\
0.7068707	3.729872\\
0.7069707	3.732823\\
0.7070707	3.734471\\
0.7071707	3.731298\\
0.7072707	3.727945\\
0.7073707	3.732746\\
0.7074707	3.736659\\
0.7075708	3.744498\\
0.7076708	3.760082\\
0.7077708	3.769355\\
0.7078708	3.774711\\
0.7079708	3.773062\\
0.7080708	3.77559\\
0.7081708	3.777722\\
0.7082708	3.779434\\
0.7083708	3.789163\\
0.7084708	3.792355\\
0.7085709	3.794566\\
0.7086709	3.794257\\
0.7087709	3.793339\\
0.7088709	3.797322\\
0.7089709	3.80858\\
0.7090709	3.808082\\
0.7091709	3.814715\\
0.7092709	3.814408\\
0.7093709	3.821848\\
0.7094709	3.832006\\
0.709571	3.831527\\
0.709671	3.841931\\
0.709771	3.846898\\
0.709871	3.850497\\
0.709971	3.857081\\
0.710071	3.872352\\
0.710171	3.868413\\
0.710271	3.860444\\
0.710371	3.866165\\
0.710471	3.868903\\
0.7105711	3.866214\\
0.7106711	3.869934\\
0.7107711	3.874954\\
0.7108711	3.873902\\
0.7109711	3.878902\\
0.7110711	3.887632\\
0.7111711	3.894058\\
0.7112711	3.907716\\
0.7113711	3.90979\\
0.7114711	3.913279\\
0.7115712	3.908833\\
0.7116712	3.915527\\
0.7117712	3.925838\\
0.7118712	3.927797\\
0.7119712	3.933193\\
0.7120712	3.935313\\
0.7121712	3.939878\\
0.7122712	3.942073\\
0.7123712	3.951787\\
0.7124712	3.949958\\
0.7125713	3.955865\\
0.7126713	3.960949\\
0.7127713	3.9587\\
0.7128713	3.965962\\
0.7129713	3.972427\\
0.7130713	3.979254\\
0.7131713	3.973047\\
0.7132713	3.986061\\
0.7133713	3.993579\\
0.7134713	3.994156\\
0.7135714	3.99425\\
0.7136714	3.993077\\
0.7137714	3.992902\\
0.7138714	3.997324\\
0.7139714	4.00405\\
0.7140714	4.005084\\
0.7141714	4.007856\\
0.7142714	4.02442\\
0.7143714	4.026682\\
0.7144714	4.025951\\
0.7145715	4.032113\\
0.7146715	4.03315\\
0.7147715	4.031118\\
0.7148715	4.038341\\
0.7149715	4.048256\\
0.7150715	4.052428\\
0.7151715	4.062398\\
0.7152715	4.06702\\
0.7153715	4.073351\\
0.7154715	4.071795\\
0.7155716	4.084405\\
0.7156716	4.082629\\
0.7157716	4.083712\\
0.7158716	4.08753\\
0.7159716	4.086497\\
0.7160716	4.086582\\
0.7161716	4.08857\\
0.7162716	4.091445\\
0.7163716	4.097023\\
0.7164716	4.101882\\
0.7165717	4.112757\\
0.7166717	4.112224\\
0.7167717	4.111157\\
0.7168717	4.11417\\
0.7169717	4.128081\\
0.7170717	4.129694\\
0.7171717	4.137812\\
0.7172717	4.146861\\
0.7173717	4.143764\\
0.7174717	4.144893\\
0.7175718	4.148178\\
0.7176718	4.147534\\
0.7177718	4.145022\\
0.7178718	4.151171\\
0.7179718	4.156509\\
0.7180718	4.170223\\
0.7181718	4.16654\\
0.7182718	4.174426\\
0.7183718	4.183859\\
0.7184718	4.182965\\
0.7185719	4.192799\\
0.7186719	4.192711\\
0.7187719	4.194626\\
0.7188719	4.20114\\
0.7189719	4.199427\\
0.7190719	4.205299\\
0.7191719	4.211901\\
0.7192719	4.217804\\
0.7193719	4.216304\\
0.7194719	4.215097\\
0.719572	4.222041\\
0.719672	4.231116\\
0.719772	4.23464\\
0.719872	4.235656\\
0.719972	4.236727\\
0.720072	4.241793\\
0.720172	4.248598\\
0.720272	4.247484\\
0.720372	4.247523\\
0.720472	4.253821\\
0.7205721	4.259463\\
0.7206721	4.260893\\
0.7207721	4.272177\\
0.7208721	4.265342\\
0.7209721	4.269274\\
0.7210721	4.272726\\
0.7211721	4.277269\\
0.7212721	4.284142\\
0.7213721	4.281224\\
0.7214721	4.294254\\
0.7215722	4.294571\\
0.7216722	4.306357\\
0.7217722	4.31426\\
0.7218722	4.318981\\
0.7219722	4.323799\\
0.7220722	4.325999\\
0.7221722	4.329605\\
0.7222722	4.327027\\
0.7223722	4.32515\\
0.7224722	4.324818\\
0.7225723	4.328346\\
0.7226723	4.332929\\
0.7227723	4.328021\\
0.7228723	4.327678\\
0.7229723	4.336435\\
0.7230723	4.343794\\
0.7231723	4.358998\\
0.7232723	4.368964\\
0.7233723	4.373275\\
0.7234723	4.375416\\
0.7235724	4.370537\\
0.7236724	4.371995\\
0.7237724	4.371693\\
0.7238724	4.37957\\
0.7239724	4.380789\\
0.7240724	4.37923\\
0.7241724	4.389596\\
0.7242724	4.390798\\
0.7243724	4.391144\\
0.7244724	4.396548\\
0.7245725	4.404595\\
0.7246725	4.408173\\
0.7247725	4.417417\\
0.7248725	4.42665\\
0.7249725	4.427045\\
0.7250725	4.428842\\
0.7251725	4.424862\\
0.7252725	4.426837\\
0.7253725	4.426398\\
0.7254725	4.433006\\
0.7255726	4.435724\\
0.7256726	4.439642\\
0.7257726	4.446229\\
0.7258726	4.441925\\
0.7259726	4.449183\\
0.7260726	4.454163\\
0.7261726	4.459974\\
0.7262726	4.470058\\
0.7263726	4.469874\\
0.7264726	4.476016\\
0.7265727	4.47232\\
0.7266727	4.468977\\
0.7267727	4.480086\\
0.7268727	4.486579\\
0.7269727	4.492383\\
0.7270727	4.486837\\
0.7271727	4.493658\\
0.7272727	4.499609\\
0.7273727	4.502795\\
0.7274727	4.498335\\
0.7275728	4.49468\\
0.7276728	4.500747\\
0.7277728	4.516261\\
0.7278728	4.516913\\
0.7279728	4.513772\\
0.7280728	4.516756\\
0.7281728	4.526758\\
0.7282728	4.541643\\
0.7283728	4.5339\\
0.7284728	4.539281\\
0.7285729	4.542127\\
0.7286729	4.546469\\
0.7287729	4.545235\\
0.7288729	4.557422\\
0.7289729	4.569187\\
0.7290729	4.566056\\
0.7291729	4.562386\\
0.7292729	4.559464\\
0.7293729	4.565201\\
0.7294729	4.568215\\
0.729573	4.569879\\
0.729673	4.575018\\
0.729773	4.570596\\
0.729873	4.578779\\
0.729973	4.583879\\
0.730073	4.581514\\
0.730173	4.579564\\
0.730273	4.579817\\
0.730373	4.583121\\
0.730473	4.590531\\
0.7305731	4.610335\\
0.7306731	4.615174\\
0.7307731	4.617308\\
0.7308731	4.622607\\
0.7309731	4.621947\\
0.7310731	4.621003\\
0.7311731	4.631185\\
0.7312731	4.633984\\
0.7313731	4.639384\\
0.7314731	4.638166\\
0.7315732	4.640614\\
0.7316732	4.641786\\
0.7317732	4.635516\\
0.7318732	4.646382\\
0.7319732	4.64746\\
0.7320732	4.650187\\
0.7321732	4.661859\\
0.7322732	4.661374\\
0.7323732	4.658513\\
0.7324732	4.651562\\
0.7325733	4.660008\\
0.7326733	4.667672\\
0.7327733	4.675111\\
0.7328733	4.674344\\
0.7329733	4.66912\\
0.7330733	4.672419\\
0.7331733	4.684625\\
0.7332733	4.686282\\
0.7333733	4.691675\\
0.7334733	4.699868\\
0.7335734	4.697566\\
0.7336734	4.708287\\
0.7337734	4.705133\\
0.7338734	4.706059\\
0.7339734	4.71246\\
0.7340734	4.72082\\
0.7341734	4.714048\\
0.7342734	4.710091\\
0.7343734	4.715895\\
0.7344734	4.717295\\
0.7345735	4.721965\\
0.7346735	4.726518\\
0.7347735	4.729288\\
0.7348735	4.738807\\
0.7349735	4.74503\\
0.7350735	4.744575\\
0.7351735	4.754957\\
0.7352735	4.752439\\
0.7353735	4.762927\\
0.7354735	4.758663\\
0.7355736	4.75344\\
0.7356736	4.761734\\
0.7357736	4.765539\\
0.7358736	4.759725\\
0.7359736	4.757429\\
0.7360736	4.763031\\
0.7361736	4.774348\\
0.7362736	4.76594\\
0.7363736	4.768535\\
0.7364736	4.776087\\
0.7365737	4.773951\\
0.7366737	4.787548\\
0.7367737	4.789952\\
0.7368737	4.796718\\
0.7369737	4.791336\\
0.7370737	4.798711\\
0.7371737	4.813297\\
0.7372737	4.810041\\
0.7373737	4.810808\\
0.7374737	4.8105\\
0.7375738	4.80507\\
0.7376738	4.810031\\
0.7377738	4.809729\\
0.7378738	4.820243\\
0.7379738	4.81994\\
0.7380738	4.825928\\
0.7381738	4.836352\\
0.7382738	4.833796\\
0.7383738	4.841231\\
0.7384738	4.840637\\
0.7385739	4.83505\\
0.7386739	4.849559\\
0.7387739	4.847601\\
0.7388739	4.853646\\
0.7389739	4.851517\\
0.7390739	4.84782\\
0.7391739	4.849446\\
0.7392739	4.849924\\
0.7393739	4.852732\\
0.7394739	4.847632\\
0.739574	4.850399\\
0.739674	4.857447\\
0.739774	4.862271\\
0.739874	4.86834\\
0.739974	4.879731\\
0.740074	4.878824\\
0.740174	4.890949\\
0.740274	4.896448\\
0.740374	4.901688\\
0.740474	4.899769\\
0.7405741	4.890252\\
0.7406741	4.885781\\
0.7407741	4.884348\\
0.7408741	4.895983\\
0.7409741	4.9018\\
0.7410741	4.902435\\
0.7411741	4.913114\\
0.7412741	4.905208\\
0.7413741	4.906121\\
0.7414741	4.915163\\
0.7415742	4.915103\\
0.7416742	4.921077\\
0.7417742	4.91705\\
0.7418742	4.918798\\
0.7419742	4.920627\\
0.7420742	4.914549\\
0.7421742	4.924022\\
0.7422742	4.927615\\
0.7423742	4.93123\\
0.7424742	4.94756\\
0.7425743	4.946246\\
0.7426743	4.945402\\
0.7427743	4.951839\\
0.7428743	4.947426\\
0.7429743	4.953293\\
0.7430743	4.950142\\
0.7431743	4.953605\\
0.7432743	4.956637\\
0.7433743	4.95568\\
0.7434743	4.95839\\
0.7435744	4.946186\\
0.7436744	4.951733\\
0.7437744	4.96024\\
0.7438744	4.969379\\
0.7439744	4.96674\\
0.7440744	4.972082\\
0.7441744	4.983544\\
0.7442744	4.992637\\
0.7443744	4.989121\\
0.7444744	4.987443\\
0.7445745	4.99358\\
0.7446745	4.990733\\
0.7447745	4.987318\\
0.7448745	4.987054\\
0.7449745	4.993548\\
0.7450745	5.001074\\
0.7451745	5.002235\\
0.7452745	5.000871\\
0.7453745	5.00547\\
0.7454745	5.00648\\
0.7455746	5.000255\\
0.7456746	5.000417\\
0.7457746	5.004178\\
0.7458746	5.003756\\
0.7459746	5.011121\\
0.7460746	5.013501\\
0.7461746	5.022127\\
0.7462746	5.020599\\
0.7463746	5.018252\\
0.7464746	5.022114\\
0.7465747	5.026198\\
0.7466747	5.03191\\
0.7467747	5.036556\\
0.7468747	5.042319\\
0.7469747	5.038394\\
0.7470747	5.0329\\
0.7471747	5.035029\\
0.7472747	5.045218\\
0.7473747	5.051327\\
0.7474747	5.053994\\
0.7475748	5.055995\\
0.7476748	5.056655\\
0.7477748	5.056856\\
0.7478748	5.058495\\
0.7479748	5.060116\\
0.7480748	5.047642\\
0.7481748	5.051609\\
0.7482748	5.063069\\
0.7483748	5.066757\\
0.7484748	5.061536\\
0.7485749	5.059942\\
0.7486749	5.068923\\
0.7487749	5.077774\\
0.7488749	5.070724\\
0.7489749	5.070305\\
0.7490749	5.076809\\
0.7491749	5.071319\\
0.7492749	5.072598\\
0.7493749	5.071239\\
0.7494749	5.075203\\
0.749575	5.083594\\
0.749675	5.087298\\
0.749775	5.089448\\
0.749875	5.086873\\
0.749975	5.096038\\
0.750075	5.104466\\
0.750175	5.103355\\
0.750275	5.101627\\
0.750375	5.10748\\
0.750475	5.111573\\
0.7505751	5.117973\\
0.7506751	5.116394\\
0.7507751	5.112401\\
0.7508751	5.107166\\
0.7509751	5.109323\\
0.7510751	5.108583\\
0.7511751	5.099656\\
0.7512751	5.103828\\
0.7513751	5.109516\\
0.7514751	5.111215\\
0.7515752	5.103358\\
0.7516752	5.123746\\
0.7517752	5.130376\\
0.7518752	5.130084\\
0.7519752	5.134414\\
0.7520752	5.134838\\
0.7521752	5.136482\\
0.7522752	5.137357\\
0.7523752	5.131252\\
0.7524752	5.13708\\
0.7525753	5.134633\\
0.7526753	5.132821\\
0.7527753	5.138434\\
0.7528753	5.134486\\
0.7529753	5.130698\\
0.7530753	5.132672\\
0.7531753	5.136106\\
0.7532753	5.14139\\
0.7533753	5.149225\\
0.7534753	5.154754\\
0.7535754	5.16199\\
0.7536754	5.15798\\
0.7537754	5.159874\\
0.7538754	5.16429\\
0.7539754	5.163387\\
0.7540754	5.165892\\
0.7541754	5.169126\\
0.7542754	5.165012\\
0.7543754	5.172222\\
0.7544754	5.163855\\
0.7545755	5.168116\\
0.7546755	5.165271\\
0.7547755	5.161595\\
0.7548755	5.162235\\
0.7549755	5.164464\\
0.7550755	5.16241\\
0.7551755	5.166684\\
0.7552755	5.177057\\
0.7553755	5.176695\\
0.7554755	5.18099\\
0.7555756	5.176685\\
0.7556756	5.180903\\
0.7557756	5.183138\\
0.7558756	5.182544\\
0.7559756	5.177115\\
0.7560756	5.186979\\
0.7561756	5.17418\\
0.7562756	5.183817\\
0.7563756	5.194008\\
0.7564756	5.200644\\
0.7565757	5.213717\\
0.7566757	5.209905\\
0.7567757	5.205485\\
0.7568757	5.19879\\
0.7569757	5.19272\\
0.7570757	5.191683\\
0.7571757	5.192527\\
0.7572757	5.192933\\
0.7573757	5.205352\\
0.7574757	5.202451\\
0.7575758	5.203743\\
0.7576758	5.20444\\
0.7577758	5.202561\\
0.7578758	5.209427\\
0.7579758	5.210614\\
0.7580758	5.211975\\
0.7581758	5.212509\\
0.7582758	5.210332\\
0.7583758	5.213524\\
0.7584758	5.210281\\
0.7585759	5.208312\\
0.7586759	5.208817\\
0.7587759	5.208912\\
0.7588759	5.216676\\
0.7589759	5.218911\\
0.7590759	5.224156\\
0.7591759	5.218855\\
0.7592759	5.221847\\
0.7593759	5.22587\\
0.7594759	5.233633\\
0.759576	5.236051\\
0.759676	5.234096\\
0.759776	5.237146\\
0.759876	5.233038\\
0.759976	5.226958\\
0.760076	5.219289\\
0.760176	5.226951\\
0.760276	5.227862\\
0.760376	5.224107\\
0.760476	5.222347\\
0.7605761	5.225768\\
0.7606761	5.231453\\
0.7607761	5.241784\\
0.7608761	5.244479\\
0.7609761	5.24863\\
0.7610761	5.245136\\
0.7611761	5.236143\\
0.7612761	5.239609\\
0.7613761	5.239444\\
0.7614761	5.235409\\
0.7615762	5.239572\\
0.7616762	5.241534\\
0.7617762	5.246833\\
0.7618762	5.246929\\
0.7619762	5.239241\\
0.7620762	5.247464\\
0.7621762	5.247244\\
0.7622762	5.246477\\
0.7623762	5.248073\\
0.7624762	5.248654\\
0.7625763	5.240325\\
0.7626763	5.234622\\
0.7627763	5.238939\\
0.7628763	5.245291\\
0.7629763	5.251876\\
0.7630763	5.254658\\
0.7631763	5.264998\\
0.7632763	5.267495\\
0.7633763	5.255121\\
0.7634763	5.255176\\
0.7635764	5.26134\\
0.7636764	5.262342\\
0.7637764	5.263916\\
0.7638764	5.273375\\
0.7639764	5.275075\\
0.7640764	5.263073\\
0.7641764	5.254869\\
0.7642764	5.255018\\
0.7643764	5.261169\\
0.7644764	5.246561\\
0.7645765	5.238948\\
0.7646765	5.248919\\
0.7647765	5.252782\\
0.7648765	5.25049\\
0.7649765	5.252916\\
0.7650765	5.252105\\
0.7651765	5.266381\\
0.7652765	5.264939\\
0.7653765	5.261749\\
0.7654765	5.267578\\
0.7655766	5.274122\\
0.7656766	5.279559\\
0.7657766	5.276886\\
0.7658766	5.273054\\
0.7659766	5.279433\\
0.7660766	5.28338\\
0.7661766	5.267586\\
0.7662766	5.262684\\
0.7663766	5.268408\\
0.7664766	5.263054\\
0.7665767	5.261104\\
0.7666767	5.265709\\
0.7667767	5.260311\\
0.7668767	5.275259\\
0.7669767	5.276259\\
0.7670767	5.276802\\
0.7671767	5.270525\\
0.7672767	5.273614\\
0.7673767	5.274142\\
0.7674767	5.277801\\
0.7675768	5.276064\\
0.7676768	5.271909\\
0.7677768	5.265684\\
0.7678768	5.263947\\
0.7679768	5.264832\\
0.7680768	5.269703\\
0.7681768	5.270134\\
0.7682768	5.268352\\
0.7683768	5.277053\\
0.7684768	5.270372\\
0.7685769	5.271904\\
0.7686769	5.269004\\
0.7687769	5.273757\\
0.7688769	5.280713\\
0.7689769	5.284274\\
0.7690769	5.279162\\
0.7691769	5.28001\\
0.7692769	5.276794\\
0.7693769	5.271539\\
0.7694769	5.273153\\
0.769577	5.283211\\
0.769677	5.271867\\
0.769777	5.270712\\
0.769877	5.274681\\
0.769977	5.275899\\
0.770077	5.282004\\
0.770177	5.274838\\
0.770277	5.275913\\
0.770377	5.275825\\
0.770477	5.27885\\
0.7705771	5.280536\\
0.7706771	5.279096\\
0.7707771	5.278351\\
0.7708771	5.271515\\
0.7709771	5.271909\\
0.7710771	5.266672\\
0.7711771	5.264609\\
0.7712771	5.270506\\
0.7713771	5.263414\\
0.7714771	5.265679\\
0.7715772	5.27203\\
0.7716772	5.275669\\
0.7717772	5.275668\\
0.7718772	5.274891\\
0.7719772	5.274666\\
0.7720772	5.272458\\
0.7721772	5.272323\\
0.7722772	5.262293\\
0.7723772	5.266128\\
0.7724772	5.271836\\
0.7725773	5.272327\\
0.7726773	5.28545\\
0.7727773	5.283177\\
0.7728773	5.283307\\
0.7729773	5.275508\\
0.7730773	5.274131\\
0.7731773	5.275056\\
0.7732773	5.272164\\
0.7733773	5.268817\\
0.7734773	5.270656\\
0.7735774	5.273544\\
0.7736774	5.268932\\
0.7737774	5.259036\\
0.7738774	5.25027\\
0.7739774	5.253862\\
0.7740774	5.248965\\
0.7741774	5.251698\\
0.7742774	5.254614\\
0.7743774	5.263863\\
0.7744774	5.27163\\
0.7745775	5.269878\\
0.7746775	5.266745\\
0.7747775	5.2694\\
0.7748775	5.269385\\
0.7749775	5.268276\\
0.7750775	5.270831\\
0.7751775	5.277993\\
0.7752775	5.269765\\
0.7753775	5.260007\\
0.7754775	5.248362\\
0.7755776	5.246035\\
0.7756776	5.262647\\
0.7757776	5.265207\\
0.7758776	5.26015\\
0.7759776	5.25925\\
0.7760776	5.257671\\
0.7761776	5.260306\\
0.7762776	5.263929\\
0.7763776	5.251806\\
0.7764776	5.251655\\
0.7765777	5.250323\\
0.7766777	5.24188\\
0.7767777	5.237221\\
0.7768777	5.245403\\
0.7769777	5.253491\\
0.7770777	5.25422\\
0.7771777	5.255973\\
0.7772777	5.263664\\
0.7773777	5.261947\\
0.7774777	5.256894\\
0.7775778	5.246512\\
0.7776778	5.2465\\
0.7777778	5.251678\\
0.7778778	5.249365\\
0.7779778	5.239115\\
0.7780778	5.226839\\
0.7781778	5.229427\\
0.7782778	5.22886\\
0.7783778	5.239372\\
0.7784778	5.242652\\
0.7785779	5.249793\\
0.7786779	5.247694\\
0.7787779	5.2455\\
0.7788779	5.245172\\
0.7789779	5.241283\\
0.7790779	5.242563\\
0.7791779	5.247965\\
0.7792779	5.243907\\
0.7793779	5.234908\\
0.7794779	5.234457\\
0.779578	5.239866\\
0.779678	5.230377\\
0.779778	5.223437\\
0.779878	5.225455\\
0.779978	5.227877\\
0.780078	5.22238\\
0.780178	5.224458\\
0.780278	5.22749\\
0.780378	5.228883\\
0.780478	5.221107\\
0.7805781	5.221741\\
0.7806781	5.223622\\
0.7807781	5.222569\\
0.7808781	5.221561\\
0.7809781	5.212653\\
0.7810781	5.21498\\
0.7811781	5.220723\\
0.7812781	5.215162\\
0.7813781	5.20785\\
0.7814781	5.209793\\
0.7815782	5.211907\\
0.7816782	5.214852\\
0.7817782	5.214557\\
0.7818782	5.20833\\
0.7819782	5.206845\\
0.7820782	5.214805\\
0.7821782	5.214087\\
0.7822782	5.214269\\
0.7823782	5.218426\\
0.7824782	5.222288\\
0.7825783	5.211195\\
0.7826783	5.204207\\
0.7827783	5.194067\\
0.7828783	5.199141\\
0.7829783	5.197535\\
0.7830783	5.193047\\
0.7831783	5.185519\\
0.7832783	5.176452\\
0.7833783	5.180227\\
0.7834783	5.186291\\
0.7835784	5.186918\\
0.7836784	5.18642\\
0.7837784	5.188798\\
0.7838784	5.187682\\
0.7839784	5.186093\\
0.7840784	5.181061\\
0.7841784	5.179374\\
0.7842784	5.180587\\
0.7843784	5.180032\\
0.7844784	5.182236\\
0.7845785	5.186204\\
0.7846785	5.174153\\
0.7847785	5.171293\\
0.7848785	5.178777\\
0.7849785	5.169408\\
0.7850785	5.179074\\
0.7851785	5.175207\\
0.7852785	5.164074\\
0.7853785	5.158738\\
0.7854785	5.165118\\
0.7855786	5.161791\\
0.7856786	5.164959\\
0.7857786	5.16608\\
0.7858786	5.157327\\
0.7859786	5.161387\\
0.7860786	5.156178\\
0.7861786	5.157412\\
0.7862786	5.148716\\
0.7863786	5.14441\\
0.7864786	5.144836\\
0.7865787	5.148139\\
0.7866787	5.14986\\
0.7867787	5.155281\\
0.7868787	5.148569\\
0.7869787	5.143958\\
0.7870787	5.134239\\
0.7871787	5.136281\\
0.7872787	5.14071\\
0.7873787	5.127053\\
0.7874787	5.123994\\
0.7875788	5.130697\\
0.7876788	5.129363\\
0.7877788	5.130539\\
0.7878788	5.125267\\
0.7879788	5.121777\\
0.7880788	5.120881\\
0.7881788	5.125044\\
0.7882788	5.127434\\
0.7883788	5.132119\\
0.7884788	5.128978\\
0.7885789	5.12837\\
0.7886789	5.129996\\
0.7887789	5.122441\\
0.7888789	5.120479\\
0.7889789	5.110227\\
0.7890789	5.110756\\
0.7891789	5.109255\\
0.7892789	5.100824\\
0.7893789	5.088099\\
0.7894789	5.096179\\
0.789579	5.101177\\
0.789679	5.097591\\
0.789779	5.093754\\
0.789879	5.104802\\
0.789979	5.09777\\
0.790079	5.092387\\
0.790179	5.081072\\
0.790279	5.076655\\
0.790379	5.072761\\
0.790479	5.080632\\
0.7905791	5.081428\\
0.7906791	5.07552\\
0.7907791	5.096139\\
0.7908791	5.091008\\
0.7909791	5.073558\\
0.7910791	5.070804\\
0.7911791	5.082318\\
0.7912791	5.089214\\
0.7913791	5.076487\\
0.7914791	5.07329\\
0.7915792	5.06629\\
0.7916792	5.066346\\
0.7917792	5.064942\\
0.7918792	5.058299\\
0.7919792	5.048516\\
0.7920792	5.051519\\
0.7921792	5.054131\\
0.7922792	5.04333\\
0.7923792	5.043635\\
0.7924792	5.053214\\
0.7925793	5.050573\\
0.7926793	5.049444\\
0.7927793	5.043116\\
0.7928793	5.034127\\
0.7929793	5.032113\\
0.7930793	5.033317\\
0.7931793	5.024874\\
0.7932793	5.016177\\
0.7933793	5.022173\\
0.7934793	5.021684\\
0.7935794	5.020764\\
0.7936794	5.018004\\
0.7937794	5.022247\\
0.7938794	5.023074\\
0.7939794	5.029676\\
0.7940794	5.033528\\
0.7941794	5.034739\\
0.7942794	5.020663\\
0.7943794	5.011629\\
0.7944794	5.008159\\
0.7945795	4.990606\\
0.7946795	4.99366\\
0.7947795	4.991526\\
0.7948795	4.986215\\
0.7949795	4.984698\\
0.7950795	4.990356\\
0.7951795	4.987581\\
0.7952795	4.985184\\
0.7953795	4.987134\\
0.7954795	4.984527\\
0.7955796	4.98971\\
0.7956796	4.987204\\
0.7957796	4.982591\\
0.7958796	4.979442\\
0.7959796	4.969281\\
0.7960796	4.963403\\
0.7961796	4.961303\\
0.7962796	4.964858\\
0.7963796	4.9643\\
0.7964796	4.95604\\
0.7965797	4.950017\\
0.7966797	4.954711\\
0.7967797	4.951219\\
0.7968797	4.940408\\
0.7969797	4.937931\\
0.7970797	4.945693\\
0.7971797	4.941993\\
0.7972797	4.943857\\
0.7973797	4.940986\\
0.7974797	4.935172\\
0.7975798	4.932199\\
0.7976798	4.936708\\
0.7977798	4.942373\\
0.7978798	4.928418\\
0.7979798	4.927252\\
0.7980798	4.933146\\
0.7981798	4.92472\\
0.7982798	4.918255\\
0.7983798	4.923241\\
0.7984798	4.919042\\
0.7985799	4.915661\\
0.7986799	4.908262\\
0.7987799	4.899206\\
0.7988799	4.890598\\
0.7989799	4.895001\\
0.7990799	4.897711\\
0.7991799	4.893785\\
0.7992799	4.893384\\
0.7993799	4.884705\\
0.7994799	4.888211\\
0.79958	4.878143\\
0.79968	4.870626\\
0.79978	4.868787\\
0.79988	4.873916\\
0.79998	4.876792\\
0.80008	4.878103\\
};
\addplot [color=mycolor1,solid]
  table[row sep=crcr]{%
0.80008	4.878103\\
0.80018	4.873618\\
0.80028	4.874526\\
0.80038	4.871356\\
0.80048	4.875637\\
0.8005801	4.864094\\
0.8006801	4.870974\\
0.8007801	4.86083\\
0.8008801	4.858067\\
0.8009801	4.851789\\
0.8010801	4.851776\\
0.8011801	4.84409\\
0.8012801	4.841027\\
0.8013801	4.836053\\
0.8014801	4.836606\\
0.8015802	4.823674\\
0.8016802	4.823836\\
0.8017802	4.834706\\
0.8018802	4.838027\\
0.8019802	4.830105\\
0.8020802	4.824568\\
0.8021802	4.819908\\
0.8022802	4.815425\\
0.8023802	4.811427\\
0.8024802	4.810978\\
0.8025803	4.796663\\
0.8026803	4.803308\\
0.8027803	4.804256\\
0.8028803	4.800649\\
0.8029803	4.801146\\
0.8030803	4.794329\\
0.8031803	4.796141\\
0.8032803	4.789794\\
0.8033803	4.786587\\
0.8034803	4.775432\\
0.8035804	4.778514\\
0.8036804	4.77687\\
0.8037804	4.779925\\
0.8038804	4.776208\\
0.8039804	4.76617\\
0.8040804	4.764479\\
0.8041804	4.770375\\
0.8042804	4.767296\\
0.8043804	4.768165\\
0.8044804	4.755043\\
0.8045805	4.75314\\
0.8046805	4.742291\\
0.8047805	4.743092\\
0.8048805	4.739957\\
0.8049805	4.731822\\
0.8050805	4.739598\\
0.8051805	4.747065\\
0.8052805	4.742538\\
0.8053805	4.735934\\
0.8054805	4.731642\\
0.8055806	4.72211\\
0.8056806	4.705352\\
0.8057806	4.701381\\
0.8058806	4.702959\\
0.8059806	4.699802\\
0.8060806	4.70134\\
0.8061806	4.706942\\
0.8062806	4.701519\\
0.8063806	4.697845\\
0.8064806	4.694479\\
0.8065807	4.68258\\
0.8066807	4.684798\\
0.8067807	4.681518\\
0.8068807	4.691723\\
0.8069807	4.67812\\
0.8070807	4.67444\\
0.8071807	4.677032\\
0.8072807	4.679046\\
0.8073807	4.679081\\
0.8074807	4.668965\\
0.8075808	4.672206\\
0.8076808	4.670552\\
0.8077808	4.664511\\
0.8078808	4.647533\\
0.8079808	4.650224\\
0.8080808	4.65179\\
0.8081808	4.64503\\
0.8082808	4.626654\\
0.8083808	4.619654\\
0.8084808	4.613345\\
0.8085809	4.611182\\
0.8086809	4.610572\\
0.8087809	4.613476\\
0.8088809	4.612522\\
0.8089809	4.612629\\
0.8090809	4.613532\\
0.8091809	4.617177\\
0.8092809	4.61561\\
0.8093809	4.618137\\
0.8094809	4.615001\\
0.809581	4.609225\\
0.809681	4.604371\\
0.809781	4.593267\\
0.809881	4.580121\\
0.809981	4.57879\\
0.810081	4.584546\\
0.810181	4.57647\\
0.810281	4.57653\\
0.810381	4.575714\\
0.810481	4.57408\\
0.8105811	4.575027\\
0.8106811	4.572612\\
0.8107811	4.564837\\
0.8108811	4.551203\\
0.8109811	4.552328\\
0.8110811	4.547364\\
0.8111811	4.543144\\
0.8112811	4.544531\\
0.8113811	4.532618\\
0.8114811	4.529358\\
0.8115812	4.537885\\
0.8116812	4.529953\\
0.8117812	4.525544\\
0.8118812	4.530755\\
0.8119812	4.527093\\
0.8120812	4.523653\\
0.8121812	4.514404\\
0.8122812	4.514388\\
0.8123812	4.510954\\
0.8124812	4.503273\\
0.8125813	4.501843\\
0.8126813	4.508354\\
0.8127813	4.492769\\
0.8128813	4.482177\\
0.8129813	4.478635\\
0.8130813	4.477191\\
0.8131813	4.480196\\
0.8132813	4.487443\\
0.8133813	4.488246\\
0.8134813	4.478671\\
0.8135814	4.480203\\
0.8136814	4.475641\\
0.8137814	4.475924\\
0.8138814	4.457948\\
0.8139814	4.455375\\
0.8140814	4.453981\\
0.8141814	4.450111\\
0.8142814	4.446436\\
0.8143814	4.44239\\
0.8144814	4.441948\\
0.8145815	4.433042\\
0.8146815	4.434859\\
0.8147815	4.427751\\
0.8148815	4.417084\\
0.8149815	4.408793\\
0.8150815	4.414931\\
0.8151815	4.409974\\
0.8152815	4.406228\\
0.8153815	4.400054\\
0.8154815	4.39551\\
0.8155816	4.405015\\
0.8156816	4.394336\\
0.8157816	4.395609\\
0.8158816	4.396588\\
0.8159816	4.389096\\
0.8160816	4.382223\\
0.8161816	4.378181\\
0.8162816	4.379161\\
0.8163816	4.372965\\
0.8164816	4.368213\\
0.8165817	4.362915\\
0.8166817	4.365446\\
0.8167817	4.357964\\
0.8168817	4.356794\\
0.8169817	4.355907\\
0.8170817	4.352059\\
0.8171817	4.346632\\
0.8172817	4.342344\\
0.8173817	4.340168\\
0.8174817	4.328898\\
0.8175818	4.317519\\
0.8176818	4.31\\
0.8177818	4.31264\\
0.8178818	4.306801\\
0.8179818	4.302254\\
0.8180818	4.29936\\
0.8181818	4.300037\\
0.8182818	4.294979\\
0.8183818	4.298907\\
0.8184818	4.300905\\
0.8185819	4.287262\\
0.8186819	4.286963\\
0.8187819	4.285293\\
0.8188819	4.279449\\
0.8189819	4.275777\\
0.8190819	4.274104\\
0.8191819	4.274925\\
0.8192819	4.275304\\
0.8193819	4.268248\\
0.8194819	4.253091\\
0.819582	4.247425\\
0.819682	4.247745\\
0.819782	4.242469\\
0.819882	4.235025\\
0.819982	4.232933\\
0.820082	4.229545\\
0.820182	4.221194\\
0.820282	4.228131\\
0.820382	4.224373\\
0.820482	4.219203\\
0.8205821	4.20917\\
0.8206821	4.2142\\
0.8207821	4.221778\\
0.8208821	4.209327\\
0.8209821	4.207256\\
0.8210821	4.198095\\
0.8211821	4.194136\\
0.8212821	4.184299\\
0.8213821	4.185509\\
0.8214821	4.187957\\
0.8215822	4.175741\\
0.8216822	4.170787\\
0.8217822	4.163868\\
0.8218822	4.166048\\
0.8219822	4.15779\\
0.8220822	4.146683\\
0.8221822	4.150372\\
0.8222822	4.150719\\
0.8223822	4.149603\\
0.8224822	4.15069\\
0.8225823	4.154679\\
0.8226823	4.141663\\
0.8227823	4.139334\\
0.8228823	4.136442\\
0.8229823	4.13799\\
0.8230823	4.126764\\
0.8231823	4.127003\\
0.8232823	4.123665\\
0.8233823	4.116099\\
0.8234823	4.112761\\
0.8235824	4.108855\\
0.8236824	4.104544\\
0.8237824	4.094178\\
0.8238824	4.083632\\
0.8239824	4.086537\\
0.8240824	4.087061\\
0.8241824	4.088561\\
0.8242824	4.075168\\
0.8243824	4.061765\\
0.8244824	4.064005\\
0.8245825	4.070352\\
0.8246825	4.064376\\
0.8247825	4.054147\\
0.8248825	4.055713\\
0.8249825	4.048797\\
0.8250825	4.048338\\
0.8251825	4.042935\\
0.8252825	4.039315\\
0.8253825	4.039162\\
0.8254825	4.0309\\
0.8255826	4.031783\\
0.8256826	4.036314\\
0.8257826	4.037809\\
0.8258826	4.032251\\
0.8259826	4.022963\\
0.8260826	4.014775\\
0.8261826	4.010283\\
0.8262826	4.000323\\
0.8263826	3.98522\\
0.8264826	3.982833\\
0.8265827	3.989582\\
0.8266827	3.988936\\
0.8267827	3.984166\\
0.8268827	3.974711\\
0.8269827	3.970713\\
0.8270827	3.969138\\
0.8271827	3.962844\\
0.8272827	3.959843\\
0.8273827	3.95478\\
0.8274827	3.950741\\
0.8275828	3.94533\\
0.8276828	3.940043\\
0.8277828	3.929888\\
0.8278828	3.931298\\
0.8279828	3.93064\\
0.8280828	3.924908\\
0.8281828	3.922695\\
0.8282828	3.918296\\
0.8283828	3.911898\\
0.8284828	3.911754\\
0.8285829	3.915997\\
0.8286829	3.913999\\
0.8287829	3.906387\\
0.8288829	3.904229\\
0.8289829	3.895823\\
0.8290829	3.882671\\
0.8291829	3.878801\\
0.8292829	3.870963\\
0.8293829	3.872118\\
0.8294829	3.868671\\
0.829583	3.855577\\
0.829683	3.862895\\
0.829783	3.854484\\
0.829883	3.854663\\
0.829983	3.845576\\
0.830083	3.846869\\
0.830183	3.847141\\
0.830283	3.840363\\
0.830383	3.833777\\
0.830483	3.821594\\
0.8305831	3.819673\\
0.8306831	3.816888\\
0.8307831	3.817719\\
0.8308831	3.81306\\
0.8309831	3.802498\\
0.8310831	3.798656\\
0.8311831	3.799318\\
0.8312831	3.78799\\
0.8313831	3.785122\\
0.8314831	3.775114\\
0.8315832	3.773168\\
0.8316832	3.774879\\
0.8317832	3.774698\\
0.8318832	3.778347\\
0.8319832	3.771858\\
0.8320832	3.768886\\
0.8321832	3.768546\\
0.8322832	3.763321\\
0.8323832	3.761539\\
0.8324832	3.754017\\
0.8325833	3.751557\\
0.8326833	3.742116\\
0.8327833	3.740628\\
0.8328833	3.740841\\
0.8329833	3.734344\\
0.8330833	3.720077\\
0.8331833	3.707525\\
0.8332833	3.704369\\
0.8333833	3.699602\\
0.8334833	3.693047\\
0.8335834	3.695847\\
0.8336834	3.701851\\
0.8337834	3.696768\\
0.8338834	3.696163\\
0.8339834	3.680309\\
0.8340834	3.685184\\
0.8341834	3.680746\\
0.8342834	3.677208\\
0.8343834	3.671694\\
0.8344834	3.670013\\
0.8345835	3.675449\\
0.8346835	3.668924\\
0.8347835	3.658408\\
0.8348835	3.653471\\
0.8349835	3.651567\\
0.8350835	3.649777\\
0.8351835	3.646601\\
0.8352835	3.646401\\
0.8353835	3.639028\\
0.8354835	3.630239\\
0.8355836	3.622452\\
0.8356836	3.61516\\
0.8357836	3.612431\\
0.8358836	3.614376\\
0.8359836	3.60784\\
0.8360836	3.605224\\
0.8361836	3.602458\\
0.8362836	3.599142\\
0.8363836	3.599653\\
0.8364836	3.589186\\
0.8365837	3.578038\\
0.8366837	3.571776\\
0.8367837	3.571275\\
0.8368837	3.560349\\
0.8369837	3.555522\\
0.8370837	3.57167\\
0.8371837	3.561449\\
0.8372837	3.557187\\
0.8373837	3.553379\\
0.8374837	3.550301\\
0.8375838	3.550233\\
0.8376838	3.538939\\
0.8377838	3.53461\\
0.8378838	3.530581\\
0.8379838	3.523601\\
0.8380838	3.517375\\
0.8381838	3.517938\\
0.8382838	3.511303\\
0.8383838	3.510529\\
0.8384838	3.500612\\
0.8385839	3.499788\\
0.8386839	3.489481\\
0.8387839	3.491642\\
0.8388839	3.48737\\
0.8389839	3.481693\\
0.8390839	3.478377\\
0.8391839	3.473451\\
0.8392839	3.47433\\
0.8393839	3.462365\\
0.8394839	3.452017\\
0.839584	3.453971\\
0.839684	3.450445\\
0.839784	3.443733\\
0.839884	3.43572\\
0.839984	3.428434\\
0.840084	3.429884\\
0.840184	3.427557\\
0.840284	3.424442\\
0.840384	3.41293\\
0.840484	3.414645\\
0.8405841	3.412145\\
0.8406841	3.397344\\
0.8407841	3.394051\\
0.8408841	3.393978\\
0.8409841	3.391167\\
0.8410841	3.387095\\
0.8411841	3.381119\\
0.8412841	3.377579\\
0.8413841	3.376414\\
0.8414841	3.371917\\
0.8415842	3.365417\\
0.8416842	3.363939\\
0.8417842	3.36852\\
0.8418842	3.361416\\
0.8419842	3.35735\\
0.8420842	3.35551\\
0.8421842	3.348348\\
0.8422842	3.341196\\
0.8423842	3.331511\\
0.8424842	3.324088\\
0.8425843	3.317853\\
0.8426843	3.310048\\
0.8427843	3.310164\\
0.8428843	3.302688\\
0.8429843	3.296773\\
0.8430843	3.289137\\
0.8431843	3.282474\\
0.8432843	3.283477\\
0.8433843	3.282543\\
0.8434843	3.285193\\
0.8435844	3.296127\\
0.8436844	3.290084\\
0.8437844	3.282544\\
0.8438844	3.280046\\
0.8439844	3.270203\\
0.8440844	3.273066\\
0.8441844	3.261679\\
0.8442844	3.253952\\
0.8443844	3.251637\\
0.8444844	3.241164\\
0.8445845	3.238763\\
0.8446845	3.229697\\
0.8447845	3.231682\\
0.8448845	3.239057\\
0.8449845	3.234622\\
0.8450845	3.226544\\
0.8451845	3.220142\\
0.8452845	3.215996\\
0.8453845	3.211614\\
0.8454845	3.205707\\
0.8455846	3.19743\\
0.8456846	3.19081\\
0.8457846	3.194564\\
0.8458846	3.19371\\
0.8459846	3.188239\\
0.8460846	3.192855\\
0.8461846	3.182555\\
0.8462846	3.178053\\
0.8463846	3.169414\\
0.8464846	3.165369\\
0.8465847	3.16167\\
0.8466847	3.151719\\
0.8467847	3.152114\\
0.8468847	3.145183\\
0.8469847	3.138132\\
0.8470847	3.143564\\
0.8471847	3.139572\\
0.8472847	3.135945\\
0.8473847	3.13502\\
0.8474847	3.128467\\
0.8475848	3.13333\\
0.8476848	3.129076\\
0.8477848	3.114756\\
0.8478848	3.10924\\
0.8479848	3.104429\\
0.8480848	3.103714\\
0.8481848	3.094404\\
0.8482848	3.088269\\
0.8483848	3.093604\\
0.8484848	3.091753\\
0.8485849	3.084613\\
0.8486849	3.073309\\
0.8487849	3.068749\\
0.8488849	3.064725\\
0.8489849	3.062407\\
0.8490849	3.055982\\
0.8491849	3.054896\\
0.8492849	3.05651\\
0.8493849	3.046572\\
0.8494849	3.029152\\
0.849585	3.020453\\
0.849685	3.022565\\
0.849785	3.025691\\
0.849885	3.021365\\
0.849985	3.011507\\
0.850085	3.007059\\
0.850185	3.005393\\
0.850285	3.000109\\
0.850385	2.998759\\
0.850485	2.997553\\
0.8505851	3.001427\\
0.8506851	2.998779\\
0.8507851	2.984896\\
0.8508851	2.980527\\
0.8509851	2.975715\\
0.8510851	2.971071\\
0.8511851	2.967703\\
0.8512851	2.963637\\
0.8513851	2.956763\\
0.8514851	2.951823\\
0.8515852	2.943596\\
0.8516852	2.942221\\
0.8517852	2.939379\\
0.8518852	2.926331\\
0.8519852	2.915204\\
0.8520852	2.911664\\
0.8521852	2.905952\\
0.8522852	2.900858\\
0.8523852	2.900376\\
0.8524852	2.90139\\
0.8525853	2.899185\\
0.8526853	2.899472\\
0.8527853	2.891262\\
0.8528853	2.886866\\
0.8529853	2.880101\\
0.8530853	2.881196\\
0.8531853	2.88436\\
0.8532853	2.869487\\
0.8533853	2.868128\\
0.8534853	2.868423\\
0.8535854	2.864109\\
0.8536854	2.85777\\
0.8537854	2.856812\\
0.8538854	2.844608\\
0.8539854	2.840859\\
0.8540854	2.8387\\
0.8541854	2.832533\\
0.8542854	2.828811\\
0.8543854	2.828826\\
0.8544854	2.821981\\
0.8545855	2.809347\\
0.8546855	2.804533\\
0.8547855	2.798234\\
0.8548855	2.79916\\
0.8549855	2.798337\\
0.8550855	2.796641\\
0.8551855	2.798683\\
0.8552855	2.796768\\
0.8553855	2.788783\\
0.8554855	2.786918\\
0.8555856	2.782488\\
0.8556856	2.780721\\
0.8557856	2.778804\\
0.8558856	2.767967\\
0.8559856	2.756803\\
0.8560856	2.751234\\
0.8561856	2.741417\\
0.8562856	2.740304\\
0.8563856	2.739503\\
0.8564856	2.738617\\
0.8565857	2.737679\\
0.8566857	2.737075\\
0.8567857	2.738527\\
0.8568857	2.737614\\
0.8569857	2.7381\\
0.8570857	2.729461\\
0.8571857	2.724889\\
0.8572857	2.722148\\
0.8573857	2.717997\\
0.8574857	2.714114\\
0.8575858	2.70491\\
0.8576858	2.699932\\
0.8577858	2.699018\\
0.8578858	2.687425\\
0.8579858	2.68185\\
0.8580858	2.675258\\
0.8581858	2.673003\\
0.8582858	2.667207\\
0.8583858	2.671942\\
0.8584858	2.666914\\
0.8585859	2.65943\\
0.8586859	2.66044\\
0.8587859	2.653272\\
0.8588859	2.650784\\
0.8589859	2.653817\\
0.8590859	2.645521\\
0.8591859	2.63689\\
0.8592859	2.629207\\
0.8593859	2.625091\\
0.8594859	2.622815\\
0.859586	2.619454\\
0.859686	2.61861\\
0.859786	2.614652\\
0.859886	2.615955\\
0.859986	2.611954\\
0.860086	2.606836\\
0.860186	2.605143\\
0.860286	2.603377\\
0.860386	2.596855\\
0.860486	2.595165\\
0.8605861	2.583291\\
0.8606861	2.568657\\
0.8607861	2.565094\\
0.8608861	2.560488\\
0.8609861	2.556028\\
0.8610861	2.554485\\
0.8611861	2.55113\\
0.8612861	2.545421\\
0.8613861	2.542437\\
0.8614861	2.536971\\
0.8615862	2.525265\\
0.8616862	2.523942\\
0.8617862	2.523563\\
0.8618862	2.521335\\
0.8619862	2.520997\\
0.8620862	2.515803\\
0.8621862	2.507956\\
0.8622862	2.49954\\
0.8623862	2.495969\\
0.8624862	2.494555\\
0.8625863	2.492426\\
0.8626863	2.49512\\
0.8627863	2.488797\\
0.8628863	2.481195\\
0.8629863	2.47895\\
0.8630863	2.473189\\
0.8631863	2.465607\\
0.8632863	2.457537\\
0.8633863	2.449626\\
0.8634863	2.439332\\
0.8635864	2.435651\\
0.8636864	2.434823\\
0.8637864	2.436766\\
0.8638864	2.434638\\
0.8639864	2.435256\\
0.8640864	2.431997\\
0.8641864	2.422584\\
0.8642864	2.414528\\
0.8643864	2.416145\\
0.8644864	2.41527\\
0.8645865	2.405897\\
0.8646865	2.403108\\
0.8647865	2.398449\\
0.8648865	2.395118\\
0.8649865	2.381681\\
0.8650865	2.377395\\
0.8651865	2.370823\\
0.8652865	2.370452\\
0.8653865	2.37672\\
0.8654865	2.371683\\
0.8655866	2.362861\\
0.8656866	2.360522\\
0.8657866	2.355718\\
0.8658866	2.349789\\
0.8659866	2.353098\\
0.8660866	2.349729\\
0.8661866	2.339945\\
0.8662866	2.340912\\
0.8663866	2.339246\\
0.8664866	2.335343\\
0.8665867	2.337949\\
0.8666867	2.337117\\
0.8667867	2.32667\\
0.8668867	2.324371\\
0.8669867	2.327657\\
0.8670867	2.323847\\
0.8671867	2.308767\\
0.8672867	2.300541\\
0.8673867	2.297343\\
0.8674867	2.292114\\
0.8675868	2.291773\\
0.8676868	2.28758\\
0.8677868	2.282991\\
0.8678868	2.272177\\
0.8679868	2.270235\\
0.8680868	2.269827\\
0.8681868	2.272382\\
0.8682868	2.269096\\
0.8683868	2.263196\\
0.8684868	2.266911\\
0.8685869	2.26819\\
0.8686869	2.265887\\
0.8687869	2.260409\\
0.8688869	2.249544\\
0.8689869	2.248864\\
0.8690869	2.249319\\
0.8691869	2.250039\\
0.8692869	2.244212\\
0.8693869	2.231029\\
0.8694869	2.228346\\
0.869587	2.227964\\
0.869687	2.224018\\
0.869787	2.216826\\
0.869887	2.210264\\
0.869987	2.208549\\
0.870087	2.205151\\
0.870187	2.200889\\
0.870287	2.198248\\
0.870387	2.195015\\
0.870487	2.196989\\
0.8705871	2.191526\\
0.8706871	2.186179\\
0.8707871	2.181774\\
0.8708871	2.178378\\
0.8709871	2.168371\\
0.8710871	2.161407\\
0.8711871	2.157672\\
0.8712871	2.154619\\
0.8713871	2.149966\\
0.8714871	2.145201\\
0.8715872	2.138645\\
0.8716872	2.140174\\
0.8717872	2.143108\\
0.8718872	2.139761\\
0.8719872	2.141375\\
0.8720872	2.133677\\
0.8721872	2.131441\\
0.8722872	2.126201\\
0.8723872	2.119572\\
0.8724872	2.11044\\
0.8725873	2.104339\\
0.8726873	2.091362\\
0.8727873	2.089321\\
0.8728873	2.090929\\
0.8729873	2.086601\\
0.8730873	2.079437\\
0.8731873	2.074735\\
0.8732873	2.075033\\
0.8733873	2.074836\\
0.8734873	2.07289\\
0.8735874	2.067436\\
0.8736874	2.064957\\
0.8737874	2.058614\\
0.8738874	2.053761\\
0.8739874	2.0421\\
0.8740874	2.040891\\
0.8741874	2.033306\\
0.8742874	2.027216\\
0.8743874	2.023673\\
0.8744874	2.017367\\
0.8745875	2.013911\\
0.8746875	2.0074\\
0.8747875	2.000788\\
0.8748875	1.999154\\
0.8749875	1.998941\\
0.8750875	2.003445\\
0.8751875	2.003199\\
0.8752875	1.99806\\
0.8753875	1.995033\\
0.8754875	1.985907\\
0.8755876	1.985114\\
0.8756876	1.979775\\
0.8757876	1.978245\\
0.8758876	1.976562\\
0.8759876	1.97077\\
0.8760876	1.963243\\
0.8761876	1.962225\\
0.8762876	1.959447\\
0.8763876	1.950462\\
0.8764876	1.943445\\
0.8765877	1.939932\\
0.8766877	1.940005\\
0.8767877	1.932583\\
0.8768877	1.929635\\
0.8769877	1.927057\\
0.8770877	1.920178\\
0.8771877	1.913627\\
0.8772877	1.916839\\
0.8773877	1.914373\\
0.8774877	1.907785\\
0.8775878	1.907515\\
0.8776878	1.906175\\
0.8777878	1.903687\\
0.8778878	1.902144\\
0.8779878	1.904139\\
0.8780878	1.901388\\
0.8781878	1.90517\\
0.8782878	1.902346\\
0.8783878	1.895882\\
0.8784878	1.885476\\
0.8785879	1.882971\\
0.8786879	1.879242\\
0.8787879	1.873136\\
0.8788879	1.865173\\
0.8789879	1.866133\\
0.8790879	1.861782\\
0.8791879	1.861333\\
0.8792879	1.864839\\
0.8793879	1.861546\\
0.8794879	1.856366\\
0.879588	1.85374\\
0.879688	1.843076\\
0.879788	1.837638\\
0.879888	1.83909\\
0.879988	1.83588\\
0.880088	1.839976\\
0.880188	1.833028\\
0.880288	1.829203\\
0.880388	1.828244\\
0.880488	1.826555\\
0.8805881	1.817755\\
0.8806881	1.81855\\
0.8807881	1.818161\\
0.8808881	1.815706\\
0.8809881	1.807618\\
0.8810881	1.802524\\
0.8811881	1.796976\\
0.8812881	1.792746\\
0.8813881	1.788162\\
0.8814881	1.7924\\
0.8815882	1.794524\\
0.8816882	1.789343\\
0.8817882	1.781764\\
0.8818882	1.779194\\
0.8819882	1.780578\\
0.8820882	1.775712\\
0.8821882	1.770839\\
0.8822882	1.769153\\
0.8823882	1.761009\\
0.8824882	1.757169\\
0.8825883	1.755462\\
0.8826883	1.747069\\
0.8827883	1.74139\\
0.8828883	1.734497\\
0.8829883	1.730402\\
0.8830883	1.728681\\
0.8831883	1.72447\\
0.8832883	1.722909\\
0.8833883	1.723046\\
0.8834883	1.719771\\
0.8835884	1.714471\\
0.8836884	1.706806\\
0.8837884	1.705263\\
0.8838884	1.700475\\
0.8839884	1.693521\\
0.8840884	1.692169\\
0.8841884	1.687292\\
0.8842884	1.679367\\
0.8843884	1.677452\\
0.8844884	1.679191\\
0.8845885	1.672329\\
0.8846885	1.668251\\
0.8847885	1.671253\\
0.8848885	1.664509\\
0.8849885	1.656586\\
0.8850885	1.653565\\
0.8851885	1.658909\\
0.8852885	1.65085\\
0.8853885	1.642095\\
0.8854885	1.643423\\
0.8855886	1.639663\\
0.8856886	1.63566\\
0.8857886	1.629969\\
0.8858886	1.618665\\
0.8859886	1.612399\\
0.8860886	1.60667\\
0.8861886	1.599122\\
0.8862886	1.593206\\
0.8863886	1.593614\\
0.8864886	1.587861\\
0.8865887	1.586465\\
0.8866887	1.589285\\
0.8867887	1.58837\\
0.8868887	1.589099\\
0.8869887	1.589535\\
0.8870887	1.588722\\
0.8871887	1.585588\\
0.8872887	1.578387\\
0.8873887	1.57433\\
0.8874887	1.567882\\
0.8875888	1.560081\\
0.8876888	1.560421\\
0.8877888	1.559398\\
0.8878888	1.553131\\
0.8879888	1.541302\\
0.8880888	1.540159\\
0.8881888	1.539265\\
0.8882888	1.541722\\
0.8883888	1.542628\\
0.8884888	1.540518\\
0.8885889	1.535673\\
0.8886889	1.533156\\
0.8887889	1.525621\\
0.8888889	1.525911\\
0.8889889	1.525045\\
0.8890889	1.519704\\
0.8891889	1.51523\\
0.8892889	1.513014\\
0.8893889	1.511925\\
0.8894889	1.509991\\
0.889589	1.508138\\
0.889689	1.505904\\
0.889789	1.502484\\
0.889889	1.495429\\
0.889989	1.496269\\
0.890089	1.493355\\
0.890189	1.492089\\
0.890289	1.491491\\
0.890389	1.49073\\
0.890489	1.48847\\
0.8905891	1.485113\\
0.8906891	1.479569\\
0.8907891	1.474569\\
0.8908891	1.471505\\
0.8909891	1.472703\\
0.8910891	1.473286\\
0.8911891	1.474613\\
0.8912891	1.46939\\
0.8913891	1.468292\\
0.8914891	1.469368\\
0.8915892	1.465388\\
0.8916892	1.460502\\
0.8917892	1.458115\\
0.8918892	1.453444\\
0.8919892	1.44241\\
0.8920892	1.442992\\
0.8921892	1.446967\\
0.8922892	1.445479\\
0.8923892	1.434092\\
0.8924892	1.427135\\
0.8925893	1.426686\\
0.8926893	1.425794\\
0.8927893	1.422154\\
0.8928893	1.41917\\
0.8929893	1.417936\\
0.8930893	1.414363\\
0.8931893	1.41073\\
0.8932893	1.407481\\
0.8933893	1.410388\\
0.8934893	1.407095\\
0.8935894	1.404567\\
0.8936894	1.401161\\
0.8937894	1.39762\\
0.8938894	1.395747\\
0.8939894	1.389108\\
0.8940894	1.384511\\
0.8941894	1.385356\\
0.8942894	1.385274\\
0.8943894	1.376639\\
0.8944894	1.367855\\
0.8945895	1.368286\\
0.8946895	1.365335\\
0.8947895	1.35981\\
0.8948895	1.356733\\
0.8949895	1.355559\\
0.8950895	1.3511\\
0.8951895	1.343391\\
0.8952895	1.335283\\
0.8953895	1.332139\\
0.8954895	1.330325\\
0.8955896	1.327585\\
0.8956896	1.323125\\
0.8957896	1.319955\\
0.8958896	1.314535\\
0.8959896	1.310078\\
0.8960896	1.309509\\
0.8961896	1.305247\\
0.8962896	1.303639\\
0.8963896	1.300436\\
0.8964896	1.29597\\
0.8965897	1.291464\\
0.8966897	1.293713\\
0.8967897	1.290788\\
0.8968897	1.287791\\
0.8969897	1.285375\\
0.8970897	1.284398\\
0.8971897	1.277984\\
0.8972897	1.267439\\
0.8973897	1.263099\\
0.8974897	1.26174\\
0.8975898	1.256862\\
0.8976898	1.251724\\
0.8977898	1.24865\\
0.8978898	1.242536\\
0.8979898	1.240578\\
0.8980898	1.238516\\
0.8981898	1.23367\\
0.8982898	1.236753\\
0.8983898	1.23624\\
0.8984898	1.237433\\
0.8985899	1.234912\\
0.8986899	1.230843\\
0.8987899	1.227476\\
0.8988899	1.220383\\
0.8989899	1.216013\\
0.8990899	1.208903\\
0.8991899	1.207426\\
0.8992899	1.205099\\
0.8993899	1.200968\\
0.8994899	1.194863\\
0.89959	1.195005\\
0.89969	1.197629\\
0.89979	1.197801\\
0.89989	1.198098\\
0.89999	1.196465\\
0.90009	1.193345\\
0.90019	1.192487\\
0.90029	1.189705\\
0.90039	1.188076\\
0.90049	1.185967\\
0.9005901	1.187357\\
0.9006901	1.186517\\
0.9007901	1.184437\\
0.9008901	1.182052\\
0.9009901	1.175888\\
0.9010901	1.174679\\
0.9011901	1.168025\\
0.9012901	1.164134\\
0.9013901	1.161318\\
0.9014901	1.158636\\
0.9015902	1.155852\\
0.9016902	1.154001\\
0.9017902	1.155741\\
0.9018902	1.156954\\
0.9019902	1.157711\\
0.9020902	1.153711\\
0.9021902	1.149799\\
0.9022902	1.14968\\
0.9023902	1.147171\\
0.9024902	1.146877\\
0.9025903	1.142783\\
0.9026903	1.14316\\
0.9027903	1.142778\\
0.9028903	1.140303\\
0.9029903	1.13954\\
0.9030903	1.135704\\
0.9031903	1.132121\\
0.9032903	1.131426\\
0.9033903	1.134469\\
0.9034903	1.132132\\
0.9035904	1.127568\\
0.9036904	1.127514\\
0.9037904	1.127653\\
0.9038904	1.123065\\
0.9039904	1.114085\\
0.9040904	1.107296\\
0.9041904	1.106514\\
0.9042904	1.105783\\
0.9043904	1.098819\\
0.9044904	1.095903\\
0.9045905	1.090377\\
0.9046905	1.091064\\
0.9047905	1.091333\\
0.9048905	1.090808\\
0.9049905	1.091595\\
0.9050905	1.089178\\
0.9051905	1.085263\\
0.9052905	1.080193\\
0.9053905	1.075966\\
0.9054905	1.0692\\
0.9055906	1.067221\\
0.9056906	1.067276\\
0.9057906	1.062643\\
0.9058906	1.060491\\
0.9059906	1.059324\\
0.9060906	1.057284\\
0.9061906	1.054738\\
0.9062906	1.050237\\
0.9063906	1.045532\\
0.9064906	1.0434\\
0.9065907	1.042503\\
0.9066907	1.032956\\
0.9067907	1.029719\\
0.9068907	1.026323\\
0.9069907	1.021034\\
0.9070907	1.019883\\
0.9071907	1.016911\\
0.9072907	1.009954\\
0.9073907	1.010011\\
0.9074907	1.005156\\
0.9075908	1.004816\\
0.9076908	1.000245\\
0.9077908	0.9945903\\
0.9078908	0.9912406\\
0.9079908	0.9898854\\
0.9080908	0.9878864\\
0.9081908	0.9867836\\
0.9082908	0.9807536\\
0.9083908	0.9755691\\
0.9084908	0.973499\\
0.9085909	0.9688547\\
0.9086909	0.9624268\\
0.9087909	0.9595387\\
0.9088909	0.9609088\\
0.9089909	0.9595458\\
0.9090909	0.9577978\\
0.9091909	0.9531198\\
0.9092909	0.9524953\\
0.9093909	0.948349\\
0.9094909	0.9472844\\
0.909591	0.94462\\
0.909691	0.9435143\\
0.909791	0.9405657\\
0.909891	0.9374142\\
0.909991	0.9364592\\
0.910091	0.9352735\\
0.910191	0.9318819\\
0.910291	0.9331659\\
0.910391	0.92722\\
0.910491	0.925131\\
0.9105911	0.9219627\\
0.9106911	0.9194646\\
0.9107911	0.9117012\\
0.9108911	0.9077448\\
0.9109911	0.9067853\\
0.9110911	0.9027292\\
0.9111911	0.9020051\\
0.9112911	0.9003534\\
0.9113911	0.9013911\\
0.9114911	0.8998968\\
0.9115912	0.9040879\\
0.9116912	0.9061214\\
0.9117912	0.9065747\\
0.9118912	0.9044433\\
0.9119912	0.9017262\\
0.9120912	0.9031054\\
0.9121912	0.9025872\\
0.9122912	0.898853\\
0.9123912	0.8967449\\
0.9124912	0.8922316\\
0.9125913	0.8920865\\
0.9126913	0.8896661\\
0.9127913	0.8847014\\
0.9128913	0.8817354\\
0.9129913	0.8859034\\
0.9130913	0.8851776\\
0.9131913	0.8839448\\
0.9132913	0.8800861\\
0.9133913	0.8780604\\
0.9134913	0.8811145\\
0.9135914	0.8798129\\
0.9136914	0.8778526\\
0.9137914	0.8773712\\
0.9138914	0.8762835\\
0.9139914	0.8762829\\
0.9140914	0.8732021\\
0.9141914	0.8701059\\
0.9142914	0.8631304\\
0.9143914	0.8607146\\
0.9144914	0.8602493\\
0.9145915	0.8595734\\
0.9146915	0.8625142\\
0.9147915	0.8625064\\
0.9148915	0.8616707\\
0.9149915	0.8604549\\
0.9150915	0.860195\\
0.9151915	0.8618426\\
0.9152915	0.8605466\\
0.9153915	0.853913\\
0.9154915	0.8444105\\
0.9155916	0.8432818\\
0.9156916	0.8414161\\
0.9157916	0.8428642\\
0.9158916	0.8382978\\
0.9159916	0.834535\\
0.9160916	0.8367003\\
0.9161916	0.8346081\\
0.9162916	0.834699\\
0.9163916	0.8285285\\
0.9164916	0.8231619\\
0.9165917	0.8224158\\
0.9166917	0.8255011\\
0.9167917	0.8195849\\
0.9168917	0.8158392\\
0.9169917	0.8110457\\
0.9170917	0.807081\\
0.9171917	0.8030814\\
0.9172917	0.8025199\\
0.9173917	0.8023338\\
0.9174917	0.7960923\\
0.9175918	0.7926747\\
0.9176918	0.787444\\
0.9177918	0.7850118\\
0.9178918	0.7840566\\
0.9179918	0.7813161\\
0.9180918	0.7778155\\
0.9181918	0.7738422\\
0.9182918	0.7714893\\
0.9183918	0.7710177\\
0.9184918	0.7688755\\
0.9185919	0.7661189\\
0.9186919	0.7614792\\
0.9187919	0.7631521\\
0.9188919	0.7608378\\
0.9189919	0.755225\\
0.9190919	0.7521316\\
0.9191919	0.7485428\\
0.9192919	0.7448688\\
0.9193919	0.7404739\\
0.9194919	0.737249\\
0.919592	0.7331814\\
0.919692	0.730196\\
0.919792	0.7288401\\
0.919892	0.7260848\\
0.919992	0.7203431\\
0.920092	0.7191087\\
0.920192	0.715072\\
0.920292	0.7110085\\
0.920392	0.7083879\\
0.920492	0.7054636\\
0.9205921	0.7027121\\
0.9206921	0.7025542\\
0.9207921	0.6987854\\
0.9208921	0.6978227\\
0.9209921	0.6955385\\
0.9210921	0.6990048\\
0.9211921	0.7004395\\
0.9212921	0.7004649\\
0.9213921	0.6989538\\
0.9214921	0.6967444\\
0.9215922	0.6910843\\
0.9216922	0.6883035\\
0.9217922	0.6869932\\
0.9218922	0.6840687\\
0.9219922	0.682182\\
0.9220922	0.6782555\\
0.9221922	0.6763654\\
0.9222922	0.6745844\\
0.9223922	0.6719879\\
0.9224922	0.6740088\\
0.9225923	0.6752104\\
0.9226923	0.6719914\\
0.9227923	0.6711129\\
0.9228923	0.6699702\\
0.9229923	0.6695076\\
0.9230923	0.6674304\\
0.9231923	0.6692125\\
0.9232923	0.6656391\\
0.9233923	0.6661772\\
0.9234923	0.666625\\
0.9235924	0.6684744\\
0.9236924	0.6655045\\
0.9237924	0.6636805\\
0.9238924	0.6572737\\
0.9239924	0.6606526\\
0.9240924	0.6594688\\
0.9241924	0.6628893\\
0.9242924	0.6623614\\
0.9243924	0.6624084\\
0.9244924	0.6609757\\
0.9245925	0.6582273\\
0.9246925	0.6520768\\
0.9247925	0.6564745\\
0.9248925	0.6544435\\
0.9249925	0.659531\\
0.9250925	0.6588029\\
0.9251925	0.6593138\\
0.9252925	0.6571462\\
0.9253925	0.6578084\\
0.9254925	0.6583894\\
0.9255926	0.6535852\\
0.9256926	0.6530781\\
0.9257926	0.6495955\\
0.9258926	0.6454609\\
0.9259926	0.6430216\\
0.9260926	0.6435603\\
0.9261926	0.6417555\\
0.9262926	0.6427309\\
0.9263926	0.6401473\\
0.9264926	0.6400162\\
0.9265927	0.6382794\\
0.9266927	0.6365296\\
0.9267927	0.635958\\
0.9268927	0.6370497\\
0.9269927	0.6341302\\
0.9270927	0.628104\\
0.9271927	0.6273794\\
0.9272927	0.6266595\\
0.9273927	0.6287023\\
0.9274927	0.6252952\\
0.9275928	0.6242825\\
0.9276928	0.6207759\\
0.9277928	0.6192804\\
0.9278928	0.6156726\\
0.9279928	0.6143126\\
0.9280928	0.6132907\\
0.9281928	0.612046\\
0.9282928	0.6127267\\
0.9283928	0.6075273\\
0.9284928	0.6068595\\
0.9285929	0.6042836\\
0.9286929	0.6014484\\
0.9287929	0.5955554\\
0.9288929	0.5907412\\
0.9289929	0.5850719\\
0.9290929	0.5806338\\
0.9291929	0.5779078\\
0.9292929	0.5729734\\
0.9293929	0.5742192\\
0.9294929	0.57121\\
0.929593	0.5690388\\
0.929693	0.5642491\\
0.929793	0.5627527\\
0.929893	0.5623196\\
0.929993	0.5615407\\
0.930093	0.5620702\\
0.930193	0.5605443\\
0.930293	0.5565652\\
0.930393	0.5510691\\
0.930493	0.5489211\\
0.9305931	0.5457418\\
0.9306931	0.5443257\\
0.9307931	0.5438068\\
0.9308931	0.5399315\\
0.9309931	0.5341045\\
0.9310931	0.5287837\\
0.9311931	0.5281897\\
0.9312931	0.5248616\\
0.9313931	0.5244337\\
0.9314931	0.5233698\\
0.9315932	0.5184968\\
0.9316932	0.5202877\\
0.9317932	0.5149979\\
0.9318932	0.512671\\
0.9319932	0.5088422\\
0.9320932	0.5070566\\
0.9321932	0.5060338\\
0.9322932	0.5033908\\
0.9323932	0.5017204\\
0.9324932	0.4975911\\
0.9325933	0.498514\\
0.9326933	0.5015552\\
0.9327933	0.4993848\\
0.9328933	0.4989799\\
0.9329933	0.497389\\
0.9330933	0.4952613\\
0.9331933	0.4936974\\
0.9332933	0.4890258\\
0.9333933	0.4904566\\
0.9334933	0.4867633\\
0.9335934	0.4853237\\
0.9336934	0.4881743\\
0.9337934	0.4857009\\
0.9338934	0.4851034\\
0.9339934	0.4838583\\
0.9340934	0.4871672\\
0.9341934	0.4871813\\
0.9342934	0.4854482\\
0.9343934	0.4878924\\
0.9344934	0.4862761\\
0.9345935	0.4868361\\
0.9346935	0.4842048\\
0.9347935	0.4843374\\
0.9348935	0.4855246\\
0.9349935	0.4840251\\
0.9350935	0.4834753\\
0.9351935	0.4813068\\
0.9352935	0.4802627\\
0.9353935	0.4804027\\
0.9354935	0.4782085\\
0.9355936	0.4761696\\
0.9356936	0.4771457\\
0.9357936	0.4752974\\
0.9358936	0.4785598\\
0.9359936	0.4759882\\
0.9360936	0.4786191\\
0.9361936	0.4795808\\
0.9362936	0.4787256\\
0.9363936	0.4854477\\
0.9364936	0.4848735\\
0.9365937	0.4854967\\
0.9366937	0.4849902\\
0.9367937	0.4808094\\
0.9368937	0.4802721\\
0.9369937	0.4780952\\
0.9370937	0.477346\\
0.9371937	0.4769411\\
0.9372937	0.4766874\\
0.9373937	0.4782207\\
0.9374937	0.4799293\\
0.9375938	0.4765222\\
0.9376938	0.4774359\\
0.9377938	0.4756087\\
0.9378938	0.472834\\
0.9379938	0.4716955\\
0.9380938	0.4696217\\
0.9381938	0.4701853\\
0.9382938	0.4658481\\
0.9383938	0.4629425\\
0.9384938	0.461478\\
0.9385939	0.4597336\\
0.9386939	0.4581291\\
0.9387939	0.4579399\\
0.9388939	0.4577537\\
0.9389939	0.4556971\\
0.9390939	0.458208\\
0.9391939	0.4545267\\
0.9392939	0.4533385\\
0.9393939	0.4525304\\
0.9394939	0.4490049\\
0.939594	0.4482888\\
0.939694	0.4465179\\
0.939794	0.4453875\\
0.939894	0.44286\\
0.939994	0.439315\\
0.940094	0.4342113\\
0.940194	0.4317932\\
0.940294	0.4311459\\
0.940394	0.4309788\\
0.940494	0.4283165\\
0.9405941	0.4208315\\
0.9406941	0.4210788\\
0.9407941	0.4203039\\
0.9408941	0.4138451\\
0.9409941	0.4081136\\
0.9410941	0.4069022\\
0.9411941	0.4024185\\
0.9412941	0.4026566\\
0.9413941	0.4039594\\
0.9414941	0.4017518\\
0.9415942	0.4006024\\
0.9416942	0.3976865\\
0.9417942	0.3952857\\
0.9418942	0.3908739\\
0.9419942	0.387136\\
0.9420942	0.3850362\\
0.9421942	0.3823094\\
0.9422942	0.3794415\\
0.9423942	0.3749022\\
0.9424942	0.373489\\
0.9425943	0.3704665\\
0.9426943	0.3668873\\
0.9427943	0.3681155\\
0.9428943	0.3644839\\
0.9429943	0.3658893\\
0.9430943	0.3651883\\
0.9431943	0.36368\\
0.9432943	0.3620988\\
0.9433943	0.3624226\\
0.9434943	0.3586696\\
0.9435944	0.3574526\\
0.9436944	0.3551845\\
0.9437944	0.3522959\\
0.9438944	0.3514212\\
0.9439944	0.3506852\\
0.9440944	0.3495434\\
0.9441944	0.3456356\\
0.9442944	0.3462289\\
0.9443944	0.343873\\
0.9444944	0.3400614\\
0.9445945	0.3417397\\
0.9446945	0.3412286\\
0.9447945	0.3418411\\
0.9448945	0.3404805\\
0.9449945	0.3397698\\
0.9450945	0.3374538\\
0.9451945	0.3376751\\
0.9452945	0.3364405\\
0.9453945	0.3360859\\
0.9454945	0.3372973\\
0.9455946	0.3395803\\
0.9456946	0.3403736\\
0.9457946	0.3401955\\
0.9458946	0.343481\\
0.9459946	0.34277\\
0.9460946	0.344533\\
0.9461946	0.3447063\\
0.9462946	0.3439077\\
0.9463946	0.3429254\\
0.9464946	0.3424712\\
0.9465947	0.3430086\\
0.9466947	0.3407054\\
0.9467947	0.3441766\\
0.9468947	0.3443989\\
0.9469947	0.3408149\\
0.9470947	0.3398746\\
0.9471947	0.3385337\\
0.9472947	0.3402778\\
0.9473947	0.3415908\\
0.9474947	0.3416195\\
0.9475948	0.3440878\\
0.9476948	0.3447154\\
0.9477948	0.3461921\\
0.9478948	0.3486087\\
0.9479948	0.3461867\\
0.9480948	0.3465007\\
0.9481948	0.3462456\\
0.9482948	0.3451811\\
0.9483948	0.3455872\\
0.9484948	0.3448051\\
0.9485949	0.3452147\\
0.9486949	0.3451354\\
0.9487949	0.3458614\\
0.9488949	0.3469765\\
0.9489949	0.3462784\\
0.9490949	0.3463053\\
0.9491949	0.3464262\\
0.9492949	0.3462018\\
0.9493949	0.3437038\\
0.9494949	0.3420335\\
0.949595	0.3429781\\
0.949695	0.3427204\\
0.949795	0.3403293\\
0.949895	0.3383276\\
0.949995	0.3396988\\
0.950095	0.3377848\\
0.950195	0.3342906\\
0.950295	0.333073\\
0.950395	0.3329109\\
0.950495	0.3295497\\
0.9505951	0.3268124\\
0.9506951	0.3268078\\
0.9507951	0.3244106\\
0.9508951	0.3238101\\
0.9509951	0.3252751\\
0.9510951	0.323995\\
0.9511951	0.3194223\\
0.9512951	0.3163753\\
0.9513951	0.3147915\\
0.9514951	0.312688\\
0.9515952	0.3079559\\
0.9516952	0.3055872\\
0.9517952	0.3063795\\
0.9518952	0.3042056\\
0.9519952	0.3015702\\
0.9520952	0.2981272\\
0.9521952	0.2950479\\
0.9522952	0.2932661\\
0.9523952	0.2907167\\
0.9524952	0.2904458\\
0.9525953	0.2913603\\
0.9526953	0.2883635\\
0.9527953	0.2843196\\
0.9528953	0.2836621\\
0.9529953	0.2830705\\
0.9530953	0.2798711\\
0.9531953	0.2754049\\
0.9532953	0.2729127\\
0.9533953	0.271261\\
0.9534953	0.2651062\\
0.9535954	0.2600804\\
0.9536954	0.2578859\\
0.9537954	0.2554107\\
0.9538954	0.2533769\\
0.9539954	0.2503741\\
0.9540954	0.2493731\\
0.9541954	0.2503319\\
0.9542954	0.2490191\\
0.9543954	0.2465965\\
0.9544954	0.2448809\\
0.9545955	0.245686\\
0.9546955	0.2454803\\
0.9547955	0.2425237\\
0.9548955	0.2397343\\
0.9549955	0.2417408\\
0.9550955	0.2408434\\
0.9551955	0.2387447\\
0.9552955	0.2363666\\
0.9553955	0.2368603\\
0.9554955	0.2362989\\
0.9555956	0.2346835\\
0.9556956	0.2330681\\
0.9557956	0.2314616\\
0.9558956	0.2315889\\
0.9559956	0.2288953\\
0.9560956	0.2274045\\
0.9561956	0.2251701\\
0.9562956	0.2277858\\
0.9563956	0.228458\\
0.9564956	0.2293095\\
0.9565957	0.228594\\
0.9566957	0.2295311\\
0.9567957	0.228874\\
0.9568957	0.2287519\\
0.9569957	0.228696\\
0.9570957	0.2287617\\
0.9571957	0.2309013\\
0.9572957	0.2311142\\
0.9573957	0.2291113\\
0.9574957	0.2304871\\
0.9575958	0.231586\\
0.9576958	0.2328919\\
0.9577958	0.2353585\\
0.9578958	0.2355978\\
0.9579958	0.2357219\\
0.9580958	0.2386371\\
0.9581958	0.2404175\\
0.9582958	0.2398432\\
0.9583958	0.2391968\\
0.9584958	0.2394328\\
0.9585959	0.2400968\\
0.9586959	0.2412187\\
0.9587959	0.2412456\\
0.9588959	0.2397077\\
0.9589959	0.2413919\\
0.9590959	0.2430388\\
0.9591959	0.2472134\\
0.9592959	0.2463478\\
0.9593959	0.2464438\\
0.9594959	0.2489333\\
0.959596	0.2494933\\
0.959696	0.2486058\\
0.959796	0.2475355\\
0.959896	0.2485614\\
0.959996	0.2480175\\
0.960096	0.2470782\\
0.960196	0.2470833\\
0.960296	0.2463797\\
0.960396	0.24376\\
0.960496	0.2446979\\
0.9605961	0.2470396\\
0.9606961	0.2477621\\
0.9607961	0.2492803\\
0.9608961	0.2478224\\
0.9609961	0.2482253\\
0.9610961	0.2493452\\
0.9611961	0.2495631\\
0.9612961	0.2479361\\
0.9613961	0.2457635\\
0.9614961	0.2448755\\
0.9615962	0.2435771\\
0.9616962	0.2436303\\
0.9617962	0.2429555\\
0.9618962	0.2391529\\
0.9619962	0.2376332\\
0.9620962	0.2380938\\
0.9621962	0.2388104\\
0.9622962	0.2355296\\
0.9623962	0.2330147\\
0.9624962	0.2311213\\
0.9625963	0.2270384\\
0.9626963	0.2255186\\
0.9627963	0.2245333\\
0.9628963	0.2240249\\
0.9629963	0.2205359\\
0.9630963	0.2179531\\
0.9631963	0.2132496\\
0.9632963	0.2112859\\
0.9633963	0.2107229\\
0.9634963	0.2079908\\
0.9635964	0.2030493\\
0.9636964	0.2019689\\
0.9637964	0.2027998\\
0.9638964	0.2020975\\
0.9639964	0.2016117\\
0.9640964	0.1999805\\
0.9641964	0.1982395\\
0.9642964	0.1954228\\
0.9643964	0.1936725\\
0.9644964	0.1915057\\
0.9645965	0.187928\\
0.9646965	0.185312\\
0.9647965	0.1833995\\
0.9648965	0.1791527\\
0.9649965	0.1750512\\
0.9650965	0.1715494\\
0.9651965	0.1714251\\
0.9652965	0.1685794\\
0.9653965	0.1666872\\
0.9654965	0.1640971\\
0.9655966	0.1623572\\
0.9656966	0.1607545\\
0.9657966	0.1600102\\
0.9658966	0.1609432\\
0.9659966	0.1610811\\
0.9660966	0.1576532\\
0.9661966	0.1554837\\
0.9662966	0.1547706\\
0.9663966	0.1532205\\
0.9664966	0.1526304\\
0.9665967	0.1519413\\
0.9666967	0.1510877\\
0.9667967	0.1485323\\
0.9668967	0.1473617\\
0.9669967	0.1465891\\
0.9670967	0.1464349\\
0.9671967	0.1463428\\
0.9672967	0.1472373\\
0.9673967	0.1481721\\
0.9674967	0.1465469\\
0.9675968	0.1448816\\
0.9676968	0.1458921\\
0.9677968	0.1467011\\
0.9678968	0.1480129\\
0.9679968	0.1489546\\
0.9680968	0.1500836\\
0.9681968	0.148998\\
0.9682968	0.1491973\\
0.9683968	0.1487751\\
0.9684968	0.149701\\
0.9685969	0.1504195\\
0.9686969	0.1509242\\
0.9687969	0.1511326\\
0.9688969	0.1530967\\
0.9689969	0.1525522\\
0.9690969	0.1520664\\
0.9691969	0.1534442\\
0.9692969	0.1552586\\
0.9693969	0.1568414\\
0.9694969	0.1582413\\
0.969597	0.1600486\\
0.969697	0.1612742\\
0.969797	0.1628609\\
0.969897	0.1618468\\
0.969997	0.1637276\\
0.970097	0.1659045\\
0.970197	0.1675823\\
0.970297	0.1692643\\
0.970397	0.1711105\\
0.970497	0.1738588\\
0.9705971	0.1746105\\
0.9706971	0.1737982\\
0.9707971	0.1746653\\
0.9708971	0.1746907\\
0.9709971	0.1762525\\
0.9710971	0.1772781\\
0.9711971	0.1775149\\
0.9712971	0.180242\\
0.9713971	0.1807373\\
0.9714971	0.1798766\\
0.9715972	0.180247\\
0.9716972	0.179226\\
0.9717972	0.1793241\\
0.9718972	0.1781555\\
0.9719972	0.1766126\\
0.9720972	0.1782937\\
0.9721972	0.1800448\\
0.9722972	0.178543\\
0.9723972	0.1781261\\
0.9724972	0.1786646\\
0.9725973	0.1788297\\
0.9726973	0.1788935\\
0.9727973	0.1785005\\
0.9728973	0.1780998\\
0.9729973	0.1776745\\
0.9730973	0.1776183\\
0.9731973	0.1772348\\
0.9732973	0.1779841\\
0.9733973	0.1772374\\
0.9734973	0.1753225\\
0.9735974	0.1738702\\
0.9736974	0.1715273\\
0.9737974	0.1676834\\
0.9738974	0.1644419\\
0.9739974	0.1629188\\
0.9740974	0.1629137\\
0.9741974	0.1607435\\
0.9742974	0.1585257\\
0.9743974	0.1591427\\
0.9744974	0.1569218\\
0.9745975	0.1545549\\
0.9746975	0.1520486\\
0.9747975	0.1491659\\
0.9748975	0.1462168\\
0.9749975	0.1444318\\
0.9750975	0.1412816\\
0.9751975	0.1380387\\
0.9752975	0.1364628\\
0.9753975	0.1344342\\
0.9754975	0.133525\\
0.9755976	0.1318575\\
0.9756976	0.1302281\\
0.9757976	0.1279328\\
0.9758976	0.126549\\
0.9759976	0.1252211\\
0.9760976	0.1223971\\
0.9761976	0.1197009\\
0.9762976	0.1163382\\
0.9763976	0.112496\\
0.9764976	0.1099891\\
0.9765977	0.1080144\\
0.9766977	0.1065687\\
0.9767977	0.1040058\\
0.9768977	0.1030872\\
0.9769977	0.1024929\\
0.9770977	0.1016184\\
0.9771977	0.1006989\\
0.9772977	0.09969075\\
0.9773977	0.09908296\\
0.9774977	0.09729033\\
0.9775978	0.09595868\\
0.9776978	0.09463953\\
0.9777978	0.09270602\\
0.9778978	0.09022318\\
0.9779978	0.08884746\\
0.9780978	0.08756641\\
0.9781978	0.08662138\\
0.9782978	0.08564624\\
0.9783978	0.08455533\\
0.9784978	0.08440477\\
0.9785979	0.08395598\\
0.9786979	0.08489617\\
0.9787979	0.0846849\\
0.9788979	0.08585329\\
0.9789979	0.08628419\\
0.9790979	0.08656906\\
0.9791979	0.08813179\\
0.9792979	0.08928918\\
0.9793979	0.08993976\\
0.9794979	0.08935775\\
0.979598	0.09085021\\
0.979698	0.09181444\\
0.979798	0.09294306\\
0.979898	0.0932424\\
0.979998	0.09409501\\
0.980098	0.09493776\\
0.980198	0.09581242\\
0.980298	0.09549563\\
0.980398	0.09654003\\
0.980498	0.09844028\\
0.9805981	0.1004191\\
0.9806981	0.1011535\\
0.9807981	0.1021603\\
0.9808981	0.1045116\\
0.9809981	0.1046854\\
0.9810981	0.1065464\\
0.9811981	0.1086026\\
0.9812981	0.1103606\\
0.9813981	0.1109999\\
0.9814981	0.1124229\\
0.9815982	0.1145485\\
0.9816982	0.1149848\\
0.9817982	0.1173526\\
0.9818982	0.1196865\\
0.9819982	0.1218392\\
0.9820982	0.1229184\\
0.9821982	0.1245977\\
0.9822982	0.1263099\\
0.9823982	0.1282369\\
0.9824982	0.1293231\\
0.9825983	0.1298028\\
0.9826983	0.1305781\\
0.9827983	0.1313713\\
0.9828983	0.1319097\\
0.9829983	0.1331825\\
0.9830983	0.1340347\\
0.9831983	0.1343423\\
0.9832983	0.1349217\\
0.9833983	0.1348261\\
0.9834983	0.1350205\\
0.9835984	0.1343863\\
0.9836984	0.1343211\\
0.9837984	0.1338821\\
0.9838984	0.1344695\\
0.9839984	0.1357064\\
0.9840984	0.1361315\\
0.9841984	0.1353406\\
0.9842984	0.1352257\\
0.9843984	0.1346178\\
0.9844984	0.1338388\\
0.9845985	0.1348415\\
0.9846985	0.1348647\\
0.9847985	0.1334617\\
0.9848985	0.1324907\\
0.9849985	0.1307586\\
0.9850985	0.1306021\\
0.9851985	0.1297994\\
0.9852985	0.1288487\\
0.9853985	0.1279404\\
0.9854985	0.1279756\\
0.9855986	0.1287686\\
0.9856986	0.1287174\\
0.9857986	0.1277705\\
0.9858986	0.1263297\\
0.9859986	0.1258216\\
0.9860986	0.1258809\\
0.9861986	0.1260837\\
0.9862986	0.1259342\\
0.9863986	0.1268857\\
0.9864986	0.1272023\\
0.9865987	0.1274872\\
0.9866987	0.1274662\\
0.9867987	0.1278216\\
0.9868987	0.1279911\\
0.9869987	0.1281659\\
0.9870987	0.1270913\\
0.9871987	0.1280616\\
0.9872987	0.1288196\\
0.9873987	0.1287312\\
0.9874987	0.1300582\\
0.9875988	0.1292708\\
0.9876988	0.1296362\\
0.9877988	0.1294739\\
0.9878988	0.1295512\\
0.9879988	0.1291705\\
0.9880988	0.1290139\\
0.9881988	0.1297047\\
0.9882988	0.1306901\\
0.9883988	0.1303628\\
0.9884988	0.1307456\\
0.9885989	0.1301117\\
0.9886989	0.1291086\\
0.9887989	0.1274045\\
0.9888989	0.1271537\\
0.9889989	0.1268138\\
0.9890989	0.1253566\\
0.9891989	0.1237835\\
0.9892989	0.1223863\\
0.9893989	0.1217012\\
0.9894989	0.1210426\\
0.989599	0.1215262\\
0.989699	0.121196\\
0.989799	0.1210106\\
0.989899	0.1197669\\
0.989999	0.1194221\\
0.990099	0.119079\\
0.990199	0.1183595\\
0.990299	0.1184255\\
0.990399	0.1194306\\
0.990499	0.1203822\\
0.9905991	0.1207705\\
0.9906991	0.1216113\\
0.9907991	0.1204397\\
0.9908991	0.1211595\\
0.9909991	0.1205558\\
0.9910991	0.1194763\\
0.9911991	0.1174895\\
0.9912991	0.1156342\\
0.9913991	0.1146506\\
0.9914991	0.1137386\\
0.9915992	0.1117798\\
0.9916992	0.1099285\\
0.9917992	0.1097512\\
0.9918992	0.1085644\\
0.9919992	0.1081617\\
0.9920992	0.1083895\\
0.9921992	0.1080801\\
0.9922992	0.1082185\\
0.9923992	0.1077733\\
0.9924992	0.1065615\\
0.9925993	0.1055003\\
0.9926993	0.1040426\\
0.9927993	0.1017291\\
0.9928993	0.100786\\
0.9929993	0.09929942\\
0.9930993	0.09695137\\
0.9931993	0.09542868\\
0.9932993	0.09301954\\
0.9933993	0.09157389\\
0.9934993	0.09140489\\
0.9935994	0.08968543\\
0.9936994	0.08902935\\
0.9937994	0.08941585\\
0.9938994	0.08840217\\
0.9939994	0.08766632\\
0.9940994	0.08629635\\
0.9941994	0.0846184\\
0.9942994	0.08292933\\
0.9943994	0.08201693\\
0.9944994	0.07928949\\
0.9945995	0.07756109\\
0.9946995	0.07588494\\
0.9947995	0.07528941\\
0.9948995	0.07473441\\
0.9949995	0.07358542\\
0.9950995	0.07206776\\
0.9951995	0.07134671\\
0.9952995	0.07035614\\
0.9953995	0.06976495\\
0.9954995	0.06841557\\
0.9955996	0.06707341\\
0.9956996	0.06548726\\
0.9957996	0.06462551\\
0.9958996	0.06400444\\
0.9959996	0.06210639\\
0.9960996	0.06235607\\
0.9961996	0.06226705\\
0.9962996	0.06116866\\
0.9963996	0.05958042\\
0.9964996	0.05876268\\
0.9965997	0.05701796\\
0.9966997	0.05505632\\
0.9967997	0.05283456\\
0.9968997	0.05149237\\
0.9969997	0.04978405\\
0.9970997	0.04870378\\
0.9971997	0.04738763\\
0.9972997	0.04638117\\
0.9973997	0.04488127\\
0.9974997	0.04478742\\
0.9975998	0.04388335\\
0.9976998	0.04177614\\
0.9977998	0.04079268\\
0.9978998	0.03926787\\
0.9979998	0.03859873\\
0.9980998	0.03818853\\
0.9981998	0.03789537\\
0.9982998	0.0378193\\
0.9983998	0.03747809\\
0.9984998	0.03697124\\
0.9985999	0.03683473\\
0.9986999	0.03554119\\
0.9987999	0.03535857\\
0.9988999	0.03527208\\
0.9989999	0.03523584\\
0.9990999	0.03512575\\
0.9991999	0.0355604\\
0.9992999	0.03528455\\
0.9993999	0.03552341\\
0.9994999	0.03455031\\
0.9996	0.03394304\\
0.9997	0.03377166\\
0.9998	0.03290953\\
0.9999	0.03312837\\
1	0.03294526\\
};
\addlegendentry{$\text{|}\psi{}_\text{n}\text{(x,t)|}^\text{2}$};

\addplot [color=mycolor2,solid,forget plot]
  table[row sep=crcr]{%
0	-0.1784732\\
0.00010001	-0.1791198\\
0.00020002	-0.1773611\\
0.00030003	-0.1750989\\
0.00040004	-0.1740345\\
0.00050005	-0.1731944\\
0.00060006	-0.1730486\\
0.00070007	-0.1737014\\
0.00080008	-0.1745322\\
0.00090009	-0.1743524\\
0.0010001	-0.1734143\\
0.00110011	-0.1718576\\
0.00120012	-0.17024\\
0.00130013	-0.1700123\\
0.00140014	-0.1687604\\
0.00150015	-0.1686385\\
0.00160016	-0.1677291\\
0.00170017	-0.1667247\\
0.00180018	-0.1633206\\
0.00190019	-0.1604784\\
0.0020002	-0.1572148\\
0.00210021	-0.1556874\\
0.00220022	-0.154226\\
0.00230023	-0.1512206\\
0.00240024	-0.1482253\\
0.00250025	-0.1435527\\
0.00260026	-0.1406803\\
0.00270027	-0.1380398\\
0.00280028	-0.1364835\\
0.00290029	-0.1345096\\
0.0030003	-0.1320232\\
0.00310031	-0.1289661\\
0.00320032	-0.1251674\\
0.00330033	-0.1217024\\
0.00340034	-0.1184506\\
0.00350035	-0.115984\\
0.00360036	-0.1135771\\
0.00370037	-0.1093029\\
0.00380038	-0.1031468\\
0.00390039	-0.09934604\\
0.0040004	-0.09574646\\
0.00410041	-0.09100841\\
0.00420042	-0.08607548\\
0.00430043	-0.08064029\\
0.00440044	-0.07624818\\
0.00450045	-0.07084748\\
0.00460046	-0.0643907\\
0.00470047	-0.05833303\\
0.00480048	-0.05576972\\
0.00490049	-0.05533438\\
0.0050005	-0.05096563\\
0.00510051	-0.04425949\\
0.00520052	-0.04102225\\
0.00530053	-0.03703001\\
0.00540054	-0.03225542\\
0.00550055	-0.02879644\\
0.00560056	-0.02131519\\
0.00570057	-0.01674746\\
0.00580058	-0.01135262\\
0.00590059	-0.008482901\\
0.0060006	-0.005018666\\
0.00610061	-8.431048e-05\\
0.00620062	0.004413624\\
0.00630063	0.01132841\\
0.00640064	0.01591903\\
0.00650065	0.01825227\\
0.00660066	0.02210799\\
0.00670067	0.02489278\\
0.00680068	0.02974965\\
0.00690069	0.03301377\\
0.0070007	0.03617218\\
0.00710071	0.03965353\\
0.00720072	0.04316366\\
0.00730073	0.04812455\\
0.00740074	0.05325768\\
0.00750075	0.0591663\\
0.00760076	0.062457\\
0.00770077	0.06812949\\
0.00780078	0.07364136\\
0.00790079	0.08134486\\
0.0080008	0.08559597\\
0.00810081	0.08816733\\
0.00820082	0.09168981\\
0.00830083	0.09432769\\
0.00840084	0.09593771\\
0.00850085	0.09850347\\
0.00860086	0.1004359\\
0.00870087	0.1014034\\
0.00880088	0.1032666\\
0.00890089	0.1057338\\
0.0090009	0.1094957\\
0.00910091	0.1132227\\
0.00920092	0.1143402\\
0.00930093	0.1185877\\
0.00940094	0.1217079\\
0.00950095	0.1241333\\
0.00960096	0.1253038\\
0.00970097	0.126805\\
0.00980098	0.1290263\\
0.00990099	0.1291122\\
0.010001	0.1312362\\
0.01010101	0.1314821\\
0.01020102	0.1308309\\
0.01030103	0.1317471\\
0.01040104	0.1353964\\
0.01050105	0.1378566\\
0.01060106	0.1398002\\
0.01070107	0.1405616\\
0.01080108	0.1427318\\
0.01090109	0.1442444\\
0.0110011	0.1439499\\
0.01110111	0.1420262\\
0.01120112	0.1419399\\
0.01130113	0.1415191\\
0.01140114	0.1416257\\
0.01150115	0.1410911\\
0.01160116	0.1390072\\
0.01170117	0.1377958\\
0.01180118	0.1376648\\
0.01190119	0.1382389\\
0.0120012	0.1399211\\
0.01210121	0.138015\\
0.01220122	0.1364121\\
0.01230123	0.1382969\\
0.01240124	0.1392624\\
0.01250125	0.1352045\\
0.01260126	0.1331379\\
0.01270127	0.1315859\\
0.01280128	0.132091\\
0.01290129	0.1287466\\
0.0130013	0.1245158\\
0.01310131	0.1229271\\
0.01320132	0.1215301\\
0.01330133	0.1168859\\
0.01340134	0.1140689\\
0.01350135	0.1110201\\
0.01360136	0.1098253\\
0.01370137	0.1091045\\
0.01380138	0.1033697\\
0.01390139	0.1003919\\
0.0140014	0.09815081\\
0.01410141	0.09533814\\
0.01420142	0.09019303\\
0.01430143	0.08768825\\
0.01440144	0.08639717\\
0.01450145	0.08544233\\
0.01460146	0.08337283\\
0.01470147	0.07981117\\
0.01480148	0.07750778\\
0.01490149	0.07629643\\
0.0150015	0.07258344\\
0.01510151	0.0677979\\
0.01520152	0.06354448\\
0.01530153	0.06120211\\
0.01540154	0.05741877\\
0.01550155	0.05303548\\
0.01560156	0.04828765\\
0.01570157	0.04552996\\
0.01580158	0.04088677\\
0.01590159	0.03576585\\
0.0160016	0.0334715\\
0.01610161	0.03111921\\
0.01620162	0.02683768\\
0.01630163	0.02334701\\
0.01640164	0.02033156\\
0.01650165	0.01505191\\
0.01660166	0.0101311\\
0.01670167	0.004520644\\
0.01680168	0.001718279\\
0.01690169	-0.001163227\\
0.0170017	-0.004681667\\
0.01710171	-0.008952129\\
0.01720172	-0.01153174\\
0.01730173	-0.01522815\\
0.01740174	-0.02044039\\
0.01750175	-0.02533239\\
0.01760176	-0.02841276\\
0.01770177	-0.03171307\\
0.01780178	-0.03552816\\
0.01790179	-0.03841937\\
0.0180018	-0.04004397\\
0.01810181	-0.04331819\\
0.01820182	-0.04593727\\
0.01830183	-0.04474421\\
0.01840184	-0.04667591\\
0.01850185	-0.05276672\\
0.01860186	-0.05631153\\
0.01870187	-0.05854873\\
0.01880188	-0.06311616\\
0.01890189	-0.06781841\\
0.0190019	-0.06964474\\
0.01910191	-0.07078413\\
0.01920192	-0.07428616\\
0.01930193	-0.07794442\\
0.01940194	-0.08218223\\
0.01950195	-0.08418649\\
0.01960196	-0.08635292\\
0.01970197	-0.08872375\\
0.01980198	-0.09123925\\
0.01990199	-0.094013\\
0.020002	-0.09476174\\
0.02010201	-0.09716567\\
0.02020202	-0.09710739\\
0.02030203	-0.0979378\\
0.02040204	-0.09856713\\
0.02050205	-0.09629236\\
0.02060206	-0.1000979\\
0.02070207	-0.1040499\\
0.02080208	-0.1064849\\
0.02090209	-0.1070974\\
0.0210021	-0.1090994\\
0.02110211	-0.1127275\\
0.02120212	-0.1142986\\
0.02130213	-0.1159033\\
0.02140214	-0.1142347\\
0.02150215	-0.1124883\\
0.02160216	-0.1128263\\
0.02170217	-0.1165865\\
0.02180218	-0.1142225\\
0.02190219	-0.114174\\
0.0220022	-0.1152734\\
0.02210221	-0.1124714\\
0.02220222	-0.1094055\\
0.02230223	-0.108904\\
0.02240224	-0.1107193\\
0.02250225	-0.1078624\\
0.02260226	-0.1075665\\
0.02270227	-0.1078969\\
0.02280228	-0.1072149\\
0.02290229	-0.108194\\
0.0230023	-0.1087294\\
0.02310231	-0.1081165\\
0.02320232	-0.1049453\\
0.02330233	-0.1041871\\
0.02340234	-0.1037544\\
0.02350235	-0.1012913\\
0.02360236	-0.1006215\\
0.02370237	-0.09936609\\
0.02380238	-0.09606537\\
0.02390239	-0.09854762\\
0.0240024	-0.09685415\\
0.02410241	-0.09451528\\
0.02420242	-0.09380667\\
0.02430243	-0.09169599\\
0.02440244	-0.08806021\\
0.02450245	-0.08642259\\
0.02460246	-0.08637829\\
0.02470247	-0.08403363\\
0.02480248	-0.07864427\\
0.02490249	-0.07687939\\
0.0250025	-0.07424804\\
0.02510251	-0.0688634\\
0.02520252	-0.06475167\\
0.02530253	-0.06065223\\
0.02540254	-0.05949109\\
0.02550255	-0.05674271\\
0.02560256	-0.05352333\\
0.02570257	-0.05227655\\
0.02580258	-0.05060867\\
0.02590259	-0.0481792\\
0.0260026	-0.0463017\\
0.02610261	-0.04502705\\
0.02620262	-0.04176905\\
0.02630263	-0.03844182\\
0.02640264	-0.0363441\\
0.02650265	-0.03225201\\
0.02660266	-0.02858011\\
0.02670267	-0.02509591\\
0.02680268	-0.01997514\\
0.02690269	-0.01870322\\
0.0270027	-0.01603965\\
0.02710271	-0.00972087\\
0.02720272	-0.006757423\\
0.02730273	-0.00601204\\
0.02740274	-0.005112096\\
0.02750275	-0.005643121\\
0.02760276	-0.004815959\\
0.02770277	5.56237e-05\\
0.02780278	0.004897445\\
0.02790279	0.007267329\\
0.0280028	0.01247844\\
0.02810281	0.0169916\\
0.02820282	0.02146909\\
0.02830283	0.02402071\\
0.02840284	0.02310589\\
0.02850285	0.02723219\\
0.02860286	0.03025358\\
0.02870287	0.0330405\\
0.02880288	0.03643788\\
0.02890289	0.03863932\\
0.0290029	0.0405907\\
0.02910291	0.04331745\\
0.02920292	0.04500852\\
0.02930293	0.04754126\\
0.02940294	0.05212636\\
0.02950295	0.05451571\\
0.02960296	0.0559105\\
0.02970297	0.05860553\\
0.02980298	0.06024206\\
0.02990299	0.06237742\\
0.030003	0.06336634\\
0.03010301	0.06418606\\
0.03020302	0.06559361\\
0.03030303	0.06932832\\
0.03040304	0.07147265\\
0.03050305	0.07358131\\
0.03060306	0.07403415\\
0.03070307	0.07700523\\
0.03080308	0.07995892\\
0.03090309	0.07840772\\
0.0310031	0.08071351\\
0.03110311	0.08230459\\
0.03120312	0.08293823\\
0.03130313	0.08250393\\
0.03140314	0.08212286\\
0.03150315	0.0822777\\
0.03160316	0.08332974\\
0.03170317	0.08038686\\
0.03180318	0.08152541\\
0.03190319	0.08257187\\
0.0320032	0.08383988\\
0.03210321	0.0886912\\
0.03220322	0.08919168\\
0.03230323	0.09009843\\
0.03240324	0.09196821\\
0.03250325	0.08953468\\
0.03260326	0.09232507\\
0.03270327	0.09275951\\
0.03280328	0.09393568\\
0.03290329	0.09319031\\
0.0330033	0.0923603\\
0.03310331	0.09295938\\
0.03320332	0.09236728\\
0.03330333	0.08838213\\
0.03340334	0.08742366\\
0.03350335	0.08575717\\
0.03360336	0.08481433\\
0.03370337	0.08539112\\
0.03380338	0.08282356\\
0.03390339	0.08124639\\
0.0340034	0.07874598\\
0.03410341	0.07802166\\
0.03420342	0.07656102\\
0.03430343	0.07490028\\
0.03440344	0.07596969\\
0.03450345	0.07490541\\
0.03460346	0.07420521\\
0.03470347	0.07134698\\
0.03480348	0.07059452\\
0.03490349	0.07083929\\
0.0350035	0.0685534\\
0.03510351	0.06684929\\
0.03520352	0.06478493\\
0.03530353	0.06241222\\
0.03540354	0.05992096\\
0.03550355	0.05663955\\
0.03560356	0.05563308\\
0.03570357	0.05435889\\
0.03580358	0.05191744\\
0.03590359	0.05083271\\
0.0360036	0.04919623\\
0.03610361	0.04769363\\
0.03620362	0.04435216\\
0.03630363	0.04360585\\
0.03640364	0.04192765\\
0.03650365	0.03908215\\
0.03660366	0.03741782\\
0.03670367	0.03302746\\
0.03680368	0.03117577\\
0.03690369	0.02776329\\
0.0370037	0.02291905\\
0.03710371	0.0231228\\
0.03720372	0.02030011\\
0.03730373	0.02043595\\
0.03740374	0.01841131\\
0.03750375	0.01378515\\
0.03760376	0.01249955\\
0.03770377	0.008410198\\
0.03780378	0.006321688\\
0.03790379	0.002696408\\
0.0380038	0.002827333\\
0.03810381	0.0009015981\\
0.03820382	-0.003011427\\
0.03830383	-0.005427831\\
0.03840384	-0.008835746\\
0.03850385	-0.01153974\\
0.03860386	-0.01470306\\
0.03870387	-0.01625738\\
0.03880388	-0.02014559\\
0.03890389	-0.02215937\\
0.0390039	-0.02282441\\
0.03910391	-0.02640033\\
0.03920392	-0.02779707\\
0.03930393	-0.02837352\\
0.03940394	-0.02792616\\
0.03950395	-0.02928564\\
0.03960396	-0.02932024\\
0.03970397	-0.03394228\\
0.03980398	-0.03547702\\
0.03990399	-0.03720897\\
0.040004	-0.04078204\\
0.04010401	-0.04154665\\
0.04020402	-0.04552428\\
0.04030403	-0.04733092\\
0.04040404	-0.05011716\\
0.04050405	-0.05112978\\
0.04060406	-0.05040409\\
0.04070407	-0.05135172\\
0.04080408	-0.05303585\\
0.04090409	-0.05391018\\
0.0410041	-0.05396468\\
0.04110411	-0.05648432\\
0.04120412	-0.05890184\\
0.04130413	-0.06193053\\
0.04140414	-0.06058214\\
0.04150415	-0.06320821\\
0.04160416	-0.0649502\\
0.04170417	-0.06765436\\
0.04180418	-0.0686263\\
0.04190419	-0.06997249\\
0.0420042	-0.07113096\\
0.04210421	-0.07005626\\
0.04220422	-0.07271902\\
0.04230423	-0.07205072\\
0.04240424	-0.07337028\\
0.04250425	-0.07151783\\
0.04260426	-0.06918077\\
0.04270427	-0.06727479\\
0.04280428	-0.06810175\\
0.04290429	-0.06425878\\
0.0430043	-0.06482672\\
0.04310431	-0.06657166\\
0.04320432	-0.06592581\\
0.04330433	-0.06762032\\
0.04340434	-0.06897323\\
0.04350435	-0.07102459\\
0.04360436	-0.06920248\\
0.04370437	-0.07229384\\
0.04380438	-0.06906449\\
0.04390439	-0.06753714\\
0.0440044	-0.066529\\
0.04410441	-0.06514304\\
0.04420442	-0.06639197\\
0.04430443	-0.06800862\\
0.04440444	-0.06662439\\
0.04450445	-0.06719995\\
0.04460446	-0.06291771\\
0.04470447	-0.06214345\\
0.04480448	-0.06053961\\
0.04490449	-0.0595035\\
0.0450045	-0.05887928\\
0.04510451	-0.05844176\\
0.04520452	-0.05494067\\
0.04530453	-0.05520517\\
0.04540454	-0.05233786\\
0.04550455	-0.05132344\\
0.04560456	-0.05048623\\
0.04570457	-0.04866819\\
0.04580458	-0.04735799\\
0.04590459	-0.04647429\\
0.0460046	-0.04713287\\
0.04610461	-0.04433747\\
0.04620462	-0.04460825\\
0.04630463	-0.04327483\\
0.04640464	-0.04082057\\
0.04650465	-0.03962222\\
0.04660466	-0.03631582\\
0.04670467	-0.03494665\\
0.04680468	-0.03165671\\
0.04690469	-0.02591893\\
0.0470047	-0.02351838\\
0.04710471	-0.02132667\\
0.04720472	-0.01888687\\
0.04730473	-0.0183637\\
0.04740474	-0.01783603\\
0.04750475	-0.0177333\\
0.04760476	-0.01765326\\
0.04770477	-0.01484551\\
0.04780478	-0.01563567\\
0.04790479	-0.01135496\\
0.0480048	-0.00970951\\
0.04810481	-0.007711804\\
0.04820482	-0.003893395\\
0.04830483	-0.003137666\\
0.04840484	-1.086657e-05\\
0.04850485	0.0002372787\\
0.04860486	0.001315171\\
0.04870487	0.002325827\\
0.04880488	0.00500459\\
0.04890489	0.00702721\\
0.0490049	0.008485616\\
0.04910491	0.009604571\\
0.04920492	0.009750197\\
0.04930493	0.01284496\\
0.04940494	0.0152364\\
0.04950495	0.01868043\\
0.04960496	0.01877152\\
0.04970497	0.02189274\\
0.04980498	0.02354759\\
0.04990499	0.02667703\\
0.050005	0.0283616\\
0.05010501	0.02957494\\
0.05020502	0.03234265\\
0.05030503	0.03190053\\
0.05040504	0.03317559\\
0.05050505	0.03549447\\
0.05060506	0.03808268\\
0.05070507	0.04065325\\
0.05080508	0.04089003\\
0.05090509	0.04199066\\
0.0510051	0.04218183\\
0.05110511	0.04340443\\
0.05120512	0.04367862\\
0.05130513	0.04232836\\
0.05140514	0.04240435\\
0.05150515	0.04417268\\
0.05160516	0.04495163\\
0.05170517	0.04673139\\
0.05180518	0.04909222\\
0.05190519	0.04945762\\
0.0520052	0.04783407\\
0.05210521	0.04810908\\
0.05220522	0.05025043\\
0.05230523	0.0514657\\
0.05240524	0.05247228\\
0.05250525	0.05195905\\
0.05260526	0.05233952\\
0.05270527	0.05423597\\
0.05280528	0.05387954\\
0.05290529	0.05429243\\
0.0530053	0.05215315\\
0.05310531	0.05442737\\
0.05320532	0.05474014\\
0.05330533	0.05458563\\
0.05340534	0.05653501\\
0.05350535	0.05672162\\
0.05360536	0.05704048\\
0.05370537	0.056736\\
0.05380538	0.05848789\\
0.05390539	0.05861187\\
0.0540054	0.05627531\\
0.05410541	0.05353223\\
0.05420542	0.05013097\\
0.05430543	0.05043155\\
0.05440544	0.04998873\\
0.05450545	0.05055124\\
0.05460546	0.04967685\\
0.05470547	0.04961074\\
0.05480548	0.04935713\\
0.05490549	0.04810098\\
0.0550055	0.04908213\\
0.05510551	0.04792568\\
0.05520552	0.04788201\\
0.05530553	0.04882512\\
0.05540554	0.04742946\\
0.05550555	0.04793835\\
0.05560556	0.04583284\\
0.05570557	0.04275178\\
0.05580558	0.04030478\\
0.05590559	0.03964935\\
0.0560056	0.03652161\\
0.05610561	0.03432274\\
0.05620562	0.03182972\\
0.05630563	0.02993116\\
0.05640564	0.03043522\\
0.05650565	0.0295819\\
0.05660566	0.02798182\\
0.05670567	0.02685817\\
0.05680568	0.02718521\\
0.05690569	0.02760648\\
0.0570057	0.02659968\\
0.05710571	0.0259541\\
0.05720572	0.02682298\\
0.05730573	0.02662055\\
0.05740574	0.0239607\\
0.05750575	0.02223825\\
0.05760576	0.02034563\\
0.05770577	0.01718926\\
0.05780578	0.01503301\\
0.05790579	0.01368498\\
0.0580058	0.01141288\\
0.05810581	0.009450221\\
0.05820582	0.008237489\\
0.05830583	0.005596995\\
0.05840584	0.004576484\\
0.05850585	0.003959781\\
0.05860586	0.00154075\\
0.05870587	0.0008838225\\
0.05880588	-0.001620288\\
0.05890589	-0.003534929\\
0.0590059	-0.004661072\\
0.05910591	-0.006889861\\
0.05920592	-0.007609789\\
0.05930593	-0.01152122\\
0.05940594	-0.01263014\\
0.05950595	-0.01532876\\
0.05960596	-0.01239348\\
0.05970597	-0.01266795\\
0.05980598	-0.01310584\\
0.05990599	-0.01576504\\
0.060006	-0.01614588\\
0.06010601	-0.01847327\\
0.06020602	-0.02073057\\
0.06030603	-0.02436295\\
0.06040604	-0.02290182\\
0.06050605	-0.02159578\\
0.06060606	-0.02240361\\
0.06070607	-0.0250799\\
0.06080608	-0.02731127\\
0.06090609	-0.02797343\\
0.0610061	-0.02855243\\
0.06110611	-0.02856199\\
0.06120612	-0.02971267\\
0.06130613	-0.03012353\\
0.06140614	-0.03105161\\
0.06150615	-0.03256968\\
0.06160616	-0.03238544\\
0.06170617	-0.03498193\\
0.06180618	-0.03810678\\
0.06190619	-0.03728747\\
0.0620062	-0.03800694\\
0.06210621	-0.03925509\\
0.06220622	-0.0394937\\
0.06230623	-0.04067501\\
0.06240624	-0.04275407\\
0.06250625	-0.04363483\\
0.06260626	-0.04326313\\
0.06270627	-0.04273223\\
0.06280628	-0.04138333\\
0.06290629	-0.04019809\\
0.0630063	-0.03923106\\
0.06310631	-0.03836992\\
0.06320632	-0.04065039\\
0.06330633	-0.04126967\\
0.06340634	-0.04316424\\
0.06350635	-0.04341464\\
0.06360636	-0.04417875\\
0.06370637	-0.04530127\\
0.06380638	-0.04730741\\
0.06390639	-0.04511555\\
0.0640064	-0.04335128\\
0.06410641	-0.04017735\\
0.06420642	-0.04100841\\
0.06430643	-0.04124671\\
0.06440644	-0.03971614\\
0.06450645	-0.0388373\\
0.06460646	-0.03846366\\
0.06470647	-0.03835447\\
0.06480648	-0.03822401\\
0.06490649	-0.037401\\
0.0650065	-0.03898423\\
0.06510651	-0.03892308\\
0.06520652	-0.03816699\\
0.06530653	-0.03596121\\
0.06540654	-0.03425717\\
0.06550655	-0.03387415\\
0.06560656	-0.03372875\\
0.06570657	-0.03173746\\
0.06580658	-0.0328893\\
0.06590659	-0.03143701\\
0.0660066	-0.03111011\\
0.06610661	-0.03235536\\
0.06620662	-0.03138758\\
0.06630663	-0.02989081\\
0.06640664	-0.02937527\\
0.06650665	-0.02929468\\
0.06660666	-0.02948129\\
0.06670667	-0.02949603\\
0.06680668	-0.02738975\\
0.06690669	-0.02548693\\
0.0670067	-0.02347098\\
0.06710671	-0.02252487\\
0.06720672	-0.02111182\\
0.06730673	-0.0175603\\
0.06740674	-0.0159488\\
0.06750675	-0.01364921\\
0.06760676	-0.01092101\\
0.06770677	-0.009905357\\
0.06780678	-0.01007792\\
0.06790679	-0.01099183\\
0.0680068	-0.008764689\\
0.06810681	-0.006604942\\
0.06820682	-0.005114084\\
0.06830683	-0.005379974\\
0.06840684	-0.00386372\\
0.06850685	-0.00185283\\
0.06860686	-0.002226752\\
0.06870687	-0.002927545\\
0.06880688	-0.00534917\\
0.06890689	-0.00542673\\
0.0690069	-0.003183973\\
0.06910691	-0.001066282\\
0.06920692	0.0009379306\\
0.06930693	0.00273031\\
0.06940694	0.005229169\\
0.06950695	0.007115165\\
0.06960696	0.01066999\\
0.06970697	0.01211214\\
0.06980698	0.01328618\\
0.06990699	0.01400176\\
0.070007	0.01324447\\
0.07010701	0.0124428\\
0.07020702	0.01071786\\
0.07030703	0.01287871\\
0.07040704	0.01443326\\
0.07050705	0.01492055\\
0.07060706	0.01685495\\
0.07070707	0.01718983\\
0.07080708	0.02009459\\
0.07090709	0.02078159\\
0.0710071	0.02045481\\
0.07110711	0.02099339\\
0.07120712	0.02424436\\
0.07130713	0.02788959\\
0.07140714	0.02790845\\
0.07150715	0.0274479\\
0.07160716	0.02781334\\
0.07170717	0.02949628\\
0.07180718	0.02690255\\
0.07190719	0.02787888\\
0.0720072	0.02714577\\
0.07210721	0.02882705\\
0.07220722	0.02960338\\
0.07230723	0.02856871\\
0.07240724	0.02875548\\
0.07250725	0.03006659\\
0.07260726	0.03162799\\
0.07270727	0.0322334\\
0.07280728	0.03142513\\
0.07290729	0.03263295\\
0.0730073	0.03293295\\
0.07310731	0.0324031\\
0.07320732	0.03257003\\
0.07330733	0.03224566\\
0.07340734	0.03273363\\
0.07350735	0.0318241\\
0.07360736	0.03166282\\
0.07370737	0.03103971\\
0.07380738	0.03128422\\
0.07390739	0.0304168\\
0.0740074	0.03087632\\
0.07410741	0.03119068\\
0.07420742	0.0324251\\
0.07430743	0.03227314\\
0.07440744	0.03046526\\
0.07450745	0.03062738\\
0.07460746	0.03166714\\
0.07470747	0.03195768\\
0.07480748	0.03220969\\
0.07490749	0.0316303\\
0.0750075	0.03218975\\
0.07510751	0.03159147\\
0.07520752	0.02953389\\
0.07530753	0.02900866\\
0.07540754	0.03100105\\
0.07550755	0.03165133\\
0.07560756	0.02952487\\
0.07570757	0.02734946\\
0.07580758	0.02868511\\
0.07590759	0.02864897\\
0.0760076	0.02496044\\
0.07610761	0.02111603\\
0.07620762	0.02081262\\
0.07630763	0.02337618\\
0.07640764	0.02239615\\
0.07650765	0.01886922\\
0.07660766	0.01595191\\
0.07670767	0.01607135\\
0.07680768	0.01359176\\
0.07690769	0.01459829\\
0.0770077	0.01452813\\
0.07710771	0.01518518\\
0.07720772	0.01475603\\
0.07730773	0.0159071\\
0.07740774	0.01743529\\
0.07750775	0.01601338\\
0.07760776	0.01406064\\
0.07770777	0.01255768\\
0.07780778	0.01102145\\
0.07790779	0.008700246\\
0.0780078	0.007867096\\
0.07810781	0.008426615\\
0.07820782	0.008251193\\
0.07830783	0.00715011\\
0.07840784	0.005834719\\
0.07850785	0.005229754\\
0.07860786	0.004124016\\
0.07870787	0.002378536\\
0.07880788	0.001820158\\
0.07890789	0.001558729\\
0.0790079	-1.255383e-05\\
0.07910791	-0.001456229\\
0.07920792	-0.003411982\\
0.07930793	-0.004309612\\
0.07940794	-0.004826715\\
0.07950795	-0.006341637\\
0.07960796	-0.007377062\\
0.07970797	-0.009140446\\
0.07980798	-0.008216576\\
0.07990799	-0.008110727\\
0.080008	-0.00980725\\
0.08010801	-0.00927047\\
0.08020802	-0.007542012\\
0.08030803	-0.007843787\\
0.08040804	-0.008764332\\
0.08050805	-0.01335136\\
0.08060806	-0.01534959\\
0.08070807	-0.01534902\\
0.08080808	-0.01625497\\
0.08090809	-0.01818831\\
0.0810081	-0.01862694\\
0.08110811	-0.01914868\\
0.08120812	-0.02048788\\
0.08130813	-0.02158355\\
0.08140814	-0.02138488\\
0.08150815	-0.02071588\\
0.08160816	-0.02035827\\
0.08170817	-0.01988686\\
0.08180818	-0.01851875\\
0.08190819	-0.01813291\\
0.0820082	-0.01873893\\
0.08210821	-0.01771352\\
0.08220822	-0.01760937\\
0.08230823	-0.01928704\\
0.08240824	-0.02145215\\
0.08250825	-0.02334729\\
0.08260826	-0.0248535\\
0.08270827	-0.0255425\\
0.08280828	-0.02532265\\
0.08290829	-0.02581223\\
0.0830083	-0.02447387\\
0.08310831	-0.02494661\\
0.08320832	-0.02666146\\
0.08330833	-0.02354313\\
0.08340834	-0.02458443\\
0.08350835	-0.02507041\\
0.08360836	-0.02633253\\
0.08370837	-0.02568911\\
0.08380838	-0.02633654\\
0.08390839	-0.02752192\\
0.0840084	-0.02620962\\
0.08410841	-0.02546339\\
0.08420842	-0.02534699\\
0.08430843	-0.02564007\\
0.08440844	-0.02487598\\
0.08450845	-0.02442788\\
0.08460846	-0.02522616\\
0.08470847	-0.02558496\\
0.08480848	-0.02406637\\
0.08490849	-0.02358596\\
0.0850085	-0.02436037\\
0.08510851	-0.02508327\\
0.08520852	-0.02279299\\
0.08530853	-0.0203642\\
0.08540854	-0.01809455\\
0.08550855	-0.01823771\\
0.08560856	-0.01536409\\
0.08570857	-0.01414038\\
0.08580858	-0.01560318\\
0.08590859	-0.01536903\\
0.0860086	-0.01736578\\
0.08610861	-0.0184058\\
0.08620862	-0.01912229\\
0.08630863	-0.01909734\\
0.08640864	-0.01937691\\
0.08650865	-0.0181688\\
0.08660866	-0.01789494\\
0.08670867	-0.0166009\\
0.08680868	-0.01707755\\
0.08690869	-0.01767403\\
0.0870087	-0.01483572\\
0.08710871	-0.01088558\\
0.08720872	-0.009783414\\
0.08730873	-0.01100308\\
0.08740874	-0.009379968\\
0.08750875	-0.008023644\\
0.08760876	-0.008922386\\
0.08770877	-0.008853345\\
0.08780878	-0.007306042\\
0.08790879	-0.006796764\\
0.0880088	-0.006230583\\
0.08810881	-0.005562056\\
0.08820882	-0.005771714\\
0.08830883	-0.003816852\\
0.08840884	-0.003020157\\
0.08850885	-0.002347909\\
0.08860886	-0.00059104\\
0.08870887	1.78139e-05\\
0.08880888	0.000762334\\
0.08890889	0.0007332713\\
0.0890089	0.001426321\\
0.08910891	0.00161314\\
0.08920892	0.001190531\\
0.08930893	0.0005824538\\
0.08940894	0.001698636\\
0.08950895	0.002931095\\
0.08960896	0.005319562\\
0.08970897	0.007609883\\
0.08980898	0.008389694\\
0.08990899	0.009770561\\
0.090009	0.01073855\\
0.09010901	0.01113743\\
0.09020902	0.01064404\\
0.09030903	0.009571467\\
0.09040904	0.008732977\\
0.09050905	0.007960501\\
0.09060906	0.0105978\\
0.09070907	0.01169222\\
0.09080908	0.01158518\\
0.09090909	0.01184701\\
0.0910091	0.010253\\
0.09110911	0.01056672\\
0.09120912	0.009502023\\
0.09130913	0.01286015\\
0.09140914	0.01433517\\
0.09150915	0.01209355\\
0.09160916	0.01413023\\
0.09170917	0.01640391\\
0.09180918	0.01605138\\
0.09190919	0.01468179\\
0.0920092	0.0168389\\
0.09210921	0.02111515\\
0.09220922	0.02100676\\
0.09230923	0.01915579\\
0.09240924	0.01992696\\
0.09250925	0.01983613\\
0.09260926	0.01860573\\
0.09270927	0.01885995\\
0.09280928	0.01952437\\
0.09290929	0.01922221\\
0.0930093	0.02122098\\
0.09310931	0.02219518\\
0.09320932	0.02254989\\
0.09330933	0.02135018\\
0.09340934	0.01876597\\
0.09350935	0.01743764\\
0.09360936	0.01577651\\
0.09370937	0.01592902\\
0.09380938	0.01669529\\
0.09390939	0.01703894\\
0.0940094	0.01777246\\
0.09410941	0.01730457\\
0.09420942	0.01786301\\
0.09430943	0.01820074\\
0.09440944	0.01674705\\
0.09450945	0.01781533\\
0.09460946	0.01787624\\
0.09470947	0.01853852\\
0.09480948	0.01859323\\
0.09490949	0.01821477\\
0.0950095	0.01785691\\
0.09510951	0.01706115\\
0.09520952	0.01512826\\
0.09530953	0.0151576\\
0.09540954	0.01606279\\
0.09550955	0.01466308\\
0.09560956	0.01366313\\
0.09570957	0.01240243\\
0.09580958	0.01405632\\
0.09590959	0.0151955\\
0.0960096	0.0169634\\
0.09610961	0.01724693\\
0.09620962	0.01782775\\
0.09630963	0.01509419\\
0.09640964	0.0133627\\
0.09650965	0.01275596\\
0.09660966	0.01299094\\
0.09670967	0.00993605\\
0.09680968	0.007013594\\
0.09690969	0.007064539\\
0.0970097	0.005669072\\
0.09710971	0.005393736\\
0.09720972	0.005102639\\
0.09730973	0.005595443\\
0.09740974	0.005278921\\
0.09750975	0.006241486\\
0.09760976	0.00727175\\
0.09770977	0.006904193\\
0.09780978	0.005759879\\
0.09790979	0.007123543\\
0.0980098	0.005276273\\
0.09810981	0.004777994\\
0.09820982	0.004501624\\
0.09830983	0.003017669\\
0.09840984	-0.0007485647\\
0.09850985	-0.00181459\\
0.09860986	-0.001428311\\
0.09870987	-0.002540984\\
0.09880988	-0.001416866\\
0.09890989	-0.002587194\\
0.0990099	-0.004771637\\
0.09910991	-0.003760599\\
0.09920992	-0.002931016\\
0.09930993	-0.003742714\\
0.09940994	-0.004380697\\
0.09950995	-0.001956908\\
0.09960996	-0.003355996\\
0.09970997	-0.004385533\\
0.09980998	-0.003072239\\
0.09990999	-0.001610172\\
0.10001	-0.003109152\\
0.10011	-0.005383281\\
0.10021	-0.004528272\\
0.10031	-0.006571203\\
0.10041	-0.007359723\\
0.1005101	-0.007776381\\
0.1006101	-0.009597252\\
0.1007101	-0.01123952\\
0.1008101	-0.01044907\\
0.1009101	-0.01078248\\
0.1010101	-0.01208155\\
0.1011101	-0.0123275\\
0.1012101	-0.01337973\\
0.1013101	-0.01628347\\
0.1014101	-0.01513492\\
0.1015102	-0.01320087\\
0.1016102	-0.01467816\\
0.1017102	-0.01350696\\
0.1018102	-0.01175547\\
0.1019102	-0.01370899\\
0.1020102	-0.01254071\\
0.1021102	-0.01364572\\
0.1022102	-0.01418018\\
0.1023102	-0.01281761\\
0.1024102	-0.01270873\\
0.1025103	-0.01184411\\
0.1026103	-0.01361177\\
0.1027103	-0.01594652\\
0.1028103	-0.01402297\\
0.1029103	-0.01266759\\
0.1030103	-0.01119063\\
0.1031103	-0.01033807\\
0.1032103	-0.01109259\\
0.1033103	-0.01285871\\
0.1034103	-0.01293296\\
0.1035104	-0.0153594\\
0.1036104	-0.01569918\\
0.1037104	-0.01529202\\
0.1038104	-0.01670097\\
0.1039104	-0.01610553\\
0.1040104	-0.01612332\\
0.1041104	-0.01450397\\
0.1042104	-0.01611789\\
0.1043104	-0.0150544\\
0.1044104	-0.01415996\\
0.1045105	-0.01245054\\
0.1046105	-0.01171111\\
0.1047105	-0.01098817\\
0.1048105	-0.01016763\\
0.1049105	-0.01081286\\
0.1050105	-0.01303148\\
0.1051105	-0.01383283\\
0.1052105	-0.01447303\\
0.1053105	-0.01527804\\
0.1054105	-0.01459342\\
0.1055106	-0.01273126\\
0.1056106	-0.01057018\\
0.1057106	-0.01058781\\
0.1058106	-0.008157185\\
0.1059106	-0.007597113\\
0.1060106	-0.007159798\\
0.1061106	-0.006300569\\
0.1062106	-0.008437759\\
0.1063106	-0.01047406\\
0.1064106	-0.009678208\\
0.1065107	-0.007948812\\
0.1066107	-0.006947955\\
0.1067107	-0.00551925\\
0.1068107	-0.00427707\\
0.1069107	-0.005621825\\
0.1070107	-0.006349453\\
0.1071107	-0.004540397\\
0.1072107	-0.004215253\\
0.1073107	-0.004820977\\
0.1074107	-0.003363729\\
0.1075108	-0.001930777\\
0.1076108	-0.001975423\\
0.1077108	-0.00218255\\
0.1078108	-0.002836447\\
0.1079108	-0.002896024\\
0.1080108	-0.00234683\\
0.1081108	-0.002558801\\
0.1082108	-0.001826645\\
0.1083108	-0.0002688672\\
0.1084108	-0.0007489465\\
0.1085109	-0.001463453\\
0.1086109	0.0005797952\\
0.1087109	0.0002578831\\
0.1088109	0.001860438\\
0.1089109	0.001789783\\
0.1090109	0.002594728\\
0.1091109	0.002979324\\
0.1092109	0.001673436\\
0.1093109	0.0008318078\\
0.1094109	0.001009891\\
0.109511	0.003059723\\
0.109611	0.004010644\\
0.109711	0.003717168\\
0.109811	0.005571772\\
0.109911	0.006110169\\
0.110011	0.006763947\\
0.110111	0.009685184\\
0.110211	0.009418036\\
0.110311	0.01175453\\
0.110411	0.01147796\\
0.1105111	0.01164186\\
0.1106111	0.009844498\\
0.1107111	0.01121876\\
0.1108111	0.009719719\\
0.1109111	0.007985638\\
0.1110111	0.007095038\\
0.1111111	0.006829085\\
0.1112111	0.007670552\\
0.1113111	0.007644624\\
0.1114111	0.005136604\\
0.1115112	0.003293855\\
0.1116112	0.005324395\\
0.1117112	0.006538952\\
0.1118112	0.008244079\\
0.1119112	0.008927717\\
0.1120112	0.0105355\\
0.1121112	0.01192901\\
0.1122112	0.01429799\\
0.1123112	0.01448753\\
0.1124112	0.01522356\\
0.1125113	0.01263163\\
0.1126113	0.0106871\\
0.1127113	0.01112291\\
0.1128113	0.01028092\\
0.1129113	0.008996325\\
0.1130113	0.008931404\\
0.1131113	0.009876721\\
0.1132113	0.009989983\\
0.1133113	0.01078234\\
0.1134113	0.01165632\\
0.1135114	0.01141371\\
0.1136114	0.01225572\\
0.1137114	0.01315899\\
0.1138114	0.01137617\\
0.1139114	0.009538778\\
0.1140114	0.0107666\\
0.1141114	0.01123859\\
0.1142114	0.01059431\\
0.1143114	0.009045374\\
0.1144114	0.008213733\\
0.1145115	0.01048775\\
0.1146115	0.01102179\\
0.1147115	0.008225928\\
0.1148115	0.00821483\\
0.1149115	0.00821724\\
0.1150115	0.008532815\\
0.1151115	0.009144158\\
0.1152115	0.008272372\\
0.1153115	0.007724836\\
0.1154115	0.007439563\\
0.1155116	0.006409112\\
0.1156116	0.006946267\\
0.1157116	0.008995164\\
0.1158116	0.005503727\\
0.1159116	0.004528187\\
0.1160116	0.003600492\\
0.1161116	0.002754178\\
0.1162116	0.003664232\\
0.1163116	0.003596085\\
0.1164116	0.005350875\\
0.1165117	0.005382127\\
0.1166117	0.006844015\\
0.1167117	0.005306301\\
0.1168117	0.004683491\\
0.1169117	0.002998374\\
0.1170117	0.003020253\\
0.1171117	0.003475375\\
0.1172117	0.002931971\\
0.1173117	0.004420412\\
0.1174117	0.003876887\\
0.1175118	0.002900873\\
0.1176118	0.003098843\\
0.1177118	0.003596409\\
0.1178118	0.002310026\\
0.1179118	0.001238931\\
0.1180118	-0.0003425202\\
0.1181118	-0.001464809\\
0.1182118	-0.001530992\\
0.1183118	-0.00188581\\
0.1184118	-0.003114384\\
0.1185119	-0.002059374\\
0.1186119	-0.002583257\\
0.1187119	-0.003929632\\
0.1188119	-0.004250944\\
0.1189119	-0.002584914\\
0.1190119	-0.001656424\\
0.1191119	-0.002886213\\
0.1192119	-0.004893069\\
0.1193119	-0.006009233\\
0.1194119	-0.006861099\\
0.119512	-0.006475281\\
0.119612	-0.005659378\\
0.119712	-0.00576811\\
0.119812	-0.004297784\\
0.119912	-0.003850239\\
0.120012	-0.005154458\\
0.120112	-0.006519781\\
0.120212	-0.005374743\\
0.120312	-0.003772176\\
0.120412	-0.004049395\\
0.1205121	-0.003509346\\
0.1206121	-0.003690123\\
0.1207121	-0.00159522\\
0.1208121	-0.002763918\\
0.1209121	-0.006000591\\
0.1210121	-0.006499002\\
0.1211121	-0.008758987\\
0.1212121	-0.009015115\\
0.1213121	-0.009091536\\
0.1214121	-0.01118021\\
0.1215122	-0.01056023\\
0.1216122	-0.009222585\\
0.1217122	-0.009246756\\
0.1218122	-0.007864754\\
0.1219122	-0.008644104\\
0.1220122	-0.009155897\\
0.1221122	-0.009619902\\
0.1222122	-0.008967288\\
0.1223122	-0.006500907\\
0.1224122	-0.007435435\\
0.1225123	-0.006569445\\
0.1226123	-0.006825547\\
0.1227123	-0.009549455\\
0.1228123	-0.009347551\\
0.1229123	-0.008444644\\
0.1230123	-0.00804505\\
0.1231123	-0.007355035\\
0.1232123	-0.006820348\\
0.1233123	-0.007283356\\
0.1234123	-0.005671114\\
0.1235124	-0.004738717\\
0.1236124	-0.008585988\\
0.1237124	-0.01085453\\
0.1238124	-0.009980286\\
0.1239124	-0.01112333\\
0.1240124	-0.008627882\\
0.1241124	-0.008734988\\
0.1242124	-0.008407321\\
0.1243124	-0.006833361\\
0.1244124	-0.005916557\\
0.1245125	-0.003823966\\
0.1246125	-0.002443942\\
0.1247125	-0.001791975\\
0.1248125	-0.0002454469\\
0.1249125	-0.001066308\\
0.1250125	-0.003166502\\
0.1251125	-0.004383243\\
0.1252125	-0.00773179\\
0.1253125	-0.007163615\\
0.1254125	-0.006749512\\
0.1255126	-0.007437727\\
0.1256126	-0.006364976\\
0.1257126	-0.00702305\\
0.1258126	-0.00666902\\
0.1259126	-0.005742053\\
0.1260126	-0.006310811\\
0.1261126	-0.004884521\\
0.1262126	-0.002080543\\
0.1263126	-0.001555706\\
0.1264126	-0.003344612\\
0.1265127	-0.001954552\\
0.1266127	-0.001004163\\
0.1267127	-0.0009936044\\
0.1268127	-0.00244968\\
0.1269127	-0.003033493\\
0.1270127	-0.00151749\\
0.1271127	-0.0001871888\\
0.1272127	-0.0002935627\\
0.1273127	-0.0009604826\\
0.1274127	-0.0009715048\\
0.1275128	0.001965749\\
0.1276128	0.001487367\\
0.1277128	0.0003783221\\
0.1278128	0.002772482\\
0.1279128	0.002150845\\
0.1280128	0.005020404\\
0.1281128	0.003677382\\
0.1282128	0.003256723\\
0.1283128	0.002545258\\
0.1284128	0.002946798\\
0.1285129	0.003985993\\
0.1286129	0.001814626\\
0.1287129	-0.000591549\\
0.1288129	-0.0005543072\\
0.1289129	-2.834861e-05\\
0.1290129	0.0002829251\\
0.1291129	0.0008504773\\
0.1292129	0.001600483\\
0.1293129	0.003170989\\
0.1294129	0.003456924\\
0.129513	0.005493122\\
0.129613	0.00573633\\
0.129713	0.003196881\\
0.129813	0.005448776\\
0.129913	0.004839108\\
0.130013	0.003922589\\
0.130113	0.004914403\\
0.130213	0.003570424\\
0.130313	0.006587276\\
0.130413	0.006107693\\
0.1305131	0.005018643\\
0.1306131	0.00601649\\
0.1307131	0.005293899\\
0.1308131	0.005792227\\
0.1309131	0.007838501\\
0.1310131	0.007559528\\
0.1311131	0.006668643\\
0.1312131	0.006477838\\
0.1313131	0.00602342\\
0.1314131	0.008400657\\
0.1315132	0.009076549\\
0.1316132	0.005476359\\
0.1317132	0.004916504\\
0.1318132	0.003905632\\
0.1319132	0.003350459\\
0.1320132	0.002670761\\
0.1321132	0.004598745\\
0.1322132	0.005833628\\
0.1323132	0.008016956\\
0.1324132	0.008147044\\
0.1325133	0.00662176\\
0.1326133	0.00787103\\
0.1327133	0.007533053\\
0.1328133	0.005764285\\
0.1329133	0.005252399\\
0.1330133	0.004752123\\
0.1331133	0.005072541\\
0.1332133	0.005259056\\
0.1333133	0.004648049\\
0.1334133	0.005026757\\
0.1335134	0.006698834\\
0.1336134	0.005579421\\
0.1337134	0.002237011\\
0.1338134	0.001107559\\
0.1339134	0.003373213\\
0.1340134	0.003644273\\
0.1341134	0.002557001\\
0.1342134	0.003152297\\
0.1343134	0.004595455\\
0.1344134	0.006127384\\
0.1345135	0.003222348\\
0.1346135	0.005028077\\
0.1347135	0.004659076\\
0.1348135	0.007303454\\
0.1349135	0.00876019\\
0.1350135	0.007585468\\
0.1351135	0.006961036\\
0.1352135	0.006553091\\
0.1353135	0.003651309\\
0.1354135	0.001673477\\
0.1355136	0.0006338206\\
0.1356136	0.001898517\\
0.1357136	0.002550827\\
0.1358136	0.00103447\\
0.1359136	-0.0009193744\\
0.1360136	-0.001417908\\
0.1361136	-0.0007963781\\
0.1362136	-0.001493725\\
0.1363136	-0.001731796\\
0.1364136	-0.001459554\\
0.1365137	-0.0003106431\\
0.1366137	0.0005964352\\
0.1367137	-0.0003168654\\
0.1368137	0.001191486\\
0.1369137	0.001680708\\
0.1370137	0.0005255067\\
0.1371137	0.001067601\\
0.1372137	-0.0009094623\\
0.1373137	-0.0005817741\\
0.1374137	-9.449187e-05\\
0.1375138	-0.001995466\\
0.1376138	-0.0007647829\\
0.1377138	0.0002135728\\
0.1378138	-0.0002293007\\
0.1379138	0.00140803\\
0.1380138	-0.0002369368\\
0.1381138	-0.0003519484\\
0.1382138	-0.002622244\\
0.1383138	-0.002714722\\
0.1384138	-0.003416034\\
0.1385139	-0.004108154\\
0.1386139	-0.002942803\\
0.1387139	-0.003208268\\
0.1388139	-0.00188164\\
0.1389139	-0.002448473\\
0.1390139	-0.002647167\\
0.1391139	-0.002534838\\
0.1392139	-0.001562042\\
0.1393139	-0.0004194555\\
0.1394139	-0.00290467\\
0.139514	-0.006859209\\
0.139614	-0.005696951\\
0.139714	-0.006096098\\
0.139814	-0.005836684\\
0.139914	-0.00630505\\
0.140014	-0.006249804\\
0.140114	-0.004074972\\
0.140214	-0.005702086\\
0.140314	-0.004315887\\
0.140414	-0.003929936\\
0.1405141	-0.004239153\\
0.1406141	-0.002697748\\
0.1407141	-0.002630279\\
0.1408141	-0.001062603\\
0.1409141	-0.00165108\\
0.1410141	-0.004166495\\
0.1411141	-0.005669584\\
0.1412141	-0.005309355\\
0.1413141	-0.004063678\\
0.1414141	-0.006263586\\
0.1415142	-0.006518687\\
0.1416142	-0.007317928\\
0.1417142	-0.007334219\\
0.1418142	-0.006329047\\
0.1419142	-0.005331982\\
0.1420142	-0.001907862\\
0.1421142	-0.0002772545\\
0.1422142	-0.0007938264\\
0.1423142	-0.0009626779\\
0.1424142	0.0009408894\\
0.1425143	-0.001170595\\
0.1426143	-0.003509715\\
0.1427143	-0.00613415\\
0.1428143	-0.006692504\\
0.1429143	-0.005344667\\
0.1430143	-0.005641025\\
0.1431143	-0.004282061\\
0.1432143	-0.005134131\\
0.1433143	-0.004680221\\
0.1434143	-0.004511351\\
0.1435144	-0.004562743\\
0.1436144	-0.003022746\\
0.1437144	-0.002792783\\
0.1438144	-0.00341976\\
0.1439144	-0.003357926\\
0.1440144	-0.005837524\\
0.1441144	-0.005151754\\
0.1442144	-0.003777573\\
0.1443144	-0.003107227\\
0.1444144	-0.00200001\\
0.1445145	-0.005144245\\
0.1446145	-0.002481087\\
0.1447145	-0.0006245195\\
0.1448145	-0.001376883\\
0.1449145	0.0001516928\\
0.1450145	0.0004069178\\
0.1451145	0.002060898\\
0.1452145	0.001674167\\
0.1453145	0.002011968\\
0.1454145	0.0009945166\\
0.1455146	-0.0002624394\\
0.1456146	0.001625202\\
0.1457146	0.0004105846\\
0.1458146	0.0004284644\\
0.1459146	-0.0007071464\\
0.1460146	-0.002076699\\
0.1461146	-0.002484855\\
0.1462146	-0.002146452\\
0.1463146	-0.003938187\\
0.1464146	-0.002115491\\
0.1465147	-0.002976959\\
0.1466147	-0.003479285\\
0.1467147	-0.000232528\\
0.1468147	-9.351901e-05\\
0.1469147	0.001726377\\
0.1470147	0.00110298\\
0.1471147	0.001791239\\
0.1472147	0.004575124\\
0.1473147	0.003998802\\
0.1474147	0.003803187\\
0.1475148	0.002266697\\
0.1476148	0.002681143\\
0.1477148	0.0006734077\\
0.1478148	-0.001301419\\
0.1479148	0.0005625798\\
0.1480148	0.001988028\\
0.1481148	0.00206473\\
0.1482148	0.001490907\\
0.1483148	0.00312068\\
0.1484148	0.003804113\\
0.1485149	0.003013968\\
0.1486149	0.002077839\\
0.1487149	0.001934369\\
0.1488149	0.001534122\\
0.1489149	0.001918304\\
0.1490149	0.004863268\\
0.1491149	0.004900897\\
0.1492149	0.004854526\\
0.1493149	0.004351911\\
0.1494149	0.003955397\\
0.149515	0.002724868\\
0.149615	0.002673878\\
0.149715	0.003571746\\
0.149815	0.003639889\\
0.149915	0.005015722\\
0.150015	0.003103714\\
0.150115	0.002353445\\
0.150215	0.00340467\\
0.150315	0.000640645\\
0.150415	0.0002746153\\
0.1505151	0.0008605176\\
0.1506151	0.002950685\\
0.1507151	0.003088338\\
0.1508151	0.002800114\\
0.1509151	0.003348406\\
0.1510151	0.004625886\\
0.1511151	0.004208872\\
0.1512151	0.00248256\\
0.1513151	0.002603508\\
0.1514151	0.003054606\\
0.1515152	0.003566632\\
0.1516152	0.002229194\\
0.1517152	0.003731172\\
0.1518152	0.00229308\\
0.1519152	0.002853974\\
0.1520152	0.002187356\\
0.1521152	0.003593173\\
0.1522152	0.003666836\\
0.1523152	0.002449968\\
0.1524152	0.003665361\\
0.1525153	0.005228095\\
0.1526153	0.005591912\\
0.1527153	0.005139767\\
0.1528153	0.004847484\\
0.1529153	0.002792014\\
0.1530153	0.002384723\\
0.1531153	0.002891866\\
0.1532153	0.003393365\\
0.1533153	0.00252004\\
0.1534153	-0.0001341044\\
0.1535154	0.0002246664\\
0.1536154	0.000255826\\
0.1537154	0.0005533063\\
0.1538154	-0.0008657223\\
0.1539154	0.000319153\\
0.1540154	0.0004585162\\
0.1541154	-9.020927e-06\\
0.1542154	-0.00106966\\
0.1543154	-0.002140155\\
0.1544154	-0.0003268033\\
0.1545155	-0.002254377\\
0.1546155	-0.0002971431\\
0.1547155	0.001081545\\
0.1548155	0.003536097\\
0.1549155	0.003300884\\
0.1550155	0.003959531\\
0.1551155	0.005187328\\
0.1552155	0.004088574\\
0.1553155	0.001823904\\
0.1554155	0.000303723\\
0.1555156	0.000388038\\
0.1556156	-0.0007149051\\
0.1557156	-0.001326236\\
0.1558156	-0.001857255\\
0.1559156	6.59543e-05\\
0.1560156	2.429958e-05\\
0.1561156	-0.001591816\\
0.1562156	-0.001563224\\
0.1563156	-0.002858217\\
0.1564156	-0.001117206\\
0.1565157	-0.002229443\\
0.1566157	0.0008639226\\
0.1567157	0.0001471482\\
0.1568157	0.0008527034\\
0.1569157	-0.002165524\\
0.1570157	-0.003032819\\
0.1571157	-0.001210206\\
0.1572157	-0.001392272\\
0.1573157	-0.0005389004\\
0.1574157	-0.003832213\\
0.1575158	-0.00391282\\
0.1576158	-0.004241399\\
0.1577158	-0.003275546\\
0.1578158	-0.001457689\\
0.1579158	0.0001272624\\
0.1580158	0.0006258002\\
0.1581158	0.0005035479\\
0.1582158	0.0006747591\\
0.1583158	-0.001284184\\
0.1584158	-0.002470834\\
0.1585159	-0.004402784\\
0.1586159	-0.006521286\\
0.1587159	-0.00482807\\
0.1588159	-0.004580564\\
0.1589159	-0.004367156\\
0.1590159	-0.003654462\\
0.1591159	-0.003884708\\
0.1592159	-0.002032802\\
0.1593159	-0.000369457\\
0.1594159	0.001383468\\
0.159516	0.0008760122\\
0.159616	0.0007027859\\
0.159716	0.002143807\\
0.159816	0.001896617\\
0.159916	-0.001443223\\
0.160016	-0.001264196\\
0.160116	-0.002068634\\
0.160216	-0.004402844\\
0.160316	-0.005361849\\
0.160416	-0.00466344\\
0.1605161	-0.003122354\\
0.1606161	-0.005634077\\
0.1607161	-0.005029023\\
0.1608161	-0.004858746\\
0.1609161	-0.002137119\\
0.1610161	-0.003603971\\
0.1611161	-0.002005342\\
0.1612161	-0.001969509\\
0.1613161	-0.002690573\\
0.1614161	-0.001490652\\
0.1615162	-0.002622947\\
0.1616162	0.0001985305\\
0.1617162	-0.001652862\\
0.1618162	-0.002041237\\
0.1619162	-0.002695189\\
0.1620162	-0.00334772\\
0.1621162	-0.004149672\\
0.1622162	-0.002636779\\
0.1623162	-0.0007789697\\
0.1624162	0.00022326\\
0.1625163	0.0005297959\\
0.1626163	0.001724864\\
0.1627163	0.001583564\\
0.1628163	-0.0003648636\\
0.1629163	-0.001225688\\
0.1630163	-0.0009077479\\
0.1631163	-0.002095741\\
0.1632163	-0.0007450674\\
0.1633163	0.0001970233\\
0.1634163	0.002428047\\
0.1635164	7.938488e-05\\
0.1636164	-0.0008408771\\
0.1637164	-1.757629e-05\\
0.1638164	-0.001325132\\
0.1639164	-0.002072043\\
0.1640164	-0.004943303\\
0.1641164	-0.003559089\\
0.1642164	-0.003382717\\
0.1643164	-0.001513915\\
0.1644164	-0.0004402856\\
0.1645165	-0.0008454223\\
0.1646165	-0.001336448\\
0.1647165	-0.0004292867\\
0.1648165	0.0009322368\\
0.1649165	0.002535114\\
0.1650165	0.001077692\\
0.1651165	0.001332373\\
0.1652165	0.0005280828\\
0.1653165	0.0009661835\\
0.1654165	0.001186532\\
0.1655166	0.001380219\\
0.1656166	0.0005947954\\
0.1657166	0.002409039\\
0.1658166	0.00330346\\
0.1659166	0.0008609645\\
0.1660166	0.001527372\\
0.1661166	0.0009991277\\
0.1662166	0.0008273606\\
0.1663166	-0.0009769594\\
0.1664166	-0.0006097279\\
0.1665167	-0.002115301\\
0.1666167	-0.001174919\\
0.1667167	0.001898223\\
0.1668167	0.003716358\\
0.1669167	0.001356196\\
0.1670167	0.002179287\\
0.1671167	0.003254659\\
0.1672167	0.004196654\\
0.1673167	0.003341764\\
0.1674167	0.00281159\\
0.1675168	0.0024503\\
0.1676168	0.002694816\\
0.1677168	0.001360779\\
0.1678168	0.00190352\\
0.1679168	3.528997e-05\\
0.1680168	-0.0006460257\\
0.1681168	-0.0002556295\\
0.1682168	-0.0003188461\\
0.1683168	0.0006358008\\
0.1684168	0.0009969649\\
0.1685169	0.0003564473\\
0.1686169	0.0008038345\\
0.1687169	0.0003335928\\
0.1688169	0.0006259953\\
0.1689169	0.001208889\\
0.1690169	0.002291517\\
0.1691169	0.001994207\\
0.1692169	0.001542406\\
0.1693169	0.001602351\\
0.1694169	0.002788716\\
0.169517	0.002724461\\
0.169617	0.003922862\\
0.169717	0.004167846\\
0.169817	0.005388437\\
0.169917	0.00554397\\
0.170017	0.002418417\\
0.170117	0.00195726\\
0.170217	0.001221132\\
0.170317	8.605004e-05\\
0.170417	0.001898823\\
0.1705171	0.002947605\\
0.1706171	0.001801433\\
0.1707171	0.0007356065\\
0.1708171	-0.0007156768\\
0.1709171	0.0003941458\\
0.1710171	0.002072124\\
0.1711171	0.0006477351\\
0.1712171	-0.0006609831\\
0.1713171	-0.0003017127\\
0.1714171	-0.001531734\\
0.1715172	-0.001330335\\
0.1716172	-0.0007041143\\
0.1717172	0.00118399\\
0.1718172	0.0007202189\\
0.1719172	0.0005613346\\
0.1720172	2.247118e-05\\
0.1721172	0.0008976092\\
0.1722172	0.002089587\\
0.1723172	0.001715185\\
0.1724172	0.002107149\\
0.1725173	0.003710057\\
0.1726173	0.004374217\\
0.1727173	0.00189321\\
0.1728173	0.0002233017\\
0.1729173	0.0005895296\\
0.1730173	-0.0001593006\\
0.1731173	-0.0008650975\\
0.1732173	-0.001828492\\
0.1733173	-0.001042325\\
0.1734173	-0.0009122719\\
0.1735174	-0.000205453\\
0.1736174	7.377059e-05\\
0.1737174	0.0007201481\\
0.1738174	0.0009636585\\
0.1739174	0.003807662\\
0.1740174	0.002929539\\
0.1741174	0.001964599\\
0.1742174	0.001223213\\
0.1743174	-0.0004820576\\
0.1744174	-0.003280121\\
0.1745175	-0.003671064\\
0.1746175	-0.00270919\\
0.1747175	-0.002975066\\
0.1748175	-0.002343341\\
0.1749175	-0.002507527\\
0.1750175	-0.002854407\\
0.1751175	-0.002208983\\
0.1752175	-3.878669e-05\\
0.1753175	0.001056576\\
0.1754175	0.0004804061\\
0.1755176	0.002338432\\
0.1756176	0.002884503\\
0.1757176	0.001435736\\
0.1758176	-0.0001829088\\
0.1759176	-0.0007147084\\
0.1760176	-0.002814917\\
0.1761176	-0.003804537\\
0.1762176	-0.002987131\\
0.1763176	-0.002482822\\
0.1764176	-0.004554612\\
0.1765177	-0.00281418\\
0.1766177	-0.001835097\\
0.1767177	-0.0009511032\\
0.1768177	-0.001213568\\
0.1769177	0.0006707857\\
0.1770177	0.0002094259\\
0.1771177	0.001921811\\
0.1772177	0.001555207\\
0.1773177	-0.0008095438\\
0.1774177	0.0009861858\\
0.1775178	0.002410714\\
0.1776178	0.002473939\\
0.1777178	-0.002013392\\
0.1778178	-0.001813895\\
0.1779178	-0.002969298\\
0.1780178	-0.004817465\\
0.1781178	-0.003072532\\
0.1782178	-0.002585723\\
0.1783178	-0.001441365\\
0.1784178	-0.000495867\\
0.1785179	-0.0004091027\\
0.1786179	-0.003847491\\
0.1787179	-0.003887629\\
0.1788179	-0.003321058\\
0.1789179	-0.004318499\\
0.1790179	-0.005938744\\
0.1791179	-0.002862138\\
0.1792179	-0.0004657696\\
0.1793179	-0.0005773477\\
0.1794179	-0.0001042768\\
0.179518	0.00153418\\
0.179618	0.0005441544\\
0.179718	0.002166187\\
0.179818	0.000728536\\
0.179918	0.001360592\\
0.180018	0.001057743\\
0.180118	0.001225564\\
0.180218	0.0005180653\\
0.180318	-0.001843404\\
0.180418	-0.001701471\\
0.1805181	-0.0009419026\\
0.1806181	-0.0003086622\\
0.1807181	0.0003676372\\
0.1808181	-0.0004732075\\
0.1809181	-0.002793388\\
0.1810181	-0.002461226\\
0.1811181	-0.0006221405\\
0.1812181	-0.002200379\\
0.1813181	-0.002449704\\
0.1814181	-0.002235944\\
0.1815182	-0.0007145324\\
0.1816182	0.001862687\\
0.1817182	0.0001814793\\
0.1818182	-6.27023e-05\\
0.1819182	-0.001073006\\
0.1820182	-0.0007254852\\
0.1821182	-0.0006656787\\
0.1822182	-0.001028992\\
0.1823182	-0.002442245\\
0.1824182	-0.0004454093\\
0.1825183	0.001657975\\
0.1826183	0.002812665\\
0.1827183	-4.730181e-05\\
0.1828183	-0.0005253527\\
0.1829183	-0.0002078626\\
0.1830183	-0.0002119992\\
0.1831183	-0.001071809\\
0.1832183	-7.550165e-05\\
0.1833183	0.0008006963\\
0.1834183	0.001477221\\
0.1835184	0.0009397915\\
0.1836184	0.001016394\\
0.1837184	-0.0003521804\\
0.1838184	0.001142194\\
0.1839184	0.001054822\\
0.1840184	0.0009949411\\
0.1841184	-2.058587e-05\\
0.1842184	0.000113548\\
0.1843184	9.96268e-05\\
0.1844184	0.001264821\\
0.1845185	0.001491529\\
0.1846185	0.001130579\\
0.1847185	0.001726598\\
0.1848185	0.002326184\\
0.1849185	0.001139522\\
0.1850185	0.001153374\\
0.1851185	0.002316881\\
0.1852185	0.001026141\\
0.1853185	-0.0006979319\\
0.1854185	-0.001583646\\
0.1855186	-0.002381433\\
0.1856186	-0.001578419\\
0.1857186	0.0002661371\\
0.1858186	0.0005438259\\
0.1859186	5.448989e-06\\
0.1860186	0.001109491\\
0.1861186	0.001831828\\
0.1862186	0.001583387\\
0.1863186	0.001466851\\
0.1864186	0.0004734211\\
0.1865187	0.001048968\\
0.1866187	-0.001482982\\
0.1867187	-0.0004398782\\
0.1868187	0.001401885\\
0.1869187	0.00168467\\
0.1870187	0.001195903\\
0.1871187	0.001270866\\
0.1872187	0.001592489\\
0.1873187	0.004639796\\
0.1874187	0.003590634\\
0.1875188	0.002694186\\
0.1876188	0.0008237635\\
0.1877188	0.0006761105\\
0.1878188	-0.0007881697\\
0.1879188	0.001077609\\
0.1880188	0.002629927\\
0.1881188	0.003368878\\
0.1882188	0.00266646\\
0.1883188	0.00255359\\
0.1884188	0.002339889\\
0.1885189	0.001019989\\
0.1886189	-0.001144152\\
0.1887189	-0.003993535\\
0.1888189	-0.003378674\\
0.1889189	-0.002291528\\
0.1890189	-0.003222224\\
0.1891189	-0.001719767\\
0.1892189	-0.002257491\\
0.1893189	-0.001833389\\
0.1894189	0.001091582\\
0.189519	0.0008594095\\
0.189619	0.0005444162\\
0.189719	0.003447932\\
0.189819	0.003527051\\
0.189919	0.005564993\\
0.190019	0.005036718\\
0.190119	0.005026928\\
0.190219	0.001954089\\
0.190319	0.0005866155\\
0.190419	-0.0006093805\\
0.1905191	-0.002273053\\
0.1906191	-0.001637027\\
0.1907191	-0.001869371\\
0.1908191	-0.001883903\\
0.1909191	-0.002226704\\
0.1910191	-0.001184259\\
0.1911191	-9.365499e-05\\
0.1912191	0.00122881\\
0.1913191	0.001205517\\
0.1914191	0.001256725\\
0.1915192	0.001936691\\
0.1916192	0.002275483\\
0.1917192	0.001282952\\
0.1918192	-0.0006045151\\
0.1919192	-0.0006535804\\
0.1920192	0.0006891222\\
0.1921192	-0.001031851\\
0.1922192	-0.001697907\\
0.1923192	-0.001749983\\
0.1924192	-0.000981364\\
0.1925193	-0.000862345\\
0.1926193	-0.001659489\\
0.1927193	-0.0006662874\\
0.1928193	0.002510838\\
0.1929193	0.002308174\\
0.1930193	-0.0007736076\\
0.1931193	-0.00263513\\
0.1932193	-0.002859054\\
0.1933193	-0.0005468382\\
0.1934193	0.0003207565\\
0.1935194	-0.002549558\\
0.1936194	-0.001465776\\
0.1937194	-0.0009251579\\
0.1938194	-0.002390601\\
0.1939194	-0.002122131\\
0.1940194	-0.0009989574\\
0.1941194	0.001525452\\
0.1942194	0.002261624\\
0.1943194	0.003458622\\
0.1944194	0.001452779\\
0.1945195	0.002279749\\
0.1946195	0.001132872\\
0.1947195	-0.0009863548\\
0.1948195	-0.002424865\\
0.1949195	-0.001090535\\
0.1950195	-0.0001699144\\
0.1951195	-0.0008535325\\
0.1952195	-0.001974234\\
0.1953195	-0.0003133633\\
0.1954195	-9.914302e-05\\
0.1955196	-0.0006304274\\
0.1956196	-0.001195671\\
0.1957196	-0.00184425\\
0.1958196	-0.0005637949\\
0.1959196	-0.001807044\\
0.1960196	-0.001439008\\
0.1961196	-0.0008828719\\
0.1962196	-0.002417242\\
0.1963196	-0.001832228\\
0.1964196	-0.002086971\\
0.1965197	-0.002607299\\
0.1966197	-0.001595324\\
0.1967197	-0.002239746\\
0.1968197	-0.001703685\\
0.1969197	0.0004217366\\
0.1970197	0.0005587028\\
0.1971197	-0.0008338749\\
0.1972197	0.001213551\\
0.1973197	0.001403723\\
0.1974197	0.002014904\\
0.1975198	0.002351623\\
0.1976198	0.0008372588\\
0.1977198	-0.0002511064\\
0.1978198	-0.0008862805\\
0.1979198	0.0002322029\\
0.1980198	-0.0002073874\\
0.1981198	-0.00103404\\
0.1982198	0.000796566\\
0.1983198	-0.0005124131\\
0.1984198	-0.0001412455\\
0.1985199	-0.0004102691\\
0.1986199	-0.0004442189\\
0.1987199	0.0004783706\\
0.1988199	-0.001454203\\
0.1989199	-0.002514393\\
0.1990199	-0.002802756\\
0.1991199	-0.001303431\\
0.1992199	-0.001101341\\
0.1993199	-8.391323e-05\\
0.1994199	-0.001720949\\
0.19952	-0.001973722\\
0.19962	-0.001033857\\
0.19972	-0.0001323427\\
0.19982	0.001149799\\
0.19992	0.00104587\\
0.20002	0.0003027992\\
0.20012	0.0008660448\\
0.20022	0.001797225\\
0.20032	0.0006451922\\
0.20042	-0.0004802411\\
0.2005201	0.0005832864\\
0.2006201	0.0005883432\\
0.2007201	0.0007725425\\
0.2008201	0.0008375568\\
0.2009201	0.0005333004\\
0.2010201	-0.0002358774\\
0.2011201	-3.995476e-05\\
0.2012201	-0.001319826\\
0.2013201	-0.001337417\\
0.2014201	-0.002524224\\
0.2015202	-0.0004340628\\
0.2016202	-0.0006100729\\
0.2017202	-0.001401966\\
0.2018202	0.0009731295\\
0.2019202	0.00265594\\
0.2020202	0.00246161\\
0.2021202	0.002212361\\
0.2022202	0.003660361\\
0.2023202	0.004285048\\
0.2024202	0.002310933\\
0.2025203	0.0008859091\\
0.2026203	0.0007076773\\
0.2027203	-0.000657669\\
0.2028203	-0.0004957923\\
0.2029203	-0.003064757\\
0.2030203	-0.001760481\\
0.2031203	-0.0001065179\\
0.2032203	-0.0007552059\\
0.2033203	-0.002551372\\
0.2034203	-0.002539952\\
0.2035204	-0.003214353\\
0.2036204	-0.002273045\\
0.2037204	-0.0003207364\\
0.2038204	0.0008459021\\
0.2039204	0.002261377\\
0.2040204	0.002723699\\
0.2041204	0.001904314\\
0.2042204	0.00353606\\
0.2043204	0.00444044\\
0.2044204	0.002018916\\
0.2045205	0.0006455437\\
0.2046205	-1.289652e-05\\
0.2047205	0.0003354565\\
0.2048205	0.0007103843\\
0.2049205	0.00136253\\
0.2050205	0.0004561528\\
0.2051205	0.001127423\\
0.2052205	0.002102478\\
0.2053205	0.001056453\\
0.2054205	-0.0005620119\\
0.2055206	-0.001486935\\
0.2056206	-0.0006811257\\
0.2057206	-0.0003286123\\
0.2058206	0.001776393\\
0.2059206	0.003854927\\
0.2060206	0.000985244\\
0.2061206	0.0003577264\\
0.2062206	0.0002810282\\
0.2063206	-0.001570533\\
0.2064206	-0.002367338\\
0.2065207	-0.001295857\\
0.2066207	-0.00222221\\
0.2067207	-0.003073153\\
0.2068207	-0.00208331\\
0.2069207	-0.001750999\\
0.2070207	0.0009920229\\
0.2071207	0.00136061\\
0.2072207	0.001187078\\
0.2073207	0.004208629\\
0.2074207	0.005269918\\
0.2075208	0.003359777\\
0.2076208	0.001557405\\
0.2077208	-0.0005531864\\
0.2078208	-0.001014689\\
0.2079208	-0.001377908\\
0.2080208	-0.001893387\\
0.2081208	-0.0001816026\\
0.2082208	-0.0002537151\\
0.2083208	-0.0004569991\\
0.2084208	-0.000211845\\
0.2085209	0.001040776\\
0.2086209	0.003762345\\
0.2087209	0.00288153\\
0.2088209	0.001033613\\
0.2089209	9.133956e-05\\
0.2090209	-0.00159153\\
0.2091209	-0.00143554\\
0.2092209	-0.001920658\\
0.2093209	-0.000253177\\
0.2094209	-0.0005685546\\
0.209521	-0.0007847903\\
0.209621	-0.001501092\\
0.209721	-0.002195993\\
0.209821	-0.0003670358\\
0.209921	0.0008558934\\
0.210021	0.001779864\\
0.210121	0.0015909\\
0.210221	0.0008269105\\
0.210321	0.001609546\\
0.210421	0.0005553724\\
0.2105211	-0.0008221706\\
0.2106211	-0.002032082\\
0.2107211	-0.001493516\\
0.2108211	-0.0009856136\\
0.2109211	-0.0007646031\\
0.2110211	-0.0005866419\\
0.2111211	-0.00223354\\
0.2112211	-0.002229405\\
0.2113211	-0.002743697\\
0.2114211	-0.001745531\\
0.2115212	-0.0009003351\\
0.2116212	0.001860113\\
0.2117212	0.002550453\\
0.2118212	0.003029453\\
0.2119212	0.002817692\\
0.2120212	0.001541736\\
0.2121212	-9.842506e-05\\
0.2122212	-0.001882633\\
0.2123212	0.0001634253\\
0.2124212	-0.0006217487\\
0.2125213	0.0008992771\\
0.2126213	0.001720851\\
0.2127213	0.0004191458\\
0.2128213	-5.970464e-05\\
0.2129213	0.0005956537\\
0.2130213	0.000146117\\
0.2131213	-0.0003621664\\
0.2132213	-0.001403231\\
0.2133213	-0.002231866\\
0.2134213	-0.002394299\\
0.2135214	-0.003467263\\
0.2136214	-0.003443256\\
0.2137214	-0.001988169\\
0.2138214	-0.002211756\\
0.2139214	-0.003436492\\
0.2140214	-0.001468677\\
0.2141214	-0.00126323\\
0.2142214	0.000131527\\
0.2143214	6.157229e-05\\
0.2144214	0.001648821\\
0.2145215	0.00350192\\
0.2146215	0.001842324\\
0.2147215	0.001505517\\
0.2148215	0.003465915\\
0.2149215	0.00277039\\
0.2150215	0.0009051415\\
0.2151215	-0.0001185797\\
0.2152215	-0.001647897\\
0.2153215	-0.0010293\\
0.2154215	-0.0008721187\\
0.2155216	-0.001635518\\
0.2156216	-0.001340388\\
0.2157216	-0.0006351808\\
0.2158216	-0.001735637\\
0.2159216	0.0007844202\\
0.2160216	0.0007764856\\
0.2161216	-0.0009273096\\
0.2162216	-0.002379738\\
0.2163216	-0.00139439\\
0.2164216	0.001379718\\
0.2165217	0.000817505\\
0.2166217	0.001251522\\
0.2167217	0.001365895\\
0.2168217	-0.0004040765\\
0.2169217	0.000205602\\
0.2170217	-8.728325e-05\\
0.2171217	-0.001559702\\
0.2172217	-0.002589062\\
0.2173217	-0.001785533\\
0.2174217	-0.0004172733\\
0.2175218	0.0006632596\\
0.2176218	-0.001158935\\
0.2177218	-0.001154029\\
0.2178218	0.001038979\\
0.2179218	-0.000373802\\
0.2180218	-0.0001148966\\
0.2181218	-0.0002033285\\
0.2182218	0.001587986\\
0.2183218	0.002462874\\
0.2184218	0.001786642\\
0.2185219	0.0008005494\\
0.2186219	0.002101333\\
0.2187219	-0.0002750239\\
0.2188219	-0.001773722\\
0.2189219	-0.0001490183\\
0.2190219	5.642273e-05\\
0.2191219	-0.001226506\\
0.2192219	-0.001157646\\
0.2193219	-0.0008272075\\
0.2194219	0.0005162221\\
0.219522	0.0008607081\\
0.219622	-0.001004138\\
0.219722	0.001748135\\
0.219822	0.002779537\\
0.219922	0.001605754\\
0.220022	0.002134162\\
0.220122	0.002952162\\
0.220222	0.002116797\\
0.220322	0.0003754184\\
0.220422	-0.003004059\\
0.2205221	-0.003562019\\
0.2206221	-0.001704722\\
0.2207221	-0.001247947\\
0.2208221	-0.0004960549\\
0.2209221	-0.0009575668\\
0.2210221	-0.0009761461\\
0.2211221	-0.001052009\\
0.2212221	-0.000910057\\
0.2213221	0.0002194795\\
0.2214221	-0.001059391\\
0.2215222	8.384058e-05\\
0.2216222	0.0005313135\\
0.2217222	0.001521091\\
0.2218222	0.001480538\\
0.2219222	0.0005639029\\
0.2220222	0.002902885\\
0.2221222	0.002835563\\
0.2222222	0.0002885955\\
0.2223222	0.002299655\\
0.2224222	-0.0001685256\\
0.2225223	0.0002435475\\
0.2226223	0.001053302\\
0.2227223	0.001315119\\
0.2228223	0.001930425\\
0.2229223	0.0004753145\\
0.2230223	0.0008871332\\
0.2231223	0.001612101\\
0.2232223	0.0001194033\\
0.2233223	-0.0004737706\\
0.2234223	-0.0004373596\\
0.2235224	-0.00122373\\
0.2236224	-0.0009471333\\
0.2237224	-0.002020491\\
0.2238224	-0.002888654\\
0.2239224	-0.002230489\\
0.2240224	-0.002203523\\
0.2241224	-0.003398193\\
0.2242224	-0.001332881\\
0.2243224	-7.496695e-05\\
0.2244224	0.001886836\\
0.2245225	0.002112052\\
0.2246225	0.003381171\\
0.2247225	0.003553636\\
0.2248225	0.003700863\\
0.2249225	0.001192799\\
0.2250225	0.0004093498\\
0.2251225	0.0004885439\\
0.2252225	1.37124e-05\\
0.2253225	-0.0001122789\\
0.2254225	-0.001399263\\
0.2255226	-0.0007510417\\
0.2256226	-0.002441163\\
0.2257226	-0.001161953\\
0.2258226	-0.0007851118\\
0.2259226	-0.0007195869\\
0.2260226	0.001457142\\
0.2261226	0.0007514673\\
0.2262226	0.001369\\
0.2263226	0.001993891\\
0.2264226	0.0001088954\\
0.2265227	0.0006827203\\
0.2266227	0.0001739406\\
0.2267227	-0.0008234683\\
0.2268227	7.604361e-05\\
0.2269227	-0.001024304\\
0.2270227	0.0001084464\\
0.2271227	-0.0004680174\\
0.2272227	-0.001764055\\
0.2273227	2.92788e-05\\
0.2274227	0.0008745793\\
0.2275228	0.001601539\\
0.2276228	0.003493956\\
0.2277228	0.001882316\\
0.2278228	0.0003140678\\
0.2279228	-0.003014\\
0.2280228	-0.005187966\\
0.2281228	-0.003761201\\
0.2282228	-0.003698596\\
0.2283228	-0.001462345\\
0.2284228	-0.001798623\\
0.2285229	0.0004897395\\
0.2286229	0.0001255203\\
0.2287229	0.001812735\\
0.2288229	0.003402532\\
0.2289229	0.003677457\\
0.2290229	0.00118234\\
0.2291229	0.0002431565\\
0.2292229	0.0008688039\\
0.2293229	-3.86706e-05\\
0.2294229	-0.001292147\\
0.229523	-0.0006495243\\
0.229623	0.0007655376\\
0.229723	-0.0009356017\\
0.229823	0.001533686\\
0.229923	-0.0003177789\\
0.230023	-0.000708393\\
0.230123	-0.0003987042\\
0.230223	-0.0008484147\\
0.230323	0.0005878796\\
0.230423	0.0001969138\\
0.2305231	-0.0007567344\\
0.2306231	0.001955811\\
0.2307231	0.002252127\\
0.2308231	0.0001294474\\
0.2309231	-0.001443583\\
0.2310231	-0.0007873145\\
0.2311231	-0.00119825\\
0.2312231	-0.003019461\\
0.2313231	-0.004455934\\
0.2314231	-0.003994267\\
0.2315232	-0.002616246\\
0.2316232	-0.002007276\\
0.2317232	0.0005298648\\
0.2318232	0.001698643\\
0.2319232	0.0005179168\\
0.2320232	0.0004020955\\
0.2321232	0.001216884\\
0.2322232	0.001665962\\
0.2323232	0.002268089\\
0.2324232	0.003174518\\
0.2325233	0.001874137\\
0.2326233	0.0002900939\\
0.2327233	0.0009736716\\
0.2328233	0.001390603\\
0.2329233	0.001164269\\
0.2330233	-0.001729182\\
0.2331233	-0.0009187303\\
0.2332233	0.0004259662\\
0.2333233	0.0008834084\\
0.2334233	-0.001782005\\
0.2335234	-0.002333113\\
0.2336234	-0.001774204\\
0.2337234	-0.001850344\\
0.2338234	-0.001662204\\
0.2339234	-0.001312403\\
0.2340234	-0.001386435\\
0.2341234	0.0008434827\\
0.2342234	0.001678711\\
0.2343234	0.001649304\\
0.2344234	0.002130089\\
0.2345235	-0.0001621263\\
0.2346235	-0.0006963649\\
0.2347235	-0.0001818749\\
0.2348235	0.0002967759\\
0.2349235	-0.0006885367\\
0.2350235	-0.0002141046\\
0.2351235	-0.0001734948\\
0.2352235	0.0003305517\\
0.2353235	-0.0008799461\\
0.2354235	-0.001587743\\
0.2355236	0.0009539597\\
0.2356236	0.002164461\\
0.2357236	-0.0003108278\\
0.2358236	-0.001889399\\
0.2359236	0.0001024141\\
0.2360236	0.001075541\\
0.2361236	0.00156154\\
0.2362236	-0.001292868\\
0.2363236	-0.002059595\\
0.2364236	0.000415195\\
0.2365237	5.035785e-05\\
0.2366237	0.0002042022\\
0.2367237	-6.705518e-05\\
0.2368237	-2.302669e-05\\
0.2369237	0.001426147\\
0.2370237	0.002034499\\
0.2371237	0.002838011\\
0.2372237	0.001240157\\
0.2373237	0.001114735\\
0.2374237	0.002785291\\
0.2375238	0.001256141\\
0.2376238	0.001372281\\
0.2377238	1.206498e-05\\
0.2378238	-0.00131515\\
0.2379238	-0.003250396\\
0.2380238	-0.004357749\\
0.2381238	-0.002257159\\
0.2382238	-0.002960978\\
0.2383238	-0.00150749\\
0.2384238	-0.0004994301\\
0.2385239	-0.0006221617\\
0.2386239	0.0001280868\\
0.2387239	0.0001562563\\
0.2388239	0.0009933118\\
0.2389239	-0.0002586496\\
0.2390239	0.0009222755\\
0.2391239	0.001436734\\
0.2392239	0.001848105\\
0.2393239	0.003182107\\
0.2394239	0.003396673\\
0.239524	0.0002299653\\
0.239624	-0.0003970944\\
0.239724	-0.0007676582\\
0.239824	-0.0007137598\\
0.239924	-0.0001984528\\
0.240024	-0.000788494\\
0.240124	0.002643046\\
0.240224	0.001384476\\
0.240324	0.0001925755\\
0.240424	6.729669e-05\\
0.2405241	0.001129035\\
0.2406241	0.001994225\\
0.2407241	-0.0008770685\\
0.2408241	-0.001004314\\
0.2409241	-0.001399957\\
0.2410241	-0.0004389508\\
0.2411241	0.0005063852\\
0.2412241	-0.0009000089\\
0.2413241	-0.0002152734\\
0.2414241	-0.00080141\\
0.2415242	-0.001593677\\
0.2416242	-0.002408385\\
0.2417242	-0.003356553\\
0.2418242	-0.0003902098\\
0.2419242	0.00138837\\
0.2420242	0.0007019232\\
0.2421242	0.002912809\\
0.2422242	0.001012156\\
0.2423242	0.0002904007\\
0.2424242	0.0008096154\\
0.2425243	-0.0001822554\\
0.2426243	0.0007758951\\
0.2427243	-0.0003132802\\
0.2428243	0.000439115\\
0.2429243	0.0002542852\\
0.2430243	-0.001203154\\
0.2431243	0.0003692891\\
0.2432243	-0.0001471127\\
0.2433243	0.001290946\\
0.2434243	0.002994214\\
0.2435244	0.003092807\\
0.2436244	0.001919666\\
0.2437244	-0.001143855\\
0.2438244	-0.002692478\\
0.2439244	-0.003026699\\
0.2440244	-0.002838347\\
0.2441244	-0.0009629185\\
0.2442244	-0.0009449401\\
0.2443244	-0.001712459\\
0.2444244	-0.0003762403\\
0.2445245	0.0004476783\\
0.2446245	0.0005996841\\
0.2447245	0.002569721\\
0.2448245	0.001951847\\
0.2449245	0.003547675\\
0.2450245	0.003356594\\
0.2451245	0.002521247\\
0.2452245	0.0005915974\\
0.2453245	-0.002421857\\
0.2454245	-0.002489331\\
0.2455246	-0.004040297\\
0.2456246	-0.002980261\\
0.2457246	-0.002282543\\
0.2458246	-0.002482152\\
0.2459246	-0.002556822\\
0.2460246	-0.001945102\\
0.2461246	-0.001075184\\
0.2462246	0.001173755\\
0.2463246	0.0006772029\\
0.2464246	0.002285209\\
0.2465247	0.002257579\\
0.2466247	0.002654576\\
0.2467247	0.002349679\\
0.2468247	0.0009905337\\
0.2469247	0.002155985\\
0.2470247	0.0003234449\\
0.2471247	-0.0009165821\\
0.2472247	-0.001175667\\
0.2473247	-0.0006132573\\
0.2474247	0.0009488939\\
0.2475248	0.0009876515\\
0.2476248	0.001984547\\
0.2477248	0.0005358468\\
0.2478248	-0.0001442152\\
0.2479248	0.001130554\\
0.2480248	-0.001928852\\
0.2481248	-0.00279143\\
0.2482248	-0.002009548\\
0.2483248	-0.0009412398\\
0.2484248	-0.001253542\\
0.2485249	-0.002274398\\
0.2486249	-0.001073339\\
0.2487249	0.0001527323\\
0.2488249	-0.001486054\\
0.2489249	-0.001722357\\
0.2490249	-0.0008492891\\
0.2491249	0.0005970937\\
0.2492249	0.0009853035\\
0.2493249	0.0006932373\\
0.2494249	0.0007210453\\
0.249525	0.001616882\\
0.249625	0.0007606142\\
0.249725	3.639786e-07\\
0.249825	-0.0004525142\\
0.249925	0.0008824529\\
0.250025	0.00363285\\
0.250125	0.001859801\\
0.250225	0.001858903\\
0.250325	0.0002502849\\
0.250425	-0.000962136\\
0.2505251	-0.001726465\\
0.2506251	-0.001759969\\
0.2507251	-0.001997433\\
0.2508251	-0.001536878\\
0.2509251	0.0009677916\\
0.2510251	-0.0005271457\\
0.2511251	-0.0007586574\\
0.2512251	0.0008147309\\
0.2513251	0.002444378\\
0.2514251	0.00173525\\
0.2515252	-0.001283248\\
0.2516252	-0.002022951\\
0.2517252	-0.001208473\\
0.2518252	0.0009420282\\
0.2519252	0.00195221\\
0.2520252	0.0003398277\\
0.2521252	-0.001763984\\
0.2522252	0.0009261444\\
0.2523252	0.000408711\\
0.2524252	-0.0008823193\\
0.2525253	-0.001628098\\
0.2526253	-0.001533327\\
0.2527253	-0.002267596\\
0.2528253	-0.001880465\\
0.2529253	9.321714e-05\\
0.2530253	0.001131325\\
0.2531253	0.002330765\\
0.2532253	0.002354986\\
0.2533253	0.002014792\\
0.2534253	0.002027143\\
0.2535254	0.002821959\\
0.2536254	-0.0004902862\\
0.2537254	-0.003315525\\
0.2538254	-0.001466144\\
0.2539254	-0.001022539\\
0.2540254	-0.002408894\\
0.2541254	-0.001460907\\
0.2542254	0.0005868816\\
0.2543254	0.001202237\\
0.2544254	0.0007611881\\
0.2545255	0.002107041\\
0.2546255	0.002495361\\
0.2547255	0.001975603\\
0.2548255	0.0001071276\\
0.2549255	0.001941476\\
0.2550255	-0.000349757\\
0.2551255	0.0006419391\\
0.2552255	0.001434012\\
0.2553255	-0.001301649\\
0.2554255	-0.0007761312\\
0.2555256	0.0005030974\\
0.2556256	-0.0005572612\\
0.2557256	-0.00342031\\
0.2558256	-0.003121402\\
0.2559256	-0.001436419\\
0.2560256	-0.002647719\\
0.2561256	-0.0009022266\\
0.2562256	-0.0006305481\\
0.2563256	0.0007024008\\
0.2564256	0.001088212\\
0.2565257	0.0004729741\\
0.2566257	-0.001521576\\
0.2567257	-0.0001152808\\
0.2568257	0.002159057\\
0.2569257	0.0009179619\\
0.2570257	0.002333008\\
0.2571257	0.001339464\\
0.2572257	0.002518252\\
0.2573257	0.0008167053\\
0.2574257	0.00201085\\
0.2575258	0.001473159\\
0.2576258	0.001296561\\
0.2577258	0.0009043335\\
0.2578258	0.001375748\\
0.2579258	0.0007088135\\
0.2580258	-0.0009813908\\
0.2581258	-0.0009594621\\
0.2582258	-0.00286537\\
0.2583258	-0.003248088\\
0.2584258	-0.001629293\\
0.2585259	-0.001902659\\
0.2586259	-0.001448968\\
0.2587259	-0.001240628\\
0.2588259	-0.001018787\\
0.2589259	-0.0009940997\\
0.2590259	0.0009135079\\
0.2591259	0.002106182\\
0.2592259	0.001541666\\
0.2593259	0.001927423\\
0.2594259	0.0005149736\\
0.259526	0.001182216\\
0.259626	0.001823821\\
0.259726	0.0003209115\\
0.259826	-0.0007758637\\
0.259926	-0.0008726287\\
0.260026	-0.0009751452\\
0.260126	-0.001062981\\
0.260226	-0.0007439466\\
0.260326	-0.0007168441\\
0.260426	-0.001134601\\
0.2605261	-0.001409467\\
0.2606261	-0.0005708684\\
0.2607261	0.001017365\\
0.2608261	0.003277632\\
0.2609261	0.003352668\\
0.2610261	0.00220702\\
0.2611261	0.001408133\\
0.2612261	-0.0004451142\\
0.2613261	-0.001978115\\
0.2614261	-0.00229583\\
0.2615262	-0.000857283\\
0.2616262	-0.0007077487\\
0.2617262	-0.0007569275\\
0.2618262	-0.00087995\\
0.2619262	-0.0001540014\\
0.2620262	0.0006589712\\
0.2621262	0.001266901\\
0.2622262	0.001797027\\
0.2623262	0.0006422874\\
0.2624262	0.002218887\\
0.2625263	0.002007973\\
0.2626263	-0.0002597078\\
0.2627263	-0.0007286727\\
0.2628263	-0.00221054\\
0.2629263	-0.001373113\\
0.2630263	-0.003166043\\
0.2631263	-0.003166137\\
0.2632263	-0.002298875\\
0.2633263	-0.002612052\\
0.2634263	0.0003501922\\
0.2635264	0.0002059214\\
0.2636264	0.0007176609\\
0.2637264	0.002879447\\
0.2638264	0.002142241\\
0.2639264	0.001194985\\
0.2640264	0.001788293\\
0.2641264	0.001468275\\
0.2642264	0.0003752139\\
0.2643264	-0.002124063\\
0.2644264	-0.0005253256\\
0.2645265	0.0006430619\\
0.2646265	-0.0001038945\\
0.2647265	-0.0004590687\\
0.2648265	-0.0007538474\\
0.2649265	0.00147942\\
0.2650265	0.002359196\\
0.2651265	0.002808783\\
0.2652265	0.000631621\\
0.2653265	0.0007698917\\
0.2654265	-0.001312005\\
0.2655266	-0.002399118\\
0.2656266	-0.000761615\\
0.2657266	-0.001226285\\
0.2658266	0.0005361997\\
0.2659266	-0.0005544369\\
0.2660266	-0.0001629642\\
0.2661266	-0.0006987812\\
0.2662266	-0.001936714\\
0.2663266	-0.002533716\\
0.2664266	-0.003245479\\
0.2665267	-0.001356619\\
0.2666267	-0.0001834954\\
0.2667267	0.001170235\\
0.2668267	-0.0007818913\\
0.2669267	0.0004669843\\
0.2670267	0.0005036356\\
0.2671267	0.002554717\\
0.2672267	0.001651467\\
0.2673267	0.0007578209\\
0.2674267	0.001620237\\
0.2675268	0.0008340872\\
0.2676268	0.003297007\\
0.2677268	0.002468315\\
0.2678268	0.002659394\\
0.2679268	0.0002040514\\
0.2680268	0.0002322088\\
0.2681268	-0.001249505\\
0.2682268	-0.003559278\\
0.2683268	-0.004020802\\
0.2684268	-0.002285659\\
0.2685269	-0.0007210167\\
0.2686269	-0.001891754\\
0.2687269	-0.000429861\\
0.2688269	2.326858e-05\\
0.2689269	0.0009939682\\
0.2690269	0.0008420014\\
0.2691269	0.001235086\\
0.2692269	0.001506906\\
0.2693269	0.001498181\\
0.2694269	0.001262033\\
0.269527	-7.162033e-05\\
0.269627	0.002017895\\
0.269727	-0.0001055586\\
0.269827	-0.001553494\\
0.269927	-0.001717732\\
0.270027	-0.0003317458\\
0.270127	-0.0002803295\\
0.270227	-0.001343111\\
0.270327	0.0003651027\\
0.270427	-0.0001331007\\
0.2705271	-2.722943e-05\\
0.2706271	-0.001889634\\
0.2707271	-0.000760784\\
0.2708271	-0.0006257916\\
0.2709271	0.0001783639\\
0.2710271	0.0008272112\\
0.2711271	-0.000699361\\
0.2712271	0.00108646\\
0.2713271	0.0012152\\
0.2714271	0.001584577\\
0.2715272	-0.0003828794\\
0.2716272	0.0001222871\\
0.2717272	0.0002726675\\
0.2718272	0.001226813\\
0.2719272	0.002772262\\
0.2720272	0.00130929\\
0.2721272	0.001089788\\
0.2722272	0.0005254804\\
0.2723272	9.085463e-05\\
0.2724272	0.001838914\\
0.2725273	7.613e-05\\
0.2726273	-0.001248555\\
0.2727273	-0.002178977\\
0.2728273	-0.002066386\\
0.2729273	-0.00334365\\
0.2730273	-0.002014582\\
0.2731273	-0.001510396\\
0.2732273	-0.0004506573\\
0.2733273	0.001242993\\
0.2734273	-0.0004034082\\
0.2735274	0.001186381\\
0.2736274	0.001191729\\
0.2737274	0.001651414\\
0.2738274	0.0001917306\\
0.2739274	1.09861e-05\\
0.2740274	-0.001650709\\
0.2741274	-0.003304307\\
0.2742274	-0.0003455276\\
0.2743274	0.0007746113\\
0.2744274	0.0001036849\\
0.2745275	0.0002907737\\
0.2746275	0.00234632\\
0.2747275	0.001233454\\
0.2748275	0.002093498\\
0.2749275	0.001847192\\
0.2750275	0.0009537757\\
0.2751275	0.002320842\\
0.2752275	0.00072747\\
0.2753275	0.0008443229\\
0.2754275	-2.852249e-05\\
0.2755276	-0.0009434757\\
0.2756276	-0.001697789\\
0.2757276	-0.0009857275\\
0.2758276	-0.001534056\\
0.2759276	0.0004829269\\
0.2760276	-0.001052853\\
0.2761276	-0.0009631136\\
0.2762276	-0.001684313\\
0.2763276	-0.002499847\\
0.2764276	-0.00300444\\
0.2765277	-0.001201173\\
0.2766277	0.00108621\\
0.2767277	0.002680895\\
0.2768277	0.003624591\\
0.2769277	0.003241879\\
0.2770277	0.001464492\\
0.2771277	-0.001192683\\
0.2772277	-0.001988198\\
0.2773277	-0.003742951\\
0.2774277	-0.002034785\\
0.2775278	-7.098086e-05\\
0.2776278	0.0017583\\
0.2777278	0.0004109748\\
0.2778278	0.0007919324\\
0.2779278	0.00187614\\
0.2780278	0.001098013\\
0.2781278	0.0004621814\\
0.2782278	0.001281479\\
0.2783278	0.001687303\\
0.2784278	0.0003936172\\
0.2785279	-0.001100899\\
0.2786279	-0.002146453\\
0.2787279	-0.001218693\\
0.2788279	0.0002093002\\
0.2789279	-0.00105831\\
0.2790279	-0.001772643\\
0.2791279	-0.001655746\\
0.2792279	-0.0009438105\\
0.2793279	-0.0003524031\\
0.2794279	0.0006046608\\
0.279528	0.001891628\\
0.279628	0.003312051\\
0.279728	0.003405804\\
0.279828	0.002416554\\
0.279928	0.002342079\\
0.280028	0.0007368697\\
0.280128	-0.001764903\\
0.280228	-0.0003847578\\
0.280328	-0.0009999322\\
0.280428	-0.002529719\\
0.2805281	-0.003967191\\
0.2806281	-0.004282545\\
0.2807281	-0.004914261\\
0.2808281	-0.001167164\\
0.2809281	0.000610514\\
0.2810281	0.0006005916\\
0.2811281	0.0009657263\\
0.2812281	0.0005873575\\
0.2813281	0.001575342\\
0.2814281	0.002590981\\
0.2815282	0.0005951878\\
0.2816282	0.0007142736\\
0.2817282	0.001077976\\
0.2818282	0.001084648\\
0.2819282	0.001999793\\
0.2820282	0.00212808\\
0.2821282	0.001284737\\
0.2822282	-0.001072352\\
0.2823282	-0.0002300336\\
0.2824282	0.001184126\\
0.2825283	0.00204476\\
0.2826283	-9.958045e-05\\
0.2827283	-0.001450837\\
0.2828283	-0.001846624\\
0.2829283	-0.0002464621\\
0.2830283	-0.002750049\\
0.2831283	-0.002918591\\
0.2832283	-0.0007816122\\
0.2833283	-1.006489e-05\\
0.2834283	-0.001969555\\
0.2835284	-0.0004082749\\
0.2836284	0.001766841\\
0.2837284	0.00151062\\
0.2838284	0.0002970445\\
0.2839284	-1.290363e-05\\
0.2840284	0.0005411049\\
0.2841284	-0.001226806\\
0.2842284	-0.0005040721\\
0.2843284	-0.0009724946\\
0.2844284	-8.37001e-05\\
0.2845285	-0.001013171\\
0.2846285	1.444273e-05\\
0.2847285	0.00118156\\
0.2848285	0.00173037\\
0.2849285	0.000630811\\
0.2850285	0.0009844692\\
0.2851285	0.001497031\\
0.2852285	0.001178288\\
0.2853285	0.0007356336\\
0.2854285	0.0006136192\\
0.2855286	0.0004884469\\
0.2856286	-0.0001423799\\
0.2857286	-0.002163912\\
0.2858286	0.0001236951\\
0.2859286	-0.00183014\\
0.2860286	-0.002019254\\
0.2861286	0.0003310211\\
0.2862286	-0.0006136398\\
0.2863286	0.002159365\\
0.2864286	0.0009953862\\
0.2865287	0.0004671463\\
0.2866287	0.0002238777\\
0.2867287	0.000458653\\
0.2868287	-0.001500958\\
0.2869287	-0.001416546\\
0.2870287	-0.0003535531\\
0.2871287	-0.001010364\\
0.2872287	-0.0007607258\\
0.2873287	0.001051946\\
0.2874287	0.0008262089\\
0.2875288	-0.0007144562\\
0.2876288	-2.705259e-05\\
0.2877288	0.002017759\\
0.2878288	0.0007815661\\
0.2879288	0.0002465384\\
0.2880288	-0.001521664\\
0.2881288	-0.0005799902\\
0.2882288	0.001519505\\
0.2883288	2.131325e-05\\
0.2884288	-0.001624461\\
0.2885289	-0.001636839\\
0.2886289	-0.0009261069\\
0.2887289	-0.0007685583\\
0.2888289	-0.000225832\\
0.2889289	-0.0007300148\\
0.2890289	-0.0001727582\\
0.2891289	0.00130635\\
0.2892289	0.001000077\\
0.2893289	0.002069133\\
0.2894289	0.002411096\\
0.289529	0.00329537\\
0.289629	0.003577188\\
0.289729	0.001333281\\
0.289829	-0.00188629\\
0.289929	-0.002531428\\
0.290029	-0.0001759847\\
0.290129	-0.0001562019\\
0.290229	-0.0004895931\\
0.290329	-0.002398234\\
0.290429	3.569093e-05\\
0.2905291	-0.001003819\\
0.2906291	-0.001331814\\
0.2907291	-0.001654859\\
0.2908291	-0.001483888\\
0.2909291	-0.0003876397\\
0.2910291	0.0002287947\\
0.2911291	-0.0006075796\\
0.2912291	0.0001177253\\
0.2913291	-0.0006147233\\
0.2914291	-0.0007232343\\
0.2915292	-0.0001950264\\
0.2916292	-0.0003487777\\
0.2917292	0.0001067645\\
0.2918292	0.0005268341\\
0.2919292	0.0008665653\\
0.2920292	0.0006058868\\
0.2921292	0.001174808\\
0.2922292	0.002321467\\
0.2923292	0.002586101\\
0.2924292	0.002853472\\
0.2925293	0.00270466\\
0.2926293	0.002828445\\
0.2927293	0.001487614\\
0.2928293	0.000474865\\
0.2929293	-0.003851179\\
0.2930293	-0.0050448\\
0.2931293	-0.003721976\\
0.2932293	-0.002916482\\
0.2933293	-0.001252767\\
0.2934293	-0.0004168679\\
0.2935294	-0.001984458\\
0.2936294	0.0003876833\\
0.2937294	0.001262724\\
0.2938294	0.0009583531\\
0.2939294	0.0008878951\\
0.2940294	0.0002388535\\
0.2941294	0.002189045\\
0.2942294	0.001229209\\
0.2943294	0.001941731\\
0.2944294	0.0003560331\\
0.2945295	0.000618583\\
0.2946295	-0.0001114452\\
0.2947295	-0.001530632\\
0.2948295	-0.003310825\\
0.2949295	-0.002330028\\
0.2950295	0.0003072054\\
0.2951295	-0.0003679807\\
0.2952295	-0.00172022\\
0.2953295	-0.000237609\\
0.2954295	0.000388317\\
0.2955296	-7.307831e-05\\
0.2956296	0.001530586\\
0.2957296	0.001865918\\
0.2958296	0.001567362\\
0.2959296	0.001858383\\
0.2960296	0.0009071709\\
0.2961296	0.001473542\\
0.2962296	0.0006239717\\
0.2963296	-0.001716845\\
0.2964296	-0.002224143\\
0.2965297	-0.002453581\\
0.2966297	-0.002299428\\
0.2967297	-0.0003941501\\
0.2968297	3.028855e-05\\
0.2969297	0.00161745\\
0.2970297	0.002124747\\
0.2971297	0.002550369\\
0.2972297	0.001897847\\
0.2973297	0.0008833332\\
0.2974297	-0.001157361\\
0.2975298	-0.001345928\\
0.2976298	-0.002515925\\
0.2977298	-0.001312303\\
0.2978298	1.418747e-05\\
0.2979298	4.511107e-05\\
0.2980298	-0.001657446\\
0.2981298	-0.00134175\\
0.2982298	-0.001086476\\
0.2983298	0.001045405\\
0.2984298	6.610153e-05\\
0.2985299	-0.0006898964\\
0.2986299	-8.632136e-05\\
0.2987299	0.0004577213\\
0.2988299	-0.001447192\\
0.2989299	-0.002189788\\
0.2990299	0.001088719\\
0.2991299	0.001302243\\
0.2992299	-0.0002198929\\
0.2993299	0.0006960775\\
0.2994299	0.002300402\\
0.29953	0.00277373\\
0.29963	-6.468351e-05\\
0.29973	0.0007203882\\
0.29983	0.001759137\\
0.29993	0.001134169\\
0.30003	0.001542434\\
0.30013	-0.0002037673\\
0.30023	-0.0007085385\\
0.30033	-0.0008457642\\
0.30043	-0.0005348616\\
0.3005301	0.0006238546\\
0.3006301	-0.001291144\\
0.3007301	-0.003206995\\
0.3008301	-0.003538659\\
0.3009301	-0.002057956\\
0.3010301	-0.001945079\\
0.3011301	0.0003044457\\
0.3012301	0.0005650013\\
0.3013301	0.0007416474\\
0.3014301	0.0001695375\\
0.3015302	-0.0001358614\\
0.3016302	0.001058271\\
0.3017302	-0.0001432019\\
0.3018302	-0.002312334\\
0.3019302	-0.0004532125\\
0.3020302	0.0006126111\\
0.3021302	0.001308023\\
0.3022302	0.001162916\\
0.3023302	0.002587727\\
0.3024302	0.00191651\\
0.3025303	0.00260592\\
0.3026303	0.001286706\\
0.3027303	0.0007697991\\
0.3028303	-0.0002154663\\
0.3029303	-0.0009461096\\
0.3030303	-0.003901308\\
0.3031303	-0.00299655\\
0.3032303	-0.003188715\\
0.3033303	-0.001247698\\
0.3034303	0.0005529436\\
0.3035304	0.001287317\\
0.3036304	0.002031469\\
0.3037304	0.0006984368\\
0.3038304	-0.0006580389\\
0.3039304	-0.0004528813\\
0.3040304	0.00177111\\
0.3041304	0.0009548545\\
0.3042304	-0.0003481398\\
0.3043304	3.700877e-06\\
0.3044304	-0.0002377998\\
0.3045305	0.0009680149\\
0.3046305	-0.0002490362\\
0.3047305	-0.001368356\\
0.3048305	-0.0009942994\\
0.3049305	-0.001884938\\
0.3050305	-2.570166e-05\\
0.3051305	0.001072113\\
0.3052305	0.0005173727\\
0.3053305	0.0006730857\\
0.3054305	0.0004306103\\
0.3055306	8.654249e-05\\
0.3056306	-0.002267251\\
0.3057306	-0.003713009\\
0.3058306	-0.002157506\\
0.3059306	-0.001057144\\
0.3060306	-4.187443e-06\\
0.3061306	-0.0002735648\\
0.3062306	0.001471907\\
0.3063306	0.001241003\\
0.3064306	0.002261241\\
0.3065307	7.961422e-05\\
0.3066307	0.002337176\\
0.3067307	0.002179761\\
0.3068307	0.002816721\\
0.3069307	0.001638335\\
0.3070307	-0.0001387829\\
0.3071307	-0.0001289019\\
0.3072307	0.001170117\\
0.3073307	-0.002159548\\
0.3074307	-0.003958594\\
0.3075308	-0.002342759\\
0.3076308	-0.001736778\\
0.3077308	-0.0009667692\\
0.3078308	-0.0001655783\\
0.3079308	-0.001022246\\
0.3080308	0.0004118113\\
0.3081308	0.000779049\\
0.3082308	-0.0006798274\\
0.3083308	-0.0004540233\\
0.3084308	0.0007520657\\
0.3085309	0.001470283\\
0.3086309	0.001902148\\
0.3087309	-0.001102154\\
0.3088309	-0.001878509\\
0.3089309	-0.001390619\\
0.3090309	-0.003238204\\
0.3091309	-0.00191728\\
0.3092309	-0.001670065\\
0.3093309	4.490895e-05\\
0.3094309	0.00220978\\
0.309531	0.0005283361\\
0.309631	0.0005762595\\
0.309731	0.002153436\\
0.309831	0.001868267\\
0.309931	0.00268817\\
0.310031	0.003039143\\
0.310131	0.002715524\\
0.310231	0.0008066011\\
0.310331	0.0005300958\\
0.310431	-0.002233818\\
0.3105311	-0.002350471\\
0.3106311	-0.002717121\\
0.3107311	-0.002914329\\
0.3108311	-0.001849366\\
0.3109311	-0.002097902\\
0.3110311	-0.001088704\\
0.3111311	5.285618e-06\\
0.3112311	-0.001437726\\
0.3113311	-0.0001019175\\
0.3114311	0.0007018093\\
0.3115312	0.001182242\\
0.3116312	0.0007814998\\
0.3117312	0.0004981519\\
0.3118312	-0.0003811491\\
0.3119312	0.0003102432\\
0.3120312	-0.0001101143\\
0.3121312	0.001445553\\
0.3122312	0.0003684172\\
0.3123312	1.162477e-05\\
0.3124312	-0.002132017\\
0.3125313	-0.001099049\\
0.3126313	-0.001457773\\
0.3127313	0.0005414135\\
0.3128313	0.001804714\\
0.3129313	0.0002264196\\
0.3130313	-0.000595771\\
0.3131313	0.001197179\\
0.3132313	3.404775e-05\\
0.3133313	-0.001589327\\
0.3134313	-0.00233221\\
0.3135314	-0.00248798\\
0.3136314	-0.0003146591\\
0.3137314	0.001122876\\
0.3138314	0.0006049748\\
0.3139314	-0.0002561598\\
0.3140314	0.0004626032\\
0.3141314	0.001071265\\
0.3142314	0.0004664466\\
0.3143314	6.71352e-06\\
0.3144314	0.001394669\\
0.3145315	0.002701357\\
0.3146315	9.6731e-05\\
0.3147315	-0.001031093\\
0.3148315	0.0006102326\\
0.3149315	-0.0001639691\\
0.3150315	-0.0002878736\\
0.3151315	-0.0008698606\\
0.3152315	-0.0005199133\\
0.3153315	-0.002643577\\
0.3154315	-0.004398449\\
0.3155316	-0.004753797\\
0.3156316	-0.001954158\\
0.3157316	0.0008828037\\
0.3158316	0.00107503\\
0.3159316	0.0001797854\\
0.3160316	-1.452045e-06\\
0.3161316	0.001182045\\
0.3162316	0.0005624831\\
0.3163316	-6.750945e-05\\
0.3164316	0.0003690849\\
0.3165317	-5.471032e-05\\
0.3166317	-0.0005193869\\
0.3167317	0.0002233079\\
0.3168317	0.000143027\\
0.3169317	0.00173417\\
0.3170317	0.001308282\\
0.3171317	0.002060326\\
0.3172317	0.002171796\\
0.3173317	0.001788709\\
0.3174317	1.002712e-05\\
0.3175318	-0.00120193\\
0.3176318	-0.003479529\\
0.3177318	-0.001990803\\
0.3178318	-0.001510567\\
0.3179318	-0.002056124\\
0.3180318	-0.001100246\\
0.3181318	-0.001163633\\
0.3182318	0.0003696569\\
0.3183318	-0.0006242913\\
0.3184318	-0.0002532779\\
0.3185319	0.0008857615\\
0.3186319	-0.0002711326\\
0.3187319	-0.0002867121\\
0.3188319	-0.00117871\\
0.3189319	-0.0001745237\\
0.3190319	-0.001007072\\
0.3191319	-0.001773197\\
0.3192319	0.0002610674\\
0.3193319	-7.487189e-05\\
0.3194319	-0.001449348\\
0.319532	0.0009367013\\
0.319632	0.001284161\\
0.319732	0.0005991779\\
0.319832	0.001838588\\
0.319932	-0.0007110719\\
0.320032	-0.0007232443\\
0.320132	0.0009813908\\
0.320232	-0.0008371141\\
0.320332	-0.0001158375\\
0.320432	-0.0006373778\\
0.3205321	-0.001228469\\
0.3206321	0.0007058475\\
0.3207321	-0.001238241\\
0.3208321	2.626148e-05\\
0.3209321	-9.619486e-05\\
0.3210321	0.0002633135\\
0.3211321	0.00033451\\
0.3212321	-0.000186467\\
0.3213321	-0.0005030736\\
0.3214321	-0.0003639396\\
0.3215322	-0.0007097202\\
0.3216322	-0.001422428\\
0.3217322	-0.0008346815\\
0.3218322	-0.0006283042\\
0.3219322	-0.001189576\\
0.3220322	-0.001194728\\
0.3221322	-0.0004013785\\
0.3222322	-0.0009148231\\
0.3223322	-0.0009538389\\
0.3224322	-0.0002712206\\
0.3225323	0.002226914\\
0.3226323	0.00339783\\
0.3227323	0.003111674\\
0.3228323	0.000810259\\
0.3229323	0.0001164954\\
0.3230323	-0.0007261058\\
0.3231323	-0.004179662\\
0.3232323	-0.006628982\\
0.3233323	-0.004334931\\
0.3234323	-0.003494967\\
0.3235324	-0.0006830396\\
0.3236324	-0.0003140338\\
0.3237324	0.0004066749\\
0.3238324	0.001389147\\
0.3239324	0.002560008\\
0.3240324	0.002684113\\
0.3241324	0.001810461\\
0.3242324	0.001306407\\
0.3243324	0.002689929\\
0.3244324	0.002379972\\
0.3245325	0.0005812463\\
0.3246325	4.728244e-05\\
0.3247325	-0.0009605069\\
0.3248325	-0.002054741\\
0.3249325	-0.003159755\\
0.3250325	-0.000853876\\
0.3251325	-0.0009820417\\
0.3252325	-0.001337458\\
0.3253325	-0.00102057\\
0.3254325	-0.003160919\\
0.3255326	-0.001504772\\
0.3256326	-0.001533446\\
0.3257326	-0.001736226\\
0.3258326	0.0009125384\\
0.3259326	0.0001978979\\
0.3260326	0.0006082437\\
0.3261326	-0.001200717\\
0.3262326	-0.001017763\\
0.3263326	0.0003656086\\
0.3264326	-0.0008792431\\
0.3265327	-0.001456963\\
0.3266327	-0.001937971\\
0.3267327	-0.001565613\\
0.3268327	0.001421598\\
0.3269327	0.002426809\\
0.3270327	0.001658314\\
0.3271327	0.001108189\\
0.3272327	0.001950307\\
0.3273327	0.002275875\\
0.3274327	-0.0002704806\\
0.3275328	-0.0008586258\\
0.3276328	-0.001144086\\
0.3277328	-0.002086015\\
0.3278328	-0.001525757\\
0.3279328	-0.0001772769\\
0.3280328	-0.0009790987\\
0.3281328	-0.001987672\\
0.3282328	-0.003042141\\
0.3283328	-0.001481522\\
0.3284328	-0.001596994\\
0.3285329	-0.0006271338\\
0.3286329	0.0008388976\\
0.3287329	-0.0003483538\\
0.3288329	0.0004430521\\
0.3289329	0.001120073\\
0.3290329	0.002380885\\
0.3291329	0.0007129257\\
0.3292329	-0.0009106375\\
0.3293329	-0.001934362\\
0.3294329	-0.002666552\\
0.329533	-0.001326958\\
0.329633	0.0004221451\\
0.329733	-0.002441974\\
0.329833	-0.001211885\\
0.329933	1.304047e-05\\
0.330033	-0.0006210105\\
0.330133	6.825042e-05\\
0.330233	0.0006438059\\
0.330333	0.0006671907\\
0.330433	-0.001695501\\
0.3305331	-0.0003348051\\
0.3306331	-0.0002423446\\
0.3307331	-0.0007082531\\
0.3308331	-0.001404322\\
0.3309331	0.0003797229\\
0.3310331	0.0003979188\\
0.3311331	-8.80966e-05\\
0.3312331	-0.0004682668\\
0.3313331	-0.001668439\\
0.3314331	-0.001209752\\
0.3315332	0.000953796\\
0.3316332	0.0004472172\\
0.3317332	-0.00149772\\
0.3318332	0.0006392673\\
0.3319332	0.001165602\\
0.3320332	-6.63478e-05\\
0.3321332	6.575669e-05\\
0.3322332	-0.0007056354\\
0.3323332	-0.001635743\\
0.3324332	-0.002024388\\
0.3325333	-0.001100421\\
0.3326333	-0.0002221204\\
0.3327333	-0.0006643074\\
0.3328333	-0.0003461006\\
0.3329333	-0.001155817\\
0.3330333	-0.00181648\\
0.3331333	-0.001141291\\
0.3332333	-0.001186706\\
0.3333333	-0.001425781\\
0.3334333	-0.001332076\\
0.3335334	-0.0002465065\\
0.3336334	-0.001303513\\
0.3337334	-0.001558624\\
0.3338334	0.0002234615\\
0.3339334	-8.406239e-05\\
0.3340334	-0.001291735\\
0.3341334	-0.001200895\\
0.3342334	-0.0003087997\\
0.3343334	0.0002359153\\
0.3344334	0.001310231\\
0.3345335	0.002722055\\
0.3346335	0.002410099\\
0.3347335	0.002525981\\
0.3348335	0.0004497361\\
0.3349335	-0.0008532369\\
0.3350335	-0.001717686\\
0.3351335	-0.0005860023\\
0.3352335	-0.001658796\\
0.3353335	-0.001219225\\
0.3354335	4.599532e-05\\
0.3355336	-0.002145859\\
0.3356336	-0.003673063\\
0.3357336	-0.004267627\\
0.3358336	-0.002204521\\
0.3359336	-0.001697548\\
0.3360336	-0.0006630645\\
0.3361336	-0.0001231815\\
0.3362336	-0.000538762\\
0.3363336	-0.001383378\\
0.3364336	0.0004945768\\
0.3365337	-0.0001146591\\
0.3366337	-0.0009087771\\
0.3367337	-0.0005344932\\
0.3368337	0.0006039033\\
0.3369337	0.002227739\\
0.3370337	0.001220614\\
0.3371337	0.0003366572\\
0.3372337	0.00016121\\
0.3373337	0.0005752386\\
0.3374337	-7.175762e-05\\
0.3375338	-0.001840667\\
0.3376338	-0.001177846\\
0.3377338	-0.001875291\\
0.3378338	-0.0033434\\
0.3379338	-0.002976645\\
0.3380338	-0.003316561\\
0.3381338	-0.001417074\\
0.3382338	-0.0009618097\\
0.3383338	0.0006178337\\
0.3384338	0.0002968012\\
0.3385339	0.002407164\\
0.3386339	0.003689435\\
0.3387339	0.001971141\\
0.3388339	0.001726024\\
0.3389339	-0.0007548207\\
0.3390339	-0.003846218\\
0.3391339	-0.004303648\\
0.3392339	-0.003470544\\
0.3393339	-0.001707529\\
0.3394339	-0.0003773942\\
0.339534	-0.0007440238\\
0.339634	-0.0002753371\\
0.339734	0.0006034007\\
0.339834	0.00129439\\
0.339934	-0.000332102\\
0.340034	-0.002025657\\
0.340134	-7.660542e-05\\
0.340234	-0.001122589\\
0.340334	-0.0005669644\\
0.340434	3.057264e-05\\
0.3405341	-0.001867739\\
0.3406341	-0.001933553\\
0.3407341	-0.004201885\\
0.3408341	-0.001975184\\
0.3409341	-0.0009468533\\
0.3410341	-0.002745793\\
0.3411341	0.0002773493\\
0.3412341	-0.000436587\\
0.3413341	0.001106927\\
0.3414341	0.002618304\\
0.3415342	0.0007899121\\
0.3416342	0.00198092\\
0.3417342	0.001777566\\
0.3418342	0.002235857\\
0.3419342	-0.000382145\\
0.3420342	-0.002416733\\
0.3421342	-0.002090829\\
0.3422342	-0.002246263\\
0.3423342	-0.002494115\\
0.3424342	-0.002200563\\
0.3425343	-0.003577086\\
0.3426343	0.0002794485\\
0.3427343	0.0005696078\\
0.3428343	-0.000623666\\
0.3429343	-0.0008984286\\
0.3430343	0.0002097239\\
0.3431343	0.0005239777\\
0.3432343	-0.001299068\\
0.3433343	-0.001348731\\
0.3434343	-0.002335718\\
0.3435344	-0.002013894\\
0.3436344	-0.002427935\\
0.3437344	-0.001562056\\
0.3438344	-0.002173438\\
0.3439344	-0.00151349\\
0.3440344	-0.002006083\\
0.3441344	0.001553677\\
0.3442344	0.0004537065\\
0.3443344	0.0003875987\\
0.3444344	0.0007436225\\
0.3445345	-0.001039503\\
0.3446345	0.0002487812\\
0.3447345	0.0007821888\\
0.3448345	0.001411222\\
0.3449345	0.002207365\\
0.3450345	0.0001428108\\
0.3451345	-0.001692177\\
0.3452345	-0.001890018\\
0.3453345	-0.001351018\\
0.3454345	0.0005388532\\
0.3455346	-0.0007896261\\
0.3456346	-0.003044921\\
0.3457346	-0.004717968\\
0.3458346	-0.003368159\\
0.3459346	-0.001102932\\
0.3460346	-0.00141762\\
0.3461346	-0.0009501017\\
0.3462346	0.0001778289\\
0.3463346	0.001748042\\
0.3464346	0.0007759074\\
0.3465347	-0.0009141882\\
0.3466347	-0.002387727\\
0.3467347	6.134463e-05\\
0.3468347	-0.001073597\\
0.3469347	-0.001908121\\
0.3470347	-0.0006093223\\
0.3471347	0.0001645893\\
0.3472347	0.0005356228\\
0.3473347	-0.0005302525\\
0.3474347	0.0003362831\\
0.3475348	-0.0002783703\\
0.3476348	0.0006018782\\
0.3477348	-0.0004861998\\
0.3478348	-0.002716002\\
0.3479348	-0.003350741\\
0.3480348	-0.003562638\\
0.3481348	-0.002766835\\
0.3482348	-0.0004840892\\
0.3483348	-0.001128133\\
0.3484348	-0.00028357\\
0.3485349	-0.0006160842\\
0.3486349	0.0008624937\\
0.3487349	0.0001065778\\
0.3488349	-0.002124075\\
0.3489349	0.0004391646\\
0.3490349	0.001964882\\
0.3491349	0.001257499\\
0.3492349	0.001468635\\
0.3493349	0.0008525721\\
0.3494349	0.0002864118\\
0.349535	-0.000815903\\
0.349635	-0.00175311\\
0.349735	-0.001257567\\
0.349835	-0.002455035\\
0.349935	-0.002608143\\
0.350035	-0.002808103\\
0.350135	-0.002676507\\
0.350235	-0.002293672\\
0.350335	-0.002635742\\
0.350435	-0.0009451124\\
0.3505351	0.0002708423\\
0.3506351	-3.710488e-05\\
0.3507351	-0.001602039\\
0.3508351	-0.00161833\\
0.3509351	-5.108705e-05\\
0.3510351	-0.001914269\\
0.3511351	3.305482e-05\\
0.3512351	0.0005521506\\
0.3513351	0.001843192\\
0.3514351	-0.001244345\\
0.3515352	-0.001125484\\
0.3516352	0.0001878435\\
0.3517352	-0.0005742415\\
0.3518352	-0.0009354352\\
0.3519352	-0.0005492271\\
0.3520352	0.00111462\\
0.3521352	-0.0002327899\\
0.3522352	-0.0001348385\\
0.3523352	-0.001012619\\
0.3524352	-0.001556223\\
0.3525353	-0.001939038\\
0.3526353	0.0005516857\\
0.3527353	0.001687816\\
0.3528353	-0.000105848\\
0.3529353	-0.001632309\\
0.3530353	-0.002182378\\
0.3531353	-0.001968053\\
0.3532353	-0.001761068\\
0.3533353	-0.003005202\\
0.3534353	-0.002250974\\
0.3535354	-0.0004430906\\
0.3536354	5.90338e-05\\
0.3537354	-0.001533099\\
0.3538354	-0.001501759\\
0.3539354	-0.0006262573\\
0.3540354	-0.001688832\\
0.3541354	-0.002273687\\
0.3542354	0.000189196\\
0.3543354	-4.540047e-05\\
0.3544354	0.0002308552\\
0.3545355	0.001915025\\
0.3546355	0.002595534\\
0.3547355	0.002212021\\
0.3548355	-0.001017352\\
0.3549355	-0.002262546\\
0.3550355	-0.003544187\\
0.3551355	-0.002495841\\
0.3552355	-0.001381642\\
0.3553355	-0.0008290144\\
0.3554355	-0.0004499328\\
0.3555356	-0.001119731\\
0.3556356	-0.001165378\\
0.3557356	-0.0006967802\\
0.3558356	-4.366458e-05\\
0.3559356	0.001322274\\
0.3560356	0.0004688842\\
0.3561356	0.0003655219\\
0.3562356	-0.0002257888\\
0.3563356	-0.0001814944\\
0.3564356	-0.001214211\\
0.3565357	-0.002398152\\
0.3566357	-0.002144337\\
0.3567357	-0.002685643\\
0.3568357	-0.002643646\\
0.3569357	-0.001537794\\
0.3570357	0.000248964\\
0.3571357	0.0003903171\\
0.3572357	0.002036326\\
0.3573357	0.001813622\\
0.3574357	0.0003741777\\
0.3575358	0.001994952\\
0.3576358	0.001442742\\
0.3577358	-0.0008598559\\
0.3578358	-0.003531891\\
0.3579358	-0.004250766\\
0.3580358	-0.003821504\\
0.3581358	-0.003368302\\
0.3582358	-0.004193292\\
0.3583358	-0.002924154\\
0.3584358	-0.0003108171\\
0.3585359	-0.0001516929\\
0.3586359	0.0002854429\\
0.3587359	0.001411552\\
0.3588359	0.002660972\\
0.3589359	0.002389261\\
0.3590359	0.001626336\\
0.3591359	0.00174188\\
0.3592359	0.001141597\\
0.3593359	0.00138824\\
0.3594359	-0.0007814123\\
0.359536	-0.002485046\\
0.359636	-0.0002696985\\
0.359736	-0.001402252\\
0.359836	-0.001930593\\
0.359936	-0.0003490029\\
0.360036	-0.001602058\\
0.360136	-0.0008472338\\
0.360236	-0.001387405\\
0.360336	-0.001684238\\
0.360436	-0.0007102378\\
0.3605361	-0.001552865\\
0.3606361	-0.00214382\\
0.3607361	-0.001686003\\
0.3608361	-0.001193935\\
0.3609361	-0.001222636\\
0.3610361	-0.0005182266\\
0.3611361	-0.0005536281\\
0.3612361	0.00114055\\
0.3613361	0.000620225\\
0.3614361	-0.0002764039\\
0.3615362	-0.001760687\\
0.3616362	-0.0005604358\\
0.3617362	0.001622908\\
0.3618362	0.001192823\\
0.3619362	0.0007211055\\
0.3620362	0.001098435\\
0.3621362	0.001691619\\
0.3622362	0.001136171\\
0.3623362	-0.001042121\\
0.3624362	-0.003220644\\
0.3625363	-0.002472554\\
0.3626363	-0.002240217\\
0.3627363	-0.001014045\\
0.3628363	-0.0006045493\\
0.3629363	0.0004062804\\
0.3630363	0.000941905\\
0.3631363	-0.0001175872\\
0.3632363	7.875066e-05\\
0.3633363	-5.494749e-05\\
0.3634363	0.0009581768\\
0.3635364	-0.0002272532\\
0.3636364	-0.001499439\\
0.3637364	-0.001640289\\
0.3638364	5.536865e-05\\
0.3639364	0.0001110795\\
0.3640364	-0.0002145461\\
0.3641364	-0.000503794\\
0.3642364	-0.001686646\\
0.3643364	-0.0004212801\\
0.3644364	0.000438484\\
0.3645365	-0.0001816311\\
0.3646365	-0.00247939\\
0.3647365	-0.0003608944\\
0.3648365	0.00112765\\
0.3649365	-0.0006871689\\
0.3650365	-1.408399e-05\\
0.3651365	0.0001058671\\
0.3652365	0.002184841\\
0.3653365	0.00197916\\
0.3654365	0.001010568\\
0.3655366	-0.001535968\\
0.3656366	-0.001696535\\
0.3657366	-0.00147548\\
0.3658366	-0.004202898\\
0.3659366	-0.0001758225\\
0.3660366	-0.00130295\\
0.3661366	-0.0001798798\\
0.3662366	0.001653968\\
0.3663366	0.002406803\\
0.3664366	0.001590618\\
0.3665367	0.001092263\\
0.3666367	0.00149145\\
0.3667367	0.0002169095\\
0.3668367	-0.0006049918\\
0.3669367	-0.000168968\\
0.3670367	0.001277898\\
0.3671367	0.00144948\\
0.3672367	-0.0006602434\\
0.3673367	3.501206e-05\\
0.3674367	-3.62188e-06\\
0.3675368	-0.002031951\\
0.3676368	-0.002290794\\
0.3677368	-0.0009309722\\
0.3678368	0.001430905\\
0.3679368	0.0004206267\\
0.3680368	-0.002063216\\
0.3681368	-0.001327338\\
0.3682368	-0.0007645718\\
0.3683368	-0.002215073\\
0.3684368	-0.001274461\\
0.3685369	-0.0002701646\\
0.3686369	0.002314657\\
0.3687369	0.001420407\\
0.3688369	0.001938579\\
0.3689369	0.001866368\\
0.3690369	0.0005407463\\
0.3691369	0.0008260983\\
0.3692369	-0.001105451\\
0.3693369	0.0005440145\\
0.3694369	0.001977371\\
0.369537	0.001561916\\
0.369637	0.001221611\\
0.369737	0.001608514\\
0.369837	0.000866924\\
0.369937	0.0008775565\\
0.370037	0.002068302\\
0.370137	-0.0008620778\\
0.370237	-0.0005855798\\
0.370337	-0.0003127474\\
0.370437	-0.0006254298\\
0.3705371	-0.001368813\\
0.3706371	-0.001839477\\
0.3707371	-0.0004798739\\
0.3708371	-0.0007190275\\
0.3709371	-0.0007360674\\
0.3710371	-0.0004943441\\
0.3711371	0.0003416616\\
0.3712371	0.0003404834\\
0.3713371	0.0003231196\\
0.3714371	0.0002066512\\
0.3715372	0.001542916\\
0.3716372	0.002491384\\
0.3717372	0.003712811\\
0.3718372	0.003187687\\
0.3719372	0.0009807786\\
0.3720372	0.001633042\\
0.3721372	0.003188187\\
0.3722372	0.0007589055\\
0.3723372	-0.001044971\\
0.3724372	-0.001711859\\
0.3725373	-0.003722701\\
0.3726373	-0.003337121\\
0.3727373	-0.0004561586\\
0.3728373	0.001350084\\
0.3729373	0.001836132\\
0.3730373	0.001839235\\
0.3731373	0.000990058\\
0.3732373	0.001298275\\
0.3733373	0.003180131\\
0.3734373	0.002750865\\
0.3735374	0.002578112\\
0.3736374	0.002165425\\
0.3737374	0.001012706\\
0.3738374	0.001112112\\
0.3739374	0.0005033328\\
0.3740374	-0.000709637\\
0.3741374	-0.0003895638\\
0.3742374	0.0002068659\\
0.3743374	-0.0005372053\\
0.3744374	-0.000914195\\
0.3745375	0.0002748997\\
0.3746375	0.002293317\\
0.3747375	0.002158033\\
0.3748375	0.001708216\\
0.3749375	0.002758487\\
0.3750375	0.003527158\\
0.3751375	0.001946315\\
0.3752375	0.000446559\\
0.3753375	-0.003012771\\
0.3754375	-0.001128158\\
0.3755376	-0.00112165\\
0.3756376	-0.0003848139\\
0.3757376	-0.00060322\\
0.3758376	0.0008535175\\
0.3759376	0.002585215\\
0.3760376	0.003095625\\
0.3761376	0.003934284\\
0.3762376	0.002534677\\
0.3763376	0.002878041\\
0.3764376	0.002464663\\
0.3765377	0.002308424\\
0.3766377	0.00179855\\
0.3767377	0.001753589\\
0.3768377	-0.0005116968\\
0.3769377	-0.0008587879\\
0.3770377	0.001704274\\
0.3771377	0.002674419\\
0.3772377	0.001489086\\
0.3773377	0.001323946\\
0.3774377	0.002452322\\
0.3775378	0.003536511\\
0.3776378	0.002425324\\
0.3777378	0.00194707\\
0.3778378	0.0001690307\\
0.3779378	0.00127659\\
0.3780378	0.0004957466\\
0.3781378	0.0003130415\\
0.3782378	-0.001330984\\
0.3783378	-4.740534e-05\\
0.3784378	-0.0007443696\\
0.3785379	2.124886e-05\\
0.3786379	0.001051345\\
0.3787379	0.00185831\\
0.3788379	0.003050161\\
0.3789379	0.00174653\\
0.3790379	0.003753836\\
0.3791379	0.003474906\\
0.3792379	0.003865123\\
0.3793379	0.002983316\\
0.3794379	0.003629002\\
0.379538	0.003505035\\
0.379638	0.003125584\\
0.379738	0.001595312\\
0.379838	0.0006318343\\
0.379938	0.002633841\\
0.380038	-0.0006003287\\
0.380138	0.002148278\\
0.380238	0.002871427\\
0.380338	0.002064824\\
0.380438	0.001148047\\
0.3805381	-0.0009475068\\
0.3806381	0.001335662\\
0.3807381	0.001145956\\
0.3808381	0.002409144\\
0.3809381	0.001659955\\
0.3810381	0.003756234\\
0.3811381	0.003319719\\
0.3812381	0.004180383\\
0.3813381	0.002683518\\
0.3814381	0.001065191\\
0.3815382	0.002127375\\
0.3816382	0.0006327266\\
0.3817382	0.002523658\\
0.3818382	0.002333716\\
0.3819382	0.00346492\\
0.3820382	0.002842769\\
0.3821382	0.001904601\\
0.3822382	0.003356975\\
0.3823382	0.00352224\\
0.3824382	0.003522143\\
0.3825383	0.001318163\\
0.3826383	0.001002239\\
0.3827383	0.0001612238\\
0.3828383	0.001413087\\
0.3829383	0.001569037\\
0.3830383	0.003049516\\
0.3831383	0.004375142\\
0.3832383	0.00250394\\
0.3833383	0.001221865\\
0.3834383	-0.0003164918\\
0.3835384	0.002005538\\
0.3836384	0.005091421\\
0.3837384	0.00523685\\
0.3838384	0.004556457\\
0.3839384	0.004849334\\
0.3840384	0.004802572\\
0.3841384	0.003997717\\
0.3842384	0.002427466\\
0.3843384	0.002226008\\
0.3844384	0.004013337\\
0.3845385	0.004441224\\
0.3846385	0.002988846\\
0.3847385	0.002731513\\
0.3848385	0.001104354\\
0.3849385	0.0002104751\\
0.3850385	8.932895e-05\\
0.3851385	0.0007763673\\
0.3852385	0.001485364\\
0.3853385	0.002885198\\
0.3854385	0.003136971\\
0.3855386	0.003557259\\
0.3856386	0.004347563\\
0.3857386	0.003001755\\
0.3858386	0.004461324\\
0.3859386	0.002483716\\
0.3860386	0.002324372\\
0.3861386	0.00503631\\
0.3862386	0.003757889\\
0.3863386	0.003200423\\
0.3864386	0.003878105\\
0.3865387	0.004231408\\
0.3866387	0.003823408\\
0.3867387	0.003684326\\
0.3868387	0.004317836\\
0.3869387	0.003963571\\
0.3870387	0.002479183\\
0.3871387	0.001685756\\
0.3872387	0.002060481\\
0.3873387	0.00304936\\
0.3874387	0.005546251\\
0.3875388	0.005832698\\
0.3876388	0.005310795\\
0.3877388	0.005139214\\
0.3878388	0.004827101\\
0.3879388	0.004508253\\
0.3880388	0.003264874\\
0.3881388	0.001797876\\
0.3882388	-0.0001774439\\
0.3883388	-0.0002099513\\
0.3884388	0.0007628947\\
0.3885389	0.002755289\\
0.3886389	0.003716643\\
0.3887389	0.00480858\\
0.3888389	0.004295938\\
0.3889389	0.00479747\\
0.3890389	0.004730058\\
0.3891389	0.005896874\\
0.3892389	0.004397617\\
0.3893389	0.004233809\\
0.3894389	0.005346073\\
0.389539	0.005462785\\
0.389639	0.004808738\\
0.389739	0.003659837\\
0.389839	0.00373145\\
0.389939	0.002672636\\
0.390039	0.003444197\\
0.390139	0.002256341\\
0.390239	0.004303495\\
0.390339	0.003444646\\
0.390439	0.003046587\\
0.3905391	0.005091161\\
0.3906391	0.003329751\\
0.3907391	0.003762922\\
0.3908391	0.006680265\\
0.3909391	0.00638969\\
0.3910391	0.00420888\\
0.3911391	0.003759794\\
0.3912391	0.002892102\\
0.3913391	0.004109329\\
0.3914391	0.00459697\\
0.3915392	0.004122899\\
0.3916392	0.003103727\\
0.3917392	0.004236032\\
0.3918392	0.004957518\\
0.3919392	0.006192865\\
0.3920392	0.006386537\\
0.3921392	0.006028829\\
0.3922392	0.006419235\\
0.3923392	0.005060193\\
0.3924392	0.005106174\\
0.3925393	0.002931696\\
0.3926393	0.0014109\\
0.3927393	0.000373408\\
0.3928393	0.00191937\\
0.3929393	0.003046959\\
0.3930393	0.002509927\\
0.3931393	0.003582544\\
0.3932393	0.003333109\\
0.3933393	0.004962089\\
0.3934393	0.006786937\\
0.3935394	0.007296598\\
0.3936394	0.008338844\\
0.3937394	0.007492192\\
0.3938394	0.006682732\\
0.3939394	0.006033241\\
0.3940394	0.00479106\\
0.3941394	0.004895283\\
0.3942394	0.003467229\\
0.3943394	0.003444985\\
0.3944394	0.004150169\\
0.3945395	0.005592372\\
0.3946395	0.005708218\\
0.3947395	0.004724381\\
0.3948395	0.005259111\\
0.3949395	0.00574961\\
0.3950395	0.00396732\\
0.3951395	0.002465412\\
0.3952395	0.005712644\\
0.3953395	0.005638503\\
0.3954395	0.005938791\\
0.3955396	0.00325482\\
0.3956396	0.003549869\\
0.3957396	0.003011293\\
0.3958396	0.001921776\\
0.3959396	0.002781013\\
0.3960396	0.004151461\\
0.3961396	0.006122751\\
0.3962396	0.006644708\\
0.3963396	0.007508937\\
0.3964396	0.007732074\\
0.3965397	0.007010094\\
0.3966397	0.007556475\\
0.3967397	0.005509834\\
0.3968397	0.004438339\\
0.3969397	0.003498894\\
0.3970397	0.003803992\\
0.3971397	0.005104605\\
0.3972397	0.005467483\\
0.3973397	0.005630093\\
0.3974397	0.004366891\\
0.3975398	0.005972675\\
0.3976398	0.004373604\\
0.3977398	0.005345387\\
0.3978398	0.004897479\\
0.3979398	0.004834778\\
0.3980398	0.00597825\\
0.3981398	0.005581994\\
0.3982398	0.004946184\\
0.3983398	0.00538057\\
0.3984398	0.00502265\\
0.3985399	0.006757185\\
0.3986399	0.005983377\\
0.3987399	0.004942052\\
0.3988399	0.006377619\\
0.3989399	0.005146364\\
0.3990399	0.00409676\\
0.3991399	0.004465482\\
0.3992399	0.002506843\\
0.3993399	0.003425105\\
0.3994399	0.006003192\\
0.39954	0.005968407\\
0.39964	0.006944333\\
0.39974	0.007123873\\
0.39984	0.005833785\\
0.39994	0.00617481\\
0.40004	0.005028364\\
};
\addplot [color=mycolor2,solid,forget plot]
  table[row sep=crcr]{%
0.40004	0.005028364\\
0.40014	0.003675153\\
0.40024	0.005747546\\
0.40034	0.006970603\\
0.40044	0.005487722\\
0.4005401	0.004372886\\
0.4006401	0.003047542\\
0.4007401	0.005759343\\
0.4008401	0.005313498\\
0.4009401	0.002799473\\
0.4010401	0.003008902\\
0.4011401	0.005454961\\
0.4012401	0.006847822\\
0.4013401	0.008491072\\
0.4014401	0.006408462\\
0.4015402	0.005644323\\
0.4016402	0.007928362\\
0.4017402	0.00653884\\
0.4018402	0.00639158\\
0.4019402	0.006897292\\
0.4020402	0.006384526\\
0.4021402	0.004644701\\
0.4022402	0.005498725\\
0.4023402	0.00555454\\
0.4024402	0.004071921\\
0.4025403	0.003028892\\
0.4026403	0.003839615\\
0.4027403	0.003613395\\
0.4028403	0.003673133\\
0.4029403	0.002829092\\
0.4030403	0.004674411\\
0.4031403	0.003545547\\
0.4032403	0.004950086\\
0.4033403	0.004905286\\
0.4034403	0.0064135\\
0.4035404	0.007342934\\
0.4036404	0.007376649\\
0.4037404	0.006110891\\
0.4038404	0.005272114\\
0.4039404	0.006655604\\
0.4040404	0.006393593\\
0.4041404	0.005399432\\
0.4042404	0.005257821\\
0.4043404	0.006359141\\
0.4044404	0.005105138\\
0.4045405	0.004889948\\
0.4046405	0.005028571\\
0.4047405	0.005813776\\
0.4048405	0.006348283\\
0.4049405	0.007074036\\
0.4050405	0.003483634\\
0.4051405	0.003385258\\
0.4052405	0.004584566\\
0.4053405	0.005735114\\
0.4054405	0.003674736\\
0.4055406	0.003386361\\
0.4056406	0.004857618\\
0.4057406	0.003730175\\
0.4058406	0.0041251\\
0.4059406	0.003948941\\
0.4060406	0.00470329\\
0.4061406	0.005295776\\
0.4062406	0.00514656\\
0.4063406	0.006016962\\
0.4064406	0.005429555\\
0.4065407	0.00555164\\
0.4066407	0.00799249\\
0.4067407	0.006682286\\
0.4068407	0.006009509\\
0.4069407	0.005814772\\
0.4070407	0.003698884\\
0.4071407	0.002225665\\
0.4072407	0.002600903\\
0.4073407	0.002366003\\
0.4074407	0.001204854\\
0.4075408	0.003819499\\
0.4076408	0.005151364\\
0.4077408	0.006278663\\
0.4078408	0.00711526\\
0.4079408	0.005264016\\
0.4080408	0.006064415\\
0.4081408	0.005090911\\
0.4082408	0.005737301\\
0.4083408	0.004539523\\
0.4084408	0.004252178\\
0.4085409	0.003973441\\
0.4086409	0.003806928\\
0.4087409	0.002127807\\
0.4088409	0.003474224\\
0.4089409	0.005499009\\
0.4090409	0.003585272\\
0.4091409	0.004435665\\
0.4092409	0.006054739\\
0.4093409	0.005287173\\
0.4094409	0.004027783\\
0.409541	0.005421515\\
0.409641	0.004024439\\
0.409741	0.003917396\\
0.409841	0.004365129\\
0.409941	0.00355805\\
0.410041	0.000629279\\
0.410141	0.001627378\\
0.410241	0.002324498\\
0.410341	0.003334103\\
0.410441	0.00390968\\
0.4105411	0.003530833\\
0.4106411	0.003165944\\
0.4107411	0.002427983\\
0.4108411	0.00300808\\
0.4109411	0.004599485\\
0.4110411	0.004732958\\
0.4111411	0.00652727\\
0.4112411	0.006313835\\
0.4113411	0.004866823\\
0.4114411	0.004772489\\
0.4115412	0.002650109\\
0.4116412	0.0008491046\\
0.4117412	0.0004889485\\
0.4118412	0.001456588\\
0.4119412	0.002062045\\
0.4120412	0.003754681\\
0.4121412	0.006107043\\
0.4122412	0.004715578\\
0.4123412	0.005997376\\
0.4124412	0.005883603\\
0.4125413	0.005168321\\
0.4126413	0.001721073\\
0.4127413	-0.0005410386\\
0.4128413	0.0009461004\\
0.4129413	0.00012291\\
0.4130413	0.0001233173\\
0.4131413	0.0003040666\\
0.4132413	0.0002556099\\
0.4133413	0.0009286269\\
0.4134413	0.001329419\\
0.4135414	0.002311894\\
0.4136414	0.00395197\\
0.4137414	0.002452347\\
0.4138414	0.003487056\\
0.4139414	0.003914131\\
0.4140414	0.00376384\\
0.4141414	0.004027575\\
0.4142414	0.00325692\\
0.4143414	0.003477704\\
0.4144414	0.003876084\\
0.4145415	-3.579043e-05\\
0.4146415	0.0009868114\\
0.4147415	0.001883384\\
0.4148415	0.001848797\\
0.4149415	0.001250673\\
0.4150415	0.0007390403\\
0.4151415	-0.0006280245\\
0.4152415	3.734683e-05\\
0.4153415	-0.000218486\\
0.4154415	0.0007567072\\
0.4155416	0.002287778\\
0.4156416	0.0004205697\\
0.4157416	0.001238161\\
0.4158416	0.0007872089\\
0.4159416	0.001011226\\
0.4160416	0.002170354\\
0.4161416	0.001333592\\
0.4162416	0.002314725\\
0.4163416	0.001768393\\
0.4164416	0.001233247\\
0.4165417	0.002837581\\
0.4166417	0.0009845742\\
0.4167417	-0.001845189\\
0.4168417	-0.002840391\\
0.4169417	-0.0008887929\\
0.4170417	0.001301057\\
0.4171417	0.0003323379\\
0.4172417	-0.0002707049\\
0.4173417	-0.0009877626\\
0.4174417	-0.001132973\\
0.4175418	-0.001200077\\
0.4176418	-0.001509592\\
0.4177418	0.0002288041\\
0.4178418	0.002149578\\
0.4179418	-0.0001239009\\
0.4180418	-0.0009940655\\
0.4181418	-0.0001758864\\
0.4182418	-0.002156103\\
0.4183418	-0.003195263\\
0.4184418	-0.001877955\\
0.4185419	0.001172773\\
0.4186419	-0.0001549103\\
0.4187419	3.818304e-05\\
0.4188419	0.001178626\\
0.4189419	0.0006902471\\
0.4190419	-0.0004861273\\
0.4191419	-0.001732684\\
0.4192419	-0.0009192713\\
0.4193419	0.0002161603\\
0.4194419	-0.0009491143\\
0.419542	-0.003386863\\
0.419642	-0.004326946\\
0.419742	-0.002743675\\
0.419842	-0.002243872\\
0.419942	-0.002318663\\
0.420042	-0.003137123\\
0.420142	-0.002056323\\
0.420242	-0.003903613\\
0.420342	-0.002616888\\
0.420442	-0.003633417\\
0.4205421	-0.004570063\\
0.4206421	-0.003409349\\
0.4207421	-0.001104719\\
0.4208421	-0.003201493\\
0.4209421	-0.002654238\\
0.4210421	-0.001541007\\
0.4211421	-0.001041763\\
0.4212421	-0.002544557\\
0.4213421	-0.003623798\\
0.4214421	-0.003142129\\
0.4215422	-0.002856355\\
0.4216422	-0.005143274\\
0.4217422	-0.006692993\\
0.4218422	-0.004083266\\
0.4219422	-0.001894954\\
0.4220422	-0.001082935\\
0.4221422	-0.002422409\\
0.4222422	-0.002804133\\
0.4223422	-0.0027842\\
0.4224422	-0.004148217\\
0.4225423	-0.004579275\\
0.4226423	-0.004927639\\
0.4227423	-0.0049779\\
0.4228423	-0.005751983\\
0.4229423	-0.005927367\\
0.4230423	-0.005777768\\
0.4231423	-0.005647142\\
0.4232423	-0.006711337\\
0.4233423	-0.006140928\\
0.4234423	-0.007283833\\
0.4235424	-0.006489554\\
0.4236424	-0.005749935\\
0.4237424	-0.004439485\\
0.4238424	-0.003485063\\
0.4239424	-0.003609734\\
0.4240424	-0.003921848\\
0.4241424	-0.004495329\\
0.4242424	-0.005616024\\
0.4243424	-0.005985463\\
0.4244424	-0.007676962\\
0.4245425	-0.007022527\\
0.4246425	-0.007030001\\
0.4247425	-0.007556402\\
0.4248425	-0.006685724\\
0.4249425	-0.007390431\\
0.4250425	-0.006034526\\
0.4251425	-0.006137092\\
0.4252425	-0.00854008\\
0.4253425	-0.005830092\\
0.4254425	-0.007959112\\
0.4255426	-0.00661762\\
0.4256426	-0.006814738\\
0.4257426	-0.006394419\\
0.4258426	-0.008265675\\
0.4259426	-0.01041125\\
0.4260426	-0.0101398\\
0.4261426	-0.009039204\\
0.4262426	-0.006743349\\
0.4263426	-0.00497391\\
0.4264426	-0.007834932\\
0.4265427	-0.007983678\\
0.4266427	-0.006973478\\
0.4267427	-0.00605739\\
0.4268427	-0.006752735\\
0.4269427	-0.00732171\\
0.4270427	-0.008455926\\
0.4271427	-0.007471889\\
0.4272427	-0.00924641\\
0.4273427	-0.01075495\\
0.4274427	-0.01408007\\
0.4275428	-0.01412023\\
0.4276428	-0.01271524\\
0.4277428	-0.01274296\\
0.4278428	-0.01135675\\
0.4279428	-0.009698353\\
0.4280428	-0.008351061\\
0.4281428	-0.008559176\\
0.4282428	-0.007810199\\
0.4283428	-0.006144337\\
0.4284428	-0.008903213\\
0.4285429	-0.01016582\\
0.4286429	-0.009205495\\
0.4287429	-0.01083431\\
0.4288429	-0.009195322\\
0.4289429	-0.01050635\\
0.4290429	-0.01104012\\
0.4291429	-0.01204904\\
0.4292429	-0.01093175\\
0.4293429	-0.01181888\\
0.4294429	-0.01242706\\
0.429543	-0.01193329\\
0.429643	-0.0105428\\
0.429743	-0.01049468\\
0.429843	-0.01119839\\
0.429943	-0.01216579\\
0.430043	-0.01107276\\
0.430143	-0.0117992\\
0.430243	-0.01324258\\
0.430343	-0.01371901\\
0.430443	-0.01303415\\
0.4305431	-0.01246378\\
0.4306431	-0.01224282\\
0.4307431	-0.01511513\\
0.4308431	-0.01528503\\
0.4309431	-0.01379924\\
0.4310431	-0.0124801\\
0.4311431	-0.01330041\\
0.4312431	-0.01154124\\
0.4313431	-0.01042184\\
0.4314431	-0.01168816\\
0.4315432	-0.01273327\\
0.4316432	-0.01223182\\
0.4317432	-0.01362348\\
0.4318432	-0.01445674\\
0.4319432	-0.01564705\\
0.4320432	-0.01548945\\
0.4321432	-0.01400074\\
0.4322432	-0.0133584\\
0.4323432	-0.01228833\\
0.4324432	-0.0128122\\
0.4325433	-0.01465145\\
0.4326433	-0.01498866\\
0.4327433	-0.0150082\\
0.4328433	-0.01381177\\
0.4329433	-0.0148422\\
0.4330433	-0.01635121\\
0.4331433	-0.01646781\\
0.4332433	-0.01621373\\
0.4333433	-0.0153325\\
0.4334433	-0.01542291\\
0.4335434	-0.01348737\\
0.4336434	-0.0133056\\
0.4337434	-0.01498647\\
0.4338434	-0.01471286\\
0.4339434	-0.01461639\\
0.4340434	-0.01602322\\
0.4341434	-0.01737187\\
0.4342434	-0.01809176\\
0.4343434	-0.01549491\\
0.4344434	-0.01616484\\
0.4345435	-0.01721004\\
0.4346435	-0.0173957\\
0.4347435	-0.01414113\\
0.4348435	-0.01638425\\
0.4349435	-0.01559558\\
0.4350435	-0.01604328\\
0.4351435	-0.01505414\\
0.4352435	-0.01704821\\
0.4353435	-0.01833173\\
0.4354435	-0.01880388\\
0.4355436	-0.01877764\\
0.4356436	-0.0185604\\
0.4357436	-0.01741409\\
0.4358436	-0.01780872\\
0.4359436	-0.01774631\\
0.4360436	-0.01633606\\
0.4361436	-0.01518939\\
0.4362436	-0.01555143\\
0.4363436	-0.01670771\\
0.4364436	-0.01792296\\
0.4365437	-0.01644542\\
0.4366437	-0.01616833\\
0.4367437	-0.01606774\\
0.4368437	-0.01514694\\
0.4369437	-0.0164322\\
0.4370437	-0.01828275\\
0.4371437	-0.01909291\\
0.4372437	-0.0193545\\
0.4373437	-0.01953133\\
0.4374437	-0.02141015\\
0.4375438	-0.0216314\\
0.4376438	-0.01985657\\
0.4377438	-0.01940999\\
0.4378438	-0.0183702\\
0.4379438	-0.01856362\\
0.4380438	-0.01822016\\
0.4381438	-0.01774094\\
0.4382438	-0.019954\\
0.4383438	-0.02014464\\
0.4384438	-0.0186483\\
0.4385439	-0.01857835\\
0.4386439	-0.0173904\\
0.4387439	-0.01766702\\
0.4388439	-0.01799007\\
0.4389439	-0.01686634\\
0.4390439	-0.0176475\\
0.4391439	-0.01842933\\
0.4392439	-0.01989078\\
0.4393439	-0.02014284\\
0.4394439	-0.01926518\\
0.439544	-0.01947833\\
0.439644	-0.01783592\\
0.439744	-0.01868533\\
0.439844	-0.01925876\\
0.439944	-0.01959491\\
0.440044	-0.02089983\\
0.440144	-0.01924754\\
0.440244	-0.02151991\\
0.440344	-0.02128018\\
0.440444	-0.02047227\\
0.4405441	-0.02003256\\
0.4406441	-0.0199373\\
0.4407441	-0.02127206\\
0.4408441	-0.01919174\\
0.4409441	-0.01785713\\
0.4410441	-0.01943039\\
0.4411441	-0.01716884\\
0.4412441	-0.01901167\\
0.4413441	-0.01841715\\
0.4414441	-0.01866436\\
0.4415442	-0.01965481\\
0.4416442	-0.0185011\\
0.4417442	-0.0202843\\
0.4418442	-0.02347769\\
0.4419442	-0.02366639\\
0.4420442	-0.02246707\\
0.4421442	-0.01996758\\
0.4422442	-0.02016643\\
0.4423442	-0.01993672\\
0.4424442	-0.01746337\\
0.4425443	-0.0181177\\
0.4426443	-0.01691423\\
0.4427443	-0.01962121\\
0.4428443	-0.02028243\\
0.4429443	-0.01857811\\
0.4430443	-0.01892012\\
0.4431443	-0.01988181\\
0.4432443	-0.0215235\\
0.4433443	-0.02096858\\
0.4434443	-0.02070126\\
0.4435444	-0.02209307\\
0.4436444	-0.02024169\\
0.4437444	-0.01888404\\
0.4438444	-0.01810318\\
0.4439444	-0.01929695\\
0.4440444	-0.01880048\\
0.4441444	-0.01738223\\
0.4442444	-0.0190548\\
0.4443444	-0.01898391\\
0.4444444	-0.01688042\\
0.4445445	-0.01843507\\
0.4446445	-0.01831632\\
0.4447445	-0.02059565\\
0.4448445	-0.0225788\\
0.4449445	-0.02150331\\
0.4450445	-0.02167189\\
0.4451445	-0.0216881\\
0.4452445	-0.02214958\\
0.4453445	-0.0202928\\
0.4454445	-0.01867941\\
0.4455446	-0.01708122\\
0.4456446	-0.01711006\\
0.4457446	-0.01757571\\
0.4458446	-0.01754915\\
0.4459446	-0.01694711\\
0.4460446	-0.02012993\\
0.4461446	-0.01828089\\
0.4462446	-0.01876365\\
0.4463446	-0.01938887\\
0.4464446	-0.01882597\\
0.4465447	-0.01799763\\
0.4466447	-0.01782866\\
0.4467447	-0.01611551\\
0.4468447	-0.01612475\\
0.4469447	-0.01858329\\
0.4470447	-0.01796532\\
0.4471447	-0.01758919\\
0.4472447	-0.01577056\\
0.4473447	-0.01733214\\
0.4474447	-0.01695709\\
0.4475448	-0.01819045\\
0.4476448	-0.02008132\\
0.4477448	-0.01915106\\
0.4478448	-0.01938344\\
0.4479448	-0.01813222\\
0.4480448	-0.01826448\\
0.4481448	-0.01843848\\
0.4482448	-0.01846146\\
0.4483448	-0.0173652\\
0.4484448	-0.01605006\\
0.4485449	-0.01643692\\
0.4486449	-0.01515566\\
0.4487449	-0.01419427\\
0.4488449	-0.01469115\\
0.4489449	-0.01452004\\
0.4490449	-0.0151031\\
0.4491449	-0.01642565\\
0.4492449	-0.01658036\\
0.4493449	-0.01495844\\
0.4494449	-0.01494657\\
0.449545	-0.01462141\\
0.449645	-0.01453882\\
0.449745	-0.01434281\\
0.449845	-0.01501252\\
0.449945	-0.01619264\\
0.450045	-0.01736606\\
0.450145	-0.01725181\\
0.450245	-0.01592943\\
0.450345	-0.01496059\\
0.450445	-0.01228178\\
0.4505451	-0.01265895\\
0.4506451	-0.01364433\\
0.4507451	-0.01306773\\
0.4508451	-0.01394637\\
0.4509451	-0.01376361\\
0.4510451	-0.01201035\\
0.4511451	-0.01139603\\
0.4512451	-0.01056592\\
0.4513451	-0.01163013\\
0.4514451	-0.01156903\\
0.4515452	-0.01253303\\
0.4516452	-0.0127433\\
0.4517452	-0.01338773\\
0.4518452	-0.01324832\\
0.4519452	-0.01222725\\
0.4520452	-0.0127856\\
0.4521452	-0.01275423\\
0.4522452	-0.01347183\\
0.4523452	-0.01255631\\
0.4524452	-0.00942628\\
0.4525453	-0.00873249\\
0.4526453	-0.008875913\\
0.4527453	-0.009680495\\
0.4528453	-0.01067343\\
0.4529453	-0.009164212\\
0.4530453	-0.01024737\\
0.4531453	-0.01012205\\
0.4532453	-0.01076852\\
0.4533453	-0.009468638\\
0.4534453	-0.008174666\\
0.4535454	-0.008380678\\
0.4536454	-0.007684285\\
0.4537454	-0.007280768\\
0.4538454	-0.004876225\\
0.4539454	-0.003827252\\
0.4540454	-0.006005657\\
0.4541454	-0.007772253\\
0.4542454	-0.006682877\\
0.4543454	-0.008546536\\
0.4544454	-0.008878737\\
0.4545455	-0.00800683\\
0.4546455	-0.008409404\\
0.4547455	-0.005301832\\
0.4548455	-0.003538409\\
0.4549455	-0.005422485\\
0.4550455	-0.005235521\\
0.4551455	-0.005547438\\
0.4552455	-0.00570259\\
0.4553455	-0.004353419\\
0.4554455	-0.004441487\\
0.4555456	-0.004560797\\
0.4556456	-0.003740056\\
0.4557456	-0.003508516\\
0.4558456	-0.004956115\\
0.4559456	-0.003976495\\
0.4560456	-0.002731592\\
0.4561456	-0.002669151\\
0.4562456	-0.001981097\\
0.4563456	0.0002015628\\
0.4564456	-0.001779596\\
0.4565457	-0.001725016\\
0.4566457	-0.0005037599\\
0.4567457	0.000779701\\
0.4568457	0.0003024823\\
0.4569457	0.0008024204\\
0.4570457	0.00150806\\
0.4571457	0.001740716\\
0.4572457	0.00141233\\
0.4573457	0.001224465\\
0.4574457	0.002476335\\
0.4575458	0.001209497\\
0.4576458	-0.001027635\\
0.4577458	-0.003454109\\
0.4578458	-0.001278841\\
0.4579458	0.002055786\\
0.4580458	0.002513897\\
0.4581458	0.003074934\\
0.4582458	0.003658165\\
0.4583458	0.004251949\\
0.4584458	0.006375466\\
0.4585459	0.004748669\\
0.4586459	0.004992969\\
0.4587459	0.005547843\\
0.4588459	0.00655625\\
0.4589459	0.008120241\\
0.4590459	0.005592106\\
0.4591459	0.00616981\\
0.4592459	0.005043186\\
0.4593459	0.004607526\\
0.4594459	0.005040277\\
0.459546	0.00610778\\
0.459646	0.007303597\\
0.459746	0.008289084\\
0.459846	0.009151793\\
0.459946	0.008133692\\
0.460046	0.008951109\\
0.460146	0.009314783\\
0.460246	0.01108766\\
0.460346	0.01030277\\
0.460446	0.01036053\\
0.4605461	0.009723304\\
0.4606461	0.00991695\\
0.4607461	0.01039644\\
0.4608461	0.009923372\\
0.4609461	0.009934901\\
0.4610461	0.01114764\\
0.4611461	0.01275735\\
0.4612461	0.01355915\\
0.4613461	0.01550519\\
0.4614461	0.01692173\\
0.4615462	0.01808322\\
0.4616462	0.01632683\\
0.4617462	0.01540152\\
0.4618462	0.01434332\\
0.4619462	0.01400513\\
0.4620462	0.01223302\\
0.4621462	0.01394086\\
0.4622462	0.01284045\\
0.4623462	0.01272469\\
0.4624462	0.01481756\\
0.4625463	0.01745157\\
0.4626463	0.01721609\\
0.4627463	0.01634174\\
0.4628463	0.01659169\\
0.4629463	0.01887327\\
0.4630463	0.02014955\\
0.4631463	0.01934938\\
0.4632463	0.02165287\\
0.4633463	0.02062681\\
0.4634463	0.02191302\\
0.4635464	0.02127417\\
0.4636464	0.02114741\\
0.4637464	0.02128711\\
0.4638464	0.02214931\\
0.4639464	0.02062174\\
0.4640464	0.02086603\\
0.4641464	0.02283146\\
0.4642464	0.02213815\\
0.4643464	0.02273726\\
0.4644464	0.02296644\\
0.4645465	0.02612196\\
0.4646465	0.02649616\\
0.4647465	0.02482941\\
0.4648465	0.02391892\\
0.4649465	0.02251818\\
0.4650465	0.02429012\\
0.4651465	0.02519081\\
0.4652465	0.02640163\\
0.4653465	0.02608124\\
0.4654465	0.02689656\\
0.4655466	0.02652497\\
0.4656466	0.02618439\\
0.4657466	0.02757963\\
0.4658466	0.02857387\\
0.4659466	0.02857902\\
0.4660466	0.02775965\\
0.4661466	0.02989101\\
0.4662466	0.03087874\\
0.4663466	0.02961877\\
0.4664466	0.02971811\\
0.4665467	0.0301071\\
0.4666467	0.03009142\\
0.4667467	0.02979553\\
0.4668467	0.03076317\\
0.4669467	0.03414691\\
0.4670467	0.0346225\\
0.4671467	0.03410994\\
0.4672467	0.03351754\\
0.4673467	0.0329793\\
0.4674467	0.03285931\\
0.4675468	0.03229013\\
0.4676468	0.03258747\\
0.4677468	0.03246244\\
0.4678468	0.03306896\\
0.4679468	0.03545932\\
0.4680468	0.03503647\\
0.4681468	0.03574091\\
0.4682468	0.03695018\\
0.4683468	0.03686416\\
0.4684468	0.03661286\\
0.4685469	0.03693709\\
0.4686469	0.0366889\\
0.4687469	0.03812335\\
0.4688469	0.03740052\\
0.4689469	0.03861034\\
0.4690469	0.03862094\\
0.4691469	0.03974497\\
0.4692469	0.03970021\\
0.4693469	0.03843782\\
0.4694469	0.03833397\\
0.469547	0.0393313\\
0.469647	0.0399028\\
0.469747	0.03954041\\
0.469847	0.04035407\\
0.469947	0.04109566\\
0.470047	0.04128504\\
0.470147	0.04133477\\
0.470247	0.04004175\\
0.470347	0.03950827\\
0.470447	0.0412532\\
0.4705471	0.04103797\\
0.4706471	0.0428182\\
0.4707471	0.04245767\\
0.4708471	0.04254374\\
0.4709471	0.04342389\\
0.4710471	0.04613651\\
0.4711471	0.04749115\\
0.4712471	0.04667167\\
0.4713471	0.04691609\\
0.4714471	0.04684007\\
0.4715472	0.04662399\\
0.4716472	0.04622712\\
0.4717472	0.04399042\\
0.4718472	0.04511341\\
0.4719472	0.04389608\\
0.4720472	0.04451099\\
0.4721472	0.04477333\\
0.4722472	0.04451413\\
0.4723472	0.0463311\\
0.4724472	0.04599894\\
0.4725473	0.04678\\
0.4726473	0.04706495\\
0.4727473	0.04739345\\
0.4728473	0.04775844\\
0.4729473	0.04869573\\
0.4730473	0.04845642\\
0.4731473	0.04875183\\
0.4732473	0.04929921\\
0.4733473	0.04889464\\
0.4734473	0.04701223\\
0.4735474	0.04728147\\
0.4736474	0.0480906\\
0.4737474	0.05032656\\
0.4738474	0.05075693\\
0.4739474	0.05196242\\
0.4740474	0.05050261\\
0.4741474	0.05166927\\
0.4742474	0.05358159\\
0.4743474	0.05159651\\
0.4744474	0.05094644\\
0.4745475	0.04846442\\
0.4746475	0.04860469\\
0.4747475	0.05006413\\
0.4748475	0.05040152\\
0.4749475	0.05066435\\
0.4750475	0.05168821\\
0.4751475	0.0502846\\
0.4752475	0.0499687\\
0.4753475	0.05138284\\
0.4754475	0.05102113\\
0.4755476	0.05241543\\
0.4756476	0.05129942\\
0.4757476	0.05161951\\
0.4758476	0.05327784\\
0.4759476	0.05344076\\
0.4760476	0.05400866\\
0.4761476	0.05364074\\
0.4762476	0.0537027\\
0.4763476	0.0530125\\
0.4764476	0.05357659\\
0.4765477	0.05114976\\
0.4766477	0.050454\\
0.4767477	0.04803617\\
0.4768477	0.04901996\\
0.4769477	0.05131777\\
0.4770477	0.05200165\\
0.4771477	0.05452249\\
0.4772477	0.05347615\\
0.4773477	0.05528662\\
0.4774477	0.0554633\\
0.4775478	0.05410744\\
0.4776478	0.05271015\\
0.4777478	0.05291201\\
0.4778478	0.05248631\\
0.4779478	0.05247676\\
0.4780478	0.05147325\\
0.4781478	0.04892182\\
0.4782478	0.05158405\\
0.4783478	0.05255283\\
0.4784478	0.05162152\\
0.4785479	0.05124583\\
0.4786479	0.05315072\\
0.4787479	0.0554052\\
0.4788479	0.05418806\\
0.4789479	0.05353326\\
0.4790479	0.0531241\\
0.4791479	0.05347968\\
0.4792479	0.05234358\\
0.4793479	0.05290125\\
0.4794479	0.05193916\\
0.479548	0.04858505\\
0.479648	0.05037416\\
0.479748	0.05031632\\
0.479848	0.04959896\\
0.479948	0.05117981\\
0.480048	0.05031941\\
0.480148	0.04772812\\
0.480248	0.04772273\\
0.480348	0.05005009\\
0.480448	0.05155974\\
0.4805481	0.05260354\\
0.4806481	0.05059499\\
0.4807481	0.04999999\\
0.4808481	0.05017295\\
0.4809481	0.04916861\\
0.4810481	0.04727359\\
0.4811481	0.04785629\\
0.4812481	0.05081445\\
0.4813481	0.05105598\\
0.4814481	0.04873355\\
0.4815482	0.04975066\\
0.4816482	0.05136019\\
0.4817482	0.04918282\\
0.4818482	0.04649334\\
0.4819482	0.04728836\\
0.4820482	0.04797522\\
0.4821482	0.0466216\\
0.4822482	0.04576591\\
0.4823482	0.04490986\\
0.4824482	0.04344477\\
0.4825483	0.04307453\\
0.4826483	0.04359845\\
0.4827483	0.04449097\\
0.4828483	0.04477625\\
0.4829483	0.04478317\\
0.4830483	0.04499499\\
0.4831483	0.04385309\\
0.4832483	0.04356289\\
0.4833483	0.04262091\\
0.4834483	0.04250233\\
0.4835484	0.04289973\\
0.4836484	0.04329353\\
0.4837484	0.04205247\\
0.4838484	0.0428176\\
0.4839484	0.04291891\\
0.4840484	0.04040789\\
0.4841484	0.04027273\\
0.4842484	0.03955861\\
0.4843484	0.04014351\\
0.4844484	0.03744731\\
0.4845485	0.03778616\\
0.4846485	0.03862821\\
0.4847485	0.03658875\\
0.4848485	0.03677427\\
0.4849485	0.03760371\\
0.4850485	0.03678015\\
0.4851485	0.03547935\\
0.4852485	0.0349922\\
0.4853485	0.03414046\\
0.4854485	0.03441997\\
0.4855486	0.03404834\\
0.4856486	0.03482579\\
0.4857486	0.03443611\\
0.4858486	0.03500547\\
0.4859486	0.03492862\\
0.4860486	0.03293101\\
0.4861486	0.03135842\\
0.4862486	0.03066038\\
0.4863486	0.02876183\\
0.4864486	0.02946718\\
0.4865487	0.02974648\\
0.4866487	0.02745972\\
0.4867487	0.02544959\\
0.4868487	0.02768575\\
0.4869487	0.02801914\\
0.4870487	0.02512697\\
0.4871487	0.02535302\\
0.4872487	0.02728626\\
0.4873487	0.02742138\\
0.4874487	0.02497148\\
0.4875488	0.02251906\\
0.4876488	0.02148941\\
0.4877488	0.02027193\\
0.4878488	0.01830221\\
0.4879488	0.01972641\\
0.4880488	0.02093143\\
0.4881488	0.02261146\\
0.4882488	0.02228529\\
0.4883488	0.01918218\\
0.4884488	0.01820175\\
0.4885489	0.01868415\\
0.4886489	0.01763332\\
0.4887489	0.01777998\\
0.4888489	0.0188695\\
0.4889489	0.01500198\\
0.4890489	0.01390324\\
0.4891489	0.01460688\\
0.4892489	0.01135658\\
0.4893489	0.01210945\\
0.4894489	0.01014746\\
0.489549	0.009553548\\
0.489649	0.007771547\\
0.489749	0.0092219\\
0.489849	0.00989197\\
0.489949	0.009454964\\
0.490049	0.008218639\\
0.490149	0.006936891\\
0.490249	0.0079943\\
0.490349	0.004006537\\
0.490449	0.003264365\\
0.4905491	0.002033683\\
0.4906491	0.002364667\\
0.4907491	0.002377611\\
0.4908491	0.0006106816\\
0.4909491	-0.001396205\\
0.4910491	-0.001187345\\
0.4911491	-0.0001803438\\
0.4912491	-0.002930931\\
0.4913491	-0.000942298\\
0.4914491	-0.002009628\\
0.4915492	-0.0005054604\\
0.4916492	-8.931829e-05\\
0.4917492	-0.002942618\\
0.4918492	-0.00375908\\
0.4919492	-0.006876634\\
0.4920492	-0.007530807\\
0.4921492	-0.00935396\\
0.4922492	-0.01016122\\
0.4923492	-0.0122979\\
0.4924492	-0.01304649\\
0.4925493	-0.01480098\\
0.4926493	-0.01395927\\
0.4927493	-0.01303254\\
0.4928493	-0.01497906\\
0.4929493	-0.01377412\\
0.4930493	-0.01654261\\
0.4931493	-0.01558267\\
0.4932493	-0.01592588\\
0.4933493	-0.01855843\\
0.4934493	-0.01861287\\
0.4935494	-0.01906009\\
0.4936494	-0.0182735\\
0.4937494	-0.01889372\\
0.4938494	-0.02003531\\
0.4939494	-0.02425724\\
0.4940494	-0.02506214\\
0.4941494	-0.02717712\\
0.4942494	-0.02759217\\
0.4943494	-0.02787951\\
0.4944494	-0.02889082\\
0.4945495	-0.02807262\\
0.4946495	-0.02940666\\
0.4947495	-0.02938564\\
0.4948495	-0.03085159\\
0.4949495	-0.03020796\\
0.4950495	-0.03320105\\
0.4951495	-0.03344873\\
0.4952495	-0.03435756\\
0.4953495	-0.03652345\\
0.4954495	-0.03651828\\
0.4955496	-0.03643635\\
0.4956496	-0.03779804\\
0.4957496	-0.03789266\\
0.4958496	-0.0388497\\
0.4959496	-0.04065917\\
0.4960496	-0.04213365\\
0.4961496	-0.04077106\\
0.4962496	-0.04024414\\
0.4963496	-0.04074562\\
0.4964496	-0.04170814\\
0.4965497	-0.04408483\\
0.4966497	-0.04491907\\
0.4967497	-0.04868917\\
0.4968497	-0.05089719\\
0.4969497	-0.05179832\\
0.4970497	-0.05163646\\
0.4971497	-0.05209655\\
0.4972497	-0.05441808\\
0.4973497	-0.05447468\\
0.4974497	-0.05191743\\
0.4975498	-0.05280194\\
0.4976498	-0.05481564\\
0.4977498	-0.05496178\\
0.4978498	-0.05458012\\
0.4979498	-0.05745767\\
0.4980498	-0.05733738\\
0.4981498	-0.05962894\\
0.4982498	-0.06183388\\
0.4983498	-0.06188306\\
0.4984498	-0.06240216\\
0.4985499	-0.06215159\\
0.4986499	-0.06104389\\
0.4987499	-0.06274664\\
0.4988499	-0.06371721\\
0.4989499	-0.06414572\\
0.4990499	-0.06549965\\
0.4991499	-0.06780021\\
0.4992499	-0.06923111\\
0.4993499	-0.06882326\\
0.4994499	-0.06968448\\
0.49955	-0.07080132\\
0.49965	-0.07148996\\
0.49975	-0.07150257\\
0.49985	-0.07357667\\
0.49995	-0.07354501\\
0.50005	-0.07439071\\
0.50015	-0.07569246\\
0.50025	-0.07771707\\
0.50035	-0.07809397\\
0.50045	-0.07879205\\
0.5005501	-0.07829745\\
0.5006501	-0.07833715\\
0.5007501	-0.07955804\\
0.5008501	-0.07972021\\
0.5009501	-0.08086914\\
0.5010501	-0.0821745\\
0.5011501	-0.08323768\\
0.5012501	-0.08446765\\
0.5013501	-0.08513865\\
0.5014501	-0.0858169\\
0.5015502	-0.08384373\\
0.5016502	-0.08456517\\
0.5017502	-0.08577067\\
0.5018502	-0.08685336\\
0.5019502	-0.08786845\\
0.5020502	-0.08875974\\
0.5021502	-0.09058365\\
0.5022502	-0.09099294\\
0.5023502	-0.09024145\\
0.5024502	-0.09348451\\
0.5025503	-0.0947248\\
0.5026503	-0.09358229\\
0.5027503	-0.09225948\\
0.5028503	-0.09437116\\
0.5029503	-0.09554766\\
0.5030503	-0.09384639\\
0.5031503	-0.09504409\\
0.5032503	-0.0962104\\
0.5033503	-0.09716924\\
0.5034503	-0.09616388\\
0.5035504	-0.09775429\\
0.5036504	-0.09950342\\
0.5037504	-0.1011858\\
0.5038504	-0.1006278\\
0.5039504	-0.09823044\\
0.5040504	-0.09931444\\
0.5041504	-0.1006148\\
0.5042504	-0.1025118\\
0.5043504	-0.1014513\\
0.5044504	-0.1024873\\
0.5045505	-0.1039082\\
0.5046505	-0.1027224\\
0.5047505	-0.1048977\\
0.5048505	-0.1062745\\
0.5049505	-0.1075863\\
0.5050505	-0.1070569\\
0.5051505	-0.1082052\\
0.5052505	-0.1072528\\
0.5053505	-0.1085983\\
0.5054505	-0.1093814\\
0.5055506	-0.1070836\\
0.5056506	-0.1063587\\
0.5057506	-0.1073153\\
0.5058506	-0.1080631\\
0.5059506	-0.1060956\\
0.5060506	-0.1041395\\
0.5061506	-0.1064943\\
0.5062506	-0.1087199\\
0.5063506	-0.1077876\\
0.5064506	-0.1094401\\
0.5065507	-0.1106226\\
0.5066507	-0.1116279\\
0.5067507	-0.113877\\
0.5068507	-0.1153805\\
0.5069507	-0.1140974\\
0.5070507	-0.1142185\\
0.5071507	-0.112293\\
0.5072507	-0.1122283\\
0.5073507	-0.110858\\
0.5074507	-0.1098676\\
0.5075508	-0.1115539\\
0.5076508	-0.112289\\
0.5077508	-0.1137803\\
0.5078508	-0.1131135\\
0.5079508	-0.1135625\\
0.5080508	-0.113801\\
0.5081508	-0.1143315\\
0.5082508	-0.1136837\\
0.5083508	-0.1138781\\
0.5084508	-0.1125325\\
0.5085509	-0.1138041\\
0.5086509	-0.1138782\\
0.5087509	-0.1123731\\
0.5088509	-0.1122135\\
0.5089509	-0.1112201\\
0.5090509	-0.111732\\
0.5091509	-0.1122618\\
0.5092509	-0.1129284\\
0.5093509	-0.1125664\\
0.5094509	-0.1141158\\
0.509551	-0.1122045\\
0.509651	-0.1116662\\
0.509751	-0.1130874\\
0.509851	-0.112903\\
0.509951	-0.1146432\\
0.510051	-0.1129525\\
0.510151	-0.1117181\\
0.510251	-0.112593\\
0.510351	-0.1115043\\
0.510451	-0.1114934\\
0.5105511	-0.1085942\\
0.5106511	-0.1065647\\
0.5107511	-0.1080362\\
0.5108511	-0.1096568\\
0.5109511	-0.1107745\\
0.5110511	-0.1096664\\
0.5111511	-0.1092761\\
0.5112511	-0.1095291\\
0.5113511	-0.1089453\\
0.5114511	-0.1078084\\
0.5115512	-0.1076006\\
0.5116512	-0.1073363\\
0.5117512	-0.1063334\\
0.5118512	-0.1059653\\
0.5119512	-0.1045706\\
0.5120512	-0.1057116\\
0.5121512	-0.1042402\\
0.5122512	-0.1038075\\
0.5123512	-0.1034962\\
0.5124512	-0.1014645\\
0.5125513	-0.1020577\\
0.5126513	-0.1026529\\
0.5127513	-0.102259\\
0.5128513	-0.1012696\\
0.5129513	-0.100655\\
0.5130513	-0.1011569\\
0.5131513	-0.09848206\\
0.5132513	-0.09722533\\
0.5133513	-0.09655723\\
0.5134513	-0.09584211\\
0.5135514	-0.09730855\\
0.5136514	-0.0944679\\
0.5137514	-0.09200661\\
0.5138514	-0.0910001\\
0.5139514	-0.08973526\\
0.5140514	-0.09065665\\
0.5141514	-0.0916986\\
0.5142514	-0.09079763\\
0.5143514	-0.09118975\\
0.5144514	-0.09159716\\
0.5145515	-0.09026405\\
0.5146515	-0.08778363\\
0.5147515	-0.08662698\\
0.5148515	-0.08559001\\
0.5149515	-0.08561129\\
0.5150515	-0.08436591\\
0.5151515	-0.08249802\\
0.5152515	-0.0817708\\
0.5153515	-0.08021887\\
0.5154515	-0.08056895\\
0.5155516	-0.0780082\\
0.5156516	-0.07777362\\
0.5157516	-0.07702079\\
0.5158516	-0.07696294\\
0.5159516	-0.07573308\\
0.5160516	-0.07208923\\
0.5161516	-0.07039719\\
0.5162516	-0.06987816\\
0.5163516	-0.06754726\\
0.5164516	-0.0668765\\
0.5165517	-0.06531547\\
0.5166517	-0.0650128\\
0.5167517	-0.06500785\\
0.5168517	-0.06393693\\
0.5169517	-0.0645431\\
0.5170517	-0.06383032\\
0.5171517	-0.06385525\\
0.5172517	-0.06202189\\
0.5173517	-0.05899304\\
0.5174517	-0.05576692\\
0.5175518	-0.054763\\
0.5176518	-0.05413847\\
0.5177518	-0.05279548\\
0.5178518	-0.05208435\\
0.5179518	-0.05331337\\
0.5180518	-0.04920273\\
0.5181518	-0.0462865\\
0.5182518	-0.04448338\\
0.5183518	-0.043721\\
0.5184518	-0.04272093\\
0.5185519	-0.04136669\\
0.5186519	-0.04046694\\
0.5187519	-0.03923853\\
0.5188519	-0.03806056\\
0.5189519	-0.03645617\\
0.5190519	-0.03433246\\
0.5191519	-0.03176805\\
0.5192519	-0.03023918\\
0.5193519	-0.03093047\\
0.5194519	-0.03115931\\
0.519552	-0.03031301\\
0.519652	-0.02863452\\
0.519752	-0.02521137\\
0.519852	-0.02278533\\
0.519952	-0.02111478\\
0.520052	-0.01909375\\
0.520152	-0.01548556\\
0.520252	-0.01401571\\
0.520352	-0.01360017\\
0.520452	-0.01264879\\
0.5205521	-0.0109349\\
0.5206521	-0.00960085\\
0.5207521	-0.007708238\\
0.5208521	-0.006706427\\
0.5209521	-0.006046011\\
0.5210521	-0.003629365\\
0.5211521	-0.003413648\\
0.5212521	0.0002009227\\
0.5213521	0.002819771\\
0.5214521	0.004480422\\
0.5215522	0.003967127\\
0.5216522	0.004791923\\
0.5217522	0.008113462\\
0.5218522	0.01076787\\
0.5219522	0.01363856\\
0.5220522	0.01498503\\
0.5221522	0.01706255\\
0.5222522	0.01723921\\
0.5223522	0.01977493\\
0.5224522	0.01898403\\
0.5225523	0.02056635\\
0.5226523	0.02073061\\
0.5227523	0.02267442\\
0.5228523	0.02541007\\
0.5229523	0.02931802\\
0.5230523	0.03227782\\
0.5231523	0.03551075\\
0.5232523	0.03749835\\
0.5233523	0.0394039\\
0.5234523	0.04025698\\
0.5235524	0.04336473\\
0.5236524	0.04395545\\
0.5237524	0.04415232\\
0.5238524	0.04548147\\
0.5239524	0.04568091\\
0.5240524	0.04964227\\
0.5241524	0.05158791\\
0.5242524	0.05258091\\
0.5243524	0.05435589\\
0.5244524	0.05654924\\
0.5245525	0.05786414\\
0.5246525	0.06032951\\
0.5247525	0.06285326\\
0.5248525	0.06594183\\
0.5249525	0.06629657\\
0.5250525	0.06757538\\
0.5251525	0.07001628\\
0.5252525	0.06993157\\
0.5253525	0.0718475\\
0.5254525	0.07438609\\
0.5255526	0.07734395\\
0.5256526	0.07858099\\
0.5257526	0.0777733\\
0.5258526	0.08180694\\
0.5259526	0.08298307\\
0.5260526	0.08660341\\
0.5261526	0.08858994\\
0.5262526	0.09107071\\
0.5263526	0.09555084\\
0.5264526	0.09842584\\
0.5265527	0.0983597\\
0.5266527	0.09673608\\
0.5267527	0.09750909\\
0.5268527	0.1007522\\
0.5269527	0.1012382\\
0.5270527	0.1027605\\
0.5271527	0.1047701\\
0.5272527	0.105659\\
0.5273527	0.1087675\\
0.5274527	0.1094748\\
0.5275528	0.1127769\\
0.5276528	0.1161907\\
0.5277528	0.1168432\\
0.5278528	0.116855\\
0.5279528	0.1202528\\
0.5280528	0.1230256\\
0.5281528	0.1243364\\
0.5282528	0.1254713\\
0.5283528	0.1289152\\
0.5284528	0.1290276\\
0.5285529	0.130483\\
0.5286529	0.1321431\\
0.5287529	0.1329954\\
0.5288529	0.1355181\\
0.5289529	0.1366411\\
0.5290529	0.1385823\\
0.5291529	0.1396928\\
0.5292529	0.1406932\\
0.5293529	0.1410369\\
0.5294529	0.1430026\\
0.529553	0.1468748\\
0.529653	0.1477597\\
0.529753	0.1489818\\
0.529853	0.1511912\\
0.529953	0.151391\\
0.530053	0.1546695\\
0.530153	0.1551757\\
0.530253	0.1546723\\
0.530353	0.1571168\\
0.530453	0.1598691\\
0.5305531	0.1618771\\
0.5306531	0.1640467\\
0.5307531	0.1649383\\
0.5308531	0.1672472\\
0.5309531	0.1693063\\
0.5310531	0.1699405\\
0.5311531	0.1708855\\
0.5312531	0.1711611\\
0.5313531	0.1731169\\
0.5314531	0.1727953\\
0.5315532	0.1720892\\
0.5316532	0.173645\\
0.5317532	0.1750681\\
0.5318532	0.1783389\\
0.5319532	0.180326\\
0.5320532	0.1800344\\
0.5321532	0.1803929\\
0.5322532	0.1813477\\
0.5323532	0.1833745\\
0.5324532	0.1860052\\
0.5325533	0.186422\\
0.5326533	0.1874922\\
0.5327533	0.1887574\\
0.5328533	0.1896989\\
0.5329533	0.189803\\
0.5330533	0.1911569\\
0.5331533	0.1921148\\
0.5332533	0.1925905\\
0.5333533	0.1939788\\
0.5334533	0.1938558\\
0.5335534	0.1960101\\
0.5336534	0.1950495\\
0.5337534	0.1979072\\
0.5338534	0.1993877\\
0.5339534	0.2001688\\
0.5340534	0.2016037\\
0.5341534	0.2032392\\
0.5342534	0.2023931\\
0.5343534	0.2016156\\
0.5344534	0.2015101\\
0.5345535	0.2019098\\
0.5346535	0.2031248\\
0.5347535	0.2033677\\
0.5348535	0.203294\\
0.5349535	0.2052726\\
0.5350535	0.2055086\\
0.5351535	0.205954\\
0.5352535	0.2067285\\
0.5353535	0.2073731\\
0.5354535	0.2062699\\
0.5355536	0.2067257\\
0.5356536	0.2080509\\
0.5357536	0.2076181\\
0.5358536	0.208648\\
0.5359536	0.2085554\\
0.5360536	0.2089435\\
0.5361536	0.2098852\\
0.5362536	0.210668\\
0.5363536	0.2124166\\
0.5364536	0.2119451\\
0.5365537	0.2120658\\
0.5366537	0.2104049\\
0.5367537	0.208553\\
0.5368537	0.2093208\\
0.5369537	0.207457\\
0.5370537	0.2075209\\
0.5371537	0.2080932\\
0.5372537	0.2093502\\
0.5373537	0.2075985\\
0.5374537	0.208489\\
0.5375538	0.2108262\\
0.5376538	0.2108083\\
0.5377538	0.2089782\\
0.5378538	0.2093694\\
0.5379538	0.2083808\\
0.5380538	0.2081082\\
0.5381538	0.2094963\\
0.5382538	0.2058612\\
0.5383538	0.2053358\\
0.5384538	0.2059191\\
0.5385539	0.2045571\\
0.5386539	0.2033547\\
0.5387539	0.2033674\\
0.5388539	0.2046278\\
0.5389539	0.2057504\\
0.5390539	0.2038689\\
0.5391539	0.2023026\\
0.5392539	0.2020528\\
0.5393539	0.2009271\\
0.5394539	0.1986671\\
0.539554	0.1957045\\
0.539654	0.1965071\\
0.539754	0.1940693\\
0.539854	0.1944587\\
0.539954	0.1934204\\
0.540054	0.1905697\\
0.540154	0.1915498\\
0.540254	0.1910961\\
0.540354	0.1890769\\
0.540454	0.1898112\\
0.5405541	0.1899032\\
0.5406541	0.1899974\\
0.5407541	0.1883815\\
0.5408541	0.1881603\\
0.5409541	0.1873256\\
0.5410541	0.1832012\\
0.5411541	0.1810492\\
0.5412541	0.1785847\\
0.5413541	0.1789238\\
0.5414541	0.1766836\\
0.5415542	0.1731302\\
0.5416542	0.1717795\\
0.5417542	0.1699084\\
0.5418542	0.1697161\\
0.5419542	0.1687178\\
0.5420542	0.1649246\\
0.5421542	0.1657607\\
0.5422542	0.1661985\\
0.5423542	0.1652269\\
0.5424542	0.1621621\\
0.5425543	0.1577417\\
0.5426543	0.1559933\\
0.5427543	0.1544138\\
0.5428543	0.1520122\\
0.5429543	0.1513334\\
0.5430543	0.1476288\\
0.5431543	0.1480977\\
0.5432543	0.1442872\\
0.5433543	0.1431364\\
0.5434543	0.1430096\\
0.5435544	0.1396606\\
0.5436544	0.1388804\\
0.5437544	0.1361469\\
0.5438544	0.1338893\\
0.5439544	0.1328048\\
0.5440544	0.1286145\\
0.5441544	0.12692\\
0.5442544	0.1234785\\
0.5443544	0.120563\\
0.5444544	0.1175958\\
0.5445545	0.1151082\\
0.5446545	0.1140905\\
0.5447545	0.1133109\\
0.5448545	0.1119232\\
0.5449545	0.1097516\\
0.5450545	0.1054935\\
0.5451545	0.1021904\\
0.5452545	0.09873992\\
0.5453545	0.09699962\\
0.5454545	0.09447395\\
0.5455546	0.09222824\\
0.5456546	0.09004662\\
0.5457546	0.08775903\\
0.5458546	0.0855558\\
0.5459546	0.08279401\\
0.5460546	0.07756245\\
0.5461546	0.07485582\\
0.5462546	0.07207671\\
0.5463546	0.06848295\\
0.5464546	0.06663937\\
0.5465547	0.06570338\\
0.5466547	0.06568328\\
0.5467547	0.06160454\\
0.5468547	0.05612479\\
0.5469547	0.05354697\\
0.5470547	0.05058864\\
0.5471547	0.04631437\\
0.5472547	0.04220746\\
0.5473547	0.03895495\\
0.5474547	0.0368479\\
0.5475548	0.03436409\\
0.5476548	0.0307826\\
0.5477548	0.03012597\\
0.5478548	0.02705473\\
0.5479548	0.02296344\\
0.5480548	0.01884551\\
0.5481548	0.01754918\\
0.5482548	0.01398767\\
0.5483548	0.01146776\\
0.5484548	0.007781877\\
0.5485549	0.0059727\\
0.5486549	0.003127629\\
0.5487549	0.0005495571\\
0.5488549	-0.003440797\\
0.5489549	-0.00887998\\
0.5490549	-0.01401496\\
0.5491549	-0.01635667\\
0.5492549	-0.01950438\\
0.5493549	-0.02328479\\
0.5494549	-0.02841144\\
0.549555	-0.02976587\\
0.549655	-0.03302161\\
0.549755	-0.0364466\\
0.549855	-0.04130736\\
0.549955	-0.04121823\\
0.550055	-0.04381741\\
0.550155	-0.04810334\\
0.550255	-0.05306583\\
0.550355	-0.05575305\\
0.550455	-0.05983674\\
0.5505551	-0.06507213\\
0.5506551	-0.06678591\\
0.5507551	-0.06771828\\
0.5508551	-0.071171\\
0.5509551	-0.07394781\\
0.5510551	-0.07794993\\
0.5511551	-0.0821168\\
0.5512551	-0.08572875\\
0.5513551	-0.0906912\\
0.5514551	-0.0923753\\
0.5515552	-0.09475954\\
0.5516552	-0.09883961\\
0.5517552	-0.104158\\
0.5518552	-0.1089348\\
0.5519552	-0.1116506\\
0.5520552	-0.1145519\\
0.5521552	-0.1171872\\
0.5522552	-0.1194545\\
0.5523552	-0.1245704\\
0.5524552	-0.1277845\\
0.5525553	-0.1313072\\
0.5526553	-0.1346134\\
0.5527553	-0.1380525\\
0.5528553	-0.1428444\\
0.5529553	-0.1444956\\
0.5530553	-0.1464633\\
0.5531553	-0.1500881\\
0.5532553	-0.1526898\\
0.5533553	-0.156418\\
0.5534553	-0.1598076\\
0.5535554	-0.1650222\\
0.5536554	-0.1682315\\
0.5537554	-0.1697827\\
0.5538554	-0.1718611\\
0.5539554	-0.178032\\
0.5540554	-0.180858\\
0.5541554	-0.1849017\\
0.5542554	-0.1883726\\
0.5543554	-0.1920059\\
0.5544554	-0.195533\\
0.5545555	-0.1983215\\
0.5546555	-0.1998951\\
0.5547555	-0.2037625\\
0.5548555	-0.2059419\\
0.5549555	-0.2087414\\
0.5550555	-0.2103509\\
0.5551555	-0.2136657\\
0.5552555	-0.2163683\\
0.5553555	-0.2197537\\
0.5554555	-0.2224435\\
0.5555556	-0.2269551\\
0.5556556	-0.2283386\\
0.5557556	-0.2331546\\
0.5558556	-0.2393114\\
0.5559556	-0.2414255\\
0.5560556	-0.2440309\\
0.5561556	-0.2458872\\
0.5562556	-0.248695\\
0.5563556	-0.249203\\
0.5564556	-0.2504216\\
0.5565557	-0.2534686\\
0.5566557	-0.2550765\\
0.5567557	-0.2578464\\
0.5568557	-0.2630747\\
0.5569557	-0.2675098\\
0.5570557	-0.2710625\\
0.5571557	-0.2732944\\
0.5572557	-0.2766077\\
0.5573557	-0.2781959\\
0.5574557	-0.2819332\\
0.5575558	-0.2818275\\
0.5576558	-0.2825333\\
0.5577558	-0.2854245\\
0.5578558	-0.2889261\\
0.5579558	-0.2906908\\
0.5580558	-0.2914225\\
0.5581558	-0.2927613\\
0.5582558	-0.2946583\\
0.5583558	-0.2969324\\
0.5584558	-0.3014727\\
0.5585559	-0.3035425\\
0.5586559	-0.3048858\\
0.5587559	-0.306062\\
0.5588559	-0.3097553\\
0.5589559	-0.3087587\\
0.5590559	-0.3130905\\
0.5591559	-0.3155184\\
0.5592559	-0.3160703\\
0.5593559	-0.3178039\\
0.5594559	-0.3204999\\
0.559556	-0.321207\\
0.559656	-0.321743\\
0.559756	-0.324238\\
0.559856	-0.3267242\\
0.559956	-0.3281002\\
0.560056	-0.3300883\\
0.560156	-0.3319503\\
0.560256	-0.3326949\\
0.560356	-0.3344379\\
0.560456	-0.3343871\\
0.5605561	-0.3361361\\
0.5606561	-0.3365905\\
0.5607561	-0.336727\\
0.5608561	-0.3375666\\
0.5609561	-0.3363012\\
0.5610561	-0.3381205\\
0.5611561	-0.3389894\\
0.5612561	-0.3391307\\
0.5613561	-0.3411697\\
0.5614561	-0.3435122\\
0.5615562	-0.3443232\\
0.5616562	-0.3451154\\
0.5617562	-0.3471586\\
0.5618562	-0.3472014\\
0.5619562	-0.3474413\\
0.5620562	-0.3477469\\
0.5621562	-0.3489832\\
0.5622562	-0.3483399\\
0.5623562	-0.3464449\\
0.5624562	-0.3454884\\
0.5625563	-0.3461399\\
0.5626563	-0.3474379\\
0.5627563	-0.3473268\\
0.5628563	-0.3460682\\
0.5629563	-0.346337\\
0.5630563	-0.3484113\\
0.5631563	-0.347188\\
0.5632563	-0.345728\\
0.5633563	-0.3453413\\
0.5634563	-0.3482701\\
0.5635564	-0.3460454\\
0.5636564	-0.3442552\\
0.5637564	-0.3441182\\
0.5638564	-0.3437115\\
0.5639564	-0.3427447\\
0.5640564	-0.3437459\\
0.5641564	-0.3420798\\
0.5642564	-0.3419425\\
0.5643564	-0.3418755\\
0.5644564	-0.3401294\\
0.5645565	-0.3368789\\
0.5646565	-0.3369173\\
0.5647565	-0.3376322\\
0.5648565	-0.336502\\
0.5649565	-0.3344278\\
0.5650565	-0.3312083\\
0.5651565	-0.3303299\\
0.5652565	-0.3291827\\
0.5653565	-0.3288954\\
0.5654565	-0.3256731\\
0.5655566	-0.3244606\\
0.5656566	-0.3224384\\
0.5657566	-0.319165\\
0.5658566	-0.3162626\\
0.5659566	-0.3158067\\
0.5660566	-0.3158731\\
0.5661566	-0.3129407\\
0.5662566	-0.3109218\\
0.5663566	-0.3111879\\
0.5664566	-0.3081425\\
0.5665567	-0.3056955\\
0.5666567	-0.3035208\\
0.5667567	-0.3020683\\
0.5668567	-0.2985287\\
0.5669567	-0.2963625\\
0.5670567	-0.2959508\\
0.5671567	-0.2927651\\
0.5672567	-0.2881254\\
0.5673567	-0.2828539\\
0.5674567	-0.2806006\\
0.5675568	-0.2813522\\
0.5676568	-0.279836\\
0.5677568	-0.2735256\\
0.5678568	-0.2714368\\
0.5679568	-0.2693278\\
0.5680568	-0.2663201\\
0.5681568	-0.2628101\\
0.5682568	-0.2578187\\
0.5683568	-0.2552358\\
0.5684568	-0.2527032\\
0.5685569	-0.2487087\\
0.5686569	-0.2472771\\
0.5687569	-0.2428038\\
0.5688569	-0.2378909\\
0.5689569	-0.2347466\\
0.5690569	-0.2299038\\
0.5691569	-0.2266815\\
0.5692569	-0.2238902\\
0.5693569	-0.2198996\\
0.5694569	-0.216302\\
0.569557	-0.2129804\\
0.569657	-0.208094\\
0.569757	-0.2068195\\
0.569857	-0.2037185\\
0.569957	-0.197854\\
0.570057	-0.1941091\\
0.570157	-0.1891108\\
0.570257	-0.185395\\
0.570357	-0.1813126\\
0.570457	-0.1764971\\
0.5705571	-0.1725102\\
0.5706571	-0.1673826\\
0.5707571	-0.1620042\\
0.5708571	-0.1582419\\
0.5709571	-0.152183\\
0.5710571	-0.1487475\\
0.5711571	-0.1441654\\
0.5712571	-0.1387153\\
0.5713571	-0.1333588\\
0.5714571	-0.1279319\\
0.5715572	-0.1248663\\
0.5716572	-0.1226155\\
0.5717572	-0.119455\\
0.5718572	-0.1145244\\
0.5719572	-0.1099285\\
0.5720572	-0.1058613\\
0.5721572	-0.0979493\\
0.5722572	-0.09160554\\
0.5723572	-0.08494424\\
0.5724572	-0.07845296\\
0.5725573	-0.07350936\\
0.5726573	-0.06929302\\
0.5727573	-0.06622973\\
0.5728573	-0.06069072\\
0.5729573	-0.05738471\\
0.5730573	-0.05147755\\
0.5731573	-0.04526979\\
0.5732573	-0.03758754\\
0.5733573	-0.03260451\\
0.5734573	-0.02822394\\
0.5735574	-0.02306297\\
0.5736574	-0.01798885\\
0.5737574	-0.01294586\\
0.5738574	-0.007760103\\
0.5739574	-0.0008801467\\
0.5740574	0.005356826\\
0.5741574	0.01151046\\
0.5742574	0.01468622\\
0.5743574	0.0198337\\
0.5744574	0.02484528\\
0.5745575	0.02976201\\
0.5746575	0.03443351\\
0.5747575	0.03954302\\
0.5748575	0.04606938\\
0.5749575	0.05191802\\
0.5750575	0.06022607\\
0.5751575	0.06727216\\
0.5752575	0.07174239\\
0.5753575	0.0791827\\
0.5754575	0.08545047\\
0.5755576	0.09068659\\
0.5756576	0.09682487\\
0.5757576	0.1018817\\
0.5758576	0.1068008\\
0.5759576	0.1105933\\
0.5760576	0.1168889\\
0.5761576	0.1211375\\
0.5762576	0.1278509\\
0.5763576	0.132289\\
0.5764576	0.1388969\\
0.5765577	0.1460127\\
0.5766577	0.1527763\\
0.5767577	0.1602765\\
0.5768577	0.1640932\\
0.5769577	0.1692045\\
0.5770577	0.1751224\\
0.5771577	0.1807963\\
0.5772577	0.1846134\\
0.5773577	0.1907603\\
0.5774577	0.194727\\
0.5775578	0.2002398\\
0.5776578	0.2088071\\
0.5777578	0.2145541\\
0.5778578	0.2185379\\
0.5779578	0.2262695\\
0.5780578	0.2308221\\
0.5781578	0.235476\\
0.5782578	0.2433736\\
0.5783578	0.2479664\\
0.5784578	0.2542896\\
0.5785579	0.2583719\\
0.5786579	0.2648796\\
0.5787579	0.2695314\\
0.5788579	0.2736215\\
0.5789579	0.2799565\\
0.5790579	0.2865304\\
0.5791579	0.2913277\\
0.5792579	0.2937816\\
0.5793579	0.2996866\\
0.5794579	0.3047057\\
0.579558	0.3111417\\
0.579658	0.3161048\\
0.579758	0.3216336\\
0.579858	0.326406\\
0.579958	0.3327001\\
0.580058	0.3381705\\
0.580158	0.3432159\\
0.580258	0.3468517\\
0.580358	0.3517432\\
0.580458	0.356201\\
0.5805581	0.3595564\\
0.5806581	0.3635008\\
0.5807581	0.3692118\\
0.5808581	0.3736709\\
0.5809581	0.3780075\\
0.5810581	0.384861\\
0.5811581	0.38916\\
0.5812581	0.3948274\\
0.5813581	0.3982319\\
0.5814581	0.4014329\\
0.5815582	0.4053515\\
0.5816582	0.410188\\
0.5817582	0.4138888\\
0.5818582	0.4170262\\
0.5819582	0.4203574\\
0.5820582	0.4251727\\
0.5821582	0.428126\\
0.5822582	0.4309976\\
0.5823582	0.4372609\\
0.5824582	0.440341\\
0.5825583	0.4428354\\
0.5826583	0.4488532\\
0.5827583	0.4505931\\
0.5828583	0.4534799\\
0.5829583	0.4578332\\
0.5830583	0.4619154\\
0.5831583	0.4670111\\
0.5832583	0.470566\\
0.5833583	0.4729014\\
0.5834583	0.4763219\\
0.5835584	0.479705\\
0.5836584	0.4812195\\
0.5837584	0.4831175\\
0.5838584	0.4838\\
0.5839584	0.4875207\\
0.5840584	0.491329\\
0.5841584	0.4929024\\
0.5842584	0.4935325\\
0.5843584	0.4958136\\
0.5844584	0.4968774\\
0.5845585	0.4989869\\
0.5846585	0.5013885\\
0.5847585	0.5036773\\
0.5848585	0.5089507\\
0.5849585	0.5123144\\
0.5850585	0.5120175\\
0.5851585	0.5134278\\
0.5852585	0.5131035\\
0.5853585	0.5146834\\
0.5854585	0.5181239\\
0.5855586	0.5192873\\
0.5856586	0.5208281\\
0.5857586	0.5220561\\
0.5858586	0.5225552\\
0.5859586	0.5224199\\
0.5860586	0.5229392\\
0.5861586	0.5236925\\
0.5862586	0.5260468\\
0.5863586	0.5258866\\
0.5864586	0.52546\\
0.5865587	0.5261342\\
0.5866587	0.5258221\\
0.5867587	0.5268462\\
0.5868587	0.5262829\\
0.5869587	0.5271998\\
0.5870587	0.5260003\\
0.5871587	0.5261775\\
0.5872587	0.5268661\\
0.5873587	0.525564\\
0.5874587	0.5253947\\
0.5875588	0.5229705\\
0.5876588	0.5230543\\
0.5877588	0.5225027\\
0.5878588	0.5229404\\
0.5879588	0.52047\\
0.5880588	0.5195721\\
0.5881588	0.51876\\
0.5882588	0.5182957\\
0.5883588	0.5188494\\
0.5884588	0.5147766\\
0.5885589	0.5144778\\
0.5886589	0.5127228\\
0.5887589	0.5084787\\
0.5888589	0.5058554\\
0.5889589	0.5038559\\
0.5890589	0.5038477\\
0.5891589	0.500401\\
0.5892589	0.496995\\
0.5893589	0.4948695\\
0.5894589	0.4944248\\
0.589559	0.4903847\\
0.589659	0.4886913\\
0.589759	0.4869552\\
0.589859	0.4851282\\
0.589959	0.4818375\\
0.590059	0.4774138\\
0.590159	0.4751792\\
0.590259	0.4707384\\
0.590359	0.4674598\\
0.590459	0.4629312\\
0.5905591	0.4598873\\
0.5906591	0.4553003\\
0.5907591	0.4518696\\
0.5908591	0.4479314\\
0.5909591	0.4450696\\
0.5910591	0.4414931\\
0.5911591	0.4379103\\
0.5912591	0.4324498\\
0.5913591	0.4293324\\
0.5914591	0.4268125\\
0.5915592	0.4215032\\
0.5916592	0.4147217\\
0.5917592	0.4085948\\
0.5918592	0.4028001\\
0.5919592	0.3974044\\
0.5920592	0.3923149\\
0.5921592	0.38778\\
0.5922592	0.3813821\\
0.5923592	0.3771103\\
0.5924592	0.372933\\
0.5925593	0.3689417\\
0.5926593	0.3642233\\
0.5927593	0.3579026\\
0.5928593	0.3531914\\
0.5929593	0.3484109\\
0.5930593	0.3426156\\
0.5931593	0.334284\\
0.5932593	0.3268897\\
0.5933593	0.3194905\\
0.5934593	0.3151861\\
0.5935594	0.3096016\\
0.5936594	0.3021226\\
0.5937594	0.2965942\\
0.5938594	0.2911584\\
0.5939594	0.2849074\\
0.5940594	0.2785561\\
0.5941594	0.2707408\\
0.5942594	0.264044\\
0.5943594	0.2560288\\
0.5944594	0.2489442\\
0.5945595	0.2409209\\
0.5946595	0.2335388\\
0.5947595	0.2257889\\
0.5948595	0.2189322\\
0.5949595	0.2119373\\
0.5950595	0.2035343\\
0.5951595	0.197209\\
0.5952595	0.1906133\\
0.5953595	0.1825784\\
0.5954595	0.1744499\\
0.5955596	0.1676699\\
0.5956596	0.1609702\\
0.5957596	0.1562704\\
0.5958596	0.1487128\\
0.5959596	0.1396869\\
0.5960596	0.1304898\\
0.5961596	0.12069\\
0.5962596	0.1103064\\
0.5963596	0.1030714\\
0.5964596	0.09560756\\
0.5965597	0.08569961\\
0.5966597	0.07734154\\
0.5967597	0.06983376\\
0.5968597	0.06278966\\
0.5969597	0.05536413\\
0.5970597	0.04708958\\
0.5971597	0.03836371\\
0.5972597	0.0289317\\
0.5973597	0.02073569\\
0.5974597	0.01243219\\
0.5975598	0.003770519\\
0.5976598	-0.004643183\\
0.5977598	-0.01166351\\
0.5978598	-0.01955461\\
0.5979598	-0.02908572\\
0.5980598	-0.0389247\\
0.5981598	-0.04710108\\
0.5982598	-0.05543948\\
0.5983598	-0.06734714\\
0.5984598	-0.07557828\\
0.5985599	-0.08475637\\
0.5986599	-0.09329067\\
0.5987599	-0.1017525\\
0.5988599	-0.1107267\\
0.5989599	-0.118564\\
0.5990599	-0.1263562\\
0.5991599	-0.1342137\\
0.5992599	-0.1448654\\
0.5993599	-0.1538065\\
0.5994599	-0.1645681\\
0.59956	-0.1736361\\
0.59966	-0.1826383\\
0.59976	-0.1915629\\
0.59986	-0.1968014\\
0.59996	-0.2053583\\
0.60006	-0.2148664\\
0.60016	-0.2249608\\
0.60026	-0.2332684\\
0.60036	-0.2413126\\
0.60046	-0.2519885\\
0.6005601	-0.2608994\\
0.6006601	-0.2671697\\
0.6007601	-0.2737899\\
0.6008601	-0.283173\\
0.6009601	-0.293874\\
0.6010601	-0.303665\\
0.6011601	-0.3107389\\
0.6012601	-0.3193661\\
0.6013601	-0.3283254\\
0.6014601	-0.3349791\\
0.6015602	-0.3444088\\
0.6016602	-0.3528409\\
0.6017602	-0.3634265\\
0.6018602	-0.3715682\\
0.6019602	-0.3798231\\
0.6020602	-0.388198\\
0.6021602	-0.3967601\\
0.6022602	-0.4044578\\
0.6023602	-0.4105187\\
0.6024602	-0.4198623\\
0.6025603	-0.4235415\\
0.6026603	-0.4339236\\
0.6027603	-0.4409716\\
0.6028603	-0.4493378\\
0.6029603	-0.4600462\\
0.6030603	-0.4674284\\
0.6031603	-0.4749686\\
0.6032603	-0.4802135\\
0.6033603	-0.4874481\\
0.6034603	-0.4934478\\
0.6035604	-0.5021716\\
0.6036604	-0.5078019\\
0.6037604	-0.515133\\
0.6038604	-0.5235213\\
0.6039604	-0.5311052\\
0.6040604	-0.5373831\\
0.6041604	-0.5419131\\
0.6042604	-0.5510133\\
0.6043604	-0.5565117\\
0.6044604	-0.5647353\\
0.6045605	-0.5722399\\
0.6046605	-0.5775432\\
0.6047605	-0.5835827\\
0.6048605	-0.5896146\\
0.6049605	-0.596506\\
0.6050605	-0.60139\\
0.6051605	-0.6074109\\
0.6052605	-0.6127523\\
0.6053605	-0.6190786\\
0.6054605	-0.6239371\\
0.6055606	-0.6289568\\
0.6056606	-0.6338606\\
0.6057606	-0.6373361\\
0.6058606	-0.642279\\
0.6059606	-0.646774\\
0.6060606	-0.6535928\\
0.6061606	-0.6594579\\
0.6062606	-0.6645544\\
0.6063606	-0.6684116\\
0.6064606	-0.6737201\\
0.6065607	-0.6789819\\
0.6066607	-0.6838391\\
0.6067607	-0.6864051\\
0.6068607	-0.6893765\\
0.6069607	-0.6932641\\
0.6070607	-0.6972586\\
0.6071607	-0.7005731\\
0.6072607	-0.7018074\\
0.6073607	-0.7055758\\
0.6074607	-0.7093173\\
0.6075608	-0.7121575\\
0.6076608	-0.7143177\\
0.6077608	-0.7170247\\
0.6078608	-0.7202933\\
0.6079608	-0.7233485\\
0.6080608	-0.7257596\\
0.6081608	-0.7273702\\
0.6082608	-0.7294282\\
0.6083608	-0.7296946\\
0.6084608	-0.731004\\
0.6085609	-0.7326752\\
0.6086609	-0.7344512\\
0.6087609	-0.7367369\\
0.6088609	-0.7350737\\
0.6089609	-0.739177\\
0.6090609	-0.7422705\\
0.6091609	-0.7435446\\
0.6092609	-0.7442296\\
0.6093609	-0.7440689\\
0.6094609	-0.7448067\\
0.609561	-0.7453612\\
0.609661	-0.74355\\
0.609761	-0.7420071\\
0.609861	-0.7409467\\
0.609961	-0.7407389\\
0.610061	-0.7386777\\
0.610161	-0.7364393\\
0.610261	-0.7371668\\
0.610361	-0.7356286\\
0.610461	-0.734206\\
0.6105611	-0.7337986\\
0.6106611	-0.7332205\\
0.6107611	-0.7327416\\
0.6108611	-0.7317293\\
0.6109611	-0.7274101\\
0.6110611	-0.7263774\\
0.6111611	-0.7241058\\
0.6112611	-0.7209852\\
0.6113611	-0.7185605\\
0.6114611	-0.7134305\\
0.6115612	-0.7089972\\
0.6116612	-0.7048988\\
0.6117612	-0.7036366\\
0.6118612	-0.7005334\\
0.6119612	-0.6981131\\
0.6120612	-0.6962795\\
0.6121612	-0.6907894\\
0.6122612	-0.6871938\\
0.6123612	-0.6842357\\
0.6124612	-0.6808451\\
0.6125613	-0.6750547\\
0.6126613	-0.6689208\\
0.6127613	-0.6628555\\
0.6128613	-0.6580939\\
0.6129613	-0.6537562\\
0.6130613	-0.6483115\\
0.6131613	-0.6418082\\
0.6132613	-0.6356345\\
0.6133613	-0.6296334\\
0.6134613	-0.6248402\\
0.6135614	-0.6184391\\
0.6136614	-0.6141783\\
0.6137614	-0.6080678\\
0.6138614	-0.6012091\\
0.6139614	-0.5926393\\
0.6140614	-0.5864748\\
0.6141614	-0.5817938\\
0.6142614	-0.5720257\\
0.6143614	-0.5651659\\
0.6144614	-0.559041\\
0.6145615	-0.5513878\\
0.6146615	-0.5440202\\
0.6147615	-0.5377377\\
0.6148615	-0.5282812\\
0.6149615	-0.5181744\\
0.6150615	-0.5104009\\
0.6151615	-0.502378\\
0.6152615	-0.4940253\\
0.6153615	-0.4854009\\
0.6154615	-0.476899\\
0.6155616	-0.469271\\
0.6156616	-0.4605978\\
0.6157616	-0.4505429\\
0.6158616	-0.4393939\\
0.6159616	-0.4313603\\
0.6160616	-0.4217738\\
0.6161616	-0.4127358\\
0.6162616	-0.4044989\\
0.6163616	-0.3960331\\
0.6164616	-0.3835985\\
0.6165617	-0.3727208\\
0.6166617	-0.3636873\\
0.6167617	-0.3549452\\
0.6168617	-0.3440993\\
0.6169617	-0.3335208\\
0.6170617	-0.3256752\\
0.6171617	-0.3133312\\
0.6172617	-0.3023026\\
0.6173617	-0.2892323\\
0.6174617	-0.2800123\\
0.6175618	-0.2689573\\
0.6176618	-0.2574243\\
0.6177618	-0.2475917\\
0.6178618	-0.2361489\\
0.6179618	-0.2236081\\
0.6180618	-0.2098054\\
0.6181618	-0.1984132\\
0.6182618	-0.1875219\\
0.6183618	-0.1741015\\
0.6184618	-0.1644376\\
0.6185619	-0.1528557\\
0.6186619	-0.1412643\\
0.6187619	-0.1301998\\
0.6188619	-0.1196064\\
0.6189619	-0.1082259\\
0.6190619	-0.09483391\\
0.6191619	-0.08346869\\
0.6192619	-0.07126268\\
0.6193619	-0.06039592\\
0.6194619	-0.04660791\\
0.619562	-0.03164452\\
0.619662	-0.02185915\\
0.619762	-0.010018\\
0.619862	0.003780405\\
0.619962	0.01671253\\
0.620062	0.02888413\\
0.620162	0.0406673\\
0.620262	0.05504047\\
0.620362	0.06878999\\
0.620462	0.08259588\\
0.6205621	0.09443003\\
0.6206621	0.1049334\\
0.6207621	0.1159583\\
0.6208621	0.129236\\
0.6209621	0.1417075\\
0.6210621	0.1555744\\
0.6211621	0.1666835\\
0.6212621	0.1802523\\
0.6213621	0.1943492\\
0.6214621	0.2080699\\
0.6215622	0.2205038\\
0.6216622	0.2325015\\
0.6217622	0.2466832\\
0.6218622	0.2573494\\
0.6219622	0.267664\\
0.6220622	0.2793688\\
0.6221622	0.2929419\\
0.6222622	0.3070093\\
0.6223622	0.3199015\\
0.6224622	0.3319564\\
0.6225623	0.3469659\\
0.6226623	0.3610631\\
0.6227623	0.3710987\\
0.6228623	0.3811631\\
0.6229623	0.3929723\\
0.6230623	0.4066831\\
0.6231623	0.4188573\\
0.6232623	0.4315508\\
0.6233623	0.4427835\\
0.6234623	0.4535105\\
0.6235624	0.466333\\
0.6236624	0.4779552\\
0.6237624	0.4898744\\
0.6238624	0.5027951\\
0.6239624	0.513034\\
0.6240624	0.5239953\\
0.6241624	0.5359945\\
0.6242624	0.5460243\\
0.6243624	0.5604814\\
0.6244624	0.571035\\
0.6245625	0.5822494\\
0.6246625	0.5942521\\
0.6247625	0.6046176\\
0.6248625	0.613831\\
0.6249625	0.6253184\\
0.6250625	0.6351377\\
0.6251625	0.6457729\\
0.6252625	0.6548539\\
0.6253625	0.6652602\\
0.6254625	0.6765801\\
0.6255626	0.6873595\\
0.6256626	0.6963746\\
0.6257626	0.706224\\
0.6258626	0.719426\\
0.6259626	0.7283306\\
0.6260626	0.7375059\\
0.6261626	0.7467948\\
0.6262626	0.7567607\\
0.6263626	0.7652886\\
0.6264626	0.7710417\\
0.6265627	0.7771716\\
0.6266627	0.7856239\\
0.6267627	0.79437\\
0.6268627	0.801866\\
0.6269627	0.8085863\\
0.6270627	0.8184805\\
0.6271627	0.8273445\\
0.6272627	0.8357644\\
0.6273627	0.845623\\
0.6274627	0.8516984\\
0.6275628	0.8597141\\
0.6276628	0.867534\\
0.6277628	0.8723359\\
0.6278628	0.8787504\\
0.6279628	0.8836968\\
0.6280628	0.8904369\\
0.6281628	0.8972112\\
0.6282628	0.9019159\\
0.6283628	0.9092639\\
0.6284628	0.9140884\\
0.6285629	0.9188302\\
0.6286629	0.9241178\\
0.6287629	0.9289611\\
0.6288629	0.9341239\\
0.6289629	0.9379766\\
0.6290629	0.9437798\\
0.6291629	0.9486591\\
0.6292629	0.9524639\\
0.6293629	0.9550681\\
0.6294629	0.9554019\\
0.629563	0.9573315\\
0.629663	0.9610261\\
0.629763	0.9636591\\
0.629863	0.967962\\
0.629963	0.9717163\\
0.630063	0.9744408\\
0.630163	0.976532\\
0.630263	0.9779149\\
0.630363	0.9810914\\
0.630463	0.9822174\\
0.6305631	0.9826691\\
0.6306631	0.9830308\\
0.6307631	0.9844442\\
0.6308631	0.9860296\\
0.6309631	0.984263\\
0.6310631	0.9841495\\
0.6311631	0.9844666\\
0.6312631	0.983149\\
0.6313631	0.9812427\\
0.6314631	0.9790949\\
0.6315632	0.9799965\\
0.6316632	0.9804686\\
0.6317632	0.979102\\
0.6318632	0.976317\\
0.6319632	0.973938\\
0.6320632	0.9723179\\
0.6321632	0.9687649\\
0.6322632	0.9667738\\
0.6323632	0.9644818\\
0.6324632	0.9623354\\
0.6325633	0.9583243\\
0.6326633	0.9569914\\
0.6327633	0.9536552\\
0.6328633	0.9469641\\
0.6329633	0.9417764\\
0.6330633	0.9385153\\
0.6331633	0.9339815\\
0.6332633	0.9277219\\
0.6333633	0.9243626\\
0.6334633	0.9181244\\
0.6335634	0.9114814\\
0.6336634	0.9040204\\
0.6337634	0.8981336\\
0.6338634	0.8923042\\
0.6339634	0.8867154\\
0.6340634	0.8809978\\
0.6341634	0.8737847\\
0.6342634	0.8674924\\
0.6343634	0.8606286\\
0.6344634	0.8507553\\
0.6345635	0.8433949\\
0.6346635	0.8359205\\
0.6347635	0.8271465\\
0.6348635	0.8175585\\
0.6349635	0.8092055\\
0.6350635	0.8008958\\
0.6351635	0.7916557\\
0.6352635	0.7838952\\
0.6353635	0.776256\\
0.6354635	0.7649181\\
0.6355636	0.7539493\\
0.6356636	0.7448599\\
0.6357636	0.7334679\\
0.6358636	0.7241723\\
0.6359636	0.7137243\\
0.6360636	0.7038756\\
0.6361636	0.692423\\
0.6362636	0.6799698\\
0.6363636	0.668648\\
0.6364636	0.6553832\\
0.6365637	0.6433157\\
0.6366637	0.6333876\\
0.6367637	0.619843\\
0.6368637	0.6074383\\
0.6369637	0.5949593\\
0.6370637	0.5818118\\
0.6371637	0.5689457\\
0.6372637	0.5563395\\
0.6373637	0.5432207\\
0.6374637	0.5292386\\
0.6375638	0.5183807\\
0.6376638	0.5036056\\
0.6377638	0.4885553\\
0.6378638	0.4738326\\
0.6379638	0.4619828\\
0.6380638	0.4472431\\
0.6381638	0.4328379\\
0.6382638	0.4181718\\
0.6383638	0.4053602\\
0.6384638	0.3905116\\
0.6385639	0.3754804\\
0.6386639	0.3602735\\
0.6387639	0.3426249\\
0.6388639	0.3296156\\
0.6389639	0.3153863\\
0.6390639	0.2986648\\
0.6391639	0.2815677\\
0.6392639	0.2638724\\
0.6393639	0.2476275\\
0.6394639	0.2331898\\
0.639564	0.2169115\\
0.639664	0.2021424\\
0.639764	0.1845734\\
0.639864	0.1694301\\
0.639964	0.1531685\\
0.640064	0.136272\\
0.640164	0.1226766\\
0.640264	0.1056902\\
0.640364	0.08884729\\
0.640464	0.07135484\\
0.6405641	0.05528891\\
0.6406641	0.03882734\\
0.6407641	0.02095163\\
0.6408641	0.003509135\\
0.6409641	-0.01370545\\
0.6410641	-0.0334195\\
0.6411641	-0.05030319\\
0.6412641	-0.06630842\\
0.6413641	-0.08460978\\
0.6414641	-0.1021427\\
0.6415642	-0.1184643\\
0.6416642	-0.1322058\\
0.6417642	-0.1495002\\
0.6418642	-0.1669041\\
0.6419642	-0.184523\\
0.6420642	-0.2028111\\
0.6421642	-0.2191042\\
0.6422642	-0.2365317\\
0.6423642	-0.2561334\\
0.6424642	-0.2722143\\
0.6425643	-0.289507\\
0.6426643	-0.3042803\\
0.6427643	-0.3222821\\
0.6428643	-0.3387062\\
0.6429643	-0.3542575\\
0.6430643	-0.3732364\\
0.6431643	-0.387892\\
0.6432643	-0.4060705\\
0.6433643	-0.4234275\\
0.6434643	-0.4376072\\
0.6435644	-0.4562312\\
0.6436644	-0.4730438\\
0.6437644	-0.4906648\\
0.6438644	-0.5070081\\
0.6439644	-0.5241321\\
0.6440644	-0.5411175\\
0.6441644	-0.5558884\\
0.6442644	-0.5739071\\
0.6443644	-0.5914144\\
0.6444644	-0.6065705\\
0.6445645	-0.6210798\\
0.6446645	-0.6360386\\
0.6447645	-0.6507856\\
0.6448645	-0.6657332\\
0.6449645	-0.6782512\\
0.6450645	-0.6928359\\
0.6451645	-0.7069228\\
0.6452645	-0.7242563\\
0.6453645	-0.7415748\\
0.6454645	-0.7573866\\
0.6455646	-0.7732614\\
0.6456646	-0.7879714\\
0.6457646	-0.8010152\\
0.6458646	-0.8146008\\
0.6459646	-0.8257998\\
0.6460646	-0.8392747\\
0.6461646	-0.851361\\
0.6462646	-0.8656624\\
0.6463646	-0.8782148\\
0.6464646	-0.8924857\\
0.6465647	-0.9058562\\
0.6466647	-0.9181726\\
0.6467647	-0.9314663\\
0.6468647	-0.9447922\\
0.6469647	-0.9563881\\
0.6470647	-0.9674343\\
0.6471647	-0.979981\\
0.6472647	-0.9918055\\
0.6473647	-1.002325\\
0.6474647	-1.012241\\
0.6475648	-1.023163\\
0.6476648	-1.03374\\
0.6477648	-1.045527\\
0.6478648	-1.05317\\
0.6479648	-1.063648\\
0.6480648	-1.073648\\
0.6481648	-1.082813\\
0.6482648	-1.092526\\
0.6483648	-1.099723\\
0.6484648	-1.107148\\
0.6485649	-1.117677\\
0.6486649	-1.126163\\
0.6487649	-1.135192\\
0.6488649	-1.141698\\
0.6489649	-1.146861\\
0.6490649	-1.155468\\
0.6491649	-1.161966\\
0.6492649	-1.16787\\
0.6493649	-1.176313\\
0.6494649	-1.183963\\
0.649565	-1.188595\\
0.649665	-1.194025\\
0.649765	-1.198123\\
0.649865	-1.202645\\
0.649965	-1.205944\\
0.650065	-1.209536\\
0.650165	-1.214698\\
0.650265	-1.217767\\
0.650365	-1.222037\\
0.650465	-1.223561\\
0.6505651	-1.2267\\
0.6506651	-1.229684\\
0.6507651	-1.230511\\
0.6508651	-1.23331\\
0.6509651	-1.235793\\
0.6510651	-1.238442\\
0.6511651	-1.239801\\
0.6512651	-1.240634\\
0.6513651	-1.240482\\
0.6514651	-1.24105\\
0.6515652	-1.240628\\
0.6516652	-1.240148\\
0.6517652	-1.238571\\
0.6518652	-1.238029\\
0.6519652	-1.236803\\
0.6520652	-1.234231\\
0.6521652	-1.232643\\
0.6522652	-1.231315\\
0.6523652	-1.227588\\
0.6524652	-1.223416\\
0.6525653	-1.221535\\
0.6526653	-1.218043\\
0.6527653	-1.210852\\
0.6528653	-1.20455\\
0.6529653	-1.200831\\
0.6530653	-1.196353\\
0.6531653	-1.193295\\
0.6532653	-1.187845\\
0.6533653	-1.181696\\
0.6534653	-1.175931\\
0.6535654	-1.170114\\
0.6536654	-1.1628\\
0.6537654	-1.155849\\
0.6538654	-1.151135\\
0.6539654	-1.1424\\
0.6540654	-1.133351\\
0.6541654	-1.125405\\
0.6542654	-1.116942\\
0.6543654	-1.107504\\
0.6544654	-1.097243\\
0.6545655	-1.089775\\
0.6546655	-1.079414\\
0.6547655	-1.067846\\
0.6548655	-1.056556\\
0.6549655	-1.047026\\
0.6550655	-1.036821\\
0.6551655	-1.02742\\
0.6552655	-1.015626\\
0.6553655	-1.00206\\
0.6554655	-0.9900925\\
0.6555656	-0.9792766\\
0.6556656	-0.9652888\\
0.6557656	-0.9511677\\
0.6558656	-0.9380983\\
0.6559656	-0.9233911\\
0.6560656	-0.9115974\\
0.6561656	-0.8986069\\
0.6562656	-0.8854656\\
0.6563656	-0.8683778\\
0.6564656	-0.8535267\\
0.6565657	-0.838646\\
0.6566657	-0.8235958\\
0.6567657	-0.8092166\\
0.6568657	-0.7951017\\
0.6569657	-0.7787834\\
0.6570657	-0.7620683\\
0.6571657	-0.744632\\
0.6572657	-0.7296495\\
0.6573657	-0.7118641\\
0.6574657	-0.6920486\\
0.6575658	-0.674339\\
0.6576658	-0.657213\\
0.6577658	-0.6402402\\
0.6578658	-0.622713\\
0.6579658	-0.6035454\\
0.6580658	-0.5864869\\
0.6581658	-0.5691233\\
0.6582658	-0.5513346\\
0.6583658	-0.5335029\\
0.6584658	-0.5129371\\
0.6585659	-0.4939207\\
0.6586659	-0.4746149\\
0.6587659	-0.4533226\\
0.6588659	-0.4344942\\
0.6589659	-0.4158371\\
0.6590659	-0.3950609\\
0.6591659	-0.3730207\\
0.6592659	-0.3514311\\
0.6593659	-0.3308382\\
0.6594659	-0.3122758\\
0.659566	-0.2924922\\
0.659666	-0.2721751\\
0.659766	-0.2529796\\
0.659866	-0.2318324\\
0.659966	-0.209038\\
0.660066	-0.1869103\\
0.660166	-0.1658425\\
0.660266	-0.1450473\\
0.660366	-0.1228431\\
0.660466	-0.1001813\\
0.6605661	-0.07911126\\
0.6606661	-0.05662403\\
0.6607661	-0.03298199\\
0.6608661	-0.01084429\\
0.6609661	0.0133016\\
0.6610661	0.03294084\\
0.6611661	0.05076606\\
0.6612661	0.07351752\\
0.6613661	0.0955517\\
0.6614661	0.114738\\
0.6615662	0.1361028\\
0.6616662	0.1589549\\
0.6617662	0.1819698\\
0.6618662	0.2062034\\
0.6619662	0.226573\\
0.6620662	0.2493311\\
0.6621662	0.2711999\\
0.6622662	0.2938666\\
0.6623662	0.3158179\\
0.6624662	0.3386016\\
0.6625663	0.3620413\\
0.6626663	0.3840602\\
0.6627663	0.4049207\\
0.6628663	0.4262206\\
0.6629663	0.4481243\\
0.6630663	0.4688207\\
0.6631663	0.4904052\\
0.6632663	0.5114209\\
0.6633663	0.5335721\\
0.6634663	0.5566396\\
0.6635664	0.5760226\\
0.6636664	0.5980527\\
0.6637664	0.618308\\
0.6638664	0.6379077\\
0.6639664	0.6597427\\
0.6640664	0.6792909\\
0.6641664	0.7008655\\
0.6642664	0.7200803\\
0.6643664	0.7386814\\
0.6644664	0.7605364\\
0.6645665	0.7798426\\
0.6646665	0.799194\\
0.6647665	0.8190941\\
0.6648665	0.838547\\
0.6649665	0.8574582\\
0.6650665	0.8786274\\
0.6651665	0.8991152\\
0.6652665	0.9170494\\
0.6653665	0.9358478\\
0.6654665	0.9515264\\
0.6655666	0.9689608\\
0.6656666	0.9854462\\
0.6657666	1.002869\\
0.6658666	1.019605\\
0.6659666	1.036826\\
0.6660666	1.054051\\
0.6661666	1.071807\\
0.6662666	1.088732\\
0.6663666	1.104144\\
0.6664666	1.121214\\
0.6665667	1.138348\\
0.6666667	1.150756\\
0.6667667	1.164691\\
0.6668667	1.17882\\
0.6669667	1.193583\\
0.6670667	1.207691\\
0.6671667	1.221531\\
0.6672667	1.235388\\
0.6673667	1.251123\\
0.6674667	1.26424\\
0.6675668	1.278123\\
0.6676668	1.290152\\
0.6677668	1.300642\\
0.6678668	1.310751\\
0.6679668	1.322029\\
0.6680668	1.332685\\
0.6681668	1.344315\\
0.6682668	1.35371\\
0.6683668	1.362686\\
0.6684668	1.373045\\
0.6685669	1.383511\\
0.6686669	1.394026\\
0.6687669	1.402748\\
0.6688669	1.408109\\
0.6689669	1.415246\\
0.6690669	1.423447\\
0.6691669	1.430389\\
0.6692669	1.436917\\
0.6693669	1.442789\\
0.6694669	1.451119\\
0.669567	1.455744\\
0.669667	1.459134\\
0.669767	1.464572\\
0.669867	1.470365\\
0.669967	1.474266\\
0.670067	1.477446\\
0.670167	1.482411\\
0.670267	1.485491\\
0.670367	1.488075\\
0.670467	1.489138\\
0.6705671	1.491571\\
0.6706671	1.494646\\
0.6707671	1.496294\\
0.6708671	1.495407\\
0.6709671	1.494522\\
0.6710671	1.492238\\
0.6711671	1.492327\\
0.6712671	1.490227\\
0.6713671	1.48778\\
0.6714671	1.486943\\
0.6715672	1.486632\\
0.6716672	1.483557\\
0.6717672	1.481701\\
0.6718672	1.476341\\
0.6719672	1.472565\\
0.6720672	1.467415\\
0.6721672	1.462196\\
0.6722672	1.456465\\
0.6723672	1.452148\\
0.6724672	1.444564\\
0.6725673	1.438746\\
0.6726673	1.431489\\
0.6727673	1.424473\\
0.6728673	1.418754\\
0.6729673	1.411033\\
0.6730673	1.401981\\
0.6731673	1.394463\\
0.6732673	1.384364\\
0.6733673	1.373808\\
0.6734673	1.366856\\
0.6735674	1.356862\\
0.6736674	1.344983\\
0.6737674	1.333369\\
0.6738674	1.318913\\
0.6739674	1.306904\\
0.6740674	1.2927\\
0.6741674	1.280891\\
0.6742674	1.267615\\
0.6743674	1.25575\\
0.6744674	1.24121\\
0.6745675	1.227801\\
0.6746675	1.213119\\
0.6747675	1.19743\\
0.6748675	1.185005\\
0.6749675	1.169569\\
0.6750675	1.153677\\
0.6751675	1.1381\\
0.6752675	1.120777\\
0.6753675	1.105873\\
0.6754675	1.087567\\
0.6755676	1.066014\\
0.6756676	1.048046\\
0.6757676	1.029125\\
0.6758676	1.010684\\
0.6759676	0.9909688\\
0.6760676	0.9704198\\
0.6761676	0.9507864\\
0.6762676	0.9326198\\
0.6763676	0.9126145\\
0.6764676	0.8925633\\
0.6765677	0.8722919\\
0.6766677	0.8524105\\
0.6767677	0.8314129\\
0.6768677	0.8089686\\
0.6769677	0.7857064\\
0.6770677	0.7641112\\
0.6771677	0.7413705\\
0.6772677	0.7191042\\
0.6773677	0.6949521\\
0.6774677	0.672985\\
0.6775678	0.6500975\\
0.6776678	0.6276704\\
0.6777678	0.6030083\\
0.6778678	0.5792675\\
0.6779678	0.5546048\\
0.6780678	0.5322738\\
0.6781678	0.5059641\\
0.6782678	0.4817021\\
0.6783678	0.4567177\\
0.6784678	0.4332941\\
0.6785679	0.4073491\\
0.6786679	0.3785134\\
0.6787679	0.3514301\\
0.6788679	0.3262424\\
0.6789679	0.3004955\\
0.6790679	0.2746969\\
0.6791679	0.2493886\\
0.6792679	0.2241776\\
0.6793679	0.1980281\\
0.6794679	0.1719182\\
0.679568	0.1439211\\
0.679668	0.1167372\\
0.679768	0.09102521\\
0.679868	0.06443987\\
0.679968	0.03772868\\
0.680068	0.009966518\\
0.680168	-0.01787349\\
0.680268	-0.0432134\\
0.680368	-0.06953062\\
0.680468	-0.098116\\
0.6805681	-0.1259532\\
0.6806681	-0.1535479\\
0.6807681	-0.1785849\\
0.6808681	-0.2049253\\
0.6809681	-0.2322054\\
0.6810681	-0.2597793\\
0.6811681	-0.2853156\\
0.6812681	-0.3137779\\
0.6813681	-0.3412622\\
0.6814681	-0.3682446\\
0.6815682	-0.3947774\\
0.6816682	-0.423746\\
0.6817682	-0.4502435\\
0.6818682	-0.4776727\\
0.6819682	-0.5042776\\
0.6820682	-0.5314602\\
0.6821682	-0.5554055\\
0.6822682	-0.5819725\\
0.6823682	-0.6059123\\
0.6824682	-0.6327752\\
0.6825683	-0.6607583\\
0.6826683	-0.6843445\\
0.6827683	-0.711527\\
0.6828683	-0.7374343\\
0.6829683	-0.762193\\
0.6830683	-0.7874027\\
0.6831683	-0.8114499\\
0.6832683	-0.836851\\
0.6833683	-0.8604731\\
0.6834683	-0.8849752\\
0.6835684	-0.9108389\\
0.6836684	-0.9354867\\
0.6837684	-0.9578882\\
0.6838684	-0.9801589\\
0.6839684	-1.004228\\
0.6840684	-1.026269\\
0.6841684	-1.050702\\
0.6842684	-1.073311\\
0.6843684	-1.096776\\
0.6844684	-1.117114\\
0.6845685	-1.138514\\
0.6846685	-1.161502\\
0.6847685	-1.181748\\
0.6848685	-1.200345\\
0.6849685	-1.219196\\
0.6850685	-1.242376\\
0.6851685	-1.260994\\
0.6852685	-1.277555\\
0.6853685	-1.297106\\
0.6854685	-1.318682\\
0.6855686	-1.33582\\
0.6856686	-1.353781\\
0.6857686	-1.373364\\
0.6858686	-1.392444\\
0.6859686	-1.407791\\
0.6860686	-1.423425\\
0.6861686	-1.439387\\
0.6862686	-1.45744\\
0.6863686	-1.473434\\
0.6864686	-1.487383\\
0.6865687	-1.50087\\
0.6866687	-1.515829\\
0.6867687	-1.529463\\
0.6868687	-1.543875\\
0.6869687	-1.558669\\
0.6870687	-1.571565\\
0.6871687	-1.582658\\
0.6872687	-1.593872\\
0.6873687	-1.603661\\
0.6874687	-1.613392\\
0.6875688	-1.621733\\
0.6876688	-1.628766\\
0.6877688	-1.639304\\
0.6878688	-1.64867\\
0.6879688	-1.659347\\
0.6880688	-1.668543\\
0.6881688	-1.676734\\
0.6882688	-1.683357\\
0.6883688	-1.690899\\
0.6884688	-1.697719\\
0.6885689	-1.702762\\
0.6886689	-1.708146\\
0.6887689	-1.712893\\
0.6888689	-1.71598\\
0.6889689	-1.718616\\
0.6890689	-1.721942\\
0.6891689	-1.724379\\
0.6892689	-1.725577\\
0.6893689	-1.726495\\
0.6894689	-1.729386\\
0.689569	-1.731669\\
0.689669	-1.730054\\
0.689769	-1.729359\\
0.689869	-1.729507\\
0.689969	-1.728435\\
0.690069	-1.727493\\
0.690169	-1.724007\\
0.690269	-1.721112\\
0.690369	-1.71802\\
0.690469	-1.71498\\
0.6905691	-1.709376\\
0.6906691	-1.703152\\
0.6907691	-1.699746\\
0.6908691	-1.691731\\
0.6909691	-1.68291\\
0.6910691	-1.674625\\
0.6911691	-1.667582\\
0.6912691	-1.660462\\
0.6913691	-1.655718\\
0.6914691	-1.647861\\
0.6915692	-1.639383\\
0.6916692	-1.628334\\
0.6917692	-1.616837\\
0.6918692	-1.604845\\
0.6919692	-1.591749\\
0.6920692	-1.57982\\
0.6921692	-1.567208\\
0.6922692	-1.555058\\
0.6923692	-1.54132\\
0.6924692	-1.528275\\
0.6925693	-1.512329\\
0.6926693	-1.498187\\
0.6927693	-1.48152\\
0.6928693	-1.465792\\
0.6929693	-1.450294\\
0.6930693	-1.434172\\
0.6931693	-1.417662\\
0.6932693	-1.400548\\
0.6933693	-1.381109\\
0.6934693	-1.362802\\
0.6935694	-1.344638\\
0.6936694	-1.32531\\
0.6937694	-1.306621\\
0.6938694	-1.283989\\
0.6939694	-1.264004\\
0.6940694	-1.243846\\
0.6941694	-1.221559\\
0.6942694	-1.196707\\
0.6943694	-1.176229\\
0.6944694	-1.155469\\
0.6945695	-1.133655\\
0.6946695	-1.110891\\
0.6947695	-1.085773\\
0.6948695	-1.061665\\
0.6949695	-1.035097\\
0.6950695	-1.011391\\
0.6951695	-0.9844254\\
0.6952695	-0.9582626\\
0.6953695	-0.9318239\\
0.6954695	-0.9048456\\
0.6955696	-0.8791715\\
0.6956696	-0.8524912\\
0.6957696	-0.8251592\\
0.6958696	-0.8001306\\
0.6959696	-0.7749749\\
0.6960696	-0.7440112\\
0.6961696	-0.7158347\\
0.6962696	-0.6888587\\
0.6963696	-0.6611983\\
0.6964696	-0.6338425\\
0.6965697	-0.6032485\\
0.6966697	-0.5723057\\
0.6967697	-0.5435137\\
0.6968697	-0.5133999\\
0.6969697	-0.4827621\\
0.6970697	-0.4523009\\
0.6971697	-0.422928\\
0.6972697	-0.3908439\\
0.6973697	-0.360547\\
0.6974697	-0.3304588\\
0.6975698	-0.2991363\\
0.6976698	-0.2672445\\
0.6977698	-0.2356696\\
0.6978698	-0.2053395\\
0.6979698	-0.1724723\\
0.6980698	-0.1414313\\
0.6981698	-0.1077704\\
0.6982698	-0.07666737\\
0.6983698	-0.04596325\\
0.6984698	-0.01264143\\
0.6985699	0.01826836\\
0.6986699	0.04932491\\
0.6987699	0.07993942\\
0.6988699	0.1128473\\
0.6989699	0.1445635\\
0.6990699	0.1769399\\
0.6991699	0.2090573\\
0.6992699	0.2414625\\
0.6993699	0.273015\\
0.6994699	0.3055225\\
0.69957	0.338338\\
0.69967	0.3704445\\
0.69977	0.4020552\\
0.69987	0.4330817\\
0.69997	0.4660505\\
0.70007	0.4994144\\
0.70017	0.530174\\
0.70027	0.5605002\\
0.70037	0.5908332\\
0.70047	0.6206875\\
0.7005701	0.6499996\\
0.7006701	0.6823187\\
0.7007701	0.712659\\
0.7008701	0.7434608\\
0.7009701	0.7732705\\
0.7010701	0.8050791\\
0.7011701	0.8344985\\
0.7012701	0.862538\\
0.7013701	0.8938721\\
0.7014701	0.9222585\\
0.7015702	0.9516828\\
0.7016702	0.9783992\\
0.7017702	1.007561\\
0.7018702	1.037293\\
0.7019702	1.066951\\
0.7020702	1.093068\\
0.7021702	1.1197\\
0.7022702	1.146083\\
0.7023702	1.173165\\
0.7024702	1.199805\\
0.7025703	1.225828\\
0.7026703	1.251916\\
0.7027703	1.275317\\
0.7028703	1.302246\\
0.7029703	1.326972\\
0.7030703	1.349648\\
0.7031703	1.373637\\
0.7032703	1.395199\\
0.7033703	1.419526\\
0.7034703	1.443939\\
0.7035704	1.466751\\
0.7036704	1.488132\\
0.7037704	1.508653\\
0.7038704	1.52857\\
0.7039704	1.549382\\
0.7040704	1.570871\\
0.7041704	1.588029\\
0.7042704	1.605289\\
0.7043704	1.622789\\
0.7044704	1.640834\\
0.7045705	1.658112\\
0.7046705	1.676384\\
0.7047705	1.694601\\
0.7048705	1.711683\\
0.7049705	1.72752\\
0.7050705	1.744953\\
0.7051705	1.759773\\
0.7052705	1.773667\\
0.7053705	1.786403\\
0.7054705	1.798846\\
0.7055706	1.812948\\
0.7056706	1.823376\\
0.7057706	1.833009\\
0.7058706	1.842055\\
0.7059706	1.852417\\
0.7060706	1.863167\\
0.7061706	1.871117\\
0.7062706	1.879431\\
0.7063706	1.888371\\
0.7064706	1.895636\\
0.7065707	1.902125\\
0.7066707	1.908723\\
0.7067707	1.915398\\
0.7068707	1.920939\\
0.7069707	1.924658\\
0.7070707	1.927677\\
0.7071707	1.929068\\
0.7072707	1.929664\\
0.7073707	1.931794\\
0.7074707	1.933036\\
0.7075708	1.934657\\
0.7076708	1.937699\\
0.7077708	1.938381\\
0.7078708	1.937534\\
0.7079708	1.934289\\
0.7080708	1.931474\\
0.7081708	1.928222\\
0.7082708	1.923901\\
0.7083708	1.921118\\
0.7084708	1.916009\\
0.7085709	1.910484\\
0.7086709	1.903116\\
0.7087709	1.895252\\
0.7088709	1.888243\\
0.7089709	1.881939\\
0.7090709	1.871834\\
0.7091709	1.863355\\
0.7092709	1.852464\\
0.7093709	1.842496\\
0.7094709	1.833383\\
0.709571	1.820851\\
0.709671	1.810468\\
0.709771	1.797882\\
0.709871	1.784792\\
0.709971	1.77157\\
0.710071	1.759229\\
0.710171	1.743442\\
0.710271	1.725242\\
0.710371	1.709391\\
0.710471	1.691114\\
0.7105711	1.672989\\
0.7106711	1.655943\\
0.7107711	1.636986\\
0.7108711	1.616419\\
0.7109711	1.597844\\
0.7110711	1.579676\\
0.7111711	1.558445\\
0.7112711	1.538704\\
0.7113711	1.515458\\
0.7114711	1.49476\\
0.7115712	1.469979\\
0.7116712	1.447652\\
0.7117712	1.424278\\
0.7118712	1.399379\\
0.7119712	1.375101\\
0.7120712	1.349684\\
0.7121712	1.324861\\
0.7122712	1.298473\\
0.7123712	1.274667\\
0.7124712	1.248028\\
0.7125713	1.221044\\
0.7126713	1.19434\\
0.7127713	1.162544\\
0.7128713	1.133309\\
0.7129713	1.105609\\
0.7130713	1.077622\\
0.7131713	1.046073\\
0.7132713	1.017082\\
0.7133713	0.9873728\\
0.7134713	0.9543833\\
0.7135714	0.9249503\\
0.7136714	0.8903955\\
0.7137714	0.8586089\\
0.7138714	0.8263054\\
0.7139714	0.795427\\
0.7140714	0.7618266\\
0.7141714	0.7295451\\
0.7142714	0.6979791\\
0.7143714	0.6645662\\
0.7144714	0.6289534\\
0.7145715	0.5937764\\
0.7146715	0.5618192\\
0.7147715	0.525251\\
0.7148715	0.4888999\\
0.7149715	0.4546392\\
0.7150715	0.4196565\\
0.7151715	0.3845942\\
0.7152715	0.348292\\
0.7153715	0.3139927\\
0.7154715	0.2792467\\
0.7155716	0.2456518\\
0.7156716	0.2094583\\
0.7157716	0.1732308\\
0.7158716	0.1376783\\
0.7159716	0.1002162\\
0.7160716	0.06217393\\
0.7161716	0.02477132\\
0.7162716	-0.01030849\\
0.7163716	-0.04899473\\
0.7164716	-0.08620704\\
0.7165717	-0.1227865\\
0.7166717	-0.1569022\\
0.7167717	-0.1938565\\
0.7168717	-0.2297956\\
0.7169717	-0.2684034\\
0.7170717	-0.3062014\\
0.7171717	-0.3412557\\
0.7172717	-0.3764469\\
0.7173717	-0.4118683\\
0.7174717	-0.4504596\\
0.7175718	-0.4839909\\
0.7176718	-0.5192101\\
0.7177718	-0.554055\\
0.7178718	-0.589946\\
0.7179718	-0.6263283\\
0.7180718	-0.6634458\\
0.7181718	-0.6993897\\
0.7182718	-0.7340372\\
0.7183718	-0.7709273\\
0.7184718	-0.8031554\\
0.7185719	-0.8368948\\
0.7186719	-0.8703826\\
0.7187719	-0.9057777\\
0.7188719	-0.9411451\\
0.7189719	-0.9739626\\
0.7190719	-1.007838\\
0.7191719	-1.040473\\
0.7192719	-1.071467\\
0.7193719	-1.101222\\
0.7194719	-1.132546\\
0.719572	-1.164508\\
0.719672	-1.196356\\
0.719772	-1.225897\\
0.719872	-1.255125\\
0.719972	-1.287416\\
0.720072	-1.317223\\
0.720172	-1.344936\\
0.720272	-1.373301\\
0.720372	-1.402633\\
0.720472	-1.432914\\
0.7205721	-1.45829\\
0.7206721	-1.485779\\
0.7207721	-1.513412\\
0.7208721	-1.537573\\
0.7209721	-1.564023\\
0.7210721	-1.587712\\
0.7211721	-1.612507\\
0.7212721	-1.637829\\
0.7213721	-1.661385\\
0.7214721	-1.685724\\
0.7215722	-1.707149\\
0.7216722	-1.73119\\
0.7217722	-1.752715\\
0.7218722	-1.772249\\
0.7219722	-1.793496\\
0.7220722	-1.812404\\
0.7221722	-1.831235\\
0.7222722	-1.847749\\
0.7223722	-1.86475\\
0.7224722	-1.88134\\
0.7225723	-1.898293\\
0.7226723	-1.914008\\
0.7227723	-1.927816\\
0.7228723	-1.942299\\
0.7229723	-1.957981\\
0.7230723	-1.972431\\
0.7231723	-1.987877\\
0.7232723	-2.001581\\
0.7233723	-2.01328\\
0.7234723	-2.023593\\
0.7235724	-2.031637\\
0.7236724	-2.040668\\
0.7237724	-2.048322\\
0.7238724	-2.057408\\
0.7239724	-2.064462\\
0.7240724	-2.070191\\
0.7241724	-2.077796\\
0.7242724	-2.082833\\
0.7243724	-2.086828\\
0.7244724	-2.091237\\
0.7245725	-2.095709\\
0.7246725	-2.098294\\
0.7247725	-2.101481\\
0.7248725	-2.10396\\
0.7249725	-2.103709\\
0.7250725	-2.103045\\
0.7251725	-2.100384\\
0.7252725	-2.098368\\
0.7253725	-2.095122\\
0.7254725	-2.092598\\
0.7255726	-2.088713\\
0.7256726	-2.083987\\
0.7257726	-2.079801\\
0.7258726	-2.07249\\
0.7259726	-2.066517\\
0.7260726	-2.059289\\
0.7261726	-2.051823\\
0.7262726	-2.044696\\
0.7263726	-2.034198\\
0.7264726	-2.024651\\
0.7265727	-2.012123\\
0.7266727	-1.999132\\
0.7267727	-1.987677\\
0.7268727	-1.974619\\
0.7269727	-1.962623\\
0.7270727	-1.947594\\
0.7271727	-1.932785\\
0.7272727	-1.917555\\
0.7273727	-1.900971\\
0.7274727	-1.881854\\
0.7275728	-1.863499\\
0.7276728	-1.846177\\
0.7277728	-1.829002\\
0.7278728	-1.808613\\
0.7279728	-1.787182\\
0.7280728	-1.766766\\
0.7281728	-1.745656\\
0.7282728	-1.725737\\
0.7283728	-1.69954\\
0.7284728	-1.677333\\
0.7285729	-1.653842\\
0.7286729	-1.629372\\
0.7287729	-1.603939\\
0.7288729	-1.578109\\
0.7289729	-1.555307\\
0.7290729	-1.528116\\
0.7291729	-1.499411\\
0.7292729	-1.472717\\
0.7293729	-1.443015\\
0.7294729	-1.413397\\
0.729573	-1.385177\\
0.729673	-1.354581\\
0.729773	-1.323003\\
0.729873	-1.291571\\
0.729973	-1.260016\\
0.730073	-1.226285\\
0.730173	-1.192694\\
0.730273	-1.160526\\
0.730373	-1.125837\\
0.730473	-1.091611\\
0.7305731	-1.060442\\
0.7306731	-1.028847\\
0.7307731	-0.994521\\
0.7308731	-0.9594558\\
0.7309731	-0.9251182\\
0.7310731	-0.8879355\\
0.7311731	-0.8514596\\
0.7312731	-0.8149601\\
0.7313731	-0.777116\\
0.7314731	-0.7397303\\
0.7315732	-0.7018243\\
0.7316732	-0.6638634\\
0.7317732	-0.6261426\\
0.7318732	-0.5889019\\
0.7319732	-0.5510667\\
0.7320732	-0.509549\\
0.7321732	-0.472026\\
0.7322732	-0.4344846\\
0.7323732	-0.3937397\\
0.7324732	-0.3529552\\
0.7325733	-0.3115046\\
0.7326733	-0.2731608\\
0.7327733	-0.2363025\\
0.7328733	-0.1959233\\
0.7329733	-0.1537957\\
0.7330733	-0.112731\\
0.7331733	-0.0735925\\
0.7332733	-0.03375756\\
0.7333733	0.007748723\\
0.7334733	0.04906312\\
0.7335734	0.08918361\\
0.7336734	0.1296297\\
0.7337734	0.1697146\\
0.7338734	0.2099757\\
0.7339734	0.2481415\\
0.7340734	0.2885647\\
0.7341734	0.3281915\\
0.7342734	0.3705921\\
0.7343734	0.4124438\\
0.7344734	0.4512187\\
0.7345735	0.4910608\\
0.7346735	0.5306837\\
0.7347735	0.5709641\\
0.7348735	0.610122\\
0.7349735	0.6491772\\
0.7350735	0.6871385\\
0.7351735	0.7265731\\
0.7352735	0.765535\\
0.7353735	0.8044071\\
0.7354735	0.8397897\\
0.7355736	0.8786557\\
0.7356736	0.9172892\\
0.7357736	0.9544933\\
0.7358736	0.9906286\\
0.7359736	1.02623\\
0.7360736	1.06354\\
0.7361736	1.099298\\
0.7362736	1.132703\\
0.7363736	1.169139\\
0.7364736	1.20402\\
0.7365737	1.237346\\
0.7366737	1.27259\\
0.7367737	1.307523\\
0.7368737	1.342272\\
0.7369737	1.37603\\
0.7370737	1.408472\\
0.7371737	1.439105\\
0.7372737	1.468332\\
0.7373737	1.499026\\
0.7374737	1.529249\\
0.7375738	1.558198\\
0.7376738	1.586916\\
0.7377738	1.615953\\
0.7378738	1.646185\\
0.7379738	1.672922\\
0.7380738	1.700855\\
0.7381738	1.728368\\
0.7382738	1.753292\\
0.7383738	1.778816\\
0.7384738	1.802774\\
0.7385739	1.825197\\
0.7386739	1.850427\\
0.7387739	1.871529\\
0.7388739	1.894362\\
0.7389739	1.914598\\
0.7390739	1.933431\\
0.7391739	1.952708\\
0.7392739	1.972588\\
0.7393739	1.991554\\
0.7394739	2.007842\\
0.739574	2.026199\\
0.739674	2.044163\\
0.739774	2.060975\\
0.739874	2.076595\\
0.739974	2.092818\\
0.740074	2.105487\\
0.740174	2.120517\\
0.740274	2.133165\\
0.740374	2.144782\\
0.740474	2.153573\\
0.7405741	2.160245\\
0.7406741	2.168276\\
0.7407741	2.175856\\
0.7408741	2.185389\\
0.7409741	2.192898\\
0.7410741	2.198527\\
0.7411741	2.20536\\
0.7412741	2.20743\\
0.7413741	2.210592\\
0.7414741	2.214853\\
0.7415742	2.216351\\
0.7416742	2.218317\\
0.7417742	2.217245\\
0.7418742	2.216682\\
0.7419742	2.215345\\
0.7420742	2.211231\\
0.7421742	2.20999\\
0.7422742	2.206662\\
0.7423742	2.202428\\
0.7424742	2.2003\\
0.7425743	2.193711\\
0.7426743	2.186405\\
0.7427743	2.179541\\
0.7428743	2.169262\\
0.7429743	2.160965\\
0.7430743	2.149699\\
0.7431743	2.13938\\
0.7432743	2.128085\\
0.7433743	2.115086\\
0.7434743	2.101415\\
0.7435744	2.084071\\
0.7436744	2.070474\\
0.7437744	2.055747\\
0.7438744	2.04059\\
0.7439744	2.023365\\
0.7440744	2.006637\\
0.7441744	1.990251\\
0.7442744	1.97221\\
0.7443744	1.951777\\
0.7444744	1.931187\\
0.7445745	1.90966\\
0.7446745	1.886663\\
0.7447745	1.861444\\
0.7448745	1.838626\\
0.7449745	1.815521\\
0.7450745	1.79169\\
0.7451745	1.766355\\
0.7452745	1.739299\\
0.7453745	1.711536\\
0.7454745	1.684718\\
0.7455746	1.655259\\
0.7456746	1.626352\\
0.7457746	1.597255\\
0.7458746	1.568309\\
0.7459746	1.539155\\
0.7460746	1.506193\\
0.7461746	1.476628\\
0.7462746	1.442267\\
0.7463746	1.40853\\
0.7464746	1.376043\\
0.7465747	1.341299\\
0.7466747	1.307594\\
0.7467747	1.272203\\
0.7468747	1.237056\\
0.7469747	1.202108\\
0.7470747	1.163949\\
0.7471747	1.128756\\
0.7472747	1.092486\\
0.7473747	1.055948\\
0.7474747	1.018102\\
0.7475748	0.978818\\
0.7476748	0.9405092\\
0.7477748	0.9016941\\
0.7478748	0.8646955\\
0.7479748	0.8243613\\
0.7480748	0.7808015\\
0.7481748	0.7393218\\
0.7482748	0.6988022\\
0.7483748	0.658684\\
0.7484748	0.6159852\\
0.7485749	0.5728115\\
0.7486749	0.5321459\\
0.7487749	0.4929758\\
0.7488749	0.4490914\\
0.7489749	0.4057668\\
0.7490749	0.3639601\\
0.7491749	0.321481\\
0.7492749	0.2784156\\
0.7493749	0.2343989\\
0.7494749	0.1915899\\
0.749575	0.1496781\\
0.749675	0.1046054\\
0.749775	0.05917672\\
0.749875	0.01675553\\
0.749975	-0.02678427\\
0.750075	-0.07037597\\
0.750175	-0.1117598\\
0.750275	-0.1535243\\
0.750375	-0.1972934\\
0.750475	-0.2408675\\
0.7505751	-0.2853478\\
0.7506751	-0.3268708\\
0.7507751	-0.3708057\\
0.7508751	-0.4168954\\
0.7509751	-0.4586895\\
0.7510751	-0.4989738\\
0.7511751	-0.5411339\\
0.7512751	-0.5850176\\
0.7513751	-0.6283714\\
0.7514751	-0.670164\\
0.7515752	-0.7114062\\
0.7516752	-0.7549476\\
0.7517752	-0.7963023\\
0.7518752	-0.836741\\
0.7519752	-0.8779207\\
0.7520752	-0.9159279\\
0.7521752	-0.9553275\\
0.7522752	-0.9958393\\
0.7523752	-1.036715\\
0.7524752	-1.075484\\
0.7525753	-1.110947\\
0.7526753	-1.149328\\
0.7527753	-1.186033\\
0.7528753	-1.223775\\
0.7529753	-1.262533\\
0.7530753	-1.29976\\
0.7531753	-1.335273\\
0.7532753	-1.37088\\
0.7533753	-1.407587\\
0.7534753	-1.444498\\
0.7535754	-1.479032\\
0.7536754	-1.509729\\
0.7537754	-1.543011\\
0.7538754	-1.576161\\
0.7539754	-1.606468\\
0.7540754	-1.636824\\
0.7541754	-1.667482\\
0.7542754	-1.696306\\
0.7543754	-1.725777\\
0.7544754	-1.752485\\
0.7545755	-1.781803\\
0.7546755	-1.808427\\
0.7547755	-1.834023\\
0.7548755	-1.860215\\
0.7549755	-1.886493\\
0.7550755	-1.910202\\
0.7551755	-1.935232\\
0.7552755	-1.960408\\
0.7553755	-1.982792\\
0.7554755	-2.004441\\
0.7555756	-2.023612\\
0.7556756	-2.044773\\
0.7557756	-2.064381\\
0.7558756	-2.082327\\
0.7559756	-2.098337\\
0.7560756	-2.11676\\
0.7561756	-2.130346\\
0.7562756	-2.148102\\
0.7563756	-2.165461\\
0.7564756	-2.180236\\
0.7565757	-2.194907\\
0.7566757	-2.206085\\
0.7567757	-2.216093\\
0.7568757	-2.224353\\
0.7569757	-2.232272\\
0.7570757	-2.240448\\
0.7571757	-2.248492\\
0.7572757	-2.255384\\
0.7573757	-2.263975\\
0.7574757	-2.268375\\
0.7575758	-2.272923\\
0.7576758	-2.276543\\
0.7577758	-2.278545\\
0.7578758	-2.281652\\
0.7579758	-2.282633\\
0.7580758	-2.282753\\
0.7581758	-2.281865\\
0.7582758	-2.279512\\
0.7583758	-2.277393\\
0.7584758	-2.272994\\
0.7585759	-2.26813\\
0.7586759	-2.262646\\
0.7587759	-2.256562\\
0.7588759	-2.250754\\
0.7589759	-2.243502\\
0.7590759	-2.23559\\
0.7591759	-2.224237\\
0.7592759	-2.214508\\
0.7593759	-2.204771\\
0.7594759	-2.193755\\
0.759576	-2.180756\\
0.759676	-2.167129\\
0.759776	-2.152764\\
0.759876	-2.135895\\
0.759976	-2.118253\\
0.760076	-2.099483\\
0.760176	-2.082669\\
0.760276	-2.064098\\
0.760376	-2.042503\\
0.760476	-2.022985\\
0.7605761	-2.001748\\
0.7606761	-1.980387\\
0.7607761	-1.959722\\
0.7608761	-1.937016\\
0.7609761	-1.91444\\
0.7610761	-1.8897\\
0.7611761	-1.861423\\
0.7612761	-1.835099\\
0.7613761	-1.808228\\
0.7614761	-1.778452\\
0.7615762	-1.749477\\
0.7616762	-1.719225\\
0.7617762	-1.689404\\
0.7618762	-1.660479\\
0.7619762	-1.627595\\
0.7620762	-1.59745\\
0.7621762	-1.565742\\
0.7622762	-1.530654\\
0.7623762	-1.497443\\
0.7624762	-1.464941\\
0.7625763	-1.428287\\
0.7626763	-1.390791\\
0.7627763	-1.354604\\
0.7628763	-1.318299\\
0.7629763	-1.281809\\
0.7630763	-1.244916\\
0.7631763	-1.207384\\
0.7632763	-1.169075\\
0.7633763	-1.129682\\
0.7634763	-1.090108\\
0.7635764	-1.049963\\
0.7636764	-1.008102\\
0.7637764	-0.9678612\\
0.7638764	-0.9289741\\
0.7639764	-0.8881276\\
0.7640764	-0.8478122\\
0.7641764	-0.8043069\\
0.7642764	-0.7598941\\
0.7643764	-0.7178374\\
0.7644764	-0.6738158\\
0.7645765	-0.6284432\\
0.7646765	-0.5854407\\
0.7647765	-0.5402878\\
0.7648765	-0.4942082\\
0.7649765	-0.44999\\
0.7650765	-0.404661\\
0.7651765	-0.3613628\\
0.7652765	-0.3160667\\
0.7653765	-0.2697508\\
0.7654765	-0.2244567\\
0.7655766	-0.1819978\\
0.7656766	-0.1355676\\
0.7657766	-0.09188825\\
0.7658766	-0.04711429\\
0.7659766	-0.002459149\\
0.7660766	0.04316331\\
0.7661766	0.08884467\\
0.7662766	0.1353978\\
0.7663766	0.1801625\\
0.7664766	0.2267082\\
0.7665767	0.2729415\\
0.7666767	0.3183943\\
0.7667767	0.3636718\\
0.7668767	0.4086104\\
0.7669767	0.4551342\\
0.7670767	0.4979944\\
0.7671767	0.5419033\\
0.7672767	0.5855742\\
0.7673767	0.6307611\\
0.7674767	0.6737918\\
0.7675768	0.7170755\\
0.7676768	0.7606744\\
0.7677768	0.8038847\\
0.7678768	0.8476394\\
0.7679768	0.89093\\
0.7680768	0.931372\\
0.7681768	0.9713099\\
0.7682768	1.013341\\
0.7683768	1.055711\\
0.7684768	1.094709\\
0.7685769	1.135459\\
0.7686769	1.176242\\
0.7687769	1.215722\\
0.7688769	1.254843\\
0.7689769	1.293482\\
0.7690769	1.329288\\
0.7691769	1.366869\\
0.7692769	1.402312\\
0.7693769	1.438712\\
0.7694769	1.474719\\
0.769577	1.508904\\
0.769677	1.541228\\
0.769777	1.576423\\
0.769877	1.610822\\
0.769977	1.643265\\
0.770077	1.675427\\
0.770177	1.705524\\
0.770277	1.737513\\
0.770377	1.766116\\
0.770477	1.795065\\
0.7705771	1.824574\\
0.7706771	1.851506\\
0.7707771	1.877516\\
0.7708771	1.902249\\
0.7709771	1.927701\\
0.7710771	1.951642\\
0.7711771	1.974871\\
0.7712771	1.99937\\
0.7713771	2.020843\\
0.7714771	2.042592\\
0.7715772	2.063476\\
0.7716772	2.084039\\
0.7717772	2.104102\\
0.7718772	2.122745\\
0.7719772	2.139857\\
0.7720772	2.155001\\
0.7721772	2.169935\\
0.7722772	2.182436\\
0.7723772	2.197617\\
0.7724772	2.211642\\
0.7725773	2.223267\\
0.7726773	2.236947\\
0.7727773	2.246832\\
0.7728773	2.255943\\
0.7729773	2.262829\\
0.7730773	2.270263\\
0.7731773	2.276817\\
0.7732773	2.28181\\
0.7733773	2.285768\\
0.7734773	2.289864\\
0.7735774	2.293292\\
0.7736774	2.294207\\
0.7737774	2.29308\\
0.7738774	2.291274\\
0.7739774	2.291227\\
0.7740774	2.288278\\
0.7741774	2.286148\\
0.7742774	2.28307\\
0.7743774	2.280489\\
0.7744774	2.276483\\
0.7745775	2.26955\\
0.7746775	2.261268\\
0.7747775	2.253535\\
0.7748775	2.244632\\
0.7749775	2.234373\\
0.7750775	2.223662\\
0.7751775	2.212627\\
0.7752775	2.198377\\
0.7753775	2.183495\\
0.7754775	2.166192\\
0.7755776	2.149337\\
0.7756776	2.136717\\
0.7757776	2.120327\\
0.7758776	2.101465\\
0.7759776	2.082708\\
0.7760776	2.062935\\
0.7761776	2.042369\\
0.7762776	2.020578\\
0.7763776	1.996426\\
0.7764776	1.973529\\
0.7765777	1.948667\\
0.7766777	1.921965\\
0.7767777	1.894797\\
0.7768777	1.869519\\
0.7769777	1.844051\\
0.7770777	1.815989\\
0.7771777	1.787078\\
0.7772777	1.759897\\
0.7773777	1.729778\\
0.7774777	1.699533\\
0.7775778	1.667202\\
0.7776778	1.634775\\
0.7777778	1.602108\\
0.7778778	1.568439\\
0.7779778	1.532286\\
0.7780778	1.496176\\
0.7781778	1.459742\\
0.7782778	1.421132\\
0.7783778	1.385615\\
0.7784778	1.348847\\
0.7785779	1.312497\\
0.7786779	1.274376\\
0.7787779	1.236263\\
0.7788779	1.19552\\
0.7789779	1.154611\\
0.7790779	1.117192\\
0.7791779	1.076854\\
0.7792779	1.034626\\
0.7793779	0.9941128\\
0.7794779	0.9513583\\
0.779578	0.9102234\\
0.779678	0.8678409\\
0.779778	0.8217751\\
0.779878	0.776069\\
0.779978	0.7320735\\
0.780078	0.6881732\\
0.780178	0.6421724\\
0.780278	0.5972339\\
0.780378	0.5531791\\
0.780478	0.5091589\\
0.7805781	0.4658719\\
0.7806781	0.4194588\\
0.7807781	0.3729492\\
0.7808781	0.3256679\\
0.7809781	0.2773003\\
0.7810781	0.2296007\\
0.7811781	0.1858153\\
0.7812781	0.1404154\\
0.7813781	0.0926553\\
0.7814781	0.04589417\\
0.7815782	-0.0005805175\\
0.7816782	-0.04669155\\
0.7817782	-0.0934121\\
0.7818782	-0.1421854\\
0.7819782	-0.1873016\\
0.7820782	-0.2332722\\
0.7821782	-0.2800365\\
0.7822782	-0.3269657\\
0.7823782	-0.3718569\\
0.7824782	-0.4155945\\
0.7825783	-0.4622371\\
0.7826783	-0.5067802\\
0.7827783	-0.5497631\\
0.7828783	-0.5945674\\
0.7829783	-0.6401298\\
0.7830783	-0.6863431\\
0.7831783	-0.7318888\\
0.7832783	-0.7775745\\
0.7833783	-0.8204314\\
0.7834783	-0.8637431\\
0.7835784	-0.9071853\\
0.7836784	-0.9494032\\
0.7837784	-0.9909246\\
0.7838784	-1.033891\\
0.7839784	-1.075621\\
0.7840784	-1.115402\\
0.7841784	-1.155475\\
0.7842784	-1.195456\\
0.7843784	-1.235692\\
0.7844784	-1.273818\\
0.7845785	-1.312599\\
0.7846785	-1.348502\\
0.7847785	-1.386161\\
0.7848785	-1.423801\\
0.7849785	-1.459302\\
0.7850785	-1.494331\\
0.7851785	-1.527253\\
0.7852785	-1.56076\\
0.7853785	-1.595064\\
0.7854785	-1.628866\\
0.7855786	-1.660359\\
0.7856786	-1.692521\\
0.7857786	-1.724467\\
0.7858786	-1.753846\\
0.7859786	-1.783001\\
0.7860786	-1.809234\\
0.7861786	-1.837879\\
0.7862786	-1.865111\\
0.7863786	-1.889667\\
0.7864786	-1.915162\\
0.7865787	-1.940889\\
0.7866787	-1.965103\\
0.7867787	-1.988365\\
0.7868787	-2.008649\\
0.7869787	-2.028708\\
0.7870787	-2.047704\\
0.7871787	-2.067828\\
0.7872787	-2.087632\\
0.7873787	-2.102652\\
0.7874787	-2.118827\\
0.7875788	-2.136632\\
0.7876788	-2.151125\\
0.7877788	-2.165132\\
0.7878788	-2.17798\\
0.7879788	-2.189393\\
0.7880788	-2.200544\\
0.7881788	-2.212073\\
0.7882788	-2.222268\\
0.7883788	-2.231781\\
0.7884788	-2.238851\\
0.7885789	-2.245132\\
0.7886789	-2.250956\\
0.7887789	-2.254002\\
0.7888789	-2.257151\\
0.7889789	-2.257684\\
0.7890789	-2.25971\\
0.7891789	-2.260267\\
0.7892789	-2.258362\\
0.7893789	-2.254491\\
0.7894789	-2.254253\\
0.789579	-2.252379\\
0.789679	-2.247694\\
0.789779	-2.241622\\
0.789879	-2.238024\\
0.789979	-2.229809\\
0.790079	-2.220836\\
0.790179	-2.20931\\
0.790279	-2.198721\\
0.790379	-2.18685\\
0.790479	-2.176982\\
0.7905791	-2.164588\\
0.7906791	-2.149804\\
0.7907791	-2.139862\\
0.7908791	-2.123713\\
0.7909791	-2.104442\\
0.7910791	-2.086542\\
0.7911791	-2.070637\\
0.7912791	-2.053177\\
0.7913791	-2.031745\\
0.7914791	-2.009632\\
0.7915792	-1.98645\\
0.7916792	-1.963762\\
0.7917792	-1.940127\\
0.7918792	-1.914674\\
0.7919792	-1.887894\\
0.7920792	-1.863024\\
0.7921792	-1.836835\\
0.7922792	-1.807706\\
0.7923792	-1.78004\\
0.7924792	-1.752085\\
0.7925793	-1.722495\\
0.7926793	-1.692463\\
0.7927793	-1.660668\\
0.7928793	-1.628387\\
0.7929793	-1.594679\\
0.7930793	-1.561103\\
0.7931793	-1.525125\\
0.7932793	-1.488508\\
0.7933793	-1.453088\\
0.7934793	-1.417617\\
0.7935794	-1.380904\\
0.7936794	-1.34302\\
0.7937794	-1.305795\\
0.7938794	-1.267277\\
0.7939794	-1.22908\\
0.7940794	-1.192088\\
0.7941794	-1.152798\\
0.7942794	-1.112962\\
0.7943794	-1.071363\\
0.7944794	-1.0302\\
0.7945795	-0.9850023\\
0.7946795	-0.9429935\\
0.7947795	-0.9000063\\
0.7948795	-0.8558181\\
0.7949795	-0.8117599\\
0.7950795	-0.7679442\\
0.7951795	-0.7223984\\
0.7952795	-0.6778728\\
0.7953795	-0.6331706\\
0.7954795	-0.5890125\\
0.7955796	-0.5450418\\
0.7956796	-0.5012033\\
0.7957796	-0.4572131\\
0.7958796	-0.4111753\\
0.7959796	-0.3634484\\
0.7960796	-0.3178387\\
0.7961796	-0.270953\\
0.7962796	-0.2232978\\
0.7963796	-0.1750663\\
0.7964796	-0.1296001\\
0.7965797	-0.08247165\\
0.7966797	-0.03738833\\
0.7967797	0.009852623\\
0.7968797	0.05667156\\
0.7969797	0.1055151\\
0.7970797	0.1517483\\
0.7971797	0.1986843\\
0.7972797	0.2446412\\
0.7973797	0.29009\\
0.7974797	0.3387757\\
0.7975798	0.3854441\\
0.7976798	0.4287559\\
0.7977798	0.4739476\\
0.7978798	0.5171808\\
0.7979798	0.5623335\\
0.7980798	0.6082143\\
0.7981798	0.6531962\\
0.7982798	0.697174\\
0.7983798	0.7407747\\
0.7984798	0.7824326\\
0.7985799	0.8265744\\
0.7986799	0.8700894\\
0.7987799	0.911912\\
0.7988799	0.9540762\\
0.7989799	0.9948053\\
0.7990799	1.03658\\
0.7991799	1.078041\\
0.7992799	1.118866\\
0.7993799	1.15762\\
0.7994799	1.19766\\
0.79958	1.236776\\
0.79968	1.275\\
0.79978	1.312645\\
0.79988	1.350342\\
0.79998	1.386437\\
0.80008	1.422565\\
};
\addplot [color=mycolor2,solid]
  table[row sep=crcr]{%
0.80008	1.422565\\
0.80018	1.457125\\
0.80028	1.490887\\
0.80038	1.525583\\
0.80048	1.558491\\
0.8005801	1.588135\\
0.8006801	1.62061\\
0.8007801	1.650073\\
0.8008801	1.681307\\
0.8009801	1.7099\\
0.8010801	1.738541\\
0.8011801	1.764826\\
0.8012801	1.792308\\
0.8013801	1.818964\\
0.8014801	1.844674\\
0.8015802	1.865536\\
0.8016802	1.890617\\
0.8017802	1.916584\\
0.8018802	1.940284\\
0.8019802	1.95943\\
0.8020802	1.978846\\
0.8021802	1.996537\\
0.8022802	2.015018\\
0.8023802	2.032078\\
0.8024802	2.048642\\
0.8025803	2.061919\\
0.8026803	2.079059\\
0.8027803	2.093817\\
0.8028803	2.10616\\
0.8029803	2.11825\\
0.8030803	2.128349\\
0.8031803	2.139084\\
0.8032803	2.147328\\
0.8033803	2.154749\\
0.8034803	2.160072\\
0.8035804	2.167248\\
0.8036804	2.172585\\
0.8037804	2.177718\\
0.8038804	2.180422\\
0.8039804	2.180793\\
0.8040804	2.182085\\
0.8041804	2.184102\\
0.8042804	2.183113\\
0.8043804	2.182051\\
0.8044804	2.176879\\
0.8045805	2.173274\\
0.8046805	2.166421\\
0.8047805	2.161187\\
0.8048805	2.154388\\
0.8049805	2.145183\\
0.8050805	2.138796\\
0.8051805	2.131552\\
0.8052805	2.120669\\
0.8053805	2.107725\\
0.8054805	2.094827\\
0.8055806	2.080204\\
0.8056806	2.062729\\
0.8057806	2.046656\\
0.8058806	2.03155\\
0.8059806	2.014157\\
0.8060806	1.997299\\
0.8061806	1.980052\\
0.8062806	1.959738\\
0.8063806	1.938542\\
0.8064806	1.916378\\
0.8065807	1.891654\\
0.8066807	1.868253\\
0.8067807	1.84392\\
0.8068807	1.821168\\
0.8069807	1.793294\\
0.8070807	1.766819\\
0.8071807	1.739894\\
0.8072807	1.711626\\
0.8073807	1.684836\\
0.8074807	1.653671\\
0.8075808	1.625149\\
0.8076808	1.594806\\
0.8077808	1.563363\\
0.8078808	1.528017\\
0.8079808	1.495668\\
0.8080808	1.46188\\
0.8081808	1.426863\\
0.8082808	1.38798\\
0.8083808	1.351368\\
0.8084808	1.313675\\
0.8085809	1.273074\\
0.8086809	1.236145\\
0.8087809	1.19947\\
0.8088809	1.160705\\
0.8089809	1.121079\\
0.8090809	1.083226\\
0.8091809	1.044116\\
0.8092809	1.004462\\
0.8093809	0.9631974\\
0.8094809	0.9223603\\
0.809581	0.8797903\\
0.809681	0.8381115\\
0.809781	0.7960772\\
0.809881	0.7503863\\
0.809981	0.7059355\\
0.810081	0.6616712\\
0.810181	0.6167766\\
0.810281	0.5719563\\
0.810381	0.5288271\\
0.810481	0.4846032\\
0.8105811	0.4403304\\
0.8106811	0.3958084\\
0.8107811	0.3514905\\
0.8108811	0.3057063\\
0.8109811	0.2605968\\
0.8110811	0.2121018\\
0.8111811	0.1667129\\
0.8112811	0.1216422\\
0.8113811	0.07694014\\
0.8114811	0.02993849\\
0.8115812	-0.01882612\\
0.8116812	-0.0631793\\
0.8117812	-0.1087812\\
0.8118812	-0.1526573\\
0.8119812	-0.1992596\\
0.8120812	-0.2432543\\
0.8121812	-0.2889729\\
0.8122812	-0.3341881\\
0.8123812	-0.3806677\\
0.8124812	-0.4247892\\
0.8125813	-0.4675192\\
0.8126813	-0.511997\\
0.8127813	-0.5561387\\
0.8128813	-0.6010222\\
0.8129813	-0.6443158\\
0.8130813	-0.6892862\\
0.8131813	-0.7330663\\
0.8132813	-0.7752796\\
0.8133813	-0.8161903\\
0.8134813	-0.8581515\\
0.8135814	-0.9000977\\
0.8136814	-0.9393388\\
0.8137814	-0.9777948\\
0.8138814	-1.015525\\
0.8139814	-1.056545\\
0.8140814	-1.094683\\
0.8141814	-1.134061\\
0.8142814	-1.17394\\
0.8143814	-1.210667\\
0.8144814	-1.24579\\
0.8145815	-1.280176\\
0.8146815	-1.315182\\
0.8147815	-1.35019\\
0.8148815	-1.383525\\
0.8149815	-1.417098\\
0.8150815	-1.452201\\
0.8151815	-1.484145\\
0.8152815	-1.514582\\
0.8153815	-1.544107\\
0.8154815	-1.573874\\
0.8155816	-1.604824\\
0.8156816	-1.632677\\
0.8157816	-1.661462\\
0.8158816	-1.688366\\
0.8159816	-1.712766\\
0.8160816	-1.736865\\
0.8161816	-1.761876\\
0.8162816	-1.785787\\
0.8163816	-1.807842\\
0.8164816	-1.829786\\
0.8165817	-1.850652\\
0.8166817	-1.871453\\
0.8167817	-1.888917\\
0.8168817	-1.907511\\
0.8169817	-1.924892\\
0.8170817	-1.940777\\
0.8171817	-1.955003\\
0.8172817	-1.969525\\
0.8173817	-1.98342\\
0.8174817	-1.994115\\
0.8175818	-2.003834\\
0.8176818	-2.014054\\
0.8177818	-2.024923\\
0.8178818	-2.032737\\
0.8179818	-2.04091\\
0.8180818	-2.047983\\
0.8181818	-2.055138\\
0.8182818	-2.05938\\
0.8183818	-2.064749\\
0.8184818	-2.068807\\
0.8185819	-2.068171\\
0.8186819	-2.069806\\
0.8187819	-2.070079\\
0.8188819	-2.068352\\
0.8189819	-2.066258\\
0.8190819	-2.063611\\
0.8191819	-2.060616\\
0.8192819	-2.056579\\
0.8193819	-2.049959\\
0.8194819	-2.040268\\
0.819582	-2.031867\\
0.819682	-2.023383\\
0.819782	-2.013115\\
0.819882	-2.00192\\
0.819982	-1.99038\\
0.820082	-1.977837\\
0.820182	-1.962245\\
0.820282	-1.949635\\
0.820382	-1.934762\\
0.820482	-1.917941\\
0.8205821	-1.899243\\
0.8206821	-1.882397\\
0.8207821	-1.86573\\
0.8208821	-1.843851\\
0.8209821	-1.823934\\
0.8210821	-1.801165\\
0.8211821	-1.779744\\
0.8212821	-1.755476\\
0.8213821	-1.731108\\
0.8214821	-1.707545\\
0.8215822	-1.68032\\
0.8216822	-1.65318\\
0.8217822	-1.623606\\
0.8218822	-1.595477\\
0.8219822	-1.565837\\
0.8220822	-1.534425\\
0.8221822	-1.506125\\
0.8222822	-1.475021\\
0.8223822	-1.444509\\
0.8224822	-1.413134\\
0.8225823	-1.382502\\
0.8226823	-1.346853\\
0.8227823	-1.314042\\
0.8228823	-1.277579\\
0.8229823	-1.242721\\
0.8230823	-1.2078\\
0.8231823	-1.17101\\
0.8232823	-1.134998\\
0.8233823	-1.095462\\
0.8234823	-1.057524\\
0.8235824	-1.017625\\
0.8236824	-0.980109\\
0.8237824	-0.9384968\\
0.8238824	-0.8964334\\
0.8239824	-0.8585996\\
0.8240824	-0.8175758\\
0.8241824	-0.7763945\\
0.8242824	-0.7339398\\
0.8243824	-0.693009\\
0.8244824	-0.6525707\\
0.8245825	-0.6087193\\
0.8246825	-0.5674026\\
0.8247825	-0.5243506\\
0.8248825	-0.479778\\
0.8249825	-0.435184\\
0.8250825	-0.3905363\\
0.8251825	-0.3476039\\
0.8252825	-0.3038877\\
0.8253825	-0.2586923\\
0.8254825	-0.2150584\\
0.8255826	-0.1696096\\
0.8256826	-0.127195\\
0.8257826	-0.08541899\\
0.8258826	-0.04285619\\
0.8259826	-0.002018526\\
0.8260826	0.04307589\\
0.8261826	0.0865712\\
0.8262826	0.1307359\\
0.8263826	0.1763963\\
0.8264826	0.2225662\\
0.8265827	0.2670852\\
0.8266827	0.3113816\\
0.8267827	0.3546565\\
0.8268827	0.3954077\\
0.8269827	0.4386819\\
0.8270827	0.4792884\\
0.8271827	0.5235243\\
0.8272827	0.5653237\\
0.8273827	0.6049344\\
0.8274827	0.6458032\\
0.8275828	0.686608\\
0.8276828	0.7307233\\
0.8277828	0.7705103\\
0.8278828	0.8119366\\
0.8279828	0.8502919\\
0.8280828	0.8892246\\
0.8281828	0.9276222\\
0.8282828	0.9640878\\
0.8283828	1.003838\\
0.8284828	1.040829\\
0.8285829	1.076695\\
0.8286829	1.111656\\
0.8287829	1.147355\\
0.8288829	1.182302\\
0.8289829	1.213156\\
0.8290829	1.245417\\
0.8291829	1.279756\\
0.8292829	1.312473\\
0.8293829	1.344239\\
0.8294829	1.376158\\
0.829583	1.406234\\
0.829683	1.437008\\
0.829783	1.464023\\
0.829883	1.493663\\
0.829983	1.520538\\
0.830083	1.546443\\
0.830183	1.57198\\
0.830283	1.595753\\
0.830383	1.619785\\
0.830483	1.640704\\
0.8305831	1.66424\\
0.8306831	1.685524\\
0.8307831	1.707505\\
0.8308831	1.726932\\
0.8309831	1.744147\\
0.8310831	1.762704\\
0.8311831	1.780249\\
0.8312831	1.795788\\
0.8313831	1.810938\\
0.8314831	1.824333\\
0.8315832	1.838829\\
0.8316832	1.852834\\
0.8317832	1.865649\\
0.8318832	1.877706\\
0.8319832	1.886963\\
0.8320832	1.89629\\
0.8321832	1.904661\\
0.8322832	1.911196\\
0.8323832	1.917289\\
0.8324832	1.921367\\
0.8325833	1.925621\\
0.8326833	1.927286\\
0.8327833	1.930236\\
0.8328833	1.932534\\
0.8329833	1.932114\\
0.8330833	1.928739\\
0.8331833	1.924819\\
0.8332833	1.922315\\
0.8333833	1.918533\\
0.8334833	1.913329\\
0.8335834	1.909286\\
0.8336834	1.905071\\
0.8337834	1.897042\\
0.8338834	1.89032\\
0.8339834	1.877641\\
0.8340834	1.869707\\
0.8341834	1.858495\\
0.8342834	1.846199\\
0.8343834	1.83317\\
0.8344834	1.819682\\
0.8345835	1.807573\\
0.8346835	1.791511\\
0.8347835	1.772616\\
0.8348835	1.754869\\
0.8349835	1.737099\\
0.8350835	1.718305\\
0.8351835	1.698533\\
0.8352835	1.67894\\
0.8353835	1.657507\\
0.8354835	1.634695\\
0.8355836	1.610416\\
0.8356836	1.584527\\
0.8357836	1.559324\\
0.8358836	1.535277\\
0.8359836	1.509213\\
0.8360836	1.481544\\
0.8361836	1.454993\\
0.8362836	1.427009\\
0.8363836	1.400313\\
0.8364836	1.368435\\
0.8365837	1.336498\\
0.8366837	1.305076\\
0.8367837	1.272412\\
0.8368837	1.238122\\
0.8369837	1.204755\\
0.8370837	1.174265\\
0.8371837	1.140061\\
0.8372837	1.104679\\
0.8373837	1.070013\\
0.8374837	1.034243\\
0.8375838	1.000385\\
0.8376838	0.9639424\\
0.8377838	0.9258256\\
0.8378838	0.8895001\\
0.8379838	0.850658\\
0.8380838	0.8116542\\
0.8381838	0.7742203\\
0.8382838	0.7338679\\
0.8383838	0.6950231\\
0.8384838	0.6546038\\
0.8385839	0.6135145\\
0.8386839	0.5753025\\
0.8387839	0.5324517\\
0.8388839	0.491537\\
0.8389839	0.4519891\\
0.8390839	0.4104314\\
0.8391839	0.3710338\\
0.8392839	0.3300314\\
0.8393839	0.2894577\\
0.8394839	0.2473644\\
0.839584	0.2062879\\
0.839684	0.1658891\\
0.839784	0.1211896\\
0.839884	0.07820487\\
0.839984	0.03715401\\
0.840084	-0.0035415\\
0.840184	-0.0452865\\
0.840284	-0.08904238\\
0.840384	-0.132411\\
0.840484	-0.1730296\\
0.8405841	-0.2153391\\
0.8406841	-0.2546125\\
0.8407841	-0.2966781\\
0.8408841	-0.3384096\\
0.8409841	-0.3771019\\
0.8410841	-0.4184522\\
0.8411841	-0.4582601\\
0.8412841	-0.4976604\\
0.8413841	-0.535901\\
0.8414841	-0.5749119\\
0.8415842	-0.6157031\\
0.8416842	-0.6565362\\
0.8417842	-0.6951079\\
0.8418842	-0.7308785\\
0.8419842	-0.7674691\\
0.8420842	-0.8052514\\
0.8421842	-0.8404988\\
0.8422842	-0.8732468\\
0.8423842	-0.9080332\\
0.8424842	-0.9433501\\
0.8425843	-0.9782609\\
0.8426843	-1.012142\\
0.8427843	-1.046566\\
0.8428843	-1.080063\\
0.8429843	-1.111707\\
0.8430843	-1.143179\\
0.8431843	-1.174859\\
0.8432843	-1.208779\\
0.8433843	-1.239287\\
0.8434843	-1.269137\\
0.8435844	-1.298561\\
0.8436844	-1.324884\\
0.8437844	-1.351747\\
0.8438844	-1.377132\\
0.8439844	-1.400662\\
0.8440844	-1.426021\\
0.8441844	-1.447648\\
0.8442844	-1.470517\\
0.8443844	-1.493905\\
0.8444844	-1.513568\\
0.8445845	-1.535126\\
0.8446845	-1.554154\\
0.8447845	-1.576188\\
0.8448845	-1.597852\\
0.8449845	-1.61415\\
0.8450845	-1.628855\\
0.8451845	-1.645127\\
0.8452845	-1.659689\\
0.8453845	-1.673266\\
0.8454845	-1.68524\\
0.8455846	-1.696711\\
0.8456846	-1.707053\\
0.8457846	-1.719485\\
0.8458846	-1.730204\\
0.8459846	-1.73895\\
0.8460846	-1.748695\\
0.8461846	-1.753743\\
0.8462846	-1.75941\\
0.8463846	-1.763045\\
0.8464846	-1.767163\\
0.8465847	-1.77042\\
0.8466847	-1.771055\\
0.8467847	-1.773332\\
0.8468847	-1.772897\\
0.8469847	-1.771475\\
0.8470847	-1.772622\\
0.8471847	-1.770175\\
0.8472847	-1.766893\\
0.8473847	-1.763632\\
0.8474847	-1.757628\\
0.8475848	-1.75414\\
0.8476848	-1.747157\\
0.8477848	-1.736432\\
0.8478848	-1.727578\\
0.8479848	-1.717951\\
0.8480848	-1.708078\\
0.8481848	-1.695325\\
0.8482848	-1.682285\\
0.8483848	-1.671047\\
0.8484848	-1.657918\\
0.8485849	-1.642401\\
0.8486849	-1.624841\\
0.8487849	-1.608721\\
0.8488849	-1.590753\\
0.8489849	-1.573449\\
0.8490849	-1.554355\\
0.8491849	-1.534475\\
0.8492849	-1.515548\\
0.8493849	-1.493386\\
0.8494849	-1.46859\\
0.849585	-1.445238\\
0.849685	-1.421705\\
0.849785	-1.399213\\
0.849885	-1.37435\\
0.849985	-1.346072\\
0.850085	-1.319324\\
0.850185	-1.292994\\
0.850285	-1.264837\\
0.850385	-1.236179\\
0.850485	-1.208856\\
0.8505851	-1.181253\\
0.8506851	-1.152959\\
0.8507851	-1.120307\\
0.8508851	-1.089205\\
0.8509851	-1.058681\\
0.8510851	-1.026299\\
0.8511851	-0.9927625\\
0.8512851	-0.960092\\
0.8513851	-0.9251878\\
0.8514851	-0.8911143\\
0.8515852	-0.8571623\\
0.8516852	-0.8212711\\
0.8517852	-0.7880079\\
0.8518852	-0.751422\\
0.8519852	-0.713472\\
0.8520852	-0.674091\\
0.8521852	-0.636463\\
0.8522852	-0.5984984\\
0.8523852	-0.5611611\\
0.8524852	-0.5248693\\
0.8525853	-0.4870576\\
0.8526853	-0.4530736\\
0.8527853	-0.4130152\\
0.8528853	-0.373036\\
0.8529853	-0.3353018\\
0.8530853	-0.2960487\\
0.8531853	-0.2586633\\
0.8532853	-0.2206225\\
0.8533853	-0.1801112\\
0.8534853	-0.1413515\\
0.8535854	-0.1023859\\
0.8536854	-0.06249691\\
0.8537854	-0.02591958\\
0.8538854	0.01331128\\
0.8539854	0.05546739\\
0.8540854	0.09283018\\
0.8541854	0.130948\\
0.8542854	0.1692879\\
0.8543854	0.2078568\\
0.8544854	0.2478174\\
0.8545855	0.2835013\\
0.8546855	0.3209576\\
0.8547855	0.3593545\\
0.8548855	0.3976102\\
0.8549855	0.435612\\
0.8550855	0.4733933\\
0.8551855	0.5103354\\
0.8552855	0.5469163\\
0.8553855	0.583618\\
0.8554855	0.6191596\\
0.8555856	0.6519664\\
0.8556856	0.6852686\\
0.8557856	0.7204672\\
0.8558856	0.754156\\
0.8559856	0.78618\\
0.8560856	0.8189079\\
0.8561856	0.8509804\\
0.8562856	0.8862781\\
0.8563856	0.918559\\
0.8564856	0.9510863\\
0.8565857	0.9809477\\
0.8566857	1.011511\\
0.8567857	1.042742\\
0.8568857	1.072105\\
0.8569857	1.098787\\
0.8570857	1.124522\\
0.8571857	1.151925\\
0.8572857	1.179111\\
0.8573857	1.203565\\
0.8574857	1.227671\\
0.8575858	1.250774\\
0.8576858	1.272958\\
0.8577858	1.296973\\
0.8578858	1.316087\\
0.8579858	1.337055\\
0.8580858	1.358081\\
0.8581858	1.378824\\
0.8582858	1.397262\\
0.8583858	1.417855\\
0.8584858	1.435203\\
0.8585859	1.451608\\
0.8586859	1.468777\\
0.8587859	1.482561\\
0.8588859	1.496829\\
0.8589859	1.51133\\
0.8590859	1.522449\\
0.8591859	1.532825\\
0.8592859	1.542464\\
0.8593859	1.552813\\
0.8594859	1.562932\\
0.859586	1.571182\\
0.859686	1.57939\\
0.859786	1.586508\\
0.859886	1.593751\\
0.859986	1.598198\\
0.860086	1.601594\\
0.860186	1.605466\\
0.860286	1.60834\\
0.860386	1.608791\\
0.860486	1.609972\\
0.8605861	1.607133\\
0.8606861	1.602618\\
0.8607861	1.600588\\
0.8608861	1.59751\\
0.8609861	1.593655\\
0.8610861	1.589748\\
0.8611861	1.584259\\
0.8612861	1.577108\\
0.8613861	1.570243\\
0.8614861	1.561641\\
0.8615862	1.549857\\
0.8616862	1.540502\\
0.8617862	1.531017\\
0.8618862	1.519954\\
0.8619862	1.509163\\
0.8620862	1.496396\\
0.8621862	1.480786\\
0.8622862	1.464322\\
0.8623862	1.449437\\
0.8624862	1.433842\\
0.8625863	1.416973\\
0.8626863	1.401385\\
0.8627863	1.382592\\
0.8628863	1.363467\\
0.8629863	1.344121\\
0.8630863	1.32313\\
0.8631863	1.300064\\
0.8632863	1.277097\\
0.8633863	1.252682\\
0.8634863	1.227661\\
0.8635864	1.204058\\
0.8636864	1.179646\\
0.8637864	1.154874\\
0.8638864	1.128929\\
0.8639864	1.104616\\
0.8640864	1.078124\\
0.8641864	1.050289\\
0.8642864	1.020608\\
0.8643864	0.9932636\\
0.8644864	0.9650403\\
0.8645865	0.935746\\
0.8646865	0.9047483\\
0.8647865	0.8750095\\
0.8648865	0.8429978\\
0.8649865	0.8093584\\
0.8650865	0.776904\\
0.8651865	0.7427233\\
0.8652865	0.7098336\\
0.8653865	0.6790066\\
0.8654865	0.6463046\\
0.8655866	0.6125185\\
0.8656866	0.5801051\\
0.8657866	0.5452069\\
0.8658866	0.5092453\\
0.8659866	0.4748579\\
0.8660866	0.4404398\\
0.8661866	0.4057592\\
0.8662866	0.370233\\
0.8663866	0.3338977\\
0.8664866	0.2998178\\
0.8665867	0.2638901\\
0.8666867	0.2303513\\
0.8667867	0.1953533\\
0.8668867	0.1592344\\
0.8669867	0.1257124\\
0.8670867	0.0900986\\
0.8671867	0.05279458\\
0.8672867	0.01617817\\
0.8673867	-0.02067052\\
0.8674867	-0.05778823\\
0.8675868	-0.09318442\\
0.8676868	-0.1261708\\
0.8677868	-0.1622909\\
0.8678868	-0.1976348\\
0.8679868	-0.2324612\\
0.8680868	-0.2673651\\
0.8681868	-0.3040513\\
0.8682868	-0.3382709\\
0.8683868	-0.3741679\\
0.8684868	-0.4102932\\
0.8685869	-0.4420184\\
0.8686869	-0.4741771\\
0.8687869	-0.5075088\\
0.8688869	-0.5408842\\
0.8689869	-0.5717116\\
0.8690869	-0.6034246\\
0.8691869	-0.6347678\\
0.8692869	-0.6645443\\
0.8693869	-0.6947816\\
0.8694869	-0.7254369\\
0.869587	-0.7565952\\
0.869687	-0.7847268\\
0.869787	-0.8129296\\
0.869887	-0.8408329\\
0.869987	-0.8694275\\
0.870087	-0.8988793\\
0.870187	-0.9254281\\
0.870287	-0.9500726\\
0.870387	-0.9763103\\
0.870487	-1.003674\\
0.8705871	-1.026525\\
0.8706871	-1.050389\\
0.8707871	-1.074026\\
0.8708871	-1.096064\\
0.8709871	-1.116401\\
0.8710871	-1.137918\\
0.8711871	-1.157559\\
0.8712871	-1.177935\\
0.8713871	-1.197316\\
0.8714871	-1.21624\\
0.8715872	-1.233343\\
0.8716872	-1.252979\\
0.8717872	-1.272004\\
0.8718872	-1.286741\\
0.8719872	-1.301938\\
0.8720872	-1.315589\\
0.8721872	-1.328942\\
0.8722872	-1.340332\\
0.8723872	-1.351607\\
0.8724872	-1.3611\\
0.8725873	-1.370224\\
0.8726873	-1.37721\\
0.8727873	-1.387473\\
0.8728873	-1.397222\\
0.8729873	-1.404418\\
0.8730873	-1.409411\\
0.8731873	-1.414742\\
0.8732873	-1.421131\\
0.8733873	-1.426217\\
0.8734873	-1.429727\\
0.8735874	-1.431454\\
0.8736874	-1.43339\\
0.8737874	-1.433176\\
0.8738874	-1.432672\\
0.8739874	-1.429021\\
0.8740874	-1.428188\\
0.8741874	-1.424275\\
0.8742874	-1.420195\\
0.8743874	-1.416127\\
0.8744874	-1.41024\\
0.8745875	-1.40477\\
0.8746875	-1.396963\\
0.8747875	-1.388124\\
0.8748875	-1.380722\\
0.8749875	-1.372687\\
0.8750875	-1.365665\\
0.8751875	-1.356329\\
0.8752875	-1.3452\\
0.8753875	-1.33353\\
0.8754875	-1.318527\\
0.8755876	-1.306151\\
0.8756876	-1.291373\\
0.8757876	-1.277105\\
0.8758876	-1.262447\\
0.8759876	-1.246189\\
0.8760876	-1.228123\\
0.8761876	-1.211593\\
0.8762876	-1.193586\\
0.8763876	-1.173028\\
0.8764876	-1.152435\\
0.8765877	-1.132773\\
0.8766877	-1.111642\\
0.8767877	-1.088383\\
0.8768877	-1.065186\\
0.8769877	-1.044655\\
0.8770877	-1.020058\\
0.8771877	-0.9938901\\
0.8772877	-0.9716415\\
0.8773877	-0.9472377\\
0.8774877	-0.9220604\\
0.8775878	-0.8963103\\
0.8776878	-0.870762\\
0.8777878	-0.8428134\\
0.8778878	-0.814444\\
0.8779878	-0.7890323\\
0.8780878	-0.763746\\
0.8781878	-0.7366711\\
0.8782878	-0.7079646\\
0.8783878	-0.678951\\
0.8784878	-0.6492066\\
0.8785879	-0.6193874\\
0.8786879	-0.5884757\\
0.8787879	-0.5560832\\
0.8788879	-0.5245168\\
0.8789879	-0.4938137\\
0.8790879	-0.4629226\\
0.8791879	-0.4334427\\
0.8792879	-0.4041339\\
0.8793879	-0.3731604\\
0.8794879	-0.3409931\\
0.879588	-0.3084175\\
0.879688	-0.2757876\\
0.879788	-0.2434428\\
0.879888	-0.2097449\\
0.879988	-0.1770686\\
0.880088	-0.1459431\\
0.880188	-0.1122731\\
0.880288	-0.08053405\\
0.880388	-0.0472989\\
0.880488	-0.01582008\\
0.8805881	0.01578841\\
0.8806881	0.0487368\\
0.8807881	0.07922211\\
0.8808881	0.1111614\\
0.8809881	0.1421228\\
0.8810881	0.1744611\\
0.8811881	0.2052139\\
0.8812881	0.2375877\\
0.8813881	0.2709957\\
0.8814881	0.3035021\\
0.8815882	0.3326097\\
0.8816882	0.3618331\\
0.8817882	0.3932631\\
0.8818882	0.4226355\\
0.8819882	0.4537252\\
0.8820882	0.482008\\
0.8821882	0.5106805\\
0.8822882	0.5387777\\
0.8823882	0.5649793\\
0.8824882	0.59276\\
0.8825883	0.6212184\\
0.8826883	0.6463133\\
0.8827883	0.6740684\\
0.8828883	0.7020705\\
0.8829883	0.7258391\\
0.8830883	0.7517458\\
0.8831883	0.7777223\\
0.8832883	0.8027038\\
0.8833883	0.8259845\\
0.8834883	0.8490816\\
0.8835884	0.8708884\\
0.8836884	0.8926044\\
0.8837884	0.914524\\
0.8838884	0.9356909\\
0.8839884	0.9563583\\
0.8840884	0.9778434\\
0.8841884	0.9958839\\
0.8842884	1.013958\\
0.8843884	1.033753\\
0.8844884	1.052753\\
0.8845885	1.067874\\
0.8846885	1.082911\\
0.8847885	1.101083\\
0.8848885	1.114767\\
0.8849885	1.12823\\
0.8850885	1.141276\\
0.8851885	1.15663\\
0.8852885	1.166388\\
0.8853885	1.176436\\
0.8854885	1.188544\\
0.8855886	1.198037\\
0.8856886	1.207665\\
0.8857886	1.215267\\
0.8858886	1.219944\\
0.8859886	1.22569\\
0.8860886	1.231013\\
0.8861886	1.235678\\
0.8862886	1.240294\\
0.8863886	1.245929\\
0.8864886	1.248364\\
0.8865887	1.251611\\
0.8866887	1.255912\\
0.8867887	1.257783\\
0.8868887	1.259648\\
0.8869887	1.260639\\
0.8870887	1.260368\\
0.8871887	1.258511\\
0.8872887	1.254263\\
0.8873887	1.250497\\
0.8874887	1.245271\\
0.8875888	1.23895\\
0.8876888	1.234597\\
0.8877888	1.228878\\
0.8878888	1.220823\\
0.8879888	1.209503\\
0.8880888	1.201658\\
0.8881888	1.192757\\
0.8882888	1.184978\\
0.8883888	1.176173\\
0.8884888	1.165565\\
0.8885889	1.152804\\
0.8886889	1.140845\\
0.8887889	1.125803\\
0.8888889	1.113212\\
0.8889889	1.098932\\
0.8890889	1.08321\\
0.8891889	1.067902\\
0.8892889	1.05132\\
0.8893889	1.034248\\
0.8894889	1.016969\\
0.889589	0.9991939\\
0.889689	0.980553\\
0.889789	0.9608818\\
0.889889	0.9401671\\
0.889989	0.9207326\\
0.890089	0.8998427\\
0.890189	0.8787238\\
0.890289	0.8585976\\
0.890389	0.8381109\\
0.890489	0.8160441\\
0.8905891	0.7920066\\
0.8906891	0.7671941\\
0.8907891	0.7418851\\
0.8908891	0.7160459\\
0.8909891	0.6910799\\
0.8910891	0.6678802\\
0.8911891	0.6448501\\
0.8912891	0.6201493\\
0.8913891	0.5949979\\
0.8914891	0.570075\\
0.8915892	0.5445928\\
0.8916892	0.515733\\
0.8917892	0.4895427\\
0.8918892	0.4611676\\
0.8919892	0.4310292\\
0.8920892	0.4030365\\
0.8921892	0.3787999\\
0.8922892	0.3516442\\
0.8923892	0.3207574\\
0.8924892	0.2901282\\
0.8925893	0.262846\\
0.8926893	0.2361658\\
0.8927893	0.2056905\\
0.8928893	0.1758543\\
0.8929893	0.1487997\\
0.8930893	0.1198738\\
0.8931893	0.09100693\\
0.8932893	0.06000178\\
0.8933893	0.03142792\\
0.8934893	0.002779611\\
0.8935894	-0.02591966\\
0.8936894	-0.0530826\\
0.8937894	-0.08194059\\
0.8938894	-0.1096781\\
0.8939894	-0.1383625\\
0.8940894	-0.1679015\\
0.8941894	-0.1957782\\
0.8942894	-0.2220104\\
0.8943894	-0.2484816\\
0.8944894	-0.2753667\\
0.8945895	-0.3021183\\
0.8946895	-0.327995\\
0.8947895	-0.3548711\\
0.8948895	-0.381093\\
0.8949895	-0.4079458\\
0.8950895	-0.4341612\\
0.8951895	-0.4602381\\
0.8952895	-0.4847139\\
0.8953895	-0.510985\\
0.8954895	-0.5358179\\
0.8955896	-0.5600983\\
0.8956896	-0.5845101\\
0.8957896	-0.608333\\
0.8958896	-0.6305973\\
0.8959896	-0.6518726\\
0.8960896	-0.6743112\\
0.8961896	-0.6971819\\
0.8962896	-0.7186255\\
0.8963896	-0.7383529\\
0.8964896	-0.7586307\\
0.8965897	-0.7780458\\
0.8966897	-0.8004301\\
0.8967897	-0.8176645\\
0.8968897	-0.835577\\
0.8969897	-0.8542767\\
0.8970897	-0.8713867\\
0.8971897	-0.8850158\\
0.8972897	-0.8979029\\
0.8973897	-0.9143206\\
0.8974897	-0.930034\\
0.8975898	-0.9438373\\
0.8976898	-0.9564985\\
0.8977898	-0.9696737\\
0.8978898	-0.9813718\\
0.8979898	-0.993727\\
0.8980898	-1.004901\\
0.8981898	-1.015215\\
0.8982898	-1.027974\\
0.8983898	-1.038534\\
0.8984898	-1.048425\\
0.8985899	-1.055272\\
0.8986899	-1.060802\\
0.8987899	-1.067384\\
0.8988899	-1.071373\\
0.8989899	-1.075337\\
0.8990899	-1.078122\\
0.8991899	-1.082879\\
0.8992899	-1.086604\\
0.8993899	-1.088017\\
0.8994899	-1.088134\\
0.89959	-1.090732\\
0.89969	-1.093504\\
0.89979	-1.094315\\
0.89989	-1.094542\\
0.89999	-1.093202\\
0.90009	-1.090637\\
0.90019	-1.088429\\
0.90029	-1.084512\\
0.90039	-1.080554\\
0.90049	-1.075699\\
0.9005901	-1.071468\\
0.9006901	-1.066335\\
0.9007901	-1.060127\\
0.9008901	-1.052687\\
0.9009901	-1.042634\\
0.9010901	-1.034347\\
0.9011901	-1.023444\\
0.9012901	-1.012179\\
0.9013901	-1.001449\\
0.9014901	-0.9894406\\
0.9015902	-0.9777434\\
0.9016902	-0.9647447\\
0.9017902	-0.9535905\\
0.9018902	-0.9415285\\
0.9019902	-0.9286746\\
0.9020902	-0.9145417\\
0.9021902	-0.8985904\\
0.9022902	-0.8827246\\
0.9023902	-0.8673122\\
0.9024902	-0.8504306\\
0.9025903	-0.8326797\\
0.9026903	-0.8166581\\
0.9027903	-0.7980608\\
0.9028903	-0.7801493\\
0.9029903	-0.7627838\\
0.9030903	-0.7431211\\
0.9031903	-0.7220874\\
0.9032903	-0.7020374\\
0.9033903	-0.6832111\\
0.9034903	-0.6636544\\
0.9035904	-0.6421948\\
0.9036904	-0.6216696\\
0.9037904	-0.6020504\\
0.9038904	-0.58021\\
0.9039904	-0.5568354\\
0.9040904	-0.5315256\\
0.9041904	-0.5078045\\
0.9042904	-0.4858586\\
0.9043904	-0.4594731\\
0.9044904	-0.4342897\\
0.9045905	-0.4091243\\
0.9046905	-0.3864143\\
0.9047905	-0.3630887\\
0.9048905	-0.3379578\\
0.9049905	-0.3137951\\
0.9050905	-0.2899586\\
0.9051905	-0.2651687\\
0.9052905	-0.2416732\\
0.9053905	-0.2151894\\
0.9054905	-0.1890896\\
0.9055906	-0.1633622\\
0.9056906	-0.1369932\\
0.9057906	-0.1119403\\
0.9058906	-0.08610143\\
0.9059906	-0.06119174\\
0.9060906	-0.03700839\\
0.9061906	-0.01149166\\
0.9062906	0.01370871\\
0.9063906	0.03802645\\
0.9064906	0.06336298\\
0.9065907	0.0894823\\
0.9066907	0.1129449\\
0.9067907	0.1358884\\
0.9068907	0.1626393\\
0.9069907	0.1873285\\
0.9070907	0.2120884\\
0.9071907	0.2376775\\
0.9072907	0.2610177\\
0.9073907	0.2862497\\
0.9074907	0.3088393\\
0.9075908	0.3328451\\
0.9076908	0.3542495\\
0.9077908	0.3744653\\
0.9078908	0.3970463\\
0.9079908	0.4229492\\
0.9080908	0.4460535\\
0.9081908	0.4649621\\
0.9082908	0.4838242\\
0.9083908	0.5052118\\
0.9084908	0.5275245\\
0.9085909	0.5457104\\
0.9086909	0.5653038\\
0.9087909	0.5856582\\
0.9088909	0.6085927\\
0.9089909	0.626992\\
0.9090909	0.6442607\\
0.9091909	0.661264\\
0.9092909	0.6796098\\
0.9093909	0.6958338\\
0.9094909	0.7123328\\
0.909591	0.7284706\\
0.909691	0.7436565\\
0.909791	0.7584011\\
0.909891	0.770981\\
0.909991	0.7849587\\
0.910091	0.7984511\\
0.910191	0.8098303\\
0.910291	0.8228157\\
0.910391	0.8335185\\
0.910491	0.8448614\\
0.9105911	0.8548903\\
0.9106911	0.8643194\\
0.9107911	0.8718712\\
0.9108911	0.8805112\\
0.9109911	0.8884397\\
0.9110911	0.894665\\
0.9111911	0.9025131\\
0.9112911	0.9098598\\
0.9113911	0.917342\\
0.9114911	0.9227096\\
0.9115912	0.9297463\\
0.9116912	0.9353914\\
0.9117912	0.9399443\\
0.9118912	0.9423623\\
0.9119912	0.9437673\\
0.9120912	0.9467853\\
0.9121912	0.948238\\
0.9122912	0.9474451\\
0.9123912	0.9469032\\
0.9124912	0.9444986\\
0.9125913	0.9437877\\
0.9126913	0.9414118\\
0.9127913	0.9370481\\
0.9128913	0.9328054\\
0.9129913	0.9317619\\
0.9130913	0.9276791\\
0.9131913	0.9232556\\
0.9132913	0.9165582\\
0.9133913	0.9099605\\
0.9134913	0.9054601\\
0.9135914	0.8981872\\
0.9136914	0.8909647\\
0.9137914	0.8828297\\
0.9138914	0.8743782\\
0.9139914	0.8658698\\
0.9140914	0.8551222\\
0.9141914	0.8444898\\
0.9142914	0.8314619\\
0.9143914	0.8188986\\
0.9144914	0.8059581\\
0.9145915	0.7934654\\
0.9146915	0.7828614\\
0.9147915	0.7703218\\
0.9148915	0.7569447\\
0.9149915	0.7440011\\
0.9150915	0.7312053\\
0.9151915	0.7167854\\
0.9152915	0.7015528\\
0.9153915	0.6858445\\
0.9154915	0.6664525\\
0.9155916	0.6483022\\
0.9156916	0.6313056\\
0.9157916	0.6152921\\
0.9158916	0.597717\\
0.9159916	0.5787638\\
0.9160916	0.5607109\\
0.9161916	0.5429814\\
0.9162916	0.5248371\\
0.9163916	0.5044361\\
0.9164916	0.4828874\\
0.9165917	0.4627191\\
0.9166917	0.4442403\\
0.9167917	0.4243986\\
0.9168917	0.4043083\\
0.9169917	0.3850307\\
0.9170917	0.3632136\\
0.9171917	0.3404716\\
0.9172917	0.3190807\\
0.9173917	0.2983779\\
0.9174917	0.2758359\\
0.9175918	0.2533214\\
0.9176918	0.230047\\
0.9177918	0.2081071\\
0.9178918	0.1865851\\
0.9179918	0.165369\\
0.9180918	0.1428047\\
0.9181918	0.1183039\\
0.9182918	0.09551076\\
0.9183918	0.07311625\\
0.9184918	0.05004763\\
0.9185919	0.03007829\\
0.9186919	0.008269126\\
0.9187919	-0.01275915\\
0.9188919	-0.03448055\\
0.9189919	-0.05519299\\
0.9190919	-0.07685766\\
0.9191919	-0.09788912\\
0.9192919	-0.1217366\\
0.9193919	-0.1423783\\
0.9194919	-0.1645607\\
0.919592	-0.1858012\\
0.919692	-0.2057844\\
0.919792	-0.2268543\\
0.919892	-0.2479616\\
0.919992	-0.2702627\\
0.920092	-0.2908486\\
0.920192	-0.3099778\\
0.920292	-0.3283096\\
0.920392	-0.3470439\\
0.920492	-0.3679502\\
0.9205921	-0.387369\\
0.9206921	-0.4087707\\
0.9207921	-0.4267337\\
0.9208921	-0.4468012\\
0.9209921	-0.4649486\\
0.9210921	-0.4843567\\
0.9211921	-0.5014645\\
0.9212921	-0.5172461\\
0.9213921	-0.5328065\\
0.9214921	-0.547928\\
0.9215922	-0.5602587\\
0.9216922	-0.5743869\\
0.9217922	-0.5891183\\
0.9218922	-0.603505\\
0.9219922	-0.6179732\\
0.9220922	-0.6301918\\
0.9221922	-0.643749\\
0.9222922	-0.6575112\\
0.9223922	-0.6684104\\
0.9224922	-0.6821993\\
0.9225923	-0.6939728\\
0.9226923	-0.704257\\
0.9227923	-0.7144639\\
0.9228923	-0.724275\\
0.9229923	-0.7339172\\
0.9230923	-0.7416328\\
0.9231923	-0.7505693\\
0.9232923	-0.7568413\\
0.9233923	-0.7645535\\
0.9234923	-0.7722149\\
0.9235924	-0.779517\\
0.9236924	-0.7842728\\
0.9237924	-0.7885006\\
0.9238924	-0.789807\\
0.9239924	-0.7965579\\
0.9240924	-0.799746\\
0.9241924	-0.8051725\\
0.9242924	-0.8078231\\
0.9243924	-0.8101866\\
0.9244924	-0.8107603\\
0.9245925	-0.8103186\\
0.9246925	-0.8073115\\
0.9247925	-0.8102197\\
0.9248925	-0.808565\\
0.9249925	-0.8108257\\
0.9250925	-0.8091342\\
0.9251925	-0.8074705\\
0.9252925	-0.8040123\\
0.9253925	-0.8014717\\
0.9254925	-0.7986967\\
0.9255926	-0.7921106\\
0.9256926	-0.7880878\\
0.9257926	-0.7806578\\
0.9258926	-0.7730362\\
0.9259926	-0.7653752\\
0.9260926	-0.758987\\
0.9261926	-0.75104\\
0.9262926	-0.7447699\\
0.9263926	-0.7351883\\
0.9264926	-0.7269011\\
0.9265927	-0.7180506\\
0.9266927	-0.7082555\\
0.9267927	-0.6993566\\
0.9268927	-0.6890224\\
0.9269927	-0.6782856\\
0.9270927	-0.6636476\\
0.9271927	-0.6530425\\
0.9272927	-0.6416724\\
0.9273927	-0.6298594\\
0.9274927	-0.6157342\\
0.9275928	-0.6029668\\
0.9276928	-0.5898353\\
0.9277928	-0.5747377\\
0.9278928	-0.5585892\\
0.9279928	-0.5435868\\
0.9280928	-0.5295024\\
0.9281928	-0.5145962\\
0.9282928	-0.5022153\\
0.9283928	-0.4850544\\
0.9284928	-0.4704957\\
0.9285929	-0.4551954\\
0.9286929	-0.4381585\\
0.9287929	-0.421259\\
0.9288929	-0.4031791\\
0.9289929	-0.3837795\\
0.9290929	-0.3633956\\
0.9291929	-0.3448757\\
0.9292929	-0.3250129\\
0.9293929	-0.3075004\\
0.9294929	-0.2882685\\
0.929593	-0.2685908\\
0.929693	-0.2506756\\
0.929793	-0.2302451\\
0.929893	-0.2118704\\
0.929993	-0.1921792\\
0.930093	-0.1752592\\
0.930193	-0.1580911\\
0.930293	-0.1400063\\
0.930393	-0.1203132\\
0.930493	-0.09872919\\
0.9305931	-0.08068705\\
0.9306931	-0.06027234\\
0.9307931	-0.04224378\\
0.9308931	-0.02219277\\
0.9309931	-0.003805226\\
0.9310931	0.01532682\\
0.9311931	0.03648722\\
0.9312931	0.05366342\\
0.9313931	0.07515263\\
0.9314931	0.09210438\\
0.9315932	0.1121144\\
0.9316932	0.1313661\\
0.9317932	0.148018\\
0.9318932	0.1675485\\
0.9319932	0.1858983\\
0.9320932	0.205855\\
0.9321932	0.2235816\\
0.9322932	0.2390681\\
0.9323932	0.2570909\\
0.9324932	0.2748862\\
0.9325933	0.2935898\\
0.9326933	0.3098846\\
0.9327933	0.3277435\\
0.9328933	0.344738\\
0.9329933	0.3590149\\
0.9330933	0.3743536\\
0.9331933	0.3892753\\
0.9332933	0.4041098\\
0.9333933	0.419444\\
0.9334933	0.4331205\\
0.9335934	0.4488325\\
0.9336934	0.4640048\\
0.9337934	0.4773924\\
0.9338934	0.4912788\\
0.9339934	0.5027112\\
0.9340934	0.5181542\\
0.9341934	0.5308504\\
0.9342934	0.5422838\\
0.9343934	0.5535456\\
0.9344934	0.563058\\
0.9345935	0.5739467\\
0.9346935	0.5812044\\
0.9347935	0.5925302\\
0.9348935	0.6026585\\
0.9349935	0.6113032\\
0.9350935	0.6187109\\
0.9351935	0.6243006\\
0.9352935	0.6319596\\
0.9353935	0.639494\\
0.9354935	0.6454222\\
0.9355936	0.6505619\\
0.9356936	0.6568359\\
0.9357936	0.6611002\\
0.9358936	0.6683714\\
0.9359936	0.6709431\\
0.9360936	0.6771258\\
0.9361936	0.6812596\\
0.9362936	0.683804\\
0.9363936	0.6910326\\
0.9364936	0.6927268\\
0.9365937	0.6946148\\
0.9366937	0.6951289\\
0.9367937	0.692932\\
0.9368937	0.6929812\\
0.9369937	0.6913584\\
0.9370937	0.6902787\\
0.9371937	0.6890349\\
0.9372937	0.6874681\\
0.9373937	0.6868005\\
0.9374937	0.6858923\\
0.9375938	0.6804772\\
0.9376938	0.6782615\\
0.9377938	0.6736859\\
0.9378938	0.6678656\\
0.9379938	0.6630653\\
0.9380938	0.6567907\\
0.9381938	0.652189\\
0.9382938	0.6442366\\
0.9383938	0.6363832\\
0.9384938	0.6287995\\
0.9385939	0.6214661\\
0.9386939	0.6123621\\
0.9387939	0.6044593\\
0.9388939	0.5971073\\
0.9389939	0.5868071\\
0.9390939	0.579648\\
0.9391939	0.5699668\\
0.9392939	0.5590801\\
0.9393939	0.5494423\\
0.9394939	0.5377022\\
0.939594	0.5258037\\
0.939694	0.5170824\\
0.939794	0.5054062\\
0.939894	0.4926254\\
0.939994	0.4801084\\
0.940094	0.4643176\\
0.940194	0.4526101\\
0.940294	0.4397209\\
0.940394	0.4263951\\
0.940494	0.4154968\\
0.9405941	0.3992301\\
0.9406941	0.386351\\
0.9407941	0.3717596\\
0.9408941	0.354719\\
0.9409941	0.3365951\\
0.9410941	0.3209485\\
0.9411941	0.3028771\\
0.9412941	0.2868951\\
0.9413941	0.2757163\\
0.9414941	0.259086\\
0.9415942	0.2430118\\
0.9416942	0.229437\\
0.9417942	0.2135866\\
0.9418942	0.1966692\\
0.9419942	0.1818452\\
0.9420942	0.1632977\\
0.9421942	0.1472352\\
0.9422942	0.1301245\\
0.9423942	0.1122909\\
0.9424942	0.09532302\\
0.9425943	0.07781573\\
0.9426943	0.05950104\\
0.9427943	0.04124896\\
0.9428943	0.0254539\\
0.9429943	0.008057716\\
0.9430943	-0.01041668\\
0.9431943	-0.02694633\\
0.9432943	-0.04176401\\
0.9433943	-0.05639745\\
0.9434943	-0.07193964\\
0.9435944	-0.08880194\\
0.9436944	-0.1020565\\
0.9437944	-0.1179563\\
0.9438944	-0.1355914\\
0.9439944	-0.1513841\\
0.9440944	-0.1667127\\
0.9441944	-0.1816671\\
0.9442944	-0.1986606\\
0.9443944	-0.2127495\\
0.9444944	-0.2274523\\
0.9445945	-0.2461043\\
0.9446945	-0.2587913\\
0.9447945	-0.274354\\
0.9448945	-0.2885013\\
0.9449945	-0.3014868\\
0.9450945	-0.3155852\\
0.9451945	-0.3308993\\
0.9452945	-0.3418049\\
0.9453945	-0.355462\\
0.9454945	-0.3696215\\
0.9455946	-0.3827467\\
0.9456946	-0.3953629\\
0.9457946	-0.4081283\\
0.9458946	-0.4208305\\
0.9459946	-0.4299903\\
0.9460946	-0.4426987\\
0.9461946	-0.452162\\
0.9462946	-0.4595803\\
0.9463946	-0.4686461\\
0.9464946	-0.4786891\\
0.9465947	-0.4881278\\
0.9466947	-0.4937415\\
0.9467947	-0.5051951\\
0.9468947	-0.5125183\\
0.9469947	-0.5170892\\
0.9470947	-0.5232365\\
0.9471947	-0.5290141\\
0.9472947	-0.5368194\\
0.9473947	-0.5449753\\
0.9474947	-0.550772\\
0.9475948	-0.5571507\\
0.9476948	-0.5625066\\
0.9477948	-0.567907\\
0.9478948	-0.5733187\\
0.9479948	-0.5742678\\
0.9480948	-0.5777067\\
0.9481948	-0.5800934\\
0.9482948	-0.5811421\\
0.9483948	-0.5835472\\
0.9484948	-0.5847174\\
0.9485949	-0.5862346\\
0.9486949	-0.5869702\\
0.9487949	-0.588034\\
0.9488949	-0.5890128\\
0.9489949	-0.5881389\\
0.9490949	-0.587564\\
0.9491949	-0.5866152\\
0.9492949	-0.5852898\\
0.9493949	-0.5815114\\
0.9494949	-0.5778659\\
0.949595	-0.5764794\\
0.949695	-0.5738932\\
0.949795	-0.5687367\\
0.949895	-0.5636865\\
0.949995	-0.5618224\\
0.950095	-0.5567372\\
0.950195	-0.5489893\\
0.950295	-0.5432611\\
0.950395	-0.5373224\\
0.950495	-0.5298693\\
0.9505951	-0.5213522\\
0.9506951	-0.5150035\\
0.9507951	-0.5079639\\
0.9508951	-0.5005443\\
0.9509951	-0.494829\\
0.9510951	-0.4870767\\
0.9511951	-0.4773586\\
0.9512951	-0.4673453\\
0.9513951	-0.4578918\\
0.9514951	-0.4479042\\
0.9515952	-0.4371615\\
0.9516952	-0.4251189\\
0.9517952	-0.4167456\\
0.9518952	-0.405164\\
0.9519952	-0.394967\\
0.9520952	-0.3828356\\
0.9521952	-0.3699015\\
0.9522952	-0.3589905\\
0.9523952	-0.3464107\\
0.9524952	-0.3337287\\
0.9525953	-0.3238851\\
0.9526953	-0.31158\\
0.9527953	-0.2985392\\
0.9528953	-0.2862772\\
0.9529953	-0.2747818\\
0.9530953	-0.2620358\\
0.9531953	-0.2476537\\
0.9532953	-0.2345116\\
0.9533953	-0.2223549\\
0.9534953	-0.2078036\\
0.9535954	-0.1912056\\
0.9536954	-0.176451\\
0.9537954	-0.162151\\
0.9538954	-0.1471081\\
0.9539954	-0.1303002\\
0.9540954	-0.1155004\\
0.9541954	-0.1002005\\
0.9542954	-0.0869588\\
0.9543954	-0.07213975\\
0.9544954	-0.05693399\\
0.9545955	-0.04201564\\
0.9546955	-0.03044921\\
0.9547955	-0.01468818\\
0.9548955	0.0009322527\\
0.9549955	0.01453861\\
0.9550955	0.0284872\\
0.9551955	0.04070618\\
0.9552955	0.05534169\\
0.9553955	0.06988208\\
0.9554955	0.08279017\\
0.9555956	0.09500604\\
0.9556956	0.1103717\\
0.9557956	0.1244758\\
0.9558956	0.1350068\\
0.9559956	0.1458697\\
0.9560956	0.1615612\\
0.9561956	0.1778002\\
0.9562956	0.1920798\\
0.9563956	0.2046407\\
0.9564956	0.2200337\\
0.9565957	0.2340579\\
0.9566957	0.2457756\\
0.9567957	0.2556848\\
0.9568957	0.2650644\\
0.9569957	0.2756193\\
0.9570957	0.2877628\\
0.9571957	0.3003273\\
0.9572957	0.311744\\
0.9573957	0.3201091\\
0.9574957	0.3313737\\
0.9575958	0.3430425\\
0.9576958	0.3541213\\
0.9577958	0.3639757\\
0.9578958	0.3735513\\
0.9579958	0.3833538\\
0.9580958	0.3935299\\
0.9581958	0.4004696\\
0.9582958	0.4071937\\
0.9583958	0.4137999\\
0.9584958	0.421178\\
0.9585959	0.4273501\\
0.9586959	0.4350438\\
0.9587959	0.4414466\\
0.9588959	0.4462327\\
0.9589959	0.4528118\\
0.9590959	0.459069\\
0.9591959	0.4670567\\
0.9592959	0.4700565\\
0.9593959	0.4743002\\
0.9594959	0.4803741\\
0.959596	0.4838832\\
0.959696	0.4855088\\
0.959796	0.4868653\\
0.959896	0.4903289\\
0.959996	0.492274\\
0.960096	0.4931409\\
0.960196	0.4942626\\
0.960296	0.4946533\\
0.960396	0.492933\\
0.960496	0.4944746\\
0.9605961	0.4970297\\
0.9606961	0.4976208\\
0.9607961	0.498759\\
0.9608961	0.4966892\\
0.9609961	0.4961983\\
0.9610961	0.4960204\\
0.9611961	0.4949377\\
0.9612961	0.4918593\\
0.9613961	0.4876585\\
0.9614961	0.4845356\\
0.9615962	0.4807442\\
0.9616962	0.4777984\\
0.9617962	0.4744884\\
0.9618962	0.4682385\\
0.9619962	0.462456\\
0.9620962	0.4585433\\
0.9621962	0.4556436\\
0.9622962	0.4486361\\
0.9623962	0.4424553\\
0.9624962	0.4362261\\
0.9625963	0.4264237\\
0.9626963	0.4196242\\
0.9627963	0.4131149\\
0.9628963	0.4083897\\
0.9629963	0.3991553\\
0.9630963	0.3901263\\
0.9631963	0.3796171\\
0.9632963	0.369596\\
0.9633963	0.362852\\
0.9634963	0.3533937\\
0.9635964	0.3403924\\
0.9636964	0.3291811\\
0.9637964	0.3223477\\
0.9638964	0.3138836\\
0.9639964	0.3052061\\
0.9640964	0.2965102\\
0.9641964	0.2866703\\
0.9642964	0.2770653\\
0.9643964	0.2662347\\
0.9644964	0.2546668\\
0.9645965	0.2434261\\
0.9646965	0.2342416\\
0.9647965	0.2231614\\
0.9648965	0.2096901\\
0.9649965	0.1964625\\
0.9650965	0.1834957\\
0.9651965	0.1727879\\
0.9652965	0.1611693\\
0.9653965	0.1476704\\
0.9654965	0.1340687\\
0.9655966	0.121338\\
0.9656966	0.1075405\\
0.9657966	0.09329792\\
0.9658966	0.0818907\\
0.9659966	0.07256819\\
0.9660966	0.06008257\\
0.9661966	0.04767925\\
0.9662966	0.03529349\\
0.9663966	0.0217918\\
0.9664966	0.009768708\\
0.9665967	-0.00200381\\
0.9666967	-0.01298044\\
0.9667967	-0.02632451\\
0.9668967	-0.03944934\\
0.9669967	-0.05260353\\
0.9670967	-0.06421152\\
0.9671967	-0.07775639\\
0.9672967	-0.08931448\\
0.9673967	-0.1001161\\
0.9674967	-0.1114681\\
0.9675968	-0.1244901\\
0.9676968	-0.1378803\\
0.9677968	-0.1502735\\
0.9678968	-0.1614506\\
0.9679968	-0.1717876\\
0.9680968	-0.1821858\\
0.9681968	-0.1906207\\
0.9682968	-0.1992302\\
0.9683968	-0.2106283\\
0.9684968	-0.2213311\\
0.9685969	-0.2308891\\
0.9686969	-0.2428489\\
0.9687969	-0.2519443\\
0.9688969	-0.2612712\\
0.9689969	-0.268912\\
0.9690969	-0.2774621\\
0.9691969	-0.2879768\\
0.9692969	-0.2969576\\
0.9693969	-0.3054783\\
0.9694969	-0.3142846\\
0.969597	-0.323468\\
0.969697	-0.3305872\\
0.969797	-0.3374242\\
0.969897	-0.3423471\\
0.969997	-0.3507728\\
0.970097	-0.3596987\\
0.970197	-0.3661942\\
0.970297	-0.3723006\\
0.970397	-0.3788382\\
0.970497	-0.3857245\\
0.9705971	-0.3893384\\
0.9706971	-0.3917698\\
0.9707971	-0.3960652\\
0.9708971	-0.3999947\\
0.9709971	-0.4040909\\
0.9710971	-0.4072901\\
0.9711971	-0.4103468\\
0.9712971	-0.4157353\\
0.9713971	-0.4182068\\
0.9714971	-0.4185705\\
0.9715972	-0.420485\\
0.9716972	-0.4204737\\
0.9717972	-0.4217343\\
0.9718972	-0.4211334\\
0.9719972	-0.4198539\\
0.9720972	-0.4221888\\
0.9721972	-0.4242988\\
0.9722972	-0.4223396\\
0.9723972	-0.4215439\\
0.9724972	-0.4216923\\
0.9725973	-0.4209795\\
0.9726973	-0.4201031\\
0.9727973	-0.4184904\\
0.9728973	-0.4163349\\
0.9729973	-0.4141077\\
0.9730973	-0.412146\\
0.9731973	-0.4102875\\
0.9732973	-0.4090167\\
0.9733973	-0.4060732\\
0.9734973	-0.4014042\\
0.9735974	-0.3970277\\
0.9736974	-0.3920365\\
0.9737974	-0.3852299\\
0.9738974	-0.3779865\\
0.9739974	-0.3719043\\
0.9740974	-0.3678421\\
0.9741974	-0.3623654\\
0.9742974	-0.3557642\\
0.9743974	-0.3519238\\
0.9744974	-0.3451432\\
0.9745975	-0.338387\\
0.9746975	-0.331317\\
0.9747975	-0.3242943\\
0.9748975	-0.3156464\\
0.9749975	-0.3067839\\
0.9750975	-0.2986403\\
0.9751975	-0.2882986\\
0.9752975	-0.2790007\\
0.9753975	-0.2702839\\
0.9754975	-0.2618637\\
0.9755976	-0.2534772\\
0.9756976	-0.246663\\
0.9757976	-0.2378979\\
0.9758976	-0.2311779\\
0.9759976	-0.2224332\\
0.9760976	-0.2116426\\
0.9761976	-0.2028448\\
0.9762976	-0.192631\\
0.9763976	-0.1806018\\
0.9764976	-0.1698679\\
0.9765977	-0.1582049\\
0.9766977	-0.1480279\\
0.9767977	-0.1355806\\
0.9768977	-0.1239904\\
0.9769977	-0.1142967\\
0.9770977	-0.1047859\\
0.9771977	-0.09370314\\
0.9772977	-0.08470296\\
0.9773977	-0.07600253\\
0.9774977	-0.06495239\\
0.9775978	-0.05657945\\
0.9776978	-0.04536228\\
0.9777978	-0.03655159\\
0.9778978	-0.02484307\\
0.9779978	-0.01298707\\
0.9780978	-0.001061766\\
0.9781978	0.01064782\\
0.9782978	0.02130736\\
0.9783978	0.0316482\\
0.9784978	0.04334533\\
0.9785979	0.05433399\\
0.9786979	0.06720639\\
0.9787979	0.07759426\\
0.9788979	0.08975966\\
0.9789979	0.1006138\\
0.9790979	0.1105342\\
0.9791979	0.1218409\\
0.9792979	0.1311086\\
0.9793979	0.1379543\\
0.9794979	0.1455527\\
0.979598	0.1565642\\
0.979698	0.167541\\
0.979798	0.1759281\\
0.979898	0.1834199\\
0.979998	0.1921495\\
0.980098	0.2008183\\
0.980198	0.2097816\\
0.980298	0.2173901\\
0.980398	0.2250704\\
0.980498	0.233218\\
0.9805981	0.2419339\\
0.9806981	0.2490882\\
0.9807981	0.255778\\
0.9808981	0.264248\\
0.9809981	0.2702995\\
0.9810981	0.2778741\\
0.9811981	0.2850838\\
0.9812981	0.2916867\\
0.9813981	0.2973934\\
0.9814981	0.3027791\\
0.9815982	0.3093554\\
0.9816982	0.3140344\\
0.9817982	0.32056\\
0.9818982	0.3262678\\
0.9819982	0.3316494\\
0.9820982	0.3354403\\
0.9821982	0.3400692\\
0.9822982	0.3444675\\
0.9823982	0.3488369\\
0.9824982	0.3522171\\
0.9825983	0.354164\\
0.9826983	0.3562352\\
0.9827983	0.3582822\\
0.9828983	0.3602299\\
0.9829983	0.3628679\\
0.9830983	0.3648848\\
0.9831983	0.3658908\\
0.9832983	0.3669327\\
0.9833983	0.3670756\\
0.9834983	0.367451\\
0.9835984	0.3665226\\
0.9836984	0.3662381\\
0.9837984	0.3652911\\
0.9838984	0.3655845\\
0.9839984	0.3665581\\
0.9840984	0.3663347\\
0.9841984	0.3643606\\
0.9842984	0.3632837\\
0.9843984	0.3612446\\
0.9844984	0.3589601\\
0.9845985	0.3587334\\
0.9846985	0.3570748\\
0.9847985	0.3536955\\
0.9848985	0.3511992\\
0.9849985	0.3469403\\
0.9850985	0.3446019\\
0.9851985	0.340773\\
0.9852985	0.33773\\
0.9853985	0.3340128\\
0.9854985	0.3310566\\
0.9855986	0.3295748\\
0.9856986	0.3270249\\
0.9857986	0.3233911\\
0.9858986	0.3193269\\
0.9859986	0.3154486\\
0.9860986	0.3117336\\
0.9861986	0.3089729\\
0.9862986	0.3058004\\
0.9863986	0.3040407\\
0.9864986	0.3017524\\
0.9865987	0.299382\\
0.9866987	0.2971416\\
0.9867987	0.2945383\\
0.9868987	0.2898725\\
0.9869987	0.2871816\\
0.9870987	0.2827325\\
0.9871987	0.2800072\\
0.9872987	0.2776064\\
0.9873987	0.2747684\\
0.9874987	0.2727094\\
0.9875988	0.2678713\\
0.9876988	0.2653315\\
0.9877988	0.2609049\\
0.9878988	0.2561691\\
0.9879988	0.2521826\\
0.9880988	0.247484\\
0.9881988	0.2427369\\
0.9882988	0.237257\\
0.9883988	0.2330031\\
0.9884988	0.2281965\\
0.9885989	0.2225634\\
0.9886989	0.2156332\\
0.9887989	0.2092343\\
0.9888989	0.2045617\\
0.9889989	0.2002445\\
0.9890989	0.1927772\\
0.9891989	0.186301\\
0.9892989	0.1810907\\
0.9893989	0.1769253\\
0.9894989	0.1726081\\
0.989599	0.1698845\\
0.989699	0.1681956\\
0.989799	0.1646312\\
0.989899	0.1611549\\
0.989999	0.1567643\\
0.990099	0.1532928\\
0.990199	0.148284\\
0.990299	0.144163\\
0.990399	0.1410361\\
0.990499	0.1384102\\
0.9905991	0.1335693\\
0.9906991	0.1302041\\
0.9907991	0.1240498\\
0.9908991	0.1209733\\
0.9909991	0.1148119\\
0.9910991	0.1099225\\
0.9911991	0.1030058\\
0.9912991	0.09634938\\
0.9913991	0.09293408\\
0.9914991	0.08597533\\
0.9915992	0.07927353\\
0.9916992	0.07447438\\
0.9917992	0.07117196\\
0.9918992	0.06739463\\
0.9919992	0.0635175\\
0.9920992	0.05893427\\
0.9921992	0.05475966\\
0.9922992	0.0498781\\
0.9923992	0.04478784\\
0.9924992	0.04138685\\
0.9925993	0.03686293\\
0.9926993	0.03018379\\
0.9927993	0.02413468\\
0.9928993	0.01656235\\
0.9929993	0.01307869\\
0.9930993	0.007982238\\
0.9931993	0.003065718\\
0.9932993	-0.000908818\\
0.9933993	-0.00388665\\
0.9934993	-0.006290269\\
0.9935994	-0.01053804\\
0.9936994	-0.01308806\\
0.9937994	-0.01759906\\
0.9938994	-0.02043064\\
0.9939994	-0.02333982\\
0.9940994	-0.02716152\\
0.9941994	-0.03313453\\
0.9942994	-0.03618255\\
0.9943994	-0.04015866\\
0.9944994	-0.04399792\\
0.9945995	-0.04624639\\
0.9946995	-0.04738769\\
0.9947995	-0.04895307\\
0.9948995	-0.0527765\\
0.9949995	-0.05609447\\
0.9950995	-0.06000801\\
0.9951995	-0.06559563\\
0.9952995	-0.07240253\\
0.9953995	-0.07624982\\
0.9954995	-0.08043311\\
0.9955996	-0.08462096\\
0.9956996	-0.08915276\\
0.9957996	-0.09300281\\
0.9958996	-0.09629444\\
0.9959996	-0.09916009\\
0.9960996	-0.1036796\\
0.9961996	-0.1075946\\
0.9962996	-0.110159\\
0.9963996	-0.1146994\\
0.9964996	-0.1194568\\
0.9965997	-0.1219368\\
0.9966997	-0.1238106\\
0.9967997	-0.1254491\\
0.9968997	-0.1253265\\
0.9969997	-0.1275029\\
0.9970997	-0.1285938\\
0.9971997	-0.1297638\\
0.9972997	-0.1336577\\
0.9973997	-0.1353209\\
0.9974997	-0.1379713\\
0.9975998	-0.1391612\\
0.9976998	-0.1399976\\
0.9977998	-0.1408983\\
0.9978998	-0.1404964\\
0.9979998	-0.1412137\\
0.9980998	-0.1432231\\
0.9981998	-0.1454826\\
0.9982998	-0.1477692\\
0.9983998	-0.1502429\\
0.9984998	-0.153985\\
0.9985999	-0.157245\\
0.9986999	-0.1569888\\
0.9987999	-0.1584671\\
0.9988999	-0.1601096\\
0.9989999	-0.1630507\\
0.9990999	-0.1657033\\
0.9991999	-0.1691009\\
0.9992999	-0.1710002\\
0.9993999	-0.1741735\\
0.9994999	-0.1745732\\
0.9996	-0.1752182\\
0.9997	-0.1759186\\
0.9998	-0.1749133\\
0.9999	-0.1771138\\
1	-0.1777804\\
};
\addlegendentry{$\text{imag(}\psi{}_\text{n}\text{(x,t))}$};

\addplot [color=red,mark size=0.7pt,only marks,mark=*,mark options={solid}]
  table[row sep=crcr]{%
0	1.64442355057845e-22\\
0.0167016701670167	1.5197240973474e-21\\
0.0334033403340334	1.33763780455174e-20\\
0.0501050105010501	1.12133413116993e-19\\
0.0668066806680668	8.95270493104685e-19\\
0.0835083508350835	6.80763438269121e-18\\
0.1002100210021	4.93015868195705e-17\\
0.116911691169117	3.40054318759744e-16\\
0.133613361336134	2.23387266838271e-15\\
0.15031503150315	1.39762691885738e-14\\
0.167016701670167	8.32811706951729e-14\\
0.183718371837184	4.72634161474762e-13\\
0.2004200420042	2.55461888610083e-12\\
0.217121712171217	1.31507296853249e-11\\
0.233823382338234	6.44757356102553e-11\\
0.250525052505251	3.01068613919693e-10\\
0.267226722672267	1.33892849120405e-09\\
0.283928392839284	5.67116137544888e-09\\
0.300630063006301	2.28775414288284e-08\\
0.317331733173317	8.78960620290514e-08\\
0.334033403340334	3.2162678389677e-07\\
0.350735073507351	1.12087638809504e-06\\
0.367436743674367	3.72036847520886e-06\\
0.384138413841384	1.17608013199936e-05\\
0.400840084008401	3.54087498300945e-05\\
0.417541754175418	0.000101532948022041\\
0.434243424342434	0.000277284837026129\\
0.450945094509451	0.000721220344715269\\
0.467646764676468	0.0017866214319633\\
0.484348434843484	0.00421521598203439\\
0.501050105010501	0.00947174173184439\\
0.517751775177518	0.020270412443154\\
0.534453445344534	0.0413159787886048\\
0.551155115511551	0.0802040370079134\\
0.567856785678568	0.148284970173713\\
0.584558455845585	0.261108351062417\\
0.601260126012601	0.437892102505805\\
0.617961796179618	0.699417002653674\\
0.634663466346635	1.06396656679291\\
0.651365136513651	1.54149634514675\\
0.668066806680668	2.12705999685486\\
0.684768476847685	2.79537259493006\\
0.701470147014701	3.4988263520665\\
0.718171817181718	4.170880980297\\
0.734873487348735	4.73539190934414\\
0.751575157515752	5.12043409924688\\
0.768276827682768	5.27327423030381\\
0.784978497849785	5.17221602143576\\
0.801680168016802	4.83165232738973\\
0.818381838183818	4.2987027920769\\
0.835083508350835	3.64251949987771\\
0.851785178517852	2.93960565285496\\
0.868486848684868	2.25943048880823\\
0.885188518851885	1.6539851970363\\
0.901890189018902	1.15315310942942\\
0.918591859185919	0.76571129420871\\
0.935293529352935	0.484245790431272\\
0.951995199519952	0.291668398961461\\
0.968696869686969	0.1673152874426\\
0.985398539853985	0.0914122908027332\\
};
\addlegendentry{$\text{|}\psi{}_\text{a}\text{(x,t)|}^\text{2}$};

\addplot [color=black,mark size=0.7pt,only marks,mark=*,mark options={solid}]
  table[row sep=crcr]{%
0	-1.27675569491282e-11\\
0.0167016701670167	-1.44320639101746e-11\\
0.0334033403340334	8.33166972643569e-11\\
0.0501050105010501	3.18312386338341e-10\\
0.0668066806680668	2.56338629711749e-10\\
0.0835083508350835	-1.48390095555069e-09\\
0.1002100210021	-6.89785251261254e-09\\
0.116911691169117	-1.62836795869618e-08\\
0.133613361336134	-2.29601284074641e-08\\
0.15031503150315	-3.97835901640648e-09\\
0.167016701670167	9.58365455090127e-08\\
0.183718371837184	3.9342699718733e-07\\
0.2004200420042	1.12247026913904e-06\\
0.217121712171217	2.71604212208874e-06\\
0.233823382338234	5.83222812402185e-06\\
0.250525052505251	1.08836798290307e-05\\
0.267226722672267	1.56401123441653e-05\\
0.283928392839284	7.80587231741371e-06\\
0.300630063006301	-4.97829872645521e-05\\
0.317331733173317	-0.000227754095430591\\
0.334033403340334	-0.000566335031112856\\
0.350735073507351	-0.00081596484089443\\
0.367436743674367	-6.75446904657783e-05\\
0.384138413841384	0.00271033068745036\\
0.400840084008401	0.00550748163683275\\
0.417541754175418	0.000335102021987193\\
0.434243424342434	-0.0156226862771179\\
0.450945094509451	-0.0145407941265323\\
0.467646764676468	0.032613847384848\\
0.484348434843484	0.0420903518300913\\
0.501050105010501	-0.0792204046795816\\
0.517751775177518	-0.059717306962364\\
0.534453445344534	0.201146948732901\\
0.551155115511551	-0.0688004273915454\\
0.567856785678568	-0.285238552995914\\
0.584558455845585	0.496161249739722\\
0.601260126012601	-0.289391448348718\\
0.617961796179618	-0.265178612190064\\
0.634663466346635	0.866007584935154\\
0.651365136513651	-1.24156790836517\\
0.668066806680668	1.30089178426692\\
0.684768476847685	-1.11760633438503\\
0.701470147014701	0.837080846817071\\
0.718171817181718	-0.598202563403678\\
0.734873487348735	0.500297253588819\\
0.751575157515752	-0.599466577921309\\
0.768276827682768	0.906250388692695\\
0.784978497849785	-1.36668112283849\\
0.801680168016802	1.83172467689519\\
0.818381838183818	-2.05191201673006\\
0.835083508350835	1.75832469348223\\
0.851785178517852	-0.861865676153611\\
0.868486848684868	-0.3271773127962\\
0.885188518851885	1.12626490495303\\
0.901890189018902	-0.969203475363891\\
0.918591859185919	0.0657930174644708\\
0.935293529352935	0.612604447859163\\
0.951995199519952	-0.391822537920453\\
0.968696869686969	-0.211318712235213\\
0.985398539853985	0.275047076370718\\
};
\addlegendentry{$\text{imag(}\psi{}_\text{a}\text{(x,t))}$};

\end{axis}
\end{tikzpicture}%
		\caption{The numerical solution ($\psi_n$) plotted against the analytical solution ($\psi_a$).}
		\label{fig:periodicInfPlot}
	\end{subfigure}
	\begin{subfigure}{.9\linewidth}
		\setlength\figureheight{.5\linewidth} 
		\setlength\figurewidth{.9\linewidth}
		% This file was created by matlab2tikz.
% Minimal pgfplots version: 1.3
%
%The latest updates can be retrieved from
%  http://www.mathworks.com/matlabcentral/fileexchange/22022-matlab2tikz
%where you can also make suggestions and rate matlab2tikz.
%
\definecolor{mycolor1}{rgb}{0.00000,0.44700,0.74100}%
%
\begin{tikzpicture}

\begin{axis}[%
width=0.95092\figurewidth,
height=\figureheight,
at={(0\figurewidth,0\figureheight)},
scale only axis,
xmin=0,
xmax=1,
xlabel={Position},
ymin=-0.05,
ymax=0.04,
ylabel={Magnitude},
title style={font=\bfseries},
title={$\text{Error of |}\psi{}_\text{n}\text{(x,t)|}^\text{2}\text{, dt = 2e-06, dx = 0.0001}$},
title style={font=\small},ticklabel style={font=\tiny}
]
\addplot [color=mycolor1,solid,forget plot]
  table[row sep=crcr]{%
0	-0.03279891\\
0.00010001	-0.0326747\\
0.00020002	-0.03186465\\
0.00030003	-0.03087951\\
0.00040004	-0.03040553\\
0.00050005	-0.0300395\\
0.00060006	-0.02995223\\
0.00070007	-0.03017902\\
0.00080008	-0.03050235\\
0.00090009	-0.03055689\\
0.0010001	-0.03044676\\
0.00110011	-0.03006607\\
0.00120012	-0.02965036\\
0.00130013	-0.02982389\\
0.00140014	-0.02958991\\
0.00150015	-0.03005099\\
0.00160016	-0.03027945\\
0.00170017	-0.03061407\\
0.00180018	-0.02989003\\
0.00190019	-0.0295733\\
0.0020002	-0.02899653\\
0.00210021	-0.02903299\\
0.00220022	-0.02924911\\
0.00230023	-0.02902951\\
0.00240024	-0.02921216\\
0.00250025	-0.02825344\\
0.00260026	-0.02756002\\
0.00270027	-0.02747151\\
0.00280028	-0.02778646\\
0.00290029	-0.02774148\\
0.0030003	-0.02800545\\
0.00310031	-0.02795833\\
0.00320032	-0.02792743\\
0.00330033	-0.02714662\\
0.00340034	-0.02759028\\
0.00350035	-0.02813572\\
0.00360036	-0.02849558\\
0.00370037	-0.02847096\\
0.00380038	-0.02843458\\
0.00390039	-0.02867554\\
0.0040004	-0.02826214\\
0.00410041	-0.0277668\\
0.00420042	-0.02778857\\
0.00430043	-0.02731028\\
0.00440044	-0.02749018\\
0.00450045	-0.02714522\\
0.00460046	-0.02627582\\
0.00470047	-0.0257735\\
0.00480048	-0.02510986\\
0.00490049	-0.02574393\\
0.0050005	-0.02594624\\
0.00510051	-0.0259037\\
0.00520052	-0.02527817\\
0.00530053	-0.02531767\\
0.00540054	-0.02539622\\
0.00550055	-0.02551154\\
0.00560056	-0.0258043\\
0.00570057	-0.02536188\\
0.00580058	-0.02516997\\
0.00590059	-0.0246524\\
0.0060006	-0.02464822\\
0.00610061	-0.02497073\\
0.00620062	-0.02504789\\
0.00630063	-0.02473015\\
0.00640064	-0.02398102\\
0.00650065	-0.02297684\\
0.00660066	-0.02354314\\
0.00670067	-0.02356506\\
0.00680068	-0.02328437\\
0.00690069	-0.02316501\\
0.0070007	-0.02315845\\
0.00710071	-0.02345513\\
0.00720072	-0.02380649\\
0.00730073	-0.02427267\\
0.00740074	-0.02428627\\
0.00750075	-0.02451394\\
0.00760076	-0.02453857\\
0.00770077	-0.02509476\\
0.00780078	-0.02548144\\
0.00790079	-0.02510888\\
0.0080008	-0.02451986\\
0.00810081	-0.02378216\\
0.00820082	-0.02357047\\
0.00830083	-0.02378276\\
0.00840084	-0.02306014\\
0.00850085	-0.02206686\\
0.00860086	-0.02150218\\
0.00870087	-0.02125111\\
0.00880088	-0.02139687\\
0.00890089	-0.02140333\\
0.0090009	-0.0217792\\
0.00910091	-0.02191427\\
0.00920092	-0.02128772\\
0.00930093	-0.02224904\\
0.00940094	-0.02238147\\
0.00950095	-0.02224286\\
0.00960096	-0.02157667\\
0.00970097	-0.02127288\\
0.00980098	-0.02116126\\
0.00990099	-0.02105036\\
0.010001	-0.02104549\\
0.01010101	-0.02040838\\
0.01020102	-0.01995105\\
0.01030103	-0.020191\\
0.01040104	-0.02085803\\
0.01050105	-0.020893\\
0.01060106	-0.02099419\\
0.01070107	-0.02101218\\
0.01080108	-0.02129707\\
0.01090109	-0.02145562\\
0.0110011	-0.02115158\\
0.01110111	-0.02039286\\
0.01120112	-0.02025078\\
0.01130113	-0.02008614\\
0.01140114	-0.02007754\\
0.01150115	-0.01990756\\
0.01160116	-0.01932657\\
0.01170117	-0.01901076\\
0.01180118	-0.01903419\\
0.01190119	-0.01933833\\
0.0120012	-0.01993611\\
0.01210121	-0.01956339\\
0.01220122	-0.01920894\\
0.01230123	-0.02004775\\
0.01240124	-0.02065795\\
0.01250125	-0.01988511\\
0.01260126	-0.01951036\\
0.01270127	-0.0195763\\
0.01280128	-0.02023035\\
0.01290129	-0.01972612\\
0.0130013	-0.01912858\\
0.01310131	-0.0188694\\
0.01320132	-0.01928717\\
0.01330133	-0.01890234\\
0.01340134	-0.01838385\\
0.01350135	-0.01793528\\
0.01360136	-0.01816563\\
0.01370137	-0.01884729\\
0.01380138	-0.01809102\\
0.01390139	-0.01761618\\
0.0140014	-0.01759306\\
0.01410141	-0.01772421\\
0.01420142	-0.01692827\\
0.01430143	-0.01640072\\
0.01440144	-0.016409\\
0.01450145	-0.0168541\\
0.01460146	-0.01687614\\
0.01470147	-0.01680702\\
0.01480148	-0.01678405\\
0.01490149	-0.01751277\\
0.0150015	-0.01820367\\
0.01510151	-0.01763165\\
0.01520152	-0.01732458\\
0.01530153	-0.01755883\\
0.01540154	-0.01785804\\
0.01550155	-0.01744876\\
0.01560156	-0.01674341\\
0.01570157	-0.01720526\\
0.01580158	-0.01725453\\
0.01590159	-0.01647132\\
0.0160016	-0.01651581\\
0.01610161	-0.01654435\\
0.01620162	-0.01658319\\
0.01630163	-0.01675631\\
0.01640164	-0.01696304\\
0.01650165	-0.01727014\\
0.01660166	-0.01696546\\
0.01670167	-0.01617824\\
0.01680168	-0.01558916\\
0.01690169	-0.01619746\\
0.0170017	-0.01634499\\
0.01710171	-0.01612091\\
0.01720172	-0.0159139\\
0.01730173	-0.01637823\\
0.01740174	-0.01614508\\
0.01750175	-0.01561616\\
0.01760176	-0.01534785\\
0.01770177	-0.01555142\\
0.01780178	-0.01525876\\
0.01790179	-0.01470636\\
0.0180018	-0.01507191\\
0.01810181	-0.01440759\\
0.01820182	-0.01389743\\
0.01830183	-0.01355486\\
0.01840184	-0.01385405\\
0.01850185	-0.01459324\\
0.01860186	-0.01479545\\
0.01870187	-0.01467257\\
0.01880188	-0.01500455\\
0.01890189	-0.01525513\\
0.0190019	-0.01491749\\
0.01910191	-0.01474385\\
0.01920192	-0.01487441\\
0.01930193	-0.01485295\\
0.01940194	-0.01488229\\
0.01950195	-0.01498013\\
0.01960196	-0.01461288\\
0.01970197	-0.01426432\\
0.01980198	-0.01434327\\
0.01990199	-0.01435977\\
0.020002	-0.01385353\\
0.02010201	-0.01386864\\
0.02020202	-0.01359846\\
0.02030203	-0.01345535\\
0.02040204	-0.01316322\\
0.02050205	-0.01263971\\
0.02060206	-0.01340157\\
0.02070207	-0.01379289\\
0.02080208	-0.01391897\\
0.02090209	-0.01377972\\
0.0210021	-0.01396197\\
0.02110211	-0.01418684\\
0.02120212	-0.01433694\\
0.02130213	-0.01448349\\
0.02140214	-0.01378539\\
0.02150215	-0.01319255\\
0.02160216	-0.01313312\\
0.02170217	-0.01385992\\
0.02180218	-0.01321704\\
0.02190219	-0.01310436\\
0.0220022	-0.01329737\\
0.02210221	-0.01265199\\
0.02220222	-0.01197205\\
0.02230223	-0.0118698\\
0.02240224	-0.01231787\\
0.02250225	-0.01173885\\
0.02260226	-0.01169175\\
0.02270227	-0.01185941\\
0.02280228	-0.0117783\\
0.02290229	-0.01201141\\
0.0230023	-0.01230409\\
0.02310231	-0.01238037\\
0.02320232	-0.01184208\\
0.02330233	-0.01185691\\
0.02340234	-0.0120895\\
0.02350235	-0.01176416\\
0.02360236	-0.01193758\\
0.02370237	-0.01185876\\
0.02380238	-0.01128476\\
0.02390239	-0.01200747\\
0.0240024	-0.01211972\\
0.02410241	-0.01217124\\
0.02420242	-0.01226636\\
0.02430243	-0.01224921\\
0.02440244	-0.01191926\\
0.02450245	-0.01190428\\
0.02460246	-0.01233165\\
0.02470247	-0.01245038\\
0.02480248	-0.01196744\\
0.02490249	-0.01184733\\
0.0250025	-0.01208723\\
0.02510251	-0.01159301\\
0.02520252	-0.01147872\\
0.02530253	-0.01118573\\
0.02540254	-0.01096636\\
0.02550255	-0.01047533\\
0.02560256	-0.01038058\\
0.02570257	-0.01037632\\
0.02580258	-0.01036965\\
0.02590259	-0.01047896\\
0.0260026	-0.01027745\\
0.02610261	-0.01090067\\
0.02620262	-0.01101785\\
0.02630263	-0.01068623\\
0.02640264	-0.01057648\\
0.02650265	-0.01076076\\
0.02660266	-0.01076977\\
0.02670267	-0.01055801\\
0.02680268	-0.01072351\\
0.02690269	-0.01018356\\
0.0270027	-0.01020656\\
0.02710271	-0.01000544\\
0.02720272	-0.009601827\\
0.02730273	-0.009584042\\
0.02740274	-0.009581976\\
0.02750275	-0.009736833\\
0.02760276	-0.01033626\\
0.02770277	-0.009997812\\
0.02780278	-0.009611788\\
0.02790279	-0.01011936\\
0.0280028	-0.01068211\\
0.02810281	-0.01017663\\
0.02820282	-0.01002172\\
0.02830283	-0.009922288\\
0.02840284	-0.009839875\\
0.02850285	-0.01002707\\
0.02860286	-0.009809572\\
0.02870287	-0.009605796\\
0.02880288	-0.009941631\\
0.02890289	-0.009701868\\
0.0290029	-0.009577669\\
0.02910291	-0.00954319\\
0.02920292	-0.009149161\\
0.02930293	-0.009879691\\
0.02940294	-0.009581105\\
0.02950295	-0.0093875\\
0.02960296	-0.009314919\\
0.02970297	-0.009365687\\
0.02980298	-0.008981108\\
0.02990299	-0.009002584\\
0.030003	-0.008668719\\
0.03010301	-0.008782852\\
0.03020302	-0.008990538\\
0.03030303	-0.009209409\\
0.03040304	-0.009155987\\
0.03050305	-0.009201231\\
0.03060306	-0.008981072\\
0.03070307	-0.009134416\\
0.03080308	-0.009125746\\
0.03090309	-0.008506506\\
0.0310031	-0.008647017\\
0.03110311	-0.008573618\\
0.03120312	-0.00841567\\
0.03130313	-0.008184082\\
0.03140314	-0.00796471\\
0.03150315	-0.00782538\\
0.03160316	-0.007801551\\
0.03170317	-0.007257149\\
0.03180318	-0.007601144\\
0.03190319	-0.007692192\\
0.0320032	-0.007754905\\
0.03210321	-0.008490003\\
0.03220322	-0.008390323\\
0.03230323	-0.008426989\\
0.03240324	-0.008650309\\
0.03250325	-0.008150308\\
0.03260326	-0.00862822\\
0.03270327	-0.008631255\\
0.03280328	-0.0088332\\
0.03290329	-0.008684659\\
0.0330033	-0.008555928\\
0.03310331	-0.008735619\\
0.03320332	-0.008732209\\
0.03330333	-0.008046671\\
0.03340334	-0.007917161\\
0.03350335	-0.007824257\\
0.03360336	-0.007737371\\
0.03370337	-0.008019023\\
0.03380338	-0.007685393\\
0.03390339	-0.007447921\\
0.0340034	-0.007173883\\
0.03410341	-0.006986045\\
0.03420342	-0.006991961\\
0.03430343	-0.00691781\\
0.03440344	-0.007182163\\
0.03450345	-0.007127342\\
0.03460346	-0.007268524\\
0.03470347	-0.007188024\\
0.03480348	-0.007007627\\
0.03490349	-0.007305441\\
0.0350035	-0.00744545\\
0.03510351	-0.007233016\\
0.03520352	-0.007477751\\
0.03530353	-0.00738384\\
0.03540354	-0.007183854\\
0.03550355	-0.006832617\\
0.03560356	-0.006798891\\
0.03570357	-0.007083925\\
0.03580358	-0.00652053\\
0.03590359	-0.006520855\\
0.0360036	-0.006728152\\
0.03610361	-0.006783296\\
0.03620362	-0.006660594\\
0.03630363	-0.00680416\\
0.03640364	-0.006986361\\
0.03650365	-0.007069497\\
0.03660366	-0.006912253\\
0.03670367	-0.007204746\\
0.03680368	-0.006947075\\
0.03690369	-0.00677217\\
0.0370037	-0.006517618\\
0.03710371	-0.006341969\\
0.03720372	-0.006378411\\
0.03730373	-0.006344911\\
0.03740374	-0.006591903\\
0.03750375	-0.006772859\\
0.03760376	-0.00686254\\
0.03770377	-0.006703892\\
0.03780378	-0.006414949\\
0.03790379	-0.00635618\\
0.0380038	-0.00631953\\
0.03810381	-0.006518432\\
0.03820382	-0.006450342\\
0.03830383	-0.006369636\\
0.03840384	-0.006477444\\
0.03850385	-0.006308992\\
0.03860386	-0.006166603\\
0.03870387	-0.006092823\\
0.03880388	-0.006379064\\
0.03890389	-0.006104841\\
0.0390039	-0.00614191\\
0.03910391	-0.005830091\\
0.03920392	-0.005623284\\
0.03930393	-0.005418884\\
0.03940394	-0.005247497\\
0.03950395	-0.005443117\\
0.03960396	-0.005417847\\
0.03970397	-0.00585313\\
0.03980398	-0.005511527\\
0.03990399	-0.005508006\\
0.040004	-0.005774232\\
0.04010401	-0.005645905\\
0.04020402	-0.005851713\\
0.04030403	-0.005834255\\
0.04040404	-0.005926942\\
0.04050405	-0.00570357\\
0.04060406	-0.005489222\\
0.04070407	-0.005346348\\
0.04080408	-0.005455909\\
0.04090409	-0.00538873\\
0.0410041	-0.005298873\\
0.04110411	-0.005276357\\
0.04120412	-0.005433797\\
0.04130413	-0.005808833\\
0.04140414	-0.005355826\\
0.04150415	-0.005776454\\
0.04160416	-0.005629201\\
0.04170417	-0.005896409\\
0.04180418	-0.005840726\\
0.04190419	-0.005834513\\
0.0420042	-0.005834301\\
0.04210421	-0.005499234\\
0.04220422	-0.005812607\\
0.04230423	-0.00554965\\
0.04240424	-0.005600339\\
0.04250425	-0.005243153\\
0.04260426	-0.00487147\\
0.04270427	-0.00456888\\
0.04280428	-0.004692562\\
0.04290429	-0.004165969\\
0.0430043	-0.004243096\\
0.04310431	-0.004443942\\
0.04320432	-0.00436417\\
0.04330433	-0.004573209\\
0.04340434	-0.004757453\\
0.04350435	-0.005045549\\
0.04360436	-0.004795178\\
0.04370437	-0.005276419\\
0.04380438	-0.004908096\\
0.04390439	-0.004719769\\
0.0440044	-0.004553008\\
0.04410441	-0.004461041\\
0.04420442	-0.004707908\\
0.04430443	-0.005034878\\
0.04440444	-0.004942825\\
0.04450445	-0.005094678\\
0.04460446	-0.004783427\\
0.04470447	-0.004703993\\
0.04480448	-0.004687545\\
0.04490449	-0.004632402\\
0.0450045	-0.004646111\\
0.04510451	-0.004729542\\
0.04520452	-0.004456228\\
0.04530453	-0.004382034\\
0.04540454	-0.004240763\\
0.04550455	-0.004239705\\
0.04560456	-0.004193251\\
0.04570457	-0.004158962\\
0.04580458	-0.004232914\\
0.04590459	-0.004123133\\
0.0460046	-0.004463883\\
0.04610461	-0.004266157\\
0.04620462	-0.004517729\\
0.04630463	-0.004678357\\
0.04640464	-0.004698267\\
0.04650465	-0.004877839\\
0.04660466	-0.004923176\\
0.04670467	-0.004709238\\
0.04680468	-0.004600252\\
0.04690469	-0.00440147\\
0.0470047	-0.004230036\\
0.04710471	-0.004059681\\
0.04720472	-0.003808869\\
0.04730473	-0.003552785\\
0.04740474	-0.003389714\\
0.04750475	-0.003472435\\
0.04760476	-0.003835628\\
0.04770477	-0.003966962\\
0.04780478	-0.00393837\\
0.04790479	-0.003950855\\
0.0480048	-0.003899902\\
0.04810481	-0.003840653\\
0.04820482	-0.003751005\\
0.04830483	-0.003753817\\
0.04840484	-0.003703659\\
0.04850485	-0.003787015\\
0.04860486	-0.003843523\\
0.04870487	-0.003756728\\
0.04880488	-0.003628575\\
0.04890489	-0.003634349\\
0.0490049	-0.003552841\\
0.04910491	-0.003679132\\
0.04920492	-0.003855274\\
0.04930493	-0.003780428\\
0.04940494	-0.003806095\\
0.04950495	-0.003832593\\
0.04960496	-0.003773936\\
0.04970497	-0.003708721\\
0.04980498	-0.003824191\\
0.04990499	-0.004005917\\
0.050005	-0.003876358\\
0.05010501	-0.003728915\\
0.05020502	-0.003794168\\
0.05030503	-0.003770764\\
0.05040504	-0.003668214\\
0.05050505	-0.003640178\\
0.05060506	-0.003782546\\
0.05070507	-0.003707704\\
0.05080508	-0.003566656\\
0.05090509	-0.003621068\\
0.0510051	-0.003439766\\
0.05110511	-0.003342905\\
0.05120512	-0.003102974\\
0.05130513	-0.00291753\\
0.05140514	-0.002845985\\
0.05150515	-0.003000869\\
0.05160516	-0.003121632\\
0.05170517	-0.003363007\\
0.05180518	-0.003297123\\
0.05190519	-0.003224094\\
0.0520052	-0.003052827\\
0.05210521	-0.003070491\\
0.05220522	-0.003157289\\
0.05230523	-0.003177816\\
0.05240524	-0.003123892\\
0.05250525	-0.003002546\\
0.05260526	-0.003105801\\
0.05270527	-0.003168233\\
0.05280528	-0.003116044\\
0.05290529	-0.00310376\\
0.0530053	-0.002859377\\
0.05310531	-0.003134774\\
0.05320532	-0.003103138\\
0.05330533	-0.003021663\\
0.05340534	-0.003211921\\
0.05350535	-0.003228424\\
0.05360536	-0.003255859\\
0.05370537	-0.003219102\\
0.05380538	-0.003427985\\
0.05390539	-0.003480507\\
0.0540054	-0.003281313\\
0.05410541	-0.003014299\\
0.05420542	-0.002632718\\
0.05430543	-0.002662939\\
0.05440544	-0.002635283\\
0.05450545	-0.002733913\\
0.05460546	-0.002704099\\
0.05470547	-0.002768946\\
0.05480548	-0.00273391\\
0.05490549	-0.002644037\\
0.0550055	-0.002843187\\
0.05510551	-0.002810305\\
0.05520552	-0.002834158\\
0.05530553	-0.002965283\\
0.05540554	-0.002952573\\
0.05550555	-0.003231828\\
0.05560556	-0.003229889\\
0.05570557	-0.003046842\\
0.05580558	-0.002803214\\
0.05590559	-0.002886716\\
0.0560056	-0.002699803\\
0.05610561	-0.00259251\\
0.05620562	-0.002497497\\
0.05630563	-0.00236598\\
0.05640564	-0.00234427\\
0.05650565	-0.002274305\\
0.05660566	-0.00224567\\
0.05670567	-0.002264785\\
0.05680568	-0.002259639\\
0.05690569	-0.002094986\\
0.0570057	-0.002118752\\
0.05710571	-0.002234975\\
0.05720572	-0.002504049\\
0.05730573	-0.002664596\\
0.05740574	-0.002505782\\
0.05750575	-0.00250284\\
0.05760576	-0.002670157\\
0.05770577	-0.002863322\\
0.05780578	-0.002758817\\
0.05790579	-0.002728493\\
0.0580058	-0.002748634\\
0.05810581	-0.002815257\\
0.05820582	-0.002671019\\
0.05830583	-0.002550152\\
0.05840584	-0.002488231\\
0.05850585	-0.002283651\\
0.05860586	-0.002256381\\
0.05870587	-0.002389128\\
0.05880588	-0.0023761\\
0.05890589	-0.002368781\\
0.0590059	-0.002382495\\
0.05910591	-0.002372575\\
0.05920592	-0.002410086\\
0.05930593	-0.002426543\\
0.05940594	-0.002248589\\
0.05950595	-0.002155983\\
0.05960596	-0.001984698\\
0.05970597	-0.001998551\\
0.05980598	-0.002206848\\
0.05990599	-0.002199414\\
0.060006	-0.002093164\\
0.06010601	-0.002218783\\
0.06020602	-0.002242471\\
0.06030603	-0.002223206\\
0.06040604	-0.002018011\\
0.06050605	-0.00193467\\
0.06060606	-0.001796589\\
0.06070607	-0.001996639\\
0.06080608	-0.002041163\\
0.06090609	-0.001991367\\
0.0610061	-0.001959344\\
0.06110611	-0.001932947\\
0.06120612	-0.002032873\\
0.06130613	-0.001997857\\
0.06140614	-0.001978261\\
0.06150615	-0.002059095\\
0.06160616	-0.001963452\\
0.06170617	-0.00207171\\
0.06180618	-0.002361791\\
0.06190619	-0.002163765\\
0.0620062	-0.002104117\\
0.06210621	-0.002106156\\
0.06220622	-0.002053985\\
0.06230623	-0.002076208\\
0.06240624	-0.002171138\\
0.06250625	-0.002168706\\
0.06260626	-0.002083409\\
0.06270627	-0.001992857\\
0.06280628	-0.001830893\\
0.06290629	-0.001701871\\
0.0630063	-0.001644819\\
0.06310631	-0.001581419\\
0.06320632	-0.001784408\\
0.06330633	-0.001803032\\
0.06340634	-0.001932399\\
0.06350635	-0.001920145\\
0.06360636	-0.00196842\\
0.06370637	-0.002053729\\
0.06380638	-0.002240904\\
0.06390639	-0.002041307\\
0.0640064	-0.001892587\\
0.06410641	-0.001637451\\
0.06420642	-0.001718805\\
0.06430643	-0.001781795\\
0.06440644	-0.001669173\\
0.06450645	-0.001589179\\
0.06460646	-0.00156879\\
0.06470647	-0.00158998\\
0.06480648	-0.001616001\\
0.06490649	-0.001575271\\
0.0650065	-0.001712378\\
0.06510651	-0.001759014\\
0.06520652	-0.001742739\\
0.06530653	-0.001651579\\
0.06540654	-0.001536079\\
0.06550655	-0.001455519\\
0.06560656	-0.001484478\\
0.06570657	-0.001442868\\
0.06580658	-0.001561539\\
0.06590659	-0.001443403\\
0.0660066	-0.001502358\\
0.06610661	-0.001558735\\
0.06620662	-0.001633075\\
0.06630663	-0.00157995\\
0.06640664	-0.001534228\\
0.06650665	-0.00167251\\
0.06660666	-0.001778973\\
0.06670667	-0.001838002\\
0.06680668	-0.001932532\\
0.06690669	-0.001795086\\
0.0670067	-0.001806466\\
0.06710671	-0.001764944\\
0.06720672	-0.001772846\\
0.06730673	-0.001680197\\
0.06740674	-0.001709618\\
0.06750675	-0.001740619\\
0.06760676	-0.001569594\\
0.06770677	-0.001394041\\
0.06780678	-0.001328509\\
0.06790679	-0.001496255\\
0.0680068	-0.0014177\\
0.06810681	-0.001411976\\
0.06820682	-0.001284749\\
0.06830683	-0.001248329\\
0.06840684	-0.001201505\\
0.06850685	-0.001197113\\
0.06860686	-0.001067194\\
0.06870687	-0.000980987399999999\\
0.06880688	-0.001180799\\
0.06890689	-0.001501283\\
0.0690069	-0.001503094\\
0.06910691	-0.00130136\\
0.06920692	-0.001515732\\
0.06930693	-0.001614412\\
0.06940694	-0.001653411\\
0.06950695	-0.001532822\\
0.06960696	-0.00156897\\
0.06970697	-0.001594978\\
0.06980698	-0.001327413\\
0.06990699	-0.001117064\\
0.070007	-0.0010896\\
0.07010701	-0.001151534\\
0.07020702	-0.00109173\\
0.07030703	-0.001236969\\
0.07040704	-0.001220059\\
0.07050705	-0.001164747\\
0.07060706	-0.001343049\\
0.07070707	-0.001316647\\
0.07080708	-0.001350395\\
0.07090709	-0.001350037\\
0.0710071	-0.001304153\\
0.07110711	-0.00140101\\
0.07120712	-0.001492532\\
0.07130713	-0.001541976\\
0.07140714	-0.001369302\\
0.07150715	-0.001307207\\
0.07160716	-0.001307626\\
0.07170717	-0.001320234\\
0.07180718	-0.001130304\\
0.07190719	-0.00115291\\
0.0720072	-0.001065816\\
0.07210721	-0.001136576\\
0.07220722	-0.001127326\\
0.07230723	-0.001103261\\
0.07240724	-0.001045486\\
0.07250725	-0.001078626\\
0.07260726	-0.001156773\\
0.07270727	-0.001209118\\
0.07280728	-0.001127894\\
0.07290729	-0.001183338\\
0.0730073	-0.001161001\\
0.07310731	-0.001101548\\
0.07320732	-0.001119629\\
0.07330733	-0.001125708\\
0.07340734	-0.001116986\\
0.07350735	-0.001023456\\
0.07360736	-0.001008934\\
0.07370737	-0.000964922499999998\\
0.07380738	-0.000978741799999998\\
0.07390739	-0.000936925699999998\\
0.0740074	-0.000954920999999998\\
0.07410741	-0.000973092199999998\\
0.07420742	-0.001053994\\
0.07430743	-0.001048891\\
0.07440744	-0.000934988099999998\\
0.07450745	-0.000946213699999998\\
0.07460746	-0.001014707\\
0.07470747	-0.001047297\\
0.07480748	-0.001070986\\
0.07490749	-0.001058692\\
0.0750075	-0.001104565\\
0.07510751	-0.001143096\\
0.07520752	-0.001010357\\
0.07530753	-0.000988448399999997\\
0.07540754	-0.001141689\\
0.07550755	-0.001244522\\
0.07560756	-0.001184422\\
0.07570757	-0.001071182\\
0.07580758	-0.001222946\\
0.07590759	-0.001316661\\
0.0760076	-0.001161674\\
0.07610761	-0.000876807299999997\\
0.07620762	-0.000895302399999997\\
0.07630763	-0.001093369\\
0.07640764	-0.001124688\\
0.07650765	-0.000900048799999997\\
0.07660766	-0.000805282299999997\\
0.07670767	-0.000767084799999997\\
0.07680768	-0.000688884599999997\\
0.07690769	-0.000661270699999997\\
0.0770077	-0.000736188299999997\\
0.07710771	-0.000791944799999997\\
0.07720772	-0.000702578699999997\\
0.07730773	-0.000780164999999997\\
0.07740774	-0.000842175199999997\\
0.07750775	-0.000884048999999997\\
0.07760776	-0.000961268199999997\\
0.07770777	-0.000939399099999997\\
0.07780778	-0.000860493999999996\\
0.07790779	-0.000883882199999996\\
0.0780078	-0.000909070999999997\\
0.07810781	-0.000871065499999996\\
0.07820782	-0.000855888499999996\\
0.07830783	-0.000884240699999996\\
0.07840784	-0.000843974299999996\\
0.07850785	-0.000920367099999996\\
0.07860786	-0.000793043299999996\\
0.07870787	-0.000778033899999996\\
0.07880788	-0.000829930799999996\\
0.07890789	-0.000849611499999996\\
0.0790079	-0.000914925999999996\\
0.07910791	-0.000997758899999996\\
0.07920792	-0.000997392299999996\\
0.07930793	-0.000849889799999996\\
0.07940794	-0.000767311999999996\\
0.07950795	-0.000810473299999996\\
0.07960796	-0.000785589899999996\\
0.07970797	-0.000805859299999996\\
0.07980798	-0.000700929999999996\\
0.07990799	-0.000731750999999996\\
0.080008	-0.000755588499999996\\
0.08010801	-0.000649774099999995\\
0.08020802	-0.000643429899999995\\
0.08030803	-0.000794699799999995\\
0.08040804	-0.000849673999999995\\
0.08050805	-0.000889905799999995\\
0.08060806	-0.000895559799999995\\
0.08070807	-0.000899889599999995\\
0.08080808	-0.000951928999999995\\
0.08090809	-0.000842600599999995\\
0.0810081	-0.000762610099999995\\
0.08110811	-0.000762639999999995\\
0.08120812	-0.000794196299999995\\
0.08130813	-0.000744996899999995\\
0.08140814	-0.000765077899999995\\
0.08150815	-0.000668528099999995\\
0.08160816	-0.000653823699999995\\
0.08170817	-0.000603651399999994\\
0.08180818	-0.000519615999999994\\
0.08190819	-0.000502757399999994\\
0.0820082	-0.000499642599999994\\
0.08210821	-0.000450549799999994\\
0.08220822	-0.000545712999999994\\
0.08230823	-0.000669167099999994\\
0.08240824	-0.000686952499999994\\
0.08250825	-0.000714533799999994\\
0.08260826	-0.000758586499999994\\
0.08270827	-0.000780002999999994\\
0.08280828	-0.000765922099999994\\
0.08290829	-0.000763946799999994\\
0.0830083	-0.000650905399999994\\
0.08310831	-0.000688230999999993\\
0.08320832	-0.000754123499999993\\
0.08330833	-0.000574589899999993\\
0.08340834	-0.000613252099999993\\
0.08350835	-0.000645951399999993\\
0.08360836	-0.000715158399999993\\
0.08370837	-0.000664543899999993\\
0.08380838	-0.000694367399999993\\
0.08390839	-0.000757603899999993\\
0.0840084	-0.000694154199999993\\
0.08410841	-0.000654995399999993\\
0.08420842	-0.000645114999999993\\
0.08430843	-0.000669339999999993\\
0.08440844	-0.000627743199999992\\
0.08450845	-0.000604213999999992\\
0.08460846	-0.000658197999999992\\
0.08470847	-0.000723171099999992\\
0.08480848	-0.000632124999999992\\
0.08490849	-0.000604712199999992\\
0.0850085	-0.000683353999999992\\
0.08510851	-0.000801312199999992\\
0.08520852	-0.000746103499999992\\
0.08530853	-0.000615095499999992\\
0.08540854	-0.000549116499999991\\
0.08550855	-0.000574119999999991\\
0.08560856	-0.000389271699999991\\
0.08570857	-0.000320023099999991\\
0.08580858	-0.000346716799999991\\
0.08590859	-0.000334506699999991\\
0.0860086	-0.000412789199999991\\
0.08610861	-0.000473252999999991\\
0.08620862	-0.000514636899999991\\
0.08630863	-0.000528922999999991\\
0.08640864	-0.00057735869999999\\
0.08650865	-0.00056688709999999\\
0.08660866	-0.00059517299999999\\
0.08670867	-0.00054975109999999\\
0.08680868	-0.00057839959999999\\
0.08690869	-0.00074769529999999\\
0.0870087	-0.00078451949999999\\
0.08710871	-0.000610535999999989\\
0.08720872	-0.000540275799999989\\
0.08730873	-0.000605273699999989\\
0.08740874	-0.000560394299999989\\
0.08750875	-0.000460169199999989\\
0.08760876	-0.000487618599999989\\
0.08770877	-0.000517941499999989\\
0.08780878	-0.000574943599999989\\
0.08790879	-0.000472120399999988\\
0.0880088	-0.000512062599999988\\
0.08810881	-0.000504429499999988\\
0.08820882	-0.000484691199999988\\
0.08830883	-0.000432157299999988\\
0.08840884	-0.000506983499999988\\
0.08850885	-0.000485422499999988\\
0.08860886	-0.000559325899999987\\
0.08870887	-0.000506729799999987\\
0.08880888	-0.000404234799999987\\
0.08890889	-0.000434046799999987\\
0.0890089	-0.000464499499999987\\
0.08910891	-0.000501015999999987\\
0.08920892	-0.000561364699999987\\
0.08930893	-0.000552088999999986\\
0.08940894	-0.000508383199999986\\
0.08950895	-0.000585829999999986\\
0.08960896	-0.000524333699999986\\
0.08970897	-0.000477215199999986\\
0.08980898	-0.000488897699999986\\
0.08990899	-0.000542707199999985\\
0.090009	-0.000431543299999985\\
0.09010901	-0.000358881199999985\\
0.09020902	-0.000357511299999985\\
0.09030903	-0.000390627199999985\\
0.09040904	-0.000327176699999985\\
0.09050905	-0.000351429399999984\\
0.09060906	-0.000402220599999984\\
0.09070907	-0.000420089599999984\\
0.09080908	-0.000403177299999984\\
0.09090909	-0.000393064299999984\\
0.0910091	-0.000316262199999983\\
0.09110911	-0.000328484599999983\\
0.09120912	-0.000338381499999983\\
0.09130913	-0.000417184199999983\\
0.09140914	-0.000437807999999983\\
0.09150915	-0.000338273499999982\\
0.09160916	-0.000405341299999982\\
0.09170917	-0.000445589499999982\\
0.09180918	-0.000471398999999982\\
0.09190919	-0.000461660299999981\\
0.0920092	-0.000492752799999981\\
0.09210921	-0.000589798199999981\\
0.09220922	-0.000551627099999981\\
0.09230923	-0.000477046499999981\\
0.09240924	-0.00048450909999998\\
0.09250925	-0.00045349399999998\\
0.09260926	-0.00038599219999998\\
0.09270927	-0.00039304859999998\\
0.09280928	-0.000400815199999979\\
0.09290929	-0.000391312599999979\\
0.0930093	-0.000471921799999979\\
0.09310931	-0.000504967199999979\\
0.09320932	-0.000511308699999978\\
0.09330933	-0.000455862899999978\\
0.09340934	-0.000352641899999978\\
0.09350935	-0.000304071999999978\\
0.09360936	-0.000249135099999977\\
0.09370937	-0.000254188799999977\\
0.09380938	-0.000278742499999977\\
0.09390939	-0.000292095099999977\\
0.0940094	-0.000317120199999976\\
0.09410941	-0.000300393699999976\\
0.09420942	-0.000324009199999976\\
0.09430943	-0.000342047099999975\\
0.09440944	-0.000286704799999975\\
0.09450945	-0.000324734899999975\\
0.09460946	-0.000330726499999974\\
0.09470947	-0.000359702199999974\\
0.09480948	-0.000368901899999974\\
0.09490949	-0.000370165799999974\\
0.0950095	-0.000397452899999973\\
0.09510951	-0.000359666299999973\\
0.09520952	-0.000287446599999973\\
0.09530953	-0.000300448599999972\\
0.09540954	-0.000345746299999972\\
0.09550955	-0.000294348899999972\\
0.09560956	-0.000264951199999971\\
0.09570957	-0.000219384899999971\\
0.09580958	-0.000242823299999971\\
0.09590959	-0.00030076569999997\\
0.0960096	-0.00037118299999997\\
0.09610961	-0.00041308049999997\\
0.09620962	-0.000487023699999969\\
0.09630963	-0.000463071599999969\\
0.09640964	-0.000367203099999968\\
0.09650965	-0.000455227099999968\\
0.09660966	-0.000567554099999968\\
0.09670967	-0.000426608099999967\\
0.09680968	-0.000303039799999967\\
0.09690969	-0.000297006499999967\\
0.0970097	-0.000271272899999966\\
0.09710971	-0.000231864899999966\\
0.09720972	-0.000249922099999965\\
0.09730973	-0.000283443099999965\\
0.09740974	-0.000226264999999964\\
0.09750975	-0.000208653999999964\\
0.09760976	-0.000265843699999964\\
0.09770977	-0.000243082199999963\\
0.09780978	-0.000247421399999963\\
0.09790979	-0.000285602399999962\\
0.0980098	-0.000325763099999962\\
0.09810981	-0.000356294699999961\\
0.09820982	-0.000389419499999961\\
0.09830983	-0.000386925399999961\\
0.09840984	-0.00033333949999996\\
0.09850985	-0.00033491049999996\\
0.09860986	-0.000333166199999959\\
0.09870987	-0.000295002199999959\\
0.09880988	-0.000252858899999958\\
0.09890989	-0.000305548399999958\\
0.0990099	-0.000291784099999957\\
0.09910991	-0.000164562799999957\\
0.09920992	-0.000144160499999956\\
0.09930993	-0.000215975099999956\\
0.09940994	-0.000224650199999955\\
0.09950995	-0.000237585899999955\\
0.09960996	-0.000174234999999954\\
0.09970997	-0.000147127999999954\\
0.09980998	-0.000151642799999953\\
0.09990999	-0.000182979899999952\\
0.10001	-0.000245780899999952\\
0.10011	-0.000241065799999951\\
0.10021	-0.000244387999999951\\
0.10031	-0.00034047089999995\\
0.10041	-0.00038706529999995\\
0.1005101	-0.000375771499999949\\
0.1006101	-0.000372732999999948\\
0.1007101	-0.000368837799999948\\
0.1008101	-0.000323681499999947\\
0.1009101	-0.000304734499999946\\
0.1010101	-0.000308902499999946\\
0.1011101	-0.000304602499999945\\
0.1012101	-0.000290006399999945\\
0.1013101	-0.000312734899999944\\
0.1014101	-0.000277059299999943\\
0.1015102	-0.000247953299999943\\
0.1016102	-0.000297407299999942\\
0.1017102	-0.000218309399999941\\
0.1018102	-0.000174476099999941\\
0.1019102	-0.00023406729999994\\
0.1020102	-0.000186904199999939\\
0.1021102	-0.000225974099999938\\
0.1022102	-0.000242432499999938\\
0.1023102	-0.000178723299999937\\
0.1024102	-0.000172344699999936\\
0.1025103	-0.000171768999999935\\
0.1026103	-0.000215130899999935\\
0.1027103	-0.000268839499999934\\
0.1028103	-0.000199716399999933\\
0.1029103	-0.000161406299999932\\
0.1030103	-0.000126309199999932\\
0.1031103	-0.000114893299999931\\
0.1032103	-0.00013310779999993\\
0.1033103	-0.000176413399999929\\
0.1034103	-0.000178390099999928\\
0.1035104	-0.000237134499999928\\
0.1036104	-0.000246940199999927\\
0.1037104	-0.000234116099999926\\
0.1038104	-0.000280893899999925\\
0.1039104	-0.000262440899999924\\
0.1040104	-0.000263396699999923\\
0.1041104	-0.000211064999999922\\
0.1042104	-0.000269336099999921\\
0.1043104	-0.00024745479999992\\
0.1044104	-0.000238214399999919\\
0.1045105	-0.000197202799999919\\
0.1046105	-0.000154755599999918\\
0.1047105	-0.000133947699999917\\
0.1048105	-0.000114350099999916\\
0.1049105	-0.000128797599999915\\
0.1050105	-0.000204935399999914\\
0.1051105	-0.000232571199999913\\
0.1052105	-0.000259473199999912\\
0.1053105	-0.000304934199999911\\
0.1054105	-0.00028548399999991\\
0.1055106	-0.000240736899999909\\
0.1056106	-0.000223774399999907\\
0.1057106	-0.000247704899999906\\
0.1058106	-0.000161148399999905\\
0.1059106	-0.000145265299999904\\
0.1060106	-0.000131222099999903\\
0.1061106	-0.000151343099999902\\
0.1062106	-0.000152872299999901\\
0.1063106	-0.0001794422999999\\
0.1064106	-0.000223806799999898\\
0.1065107	-0.000223995099999897\\
0.1066107	-0.000219053299999896\\
0.1067107	-0.000212601399999895\\
0.1068107	-0.000151327799999894\\
0.1069107	-0.000144773499999892\\
0.1070107	-0.000174889699999891\\
0.1071107	-0.00016400819999989\\
0.1072107	-0.000157234699999889\\
0.1073107	-0.000115116699999887\\
0.1074107	-0.000115507099999886\\
0.1075108	-0.000116596199999885\\
0.1076108	-0.000181257599999883\\
0.1077108	-0.000132288599999882\\
0.1078108	-0.000137873299999881\\
0.1079108	-0.000152722699999879\\
0.1080108	-0.000150272899999878\\
0.1081108	-0.000174508499999876\\
0.1082108	-0.000163434099999875\\
0.1083108	-0.000128839399999873\\
0.1084108	-0.000138973699999872\\
0.1085109	-0.00016748969999987\\
0.1086109	-0.000185364599999869\\
0.1087109	-0.000121129099999867\\
0.1088109	-0.000155364699999866\\
0.1089109	-0.000155171399999864\\
0.1090109	-0.000111063699999863\\
0.1091109	-0.000122604899999861\\
0.1092109	-0.00014166959999986\\
0.1093109	-0.000145556899999858\\
0.1094109	-0.000133965099999856\\
0.109511	-0.000217247299999855\\
0.109611	-0.000257556999999853\\
0.109711	-0.000236833899999851\\
0.109811	-0.00024534659999985\\
0.109911	-0.000209871299999848\\
0.110011	-0.000179586999999846\\
0.110111	-0.000231886699999844\\
0.110211	-0.000215815099999842\\
0.110311	-0.000247033699999841\\
0.110411	-0.000196356299999839\\
0.1105111	-0.000183045599999837\\
0.1106111	-0.000138823899999835\\
0.1107111	-0.000159059799999833\\
0.1108111	-0.000117962799999831\\
0.1109111	-7.62803299998291e-05\\
0.1110111	-6.83569299998272e-05\\
0.1111111	-6.96941499998251e-05\\
0.1112111	-7.30874699998231e-05\\
0.1113111	-7.36980199998211e-05\\
0.1114111	-3.4981439999819e-05\\
0.1115112	-4.99493099998169e-05\\
0.1116112	-8.53323799998148e-05\\
0.1117112	-9.91395699998126e-05\\
0.1118112	-0.00014695899999981\\
0.1119112	-0.000183969499999808\\
0.1120112	-0.000183982599999806\\
0.1121112	-0.000201105099999804\\
0.1122112	-0.000258687399999802\\
0.1123112	-0.000224950099999799\\
0.1124112	-0.000236451799999797\\
0.1125113	-0.000159654699999795\\
0.1126113	-0.000115463699999792\\
0.1127113	-0.00012440119999979\\
0.1128113	-0.000105760199999787\\
0.1129113	-8.18118499997849e-05\\
0.1130113	-8.12522399997824e-05\\
0.1131113	-9.81588299997799e-05\\
0.1132113	-0.000100112299999777\\
0.1133113	-0.000116772699999775\\
0.1134113	-0.000136269899999772\\
0.1135114	-0.00013086739999977\\
0.1136114	-0.000153605099999767\\
0.1137114	-0.000173165899999764\\
0.1138114	-0.000131684299999761\\
0.1139114	-9.50654599997588e-05\\
0.1140114	-0.000138346699999756\\
0.1141114	-0.000141187799999753\\
0.1142114	-0.00012867279999975\\
0.1143114	-0.000103763999999747\\
0.1144114	-8.33665799997445e-05\\
0.1145115	-0.000124569299999742\\
0.1146115	-0.000159925099999739\\
0.1147115	-0.000105695899999736\\
0.1148115	-0.000122067299999733\\
0.1149115	-0.000100220199999729\\
0.1150115	-0.000100591999999726\\
0.1151115	-0.000115212999999723\\
0.1152115	-0.00014886769999972\\
0.1153115	-0.000132054699999717\\
0.1154115	-9.31299399997135e-05\\
0.1155116	-8.78568299997102e-05\\
0.1156116	-7.79884399997069e-05\\
0.1157116	-0.000139601799999704\\
0.1158116	-0.0001068271999997\\
0.1159116	-7.27083299996967e-05\\
0.1160116	-5.63893799996932e-05\\
0.1161116	-5.25347899996897e-05\\
0.1162116	-6.34417899996861e-05\\
0.1163116	-7.76815899996825e-05\\
0.1164116	-6.66326299996788e-05\\
0.1165117	-8.90719099996751e-05\\
0.1166117	-0.000124751599999671\\
0.1167117	-0.000121513399999668\\
0.1168117	-9.00875999996638e-05\\
0.1169117	-9.27184599996599e-05\\
0.1170117	-8.0125729999656e-05\\
0.1171117	-5.77874399996521e-05\\
0.1172117	-5.85158499996481e-05\\
0.1173117	-9.07932699996441e-05\\
0.1174117	-9.291803999964e-05\\
0.1175118	-8.91464699996358e-05\\
0.1176118	-9.15694099996317e-05\\
0.1177118	-0.000154832499999627\\
0.1178118	-0.000148415299999623\\
0.1179118	-0.000124549599999619\\
0.1180118	-0.000164198299999614\\
0.1181118	-0.00011771919999961\\
0.1182118	-0.000105011199999606\\
0.1183118	-0.000130315999999601\\
0.1184118	-0.000126777999999597\\
0.1185119	-0.000119288599999592\\
0.1186119	-0.000122680399999587\\
0.1187119	-7.96257399995825e-05\\
0.1188119	-5.61005899995777e-05\\
0.1189119	-6.02873699995729e-05\\
0.1190119	-9.5409159999568e-05\\
0.1191119	-0.000117445899999563\\
0.1192119	-0.000121044199999558\\
0.1193119	-8.8925029999553e-05\\
0.1194119	-0.000100340299999548\\
0.119512	-8.12599899995427e-05\\
0.119612	-6.45151999995375e-05\\
0.119712	-6.21784899995322e-05\\
0.119812	-5.53413199995268e-05\\
0.119912	-5.65168699995214e-05\\
0.120012	-7.24988099995159e-05\\
0.120112	-5.67019099995104e-05\\
0.120212	-4.00654399995048e-05\\
0.120312	-3.40591499994992e-05\\
0.120412	-2.78883299994934e-05\\
0.1205121	-4.38560099994876e-05\\
0.1206121	-6.53711299994818e-05\\
0.1207121	-5.26449199994759e-05\\
0.1208121	-6.85967399994699e-05\\
0.1209121	-9.77895099994639e-05\\
0.1210121	-8.88783999994577e-05\\
0.1211121	-0.000141080399999452\\
0.1212121	-0.000122799699999445\\
0.1213121	-0.000129442199999439\\
0.1214121	-0.000140143099999433\\
0.1215122	-0.000119049499999426\\
0.1216122	-8.74775399994195e-05\\
0.1217122	-0.000103801699999413\\
0.1218122	-7.43546999994062e-05\\
0.1219122	-7.82495099993995e-05\\
0.1220122	-8.58708199993926e-05\\
0.1221122	-9.53953499993857e-05\\
0.1222122	-8.09086499993787e-05\\
0.1223122	-4.35535899993716e-05\\
0.1224122	-5.53205899993645e-05\\
0.1225123	-4.47674199993572e-05\\
0.1226123	-4.70721899993499e-05\\
0.1227123	-9.11931199993425e-05\\
0.1228123	-8.7534549999335e-05\\
0.1229123	-7.37593199993275e-05\\
0.1230123	-6.56893199993198e-05\\
0.1231123	-5.54479099993121e-05\\
0.1232123	-4.74816199993043e-05\\
0.1233123	-5.44132799992964e-05\\
0.1234123	-3.71195199992884e-05\\
0.1235124	-2.24948999992803e-05\\
0.1236124	-7.37428799992721e-05\\
0.1237124	-0.000122292299999264\\
0.1238124	-0.000108917799999255\\
0.1239124	-0.000150671799999247\\
0.1240124	-0.000103668999999238\\
0.1241124	-0.00010518229999923\\
0.1242124	-0.000114502999999221\\
0.1243124	-0.000116266599999212\\
0.1244124	-8.38953299992033e-05\\
0.1245125	-4.86196999991942e-05\\
0.1246125	-3.23195799991851e-05\\
0.1247125	-4.00518699991758e-05\\
0.1248125	-1.81723999991665e-05\\
0.1249125	-1.36283399915703e-06\\
0.1250125	-1.01725899991475e-05\\
0.1251125	-1.95994899991378e-05\\
0.1252125	-6.0528619999128e-05\\
0.1253125	-6.03696199991181e-05\\
0.1254125	-6.47305699991081e-05\\
0.1255126	-8.0412319999098e-05\\
0.1256126	-8.23067399990878e-05\\
0.1257126	-7.83495799990775e-05\\
0.1258126	-8.3037219999067e-05\\
0.1259126	-9.38620999990565e-05\\
0.1260126	-0.000115279099999046\\
0.1261126	-0.000104096299999035\\
0.1262126	-5.55931099990241e-05\\
0.1263126	-6.4240339999013e-05\\
0.1264126	-7.40427499990019e-05\\
0.1265127	-6.49528199989906e-05\\
0.1266127	-7.15488799989791e-05\\
0.1267127	-4.66489199989676e-05\\
0.1268127	-6.48214499989559e-05\\
0.1269127	-8.54000699989441e-05\\
0.1270127	-6.51572399989322e-05\\
0.1271127	-3.74518099989201e-05\\
0.1272127	-3.98425499989079e-05\\
0.1273127	-5.84813499988956e-05\\
0.1274127	-6.78754299988831e-05\\
0.1275128	-6.27134099988704e-05\\
0.1276128	-4.95530099988577e-05\\
0.1277128	-5.49333799988448e-05\\
0.1278128	-8.12027099988318e-05\\
0.1279128	-8.38616799988186e-05\\
0.1280128	-8.24724399988052e-05\\
0.1281128	-5.94697999987917e-05\\
0.1282128	-4.92698499987781e-05\\
0.1283128	-3.04177799987643e-05\\
0.1284128	-1.80596799987504e-05\\
0.1285129	-2.60900299987363e-05\\
0.1286129	-1.6346109998722e-05\\
0.1287129	-1.73041799987076e-05\\
0.1288129	-1.88454999986931e-05\\
0.1289129	-8.54036399867833e-06\\
0.1290129	-2.12315399986634e-05\\
0.1291129	-4.45541999986484e-05\\
0.1292129	-3.98700499986332e-05\\
0.1293129	-5.48768999986178e-05\\
0.1294129	-6.89634999986022e-05\\
0.129513	-9.10484299985864e-05\\
0.129613	-9.90187899985705e-05\\
0.129713	-2.94411099985544e-05\\
0.129813	-5.65992199985382e-05\\
0.129913	-4.89370699985217e-05\\
0.130013	-5.56840799985051e-05\\
0.130113	-5.12653799984883e-05\\
0.130213	-3.88262899984713e-05\\
0.130313	-6.27158999984541e-05\\
0.130413	-4.48528899984367e-05\\
0.1305131	-3.43063799984191e-05\\
0.1306131	-4.63227199984013e-05\\
0.1307131	-3.64051699983833e-05\\
0.1308131	-5.27783699983652e-05\\
0.1309131	-7.66033299983468e-05\\
0.1310131	-6.00898799983282e-05\\
0.1311131	-5.16341999983095e-05\\
0.1312131	-4.94680699982905e-05\\
0.1313131	-4.28745199982713e-05\\
0.1314131	-7.10909899982519e-05\\
0.1315132	-8.26331199982322e-05\\
0.1316132	-2.99929199982124e-05\\
0.1317132	-2.47931399981923e-05\\
0.1318132	-1.55233399981721e-05\\
0.1319132	-1.24897999981516e-05\\
0.1320132	-7.67255099813082e-06\\
0.1321132	-2.24611499981099e-05\\
0.1322132	-3.73087499980887e-05\\
0.1323132	-6.98309699980672e-05\\
0.1324132	-6.66578899980456e-05\\
0.1325133	-4.50400799980236e-05\\
0.1326133	-6.30540599980015e-05\\
0.1327133	-6.58073199979791e-05\\
0.1328133	-4.49788699979565e-05\\
0.1329133	-3.90870599979336e-05\\
0.1330133	-3.28298199979104e-05\\
0.1331133	-2.6336379997887e-05\\
0.1332133	-2.92359799978634e-05\\
0.1333133	-2.28159699978395e-05\\
0.1334133	-2.93260899978153e-05\\
0.1335134	-5.27703199977908e-05\\
0.1336134	-4.72490499977661e-05\\
0.1337134	-1.66551499977411e-05\\
0.1338134	-9.32522299771587e-06\\
0.1339134	-1.65813799976903e-05\\
0.1340134	-2.14658299976645e-05\\
0.1341134	-9.56687199763839e-06\\
0.1342134	-1.5023969997612e-05\\
0.1343134	-2.19448999975853e-05\\
0.1344134	-3.83905799975583e-05\\
0.1345135	-1.5957699997531e-05\\
0.1346135	-3.06084799975035e-05\\
0.1347135	-2.18401399974756e-05\\
0.1348135	-6.43872499974474e-05\\
0.1349135	-0.000112056999997419\\
0.1350135	-9.80611299973901e-05\\
0.1351135	-7.30027899973609e-05\\
0.1352135	-9.67276699973315e-05\\
0.1353135	-8.99457399973017e-05\\
0.1354135	-5.97872099972716e-05\\
0.1355136	-5.98129499972411e-05\\
0.1356136	-6.73379999972104e-05\\
0.1357136	-9.15756299971793e-05\\
0.1358136	-8.23908199971478e-05\\
0.1359136	-4.1571329997116e-05\\
0.1360136	-3.10209999970839e-05\\
0.1361136	-1.78479299970514e-05\\
0.1362136	-1.24876499970185e-05\\
0.1363136	-8.55203399698533e-06\\
0.1364136	-1.68749199969518e-05\\
0.1365137	-2.33750499969178e-05\\
0.1366137	-8.35547399688345e-06\\
0.1367137	-1.53942399968487e-05\\
0.1368137	-9.07094499681367e-06\\
0.1369137	-1.93899599967782e-05\\
0.1370137	-3.36834199967424e-05\\
0.1371137	-1.28398999967061e-05\\
0.1372137	-3.95804499966695e-05\\
0.1373137	-1.71960599966324e-05\\
0.1374137	-1.7032349996595e-05\\
0.1375138	-4.37321199965571e-05\\
0.1376138	-1.97896199965188e-05\\
0.1377138	-2.46842199964801e-05\\
0.1378138	-1.8563899996441e-05\\
0.1379138	-2.83466199964015e-05\\
0.1380138	-2.67965299963615e-05\\
0.1381138	-4.15719099963211e-05\\
0.1382138	-4.69348999962802e-05\\
0.1383138	-2.15459899962389e-05\\
0.1384138	-1.45482699961971e-05\\
0.1385139	-2.93287699961549e-05\\
0.1386139	-2.51048999961122e-05\\
0.1387139	-2.20268899960691e-05\\
0.1388139	-1.16650999960254e-05\\
0.1389139	-3.08179899959813e-05\\
0.1390139	-4.44462799959368e-05\\
0.1391139	-1.57859499958917e-05\\
0.1392139	-3.03149099958461e-05\\
0.1393139	-4.27324199958001e-05\\
0.1394139	-6.08422099957535e-05\\
0.139514	-7.16271999957064e-05\\
0.139614	-4.37083199956588e-05\\
0.139714	-5.76539299956107e-05\\
0.139814	-4.43694299955621e-05\\
0.139914	-4.2332889995513e-05\\
0.140014	-3.95506599954633e-05\\
0.140114	-1.6615199995413e-05\\
0.140214	-3.26368699953623e-05\\
0.140314	-1.86767699953109e-05\\
0.140414	-1.6084029995259e-05\\
0.1405141	-1.88218999952065e-05\\
0.1406141	-1.05017699951535e-05\\
0.1407141	-9.31962199509986e-06\\
0.1408141	-1.43513499950457e-05\\
0.1409141	-1.74984199949909e-05\\
0.1410141	-3.23271199949355e-05\\
0.1411141	-3.46169899948795e-05\\
0.1412141	-2.83813599948229e-05\\
0.1413141	-1.96769899947657e-05\\
0.1414141	-3.93232199947079e-05\\
0.1415142	-4.24941399946493e-05\\
0.1416142	-5.37353099945902e-05\\
0.1417142	-5.54292599945305e-05\\
0.1418142	-4.41499299944701e-05\\
0.1419142	-4.7618139994409e-05\\
0.1420142	-1.92037799943473e-05\\
0.1421142	-4.29338099428491e-06\\
0.1422142	-2.08140299422184e-06\\
0.1423142	-2.03507699415808e-06\\
0.1424142	-1.04292399409363e-06\\
0.1425143	-6.09991499402842e-06\\
0.1426143	-1.68369099939626e-05\\
0.1427143	-3.7707729993896e-05\\
0.1428143	-4.58350099938287e-05\\
0.1429143	-2.95968699937607e-05\\
0.1430143	-3.19215599936919e-05\\
0.1431143	-1.90683399936224e-05\\
0.1432143	-2.68920299935521e-05\\
0.1433143	-2.23248499934811e-05\\
0.1434143	-2.13213599934093e-05\\
0.1435144	-2.24520399933366e-05\\
0.1436144	-2.49218199932633e-05\\
0.1437144	-1.71009399931891e-05\\
0.1438144	-1.57370499931141e-05\\
0.1439144	-4.35342499930383e-05\\
0.1440144	-7.01350699929617e-05\\
0.1441144	-5.01718699928843e-05\\
0.1442144	-2.1232019992806e-05\\
0.1443144	-2.70117499927269e-05\\
0.1444144	-3.67946699926469e-05\\
0.1445145	-6.5194169992566e-05\\
0.1446145	-4.11160399924843e-05\\
0.1447145	-2.24706899924017e-05\\
0.1448145	-1.87084799923182e-05\\
0.1449145	-2.65824699922337e-05\\
0.1450145	-1.64204799921484e-05\\
0.1451145	-2.27729999920622e-05\\
0.1452145	-1.3719279991975e-05\\
0.1453145	-1.75763399918869e-05\\
0.1454145	-5.74271399179781e-06\\
0.1455146	-3.41132199170768e-06\\
0.1456146	-1.43511499916167e-05\\
0.1457146	-2.16012099152467e-06\\
0.1458146	-1.54018899143168e-06\\
0.1459146	-2.04344399133768e-06\\
0.1460146	-8.49610999124267e-06\\
0.1461146	-8.74758699114663e-06\\
0.1462146	-1.19900099910496e-05\\
0.1463146	-2.69061999909514e-05\\
0.1464146	-1.26385999908522e-05\\
0.1465147	-1.21181799907519e-05\\
0.1466147	-3.47223099906505e-05\\
0.1467147	-4.69576199905481e-05\\
0.1468147	-4.31590999904446e-05\\
0.1469147	-3.36452299903399e-05\\
0.1470147	-3.27825299902341e-05\\
0.1471147	-7.25682499901272e-05\\
0.1472147	-4.64411299900191e-05\\
0.1473147	-2.98972399899099e-05\\
0.1474147	-2.26494299897995e-05\\
0.1475148	-5.68927098968773e-06\\
0.1476148	-7.20997598957491e-06\\
0.1477148	-3.00245998946088e-06\\
0.1478148	-9.81073298934561e-06\\
0.1479148	-9.92706098922911e-06\\
0.1480148	-1.99463899891113e-05\\
0.1481148	-1.32952299889923e-05\\
0.1482148	-1.2800999988872e-05\\
0.1483148	-3.75481899887504e-05\\
0.1484148	-3.23522399886275e-05\\
0.1485149	-2.08661699885031e-05\\
0.1486149	-5.50569398837756e-06\\
0.1487149	-4.98822298825063e-06\\
0.1488149	-1.69751099881223e-05\\
0.1489149	-1.98973599879927e-05\\
0.1490149	-4.27056299878616e-05\\
0.1491149	-2.90152899877291e-05\\
0.1492149	-2.91014499875952e-05\\
0.1493149	-2.07537199874599e-05\\
0.1494149	-1.85305999873231e-05\\
0.149515	-8.7808119871847e-06\\
0.149615	-7.33766498704495e-06\\
0.149715	-1.28373299869037e-05\\
0.149815	-1.33131999867609e-05\\
0.149915	-2.52894299866166e-05\\
0.150015	-9.97653498647078e-06\\
0.150115	-8.46979798632337e-06\\
0.150215	-1.84596799861744e-05\\
0.150315	-1.07744498602378e-06\\
0.150415	-7.37622298587157e-06\\
0.1505151	-1.58796399857176e-05\\
0.1506151	-1.24277699855621e-05\\
0.1507151	-9.71434898540491e-06\\
0.1508151	-1.03981099852461e-05\\
0.1509151	-1.41666999850855e-05\\
0.1510151	-2.37232099849232e-05\\
0.1511151	-1.77200099847592e-05\\
0.1512151	-6.54897198459346e-06\\
0.1513151	-6.78476598442592e-06\\
0.1514151	-9.92689498425658e-06\\
0.1515152	-1.73463699840853e-05\\
0.1516152	-8.70461898391228e-06\\
0.1517152	-1.64069399837374e-05\\
0.1518152	-1.14115399835607e-05\\
0.1519152	-8.65171098338213e-06\\
0.1520152	-4.91669198320162e-06\\
0.1521152	-1.56830399830192e-05\\
0.1522152	-1.35193099828348e-05\\
0.1523152	-6.11541698264842e-06\\
0.1524152	-1.38618399824601e-05\\
0.1525153	-2.73350099822695e-05\\
0.1526153	-3.27393299820771e-05\\
0.1527153	-3.46052799818826e-05\\
0.1528153	-3.13874799816861e-05\\
0.1529153	-2.57023799814874e-05\\
0.1530153	-3.02351499812867e-05\\
0.1531153	-2.14759899810838e-05\\
0.1532153	-3.78703299808787e-05\\
0.1533153	-3.09970699806714e-05\\
0.1534153	-3.21871299804619e-05\\
0.1535154	-1.462252998025e-05\\
0.1536154	-6.70597898003605e-06\\
0.1537154	-6.4376129798198e-06\\
0.1538154	-1.40595799796012e-05\\
0.1539154	-1.09024999793803e-05\\
0.1540154	-2.08285699791571e-05\\
0.1541154	-8.78055397893145e-06\\
0.1542154	-1.20216997870342e-06\\
0.1543154	-4.87985897847295e-06\\
0.1544154	-7.86381278240027e-07\\
0.1545155	-5.29916597800439e-06\\
0.1546155	-1.11345199777665e-05\\
0.1547155	-1.18789297752602e-06\\
0.1548155	-1.2582879977283e-05\\
0.1549155	-1.13601699770374e-05\\
0.1550155	-1.63987799767892e-05\\
0.1551155	-3.37662499765383e-05\\
0.1552155	-4.80766799762848e-05\\
0.1553155	-4.31120799760286e-05\\
0.1554155	-3.82140899757697e-05\\
0.1555156	-4.21148699755077e-05\\
0.1556156	-2.40174099752432e-05\\
0.1557156	-3.9488139749759e-06\\
0.1558156	-9.21601897470576e-06\\
0.1559156	-1.28406099744327e-05\\
0.1560156	-1.12127399741568e-05\\
0.1561156	-9.27421497387798e-06\\
0.1562156	-1.08050699735962e-05\\
0.1563156	-9.34427397331137e-06\\
0.1564156	-2.23992797302354e-06\\
0.1565157	-7.79058297273236e-06\\
0.1566157	-1.35857499724384e-05\\
0.1567157	-1.16638999721413e-05\\
0.1568157	-5.69576397184104e-06\\
0.1569157	-1.98265599715376e-05\\
0.1570157	-2.21455999712309e-05\\
0.1571157	-7.90504397092104e-06\\
0.1572157	-8.33145497060784e-06\\
0.1573157	-7.35156597029133e-06\\
0.1574157	-2.46272999699715e-05\\
0.1575158	-2.74705199696479e-05\\
0.1576158	-2.05271499693212e-05\\
0.1577158	-1.1113599968991e-05\\
0.1578158	-3.43203796865736e-06\\
0.1579158	-1.05220796832017e-06\\
0.1580158	-7.26591996797941e-06\\
0.1581158	-4.35559496763504e-06\\
0.1582158	-7.81703096728702e-06\\
0.1583158	-6.64373096693532e-06\\
0.1584158	-1.55798499665799e-05\\
0.1585159	-3.05243299662204e-05\\
0.1586159	-4.36233999658574e-05\\
0.1587159	-2.37998899654905e-05\\
0.1588159	-2.15591299651198e-05\\
0.1589159	-2.15253999647452e-05\\
0.1590159	-1.76384299643666e-05\\
0.1591159	-1.5510329963984e-05\\
0.1592159	-6.38166996359737e-06\\
0.1593159	-1.38599663206643e-07\\
0.1594159	-3.82237296281179e-06\\
0.159516	-7.93664762412362e-07\\
0.159616	-2.68490196200912e-06\\
0.159716	-5.50101596160161e-06\\
0.159816	-3.60080896118981e-06\\
0.159916	-1.23510099607736e-05\\
0.160016	-1.73032099603531e-05\\
0.160116	-1.25509299599281e-05\\
0.160216	-2.23227499594986e-05\\
0.160316	-2.93342099590646e-05\\
0.160416	-2.29034099586261e-05\\
0.1605161	-1.56436699581824e-05\\
0.1606161	-3.38655299577345e-05\\
0.1607161	-2.53424199572819e-05\\
0.1608161	-2.49868199568245e-05\\
0.1609161	-7.42667195636233e-06\\
0.1610161	-1.69825499558953e-05\\
0.1611161	-2.19416199554233e-05\\
0.1612161	-4.43057495494631e-06\\
0.1613161	-8.33172095446432e-06\\
0.1614161	-9.46083495397725e-06\\
0.1615162	-1.81689199534846e-05\\
0.1616162	-2.58513495298719e-06\\
0.1617162	-8.41456095248457e-06\\
0.1618162	-5.45531495197667e-06\\
0.1619162	-8.39783195146342e-06\\
0.1620162	-1.54124999509448e-05\\
0.1621162	-2.21157099504207e-05\\
0.1622162	-1.1868079949891e-05\\
0.1623162	-1.08687594935587e-06\\
0.1624162	-3.78719594881505e-06\\
0.1625163	-5.55052494826801e-06\\
0.1626163	-1.03825399477158e-05\\
0.1627163	-5.57654794715772e-06\\
0.1628163	-4.2848759465938e-06\\
0.1629163	-1.6330139946024e-05\\
0.1630163	-1.38047999454482e-05\\
0.1631163	-5.81168794486629e-06\\
0.1632163	-2.60671094427831e-06\\
0.1633163	-4.93120643684165e-07\\
0.1634163	-5.93237194308378e-06\\
0.1635164	-3.9768484247649e-07\\
0.1636164	-1.66928094186344e-06\\
0.1637164	-1.16775941243952e-07\\
0.1638164	-3.09830194061797e-06\\
0.1639164	-9.56982893998542e-06\\
0.1640164	-2.46447599393462e-05\\
0.1641164	-1.81182499387004e-05\\
0.1642164	-2.46682699380477e-05\\
0.1643164	-2.28847299373882e-05\\
0.1644164	-5.91526393672183e-06\\
0.1645165	-5.32447393604778e-06\\
0.1646165	-1.98888599353673e-05\\
0.1647165	-3.20599499346798e-05\\
0.1648165	-1.7052139933985e-05\\
0.1649165	-2.6072309933283e-05\\
0.1650165	-8.3314079325736e-06\\
0.1651165	-1.28572299318568e-05\\
0.1652165	-6.43316193113249e-06\\
0.1653165	-4.4283719304006e-06\\
0.1654165	-2.48223692966106e-06\\
0.1655166	-5.75222992891304e-06\\
0.1656166	-5.02959392815795e-06\\
0.1657166	-8.39168992739496e-06\\
0.1658166	-1.1043959926624e-05\\
0.1659166	-1.39730592584498e-06\\
0.1660166	-2.67358692505782e-06\\
0.1661166	-1.52359792426244e-06\\
0.1662166	-7.03594723458747e-07\\
0.1663166	-1.53608499226467e-05\\
0.1664166	-4.7742709218261e-06\\
0.1665167	-9.55261892099614e-06\\
0.1666167	-1.00592499201583e-05\\
0.1667167	-1.72008499193118e-05\\
0.1668167	-1.84122999184564e-05\\
0.1669167	-1.34584199175922e-05\\
0.1670167	-2.34611599167188e-05\\
0.1671167	-1.43227099158364e-05\\
0.1672167	-1.84177899149448e-05\\
0.1673167	-1.11889499140439e-05\\
0.1674167	-8.9001749131336e-06\\
0.1675168	-9.10816291221289e-06\\
0.1676168	-1.32325299112835e-05\\
0.1677168	-1.47582999103444e-05\\
0.1678168	-4.35014290939559e-06\\
0.1679168	-8.88802808436865e-07\\
0.1680168	-4.80361707468154e-07\\
0.1681168	-7.2113496489357e-08\\
0.1682168	-2.26375390550037e-06\\
0.1683168	-2.04898290450109e-06\\
0.1684168	-2.09497190349142e-06\\
0.1685169	-5.31596290247021e-06\\
0.1686169	-1.81907999014394e-05\\
0.1687169	-7.81901490039789e-06\\
0.1688169	-3.37808889934553e-06\\
0.1689169	-1.65227189828224e-06\\
0.1690169	-6.04812089720789e-06\\
0.1691169	-4.13847889612238e-06\\
0.1692169	-2.47375989502559e-06\\
0.1693169	-3.3506688939174e-06\\
0.1694169	-1.19644298927977e-05\\
0.169517	-2.05432698916652e-05\\
0.169617	-1.65458498905221e-05\\
0.169717	-1.74794898893672e-05\\
0.169817	-3.21679098882002e-05\\
0.169917	-4.08634398870212e-05\\
0.170017	-1.19845498858299e-05\\
0.170117	-6.70406488462625e-06\\
0.170217	-7.08714188341011e-06\\
0.170317	-1.03867998821814e-05\\
0.170417	-4.14934888093987e-06\\
0.1705171	-9.96463687968425e-06\\
0.1706171	-8.58926387841687e-06\\
0.1707171	-3.82597687713635e-06\\
0.1708171	-5.31615475842568e-07\\
0.1709171	-7.5657187453538e-07\\
0.1710171	-1.17823898732147e-05\\
0.1711171	-1.07095298718802e-05\\
0.1712171	-5.51931787053202e-06\\
0.1713171	-8.84370986916983e-06\\
0.1714171	-2.54241486779354e-06\\
0.1715172	-3.14065486640161e-06\\
0.1716172	-9.90011464996679e-07\\
0.1717172	-3.14381886357721e-06\\
0.1718172	-1.00543186214305e-06\\
0.1719172	-3.56000160694066e-07\\
0.1720172	-4.11060559230094e-07\\
0.1721172	-1.09008985775099e-06\\
0.1722172	-5.66637585625659e-06\\
0.1723172	-3.78622085474674e-06\\
0.1724172	-5.0062858532213e-06\\
0.1725173	-1.58067398516785e-05\\
0.1726173	-2.05072498501214e-05\\
0.1727173	-1.40846098485482e-05\\
0.1728173	-1.28326598469587e-05\\
0.1729173	-3.35819684535281e-06\\
0.1730173	-1.00639598437304e-05\\
0.1731173	-4.43190384209115e-06\\
0.1732173	-6.16157884043502e-06\\
0.1733173	-8.68299083876181e-06\\
0.1734173	-1.06530783707134e-06\\
0.1735174	-2.81610235361717e-07\\
0.1736174	-1.22561733636181e-07\\
0.1737174	-5.57561631892853e-07\\
0.1738174	-1.90908783013155e-06\\
0.1739174	-1.70239598283521e-05\\
0.1740174	-1.56919098265543e-05\\
0.1741174	-1.3305269824738e-05\\
0.1742174	-4.6885629822903e-05\\
0.1743174	-3.8144579821049e-05\\
0.1744174	-3.4172939819176e-05\\
0.1745175	-1.56133398172818e-05\\
0.1746175	-8.13171581537003e-06\\
0.1747175	-8.93368981343855e-06\\
0.1748175	-5.5214638114872e-06\\
0.1749175	-6.29077080951577e-06\\
0.1750175	-8.15336980752406e-06\\
0.1751175	-5.63144480551186e-06\\
0.1752175	-8.06375034789715e-08\\
0.1753175	-1.77778380142518e-06\\
0.1754175	-2.31599399350277e-07\\
0.1755176	-6.10501179725194e-06\\
0.1756176	-8.69586679513415e-06\\
0.1757176	-1.03672497929946e-05\\
0.1758176	-9.72597379083306e-06\\
0.1759176	-1.17979497886493e-05\\
0.1760176	-8.46074978644317e-06\\
0.1761176	-1.44760097842144e-05\\
0.1762176	-1.26756997819627e-05\\
0.1763176	-9.25407077968787e-06\\
0.1764176	-2.09015097773897e-05\\
0.1765177	-8.26216877506567e-06\\
0.1766177	-3.4244617727201e-06\\
0.1767177	-1.02154477035048e-06\\
0.1768177	-2.24977676795656e-06\\
0.1769177	-2.79426476553809e-06\\
0.1770177	-4.38953776309483e-06\\
0.1771177	-3.70837876062653e-06\\
0.1772177	-2.56922175813293e-06\\
0.1773177	-2.67999375561378e-06\\
0.1774177	-1.11431875306883e-06\\
0.1775178	-8.53543475049523e-06\\
0.1776178	-1.72504797478979e-05\\
0.1777178	-9.15933274527389e-06\\
0.1778178	-1.47896897426231e-05\\
0.1779178	-2.44353097399451e-05\\
0.1780178	-3.05567897372397e-05\\
0.1781178	-1.39486097345067e-05\\
0.1782178	-9.2766277317457e-06\\
0.1783178	-3.33273872895645e-06\\
0.1784178	-8.4514292613869e-07\\
0.1785179	-6.85385572328926e-06\\
0.1786179	-1.49149997204136e-05\\
0.1787179	-2.15580597175085e-05\\
0.1788179	-1.88434497145737e-05\\
0.1789179	-2.32786297116089e-05\\
0.1790179	-4.61776497086139e-05\\
0.1791179	-2.94219497055882e-05\\
0.1792179	-1.85092197025317e-05\\
0.1793179	-9.42282869944393e-06\\
0.1794179	-1.56631296963247e-05\\
0.179518	-7.2145406931704e-06\\
0.179618	-4.06476789987105e-07\\
0.179718	-4.77952368677134e-06\\
0.179818	-8.86833183522762e-07\\
0.179918	-3.43578168024106e-06\\
0.180018	-1.30847767692589e-06\\
0.180118	-1.84433167357691e-06\\
0.180218	-4.63120370193804e-07\\
0.180318	-3.60201266677621e-06\\
0.180418	-2.9163406633238e-06\\
0.1805181	-9.74957159832701e-07\\
0.1806181	-1.31695265630955e-06\\
0.1807181	-2.19161852750511e-07\\
0.1808181	-1.08733964915523e-06\\
0.1809181	-9.32679964552335e-06\\
0.1810181	-6.3644526418545e-06\\
0.1811181	-7.62182938148304e-07\\
0.1812181	-4.8605706344044e-06\\
0.1813181	-8.1258266306224e-06\\
0.1814181	-1.29802296268019e-05\\
0.1815182	-5.70595562293873e-06\\
0.1816182	-6.28270561904012e-06\\
0.1817182	-6.08451515101872e-07\\
0.1818182	-3.81359011235942e-08\\
0.1819182	-1.15516360710488e-06\\
0.1820182	-7.05539603045338e-07\\
0.1821182	-4.88786798944548e-07\\
0.1822182	-6.8884105948021e-06\\
0.1823182	-1.21596895906176e-05\\
0.1824182	-3.42596258639058e-06\\
0.1825183	-3.81183058211636e-06\\
0.1826183	-1.04927395778031e-05\\
0.1827183	-4.78407757344597e-06\\
0.1828183	-5.27998556904468e-06\\
0.1829183	-3.48671656459873e-06\\
0.1830183	-1.30161756010768e-06\\
0.1831183	-3.58503655557109e-06\\
0.1832183	-1.21275595509885e-05\\
0.1833183	-8.88633954635944e-06\\
0.1834183	-2.49157054168347e-06\\
0.1835184	-9.36034836955357e-07\\
0.1836184	-1.03330253218408e-06\\
0.1837184	-2.25130227364474e-07\\
0.1838184	-5.08593152249605e-06\\
0.1839184	-1.18668051757831e-06\\
0.1840184	-1.19679051261079e-06\\
0.1841184	-1.24986407592967e-07\\
0.1842184	-4.52409702524358e-07\\
0.1843184	-1.79158549740445e-06\\
0.1844184	-7.75911449223274e-06\\
0.1845185	-7.91465748700347e-06\\
0.1846185	-2.96066648172654e-06\\
0.1847185	-3.59752747639626e-06\\
0.1848185	-5.47935247101207e-06\\
0.1849185	-1.65252746557346e-06\\
0.1850185	-3.04073146007987e-06\\
0.1851185	-1.40325294545308e-05\\
0.1852185	-3.6017434489256e-06\\
0.1853185	-2.2153084432638e-06\\
0.1854185	-3.82954043754482e-06\\
0.1855186	-7.34172043176228e-06\\
0.1856186	-2.94094542592715e-06\\
0.1857186	-2.58944020033125e-07\\
0.1858186	-3.70061414079607e-07\\
0.1859186	-1.1178329408066e-05\\
0.1860186	-3.94386740199174e-06\\
0.1861186	-3.35802439585619e-06\\
0.1862186	-4.51764038965875e-06\\
0.1863186	-3.57910038339882e-06\\
0.1864186	-2.38839037707578e-06\\
0.1865187	-1.33567037068257e-06\\
0.1866187	-8.49927636423135e-06\\
0.1867187	-4.53703935771512e-06\\
0.1868187	-2.88893935113323e-06\\
0.1869187	-3.22765234448504e-06\\
0.1870187	-1.53367733776989e-06\\
0.1871187	-2.36258233098713e-06\\
0.1872187	-5.79518632413607e-06\\
0.1873187	-2.2683009317216e-05\\
0.1874187	-1.34144993102264e-05\\
0.1875188	-7.57866630315928e-06\\
0.1876188	-7.4673159602817e-07\\
0.1877188	-7.53942288825331e-07\\
0.1878188	-4.35194228155005e-06\\
0.1879188	-6.40575227420161e-06\\
0.1880188	-7.60486026677929e-06\\
0.1881188	-1.20013892592824e-05\\
0.1882188	-8.3231622517101e-06\\
0.1883188	-1.40507492440617e-05\\
0.1884188	-1.12920592363365e-05\\
0.1885189	-1.13247492285259e-05\\
0.1886189	-1.29112392206447e-05\\
0.1887189	-2.12875992126843e-05\\
0.1888189	-1.1926249204644e-05\\
0.1889189	-6.73027319652298e-06\\
0.1890189	-1.24941991883205e-05\\
0.1891189	-3.17043418003568e-06\\
0.1892189	-9.53102617166776e-06\\
0.1893189	-1.07371191632159e-05\\
0.1894189	-1.44215891546793e-05\\
0.189519	-1.06982891460484e-05\\
0.189619	-9.82815713733968e-06\\
0.189719	-2.01521191285437e-05\\
0.189819	-1.75861791196595e-05\\
0.189919	-3.13287791106863e-05\\
0.190019	-2.88240591016232e-05\\
0.190119	-3.90971190924694e-05\\
0.190219	-2.76799390832238e-05\\
0.190319	-2.89674290738857e-05\\
0.190419	-1.31291990644542e-05\\
0.1905191	-8.26174605491862e-06\\
0.1906191	-4.98176504529722e-06\\
0.1907191	-8.16864703557955e-06\\
0.1908191	-7.32769602576468e-06\\
0.1909191	-5.14533801585164e-06\\
0.1910191	-5.06708300583947e-06\\
0.1911191	-2.3448979957272e-06\\
0.1912191	-3.63710398551385e-06\\
0.1913191	-4.22296497519841e-06\\
0.1914191	-1.5973549647799e-06\\
0.1915192	-5.23422095424672e-06\\
0.1916192	-1.00170589436189e-05\\
0.1917192	-8.65024693288494e-06\\
0.1918192	-9.78142422043789e-07\\
0.1919192	-3.03570491109441e-06\\
0.1920192	-4.31006790003573e-06\\
0.1921192	-1.81347288886668e-06\\
0.1922192	-5.27044687758619e-06\\
0.1923192	-1.44287288661932e-05\\
0.1924192	-8.18780985468648e-06\\
0.1925193	-4.42258784305336e-06\\
0.1926193	-2.93739483131594e-06\\
0.1927193	-4.57465019461493e-07\\
0.1928193	-6.35880080748889e-06\\
0.1929193	-1.2113018795397e-05\\
0.1930193	-3.31639578318457e-06\\
0.1931193	-8.45875877085051e-06\\
0.1932193	-1.60378087583936e-05\\
0.1933193	-9.95330745812619e-07\\
0.1934193	-1.87663933106384e-07\\
0.1935194	-6.50246772026076e-06\\
0.1936194	-3.08086170730018e-06\\
0.1937194	-3.48638169421062e-06\\
0.1938194	-5.87566768099083e-06\\
0.1939194	-6.48453366763953e-06\\
0.1940194	-1.61386665415544e-06\\
0.1941194	-2.34198364053726e-06\\
0.1942194	-5.34482362678368e-06\\
0.1943194	-1.39074486128934e-05\\
0.1944194	-9.53605859886503e-06\\
0.1945195	-5.63963158468304e-06\\
0.1946195	-4.20507857037439e-06\\
0.1947195	-3.0950145559236e-06\\
0.1948195	-7.0318915413293e-06\\
0.1949195	-1.50981452659007e-06\\
0.1950195	-1.27625611704514e-07\\
0.1951195	-7.43804296671199e-07\\
0.1952195	-4.10560848148869e-06\\
0.1953195	-1.18001346615552e-06\\
0.1954195	-2.66038745067024e-06\\
0.1955196	-7.55180443501564e-06\\
0.1956196	-8.06870341922151e-06\\
0.1957196	-1.43873484032708e-05\\
0.1958196	-2.07493138716191e-06\\
0.1959196	-3.31481837089337e-06\\
0.1960196	-2.11607235446361e-06\\
0.1961196	-7.87552937871057e-07\\
0.1962196	-5.91739832111413e-06\\
0.1963196	-3.56678630419123e-06\\
0.1964196	-1.48018882871008e-05\\
0.1965197	-1.19088082698237e-05\\
0.1966197	-3.50545025239298e-06\\
0.1967197	-5.16048223478973e-06\\
0.1968197	-2.93906621701229e-06\\
0.1969197	-8.87694919905895e-06\\
0.1970197	-7.98740418092802e-06\\
0.1971197	-2.59960316261777e-06\\
0.1972197	-9.22820714412646e-06\\
0.1973197	-6.61850712545232e-06\\
0.1974197	-5.7618621065936e-06\\
0.1975198	-5.61477008752935e-06\\
0.1976198	-2.13675906829586e-06\\
0.1977198	-4.24551148872358e-07\\
0.1978198	-1.39020002925699e-06\\
0.1979198	-4.9860410094479e-06\\
0.1980198	-5.21907794432188e-08\\
0.1981198	-2.43583896924104e-06\\
0.1982198	-1.75292694883945e-06\\
0.1983198	-4.58729392823653e-06\\
0.1984198	-9.53802974303177e-08\\
0.1985199	-4.07047986397733e-07\\
0.1986199	-1.97637665178811e-07\\
0.1987199	-1.34907384375064e-06\\
0.1988199	-9.23539982211119e-06\\
0.1989199	-7.33629380025841e-06\\
0.1990199	-1.54971377781903e-05\\
0.1991199	-3.26389175590463e-06\\
0.1992199	-1.23838373339943e-06\\
0.1993199	-2.6855071067254e-07\\
0.1994199	-2.98739668772181e-06\\
0.19952	-5.11124866452179e-06\\
0.19962	-9.47618564111665e-06\\
0.19972	-8.59694061748111e-06\\
0.19982	-1.73318259361296e-06\\
0.19992	-2.05431056950995e-06\\
0.20002	-1.10689845169801e-07\\
0.20012	-1.10731352059023e-06\\
0.20022	-4.02567649576894e-06\\
0.20032	-1.39037447070358e-06\\
0.20042	-7.4690704453918e-06\\
0.2005201	-4.00007641980554e-06\\
0.2006201	-9.73496093993515e-07\\
0.2007201	-7.66830767927872e-07\\
0.2008201	-5.27047734160616e-06\\
0.2009201	-7.82587331502592e-06\\
0.2010201	-1.41881988184658e-07\\
0.2011201	-3.5574291079852e-08\\
0.2012201	-5.43759023370897e-06\\
0.2013201	-2.57859720606944e-06\\
0.2014201	-6.50010817815869e-06\\
0.2015202	-3.12115349945768e-07\\
0.2016202	-1.16643771214844e-05\\
0.2017202	-1.50694770927439e-05\\
0.2018202	-9.96346106372161e-06\\
0.2019202	-1.01953570344148e-05\\
0.2020202	-6.14863600482063e-06\\
0.2021202	-6.94863997493647e-06\\
0.2022202	-1.34316369447595e-05\\
0.2023202	-2.75878869142868e-05\\
0.2024202	-1.09537468835157e-05\\
0.2025203	-3.18320185241204e-06\\
0.2026203	-6.87398982103502e-06\\
0.2027203	-9.65230078935083e-06\\
0.2028203	-8.94222375735652e-06\\
0.2029203	-1.11626667250491e-05\\
0.2030203	-3.51180969242559e-06\\
0.2031203	-2.04237994829316e-08\\
0.2032203	-7.54173426218069e-07\\
0.2033203	-1.03781265926279e-05\\
0.2034203	-9.61304955870934e-06\\
0.2035204	-1.37842565244248e-05\\
0.2036204	-7.72010348983956e-06\\
0.2037204	-8.05191745491638e-06\\
0.2038204	-2.4685766419652e-05\\
0.2039204	-2.06376063840432e-05\\
0.2040204	-9.36253634808655e-06\\
0.2041204	-6.21977631177883e-06\\
0.2042204	-1.64322662751167e-05\\
0.2043204	-1.99345562380966e-05\\
0.2044204	-1.12444162007153e-05\\
0.2045205	-5.5842861629313e-06\\
0.2046205	-1.45644912481663e-06\\
0.2047205	-1.3305230863302e-06\\
0.2048205	-2.35344304746845e-06\\
0.2049205	-2.21121500822779e-06\\
0.2050205	-2.47551868604598e-07\\
0.2051205	-1.13197659285952e-05\\
0.2052205	-8.40589388819595e-06\\
0.2053205	-1.13659784740307e-06\\
0.2054205	-7.01510506212819e-07\\
0.2055206	-2.46485276457961e-06\\
0.2056206	-3.49019372258278e-06\\
0.2057206	-3.18549568017708e-06\\
0.2058206	-5.11963863735861e-06\\
0.2059206	-1.50641955941234e-05\\
0.2060206	-8.97212155046752e-06\\
0.2061206	-1.28643355063869e-05\\
0.2062206	-3.43502246187755e-06\\
0.2063206	-6.48474641693532e-06\\
0.2064206	-6.02113537155612e-06\\
0.2065207	-4.77218532568972e-06\\
0.2066207	-1.28892852794236e-05\\
0.2067207	-9.79649523270779e-06\\
0.2068207	-6.19044818553812e-06\\
0.2069207	-1.25686251379102e-05\\
0.2070207	-1.60197550898198e-05\\
0.2071207	-7.11403004126233e-06\\
0.2072207	-7.78325999223344e-06\\
0.2073207	-2.33406749427286e-05\\
0.2074207	-2.78525048927433e-05\\
0.2075208	-1.19315548422223e-05\\
0.2076208	-5.31917579126188e-06\\
0.2077208	-2.86184073980713e-06\\
0.2078208	-2.94401668785337e-06\\
0.2079208	-2.85151363539585e-06\\
0.2080208	-7.28867158242978e-06\\
0.2081208	-4.76051928950322e-07\\
0.2082208	-1.51193747495261e-06\\
0.2083208	-8.876864204317e-07\\
0.2084208	-6.28777865382635e-07\\
0.2085209	-1.14726530974454e-06\\
0.2086209	-1.41671442536235e-05\\
0.2087209	-1.06768341969591e-05\\
0.2088209	-1.22877341397462e-05\\
0.2089209	-1.93684081979611e-07\\
0.2090209	-4.85247002365401e-06\\
0.2091209	-4.44212696476412e-06\\
0.2092209	-7.8928869053046e-06\\
0.2093209	-8.44775245270039e-07\\
0.2094209	-1.20030478465498e-06\\
0.209521	-6.56719223392434e-07\\
0.209621	-3.15259666159923e-06\\
0.209721	-4.83702359920887e-06\\
0.209821	-8.26845136215682e-07\\
0.209921	-8.56117272613959e-07\\
0.210021	-3.57526840839794e-06\\
0.210121	-2.5592403435618e-06\\
0.210221	-3.49292827809967e-06\\
0.210321	-1.00890932120056e-05\\
0.210421	-6.27651814527369e-06\\
0.2105211	-6.32354307783011e-06\\
0.2106211	-4.73632300980356e-06\\
0.2107211	-2.60372094112083e-06\\
0.2108211	-3.30713087177571e-06\\
0.2109211	-3.76346080176194e-06\\
0.2110211	-1.3888527310732e-06\\
0.2111211	-8.00339565970311e-06\\
0.2112211	-5.85789658764522e-06\\
0.2113211	-8.67477951489305e-06\\
0.2114211	-5.74301744144002e-06\\
0.2115212	-8.17352667205017e-07\\
0.2116212	-3.48109529232966e-06\\
0.2117212	-6.8678142167334e-06\\
0.2118212	-1.10820521404094e-05\\
0.2119212	-1.08750920633509e-05\\
0.2120212	-5.08366698555089e-06\\
0.2121212	-1.06739070023563e-08\\
0.2122212	-4.11544882769827e-06\\
0.2123212	-2.9112007476315e-06\\
0.2124212	-8.13865066794858e-07\\
0.2125213	-5.81269458509908e-06\\
0.2126213	-8.56245250270006e-06\\
0.2127213	-3.12341241950919e-06\\
0.2128213	-1.443841335519e-06\\
0.2129213	-5.59214025072197e-06\\
0.2130213	-9.0917381651105e-06\\
0.2131213	-2.40335007867692e-06\\
0.2132213	-6.39880499141348e-06\\
0.2133213	-5.85553290331239e-06\\
0.2134213	-6.28472081436575e-06\\
0.2135214	-1.20673007244754e-05\\
0.2136214	-1.29226606338129e-05\\
0.2137214	-4.43309454228069e-06\\
0.2138214	-7.66018444987072e-06\\
0.2139214	-2.21175403565747e-05\\
0.2140214	-1.24347602623842e-05\\
0.2141214	-3.55394916729099e-06\\
0.2142214	-8.06330607128647e-06\\
0.2143214	-6.8163389743621e-06\\
0.2144214	-6.28007587650926e-06\\
0.2145215	-1.23712897776199e-05\\
0.2146215	-3.64004467788295e-06\\
0.2147215	-3.69834557719107e-06\\
0.2148215	-1.20126094755353e-05\\
0.2149215	-8.68542737290672e-06\\
0.2150215	-2.49029126929608e-06\\
0.2151215	-1.67030316469422e-06\\
0.2152215	-6.18197905909183e-06\\
0.2153215	-4.40395995247953e-06\\
0.2154215	-1.85046184484787e-06\\
0.2155216	-2.69650673607811e-06\\
0.2156216	-1.96810862637791e-06\\
0.2157216	-4.24933915629419e-07\\
0.2158216	-3.0243124038228e-06\\
0.2159216	-1.00944529094816e-06\\
0.2160216	-2.20311717699548e-06\\
0.2161216	-2.66234206195468e-06\\
0.2162216	-1.17034779458156e-05\\
0.2163216	-6.0383038285679e-06\\
0.2164216	-4.28525871020127e-06\\
0.2165217	-6.74588290585157e-07\\
0.2166217	-1.65862446994798e-06\\
0.2167217	-3.59138234816016e-06\\
0.2168217	-9.04860122521093e-06\\
0.2169217	-9.67937401089448e-07\\
0.2170217	-1.20029697578475e-06\\
0.2171217	-4.14215784928578e-06\\
0.2172217	-7.8529647215814e-06\\
0.2173217	-3.19248659266033e-06\\
0.2174217	-3.77223146251123e-06\\
0.2175218	-9.81327230990611e-07\\
0.2176218	-1.14190161983497e-05\\
0.2177218	-1.4211746064446e-05\\
0.2178218	-7.44239592926776e-06\\
0.2179218	-7.26717879280311e-06\\
0.2180218	-2.95357465504003e-06\\
0.2181218	-2.48968315966429e-07\\
0.2182218	-2.72369737557008e-06\\
0.2183218	-6.78039723383865e-06\\
0.2184218	-3.21961909075971e-06\\
0.2185219	-2.57928394617558e-06\\
0.2186219	-4.56577680036245e-06\\
0.2187219	-1.1022855316379e-07\\
0.2188219	-3.24185650456669e-06\\
0.2189219	-1.06316935455814e-06\\
0.2190219	-5.37307820312499e-06\\
0.2191219	-1.539029050254e-06\\
0.2192219	-1.38261289593178e-06\\
0.2193219	-2.38220874014483e-06\\
0.2194219	-5.07756082879542e-07\\
0.219522	-1.30487842396265e-06\\
0.219622	-1.06421532636978e-05\\
0.219722	-6.99979510191295e-06\\
0.219822	-7.74723093859401e-06\\
0.219922	-2.60142977372671e-06\\
0.220022	-5.15394460729668e-06\\
0.220122	-9.1755194392894e-06\\
0.220222	-1.47711822696902e-05\\
0.220322	-1.45805320984844e-05\\
0.220422	-1.05096619256569e-05\\
0.2205221	-1.64608117510176e-05\\
0.2206221	-3.80507657490002e-06\\
0.2207221	-1.93497839711531e-06\\
0.2208221	-4.91778917647981e-07\\
0.2209221	-9.20463036482428e-07\\
0.2210221	-9.53446353602908e-07\\
0.2211221	-1.63626866899354e-06\\
0.2212221	-8.80744682638281e-07\\
0.2213221	-8.39868494520966e-07\\
0.2214221	-9.51863110462527e-06\\
0.2215222	-1.71691099127421e-05\\
0.2216222	-1.19653097192383e-05\\
0.2217222	-1.08519495239062e-05\\
0.2218222	-4.88011032672889e-06\\
0.2219222	-6.69781927689352e-07\\
0.2220222	-1.05011589267703e-05\\
0.2221222	-8.19712072395441e-06\\
0.2222222	-2.1829401922409e-07\\
0.2223222	-8.36082431256164e-06\\
0.2224222	-1.31948781039492e-05\\
0.2225223	-1.01073478931572e-05\\
0.2226223	-3.24623568058858e-06\\
0.2227223	-2.89002146601539e-06\\
0.2228223	-5.02645624941912e-06\\
0.2229223	-2.86268130781075e-07\\
0.2230223	-1.03054481008241e-06\\
0.2231223	-2.65448658730411e-06\\
0.2232223	-1.77351836242696e-06\\
0.2233223	-5.5256891354316e-06\\
0.2234223	-3.81880690629849e-06\\
0.2235224	-4.45368167477553e-06\\
0.2236224	-1.26322044130538e-06\\
0.2237224	-5.17059320563776e-06\\
0.2238224	-9.05011796775241e-06\\
0.2239224	-5.50976572762889e-06\\
0.2240224	-5.06269348524657e-06\\
0.2241224	-1.72477542405846e-05\\
0.2242224	-4.5235189936221e-06\\
0.2243224	-6.71798944337762e-07\\
0.2244224	-5.82982249271024e-06\\
0.2245225	-1.10129332384628e-05\\
0.2246225	-1.17305129820816e-05\\
0.2247225	-1.26967227232919e-05\\
0.2248225	-1.50789924620715e-05\\
0.2249225	-2.66217219839817e-06\\
0.2250225	-9.2414609322492e-06\\
0.2251225	-1.16549166360188e-06\\
0.2252225	-3.0878559243325e-07\\
0.2253225	-1.56066411872012e-06\\
0.2254225	-1.95818484243911e-06\\
0.2255226	-1.77223456328646e-06\\
0.2256226	-5.96362628179607e-06\\
0.2257226	-1.35060199766637e-06\\
0.2258226	-6.77943810873104e-07\\
0.2259226	-2.26860542139183e-06\\
0.2260226	-2.61438312919787e-06\\
0.2261226	-1.11745883426633e-06\\
0.2262226	-1.99764053657209e-06\\
0.2263226	-4.50937723608982e-06\\
0.2264226	-1.74920293279393e-06\\
0.2265227	-3.02981262635106e-06\\
0.2266227	-1.04909631734744e-06\\
0.2267227	-7.87365005452083e-07\\
0.2268227	-1.44444490638491e-07\\
0.2269227	-1.97300037287993e-06\\
0.2270227	-1.27045405214941e-06\\
0.2271227	-2.66453328419724e-07\\
0.2272227	-3.8604334016634e-06\\
0.2273227	-1.27685371852743e-07\\
0.2274227	-2.17193073895978e-06\\
0.2275228	-3.77356540261874e-06\\
0.2276228	-1.26119040634732e-05\\
0.2277228	-9.12008072115986e-06\\
0.2278228	-2.07799933756499e-05\\
0.2279228	-1.1778423026914e-05\\
0.2280228	-2.86431226749227e-05\\
0.2281228	-2.05831623196462e-05\\
0.2282228	-1.95572219610546e-05\\
0.2283228	-6.85908459911749e-06\\
0.2284228	-6.36803323380435e-06\\
0.2285229	-7.07520286471391e-06\\
0.2286229	-9.50915849255248e-06\\
0.2287229	-7.02581911692166e-06\\
0.2288229	-1.20159697377898e-05\\
0.2289229	-1.41994293551249e-05\\
0.2290229	-3.49976896889476e-06\\
0.2291229	-3.44986257906686e-06\\
0.2292229	-7.60092318560838e-06\\
0.2293229	-1.84013678848623e-06\\
0.2294229	-2.72710738766699e-06\\
0.229523	-4.32631482710545e-07\\
0.229623	-5.91839974391969e-07\\
0.229723	-8.9214596227428e-07\\
0.229823	-4.42059474632287e-06\\
0.229923	-1.9904032650283e-07\\
0.230023	-5.06676002778928e-07\\
0.230123	-1.69871675115626e-07\\
0.230223	-1.65343104347707e-06\\
0.230323	-5.95522907827085e-07\\
0.230423	-5.39591968129174e-07\\
0.2305231	-8.28675123900677e-07\\
0.2306231	-3.85788527599198e-06\\
0.2307231	-1.02109418239239e-05\\
0.2308231	-1.9467247658547e-08\\
0.2309231	-4.06585590715785e-06\\
0.2310231	-6.27742844238328e-06\\
0.2311231	-1.1958019973296e-05\\
0.2312231	-1.25306794998569e-05\\
0.2313231	-2.43426990220264e-05\\
0.2314231	-3.15697485397647e-05\\
0.2315232	-8.66414605254245e-06\\
0.2316232	-6.88616556129261e-06\\
0.2317232	-3.82661265489722e-07\\
0.2318232	-3.68118756509248e-06\\
0.2319232	-6.1343260600592e-06\\
0.2320232	-1.11484455503478e-05\\
0.2321232	-1.3021865035916e-05\\
0.2322232	-5.53862951672087e-06\\
0.2323232	-1.01923239927193e-05\\
0.2324232	-1.32697634638677e-05\\
0.2325233	-4.18862392958597e-06\\
0.2326233	-5.33167339089722e-06\\
0.2327233	-6.72529384722542e-06\\
0.2328233	-4.61198029852547e-06\\
0.2329233	-1.69737907447519e-05\\
0.2330233	-1.90545901858586e-05\\
0.2331233	-2.15366962179953e-06\\
0.2332233	-6.50586305252776e-06\\
0.2333233	-2.85196747799621e-06\\
0.2334233	-3.90113689815729e-06\\
0.2335234	-6.04900931237514e-06\\
0.2336234	-9.7543467217717e-06\\
0.2337234	-5.68080912571563e-06\\
0.2338234	-5.12292552415766e-06\\
0.2339234	-3.35576591704807e-06\\
0.2340234	-4.31763730433674e-06\\
0.2341234	-8.98537268597307e-06\\
0.2342234	-3.00073206190604e-06\\
0.2343234	-6.02801143208417e-06\\
0.2344234	-4.54268379645552e-06\\
0.2345235	-7.67652543232495e-08\\
0.2346235	-5.01585906917455e-07\\
0.2347235	-2.51333985354626e-06\\
0.2348235	-1.59927194155866e-07\\
0.2349235	-2.43590352869201e-06\\
0.2350235	-1.14849785709996e-06\\
0.2351235	-2.63699279324476e-07\\
0.2352235	-1.29663195309852e-07\\
0.2353235	-2.09439480499988e-06\\
0.2354235	-6.12358410833788e-06\\
0.2355236	-1.87158840456034e-06\\
0.2356236	-5.16733969501566e-06\\
0.2357236	-7.00408478945783e-07\\
0.2358236	-8.77041725629199e-06\\
0.2359236	-9.86317452699503e-06\\
0.2360236	-1.20965379099517e-06\\
0.2361236	-3.37523504823211e-06\\
0.2362236	-3.17534029864505e-06\\
0.2363236	-4.25190054217264e-06\\
0.2364236	-2.16943377875299e-06\\
0.2365237	-1.01344500754972e-06\\
0.2366237	-4.76666823004067e-06\\
0.2367237	-7.36719839538732e-09\\
0.2368237	-1.20655853549789e-07\\
0.2369237	-2.66758585443923e-06\\
0.2370237	-1.34346530479985e-05\\
0.2371237	-9.05818623416177e-06\\
0.2372237	-4.15146641286272e-06\\
0.2373237	-2.36411458403439e-06\\
0.2374237	-7.82000374760925e-06\\
0.2375238	-2.88067390267122e-06\\
0.2376238	-3.83674205083971e-06\\
0.2377238	-5.10116619120514e-06\\
0.2378238	-7.70427832369759e-06\\
0.2379238	-1.05855454482465e-05\\
0.2380238	-1.91531745647807e-05\\
0.2381238	-7.2913966732285e-06\\
0.2382238	-9.19816277351738e-06\\
0.2383238	-3.67285886557432e-06\\
0.2384238	-7.2535554932564e-07\\
0.2385239	-7.71568723768173e-07\\
0.2386239	-5.51393809067614e-06\\
0.2387239	-3.70308814905347e-06\\
0.2388239	-1.25411471988239e-05\\
0.2389239	-1.3122742399104e-06\\
0.2390239	-4.04945127223539e-06\\
0.2391239	-1.00984042957206e-05\\
0.2392239	-3.49870631028689e-06\\
0.2393239	-1.02924023158547e-05\\
0.2394239	-1.15788113123436e-05\\
0.239524	-5.3072442986553e-06\\
0.239624	-3.15837476733206e-07\\
0.239724	-3.73417524548676e-06\\
0.239824	-7.69867304832737e-07\\
0.239924	-8.46327446871963e-08\\
0.240024	-1.91241409496546e-06\\
0.240124	-9.4135000255821e-06\\
0.240224	-2.94932094645097e-06\\
0.240324	-2.10058485748514e-06\\
0.240424	-8.02275785969617e-08\\
0.2405241	-4.76934264858405e-06\\
0.2406241	-5.10999952957494e-06\\
0.2407241	-1.42809840037578e-06\\
0.2408241	-1.15767026089579e-06\\
0.2409241	-4.77808311104343e-06\\
0.2410241	-1.59221595072635e-06\\
0.2411241	-3.42575579851403e-07\\
0.2412241	-1.32040259832463e-06\\
0.2413241	-2.17642740605125e-06\\
0.2414241	-3.25362920293566e-06\\
0.2415242	-1.48109379876619e-05\\
0.2416242	-6.58690676256067e-06\\
0.2417242	-1.60536155263253e-05\\
0.2418242	-2.79076127885695e-06\\
0.2419242	-4.95634202005565e-06\\
0.2420242	-3.22720274982075e-06\\
0.2421242	-1.05623604680507e-05\\
0.2422242	-1.37676617464305e-06\\
0.2423242	-5.3408576949446e-07\\
0.2424242	-2.10654555250071e-06\\
0.2425243	-1.9907702222217e-06\\
0.2426243	-7.13982881209204e-07\\
0.2427243	-4.0380865280333e-06\\
0.2428243	-3.06174062586093e-07\\
0.2429243	-1.72432984758763e-07\\
0.2430243	-1.65347839444154e-06\\
0.2431243	-3.09265491523716e-07\\
0.2432243	-6.07580758936062e-08\\
0.2433243	-1.99211414743857e-06\\
0.2434243	-9.080988706045e-06\\
0.2435244	-1.36615412501373e-05\\
0.2436244	-2.14095597825086e-05\\
0.2437244	-1.43345283015945e-05\\
0.2438244	-1.05537268072774e-05\\
0.2439244	-9.36063029943859e-06\\
0.2440244	-8.40730877795852e-06\\
0.2441244	-1.4803482427165e-06\\
0.2442244	-5.13889769359081e-06\\
0.2443244	-6.67302013045868e-06\\
0.2444244	-9.78618855319629e-06\\
0.2445245	-5.19480596008004e-06\\
0.2446245	-1.95529835416693e-06\\
0.2447245	-7.13476673374549e-06\\
0.2448245	-6.0521030986876e-06\\
0.2449245	-1.4254799448864e-05\\
0.2450245	-2.17192777841442e-05\\
0.2451245	-1.40610461043969e-05\\
0.2452245	-5.80574040948934e-06\\
0.2453245	-1.57775226992878e-05\\
0.2454245	-1.20688509736573e-05\\
0.2455246	-1.73638492307127e-05\\
0.2456246	-1.07443874737991e-05\\
0.2457246	-6.19610870104471e-06\\
0.2458246	-6.40636191230979e-06\\
0.2459246	-1.23034421074535e-05\\
0.2460246	-8.91130628633389e-06\\
0.2461246	-1.43936484488076e-05\\
0.2462246	-1.04390565947302e-05\\
0.2463246	-9.20974172395586e-06\\
0.2464246	-9.41281083633766e-06\\
0.2465247	-9.51759892981415e-06\\
0.2466247	-7.28585800804492e-06\\
0.2467247	-6.35110606898304e-06\\
0.2468247	-4.04727711247641e-06\\
0.2469247	-8.69598313837161e-06\\
0.2470247	-1.42970711465139e-05\\
0.2471247	-8.73671636747208e-07\\
0.2472247	-2.40955910891415e-06\\
0.2473247	-3.12445106285596e-06\\
0.2474247	-7.45275899841254e-06\\
0.2475248	-5.02396891333006e-06\\
0.2476248	-3.93875481161152e-06\\
0.2477248	-5.41676669101895e-06\\
0.2478248	-1.08161251386677e-07\\
0.2479248	-6.8907673925476e-06\\
0.2480248	-7.91400021433321e-06\\
0.2481248	-9.48581501657352e-06\\
0.2482248	-8.41279979909711e-06\\
0.2483248	-1.8593815617311e-06\\
0.2484248	-1.90197230430111e-06\\
0.2485249	-9.18862102434341e-06\\
0.2486249	-4.95090672623587e-06\\
0.2487249	-5.65981707532055e-07\\
0.2488249	-2.36098406805156e-06\\
0.2489249	-3.00608370761243e-06\\
0.2490249	-8.54460226031143e-07\\
0.2491249	-5.15284923122634e-07\\
0.2492249	-9.91400498700238e-07\\
0.2493249	-7.03089452575702e-07\\
0.2494249	-1.00139358455917e-06\\
0.249525	-4.68514509195793e-06\\
0.249625	-3.30578357955894e-06\\
0.249725	-5.1608770446884e-06\\
0.249825	-5.54733448714987e-06\\
0.249925	-1.53740290674526e-06\\
0.250025	-1.32161323032748e-05\\
0.250125	-5.25888167653692e-06\\
0.250225	-5.98760002632849e-06\\
0.250325	-4.27692052444543e-07\\
0.250425	-3.74771965467838e-06\\
0.2505251	-7.89740093008758e-06\\
0.2506251	-6.21769718390547e-06\\
0.2507251	-6.84671441321025e-06\\
0.2508251	-2.48696861778811e-06\\
0.2509251	-1.0501997974234e-06\\
0.2510251	-2.84284351898643e-07\\
0.2511251	-4.78859108099452e-06\\
0.2512251	-4.21368218448985e-06\\
0.2513251	-6.18980626216156e-06\\
0.2514251	-3.04397131378467e-06\\
0.2515252	-3.19059433614444e-06\\
0.2516252	-5.1331093349612e-06\\
0.2517252	-2.20831330704271e-06\\
0.2518252	-1.15636425215626e-06\\
0.2519252	-4.71720817006716e-06\\
0.2520252	-2.14937806053874e-06\\
0.2521252	-3.2533339233323e-06\\
0.2522252	-8.73186758207144e-07\\
0.2523252	-1.7306146492051e-07\\
0.2524252	-4.27014834322758e-06\\
0.2525253	-2.83987208961671e-06\\
0.2526253	-2.35073081033942e-06\\
0.2527253	-5.72780750190863e-06\\
0.2528253	-6.72997316407111e-06\\
0.2529253	-5.15410665714988e-08\\
0.2530253	-2.42597939915227e-06\\
0.2531253	-7.1371719715537e-06\\
0.2532253	-7.5359875135139e-06\\
0.2533253	-4.56781702476872e-06\\
0.2534253	-4.18572850505182e-06\\
0.2535254	-8.38624195052789e-06\\
0.2536254	-4.40983336802777e-06\\
0.2537254	-1.1780727753743e-05\\
0.2538254	-3.130415107398e-06\\
0.2539254	-2.24909642871504e-06\\
0.2540254	-6.38040171741393e-06\\
0.2541254	-2.93915297321213e-06\\
0.2542254	-4.83219819582472e-06\\
0.2543254	-1.96569638496438e-06\\
0.2544254	-9.54137240341362e-07\\
0.2545255	-4.47746565776764e-06\\
0.2546255	-6.39371574470569e-06\\
0.2547255	-3.91809879699676e-06\\
0.2548255	-2.31865514341166e-07\\
0.2549255	-5.87722079643674e-06\\
0.2550255	-1.65701142978738e-07\\
0.2551255	-8.79977353659863e-07\\
0.2552255	-9.66058052817022e-06\\
0.2553255	-5.5344153661973e-06\\
0.2554255	-1.19084816742597e-06\\
0.2555256	-1.66359192728387e-06\\
0.2556256	-2.53781465392208e-06\\
0.2557256	-1.58024283428001e-05\\
0.2558256	-1.21732039935921e-05\\
0.2559256	-3.41262860596944e-06\\
0.2560256	-8.81829817960069e-06\\
0.2561256	-9.59751671415169e-06\\
0.2562256	-5.89388420928544e-06\\
0.2563256	-6.08094066466211e-06\\
0.2564256	-4.48104507993903e-06\\
0.2565257	-1.02262824501251e-05\\
0.2566257	-1.0933277784122e-05\\
0.2567257	-1.50460107697328e-06\\
0.2568257	-5.11783632832473e-06\\
0.2569257	-1.32429453781898e-06\\
0.2570257	-9.32071770509572e-06\\
0.2571257	-6.1810438297916e-06\\
0.2572257	-7.97868591154022e-06\\
0.2573257	-1.57175739499721e-05\\
0.2574257	-1.52630389447147e-05\\
0.2575258	-5.04461689032088e-06\\
0.2576258	-4.95269379651008e-06\\
0.2577258	-1.05254336578731e-05\\
0.2578258	-6.64470847402463e-06\\
0.2579258	-5.45728544576321e-07\\
0.2580258	-2.86955496913645e-06\\
0.2581258	-3.83684764731003e-06\\
0.2582258	-8.44701527869878e-06\\
0.2583258	-1.71788318629011e-05\\
0.2584258	-2.66898339951196e-06\\
0.2585259	-4.79674888258751e-06\\
0.2586259	-3.15363032273841e-06\\
0.2587259	-1.85598671406211e-06\\
0.2588259	-3.84925905613989e-06\\
0.2589259	-6.15296434854957e-06\\
0.2590259	-5.61228059086541e-06\\
0.2591259	-5.02172378265813e-06\\
0.2592259	-4.28950492349487e-06\\
0.2593259	-4.09850101293914e-06\\
0.2594259	-5.52728405055081e-06\\
0.259526	-8.6695540298451e-06\\
0.259626	-7.32969296240352e-06\\
0.259726	-5.07299741786354e-07\\
0.259826	-2.44585766753859e-06\\
0.259926	-1.12199143920142e-06\\
0.260026	-2.38358915631223e-06\\
0.260126	-1.14916481840454e-06\\
0.260226	-7.37247925007973e-07\\
0.260326	-5.78810197564825e-06\\
0.260426	-4.94953846984713e-06\\
0.2605261	-2.99617190053102e-06\\
0.2606261	-6.40701728033878e-06\\
0.2607261	-1.60687360224591e-06\\
0.2608261	-1.0833489865758e-05\\
0.2609261	-1.12950730703767e-05\\
0.2610261	-6.98305821559927e-06\\
0.2611261	-6.40277730091893e-06\\
0.2612261	-4.60950332582465e-06\\
0.2613261	-1.28705052898012e-05\\
0.2614261	-7.32113219232897e-06\\
0.2615262	-2.08745602569351e-06\\
0.2616262	-1.51764280368508e-06\\
0.2617262	-2.10104751864269e-06\\
0.2618262	-8.28105770029282e-07\\
0.2619262	-5.59407857303383e-07\\
0.2620262	-1.403174279919e-06\\
0.2621262	-1.65352173732563e-06\\
0.2622262	-4.29336912896818e-06\\
0.2623262	-4.53010945428697e-06\\
0.2624262	-5.32314471271767e-06\\
0.2625263	-7.7394548958483e-06\\
0.2626263	-3.02129501872278e-06\\
0.2627263	-3.7133140729874e-06\\
0.2628263	-1.22737020580589e-05\\
0.2629263	-2.39465097334912e-06\\
0.2630263	-1.00257758182652e-05\\
0.2631263	-1.46142075922091e-05\\
0.2632263	-5.88660829457813e-06\\
0.2633263	-7.13957892476443e-06\\
0.2634263	-3.05900948215515e-06\\
0.2635264	-5.06717895757941e-06\\
0.2636264	-4.18652536744578e-06\\
0.2637264	-8.30552470264683e-06\\
0.2638264	-4.5883339625492e-06\\
0.2639264	-1.50139914651428e-06\\
0.2640264	-7.05016625389819e-06\\
0.2641264	-4.65725528405172e-06\\
0.2642264	-1.65222936320335e-07\\
0.2643264	-5.06055911004406e-06\\
0.2644264	-1.99389590455748e-06\\
0.2645265	-4.22409109864131e-07\\
0.2646265	-7.78893843857824e-07\\
0.2647265	-5.50887096611184e-07\\
0.2648265	-1.08701226743657e-06\\
0.2649265	-2.21201965564066e-06\\
0.2650265	-1.39159869605244e-05\\
0.2651265	-1.33817871813831e-05\\
0.2652265	-8.22563931750607e-06\\
0.2653265	-4.69773936817689e-06\\
0.2654265	-4.05207833267322e-06\\
0.2655266	-5.90238720010061e-06\\
0.2656266	-1.05909698996907e-06\\
0.2657266	-1.57207169145938e-06\\
0.2658266	-1.7036773038251e-06\\
0.2659266	-4.20879226313687e-07\\
0.2660266	-1.09049625816637e-06\\
0.2661266	-8.18595798618155e-07\\
0.2662266	-3.82330484689776e-06\\
0.2663266	-6.43332200222757e-06\\
0.2664266	-1.05562520638236e-05\\
0.2665267	-2.68609201981483e-06\\
0.2666267	-5.60235814696661e-08\\
0.2667267	-2.68593366699902e-06\\
0.2668267	-1.09382473455928e-05\\
0.2669267	-1.26982459264343e-05\\
0.2670267	-1.42524640870005e-06\\
0.2671267	-8.70261079155989e-06\\
0.2672267	-4.61531407417682e-06\\
0.2673267	-8.60150955706982e-07\\
0.2674267	-2.6750813352996e-06\\
0.2675268	-1.47467430002199e-06\\
0.2676268	-1.11470631730552e-05\\
0.2677268	-8.56760094155554e-06\\
0.2678268	-1.42504786046441e-05\\
0.2679268	-1.42738761614347e-05\\
0.2680268	-6.52328961103385e-06\\
0.2681268	-9.25851795254078e-06\\
0.2682268	-1.93826881850474e-05\\
0.2683268	-1.6665415307638e-05\\
0.2684268	-5.78046231938971e-06\\
0.2685269	-6.6419110621558e-07\\
0.2686269	-5.01488799337678e-06\\
0.2687269	-1.27782166688339e-06\\
0.2688269	-4.58363722578181e-06\\
0.2689269	-9.05070466911065e-06\\
0.2690269	-3.32220899590066e-06\\
0.2691269	-3.04447520517467e-06\\
0.2692269	-2.29812329594756e-06\\
0.2693269	-4.70750426722615e-06\\
0.2694269	-1.35648661180092e-05\\
0.269527	-7.40821432955267e-07\\
0.269627	-4.20508243958751e-06\\
0.269727	-8.56758022670051e-07\\
0.269827	-2.53717828116854e-06\\
0.269927	-2.9895685140402e-06\\
0.270027	-2.12242320233756e-07\\
0.270127	-1.01778359868939e-06\\
0.270227	-2.04057644833862e-06\\
0.270327	-3.60916668104298e-07\\
0.270427	-1.50740475690049e-06\\
0.2705271	-2.20266898022683e-07\\
0.2706271	-4.71666452145296e-06\\
0.2707271	-1.56030807106015e-05\\
0.2708271	-1.30989647643467e-05\\
0.2709271	-3.48354468155762e-06\\
0.2710271	-8.6854216109429e-07\\
0.2711271	-3.23071710180741e-06\\
0.2712271	-4.25285760253836e-06\\
0.2713271	-3.45492396211912e-06\\
0.2714271	-3.17988017937215e-06\\
0.2715272	-1.3818542361119e-06\\
0.2716272	-2.63981216499332e-06\\
0.2717272	-7.38625094795558e-06\\
0.2718272	-7.11953058378248e-06\\
0.2719272	-1.36907700712479e-05\\
0.2720272	-2.62999840911588e-06\\
0.2721272	-3.6318015961403e-06\\
0.2722272	-4.22077731065029e-07\\
0.2723272	-1.45735526237409e-08\\
0.2724272	-3.38600023953985e-06\\
0.2725273	-6.39159779201892e-06\\
0.2726273	-3.16256020562073e-06\\
0.2727273	-5.91382646068647e-06\\
0.2728273	-1.23612555558977e-05\\
0.2729273	-1.37526364899253e-05\\
0.2730273	-4.06559826142946e-06\\
0.2731273	-6.70944986905938e-06\\
0.2732273	-2.89044411453456e-07\\
0.2733273	-1.84869958723906e-06\\
0.2734273	-3.33419595032455e-07\\
0.2735274	-1.71569761329188e-06\\
0.2736274	-2.11042538073334e-06\\
0.2737274	-3.52220097596252e-06\\
0.2738274	-1.19601239755042e-06\\
0.2739274	-9.66323644056503e-07\\
0.2740274	-3.67287671402861e-06\\
0.2741274	-1.87042946060029e-05\\
0.2742274	-8.92164718503554e-07\\
0.2743274	-1.9039288500431e-06\\
0.2744274	-4.00680999121923e-07\\
0.2745275	-6.9445944230085e-07\\
0.2746275	-5.69912532172472e-06\\
0.2747275	-1.81646811411464e-06\\
0.2748275	-6.01834871792207e-06\\
0.2749275	-3.41177613158602e-06\\
0.2750275	-2.10271935353292e-06\\
0.2751275	-5.39508938217652e-06\\
0.2752275	-2.28866121591784e-06\\
0.2753275	-1.4171176853145e-05\\
0.2754275	-3.94719529223316e-06\\
0.2755276	-4.79368850768319e-06\\
0.2756276	-8.15669154536421e-06\\
0.2757276	-3.08076937995125e-06\\
0.2758276	-4.50743900976655e-06\\
0.2759276	-3.55631733118917e-07\\
0.2760276	-2.48378664830353e-06\\
0.2761276	-1.11387365360191e-06\\
0.2762276	-3.44090944728176e-06\\
0.2763276	-1.36925220275969e-05\\
0.2764276	-1.01989663927871e-05\\
0.2765277	-2.53483151511711e-06\\
0.2766277	-6.35888344450058e-06\\
0.2767277	-1.18726081533915e-05\\
0.2768277	-1.61236016399725e-05\\
0.2769277	-1.16858649024116e-05\\
0.2770277	-4.69397793886211e-06\\
0.2771277	-9.66797374746273e-06\\
0.2772277	-5.38979132633705e-06\\
0.2773277	-1.78477056735937e-05\\
0.2774277	-4.83947078732627e-06\\
0.2775278	-2.86561937372686e-07\\
0.2776278	-3.10177827803788e-06\\
0.2777278	-1.71507079365418e-07\\
0.2778278	-8.62666139387011e-07\\
0.2779278	-3.78506675611865e-06\\
0.2780278	-2.96165142756044e-06\\
0.2781278	-1.36295185169654e-06\\
0.2782278	-1.75879202649496e-06\\
0.2783278	-6.0384839499075e-06\\
0.2784278	-1.16659361986959e-06\\
0.2785279	-1.26047700358602e-06\\
0.2786279	-4.95290716012875e-06\\
0.2787279	-3.57397205692579e-06\\
0.2788279	-1.13740369184582e-06\\
0.2789279	-2.92472106274055e-06\\
0.2790279	-3.45547716744454e-06\\
0.2791279	-4.42667400377509e-06\\
0.2792279	-1.03212056953209e-06\\
0.2793279	-3.92289886249791e-06\\
0.2794279	-7.3206438043723e-07\\
0.279528	-6.83491058769806e-06\\
0.279628	-1.09868410485264e-05\\
0.279728	-1.17989772275126e-05\\
0.279828	-1.10274431223494e-05\\
0.279928	-8.67302673071082e-06\\
0.280028	-5.59736205025278e-06\\
0.280128	-7.16597207861232e-06\\
0.280228	-8.04878113407764e-07\\
0.280328	-6.30525225223846e-06\\
0.280428	-8.30142939268471e-06\\
0.2805281	-1.60279961959958e-05\\
0.2806281	-2.35223997320324e-05\\
0.2807281	-3.18505929623069e-05\\
0.2808281	-3.08726188432185e-06\\
0.2809281	-4.35690695559783e-07\\
0.2810281	-1.64556279348337e-06\\
0.2811281	-1.6258117755351e-06\\
0.2812281	-1.01491244391372e-05\\
0.2813281	-7.86944278169144e-06\\
0.2814281	-9.01211780057897e-06\\
0.2815282	-3.59941753688567e-07\\
0.2816282	-4.47049981697266e-06\\
0.2817282	-1.82700384860562e-06\\
0.2818282	-3.12004954588458e-06\\
0.2819282	-6.35190790608527e-06\\
0.2820282	-5.22018692646183e-06\\
0.2821282	-5.65450260424667e-06\\
0.2822282	-8.1624829366503e-06\\
0.2823282	-5.10990492086115e-06\\
0.2824282	-8.60980255404541e-06\\
0.2825283	-1.18920657904481e-05\\
0.2826283	-2.0417451262969e-07\\
0.2827283	-2.11244627514521e-06\\
0.2828283	-6.82227647507011e-06\\
0.2829283	-7.37387694567679e-08\\
0.2830283	-8.97440377533427e-06\\
0.2831283	-1.00898118697082e-05\\
0.2832283	-9.14277189560668e-07\\
0.2833283	-8.95217431849704e-07\\
0.2834283	-1.22139348935095e-05\\
0.2835284	-7.67246642483489e-06\\
0.2836284	-3.11857161555334e-06\\
0.2837284	-3.66084141629598e-06\\
0.2838284	-4.28390623899079e-07\\
0.2839284	-1.19354951739493e-08\\
0.2840284	-2.92168044690678e-06\\
0.2841284	-3.68387865585845e-06\\
0.2842284	-5.6788554587643e-06\\
0.2843284	-1.063801852334e-06\\
0.2844284	-1.67000733251284e-07\\
0.2845285	-6.40765834752939e-06\\
0.2846285	-9.90545349266742e-06\\
0.2847285	-5.67059021504346e-06\\
0.2848285	-5.37771251123561e-06\\
0.2849285	-2.17743037779509e-06\\
0.2850285	-3.18918981124599e-06\\
0.2851285	-2.63205280808509e-06\\
0.2852285	-2.33962036478162e-06\\
0.2853285	-3.39200447777712e-06\\
0.2854285	-5.19984414348513e-06\\
0.2855286	-2.00220630327898e-06\\
0.2856286	-3.13117363083505e-07\\
0.2857286	-5.08560936466774e-06\\
0.2858286	-1.09990220433131e-06\\
0.2859286	-8.32445857834478e-06\\
0.2860286	-5.63020348294948e-06\\
0.2861286	-1.39797814357295e-07\\
0.2862286	-2.04505486875039e-06\\
0.2863286	-8.48185034228101e-06\\
0.2864286	-1.02166133107128e-06\\
0.2865287	-2.73700771467187e-07\\
0.2866287	-2.04146977852678e-06\\
0.2867287	-1.11146228902519e-06\\
0.2868287	-2.31861629896161e-06\\
0.2869287	-2.14808480430394e-06\\
0.2870287	-6.48408000988522e-07\\
0.2871287	-2.44696928491991e-06\\
0.2872287	-9.09311725197065e-06\\
0.2873287	-7.21744869798101e-06\\
0.2874287	-2.21230961875876e-06\\
0.2875288	-6.04664745203821e-07\\
0.2876288	-1.02114528022725e-05\\
0.2877288	-5.89127012133005e-06\\
0.2878288	-2.19209289805165e-06\\
0.2879288	-4.12524628078758e-07\\
0.2880288	-2.39995980701882e-06\\
0.2881288	-3.49426130445013e-07\\
0.2882288	-2.51197649389597e-06\\
0.2883288	4.97809112445779e-09\\
0.2884288	-2.63273792285253e-06\\
0.2885289	-5.33747770882822e-06\\
0.2886289	-3.28847498648441e-06\\
0.2887289	-4.91711068132678e-06\\
0.2888289	-2.54135978868121e-06\\
0.2889289	-2.74359430383719e-06\\
0.2890289	-7.5288092204755e-07\\
0.2891289	-1.87737853852815e-06\\
0.2892289	-2.48241724845764e-06\\
0.2893289	-6.06253034697716e-06\\
0.2894289	-5.87642882919007e-06\\
0.289529	-1.147613761371e-05\\
0.289629	-2.55073808478385e-05\\
0.289729	-2.17704934507358e-05\\
0.289829	-5.51992241735121e-06\\
0.289929	-7.43085474259485e-06\\
0.290029	-5.03361221337223e-07\\
0.290129	-1.44199944840906e-06\\
0.290229	-8.6373881860096e-07\\
0.290329	-5.80612052666309e-06\\
0.290429	-7.66767304882423e-10\\
0.2905291	-1.48881385222415e-06\\
0.2906291	-2.17307854130834e-06\\
0.2907291	-2.84811754684767e-06\\
0.2908291	-2.43811386338554e-06\\
0.2909291	-3.63052848542307e-06\\
0.2910291	-9.6708214074188e-06\\
0.2911291	-3.58834123788357e-07\\
0.2912291	-2.82451228904107e-07\\
0.2913291	-4.71488291709487e-06\\
0.2914291	-3.64903598264559e-06\\
0.2915292	-2.76128329767611e-07\\
0.2916292	-2.07993783197869e-06\\
0.2917292	-2.24283894131997e-07\\
0.2918292	-4.48302510333569e-07\\
0.2919292	-3.70440317464387e-06\\
0.2920292	-6.28638181077465e-07\\
0.2921292	-2.12987862360269e-06\\
0.2922292	-6.48024969614129e-06\\
0.2923292	-6.67818999256807e-06\\
0.2924292	-9.89625150671057e-06\\
0.2925293	-7.43191113467757e-06\\
0.2926293	-8.26351806474525e-06\\
0.2927293	-1.03558511937176e-05\\
0.2928293	-2.59316015152295e-05\\
0.2929293	-2.81042510228666e-05\\
0.2930293	-2.69575197101651e-05\\
0.2931293	-1.43719375706116e-05\\
0.2932293	-9.09954259764233e-06\\
0.2933293	-1.69962178464294e-06\\
0.2934293	-6.82226724948267e-07\\
0.2935294	-6.40051450589928e-06\\
0.2936294	-2.49043813174917e-06\\
0.2937294	-1.62413589059188e-06\\
0.2938294	-3.29372677555459e-06\\
0.2939294	-2.57580977971163e-06\\
0.2940294	-2.71566896084055e-07\\
0.2941294	-4.79975611763931e-06\\
0.2942294	-1.81145843729076e-06\\
0.2943294	-4.20751484789734e-06\\
0.2944294	-6.54715642263136e-07\\
0.2945295	-1.10605967982432e-05\\
0.2946295	-2.09948543738373e-06\\
0.2947295	-2.61372513835512e-06\\
0.2948295	-1.10077078937379e-05\\
0.2949295	-8.25909669605558e-06\\
0.2950295	-1.1544403777449e-07\\
0.2951295	-1.44179941130308e-06\\
0.2952295	-6.84494330899163e-06\\
0.2953295	-7.75359722313181e-06\\
0.2954295	-3.4635781459562e-06\\
0.2955296	-8.32571145058889e-07\\
0.2956296	-3.26978486070006e-06\\
0.2957296	-5.5277907613565e-06\\
0.2958296	-2.54981663902007e-06\\
0.2959296	-5.67850148562138e-06\\
0.2960296	-6.93669229302929e-06\\
0.2961296	-6.19884905305047e-06\\
0.2962296	-4.92828275742893e-06\\
0.2963296	-4.76845039784558e-06\\
0.2964296	-5.79011796591775e-06\\
0.2965297	-7.2086553181422e-06\\
0.2966297	-5.48338871502708e-06\\
0.2967297	-1.32931101402462e-06\\
0.2968297	-3.84151006493021e-07\\
0.2969297	-3.6089462837245e-06\\
0.2970297	-4.55538023694485e-06\\
0.2971297	-6.6685880573129e-06\\
0.2972297	-3.67653273592005e-06\\
0.2973297	-8.62590263789776e-07\\
0.2974297	-2.2367456318771e-06\\
0.2975298	-2.11280368467923e-06\\
0.2976298	-6.5770697046079e-06\\
0.2977298	-2.32510553719433e-06\\
0.2978298	-1.26145717311465e-06\\
0.2979298	-3.03637702974026e-07\\
0.2980298	-6.1575238173061e-06\\
0.2981298	-3.65806980657246e-06\\
0.2982298	-1.92975256116214e-06\\
0.2983298	-1.10080907139106e-06\\
0.2984298	-1.71208927501509e-07\\
0.2985299	-6.52997061017833e-07\\
0.2986299	-8.77516978042057e-07\\
0.2987299	-3.27104711217345e-07\\
0.2988299	-3.32963845048588e-06\\
0.2989299	-1.36174462857135e-05\\
0.2990299	-1.55173080668896e-06\\
0.2991299	-1.94692500312368e-06\\
0.2992299	-1.02515788646509e-05\\
0.2993299	-1.08791503808252e-05\\
0.2994299	-7.34951254112198e-06\\
0.29953	-1.16383991630431e-05\\
0.29963	-4.73154357830855e-06\\
0.29973	-9.57022105630547e-07\\
0.29983	-4.14213023416162e-06\\
0.29993	-1.51805304529721e-05\\
0.30003	-1.18199822510496e-05\\
0.30013	-1.28467526172982e-05\\
0.30023	-7.92565454053818e-06\\
0.30033	-2.56703000950506e-06\\
0.30043	-1.51366301284922e-06\\
0.3005301	-1.11927435291833e-06\\
0.3006301	-1.71423538912968e-06\\
0.3007301	-1.07742839251401e-05\\
0.3008301	-1.33744239492526e-05\\
0.3009301	-1.2332122449682e-05\\
0.3010301	-9.76702041455388e-06\\
0.3011301	-9.88690831904224e-07\\
0.3012301	-3.34323889678697e-07\\
0.3013301	-5.34623475731944e-07\\
0.3014301	-9.25743177826959e-07\\
0.3015302	-1.22010258193713e-06\\
0.3016302	-1.50986207741973e-06\\
0.3017302	4.33234833615131e-09\\
0.3018302	-5.32669019205904e-06\\
0.3019302	-1.05117878589978e-06\\
0.3020302	-4.16813620384822e-07\\
0.3021302	-2.70113198261649e-06\\
0.3022302	-1.34606555959995e-06\\
0.3023302	-7.11041243824254e-06\\
0.3024302	-3.65047260535304e-06\\
0.3025303	-6.76578282921551e-06\\
0.3026303	-2.1198175315454e-06\\
0.3027303	-6.01313348215746e-06\\
0.3028303	-8.27462566745894e-06\\
0.3029303	-4.44522307375489e-06\\
0.3030303	-1.81746866872474e-05\\
0.3031303	-1.7436328494035e-05\\
0.3032303	-1.13055684801115e-05\\
0.3033303	-1.52972863136597e-06\\
0.3034303	-1.01468893358112e-06\\
0.3035304	-2.34243813593458e-06\\
0.3036304	-4.09767269510709e-06\\
0.3037304	-4.59615561930936e-07\\
0.3038304	-1.08257812175733e-06\\
0.3039304	-3.64491859827654e-07\\
0.3040304	-4.01605486127264e-06\\
0.3041304	-1.41489481111159e-06\\
0.3042304	-2.42004479425154e-06\\
0.3043304	-2.08105079548647e-06\\
0.3044304	-3.64926894965003e-08\\
0.3045305	-1.00303053482586e-06\\
0.3046305	-1.25327795930714e-07\\
0.3047305	-2.98777141314372e-06\\
0.3048305	-9.59375170681068e-07\\
0.3049305	-3.74564205264088e-06\\
0.3050305	-6.78325430023731e-08\\
0.3051305	-1.14010832562502e-06\\
0.3052305	-2.43357884247646e-07\\
0.3053305	-1.36927210248759e-06\\
0.3054305	-1.0941592638398e-06\\
0.3055306	-8.7246629745704e-06\\
0.3056306	-5.27382076993943e-06\\
0.3057306	-1.57828923581456e-05\\
0.3058306	-9.31643572218525e-06\\
0.3059306	-1.2907568449281e-06\\
0.3060306	-4.34005870911594e-06\\
0.3061306	-7.32584729736193e-06\\
0.3062306	-8.70783159214963e-06\\
0.3063306	-1.70095357583203e-06\\
0.3064306	-5.95041523063067e-06\\
0.3065307	-4.08105923876208e-06\\
0.3066307	-1.01717161795532e-05\\
0.3067307	-9.01240773730861e-06\\
0.3068307	-1.05729168937142e-05\\
0.3069307	-6.7532266303196e-06\\
0.3070307	-3.12669992853735e-06\\
0.3071307	-3.41042964161071e-09\\
0.3072307	-1.40451813476736e-06\\
0.3073307	-9.72791400490934e-06\\
0.3074307	-1.63398503609211e-05\\
0.3075308	-7.54417285906198e-06\\
0.3076308	-3.04237412624114e-06\\
0.3077308	-1.5314568209734e-06\\
0.3078308	-3.46365435370034e-08\\
0.3079308	-1.01180941406383e-06\\
0.3080308	-2.17855672538334e-07\\
0.3081308	-5.65035578796522e-07\\
0.3082308	-3.81450101252485e-06\\
0.3083308	-4.64140385325919e-06\\
0.3084308	-9.75314980383715e-07\\
0.3085309	-2.19641502214707e-06\\
0.3086309	-7.61034565682655e-06\\
0.3087309	-3.74734251510665e-06\\
0.3088309	-5.23611457575378e-06\\
0.3089309	-1.91659281737714e-06\\
0.3090309	-1.18169412184276e-05\\
0.3091309	-8.93877475719679e-06\\
0.3092309	-3.05649341181549e-06\\
0.3093309	3.92793747470445e-08\\
0.3094309	-4.85408898031552e-06\\
0.309531	-1.56554447003049e-06\\
0.309631	-3.03045136312044e-06\\
0.309731	-9.30121426024289e-06\\
0.309831	-3.71430113854088e-06\\
0.309931	-7.21270397498867e-06\\
0.310031	-1.35107037463906e-05\\
0.310131	-8.17109942937975e-06\\
0.310231	-3.03180600041695e-06\\
0.310331	-1.40496454357893e-05\\
0.310431	-8.49617471160924e-06\\
0.3105311	-1.19101703933036e-05\\
0.3106311	-1.33295682744299e-05\\
0.3107311	-1.99941529232545e-05\\
0.3108311	-7.40479631517785e-06\\
0.3109311	-4.91525742541898e-06\\
0.3110311	-1.64986022901457e-06\\
0.3111311	-3.41198270081757e-06\\
0.3112311	-6.50818381549591e-06\\
0.3113311	-3.71058054753116e-06\\
0.3114311	-7.16658187121726e-06\\
0.3115312	-9.28222531682103e-06\\
0.3116312	-6.69484074245979e-06\\
0.3117312	-3.06262681465444e-07\\
0.3118312	-1.01515310736767e-06\\
0.3119312	-1.40951099350174e-06\\
0.3120312	-4.40326831300715e-06\\
0.3121312	-2.4733700388262e-06\\
0.3122312	-1.43755614370266e-06\\
0.3123312	-4.73466100180305e-07\\
0.3124312	-5.82663738060154e-06\\
0.3125313	-1.22020097731931e-06\\
0.3126313	-2.27248231809619e-06\\
0.3127313	-2.31014489859291e-06\\
0.3128313	-3.4463706903318e-06\\
0.3129313	-1.7090416646266e-06\\
0.3130313	-5.04123192580963e-07\\
0.3131313	-1.89342904508699e-06\\
0.3132313	-2.71745439282374e-06\\
0.3133313	-3.08014880625572e-06\\
0.3134313	-8.07741325563137e-06\\
0.3135314	-6.92673119242783e-06\\
0.3136314	-3.83570361952488e-06\\
0.3137314	-5.75243299196876e-06\\
0.3138314	-6.18675079128086e-07\\
0.3139314	-1.21743545014773e-06\\
0.3140314	-4.19035273947171e-07\\
0.3141314	-3.59556131921897e-06\\
0.3142314	-2.33238595442712e-06\\
0.3143314	-6.2437983478055e-06\\
0.3144314	-7.51216646735619e-06\\
0.3145315	-1.52494617204954e-05\\
0.3146315	-3.5064511911069e-06\\
0.3147315	-1.54337429045724e-06\\
0.3148315	-3.26818685604796e-07\\
0.3149315	-4.57164743367928e-07\\
0.3150315	-2.88853903032329e-06\\
0.3151315	-3.83434031280413e-06\\
0.3152315	-2.33611005689859e-06\\
0.3153315	-9.26940822844797e-06\\
0.3154315	-1.9863388793045e-05\\
0.3155316	-2.87114551106204e-05\\
0.3156316	-1.0343867352394e-05\\
0.3157316	-1.16475488244929e-06\\
0.3158316	-1.08896066536708e-06\\
0.3159316	2.92955245293689e-08\\
0.3160316	-4.23655146824072e-07\\
0.3161316	-1.53697117323051e-06\\
0.3162316	-4.11762208230222e-07\\
0.3163316	-1.13838811509882e-06\\
0.3164316	-5.4921956845375e-08\\
0.3165317	-3.90419742233773e-07\\
0.3166317	-2.13019236629334e-07\\
0.3167317	-1.44859745327437e-06\\
0.3168317	-5.85806754092859e-07\\
0.3169317	-3.10202190073282e-06\\
0.3170317	-2.35964085456437e-06\\
0.3171317	-5.11114457667783e-06\\
0.3172317	-9.34588002788172e-06\\
0.3173317	-5.42849416870082e-06\\
0.3174317	-4.6376469593742e-06\\
0.3175318	-5.49601465354245e-06\\
0.3176318	-1.38897126180379e-05\\
0.3177318	-3.96019211132977e-06\\
0.3178318	-6.12043109249387e-06\\
0.3179318	-4.83527452031025e-06\\
0.3180318	-1.28826235326109e-06\\
0.3181318	-5.09101554952859e-06\\
0.3182318	-2.75818206699297e-06\\
0.3183318	-2.971584632303e-07\\
0.3184318	-1.4099958955104e-06\\
0.3185319	-3.52974435809919e-06\\
0.3186319	-5.30507272716658e-06\\
0.3187319	-3.3731270218363e-07\\
0.3188319	-2.00566873917348e-06\\
0.3189319	-5.74042829384223e-06\\
0.3190319	-3.40820182157676e-06\\
0.3191319	-3.32444527744253e-06\\
0.3192319	9.63647381868052e-09\\
0.3193319	-6.46788992208988e-07\\
0.3194319	-3.69889975961324e-06\\
0.319532	-1.73720364871779e-06\\
0.319632	-1.71640705348938e-06\\
0.319732	-7.25494909861756e-07\\
0.319832	-3.6943747705867e-06\\
0.319932	-2.78223798807634e-06\\
0.320032	-7.00055014400717e-07\\
0.320132	-2.1210779012855e-06\\
0.320232	-6.98693900109572e-07\\
0.320332	-8.05291061902607e-07\\
0.320432	-1.1104597373427e-06\\
0.3205321	-1.39620838790527e-06\\
0.3206321	-4.11177034443968e-07\\
0.3207321	-2.73875804447947e-06\\
0.3208321	1.15133573741335e-07\\
0.3209321	-5.25879051664533e-07\\
0.3210321	-3.72205224621365e-06\\
0.3211321	-9.64417299053626e-07\\
0.3212321	-2.15361325796062e-06\\
0.3213321	-1.29216707033672e-06\\
0.3214321	-4.2897726832074e-06\\
0.3215322	-1.01691808393096e-06\\
0.3216322	-2.41389913001413e-06\\
0.3217322	-6.42304715328225e-07\\
0.3218322	-2.69617885365734e-07\\
0.3219322	-5.58993688522937e-06\\
0.3220322	-5.44421815962942e-06\\
0.3221322	-3.93922085288098e-06\\
0.3222322	-3.94170090890134e-06\\
0.3223322	-1.37458927120715e-06\\
0.3224322	-1.73199188291173e-06\\
0.3225323	-4.83466065159903e-06\\
0.3226323	-1.24537685819193e-05\\
0.3227323	-1.3512581588472e-05\\
0.3228323	-2.61992861272896e-06\\
0.3229323	-1.02835659574433e-06\\
0.3230323	-5.17478247815214e-06\\
0.3231323	-2.74658732001633e-05\\
0.3232323	-4.40001357015625e-05\\
0.3233323	-3.66308799217059e-05\\
0.3234323	-2.50114657995174e-05\\
0.3235324	-2.7312051567334e-06\\
0.3236324	-2.16303215641393e-06\\
0.3237324	-1.03462062785111e-06\\
0.3238324	-5.49238650821121e-06\\
0.3239324	-7.15331573421369e-06\\
0.3240324	-7.24541724212803e-06\\
0.3241324	-3.46460996777077e-06\\
0.3242324	-1.58413684650232e-06\\
0.3243324	-1.11785518132239e-05\\
0.3244324	-8.36367080237441e-06\\
0.3245325	-1.08977795433243e-05\\
0.3246325	-1.41643203696406e-05\\
0.3247325	-1.79880720188295e-05\\
0.3248325	-5.41821442344826e-06\\
0.3249325	-1.00328875155759e-05\\
0.3250325	-8.32781726809714e-07\\
0.3251325	-8.43817248826218e-06\\
0.3252325	-3.98806123055757e-06\\
0.3253325	-1.62915238382869e-06\\
0.3254325	-1.04260608777135e-05\\
0.3255326	-7.69889334222149e-06\\
0.3256326	-3.08720729441363e-06\\
0.3257326	-7.34393137305823e-06\\
0.3258326	-7.71738250577942e-06\\
0.3259326	-8.76768161968941e-06\\
0.3260326	-4.2580794138503e-07\\
0.3261326	-5.6718374969442e-06\\
0.3262326	-1.50098211192243e-06\\
0.3263326	-1.73150041134927e-06\\
0.3264326	-2.46230931972472e-06\\
0.3265327	-2.37809836019266e-06\\
0.3266327	-4.63606024724707e-06\\
0.3267327	-5.30139751345946e-06\\
0.3268327	-2.87593508117499e-06\\
0.3269327	-8.70966187219107e-06\\
0.3270327	-2.56312580775365e-06\\
0.3271327	-1.2729368085535e-06\\
0.3272327	-3.61294679472243e-06\\
0.3273327	-6.32224268582955e-06\\
0.3274327	-2.81496640087739e-06\\
0.3275328	-9.34655348093739e-07\\
0.3276328	-1.16332045436426e-06\\
0.3277328	-4.15762413806246e-06\\
0.3278328	-2.28274131588555e-06\\
0.3279328	-5.84191490394481e-06\\
0.3280328	-8.4344878177616e-06\\
0.3281328	-3.90646897226336e-06\\
0.3282328	-9.31675728177965e-06\\
0.3283328	-5.74937366003805e-06\\
0.3284328	-2.60573702016017e-06\\
0.3285329	-6.99636646823775e-07\\
0.3286329	-5.82868795355576e-07\\
0.3287329	-9.30666461349787e-07\\
0.3288329	-4.26380185546221e-06\\
0.3289329	-1.90740778772192e-06\\
0.3290329	-6.95753916752709e-06\\
0.3291329	-2.87393103640668e-07\\
0.3292329	-1.26529890418618e-06\\
0.3293329	-3.73715807664336e-06\\
0.3294329	-7.06796932784387e-06\\
0.329533	-2.65523480965752e-06\\
0.329633	-4.12903523068419e-07\\
0.329733	-6.04897483168988e-06\\
0.329833	-3.67816543971666e-06\\
0.329933	-1.98396265067334e-06\\
0.330033	-3.31391567409856e-07\\
0.330133	-1.37109692096917e-07\\
0.330233	-1.00240892622149e-06\\
0.330333	-8.21051570582213e-07\\
0.330433	-4.42218832528473e-06\\
0.3305331	-4.03505999465694e-07\\
0.3306331	1.87367171768754e-07\\
0.3307331	-4.90704823348877e-07\\
0.3308331	-2.12140549210495e-06\\
0.3309331	-5.39185261069318e-07\\
0.3310331	9.71634739094325e-08\\
0.3311331	1.19865017708842e-07\\
0.3312331	-2.31246512406257e-06\\
0.3313331	-2.70359224506044e-06\\
0.3314331	-1.65309843819961e-06\\
0.3315332	-1.12130555924508e-06\\
0.3316332	5.12082427754305e-08\\
0.3317332	-2.1232743014858e-06\\
0.3318332	-1.98550881929897e-07\\
0.3319332	-4.99725958769211e-06\\
0.3320332	-4.07281190713672e-06\\
0.3321332	-9.55045278518996e-08\\
0.3322332	-2.43001033664452e-06\\
0.3323332	-3.18412621953496e-06\\
0.3324332	-5.75861306175184e-06\\
0.3325333	-9.24278554287757e-07\\
0.3326333	-1.42088539513189e-05\\
0.3327333	-3.22744095844133e-06\\
0.3328333	1.01919842337146e-07\\
0.3329333	-1.73374933017272e-06\\
0.3330333	-3.31208545633628e-06\\
0.3331333	-1.13820371568949e-06\\
0.3332333	-2.34013198693341e-06\\
0.3333333	-1.73798714792864e-06\\
0.3334333	-1.92894207568972e-06\\
0.3335334	-7.87226284221482e-07\\
0.3336334	-1.96027835560118e-06\\
0.3337334	-3.61954181953177e-06\\
0.3338334	-4.19845554958622e-06\\
0.3339334	-5.51020041846295e-06\\
0.3340334	-1.90691329798011e-06\\
0.3341334	-3.93157805906969e-06\\
0.3342334	-2.46226457177165e-06\\
0.3343334	-1.67202870522804e-06\\
0.3344334	-6.71179732767701e-06\\
0.3345335	-1.392012576305e-05\\
0.3346335	-5.89358594570866e-06\\
0.3347335	-6.76496921645613e-06\\
0.3348335	-7.07351398538415e-08\\
0.3349335	-2.65993247952906e-06\\
0.3350335	-6.19473419816869e-06\\
0.3351335	-6.08979445751305e-06\\
0.3352335	-6.4695471183496e-06\\
0.3353335	-1.31526304050667e-06\\
0.3354335	-1.06219708284711e-06\\
0.3355336	-9.83505736521369e-06\\
0.3356336	-1.43119392003787e-05\\
0.3357336	-1.95189707263143e-05\\
0.3358336	-5.78262079795623e-06\\
0.3359336	-3.17544926924268e-06\\
0.3360336	-2.54370499310732e-06\\
0.3361336	-3.96001782147295e-06\\
0.3362336	-2.69055560524474e-06\\
0.3363336	-5.65742419430354e-06\\
0.3364336	-3.11906243749917e-06\\
0.3365337	-8.2971923557788e-07\\
0.3366337	-3.92844730771004e-06\\
0.3367337	-1.89830174119221e-07\\
0.3368337	-2.62601587946165e-06\\
0.3369337	-4.74097006732814e-06\\
0.3370337	-1.52214498023709e-06\\
0.3371337	2.13427140372578e-07\\
0.3372337	-6.9098803458515e-06\\
0.3373337	-2.23029347816799e-06\\
0.3374337	4.00678435265272e-07\\
0.3375338	-3.76682266113372e-06\\
0.3376338	-2.45913553290946e-06\\
0.3377338	-3.89836999656546e-06\\
0.3378338	-1.17335788857822e-05\\
0.3379338	-2.18229550331029e-05\\
0.3380338	-1.97039272699255e-05\\
0.3381338	-4.07875742649603e-06\\
0.3382338	-8.7372833190022e-07\\
0.3383338	-1.02495681405643e-06\\
0.3384338	-1.58002997078102e-08\\
0.3385339	-5.434221402073e-06\\
0.3386339	-1.34983255451702e-05\\
0.3387339	-1.16434055646865e-05\\
0.3388339	-1.59417302825859e-05\\
0.3389339	-1.01734595196182e-05\\
0.3390339	-3.01346630953112e-05\\
0.3391339	-1.80917408279629e-05\\
0.3392339	-1.2044952534633e-05\\
0.3393339	-2.68716303113516e-06\\
0.3394339	-3.12256713202901e-06\\
0.339534	-3.08712497976309e-06\\
0.339634	-1.73130470119609e-06\\
0.339734	-3.0594936290197e-07\\
0.339834	-1.92368407433974e-06\\
0.339934	-1.3929163436728e-06\\
0.340034	-3.98552607776049e-06\\
0.340134	-3.33685008214956e-06\\
0.340234	-1.28582116106572e-06\\
0.340334	7.31982595028811e-10\\
0.340434	-2.78652875272499e-06\\
0.3405341	-3.34936291900922e-06\\
0.3406341	-1.04836602827439e-05\\
0.3407341	-1.77389187218119e-05\\
0.3408341	-3.57287803232874e-06\\
0.3409341	-6.35966009027545e-07\\
0.3410341	-9.31374444525006e-06\\
0.3411341	-2.60637032937667e-07\\
0.3412341	-8.50963862622323e-07\\
0.3413341	-1.10644542341738e-06\\
0.3414341	-6.64359160300844e-06\\
0.3415342	-6.52787941844182e-07\\
0.3416342	-1.10727946854082e-05\\
0.3417342	-2.58834340194816e-06\\
0.3418342	-7.16826187334394e-06\\
0.3419342	-4.33489880000889e-07\\
0.3420342	-6.23666720084017e-06\\
0.3421342	-3.8326816132892e-06\\
0.3422342	-5.322088893272e-06\\
0.3423342	-5.85882181519948e-06\\
0.3424342	-4.23633715195962e-06\\
0.3425343	-1.26224161099306e-05\\
0.3426343	-8.90128555719616e-07\\
0.3427343	-2.54121972553674e-06\\
0.3428343	-3.40105886077917e-07\\
0.3429343	-8.1000430246636e-07\\
0.3430343	3.58490961757541e-07\\
0.3431343	-1.30367395535357e-06\\
0.3432343	-8.56709721414376e-06\\
0.3433343	-5.72913777334346e-06\\
0.3434343	-5.49869739005909e-06\\
0.3435344	-9.37840991249917e-06\\
0.3436344	-1.06140548734606e-05\\
0.3437344	-1.77570315315961e-06\\
0.3438344	-7.37871550209971e-06\\
0.3439344	-7.57729166910746e-06\\
0.3440344	-4.03407940132171e-06\\
0.3441344	-4.79622144418272e-06\\
0.3442344	2.64680558578694e-07\\
0.3443344	4.35802064952186e-07\\
0.3444344	-1.13081965341101e-07\\
0.3445345	-1.14672529657393e-06\\
0.3446345	-2.6464359759354e-06\\
0.3447345	-5.19574020339425e-08\\
0.3448345	-1.26698730810886e-06\\
0.3449345	-4.33353842561172e-06\\
0.3450345	-1.08584248419489e-06\\
0.3451345	-2.74980621170004e-06\\
0.3452345	-4.1132023341465e-06\\
0.3453345	-1.35554057571965e-06\\
0.3454345	4.67787641240844e-07\\
0.3455346	-4.72880563605912e-06\\
0.3456346	-1.37918575207475e-05\\
0.3457346	-2.51270284024443e-05\\
0.3458346	-1.32680079959857e-05\\
0.3459346	-1.47224801430249e-06\\
0.3460346	-3.72002516840758e-06\\
0.3461346	-1.46882167384181e-07\\
0.3462346	3.52484781626721e-07\\
0.3463346	-4.30916252656125e-06\\
0.3464346	-4.03514829516744e-06\\
0.3465347	-3.57608863607077e-06\\
0.3466347	-6.0855433834501e-06\\
0.3467347	7.48934601326205e-07\\
0.3468347	-3.1173574696477e-07\\
0.3469347	-2.98890975151543e-06\\
0.3470347	4.16921806525288e-07\\
0.3471347	7.82087858064899e-07\\
0.3472347	4.93520656080458e-07\\
0.3473347	3.57048275631734e-07\\
0.3474347	-5.01072358612558e-06\\
0.3475348	2.88689531299006e-07\\
0.3476348	2.88058413763001e-07\\
0.3477348	-8.30444720580642e-07\\
0.3478348	-6.72837654605629e-06\\
0.3479348	-1.1648872634825e-05\\
0.3480348	-1.30993776568706e-05\\
0.3481348	-6.72954827998621e-06\\
0.3482348	-3.37432169760081e-07\\
0.3483348	-4.1400419895617e-06\\
0.3484348	-1.85165404005277e-05\\
0.3485349	-1.48958630361947e-05\\
0.3486349	-3.5717195536316e-06\\
0.3487349	6.7598566824254e-07\\
0.3488349	-3.64685492263308e-06\\
0.3489349	-6.97017076014899e-07\\
0.3490349	-3.74825573934038e-06\\
0.3491349	-6.74204557713344e-07\\
0.3492349	-1.34983717388926e-06\\
0.3493349	-8.52239522826035e-06\\
0.3494349	-6.36296335884072e-06\\
0.349535	-7.11533165730075e-06\\
0.349635	-2.20524479091491e-06\\
0.349735	-5.7378789999077e-07\\
0.349835	-5.62934861287639e-06\\
0.349935	-7.10267755546568e-06\\
0.350035	-1.5383740351183e-05\\
0.350135	-2.04226656209674e-05\\
0.350235	-4.6623909832574e-06\\
0.350335	-1.09741180539747e-05\\
0.350435	-1.84332444650864e-06\\
0.3505351	-6.8756636723328e-06\\
0.3506351	-3.072870480803e-06\\
0.3507351	-4.14651843550824e-06\\
0.3508351	-1.89743213954543e-06\\
0.3509351	1.09247603660162e-06\\
0.3510351	-3.03877719492149e-06\\
0.3511351	-7.39871273932313e-06\\
0.3512351	6.92558780850886e-07\\
0.3513351	-2.82887782426597e-06\\
0.3514351	-5.78138541844072e-07\\
0.3515352	-4.02291462233822e-07\\
0.3516352	-1.21274936449567e-08\\
0.3517352	1.12434417740102e-07\\
0.3518352	-8.72778304291757e-07\\
0.3519352	5.45576682806263e-09\\
0.3520352	-4.05118939532747e-07\\
0.3521352	3.56197808810119e-07\\
0.3522352	-1.48639795312478e-06\\
0.3523352	-4.61844238746693e-06\\
0.3524352	-3.453649853476e-06\\
0.3525353	-3.33388557669044e-06\\
0.3526353	9.76008493839644e-07\\
0.3527353	-3.6639615272265e-06\\
0.3528353	1.22767034251409e-06\\
0.3529353	-1.54123143157504e-06\\
0.3530353	-4.44522960115126e-06\\
0.3531353	-3.60737163487434e-06\\
0.3532353	-1.94291106838764e-06\\
0.3533353	-8.58097943429941e-06\\
0.3534353	-3.70395826216386e-06\\
0.3535354	-5.35593066078725e-07\\
0.3536354	1.60794676528205e-07\\
0.3537354	-1.63885716124836e-05\\
0.3538354	-1.14482814502575e-05\\
0.3539354	-4.71984135078655e-06\\
0.3540354	-1.30440648248939e-05\\
0.3541354	-9.54089738021295e-06\\
0.3542354	-1.19576865211677e-05\\
0.3543354	-1.44637017489528e-05\\
0.3544354	-5.31594156151315e-06\\
0.3545355	-2.1960377117491e-06\\
0.3546355	-6.80838809891175e-06\\
0.3547355	-6.7878965444722e-06\\
0.3548355	-4.84830853317133e-06\\
0.3549355	-5.00899254639848e-06\\
0.3550355	-1.10537650621706e-05\\
0.3551355	-8.80669055511136e-06\\
0.3552355	-4.07895649643026e-06\\
0.3553355	-3.8439963539013e-06\\
0.3554355	8.85772208158129e-07\\
0.3555356	3.19219851129794e-07\\
0.3556356	-4.80036544579397e-06\\
0.3557356	-2.09470049463678e-06\\
0.3558356	1.40186155432943e-06\\
0.3559356	-2.48856964560233e-06\\
0.3560356	-3.99656463755218e-06\\
0.3561356	5.07520838968903e-07\\
0.3562356	1.01662674808131e-06\\
0.3563356	2.80204457559199e-07\\
0.3564356	-2.00443426114693e-06\\
0.3565357	-4.05196987575006e-06\\
0.3566357	-3.47772843503718e-06\\
0.3567357	-5.581454085287e-06\\
0.3568357	-5.51999924011191e-06\\
0.3569357	-2.7524273093332e-06\\
0.3570357	6.19717301042175e-07\\
0.3571357	1.59615068884514e-06\\
0.3572357	-4.79549404423189e-06\\
0.3573357	-3.50336379261126e-06\\
0.3574357	-6.49687344680622e-06\\
0.3575358	-4.47838414323803e-06\\
0.3576358	-3.24028667221185e-06\\
0.3577358	1.02679054577807e-06\\
0.3578358	-1.11235192638994e-05\\
0.3579358	-1.61883020718445e-05\\
0.3580358	-1.26988000446023e-05\\
0.3581358	-1.88106325446376e-05\\
0.3582358	-3.15771089303095e-05\\
0.3583358	-2.52311285558472e-05\\
0.3584358	-3.53171177132348e-06\\
0.3585359	-5.14051717481165e-07\\
0.3586359	-1.78560704725028e-06\\
0.3587359	-9.08351991322711e-07\\
0.3588359	-5.90022588288808e-06\\
0.3589359	-5.49155405085145e-06\\
0.3590359	-2.32874981980708e-06\\
0.3591359	-2.29743751001211e-06\\
0.3592359	-6.22599437360064e-07\\
0.3593359	-1.76297791335413e-06\\
0.3594359	1.23013965491957e-06\\
0.359536	-4.10114550896394e-06\\
0.359636	2.03222545028659e-06\\
0.359736	-5.27596693997181e-06\\
0.359836	-1.56090056888109e-06\\
0.359936	8.67431478970329e-07\\
0.360036	-3.44261907642472e-06\\
0.360136	1.49750068953575e-06\\
0.360236	-2.16770409390899e-06\\
0.360336	-5.34545109285421e-06\\
0.360436	-6.4389937687013e-06\\
0.3605361	-4.6955640604001e-06\\
0.3606361	-7.74796574214624e-06\\
0.3607361	-9.2428378557936e-06\\
0.3608361	-2.66467156436761e-05\\
0.3609361	-8.11473834328923e-06\\
0.3610361	-7.39926518726091e-06\\
0.3611361	-5.12286140332172e-06\\
0.3612361	-6.39612321427528e-06\\
0.3613361	1.96245076203168e-06\\
0.3614361	9.13684512739388e-07\\
0.3615362	-2.60561288566867e-06\\
0.3616362	-1.50804008367534e-06\\
0.3617362	-7.55794914831013e-07\\
0.3618362	9.91922428261465e-07\\
0.3619362	-2.87279241860866e-07\\
0.3620362	-3.52482910748999e-06\\
0.3621362	-1.95999734571565e-06\\
0.3622362	-1.41817212839382e-06\\
0.3623362	6.68119377884878e-07\\
0.3624362	-1.1130765988173e-05\\
0.3625363	-7.75690065159302e-06\\
0.3626363	-1.02480150956072e-05\\
0.3627363	5.95358137123395e-07\\
0.3628363	-5.96733093178511e-07\\
0.3629363	-6.59572920831182e-07\\
0.3630363	1.29920152533913e-06\\
0.3631363	1.26032212203131e-06\\
0.3632363	2.91689751502606e-07\\
0.3633363	2.33628650160221e-06\\
0.3634363	1.83381136580489e-06\\
0.3635364	1.62680880604886e-06\\
0.3636364	-6.89146360373337e-07\\
0.3637364	-4.98673999726161e-06\\
0.3638364	2.16363721170752e-06\\
0.3639364	2.8776055186544e-06\\
0.3640364	1.53428346153115e-06\\
0.3641364	-5.34894683584338e-06\\
0.3642364	-1.13328009637483e-05\\
0.3643364	-1.03737899765254e-05\\
0.3644364	-2.23449225431318e-08\\
0.3645365	9.67055641323435e-07\\
0.3646365	-1.10696011447376e-05\\
0.3647365	-1.99533039719746e-05\\
0.3648365	-1.18575778644975e-05\\
0.3649365	-4.9681598402618e-06\\
0.3650365	-2.23808191103175e-06\\
0.3651365	-3.38552082343516e-07\\
0.3652365	-1.92304035346764e-06\\
0.3653365	-1.10797071737174e-06\\
0.3654365	2.06391683931728e-06\\
0.3655366	1.0282534123837e-07\\
0.3656366	2.59128962267782e-07\\
0.3657366	9.58564583469846e-07\\
0.3658366	-2.18959827560319e-05\\
0.3659366	-1.18248520105813e-05\\
0.3660366	-1.34327101279511e-05\\
0.3661366	-4.10956204930325e-06\\
0.3662366	-1.11974387091497e-05\\
0.3663366	-1.37631470353127e-05\\
0.3664366	-1.00322999488851e-05\\
0.3665367	-9.33903736908786e-07\\
0.3666367	6.57363606680372e-07\\
0.3667367	3.38844704146383e-06\\
0.3668367	2.49841247367699e-06\\
0.3669367	3.29776631648754e-06\\
0.3670367	1.02064999003666e-06\\
0.3671367	1.5092189214805e-06\\
0.3672367	3.13774074503186e-06\\
0.3673367	5.02078802002057e-07\\
0.3674367	2.62820304084316e-06\\
0.3675368	-1.96284462348811e-06\\
0.3676368	-6.7485675672892e-06\\
0.3677368	-2.43418043942011e-06\\
0.3678368	3.68220937567696e-07\\
0.3679368	1.49306748477737e-07\\
0.3680368	-1.58438588144878e-05\\
0.3681368	-3.08820875516841e-05\\
0.3682368	-1.004289025598e-05\\
0.3683368	-9.56058571271339e-06\\
0.3684368	-1.20216126996467e-05\\
0.3685369	-2.25794778852376e-07\\
0.3686369	-9.5501379373879e-06\\
0.3687369	-1.34366619122598e-05\\
0.3688369	-1.17544224504705e-05\\
0.3689369	-1.15775729122034e-06\\
0.3690369	7.76511834138245e-07\\
0.3691369	-1.75371479785265e-06\\
0.3692369	2.97765709729004e-06\\
0.3693369	3.48756631203718e-06\\
0.3694369	1.48742646894467e-07\\
0.369537	1.01281091467952e-07\\
0.369637	2.2001063048467e-06\\
0.369737	-3.67822341013382e-06\\
0.369837	2.2801227965623e-07\\
0.369937	2.02852371562099e-06\\
0.370037	-1.02187927525123e-05\\
0.370137	-3.4055107666445e-06\\
0.370237	-8.91226960255133e-07\\
0.370337	-1.39386495835311e-06\\
0.370437	5.30470622573129e-07\\
0.3705371	1.3162794607181e-06\\
0.3706371	-8.90593973645203e-08\\
0.3707371	4.30285722715002e-06\\
0.3708371	2.55596865236695e-06\\
0.3709371	-1.1660209594835e-05\\
0.3710371	-8.60585077875212e-06\\
0.3711371	-5.29703745480429e-06\\
0.3712371	-3.26692169482644e-07\\
0.3713371	-7.0011614602975e-06\\
0.3714371	-1.51564858557257e-05\\
0.3715372	-1.60153853433774e-05\\
0.3716372	-4.87945026445008e-06\\
0.3717372	-1.00450858193324e-05\\
0.3718372	-1.64863594998266e-05\\
0.3719372	4.6475021156553e-07\\
0.3720372	-1.27301158302507e-07\\
0.3721372	-6.01577607316771e-06\\
0.3722372	4.62799341269753e-06\\
0.3723372	3.14004165454056e-06\\
0.3724372	-4.73044582815907e-07\\
0.3725373	-2.10588831979618e-05\\
0.3726373	-1.41368822125932e-05\\
0.3727373	-4.661952352346e-06\\
0.3728373	2.40885798647493e-06\\
0.3729373	-6.38583958257791e-06\\
0.3730373	-7.8522734360372e-06\\
0.3731373	-1.0251961940463e-05\\
0.3732373	-8.57050345238468e-06\\
0.3733373	-9.95955631824295e-06\\
0.3734373	-5.45963887433087e-06\\
0.3735374	-5.84353996562336e-06\\
0.3736374	4.10464394349254e-07\\
0.3737374	3.93798011838856e-06\\
0.3738374	4.39889391139445e-06\\
0.3739374	5.58154878871238e-06\\
0.3740374	-9.75116242380621e-06\\
0.3741374	-7.86149508974292e-06\\
0.3742374	2.24860123794887e-06\\
0.3743374	2.66250431700542e-06\\
0.3744374	1.33998291591465e-06\\
0.3745375	4.46210401826236e-06\\
0.3746375	-3.52344771235244e-06\\
0.3747375	-1.28752325507106e-05\\
0.3748375	-1.62229496846813e-05\\
0.3749375	-1.57257402910672e-05\\
0.3750375	-1.2053322535541e-05\\
0.3751375	-3.64275725812872e-08\\
0.3752375	-8.70434454540736e-06\\
0.3753375	-7.47949058591465e-06\\
0.3754375	-9.9974681460936e-07\\
0.3755376	-4.88843823343425e-06\\
0.3756376	-1.24671608535752e-05\\
0.3757376	1.02911004749184e-06\\
0.3758376	4.72993739548978e-06\\
0.3759376	-2.01623087213732e-06\\
0.3760376	-3.67200680615668e-06\\
0.3761376	-1.2957292445479e-05\\
0.3762376	-1.37240281709028e-06\\
0.3763376	-7.89835793598319e-06\\
0.3764376	-5.15850380508832e-06\\
0.3765377	-3.1510442143529e-06\\
0.3766377	-1.20896932237822e-05\\
0.3767377	-4.10855091710764e-06\\
0.3768377	-6.71405248301932e-07\\
0.3769377	-6.81445415892538e-06\\
0.3770377	-8.00741557805468e-06\\
0.3771377	-2.53498422211379e-07\\
0.3772377	4.94663040471014e-06\\
0.3773377	5.55081501151482e-06\\
0.3774377	-4.37010448022008e-06\\
0.3775378	-8.37669543862537e-06\\
0.3776378	1.66272663000875e-06\\
0.3777378	2.59025004436726e-06\\
0.3778378	6.05934997786396e-06\\
0.3779378	4.06645461705416e-06\\
0.3780378	3.79519516170986e-06\\
0.3781378	-7.54342175104692e-07\\
0.3782378	2.94259783304267e-06\\
0.3783378	-1.40107225739715e-05\\
0.3784378	-1.63410641427498e-05\\
0.3785379	-9.57069059676701e-06\\
0.3786379	-1.93393193108788e-05\\
0.3787379	-2.24702933423617e-05\\
0.3788379	-1.95593503829452e-05\\
0.3789379	-6.85660110448371e-07\\
0.3790379	-1.14265941887002e-05\\
0.3791379	-9.3649162674602e-06\\
0.3792379	-1.05863819823389e-05\\
0.3793379	-1.75780895471649e-06\\
0.3794379	-5.03219479166275e-06\\
0.379538	-4.50636833537908e-06\\
0.379638	-1.49804427784043e-06\\
0.379738	6.01636818304353e-06\\
0.379838	5.53059849827781e-06\\
0.379938	-1.137239866411e-06\\
0.380038	6.41871056950873e-06\\
0.380138	-2.33713869854249e-06\\
0.380238	-2.04130716017072e-06\\
0.380338	3.33178171007578e-06\\
0.380438	7.1339364527923e-06\\
0.3805381	3.12562135824946e-06\\
0.3806381	-1.13114905901479e-06\\
0.3807381	-2.58045348877101e-05\\
0.3808381	-1.7468454525791e-05\\
0.3809381	-1.11212253556358e-05\\
0.3810381	-1.44479747439581e-05\\
0.3811381	-1.73615004171908e-06\\
0.3812381	-1.02828485840379e-05\\
0.3813381	2.34350230989719e-06\\
0.3814381	8.58901533691892e-06\\
0.3815382	4.77251918639448e-06\\
0.3816382	9.42938607356998e-06\\
0.3817382	1.85832828411498e-06\\
0.3818382	-1.44957284201298e-05\\
0.3819382	-8.51974326084635e-06\\
0.3820382	-7.62425944314644e-06\\
0.3821382	-6.47583415547874e-06\\
0.3822382	-9.70677456953492e-06\\
0.3823382	-1.24372078401556e-05\\
0.3824382	-1.35237711052356e-05\\
0.3825383	5.50997500662993e-06\\
0.3826383	5.56231987444243e-06\\
0.3827383	-1.8592295603381e-06\\
0.3828383	1.53454863315063e-06\\
0.3829383	5.74629040319744e-06\\
0.3830383	2.7902571561551e-07\\
0.3831383	-9.10419544615938e-06\\
0.3832383	-8.70963080970384e-07\\
0.3833383	8.03954083016e-06\\
0.3834383	-7.28135675870855e-07\\
0.3835384	-4.81016824542364e-06\\
0.3836384	-1.51932509056471e-05\\
0.3837384	-1.60609047607127e-05\\
0.3838384	-1.07110067010165e-05\\
0.3839384	-3.10027135985274e-05\\
0.3840384	-1.41342823066852e-05\\
0.3841384	-4.60589966029606e-06\\
0.3842384	5.62165352457033e-06\\
0.3843384	5.313183450686e-06\\
0.3844384	-4.12869166023079e-06\\
0.3845385	-8.39401015386596e-06\\
0.3846385	-6.13220682760944e-08\\
0.3847385	3.582084763463e-06\\
0.3848385	7.7637706399364e-06\\
0.3849385	5.57662687920919e-06\\
0.3850385	8.03417181893463e-06\\
0.3851385	8.52760181646339e-06\\
0.3852385	1.54078824895268e-06\\
0.3853385	-1.19785848652344e-06\\
0.3854385	-4.64424097286378e-06\\
0.3855386	-9.40660891735401e-06\\
0.3856386	-1.12412120105722e-05\\
0.3857386	3.52460452141561e-06\\
0.3858386	-1.22774308230421e-05\\
0.3859386	-5.15053952500794e-06\\
0.3860386	-7.0365430448414e-06\\
0.3861386	-2.04955978220843e-05\\
0.3862386	-3.77627027534546e-06\\
0.3863386	6.76863197815055e-07\\
0.3864386	-2.6236837789976e-06\\
0.3865387	-4.24437491656171e-06\\
0.3866387	-1.97906123855537e-06\\
0.3867387	-2.91590700784363e-06\\
0.3868387	-6.02143851496261e-06\\
0.3869387	-2.37038202869179e-06\\
0.3870387	7.93831220406563e-06\\
0.3871387	3.54975995840391e-06\\
0.3872387	1.48235303153564e-06\\
0.3873387	1.59753924291387e-06\\
0.3874387	-1.77723475656453e-05\\
0.3875388	-2.67821947287708e-05\\
0.3876388	-1.83854653250045e-05\\
0.3877388	-1.35872972763943e-05\\
0.3878388	-9.53451665041147e-06\\
0.3879388	-7.78031949154236e-06\\
0.3880388	4.48917817883841e-06\\
0.3881388	1.00546643625946e-05\\
0.3882388	1.45757608849562e-05\\
0.3883388	1.43180923946474e-05\\
0.3884388	1.05162193640161e-05\\
0.3885389	2.69570043697467e-06\\
0.3886389	-6.21700328517092e-06\\
0.3887389	-3.01460379834007e-05\\
0.3888389	-1.3554629489547e-05\\
0.3889389	-1.02612636111638e-05\\
0.3890389	-7.34805613139446e-06\\
0.3891389	-2.91062728088377e-05\\
0.3892389	-1.68714293774134e-05\\
0.3893389	-1.15422515462277e-05\\
0.3894389	-1.21138649994373e-05\\
0.389539	-1.30399530881698e-05\\
0.389639	-8.22244534254417e-06\\
0.389739	3.46755422168304e-06\\
0.389839	2.53302002154834e-06\\
0.389939	1.01978164997261e-05\\
0.390039	4.76821812466901e-06\\
0.390139	1.24476643907474e-05\\
0.390239	-1.07528518160812e-06\\
0.390339	5.77864895423399e-06\\
0.390439	8.29469537124989e-06\\
0.3905391	-9.95840862493803e-06\\
0.3906391	1.93539994427839e-06\\
0.3907391	-7.9676743222352e-06\\
0.3908391	-2.91628367446583e-05\\
0.3909391	-2.32434126161021e-05\\
0.3910391	7.49222797537722e-07\\
0.3911391	4.66614257728451e-07\\
0.3912391	9.29589555344891e-06\\
0.3913391	1.66358450133899e-06\\
0.3914391	-5.8772170541506e-06\\
0.3915392	1.0444543279268e-06\\
0.3916392	9.43885422375954e-06\\
0.3917392	1.01550932288376e-06\\
0.3918392	-6.81091041746279e-06\\
0.3919392	-1.85658750114705e-05\\
0.3920392	-2.32712794446025e-05\\
0.3921392	-1.901115867344e-05\\
0.3922392	-2.3730067625526e-05\\
0.3923392	-5.47778119920817e-06\\
0.3924392	-1.13494242634815e-05\\
0.3925393	8.98452426476912e-06\\
0.3926393	1.87382955940011e-05\\
0.3927393	1.00350490085975e-05\\
0.3928393	-1.11190924140245e-06\\
0.3929393	4.2144791241931e-06\\
0.3930393	1.4615197415892e-05\\
0.3931393	7.15462697468098e-06\\
0.3932393	7.74710717218948e-06\\
0.3933393	-4.0737245891449e-06\\
0.3934393	-2.43381228759113e-05\\
0.3935394	-3.147911742615e-05\\
0.3936394	-4.74524814216236e-05\\
0.3937394	-3.58951354492422e-05\\
0.3938394	-2.32099239499956e-05\\
0.3939394	-1.44456213330587e-05\\
0.3940394	-7.01613197562112e-06\\
0.3941394	-5.23974022271548e-06\\
0.3942394	8.97723961295486e-06\\
0.3943394	1.13018532511888e-05\\
0.3944394	5.03777644446216e-06\\
0.3945395	-8.49780079827173e-06\\
0.3946395	-1.05929701318284e-05\\
0.3947395	-2.27212449236432e-07\\
0.3948395	-7.46782186552207e-06\\
0.3949395	-9.30496246215087e-06\\
0.3950395	8.31370171315097e-06\\
0.3951395	1.83557056465772e-05\\
0.3952395	-8.06452164157711e-06\\
0.3953395	-9.41404709673477e-06\\
0.3954395	-1.34225536298521e-05\\
0.3955396	1.44092091163649e-05\\
0.3956396	1.07512498675399e-05\\
0.3957396	1.6324840022734e-05\\
0.3958396	2.14269808107362e-05\\
0.3959396	1.78453604957294e-05\\
0.3960396	4.15970937747887e-06\\
0.3961396	-1.16739902084783e-05\\
0.3962396	-2.33195408906433e-05\\
0.3963396	-3.53473362613646e-05\\
0.3964396	-3.48190298766454e-05\\
0.3965397	-2.36980703944953e-05\\
0.3966397	-3.20182499225311e-05\\
0.3967397	-3.24860813655826e-06\\
0.3968397	4.3524015547768e-06\\
0.3969397	1.38872457801472e-05\\
0.3970397	1.07803112057441e-05\\
0.3971397	1.45178453547442e-06\\
0.3972397	-4.24719748883877e-06\\
0.3973397	-9.97412808725558e-06\\
0.3974397	-1.10046004415183e-05\\
0.3975398	-1.12566415516745e-05\\
0.3976398	9.60194435689305e-06\\
0.3977398	-1.75452079671608e-06\\
0.3978398	-3.92936003889515e-06\\
0.3979398	5.42784364330381e-06\\
0.3980398	-7.05481269762149e-06\\
0.3981398	-4.57911196941579e-06\\
0.3982398	5.24007296014393e-06\\
0.3983398	6.93499263458648e-07\\
0.3984398	2.0330341533191e-06\\
0.3985399	-1.72442770001871e-05\\
0.3986399	-9.72009189944031e-06\\
0.3987399	1.88955367805166e-06\\
0.3988399	-1.54730828911822e-05\\
0.3989399	-1.61879418914943e-06\\
0.3990399	1.11098272438248e-05\\
0.3991399	8.46042890931537e-06\\
0.3992399	2.55198823510188e-05\\
0.3993399	1.79188031549729e-05\\
0.3994399	-1.05893190502207e-05\\
0.39954	-1.00771297711624e-05\\
0.39964	-1.54830016252202e-05\\
0.39974	-1.79046773372191e-05\\
0.39984	-1.09743910601482e-06\\
0.39994	-5.32653908677179e-06\\
0.40004	2.30538060926468e-06\\
};
\addplot [color=mycolor1,solid,forget plot]
  table[row sep=crcr]{%
0.40004	2.30538060926468e-06\\
0.40014	1.35622479149993e-05\\
0.40024	-6.99242919228302e-06\\
0.40034	-3.11309026906855e-05\\
0.40044	-6.27020451346543e-06\\
0.4005401	1.00193477524661e-05\\
0.4006401	7.5560650537532e-06\\
0.4007401	-4.8446714894657e-06\\
0.4008401	3.50808637133198e-06\\
0.4009401	2.31183369306966e-05\\
0.4010401	2.20102285294455e-05\\
0.4011401	4.78960955490531e-06\\
0.4012401	-1.21833215588472e-05\\
0.4013401	-4.11087463307409e-05\\
0.4014401	-1.02008962324661e-05\\
0.4015402	-1.280204089356e-06\\
0.4016402	-4.3106876999232e-05\\
0.4017402	-2.1065699159423e-05\\
0.4018402	-2.1437751850149e-05\\
0.4019402	-2.5290706303152e-05\\
0.4020402	-2.89961437014437e-05\\
0.4021402	3.9171248209473e-06\\
0.4022402	-4.2935118207707e-06\\
0.4023402	-7.7071646618978e-06\\
0.4024402	7.38097531201471e-06\\
0.4025403	2.53949809270285e-05\\
0.4026403	5.72218534022049e-06\\
0.4027403	-8.57808007794017e-06\\
0.4028403	1.44911038886269e-05\\
0.4029403	3.00768765070513e-05\\
0.4030403	1.3460947095775e-05\\
0.4031403	2.44249950248168e-05\\
0.4032403	1.57249997160401e-05\\
0.4033403	1.7345110643419e-05\\
0.4034403	7.20137333307615e-07\\
0.4035404	-1.26370507965538e-05\\
0.4036404	-1.64273521318999e-05\\
0.4037404	-1.44537687989757e-05\\
0.4038404	-8.11288105914039e-06\\
0.4039404	-4.14254591200103e-05\\
0.4040404	-5.32794031351814e-05\\
0.4041404	-3.22923232039536e-05\\
0.4042404	1.570890628951e-06\\
0.4043404	-4.27212716263337e-05\\
0.4044404	-3.32877990453458e-06\\
0.4045405	-9.51117202824267e-07\\
0.4046405	-1.41023453496827e-05\\
0.4047405	-1.57986489797106e-05\\
0.4048405	-1.72765378041808e-05\\
0.4049405	-2.1688181477799e-05\\
0.4050405	2.96682104015861e-05\\
0.4051405	3.30935582932736e-05\\
0.4052405	1.20671527140036e-05\\
0.4053405	-7.94219576174956e-06\\
0.4054405	2.3230349851789e-07\\
0.4055406	1.76790061320443e-05\\
0.4056406	-5.43763513995706e-06\\
0.4057406	9.48133583561467e-07\\
0.4058406	1.01134031702607e-05\\
0.4059406	1.47040245473284e-05\\
0.4060406	-1.70121212982183e-05\\
0.4061406	-1.19110333192261e-05\\
0.4062406	-1.63176404081015e-05\\
0.4063406	-7.74480139650101e-06\\
0.4064406	1.26682549449777e-05\\
0.4065407	5.90625399614169e-06\\
0.4066407	-3.52335390929534e-05\\
0.4067407	-1.63903807830542e-05\\
0.4068407	-1.16679095973413e-05\\
0.4069407	-8.6531339963679e-06\\
0.4070407	5.20085762225685e-06\\
0.4071407	-7.08032307581184e-06\\
0.4072407	1.61653456386656e-05\\
0.4073407	4.1414445558837e-05\\
0.4074407	4.3725678542081e-05\\
0.4075408	9.05601088082725e-06\\
0.4076408	-7.00175208610097e-06\\
0.4077408	-2.16145960170872e-05\\
0.4078408	-3.0277448794493e-05\\
0.4079408	-3.06793382348063e-05\\
0.4080408	-4.59190820883092e-05\\
0.4081408	-3.48544680387417e-05\\
0.4082408	-2.4598703702968e-05\\
0.4083408	-1.72738666306367e-05\\
0.4084408	-9.99371430384316e-06\\
0.4085409	4.98011171761693e-06\\
0.4086409	1.92224445958202e-05\\
0.4087409	2.5537210703695e-05\\
0.4088409	1.58632328328521e-05\\
0.4089409	-9.34340615582297e-06\\
0.4090409	2.69782566681763e-05\\
0.4091409	6.12234305328341e-07\\
0.4092409	-3.32645801735637e-05\\
0.4093409	-2.38881536270205e-05\\
0.4094409	-7.29173284253e-06\\
0.409541	-2.12644359285075e-05\\
0.409641	1.25837560291373e-06\\
0.409741	4.33111453050647e-07\\
0.409841	-3.80203848778014e-05\\
0.409941	-2.98407298165133e-05\\
0.410041	2.730771028324e-05\\
0.410141	4.08697791413474e-05\\
0.410241	2.26091905516298e-05\\
0.410341	2.73609883822085e-05\\
0.410441	2.41529365758789e-05\\
0.4105411	-6.28968814882194e-06\\
0.4106411	-8.68534461446971e-06\\
0.4107411	1.91420615773459e-05\\
0.4108411	1.59271546719373e-05\\
0.4109411	3.38783899117416e-06\\
0.4110411	-6.14781066135317e-07\\
0.4111411	-2.86671310238614e-05\\
0.4112411	-4.18754463281669e-05\\
0.4113411	-1.69919323471215e-05\\
0.4114411	-3.06102643703107e-05\\
0.4115412	-1.66200794010316e-05\\
0.4116412	-1.13938363544411e-05\\
0.4117412	1.48404453089559e-05\\
0.4118412	1.86188506175073e-05\\
0.4119412	2.82880646800146e-05\\
0.4120412	-5.49628773138675e-05\\
0.4121412	-8.46796100932304e-05\\
0.4122412	-1.72828583055084e-05\\
0.4123412	-1.63570665160744e-05\\
0.4124412	-2.03994192078323e-05\\
0.4125413	-5.63707755752139e-05\\
0.4126413	-2.88785756147091e-06\\
0.4127413	8.11104303708073e-06\\
0.4128413	8.32620715991825e-08\\
0.4129413	2.07138254777698e-05\\
0.4130413	2.4376689276227e-05\\
0.4131413	6.76777957297133e-06\\
0.4132413	4.23732255979208e-06\\
0.4133413	3.27358845146897e-05\\
0.4134413	2.21777618023023e-05\\
0.4135414	1.53659246512701e-05\\
0.4136414	3.60595499354074e-06\\
0.4137414	2.16330030321268e-06\\
0.4138414	1.43777772958504e-05\\
0.4139414	1.60980427758103e-05\\
0.4140414	-2.38725463633223e-05\\
0.4141414	2.06460768617059e-05\\
0.4142414	6.45219524256295e-07\\
0.4143414	-1.98071312117585e-05\\
0.4144414	-2.4134068091431e-05\\
0.4145415	-3.27085897611278e-05\\
0.4146415	-2.50560236797761e-05\\
0.4147415	-6.02868314093335e-05\\
0.4148415	-4.89886953263315e-05\\
0.4149415	-1.83615977140532e-05\\
0.4150415	-4.00452076207681e-06\\
0.4151415	1.84172343418305e-06\\
0.4152415	1.20111328739373e-05\\
0.4153415	1.70113156514824e-05\\
0.4154415	9.69022995666781e-06\\
0.4155416	-1.8366599378514e-06\\
0.4156416	4.21996156741308e-05\\
0.4157416	2.10195779798151e-05\\
0.4158416	2.55504955560426e-05\\
0.4159416	1.38191970775612e-05\\
0.4160416	-2.0194628682498e-05\\
0.4161416	-1.11233728521405e-05\\
0.4162416	6.48319353997221e-06\\
0.4163416	-1.68356404349939e-05\\
0.4164416	-8.70632256055592e-05\\
0.4165417	-6.21139628893109e-05\\
0.4166417	-3.39398090457626e-05\\
0.4167417	-3.54769582570815e-05\\
0.4168417	-5.9013909455351e-06\\
0.4169417	2.37997225693846e-05\\
0.4170417	3.74781320714564e-05\\
0.4171417	1.76915474482326e-05\\
0.4172417	2.47802386915411e-05\\
0.4173417	3.28993158979909e-05\\
0.4174417	3.03629392694793e-05\\
0.4175418	1.7119304690415e-05\\
0.4176418	2.65814351085123e-05\\
0.4177418	-1.34207270452523e-05\\
0.4178418	-6.95439114260489e-06\\
0.4179418	2.83408835525876e-05\\
0.4180418	-1.68898211516335e-06\\
0.4181418	-3.64466671924453e-05\\
0.4182418	-1.02233506164164e-05\\
0.4183418	-1.58151112142819e-05\\
0.4184418	-8.97812770276162e-06\\
0.4185419	-5.5950115296689e-05\\
0.4186419	-5.05290753741463e-05\\
0.4187419	-2.93762680246428e-06\\
0.4188419	-2.64885478509682e-05\\
0.4189419	-2.27908166758144e-05\\
0.4190419	1.30041886805535e-05\\
0.4191419	1.26947502899493e-05\\
0.4192419	1.61606203389949e-05\\
0.4193419	4.38807112966924e-06\\
0.4194419	1.30542950798623e-05\\
0.419542	3.26331480860172e-05\\
0.419642	3.0381301950635e-06\\
0.419742	-1.61742165588403e-05\\
0.419842	-5.64271092884626e-05\\
0.419942	-3.35520249878582e-05\\
0.420042	-1.22298405318009e-05\\
0.420142	2.54961073247858e-05\\
0.420242	6.65072194740191e-06\\
0.420342	-2.55446531774526e-05\\
0.420442	-3.37126344417035e-05\\
0.4205421	-3.87564525098861e-05\\
0.4206421	9.12042961392562e-06\\
0.4207421	2.19467771818713e-05\\
0.4208421	1.21233142942778e-05\\
0.4209421	5.97207051759742e-05\\
0.4210421	5.29592441768814e-05\\
0.4211421	4.166705577261e-05\\
0.4212421	3.32372145650554e-05\\
0.4213421	1.59141952829996e-05\\
0.4214421	-1.18971721728629e-06\\
0.4215422	-3.95943777106356e-05\\
0.4216422	-1.74827369660956e-05\\
0.4217422	-2.77396840746705e-05\\
0.4218422	-4.49734936639274e-05\\
0.4219422	-5.24534402308921e-05\\
0.4220422	-6.89885981414314e-05\\
0.4221422	8.80475837036806e-06\\
0.4222422	2.6779255202822e-05\\
0.4223422	9.08861838727612e-06\\
0.4224422	-9.53312591126499e-06\\
0.4225423	-7.63761399836017e-06\\
0.4226423	-1.34488893715624e-05\\
0.4227423	-1.37492897836006e-07\\
0.4228423	-1.58615978710664e-05\\
0.4229423	-4.58292774482981e-05\\
0.4230423	-4.48695046490882e-05\\
0.4231423	2.60777476451418e-05\\
0.4232423	-4.32939330824653e-06\\
0.4233423	-7.76210011245156e-06\\
0.4234423	1.14870547694212e-05\\
0.4235424	1.93780862157669e-05\\
0.4236424	4.23544527753049e-05\\
0.4237424	9.49616450680055e-06\\
0.4238424	1.19734495142795e-05\\
0.4239424	1.52559360451829e-05\\
0.4240424	2.95722524910401e-05\\
0.4241424	8.78382738814194e-06\\
0.4242424	-2.22100105817788e-05\\
0.4243424	-3.42293325908694e-05\\
0.4244424	-4.88434096644695e-05\\
0.4245425	-5.08487288396867e-06\\
0.4246425	4.48993284607615e-06\\
0.4247425	1.29293713527985e-05\\
0.4248425	3.43074722051244e-05\\
0.4249425	-6.30534877749946e-07\\
0.4250425	1.51434799743402e-05\\
0.4251425	5.54064678323282e-06\\
0.4252425	-5.60296042768519e-05\\
0.4253425	-1.46242878606653e-07\\
0.4254425	-5.17863854091811e-06\\
0.4255426	1.92764346996486e-05\\
0.4256426	-5.47487730181504e-06\\
0.4257426	-1.63494539594804e-05\\
0.4258426	-3.42058641696783e-05\\
0.4259426	-4.81726766713251e-05\\
0.4260426	-2.46167600451942e-05\\
0.4261426	-3.6128827131818e-06\\
0.4262426	5.73951870624312e-05\\
0.4263426	5.8137581179715e-05\\
0.4264426	-5.63720683021597e-05\\
0.4265427	-2.41553752149383e-05\\
0.4266427	3.39104909677662e-05\\
0.4267427	3.54862107355057e-05\\
0.4268427	1.81495167994932e-05\\
0.4269427	1.98286420357992e-05\\
0.4270427	-2.43575805138852e-05\\
0.4271427	-2.41955176414831e-05\\
0.4272427	-4.00486359717673e-05\\
0.4273427	-3.51738019615846e-05\\
0.4274427	-0.000101342381899087\\
0.4275428	-9.14572260884747e-05\\
0.4276428	-3.72181257360358e-05\\
0.4277428	-1.1966337141452e-05\\
0.4278428	7.81357408989161e-06\\
0.4279428	9.56084252522614e-06\\
0.4280428	5.46684029051942e-05\\
0.4281428	6.24233901446428e-05\\
0.4282428	5.72258393334261e-05\\
0.4283428	5.41387857372072e-05\\
0.4284428	3.75486479826237e-06\\
0.4285429	-3.12388067669578e-05\\
0.4286429	4.77885136804476e-06\\
0.4287429	-4.62480504089729e-05\\
0.4288429	2.79947240590374e-05\\
0.4289429	-1.54935888901801e-05\\
0.4290429	-3.2995527373575e-06\\
0.4291429	-3.81052307808726e-05\\
0.4292429	-2.46177861359194e-05\\
0.4293429	-2.32321817336688e-05\\
0.4294429	-2.58196803204325e-05\\
0.429543	-5.8301694487664e-05\\
0.429643	8.66302054325084e-06\\
0.429743	1.33073455017449e-05\\
0.429843	3.20020183896084e-05\\
0.429943	3.61701773977416e-05\\
0.430043	5.98176609070168e-05\\
0.430143	2.55790074891413e-05\\
0.430243	-4.2274244092473e-05\\
0.430343	-3.2612754881843e-05\\
0.430443	-4.30651857295364e-05\\
0.4305431	-3.0760374676825e-05\\
0.4306431	-8.91131993601291e-06\\
0.4307431	-7.41758665959203e-05\\
0.4308431	-8.53566747243156e-05\\
0.4309431	-2.96403041910979e-05\\
0.4310431	1.60335853326004e-05\\
0.4311431	6.82693437531419e-06\\
0.4312431	1.9334783666147e-05\\
0.4313431	8.88221741356771e-05\\
0.4314431	7.60059469168673e-05\\
0.4315432	4.16633425719916e-05\\
0.4316432	6.00218120755437e-05\\
0.4317432	-4.1888114450505e-06\\
0.4318432	-2.00246860379131e-05\\
0.4319432	-3.89277695441773e-05\\
0.4320432	-2.40104195970585e-05\\
0.4321432	-2.9698936209128e-06\\
0.4322432	-4.63287488303022e-05\\
0.4323432	6.7920577709489e-06\\
0.4324432	1.56503693907019e-05\\
0.4325433	-2.24663905475988e-05\\
0.4326433	-3.90980301017861e-05\\
0.4327433	-3.70260336903053e-05\\
0.4328433	2.28594427512005e-05\\
0.4329433	-1.89168564964383e-05\\
0.4330433	-5.08603869349458e-05\\
0.4331433	-2.88132038476304e-05\\
0.4332433	-2.54308622984046e-05\\
0.4333433	1.28327828691937e-05\\
0.4334433	9.03307703298744e-06\\
0.4335434	5.00337309488316e-05\\
0.4336434	6.2911268892296e-05\\
0.4337434	2.02295933496454e-05\\
0.4338434	2.60014505942898e-05\\
0.4339434	2.00866871259851e-05\\
0.4340434	-1.47509503281653e-05\\
0.4341434	-4.31524148126965e-05\\
0.4342434	-7.71128591427618e-05\\
0.4343434	1.26803640968819e-05\\
0.4344434	-3.21352974471004e-05\\
0.4345435	-3.81064409918663e-05\\
0.4346435	-2.55919189969004e-05\\
0.4347435	5.77445626017379e-05\\
0.4348435	-6.9903476130415e-06\\
0.4349435	3.29044991782767e-05\\
0.4350435	1.32401520328604e-05\\
0.4351435	5.83936602465769e-05\\
0.4352435	-3.10742664494271e-06\\
0.4353435	-4.27666588652474e-05\\
0.4354435	-6.03764863960018e-05\\
0.4355436	-6.52917094993531e-05\\
0.4356436	-5.59164668922956e-05\\
0.4357436	-1.19540680277908e-05\\
0.4358436	-2.07311619089878e-05\\
0.4359436	-1.05376972917654e-05\\
0.4360436	3.41646773163577e-05\\
0.4361436	6.46508136573186e-05\\
0.4362436	4.88434637236019e-05\\
0.4363436	1.50960797593501e-05\\
0.4364436	-2.40378857385347e-05\\
0.4365437	3.04665690948077e-05\\
0.4366437	5.19279092878007e-05\\
0.4367437	5.82556270189444e-05\\
0.4368437	9.33112758075925e-05\\
0.4369437	5.00201094314534e-05\\
0.4370437	-1.37810180722769e-05\\
0.4371437	-3.73573524057723e-05\\
0.4372437	-4.68036390094443e-05\\
0.4373437	-5.11596230607868e-05\\
0.4374437	-0.000124129249473237\\
0.4375438	-0.000132518908828038\\
0.4376438	-5.76351428444481e-05\\
0.4377438	-3.76799523103189e-05\\
0.4378438	4.42191892901622e-06\\
0.4379438	-1.1581727019e-06\\
0.4380438	1.4093029492528e-05\\
0.4381438	3.32740824801291e-05\\
0.4382438	-4.93547564978513e-05\\
0.4383438	-6.04570299255554e-05\\
0.4384438	-1.20298001132465e-06\\
0.4385439	7.48441591062483e-06\\
0.4386439	5.32995983619597e-05\\
0.4387439	4.80987792557954e-05\\
0.4388439	3.86125174993096e-05\\
0.4389439	7.84515722815283e-05\\
0.4390439	5.53331030745363e-05\\
0.4391439	2.92344696347129e-05\\
0.4392439	-2.46348679960448e-05\\
0.4393439	-3.8874549489085e-05\\
0.4394439	-1.01121422775688e-06\\
0.439544	-1.86452032807579e-06\\
0.439644	6.00048434197284e-05\\
0.439744	2.29099075739311e-05\\
0.439844	-1.41546608405254e-06\\
0.439944	-1.82302154785326e-05\\
0.440044	-5.50243782382862e-05\\
0.440144	1.85262083047182e-05\\
0.440244	-7.47107928834013e-05\\
0.440344	-7.38570185371578e-05\\
0.440444	-4.09581050904089e-05\\
0.4405441	-2.65025851795149e-05\\
0.4406441	-2.22013890248775e-05\\
0.4407441	-6.16793611897932e-05\\
0.4408441	1.39262631077778e-05\\
0.4409441	6.30870489570463e-05\\
0.4410441	2.11107617557038e-05\\
0.4411441	8.62422672112466e-05\\
0.4412441	4.02078313423003e-05\\
0.4413441	5.13717204799522e-05\\
0.4414441	3.02975012690848e-05\\
0.4415442	2.19301731174394e-05\\
0.4416442	5.87285516657503e-05\\
0.4417442	-1.96854762358351e-05\\
0.4418442	-0.000159500742673465\\
0.4419442	-0.000183445379412736\\
0.4420442	-0.000147673817897326\\
0.4421442	-3.98982892476218e-05\\
0.4422442	-1.07615242593425e-05\\
0.4423442	1.34907465978359e-05\\
0.4424442	5.68951931816709e-05\\
0.4425443	4.47693538246374e-05\\
0.4426443	8.50876767035923e-05\\
0.4427443	1.34143869320317e-05\\
0.4428443	-1.38281443082538e-05\\
0.4429443	8.3933454498964e-05\\
0.4430443	7.90034552057743e-05\\
0.4431443	2.69303300015481e-05\\
0.4432443	-2.9270148585636e-05\\
0.4433443	-1.73989076875586e-05\\
0.4434443	-2.50086740944212e-05\\
0.4435444	-0.000135626863338516\\
0.4436444	-8.27017086750841e-05\\
0.4437444	-3.48022395488535e-05\\
0.4438444	4.22208186311212e-05\\
0.4439444	7.68024080463207e-06\\
0.4440444	3.16236022617847e-05\\
0.4441444	4.69216786444827e-05\\
0.4442444	2.81200459478981e-05\\
0.4443444	3.47827805219626e-05\\
0.4444444	0.000123852659072858\\
0.4445445	6.82268197580931e-05\\
0.4446445	5.30998332509979e-05\\
0.4447445	-3.58091769302842e-05\\
0.4448445	-0.000123070132639199\\
0.4449445	-7.88714553654056e-05\\
0.4450445	-0.00010103946623326\\
0.4451445	-0.000137760586000275\\
0.4452445	-0.00014211933505559\\
0.4453445	-6.35596334184357e-05\\
0.4454445	3.46267992634187e-05\\
0.4455446	8.0433762750642e-05\\
0.4456446	2.23816163057532e-05\\
0.4457446	-3.02833557118031e-05\\
0.4458446	-4.66402714480226e-05\\
0.4459446	6.05874513303699e-05\\
0.4460446	-7.90716047625088e-05\\
0.4461446	-1.34240567300971e-05\\
0.4462446	-2.10503211918153e-05\\
0.4463446	1.53914856185342e-05\\
0.4464446	3.05334785437534e-06\\
0.4465447	4.65922351669335e-05\\
0.4466447	9.53864793384831e-05\\
0.4467447	0.000152419933820738\\
0.4468447	0.000109936584332527\\
0.4469447	2.99867169879992e-05\\
0.4470447	6.44394182982549e-05\\
0.4471447	6.20270751729851e-05\\
0.4472447	6.04972749221292e-05\\
0.4473447	-4.44316947424847e-05\\
0.4474447	1.23115429453296e-06\\
0.4475448	5.31887025221429e-06\\
0.4476448	-0.000112371859398006\\
0.4477448	-0.00020450570347289\\
0.4478448	-0.00020673087222571\\
0.4479448	-8.26617754978046e-05\\
0.4480448	-1.92427227168662e-05\\
0.4481448	-3.82215228952656e-05\\
0.4482448	-9.67336846283222e-05\\
0.4483448	-1.87111160926017e-05\\
0.4484448	9.02110749558e-05\\
0.4485449	3.87685243872062e-05\\
0.4486449	7.21565773821597e-06\\
0.4487449	7.33853929674682e-05\\
0.4488449	9.0942024018496e-05\\
0.4489449	3.3873745263987e-05\\
0.4490449	1.30026515075447e-05\\
0.4491449	-7.16113620145601e-05\\
0.4492449	-5.33223996316106e-05\\
0.4493449	-1.22658652366817e-05\\
0.4494449	-8.3047622848508e-06\\
0.449545	7.04561422521638e-05\\
0.449645	2.4221293525339e-05\\
0.449745	5.27023057341057e-05\\
0.449845	8.45557771937139e-05\\
0.449945	-1.2544493333598e-05\\
0.450045	-9.05433066367968e-05\\
0.450145	-0.000114360463054225\\
0.450245	-2.9002662471775e-05\\
0.450345	4.6877956789475e-06\\
0.450445	-3.5648887577534e-05\\
0.4505451	-4.81117210753554e-06\\
0.4506451	-5.0864124578969e-06\\
0.4507451	0.000112971311716025\\
0.4508451	3.8701003282278e-05\\
0.4509451	2.5824565574339e-05\\
0.4510451	2.23515023930003e-05\\
0.4511451	2.91775180082524e-05\\
0.4512451	5.20400171611802e-05\\
0.4513451	1.60527050658548e-05\\
0.4514451	-8.12501258874666e-06\\
0.4515452	1.04645218428213e-05\\
0.4516452	-6.10996609347705e-06\\
0.4517452	-7.77894395462355e-05\\
0.4518452	-9.19945909072257e-05\\
0.4519452	2.93753879163091e-05\\
0.4520452	3.49735055035925e-05\\
0.4521452	-0.000103335429077728\\
0.4522452	-4.51069062696343e-05\\
0.4523452	-7.87397160217597e-05\\
0.4524452	1.35668522105799e-05\\
0.4525453	-2.98952155312883e-05\\
0.4526453	-8.57343347522685e-05\\
0.4527453	-5.81012412567566e-05\\
0.4528453	-2.80492250246322e-06\\
0.4529453	8.19793455717019e-06\\
0.4530453	4.75894347498314e-06\\
0.4531453	3.78209183121244e-05\\
0.4532453	4.4608073640065e-05\\
0.4533453	-7.48239754573685e-05\\
0.4534453	1.26023866180494e-05\\
0.4535454	7.80224856848514e-05\\
0.4536454	6.97225430709393e-05\\
0.4537454	5.23100616488417e-05\\
0.4538454	6.75486590954075e-05\\
0.4539454	0.00016258385361219\\
0.4540454	7.63894639275097e-05\\
0.4541454	2.46460929854758e-06\\
0.4542454	-1.55429904865596e-05\\
0.4543454	-4.35632151066319e-05\\
0.4544454	-0.00011003254370532\\
0.4545455	-0.000186779998788262\\
0.4546455	-4.4231245323373e-05\\
0.4547455	5.01197700868231e-05\\
0.4548455	3.61244704610643e-05\\
0.4549455	9.48797936389367e-06\\
0.4550455	-1.28194790921862e-05\\
0.4551455	-3.50992802445719e-05\\
0.4552455	0.000107629701121618\\
0.4553455	0.000117066890775705\\
0.4554455	5.19461504362891e-06\\
0.4555456	-8.17771844676669e-05\\
0.4556456	-0.000100781183203907\\
0.4557456	-0.000104116564858581\\
0.4558456	-0.000111158200856775\\
0.4559456	-4.52343620559635e-05\\
0.4560456	1.08688812562243e-05\\
0.4561456	-5.6747340635773e-05\\
0.4562456	-6.90766968732147e-05\\
0.4563456	3.58835439791868e-05\\
0.4564456	-3.57270860644912e-05\\
0.4565457	-0.000102587468288368\\
0.4566457	-5.86004138937576e-05\\
0.4567457	7.79990692764522e-05\\
0.4568457	0.00013871931557476\\
0.4569457	9.11481599438181e-05\\
0.4570457	6.70561379187236e-05\\
0.4571457	7.29330856293339e-05\\
0.4572457	4.05965398025607e-05\\
0.4573457	0.000134039937764698\\
0.4574457	9.05729174437297e-05\\
0.4575458	0.000126910888242642\\
0.4576458	2.49642767871331e-05\\
0.4577458	-5.38469353939092e-05\\
0.4578458	-8.33117079416929e-05\\
0.4579458	-8.37408998839997e-05\\
0.4580458	8.74963036718917e-06\\
0.4581458	-7.96589749819251e-05\\
0.4582458	-0.000135343573104585\\
0.4583458	-0.000225333020551072\\
0.4584458	-0.000147321173246332\\
0.4585459	-4.60059169173691e-05\\
0.4586459	-0.000122182014713791\\
0.4587459	8.93466183887034e-05\\
0.4588459	9.09261289913641e-05\\
0.4589459	2.42656643329032e-05\\
0.4590459	0.000117943372291937\\
0.4591459	0.000176302301389448\\
0.4592459	0.000122498900791235\\
0.4593459	0.000109123020310397\\
0.4594459	-6.482908959021e-05\\
0.459546	-9.3980994956266e-05\\
0.459646	4.30692245329445e-06\\
0.459746	-5.14379340921163e-05\\
0.459846	-0.000143236411504993\\
0.459946	-0.000138427356035572\\
0.460046	-0.000132354613269306\\
0.460146	-0.000129427028124322\\
0.460246	-1.83204448488874e-05\\
0.460346	-3.5798707018868e-05\\
0.460446	5.41053424648332e-05\\
0.4605461	0.000111826472702036\\
0.4606461	0.00013305965321046\\
0.4607461	0.000204314620904751\\
0.4608461	0.000121025535610241\\
0.4609461	-1.64144421598905e-05\\
0.4610461	-5.555715120201e-05\\
0.4611461	-2.51442961940657e-06\\
0.4612461	0.000143935885180315\\
0.4613461	4.37489564878624e-05\\
0.4614461	-0.000108653051705104\\
0.4615462	-0.00019224001901104\\
0.4616462	-8.68596560794324e-05\\
0.4617462	-9.9437876274735e-05\\
0.4618462	-7.24375127742663e-05\\
0.4619462	3.73326019588948e-05\\
0.4620462	0.000155709636178615\\
0.4621462	0.000101860758858378\\
0.4622462	-5.1198860306011e-05\\
0.4623462	-8.72070508937039e-05\\
0.4624462	2.40343582438903e-05\\
0.4625463	4.94855628340937e-07\\
0.4626463	-7.33729823822178e-05\\
0.4627463	3.40332973458087e-05\\
0.4628463	0.000114877868900087\\
0.4629463	3.90319071097468e-05\\
0.4630463	-7.77894124518463e-05\\
0.4631463	-5.88989134643308e-05\\
0.4632463	-0.000111146418857564\\
0.4633463	-8.77975080880509e-06\\
0.4634463	-5.44787307399453e-05\\
0.4635464	-0.000142611484418254\\
0.4636464	-1.3625182737448e-05\\
0.4637464	-5.86059885605567e-05\\
0.4638464	-4.05247202590201e-05\\
0.4639464	0.000118719804565253\\
0.4640464	6.60227690827049e-05\\
0.4641464	3.2523572389856e-06\\
0.4642464	0.000165076753757864\\
0.4643464	9.72521441440566e-05\\
0.4644464	-5.74328531386097e-06\\
0.4645465	-0.000110390256336348\\
0.4646465	5.54262771825223e-05\\
0.4647465	0.000153540554644152\\
0.4648465	-2.94312344976393e-05\\
0.4649465	-2.57448999906318e-05\\
0.4650465	-4.30832507812592e-05\\
0.4651465	-7.80670950116479e-05\\
0.4652465	-0.000103098240016694\\
0.4653465	7.08785076789361e-05\\
0.4654465	-0.000116114657636249\\
0.4655466	-2.16620345132586e-05\\
0.4656466	0.000201881602859487\\
0.4657466	0.000113748915193074\\
0.4658466	2.40181000594917e-05\\
0.4659466	-1.32036441411105e-05\\
0.4660466	3.45648818226355e-05\\
0.4661466	-0.000159226121983667\\
0.4662466	-0.000166543454657145\\
0.4663466	-0.000126990914454583\\
0.4664466	-1.80322987893718e-05\\
0.4665467	5.04195368958367e-05\\
0.4666467	3.33089592208103e-05\\
0.4667467	9.10120700649099e-05\\
0.4668467	2.35030754199526e-05\\
0.4669467	-0.000143634817863409\\
0.4670467	-0.00015704240207258\\
0.4671467	-7.5521468629899e-05\\
0.4672467	-6.88018080895088e-05\\
0.4673467	8.7157789865788e-05\\
0.4674467	0.0001111925364277\\
0.4675468	0.00010218804004733\\
0.4676468	8.14017675918061e-05\\
0.4677468	0.000126114282810826\\
0.4678468	0.000142518800426612\\
0.4679468	4.55475360517804e-05\\
0.4680468	5.8163706192515e-05\\
0.4681468	-5.39904717481924e-05\\
0.4682468	-0.000110000779467326\\
0.4683468	-0.000123060997758633\\
0.4684468	-4.18189065093542e-05\\
0.4685469	-2.21574117116108e-05\\
0.4686469	-2.15988536760936e-07\\
0.4687469	-0.000103716589788222\\
0.4688469	-1.12289916937478e-05\\
0.4689469	-4.61419695582479e-05\\
0.4690469	2.96627022395115e-05\\
0.4691469	-8.81477497497911e-05\\
0.4692469	-8.4301098042721e-05\\
0.4693469	2.33918857802289e-05\\
0.4694469	8.48014310779945e-05\\
0.469547	3.49721399877535e-05\\
0.469647	5.79495512254365e-05\\
0.469747	0.000140422217917403\\
0.469847	5.7586373214729e-05\\
0.469947	9.29825122472638e-06\\
0.470047	-4.22969129856609e-05\\
0.470147	-5.9328883386492e-05\\
0.470247	1.02805770186305e-05\\
0.470347	9.01747061959205e-05\\
0.470447	-4.54625691510878e-06\\
0.4705471	0.000130028821466997\\
0.4706471	-3.48265519824493e-05\\
0.4707471	1.75017052147417e-05\\
0.4708471	4.17838359266288e-05\\
0.4709471	8.06608401182846e-06\\
0.4710471	-0.000165977305676994\\
0.4711471	-0.000227420087289635\\
0.4712471	-0.000149290013974836\\
0.4713471	-0.000186794837876795\\
0.4714471	-0.000146599310131633\\
0.4715472	-0.000132175740822344\\
0.4716472	-9.85137031629512e-05\\
0.4717472	0.000144020439071502\\
0.4718472	6.70219388136835e-05\\
0.4719472	0.000206947050022087\\
0.4720472	9.99280276846358e-05\\
0.4721472	0.000124006127822282\\
0.4722472	0.000176226607492586\\
0.4723472	-2.73322752066532e-05\\
0.4724472	6.80987388661835e-05\\
0.4725473	3.42199212659375e-05\\
0.4726473	1.58415690507123e-05\\
0.4727473	-1.58241024052581e-05\\
0.4728473	-2.35798297474887e-05\\
0.4729473	-9.4464348559406e-05\\
0.4730473	-9.29039335865328e-06\\
0.4731473	-3.50356975933981e-05\\
0.4732473	-6.26879936386619e-05\\
0.4733473	-1.52801279257927e-06\\
0.4734473	0.00019204451472728\\
0.4735474	0.000182829468662626\\
0.4736474	0.000118240964444376\\
0.4737474	-9.94341769582933e-05\\
0.4738474	-0.000126673681402106\\
0.4739474	-0.000226826273644436\\
0.4740474	-6.19376773395857e-05\\
0.4741474	-0.000165401615034933\\
0.4742474	-0.000354086808167171\\
0.4743474	-0.000134101977058478\\
0.4744474	-4.66048409127405e-05\\
0.4745475	0.000213178115831082\\
0.4746475	0.000213100772943753\\
0.4747475	8.24195845148231e-05\\
0.4748475	5.73168358532022e-05\\
0.4749475	3.60358134053576e-05\\
0.4750475	-6.09471952408179e-05\\
0.4751475	9.46770986479033e-05\\
0.4752475	0.000151912984953961\\
0.4753475	1.97807547129307e-05\\
0.4754475	6.65057001174316e-05\\
0.4755476	-7.73789986985894e-05\\
0.4756476	5.77242461339538e-05\\
0.4757476	2.63925504677518e-05\\
0.4758476	-0.000111363788835845\\
0.4759476	-0.000116947473738695\\
0.4760476	-0.0001708942050219\\
0.4761476	-0.000130936682281784\\
0.4762476	-0.000132168603925933\\
0.4763476	-8.03886671691643e-05\\
0.4764476	-0.000162621568029503\\
0.4765477	8.93475681132232e-05\\
0.4766477	0.00017829197869954\\
0.4767477	0.000449738471143884\\
0.4768477	0.000388321354256964\\
0.4769477	0.000186185938066523\\
0.4770477	2.91585338214289e-05\\
0.4771477	-0.000222885546004267\\
0.4772477	-4.79249877071808e-05\\
0.4773477	-0.000254489476350522\\
0.4774477	-0.000326160695759952\\
0.4775478	-0.000122780045716321\\
0.4776478	3.48629990112436e-06\\
0.4777478	-6.28431289373655e-05\\
0.4778478	-7.07310110646072e-05\\
0.4779478	-6.46002405511447e-06\\
0.4780478	0.000149945155779082\\
0.4781478	0.000401703853392645\\
0.4782478	9.35183950111309e-05\\
0.4783478	-0.000114900891864822\\
0.4784478	9.53403215447905e-05\\
0.4785479	0.000170481393410875\\
0.4786479	-0.000111839324801976\\
0.4787479	-0.000443613548337633\\
0.4788479	-0.000198696943255049\\
0.4789479	-0.000111004174312556\\
0.4790479	-0.00012427990496356\\
0.4791479	-0.000145183797352302\\
0.4792479	-4.60985123094974e-05\\
0.4793479	-0.000174284709348027\\
0.4794479	-3.38920466586303e-05\\
0.479548	0.000316353625515949\\
0.479648	0.000171597118119557\\
0.479748	5.36155042017597e-05\\
0.479848	0.000195374130905311\\
0.479948	2.79193467169255e-05\\
0.480048	7.44715014716567e-05\\
0.480148	0.000299718946357358\\
0.480248	0.000374956033918996\\
0.480348	9.14881180631914e-05\\
0.480448	-0.000203924445937435\\
0.4805481	-0.00026617268178027\\
0.4806481	6.05561247990884e-06\\
0.4807481	6.35203330139359e-05\\
0.4808481	-5.41311593954662e-05\\
0.4809481	2.60744974227697e-05\\
0.4810481	0.000268885667032582\\
0.4811481	0.000236488714395289\\
0.4812481	-9.4966994125928e-05\\
0.4813481	-0.000221046090761424\\
0.4814481	-0.00018285420633061\\
0.4815482	-0.000372358501658229\\
0.4816482	-0.000539109454976079\\
0.4817482	-0.00031894631080037\\
0.4818482	-6.46066942596103e-05\\
0.4819482	-0.000175032229048483\\
0.4820482	-0.000196829537423248\\
0.4821482	9.98617598028991e-05\\
0.4822482	4.60320432644143e-05\\
0.4823482	0.000155786695048067\\
0.4824482	0.000251748098697595\\
0.4825483	0.000449803994001247\\
0.4826483	0.000260449148582067\\
0.4827483	0.000290010214836991\\
0.4828483	0.000117225582185691\\
0.4829483	1.53641528237783e-07\\
0.4830483	-2.02162147502584e-05\\
0.4831483	-3.89315927748733e-05\\
0.4832483	4.06809028237988e-05\\
0.4833483	-3.83013310855679e-05\\
0.4834483	-3.08889613027594e-06\\
0.4835484	-8.35841127098703e-05\\
0.4836484	-0.000207269044195753\\
0.4837484	-9.91291022234431e-05\\
0.4838484	-0.000110406882354941\\
0.4839484	-0.000251221978624268\\
0.4840484	6.99680164672981e-05\\
0.4841484	9.64765119560781e-05\\
0.4842484	-6.47430815791609e-05\\
0.4843484	-0.000107364352012407\\
0.4844484	0.000247700114334525\\
0.4845485	5.72579775792803e-05\\
0.4846485	-1.6676736468023e-05\\
0.4847485	0.00021729753708016\\
0.4848485	0.000113854218160954\\
0.4849485	-1.95382717118799e-05\\
0.4850485	0.000116035490556946\\
0.4851485	1.17629296492059e-05\\
0.4852485	8.91054718380911e-05\\
0.4853485	5.76355449931377e-05\\
0.4854485	-0.000127994421414727\\
0.4855486	-0.000258419744491915\\
0.4856486	-0.000457352392141299\\
0.4857486	-0.000207552780847836\\
0.4858486	-0.000406399474683716\\
0.4859486	-0.00014002703609478\\
0.4860486	-2.13530258954655e-05\\
0.4861486	4.07299967362098e-05\\
0.4862486	2.75374742637179e-05\\
0.4863486	0.000218691850797067\\
0.4864486	0.000337980572097914\\
0.4865487	0.000183298387697356\\
0.4866487	0.000286189249916642\\
0.4867487	0.000350775804600609\\
0.4868487	0.000181799504170071\\
0.4869487	-0.000100175197277048\\
0.4870487	-5.6607843960439e-05\\
0.4871487	-0.000111461978412362\\
0.4872487	-0.000463912141472445\\
0.4873487	-0.00021248387228251\\
0.4874487	9.90022917185633e-05\\
0.4875488	0.00019088421221386\\
0.4876488	3.13006724174782e-05\\
0.4877488	-3.14025741458219e-05\\
0.4878488	0.000185998941950811\\
0.4879488	0.000265118691863289\\
0.4880488	2.94651484818564e-05\\
0.4881488	-0.000106004213563667\\
0.4882488	-8.29079178986166e-05\\
0.4883488	-8.22124863981977e-05\\
0.4884488	1.48795608177583e-05\\
0.4885489	1.99042448027157e-05\\
0.4886489	0.000117007086961875\\
0.4887489	-0.000171069004451192\\
0.4888489	-0.000298857542479706\\
0.4889489	-0.000150773038384831\\
0.4890489	6.15899983595868e-05\\
0.4891489	-0.000187971939927482\\
0.4892489	-0.000249649359128932\\
0.4893489	-0.000137306763324137\\
0.4894489	-0.000296943654783543\\
0.489549	-2.43148041086131e-05\\
0.489649	0.0004120089521201\\
0.489749	-1.31727431498026e-06\\
0.489849	-0.000183499978392586\\
0.489949	9.60034674466013e-06\\
0.490049	0.000477800209795615\\
0.490149	0.00021010812130621\\
0.490249	-6.96364063251468e-05\\
0.490349	5.05071411579846e-05\\
0.490449	0.0001685432798755\\
0.4905491	0.00032569349835531\\
0.4906491	0.000145659502272839\\
0.4907491	2.3654657543391e-05\\
0.4908491	-8.93665122019254e-05\\
0.4909491	7.52695185591585e-05\\
0.4910491	-5.52137227543478e-05\\
0.4911491	-0.000185334706821041\\
0.4912491	2.88190975883708e-05\\
0.4913491	-1.18557763832353e-05\\
0.4914491	-0.000507820793673858\\
0.4915492	-0.000155032153978115\\
0.4916492	0.000128986287186829\\
0.4917492	0.000223763202331717\\
0.4918492	0.000272831134253393\\
0.4919492	0.000170310627696099\\
0.4920492	3.04692293571088e-05\\
0.4921492	-0.000303946512107595\\
0.4922492	-0.000312866046077561\\
0.4923492	-0.000527970819962239\\
0.4924492	-0.000379187279195187\\
0.4925493	-0.000330506257126112\\
0.4926493	0.000149898722285617\\
0.4927493	0.000352527692522298\\
0.4928493	0.00015011321611313\\
0.4929493	5.58985759191964e-06\\
0.4930493	0.000111781183502883\\
0.4931493	0.000118753762406358\\
0.4932493	0.000180249164884695\\
0.4933493	0.000283100963548033\\
0.4934493	0.000162517733040098\\
0.4935494	7.32470628991393e-05\\
0.4936494	-8.04193504399357e-05\\
0.4937494	-2.38320561772682e-05\\
0.4938494	-3.67147147416636e-06\\
0.4939494	0.00014534398857092\\
0.4940494	-0.00031306408907173\\
0.4941494	-0.000415680115357159\\
0.4942494	-3.80334991598461e-05\\
0.4943494	2.90433527322309e-05\\
0.4944494	2.00390356234923e-05\\
0.4945495	-0.000289702379483656\\
0.4946495	-3.81240909528135e-05\\
0.4947495	0.00035494782770696\\
0.4948495	0.000201789980231281\\
0.4949495	0.000359341972477976\\
0.4950495	8.98764124330443e-05\\
0.4951495	5.20139102166877e-05\\
0.4952495	4.14900780892182e-05\\
0.4953495	0.000146879530457257\\
0.4954495	-0.000185986116120408\\
0.4955496	-0.000217564248509285\\
0.4956496	-0.000324216079210645\\
0.4957496	-0.000132106145583339\\
0.4958496	2.35117760307661e-06\\
0.4959496	-0.000105666482238448\\
0.4960496	-0.000315507495506364\\
0.4961496	-9.99623040435615e-06\\
0.4962496	-7.81830529353426e-05\\
0.4963496	1.08586731047602e-05\\
0.4964496	0.000427324586133091\\
0.4965497	0.000159557904235339\\
0.4966497	7.80732771676037e-05\\
0.4967497	-0.000261445233674336\\
0.4968497	-0.000206721980957277\\
0.4969497	0.000101299684895863\\
0.4970497	8.85114157122254e-05\\
0.4971497	0.00024855786557526\\
0.4972497	7.90116908307974e-05\\
0.4973497	-0.000205855449906533\\
0.4974497	0.000126500104252792\\
0.4975498	8.84872413763407e-05\\
0.4976498	-0.000109921654514266\\
0.4977498	-9.28978584587431e-05\\
0.4978498	0.000207813299596833\\
0.4979498	-0.000171912507987291\\
0.4980498	-2.91316065383802e-05\\
0.4981498	-0.000340854319064892\\
0.4982498	-0.00037299096625016\\
0.4983498	-5.21718664461355e-05\\
0.4984498	4.65536643330169e-05\\
0.4985499	0.000322339251440736\\
0.4986499	0.000584617880661836\\
0.4987499	0.000361897007597747\\
0.4988499	0.000105060325651109\\
0.4989499	7.87195305941377e-05\\
0.4990499	-3.2012679425035e-05\\
0.4991499	-0.000284202603875976\\
0.4992499	-0.00015031853983948\\
0.4993499	4.27021799868053e-06\\
0.4994499	-0.000134972622645661\\
0.49955	-0.000208562629493633\\
0.49965	-0.000154986352135462\\
0.49975	-4.06625349224591e-05\\
0.49985	-5.87004604657802e-05\\
0.49995	4.59515910572663e-05\\
0.50005	-0.000119414658090393\\
0.50015	-0.000134779483199057\\
0.50025	-0.000223694157105684\\
0.50035	-7.18599501875992e-05\\
0.50045	0.000120428869644282\\
0.5005501	0.000132936613632867\\
0.5006501	0.00014586605491141\\
0.5007501	0.000108940320974631\\
0.5008501	0.000243281153844491\\
0.5009501	0.00020174629804201\\
0.5010501	0.00015441750059382\\
0.5011501	-7.89814889614569e-05\\
0.5012501	-6.8873918566293e-05\\
0.5013501	-0.000171100033637904\\
0.5014501	-0.000249097077061541\\
0.5015502	0.000191386215871727\\
0.5016502	0.000117809794224964\\
0.5017502	0.000127330731594185\\
0.5018502	7.36347952907546e-05\\
0.5019502	5.76987551910626e-05\\
0.5020502	-0.00014382261625673\\
0.5021502	-0.000175194544026017\\
0.5022502	-0.000173775250505078\\
0.5023502	0.000179081044509557\\
0.5024502	-0.000262179876180088\\
0.5025503	-0.000300344717368339\\
0.5026503	6.53574941198581e-05\\
0.5027503	0.000407486854759637\\
0.5028503	-0.000126611842183297\\
0.5029503	-0.000318941800811193\\
0.5030503	0.000261905777412551\\
0.5031503	8.44716936701777e-05\\
0.5032503	1.34567517960978e-05\\
0.5033503	-9.34082417159473e-05\\
0.5034503	0.000122727522293527\\
0.5035504	-1.16555508476879e-05\\
0.5036504	-2.77556208264085e-05\\
0.5037504	-0.000233502488923133\\
0.5038504	-0.000157235335244026\\
0.5039504	0.000399866662804666\\
0.5040504	0.000359864330523365\\
0.5041504	0.000207088495926114\\
0.5042504	-0.000142000010252614\\
0.5043504	0.000187139645448949\\
0.5044504	0.00012363829922668\\
0.5045505	-0.000136951451182487\\
0.5046505	0.000212127939407104\\
0.5047505	-9.54611460200691e-05\\
0.5048505	-0.00032121786025999\\
0.5049505	-0.000501601353340105\\
0.5050505	-0.000309080772512424\\
0.5051505	-0.000469345262246455\\
0.5052505	-0.000195403964222468\\
0.5053505	-0.000402996017324554\\
0.5054505	-0.000452140557633533\\
0.5055506	0.000112907290013113\\
0.5056506	0.000301630607980583\\
0.5057506	0.000177924048261125\\
0.5058506	0.000119718486058894\\
0.5059506	0.000583894799416226\\
0.5060506	0.00103812386922033\\
0.5061506	0.000617496579210453\\
0.5062506	0.000234083815984771\\
0.5063506	0.000503666469007479\\
0.5064506	0.000200145430615702\\
0.5065507	-4.62205878790263e-06\\
0.5066507	-0.000170857552917738\\
0.5067507	-0.000625814043239999\\
0.5068507	-0.000913590625852362\\
0.5069507	-0.000570696393944385\\
0.5070507	-0.000581220437790389\\
0.5071507	-0.000121171844742389\\
0.5072507	-1.76496992230265e-05\\
0.5073507	0.000335356917281348\\
0.5074507	0.000566078926228078\\
0.5075508	0.000175506023941722\\
0.5076508	-1.36381583361253e-05\\
0.5077508	-0.000294164165135931\\
0.5078508	-9.78010631523425e-05\\
0.5079508	-0.000134157916101146\\
0.5080508	-0.000134763784711776\\
0.5081508	-0.000234697726720377\\
0.5082508	-1.8538796862767e-05\\
0.5083508	-5.58760468671021e-05\\
0.5084508	0.000185061474553149\\
0.5085509	-6.92239867362832e-05\\
0.5086509	-0.000139433799349218\\
0.5087509	0.000175664032443948\\
0.5088509	0.000263750472063361\\
0.5089509	0.000485186485979758\\
0.5090509	0.000208743043721621\\
0.5091509	0.000114861117882385\\
0.5092509	-0.000117918315872345\\
0.5093509	3.6065721202478e-05\\
0.5094509	-0.000265025789061719\\
0.509551	0.000116582045368848\\
0.509651	0.000173312665224955\\
0.509751	-6.81292971788788e-05\\
0.509851	-2.20628475893665e-05\\
0.509951	-0.000483356988630688\\
0.510051	-6.87307197968466e-05\\
0.510151	6.56369625552673e-05\\
0.510251	-0.000158122934786634\\
0.510351	-8.08094018832362e-05\\
0.510451	-5.04314256360838e-05\\
0.5105511	0.000470048631863011\\
0.5106511	0.000801858823755662\\
0.5107511	0.000388836518338338\\
0.5108511	-0.000229787258578459\\
0.5109511	-0.000461341477988856\\
0.5110511	-0.000257875107684172\\
0.5111511	-0.000259047112245692\\
0.5112511	-0.00025563645303738\\
0.5113511	-0.000158312088198484\\
0.5114511	-0.000119532972636156\\
0.5115512	-0.000233658619710725\\
0.5116512	-0.000255506567080907\\
0.5117512	0.000166293392072194\\
0.5118512	0.000364412315845906\\
0.5119512	0.00023664126560626\\
0.5120512	-0.000218138694004895\\
0.5121512	0.00025554350493791\\
0.5122512	-1.28510663492364e-05\\
0.5123512	-9.31513333521129e-05\\
0.5124512	0.000385303781748988\\
0.5125513	0.000270207720406555\\
0.5126513	-0.000130852855475126\\
0.5127513	-0.000237694794797811\\
0.5128513	-0.000140297006442913\\
0.5129513	-0.000322968395949464\\
0.5130513	-0.000795967865506664\\
0.5131513	-0.000364364313946659\\
0.5132513	0.000190933363263038\\
0.5133513	0.000335666274026831\\
0.5134513	0.000248265529628575\\
0.5135514	-0.000169122364289814\\
0.5136514	3.646323753171e-05\\
0.5137514	0.000832148539515871\\
0.5138514	0.00117631466654149\\
0.5139514	0.00105636274690378\\
0.5140514	0.000526263912321922\\
0.5141514	2.03492979461879e-05\\
0.5142514	-0.000160259957634083\\
0.5143514	-0.0005131327123836\\
0.5144514	-0.0009722978208133\\
0.5145515	-0.000574185600344786\\
0.5146515	-0.000286631645290449\\
0.5147515	-0.000368826585495415\\
0.5148515	-0.00023021926157819\\
0.5149515	-0.00039656851066636\\
0.5150515	-7.53931663891906e-05\\
0.5151515	-0.000572332058870199\\
0.5152515	-0.000299784014719476\\
0.5153515	4.61214297343623e-06\\
0.5154515	-0.000142082405352646\\
0.5155516	0.000131995316006741\\
0.5156516	0.000190753243370107\\
0.5157516	0.000107084028467576\\
0.5158516	0.000125468865929079\\
0.5159516	-0.000131481046049979\\
0.5160516	0.000565315494299161\\
0.5161516	0.00026951969232537\\
0.5162516	0.000352162756965976\\
0.5163516	0.000172755900753645\\
0.5164516	0.000227980339824192\\
0.5165517	0.000745452462714784\\
0.5166517	0.000423793499559833\\
0.5167517	-5.3590496983183e-05\\
0.5168517	-5.06682962877619e-05\\
0.5169517	0.000111641335913088\\
0.5170517	-8.80103624657305e-05\\
0.5171517	-0.000456592149853701\\
0.5172517	-0.000376622781017348\\
0.5173517	-0.000178951007052678\\
0.5174517	-0.000234715575377589\\
0.5175518	-0.000111946561324152\\
0.5176518	-0.000573659684926897\\
0.5177518	0.000217594630335577\\
0.5178518	0.000358207651836071\\
0.5179518	-5.6369349336928e-05\\
0.5180518	-1.05750983718962e-05\\
0.5181518	0.000118861683273429\\
0.5182518	0.000195952277879011\\
0.5183518	-6.25520285292246e-05\\
0.5184518	-0.000583689946172309\\
0.5185519	-0.000719527887288369\\
0.5186519	-1.76387733841175e-05\\
0.5187519	0.000192450622518773\\
0.5188519	-0.000347378394709361\\
0.5189519	0.000371795483592487\\
0.5190519	0.000471063569883544\\
0.5191519	0.00036447718042866\\
0.5192519	0.000562117635305937\\
0.5193519	0.000234926258414155\\
0.5194519	-0.000551315622519592\\
0.519552	-0.000697930625936167\\
0.519652	-3.71691335895909e-05\\
0.519752	0.000259595862281834\\
0.519852	-0.000240204295202835\\
0.519952	-0.00070903825905895\\
0.520052	-0.000112394678428454\\
0.520152	0.000297177801427486\\
0.520252	2.03005391360422e-05\\
0.520352	0.000199194897219961\\
0.520452	0.000399902242105506\\
0.5205521	0.000657973883665945\\
0.5206521	0.000973931713504721\\
0.5207521	0.000107856654298848\\
0.5208521	-8.8059911833565e-05\\
0.5209521	9.25634011667287e-05\\
0.5210521	0.000360677983307144\\
0.5211521	-0.000111744771449138\\
0.5212521	-0.000600243465175742\\
0.5213521	-0.000505866695975417\\
0.5214521	-0.000389853057972775\\
0.5215522	-3.28071745748608e-05\\
0.5216522	-0.00061499515498958\\
0.5217522	-0.000511132023611777\\
0.5218522	0.000102463641425705\\
0.5219522	0.000124733266005598\\
0.5220522	-0.000501781719966305\\
0.5221522	-3.125988255364e-05\\
0.5222522	8.09502162181508e-05\\
0.5223522	0.000440770018368619\\
0.5224522	-1.4999030029314e-05\\
0.5225523	0.000517842656704896\\
0.5226523	0.000573120686460499\\
0.5227523	0.000217884229738497\\
0.5228523	-0.000243105251097928\\
0.5229523	-0.000721666289595701\\
0.5230523	0.000186232584796182\\
0.5231523	0.000218092846733627\\
0.5232523	0.000228585974985072\\
0.5233523	-0.000263666547561232\\
0.5234523	-0.000540023233889669\\
0.5235524	-0.000516050142827559\\
0.5236524	-0.000284130239554547\\
0.5237524	-7.25300126772008e-05\\
0.5238524	0.000245472041407728\\
0.5239524	1.9337430467694e-05\\
0.5240524	0.000405817666442106\\
0.5241524	0.000491934265449398\\
0.5242524	0.000802468747794653\\
0.5243524	0.000438792637976901\\
0.5244524	0.000186877464696176\\
0.5245525	0.000181201675131349\\
0.5246525	0.000785923432645021\\
0.5247525	0.000411890739621595\\
0.5248525	-0.000271594858355508\\
0.5249525	-0.000711961811465048\\
0.5250525	-0.000890568565639942\\
0.5251525	-0.0011395935625605\\
0.5252525	-0.00047685523964688\\
0.5253525	-0.000716352030051777\\
0.5254525	-0.000613192362653066\\
0.5255526	0.000107946870431728\\
0.5256526	-4.75053456554936e-05\\
0.5257526	0.000923213636735077\\
0.5258526	0.000695865405902031\\
0.5259526	0.000654861554455252\\
0.5260526	-0.000153836320676753\\
0.5261526	9.13533817603306e-05\\
0.5262526	0.000270522267353971\\
0.5263526	-0.00019320805396799\\
0.5264526	-0.000608105967929866\\
0.5265527	-0.000546693546993091\\
0.5266527	0.000145208700480879\\
0.5267527	0.000203098225929271\\
0.5268527	-0.00045441333890513\\
0.5269527	0.000525645642104652\\
0.5270527	0.000233126809476077\\
0.5271527	-5.62881918746705e-05\\
0.5272527	0.000560582287373827\\
0.5273527	-0.000204040099044313\\
0.5274527	0.000335956307024786\\
0.5275528	0.000188544416057167\\
0.5276528	-0.000277326097764492\\
0.5277528	3.68971836270347e-05\\
0.5278528	0.000316555936140991\\
0.5279528	8.00218401421776e-05\\
0.5280528	-0.000185703419541869\\
0.5281528	-0.000311898153613696\\
0.5282528	-0.000255690668299002\\
0.5283528	-0.000325479265339337\\
0.5284528	0.000321477758014777\\
0.5285529	0.000153428462855834\\
0.5286529	0.000123492373145344\\
0.5287529	-0.000161572959185743\\
0.5288529	-0.000724215813347952\\
0.5289529	-0.000278704464025385\\
0.5290529	1.75628186315571e-05\\
0.5291529	-0.000343052230987385\\
0.5292529	-2.87787394450539e-06\\
0.5293529	0.000367027633252262\\
0.5294529	0.00035413603865668\\
0.529553	-0.000393419273873735\\
0.529653	0.00017996072767721\\
0.529753	0.000298728900800141\\
0.529853	7.10470118692949e-05\\
0.529953	0.000711706831855424\\
0.530053	5.13401363325047e-05\\
0.530153	0.00013803870548474\\
0.530253	0.000410344324113521\\
0.530353	0.000244268781644338\\
0.530453	-1.39661278664716e-05\\
0.5305531	-7.84715211088263e-05\\
0.5306531	-0.000576509208836624\\
0.5307531	-1.53308517723272e-05\\
0.5308531	-0.000520554637262631\\
0.5309531	-0.000744928747988927\\
0.5310531	-0.000581871361960667\\
0.5311531	-0.000511380652508296\\
0.5312531	-8.07547882772444e-05\\
0.5313531	-0.000529981933219931\\
0.5314531	0.000181239753410091\\
0.5315532	0.00123017483566979\\
0.5316532	0.000794890299969771\\
0.5317532	0.000721500148201167\\
0.5318532	-0.000199483760020741\\
0.5319532	-0.000537139560348156\\
0.5320532	0.000121984616306317\\
0.5321532	0.000174450643776074\\
0.5322532	0.000368020400647304\\
0.5323532	0.000190345770265631\\
0.5324532	-0.000313921359257237\\
0.5325533	-0.000105740557148022\\
0.5326533	-0.000161176409278312\\
0.5327533	-0.000142469065647577\\
0.5328533	-0.000214826619009206\\
0.5329533	0.000211222842682476\\
0.5330533	8.59612362785755e-05\\
0.5331533	9.92704834425692e-05\\
0.5332533	0.000230302510655866\\
0.5333533	-9.14407507744547e-05\\
0.5334533	0.00033275263529\\
0.5335534	-0.000148460842952246\\
0.5336534	0.000613952277385386\\
0.5337534	-1.01021153601777e-05\\
0.5338534	-0.000482612065654878\\
0.5339534	-0.000554555613100943\\
0.5340534	-0.000732490792430596\\
0.5341534	-0.00113782563349957\\
0.5342534	-0.000474618161280645\\
0.5343534	2.03636041424052e-05\\
0.5344534	0.000352041647581444\\
0.5345535	0.000462538710174876\\
0.5346535	0.000186795914957162\\
0.5347535	0.000340275383628956\\
0.5348535	0.000589209120651787\\
0.5349535	-2.89086458582499e-06\\
0.5350535	0.000102207442237774\\
0.5351535	9.99860603833058e-05\\
0.5352535	-4.00629859425572e-05\\
0.5353535	-0.000131157667582274\\
0.5354535	0.000491694049579994\\
0.5355536	0.000447981360004013\\
0.5356536	6.82826484859969e-05\\
0.5357536	0.000391956469976938\\
0.5358536	4.0224878489252e-05\\
0.5359536	0.000187639933024487\\
0.5360536	0.000172653697578032\\
0.5361536	-0.000112451758854257\\
0.5362536	-0.000504054362270287\\
0.5363536	-0.00130389203365563\\
0.5364536	-0.00108750268897761\\
0.5365537	-0.000971000475143875\\
0.5366537	-0.000148770153889775\\
0.5367537	0.000453683467796\\
0.5368537	7.09424940831885e-05\\
0.5369537	0.000859649034189475\\
0.5370537	0.000921435202386134\\
0.5371537	0.000857803118003694\\
0.5372537	0.000154464905438051\\
0.5373537	0.000867562694156131\\
0.5374537	0.00050962861870122\\
0.5375538	-0.000533424598902071\\
0.5376538	-0.000783895285240969\\
0.5377538	-0.000122481399451743\\
0.5378538	-0.000295850786621442\\
0.5379538	-9.27612867324262e-05\\
0.5380538	-0.000409720734655512\\
0.5381538	-0.000946436960145257\\
0.5382538	0.000140762212165788\\
0.5383538	-0.000164341037227393\\
0.5384538	-0.000600454522696518\\
0.5385539	-4.92884379394293e-05\\
0.5386539	0.000788304897135828\\
0.5387539	0.000525194586077499\\
0.5388539	-0.000173067164893445\\
0.5389539	-0.0011296381443955\\
0.5390539	-0.00075607613588112\\
0.5391539	-0.000120448917631688\\
0.5392539	-0.000285684262751613\\
0.5393539	-0.000294609939163262\\
0.5394539	0.000437836290397772\\
0.539554	0.00119909153695798\\
0.539654	0.000951349048557447\\
0.539754	0.00136923521135617\\
0.539854	0.000936092283426067\\
0.539954	0.000831872528052821\\
0.540054	0.0010768282137399\\
0.540154	0.000503321614215638\\
0.540254	9.76650084367203e-05\\
0.540354	0.000234490680594043\\
0.540454	-0.000661169079882437\\
0.5405541	-0.00157649963164029\\
0.5406541	-0.00178148260779838\\
0.5407541	-0.00114905411450793\\
0.5408541	-0.00120250184132728\\
0.5409541	-0.00144678347255083\\
0.5410541	-0.000362766687204023\\
0.5411541	0.000419010840961694\\
0.5412541	0.000942581443473284\\
0.5413541	-0.000195362542858693\\
0.5414541	-8.6778775935048e-05\\
0.5415542	0.000425725146728423\\
0.5416542	0.000380282251268682\\
0.5417542	0.00019281416949888\\
0.5418542	0.000843993264723904\\
0.5419542	0.000780191905560612\\
0.5420542	0.000237102465941774\\
0.5421542	-0.000604842674877863\\
0.5422542	-0.000906421132318286\\
0.5423542	-0.00112214051646899\\
0.5424542	-0.000192088432084944\\
0.5425543	0.000947375520511363\\
0.5426543	0.00102154855654334\\
0.5427543	0.000213690281313295\\
0.5428543	0.000460323111455684\\
0.5429543	0.000354029468961813\\
0.5430543	0.000791191781184433\\
0.5431543	0.000481802480841925\\
0.5432543	0.000312744006022603\\
0.5433543	3.28800189047795e-07\\
0.5434543	-0.000205750687817842\\
0.5435544	-0.000491565819769453\\
0.5436544	-0.00114516567220085\\
0.5437544	-0.000524732425502474\\
0.5438544	-0.000419493609272667\\
0.5439544	1.17232522895902e-05\\
0.5440544	0.000651010640388701\\
0.5441544	0.000204811041635271\\
0.5442544	-0.000455463051948032\\
0.5443544	-0.000620089142923545\\
0.5444544	-5.05947284347094e-05\\
0.5445545	0.000485437314998408\\
0.5446545	0.000564901127468674\\
0.5447545	-0.000825837010952249\\
0.5448545	-0.00114993457568754\\
0.5449545	-0.000673199036722291\\
0.5450545	0.000240282141400094\\
0.5451545	0.000640401499582097\\
0.5452545	7.62915841752559e-05\\
0.5453545	0.000962834946983807\\
0.5454545	0.000344364145268838\\
0.5455546	-0.000266114960157202\\
0.5456546	-0.000396085515039959\\
0.5457546	-0.000351932527313331\\
0.5458546	-0.000423293417850079\\
0.5459546	-0.000160995602048908\\
0.5460546	0.000412983510168691\\
0.5461546	0.000203216514362017\\
0.5462546	1.96460115738789e-05\\
0.5463546	0.00121077460833457\\
0.5464546	0.00126020491666509\\
0.5465547	-0.000124408207790791\\
0.5466547	-0.000402265710013117\\
0.5467547	-0.00039022362899746\\
0.5468547	0.000707040669276779\\
0.5469547	3.26598243355763e-05\\
0.5470547	-0.000334863518785505\\
0.5471547	-0.000286846709538768\\
0.5472547	-0.000315997091861558\\
0.5473547	-0.00112567200417266\\
0.5474547	-0.000486488779370395\\
0.5475548	0.000281636695969784\\
0.5476548	0.000456395153233188\\
0.5477548	0.000388379784756615\\
0.5478548	0.000771623279759479\\
0.5479548	0.000634738332996376\\
0.5480548	0.000430687644759023\\
0.5481548	-0.000582856079120078\\
0.5482548	-0.00135631012726672\\
0.5483548	7.06782172393561e-05\\
0.5484548	0.000169081676866079\\
0.5485549	0.000497943891246799\\
0.5486549	-0.000187363267331389\\
0.5487549	-0.00121512710733204\\
0.5488549	-0.00176782488406675\\
0.5489549	-0.00053557384728678\\
0.5490549	-0.000203881241181522\\
0.5491549	0.000143855695623932\\
0.5492549	-0.000235750269929308\\
0.5493549	0.000211713634668709\\
0.5494549	0.000623780187498493\\
0.549555	0.000149362828337413\\
0.549655	0.000497224023784271\\
0.549755	0.00177472623794669\\
0.549855	0.00076464227121377\\
0.549955	0.000406404929555348\\
0.550055	0.000544197024523727\\
0.550155	0.000485151373255471\\
0.550255	0.00113814079847367\\
0.550355	0.000230638128488295\\
0.550455	-0.000671793802800497\\
0.5505551	-0.000421771476009136\\
0.5506551	0.000160247612011388\\
0.5507551	-0.00108792402316366\\
0.5508551	-0.000973123525254985\\
0.5509551	-0.000774128032384996\\
0.5510551	-0.000834724677077486\\
0.5511551	-0.000459520586256187\\
0.5512551	-7.97928812431847e-05\\
0.5513551	-0.00110306867775754\\
0.5514551	-2.56050859144402e-05\\
0.5515552	0.00029081177828344\\
0.5516552	0.000814973885557888\\
0.5517552	0.000567974087390982\\
0.5518552	0.000864065296101557\\
0.5519552	0.00156317042961837\\
0.5520552	8.26524114807409e-05\\
0.5521552	-0.000916495829160091\\
0.5522552	7.22864245966681e-06\\
0.5523552	0.000245218766717778\\
0.5524552	-0.000328652510394864\\
0.5525553	-0.00105045064970066\\
0.5526553	-0.000566474793128768\\
0.5527553	-0.00156413146329344\\
0.5528553	-0.000664587691732429\\
0.5529553	0.000655969495632888\\
0.5530553	-9.85769214994214e-05\\
0.5531553	7.38604218690486e-06\\
0.5532553	0.000698481377627402\\
0.5533553	0.00103441208137715\\
0.5534553	0.000953331155611067\\
0.5535554	0.000815494090822433\\
0.5536554	0.00116240004093224\\
0.5537554	0.000923183404795164\\
0.5538554	0.00114677720707353\\
0.5539554	4.94944780508044e-05\\
0.5540554	-0.000341641746368457\\
0.5541554	-0.00128371842465892\\
0.5542554	-0.00177738250967435\\
0.5543554	-0.000590310948647324\\
0.5544554	-0.0014326006831908\\
0.5545555	-0.000903994969788047\\
0.5546555	0.000479603038549548\\
0.5547555	-0.00074299103672007\\
0.5548555	-0.000103724114722911\\
0.5549555	0.000176626891033871\\
0.5550555	0.000329365072661347\\
0.5551555	0.000409203527889465\\
0.5552555	0.000961715360066007\\
0.5553555	0.000355813678155403\\
0.5554555	0.00100930159673883\\
0.5555556	0.000211197419637404\\
0.5556556	0.000991205072827614\\
0.5557556	3.43917070808469e-05\\
0.5558556	-0.000862269540557722\\
0.5559556	-0.00135307552743036\\
0.5560556	-0.00163628310526825\\
0.5561556	-0.000499809120192013\\
0.5562556	-2.04804127126729e-05\\
0.5563556	0.000709336182266887\\
0.5564556	0.000166103835451362\\
0.5565557	0.000834992723813752\\
0.5566557	0.00132038322695262\\
0.5567557	0.00183152433722907\\
0.5568557	0.000580929247768661\\
0.5569557	0.000111221157297753\\
0.5570557	-0.000774406729859886\\
0.5571557	-0.000683571203781444\\
0.5572557	-0.00151401904894899\\
0.5573557	-0.000963347044250426\\
0.5574557	-0.00178785196298145\\
0.5575558	-0.000475851436608235\\
0.5576558	0.000658900731445228\\
0.5577558	0.000740485692965351\\
0.5578558	-0.000472193302986507\\
0.5579558	-0.000696633001766145\\
0.5580558	0.000323469856851638\\
0.5581558	0.00104381853867121\\
0.5582558	0.000834216315072911\\
0.5583558	0.000953366463012181\\
0.5584558	-1.97277349839875e-05\\
0.5585559	-6.70713948743989e-05\\
0.5586559	-5.38431509962689e-05\\
0.5587559	0.00049675063123937\\
0.5588559	-0.000432886743369831\\
0.5589559	0.00154164803553354\\
0.5590559	-0.000167041716135685\\
0.5591559	-0.000586152676909174\\
0.5592559	3.24184802345528e-05\\
0.5593559	-0.000325824912134412\\
0.5594559	-0.000543579515899936\\
0.559556	-2.30376020441125e-05\\
0.559656	0.000715596705075219\\
0.559756	0.000238529851751476\\
0.559856	-0.000308134801738608\\
0.559956	-0.000308793889587544\\
0.560056	-0.000686544040460146\\
0.560156	-0.00107508187749653\\
0.560256	-0.000648204018315654\\
0.560356	-0.00104170707501887\\
0.560456	-0.000232087654193441\\
0.5605561	-0.000804124846880819\\
0.5606561	-0.000286248937732839\\
0.5607561	0.000281459665645495\\
0.5608561	0.000434904378694034\\
0.5609561	0.00181788862234838\\
0.5610561	0.00121501582303686\\
0.5611561	0.0013705894126759\\
0.5612561	0.00184621282866558\\
0.5613561	0.000939289513887345\\
0.5614561	-0.00018057708330145\\
0.5615562	-0.000283452533577158\\
0.5616562	-0.000404193963266186\\
0.5617562	-0.00143996829401052\\
0.5618562	-0.00110527205558414\\
0.5619562	-0.000913101772305538\\
0.5620562	-0.000902653963044078\\
0.5621562	-0.00167022514122184\\
0.5622562	-0.000992011814820434\\
0.5623562	0.000416389513615611\\
0.5624562	0.00139638234697374\\
0.5625563	0.000976614980792101\\
0.5626563	2.13027546482891e-05\\
0.5627563	0.000166792578154454\\
0.5628563	0.00117268797588327\\
0.5629563	0.000879392477814447\\
0.5630563	-0.000599590380669535\\
0.5631563	0.000203742941210866\\
0.5632563	0.00125019598962631\\
0.5633563	0.00102757231613504\\
0.5634563	-0.0014971245223237\\
0.5635564	-0.000178032012771806\\
0.5636564	0.000424636946902396\\
0.5637564	-1.68565517640884e-05\\
0.5638564	-9.90089303577468e-05\\
0.5639564	0.000320283394889681\\
0.5640564	-0.00109767598690594\\
0.5641564	-0.000164383481287472\\
0.5642564	-0.00120633548846147\\
0.5643564	-0.00198642840330512\\
0.5644564	-0.00104525861537114\\
0.5645565	0.000824450962312318\\
0.5646565	0.000414058480160573\\
0.5647565	-0.000800160432362335\\
0.5648565	-0.000729902143570355\\
0.5649565	0.000168936983514972\\
0.5650565	0.00122566059115786\\
0.5651565	0.00090927232690291\\
0.5652565	0.000341575843567393\\
0.5653565	-0.00077712520076606\\
0.5654565	0.000707072857246305\\
0.5655566	-6.74379596083408e-05\\
0.5656566	-0.000327729177585889\\
0.5657566	0.00127638973198305\\
0.5658566	0.00251522245341496\\
0.5659566	0.00106457267625334\\
0.5660566	-0.000238355904741194\\
0.5661566	0.000545340410402273\\
0.5662566	0.000832865326857529\\
0.5663566	-0.00132037744500535\\
0.5664566	-0.000808984189628242\\
0.5665567	-0.000544647581381369\\
0.5666567	-0.000513069558793627\\
0.5667567	-0.000866844327503336\\
0.5668567	0.000650431848730065\\
0.5669567	6.75627112975141e-05\\
0.5670567	-0.00140004799326821\\
0.5671567	-0.000462996513300201\\
0.5672567	0.00057452090799473\\
0.5673567	0.00114560803252847\\
0.5674567	0.000787468627320376\\
0.5675568	-0.00127787430699855\\
0.5676568	-0.00234105386731542\\
0.5677568	-0.00144894862194933\\
0.5678568	-0.000231154783517012\\
0.5679568	-0.000611368559570086\\
0.5680568	-0.00129308615259807\\
0.5681568	-0.000528003760041307\\
0.5682568	0.000774082425701539\\
0.5683568	0.000951676217265524\\
0.5684568	0.00104478143230899\\
0.5685569	0.00188323712367514\\
0.5686569	2.74782796954309e-05\\
0.5687569	0.00135604234605985\\
0.5688569	0.00090833316043698\\
0.5689569	0.000499054565474277\\
0.5690569	0.00105221040878739\\
0.5691569	0.00137140454295182\\
0.5692569	0.000493640825492664\\
0.5693569	-0.000179776881123583\\
0.5694569	-0.000562644709502319\\
0.569557	-0.000612307172407761\\
0.569657	-0.000258961962878307\\
0.569757	-0.00112475523955191\\
0.569857	-0.00169058311540368\\
0.569957	-0.000298541698526861\\
0.570057	0.000523672907854128\\
0.570157	-0.000287335394622862\\
0.570257	-0.000341562699469472\\
0.570357	-4.81050953583528e-05\\
0.570457	-0.000327758666132161\\
0.5705571	-0.0012157511031094\\
0.5706571	-0.00026261355727783\\
0.5707571	0.000629624594944589\\
0.5708571	0.00041466728893147\\
0.5709571	-0.000731481535167638\\
0.5710571	-0.000435017932437692\\
0.5711571	0.000637762046789475\\
0.5712571	0.0010796623569258\\
0.5713571	0.0002016869571142\\
0.5714571	0.00108303981121899\\
0.5715572	-0.000307189558808235\\
0.5716572	-0.000777666548269912\\
0.5717572	0.00038139663996456\\
0.5718572	0.000212503988531426\\
0.5719572	0.000614759484729832\\
0.5720572	-0.00290613287949096\\
0.5721572	-0.00204426910754096\\
0.5722572	0.000577554801796443\\
0.5723572	0.00198434285435392\\
0.5724572	0.000509199060566928\\
0.5725573	0.000254330552149823\\
0.5726573	0.000166636893535882\\
0.5727573	0.000832323451409572\\
0.5728573	0.000582094254502674\\
0.5729573	-0.000969946663912258\\
0.5730573	-0.000329895266032137\\
0.5731573	0.000905352490461808\\
0.5732573	0.00145060065239258\\
0.5733573	0.000861953271073551\\
0.5734573	7.46144022968354e-05\\
0.5735574	-0.00135559081142669\\
0.5736574	-0.000873298651222126\\
0.5737574	0.00146711421971038\\
0.5738574	0.00092155187494869\\
0.5739574	0.000771718392483217\\
0.5740574	0.000595717854704164\\
0.5741574	-0.000965145651615179\\
0.5742574	-0.0025205680353278\\
0.5743574	-0.00227794520092872\\
0.5744574	-0.000720573048568318\\
0.5745575	-0.0005865080194834\\
0.5746575	-0.00103862305475669\\
0.5747575	-0.000127076442478002\\
0.5748575	-0.000770464065555559\\
0.5749575	-0.000887181802622511\\
0.5750575	-0.000146625528052086\\
0.5751575	0.000776008888028229\\
0.5752575	0.0010946255797227\\
0.5753575	0.000534428685352167\\
0.5754575	0.00165042234743981\\
0.5755576	0.00236006895074664\\
0.5756576	0.00133275807098931\\
0.5757576	-0.000668549795516477\\
0.5758576	0.000595249510418061\\
0.5759576	0.00140556015210971\\
0.5760576	-0.000293413703011863\\
0.5761576	-0.00150076788341899\\
0.5762576	-0.00143129821350343\\
0.5763576	0.000262899486411977\\
0.5764576	0.00055872940005533\\
0.5765577	-0.000683226847927126\\
0.5766577	-0.000809417996823159\\
0.5767577	-0.000522864352457852\\
0.5768577	0.00101283828495072\\
0.5769577	-1.95058808430681e-05\\
0.5770577	-0.00114869264212869\\
0.5771577	-0.000189917787254884\\
0.5772577	0.00135662289935465\\
0.5773577	-0.000524666362816478\\
0.5774577	0.0001474186496081\\
0.5775578	0.0013351792190465\\
0.5776578	0.00105702745112676\\
0.5777578	0.000447562655353423\\
0.5778578	0.000175389070508292\\
0.5779578	-0.00126148906081849\\
0.5780578	-0.00109466749224463\\
0.5781578	-0.000676441973612218\\
0.5782578	-0.00125660825100818\\
0.5783578	-0.000195162066776389\\
0.5784578	-0.00178629915953868\\
0.5785579	-0.000567498167386132\\
0.5786579	-0.000511586987799295\\
0.5787579	-0.00140074627462988\\
0.5788579	0.0004625282482216\\
0.5789579	0.000352940860492423\\
0.5790579	0.00159259584553678\\
0.5791579	0.0022431974903093\\
0.5792579	0.00276075008534604\\
0.5793579	0.000508157924746983\\
0.5794579	0.000719825306157912\\
0.579558	0.000497194096367309\\
0.579658	0.00151879553913023\\
0.579758	0.000867569442241256\\
0.579858	-0.000438079882650988\\
0.579958	-0.00038044812044527\\
0.580058	-0.0023820309526045\\
0.580158	-0.00328602405717657\\
0.580258	-0.00228122310881113\\
0.580358	-0.00112432377878263\\
0.580458	-0.00151852173500377\\
0.5805581	0.00121304582307022\\
0.5806581	0.00242166841750879\\
0.5807581	0.00187310674626218\\
0.5808581	0.00224916515466095\\
0.5809581	0.00128314799130064\\
0.5810581	-0.000910740391982329\\
0.5811581	-7.56956401286313e-05\\
0.5812581	-0.000887113394875932\\
0.5813581	-0.00182668929477911\\
0.5814581	-0.00199511897523055\\
0.5815582	-0.000659118269744408\\
0.5816582	-0.000432640247771177\\
0.5817582	-0.000366402889372008\\
0.5818582	-9.06018174862688e-05\\
0.5819582	0.000977767348007086\\
0.5820582	0.000411408990269951\\
0.5821582	0.000703527495484829\\
0.5822582	0.00203022725283239\\
0.5823582	0.000291712654473392\\
0.5824582	0.000752188095523187\\
0.5825583	0.00122675954369872\\
0.5826583	-0.0010538695383156\\
0.5827583	0.00103500462636882\\
0.5828583	0.00172038644449099\\
0.5829583	0.00116498032563739\\
0.5830583	0.000218590682227432\\
0.5831583	-0.00137637807051258\\
0.5832583	-0.00211532151457045\\
0.5833583	-0.00129453522916884\\
0.5834583	-0.0019617147907891\\
0.5835584	-0.00282803199228604\\
0.5836584	-0.00143852772130676\\
0.5837584	-0.000750076006124478\\
0.5838584	0.000932627587531321\\
0.5839584	-8.52125034382922e-05\\
0.5840584	-0.00105969183952154\\
0.5841584	-0.000424905978622325\\
0.5842584	0.00161934952392012\\
0.5843584	0.00166557911530746\\
0.5844584	0.00308678724525985\\
0.5845585	0.00274772480123353\\
0.5846585	0.00206730565710317\\
0.5847585	0.00191927842003031\\
0.5848585	-0.00118355245042251\\
0.5849585	-0.00267508249228948\\
0.5850585	-0.000935407241229702\\
0.5851585	-0.000590022230543052\\
0.5852585	0.00113247700879865\\
0.5853585	0.00109389494813406\\
0.5854585	-0.000981463938916238\\
0.5855586	-0.000837125641225511\\
0.5856586	-0.00113282241679957\\
0.5857586	-0.00135558257995244\\
0.5858586	-0.000816401648166265\\
0.5859586	0.000263324863239878\\
0.5860586	0.000565901441085981\\
0.5861586	0.000527932574304391\\
0.5862586	-0.00131667724608109\\
0.5863586	-0.000505623526982801\\
0.5864586	0.000364098226727505\\
0.5865587	-9.89144045707713e-05\\
0.5866587	0.000781359291061878\\
0.5867587	8.94145231798427e-05\\
0.5868587	0.000536755794822652\\
0.5869587	-0.000425012389055901\\
0.5870587	0.000604914478383789\\
0.5871587	0.000209940905848749\\
0.5872587	-0.000662728596118101\\
0.5873587	9.39104848386996e-05\\
0.5874587	-0.000623037337140031\\
0.5875588	0.00102604953516267\\
0.5876588	-3.11561188542364e-05\\
0.5877588	0.00039706488159641\\
0.5878588	-0.000431482942457984\\
0.5879588	0.0013677049316671\\
0.5880588	0.000670233028284672\\
0.5881588	-0.000248594126689095\\
0.5882588	-0.000759472005762207\\
0.5883588	-0.00251099607989286\\
0.5884588	-5.39618185164126e-05\\
0.5885589	-0.00180842315826346\\
0.5886589	-0.0012066561585517\\
0.5887589	0.00149478278176485\\
0.5888589	0.00242159819907278\\
0.5889589	0.00225229463114673\\
0.5890589	2.74766171229346e-05\\
0.5891589	0.000549748697465602\\
0.5892589	0.00184391541394779\\
0.5893589	0.00205338130961957\\
0.5894589	-0.000145849071221116\\
0.589559	-3.23047482583272e-05\\
0.589659	-0.00121341152927745\\
0.589759	-0.0021531009421466\\
0.589859	-0.00255866843784802\\
0.589959	-0.00200050946625435\\
0.590059	-0.000745219476158465\\
0.590159	-0.000258493915299751\\
0.590259	-0.000559628230395515\\
0.590359	-0.000230017867166199\\
0.590459	0.00121124172963261\\
0.5905591	0.000777246909263785\\
0.5906591	0.00174022120328521\\
0.5907591	0.00145915840667765\\
0.5908591	0.000604163078262554\\
0.5909591	-0.000821560222318496\\
0.5910591	0.000357893065368375\\
0.5911591	-0.000237372497479849\\
0.5912591	-0.000299552348937693\\
0.5913591	-0.00298694192637206\\
0.5914591	-0.00441753666647965\\
0.5915592	-0.00201201439812732\\
0.5916592	-0.000842903164872122\\
0.5917592	-2.26833960728645e-05\\
0.5918592	0.00203234947397013\\
0.5919592	0.00281140001147728\\
0.5920592	0.00294707278316153\\
0.5921592	0.0035389723561986\\
0.5922592	0.00341570329819507\\
0.5923592	0.00403637017715919\\
0.5924592	0.00214167756146844\\
0.5925593	0.000549273897683722\\
0.5926593	-0.0013157213489266\\
0.5927593	-0.00240996237857732\\
0.5928593	-0.0011776446217181\\
0.5929593	-0.00258036350858526\\
0.5930593	-0.00286911446923144\\
0.5931593	-0.00189539293355623\\
0.5932593	0.00131020566866141\\
0.5933593	-0.000221414092270666\\
0.5934593	-0.00267554764598593\\
0.5935594	-0.00301941981660758\\
0.5936594	-0.00160936454708188\\
0.5937594	-3.34093546425773e-05\\
0.5938594	0.00054065033098416\\
0.5939594	-0.000930180920027002\\
0.5940594	-0.000531698537638126\\
0.5941594	-0.000205097951978594\\
0.5942594	-0.000313574593379162\\
0.5943594	-0.000783623892401497\\
0.5944594	0.00311655872012573\\
0.5945595	0.00324027560330342\\
0.5946595	0.00336713848855413\\
0.5947595	0.00268384699551655\\
0.5948595	-4.99943080300436e-05\\
0.5949595	0.000402219145264227\\
0.5950595	0.000708091922284659\\
0.5951595	0.000574028589420239\\
0.5952595	-0.00103856628747168\\
0.5953595	-0.00111188814309976\\
0.5954595	-0.00201763241276842\\
0.5955596	0.000899830899454723\\
0.5956596	-0.00118794171777453\\
0.5957596	-0.0018179230543251\\
0.5958596	-0.00256900854823239\\
0.5959596	0.000697806361694919\\
0.5960596	0.00135752623585\\
0.5961596	0.00227475563378821\\
0.5962596	0.00288289911419398\\
0.5963596	-0.0002368387651554\\
0.5964596	-0.00012985344742189\\
0.5965597	-0.00107228684686139\\
0.5966597	6.92613662010011e-05\\
0.5967597	6.90464454208284e-05\\
0.5968597	-0.00210022705647667\\
0.5969597	-0.00259475458788117\\
0.5970597	-0.00214603159833321\\
0.5971597	0.00108834646144135\\
0.5972597	0.00101978413949383\\
0.5973597	-0.000792114017378942\\
0.5974597	-0.00131904346367689\\
0.5975598	0.000505882428163573\\
0.5976598	0.00288960691424583\\
0.5977598	0.000449813743838956\\
0.5978598	0.000243207456915306\\
0.5979598	-0.000190707408028068\\
0.5980598	-0.00123592631399772\\
0.5981598	-0.00100604472554794\\
0.5982598	-0.00053325810881405\\
0.5983598	-0.000629261931548053\\
0.5984598	0.00151394833684171\\
0.5985599	-1.71168357243623e-06\\
0.5986599	-0.00104885672254451\\
0.5987599	0.00134312591561686\\
0.5988599	0.000870040754516466\\
0.5989599	0.00181579231592122\\
0.5990599	0.00236928511972134\\
0.5991599	0.00223842368389332\\
0.5992599	-0.000376387475535811\\
0.5993599	0.000302256155471525\\
0.5994599	0.000145459088929489\\
0.59956	-0.00187113361357749\\
0.59966	-0.00155039557690123\\
0.59976	-0.000204384709765915\\
0.59986	0.000263303491359335\\
0.59996	-9.68264722104317e-05\\
0.60006	0.000787029898584146\\
0.60016	-0.00125502289948531\\
0.60026	-0.00435258037197694\\
0.60036	-0.000941138026819222\\
0.60046	-0.000826391374345092\\
0.6005601	0.000211334535513197\\
0.6006601	-0.00086339372267763\\
0.6007601	0.000440195785102515\\
0.6008601	0.0017850075385038\\
0.6009601	0.000336746014577116\\
0.6010601	-0.000839984312266562\\
0.6011601	0.00134542102970459\\
0.6012601	0.00180366650950575\\
0.6013601	-0.000249543406606711\\
0.6014601	0.0021698957448304\\
0.6015602	0.00125488924689621\\
0.6016602	0.00206934297264105\\
0.6017602	-0.000929140856407329\\
0.6018602	-0.00134655778837273\\
0.6019602	-0.00206860337437542\\
0.6020602	-0.00200737316856914\\
0.6021602	-0.00287936272818745\\
0.6022602	-0.000813467613574148\\
0.6023602	-0.00152588338823112\\
0.6024602	-0.000641505618854787\\
0.6025603	0.00438630175237942\\
0.6026603	0.00265938300135871\\
0.6027603	0.00389427107912116\\
0.6028603	0.00318287040577664\\
0.6029603	-0.000541514601962778\\
0.6030603	-0.000343979530826488\\
0.6031603	-0.00233271997102252\\
0.6032603	-0.00148463151628253\\
0.6033603	-0.00170820976390035\\
0.6034603	-0.000947050314775988\\
0.6035604	-0.00175948589782593\\
0.6036604	-2.323472385235e-05\\
0.6037604	0.000814567326515747\\
0.6038604	0.000244424637544594\\
0.6039604	-0.000452658410310447\\
0.6040604	-0.00104707744044702\\
0.6041604	0.00207937191983765\\
0.6042604	-0.00061500596068742\\
0.6043604	0.00110639328276735\\
0.6044604	0.000648374010969011\\
0.6045605	-0.00072316485684315\\
0.6046605	-0.00102910490077007\\
0.6047605	3.49496021445628e-05\\
0.6048605	-8.419700384249e-05\\
0.6049605	-0.000441640378711972\\
0.6050605	0.000283523813282227\\
0.6051605	2.38999035648835e-05\\
0.6052605	0.000419692219200407\\
0.6053605	-0.00116359491715118\\
0.6054605	-0.000896757187277497\\
0.6055606	-0.000440763592903182\\
0.6056606	0.000403140041696226\\
0.6057606	0.0014952814743438\\
0.6058606	0.00214136500502621\\
0.6059606	0.00374809492906564\\
0.6060606	0.00157497553707808\\
0.6061606	2.60111149300579e-05\\
0.6062606	-0.000700194056312053\\
0.6063606	-0.000123035700413165\\
0.6064606	-0.00133620954602354\\
0.6065607	-0.00150795208940813\\
0.6066607	-0.00352887426527282\\
0.6067607	-0.0018313158540737\\
0.6068607	7.2427395569119e-05\\
0.6069607	0.000189059729931218\\
0.6070607	-0.000865814609858351\\
0.6071607	-0.00162729138786177\\
0.6072607	0.00139713362662475\\
0.6073607	0.000664064659023467\\
0.6074607	-0.000235694070568115\\
0.6075608	-0.000614046131407364\\
0.6076608	0.00113563157456076\\
0.6077608	0.00119133215063916\\
0.6078608	0.000202559795173807\\
0.6079608	-0.000673281299037876\\
0.6080608	-0.000844686944793005\\
0.6081608	6.25470394715055e-05\\
0.6082608	5.96248296368396e-05\\
0.6083608	0.00251115059584817\\
0.6084608	0.00295492850248058\\
0.6085609	0.00270938832350609\\
0.6086609	0.0021633863433006\\
0.6087609	0.000763048965674473\\
0.6088609	0.00485138033143462\\
0.6089609	0.000261384575388601\\
0.6090609	-0.0031086341737071\\
0.6091609	-0.00406137179319033\\
0.6092609	-0.00417762416654321\\
0.6093609	-0.00331758718343322\\
0.6094609	-0.00404385673976349\\
0.609561	-0.00484796930786902\\
0.609661	-0.00231913625225888\\
0.609761	-0.000234997457581709\\
0.609861	0.000774351154855135\\
0.609961	0.000485613657275952\\
0.610061	0.00310209411539286\\
0.610161	0.00518809658836017\\
0.610261	0.0020467251287255\\
0.610361	0.00196658378238013\\
0.610461	0.00210497658851772\\
0.6105611	0.00134680123273123\\
0.6106611	5.06778786040218e-05\\
0.6107611	-0.00153589924211583\\
0.6108611	-0.00289002611751032\\
0.6109611	0.000763101257395205\\
0.6110611	-0.00127061311940113\\
0.6111611	-0.00149836525693492\\
0.6112611	-0.00106415117131875\\
0.6113611	-0.0016844668857916\\
0.6114611	0.00219379156923349\\
0.6115612	0.00425285643483608\\
0.6116612	0.00533027859275026\\
0.6117612	0.00211878678934985\\
0.6118612	0.000873984965042762\\
0.6119612	-0.000903022947175702\\
0.6120612	-0.00298803302177275\\
0.6121612	-0.00140784134072103\\
0.6122612	-0.00132614399355502\\
0.6123612	-0.00202993707740939\\
0.6124612	-0.00521781669707988\\
0.6125613	-0.00338161566559747\\
0.6126613	-9.16531789589836e-05\\
0.6127613	0.00211633441495074\\
0.6128613	0.000256950980234039\\
0.6129613	-0.00105979962691005\\
0.6130613	0.00031838644177884\\
0.6131613	0.00182551302656531\\
0.6132613	0.00346548395967017\\
0.6133613	0.00130370306522276\\
0.6134613	-0.00105442584078841\\
0.6135614	-0.000323400242728811\\
0.6136614	-0.00322171019521678\\
0.6137614	-0.00233295675685252\\
0.6138614	-0.000597636144710312\\
0.6139614	0.00251185541574817\\
0.6140614	0.00331072169062507\\
0.6141614	0.00106256643753744\\
0.6142614	0.00195539340556683\\
0.6143614	0.00236460633521074\\
0.6144614	9.79089583340587e-05\\
0.6145615	0.00019363949389517\\
0.6146615	0.000362136263761692\\
0.6147615	-0.00364816612590668\\
0.6148615	-0.00391576397829008\\
0.6149615	-0.000278653605450208\\
0.6150615	0.00162776867161363\\
0.6151615	0.000282506522924275\\
0.6152615	-0.000685336390531144\\
0.6153615	0.000906743583151237\\
0.6154615	0.00131465008674247\\
0.6155616	-0.000511142591394953\\
0.6156616	-0.00356856850086118\\
0.6157616	0.000622942996201026\\
0.6158616	0.00307579550551451\\
0.6159616	0.00185229262341946\\
0.6160616	0.00469383793681821\\
0.6161616	0.00234713502312867\\
0.6162616	-0.0036322125497702\\
0.6163616	-0.00314720122358481\\
0.6164616	-0.000387427449660027\\
0.6165617	0.000356919486730889\\
0.6166617	-0.000702267565437897\\
0.6167617	-0.00149188157872959\\
0.6168617	-0.00079431904356464\\
0.6169617	-0.00364237646025145\\
0.6170617	-0.00449755033904342\\
0.6171617	-0.00326593720019297\\
0.6172617	8.92664260061515e-05\\
0.6173617	0.000605063999160937\\
0.6174617	-0.00127824103126895\\
0.6175618	-0.000755101176804773\\
0.6176618	0.00129800259909818\\
0.6177618	0.00225231406180326\\
0.6178618	0.00384583661965121\\
0.6179618	0.00274947367057821\\
0.6180618	0.00556872860206281\\
0.6181618	0.00363730479107638\\
0.6182618	0.00584300560402884\\
0.6183618	0.00452513439672042\\
0.6184618	-0.000317305485713626\\
0.6185619	0.000481270556232039\\
0.6186619	-0.00266339294431783\\
0.6187619	-0.00277541514577606\\
0.6188619	-0.00589919274619155\\
0.6189619	-0.00496932245453086\\
0.6190619	-0.00313260099073809\\
0.6191619	-0.00190182508578596\\
0.6192619	-0.00126559148172245\\
0.6193619	-0.000868696931725421\\
0.6194619	0.00234286179984333\\
0.619562	-0.000685493249816704\\
0.619662	-0.00212599272095715\\
0.619762	-3.58183680362334e-05\\
0.619862	0.00358213299932253\\
0.619962	0.00301386456004982\\
0.620062	0.00208577948157918\\
0.620162	-0.00257721908020425\\
0.620262	-0.0016574279810162\\
0.620362	0.000582355911775134\\
0.620462	1.04357190937598e-05\\
0.6205621	-0.00087852876861616\\
0.6206621	-0.0013284440128869\\
0.6207621	-0.000695254047536564\\
0.6208621	-5.62557990033685e-05\\
0.6209621	0.003092253794309\\
0.6210621	0.000917977781973001\\
0.6211621	-0.000601980798509594\\
0.6212621	0.000944381078226475\\
0.6213621	0.000621566425375697\\
0.6214621	-0.000799621756102931\\
0.6215622	-0.000257985618619228\\
0.6216622	-0.000295308069748645\\
0.6217622	0.000326004882401087\\
0.6218622	0.00210335618931801\\
0.6219622	0.0035997487899917\\
0.6220622	0.00203638561086195\\
0.6221622	0.00216436956576316\\
0.6222622	-0.00101989644412459\\
0.6223622	-8.56095303328752e-05\\
0.6224622	-0.000889566817152576\\
0.6225623	-9.03321081042741e-05\\
0.6226623	-0.00226496535740894\\
0.6227623	-0.00758283425448136\\
0.6228623	-0.00340013597520705\\
0.6229623	0.00112023229149549\\
0.6230623	-0.000698226656369649\\
0.6231623	-0.00071191003393567\\
0.6232623	-0.00124311506952202\\
0.6233623	0.00164506099530393\\
0.6234623	0.0020952209056877\\
0.6235624	0.000726039486050323\\
0.6236624	0.00163687916018129\\
0.6237624	0.00177321083769399\\
0.6238624	0.00039533720999918\\
0.6239624	0.000847860954952151\\
0.6240624	-0.000344915263209344\\
0.6241624	-0.00241008879391647\\
0.6242624	-0.0016334570003208\\
0.6243624	-0.00159971725935459\\
0.6244624	0.00238133303822674\\
0.6245625	0.00128980761062525\\
0.6246625	-0.00226280928940747\\
0.6247625	-0.000113207886054179\\
0.6248625	0.00270891437406029\\
0.6249625	-0.00144773996978853\\
0.6250625	0.00318133160752687\\
0.6251625	0.00301023161692682\\
0.6252625	0.00286656255507856\\
0.6253625	0.0013711269043285\\
0.6254625	-0.000484772867334837\\
0.6255626	-0.000812283895832655\\
0.6256626	2.80993788952522e-05\\
0.6257626	0.000265425852357781\\
0.6258626	-0.00280150206546992\\
0.6259626	-0.00222068197925118\\
0.6260626	-0.00274301150833267\\
0.6261626	-0.00462578828680327\\
0.6262626	-0.00578830996354618\\
0.6263626	-0.00560147420229473\\
0.6264626	-0.00148527868168424\\
0.6265627	0.00311086884606226\\
0.6266627	0.00358619475510724\\
0.6267627	0.00516858729992276\\
0.6268627	0.0050297487417128\\
0.6269627	0.00701718132650853\\
0.6270627	0.00278858728511677\\
0.6271627	0.0021556688330655\\
0.6272627	0.00105862817055058\\
0.6273627	-0.00478953251761505\\
0.6274627	-0.00209781106206774\\
0.6275628	-0.00328537560832043\\
0.6276628	-0.00480367943176019\\
0.6277628	0.000500105305562881\\
0.6278628	-0.000391519289290598\\
0.6279628	0.000481048875022205\\
0.6280628	0.000831611874088645\\
0.6281628	0.000221371767680001\\
0.6282628	0.00271543059969437\\
0.6283628	-0.00167910960188966\\
0.6284628	0.000238453174926323\\
0.6285629	0.000925590599014092\\
0.6286629	-0.000273730687611273\\
0.6287629	0.000167556945715353\\
0.6288629	-0.000217944553469351\\
0.6289629	0.00073526674612534\\
0.6290629	-0.00264040724050885\\
0.6291629	-0.00462086461473366\\
0.6292629	-0.00490200349431735\\
0.6293629	-0.00346112201348936\\
0.6294629	0.00321088167701034\\
0.629563	0.00571001925223835\\
0.629663	0.00466237687028237\\
0.629763	0.00593956214793834\\
0.629863	0.00281347686785283\\
0.629963	0.000193922795895363\\
0.630063	-0.000875698318908213\\
0.630163	-0.000969284744414267\\
0.630263	-0.000178534765423288\\
0.630363	-0.00331744668373268\\
0.630463	-0.0029179188181887\\
0.6305631	-0.00157129931213063\\
0.6306631	-0.000357182859972549\\
0.6307631	-0.00175872168171121\\
0.6308631	-0.0039723141649427\\
0.6309631	2.27412854186815e-05\\
0.6310631	0.000578146247090738\\
0.6311631	-0.000266197719632766\\
0.6312631	0.00109011092806488\\
0.6313631	0.00366747371547071\\
0.6314631	0.00636849215029123\\
0.6315632	0.00264885842925033\\
0.6316632	-0.00089980332925399\\
0.6317632	-0.00214370502181993\\
0.6318632	-0.000100645211626338\\
0.6319632	0.000700377520299611\\
0.6320632	-0.000192535424971663\\
0.6321632	0.00172251733567008\\
0.6322632	0.000119637167333986\\
0.6323632	-0.00091117458293577\\
0.6324632	-0.00237791658620856\\
0.6325633	-0.0011741561646692\\
0.6326633	-0.00501975068688987\\
0.6327633	-0.00570027158327258\\
0.6328633	0.000682282402105328\\
0.6329633	0.00177791250679227\\
0.6330633	-0.00227538005009853\\
0.6331633	-0.000176594067936353\\
0.6332633	0.00283027163537097\\
0.6333633	-0.00142278177667299\\
0.6334633	-0.000530753159207809\\
0.6335634	0.00106183076969746\\
0.6336634	0.00425603089074089\\
0.6337634	0.00423331636457336\\
0.6338634	0.00464168826094125\\
0.6339634	0.00280214763068254\\
0.6340634	-0.00328430449431361\\
0.6341634	-0.00176866710118095\\
0.6342634	-0.0029579391961061\\
0.6343634	-0.00199311980439476\\
0.6344634	0.00102779202948478\\
0.6345635	-0.000234689703904944\\
0.6346635	0.000759413898717476\\
0.6347635	0.00138461269545531\\
0.6348635	0.00628290756463135\\
0.6349635	0.0018192993651398\\
0.6350635	0.000908788936393012\\
0.6351635	0.0023423770982729\\
0.6352635	-0.00202993534892237\\
0.6353635	-0.00508314762452922\\
0.6354635	-6.32589675706452e-05\\
0.6355636	-0.00174271459334729\\
0.6356636	-0.00564261776458719\\
0.6357636	-0.00692841783828069\\
0.6358636	-0.00261111413273496\\
0.6359636	0.000294294013795326\\
0.6360636	-0.00187719275693854\\
0.6361636	-0.00156057382323271\\
0.6362636	-0.00142384858349054\\
0.6363636	0.00219198354374339\\
0.6364636	0.00525792311971851\\
0.6365637	0.00183656578921987\\
0.6366637	-0.00143227402586987\\
0.6367637	-0.000538004829705452\\
0.6368637	0.00313237385762632\\
0.6369637	0.00242686249557855\\
0.6370637	0.00310946152309555\\
0.6371637	0.00593217135856072\\
0.6372637	0.00175899239975186\\
0.6373637	0.000681925023782615\\
0.6374637	0.00167796958705568\\
0.6375638	-0.0012152373596479\\
0.6376638	-0.000315963819036646\\
0.6377638	0.000381422605536086\\
0.6378638	-0.000725077812897101\\
0.6379638	0.000287535177732723\\
0.6380638	-4.97381915214312e-05\\
0.6381638	-0.000897897710667728\\
0.6382638	-0.00345294319082945\\
0.6383638	-0.00267687446429932\\
0.6384638	-0.00595469138457827\\
0.6385639	-0.00116471646954386\\
0.6386639	0.000939699785571957\\
0.6387639	-0.000112769294339454\\
0.6388639	-0.00165112364816311\\
0.6389639	-0.00126536323623982\\
0.6390639	-0.00209348804042597\\
0.6391639	0.00340450193586195\\
0.6392639	0.000509606667593809\\
0.6393639	0.00184782610807765\\
0.6394639	0.0027971601889123\\
0.639564	-0.00379467267213052\\
0.639664	-0.000922105488507352\\
0.639764	0.000796575999367777\\
0.639864	-0.00125662836426033\\
0.639964	0.0026772812428979\\
0.640064	0.00545430462112728\\
0.640164	-0.000157558451338469\\
0.640264	-0.00155030821836566\\
0.640364	0.00319105505402151\\
0.640464	0.00463353107760867\\
0.6405641	-0.00339912081143834\\
0.6406641	-0.0036814161272043\\
0.6407641	-0.00167059969046224\\
0.6408641	-0.000699671878724661\\
0.6409641	-0.0042266330919527\\
0.6410641	-0.00192448375258802\\
0.6411641	0.00221177569439246\\
0.6412641	0.00130514478142674\\
0.6413641	0.00120962301832406\\
0.6414641	0.00540220989221307\\
0.6415642	0.00312270561839667\\
0.6416642	0.00365351224396604\\
0.6417642	0.000585425830781139\\
0.6418642	-0.00292755422523783\\
0.6419642	0.00304057144891368\\
0.6420642	0.00140980220328091\\
0.6421642	-6.08626351035202e-05\\
0.6422642	-0.00377142376225614\\
0.6423642	-0.00106088189730125\\
0.6424642	-0.00585523778251495\\
0.6425643	-0.00946965038523029\\
0.6426643	-0.00382279999012547\\
0.6427643	-0.00233284971218106\\
0.6428643	-0.00282580038675184\\
0.6429643	-8.66528725789184e-05\\
0.6430643	0.000864591948178228\\
0.6431643	0.00549993316988684\\
0.6432643	0.0047843698633927\\
0.6433643	1.29010759775117e-05\\
0.6434643	0.00139552583131652\\
0.6435644	-0.00068387410529791\\
0.6436644	0.00688093880289808\\
0.6437644	0.00582984218169469\\
0.6438644	0.00478483495984761\\
0.6439644	0.000874916042276963\\
0.6440644	0.000280084310022355\\
0.6441644	-0.000185661379802804\\
0.6442644	-0.002734322194057\\
0.6443644	-0.00414489932361461\\
0.6444644	-0.00400339398340455\\
0.6445645	-0.00494788378366517\\
0.6446645	-0.00318021316586314\\
0.6447645	-0.00301946386920071\\
0.6448645	-0.00115863720531739\\
0.6449645	-0.000162734510131823\\
0.6450645	0.0011072428561052\\
0.6451645	0.00972629350877985\\
0.6452645	0.00819541603886575\\
0.6453645	0.0040336090128803\\
0.6454645	0.000302870972841873\\
0.6455646	-0.00520983519902396\\
0.6456646	-0.0046334356740918\\
0.6457646	-0.00432597168612614\\
0.6458646	-0.0049604447915621\\
0.6459646	0.0012081434284521\\
0.6460646	0.000223791368014314\\
0.6461646	-0.000487502603581103\\
0.6462646	-0.00254974014187881\\
0.6463646	0.001714077072692\\
0.6464646	0.00101094733477569\\
0.6465647	0.00195087386273052\\
0.6466647	0.00204184905253624\\
0.6467647	-0.000197127977979994\\
0.6468647	-0.000369059034321939\\
0.6469647	0.0021230540528745\\
0.6470647	-0.000208790572204576\\
0.6471647	-4.4594790596264e-05\\
0.6472647	-0.000853360508588308\\
0.6473647	0.001848910342229\\
0.6474647	0.000983215804924553\\
0.6475648	-0.000233400747055912\\
0.6476648	0.000427971996649124\\
0.6477648	-0.00379162667305422\\
0.6478648	0.00131380118511104\\
0.6479648	0.000467253486877928\\
0.6480648	-0.000507271877602244\\
0.6481648	-0.000593777043797816\\
0.6482648	-0.00129326417284603\\
0.6483648	0.00170626454841694\\
0.6484648	0.00301280690741801\\
0.6485649	-0.00122055377349084\\
0.6486649	0.000391013128083095\\
0.6487649	-0.0029064131274823\\
0.6488649	-0.000468834856145905\\
0.6489649	0.00376874560019602\\
0.6490649	0.000450325873669888\\
0.6491649	0.00169990357039462\\
0.6492649	0.00308247627042579\\
0.6493649	-0.00312395847226088\\
0.6494649	-0.00721440312982402\\
0.649565	-0.00431473466270749\\
0.649665	-0.00371220268739436\\
0.649765	-0.000795688203139688\\
0.649865	-0.000856193787009518\\
0.649965	0.00150027795760499\\
0.650065	0.00359272440096192\\
0.650165	0.000425142886932184\\
0.650265	0.00165553073295421\\
0.650365	0.000191885230005573\\
0.650465	0.00422020364256359\\
0.6505651	0.00365064846976471\\
0.6506651	0.00235389035754507\\
0.6507651	0.00578708779207049\\
0.6508651	0.00365023793142805\\
0.6509651	0.00146733790702425\\
0.6510651	-0.00191061517644853\\
0.6511651	-0.00284662424104631\\
0.6512651	-0.00319669223562546\\
0.6513651	-0.00186182213585084\\
0.6514651	-0.00281501694424691\\
0.6515652	-0.00207707498600018\\
0.6516652	-0.00259240479844758\\
0.6517652	-0.00105880869122288\\
0.6518652	-0.00236328977466016\\
0.6519652	-0.00314685118610636\\
0.6520652	-0.00119249608998118\\
0.6521652	-0.00233922767779737\\
0.6522652	-0.00476204916819811\\
0.6523652	-0.00152296380699157\\
0.6524652	0.00111002513281844\\
0.6525653	-0.00276584180880191\\
0.6526653	-0.00259205174447508\\
0.6527653	0.00618263191256241\\
0.6528653	0.0112812057801617\\
0.6529653	0.00824666644881433\\
0.6530653	0.00643801048161619\\
0.6531653	0.000422234414241052\\
0.6532653	0.00237433475490834\\
0.6533653	0.00304230798433958\\
0.6534653	0.000983150555742052\\
0.6535654	7.71415367284423e-05\\
0.6536654	0.000264715901815205\\
0.6537654	-0.001223851200578\\
0.6538654	-0.00822456342767097\\
0.6539654	-0.00536242446434909\\
0.6540654	-0.00307343802320692\\
0.6541654	-0.00335160784457522\\
0.6542654	-0.00271393769654971\\
0.6543654	-1.4313750218431e-06\\
0.6544654	-4.1092703712664e-05\\
0.6545655	-0.00189960445215354\\
0.6546655	0.000117391158783509\\
0.6547655	0.00273020747717578\\
0.6548655	0.00297184056768018\\
0.6549655	0.00276328646698376\\
0.6550655	0.000587541183768137\\
0.6551655	-0.0071493993013112\\
0.6552655	-0.00317353903565398\\
0.6553655	0.00277511790525509\\
0.6554655	0.00286856741780595\\
0.6555656	-0.000740835497056835\\
0.6556656	-0.000950809482413106\\
0.6557656	0.00237299661991952\\
0.6558656	2.75785935908157e-05\\
0.6559656	0.00357493219398441\\
0.6560656	0.0014180531482253\\
0.6561656	-0.00339606284487592\\
0.6562656	-0.00439042011483193\\
0.6563656	-0.000205023019512751\\
0.6564656	0.00161812405483008\\
0.6565657	0.00292841345935058\\
0.6566657	0.00211105095821806\\
0.6567657	0.00482042510232206\\
0.6568657	0.00211653139160206\\
0.6569657	0.000863365297486096\\
0.6570657	0.00177892226287391\\
0.6571657	0.00250719770210317\\
0.6572657	-0.00324281299906981\\
0.6573657	-0.00275911448348665\\
0.6574657	-0.00132571142263682\\
0.6575658	-0.00197217456354726\\
0.6576658	-0.00128837284848871\\
0.6577658	-0.00320288077938646\\
0.6578658	-0.00405270314249973\\
0.6579658	0.00192715524716203\\
0.6580658	-0.00109131045418698\\
0.6581658	-0.00174210511911665\\
0.6582658	0.00106476635098773\\
0.6583658	-0.00264370097409694\\
0.6584658	0.000935487946545743\\
0.6585659	0.000105798790919875\\
0.6586659	0.00438728885395512\\
0.6587659	0.00396842010705023\\
0.6588659	0.00132718747544325\\
0.6589659	-0.00447141414458962\\
0.6590659	-0.00496638988575437\\
0.6591659	0.000927255090231505\\
0.6592659	0.00621251559262959\\
0.6593659	0.00260038640164972\\
0.6594659	-0.00366813773155461\\
0.659566	-0.00568055520936483\\
0.659666	-0.00409088149814263\\
0.659766	-0.00324661865564813\\
0.659866	-0.000233772047218528\\
0.659966	0.00218765293265077\\
0.660066	0.0055046508602703\\
0.660166	0.00404121628276877\\
0.660266	0.00178134371805094\\
0.660366	0.00299302765479004\\
0.660466	0.00204526255240411\\
0.6605661	0.00296358539394048\\
0.6606661	0.00347190901145722\\
0.6607661	0.00518376678678933\\
0.6608661	0.00927915306214921\\
0.6609661	-0.00282193784958551\\
0.6610661	-0.00821251166488923\\
0.6611661	-0.00393757412960527\\
0.6612661	-0.00177113101895832\\
0.6613661	-0.00482718813758032\\
0.6614661	-0.00562175131950848\\
0.6615662	-0.00558924875950662\\
0.6616662	-0.00453183820869008\\
0.6617662	-0.00640395140588468\\
0.6618662	-0.00645459430297124\\
0.6619662	0.00072422711868203\\
0.6620662	0.00323050684820969\\
0.6621662	0.00645223884521551\\
0.6622662	0.00729841703978629\\
0.6623662	0.00344503533246376\\
0.6624662	0.000440087594233196\\
0.6625663	0.00472917986014032\\
0.6626663	0.000555084973277165\\
0.6627663	-0.00162659451523472\\
0.6628663	-0.00315186485278485\\
0.6629663	-0.00123073231638648\\
0.6630663	0.00243679678732045\\
0.6631663	-0.000270283877943367\\
0.6632663	0.00416401932189281\\
0.6633663	0.00332869999123342\\
0.6634663	-0.00276524829518943\\
0.6635664	0.0017278141057937\\
0.6636664	1.15918681500293e-05\\
0.6637664	-0.00124727878572739\\
0.6638664	0.00076919560006905\\
0.6639664	0.000387008451730342\\
0.6640664	-0.0023208468342939\\
0.6641664	-0.00372537689128505\\
0.6642664	0.000916411617714497\\
0.6643664	0.00200351199991244\\
0.6644664	-0.00129908246724875\\
0.6645665	-0.000178699183591702\\
0.6646665	-0.000370700348048647\\
0.6647665	0.00214658328494499\\
0.6648665	1.41448736559191e-05\\
0.6649665	0.00114997754654933\\
0.6650665	-0.00412692559772276\\
0.6651665	-0.00166557149032709\\
0.6652665	0.000781032907748447\\
0.6653665	-0.00354511939431523\\
0.6654665	-0.00414403541717068\\
0.6655666	-0.00579301028423584\\
0.6656666	-0.00109047171140819\\
0.6657666	0.00400928189247463\\
0.6658666	0.00416924338733971\\
0.6659666	0.00565840560324737\\
0.6660666	0.00502076134039786\\
0.6661666	0.000655303369125004\\
0.6662666	-0.00206697557012214\\
0.6663666	-0.000743082766757386\\
0.6664666	-0.000660025540078379\\
0.6665667	-0.00628206744730919\\
0.6666667	-0.000210700305429246\\
0.6667667	0.00139780911515253\\
0.6668667	0.0045934533755565\\
0.6669667	0.00330722500702763\\
0.6670667	0.0031431165109117\\
0.6671667	0.00412612035865711\\
0.6672667	0.00479522899182339\\
0.6673667	-0.000593565177934963\\
0.6674667	-0.000497269768852782\\
0.6675668	-0.00745711751177636\\
0.6676668	-0.00565766284891733\\
0.6677668	-0.00336514164904367\\
0.6678668	-0.00125256165001009\\
0.6679668	-0.00114693061957238\\
0.6680668	0.000749743644621326\\
0.6681668	-0.0025805466849782\\
0.6682668	-0.000471809465800277\\
0.6683668	0.00336294741481558\\
0.6684668	0.000479716039656797\\
0.6685669	-0.000595706265224649\\
0.6686669	-0.00542393503131855\\
0.6687669	-0.00734917599217511\\
0.6688669	-9.84371845800247e-05\\
0.6689669	0.00420527332481058\\
0.6690669	0.00375594743947349\\
0.6691669	0.00429657703300768\\
0.6692669	0.00307215394915428\\
0.6693669	0.00474767000179765\\
0.6694669	-0.00139888302504421\\
0.669567	0.00038032145249689\\
0.669667	0.00568060840961948\\
0.669767	0.00387380145193417\\
0.669867	0.000385892244069375\\
0.669967	0.00116287242082302\\
0.670067	0.00268473358717358\\
0.670167	-0.00236953268171591\\
0.670267	-0.00301993484049179\\
0.670367	-0.00314848137359514\\
0.670467	0.000119819204721594\\
0.6705671	-0.00214117809257353\\
0.6706671	-0.0074712061275668\\
0.6707671	-0.00979041278115833\\
0.6708671	-0.00570480668671047\\
0.6709671	-0.00266439650734807\\
0.6710671	0.00313080906406782\\
0.6711671	0.000854801304941422\\
0.6712671	0.00371957146294966\\
0.6713671	0.00697511075605295\\
0.6714671	0.00405541037251123\\
0.6715672	-0.00206364710385509\\
0.6716672	-0.00189385065652647\\
0.6717672	-0.00452532050811127\\
0.6718672	0.00110693441090071\\
0.6719672	0.000509905140372968\\
0.6720672	0.00251758269053148\\
0.6721672	0.00465395804197311\\
0.6722672	0.00558102214568246\\
0.6723672	0.00267476592304394\\
0.6724672	0.00578918026584363\\
0.6725673	0.00176417444105903\\
0.6726673	0.00222390511949566\\
0.6727673	0.00113827885214413\\
0.6728673	-0.00311171358745144\\
0.6729673	-0.00498408145526863\\
0.6730673	-0.0035838340367933\\
0.6731673	-0.00676698064699588\\
0.6732673	-0.0023505306303262\\
0.6733673	0.00109750663931019\\
0.6734673	-0.00481487824144278\\
0.6735674	-0.00793375023948784\\
0.6736674	-0.00574800519456131\\
0.6737674	-0.00305771069609007\\
0.6738674	0.00794512373492662\\
0.6739674	0.00424748854813384\\
0.6740674	0.00892637416382636\\
0.6741674	0.00685477097297849\\
0.6742674	0.00422166933725299\\
0.6743674	-9.40410975314165e-07\\
0.6744674	-2.10679686034609e-05\\
0.6745675	-0.00136775348266926\\
0.6746675	0.00176405659102574\\
0.6747675	0.00183331957938782\\
0.6748675	-0.00555597433140687\\
0.6749675	-0.004942834984349\\
0.6750675	-0.00630527225159527\\
0.6751675	-0.00743429603442669\\
0.6752675	-0.00248291626323738\\
0.6753675	-0.00670814289751487\\
0.6754675	-0.00176498592581442\\
0.6755676	0.0096295383494498\\
0.6756676	0.00434743480976651\\
0.6757676	0.00343168472610111\\
0.6758676	3.92779938067633e-05\\
0.6759676	3.22044792824272e-05\\
0.6760676	0.00639745401998804\\
0.6761676	0.0077840164244769\\
0.6762676	0.00273788147240994\\
0.6763676	0.00354403891458688\\
0.6764676	0.00602347847296913\\
0.6765677	0.000339206685664006\\
0.6766677	-0.00475081821660961\\
0.6767677	-0.00526859201996421\\
0.6768677	-0.000280125117575469\\
0.6769677	-0.00389442793133643\\
0.6770677	-0.00545751091181712\\
0.6771677	0.00040461546174031\\
0.6772677	-0.000506059318535979\\
0.6773677	-0.00307754578911945\\
0.6774677	-0.000934854515067585\\
0.6775678	0.00147604284955394\\
0.6776678	-0.000593940045270802\\
0.6777678	0.0011172229275056\\
0.6778678	-0.00154647891124071\\
0.6779678	-0.00583505626907366\\
0.6780678	-0.00700451988196349\\
0.6781678	-0.00342988051426385\\
0.6782678	0.000790851041323926\\
0.6783678	-0.00282933603622437\\
0.6784678	-0.00696045259622258\\
0.6785679	0.00229055045405202\\
0.6786679	0.0141915443114757\\
0.6787679	0.0100675759581277\\
0.6788679	0.00204963443181549\\
0.6789679	0.00307470874220694\\
0.6790679	0.00543078787087614\\
0.6791679	0.00495086077132179\\
0.6792679	0.0032469163690001\\
0.6793679	0.000123943561370687\\
0.6794679	-0.00613706878208164\\
0.679568	-0.00300805191095188\\
0.679668	-0.0031171748967922\\
0.679768	-0.000444370988724963\\
0.679868	0.000459348571167695\\
0.679968	0.000988972512995456\\
0.680068	-0.00272051046090205\\
0.680168	-0.000118111675906363\\
0.680268	-0.00100584248510183\\
0.680368	0.000888285730772154\\
0.680468	-0.00327873843664994\\
0.6805681	-0.000184827696120937\\
0.6806681	-0.000558189146561361\\
0.6807681	0.00497626261773565\\
0.6808681	-0.00211048392174362\\
0.6809681	0.000327559689051515\\
0.6810681	-0.0114706181232336\\
0.6811681	-0.00981202895930355\\
0.6812681	0.000827315552799579\\
0.6813681	0.00337240375776027\\
0.6814681	0.00485922397301408\\
0.6815682	0.00179088088640844\\
0.6816682	-0.00301586833153111\\
0.6817682	-0.00171092076170476\\
0.6818682	0.00335471180471147\\
0.6819682	-0.000837982450493602\\
0.6820682	0.00369698462746681\\
0.6821682	0.00335560116641176\\
0.6822682	-0.000602144732754883\\
0.6823682	0.00591273500399403\\
0.6824682	0.00681422842384372\\
0.6825683	-0.00247154356003687\\
0.6826683	0.00117414284412831\\
0.6827683	-0.0004555930996748\\
0.6828683	0.00144523654879825\\
0.6829683	0.00360961970317053\\
0.6830683	0.00174954425049156\\
0.6831683	-0.0027120019487219\\
0.6832683	-0.00412603106043719\\
0.6833683	-0.00342555527707988\\
0.6834683	-0.000832586817456793\\
0.6835684	-0.00374798974098134\\
0.6836684	-0.00546507122861817\\
0.6837684	-0.00481769686666356\\
0.6838684	-0.000900878979095854\\
0.6839684	0.00214237008394491\\
0.6840684	0.00649903794620688\\
0.6841684	0.00676911220535237\\
0.6842684	0.00457658043302356\\
0.6843684	-0.00364856982511386\\
0.6844684	-0.00310835104932439\\
0.6845685	0.00095638650431118\\
0.6846685	-0.00294569285860291\\
0.6847685	-0.00549344079604763\\
0.6848685	-0.00694586989154233\\
0.6849685	-0.00213899275429297\\
0.6850685	-0.00270382201913977\\
0.6851685	2.6296534945125e-06\\
0.6852685	0.00329034957765328\\
0.6853685	0.0048773250419063\\
0.6854685	0.000504543309368888\\
0.6855686	0.00695316667747425\\
0.6856686	0.00530183345013624\\
0.6857686	0.00107570465050877\\
0.6858686	-0.0012112325595095\\
0.6859686	0.00334400895679821\\
0.6860686	0.00641941631104848\\
0.6861686	0.00725597658978172\\
0.6862686	-0.0016603231454706\\
0.6863686	-0.00486049585813353\\
0.6864686	9.9445463463077e-05\\
0.6865687	0.00341767439519591\\
0.6866687	-0.00160719419756372\\
0.6867687	-0.00235698788300231\\
0.6868687	-0.00313271974853269\\
0.6869687	-0.0109774029062057\\
0.6870687	-0.0130260504926438\\
0.6871687	-0.0111446756689793\\
0.6872687	-0.00727729162081125\\
0.6873687	-0.00325291155812746\\
0.6874687	0.0002684512847142\\
0.6875688	0.00626898046226199\\
0.6876688	0.0153382700148876\\
0.6877688	0.0119175024858271\\
0.6878688	0.0114306645439104\\
0.6879688	0.00534674283391023\\
0.6880688	0.00210772397659209\\
0.6881688	-0.000583405431203943\\
0.6882688	0.000550341183484182\\
0.6883688	-0.00304504963020635\\
0.6884688	-0.00633159134689398\\
0.6885689	-0.00338209175730464\\
0.6886689	-0.00511097498326896\\
0.6887689	-0.0055620496915898\\
0.6888689	-0.00257532945133132\\
0.6889689	0.00141117214498188\\
0.6890689	0.000722441481434632\\
0.6891689	0.00167246491878581\\
0.6892689	0.0050032287945192\\
0.6893689	0.00730271942292138\\
0.6894689	0.00119992309513961\\
0.689569	-0.00471696067237737\\
0.689669	0.00105662854730015\\
0.689769	0.00243588953220542\\
0.689869	-0.00109619151846951\\
0.689969	-0.0025706284283542\\
0.690069	-0.00610243504384966\\
0.690169	-0.00224462523404112\\
0.690269	-0.00165721289063514\\
0.690369	-0.00261321192791009\\
0.690469	-0.00587963628263299\\
0.6905691	-0.00180328050165812\\
0.6906691	0.00214440314849318\\
0.6907691	-0.00500338047744142\\
0.6908691	0.00184435459442378\\
0.6909691	0.0105635943158187\\
0.6910691	0.0145633246163719\\
0.6911691	0.0110105314036866\\
0.6912691	0.0057462005634159\\
0.6913691	-0.00720468204069169\\
0.6914691	-0.0098041305666956\\
0.6915692	-0.00937193501762046\\
0.6916692	-0.00567055754997314\\
0.6917692	-0.00184978862288965\\
0.6918692	-0.000510642481047263\\
0.6919692	0.00257786660940162\\
0.6920692	0.0060447243609052\\
0.6921692	0.00265291646457033\\
0.6922692	-0.0011865714097441\\
0.6923692	-6.77536133824042e-05\\
0.6924692	-0.00120764451879207\\
0.6925693	0.00214296900030364\\
0.6926693	-0.000769382254866002\\
0.6927693	0.00476651453038768\\
0.6928693	0.00416064489983903\\
0.6929693	0.00102699437650422\\
0.6930693	-0.00403645153728238\\
0.6931693	0.000663292640207036\\
0.6932693	-0.0011047876298429\\
0.6933693	-0.00263470690670031\\
0.6934693	-0.000303479770023429\\
0.6935694	-0.00276189139973182\\
0.6936694	-0.00362341514626063\\
0.6937694	-0.00483283635445186\\
0.6938694	0.00153283031521934\\
0.6939694	-0.00163742981773041\\
0.6940694	-0.00213563145372486\\
0.6941694	-0.000365789313032217\\
0.6942694	0.0110570818642706\\
0.6943694	0.00327796731843\\
0.6944694	-0.00813314772994733\\
0.6945695	-0.00965504822276309\\
0.6946695	-0.00482720873005693\\
0.6947695	0.000614585790097433\\
0.6948695	-0.0033646795194473\\
0.6949695	0.00247498046493044\\
0.6950695	0.00259555084769181\\
0.6951695	0.0052800167142113\\
0.6952695	0.003400363130873\\
0.6953695	0.00878257514513203\\
0.6954695	0.0122786377856108\\
0.6955696	0.0120987648731652\\
0.6956696	0.0114484835900717\\
0.6957696	0.00301200789176992\\
0.6958696	-0.00583867726770571\\
0.6959696	-0.00287958695274781\\
0.6960696	-0.000692736246112879\\
0.6961696	-0.000296140248824184\\
0.6962696	-0.00911881408011395\\
0.6963696	-0.00840277287730107\\
0.6964696	-0.0157420317957486\\
0.6965697	-0.0162723797458475\\
0.6966697	-0.00755528478267609\\
0.6967697	-0.00698453552938227\\
0.6968697	0.00146085278736763\\
0.6969697	0.00106586492331484\\
0.6970697	0.00349448561667076\\
0.6971697	0.00279069958819722\\
0.6972697	0.00354649154130815\\
0.6973697	0.00229584616213607\\
0.6974697	0.00604374811963382\\
0.6975698	0.012182404260229\\
0.6976698	0.0102273543385172\\
0.6977698	0.00531580564266143\\
0.6978698	-0.00385225722629956\\
0.6979698	-0.00326184968407972\\
0.6980698	0.000329012836930431\\
0.6981698	-0.000302685112245626\\
0.6982698	-0.00264295899706957\\
0.6983698	0.00798317570056639\\
0.6984698	-4.82965175305772e-05\\
0.6985699	-0.000540174577156272\\
0.6986699	-0.0026689078319091\\
0.6987699	-0.00499829460959633\\
0.6988699	-0.00269135047285074\\
0.6989699	-0.00972209100016741\\
0.6990699	-0.00451953178583198\\
0.6991699	0.00189331156015848\\
0.6992699	0.0058204234122079\\
0.6993699	0.00501078812919209\\
0.6994699	0.00882739005455901\\
0.69957	0.00751942294502417\\
0.69967	0.00358945145422096\\
0.69977	-0.000649329905243246\\
0.69987	-0.00234693685075271\\
0.69997	-0.00455538511466447\\
0.70007	-0.00455669044421736\\
0.70017	-0.0117298686014506\\
0.70027	-0.00708093536309917\\
0.70037	-0.00589090652049329\\
0.70047	-0.00981979787950626\\
0.7005701	-0.0027144245615518\\
0.7006701	0.00222579524127076\\
0.7007701	0.00679104732289693\\
0.7008701	0.00651231582034173\\
0.7009701	0.00401558485654174\\
0.7010701	0.000969838540508139\\
0.7011701	0.0044180609673794\\
0.7012701	0.00606723621852989\\
0.7013701	-0.00378965163832667\\
0.7014701	0.00708138145093429\\
0.7015702	0.00775750991166069\\
0.7016702	0.0028073358827454\\
0.7017702	-0.00233396513414608\\
0.7018702	-0.000868409138237602\\
0.7019702	-0.00698301214185548\\
0.7020702	-0.00228579017032127\\
0.7021702	0.000988240738132973\\
0.7022702	-0.00360693546751634\\
0.7023702	-0.00247533485101981\\
0.7024702	0.00258402651127954\\
0.7025703	0.00432931100284328\\
0.7026703	-0.0011928557018015\\
0.7027703	0.00460969001946987\\
0.7028703	0.00106493204082625\\
0.7029703	-0.00267714577569222\\
0.7030703	-0.00289455958007379\\
0.7031703	-0.00314232553421956\\
0.7032703	0.00194854018814139\\
0.7033703	-0.00337797859845068\\
0.7034703	-0.00168189809107711\\
0.7035704	-0.00236106954933568\\
0.7036704	-0.00577684053254623\\
0.7037704	-0.00661006089804106\\
0.7038704	-0.00340474688841708\\
0.7039704	-0.000555914757399023\\
0.7040704	-0.00820358076974026\\
0.7041704	-4.37612011139343e-05\\
0.7042704	0.00258852766198725\\
0.7043704	0.0108262695223473\\
0.7044704	0.0107664480721334\\
0.7045705	0.0135341967958515\\
0.7046705	0.0117851981548753\\
0.7047705	0.00683158720077381\\
0.7048705	0.00547234758425619\\
0.7049705	0.000288462945917445\\
0.7050705	-0.00855508308366693\\
0.7051705	-0.00694930688380557\\
0.7052705	-0.00483722484360571\\
0.7053705	-0.00669285336187064\\
0.7054705	-0.00600420884700092\\
0.7055706	-0.0123781746937928\\
0.7056706	-0.00714603514371381\\
0.7057706	-0.00137167185863518\\
0.7058706	0.00395489871572829\\
0.7059706	0.00559966012457647\\
0.7060706	0.00217559590415384\\
0.7061706	0.00557368958184057\\
0.7062706	0.00435292467626791\\
0.7063706	0.000865284697419977\\
0.7064706	0.000569753146733021\\
0.7065707	0.00110742811707976\\
0.7066707	-0.00129293803946062\\
0.7067707	-0.00514924533023997\\
0.7068707	-0.00670251028692981\\
0.7069707	-0.00554574944921704\\
0.7070707	-0.00308797936468785\\
0.7071707	0.00418878341126216\\
0.7072707	0.0116435223155649\\
0.7073707	0.0109422207775625\\
0.7074707	0.0111268622191254\\
0.7075708	0.00738752457920944\\
0.7076708	-0.00410299988236185\\
0.7077708	-0.0092846311591348\\
0.7078708	-0.0105513858580659\\
0.7079708	-0.00481528059304415\\
0.7080708	-0.0032583319847852\\
0.7081708	-0.00130755666071858\\
0.7082708	0.00106102874512226\\
0.7083708	-0.00458959240779722\\
0.7084708	-0.00370543676640578\\
0.7085709	-0.00183844819472467\\
0.7086709	0.00254220880466827\\
0.7087709	0.00752959260370423\\
0.7088709	0.00761368653100503\\
0.7089709	0.000420473909343411\\
0.7090709	0.00498093805577327\\
0.7091709	0.00240806228172907\\
0.7092709	0.00677282989312911\\
0.7093709	0.00338822419050411\\
0.7094709	-0.00271677153090044\\
0.709571	0.00181687540639075\\
0.709671	-0.00453895291922235\\
0.709771	-0.00546022142520064\\
0.709871	-0.00501594683632423\\
0.709971	-0.00755914588210072\\
0.710071	-0.0187918352966596\\
0.710171	-0.010817031818628\\
0.710271	0.001185247808976\\
0.710371	-0.000505013161141932\\
0.710471	0.000785168519558788\\
0.7105711	0.00750380040921028\\
0.7106711	0.00780681452181753\\
0.7107711	0.00680722099023701\\
0.7108711	0.011877003047478\\
0.7109711	0.0108921439229461\\
0.7110711	0.00617462684256775\\
0.7111711	0.00375843502889239\\
0.7112711	-0.00589244829878455\\
0.7113711	-0.00396203992431543\\
0.7114711	-0.00344935663457813\\
0.7115712	0.00499958234413178\\
0.7116712	0.00230176232515067\\
0.7117712	-0.00401583317302423\\
0.7118712	-0.00198422094825679\\
0.7119712	-0.00339241780085242\\
0.7120712	-0.00152744053346288\\
0.7121712	-0.00211030595096329\\
0.7122712	-0.000326030860348325\\
0.7123712	-0.00606363207059557\\
0.7124712	-0.000261126392572919\\
0.7125713	-0.00219356150518157\\
0.7126713	-0.00330989542530258\\
0.7127713	0.00290382708332348\\
0.7128713	-0.000396410796532631\\
0.7129713	-0.00290262588339552\\
0.7130713	-0.00577383499697204\\
0.7131713	0.00438594504196077\\
0.7132713	-0.00467830258831281\\
0.7133713	-0.00824959471036424\\
0.7134713	-0.00488294814746704\\
0.7135714	-0.0010324407012936\\
0.7136714	0.00407802965601567\\
0.7137714	0.00818738820849685\\
0.7138714	0.00769661813118461\\
0.7139714	0.00489870259896552\\
0.7140714	0.00778962478672884\\
0.7141714	0.00893936786946981\\
0.7142714	-0.00370608497760294\\
0.7143714	-0.00205275057893317\\
0.7144714	0.00259035424149268\\
0.7145715	0.000341119888243036\\
0.7146715	0.0032097118109915\\
0.7147715	0.00914402367064859\\
0.7148715	0.00582003864623903\\
0.7149715	-0.000199260082154318\\
0.7150715	-0.000478889333282773\\
0.7151715	-0.00655986592460867\\
0.7152715	-0.00729620667218356\\
0.7153715	-0.00974492839053553\\
0.7154715	-0.00431004789255418\\
0.7155716	-0.0130407082380737\\
0.7156716	-0.00739267717871162\\
0.7157716	-0.0046070943474108\\
0.7158716	-0.00455997654927298\\
0.7159716	0.000334659412828842\\
0.7160716	0.00410779673838668\\
0.7161716	0.00597441862937842\\
0.7162716	0.00695050829038646\\
0.7163716	0.00522004892870598\\
0.7164716	0.00420502375447906\\
0.7165717	-0.0028257454256595\\
0.7166717	0.00154404380962436\\
0.7167717	0.00644421686424668\\
0.7168717	0.00726075696093087\\
0.7169717	-0.00282435267418091\\
0.7170717	-0.000615128811361387\\
0.7171717	-0.0049145882172148\\
0.7172717	-0.0101487476545357\\
0.7173717	-0.00324062388219293\\
0.7174717	-0.000562233654996191\\
0.7175718	-3.97919663006618e-05\\
0.7176718	0.0044040771474787\\
0.7177718	0.0107121624603872\\
0.7178718	0.00835544723500181\\
0.7179718	0.00680591473855685\\
0.7180718	-0.00312345175695139\\
0.7181718	0.00434033102538223\\
0.7182718	0.000231246367467186\\
0.7183718	-0.00542872244363934\\
0.7184718	-0.000765592115630653\\
0.7185719	-0.00683061610435676\\
0.7186719	-0.00298134154214491\\
0.7187719	-0.00113901794779458\\
0.7188719	-0.00389966200674241\\
0.7189719	0.00156270960145743\\
0.7190719	-0.00056391979677084\\
0.7191719	-0.00342456686887704\\
0.7192719	-0.00559024827609633\\
0.7193719	-0.000356980673295482\\
0.7194719	0.00457921929114047\\
0.719572	0.00136405804136253\\
0.719672	-0.00398993129882719\\
0.719772	-0.00379703819909238\\
0.719872	-0.0011002792806778\\
0.719972	0.00153732884225644\\
0.720072	0.000175769562769545\\
0.720172	-0.00292897371875434\\
0.720272	0.00188108240553131\\
0.720372	0.00553392135105391\\
0.720472	0.00292352654094774\\
0.7205721	0.000968562628210101\\
0.7206721	0.00321764633247401\\
0.7207721	-0.00439155341861053\\
0.7208721	0.00611394683012545\\
0.7209721	0.00584813054217559\\
0.7210721	0.00605798118947121\\
0.7211721	0.00517248225250366\\
0.7212721	0.00195261722045004\\
0.7213721	0.00851936959129507\\
0.7214721	-0.000866277128042547\\
0.7215722	0.00246029830061811\\
0.7216722	-0.00569020048229341\\
0.7217722	-0.00996214779629323\\
0.7218722	-0.0110565600974919\\
0.7219722	-0.0122524538324704\\
0.7220722	-0.0108348454381328\\
0.7221722	-0.0108277513415933\\
0.7222722	-0.00464118796005497\\
0.7223722	0.000839828299327827\\
0.7224722	0.00477128103955327\\
0.7225723	0.00484174644903401\\
0.7226723	0.00384901839674789\\
0.7227723	0.0123426776801523\\
0.7228723	0.0162667079441361\\
0.7229723	0.0110860928443755\\
0.7230723	0.00729881604743543\\
0.7231723	-0.00433813876908662\\
0.7232723	-0.0107417879164471\\
0.7233723	-0.0114951476946379\\
0.7234723	-0.0100832343922521\\
0.7235724	-0.00165251849565706\\
0.7236724	0.000432887380703129\\
0.7237724	0.00427351752520977\\
0.7238724	-6.96443039140959e-05\\
0.7239724	0.00224038566357621\\
0.7240724	0.00732359121005022\\
0.7241724	0.000476956130134987\\
0.7242724	0.0027894642308377\\
0.7243724	0.00595309933167254\\
0.7244724	0.00405384526476738\\
0.7245725	-0.000489816744878802\\
0.7246725	-0.000572902529254904\\
0.7247725	-0.00632692592408546\\
0.7248725	-0.0120749030454892\\
0.7249725	-0.00898984999632457\\
0.7250725	-0.0073117828661049\\
0.7251725	0.000138282269157131\\
0.7252725	0.00162832934705914\\
0.7253725	0.00552734231892593\\
0.7254725	0.00237430514994674\\
0.7255726	0.00310964917500733\\
0.7256726	0.00263645858577011\\
0.7257726	-0.000510830180355804\\
0.7258726	0.00722776689848459\\
0.7259726	0.00339923385863017\\
0.7260726	0.00184355475100606\\
0.7261726	-0.000548286358733385\\
0.7262726	-0.00721830539010249\\
0.7263726	-0.00362551824765056\\
0.7264726	-0.00636394082084113\\
0.7265727	0.000733806747209798\\
0.7266727	0.0074699118859316\\
0.7267727	-0.0002512402834558\\
0.7268727	-0.00336166558883999\\
0.7269727	-0.00578837984239211\\
0.7270727	0.00312960115954031\\
0.7271727	-0.000324738363454813\\
0.7272727	-0.00291441417572269\\
0.7273727	-0.00274444202541169\\
0.7274727	0.0050661623556314\\
0.7275728	0.0120697257732214\\
0.7276728	0.00934254207902807\\
0.7277728	-0.00283705649654387\\
0.7278728	-0.000160085618849948\\
0.7279728	0.00630443906366729\\
0.7280728	0.00663850191961757\\
0.7281728	-5.0912665225944e-05\\
0.7282728	-0.0116288202877985\\
0.7283728	-0.000584236527640059\\
0.7284728	-0.00266917694674706\\
0.7285729	-0.00222136934742956\\
0.7286729	-0.00327841030334852\\
0.7287729	0.00123497796637029\\
0.7288729	-0.00767822002890739\\
0.7289729	-0.0161750197617057\\
0.7290729	-0.00978143668624742\\
0.7291729	-0.00285448623840789\\
0.7292729	0.00331881616445884\\
0.7293729	0.00082745512361182\\
0.7294729	0.00105341525906155\\
0.729573	0.00262691262056869\\
0.729673	0.000716463327013805\\
0.729773	0.00836128916813728\\
0.729873	0.00339537484015118\\
0.729973	0.00150670505861594\\
0.730073	0.00707726455855795\\
0.730173	0.012227038094597\\
0.730273	0.0151680104410641\\
0.730373	0.015052166392115\\
0.730473	0.0108244907618511\\
0.7305731	-0.00579985806928107\\
0.7306731	-0.00746824820891234\\
0.7307731	-0.00643751538175152\\
0.7308731	-0.00857767469266424\\
0.7309731	-0.00476474122598436\\
0.7310731	-0.000673730045365417\\
0.7311731	-0.0077146561936674\\
0.7312731	-0.00737853469285632\\
0.7313731	-0.00964938054386177\\
0.7314731	-0.00530820872646753\\
0.7315732	-0.00463592003129865\\
0.7316732	-0.00269676375098005\\
0.7317732	0.00667836537153921\\
0.7318732	-0.00108854755771848\\
0.7319732	0.000926482588991639\\
0.7320732	0.00128644096127672\\
0.7321732	-0.00730468726926858\\
0.7322732	-0.00374491690895962\\
0.7323732	0.0021847372581032\\
0.7324732	0.0121982604701509\\
0.7325733	0.0068116912845424\\
0.7326733	0.00219790222340066\\
0.7327733	-0.00219706195757485\\
0.7328733	0.00160778407038187\\
0.7329733	0.00986342565895804\\
0.7330733	0.00958984818286357\\
0.7331733	0.0004030370399839\\
0.7332733	0.00175897765146615\\
0.7333733	-0.000627344538147945\\
0.7334733	-0.00581994406080799\\
0.7335734	-0.000520844470116799\\
0.7336734	-0.0082540484736997\\
0.7337734	-0.00211857327843479\\
0.7338734	-6.94333210597975e-05\\
0.7339734	-0.00350164301422229\\
0.7340734	-0.00889921674636618\\
0.7341734	0.000828831118385942\\
0.7342734	0.00773548624035048\\
0.7343734	0.00487473430439422\\
0.7344734	0.00641156102005436\\
0.7345735	0.00467487928696375\\
0.7346735	0.00304581407672799\\
0.7347735	0.00319328478166359\\
0.7348735	-0.00341472278883348\\
0.7349735	-0.00673322280015043\\
0.7350735	-0.00338022939229443\\
0.7351735	-0.0108707566798074\\
0.7352735	-0.00546781875165436\\
0.7353735	-0.0130774296710872\\
0.7354735	-0.0059416034755424\\
0.7355736	0.00214950777635803\\
0.7356736	-0.00328584040451307\\
0.7357736	-0.00423879344065536\\
0.7358736	0.00442063473491583\\
0.7359736	0.00955543021556693\\
0.7360736	0.00678557912115441\\
0.7361736	-0.00170593240185468\\
0.7362736	0.00952088181972943\\
0.7363736	0.00973800798592883\\
0.7364736	0.00499143232370969\\
0.7365737	0.00992893643014003\\
0.7366737	-0.000876090835811638\\
0.7367737	-0.000494861100154509\\
0.7368737	-0.00448238802699219\\
0.7369737	0.00367131474708859\\
0.7370737	-0.000938766386894052\\
0.7371737	-0.0127666450101964\\
0.7372737	-0.00675933467622247\\
0.7373737	-0.00478184891043476\\
0.7374737	-0.00173620121023355\\
0.7375738	0.00642732231762899\\
0.7376738	0.00419024663555057\\
0.7377738	0.00720929255174685\\
0.7378738	-0.000594553317806223\\
0.7379738	0.00241169567146393\\
0.7380738	-0.000879973807130341\\
0.7381738	-0.00861457505146301\\
0.7382738	-0.00337612133045884\\
0.7383738	-0.00813562588400529\\
0.7384738	-0.00487310192282564\\
0.7385739	0.00337809541070033\\
0.7386739	-0.00847637011806501\\
0.7387739	-0.00387084660136239\\
0.7388739	-0.00727534713218336\\
0.7389739	-0.00251288477381095\\
0.7390739	0.00381052744026267\\
0.7391739	0.00480387650647351\\
0.7392739	0.00693814945126281\\
0.7393739	0.00673533333120524\\
0.7394739	0.0144334152331078\\
0.739574	0.014259969675873\\
0.739674	0.00979580186944862\\
0.739774	0.00754849351505094\\
0.739874	0.00404903182152339\\
0.739974	-0.00477959597153887\\
0.740074	-0.00131740259367419\\
0.740174	-0.0108944007434379\\
0.740274	-0.0138526030883153\\
0.740374	-0.0165590222646026\\
0.740474	-0.0121136708772989\\
0.7405741	-7.50460194645086e-05\\
0.7406741	0.00690780154484472\\
0.7407741	0.0108453820339962\\
0.7408741	0.00170768296921864\\
0.7409741	-0.00161930809646282\\
0.7410741	0.00022839642189787\\
0.7411741	-0.00797521585874161\\
0.7412741	0.00239884271070157\\
0.7413741	0.00394655981151448\\
0.7414741	-0.00264207684268136\\
0.7415742	-0.00013363719927284\\
0.7416742	-0.0036690254771532\\
0.7417742	0.0027891958136097\\
0.7418742	0.00346501451697012\\
0.7419742	0.00405241850973148\\
0.7420742	0.0125393957016167\\
0.7421742	0.00546793403539247\\
0.7422742	0.00426902148694541\\
0.7423742	0.00304064606541221\\
0.7424742	-0.0109102041867501\\
0.7425743	-0.00722217328174235\\
0.7426743	-0.00401401643852495\\
0.7427743	-0.00809437010938385\\
0.7428743	-0.00133224611750205\\
0.7429743	-0.00485765625221823\\
0.7430743	0.000627387731058526\\
0.7431743	-0.000509125889065487\\
0.7432743	-0.00122220879984525\\
0.7433743	0.00204612734567\\
0.7434743	0.00163987092877704\\
0.7435744	0.0161423026948651\\
0.7436744	0.0128838188132825\\
0.7437744	0.00665770770852969\\
0.7438744	-0.000208042099828276\\
0.7439744	0.00469655794258284\\
0.7440744	0.00161249642503059\\
0.7441744	-0.00759923802820683\\
0.7442744	-0.0144496567577752\\
0.7443744	-0.0086987710691222\\
0.7444744	-0.00479359223241982\\
0.7445745	-0.00870891588835576\\
0.7446745	-0.00365019215934748\\
0.7447745	0.00196879110868853\\
0.7448745	0.00442902278768464\\
0.7449745	0.000123491785334551\\
0.7450745	-0.00522181295483204\\
0.7451745	-0.00420990245334174\\
0.7452745	-0.000680787694691709\\
0.7453745	-0.00312247962725287\\
0.7454745	-0.00198298916317796\\
0.7455746	0.00638581056188414\\
0.7456746	0.00835762538334794\\
0.7457746	0.00672259007240772\\
0.7458746	0.00926269386272516\\
0.7459746	0.00400792602463529\\
0.7460746	0.00373027586524088\\
0.7461746	-0.00280126727150964\\
0.7462746	0.00081328599528252\\
0.7463746	0.00523892508355317\\
0.7464746	0.00344763944835336\\
0.7465747	0.00142847738600338\\
0.7466747	-0.00222869713296969\\
0.7467747	-0.00482782779703594\\
0.7468747	-0.00855192500171942\\
0.7469747	-0.00259599910498931\\
0.7470747	0.00492093957285089\\
0.7471747	0.00480688074923474\\
0.7472747	-0.00337518582058038\\
0.7473747	-0.0074852703434134\\
0.7474747	-0.00816138298809133\\
0.7475748	-0.00817755506276896\\
0.7476748	-0.00686276235855487\\
0.7477748	-0.00509702806382162\\
0.7478748	-0.0047773621944911\\
0.7479748	-0.00444777472805047\\
0.7480748	0.00996872439651586\\
0.7481748	0.00793612527883347\\
0.7482748	-0.00159758194279558\\
0.7483748	-0.00336740709130368\\
0.7484748	0.0037636400492298\\
0.7485749	0.00726144756730118\\
0.7486749	0.000174201937484852\\
0.7487749	-0.00679120053125892\\
0.7488749	0.00213623053326373\\
0.7489749	0.00442448554249886\\
0.7490749	-0.000218445052769134\\
0.7491749	0.00712442923768286\\
0.7492749	0.00769009894357886\\
0.7493749	0.0108855546342257\\
0.7494749	0.00874978691859951\\
0.749575	0.00218060232134665\\
0.749675	0.000288351532460318\\
0.749775	-5.81506062200532e-05\\
0.749875	0.00431208667379135\\
0.749975	-0.00306594581895769\\
0.750075	-0.00971525723578104\\
0.750175	-0.00683385668776371\\
0.750275	-0.00334375324570235\\
0.750375	-0.00744295594001709\\
0.750475	-0.00979047376066333\\
0.7505751	-0.0144515826681486\\
0.7506751	-0.0111437658865832\\
0.7507751	-0.00543029096660419\\
0.7508751	0.00151683326444108\\
0.7509751	0.00106359802004707\\
0.7510751	0.00349899455460267\\
0.7511751	0.0141130141634971\\
0.7512751	0.0116196481831592\\
0.7513751	0.00760188799117056\\
0.7514751	0.00756472500630156\\
0.7515752	0.0170767999015791\\
0.7516752	-0.00166620267159345\\
0.7517752	-0.00665963354178434\\
0.7518752	-0.00473950112461807\\
0.7519752	-0.00744981379414078\\
0.7520752	-0.00626257988277779\\
0.7521752	-0.00630380768122496\\
0.7522752	-0.00558450543840028\\
0.7523752	0.00210631863863586\\
0.7524752	-0.00214434361525218\\
0.7525753	0.00187306426590972\\
0.7526753	0.00524539651626021\\
0.7527753	0.00118421818450809\\
0.7528753	0.00667552127396842\\
0.7529753	0.011998297830214\\
0.7530753	0.0115505399411662\\
0.7531753	0.00963423973715649\\
0.7532753	0.00585938939098263\\
0.7533753	-0.000475018881999745\\
0.7534753	-0.00451199282384174\\
0.7535754	-0.0102630609747401\\
0.7536754	-0.00477819789447942\\
0.7537754	-0.00520592346046822\\
0.7538754	-0.00816424524359327\\
0.7539754	-0.00581217077181151\\
0.7540754	-0.00687670753007641\\
0.7541754	-0.00867886296026121\\
0.7542754	-0.00314164446112031\\
0.7543754	-0.00893705938820499\\
0.7544754	0.000835884946200238\\
0.7545755	-0.00202642576129541\\
0.7546755	0.00220720667373175\\
0.7547755	0.00726317668640153\\
0.7548755	0.00799447713813706\\
0.7549755	0.00712810093392768\\
0.7550755	0.0105360410223989\\
0.7551755	0.00760729039587282\\
0.7552755	-0.00142915790958131\\
0.7553755	0.00026068918592248\\
0.7554755	-0.00271517519385611\\
0.7555756	0.0029015481703123\\
0.7556756	-1.47683414288835e-05\\
0.7557756	-0.000956816700057317\\
0.7558756	0.000921396394318208\\
0.7559756	0.00762586428575407\\
0.7560756	-0.000971419637470383\\
0.7561756	0.013085538057215\\
0.7562756	0.00469773084670422\\
0.7563756	-0.0042528477477024\\
0.7564756	-0.00965720416026628\\
0.7565757	-0.0215061263199621\\
0.7566757	-0.0164800662872207\\
0.7567757	-0.0108548031144098\\
0.7568757	-0.00296334305738277\\
0.7569757	0.00429430767272976\\
0.7570757	0.00651014290957885\\
0.7571757	0.00683615653164615\\
0.7572757	0.00759134246230264\\
0.7573757	-0.00367530533014992\\
0.7574757	0.000369207167588037\\
0.7575758	0.000213004251475191\\
0.7576758	0.000641810695207745\\
0.7577758	0.00363775973363989\\
0.7578758	-0.00212015443940317\\
0.7579758	-0.00220793758487758\\
0.7580758	-0.002478595418415\\
0.7581758	-0.00193113361032271\\
0.7582758	0.00131844221450272\\
0.7583758	-0.00080987352341122\\
0.7584758	0.00348791364196988\\
0.7585759	0.00650383964922696\\
0.7586759	0.00703580729074638\\
0.7587759	0.00796886150169485\\
0.7588759	0.00122399693058739\\
0.7589759	-7.91728335336472e-07\\
0.7590759	-0.0042445097350603\\
0.7591759	0.0020488376962442\\
0.7592759	4.02453972832006e-05\\
0.7593759	-0.00300829175434014\\
0.7594759	-0.00980577883507028\\
0.759576	-0.0112662688010072\\
0.759676	-0.00836367974764851\\
0.759776	-0.0104750555814697\\
0.759876	-0.00543740119461322\\
0.759976	0.0015632785669526\\
0.760076	0.0101439789034581\\
0.760176	0.00338469506140004\\
0.760276	0.00336742233356802\\
0.760376	0.00800715605908486\\
0.760476	0.0106428916234442\\
0.7605761	0.0080894866840282\\
0.7606761	0.00326220325879056\\
0.7607761	-0.00622009189689443\\
0.7608761	-0.00807540321157596\\
0.7609761	-0.0113957350672349\\
0.7610761	-0.00708009179923064\\
0.7611761	0.00272552230370504\\
0.7612761	6.31029995563637e-05\\
0.7613761	0.00102264609300651\\
0.7614761	0.00584314743550873\\
0.7615762	0.00245737485199182\\
0.7616762	0.00126277138224662\\
0.7617762	-0.00327788600715984\\
0.7618762	-0.00262460127733721\\
0.7619762	0.00580362165754966\\
0.7620762	-0.00168822106970534\\
0.7621762	-0.00074613327929729\\
0.7622762	0.000733881255616353\\
0.7623762	-0.000158181191055817\\
0.7624762	-4.43242982957059e-05\\
0.7625763	0.00897112952674828\\
0.7626763	0.015350805160316\\
0.7627763	0.0117013893757703\\
0.7628763	0.00600787868290009\\
0.7629763	7.22696387445154e-05\\
0.7630763	-0.0020694411523543\\
0.7631763	-0.0117782570387526\\
0.7632763	-0.0136531813214464\\
0.7633763	-0.00066621725406435\\
0.7634763	-0.000117368042822363\\
0.7635764	-0.00568604667992112\\
0.7636764	-0.00610244573253294\\
0.7637764	-0.00709996897828358\\
0.7638764	-0.0159916194334517\\
0.7639764	-0.0171334000667738\\
0.7640764	-0.00458231379942386\\
0.7641764	0.00416163649501478\\
0.7642764	0.00454344799057971\\
0.7643764	-0.00108588209105864\\
0.7644764	0.0140346435194552\\
0.7645765	0.0221515209388006\\
0.7646765	0.0126747407814758\\
0.7647765	0.00929680840798497\\
0.7648765	0.0120647212786116\\
0.7649765	0.0101054769014324\\
0.7650765	0.0113740728323037\\
0.7651765	-0.00245349332507505\\
0.7652765	-0.000572223919171222\\
0.7653765	0.00304787874941947\\
0.7654765	-0.00236018757201606\\
0.7655766	-0.00849201791671206\\
0.7656766	-0.0135264379591202\\
0.7657766	-0.0104600334642795\\
0.7658766	-0.00624380649311451\\
0.7659766	-0.0122477590585843\\
0.7660766	-0.0158288931256472\\
0.7661766	0.00032178938876104\\
0.7662766	0.00557128661572115\\
0.7663766	0.000185596734371352\\
0.7664766	0.00586871797190813\\
0.7665767	0.00813896393378144\\
0.7666767	0.00384469308991342\\
0.7667767	0.0095442283334517\\
0.7668767	-0.00511143191595664\\
0.7669767	-0.00582828919052325\\
0.7670767	-0.00609734497431624\\
0.7671767	0.000444399296759102\\
0.7672767	-0.00238905776502385\\
0.7673767	-0.00267071749920422\\
0.7674767	-0.00609258119712308\\
0.7675768	-0.00412742677843436\\
0.7676768	0.000246288708001252\\
0.7677768	0.00668079664485166\\
0.7678768	0.00861809593371987\\
0.7679768	0.0079241855244776\\
0.7680768	0.00323506441523858\\
0.7681768	0.00297673165239321\\
0.7682768	0.0049221863306137\\
0.7683768	-0.00362457240715308\\
0.7684768	0.00320145463034738\\
0.7685769	0.00180539788236178\\
0.7686769	0.00483198502128612\\
0.7687769	0.000196355798399495\\
0.7688769	-0.00665149040177315\\
0.7689769	-0.010113554146379\\
0.7690769	-0.00491183595422484\\
0.7691769	-0.0056793362957972\\
0.7692769	-0.0023920555932424\\
0.7693769	0.00292500577963395\\
0.7694769	0.00136384749738028\\
0.769577	-0.00865049170999654\\
0.769677	0.00272790069422868\\
0.769777	0.00390807275688143\\
0.769877	-4.49756540579216e-05\\
0.769977	-0.00125624462223328\\
0.770077	-0.00736373418294889\\
0.770177	-0.00020944432315062\\
0.770277	-0.00130537498142402\\
0.770377	-0.00124752604800538\\
0.770477	-0.00431189736476956\\
0.7705771	-0.00604654193116172\\
0.7706771	-0.00466436230017564\\
0.7707771	-0.00398640215458101\\
0.7708771	0.00277333885718178\\
0.7709771	0.00229386113503072\\
0.7710771	0.00743616512721168\\
0.7711771	0.00939525133030195\\
0.7712771	0.00338512028922366\\
0.7713771	0.0103547725972106\\
0.7714771	0.00795820889582455\\
0.7715772	0.00146628448389752\\
0.7716772	-0.00232271833248454\\
0.7717772	-0.00248093494287716\\
0.7718772	-0.00187236451244832\\
0.7719772	-0.00182500615807069\\
0.7720772	0.000196141051657328\\
0.7721772	0.000135078096407781\\
0.7722772	0.00995980600412594\\
0.7723772	0.00591032585100315\\
0.7724772	-2.13612385149275e-05\\
0.7725773	-0.000745491591689351\\
0.7726773	-0.0141105981916594\\
0.7727773	-0.0120889080640296\\
0.7728773	-0.0124794198914371\\
0.7729773	-0.00495013230831365\\
0.7730773	-0.00385204390091243\\
0.7731773	-0.00506515320732159\\
0.7732773	-0.00247045871747353\\
0.7733773	0.000570041126842113\\
0.7734773	-0.00158465206803449\\
0.7735774	-0.0047978661319581\\
0.7736774	-0.000519949582570511\\
0.7737774	0.00903277903704147\\
0.7738774	0.0174463215256448\\
0.7739774	0.0134926797300512\\
0.7740774	0.0180188555451082\\
0.7741774	0.014905850913685\\
0.7742774	0.0116006678266674\\
0.7743774	0.00195330832291418\\
0.7744774	-0.00622122551074167\\
0.7745775	-0.00488635283549854\\
0.7746775	-0.00217923804556985\\
0.7747775	-0.00526929103514995\\
0.7748775	-0.00569850952566497\\
0.7749775	-0.00504289119067547\\
0.7750775	-0.00806043365586628\\
0.7751775	-0.0156941344991113\\
0.7752775	-0.00794699125046083\\
0.7753775	0.00132099860782287\\
0.7754775	0.0124668376412487\\
0.7755776	0.0142850155760952\\
0.7756776	-0.00284444829692276\\
0.7757776	-0.0059310550091638\\
0.7758776	-0.00140980180422545\\
0.7759776	-0.00105468587805735\\
0.7760776	-2.97043789778684e-05\\
0.7761776	-0.00322785440770357\\
0.7762776	-0.00742313301737063\\
0.7763776	0.00411846278643502\\
0.7764776	0.00367893604564973\\
0.7765777	0.00441068564027525\\
0.7766777	0.0122449140113714\\
0.7767777	0.0162860292517939\\
0.7768777	0.00747703459336524\\
0.7769777	-0.00124706668470065\\
0.7770777	-0.00262127125583334\\
0.7771777	-0.00502857574612925\\
0.7772777	-0.0133829767343832\\
0.7773777	-0.0123384707521144\\
0.7774777	-0.00796705428360589\\
0.7775778	0.00172358101827896\\
0.7776778	0.0010358201147298\\
0.7777778	-0.0048510194685889\\
0.7778778	-0.00325593402725133\\
0.7779778	0.00626708019025024\\
0.7780778	0.017807026982485\\
0.7781778	0.0144739101950329\\
0.7782778	0.014286733720458\\
0.7783778	0.0030115014982881\\
0.7784778	-0.00104078248504358\\
0.7785779	-0.00896390005496173\\
0.7786779	-0.00765528445539321\\
0.7787779	-0.00626070837237158\\
0.7788779	-0.00674116763208676\\
0.7789779	-0.00366965801399832\\
0.7790779	-0.0057761752508565\\
0.7791779	-0.0120137150287496\\
0.7792779	-0.00880027298714037\\
0.7793779	-0.000654844718889436\\
0.7794779	-0.00106642577032101\\
0.779578	-0.00734788773249928\\
0.779678	0.00126051712579667\\
0.779778	0.00731092630901298\\
0.779878	0.00439434445669029\\
0.779978	0.00106477625473378\\
0.780078	0.00564522643536058\\
0.780178	0.00264169977705997\\
0.780278	-0.00132479889544523\\
0.780378	-0.00366126471124595\\
0.780478	0.00316230724671396\\
0.7805781	0.00156595607485155\\
0.7806781	-0.00128539043385079\\
0.7807781	-0.00121168413608963\\
0.7808781	-0.00119191993060408\\
0.7809781	0.0067189073297973\\
0.7810781	0.00338580283819123\\
0.7811781	-0.00337222816651472\\
0.7812781	0.00116481960038683\\
0.7813781	0.00744395146936316\\
0.7814781	0.00445917281656616\\
0.7815782	0.00129343392487868\\
0.7816782	-0.00271115835729141\\
0.7817782	-0.00348464475347576\\
0.7818782	0.00166498029486384\\
0.7819782	0.00206372239174524\\
0.7820782	-0.00699141281341564\\
0.7821782	-0.00737741962582827\\
0.7822782	-0.00867229230537259\\
0.7823782	-0.0139510250666781\\
0.7824782	-0.0189436120791502\\
0.7825783	-0.00899119132903881\\
0.7826783	-0.00315147801090809\\
0.7827783	0.00583139882464856\\
0.7828783	-0.000408554811374806\\
0.7829783	2.26671369984999e-05\\
0.7830783	0.00332707077063255\\
0.7831783	0.00966266223527068\\
0.7832783	0.0175284477214461\\
0.7833783	0.0125434334644456\\
0.7834783	0.00526062574425357\\
0.7835784	0.00340479889479184\\
0.7836784	0.00266641448910132\\
0.7837784	-0.000956744265759646\\
0.7838784	-0.00109467091154425\\
0.7839784	-0.000768358945554048\\
0.7840784	0.00299219817925334\\
0.7841784	0.00339900705423002\\
0.7842784	0.000897074314991642\\
0.7843784	0.000154406641335925\\
0.7844784	-0.00335598924280323\\
0.7845785	-0.0086404260495474\\
0.7846785	0.00208673328220055\\
0.7847785	0.00361418484638776\\
0.7848785	-0.00521106445705399\\
0.7849785	0.00280799231566409\\
0.7850785	-0.00821663784790161\\
0.7851785	-0.00571694791646582\\
0.7852785	0.0040400691849154\\
0.7853785	0.00799142057478353\\
0.7854785	0.000218113415228594\\
0.7855786	0.00214174862571781\\
0.7856786	-0.00243686261305065\\
0.7857786	-0.00497711064707573\\
0.7858786	0.00234801185930777\\
0.7859786	-0.00314848771501897\\
0.7860786	0.000615398051991178\\
0.7861786	-0.0020723233745148\\
0.7862786	0.00516135551364183\\
0.7863786	0.00799644226761931\\
0.7864786	0.00609094448145964\\
0.7865787	0.00129837742720085\\
0.7866787	-0.00191927505112499\\
0.7867787	-0.00884548902290039\\
0.7868787	-0.00364725672306321\\
0.7869787	-0.000558570343978104\\
0.7870787	0.00762957796449015\\
0.7871787	0.00404819609492701\\
0.7872787	-0.00192870801771594\\
0.7873787	0.0101718736038219\\
0.7874787	0.0116659489790329\\
0.7875788	0.00338794849612079\\
0.7876788	0.00314002711973416\\
0.7877788	0.00037362381638939\\
0.7878788	0.00404674677386474\\
0.7879788	0.00592940422183474\\
0.7880788	0.00520960443178708\\
0.7881788	-0.000577644283046297\\
0.7882788	-0.00460033356775291\\
0.7883788	-0.0109264550258192\\
0.7884788	-0.00943500021920229\\
0.7885789	-0.010486622837842\\
0.7886789	-0.0137789984244243\\
0.7887789	-0.00789877217514334\\
0.7888789	-0.00761993548652207\\
0.7889789	0.000940520286091484\\
0.7890789	-0.00128839617155929\\
0.7891789	-0.00149567613271451\\
0.7892789	0.00521868917032364\\
0.7893789	0.0162187085461323\\
0.7894789	0.00640539084407354\\
0.789579	-0.0003360008569393\\
0.789679	0.00150002568137975\\
0.789779	0.00357874194291519\\
0.789879	-0.00923584306051151\\
0.789979	-0.00397872027667834\\
0.790079	-0.000378880613016719\\
0.790179	0.0091446850632888\\
0.790279	0.0117619859252347\\
0.790379	0.0138480311859173\\
0.790479	0.00416083009846435\\
0.7905791	0.00153856339825431\\
0.7906791	0.00561388931052598\\
0.7907791	-0.016846003117247\\
0.7908791	-0.0135641044726231\\
0.7909791	0.00202859469648065\\
0.7910791	0.00291710388168642\\
0.7911791	-0.0104705673859264\\
0.7912791	-0.019248409535888\\
0.7913791	-0.00841141295843872\\
0.7914791	-0.0071125680045947\\
0.7915792	-0.00202077535501388\\
0.7916792	-0.00399121267229763\\
0.7917792	-0.00450977242196604\\
0.7918792	0.000202555201050281\\
0.7919792	0.00804678004062787\\
0.7920792	0.00309691197936068\\
0.7921792	-0.0014700390615161\\
0.7922792	0.0073679368777757\\
0.7923792	0.0050918497955017\\
0.7924792	-0.00646629027171208\\
0.7925793	-0.00581446445400147\\
0.7926793	-0.00668068825578505\\
0.7927793	-0.00235593469297513\\
0.7928793	0.00462180642378573\\
0.7929793	0.00461654532180322\\
0.7930793	0.00138529226625916\\
0.7931793	0.00779305756010729\\
0.7932793	0.0144468515440019\\
0.7933793	0.00639968459619844\\
0.7934793	0.00482956713247784\\
0.7935794	0.00368043857801048\\
0.7936794	0.00436544355510282\\
0.7937794	-0.00196047050162118\\
0.7938794	-0.00487829302715515\\
0.7939794	-0.0135790134194274\\
0.7940794	-0.0195376210393725\\
0.7941794	-0.0228631052110257\\
0.7942794	-0.0109094552216265\\
0.7943794	-0.00400566032169536\\
0.7944794	-0.00267370972512992\\
0.7945795	0.0127312575907874\\
0.7946795	0.00752354426779789\\
0.7947795	0.00749601922950394\\
0.7948795	0.0106376934076273\\
0.7949795	0.00997757777005681\\
0.7950795	0.00213468332076161\\
0.7951795	0.00271702109968253\\
0.7952795	0.00291360218265702\\
0.7953795	-0.00124456231868653\\
0.7954795	-0.000853461257016797\\
0.7955796	-0.00826231093332908\\
0.7956796	-0.0079876528677012\\
0.7957796	-0.00561369557252167\\
0.7958796	-0.00471142775855604\\
0.7959796	0.00319516189866587\\
0.7960796	0.00681108475873948\\
0.7961796	0.00664135221630513\\
0.7962796	0.000808975700951287\\
0.7963796	-0.00091803332289242\\
0.7964796	0.00504933664396034\\
0.7965797	0.00876979309115722\\
0.7966797	0.00176794808338254\\
0.7967797	0.00294451678302998\\
0.7968797	0.0114325108273627\\
0.7969797	0.0115789418879171\\
0.7970797	0.00147882167038915\\
0.7971797	0.00283316191455008\\
0.7972797	-0.00138402560584527\\
0.7973797	-0.000873729083177288\\
0.7974797	0.00257206332395743\\
0.7975798	0.00316698404490356\\
0.7976798	-0.00372520359458495\\
0.7977798	-0.0117808596117719\\
0.7978798	-0.000223972031110797\\
0.7979798	-0.00146352884377343\\
0.7980798	-0.00977051800775008\\
0.7981798	-0.00376492744794898\\
0.7982798	0.00027225494369354\\
0.7983798	-0.00714895869186538\\
0.7984798	-0.00539255618091339\\
0.7985799	-0.00446397897161166\\
0.7986799	0.000477685130868544\\
0.7987799	0.00706900210896855\\
0.7988799	0.0132049842665296\\
0.7989799	0.00632264393967397\\
0.7990799	0.00112599349667875\\
0.7991799	0.00255804533788062\\
0.7992799	0.000457811895592464\\
0.7993799	0.00662830563396799\\
0.7994799	0.000606539048918719\\
0.79958	0.00814899803029334\\
0.79968	0.0131357411836843\\
0.79978	0.0124372617034654\\
0.79988	0.00476357221156398\\
0.79998	-0.000664314638876284\\
0.80008	-0.00453438616358071\\
};
\addplot [color=mycolor1,solid,forget plot]
  table[row sep=crcr]{%
0.80008	-0.00453438616358071\\
0.80018	-0.00261562964724327\\
0.80028	-0.00609703234365355\\
0.80038	-0.00550758147578101\\
0.80048	-0.0123762642358862\\
0.8005801	-0.00343066614884702\\
0.8006801	-0.0129125847208815\\
0.8007801	-0.00537759830158713\\
0.8008801	-0.00523069396144127\\
0.8009801	-0.0015758587407273\\
0.8010801	-0.00419307964967963\\
0.8011801	0.00085565633145368\\
0.8012801	0.00127436225225264\\
0.8013801	0.00359705119206666\\
0.8014801	0.000385736259866043\\
0.8015802	0.0106497617940331\\
0.8016802	0.00781547159180018\\
0.8017802	-0.00573378296554772\\
0.8018802	-0.0117409886515176\\
0.8019802	-0.00651213221053304\\
0.8020802	-0.00367520035802116\\
0.8021802	-0.00172217978053535\\
0.8022802	4.69428641443059e-05\\
0.8023802	0.00132418094690134\\
0.8024802	-0.000954452132834405\\
0.8025803	0.0106233191341945\\
0.8026803	0.00123697720389959\\
0.8027803	-0.00245919550687557\\
0.8028803	-0.00160718548557259\\
0.8029803	-0.00486597919161191\\
0.8030803	-0.000817563056539683\\
0.8031803	-0.00540492348412425\\
0.8032803	-0.00184004685045469\\
0.8033803	-0.00142191950407611\\
0.8034803	0.00693747223392283\\
0.8035804	0.00105033637989038\\
0.8036804	-0.00011470865421348\\
0.8037804	-0.00598544808265267\\
0.8038804	-0.00509086811782655\\
0.8039804	0.00211804505474333\\
0.8040804	0.000973305276317227\\
0.8041804	-0.00776507358519574\\
0.8042804	-0.00753507763533534\\
0.8043804	-0.0112596929532005\\
0.8044804	-0.000999905591565486\\
0.8045805	-0.00196857366329528\\
0.8046805	0.0060050544413599\\
0.8047805	0.00232212723799563\\
0.8048805	0.00256865877800649\\
0.8049805	0.0078086631385359\\
0.8050805	-0.00286884557761802\\
0.8051805	-0.0132438532421508\\
0.8052805	-0.0116313457013382\\
0.8053805	-0.00794830877616626\\
0.8054805	-0.00658372826242637\\
0.8055806	1.14729879214792e-05\\
0.8056806	0.0138291769707557\\
0.8057806	0.0148534673189848\\
0.8058806	0.010322358336408\\
0.8059806	0.0105198643514264\\
0.8060806	0.00601599971694\\
0.8061806	-0.00255822118977722\\
0.8062806	-0.000113783967204739\\
0.8063806	0.000575325810335592\\
0.8064806	0.000950122592534797\\
0.8065807	0.00984862020368649\\
0.8066807	0.00462682816305815\\
0.8067807	0.00489676663371164\\
0.8068807	-0.00832454983969999\\
0.8069807	0.00225589331093357\\
0.8070807	0.00290711067704752\\
0.8071807	-0.00271988312672189\\
0.8072807	-0.00777507346264272\\
0.8073807	-0.010857445670001\\
0.8074807	-0.00379498506523657\\
0.8075808	-0.0100987397083143\\
0.8076808	-0.0115105754681242\\
0.8077808	-0.00854153421437776\\
0.8078808	0.00535839882677003\\
0.8079808	-0.000416761548573952\\
0.8080808	-0.00507300052214266\\
0.8081808	-0.00140930325364153\\
0.8082808	0.0138643451191349\\
0.8083808	0.0177559594801862\\
0.8084808	0.0209505547351956\\
0.8085809	0.0199900224023732\\
0.8086809	0.0174736182667088\\
0.8087809	0.0114372398858897\\
0.8088809	0.00925290225114495\\
0.8089809	0.00600162037476437\\
0.8090809	0.00194840929002016\\
0.8091809	-0.00485271594898418\\
0.8092809	-0.00644774026741324\\
0.8093809	-0.0121426485698377\\
0.8094809	-0.0121804257403326\\
0.809581	-0.00958723919825477\\
0.809681	-0.00791871450677828\\
0.809781	-6.01319871229578e-06\\
0.809881	0.00994287992261889\\
0.809981	0.00807098007378038\\
0.810081	-0.000893697508894675\\
0.810181	0.00396786243057168\\
0.810281	0.000687675167673873\\
0.810381	-0.0017222440026794\\
0.810481	-0.00331987976631165\\
0.8105811	-0.00750745697525002\\
0.8106811	-0.00833548558455721\\
0.8107811	-0.0038091847147097\\
0.8108811	0.00657046102442571\\
0.8109811	0.00218546704165679\\
0.8110811	0.00388384876434422\\
0.8111811	0.00483262163830123\\
0.8112811	0.000168801127639284\\
0.8113811	0.0087994027146765\\
0.8114811	0.00877144189979528\\
0.8115812	-0.00305236207739146\\
0.8116812	0.00158059335317873\\
0.8117812	0.00268503300577727\\
0.8118812	-0.00583602754811352\\
0.8119812	-0.00548957271954542\\
0.8120812	-0.00537058690222914\\
0.8121812	0.000551945527346831\\
0.8122812	-0.00276395979019828\\
0.8123812	-0.00266728719728171\\
0.8124812	0.00167097898054447\\
0.8125813	-0.000250496382901311\\
0.8126813	-0.0101150013122533\\
0.8127813	0.00211113448228861\\
0.8128813	0.00933892674139347\\
0.8129813	0.00951139122197109\\
0.8130813	0.00758054369705263\\
0.8131813	0.00119539995567308\\
0.8132813	-0.00943702419725412\\
0.8133813	-0.0136307129410609\\
0.8134813	-0.00745165043944862\\
0.8135814	-0.0123882246247797\\
0.8136814	-0.011232617270589\\
0.8137814	-0.0149272110536094\\
0.8138814	-0.000367990076220792\\
0.8139814	-0.00121693842577564\\
0.8140814	-0.00325004017473685\\
0.8141814	-0.00281227938078832\\
0.8142814	-0.00257464008695951\\
0.8143814	-0.00197110632175779\\
0.8144814	-0.00497666209927683\\
0.8145815	0.000473253417381692\\
0.8146815	-0.00480143848037606\\
0.8147815	-0.00115617186191486\\
0.8148815	0.00604306931508436\\
0.8149815	0.0108613011067211\\
0.8150815	0.00124553958276774\\
0.8151815	0.00271980082653922\\
0.8152815	0.00297810093476869\\
0.8153815	0.00565945601749718\\
0.8154815	0.00670588219793355\\
0.8155816	-0.00630510932769557\\
0.8156816	0.000866502581350126\\
0.8157816	-0.00391876594826712\\
0.8158816	-0.00841489874185264\\
0.8159816	-0.0044448796121479\\
0.8160816	-0.00109869235946203\\
0.8161816	-0.000588320771789164\\
0.8162816	-0.00510474862493204\\
0.8163816	-0.00244995968262884\\
0.8164816	-0.00124393769666664\\
0.8165817	0.000499780491976409\\
0.8166817	-0.00558668736924961\\
0.8167817	-0.00166487335548204\\
0.8168817	-0.00405976117195284\\
0.8169817	-0.00674233451257766\\
0.8170817	-0.00646857706006365\\
0.8171817	-0.00462047248603437\\
0.8172817	-0.00391600445116236\\
0.8173817	-0.00532815660527675\\
0.8174817	0.00234908741249562\\
0.8175818	0.0101271443396298\\
0.8176818	0.0140442252631718\\
0.8177818	0.00779775152066353\\
0.8178818	0.0100257395146146\\
0.8179818	0.0109572056576148\\
0.8180818	0.0102311663722352\\
0.8181818	0.00592963809089397\\
0.8182818	0.00735863725572461\\
0.8183818	-0.000202819681521582\\
0.8184818	-0.00583871625961496\\
0.8185819	0.00415831946397738\\
0.8186819	0.000810588626720232\\
0.8187819	-0.00117053239557574\\
0.8188819	0.00101797289501437\\
0.8189819	0.00103012100527877\\
0.8190819	-0.000961071549253845\\
0.8191819	-0.005450588244444\\
0.8192819	-0.00950241254764084\\
0.8193819	-0.0061235279178371\\
0.8194819	0.005352082194233\\
0.819582	0.00732874657254001\\
0.819682	0.0033188530962267\\
0.819782	0.00490073481483133\\
0.819882	0.00864640830939312\\
0.819982	0.00703589016859585\\
0.820082	0.00671719698863082\\
0.820182	0.0113573453730957\\
0.820282	0.000705351932832166\\
0.820382	0.000744233285852935\\
0.820482	0.00219100605717149\\
0.8205821	0.00849295751708645\\
0.8206821	-0.000268441039401601\\
0.8207821	-0.0115818982449376\\
0.8208821	-0.00287039744762208\\
0.8209821	-0.00454292198913642\\
0.8210821	0.000870544795128936\\
0.8211821	0.001078019575945\\
0.8212821	0.00715951903013412\\
0.8213821	0.00219005984043807\\
0.8214821	-0.00402134130458798\\
0.8215822	0.00442356300384805\\
0.8216822	0.00560632413618745\\
0.8217822	0.00875019342913852\\
0.8218822	0.00279118759324071\\
0.8219822	0.00726632334423183\\
0.8220822	0.014586617402923\\
0.8221822	0.00710708649507374\\
0.8222822	0.00296574735127209\\
0.8223822	0.000283616706813206\\
0.8224822	-0.00460528869842314\\
0.8225823	-0.0124037596587936\\
0.8226823	-0.00319716808107007\\
0.8227823	-0.00468130100140574\\
0.8228823	-0.0056061416627804\\
0.8229823	-0.0109746733041982\\
0.8230823	-0.00357287916078342\\
0.8231823	-0.00763974246394028\\
0.8232823	-0.00813324644144675\\
0.8233823	-0.00440237431758206\\
0.8234823	-0.00490310931324878\\
0.8235824	-0.00484327876278279\\
0.8236824	-0.00437818121245748\\
0.8237824	0.00213835959148945\\
0.8238824	0.0088313604404604\\
0.8239824	0.0020698381286417\\
0.8240824	-0.00231419054712401\\
0.8241824	-0.00767770878745733\\
0.8242824	0.0018483002094376\\
0.8243824	0.0113808532476494\\
0.8244824	0.00526696713345221\\
0.8245825	-0.00496122034055979\\
0.8246825	-0.00286593772977728\\
0.8247825	0.00347895617609151\\
0.8248825	-0.00197452180917246\\
0.8249825	0.00105064512979958\\
0.8250825	-0.00238452619016538\\
0.8251825	-0.000879018950890043\\
0.8252825	-0.00115981633297935\\
0.8253825	-0.00491090151593276\\
0.8254825	-0.00055625767826406\\
0.8255826	-0.0053537802310295\\
0.8256826	-0.0137986311132128\\
0.8257826	-0.0192107024887544\\
0.8258826	-0.017572977533403\\
0.8259826	-0.0122084394225084\\
0.8260826	-0.00794707133115935\\
0.8261826	-0.00738485643426756\\
0.8262826	-0.00135777790671332\\
0.8263826	0.00980918107654771\\
0.8264826	0.00825703734037164\\
0.8265827	-0.00243813605921428\\
0.8266827	-0.00573743782172942\\
0.8267827	-0.00491579181464186\\
0.8268827	0.000587818784952177\\
0.8269827	0.000631410799171839\\
0.8270827	-0.00175099895075181\\
0.8271827	0.000582606355388204\\
0.8272827	-0.00037975646331434\\
0.8273827	0.000716929410968792\\
0.8274827	0.000786680794719885\\
0.8275828	0.00222154088195659\\
0.8276828	0.00353347083557853\\
0.8277828	0.00971051675556778\\
0.8278828	0.00431969545178212\\
0.8279828	0.000994023732149518\\
0.8280828	0.00273951840252451\\
0.8281828	0.000963196266599908\\
0.8282828	0.0013700741257785\\
0.8283828	0.00377316877905987\\
0.8284828	-8.05029770782895e-05\\
0.8285829	-0.0083279261407383\\
0.8286829	-0.0103330830613371\\
0.8287829	-0.00672695600138828\\
0.8288829	-0.00857752817562307\\
0.8289829	-0.00418278280188744\\
0.8290829	0.00495529689878582\\
0.8291829	0.00480872770204011\\
0.8292829	0.00862752638007835\\
0.8293829	0.00345070970156813\\
0.8294829	0.00287329443151529\\
0.829583	0.0119362690474301\\
0.829683	0.000588704317394217\\
0.829783	0.00496759128446689\\
0.829883	0.000753946697889418\\
0.829983	0.0058037873026513\\
0.830083	0.000471129839414886\\
0.830183	-0.00384300895560896\\
0.830283	-0.00110961235077056\\
0.830383	0.00142933638090037\\
0.830483	0.00956285396158973\\
0.8305831	0.00742790400888493\\
0.8306831	0.00615860704510274\\
0.8307831	0.00127092908375692\\
0.8308831	0.00187088682676961\\
0.8309831	0.00837149697070494\\
0.8310831	0.00814977620667712\\
0.8311831	0.00342174122024108\\
0.8312831	0.0106814086912563\\
0.8313831	0.0094787952938189\\
0.8314831	0.0154139176961055\\
0.8315832	0.0132807163158337\\
0.8316832	0.0074923580754418\\
0.8317832	0.00359378561948676\\
0.8318832	-0.00413698440854171\\
0.8319832	-0.00173193537160321\\
0.8320832	-0.00284605063921539\\
0.8321832	-0.00659431358756368\\
0.8322832	-0.00545970759960124\\
0.8323832	-0.00777021606517536\\
0.8324832	-0.00434282238111372\\
0.8325833	-0.00598360767505701\\
0.8326833	-0.000641361967105514\\
0.8327833	-0.00325416432647829\\
0.8328833	-0.00756999817906623\\
0.8329833	-0.0051778469582926\\
0.8330833	0.00498230589477711\\
0.8331833	0.0134254769313324\\
0.8332833	0.012470682694711\\
0.8333833	0.0131249397202766\\
0.8334833	0.0155652645353479\\
0.8335834	0.00864455611488468\\
0.8336834	-0.00147793583285516\\
0.8337834	-0.000515310442155581\\
0.8338834	-0.00403255121906021\\
0.8339834	0.00769735832180496\\
0.8340834	-0.00130356534291654\\
0.8341834	-0.000993305745372375\\
0.8342834	-0.00158484642663304\\
0.8343834	-0.000202170936789425\\
0.8344834	-0.0026542628350561\\
0.8345835	-0.0122292414036034\\
0.8346835	-0.00984082051907897\\
0.8347835	-0.00346311773969887\\
0.8348835	-0.00266611666223326\\
0.8349835	-0.00490380089309905\\
0.8350835	-0.00725715404846428\\
0.8351835	-0.00822615975435159\\
0.8352835	-0.0121728016467362\\
0.8353835	-0.00894806337165921\\
0.8354835	-0.0043089285853215\\
0.8355836	-0.00067753319547359\\
0.8356836	0.00246144204096321\\
0.8357836	0.00103586277436429\\
0.8358836	-0.00506525469315511\\
0.8359836	-0.0026868940701581\\
0.8360836	-0.00423003907597508\\
0.8361836	-0.00562467344082274\\
0.8362836	-0.00647078090588682\\
0.8363836	-0.0111453452234427\\
0.8364836	-0.00484335015694004\\
0.8365837	0.0021340533820764\\
0.8366837	0.00422821448106925\\
0.8367837	0.00055998362188836\\
0.8368837	0.00731537699535156\\
0.8369837	0.00797041078059824\\
0.8370837	-0.0123508988550172\\
0.8371837	-0.00630453575600898\\
0.8372837	-0.00621848377888323\\
0.8373837	-0.00658772679221231\\
0.8374837	-0.00768824867675244\\
0.8375838	-0.0118042137368017\\
0.8376838	-0.00469124629380779\\
0.8377838	-0.00454450942222628\\
0.8378838	-0.0046989870525036\\
0.8379838	-0.00190366312774559\\
0.8380838	0.000136478396171924\\
0.8381838	-0.00461354644947187\\
0.8382838	-0.00216672164636478\\
0.8383838	-0.00558203118924716\\
0.8384838	0.000144540913998625\\
0.8385839	-0.00322718143444112\\
0.8386839	0.00288720080590954\\
0.8387839	-0.00346748739517277\\
0.8388839	-0.00339023009913442\\
0.8389839	-0.00190901138105426\\
0.8390839	-0.002789815329729\\
0.8391839	-0.00206162604777438\\
0.8392839	-0.00713942765171005\\
0.8393839	0.000625795727932754\\
0.8394839	0.00677305994652988\\
0.839584	0.000613178699306438\\
0.839684	-6.34288145135997e-05\\
0.839784	0.00244505200701672\\
0.839884	0.00625363696192327\\
0.839984	0.00933434183367554\\
0.840084	0.00367818239107587\\
0.840184	0.00179817438818741\\
0.840284	0.000705333564227306\\
0.840384	0.00800867564347296\\
0.840484	0.0020842163351853\\
0.8405841	0.000369760700500876\\
0.8406841	0.0109597449221823\\
0.8407841	0.0100409748087005\\
0.8408841	0.00590146600828234\\
0.8409841	0.00449923415368003\\
0.8410841	0.00435729486207492\\
0.8411841	0.00611866373499081\\
0.8412841	0.00544335635820259\\
0.8413841	0.00239238830164545\\
0.8414841	0.00267277511933184\\
0.8415842	0.0049513147965552\\
0.8416842	0.00221145735429484\\
0.8417842	-0.00658799863264914\\
0.8418842	-0.00370303767487812\\
0.8419842	-0.00385664429936261\\
0.8420842	-0.00623680304953123\\
0.8421842	-0.00329549848536281\\
0.8422842	-0.000364715183444719\\
0.8423842	0.00509856226289562\\
0.8424842	0.0082993492435306\\
0.8425843	0.0103074382106829\\
0.8426843	0.0138892899098289\\
0.8427843	0.0095496972265452\\
0.8428843	0.0128016754826072\\
0.8429843	0.0144922399825615\\
0.8430843	0.017903406013636\\
0.8431843	0.0203411888456557\\
0.8432843	0.0151126037309561\\
0.8433843	0.0118206659043074\\
0.8434843	0.00494439058281637\\
0.8435844	-0.0102204337880081\\
0.8436844	-0.00840433881842229\\
0.8437844	-0.00509153578396804\\
0.8438844	-0.00682100953901266\\
0.8439844	-0.00120574495599435\\
0.8440844	-0.00829672692550742\\
0.8441844	-0.00113794035638204\\
0.8442844	0.00236062982422958\\
0.8443844	0.000446998670788457\\
0.8444844	0.00669118121926182\\
0.8445845	0.00485896341760794\\
0.8446845	0.00969581825470822\\
0.8447845	0.00348153180489907\\
0.8448845	-0.00812288097058511\\
0.8449845	-0.00791740512942285\\
0.8450845	-0.00406902574823942\\
0.8451845	-0.00189672792270468\\
0.8452845	-0.00198049676762047\\
0.8453845	-0.00182831741697642\\
0.8454845	-0.000151175024052819\\
0.8455846	0.00389171535264365\\
0.8456846	0.00628182829286983\\
0.8457846	-0.00170205128613787\\
0.8458846	-0.00507790861557877\\
0.8459846	-0.00383672894631282\\
0.8460846	-0.0126824975489574\\
0.8461846	-0.00661219971394633\\
0.8462846	-0.00633982075161921\\
0.8463846	-0.0019303459922777\\
0.8464846	-0.00211476078629014\\
0.8465847	-0.00264927972637619\\
0.8466847	0.00307257038823128\\
0.8467847	-0.0015514252080826\\
0.8468847	0.00115074805333171\\
0.8469847	0.00397310472071943\\
0.8470847	-0.00568734067819543\\
0.8471847	-0.00592357363625728\\
0.8472847	-0.00652457966696129\\
0.8473847	-0.00982734430452314\\
0.8474847	-0.00750185310395501\\
0.8475848	-0.0165963187395968\\
0.8476848	-0.0165692723194786\\
0.8477848	-0.00647592683738907\\
0.8478848	-0.00518626793227028\\
0.8479848	-0.00460128126421733\\
0.8480848	-0.00811195251455521\\
0.8481848	-0.0030272673859062\\
0.8482848	-0.00111721160226708\\
0.8483848	-0.010676770909078\\
0.8484848	-0.0130499310732719\\
0.8485849	-0.0101379014185703\\
0.8486849	-0.00305722025013777\\
0.8487849	-0.00272009735554413\\
0.8488849	-0.00291851858850123\\
0.8489849	-0.00482246982457069\\
0.8490849	-0.00261893696124993\\
0.8491849	-0.00575390591803071\\
0.8492849	-0.0115883626364677\\
0.8493849	-0.00587029308024745\\
0.8494849	0.00733031676474782\\
0.849585	0.0118062623365724\\
0.849685	0.00547599528756981\\
0.849785	-0.00186768955023942\\
0.849885	-0.00175877825221837\\
0.849985	0.0038827430837296\\
0.850085	0.00411488833710427\\
0.850185	0.0015656713647445\\
0.850285	0.002635106000763\\
0.850385	-0.0002287939435206\\
0.850485	-0.00323601467965418\\
0.8505851	-0.0113267546184219\\
0.8506851	-0.0128905749510144\\
0.8507851	-0.00321867483414051\\
0.8508851	-0.00306004057143339\\
0.8509851	-0.00245765848970692\\
0.8510851	-0.00202251493899297\\
0.8511851	-0.00286259629264185\\
0.8512851	-0.00300388894736558\\
0.8513851	-0.000336379323297997\\
0.8514851	0.000397946135938909\\
0.8515852	0.00441589653761065\\
0.8516852	0.00158689509169818\\
0.8517852	0.000225750022031512\\
0.8518852	0.00907147479040438\\
0.8519852	0.0159970828348257\\
0.8520852	0.0153365875694451\\
0.8521852	0.0168490023845118\\
0.8522852	0.0177443406462992\\
0.8523852	0.014028615697053\\
0.8524852	0.00881784085492665\\
0.8525853	0.00682283408912232\\
0.8526853	0.0023410003023967\\
0.8527853	0.00635715644539703\\
0.8528853	0.00656031573933236\\
0.8529853	0.00913349138103214\\
0.8530853	0.0038476965428802\\
0.8531853	-0.00350605562720441\\
0.8532853	0.00717824799412847\\
0.8533853	0.00434962050563348\\
0.8534853	-0.000131925018588941\\
0.8535854	-7.56042726512973e-06\\
0.8536854	0.00214709821498493\\
0.8537854	-0.0010781220917182\\
0.8538854	0.00694379162751568\\
0.8539854	0.00651185232263973\\
0.8540854	0.00449107291862472\\
0.8541854	0.00647946631539531\\
0.8542854	0.00602404538778956\\
0.8543854	0.00183282298549603\\
0.8544854	0.00450281193300128\\
0.8545855	0.0129588518595916\\
0.8546855	0.0136003031224092\\
0.8547855	0.0157280040695111\\
0.8548855	0.0106319674240365\\
0.8549855	0.00728620588367379\\
0.8550855	0.00481473212058958\\
0.8551855	-0.0013934412185912\\
0.8552855	-0.0036433015128714\\
0.8553855	0.000178163833094214\\
0.8554855	-0.00211903261106761\\
0.8555856	-0.0018540384669401\\
0.8556856	-0.00424651951602772\\
0.8557856	-0.00648762478862563\\
0.8558856	0.000192658181534799\\
0.8559856	0.00720134183476295\\
0.8560856	0.00861643858537198\\
0.8561856	0.0142809608216181\\
0.8562856	0.0112429209056653\\
0.8563856	0.00789433117352356\\
0.8564856	0.00463220393501951\\
0.8565857	0.00141940556348885\\
0.8566857	-0.00212175837012163\\
0.8567857	-0.00771742302603018\\
0.8568857	-0.010946576199677\\
0.8569857	-0.0155732057129194\\
0.8570857	-0.0110732994140825\\
0.8571857	-0.0106388451780091\\
0.8572857	-0.0120338309060832\\
0.8573857	-0.0120172445263065\\
0.8574857	-0.0122670739933146\\
0.8575858	-0.00719843771883966\\
0.8576858	-0.00635006123597881\\
0.8577858	-0.0095640646121633\\
0.8578858	-0.00209743590907197\\
0.8579858	-0.000647163215236279\\
0.8580858	0.0018217653539061\\
0.8581858	-4.46383440175779e-05\\
0.8582858	0.0016316375216272\\
0.8583858	-0.00722139524554199\\
0.8584858	-0.00630972486897541\\
0.8585859	-0.00294445335174931\\
0.8586859	-0.00806733973406537\\
0.8587859	-0.00501048779411972\\
0.8588859	-0.00663188586405861\\
0.8589859	-0.0137725223032921\\
0.8590859	-0.00958238549853663\\
0.8591859	-0.00505546386384648\\
0.8592859	-0.00147474584067098\\
0.8593859	-0.00145921989785558\\
0.8594859	-0.00328187453172779\\
0.859586	-0.00402179416964055\\
0.859686	-0.00727277370778268\\
0.859786	-0.00740789946528331\\
0.859886	-0.0128021600486758\\
0.859986	-0.0128905440921576\\
0.860086	-0.0118600402575977\\
0.860186	-0.0142526372345793\\
0.860286	-0.016570323740436\\
0.860386	-0.0141300885202855\\
0.860486	-0.0165199203470623\\
0.8605861	-0.00872788493245613\\
0.8606861	0.00183018467755103\\
0.8607861	0.00131922076568713\\
0.8608861	0.00185323444736563\\
0.8609861	0.00224323681001204\\
0.8610861	-0.000281761086976751\\
0.8611861	-0.000992748212244265\\
0.8612861	0.00065228643745785\\
0.8613861	-0.000425646162721005\\
0.8614861	0.000980464934206005\\
0.8615862	0.00862457384472703\\
0.8616862	0.00589180713501003\\
0.8617862	0.00221711680498293\\
0.8618862	0.0003935136886839\\
0.8619862	-0.00331799140814448\\
0.8620862	-0.00217138770810932\\
0.8621862	0.00163033553781888\\
0.8622862	0.00600318905027697\\
0.8623862	0.00553318352147203\\
0.8624862	0.0029083296151593\\
0.8625863	0.000996602360716192\\
0.8626863	-0.00573191424513908\\
0.8627863	-0.00344124739768015\\
0.8628863	0.000130613452720585\\
0.8629863	-0.00165232117290781\\
0.8630863	8.29592178517302e-05\\
0.8631863	0.00364146508875907\\
0.8632863	0.00769020687490363\\
0.8633863	0.0115821949826675\\
0.8634863	0.017859439789722\\
0.8635864	0.0175219382949323\\
0.8636864	0.0143377298010261\\
0.8637864	0.00838480897704574\\
0.8638864	0.00650518608545791\\
0.8639864	0.00188187135988649\\
0.8640864	0.00113787500507012\\
0.8641864	0.00655020719684751\\
0.8642864	0.0106078780821339\\
0.8643864	0.00499489777890005\\
0.8644864	0.00187627637614041\\
0.8645865	0.00725403387043411\\
0.8646865	0.00605416280362769\\
0.8647865	0.00672668074022509\\
0.8648865	0.00607359765309257\\
0.8649865	0.0155289234860092\\
0.8650865	0.0158356681536511\\
0.8651865	0.0184308415415724\\
0.8652865	0.0148274535061823\\
0.8653865	0.00458751387472534\\
0.8654865	0.00565503244525489\\
0.8655866	0.0105060532116523\\
0.8656866	0.00888051994608441\\
0.8657866	0.00972247411107263\\
0.8658866	0.0116919253877117\\
0.8659866	0.00442588342780814\\
0.8660866	0.00384035785388592\\
0.8661866	0.00967235825913404\\
0.8662866	0.00475589420740796\\
0.8663866	0.00247497523321671\\
0.8664866	0.00243361084169225\\
0.8665867	-0.00411813000528882\\
0.8666867	-0.0072253542554197\\
0.8667867	-0.000714995574475186\\
0.8668867	-0.0023500445749951\\
0.8669867	-0.0095674918989781\\
0.8670867	-0.00968632821789539\\
0.8671867	0.00146745576730822\\
0.8672867	0.00576986932618295\\
0.8673867	0.00504692169876542\\
0.8674867	0.00635762209556079\\
0.8675868	0.00277906538813877\\
0.8676868	0.0030590920176512\\
0.8677868	0.00373779413471098\\
0.8678868	0.0106441808315494\\
0.8679868	0.00868126117080026\\
0.8680868	0.00518704418548532\\
0.8681868	-0.00126746112097864\\
0.8682868	-0.00187824577482454\\
0.8683868	0.000127699168070095\\
0.8684868	-0.00747861737782118\\
0.8685869	-0.0126500737188167\\
0.8686869	-0.0142328838568759\\
0.8687869	-0.0126379289095544\\
0.8688869	-0.00565320008114378\\
0.8689869	-0.00885068860565363\\
0.8690869	-0.013180385746812\\
0.8691869	-0.0177722827980755\\
0.8692869	-0.0158143710826426\\
0.8693869	-0.00649764195346147\\
0.8694869	-0.00767808679323245\\
0.869587	-0.0111605562030683\\
0.869687	-0.0110723204004852\\
0.869787	-0.00773523288508393\\
0.869887	-0.00502528515871514\\
0.869987	-0.00715946875302143\\
0.870087	-0.00760777522945943\\
0.870187	-0.00718919617927982\\
0.870287	-0.00838872322356821\\
0.870387	-0.00899334801323137\\
0.870487	-0.0148020622290042\\
0.8705871	-0.0131746879131582\\
0.8706871	-0.0116565532114699\\
0.8707871	-0.011077483148926\\
0.8708871	-0.0115044695256752\\
0.8709871	-0.00531750417172239\\
0.8710871	-0.00217057894693307\\
0.8711871	-0.00224968574103723\\
0.8712871	-0.00300781647364001\\
0.8713871	-0.00216296309421793\\
0.8714871	-0.00120311758212077\\
0.8715872	0.00154692740326956\\
0.8716872	-0.00378121586481495\\
0.8717872	-0.010511343109664\\
0.8718872	-0.0109574464302011\\
0.8719872	-0.016361517955239\\
0.8720872	-0.0124505498434786\\
0.8721872	-0.0139985342835249\\
0.8722872	-0.0125394634938623\\
0.8723872	-0.0096883297228838\\
0.8724872	-0.00433112524887758\\
0.8725873	-0.00200561255385567\\
0.8726873	0.0072027594616384\\
0.8727873	0.00547822517353769\\
0.8728873	0.000107792183933864\\
0.8729873	0.000676468065021307\\
0.8730873	0.0040842603591007\\
0.8731873	0.0050331765785776\\
0.8732873	0.000985224205964919\\
0.8733873	-0.00256458930611769\\
0.8734873	-0.00436225653493993\\
0.8735874	-0.00265250902034131\\
0.8736874	-0.00391085836929417\\
0.8737874	-0.00130203933880058\\
0.8738874	-0.000180044625697207\\
0.8739874	0.00775313304329828\\
0.8740874	0.00523750091159725\\
0.8741874	0.00910106619274131\\
0.8742874	0.0114728360703991\\
0.8743874	0.011300817698392\\
0.8744874	0.0138950182006679\\
0.8745875	0.0136387377147638\\
0.8746875	0.0164444004546223\\
0.8747875	0.019354303268543\\
0.8748875	0.0172894531610763\\
0.8749875	0.0138068571069307\\
0.8750875	0.00561052205098367\\
0.8751875	0.00216745490830128\\
0.8752875	0.0036206625641324\\
0.8753875	0.00296515187392177\\
0.8754875	0.0084119296633085\\
0.8755876	0.00552532845275344\\
0.8756876	0.00719170686451909\\
0.8757876	0.00505239406080449\\
0.8758876	0.00306939674809392\\
0.8759876	0.00519872160311285\\
0.8760876	0.00906637527282972\\
0.8761876	0.00642836437447114\\
0.8762876	0.0055536954955222\\
0.8763876	0.010889375193752\\
0.8764876	0.0142604099972012\\
0.8765877	0.0141271654852602\\
0.8766877	0.0104149333357586\\
0.8767877	0.0142010757038249\\
0.8768877	0.0135165989986983\\
0.8769877	0.0124655095999591\\
0.8770877	0.0157188138575219\\
0.8771877	0.018647518091661\\
0.8772877	0.0118166285930146\\
0.8773877	0.0106671516225982\\
0.8774877	0.0136430934118141\\
0.8775878	0.0103008532454771\\
0.8776878	0.00803565456400701\\
0.8777878	0.00692189316496883\\
0.8778878	0.00486657516141431\\
0.8779878	-0.0007232933631518\\
0.8780878	-0.00156370635475889\\
0.8781878	-0.00893365778896293\\
0.8782878	-0.00969414167082361\\
0.8783878	-0.00681115203490368\\
0.8784878	1.73170547703716e-05\\
0.8785879	-0.00105530079447247\\
0.8786879	-0.000896851613096938\\
0.8787879	0.001642094659007\\
0.8788879	0.00604154383999145\\
0.8789879	0.00152150171861609\\
0.8790879	0.00231597405423134\\
0.8791879	-0.000788033423179568\\
0.8792879	-0.00784351501299407\\
0.8793879	-0.00809646504393324\\
0.8794879	-0.00645887787404642\\
0.879588	-0.00737528498566609\\
0.879688	-0.000246603054316452\\
0.879788	0.00165963283345771\\
0.879888	-0.00332057179756373\\
0.879988	-0.00363521145184098\\
0.880088	-0.0112522806630593\\
0.880188	-0.00782177399410866\\
0.880288	-0.00751068603707195\\
0.880388	-0.0100620114131948\\
0.880488	-0.0118797447728873\\
0.8805881	-0.00658638212945495\\
0.8806881	-0.0108809119187161\\
0.8807881	-0.0139878338122135\\
0.8808881	-0.0150251425767942\\
0.8809881	-0.0104258330083613\\
0.8810881	-0.00881689993184365\\
0.8811881	-0.00675033820119508\\
0.8812881	-0.00599814269936227\\
0.8813881	-0.00488830833826404\\
0.8814881	-0.0125968300587735\\
0.8815882	-0.0181911678755011\\
0.8816882	-0.0164733830411141\\
0.8817882	-0.0123539392795085\\
0.8818882	-0.0132398316471891\\
0.8819882	-0.0180760552295072\\
0.8820882	-0.0166586051406397\\
0.8821882	-0.0152304765235631\\
0.8822882	-0.0169856645500377\\
0.8823882	-0.01227916442058\\
0.8824882	-0.0118729713644377\\
0.8825883	-0.0135995088965912\\
0.8826883	-0.008632912084924\\
0.8827883	-0.00637660820182839\\
0.8828883	-0.00290259259128955\\
0.8829883	-0.00222286062591071\\
0.8830883	-0.00391340770688764\\
0.8831883	-0.00311022926397864\\
0.8832883	-0.00495332075549371\\
0.8833883	-0.00849067766826095\\
0.8834883	-0.00861229551760223\\
0.8835884	-0.00670856084653138\\
0.8836884	-0.0024326834786903\\
0.8837884	-0.00427505375972359\\
0.8838884	-0.00286866731870949\\
0.8839884	0.000707480186908915\\
0.8840884	-0.00131460692865581\\
0.8841884	0.000192075620494281\\
0.8842884	0.00475053209196896\\
0.8843884	0.0033027667151162\\
0.8844884	-0.00179521630895096\\
0.8845885	0.00170823389271235\\
0.8846885	0.00243483185748294\\
0.8847885	-0.00391477538666196\\
0.8848885	-0.000514583751304531\\
0.8849885	0.00406841082386888\\
0.8850885	0.0037532123711097\\
0.8851885	-0.00492317510538109\\
0.8852885	-0.000192747629404799\\
0.8853885	0.00523749874724122\\
0.8854885	0.000588567944808771\\
0.8855886	0.00102814866809586\\
0.8856886	0.00171787898907683\\
0.8857886	0.00409944372810966\\
0.8858886	0.0120978466938795\\
0.8859886	0.0150620916672655\\
0.8860886	0.0174931824013658\\
0.8861886	0.0217471226215316\\
0.8862886	0.0243729160253907\\
0.8863886	0.0206785662828834\\
0.8864886	0.0231490770362921\\
0.8865887	0.0212631752103174\\
0.8866887	0.0151684216414034\\
0.8867887	0.0128125393331509\\
0.8868887	0.00881653181748687\\
0.8869887	0.00511740259883697\\
0.8870887	0.00267115515415606\\
0.8871887	0.00254979293295388\\
0.8872887	0.00649931935734172\\
0.8873887	0.00730873782204755\\
0.8874887	0.0105130516944605\\
0.8875888	0.0150710264797072\\
0.8876888	0.0114951450641905\\
0.8877888	0.00928616899807588\\
0.8878888	0.0123251015396848\\
0.8879888	0.0209299459201473\\
0.8880888	0.0188527053434533\\
0.8881888	0.0165303829864805\\
0.8882888	0.0108609819990264\\
0.8883888	0.0067465055038467\\
0.8884888	0.00565195659668438\\
0.8885889	0.00729313969636669\\
0.8886889	0.00661345907981059\\
0.8887889	0.010955715179787\\
0.8888889	0.00747691098431136\\
0.8889889	0.00515804945456488\\
0.8890889	0.00731813352492683\\
0.8891889	0.00861516610300495\\
0.8892889	0.00765815006967574\\
0.8893889	0.00557808827912809\\
0.8894889	0.00434698355887631\\
0.889589	0.00303567954781192\\
0.889689	0.00211250130827234\\
0.889789	0.0023792884647118\\
0.889889	0.00628504373834571\\
0.889989	0.00229976982389712\\
0.890089	0.00207246938965255\\
0.890189	0.000201145077488363\\
0.890289	-0.00233420049709165\\
0.890389	-0.00470256474491282\\
0.890489	-0.00556794510309699\\
0.8905891	-0.00533545843291394\\
0.8906891	-0.00290885943800623\\
0.8907891	-0.00102226901976188\\
0.8908891	-0.00106768472013874\\
0.8909891	-0.00537110410720176\\
0.8910891	-0.00905552477508409\\
0.8911891	-0.0134799443439564\\
0.8912891	-0.0113503604599745\\
0.8913891	-0.0133417707952588\\
0.8914891	-0.0175031730478288\\
0.8915892	-0.0166076443255507\\
0.8916892	-0.014799019596534\\
0.8917892	-0.0154853800317889\\
0.8918892	-0.013883723432649\\
0.8919892	-0.00591504762615402\\
0.8920892	-0.0095583504650405\\
0.8921892	-0.0165906298276739\\
0.8922892	-0.0181558836180258\\
0.8923892	-0.00981810976562891\\
0.8924892	-0.00590630622553734\\
0.8925893	-0.00850151012449918\\
0.8926893	-0.0106466371413645\\
0.8927893	-0.0100397284864582\\
0.8928893	-0.0100847822165409\\
0.8929893	-0.0118757964136857\\
0.8930893	-0.0113237691852497\\
0.8931893	-0.010707698663821\\
0.8932893	-0.0104715830071871\\
0.8933893	-0.0163874203982901\\
0.8934893	-0.0160992090451768\\
0.8935894	-0.016574945891241\\
0.8936894	-0.0161656277209818\\
0.8937894	-0.0156172555792184\\
0.8938894	-0.0167328277740622\\
0.8939894	-0.0130783426385281\\
0.8940894	-0.0114617985304852\\
0.8941894	-0.0152831938326312\\
0.8942894	-0.0181735269524335\\
0.8943894	-0.0125067963220966\\
0.8944894	-0.00668700039851333\\
0.8945895	-0.0100810957645545\\
0.8946895	-0.0100861606546687\\
0.8947895	-0.00751315576848\\
0.8948895	-0.00738407966124188\\
0.8949895	-0.00915393091268313\\
0.8950895	-0.00763470812696565\\
0.8951895	-0.00286140993263673\\
0.8952895	0.00231496501740569\\
0.8953895	0.00253141804596901\\
0.8954895	0.00142195045159577\\
0.8955896	0.00123964616446948\\
0.8956896	0.00278434520558402\\
0.8957896	0.00304312737551893\\
0.8958896	0.00555599387620753\\
0.8959896	0.00710994588554681\\
0.8960896	0.00477998455745166\\
0.8961896	0.00614711102188004\\
0.8962896	0.00486432638489132\\
0.8963896	0.00518063172868932\\
0.8964896	0.00676402811166299\\
0.8965897	0.00838864010531504\\
0.8966897	0.00326522574037469\\
0.8967897	0.00331990544833927\\
0.8968897	0.00345068019278028\\
0.8969897	0.0030045509136849\\
0.8970897	0.00112351852750003\\
0.8971897	0.00468358392718082\\
0.8972897	0.0123787479822322\\
0.8973897	0.0138730115387589\\
0.8974897	0.0123903754195089\\
0.8975898	0.0144280049417989\\
0.8976898	0.0167325759483312\\
0.8977898	0.0169772496084035\\
0.8978898	0.0202660266517798\\
0.8979898	0.0194029077851039\\
0.8980898	0.0186478936919354\\
0.8981898	0.0206809850328165\\
0.8982898	0.0147891824452948\\
0.8983898	0.0124974865439844\\
0.8984898	0.00850389792061135\\
0.8985899	0.0082256227194426\\
0.8986899	0.00950225444445851\\
0.8987899	0.0100809950862355\\
0.8988899	0.0143898451453921\\
0.8989899	0.0159798050998936\\
0.8990899	0.0203138754051011\\
0.8991899	0.0190190564938171\\
0.8992899	0.0185783487763347\\
0.8993899	0.0199457526404885\\
0.8994899	0.0232912684516953\\
0.89959	0.0203911432394195\\
0.89969	0.0150158880643394\\
0.89979	0.0120967457988588\\
0.89989	0.0090567167191975\\
0.89999	0.0079508010794096\\
0.90009	0.00833599911142269\\
0.90019	0.00646331102508557\\
0.90029	0.00651873700823158\\
0.90039	0.00542527722670516\\
0.90049	0.00481593182442541\\
0.9005901	0.000708988751875861\\
0.9006901	-0.00116112343298291\\
0.9007901	-0.00178712093785061\\
0.9008901	-0.00210400370605912\\
0.9009901	0.00136222829737886\\
0.9010901	-0.00012242491418224\\
0.9011901	0.00384203665102389\\
0.9012901	0.0050476129632322\\
0.9013901	0.00518230397118713\\
0.9014901	0.00518710960220203\\
0.9015902	0.00529535874164044\\
0.9016902	0.00447739742955422\\
0.9017902	7.25503941880756e-05\\
0.9018902	-0.00380118252241712\\
0.9019902	-0.0072148014994069\\
0.9020902	-0.00586730673706581\\
0.9021902	-0.00460369845676678\\
0.9022902	-0.00712897690091197\\
0.9023902	-0.0072601423329004\\
0.9024902	-0.00960219503706194\\
0.9025903	-0.0081427652007342\\
0.9026903	-0.0111475892738049\\
0.9027903	-0.0133893015975182\\
0.9028903	-0.0135339025394614\\
0.9029903	-0.0153863924879163\\
0.9030903	-0.0141617718517917\\
0.9031903	-0.0131860410605842\\
0.9032903	-0.015094200564316\\
0.9033903	-0.0207362508334941\\
0.9034903	-0.0209941923590555\\
0.9035904	-0.0190236144296141\\
0.9036904	-0.0215563359147843\\
0.9037904	-0.0242779502517763\\
0.9038904	-0.0222684580127552\\
0.9039904	-0.0158628597900601\\
0.9040904	-0.011644156196154\\
0.9041904	-0.0134283478635715\\
0.9042904	-0.0152594354448687\\
0.9043904	-0.0108534196125654\\
0.9044904	-0.010491301059103\\
0.9045905	-0.00751762822341306\\
0.9046905	-0.0107503022834652\\
0.9047905	-0.0135608758194345\\
0.9048905	-0.0155733496029276\\
0.9049905	-0.0188937244252223\\
0.9050905	-0.0190060010971973\\
0.9051905	-0.0176161804492883\\
0.9052905	-0.0150672633314335\\
0.9053905	-0.0133572506130235\\
0.9054905	-0.00910414318284758\\
0.9055906	-0.00963644869916891\\
0.9056906	-0.0121961504967618\\
0.9057906	-0.0100637603657896\\
0.9058906	-0.0104082792721383\\
0.9059906	-0.0117337082008322\\
0.9060906	-0.0121820481559816\\
0.9061906	-0.0121203001607313\\
0.9062906	-0.0100994652572093\\
0.9063906	-0.0078705445064724\\
0.9064906	-0.00821053898844926\\
0.9065907	-0.00978391566920722\\
0.9066907	-0.002700739849675\\
0.9067907	-0.00192348261653885\\
0.9068907	-0.000983145124736851\\
0.9069907	0.00185427145218831\\
0.9070907	0.000557765922125197\\
0.9071907	0.00108633707445605\\
0.9072907	0.00560398368010451\\
0.9073907	0.00311170449159581\\
0.9074907	0.00553549824310062\\
0.9075908	0.00344593855333053\\
0.9076908	0.00559387838393777\\
0.9077908	0.008829587246173\\
0.9078908	0.00976436380131629\\
0.9079908	0.00870870669256418\\
0.9080908	0.00830091454507942\\
0.9081908	0.00700098596604493\\
0.9082908	0.0106323195447172\\
0.9083908	0.0134222138524756\\
0.9084908	0.013101767442889\\
0.9085909	0.0153571943936042\\
0.9086909	0.0194026661945806\\
0.9087909	0.0199123928309964\\
0.9088909	0.0161679727856459\\
0.9089909	0.0151607045237666\\
0.9090909	0.0145424864931029\\
0.9091909	0.0168583171239427\\
0.9092909	0.0151246948291833\\
0.9093909	0.0169169180043802\\
0.9094909	0.0156314850278\\
0.909591	0.0159475502922822\\
0.909691	0.0147113041177859\\
0.909791	0.0153219968214405\\
0.909891	0.0161396267128796\\
0.909991	0.0147647920847155\\
0.910091	0.0136246912125932\\
0.910191	0.0146945223552455\\
0.910291	0.0110927837545428\\
0.910391	0.0147249736355499\\
0.910491	0.0145042902065768\\
0.9105911	0.0153646280145394\\
0.9106911	0.0155610965459413\\
0.9107911	0.0210268862904679\\
0.9108911	0.0226896953899371\\
0.9109911	0.0213596219696806\\
0.9110911	0.023130164138606\\
0.9111911	0.0215727199892345\\
0.9112911	0.0209468875977669\\
0.9113911	0.0176356650241317\\
0.9114911	0.0168604503120428\\
0.9115912	0.01040157798451\\
0.9116912	0.00610657706495321\\
0.9117912	0.00339577803930424\\
0.9118912	0.00327367888692909\\
0.9119912	0.00374127757124687\\
0.9120912	0.000116572039780549\\
0.9121912	-0.00160673977578707\\
0.9122912	-0.000110059959560038\\
0.9123912	-0.000235490611375178\\
0.9124912	0.00204826615324805\\
0.9125913	-3.4415360862794e-05\\
0.9126913	0.000164413809696784\\
0.9127913	0.00291152395443328\\
0.9128913	0.00366391289563373\\
0.9129913	-0.00271372155982852\\
0.9130913	-0.00419358162043648\\
0.9131913	-0.0051624695099709\\
0.9132913	-0.00350148746746926\\
0.9133913	-0.00366953774716239\\
0.9134913	-0.00891342261842298\\
0.9135914	-0.00979982820468384\\
0.9136914	-0.0100213851678294\\
0.9137914	-0.0117178836232984\\
0.9138914	-0.0128041259005105\\
0.9139914	-0.0149735143437572\\
0.9140914	-0.0140587513121504\\
0.9141914	-0.0131246391795681\\
0.9142914	-0.00830728033460115\\
0.9143914	-0.00804567718049731\\
0.9144914	-0.00973063213511105\\
0.9145915	-0.0112031919754587\\
0.9146915	-0.0162863665234351\\
0.9147915	-0.0184170065232494\\
0.9148915	-0.0197158144506526\\
0.9149915	-0.0206305927957312\\
0.9150915	-0.0224973440628541\\
0.9151915	-0.0262676707706239\\
0.9152915	-0.0270904754518093\\
0.9153915	-0.0225717606533086\\
0.9154915	-0.0151802289360858\\
0.9155916	-0.016160687970794\\
0.9156916	-0.0163981262445755\\
0.9157916	-0.0199454553690113\\
0.9158916	-0.0174743779608186\\
0.9159916	-0.0158029966505138\\
0.9160916	-0.0200558140823601\\
0.9161916	-0.0200472329143152\\
0.9162916	-0.0222178558179761\\
0.9163916	-0.0181231854785253\\
0.9164916	-0.0148285245946794\\
0.9165917	-0.0161525419849691\\
0.9166917	-0.0213020042786108\\
0.9167917	-0.0174460842074174\\
0.9168917	-0.0157567845237175\\
0.9169917	-0.0130158079931054\\
0.9170917	-0.0110997573943877\\
0.9171917	-0.00914493551953421\\
0.9172917	-0.0106243451736132\\
0.9173917	-0.0124752891747516\\
0.9174917	-0.00826697035407697\\
0.9175918	-0.00688071894589548\\
0.9176918	-0.00367547917101607\\
0.9177918	-0.00326488514804546\\
0.9178918	-0.00432743975958427\\
0.9179918	-0.00360084590097387\\
0.9180918	-0.0021103064802348\\
0.9181918	-0.000143224418014731\\
0.9182918	0.000207297352462721\\
0.9183918	-0.0013196441145451\\
0.9184918	-0.00117215177725349\\
0.9185919	-0.000408417566782782\\
0.9186919	0.00224423728121426\\
0.9187919	-0.00141188301841155\\
0.9188919	-0.00107698147323187\\
0.9189919	0.00256023889697787\\
0.9190919	0.00368187506029205\\
0.9191919	0.00530272397268494\\
0.9192919	0.00701258257808324\\
0.9193919	0.00944714780842615\\
0.9194919	0.010715516583707\\
0.919592	0.0128284349823548\\
0.919692	0.0138649053557904\\
0.919792	0.0132756699599093\\
0.919892	0.0140896256674629\\
0.919992	0.017893769339518\\
0.920092	0.0171943978255202\\
0.920192	0.0193011079633415\\
0.920292	0.0214383965793326\\
0.920392	0.0221365604883783\\
0.920492	0.0231421964939453\\
0.9205921	0.0239768883782872\\
0.9206921	0.0222236627059561\\
0.9207921	0.024085099469343\\
0.9208921	0.0231441954267655\\
0.9209921	0.0235285473253859\\
0.9210921	0.0181661519012637\\
0.9211921	0.0148391058794074\\
0.9212921	0.0129251059738296\\
0.9213921	0.0125513488875917\\
0.9214921	0.0128796313128684\\
0.9215922	0.0166604744184576\\
0.9216922	0.0175676296311439\\
0.9217922	0.0170080143636577\\
0.9218922	0.0180663252650984\\
0.9219922	0.0180905589739329\\
0.9220922	0.0201583121180403\\
0.9221922	0.0201933813147688\\
0.9222922	0.0201230631709863\\
0.9223922	0.0208719542831316\\
0.9224922	0.0170071512372653\\
0.9225923	0.0139635122604079\\
0.9226923	0.0153460143127097\\
0.9227923	0.0143917118999815\\
0.9228923	0.0137053015672796\\
0.9229923	0.0123424798495357\\
0.9230923	0.0125979432716152\\
0.9231923	0.00899778834836329\\
0.9232923	0.010756811584658\\
0.9233923	0.00840800947546061\\
0.9234923	0.00615317850587072\\
0.9235924	0.00249861362227777\\
0.9236924	0.00366881801029117\\
0.9237924	0.00369678293096987\\
0.9238924	0.00831120483044723\\
0.9239924	0.00314358014524752\\
0.9240924	0.00254230530233646\\
0.9241924	-0.00265962328083325\\
0.9242924	-0.00390950919626554\\
0.9243924	-0.0057306560453747\\
0.9244924	-0.00606846743893263\\
0.9245925	-0.00508871206005401\\
0.9246925	-0.000701459785616865\\
0.9247925	-0.00685878294722897\\
0.9248925	-0.00658378519260006\\
0.9249925	-0.0134236701785464\\
0.9250925	-0.014444341570938\\
0.9251925	-0.0167004030446533\\
0.9252925	-0.0162743582835265\\
0.9253925	-0.0186745109802968\\
0.9254925	-0.0209898648365641\\
0.9255926	-0.0179181525233282\\
0.9256926	-0.0191382162489979\\
0.9257926	-0.0173791922953205\\
0.9258926	-0.014964584398829\\
0.9259926	-0.0142416963046629\\
0.9260926	-0.0164932317665156\\
0.9261926	-0.0163976945465943\\
0.9262926	-0.0190787884155624\\
0.9263926	-0.018197317152495\\
0.9264926	-0.0197647845448243\\
0.9265927	-0.0197246876188563\\
0.9266927	-0.0196663401656375\\
0.9267927	-0.020782642783602\\
0.9268927	-0.0235586992931414\\
0.9269927	-0.0223200135227589\\
0.9270927	-0.0179710893090266\\
0.9271927	-0.0189202304965309\\
0.9272927	-0.0198705409378267\\
0.9273927	-0.0235800244933873\\
0.9274927	-0.0218360850315572\\
0.9275928	-0.0224846843098662\\
0.9276928	-0.0206342069362151\\
0.9277928	-0.020791318200837\\
0.9278928	-0.0188326220030793\\
0.9279928	-0.0191182222499177\\
0.9280928	-0.0197384228559085\\
0.9281928	-0.0201323277431336\\
0.9282928	-0.0224481408411642\\
0.9283928	-0.0188803660870043\\
0.9284928	-0.0198407074250451\\
0.9285929	-0.0188910917280017\\
0.9286929	-0.0176770736389106\\
0.9287929	-0.013401783523004\\
0.9288929	-0.0102018253537504\\
0.9289929	-0.00614330311176392\\
0.9290929	-0.00331252078475919\\
0.9291929	-0.00219038236750424\\
0.9292929	0.00114360813822845\\
0.9293929	-0.00169915327629167\\
0.9294929	-0.000283470626705085\\
0.929593	0.000296063707570071\\
0.929693	0.00349912584503487\\
0.929793	0.00341231995465274\\
0.929893	0.0022656419936089\\
0.929993	0.00146818791240877\\
0.930093	-0.000634246345066147\\
0.930193	-0.000677864841518994\\
0.930293	0.00173512835381784\\
0.930393	0.00566852916522098\\
0.930493	0.00625723351052476\\
0.9305931	0.00787908310501972\\
0.9306931	0.00774268563940106\\
0.9307931	0.00671247941890829\\
0.9308931	0.00904206033607058\\
0.9309931	0.0133267242772092\\
0.9310931	0.0171085671224818\\
0.9311931	0.0161669847459305\\
0.9312931	0.0179628730155269\\
0.9313931	0.0168619277932163\\
0.9314931	0.0164003449349657\\
0.9315932	0.019749699845563\\
0.9316932	0.0164400326116224\\
0.9317932	0.0202144152700613\\
0.9318932	0.0210292436533819\\
0.9319932	0.0233493135883359\\
0.9320932	0.0236295208959686\\
0.9321932	0.0231502613916641\\
0.9322932	0.0242945308851958\\
0.9323932	0.0244695251807634\\
0.9324932	0.027106740077043\\
0.9325933	0.0246935842569715\\
0.9326933	0.020166931038901\\
0.9327933	0.0208551857806503\\
0.9328933	0.0197812442592435\\
0.9329933	0.0198966022463989\\
0.9330933	0.0205520555085825\\
0.9331933	0.0206469998070463\\
0.9332933	0.0238529308978767\\
0.9333933	0.0209597445320375\\
0.9334933	0.0231939364554143\\
0.9335934	0.0231762482120388\\
0.9336934	0.0188730871990392\\
0.9337934	0.01989719167861\\
0.9338934	0.0190486573767596\\
0.9339934	0.0188509800146378\\
0.9340934	0.014102555308579\\
0.9341934	0.0126521789701479\\
0.9342934	0.0129522467061799\\
0.9343934	0.00907825421883229\\
0.9344934	0.00926799720561661\\
0.9345935	0.0072832496493635\\
0.9346935	0.00849445388330328\\
0.9347935	0.00694498065218668\\
0.9348935	0.0043441256353976\\
0.9349935	0.00443318450789154\\
0.9350935	0.0035757529402427\\
0.9351935	0.00434022659868905\\
0.9352935	0.00398350114517071\\
0.9353935	0.00244587223737558\\
0.9354935	0.00324563552878199\\
0.9355936	0.00389189701390291\\
0.9356936	0.0015277348288274\\
0.9357936	0.00199115177418585\\
0.9358936	-0.00265295651296482\\
0.9359936	-0.00145989439956778\\
0.9360936	-0.00546616625652313\\
0.9361936	-0.00780007645864511\\
0.9362936	-0.00831392938462061\\
0.9363936	-0.0164019294169699\\
0.9364936	-0.017190480941997\\
0.9365937	-0.0191746463850014\\
0.9366937	-0.0200246109317361\\
0.9367937	-0.0171971401567921\\
0.9368937	-0.0180100384612241\\
0.9369937	-0.0171802102496631\\
0.9370937	-0.0177749599302878\\
0.9371937	-0.0187108919147734\\
0.9372937	-0.0197949106182531\\
0.9373937	-0.0226628204592802\\
0.9374937	-0.0257029258597829\\
0.9375938	-0.0236255581003337\\
0.9376938	-0.0258645648055106\\
0.9377938	-0.0253595803602387\\
0.9378938	-0.0239040091995514\\
0.9379938	-0.0240815557616572\\
0.9380938	-0.0233207244878965\\
0.9381938	-0.0251942198227056\\
0.9382938	-0.0221638462135731\\
0.9383938	-0.0205620081110008\\
0.9384938	-0.0203982099684665\\
0.9385939	-0.0199527523608186\\
0.9386939	-0.0196428444615869\\
0.9387939	-0.0207451899047271\\
0.9388939	-0.0218474931552142\\
0.9389939	-0.0210763586808012\\
0.9390939	-0.0248696909519715\\
0.9391939	-0.0224677944419111\\
0.9392939	-0.0225559736264604\\
0.9393939	-0.0230212329840794\\
0.9394939	-0.0207660769958068\\
0.939594	-0.0213185759729328\\
0.939694	-0.0208119997419867\\
0.939794	-0.0209429216278425\\
0.939894	-0.0196737461214715\\
0.939994	-0.0173840777162299\\
0.940094	-0.0135327209078214\\
0.940194	-0.0123639801942579\\
0.940294	-0.0129630600758232\\
0.940394	-0.0140393650550326\\
0.940494	-0.0126174996365978\\
0.9405941	-0.00637120431314908\\
0.9406941	-0.00785300866301508\\
0.9407941	-0.00830965614666446\\
0.9408941	-0.00307945127720921\\
0.9409941	0.0014264014302342\\
0.9410941	0.00141509745858348\\
0.9411941	0.0046790322888221\\
0.9412941	0.00322410140004248\\
0.9413941	0.000707400269480463\\
0.9414941	0.00170402437255396\\
0.9415942	0.0016441625880087\\
0.9416942	0.00335492649142416\\
0.9417942	0.00455350203965138\\
0.9418942	0.00776598470119999\\
0.9419942	0.0103074699429445\\
0.9420942	0.0112137532301618\\
0.9421942	0.0127499300265662\\
0.9422942	0.0144300957943479\\
0.9423942	0.0177845459942075\\
0.9424942	0.0180157760853934\\
0.9425943	0.0198580038687711\\
0.9426943	0.0222609829832522\\
0.9427943	0.0198594283539006\\
0.9428943	0.0223205354345112\\
0.9429943	0.0197474996776055\\
0.9430943	0.0192837165344669\\
0.9431943	0.0196300814551789\\
0.9432943	0.0200521898886551\\
0.9433943	0.0185721372826783\\
0.9434943	0.0211717190839358\\
0.9435944	0.0212369815645267\\
0.9436944	0.0223573213391188\\
0.9437944	0.0241009818502208\\
0.9438944	0.023833558540482\\
0.9439944	0.0234302468516314\\
0.9440944	0.0234355422245103\\
0.9441944	0.0262096400991071\\
0.9442944	0.0244854359145925\\
0.9443944	0.0257132251093528\\
0.9444944	0.028399503121025\\
0.9445945	0.0255975442407911\\
0.9446945	0.0249888889737753\\
0.9447945	0.0232594088278607\\
0.9448945	0.0235057992380246\\
0.9449945	0.0231050556386653\\
0.9450945	0.0243123734636351\\
0.9451945	0.0229851481462711\\
0.9452945	0.023116575119436\\
0.9453945	0.0223707498155419\\
0.9454945	0.0200616676665906\\
0.9455946	0.0166826305297793\\
0.9456946	0.0147971237169617\\
0.9457946	0.0138857463483767\\
0.9458946	0.0095134938547049\\
0.9459946	0.00914046166637777\\
0.9460946	0.00629614521361466\\
0.9461946	0.0050442399264492\\
0.9462946	0.00476694123476706\\
0.9463946	0.00467604456833542\\
0.9464946	0.00405974535683529\\
0.9465947	0.00245347256983119\\
0.9466947	0.00369155724308534\\
0.9467947	-0.000842074344446275\\
0.9468947	-0.00212412676310247\\
0.9469947	0.000402795416849844\\
0.9470947	0.000288687625252859\\
0.9471947	0.000577845292084489\\
0.9472947	-0.00221533615250868\\
0.9473947	-0.00457476127817708\\
0.9474947	-0.00564723465434097\\
0.9475948	-0.00915770065296551\\
0.9476948	-0.0108237815969743\\
0.9477948	-0.0133363245026075\\
0.9478948	-0.0167861339380443\\
0.9479948	-0.015394714471081\\
0.9480948	-0.0167366706690951\\
0.9481948	-0.0175069070990161\\
0.9482948	-0.0174651283272956\\
0.9483948	-0.0188913389198758\\
0.9484948	-0.0191267434421587\\
0.9485949	-0.0205522600614664\\
0.9486949	-0.0214852635424003\\
0.9487949	-0.0232209746502728\\
0.9488949	-0.0253431979480291\\
0.9489949	-0.0256496379979238\\
0.9490949	-0.0266784993614936\\
0.9491949	-0.027798786599526\\
0.9492949	-0.0285712042720321\\
0.9493949	-0.0270674569382125\\
0.9494949	-0.0263888491564331\\
0.949595	-0.0283235733428193\\
0.949695	-0.029052455787672\\
0.949795	-0.0276453914589053\\
0.949895	-0.0266251849112475\\
0.949995	-0.028975340698441\\
0.950095	-0.0280377633732215\\
0.950195	-0.025517457487284\\
0.950295	-0.0252712275912564\\
0.950395	-0.0260779782346702\\
0.950495	-0.0236831139659328\\
0.9505951	-0.0219106019027459\\
0.9506951	-0.0228673189467453\\
0.9507951	-0.021428934721332\\
0.9508951	-0.0217847537701328\\
0.9509951	-0.0242035806355114\\
0.9510951	-0.023874819858538\\
0.9511951	-0.0202509759789639\\
0.9512951	-0.0181503535351926\\
0.9513951	-0.0175104570642541\\
0.9514951	-0.0163483911017732\\
0.9515952	-0.0125561979188301\\
0.9516952	-0.0111240041162456\\
0.9517952	-0.0128503544248462\\
0.9518952	-0.011608053374379\\
0.9519952	-0.0099018054930497\\
0.9520952	-0.00738551530750431\\
0.9521952	-0.00523048734279596\\
0.9522952	-0.00437052612236211\\
0.9523952	-0.00274053616799491\\
0.9524952	-0.0033866219998181\\
0.9525953	-0.00521660149284581\\
0.9526953	-0.00313195003768874\\
0.9527953	2.21207667666645e-06\\
0.9528953	-0.000247619662930076\\
0.9529953	-0.000560949767888619\\
0.9530953	0.00173591725224687\\
0.9531953	0.00530197688977452\\
0.9532953	0.00689642463886875\\
0.9533953	0.00765275599560711\\
0.9534953	0.0129145664579943\\
0.9535954	0.0170488620980471\\
0.9536954	0.0183551196414449\\
0.9537954	0.0199444427918332\\
0.9538954	0.0210947270551829\\
0.9539954	0.0232163679395171\\
0.9540954	0.0233385609549402\\
0.9541954	0.0215033016136558\\
0.9542954	0.0219419854299966\\
0.9543954	0.0234928079204467\\
0.9544954	0.0243389646036665\\
0.9545955	0.0226658850514404\\
0.9546955	0.022006799008003\\
0.9547955	0.0241009337221429\\
0.9548955	0.0260301847214685\\
0.9549955	0.0231658475358859\\
0.9550955	0.0232077176976213\\
0.9551955	0.0244531907412453\\
0.9552955	0.0259803622036964\\
0.9553955	0.0246380276243039\\
0.9554955	0.0243530825448112\\
0.9555956	0.0251235795915445\\
0.9556956	0.0258972024252243\\
0.9557956	0.0266642013943083\\
0.9558956	0.0256996720504341\\
0.9559956	0.0275583099477693\\
0.9560956	0.0282164106430343\\
0.9561956	0.0296203696955275\\
0.9562956	0.0261764826671436\\
0.9563956	0.0246783451223986\\
0.9564956	0.0230031526284524\\
0.9565957	0.0228963804232094\\
0.9566957	0.0211400669770039\\
0.9567957	0.0209801852947371\\
0.9568957	0.0202875309543677\\
0.9569957	0.0195308995366096\\
0.9570957	0.0186548866249518\\
0.9571957	0.0157071878056815\\
0.9572957	0.0146883986679058\\
0.9573957	0.0158876148035728\\
0.9574957	0.0137103318074934\\
0.9575958	0.0118113470886937\\
0.9576958	0.00970835481497673\\
0.9577958	0.00644685020693986\\
0.9578958	0.0054148288710609\\
0.9579958	0.0045001864167856\\
0.9580958	0.000796618456549325\\
0.9581958	-0.00176997939420062\\
0.9582958	-0.00197971151698753\\
0.9583958	-0.00211518229028165\\
0.9584958	-0.00313089608948217\\
0.9585959	-0.00457323377217334\\
0.9586959	-0.00647054459094398\\
0.9587959	-0.00727071154754183\\
0.9588959	-0.00650393900487015\\
0.9589959	-0.00895713132265777\\
0.9590959	-0.0113708928574393\\
0.9591959	-0.0163102279625372\\
0.9592959	-0.0162072409880389\\
0.9593959	-0.0170637362807805\\
0.9594959	-0.0203116181843242\\
0.959596	-0.0216286462576653\\
0.959696	-0.0214953122977715\\
0.959796	-0.0211770779638702\\
0.959896	-0.0229529475862358\\
0.959996	-0.0231569254917745\\
0.960096	-0.0229634160040031\\
0.960196	-0.023712223443033\\
0.960296	-0.0237502521255467\\
0.960396	-0.021870106364784\\
0.960496	-0.0235454904705207\\
0.9605961	-0.026623343134819\\
0.9606961	-0.0280791978295641\\
0.9607961	-0.0303286953034664\\
0.9608961	-0.0296000398522601\\
0.9609961	-0.030730135768123\\
0.9610961	-0.0325751873396588\\
0.9611961	-0.0335161988518809\\
0.9612961	-0.0326102745861917\\
0.9613961	-0.0311567188203668\\
0.9614961	-0.0309857358285367\\
0.9615962	-0.0304030438641572\\
0.9616962	-0.0311692172114841\\
0.9617962	-0.0312053761374106\\
0.9618962	-0.0281117249013068\\
0.9619962	-0.0272989677588103\\
0.9620962	-0.0284645089618085\\
0.9621962	-0.0298840527584225\\
0.9622962	-0.0273042033929896\\
0.9623962	-0.0254882651060441\\
0.9624962	-0.0242918421343048\\
0.9625963	-0.0209046327174273\\
0.9626963	-0.0200778510967921\\
0.9627963	-0.0197835974826678\\
0.9628963	-0.0199642760963289\\
0.9629963	-0.0171623911551519\\
0.9630963	-0.0152647468725955\\
0.9631963	-0.0112444474581853\\
0.9632963	-0.00996199711749723\\
0.9633963	-0.0100783000521423\\
0.9634963	-0.00802356045974936\\
0.9635964	-0.00375815698728854\\
0.9636964	-0.00335124298565559\\
0.9637964	-0.00485369903000998\\
0.9638964	-0.00482102930189399\\
0.9639964	-0.00500293797878998\\
0.9640964	-0.00403752923410475\\
0.9641964	-0.00296040723715449\\
0.9642964	-0.000805676153150997\\
0.9643964	0.000284559856818106\\
0.9644964	0.00179319663579863\\
0.9645965	0.00471397471225449\\
0.9646965	0.00667560246540908\\
0.9647965	0.00793561853753827\\
0.9648965	0.0115318187883857\\
0.9649965	0.0149845990819071\\
0.9650965	0.0178395552862806\\
0.9651965	0.0173188832739249\\
0.9652965	0.0195214789215118\\
0.9653965	0.0207724381099807\\
0.9654965	0.0227231567245542\\
0.9655966	0.0238248940558939\\
0.9656966	0.0247919210446961\\
0.9657966	0.024902395137002\\
0.9658966	0.0233374122353061\\
0.9659966	0.0225693682464563\\
0.9660966	0.0253689590816644\\
0.9661966	0.0269119806565223\\
0.9662966	0.0270004288910145\\
0.9663966	0.027927699709531\\
0.9664966	0.0278967890408831\\
0.9665967	0.0279660745288647\\
0.9666967	0.0282022904984139\\
0.9667967	0.030142112789822\\
0.9668967	0.0306989373497139\\
0.9669967	0.0308595601292\\
0.9670967	0.0304035770838872\\
0.9671967	0.0298872841738945\\
0.9672967	0.0283861773638632\\
0.9673967	0.0268465526229712\\
0.9674967	0.0278687059249448\\
0.9675968	0.0289321328619503\\
0.9676968	0.0273221319570851\\
0.9677968	0.0259153970396694\\
0.9678968	0.0240076241017515\\
0.9679968	0.0224717091399903\\
0.9680968	0.0207502481556659\\
0.9681968	0.0212451371546927\\
0.9682968	0.020456872147631\\
0.9683968	0.020291849149698\\
0.9684968	0.0187804641807801\\
0.9685969	0.0174776303810994\\
0.9686969	0.0163909112767064\\
0.9687969	0.0156022182851574\\
0.9688969	0.0130595474451362\\
0.9689969	0.0130271948000527\\
0.9690969	0.0129378563980567\\
0.9691969	0.010986628292044\\
0.9692969	0.00860050653967237\\
0.9693969	0.00644768720336938\\
0.9694969	0.00447946635034521\\
0.969597	0.00210497427311893\\
0.969697	0.000314440296133089\\
0.969797	-0.00183550697106177\\
0.969897	-0.0013829714420161\\
0.969997	-0.0038236570254456\\
0.970097	-0.0065587676252204\\
0.970197	-0.00879310714035403\\
0.970297	-0.0110299794649948\\
0.970397	-0.0134293884884139\\
0.970497	-0.0167292380949975\\
0.9705971	-0.018031381230999\\
0.9706971	-0.0177673219877434\\
0.9707971	-0.0191810149552552\\
0.9708971	-0.0197513639982693\\
0.9709971	-0.0218564729765848\\
0.9710971	-0.0234237457450557\\
0.9711971	-0.0242005861535791\\
0.9712971	-0.0274660980470899\\
0.9713971	-0.0284981852655476\\
0.9714971	-0.0281726516439295\\
0.9715972	-0.0290771337535518\\
0.9716972	-0.0285880683254597\\
0.9717972	-0.0292164935360472\\
0.9718972	-0.0285766132002611\\
0.9719972	-0.0275608311280191\\
0.9720972	-0.0297674511242014\\
0.9721972	-0.0320424769886419\\
0.9722972	-0.0310630125161189\\
0.9723972	-0.0311668614963468\\
0.9724972	-0.0322245277139677\\
0.9725973	-0.032907731746757\\
0.9726973	-0.0334875421994366\\
0.9727973	-0.0336089812166828\\
0.9728973	-0.0337211525627578\\
0.9729973	-0.0338071599968106\\
0.9730973	-0.0342607072728675\\
0.9731973	-0.0343853981398228\\
0.9732973	-0.0356413363414344\\
0.9733973	-0.0353997256163128\\
0.9734973	-0.0339883696979145\\
0.9735974	-0.0330385735468904\\
0.9736974	-0.0311961368857728\\
0.9737974	-0.0278511662044601\\
0.9738974	-0.0251070652157101\\
0.9739974	-0.0240798376270878\\
0.9740974	-0.0245690871409589\\
0.9741974	-0.022891717454481\\
0.9742974	-0.0211652322595983\\
0.9743974	-0.022272035243033\\
0.9744974	-0.0205394300862786\\
0.9745975	-0.0186598065042141\\
0.9746975	-0.0166387945916549\\
0.9747975	-0.0142398855556081\\
0.9748975	-0.0117730830565775\\
0.9749975	-0.0104688907498035\\
0.9750975	-0.00779801228525681\\
0.9751975	-0.00503295130763179\\
0.9752975	-0.00393341145633833\\
0.9753975	-0.0023796963654982\\
0.9754975	-0.00194390966393637\\
0.9755976	-0.000748826186967533\\
0.9756976	0.000410094333336913\\
0.9757976	0.00223637560611725\\
0.9758976	0.00315261402379993\\
0.9759976	0.00401440598413927\\
0.9760976	0.00637374789022284\\
0.9761976	0.00860673615047686\\
0.9762976	0.0115076671786731\\
0.9763976	0.0148895373939334\\
0.9764976	0.0169375432207365\\
0.9765977	0.0184543243423374\\
0.9766977	0.0194439921138106\\
0.9767977	0.0215522847989012\\
0.9768977	0.0220176988435751\\
0.9769977	0.0221602306991825\\
0.9770977	0.0225843768224632\\
0.9771977	0.0230549336755514\\
0.9772977	0.0236155477259812\\
0.9773977	0.023777205446693\\
0.9774977	0.0251251033160361\\
0.9775978	0.0260129751801192\\
0.9776978	0.0268901841948058\\
0.9777978	0.0283831428221804\\
0.9778978	0.0304268175622892\\
0.9779978	0.0313647549206116\\
0.9780978	0.0322094014080665\\
0.9781978	0.0327194035410174\\
0.9782978	0.033260887841275\\
0.9783978	0.0339195108361039\\
0.9784978	0.0336391490582255\\
0.9785979	0.0336579501662198\\
0.9786979	0.0322895598195238\\
0.9787979	0.0320739843276243\\
0.9788979	0.0304801002451208\\
0.9789979	0.0296250541320907\\
0.9790979	0.0289173825540949\\
0.9791979	0.02693319208218\\
0.9792979	0.0253556792928828\\
0.9793979	0.024286310768236\\
0.9794979	0.0244508630957688\\
0.979598	0.0225418574015358\\
0.979698	0.021162822540371\\
0.979798	0.0196207183236073\\
0.979898	0.0189092113608084\\
0.979998	0.0176457482670561\\
0.980098	0.0163934556629498\\
0.980198	0.0151105601746125\\
0.980298	0.0150204184336935\\
0.980398	0.0135703870773715\\
0.980498	0.011265802748358\\
0.9805981	0.00888353970061372\\
0.9806981	0.00774739066515426\\
0.9807981	0.0063401286150269\\
0.9808981	0.00358965021510986\\
0.9809981	0.00301795213582937\\
0.9810981	0.000760331053162006\\
0.9811981	-0.00169121635136224\\
0.9812981	-0.00384329339065724\\
0.9813981	-0.00487540337207902\\
0.9814981	-0.0066899495974205\\
0.9815982	-0.00920622501891656\\
0.9816982	-0.0100315523596989\\
0.9817982	-0.0127871258196733\\
0.9818982	-0.0155075486783586\\
0.9819982	-0.0180455242096992\\
0.9820982	-0.0195087556820652\\
0.9821982	-0.0215708463582484\\
0.9822982	-0.0236645994954604\\
0.9823982	-0.0259719183453311\\
0.9824982	-0.0274372061539062\\
0.9825983	-0.0282951434082195\\
0.9826983	-0.0294470776270458\\
0.9827983	-0.0306156905124164\\
0.9828983	-0.0315282852880163\\
0.9829983	-0.0331740651719366\\
0.9830983	-0.0343980333766741\\
0.9831983	-0.0350761931091285\\
0.9832983	-0.0360249475706012\\
0.9833983	-0.0362974999567944\\
0.9834983	-0.0368588534578089\\
0.9835984	-0.0365907764185555\\
0.9836984	-0.0368901405061688\\
0.9837984	-0.0368145152484543\\
0.9838984	-0.0377641038130884\\
0.9839984	-0.0393620093621405\\
0.9840984	-0.0401469350520696\\
0.9841984	-0.0397146840337261\\
0.9842984	-0.0399572594523489\\
0.9843984	-0.039705664447565\\
0.9844984	-0.0392818021533882\\
0.9845985	-0.0406388290901441\\
0.9846985	-0.0410148404372904\\
0.9847985	-0.0399634938665138\\
0.9848985	-0.0393429924893629\\
0.9849985	-0.0379602394117691\\
0.9850985	-0.0381519377340464\\
0.9851985	-0.0376962905508907\\
0.9852985	-0.0370915009513796\\
0.9853985	-0.0365279720189718\\
0.9854985	-0.0369068068315063\\
0.9855986	-0.0380426503966977\\
0.9856986	-0.0383328207815706\\
0.9857986	-0.0377261641142411\\
0.9858986	-0.0366244834500619\\
0.9859986	-0.0364543818387647\\
0.9860986	-0.0368505623244608\\
0.9861986	-0.0373891279456396\\
0.9862986	-0.0375742817351708\\
0.9863986	-0.0388593267203024\\
0.9864986	-0.0395083659226617\\
0.9865987	-0.0401249331437619\\
0.9866987	-0.0404341687247244\\
0.9867987	-0.0411187075570754\\
0.9868987	-0.0416162526399548\\
0.9869987	-0.042118006966883\\
0.9870987	-0.0413692735257614\\
0.9871987	-0.042664355298872\\
0.9872987	-0.0437460552628784\\
0.9873987	-0.0439802763888268\\
0.9874987	-0.0456288216421452\\
0.9875988	-0.0451622139189801\\
0.9876988	-0.0458470152323759\\
0.9877988	-0.0460030495386748\\
0.9878988	-0.046397619780837\\
0.9879988	-0.0463331288962081\\
0.9880988	-0.0464916798165215\\
0.9881988	-0.0474965754678971\\
0.9882988	-0.0487950187708447\\
0.9883988	-0.0487797126402637\\
0.9884988	-0.0494734599854441\\
0.9885989	-0.0491497730924279\\
0.9886989	-0.0484555350552959\\
0.9887989	-0.04705925919104\\
0.9888989	-0.047115248386519\\
0.9889989	-0.0470811055229882\\
0.9890989	-0.0459286334761005\\
0.9891989	-0.0446592351159077\\
0.9892989	-0.0435647133068628\\
0.9893989	-0.04318127090782\\
0.9894989	-0.0428233107720376\\
0.989599	-0.0436068348651508\\
0.989699	-0.0435752467826576\\
0.989799	-0.0436874494924679\\
0.989899	-0.0427403458254634\\
0.989999	-0.042691138606937\\
0.990099	-0.0426426306565946\\
0.990199	-0.0422167247885574\\
0.990299	-0.0425753238113623\\
0.990399	-0.0438720305279657\\
0.990499	-0.045114247735744\\
0.9905991	-0.0457924673640651\\
0.9906991	-0.0469219129414757\\
0.9907991	-0.0460379773715096\\
0.9908991	-0.0470444634292394\\
0.9909991	-0.0467264738841676\\
0.9910991	-0.0459317115002293\\
0.9911991	-0.0442286790357943\\
0.9912991	-0.0426561792436695\\
0.9913991	-0.0419544148711008\\
0.9914991	-0.0413232886597759\\
0.9915992	-0.0396446827813935\\
0.9916992	-0.0380723401403891\\
0.9917992	-0.038173043855086\\
0.9918992	-0.0372632966449558\\
0.9919992	-0.0371367012239238\\
0.9920992	-0.0376396603003715\\
0.9921992	-0.0376044765771388\\
0.9922992	-0.0380161527515276\\
0.9923992	-0.0378432915153023\\
0.9924992	-0.0369028955546952\\
0.9925993	-0.0361124375568085\\
0.9926993	-0.0349242792559816\\
0.9927993	-0.0328793942589625\\
0.9928993	-0.0322039852298708\\
0.9929993	-0.0309841748273076\\
0.9930993	-0.0289019757043571\\
0.9931993	-0.0276442205085902\\
0.9932993	-0.0254991018820668\\
0.9933993	-0.0243165624613394\\
0.9934993	-0.0244097648774561\\
0.9935994	-0.0229518626004955\\
0.9936994	-0.0225561756598334\\
0.9937994	-0.0232021684187749\\
0.9938994	-0.0224470834863754\\
0.9939994	-0.0219689334662018\\
0.9940994	-0.0208557709563356\\
0.9941994	-0.0194337385493758\\
0.9942994	-0.0179996988324432\\
0.9943994	-0.017341444387182\\
0.9944994	-0.0148672677897654\\
0.9945995	-0.013391503555345\\
0.9946995	-0.0119668594845309\\
0.9947995	-0.0116219609598393\\
0.9948995	-0.0113167205355843\\
0.9949995	-0.0104166207606248\\
0.9950995	-0.00914698417836958\\
0.9951995	-0.00867309332677987\\
0.9952995	-0.00792882073837329\\
0.9953995	-0.00758306894022775\\
0.9954995	-0.0064782704539848\\
0.9955996	-0.00538008109652154\\
0.9956996	-0.00403680592687716\\
0.9957996	-0.00341708160398429\\
0.9958996	-0.00303719062777272\\
0.9959996	-0.00137947549275535\\
0.9960996	-0.00186864868803215\\
0.9961996	-0.00201828269729395\\
0.9962996	-0.00115770999882658\\
0.9963996	0.000193546934484652\\
0.9964996	0.000775135635152469\\
0.9965997	0.00228429873333431\\
0.9966997	0.00401144441343893\\
0.9967997	0.00599953448832777\\
0.9968997	0.0071088765120647\\
0.9969997	0.00858516804409007\\
0.9970997	0.00943422664921582\\
0.9971997	0.0105199798976214\\
0.9972997	0.0112968553648499\\
0.9973997	0.012567980631803\\
0.9974997	0.0124338632847374\\
0.9975998	0.013110544154979\\
0.9976998	0.0149913951614157\\
0.9977998	0.0157492963422922\\
0.9978998	0.0170493453052443\\
0.9979998	0.0174945196632407\\
0.9980998	0.0176815470345777\\
0.9981998	0.0177523250428749\\
0.9982998	0.0176068013170705\\
0.9983998	0.0177272034914168\\
0.9984998	0.0180140292054753\\
0.9985999	0.0179310772512095\\
0.9986999	0.0190061537622524\\
0.9987999	0.0189710867611603\\
0.9988999	0.0188406639086904\\
0.9989999	0.0186607628708857\\
0.9990999	0.0185554813190705\\
0.9991999	0.0179062269298448\\
0.9992999	0.0179682373850799\\
0.9993999	0.0175163003719131\\
0.9994999	0.0182770835827433\\
0.9996	0.0186725835349308\\
0.9997	0.0186331610464491\\
0.9998	0.0192852418883904\\
0.9999	0.0188571037741545\\
1	0.0188316644223776\\
};
\end{axis}
\end{tikzpicture}%
		\caption{The error is defined as the difference between the analytical solution and numerical solution.}
		\label{fig:periodicInfError}
	\end{subfigure}
	\label{fig:periodicInf}
	\caption{The numerical and analytical solution as it passes its solution space of $x=[0,1]$. One can clearly see in figure \ref{fig:periodicInfPlot} that the numerical solution is periodic. The error caused by this is highlighted in figure \ref{fig:periodicInfError}}
\end{figure}
\fi
\section{Potential Well}



\end{document}
