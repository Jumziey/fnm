\documentclass[11pt]{article}
\usepackage{report}
\newlength\figureheight
\newlength\figurewidth
\newif\iftikz
\tikztrue

\begin{document}

\newpage
\section{Free Evolution}
\subsection{Selection of Time Step}
In order to find the biggest possible time steps we could use we simply started with reducing increasing the timestep until it we saw significant error. But we went so far that the the time step got so large that it went through the solution space ($x=[0,1]$) after just a few steps. Thus the smallest time step we used was  $3.45\cdot 10^{-4}$. The smallest spacial step we used before the solution started to get jaggy was $0.01$. Since the solution took such a small time to solve for we will use significantly smaller time steps and spacial time steps for the rest of the exercises, such a solution is given as an example in figure \ref{fig:smallError}. In order to look at what $k_0$ to use wee looked at

\iftikz
\begin{figure}[H]
	\centering
	\begin{subfigure}{.9\linewidth}
		\setlength\figureheight{.5\linewidth}
		\setlength\figurewidth{.9\linewidth}
		% This file was created by matlab2tikz.
% Minimal pgfplots version: 1.3
%
%The latest updates can be retrieved from
%  http://www.mathworks.com/matlabcentral/fileexchange/22022-matlab2tikz
%where you can also make suggestions and rate matlab2tikz.
%
\definecolor{mycolor1}{rgb}{0.00000,0.44700,0.74100}%
\definecolor{mycolor2}{rgb}{0.85000,0.32500,0.09800}%
%
\begin{tikzpicture}

\begin{axis}[%
width=0.95092\figurewidth,
height=\figureheight,
at={(0\figurewidth,0\figureheight)},
scale only axis,
xmin=0,
xmax=1,
xlabel={Position},
ymin=-3,
ymax=7,
title style={font=\bfseries},
title={Analytical Solution over Numerical Solution},
legend style={at={(0.03,0.97)},anchor=north west,legend cell align=left,align=left,draw=white!15!black},
title style={font=\small},ticklabel style={font=\tiny}
]
\addplot [color=mycolor1,solid]
  table[row sep=crcr]{%
0	4.983509e-05\\
0.01010101	4.097621e-05\\
0.02020202	3.436718e-05\\
0.03030303	2.931579e-05\\
0.04040404	2.537406e-05\\
0.05050505	2.224358e-05\\
0.06060606	1.971999e-05\\
0.07070707	1.76598e-05\\
0.08080808	1.59599e-05\\
0.09090909	1.454472e-05\\
0.1010101	1.335777e-05\\
0.1111111	1.235629e-05\\
0.1212121	1.150733e-05\\
0.1313131	1.078516e-05\\
0.1414141	1.017007e-05\\
0.1515152	9.644921e-06\\
0.1616162	9.199423e-06\\
0.1717172	8.818608e-06\\
0.1818182	8.502767e-06\\
0.1919192	8.230499e-06\\
0.2020202	8.021692e-06\\
0.2121212	7.828931e-06\\
0.2222222	7.724801e-06\\
0.2323232	7.568721e-06\\
0.2424242	7.619103e-06\\
0.2525253	7.374119e-06\\
0.2626263	7.800733e-06\\
0.2727273	7.072009e-06\\
0.2828283	8.51058e-06\\
0.2929293	6.524174e-06\\
0.3030303	9.695808e-06\\
0.3131313	7.235223e-06\\
0.3232323	8.796613e-06\\
0.3333333	1.725645e-05\\
0.3434343	1.457853e-06\\
0.3535354	5.023403e-05\\
0.3636364	2.094688e-05\\
0.3737374	9.912553e-05\\
0.3838384	0.0002442566\\
0.3939394	0.000261745\\
0.4040404	0.0009475293\\
0.4141414	0.001539987\\
0.4242424	0.002714287\\
0.4343434	0.006251136\\
0.4444444	0.01006635\\
0.4545455	0.0178519\\
0.4646465	0.03377363\\
0.4747475	0.05364224\\
0.4848485	0.08722348\\
0.4949495	0.1466257\\
0.5050505	0.2232999\\
0.5151515	0.3307566\\
0.5252525	0.5013224\\
0.5353535	0.7229525\\
0.5454545	0.9865734\\
0.5555556	1.344122\\
0.5656566	1.809075\\
0.5757576	2.315648\\
0.5858586	2.856428\\
0.5959596	3.495926\\
0.6060606	4.199714\\
0.6161616	4.833282\\
0.6262626	5.356597\\
0.6363636	5.841853\\
0.6464646	6.28096\\
0.6565657	6.539182\\
0.6666667	6.532512\\
0.6767677	6.327717\\
0.6868687	6.028471\\
0.6969697	5.649344\\
0.7070707	5.145246\\
0.7171717	4.512473\\
0.7272727	3.816756\\
0.7373737	3.140915\\
0.7474747	2.534426\\
0.7575758	2.008535\\
0.7676768	1.557409\\
0.7777778	1.174872\\
0.7878788	0.8581664\\
0.7979798	0.6051625\\
0.8080808	0.411551\\
0.8181818	0.2700715\\
0.8282828	0.171357\\
0.8383838	0.1054464\\
0.8484848	0.06317496\\
0.8585859	0.03701439\\
0.8686869	0.02131438\\
0.8787879	0.0121304\\
0.8888889	0.006865881\\
0.8989899	0.003892006\\
0.9090909	0.002226652\\
0.9191919	0.001296303\\
0.9292929	0.000774351\\
0.9393939	0.000478232\\
0.9494949	0.0003071622\\
0.959596	0.0002058649\\
0.969697	0.0001440401\\
0.979798	0.000104991\\
0.989899	7.941556e-05\\
1	6.204587e-05\\
};
\addlegendentry{$\text{|}\psi{}_\text{n}\text{(x,t)|}^\text{2}$};

\addplot [color=mycolor2,solid]
  table[row sep=crcr]{%
0	0.006382258\\
0.01010101	-0.005693388\\
0.02020202	0.005127128\\
0.03030303	-0.004655344\\
0.04040404	0.004257565\\
0.05050505	-0.003918664\\
0.06060606	0.00362728\\
0.07070707	-0.003374744\\
0.08080808	0.003154344\\
0.09090909	-0.002960805\\
0.1010101	0.002789925\\
0.1111111	-0.002638334\\
0.1212121	0.002503263\\
0.1313131	-0.002382463\\
0.1414141	0.002274064\\
0.1515152	-0.002176413\\
0.1616162	0.002088442\\
0.1717172	-0.002008374\\
0.1818182	0.001936676\\
0.1919192	-0.001869571\\
0.2020202	0.001812087\\
0.2121212	-0.00175266\\
0.2222222	0.001711034\\
0.2323232	-0.001649596\\
0.2424242	0.001634318\\
0.2525253	-0.001549213\\
0.2626263	0.001586159\\
0.2727273	-0.001446619\\
0.2828283	0.001541094\\
0.2929293	-0.001421135\\
0.3030303	0.001297681\\
0.3131313	-0.001836702\\
0.3232323	0.0003524648\\
0.3333333	-0.003030176\\
0.3434343	-0.0007223738\\
0.3535354	-0.002324622\\
0.3636364	0.003607669\\
0.3737374	0.006917565\\
0.3838384	0.01415357\\
0.3939394	0.008440161\\
0.4040404	-0.007736087\\
0.4141414	-0.0377529\\
0.4242424	-0.04553784\\
0.4343434	-0.00416396\\
0.4444444	0.08495478\\
0.4545455	0.1210858\\
0.4646465	0.0009741574\\
0.4747475	-0.2130927\\
0.4848485	-0.2134305\\
0.4949495	0.1629556\\
0.5050505	0.4654812\\
0.5151515	0.04552584\\
0.5252525	-0.6806516\\
0.5353535	-0.3426896\\
0.5454545	0.8629723\\
0.5555556	0.6414282\\
0.5656566	-1.112356\\
0.5757576	-0.8042046\\
0.5858586	1.497562\\
0.5959596	0.6698328\\
0.6060606	-2.009756\\
0.6161616	0.04957755\\
0.6262626	2.224081\\
0.6363636	-1.319297\\
0.6464646	-1.510677\\
0.6565657	2.478664\\
0.6666667	-0.4948481\\
0.6767677	-1.961102\\
0.6868687	2.288436\\
0.6969697	-0.4743964\\
0.7070707	-1.541135\\
0.7171717	2.118017\\
0.7272727	-1.160003\\
0.7373737	-0.3219549\\
0.7474747	1.30327\\
0.7575758	-1.407246\\
0.7676768	0.8689363\\
0.7777778	-0.1489086\\
0.7878788	-0.3996136\\
0.7979798	0.6473737\\
0.8080808	-0.6386003\\
0.8181818	0.4874338\\
0.8282828	-0.2991896\\
0.8383838	0.1377752\\
0.8484848	-0.02674833\\
0.8585859	-0.03589961\\
0.8686869	0.06289834\\
0.8787879	-0.06813551\\
0.8888889	0.06241376\\
0.8989899	-0.05262796\\
0.9090909	0.04249126\\
0.9191919	-0.03362744\\
0.9292929	0.0264944\\
0.9393939	-0.02100171\\
0.9494949	0.01686189\\
0.959596	-0.01376213\\
0.969697	0.01143267\\
0.979798	-0.009663701\\
0.989899	0.008300678\\
1	-0.007232945\\
};
\addlegendentry{$\text{imag(}\psi{}_\text{n}\text{(x,t))}$};

\addplot [color=red,mark size=0.7pt,only marks,mark=*,mark options={solid}]
  table[row sep=crcr]{%
0	1.65412229097909e-25\\
0.0202020202020202	5.65418478863692e-24\\
0.0404040404040404	1.73294987241482e-22\\
0.0606060606060606	4.76228762688986e-21\\
0.0808080808080808	1.17343452559411e-19\\
0.101010101010101	2.59248183105009e-18\\
0.121212121212121	5.13554098478904e-17\\
0.141414141414141	9.12158472755633e-16\\
0.161616161616162	1.45267349434318e-14\\
0.181818181818182	2.07433715889589e-13\\
0.202020202020202	2.65585463580452e-12\\
0.222222222222222	3.04889784599049e-11\\
0.242424242424242	3.13830439450614e-10\\
0.262626262626263	2.89641558271376e-09\\
0.282828282828283	2.39684695781992e-08\\
0.303030303030303	1.77841598272928e-07\\
0.323232323232323	1.18315057680938e-06\\
0.343434343434343	7.05765306368423e-06\\
0.363636363636364	3.77480229633893e-05\\
0.383838383838384	0.000181026324739673\\
0.404040404040404	0.000778400089026891\\
0.424242424242424	0.00300108045506635\\
0.444444444444444	0.0103744723815011\\
0.464646464646465	0.0321564441201201\\
0.484848484848485	0.0893683287447141\\
0.505050505050505	0.222696234357961\\
0.525252525252525	0.497571766572944\\
0.545454545454545	0.996809582644575\\
0.565656565656566	1.79053287028574\\
0.585858585858586	2.88380579888758\\
0.606060606060606	4.16450456383034\\
0.626262626262626	5.39230306301861\\
0.646464646464647	6.26035357491163\\
0.666666666666667	6.51683987985372\\
0.686868686868687	6.08259439701959\\
0.707070707070707	5.09042781848946\\
0.727272727272727	3.81973583040548\\
0.747474747474748	2.56995780171507\\
0.767676767676768	1.55035900118948\\
0.787878787878788	0.838594822382505\\
0.808080808080808	0.406710792647341\\
0.828282828282828	0.176861309631007\\
0.848484848484849	0.0689594191703558\\
0.868686868686869	0.024108376879539\\
0.888888888888889	0.00755711390395282\\
0.909090909090909	0.00212401501788712\\
0.929292929292929	0.000535269893495758\\
0.94949494949495	0.000120948829490799\\
0.96969696969697	2.45044038529882e-05\\
0.98989898989899	4.45143700656034e-06\\
};
\addlegendentry{$\text{|}\psi{}_\text{a}\text{(x,t)|}^\text{2}$};

\addplot [color=black,mark size=0.7pt,only marks,mark=*,mark options={solid}]
  table[row sep=crcr]{%
0	-3.48584777067594e-13\\
0.0202020202020202	-7.1155057653272e-13\\
0.0404040404040404	1.27637485171471e-11\\
0.0606060606060606	2.50408208561786e-11\\
0.0808080808080808	-2.67201294681666e-10\\
0.101010101010101	-1.47714312353884e-09\\
0.121212121212121	-1.82056037794223e-09\\
0.141414141414141	1.36004186618439e-08\\
0.161616161616162	1.02092343642842e-07\\
0.181818181818182	4.4656986824771e-07\\
0.202020202020202	1.6294741826555e-06\\
0.222222222222222	5.52109095603838e-06\\
0.242424242424242	1.73825991111427e-05\\
0.262626262626263	4.57615567101897e-05\\
0.282828282828283	7.09019433146301e-05\\
0.303030303030303	-0.000102621858997162\\
0.323232323232323	-0.000991959412866287\\
0.343434343434343	-0.00209849871392245\\
0.363636363636364	0.0021236122116585\\
0.383838383838384	0.0131112358923495\\
0.404040404040404	-0.00772567269742116\\
0.424242424242424	-0.0476648290951618\\
0.444444444444444	0.081708570418146\\
0.464646464646465	0.00781345708718012\\
0.484848484848485	-0.222100221350392\\
0.505050505050505	0.468504947044059\\
0.525252525252525	-0.671256889562341\\
0.545454545454545	0.843688882284122\\
0.565656565656566	-1.07704854861936\\
0.585858585858586	1.4688629312828\\
0.606060606060606	-1.9844012936074\\
0.626262626262626	2.25804866969184\\
0.646464646464647	-1.58044703493243\\
0.666666666666667	-0.378031090159871\\
0.686868686868687	2.2627132548072\\
0.707070707070707	-1.59674874009463\\
0.727272727272727	-1.08430724770132\\
0.747474747474748	1.34758253004581\\
0.767676767676768	0.837146683904619\\
0.787878787878788	-0.42918635583009\\
0.808080808080808	-0.637264980186851\\
0.828282828282828	-0.285298700105815\\
0.848484848484849	-0.0164008145131732\\
0.868686868686869	0.0661206863804965\\
0.888888888888889	0.0599900094185233\\
0.909090909090909	0.0361761540616706\\
0.929292929292929	0.0175778117298425\\
0.94949494949495	0.00659058768232577\\
0.96969696969697	0.00118409228734464\\
0.98989898989899	-0.000703987334150138\\
};
\addlegendentry{$\text{imag(}\psi{}_\text{a}\text{(x,t))}$};

\end{axis}
\end{tikzpicture}%
		\caption{The numerical solution ($\psi_n$) plotted against the analytical solution ($\psi_a$).}
		\label{fig:biggestStepPlot}
	\end{subfigure}
	\begin{subfigure}{.9\linewidth}
		\setlength\figureheight{.5\linewidth}
		\setlength\figurewidth{.9\linewidth}
		% This file was created by matlab2tikz.
% Minimal pgfplots version: 1.3
%
%The latest updates can be retrieved from
%  http://www.mathworks.com/matlabcentral/fileexchange/22022-matlab2tikz
%where you can also make suggestions and rate matlab2tikz.
%
\definecolor{mycolor1}{rgb}{0.00000,0.44700,0.74100}%
%
\begin{tikzpicture}

\begin{axis}[%
width=0.95092\figurewidth,
height=\figureheight,
at={(0\figurewidth,0\figureheight)},
scale only axis,
xmin=0,
xmax=1,
xlabel={Position},
ymin=-0.08,
ymax=0.06,
ylabel={Magnitude},
title style={font=\bfseries},
title={$\text{Error of |}\psi{}_\text{n}\text{(x,t)|}^\text{2}\text{, dt = 0.000345, dx = 0.01}$},
title style={font=\small},ticklabel style={font=\tiny}
]
\addplot [color=mycolor1,solid,forget plot]
  table[row sep=crcr]{%
0	-4.983509e-05\\
0.01010101	-4.097621e-05\\
0.02020202	-3.436718e-05\\
0.03030303	-2.931579e-05\\
0.04040404	-2.537406e-05\\
0.05050505	-2.224358e-05\\
0.06060606	-1.971999e-05\\
0.07070707	-1.76598e-05\\
0.08080808	-1.59598999999999e-05\\
0.09090909	-1.45447199999994e-05\\
0.1010101	-1.33577699999974e-05\\
0.1111111	-1.23562899999883e-05\\
0.1212121	-1.15073299999486e-05\\
0.1313131	-1.07851599997806e-05\\
0.1414141	-1.01700699990878e-05\\
0.1515152	-9.64492099630984e-06\\
0.1616162	-9.19942298547319e-06\\
0.1717172	-8.81860794435212e-06\\
0.1818182	-8.5027667925658e-06\\
0.1919192	-8.23049824757093e-06\\
0.2020202	-8.02168934414603e-06\\
0.2121212	-7.82892187787017e-06\\
0.2222222	-7.72477051110156e-06\\
0.2323232	-7.56862183905732e-06\\
0.2424242	-7.61878917106114e-06\\
0.2525253	-7.3731524962667e-06\\
0.2626263	-7.79783657280128e-06\\
0.2727273	-7.06356253228115e-06\\
0.2828283	-8.48661148847926e-06\\
0.2929293	-6.45798886928311e-06\\
0.3030303	-9.51796645373502e-06\\
0.3131313	-6.77021604460534e-06\\
0.3232323	-7.61346492739771e-06\\
0.3333333	-1.43270938204716e-05\\
0.3434343	5.59977379190962e-06\\
0.3535354	-3.36876790803057e-05\\
0.3636364	1.68012531906737e-05\\
0.3737374	-1.53258895645637e-05\\
0.3838384	-6.32300561236675e-05\\
0.3939394	0.000118791038999799\\
0.4040404	-0.000169129429554457\\
0.4141414	9.41154957046354e-06\\
0.4242424	0.000286788791637682\\
0.4343434	-0.000594682639952643\\
0.4444444	0.000308095316563018\\
0.4545455	0.000663854829978203\\
0.4646465	-0.0016171252893417\\
0.4747475	0.000701519525555902\\
0.4848485	0.00214491359919021\\
0.4949495	-0.00361397283377782\\
0.5050505	-0.000603713437475339\\
0.5151515	0.00669177199486548\\
0.5252525	-0.00375109951202501\\
0.5353535	-0.00902068404955403\\
0.5454545	0.0102347466465921\\
0.5555556	0.0101974277804118\\
0.5656566	-0.0185405129334668\\
0.5757576	-0.0120976968631394\\
0.5858586	0.0273786508486924\\
0.5959596	0.0171537286738053\\
0.6060606	-0.0352098271302488\\
0.6161616	-0.0294033806247169\\
0.6262626	0.035704634243606\\
0.6363636	0.04807191540828\\
0.6464646	-0.0206077887800227\\
0.6565657	-0.0641593577166102\\
0.6666667	-0.015672275016124\\
0.6767677	0.0547153051813796\\
0.6868687	0.0541229086844499\\
0.6969697	-0.0084900335490099\\
0.7070707	-0.0548177670542502\\
0.7171717	-0.0423732992488821\\
0.7272727	0.0029815926172132\\
0.7373737	0.0352516146071236\\
0.7474747	0.0355345245637482\\
0.7575758	0.0149592303173445\\
0.7676768	-0.00705138782896197\\
0.7777778	-0.0189853976601937\\
0.7878788	-0.0195719142644879\\
0.7979798	-0.0131345600566658\\
0.8080808	-0.00484008075481551\\
0.8181818	0.00181202377254269\\
0.8282828	0.00550452932981824\\
0.8383838	0.006507000771152\\
0.8484848	0.00578462407786548\\
0.8585859	0.0043192017286811\\
0.8686869	0.00279395556872581\\
0.8787879	0.00155270749253078\\
0.8888889	0.000691227855469169\\
0.8989899	0.000169434036944505\\
0.9090909	-0.000102635716875982\\
0.9191919	-0.000215394165758254\\
0.9292929	-0.000239079994410615\\
0.9393939	-0.000220296095025064\\
0.9494949	-0.000186212913587908\\
0.959596	-0.000150676879557937\\
0.969697	-0.00011953575683468\\
0.979798	-9.44034678147457e-05\\
0.989899	-7.49641269111064e-05\\
1	-6.0224666885653e-05\\
};
\end{axis}
\end{tikzpicture}%
		\caption{The error is defined as the difference between the analytical solution and numerical solution.}
		\label{fig:biggestStepError}
	\end{subfigure}
	\label{fig:freeEvo}
	\caption{The numerical solution at the largest tested time and spacial step. One can clearly see in figure \ref{fig:biggestStepError} that the solution is starting to get jaggy from the spacial step. The larger time step never gets to the point where the solution seem to go bad before just running off our solution space $x=[0,1]$ in just a few steps. The error is not particularly large compared to the solution values, as seen in figure \ref{fig:biggestStepError}}
\end{figure}
\fi
\iftikz
\begin{figure}[H]
	\centering
	\begin{subfigure}{.9\linewidth}
		\setlength\figureheight{.5\linewidth}
		\setlength\figurewidth{.9\linewidth}
		% This file was created by matlab2tikz.
% Minimal pgfplots version: 1.3
%
%The latest updates can be retrieved from
%  http://www.mathworks.com/matlabcentral/fileexchange/22022-matlab2tikz
%where you can also make suggestions and rate matlab2tikz.
%
\definecolor{mycolor1}{rgb}{0.00000,0.44700,0.74100}%
\definecolor{mycolor2}{rgb}{0.85000,0.32500,0.09800}%
%
\begin{tikzpicture}

\begin{axis}[%
width=0.95092\figurewidth,
height=\figureheight,
at={(0\figurewidth,0\figureheight)},
scale only axis,
xmin=0,
xmax=1,
xlabel={Position},
ymin=-4,
ymax=8,
title style={font=\bfseries},
title={Analytical Solution over Numerical Solution},
legend style={at={(0.03,0.97)},anchor=north west,legend cell align=left,align=left,draw=white!15!black},
title style={font=\small},ticklabel style={font=\tiny}
]
\addplot [color=mycolor1,solid,forget plot]
  table[row sep=crcr]{%
0	1.979135e-11\\
0.00010001	2.001106e-11\\
0.00020002	2.013798e-11\\
0.00030003	2.030288e-11\\
0.00040004	2.00696e-11\\
0.00050005	1.94628e-11\\
0.00060006	1.873618e-11\\
0.00070007	1.820298e-11\\
0.00080008	1.784599e-11\\
0.00090009	1.771746e-11\\
0.0010001	1.781484e-11\\
0.00110011	1.774644e-11\\
0.00120012	1.739919e-11\\
0.00130013	1.695124e-11\\
0.00140014	1.626287e-11\\
0.00150015	1.581651e-11\\
0.00160016	1.573756e-11\\
0.00170017	1.592192e-11\\
0.00180018	1.615899e-11\\
0.00190019	1.619832e-11\\
0.0020002	1.59922e-11\\
0.00210021	1.555577e-11\\
0.00220022	1.517869e-11\\
0.00230023	1.490197e-11\\
0.00240024	1.504665e-11\\
0.00250025	1.514703e-11\\
0.00260026	1.515247e-11\\
0.00270027	1.486706e-11\\
0.00280028	1.410769e-11\\
0.00290029	1.339793e-11\\
0.0030003	1.292548e-11\\
0.00310031	1.262472e-11\\
0.00320032	1.256595e-11\\
0.00330033	1.221843e-11\\
0.00340034	1.17284e-11\\
0.00350035	1.109323e-11\\
0.00360036	1.054082e-11\\
0.00370037	1.006407e-11\\
0.00380038	1.00371e-11\\
0.00390039	1.026417e-11\\
0.0040004	1.028971e-11\\
0.00410041	1.010625e-11\\
0.00420042	9.869778e-12\\
0.00430043	9.701126e-12\\
0.00440044	9.954087e-12\\
0.00450045	1.037724e-11\\
0.00460046	1.072019e-11\\
0.00470047	1.083342e-11\\
0.00480048	1.062668e-11\\
0.00490049	1.045079e-11\\
0.0050005	1.047256e-11\\
0.00510051	1.075032e-11\\
0.00520052	1.079974e-11\\
0.00530053	1.091013e-11\\
0.00540054	1.056258e-11\\
0.00550055	1.012015e-11\\
0.00560056	9.935881e-12\\
0.00570057	9.914937e-12\\
0.00580058	1.016725e-11\\
0.00590059	1.002723e-11\\
0.0060006	9.744337e-12\\
0.00610061	9.340923e-12\\
0.00620062	9.151418e-12\\
0.00630063	9.109164e-12\\
0.00640064	9.288624e-12\\
0.00650065	9.131997e-12\\
0.00660066	8.716563e-12\\
0.00670067	8.245757e-12\\
0.00680068	7.916234e-12\\
0.00690069	7.764897e-12\\
0.0070007	7.692801e-12\\
0.00710071	7.378082e-12\\
0.00720072	6.830167e-12\\
0.00730073	6.378695e-12\\
0.00740074	6.083001e-12\\
0.00750075	6.210472e-12\\
0.00760076	6.137054e-12\\
0.00770077	6.111324e-12\\
0.00780078	5.716687e-12\\
0.00790079	5.760047e-12\\
0.0080008	5.897477e-12\\
0.00810081	6.196999e-12\\
0.00820082	6.391898e-12\\
0.00830083	6.212512e-12\\
0.00840084	6.034506e-12\\
0.00850085	6.128531e-12\\
0.00860086	6.343514e-12\\
0.00870087	6.374792e-12\\
0.00880088	6.168623e-12\\
0.00890089	5.803961e-12\\
0.0090009	5.730319e-12\\
0.00910091	5.779149e-12\\
0.00920092	5.852532e-12\\
0.00930093	5.726741e-12\\
0.00940094	5.520773e-12\\
0.00950095	5.32363e-12\\
0.00960096	5.556461e-12\\
0.00970097	5.687112e-12\\
0.00980098	5.681423e-12\\
0.00990099	5.368825e-12\\
0.010001	5.239001e-12\\
0.01010101	5.414369e-12\\
0.01020102	5.472339e-12\\
0.01030103	5.355335e-12\\
0.01040104	5.030297e-12\\
0.01050105	4.867331e-12\\
0.01060106	4.932672e-12\\
0.01070107	4.951117e-12\\
0.01080108	4.854511e-12\\
0.01090109	4.572396e-12\\
0.0110011	4.366981e-12\\
0.01110111	4.544271e-12\\
0.01120112	4.581339e-12\\
0.01130113	4.51376e-12\\
0.01140114	4.303834e-12\\
0.01150115	4.288212e-12\\
0.01160116	4.475242e-12\\
0.01170117	4.549516e-12\\
0.01180118	4.305866e-12\\
0.01190119	4.054726e-12\\
0.0120012	3.962449e-12\\
0.01210121	3.895654e-12\\
0.01220122	3.672542e-12\\
0.01230123	3.249599e-12\\
0.01240124	2.912094e-12\\
0.01250125	2.846484e-12\\
0.01260126	2.804483e-12\\
0.01270127	2.680607e-12\\
0.01280128	2.549466e-12\\
0.01290129	2.666116e-12\\
0.0130013	2.914731e-12\\
0.01310131	2.955983e-12\\
0.01320132	2.866019e-12\\
0.01330133	2.816352e-12\\
0.01340134	2.92058e-12\\
0.01350135	3.00315e-12\\
0.01360136	2.863973e-12\\
0.01370137	2.823131e-12\\
0.01380138	2.969006e-12\\
0.01390139	3.180118e-12\\
0.0140014	3.322963e-12\\
0.01410141	3.282505e-12\\
0.01420142	3.412267e-12\\
0.01430143	3.648007e-12\\
0.01440144	3.766139e-12\\
0.01450145	3.586935e-12\\
0.01460146	3.439374e-12\\
0.01470147	3.526346e-12\\
0.01480148	3.536009e-12\\
0.01490149	3.24039e-12\\
0.0150015	2.943962e-12\\
0.01510151	2.823436e-12\\
0.01520152	2.754822e-12\\
0.01530153	2.48134e-12\\
0.01540154	2.175781e-12\\
0.01550155	2.191782e-12\\
0.01560156	2.276727e-12\\
0.01570157	2.252617e-12\\
0.01580158	2.0911e-12\\
0.01590159	2.143163e-12\\
0.0160016	2.190044e-12\\
0.01610161	2.028517e-12\\
0.01620162	1.785357e-12\\
0.01630163	1.721192e-12\\
0.01640164	1.809357e-12\\
0.01650165	1.740618e-12\\
0.01660166	1.631899e-12\\
0.01670167	1.704737e-12\\
0.01680168	1.828199e-12\\
0.01690169	1.801025e-12\\
0.0170017	1.728275e-12\\
0.01710171	1.779495e-12\\
0.01720172	1.899921e-12\\
0.01730173	1.789859e-12\\
0.01740174	1.667581e-12\\
0.01750175	1.648179e-12\\
0.01760176	1.703859e-12\\
0.01770177	1.49872e-12\\
0.01780178	1.417719e-12\\
0.01790179	1.57126e-12\\
0.0180018	1.669202e-12\\
0.01810181	1.737487e-12\\
0.01820182	1.745883e-12\\
0.01830183	1.873749e-12\\
0.01840184	1.925694e-12\\
0.01850185	1.748385e-12\\
0.01860186	1.685984e-12\\
0.01870187	1.751117e-12\\
0.01880188	1.719084e-12\\
0.01890189	1.521551e-12\\
0.0190019	1.509698e-12\\
0.01910191	1.581585e-12\\
0.01920192	1.482492e-12\\
0.01930193	1.402667e-12\\
0.01940194	1.58509e-12\\
0.01950195	1.654704e-12\\
0.01960196	1.50985e-12\\
0.01970197	1.412838e-12\\
0.01980198	1.44391e-12\\
0.01990199	1.373369e-12\\
0.020002	1.213924e-12\\
0.02010201	1.286449e-12\\
0.02020202	1.388807e-12\\
0.02030203	1.343935e-12\\
0.02040204	1.292197e-12\\
0.02050205	1.423695e-12\\
0.02060206	1.405288e-12\\
0.02070207	1.315418e-12\\
0.02080208	1.264681e-12\\
0.02090209	1.24765e-12\\
0.0210021	1.060577e-12\\
0.02110211	8.601617e-13\\
0.02120212	8.225582e-13\\
0.02130213	7.806143e-13\\
0.02140214	6.645413e-13\\
0.02150215	6.852008e-13\\
0.02160216	7.850043e-13\\
0.02170217	6.550245e-13\\
0.02180218	6.467324e-13\\
0.02190219	7.098659e-13\\
0.0220022	7.200472e-13\\
0.02210221	6.45784e-13\\
0.02220222	7.147592e-13\\
0.02230223	7.722388e-13\\
0.02240224	7.440778e-13\\
0.02250225	8.904364e-13\\
0.02260226	1.062149e-12\\
0.02270227	1.065505e-12\\
0.02280228	1.048584e-12\\
0.02290229	1.119624e-12\\
0.0230023	1.099488e-12\\
0.02310231	9.743292e-13\\
0.02320232	1.051328e-12\\
0.02330233	1.048869e-12\\
0.02340234	9.331695e-13\\
0.02350235	9.921633e-13\\
0.02360236	1.113459e-12\\
0.02370237	1.08457e-12\\
0.02380238	1.106505e-12\\
0.02390239	1.144826e-12\\
0.0240024	1.047128e-12\\
0.02410241	8.858226e-13\\
0.02420242	9.016643e-13\\
0.02430243	8.292417e-13\\
0.02440244	6.842427e-13\\
0.02450245	6.725165e-13\\
0.02460246	6.411992e-13\\
0.02470247	5.573129e-13\\
0.02480248	5.468596e-13\\
0.02490249	5.765338e-13\\
0.0250025	4.080176e-13\\
0.02510251	3.672776e-13\\
0.02520252	3.694407e-13\\
0.02530253	3.432863e-13\\
0.02540254	3.469602e-13\\
0.02550255	4.547717e-13\\
0.02560256	4.472145e-13\\
0.02570257	5.100607e-13\\
0.02580258	6.361349e-13\\
0.02590259	6.072464e-13\\
0.0260026	5.44659e-13\\
0.02610261	5.717862e-13\\
0.02620262	5.094952e-13\\
0.02630263	4.483122e-13\\
0.02640264	5.331668e-13\\
0.02650265	4.809567e-13\\
0.02660266	4.301597e-13\\
0.02670267	5.30307e-13\\
0.02680268	4.798618e-13\\
0.02690269	4.154147e-13\\
0.0270027	4.303016e-13\\
0.02710271	3.676037e-13\\
0.02720272	3.555255e-13\\
0.02730273	4.465572e-13\\
0.02740274	4.553807e-13\\
0.02750275	4.575093e-13\\
0.02760276	5.381889e-13\\
0.02770277	5.429591e-13\\
0.02780278	5.401277e-13\\
0.02790279	5.509587e-13\\
0.0280028	4.307739e-13\\
0.02810281	4.20345e-13\\
0.02820282	5.033834e-13\\
0.02830283	4.439443e-13\\
0.02840284	4.792879e-13\\
0.02850285	5.237549e-13\\
0.02860286	5.120111e-13\\
0.02870287	5.251715e-13\\
0.02880288	5.207326e-13\\
0.02890289	4.123481e-13\\
0.0290029	4.097448e-13\\
0.02910291	4.450259e-13\\
0.02920292	3.868428e-13\\
0.02930293	4.159305e-13\\
0.02940294	4.162158e-13\\
0.02950295	3.627908e-13\\
0.02960296	3.773254e-13\\
0.02970297	3.026744e-13\\
0.02980298	2.03432e-13\\
0.02990299	2.238597e-13\\
0.030003	1.810781e-13\\
0.03010301	1.695233e-13\\
0.03020302	2.139371e-13\\
0.03030303	1.823417e-13\\
0.03040304	2.094664e-13\\
0.03050305	2.309308e-13\\
0.03060306	1.632861e-13\\
0.03070307	1.655833e-13\\
0.03080308	1.668863e-13\\
0.03090309	1.669704e-13\\
0.0310031	2.038027e-13\\
0.03110311	1.905742e-13\\
0.03120312	1.970997e-13\\
0.03130313	2.315422e-13\\
0.03140314	1.549137e-13\\
0.03150315	1.844455e-13\\
0.03160316	2.055944e-13\\
0.03170317	2.208006e-13\\
0.03180318	3.145331e-13\\
0.03190319	3.331561e-13\\
0.0320032	4.142694e-13\\
0.03210321	4.910827e-13\\
0.03220322	4.091916e-13\\
0.03230323	4.453398e-13\\
0.03240324	4.443101e-13\\
0.03250325	3.806322e-13\\
0.03260326	4.418101e-13\\
0.03270327	3.770176e-13\\
0.03280328	3.554571e-13\\
0.03290329	3.908809e-13\\
0.0330033	2.960702e-13\\
0.03310331	2.388414e-13\\
0.03320332	2.172008e-13\\
0.03330333	1.56895e-13\\
0.03340334	1.804727e-13\\
0.03350335	1.60041e-13\\
0.03360336	1.749393e-13\\
0.03370337	1.895779e-13\\
0.03380338	1.367345e-13\\
0.03390339	1.145229e-13\\
0.0340034	8.574383e-14\\
0.03410341	6.318322e-14\\
0.03420342	7.73087e-14\\
0.03430343	6.063893e-14\\
0.03440344	1.209487e-13\\
0.03450345	1.121673e-13\\
0.03460346	1.20075e-13\\
0.03470347	1.394091e-13\\
0.03480348	1.374319e-13\\
0.03490349	1.573787e-13\\
0.0350035	1.644963e-13\\
0.03510351	1.756601e-13\\
0.03520352	1.992569e-13\\
0.03530353	1.349012e-13\\
0.03540354	1.342956e-13\\
0.03550355	1.087678e-13\\
0.03560356	6.270509e-14\\
0.03570357	8.347131e-14\\
0.03580358	8.635929e-14\\
0.03590359	1.597633e-13\\
0.0360036	1.489808e-13\\
0.03610361	1.423669e-13\\
0.03620362	1.805602e-13\\
0.03630363	1.244529e-13\\
0.03640364	1.435057e-13\\
0.03650365	1.236362e-13\\
0.03660366	1.382131e-13\\
0.03670367	1.497984e-13\\
0.03680368	1.123753e-13\\
0.03690369	1.362849e-13\\
0.0370037	9.578127e-14\\
0.03710371	1.108092e-13\\
0.03720372	1.22267e-13\\
0.03730373	1.456496e-13\\
0.03740374	1.969559e-13\\
0.03750375	1.800654e-13\\
0.03760376	2.151173e-13\\
0.03770377	1.92609e-13\\
0.03780378	1.510589e-13\\
0.03790379	1.678668e-13\\
0.0380038	1.314482e-13\\
0.03810381	1.715969e-13\\
0.03820382	1.497706e-13\\
0.03830383	1.622399e-13\\
0.03840384	1.484375e-13\\
0.03850385	1.000146e-13\\
0.03860386	1.051893e-13\\
0.03870387	6.626921e-14\\
0.03880388	7.447862e-14\\
0.03890389	5.614452e-14\\
0.0390039	6.476783e-14\\
0.03910391	4.62559e-14\\
0.03920392	1.559285e-14\\
0.03930393	1.377689e-14\\
0.03940394	1.734664e-15\\
0.03950395	1.905586e-14\\
0.03960396	2.182474e-14\\
0.03970397	4.787657e-14\\
0.03980398	4.293798e-14\\
0.03990399	6.556501e-14\\
0.040004	7.13583e-14\\
0.04010401	7.814592e-14\\
0.04020402	1.171868e-13\\
0.04030403	9.959317e-14\\
0.04040404	1.229949e-13\\
0.04050405	8.239843e-14\\
0.04060406	7.814006e-14\\
0.04070407	5.435878e-14\\
0.04080408	5.993792e-14\\
0.04090409	9.305767e-14\\
0.0410041	1.070815e-13\\
0.04110411	1.489485e-13\\
0.04120412	1.264462e-13\\
0.04130413	1.496471e-13\\
0.04140414	1.095995e-13\\
0.04150415	1.352902e-13\\
0.04160416	1.21101e-13\\
0.04170417	1.176095e-13\\
0.04180418	1.363441e-13\\
0.04190419	9.71981e-14\\
0.0420042	1.096835e-13\\
0.04210421	6.82277e-14\\
0.04220422	8.928772e-14\\
0.04230423	7.735579e-14\\
0.04240424	9.643094e-14\\
0.04250425	9.427386e-14\\
0.04260426	5.56173e-14\\
0.04270427	5.293571e-14\\
0.04280428	1.596754e-14\\
0.04290429	1.079068e-14\\
0.0430043	5.926238e-15\\
0.04310431	1.911124e-14\\
0.04320432	9.286236e-15\\
0.04330433	2.644815e-14\\
0.04340434	9.776996e-15\\
0.04350435	3.251984e-14\\
0.04360436	2.340328e-14\\
0.04370437	4.746065e-14\\
0.04380438	3.124076e-14\\
0.04390439	4.178834e-14\\
0.0440044	2.654641e-14\\
0.04410441	1.833925e-14\\
0.04420442	2.153565e-14\\
0.04430443	3.58662e-14\\
0.04440444	4.137477e-14\\
0.04450445	5.150753e-14\\
0.04460446	5.226017e-14\\
0.04470447	3.965832e-14\\
0.04480448	4.962215e-14\\
0.04490449	3.809025e-14\\
0.0450045	6.387419e-14\\
0.04510451	3.995413e-14\\
0.04520452	3.659268e-14\\
0.04530453	1.536033e-14\\
0.04540454	1.441925e-14\\
0.04550455	1.078749e-14\\
0.04560456	3.884471e-14\\
0.04570457	3.136948e-14\\
0.04580458	4.962397e-14\\
0.04590459	3.766647e-14\\
0.0460046	3.651174e-14\\
0.04610461	2.675136e-14\\
0.04620462	5.478365e-14\\
0.04630463	5.56204e-14\\
0.04640464	7.09876e-14\\
0.04650465	7.744748e-14\\
0.04660466	7.469931e-14\\
0.04670467	8.754976e-14\\
0.04680468	1.028224e-13\\
0.04690469	1.125338e-13\\
0.0470047	1.155964e-13\\
0.04710471	1.065266e-13\\
0.04720472	6.635721e-14\\
0.04730473	5.670216e-14\\
0.04740474	2.291814e-14\\
0.04750475	2.232078e-14\\
0.04760476	1.641076e-14\\
0.04770477	1.869998e-14\\
0.04780478	9.158743e-15\\
0.04790479	8.016175e-15\\
0.0480048	1.912913e-15\\
0.04810481	5.548156e-15\\
0.04820482	3.484598e-15\\
0.04830483	5.405379e-15\\
0.04840484	2.109249e-15\\
0.04850485	4.103561e-17\\
0.04860486	2.22431e-15\\
0.04870487	8.54337e-15\\
0.04880488	1.198452e-14\\
0.04890489	1.637783e-14\\
0.0490049	1.090828e-14\\
0.04910491	1.533346e-14\\
0.04920492	2.031887e-14\\
0.04930493	2.958089e-14\\
0.04940494	2.765866e-14\\
0.04950495	3.819689e-14\\
0.04960496	3.552753e-14\\
0.04970497	3.904439e-14\\
0.04980498	5.822493e-14\\
0.04990499	8.555226e-14\\
0.050005	9.188719e-14\\
0.05010501	9.087042e-14\\
0.05020502	7.545503e-14\\
0.05030503	4.827833e-14\\
0.05040504	3.779878e-14\\
0.05050505	3.362462e-14\\
0.05060506	3.235114e-14\\
0.05070507	3.819706e-14\\
0.05080508	3.092271e-14\\
0.05090509	2.8615e-14\\
0.0510051	2.468672e-14\\
0.05110511	3.007115e-14\\
0.05120512	3.390928e-14\\
0.05130513	2.773678e-14\\
0.05140514	1.961141e-14\\
0.05150515	9.715114e-15\\
0.05160516	3.562603e-15\\
0.05170517	4.278515e-15\\
0.05180518	5.634511e-15\\
0.05190519	4.800554e-15\\
0.0520052	4.038082e-15\\
0.05210521	2.946022e-15\\
0.05220522	4.945819e-15\\
0.05230523	8.18096e-15\\
0.05240524	1.004106e-14\\
0.05250525	6.457853e-15\\
0.05260526	3.131125e-15\\
0.05270527	2.751887e-15\\
0.05280528	8.304417e-15\\
0.05290529	1.219947e-14\\
0.0530053	1.089678e-14\\
0.05310531	1.221936e-14\\
0.05320532	3.73614e-15\\
0.05330533	2.778024e-16\\
0.05340534	9.004666e-16\\
0.05350535	5.131722e-15\\
0.05360536	7.546117e-15\\
0.05370537	1.156568e-14\\
0.05380538	2.366535e-14\\
0.05390539	3.036654e-14\\
0.0540054	3.634431e-14\\
0.05410541	4.169066e-14\\
0.05420542	2.437512e-14\\
0.05430543	1.628011e-14\\
0.05440544	7.746832e-15\\
0.05450545	8.33806e-15\\
0.05460546	8.043338e-15\\
0.05470547	6.555132e-15\\
0.05480548	5.240663e-15\\
0.05490549	2.822891e-15\\
0.0550055	1.035007e-14\\
0.05510551	1.478899e-14\\
0.05520552	2.03394e-14\\
0.05530553	1.990546e-14\\
0.05540554	1.564382e-14\\
0.05550555	1.982851e-14\\
0.05560556	3.261757e-14\\
0.05570557	4.085775e-14\\
0.05580558	4.647121e-14\\
0.05590559	4.752756e-14\\
0.0560056	3.530919e-14\\
0.05610561	2.910571e-14\\
0.05620562	2.538305e-14\\
0.05630563	2.481019e-14\\
0.05640564	2.284208e-14\\
0.05650565	1.410108e-14\\
0.05660566	1.071122e-14\\
0.05670567	1.583631e-14\\
0.05680568	2.616608e-14\\
0.05690569	2.877337e-14\\
0.0570057	2.754808e-14\\
0.05710571	2.574093e-14\\
0.05720572	1.941226e-14\\
0.05730573	8.812546e-15\\
0.05740574	9.238845e-15\\
0.05750575	1.223061e-14\\
0.05760576	1.41327e-14\\
0.05770577	1.671273e-14\\
0.05780578	1.959078e-14\\
0.05790579	1.870992e-14\\
0.0580058	8.55303e-15\\
0.05810581	8.76147e-15\\
0.05820582	1.030085e-14\\
0.05830583	1.848607e-14\\
0.05840584	2.363639e-14\\
0.05850585	1.630695e-14\\
0.05860586	1.957726e-14\\
0.05870587	1.70274e-14\\
0.05880588	2.548358e-14\\
0.05890589	2.873397e-14\\
0.0590059	2.444215e-14\\
0.05910591	2.378606e-14\\
0.05920592	2.59608e-14\\
0.05930593	3.396155e-14\\
0.05940594	4.264335e-14\\
0.05950595	5.920884e-14\\
0.05960596	3.43221e-14\\
0.05970597	2.693124e-14\\
0.05980598	1.041816e-14\\
0.05990599	6.142929e-15\\
0.060006	2.924285e-15\\
0.06010601	3.319343e-15\\
0.06020602	5.687823e-16\\
0.06030603	1.398396e-15\\
0.06040604	4.393774e-15\\
0.06050605	7.208062e-15\\
0.06060606	8.83144e-15\\
0.06070607	4.016173e-15\\
0.06080608	1.762046e-15\\
0.06090609	9.444209e-16\\
0.0610061	1.574292e-15\\
0.06110611	3.718919e-15\\
0.06120612	1.608561e-15\\
0.06130613	1.298782e-15\\
0.06140614	2.297607e-15\\
0.06150615	2.278357e-15\\
0.06160616	3.221828e-15\\
0.06170617	1.04311e-15\\
0.06180618	1.530602e-15\\
0.06190619	3.223509e-15\\
0.0620062	3.047311e-15\\
0.06210621	2.25119e-15\\
0.06220622	1.024681e-15\\
0.06230623	2.782173e-16\\
0.06240624	1.611816e-15\\
0.06250625	2.175779e-15\\
0.06260626	1.238754e-15\\
0.06270627	2.536675e-15\\
0.06280628	4.834942e-15\\
0.06290629	7.250847e-15\\
0.0630063	7.402377e-15\\
0.06310631	7.400566e-15\\
0.06320632	7.22887e-16\\
0.06330633	7.401684e-17\\
0.06340634	1.389444e-15\\
0.06350635	3.029801e-15\\
0.06360636	2.52622e-15\\
0.06370637	4.893716e-15\\
0.06380638	1.435835e-14\\
0.06390639	2.07097e-14\\
0.0640064	1.591603e-14\\
0.06410641	1.318706e-14\\
0.06420642	1.35832e-14\\
0.06430643	2.148368e-14\\
0.06440644	3.042096e-14\\
0.06450645	3.833417e-14\\
0.06460646	2.818647e-14\\
0.06470647	2.392648e-14\\
0.06480648	1.119198e-14\\
0.06490649	1.118233e-14\\
0.0650065	8.065365e-15\\
0.06510651	6.166135e-15\\
0.06520652	9.671108e-16\\
0.06530653	3.391306e-15\\
0.06540654	8.545584e-15\\
0.06550655	1.279148e-14\\
0.06560656	1.441435e-14\\
0.06570657	9.657981e-15\\
0.06580658	7.171624e-15\\
0.06590659	7.848735e-15\\
0.0660066	4.34467e-15\\
0.06610661	5.449191e-15\\
0.06620662	2.688207e-15\\
0.06630663	9.405409e-17\\
0.06640664	1.168631e-15\\
0.06650665	3.690248e-15\\
0.06660666	6.197975e-15\\
0.06670667	7.89595e-15\\
0.06680668	7.463429e-15\\
0.06690669	2.811055e-15\\
0.0670067	3.924399e-15\\
0.06710671	9.498134e-15\\
0.06720672	1.906972e-14\\
0.06730673	1.726255e-14\\
0.06740674	1.476465e-14\\
0.06750675	1.719118e-14\\
0.06760676	2.781934e-14\\
0.06770677	3.535428e-14\\
0.06780678	3.556358e-14\\
0.06790679	2.444092e-14\\
0.0680068	1.59867e-14\\
0.06810681	1.163195e-14\\
0.06820682	1.258519e-14\\
0.06830683	9.496174e-15\\
0.06840684	9.387139e-15\\
0.06850685	1.246278e-14\\
0.06860686	1.916813e-14\\
0.06870687	3.060005e-14\\
0.06880688	2.989107e-14\\
0.06890689	2.027445e-14\\
0.0690069	1.198949e-14\\
0.06910691	7.079639e-15\\
0.06920692	5.676959e-15\\
0.06930693	7.87504e-15\\
0.06940694	5.915716e-15\\
0.06950695	2.441671e-15\\
0.06960696	1.708664e-16\\
0.06970697	1.318531e-15\\
0.06980698	1.233777e-15\\
0.06990699	3.149258e-15\\
0.070007	7.750539e-15\\
0.07010701	8.990115e-15\\
0.07020702	7.163372e-15\\
0.07030703	5.08254e-15\\
0.07040704	3.66085e-15\\
0.07050705	3.914818e-15\\
0.07060706	3.040987e-15\\
0.07070707	1.743086e-15\\
0.07080708	7.739459e-17\\
0.07090709	7.517932e-16\\
0.0710071	3.465241e-15\\
0.07110711	5.075276e-15\\
0.07120712	4.959327e-15\\
0.07130713	4.260037e-15\\
0.07140714	1.470998e-15\\
0.07150715	1.485197e-15\\
0.07160716	2.411021e-15\\
0.07170717	6.725184e-15\\
0.07180718	3.590018e-15\\
0.07190719	2.732767e-15\\
0.0720072	1.986882e-16\\
0.07210721	8.644567e-16\\
0.07220722	5.658615e-15\\
0.07230723	2.509556e-15\\
0.07240724	2.254868e-15\\
0.07250725	2.626901e-16\\
0.07260726	6.371856e-16\\
0.07270727	3.626938e-15\\
0.07280728	4.752386e-15\\
0.07290729	8.630138e-15\\
0.0730073	1.674314e-14\\
0.07310731	2.941974e-14\\
0.07320732	3.018386e-14\\
0.07330733	2.186554e-14\\
0.07340734	1.428992e-14\\
0.07350735	9.345983e-15\\
0.07360736	1.019584e-14\\
0.07370737	7.436776e-15\\
0.07380738	3.266683e-15\\
0.07390739	2.426061e-15\\
0.0740074	8.112418e-15\\
0.07410741	9.944023e-15\\
0.07420742	9.142254e-15\\
0.07430743	5.178305e-15\\
0.07440744	3.771099e-15\\
0.07450745	5.389111e-15\\
0.07460746	9.545843e-15\\
0.07470747	7.416665e-15\\
0.07480748	7.266389e-15\\
0.07490749	5.018969e-15\\
0.0750075	8.235718e-15\\
0.07510751	5.997483e-15\\
0.07520752	3.982927e-15\\
0.07530753	6.922229e-16\\
0.07540754	2.243826e-15\\
0.07550755	3.835789e-15\\
0.07560756	7.703243e-15\\
0.07570757	6.533095e-15\\
0.07580758	4.995896e-15\\
0.07590759	1.158957e-14\\
0.0760076	1.232087e-14\\
0.07610761	8.140123e-15\\
0.07620762	4.973147e-15\\
0.07630763	1.101255e-15\\
0.07640764	3.350129e-15\\
0.07650765	8.702456e-15\\
0.07660766	1.229851e-14\\
0.07670767	7.427546e-15\\
0.07680768	1.085414e-14\\
0.07690769	1.398677e-14\\
0.0770077	1.575111e-14\\
0.07710771	1.502161e-14\\
0.07720772	1.193751e-14\\
0.07730773	1.140899e-14\\
0.07740774	1.620129e-14\\
0.07750775	2.084849e-14\\
0.07760776	1.45967e-14\\
0.07770777	1.033238e-14\\
0.07780778	7.066863e-15\\
0.07790779	3.513222e-15\\
0.0780078	2.905744e-15\\
0.07810781	1.833326e-15\\
0.07820782	4.646209e-15\\
0.07830783	5.250164e-15\\
0.07840784	4.983456e-15\\
0.07850785	2.992977e-15\\
0.07860786	1.593353e-15\\
0.07870787	5.212159e-15\\
0.07880788	5.350537e-15\\
0.07890789	4.582577e-15\\
0.0790079	6.100939e-16\\
0.07910791	1.100119e-15\\
0.07920792	3.240871e-15\\
0.07930793	3.047849e-15\\
0.07940794	3.61217e-15\\
0.07950795	1.390903e-15\\
0.07960796	9.422442e-16\\
0.07970797	9.083915e-17\\
0.07980798	3.600537e-16\\
0.07990799	3.08651e-16\\
0.080008	5.119255e-16\\
0.08010801	9.767116e-16\\
0.08020802	3.292424e-15\\
0.08030803	4.27505e-15\\
0.08040804	1.072162e-14\\
0.08050805	6.392348e-15\\
0.08060806	2.84901e-15\\
0.08070807	3.316536e-15\\
0.08080808	5.627262e-15\\
0.08090809	4.011244e-15\\
0.0810081	4.601197e-15\\
0.08110811	2.18383e-16\\
0.08120812	8.574785e-16\\
0.08130813	4.798219e-15\\
0.08140814	8.084089e-15\\
0.08150815	7.930681e-15\\
0.08160816	5.065948e-15\\
0.08170817	1.075892e-14\\
0.08180818	1.055709e-14\\
0.08190819	1.835887e-14\\
0.0820082	1.474759e-14\\
0.08210821	1.96113e-14\\
0.08220822	2.664314e-14\\
0.08230823	3.029178e-14\\
0.08240824	2.910307e-14\\
0.08250825	1.876563e-14\\
0.08260826	1.256816e-14\\
0.08270827	1.174157e-14\\
0.08280828	7.166766e-15\\
0.08290829	2.721928e-15\\
0.0830083	4.575813e-15\\
0.08310831	8.255664e-15\\
0.08320832	1.11776e-14\\
0.08330833	4.251588e-15\\
0.08340834	2.180477e-15\\
0.08350835	4.279683e-15\\
0.08360836	9.426772e-15\\
0.08370837	4.787527e-15\\
0.08380838	3.607494e-15\\
0.08390839	5.511343e-15\\
0.0840084	7.674991e-15\\
0.08410841	5.268862e-15\\
0.08420842	2.145647e-15\\
0.08430843	3.74844e-16\\
0.08440844	2.188128e-15\\
0.08450845	5.006234e-15\\
0.08460846	4.392247e-15\\
0.08470847	5.460212e-15\\
0.08480848	6.160478e-15\\
0.08490849	4.641859e-15\\
0.0850085	3.223084e-15\\
0.08510851	6.101997e-16\\
0.08520852	1.029586e-15\\
0.08530853	1.662699e-15\\
0.08540854	4.204079e-15\\
0.08550855	3.65941e-15\\
0.08560856	9.376864e-15\\
0.08570857	1.500509e-14\\
0.08580858	7.824488e-15\\
0.08590859	5.390061e-15\\
0.0860086	6.834341e-15\\
0.08610861	1.135304e-14\\
0.08620862	1.500336e-14\\
0.08630863	9.148417e-15\\
0.08640864	4.564209e-15\\
0.08650865	2.850182e-15\\
0.08660866	3.80564e-15\\
0.08670867	2.023332e-15\\
0.08680868	1.634978e-15\\
0.08690869	3.669796e-15\\
0.0870087	6.35032e-15\\
0.08710871	4.782592e-15\\
0.08720872	1.362018e-15\\
0.08730873	1.272712e-15\\
0.08740874	8.126798e-16\\
0.08750875	1.332796e-16\\
0.08760876	5.415842e-17\\
0.08770877	1.477935e-15\\
0.08780878	1.413995e-15\\
0.08790879	2.813695e-16\\
0.0880088	2.67795e-16\\
0.08810881	8.427263e-16\\
0.08820882	1.246409e-15\\
0.08830883	1.195013e-16\\
0.08840884	7.053291e-16\\
0.08850885	2.564075e-16\\
0.08860886	9.977799e-16\\
0.08870887	2.278406e-16\\
0.08880888	2.564703e-15\\
0.08890889	5.486531e-15\\
0.0890089	6.918467e-15\\
0.08910891	3.566619e-15\\
0.08920892	3.22971e-15\\
0.08930893	1.88533e-15\\
0.08940894	4.099768e-15\\
0.08950895	5.140152e-15\\
0.08960896	3.051459e-15\\
0.08970897	3.482352e-15\\
0.08980898	2.152361e-15\\
0.08990899	4.159076e-15\\
0.090009	5.293284e-15\\
0.09010901	6.684681e-15\\
0.09020902	8.603395e-15\\
0.09030903	5.181357e-15\\
0.09040904	1.077591e-14\\
0.09050905	1.478757e-14\\
0.09060906	1.56958e-14\\
0.09070907	3.923979e-15\\
0.09080908	4.156706e-15\\
0.09090909	5.590733e-15\\
0.0910091	8.142698e-15\\
0.09110911	3.19485e-15\\
0.09120912	7.104333e-15\\
0.09130913	9.675723e-15\\
0.09140914	8.573541e-15\\
0.09150915	2.48349e-15\\
0.09160916	2.095708e-15\\
0.09170917	4.088019e-15\\
0.09180918	6.255739e-15\\
0.09190919	6.792664e-15\\
0.0920092	6.955869e-15\\
0.09210921	1.239942e-14\\
0.09220922	7.415099e-15\\
0.09230923	4.154252e-15\\
0.09240924	7.188549e-15\\
0.09250925	1.041416e-14\\
0.09260926	8.687483e-15\\
0.09270927	5.612015e-15\\
0.09280928	5.766622e-15\\
0.09290929	3.821414e-15\\
0.0930093	2.590911e-15\\
0.09310931	1.227938e-15\\
0.09320932	8.288325e-15\\
0.09330933	2.092363e-14\\
0.09340934	1.666433e-14\\
0.09350935	1.179364e-14\\
0.09360936	1.39987e-14\\
0.09370937	1.506501e-14\\
0.09380938	9.24352e-15\\
0.09390939	3.506403e-15\\
0.0940094	2.773136e-15\\
0.09410941	5.46217e-15\\
0.09420942	2.840234e-15\\
0.09430943	1.279626e-15\\
0.09440944	2.063988e-15\\
0.09450945	3.269659e-15\\
0.09460946	2.051866e-15\\
0.09470947	4.263444e-15\\
0.09480948	7.567094e-15\\
0.09490949	1.009635e-14\\
0.0950095	5.41355e-15\\
0.09510951	1.128393e-15\\
0.09520952	1.318651e-15\\
0.09530953	1.719786e-15\\
0.09540954	9.154488e-16\\
0.09550955	2.055189e-16\\
0.09560956	1.458576e-15\\
0.09570957	2.450802e-15\\
0.09580958	3.957021e-15\\
0.09590959	3.190184e-15\\
0.0960096	3.662328e-15\\
0.09610961	3.576053e-15\\
0.09620962	4.013429e-15\\
0.09630963	6.730054e-15\\
0.09640964	7.420676e-15\\
0.09650965	3.869408e-15\\
0.09660966	1.2556e-14\\
0.09670967	1.520756e-14\\
0.09680968	1.00375e-14\\
0.09690969	5.307621e-15\\
0.0970097	6.210745e-15\\
0.09710971	2.867709e-15\\
0.09720972	1.112358e-15\\
0.09730973	2.152467e-16\\
0.09740974	2.046352e-15\\
0.09750975	2.219068e-15\\
0.09760976	2.900851e-16\\
0.09770977	2.10303e-16\\
0.09780978	3.762441e-16\\
0.09790979	9.468945e-16\\
0.0980098	2.006173e-15\\
0.09810981	1.068991e-15\\
0.09820982	5.32253e-16\\
0.09830983	2.779805e-16\\
0.09840984	1.606866e-15\\
0.09850985	2.528456e-15\\
0.09860986	1.449721e-15\\
0.09870987	8.115685e-16\\
0.09880988	1.539688e-15\\
0.09890989	1.275288e-16\\
0.0990099	6.050939e-17\\
0.09910991	2.613166e-15\\
0.09920992	5.245465e-15\\
0.09930993	3.552679e-15\\
0.09940994	2.850212e-15\\
0.09950995	3.580722e-15\\
0.09960996	2.717284e-15\\
0.09970997	2.759823e-16\\
0.09980998	1.923615e-15\\
0.09990999	5.280833e-15\\
0.10001	4.820228e-15\\
0.10011	2.456074e-15\\
0.10021	9.00719e-15\\
0.10031	1.009478e-14\\
0.10041	1.305203e-14\\
0.1005101	1.197e-14\\
0.1006101	1.992274e-14\\
0.1007101	2.021371e-14\\
0.1008101	1.477355e-14\\
0.1009101	3.714665e-15\\
0.1010101	1.558801e-15\\
0.1011101	2.029384e-15\\
0.1012101	1.746926e-15\\
0.1013101	3.136201e-15\\
0.1014101	7.423978e-15\\
0.1015102	5.642083e-15\\
0.1016102	3.057232e-15\\
0.1017102	5.630522e-15\\
0.1018102	8.181124e-15\\
0.1019102	5.176252e-15\\
0.1020102	1.149319e-14\\
0.1021102	1.06844e-14\\
0.1022102	7.012049e-15\\
0.1023102	2.755663e-15\\
0.1024102	5.336286e-15\\
0.1025103	8.345238e-15\\
0.1026103	4.328549e-15\\
0.1027103	8.812379e-16\\
0.1028103	2.374686e-15\\
0.1029103	1.194311e-15\\
0.1030103	6.603373e-16\\
0.1031103	3.794628e-15\\
0.1032103	6.079442e-15\\
0.1033103	5.228277e-15\\
0.1034103	5.090611e-15\\
0.1035104	5.279857e-15\\
0.1036104	6.665737e-15\\
0.1037104	2.212649e-15\\
0.1038104	5.979379e-15\\
0.1039104	5.286531e-15\\
0.1040104	3.739787e-15\\
0.1041104	6.5936e-16\\
0.1042104	1.619301e-15\\
0.1043104	2.476388e-15\\
0.1044104	1.251631e-15\\
0.1045105	4.901858e-15\\
0.1046105	9.044837e-15\\
0.1047105	8.102035e-15\\
0.1048105	3.847241e-15\\
0.1049105	1.276604e-15\\
0.1050105	9.485287e-16\\
0.1051105	2.515518e-15\\
0.1052105	9.218653e-16\\
0.1053105	5.917189e-17\\
0.1054105	1.067866e-15\\
0.1055106	3.427482e-15\\
0.1056106	4.465572e-15\\
0.1057106	6.42843e-16\\
0.1058106	9.786581e-16\\
0.1059106	2.148465e-15\\
0.1060106	2.942918e-16\\
0.1061106	5.467379e-16\\
0.1062106	1.946533e-15\\
0.1063106	1.635967e-15\\
0.1064106	6.160534e-16\\
0.1065107	6.53247e-16\\
0.1066107	5.23261e-16\\
0.1067107	3.519134e-15\\
0.1068107	6.717417e-15\\
0.1069107	4.75391e-15\\
0.1070107	3.663782e-15\\
0.1071107	2.161688e-15\\
0.1072107	3.992742e-15\\
0.1073107	2.709768e-15\\
0.1074107	1.027445e-15\\
0.1075108	8.259179e-16\\
0.1076108	1.510068e-15\\
0.1077108	2.891355e-16\\
0.1078108	2.839271e-15\\
0.1079108	7.985709e-15\\
0.1080108	6.633978e-15\\
0.1081108	6.037224e-15\\
0.1082108	4.242658e-15\\
0.1083108	3.849128e-15\\
0.1084108	1.69557e-15\\
0.1085109	4.603901e-15\\
0.1086109	9.636095e-15\\
0.1087109	6.018107e-15\\
0.1088109	4.676727e-15\\
0.1089109	3.746462e-15\\
0.1090109	2.245909e-15\\
0.1091109	9.267135e-17\\
0.1092109	2.067712e-15\\
0.1093109	7.785435e-16\\
0.1094109	1.623151e-16\\
0.109511	1.099947e-15\\
0.109611	1.395228e-15\\
0.109711	7.01133e-16\\
0.109811	2.051062e-15\\
0.109911	2.909822e-15\\
0.110011	6.218901e-16\\
0.110111	7.649235e-16\\
0.110211	6.297075e-15\\
0.110311	9.168649e-15\\
0.110411	7.149816e-15\\
0.1105111	1.441037e-14\\
0.1106111	1.490987e-14\\
0.1107111	9.371241e-15\\
0.1108111	8.831367e-15\\
0.1109111	1.213451e-14\\
0.1110111	6.055846e-15\\
0.1111111	5.195453e-15\\
0.1112111	2.94524e-15\\
0.1113111	1.572642e-15\\
0.1114111	2.264742e-15\\
0.1115112	7.322303e-15\\
0.1116112	7.092878e-15\\
0.1117112	4.220584e-15\\
0.1118112	1.999466e-15\\
0.1119112	4.168113e-15\\
0.1120112	2.753799e-15\\
0.1121112	6.393715e-15\\
0.1122112	1.300689e-14\\
0.1123112	1.243613e-14\\
0.1124112	9.164543e-15\\
0.1125113	4.73605e-15\\
0.1126113	5.540187e-15\\
0.1127113	4.45929e-15\\
0.1128113	3.053181e-15\\
0.1129113	5.066356e-15\\
0.1130113	3.779563e-15\\
0.1131113	1.339374e-16\\
0.1132113	3.658974e-17\\
0.1133113	3.349814e-16\\
0.1134113	9.59602e-17\\
0.1135114	1.272974e-15\\
0.1136114	1.907724e-15\\
0.1137114	2.344339e-15\\
0.1138114	2.709472e-15\\
0.1139114	6.140032e-16\\
0.1140114	1.146305e-15\\
0.1141114	2.409356e-15\\
0.1142114	3.75391e-16\\
0.1143114	1.251602e-16\\
0.1144114	1.376171e-15\\
0.1145115	1.368166e-15\\
0.1146115	5.405492e-16\\
0.1147115	3.174228e-18\\
0.1148115	5.875563e-17\\
0.1149115	2.118239e-16\\
0.1150115	6.892102e-16\\
0.1151115	1.891245e-16\\
0.1152115	1.928035e-15\\
0.1153115	3.295921e-15\\
0.1154115	1.452358e-15\\
0.1155116	2.209884e-15\\
0.1156116	1.786121e-15\\
0.1157116	4.434211e-16\\
0.1158116	3.983942e-16\\
0.1159116	1.628803e-15\\
0.1160116	1.552766e-15\\
0.1161116	8.166322e-16\\
0.1162116	6.464714e-15\\
0.1163116	6.005131e-15\\
0.1164116	7.108585e-15\\
0.1165117	1.288933e-14\\
0.1166117	1.714369e-14\\
0.1167117	1.119616e-14\\
0.1168117	3.190975e-15\\
0.1169117	1.235993e-15\\
0.1170117	9.518179e-16\\
0.1171117	6.044459e-16\\
0.1172117	2.000358e-15\\
0.1173117	1.565618e-15\\
0.1174117	5.746848e-16\\
0.1175118	2.922507e-15\\
0.1176118	4.153199e-15\\
0.1177118	3.881587e-15\\
0.1178118	1.015224e-14\\
0.1179118	4.564005e-15\\
0.1180118	7.230058e-15\\
0.1181118	8.132082e-15\\
0.1182118	1.150484e-14\\
0.1183118	7.725976e-15\\
0.1184118	8.663711e-15\\
0.1185119	3.482819e-15\\
0.1186119	8.891647e-16\\
0.1187119	2.230375e-15\\
0.1188119	4.009538e-15\\
0.1189119	3.215594e-15\\
0.1190119	6.264079e-16\\
0.1191119	9.462054e-16\\
0.1192119	6.351822e-16\\
0.1193119	3.751628e-16\\
0.1194119	5.24737e-16\\
0.119512	1.834728e-16\\
0.119612	2.277798e-15\\
0.119712	3.797008e-15\\
0.119812	5.279615e-15\\
0.119912	4.496837e-15\\
0.120012	7.97842e-15\\
0.120112	1.775115e-15\\
0.120212	9.858062e-17\\
0.120312	1.816123e-15\\
0.120412	1.262398e-15\\
0.1205121	1.733311e-15\\
0.1206121	1.682834e-15\\
0.1207121	2.636443e-16\\
0.1208121	1.488491e-15\\
0.1209121	2.78331e-15\\
0.1210121	1.142219e-15\\
0.1211121	3.310814e-15\\
0.1212121	3.155621e-15\\
0.1213121	2.211346e-15\\
0.1214121	2.394771e-15\\
0.1215122	5.456338e-15\\
0.1216122	4.13436e-15\\
0.1217122	1.167567e-16\\
0.1218122	5.46022e-17\\
0.1219122	2.480723e-16\\
0.1220122	2.053953e-15\\
0.1221122	6.582837e-15\\
0.1222122	2.771286e-15\\
0.1223122	2.123337e-15\\
0.1224122	1.354705e-15\\
0.1225123	2.847927e-15\\
0.1226123	5.324103e-16\\
0.1227123	9.582122e-16\\
0.1228123	1.011163e-15\\
0.1229123	1.096817e-15\\
0.1230123	4.560874e-16\\
0.1231123	8.069727e-16\\
0.1232123	2.294174e-16\\
0.1233123	4.970726e-16\\
0.1234123	1.188876e-16\\
0.1235124	1.279604e-15\\
0.1236124	1.552303e-15\\
0.1237124	7.842203e-16\\
0.1238124	4.518884e-16\\
0.1239124	1.76677e-15\\
0.1240124	4.777673e-16\\
0.1241124	1.447628e-16\\
0.1242124	1.717837e-15\\
0.1243124	3.30989e-15\\
0.1244124	2.139844e-15\\
0.1245125	1.543148e-15\\
0.1246125	1.99719e-15\\
0.1247125	3.732531e-15\\
0.1248125	7.379135e-15\\
0.1249125	1.437743e-14\\
0.1250125	2.201162e-14\\
0.1251125	1.657326e-14\\
0.1252125	1.617314e-14\\
0.1253125	1.107405e-14\\
0.1254125	5.549958e-15\\
0.1255126	6.533907e-15\\
0.1256126	9.424486e-15\\
0.1257126	2.764069e-15\\
0.1258126	2.573639e-15\\
0.1259126	1.470796e-15\\
0.1260126	1.293699e-15\\
0.1261126	3.701524e-15\\
0.1262126	3.151243e-15\\
0.1263126	3.07563e-16\\
0.1264126	2.598998e-15\\
0.1265127	4.424968e-15\\
0.1266127	3.411187e-15\\
0.1267127	6.243832e-15\\
0.1268127	8.338769e-15\\
0.1269127	4.874938e-16\\
0.1270127	1.806249e-15\\
0.1271127	1.227412e-15\\
0.1272127	5.624055e-16\\
0.1273127	2.139026e-15\\
0.1274127	1.193113e-15\\
0.1275128	2.709897e-15\\
0.1276128	4.24192e-15\\
0.1277128	1.90515e-15\\
0.1278128	2.775559e-17\\
0.1279128	2.7504e-16\\
0.1280128	3.831595e-15\\
0.1281128	8.423131e-15\\
0.1282128	8.394911e-15\\
0.1283128	4.096766e-15\\
0.1284128	2.547051e-15\\
0.1285129	2.827734e-15\\
0.1286129	7.522463e-16\\
0.1287129	2.871188e-15\\
0.1288129	2.324277e-15\\
0.1289129	9.034881e-16\\
0.1290129	2.373691e-15\\
0.1291129	2.89797e-15\\
0.1292129	1.634134e-15\\
0.1293129	4.811805e-15\\
0.1294129	1.665076e-15\\
0.129513	6.580745e-16\\
0.129613	3.332248e-15\\
0.129713	2.630578e-15\\
0.129813	2.704919e-15\\
0.129913	4.450766e-15\\
0.130013	1.152348e-15\\
0.130113	3.88431e-17\\
0.130213	2.86574e-16\\
0.130313	6.744138e-16\\
0.130413	1.072946e-15\\
0.1305131	3.224241e-15\\
0.1306131	2.49198e-15\\
0.1307131	4.25435e-15\\
0.1308131	3.337535e-15\\
0.1309131	2.214955e-16\\
0.1310131	2.132191e-17\\
0.1311131	8.116986e-16\\
0.1312131	1.402384e-15\\
0.1313131	3.268373e-16\\
0.1314131	1.531822e-15\\
0.1315132	4.456075e-15\\
0.1316132	1.208193e-15\\
0.1317132	2.845569e-15\\
0.1318132	2.084815e-15\\
0.1319132	7.371335e-16\\
0.1320132	2.300571e-15\\
0.1321132	2.750001e-15\\
0.1322132	6.504076e-16\\
0.1323132	2.521014e-16\\
0.1324132	7.222937e-19\\
0.1325133	2.622276e-16\\
0.1326133	1.473776e-15\\
0.1327133	3.412544e-15\\
0.1328133	3.815283e-15\\
0.1329133	2.676758e-15\\
0.1330133	3.249006e-15\\
0.1331133	3.527411e-15\\
0.1332133	2.572822e-16\\
0.1333133	1.368843e-15\\
0.1334133	2.108211e-15\\
0.1335134	1.703418e-15\\
0.1336134	1.830355e-16\\
0.1337134	2.617815e-16\\
0.1338134	1.072063e-15\\
0.1339134	2.080579e-15\\
0.1340134	3.994146e-15\\
0.1341134	4.745504e-16\\
0.1342134	1.535788e-15\\
0.1343134	5.566163e-16\\
0.1344134	2.154478e-16\\
0.1345135	2.104042e-15\\
0.1346135	6.744259e-16\\
0.1347135	6.001391e-16\\
0.1348135	5.63502e-16\\
0.1349135	1.002286e-15\\
0.1350135	4.162778e-15\\
0.1351135	2.568083e-15\\
0.1352135	8.656564e-15\\
0.1353135	1.361558e-14\\
0.1354135	8.944023e-15\\
0.1355136	4.337088e-15\\
0.1356136	4.803535e-15\\
0.1357136	1.296513e-15\\
0.1358136	6.675339e-15\\
0.1359136	5.174661e-15\\
0.1360136	2.351573e-15\\
0.1361136	3.552036e-15\\
0.1362136	2.084757e-15\\
0.1363136	1.489411e-15\\
0.1364136	4.958365e-15\\
0.1365137	2.258056e-15\\
0.1366137	2.842601e-16\\
0.1367137	1.551228e-15\\
0.1368137	3.467385e-15\\
0.1369137	5.465453e-15\\
0.1370137	6.231201e-15\\
0.1371137	1.743366e-15\\
0.1372137	3.456105e-15\\
0.1373137	5.260191e-15\\
0.1374137	6.558899e-15\\
0.1375138	4.672999e-15\\
0.1376138	9.071184e-16\\
0.1377138	1.26554e-15\\
0.1378138	6.163103e-16\\
0.1379138	1.843964e-15\\
0.1380138	3.942537e-15\\
0.1381138	3.865525e-15\\
0.1382138	9.928299e-15\\
0.1383138	1.169717e-14\\
0.1384138	6.095891e-15\\
0.1385139	3.363015e-15\\
0.1386139	2.818726e-15\\
0.1387139	9.69927e-16\\
0.1388139	6.99429e-16\\
0.1389139	1.346281e-15\\
0.1390139	1.400715e-15\\
0.1391139	3.666905e-16\\
0.1392139	2.026633e-15\\
0.1393139	1.409338e-15\\
0.1394139	3.116913e-15\\
0.139514	2.447539e-15\\
0.139614	3.841405e-16\\
0.139714	1.579774e-16\\
0.139814	1.71889e-16\\
0.139914	6.793155e-17\\
0.140014	1.730334e-15\\
0.140114	1.870095e-15\\
0.140214	9.854576e-16\\
0.140314	5.592201e-16\\
0.140414	1.433359e-15\\
0.1405141	1.454034e-16\\
0.1406141	1.715684e-15\\
0.1407141	8.512558e-16\\
0.1408141	1.40962e-15\\
0.1409141	2.053231e-15\\
0.1410141	1.802199e-15\\
0.1411141	1.360623e-15\\
0.1412141	8.563345e-16\\
0.1413141	9.587604e-16\\
0.1414141	4.050192e-15\\
0.1415142	2.231821e-15\\
0.1416142	1.519434e-15\\
0.1417142	1.184998e-15\\
0.1418142	1.097174e-16\\
0.1419142	2.361675e-16\\
0.1420142	1.669176e-15\\
0.1421142	3.915415e-15\\
0.1422142	3.562549e-15\\
0.1423142	7.536765e-15\\
0.1424142	5.80236e-15\\
0.1425143	7.542866e-15\\
0.1426143	1.512967e-14\\
0.1427143	9.232795e-15\\
0.1428143	5.967243e-15\\
0.1429143	5.794859e-15\\
0.1430143	8.736524e-16\\
0.1431143	7.758412e-16\\
0.1432143	1.233551e-15\\
0.1433143	4.036476e-16\\
0.1434143	1.491346e-15\\
0.1435144	1.670782e-15\\
0.1436144	2.106259e-15\\
0.1437144	5.841508e-15\\
0.1438144	3.713808e-15\\
0.1439144	4.251684e-16\\
0.1440144	4.638243e-16\\
0.1441144	1.746445e-15\\
0.1442144	4.205412e-15\\
0.1443144	8.664234e-16\\
0.1444144	2.169834e-15\\
0.1445145	1.273677e-15\\
0.1446145	1.462722e-15\\
0.1447145	7.241021e-16\\
0.1448145	9.058681e-16\\
0.1449145	6.561929e-15\\
0.1450145	4.123846e-15\\
0.1451145	4.379861e-15\\
0.1452145	3.957486e-15\\
0.1453145	5.211472e-15\\
0.1454145	1.408378e-14\\
0.1455146	7.535692e-15\\
0.1456146	3.347192e-16\\
0.1457146	1.20786e-15\\
0.1458146	6.509556e-16\\
0.1459146	9.922259e-16\\
0.1460146	3.110148e-16\\
0.1461146	2.84113e-15\\
0.1462146	1.164679e-15\\
0.1463146	2.590435e-15\\
0.1464146	1.141615e-15\\
0.1465147	1.277314e-15\\
0.1466147	3.381087e-16\\
0.1467147	2.441036e-15\\
0.1468147	1.559958e-15\\
0.1469147	1.984549e-15\\
0.1470147	2.510574e-15\\
0.1471147	3.949595e-15\\
0.1472147	9.742242e-15\\
0.1473147	6.44195e-15\\
0.1474147	7.321485e-15\\
0.1475148	8.423961e-15\\
0.1476148	4.139505e-15\\
0.1477148	7.881009e-15\\
0.1478148	1.91445e-15\\
0.1479148	4.373142e-16\\
0.1480148	7.980534e-16\\
0.1481148	4.496777e-16\\
0.1482148	6.676228e-16\\
0.1483148	6.551618e-16\\
0.1484148	5.016369e-15\\
0.1485149	1.488048e-15\\
0.1486149	2.277335e-15\\
0.1487149	1.211172e-15\\
0.1488149	3.297643e-17\\
0.1489149	2.552525e-16\\
0.1490149	3.500131e-15\\
0.1491149	4.539062e-15\\
0.1492149	5.970445e-15\\
0.1493149	7.21108e-15\\
0.1494149	1.502426e-15\\
0.149515	4.109373e-15\\
0.149615	3.655837e-15\\
0.149715	2.503181e-15\\
0.149815	1.038088e-15\\
0.149915	4.702193e-16\\
0.150015	1.781903e-15\\
0.150115	5.665464e-17\\
0.150215	1.792694e-15\\
0.150315	6.919e-16\\
0.150415	2.740696e-16\\
0.1505151	3.428895e-16\\
0.1506151	2.427448e-15\\
0.1507151	4.784884e-15\\
0.1508151	3.85921e-16\\
0.1509151	4.196348e-16\\
0.1510151	8.329788e-16\\
0.1511151	4.189036e-15\\
0.1512151	3.122511e-15\\
0.1513151	9.450729e-15\\
0.1514151	9.433704e-15\\
0.1515152	7.350291e-15\\
0.1516152	4.126148e-15\\
0.1517152	3.857837e-15\\
0.1518152	1.150689e-14\\
0.1519152	3.325146e-15\\
0.1520152	1.891531e-15\\
0.1521152	1.11859e-15\\
0.1522152	2.816485e-15\\
0.1523152	1.24193e-15\\
0.1524152	3.712175e-16\\
0.1525153	3.057804e-15\\
0.1526153	2.162606e-15\\
0.1527153	6.945192e-15\\
0.1528153	2.009007e-15\\
0.1529153	4.084359e-15\\
0.1530153	3.184177e-15\\
0.1531153	1.944405e-16\\
0.1532153	1.666782e-16\\
0.1533153	3.355887e-15\\
0.1534153	3.429823e-15\\
0.1535154	3.614935e-16\\
0.1536154	5.807361e-16\\
0.1537154	1.219821e-15\\
0.1538154	1.544436e-15\\
0.1539154	3.474226e-15\\
0.1540154	7.579053e-15\\
0.1541154	5.949915e-15\\
0.1542154	7.160279e-15\\
0.1543154	6.137385e-15\\
0.1544154	5.638516e-15\\
0.1545155	2.577592e-15\\
0.1546155	1.168584e-15\\
0.1547155	4.318257e-15\\
0.1548155	6.608801e-15\\
0.1549155	1.360809e-14\\
0.1550155	7.549588e-15\\
0.1551155	1.174322e-14\\
0.1552155	1.08611e-14\\
0.1553155	9.19659e-15\\
0.1554155	1.063848e-14\\
0.1555156	6.426603e-15\\
0.1556156	8.457923e-15\\
0.1557156	2.100668e-15\\
0.1558156	1.130988e-15\\
0.1559156	1.189366e-15\\
0.1560156	1.812714e-15\\
0.1561156	2.608819e-16\\
0.1562156	4.611666e-15\\
0.1563156	3.032229e-15\\
0.1564156	4.802861e-15\\
0.1565157	2.782438e-15\\
0.1566157	7.910682e-15\\
0.1567157	5.772037e-15\\
0.1568157	2.064635e-15\\
0.1569157	4.537297e-15\\
0.1570157	2.040191e-15\\
0.1571157	2.937355e-15\\
0.1572157	5.631959e-16\\
0.1573157	9.433066e-15\\
0.1574157	6.478728e-15\\
0.1575158	4.262175e-15\\
0.1576158	1.197394e-15\\
0.1577158	4.132358e-15\\
0.1578158	5.332396e-15\\
0.1579158	5.853433e-15\\
0.1580158	1.467163e-14\\
0.1581158	9.880937e-15\\
0.1582158	8.969643e-15\\
0.1583158	1.861939e-15\\
0.1584158	5.198156e-15\\
0.1585159	1.962069e-15\\
0.1586159	7.48422e-16\\
0.1587159	2.686397e-15\\
0.1588159	7.023519e-15\\
0.1589159	3.602494e-15\\
0.1590159	5.693953e-15\\
0.1591159	6.979302e-15\\
0.1592159	6.975818e-15\\
0.1593159	8.688605e-15\\
0.1594159	9.018642e-15\\
0.159516	1.449731e-14\\
0.159616	3.384287e-15\\
0.159716	7.165659e-15\\
0.159816	1.259493e-14\\
0.159916	1.702921e-14\\
0.160016	1.293703e-14\\
0.160116	5.625289e-15\\
0.160216	2.031758e-15\\
0.160316	5.120391e-16\\
0.160416	1.25959e-16\\
0.1605161	6.419633e-15\\
0.1606161	4.43233e-15\\
0.1607161	1.729378e-15\\
0.1608161	2.819174e-15\\
0.1609161	7.273436e-15\\
0.1610161	8.339204e-15\\
0.1611161	7.082602e-15\\
0.1612161	1.57735e-14\\
0.1613161	1.939723e-14\\
0.1614161	2.522151e-14\\
0.1615162	1.27219e-14\\
0.1616162	1.687579e-14\\
0.1617162	8.701797e-15\\
0.1618162	1.041773e-14\\
0.1619162	9.81837e-15\\
0.1620162	1.395648e-14\\
0.1621162	1.002682e-14\\
0.1622162	2.893404e-15\\
0.1623162	9.144162e-15\\
0.1624162	1.108557e-14\\
0.1625163	1.497106e-14\\
0.1626163	7.392532e-15\\
0.1627163	1.367267e-14\\
0.1628163	5.27916e-15\\
0.1629163	7.804983e-15\\
0.1630163	6.357188e-15\\
0.1631163	3.338815e-15\\
0.1632163	1.20449e-16\\
0.1633163	6.021387e-15\\
0.1634163	9.567914e-15\\
0.1635164	1.221814e-14\\
0.1636164	1.330567e-14\\
0.1637164	2.749088e-14\\
0.1638164	1.686832e-14\\
0.1639164	1.408857e-14\\
0.1640164	4.111007e-15\\
0.1641164	4.814651e-15\\
0.1642164	2.883364e-15\\
0.1643164	1.432783e-14\\
0.1644164	2.096458e-14\\
0.1645165	1.41345e-14\\
0.1646165	1.383009e-14\\
0.1647165	1.460787e-14\\
0.1648165	2.654831e-14\\
0.1649165	1.62084e-14\\
0.1650165	1.627352e-14\\
0.1651165	1.254288e-14\\
0.1652165	1.986066e-14\\
0.1653165	6.401338e-15\\
0.1654165	1.32259e-14\\
0.1655166	1.241832e-14\\
0.1656166	1.237506e-14\\
0.1657166	8.469782e-15\\
0.1658166	2.162948e-14\\
0.1659166	1.394449e-14\\
0.1660166	1.172358e-14\\
0.1661166	6.793008e-15\\
0.1662166	1.865647e-14\\
0.1663166	1.541255e-14\\
0.1664166	1.303895e-14\\
0.1665167	5.607713e-15\\
0.1666167	8.150325e-15\\
0.1667167	1.093682e-14\\
0.1668167	2.82821e-14\\
0.1669167	2.543428e-14\\
0.1670167	1.587302e-14\\
0.1671167	1.322202e-14\\
0.1672167	2.202725e-14\\
0.1673167	2.482499e-14\\
0.1674167	1.790043e-14\\
0.1675168	2.346118e-14\\
0.1676168	1.403873e-14\\
0.1677168	8.352092e-15\\
0.1678168	1.301715e-14\\
0.1679168	1.585923e-14\\
0.1680168	1.041448e-14\\
0.1681168	2.102037e-14\\
0.1682168	2.387995e-14\\
0.1683168	2.43855e-14\\
0.1684168	7.691068e-15\\
0.1685169	1.490572e-14\\
0.1686169	2.759582e-14\\
0.1687169	3.506503e-14\\
0.1688169	3.194525e-14\\
0.1689169	4.410338e-14\\
0.1690169	2.893838e-14\\
0.1691169	2.507857e-14\\
0.1692169	2.506108e-14\\
0.1693169	2.79173e-14\\
0.1694169	1.51146e-14\\
0.169517	2.506294e-14\\
0.169617	2.790203e-14\\
0.169717	4.086186e-14\\
0.169817	2.515566e-14\\
0.169917	3.125423e-14\\
0.170017	2.1501e-14\\
0.170117	1.866394e-14\\
0.170217	9.110447e-15\\
0.170317	1.043534e-14\\
0.170417	1.62522e-14\\
0.1705171	3.237433e-14\\
0.1706171	3.163728e-14\\
0.1707171	2.827926e-14\\
0.1708171	1.743219e-14\\
0.1709171	2.264751e-14\\
0.1710171	2.920341e-14\\
0.1711171	2.959617e-14\\
0.1712171	2.841602e-14\\
0.1713171	3.379288e-14\\
0.1714171	2.583238e-14\\
0.1715172	4.78522e-14\\
0.1716172	5.110021e-14\\
0.1717172	4.960811e-14\\
0.1718172	4.274325e-14\\
0.1719172	5.013845e-14\\
0.1720172	4.84041e-14\\
0.1721172	3.325907e-14\\
0.1722172	3.340097e-14\\
0.1723172	3.75411e-14\\
0.1724172	3.625645e-14\\
0.1725173	4.361058e-14\\
0.1726173	4.586248e-14\\
0.1727173	3.144315e-14\\
0.1728173	2.944526e-14\\
0.1729173	4.204722e-14\\
0.1730173	3.926793e-14\\
0.1731173	1.978841e-14\\
0.1732173	1.832179e-14\\
0.1733173	2.957273e-14\\
0.1734173	3.89138e-14\\
0.1735174	4.135985e-14\\
0.1736174	5.123255e-14\\
0.1737174	4.572515e-14\\
0.1738174	4.464274e-14\\
0.1739174	7.272255e-14\\
0.1740174	7.249925e-14\\
0.1741174	6.81802e-14\\
0.1742174	5.793608e-14\\
0.1743174	5.367768e-14\\
0.1744174	3.524818e-14\\
0.1745175	2.104899e-14\\
0.1746175	2.850786e-14\\
0.1747175	4.244503e-14\\
0.1748175	5.525857e-14\\
0.1749175	6.639647e-14\\
0.1750175	6.087821e-14\\
0.1751175	5.053883e-14\\
0.1752175	7.480867e-14\\
0.1753175	7.063189e-14\\
0.1754175	6.434558e-14\\
0.1755176	6.41403e-14\\
0.1756176	6.542362e-14\\
0.1757176	6.244644e-14\\
0.1758176	5.177149e-14\\
0.1759176	6.269076e-14\\
0.1760176	6.130529e-14\\
0.1761176	5.63287e-14\\
0.1762176	5.930604e-14\\
0.1763176	6.333581e-14\\
0.1764176	5.188254e-14\\
0.1765177	7.510682e-14\\
0.1766177	8.094159e-14\\
0.1767177	7.102433e-14\\
0.1768177	6.445374e-14\\
0.1769177	7.67402e-14\\
0.1770177	7.845778e-14\\
0.1771177	7.767305e-14\\
0.1772177	7.026888e-14\\
0.1773177	5.75423e-14\\
0.1774177	5.223632e-14\\
0.1775178	8.548184e-14\\
0.1776178	1.032169e-13\\
0.1777178	7.480031e-14\\
0.1778178	7.52087e-14\\
0.1779178	6.578425e-14\\
0.1780178	5.74015e-14\\
0.1781178	7.232011e-14\\
0.1782178	7.543477e-14\\
0.1783178	9.001751e-14\\
0.1784178	1.020852e-13\\
0.1785179	1.093136e-13\\
0.1786179	9.639462e-14\\
0.1787179	9.211909e-14\\
0.1788179	9.976642e-14\\
0.1789179	1.166767e-13\\
0.1790179	1.07407e-13\\
0.1791179	1.01432e-13\\
0.1792179	1.116967e-13\\
0.1793179	1.119827e-13\\
0.1794179	1.225728e-13\\
0.179518	1.212262e-13\\
0.179618	9.064354e-14\\
0.179718	1.077209e-13\\
0.179818	1.322646e-13\\
0.179918	1.372652e-13\\
0.180018	1.048421e-13\\
0.180118	1.035871e-13\\
0.180218	8.972207e-14\\
0.180318	7.44086e-14\\
0.180418	8.726616e-14\\
0.1805181	1.047439e-13\\
0.1806181	1.034832e-13\\
0.1807181	1.330687e-13\\
0.1808181	1.203827e-13\\
0.1809181	1.074161e-13\\
0.1810181	1.183623e-13\\
0.1811181	1.700678e-13\\
0.1812181	1.414675e-13\\
0.1813181	1.319636e-13\\
0.1814181	1.460519e-13\\
0.1815182	1.561679e-13\\
0.1816182	1.663331e-13\\
0.1817182	1.702459e-13\\
0.1818182	1.741912e-13\\
0.1819182	1.509215e-13\\
0.1820182	1.465072e-13\\
0.1821182	1.385229e-13\\
0.1822182	1.33216e-13\\
0.1823182	1.61876e-13\\
0.1824182	1.70287e-13\\
0.1825183	1.717592e-13\\
0.1826183	1.877214e-13\\
0.1827183	1.741292e-13\\
0.1828183	1.599864e-13\\
0.1829183	1.661475e-13\\
0.1830183	1.535581e-13\\
0.1831183	1.223379e-13\\
0.1832183	1.660003e-13\\
0.1833183	1.998307e-13\\
0.1834183	2.162149e-13\\
0.1835184	1.982949e-13\\
0.1836184	1.974965e-13\\
0.1837184	1.814215e-13\\
0.1838184	1.963064e-13\\
0.1839184	2.068643e-13\\
0.1840184	2.121667e-13\\
0.1841184	1.866295e-13\\
0.1842184	2.188468e-13\\
0.1843184	2.061934e-13\\
0.1844184	1.86493e-13\\
0.1845185	1.972117e-13\\
0.1846185	2.464599e-13\\
0.1847185	2.21805e-13\\
0.1848185	2.362134e-13\\
0.1849185	2.266569e-13\\
0.1850185	2.184822e-13\\
0.1851185	2.325658e-13\\
0.1852185	2.493139e-13\\
0.1853185	2.245888e-13\\
0.1854185	2.147747e-13\\
0.1855186	2.44801e-13\\
0.1856186	2.42865e-13\\
0.1857186	2.369154e-13\\
0.1858186	2.856736e-13\\
0.1859186	2.820176e-13\\
0.1860186	3.192463e-13\\
0.1861186	3.469313e-13\\
0.1862186	3.298967e-13\\
0.1863186	2.995605e-13\\
0.1864186	2.936022e-13\\
0.1865187	2.771485e-13\\
0.1866187	2.573229e-13\\
0.1867187	3.000449e-13\\
0.1868187	2.995975e-13\\
0.1869187	3.036859e-13\\
0.1870187	2.836225e-13\\
0.1871187	2.834887e-13\\
0.1872187	3.162647e-13\\
0.1873187	3.684567e-13\\
0.1874187	3.569527e-13\\
0.1875188	3.095831e-13\\
0.1876188	2.899114e-13\\
0.1877188	3.250955e-13\\
0.1878188	2.842467e-13\\
0.1879188	3.269163e-13\\
0.1880188	3.803925e-13\\
0.1881188	4.03086e-13\\
0.1882188	4.207182e-13\\
0.1883188	4.062031e-13\\
0.1884188	3.706329e-13\\
0.1885189	3.859089e-13\\
0.1886189	3.883885e-13\\
0.1887189	3.529743e-13\\
0.1888189	3.668511e-13\\
0.1889189	4.460648e-13\\
0.1890189	4.427948e-13\\
0.1891189	4.807736e-13\\
0.1892189	4.29353e-13\\
0.1893189	4.333566e-13\\
0.1894189	4.626354e-13\\
0.189519	5.285862e-13\\
0.189619	4.945129e-13\\
0.189719	4.558888e-13\\
0.189819	4.83616e-13\\
0.189919	4.991146e-13\\
0.190019	4.65322e-13\\
0.190119	5.023363e-13\\
0.190219	4.659655e-13\\
0.190319	4.53038e-13\\
0.190419	4.65258e-13\\
0.1905191	4.468889e-13\\
0.1906191	4.64986e-13\\
0.1907191	5.080804e-13\\
0.1908191	5.351178e-13\\
0.1909191	5.192857e-13\\
0.1910191	5.575891e-13\\
0.1911191	5.757148e-13\\
0.1912191	6.200986e-13\\
0.1913191	6.45839e-13\\
0.1914191	6.276635e-13\\
0.1915192	6.331803e-13\\
0.1916192	6.456278e-13\\
0.1917192	6.340708e-13\\
0.1918192	5.81522e-13\\
0.1919192	6.069987e-13\\
0.1920192	6.549948e-13\\
0.1921192	6.251275e-13\\
0.1922192	6.814686e-13\\
0.1923192	6.684894e-13\\
0.1924192	6.622423e-13\\
0.1925193	6.68119e-13\\
0.1926193	6.740159e-13\\
0.1927193	7.04387e-13\\
0.1928193	8.038897e-13\\
0.1929193	7.707017e-13\\
0.1930193	7.270174e-13\\
0.1931193	7.331544e-13\\
0.1932193	7.672497e-13\\
0.1933193	7.370509e-13\\
0.1934193	7.82058e-13\\
0.1935194	7.747986e-13\\
0.1936194	7.975548e-13\\
0.1937194	8.600523e-13\\
0.1938194	8.678966e-13\\
0.1939194	8.772744e-13\\
0.1940194	8.906894e-13\\
0.1941194	9.363372e-13\\
0.1942194	8.798375e-13\\
0.1943194	8.769767e-13\\
0.1944194	8.895371e-13\\
0.1945195	9.287043e-13\\
0.1946195	9.138826e-13\\
0.1947195	8.391994e-13\\
0.1948195	8.434309e-13\\
0.1949195	9.479628e-13\\
0.1950195	9.540859e-13\\
0.1951195	9.498266e-13\\
0.1952195	1.01039e-12\\
0.1953195	1.127521e-12\\
0.1954195	1.122487e-12\\
0.1955196	1.157497e-12\\
0.1956196	1.081287e-12\\
0.1957196	9.954571e-13\\
0.1958196	1.078326e-12\\
0.1959196	1.076997e-12\\
0.1960196	1.142767e-12\\
0.1961196	1.292433e-12\\
0.1962196	1.252489e-12\\
0.1963196	1.237933e-12\\
0.1964196	1.255505e-12\\
0.1965197	1.26862e-12\\
0.1966197	1.228298e-12\\
0.1967197	1.291482e-12\\
0.1968197	1.243641e-12\\
0.1969197	1.233138e-12\\
0.1970197	1.311845e-12\\
0.1971197	1.304484e-12\\
0.1972197	1.342727e-12\\
0.1973197	1.379688e-12\\
0.1974197	1.42658e-12\\
0.1975198	1.415796e-12\\
0.1976198	1.384108e-12\\
0.1977198	1.379852e-12\\
0.1978198	1.388778e-12\\
0.1979198	1.406361e-12\\
0.1980198	1.441991e-12\\
0.1981198	1.514625e-12\\
0.1982198	1.681365e-12\\
0.1983198	1.576824e-12\\
0.1984198	1.589457e-12\\
0.1985199	1.628681e-12\\
0.1986199	1.677737e-12\\
0.1987199	1.74032e-12\\
0.1988199	1.749595e-12\\
0.1989199	1.721181e-12\\
0.1990199	1.764475e-12\\
0.1991199	1.848001e-12\\
0.1992199	1.759869e-12\\
0.1993199	1.816976e-12\\
0.1994199	1.853618e-12\\
0.19952	1.813522e-12\\
0.19962	1.874112e-12\\
0.19972	1.951328e-12\\
0.19982	1.997183e-12\\
0.19992	2.107749e-12\\
0.20002	2.049617e-12\\
0.20012	1.998602e-12\\
0.20022	2.014047e-12\\
0.20032	2.086711e-12\\
0.20042	2.066764e-12\\
0.2005201	2.125451e-12\\
0.2006201	2.184308e-12\\
0.2007201	2.246932e-12\\
0.2008201	2.303886e-12\\
0.2009201	2.194712e-12\\
0.2010201	2.174441e-12\\
0.2011201	2.34488e-12\\
0.2012201	2.401483e-12\\
0.2013201	2.450177e-12\\
0.2014201	2.487352e-12\\
0.2015202	2.543127e-12\\
0.2016202	2.451125e-12\\
0.2017202	2.519056e-12\\
0.2018202	2.581056e-12\\
0.2019202	2.67757e-12\\
0.2020202	2.736564e-12\\
0.2021202	2.660977e-12\\
0.2022202	2.750537e-12\\
0.2023202	2.862105e-12\\
0.2024202	2.887292e-12\\
0.2025203	2.889236e-12\\
0.2026203	2.824589e-12\\
0.2027203	2.752527e-12\\
0.2028203	2.924617e-12\\
0.2029203	2.946049e-12\\
0.2030203	3.021142e-12\\
0.2031203	3.106665e-12\\
0.2032203	3.167325e-12\\
0.2033203	3.174343e-12\\
0.2034203	3.403159e-12\\
0.2035204	3.356927e-12\\
0.2036204	3.381068e-12\\
0.2037204	3.520289e-12\\
0.2038204	3.513013e-12\\
0.2039204	3.587974e-12\\
0.2040204	3.58672e-12\\
0.2041204	3.479514e-12\\
0.2042204	3.615567e-12\\
0.2043204	3.791776e-12\\
0.2044204	3.685957e-12\\
0.2045205	3.707037e-12\\
0.2046205	3.660976e-12\\
0.2047205	3.625083e-12\\
0.2048205	3.807699e-12\\
0.2049205	3.984403e-12\\
0.2050205	4.051776e-12\\
0.2051205	4.227402e-12\\
0.2052205	4.203738e-12\\
0.2053205	4.089522e-12\\
0.2054205	4.218167e-12\\
0.2055206	4.318826e-12\\
0.2056206	4.411838e-12\\
0.2057206	4.476477e-12\\
0.2058206	4.554105e-12\\
0.2059206	4.813678e-12\\
0.2060206	4.808783e-12\\
0.2061206	4.74515e-12\\
0.2062206	4.8551e-12\\
0.2063206	4.934423e-12\\
0.2064206	5.000296e-12\\
0.2065207	4.998575e-12\\
0.2066207	4.931289e-12\\
0.2067207	5.053238e-12\\
0.2068207	5.227499e-12\\
0.2069207	5.27506e-12\\
0.2070207	5.373831e-12\\
0.2071207	5.540958e-12\\
0.2072207	5.506435e-12\\
0.2073207	5.566914e-12\\
0.2074207	5.799495e-12\\
0.2075208	5.76261e-12\\
0.2076208	5.807009e-12\\
0.2077208	5.679111e-12\\
0.2078208	5.609262e-12\\
0.2079208	5.953825e-12\\
0.2080208	6.112339e-12\\
0.2081208	6.169365e-12\\
0.2082208	6.402147e-12\\
0.2083208	6.551903e-12\\
0.2084208	6.541353e-12\\
0.2085209	6.74291e-12\\
0.2086209	6.786211e-12\\
0.2087209	6.787768e-12\\
0.2088209	6.820817e-12\\
0.2089209	6.948728e-12\\
0.2090209	7.112746e-12\\
0.2091209	7.133635e-12\\
0.2092209	7.001356e-12\\
0.2093209	7.333704e-12\\
0.2094209	7.57754e-12\\
0.209521	7.705969e-12\\
0.209621	7.649164e-12\\
0.209721	7.66733e-12\\
0.209821	7.860468e-12\\
0.209921	8.084378e-12\\
0.210021	8.254363e-12\\
0.210121	8.346763e-12\\
0.210221	8.380101e-12\\
0.210321	8.357633e-12\\
0.210421	8.360225e-12\\
0.2105211	8.660039e-12\\
0.2106211	8.795869e-12\\
0.2107211	8.904744e-12\\
0.2108211	9.011516e-12\\
0.2109211	9.153078e-12\\
0.2110211	9.597422e-12\\
0.2111211	9.423315e-12\\
0.2112211	9.4328e-12\\
0.2113211	9.559285e-12\\
0.2114211	9.802253e-12\\
0.2115212	9.933897e-12\\
0.2116212	1.025912e-11\\
0.2117212	1.041992e-11\\
0.2118212	1.065632e-11\\
0.2119212	1.051771e-11\\
0.2120212	1.041686e-11\\
0.2121212	1.048739e-11\\
0.2122212	1.067103e-11\\
0.2123212	1.090438e-11\\
0.2124212	1.129699e-11\\
0.2125213	1.154457e-11\\
0.2126213	1.171647e-11\\
0.2127213	1.183133e-11\\
0.2128213	1.173232e-11\\
0.2129213	1.23052e-11\\
0.2130213	1.246055e-11\\
0.2131213	1.240175e-11\\
0.2132213	1.234973e-11\\
0.2133213	1.262425e-11\\
0.2134213	1.283762e-11\\
0.2135214	1.31211e-11\\
0.2136214	1.332251e-11\\
0.2137214	1.361848e-11\\
0.2138214	1.388769e-11\\
0.2139214	1.392775e-11\\
0.2140214	1.410352e-11\\
0.2141214	1.435045e-11\\
0.2142214	1.443059e-11\\
0.2143214	1.459465e-11\\
0.2144214	1.497408e-11\\
0.2145215	1.505668e-11\\
0.2146215	1.509848e-11\\
0.2147215	1.516091e-11\\
0.2148215	1.553671e-11\\
0.2149215	1.575885e-11\\
0.2150215	1.583831e-11\\
0.2151215	1.608947e-11\\
0.2152215	1.622793e-11\\
0.2153215	1.653643e-11\\
0.2154215	1.70512e-11\\
0.2155216	1.716272e-11\\
0.2156216	1.743215e-11\\
0.2157216	1.769275e-11\\
0.2158216	1.771874e-11\\
0.2159216	1.84992e-11\\
0.2160216	1.875939e-11\\
0.2161216	1.8761e-11\\
0.2162216	1.880422e-11\\
0.2163216	1.913739e-11\\
0.2164216	1.973366e-11\\
0.2165217	2.007445e-11\\
0.2166217	2.012055e-11\\
0.2167217	2.043287e-11\\
0.2168217	2.055959e-11\\
0.2169217	2.053241e-11\\
0.2170217	2.065296e-11\\
0.2171217	2.104901e-11\\
0.2172217	2.137597e-11\\
0.2173217	2.190219e-11\\
0.2174217	2.23167e-11\\
0.2175218	2.270404e-11\\
0.2176218	2.305663e-11\\
0.2177218	2.286263e-11\\
0.2178218	2.366509e-11\\
0.2179218	2.40782e-11\\
0.2180218	2.413796e-11\\
0.2181218	2.427489e-11\\
0.2182218	2.46288e-11\\
0.2183218	2.518e-11\\
0.2184218	2.559051e-11\\
0.2185219	2.563267e-11\\
0.2186219	2.623937e-11\\
0.2187219	2.650979e-11\\
0.2188219	2.686516e-11\\
0.2189219	2.745507e-11\\
0.2190219	2.747618e-11\\
0.2191219	2.790396e-11\\
0.2192219	2.812829e-11\\
0.2193219	2.83985e-11\\
0.2194219	2.919029e-11\\
0.219522	2.955373e-11\\
0.219622	2.966685e-11\\
0.219722	3.063251e-11\\
0.219822	3.091732e-11\\
0.219922	3.126052e-11\\
0.220022	3.186191e-11\\
0.220122	3.219034e-11\\
0.220222	3.246399e-11\\
0.220322	3.275469e-11\\
0.220422	3.276757e-11\\
0.2205221	3.3433e-11\\
0.2206221	3.404167e-11\\
0.2207221	3.468166e-11\\
0.2208221	3.586916e-11\\
0.2209221	3.588551e-11\\
0.2210221	3.623647e-11\\
0.2211221	3.652007e-11\\
0.2212221	3.709443e-11\\
0.2213221	3.808179e-11\\
0.2214221	3.802746e-11\\
0.2215222	3.84472e-11\\
0.2216222	3.924235e-11\\
0.2217222	3.94859e-11\\
0.2218222	3.987294e-11\\
0.2219222	4.044075e-11\\
0.2220222	4.130143e-11\\
0.2221222	4.183338e-11\\
0.2222222	4.223561e-11\\
0.2223222	4.309272e-11\\
0.2224222	4.334505e-11\\
0.2225223	4.370578e-11\\
0.2226223	4.451689e-11\\
0.2227223	4.536365e-11\\
0.2228223	4.587369e-11\\
0.2229223	4.660495e-11\\
0.2230223	4.733648e-11\\
0.2231223	4.841616e-11\\
0.2232223	4.85464e-11\\
0.2233223	4.839416e-11\\
0.2234223	4.929139e-11\\
0.2235224	5.046475e-11\\
0.2236224	5.171657e-11\\
0.2237224	5.249408e-11\\
0.2238224	5.260525e-11\\
0.2239224	5.309268e-11\\
0.2240224	5.388799e-11\\
0.2241224	5.459348e-11\\
0.2242224	5.571829e-11\\
0.2243224	5.571794e-11\\
0.2244224	5.656181e-11\\
0.2245225	5.785255e-11\\
0.2246225	5.835047e-11\\
0.2247225	5.921998e-11\\
0.2248225	6.027816e-11\\
0.2249225	6.050552e-11\\
0.2250225	6.153194e-11\\
0.2251225	6.173361e-11\\
0.2252225	6.230664e-11\\
0.2253225	6.366245e-11\\
0.2254225	6.452877e-11\\
0.2255226	6.600653e-11\\
0.2256226	6.67098e-11\\
0.2257226	6.767708e-11\\
0.2258226	6.827909e-11\\
0.2259226	6.918763e-11\\
0.2260226	7.074262e-11\\
0.2261226	7.147969e-11\\
0.2262226	7.212186e-11\\
0.2263226	7.354287e-11\\
0.2264226	7.440212e-11\\
0.2265227	7.532417e-11\\
0.2266227	7.56641e-11\\
0.2267227	7.645737e-11\\
0.2268227	7.808796e-11\\
0.2269227	7.850763e-11\\
0.2270227	8.036684e-11\\
0.2271227	8.166544e-11\\
0.2272227	8.22272e-11\\
0.2273227	8.351297e-11\\
0.2274227	8.440612e-11\\
0.2275228	8.568927e-11\\
0.2276228	8.696143e-11\\
0.2277228	8.751463e-11\\
0.2278228	8.911437e-11\\
0.2279228	9.017716e-11\\
0.2280228	9.038237e-11\\
0.2281228	9.237048e-11\\
0.2282228	9.408189e-11\\
0.2283228	9.573498e-11\\
0.2284228	9.629735e-11\\
0.2285229	9.789175e-11\\
0.2286229	9.96574e-11\\
0.2287229	1.007453e-10\\
0.2288229	1.017792e-10\\
0.2289229	1.034267e-10\\
0.2290229	1.041873e-10\\
0.2291229	1.050409e-10\\
0.2292229	1.064938e-10\\
0.2293229	1.080922e-10\\
0.2294229	1.091545e-10\\
0.229523	1.101947e-10\\
0.229623	1.125324e-10\\
0.229723	1.143578e-10\\
0.229823	1.160812e-10\\
0.229923	1.174196e-10\\
0.230023	1.192427e-10\\
0.230123	1.209091e-10\\
0.230223	1.220419e-10\\
0.230323	1.23121e-10\\
0.230423	1.25543e-10\\
0.2305231	1.266657e-10\\
0.2306231	1.289599e-10\\
0.2307231	1.310552e-10\\
0.2308231	1.321972e-10\\
0.2309231	1.339784e-10\\
0.2310231	1.359353e-10\\
0.2311231	1.374025e-10\\
0.2312231	1.380102e-10\\
0.2313231	1.395353e-10\\
0.2314231	1.425391e-10\\
0.2315232	1.450931e-10\\
0.2316232	1.469751e-10\\
0.2317232	1.494305e-10\\
0.2318232	1.50311e-10\\
0.2319232	1.519313e-10\\
0.2320232	1.541126e-10\\
0.2321232	1.562441e-10\\
0.2322232	1.582938e-10\\
0.2323232	1.597598e-10\\
0.2324232	1.628292e-10\\
0.2325233	1.646093e-10\\
0.2326233	1.656799e-10\\
0.2327233	1.685958e-10\\
0.2328233	1.71322e-10\\
0.2329233	1.738898e-10\\
0.2330233	1.757457e-10\\
0.2331233	1.774957e-10\\
0.2332233	1.807579e-10\\
0.2333233	1.827673e-10\\
0.2334233	1.845421e-10\\
0.2335234	1.877418e-10\\
0.2336234	1.898145e-10\\
0.2337234	1.924661e-10\\
0.2338234	1.952846e-10\\
0.2339234	1.988303e-10\\
0.2340234	2.003476e-10\\
0.2341234	2.021937e-10\\
0.2342234	2.059965e-10\\
0.2343234	2.086541e-10\\
0.2344234	2.108286e-10\\
0.2345235	2.129028e-10\\
0.2346235	2.155768e-10\\
0.2347235	2.185184e-10\\
0.2348235	2.21007e-10\\
0.2349235	2.242179e-10\\
0.2350235	2.280776e-10\\
0.2351235	2.299451e-10\\
0.2352235	2.339391e-10\\
0.2353235	2.366386e-10\\
0.2354235	2.398523e-10\\
0.2355236	2.43577e-10\\
0.2356236	2.458602e-10\\
0.2357236	2.49527e-10\\
0.2358236	2.512392e-10\\
0.2359236	2.551738e-10\\
0.2360236	2.591684e-10\\
0.2361236	2.625572e-10\\
0.2362236	2.659042e-10\\
0.2363236	2.690658e-10\\
0.2364236	2.729447e-10\\
0.2365237	2.764786e-10\\
0.2366237	2.795928e-10\\
0.2367237	2.838858e-10\\
0.2368237	2.866522e-10\\
0.2369237	2.904565e-10\\
0.2370237	2.954621e-10\\
0.2371237	2.976253e-10\\
0.2372237	3.012286e-10\\
0.2373237	3.052521e-10\\
0.2374237	3.108559e-10\\
0.2375238	3.14673e-10\\
0.2376238	3.179421e-10\\
0.2377238	3.228031e-10\\
0.2378238	3.264416e-10\\
0.2379238	3.293444e-10\\
0.2380238	3.334722e-10\\
0.2381238	3.389311e-10\\
0.2382238	3.44234e-10\\
0.2383238	3.487223e-10\\
0.2384238	3.536512e-10\\
0.2385239	3.57349e-10\\
0.2386239	3.610883e-10\\
0.2387239	3.674988e-10\\
0.2388239	3.715487e-10\\
0.2389239	3.752289e-10\\
0.2390239	3.794657e-10\\
0.2391239	3.857451e-10\\
0.2392239	3.913305e-10\\
0.2393239	3.948028e-10\\
0.2394239	4.002384e-10\\
0.239524	4.044425e-10\\
0.239624	4.094617e-10\\
0.239724	4.141572e-10\\
0.239824	4.20298e-10\\
0.239924	4.269149e-10\\
0.240024	4.321015e-10\\
0.240124	4.391173e-10\\
0.240224	4.449437e-10\\
0.240324	4.490504e-10\\
0.240424	4.549814e-10\\
0.2405241	4.609286e-10\\
0.2406241	4.661465e-10\\
0.2407241	4.723813e-10\\
0.2408241	4.7733e-10\\
0.2409241	4.858576e-10\\
0.2410241	4.914088e-10\\
0.2411241	4.974664e-10\\
0.2412241	5.049724e-10\\
0.2413241	5.115851e-10\\
0.2414241	5.174021e-10\\
0.2415242	5.223972e-10\\
0.2416242	5.292893e-10\\
0.2417242	5.346891e-10\\
0.2418242	5.419205e-10\\
0.2419242	5.506537e-10\\
0.2420242	5.576376e-10\\
0.2421242	5.648786e-10\\
0.2422242	5.71552e-10\\
0.2423242	5.780337e-10\\
0.2424242	5.858509e-10\\
0.2425243	5.934661e-10\\
0.2426243	6.031969e-10\\
0.2427243	6.083275e-10\\
0.2428243	6.159474e-10\\
0.2429243	6.23987e-10\\
0.2430243	6.306461e-10\\
0.2431243	6.413583e-10\\
0.2432243	6.487709e-10\\
0.2433243	6.582749e-10\\
0.2434243	6.651867e-10\\
0.2435244	6.741869e-10\\
0.2436244	6.845981e-10\\
0.2437244	6.894809e-10\\
0.2438244	6.986565e-10\\
0.2439244	7.077003e-10\\
0.2440244	7.168292e-10\\
0.2441244	7.265158e-10\\
0.2442244	7.354584e-10\\
0.2443244	7.454699e-10\\
0.2444244	7.557671e-10\\
0.2445245	7.655089e-10\\
0.2446245	7.750196e-10\\
0.2447245	7.827364e-10\\
0.2448245	7.935861e-10\\
0.2449245	8.048216e-10\\
0.2450245	8.140987e-10\\
0.2451245	8.239043e-10\\
0.2452245	8.341114e-10\\
0.2453245	8.454228e-10\\
0.2454245	8.530547e-10\\
0.2455246	8.636536e-10\\
0.2456246	8.766692e-10\\
0.2457246	8.895915e-10\\
0.2458246	9.013745e-10\\
0.2459246	9.113198e-10\\
0.2460246	9.222988e-10\\
0.2461246	9.334087e-10\\
0.2462246	9.481698e-10\\
0.2463246	9.598458e-10\\
0.2464246	9.713477e-10\\
0.2465247	9.82487e-10\\
0.2466247	9.947411e-10\\
0.2467247	1.007516e-09\\
0.2468247	1.016369e-09\\
0.2469247	1.031566e-09\\
0.2470247	1.044131e-09\\
0.2471247	1.057367e-09\\
0.2472247	1.07193e-09\\
0.2473247	1.084272e-09\\
0.2474247	1.09962e-09\\
0.2475248	1.113299e-09\\
0.2476248	1.128519e-09\\
0.2477248	1.14272e-09\\
0.2478248	1.154574e-09\\
0.2479248	1.171699e-09\\
0.2480248	1.183908e-09\\
0.2481248	1.196791e-09\\
0.2482248	1.213183e-09\\
0.2483248	1.230473e-09\\
0.2484248	1.248811e-09\\
0.2485249	1.261178e-09\\
0.2486249	1.277462e-09\\
0.2487249	1.293493e-09\\
0.2488249	1.308228e-09\\
0.2489249	1.32579e-09\\
0.2490249	1.34081e-09\\
0.2491249	1.358387e-09\\
0.2492249	1.373658e-09\\
0.2493249	1.393415e-09\\
0.2494249	1.409776e-09\\
0.249525	1.427487e-09\\
0.249625	1.445684e-09\\
0.249725	1.460525e-09\\
0.249825	1.479107e-09\\
0.249925	1.499452e-09\\
0.250025	1.521746e-09\\
0.250125	1.53735e-09\\
0.250225	1.556372e-09\\
0.250325	1.576165e-09\\
0.250425	1.594856e-09\\
0.2505251	1.615708e-09\\
0.2506251	1.636462e-09\\
0.2507251	1.656635e-09\\
0.2508251	1.676831e-09\\
0.2509251	1.69885e-09\\
0.2510251	1.720396e-09\\
0.2511251	1.738875e-09\\
0.2512251	1.763338e-09\\
0.2513251	1.78727e-09\\
0.2514251	1.808257e-09\\
0.2515252	1.829235e-09\\
0.2516252	1.850092e-09\\
0.2517252	1.873815e-09\\
0.2518252	1.894856e-09\\
0.2519252	1.920125e-09\\
0.2520252	1.945397e-09\\
0.2521252	1.967814e-09\\
0.2522252	1.99469e-09\\
0.2523252	2.016572e-09\\
0.2524252	2.042155e-09\\
0.2525253	2.067387e-09\\
0.2526253	2.096753e-09\\
0.2527253	2.118807e-09\\
0.2528253	2.142331e-09\\
0.2529253	2.17268e-09\\
0.2530253	2.200336e-09\\
0.2531253	2.232217e-09\\
0.2532253	2.255212e-09\\
0.2533253	2.283316e-09\\
0.2534253	2.310455e-09\\
0.2535254	2.338866e-09\\
0.2536254	2.366623e-09\\
0.2537254	2.392059e-09\\
0.2538254	2.426446e-09\\
0.2539254	2.456614e-09\\
0.2540254	2.486868e-09\\
0.2541254	2.516793e-09\\
0.2542254	2.549208e-09\\
0.2543254	2.582146e-09\\
0.2544254	2.610257e-09\\
0.2545255	2.644772e-09\\
0.2546255	2.677158e-09\\
0.2547255	2.708709e-09\\
0.2548255	2.741471e-09\\
0.2549255	2.779834e-09\\
0.2550255	2.810179e-09\\
0.2551255	2.844013e-09\\
0.2552255	2.879864e-09\\
0.2553255	2.910393e-09\\
0.2554255	2.951076e-09\\
0.2555256	2.992686e-09\\
0.2556256	3.030494e-09\\
0.2557256	3.062426e-09\\
0.2558256	3.096974e-09\\
0.2559256	3.135206e-09\\
0.2560256	3.170285e-09\\
0.2561256	3.213492e-09\\
0.2562256	3.255286e-09\\
0.2563256	3.295063e-09\\
0.2564256	3.336603e-09\\
0.2565257	3.376798e-09\\
0.2566257	3.41826e-09\\
0.2567257	3.456661e-09\\
0.2568257	3.500185e-09\\
0.2569257	3.540879e-09\\
0.2570257	3.583769e-09\\
0.2571257	3.630079e-09\\
0.2572257	3.67542e-09\\
0.2573257	3.717139e-09\\
0.2574257	3.762489e-09\\
0.2575258	3.813689e-09\\
0.2576258	3.858674e-09\\
0.2577258	3.904962e-09\\
0.2578258	3.952565e-09\\
0.2579258	3.997824e-09\\
0.2580258	4.048277e-09\\
0.2581258	4.099218e-09\\
0.2582258	4.148937e-09\\
0.2583258	4.197307e-09\\
0.2584258	4.25219e-09\\
0.2585259	4.303349e-09\\
0.2586259	4.354626e-09\\
0.2587259	4.41007e-09\\
0.2588259	4.462608e-09\\
0.2589259	4.516664e-09\\
0.2590259	4.570418e-09\\
0.2591259	4.631233e-09\\
0.2592259	4.685083e-09\\
0.2593259	4.741052e-09\\
0.2594259	4.791367e-09\\
0.259526	4.847553e-09\\
0.259626	4.911758e-09\\
0.259726	4.973035e-09\\
0.259826	5.033974e-09\\
0.259926	5.088445e-09\\
0.260026	5.156035e-09\\
0.260126	5.21918e-09\\
0.260226	5.280431e-09\\
0.260326	5.349923e-09\\
0.260426	5.412357e-09\\
0.2605261	5.479382e-09\\
0.2606261	5.540281e-09\\
0.2607261	5.608895e-09\\
0.2608261	5.675729e-09\\
0.2609261	5.747107e-09\\
0.2610261	5.822224e-09\\
0.2611261	5.888551e-09\\
0.2612261	5.95675e-09\\
0.2613261	6.028151e-09\\
0.2614261	6.102042e-09\\
0.2615262	6.169251e-09\\
0.2616262	6.2512e-09\\
0.2617262	6.326259e-09\\
0.2618262	6.403074e-09\\
0.2619262	6.478696e-09\\
0.2620262	6.55618e-09\\
0.2621262	6.636952e-09\\
0.2622262	6.716841e-09\\
0.2623262	6.797475e-09\\
0.2624262	6.875038e-09\\
0.2625263	6.965523e-09\\
0.2626263	7.049722e-09\\
0.2627263	7.131717e-09\\
0.2628263	7.212636e-09\\
0.2629263	7.301354e-09\\
0.2630263	7.388093e-09\\
0.2631263	7.472906e-09\\
0.2632263	7.573862e-09\\
0.2633263	7.658339e-09\\
0.2634263	7.756333e-09\\
0.2635264	7.846077e-09\\
0.2636264	7.938662e-09\\
0.2637264	8.037162e-09\\
0.2638264	8.134348e-09\\
0.2639264	8.233681e-09\\
0.2640264	8.327316e-09\\
0.2641264	8.428751e-09\\
0.2642264	8.522857e-09\\
0.2643264	8.623297e-09\\
0.2644264	8.725386e-09\\
0.2645265	8.83558e-09\\
0.2646265	8.942187e-09\\
0.2647265	9.048583e-09\\
0.2648265	9.159606e-09\\
0.2649265	9.266997e-09\\
0.2650265	9.38815e-09\\
0.2651265	9.498409e-09\\
0.2652265	9.60651e-09\\
0.2653265	9.715906e-09\\
0.2654265	9.830038e-09\\
0.2655266	9.948213e-09\\
0.2656266	1.006746e-08\\
0.2657266	1.019911e-08\\
0.2658266	1.031679e-08\\
0.2659266	1.043502e-08\\
0.2660266	1.055726e-08\\
0.2661266	1.069517e-08\\
0.2662266	1.081362e-08\\
0.2663266	1.094183e-08\\
0.2664266	1.106709e-08\\
0.2665267	1.120022e-08\\
0.2666267	1.134024e-08\\
0.2667267	1.147111e-08\\
0.2668267	1.160435e-08\\
0.2669267	1.174635e-08\\
0.2670267	1.189179e-08\\
0.2671267	1.202504e-08\\
0.2672267	1.216925e-08\\
0.2673267	1.231329e-08\\
0.2674267	1.245779e-08\\
0.2675268	1.260295e-08\\
0.2676268	1.27562e-08\\
0.2677268	1.29109e-08\\
0.2678268	1.306466e-08\\
0.2679268	1.322534e-08\\
0.2680268	1.337231e-08\\
0.2681268	1.35326e-08\\
0.2682268	1.368651e-08\\
0.2683268	1.384117e-08\\
0.2684268	1.401066e-08\\
0.2685269	1.418713e-08\\
0.2686269	1.436199e-08\\
0.2687269	1.452595e-08\\
0.2688269	1.469645e-08\\
0.2689269	1.486428e-08\\
0.2690269	1.505079e-08\\
0.2691269	1.522473e-08\\
0.2692269	1.540075e-08\\
0.2693269	1.557359e-08\\
0.2694269	1.576676e-08\\
0.269527	1.595069e-08\\
0.269627	1.613506e-08\\
0.269727	1.632604e-08\\
0.269827	1.652049e-08\\
0.269927	1.671111e-08\\
0.270027	1.691109e-08\\
0.270127	1.71159e-08\\
0.270227	1.73097e-08\\
0.270327	1.752154e-08\\
0.270427	1.772065e-08\\
0.2705271	1.793572e-08\\
0.2706271	1.814084e-08\\
0.2707271	1.83521e-08\\
0.2708271	1.856719e-08\\
0.2709271	1.878883e-08\\
0.2710271	1.901942e-08\\
0.2711271	1.923216e-08\\
0.2712271	1.946265e-08\\
0.2713271	1.96906e-08\\
0.2714271	1.992661e-08\\
0.2715272	2.014234e-08\\
0.2716272	2.038445e-08\\
0.2717272	2.06231e-08\\
0.2718272	2.086917e-08\\
0.2719272	2.111922e-08\\
0.2720272	2.135814e-08\\
0.2721272	2.161326e-08\\
0.2722272	2.185221e-08\\
0.2723272	2.21137e-08\\
0.2724272	2.237337e-08\\
0.2725273	2.264758e-08\\
0.2726273	2.290485e-08\\
0.2727273	2.31657e-08\\
0.2728273	2.343806e-08\\
0.2729273	2.3715e-08\\
0.2730273	2.399236e-08\\
0.2731273	2.427449e-08\\
0.2732273	2.456687e-08\\
0.2733273	2.48454e-08\\
0.2734273	2.513431e-08\\
0.2735274	2.542345e-08\\
0.2736274	2.571234e-08\\
0.2737274	2.602653e-08\\
0.2738274	2.632981e-08\\
0.2739274	2.662431e-08\\
0.2740274	2.693418e-08\\
0.2741274	2.724656e-08\\
0.2742274	2.756388e-08\\
0.2743274	2.788855e-08\\
0.2744274	2.821835e-08\\
0.2745275	2.854145e-08\\
0.2746275	2.886661e-08\\
0.2747275	2.920867e-08\\
0.2748275	2.954917e-08\\
0.2749275	2.989444e-08\\
0.2750275	3.023237e-08\\
0.2751275	3.058718e-08\\
0.2752275	3.094157e-08\\
0.2753275	3.131738e-08\\
0.2754275	3.166244e-08\\
0.2755276	3.202584e-08\\
0.2756276	3.239324e-08\\
0.2757276	3.277407e-08\\
0.2758276	3.316282e-08\\
0.2759276	3.354154e-08\\
0.2760276	3.393703e-08\\
0.2761276	3.432364e-08\\
0.2762276	3.472809e-08\\
0.2763276	3.510791e-08\\
0.2764276	3.552705e-08\\
0.2765277	3.593134e-08\\
0.2766277	3.634539e-08\\
0.2767277	3.677165e-08\\
0.2768277	3.720385e-08\\
0.2769277	3.76304e-08\\
0.2770277	3.805513e-08\\
0.2771277	3.850539e-08\\
0.2772277	3.89363e-08\\
0.2773277	3.936896e-08\\
0.2774277	3.982047e-08\\
0.2775278	4.028801e-08\\
0.2776278	4.07676e-08\\
0.2777278	4.125103e-08\\
0.2778278	4.170747e-08\\
0.2779278	4.219028e-08\\
0.2780278	4.267772e-08\\
0.2781278	4.316558e-08\\
0.2782278	4.366334e-08\\
0.2783278	4.417265e-08\\
0.2784278	4.466404e-08\\
0.2785279	4.51678e-08\\
0.2786279	4.569032e-08\\
0.2787279	4.621631e-08\\
0.2788279	4.675957e-08\\
0.2789279	4.728662e-08\\
0.2790279	4.783818e-08\\
0.2791279	4.838345e-08\\
0.2792279	4.894601e-08\\
0.2793279	4.948456e-08\\
0.2794279	5.006066e-08\\
0.279528	5.064058e-08\\
0.279628	5.12189e-08\\
0.279728	5.178955e-08\\
0.279828	5.239616e-08\\
0.279928	5.299631e-08\\
0.280028	5.359228e-08\\
0.280128	5.420618e-08\\
0.280228	5.482509e-08\\
0.280328	5.546135e-08\\
0.280428	5.609072e-08\\
0.2805281	5.672225e-08\\
0.2806281	5.736749e-08\\
0.2807281	5.801966e-08\\
0.2808281	5.867505e-08\\
0.2809281	5.936815e-08\\
0.2810281	6.003844e-08\\
0.2811281	6.073025e-08\\
0.2812281	6.140138e-08\\
0.2813281	6.211908e-08\\
0.2814281	6.283069e-08\\
0.2815282	6.3544e-08\\
0.2816282	6.424945e-08\\
0.2817282	6.497386e-08\\
0.2818282	6.571817e-08\\
0.2819282	6.646791e-08\\
0.2820282	6.722312e-08\\
0.2821282	6.797629e-08\\
0.2822282	6.875479e-08\\
0.2823282	6.952752e-08\\
0.2824282	7.032684e-08\\
0.2825283	7.113344e-08\\
0.2826283	7.193555e-08\\
0.2827283	7.275173e-08\\
0.2828283	7.35756e-08\\
0.2829283	7.440386e-08\\
0.2830283	7.524685e-08\\
0.2831283	7.607776e-08\\
0.2832283	7.696733e-08\\
0.2833283	7.78539e-08\\
0.2834283	7.871903e-08\\
0.2835284	7.959198e-08\\
0.2836284	8.051132e-08\\
0.2837284	8.141563e-08\\
0.2838284	8.233115e-08\\
0.2839284	8.326314e-08\\
0.2840284	8.419916e-08\\
0.2841284	8.516879e-08\\
0.2842284	8.612831e-08\\
0.2843284	8.708527e-08\\
0.2844284	8.803665e-08\\
0.2845285	8.903403e-08\\
0.2846285	9.002929e-08\\
0.2847285	9.107361e-08\\
0.2848285	9.211002e-08\\
0.2849285	9.315119e-08\\
0.2850285	9.41847e-08\\
0.2851285	9.5259e-08\\
0.2852285	9.631614e-08\\
0.2853285	9.740768e-08\\
0.2854285	9.84844e-08\\
0.2855286	9.960042e-08\\
0.2856286	1.007172e-07\\
0.2857286	1.018518e-07\\
0.2858286	1.029847e-07\\
0.2859286	1.041617e-07\\
0.2860286	1.053336e-07\\
0.2861286	1.064921e-07\\
0.2862286	1.076841e-07\\
0.2863286	1.089027e-07\\
0.2864286	1.101267e-07\\
0.2865287	1.113727e-07\\
0.2866287	1.126216e-07\\
0.2867287	1.138609e-07\\
0.2868287	1.151309e-07\\
0.2869287	1.164118e-07\\
0.2870287	1.17726e-07\\
0.2871287	1.190429e-07\\
0.2872287	1.203608e-07\\
0.2873287	1.216973e-07\\
0.2874287	1.231055e-07\\
0.2875288	1.244504e-07\\
0.2876288	1.258582e-07\\
0.2877288	1.27251e-07\\
0.2878288	1.286852e-07\\
0.2879288	1.300953e-07\\
0.2880288	1.315497e-07\\
0.2881288	1.329993e-07\\
0.2882288	1.345077e-07\\
0.2883288	1.359999e-07\\
0.2884288	1.375044e-07\\
0.2885289	1.390387e-07\\
0.2886289	1.406129e-07\\
0.2887289	1.42159e-07\\
0.2888289	1.437267e-07\\
0.2889289	1.453345e-07\\
0.2890289	1.469596e-07\\
0.2891289	1.485894e-07\\
0.2892289	1.502404e-07\\
0.2893289	1.519006e-07\\
0.2894289	1.535746e-07\\
0.289529	1.552976e-07\\
0.289629	1.570193e-07\\
0.289729	1.587973e-07\\
0.289829	1.605166e-07\\
0.289929	1.622593e-07\\
0.290029	1.640834e-07\\
0.290129	1.659031e-07\\
0.290229	1.677406e-07\\
0.290329	1.696194e-07\\
0.290429	1.714943e-07\\
0.2905291	1.733904e-07\\
0.2906291	1.753126e-07\\
0.2907291	1.77252e-07\\
0.2908291	1.791815e-07\\
0.2909291	1.811866e-07\\
0.2910291	1.831483e-07\\
0.2911291	1.852131e-07\\
0.2912291	1.872514e-07\\
0.2913291	1.893024e-07\\
0.2914291	1.913957e-07\\
0.2915292	1.935383e-07\\
0.2916292	1.956388e-07\\
0.2917292	1.977944e-07\\
0.2918292	1.999677e-07\\
0.2919292	2.02184e-07\\
0.2920292	2.044014e-07\\
0.2921292	2.066478e-07\\
0.2922292	2.08905e-07\\
0.2923292	2.112406e-07\\
0.2924292	2.135501e-07\\
0.2925293	2.15902e-07\\
0.2926293	2.18281e-07\\
0.2927293	2.20684e-07\\
0.2928293	2.231166e-07\\
0.2929293	2.255344e-07\\
0.2930293	2.279901e-07\\
0.2931293	2.30492e-07\\
0.2932293	2.33023e-07\\
0.2933293	2.356057e-07\\
0.2934293	2.382128e-07\\
0.2935294	2.408031e-07\\
0.2936294	2.434264e-07\\
0.2937294	2.461212e-07\\
0.2938294	2.488301e-07\\
0.2939294	2.515285e-07\\
0.2940294	2.542747e-07\\
0.2941294	2.57043e-07\\
0.2942294	2.598513e-07\\
0.2943294	2.626917e-07\\
0.2944294	2.655792e-07\\
0.2945295	2.684932e-07\\
0.2946295	2.714386e-07\\
0.2947295	2.743395e-07\\
0.2948295	2.773299e-07\\
0.2949295	2.803711e-07\\
0.2950295	2.834125e-07\\
0.2951295	2.86517e-07\\
0.2952295	2.896852e-07\\
0.2953295	2.928013e-07\\
0.2954295	2.960203e-07\\
0.2955296	2.99242e-07\\
0.2956296	3.024946e-07\\
0.2957296	3.057564e-07\\
0.2958296	3.091023e-07\\
0.2959296	3.124634e-07\\
0.2960296	3.159213e-07\\
0.2961296	3.192977e-07\\
0.2962296	3.227509e-07\\
0.2963296	3.262501e-07\\
0.2964296	3.298138e-07\\
0.2965297	3.333809e-07\\
0.2966297	3.370248e-07\\
0.2967297	3.406289e-07\\
0.2968297	3.443459e-07\\
0.2969297	3.480758e-07\\
0.2970297	3.518843e-07\\
0.2971297	3.556849e-07\\
0.2972297	3.595316e-07\\
0.2973297	3.634221e-07\\
0.2974297	3.673603e-07\\
0.2975298	3.713092e-07\\
0.2976298	3.753255e-07\\
0.2977298	3.794079e-07\\
0.2978298	3.834906e-07\\
0.2979298	3.87618e-07\\
0.2980298	3.918415e-07\\
0.2981298	3.960649e-07\\
0.2982298	4.003993e-07\\
0.2983298	4.046642e-07\\
0.2984298	4.090036e-07\\
0.2985299	4.134144e-07\\
0.2986299	4.1789e-07\\
0.2987299	4.223815e-07\\
0.2988299	4.269679e-07\\
0.2989299	4.315313e-07\\
0.2990299	4.362009e-07\\
0.2991299	4.408896e-07\\
0.2992299	4.456295e-07\\
0.2993299	4.503727e-07\\
0.2994299	4.552693e-07\\
0.29953	4.601585e-07\\
0.29963	4.651509e-07\\
0.29973	4.700911e-07\\
0.29983	4.751406e-07\\
0.29993	4.802719e-07\\
0.30003	4.853968e-07\\
0.30013	4.905673e-07\\
0.30023	4.95861e-07\\
0.30033	5.011718e-07\\
0.30043	5.06589e-07\\
0.3005301	5.120055e-07\\
0.3006301	5.17493e-07\\
0.3007301	5.230305e-07\\
0.3008301	5.286505e-07\\
0.3009301	5.342251e-07\\
0.3010301	5.399541e-07\\
0.3011301	5.457322e-07\\
0.3012301	5.51574e-07\\
0.3013301	5.575297e-07\\
0.3014301	5.634861e-07\\
0.3015302	5.694435e-07\\
0.3016302	5.755572e-07\\
0.3017302	5.81716e-07\\
0.3018302	5.879207e-07\\
0.3019302	5.941486e-07\\
0.3020302	6.004677e-07\\
0.3021302	6.068901e-07\\
0.3022302	6.133548e-07\\
0.3023302	6.198601e-07\\
0.3024302	6.264769e-07\\
0.3025303	6.331381e-07\\
0.3026303	6.398851e-07\\
0.3027303	6.46745e-07\\
0.3028303	6.536452e-07\\
0.3029303	6.605792e-07\\
0.3030303	6.675703e-07\\
0.3031303	6.74601e-07\\
0.3032303	6.817907e-07\\
0.3033303	6.889936e-07\\
0.3034303	6.963431e-07\\
0.3035304	7.03794e-07\\
0.3036304	7.1123e-07\\
0.3037304	7.187649e-07\\
0.3038304	7.264164e-07\\
0.3039304	7.341077e-07\\
0.3040304	7.418531e-07\\
0.3041304	7.496996e-07\\
0.3042304	7.576854e-07\\
0.3043304	7.656827e-07\\
0.3044304	7.737418e-07\\
0.3045305	7.819101e-07\\
0.3046305	7.90224e-07\\
0.3047305	7.985309e-07\\
0.3048305	8.069938e-07\\
0.3049305	8.155637e-07\\
0.3050305	8.241037e-07\\
0.3051305	8.327812e-07\\
0.3052305	8.41627e-07\\
0.3053305	8.505174e-07\\
0.3054305	8.595122e-07\\
0.3055306	8.685467e-07\\
0.3056306	8.776675e-07\\
0.3057306	8.869056e-07\\
0.3058306	8.962327e-07\\
0.3059306	9.056749e-07\\
0.3060306	9.152438e-07\\
0.3061306	9.248487e-07\\
0.3062306	9.345146e-07\\
0.3063306	9.444306e-07\\
0.3064306	9.543504e-07\\
0.3065307	9.643962e-07\\
0.3066307	9.745148e-07\\
0.3067307	9.847138e-07\\
0.3068307	9.950645e-07\\
0.3069307	1.005501e-06\\
0.3070307	1.016033e-06\\
0.3071307	1.026681e-06\\
0.3072307	1.037379e-06\\
0.3073307	1.048342e-06\\
0.3074307	1.059317e-06\\
0.3075308	1.070334e-06\\
0.3076308	1.081543e-06\\
0.3077308	1.092927e-06\\
0.3078308	1.104347e-06\\
0.3079308	1.115918e-06\\
0.3080308	1.127578e-06\\
0.3081308	1.13938e-06\\
0.3082308	1.151248e-06\\
0.3083308	1.163275e-06\\
0.3084308	1.175423e-06\\
0.3085309	1.187695e-06\\
0.3086309	1.200026e-06\\
0.3087309	1.212614e-06\\
0.3088309	1.225288e-06\\
0.3089309	1.238032e-06\\
0.3090309	1.250908e-06\\
0.3091309	1.263897e-06\\
0.3092309	1.277027e-06\\
0.3093309	1.290332e-06\\
0.3094309	1.303777e-06\\
0.309531	1.317428e-06\\
0.309631	1.331073e-06\\
0.309731	1.344803e-06\\
0.309831	1.358882e-06\\
0.309931	1.372992e-06\\
0.310031	1.387241e-06\\
0.310131	1.401692e-06\\
0.310231	1.416246e-06\\
0.310331	1.430893e-06\\
0.310431	1.445729e-06\\
0.3105311	1.460684e-06\\
0.3106311	1.475811e-06\\
0.3107311	1.491092e-06\\
0.3108311	1.506593e-06\\
0.3109311	1.522231e-06\\
0.3110311	1.537902e-06\\
0.3111311	1.553779e-06\\
0.3112311	1.569968e-06\\
0.3113311	1.586105e-06\\
0.3114311	1.602548e-06\\
0.3115312	1.619124e-06\\
0.3116312	1.635817e-06\\
0.3117312	1.652719e-06\\
0.3118312	1.669756e-06\\
0.3119312	1.687007e-06\\
0.3120312	1.704486e-06\\
0.3121312	1.721917e-06\\
0.3122312	1.739659e-06\\
0.3123312	1.757601e-06\\
0.3124312	1.775792e-06\\
0.3125313	1.79403e-06\\
0.3126313	1.8125e-06\\
0.3127313	1.831092e-06\\
0.3128313	1.849973e-06\\
0.3129313	1.869024e-06\\
0.3130313	1.888222e-06\\
0.3131313	1.9076e-06\\
0.3132313	1.927168e-06\\
0.3133313	1.947027e-06\\
0.3134313	1.967052e-06\\
0.3135314	1.987208e-06\\
0.3136314	2.007598e-06\\
0.3137314	2.028117e-06\\
0.3138314	2.04891e-06\\
0.3139314	2.069966e-06\\
0.3140314	2.091143e-06\\
0.3141314	2.112575e-06\\
0.3142314	2.13417e-06\\
0.3143314	2.156094e-06\\
0.3144314	2.178156e-06\\
0.3145315	2.200392e-06\\
0.3146315	2.222916e-06\\
0.3147315	2.245579e-06\\
0.3148315	2.268467e-06\\
0.3149315	2.291694e-06\\
0.3150315	2.315061e-06\\
0.3151315	2.338673e-06\\
0.3152315	2.362616e-06\\
0.3153315	2.386787e-06\\
0.3154315	2.41109e-06\\
0.3155316	2.435641e-06\\
0.3156316	2.460279e-06\\
0.3157316	2.48543e-06\\
0.3158316	2.510754e-06\\
0.3159316	2.536339e-06\\
0.3160316	2.562218e-06\\
0.3161316	2.588226e-06\\
0.3162316	2.614565e-06\\
0.3163316	2.641261e-06\\
0.3164316	2.667914e-06\\
0.3165317	2.695053e-06\\
0.3166317	2.72237e-06\\
0.3167317	2.749989e-06\\
0.3168317	2.777953e-06\\
0.3169317	2.806088e-06\\
0.3170317	2.834668e-06\\
0.3171317	2.863342e-06\\
0.3172317	2.892245e-06\\
0.3173317	2.92161e-06\\
0.3174317	2.951337e-06\\
0.3175318	2.981148e-06\\
0.3176318	3.011248e-06\\
0.3177318	3.041657e-06\\
0.3178318	3.072478e-06\\
0.3179318	3.103546e-06\\
0.3180318	3.134897e-06\\
0.3181318	3.166539e-06\\
0.3182318	3.198543e-06\\
0.3183318	3.230867e-06\\
0.3184318	3.263539e-06\\
0.3185319	3.296411e-06\\
0.3186319	3.329633e-06\\
0.3187319	3.363251e-06\\
0.3188319	3.397147e-06\\
0.3189319	3.431236e-06\\
0.3190319	3.465802e-06\\
0.3191319	3.500817e-06\\
0.3192319	3.536051e-06\\
0.3193319	3.571493e-06\\
0.3194319	3.607558e-06\\
0.319532	3.643787e-06\\
0.319632	3.680311e-06\\
0.319732	3.717336e-06\\
0.319832	3.754679e-06\\
0.319932	3.792382e-06\\
0.320032	3.830444e-06\\
0.320132	3.868769e-06\\
0.320232	3.907599e-06\\
0.320332	3.946812e-06\\
0.320432	3.986491e-06\\
0.3205321	4.026505e-06\\
0.3206321	4.066529e-06\\
0.3207321	4.107382e-06\\
0.3208321	4.14848e-06\\
0.3209321	4.189988e-06\\
0.3210321	4.231931e-06\\
0.3211321	4.274246e-06\\
0.3212321	4.316948e-06\\
0.3213321	4.360099e-06\\
0.3214321	4.403573e-06\\
0.3215322	4.447506e-06\\
0.3216322	4.491913e-06\\
0.3217322	4.536864e-06\\
0.3218322	4.582095e-06\\
0.3219322	4.627703e-06\\
0.3220322	4.673825e-06\\
0.3221322	4.720382e-06\\
0.3222322	4.767339e-06\\
0.3223322	4.814834e-06\\
0.3224322	4.862843e-06\\
0.3225323	4.911149e-06\\
0.3226323	4.960021e-06\\
0.3227323	5.009267e-06\\
0.3228323	5.059114e-06\\
0.3229323	5.109385e-06\\
0.3230323	5.160149e-06\\
0.3231323	5.211585e-06\\
0.3232323	5.263233e-06\\
0.3233323	5.315268e-06\\
0.3234323	5.367988e-06\\
0.3235324	5.421004e-06\\
0.3236324	5.474887e-06\\
0.3237324	5.529152e-06\\
0.3238324	5.584003e-06\\
0.3239324	5.639276e-06\\
0.3240324	5.694964e-06\\
0.3241324	5.751428e-06\\
0.3242324	5.80844e-06\\
0.3243324	5.865703e-06\\
0.3244324	5.923787e-06\\
0.3245325	5.982289e-06\\
0.3246325	6.041346e-06\\
0.3247325	6.100975e-06\\
0.3248325	6.16117e-06\\
0.3249325	6.221919e-06\\
0.3250325	6.28308e-06\\
0.3251325	6.345178e-06\\
0.3252325	6.407758e-06\\
0.3253325	6.470811e-06\\
0.3254325	6.534674e-06\\
0.3255326	6.599008e-06\\
0.3256326	6.664044e-06\\
0.3257326	6.729574e-06\\
0.3258326	6.795529e-06\\
0.3259326	6.862399e-06\\
0.3260326	6.929803e-06\\
0.3261326	6.998073e-06\\
0.3262326	7.066805e-06\\
0.3263326	7.135978e-06\\
0.3264326	7.206065e-06\\
0.3265327	7.276923e-06\\
0.3266327	7.34824e-06\\
0.3267327	7.420253e-06\\
0.3268327	7.492862e-06\\
0.3269327	7.566281e-06\\
0.3270327	7.640398e-06\\
0.3271327	7.71519e-06\\
0.3272327	7.790621e-06\\
0.3273327	7.866831e-06\\
0.3274327	7.943868e-06\\
0.3275328	8.021588e-06\\
0.3276328	8.099865e-06\\
0.3277328	8.179031e-06\\
0.3278328	8.258852e-06\\
0.3279328	8.339261e-06\\
0.3280328	8.420859e-06\\
0.3281328	8.503084e-06\\
0.3282328	8.586082e-06\\
0.3283328	8.669498e-06\\
0.3284328	8.753966e-06\\
0.3285329	8.83941e-06\\
0.3286329	8.925435e-06\\
0.3287329	9.012315e-06\\
0.3288329	9.100015e-06\\
0.3289329	9.18848e-06\\
0.3290329	9.277845e-06\\
0.3291329	9.368075e-06\\
0.3292329	9.45893e-06\\
0.3293329	9.551012e-06\\
0.3294329	9.643833e-06\\
0.329533	9.737341e-06\\
0.329633	9.831624e-06\\
0.329733	9.927054e-06\\
0.329833	1.002328e-05\\
0.329933	1.012036e-05\\
0.330033	1.021843e-05\\
0.330133	1.031754e-05\\
0.330233	1.041724e-05\\
0.330333	1.051825e-05\\
0.330433	1.062003e-05\\
0.3305331	1.072268e-05\\
0.3306331	1.082629e-05\\
0.3307331	1.093085e-05\\
0.3308331	1.103663e-05\\
0.3309331	1.114307e-05\\
0.3310331	1.125084e-05\\
0.3311331	1.135932e-05\\
0.3312331	1.146859e-05\\
0.3313331	1.157945e-05\\
0.3314331	1.169122e-05\\
0.3315332	1.180365e-05\\
0.3316332	1.191739e-05\\
0.3317332	1.203225e-05\\
0.3318332	1.214824e-05\\
0.3319332	1.22648e-05\\
0.3320332	1.23827e-05\\
0.3321332	1.250183e-05\\
0.3322332	1.262204e-05\\
0.3323332	1.274346e-05\\
0.3324332	1.2866e-05\\
0.3325333	1.298937e-05\\
0.3326333	1.3114e-05\\
0.3327333	1.32396e-05\\
0.3328333	1.336666e-05\\
0.3329333	1.349479e-05\\
0.3330333	1.362427e-05\\
0.3331333	1.375501e-05\\
0.3332333	1.388654e-05\\
0.3333333	1.401965e-05\\
0.3334333	1.415362e-05\\
0.3335334	1.428878e-05\\
0.3336334	1.442584e-05\\
0.3337334	1.456376e-05\\
0.3338334	1.470285e-05\\
0.3339334	1.484321e-05\\
0.3340334	1.498485e-05\\
0.3341334	1.512808e-05\\
0.3342334	1.52723e-05\\
0.3343334	1.541822e-05\\
0.3344334	1.556502e-05\\
0.3345335	1.571321e-05\\
0.3346335	1.586324e-05\\
0.3347335	1.601419e-05\\
0.3348335	1.616677e-05\\
0.3349335	1.632079e-05\\
0.3350335	1.647606e-05\\
0.3351335	1.663266e-05\\
0.3352335	1.679053e-05\\
0.3353335	1.695026e-05\\
0.3354335	1.711132e-05\\
0.3355336	1.72737e-05\\
0.3356336	1.743809e-05\\
0.3357336	1.760351e-05\\
0.3358336	1.777033e-05\\
0.3359336	1.79389e-05\\
0.3360336	1.810883e-05\\
0.3361336	1.828051e-05\\
0.3362336	1.845369e-05\\
0.3363336	1.862841e-05\\
0.3364336	1.880464e-05\\
0.3365337	1.898285e-05\\
0.3366337	1.916254e-05\\
0.3367337	1.934334e-05\\
0.3368337	1.952628e-05\\
0.3369337	1.971107e-05\\
0.3370337	1.989686e-05\\
0.3371337	2.008482e-05\\
0.3372337	2.027434e-05\\
0.3373337	2.04658e-05\\
0.3374337	2.065868e-05\\
0.3375338	2.085345e-05\\
0.3376338	2.105007e-05\\
0.3377338	2.124811e-05\\
0.3378338	2.144836e-05\\
0.3379338	2.165043e-05\\
0.3380338	2.185399e-05\\
0.3381338	2.20596e-05\\
0.3382338	2.226671e-05\\
0.3383338	2.247618e-05\\
0.3384338	2.268768e-05\\
0.3385339	2.290074e-05\\
0.3386339	2.311581e-05\\
0.3387339	2.333222e-05\\
0.3388339	2.355159e-05\\
0.3389339	2.377261e-05\\
0.3390339	2.399539e-05\\
0.3391339	2.422041e-05\\
0.3392339	2.444738e-05\\
0.3393339	2.467603e-05\\
0.3394339	2.490709e-05\\
0.339534	2.513999e-05\\
0.339634	2.537527e-05\\
0.339734	2.561238e-05\\
0.339834	2.585201e-05\\
0.339934	2.609363e-05\\
0.340034	2.633722e-05\\
0.340134	2.658355e-05\\
0.340234	2.683132e-05\\
0.340334	2.708175e-05\\
0.340434	2.733428e-05\\
0.3405341	2.758908e-05\\
0.3406341	2.784631e-05\\
0.3407341	2.810578e-05\\
0.3408341	2.836735e-05\\
0.3409341	2.86315e-05\\
0.3410341	2.889792e-05\\
0.3411341	2.916713e-05\\
0.3412341	2.943829e-05\\
0.3413341	2.971161e-05\\
0.3414341	2.998755e-05\\
0.3415342	3.026614e-05\\
0.3416342	3.054748e-05\\
0.3417342	3.083076e-05\\
0.3418342	3.111669e-05\\
0.3419342	3.140541e-05\\
0.3420342	3.169699e-05\\
0.3421342	3.19908e-05\\
0.3422342	3.228699e-05\\
0.3423342	3.258599e-05\\
0.3424342	3.288758e-05\\
0.3425343	3.319157e-05\\
0.3426343	3.349849e-05\\
0.3427343	3.380845e-05\\
0.3428343	3.412094e-05\\
0.3429343	3.443637e-05\\
0.3430343	3.475445e-05\\
0.3431343	3.507541e-05\\
0.3432343	3.539934e-05\\
0.3433343	3.572605e-05\\
0.3434343	3.605502e-05\\
0.3435344	3.638751e-05\\
0.3436344	3.672282e-05\\
0.3437344	3.706085e-05\\
0.3438344	3.740275e-05\\
0.3439344	3.774732e-05\\
0.3440344	3.809428e-05\\
0.3441344	3.844398e-05\\
0.3442344	3.879777e-05\\
0.3443344	3.915469e-05\\
0.3444344	3.95142e-05\\
0.3445345	3.987747e-05\\
0.3446345	4.024349e-05\\
0.3447345	4.061281e-05\\
0.3448345	4.098548e-05\\
0.3449345	4.136097e-05\\
0.3450345	4.174029e-05\\
0.3451345	4.212264e-05\\
0.3452345	4.250877e-05\\
0.3453345	4.289783e-05\\
0.3454345	4.329022e-05\\
0.3455346	4.368656e-05\\
0.3456346	4.408574e-05\\
0.3457346	4.44894e-05\\
0.3458346	4.489629e-05\\
0.3459346	4.530624e-05\\
0.3460346	4.571979e-05\\
0.3461346	4.613692e-05\\
0.3462346	4.655811e-05\\
0.3463346	4.698252e-05\\
0.3464346	4.741095e-05\\
0.3465347	4.784362e-05\\
0.3466347	4.827918e-05\\
0.3467347	4.871875e-05\\
0.3468347	4.916253e-05\\
0.3469347	4.961006e-05\\
0.3470347	5.006184e-05\\
0.3471347	5.051657e-05\\
0.3472347	5.097586e-05\\
0.3473347	5.143907e-05\\
0.3474347	5.190632e-05\\
0.3475348	5.237741e-05\\
0.3476348	5.28526e-05\\
0.3477348	5.333229e-05\\
0.3478348	5.381592e-05\\
0.3479348	5.430426e-05\\
0.3480348	5.479686e-05\\
0.3481348	5.529307e-05\\
0.3482348	5.579338e-05\\
0.3483348	5.629841e-05\\
0.3484348	5.680748e-05\\
0.3485349	5.732181e-05\\
0.3486349	5.784003e-05\\
0.3487349	5.836274e-05\\
0.3488349	5.888998e-05\\
0.3489349	5.942239e-05\\
0.3490349	5.995813e-05\\
0.3491349	6.049931e-05\\
0.3492349	6.104541e-05\\
0.3493349	6.159593e-05\\
0.3494349	6.215125e-05\\
0.349535	6.271102e-05\\
0.349635	6.327613e-05\\
0.349735	6.384545e-05\\
0.349835	6.442061e-05\\
0.349935	6.49999e-05\\
0.350035	6.558415e-05\\
0.350135	6.617395e-05\\
0.350235	6.676846e-05\\
0.350335	6.736842e-05\\
0.350435	6.797331e-05\\
0.3505351	6.858329e-05\\
0.3506351	6.919893e-05\\
0.3507351	6.981924e-05\\
0.3508351	7.044519e-05\\
0.3509351	7.107642e-05\\
0.3510351	7.171336e-05\\
0.3511351	7.235637e-05\\
0.3512351	7.300345e-05\\
0.3513351	7.365649e-05\\
0.3514351	7.431506e-05\\
0.3515352	7.497981e-05\\
0.3516352	7.565032e-05\\
0.3517352	7.632597e-05\\
0.3518352	7.700787e-05\\
0.3519352	7.769501e-05\\
0.3520352	7.838822e-05\\
0.3521352	7.908737e-05\\
0.3522352	7.979252e-05\\
0.3523352	8.05042e-05\\
0.3524352	8.122148e-05\\
0.3525353	8.194462e-05\\
0.3526353	8.267444e-05\\
0.3527353	8.340987e-05\\
0.3528353	8.415154e-05\\
0.3529353	8.489982e-05\\
0.3530353	8.565499e-05\\
0.3531353	8.64159e-05\\
0.3532353	8.718384e-05\\
0.3533353	8.795815e-05\\
0.3534353	8.873848e-05\\
0.3535354	8.95257e-05\\
0.3536354	9.03193e-05\\
0.3537354	9.112018e-05\\
0.3538354	9.192733e-05\\
0.3539354	9.274181e-05\\
0.3540354	9.356264e-05\\
0.3541354	9.439091e-05\\
0.3542354	9.522717e-05\\
0.3543354	9.60686e-05\\
0.3544354	9.691767e-05\\
0.3545355	9.777403e-05\\
0.3546355	9.863758e-05\\
0.3547355	9.950873e-05\\
0.3548355	0.0001003869\\
0.3549355	0.0001012726\\
0.3550355	0.0001021658\\
0.3551355	0.0001030672\\
0.3552355	0.0001039753\\
0.3553355	0.000104891\\
0.3554355	0.0001058145\\
0.3555356	0.0001067456\\
0.3556356	0.0001076851\\
0.3557356	0.0001086324\\
0.3558356	0.0001095868\\
0.3559356	0.0001105501\\
0.3560356	0.0001115218\\
0.3561356	0.0001125012\\
0.3562356	0.000113488\\
0.3563356	0.0001144845\\
0.3564356	0.0001154881\\
0.3565357	0.0001165006\\
0.3566357	0.0001175219\\
0.3567357	0.0001185509\\
0.3568357	0.0001195903\\
0.3569357	0.0001206367\\
0.3570357	0.0001216922\\
0.3571357	0.0001227558\\
0.3572357	0.0001238294\\
0.3573357	0.0001249118\\
0.3574357	0.0001260029\\
0.3575358	0.0001271036\\
0.3576358	0.0001282127\\
0.3577358	0.0001293317\\
0.3578358	0.0001304601\\
0.3579358	0.000131597\\
0.3580358	0.0001327434\\
0.3581358	0.0001339006\\
0.3582358	0.0001350672\\
0.3583358	0.000136243\\
0.3584358	0.0001374278\\
0.3585359	0.0001386228\\
0.3586359	0.0001398281\\
0.3587359	0.000141044\\
0.3588359	0.0001422688\\
0.3589359	0.0001435041\\
0.3590359	0.0001447504\\
0.3591359	0.0001460063\\
0.3592359	0.0001472723\\
0.3593359	0.0001485493\\
0.3594359	0.0001498372\\
0.359536	0.0001511343\\
0.359636	0.000152444\\
0.359736	0.0001537644\\
0.359836	0.000155094\\
0.359936	0.0001564352\\
0.360036	0.0001577871\\
0.360136	0.0001591513\\
0.360236	0.0001605264\\
0.360336	0.0001619131\\
0.360436	0.0001633102\\
0.3605361	0.0001647193\\
0.3606361	0.0001661395\\
0.3607361	0.0001675714\\
0.3608361	0.0001690172\\
0.3609361	0.0001704727\\
0.3610361	0.0001719405\\
0.3611361	0.0001734203\\
0.3612361	0.0001749129\\
0.3613361	0.0001764167\\
0.3614361	0.0001779335\\
0.3615362	0.000179463\\
0.3616362	0.0001810042\\
0.3617362	0.0001825591\\
0.3618362	0.0001841255\\
0.3619362	0.0001857051\\
0.3620362	0.0001872986\\
0.3621362	0.0001889036\\
0.3622362	0.0001905223\\
0.3623362	0.0001921549\\
0.3624362	0.0001938001\\
0.3625363	0.0001954583\\
0.3626363	0.0001971306\\
0.3627363	0.0001988177\\
0.3628363	0.0002005167\\
0.3629363	0.0002022295\\
0.3630363	0.0002039569\\
0.3631363	0.0002056983\\
0.3632363	0.0002074537\\
0.3633363	0.0002092234\\
0.3634363	0.0002110081\\
0.3635364	0.000212807\\
0.3636364	0.0002146202\\
0.3637364	0.0002164479\\
0.3638364	0.000218292\\
0.3639364	0.0002201494\\
0.3640364	0.0002220206\\
0.3641364	0.0002239096\\
0.3642364	0.0002258125\\
0.3643364	0.0002277318\\
0.3644364	0.0002296651\\
0.3645365	0.0002316151\\
0.3646365	0.0002335803\\
0.3647365	0.0002355626\\
0.3648365	0.0002375601\\
0.3649365	0.0002395726\\
0.3650365	0.0002416024\\
0.3651365	0.0002436482\\
0.3652365	0.0002457119\\
0.3653365	0.0002477908\\
0.3654365	0.0002498867\\
0.3655366	0.0002519974\\
0.3656366	0.0002541279\\
0.3657366	0.0002562756\\
0.3658366	0.0002584388\\
0.3659366	0.0002606222\\
0.3660366	0.0002628207\\
0.3661366	0.0002650373\\
0.3662366	0.0002672718\\
0.3663366	0.000269523\\
0.3664366	0.0002717931\\
0.3665367	0.0002740821\\
0.3666367	0.000276389\\
0.3667367	0.0002787138\\
0.3668367	0.0002810587\\
0.3669367	0.0002834207\\
0.3670367	0.0002858021\\
0.3671367	0.0002882028\\
0.3672367	0.0002906232\\
0.3673367	0.0002930625\\
0.3674367	0.0002955205\\
0.3675368	0.000297998\\
0.3676368	0.000300495\\
0.3677368	0.0003030139\\
0.3678368	0.0003055522\\
0.3679368	0.0003081087\\
0.3680368	0.0003106872\\
0.3681368	0.0003132865\\
0.3682368	0.0003159061\\
0.3683368	0.0003185463\\
0.3684368	0.0003212065\\
0.3685369	0.0003238887\\
0.3686369	0.000326593\\
0.3687369	0.0003293178\\
0.3688369	0.0003320646\\
0.3689369	0.0003348337\\
0.3690369	0.0003376233\\
0.3691369	0.0003404351\\
0.3692369	0.0003432696\\
0.3693369	0.0003461268\\
0.3694369	0.0003490064\\
0.369537	0.0003519086\\
0.369637	0.0003548348\\
0.369737	0.0003577825\\
0.369837	0.000360754\\
0.369937	0.0003637495\\
0.370037	0.0003667678\\
0.370137	0.0003698098\\
0.370237	0.0003728755\\
0.370337	0.0003759672\\
0.370437	0.0003790818\\
0.3705371	0.0003822197\\
0.3706371	0.0003853835\\
0.3707371	0.000388574\\
0.3708371	0.0003917877\\
0.3709371	0.0003950265\\
0.3710371	0.0003982912\\
0.3711371	0.0004015805\\
0.3712371	0.0004048965\\
0.3713371	0.0004082375\\
0.3714371	0.0004116058\\
0.3715372	0.0004150007\\
0.3716372	0.0004184212\\
0.3717372	0.0004218698\\
0.3718372	0.000425344\\
0.3719372	0.0004288444\\
0.3720372	0.0004323731\\
0.3721372	0.0004359321\\
0.3722372	0.000439515\\
0.3723372	0.0004431271\\
0.3724372	0.0004467678\\
0.3725373	0.0004504361\\
0.3726373	0.0004541333\\
0.3727373	0.00045786\\
0.3728373	0.0004616167\\
0.3729373	0.0004653993\\
0.3730373	0.0004692131\\
0.3731373	0.0004730552\\
0.3732373	0.000476929\\
0.3733373	0.000480833\\
0.3734373	0.000484765\\
0.3735374	0.00048873\\
0.3736374	0.0004927247\\
0.3737374	0.0004967495\\
0.3738374	0.0005008053\\
0.3739374	0.0005048954\\
0.3740374	0.0005090145\\
0.3741374	0.0005131666\\
0.3742374	0.0005173521\\
0.3743374	0.0005215659\\
0.3744374	0.0005258149\\
0.3745375	0.0005300957\\
0.3746375	0.0005344107\\
0.3747375	0.0005387576\\
0.3748375	0.0005431393\\
0.3749375	0.000547554\\
0.3750375	0.0005520037\\
0.3751375	0.0005564866\\
0.3752375	0.000561003\\
0.3753375	0.0005655553\\
0.3754375	0.0005701426\\
0.3755376	0.0005747662\\
0.3756376	0.0005794231\\
0.3757376	0.0005841177\\
0.3758376	0.0005888467\\
0.3759376	0.0005936133\\
0.3760376	0.000598417\\
0.3761376	0.0006032546\\
0.3762376	0.0006081306\\
0.3763376	0.0006130437\\
0.3764376	0.0006179952\\
0.3765377	0.0006229846\\
0.3766377	0.0006280132\\
0.3767377	0.0006330775\\
0.3768377	0.000638181\\
0.3769377	0.0006433242\\
0.3770377	0.0006485076\\
0.3771377	0.0006537307\\
0.3772377	0.0006589891\\
0.3773377	0.0006642919\\
0.3774377	0.000669634\\
0.3775378	0.0006750177\\
0.3776378	0.0006804405\\
0.3777378	0.0006859051\\
0.3778378	0.000691411\\
0.3779378	0.0006969593\\
0.3780378	0.000702549\\
0.3781378	0.0007081807\\
0.3782378	0.0007138584\\
0.3783378	0.0007195759\\
0.3784378	0.0007253383\\
0.3785379	0.0007311432\\
0.3786379	0.0007369911\\
0.3787379	0.0007428846\\
0.3788379	0.000748823\\
0.3789379	0.0007548066\\
0.3790379	0.0007608337\\
0.3791379	0.0007669082\\
0.3792379	0.0007730262\\
0.3793379	0.0007791909\\
0.3794379	0.0007854044\\
0.379538	0.000791663\\
0.379638	0.0007979684\\
0.379738	0.0008043202\\
0.379838	0.0008107219\\
0.379938	0.0008171721\\
0.380038	0.0008236705\\
0.380138	0.0008302162\\
0.380238	0.0008368125\\
0.380338	0.0008434578\\
0.380438	0.000850152\\
0.3805381	0.000856896\\
0.3806381	0.0008636918\\
0.3807381	0.0008705381\\
0.3808381	0.000877437\\
0.3809381	0.0008843871\\
0.3810381	0.0008913878\\
0.3811381	0.0008984422\\
0.3812381	0.0009055474\\
0.3813381	0.0009127075\\
0.3814381	0.0009199203\\
0.3815382	0.0009271861\\
0.3816382	0.0009345078\\
0.3817382	0.0009418855\\
0.3818382	0.000949313\\
0.3819382	0.0009567992\\
0.3820382	0.0009643417\\
0.3821382	0.0009719382\\
0.3822382	0.0009795938\\
0.3823382	0.0009873042\\
0.3824382	0.0009950732\\
0.3825383	0.001002899\\
0.3826383	0.001010782\\
0.3827383	0.001018722\\
0.3828383	0.001026725\\
0.3829383	0.001034786\\
0.3830383	0.001042907\\
0.3831383	0.001051089\\
0.3832383	0.001059328\\
0.3833383	0.001067631\\
0.3834383	0.00107599\\
0.3835384	0.001084415\\
0.3836384	0.001092903\\
0.3837384	0.001101456\\
0.3838384	0.001110069\\
0.3839384	0.001118743\\
0.3840384	0.001127484\\
0.3841384	0.001136288\\
0.3842384	0.001145159\\
0.3843384	0.001154091\\
0.3844384	0.001163093\\
0.3845385	0.001172159\\
0.3846385	0.001181293\\
0.3847385	0.001190493\\
0.3848385	0.001199761\\
0.3849385	0.001209096\\
0.3850385	0.001218499\\
0.3851385	0.001227975\\
0.3852385	0.001237516\\
0.3853385	0.001247129\\
0.3854385	0.001256811\\
0.3855386	0.001266566\\
0.3856386	0.00127639\\
0.3857386	0.001286287\\
0.3858386	0.001296255\\
0.3859386	0.001306294\\
0.3860386	0.00131641\\
0.3861386	0.001326599\\
0.3862386	0.001336865\\
0.3863386	0.0013472\\
0.3864386	0.001357613\\
0.3865387	0.001368104\\
0.3866387	0.001378667\\
0.3867387	0.001389308\\
0.3868387	0.001400029\\
0.3869387	0.001410826\\
0.3870387	0.0014217\\
0.3871387	0.001432652\\
0.3872387	0.001443687\\
0.3873387	0.001454802\\
0.3874387	0.001465996\\
0.3875388	0.001477271\\
0.3876388	0.001488629\\
0.3877388	0.001500066\\
0.3878388	0.001511587\\
0.3879388	0.001523196\\
0.3880388	0.00153488\\
0.3881388	0.001546653\\
0.3882388	0.001558511\\
0.3883388	0.001570454\\
0.3884388	0.001582484\\
0.3885389	0.001594603\\
0.3886389	0.001606805\\
0.3887389	0.001619094\\
0.3888389	0.001631475\\
0.3889389	0.001643945\\
0.3890389	0.001656506\\
0.3891389	0.001669151\\
0.3892389	0.001681887\\
0.3893389	0.00169472\\
0.3894389	0.001707642\\
0.389539	0.001720658\\
0.389639	0.001733765\\
0.389739	0.001746969\\
0.389839	0.001760265\\
0.389939	0.001773658\\
0.390039	0.001787144\\
0.390139	0.001800728\\
0.390239	0.001814409\\
0.390339	0.001828189\\
0.390439	0.001842067\\
0.3905391	0.00185604\\
0.3906391	0.001870114\\
0.3907391	0.001884291\\
0.3908391	0.001898568\\
0.3909391	0.001912945\\
0.3910391	0.001927428\\
0.3911391	0.001942009\\
0.3912391	0.001956696\\
0.3913391	0.001971487\\
0.3914391	0.001986385\\
0.3915392	0.002001384\\
0.3916392	0.002016494\\
0.3917392	0.002031709\\
0.3918392	0.002047033\\
0.3919392	0.002062461\\
0.3920392	0.002078003\\
0.3921392	0.002093655\\
0.3922392	0.002109416\\
0.3923392	0.00212529\\
0.3924392	0.00214127\\
0.3925393	0.002157368\\
0.3926393	0.002173581\\
0.3927393	0.002189908\\
0.3928393	0.002206345\\
0.3929393	0.002222901\\
0.3930393	0.002239572\\
0.3931393	0.002256362\\
0.3932393	0.002273272\\
0.3933393	0.002290298\\
0.3934393	0.002307445\\
0.3935394	0.00232471\\
0.3936394	0.002342097\\
0.3937394	0.002359607\\
0.3938394	0.002377239\\
0.3939394	0.002394994\\
0.3940394	0.002412876\\
0.3941394	0.00243088\\
0.3942394	0.002449011\\
0.3943394	0.002467267\\
0.3944394	0.002485652\\
0.3945395	0.002504165\\
0.3946395	0.002522812\\
0.3947395	0.002541585\\
0.3948395	0.00256049\\
0.3949395	0.002579525\\
0.3950395	0.00259869\\
0.3951395	0.002617993\\
0.3952395	0.002637426\\
0.3953395	0.002657\\
0.3954395	0.002676708\\
0.3955396	0.002696551\\
0.3956396	0.00271653\\
0.3957396	0.002736654\\
0.3958396	0.002756913\\
0.3959396	0.002777311\\
0.3960396	0.002797851\\
0.3961396	0.002818537\\
0.3962396	0.002839366\\
0.3963396	0.002860335\\
0.3964396	0.002881451\\
0.3965397	0.002902714\\
0.3966397	0.002924121\\
0.3967397	0.002945678\\
0.3968397	0.002967385\\
0.3969397	0.002989237\\
0.3970397	0.003011243\\
0.3971397	0.003033398\\
0.3972397	0.003055707\\
0.3973397	0.00307817\\
0.3974397	0.003100792\\
0.3975398	0.003123563\\
0.3976398	0.003146491\\
0.3977398	0.003169577\\
0.3978398	0.003192823\\
0.3979398	0.003216226\\
0.3980398	0.00323979\\
0.3981398	0.00326352\\
0.3982398	0.003287406\\
0.3983398	0.003311457\\
0.3984398	0.003335674\\
0.3985399	0.003360055\\
0.3986399	0.003384605\\
0.3987399	0.003409323\\
0.3988399	0.003434207\\
0.3989399	0.00345926\\
0.3990399	0.003484486\\
0.3991399	0.003509884\\
0.3992399	0.003535449\\
0.3993399	0.003561196\\
0.3994399	0.003587118\\
0.39954	0.003613216\\
0.39964	0.003639485\\
0.39974	0.003665935\\
0.39984	0.003692568\\
0.39994	0.003719375\\
0.40004	0.003746372\\
};
\addplot [color=mycolor1,solid,forget plot]
  table[row sep=crcr]{%
0.40004	0.003746372\\
0.40014	0.003773548\\
0.40024	0.003800909\\
0.40034	0.003828451\\
0.40044	0.003856185\\
0.4005401	0.003884103\\
0.4006401	0.003912215\\
0.4007401	0.003940507\\
0.4008401	0.003968993\\
0.4009401	0.00399768\\
0.4010401	0.004026555\\
0.4011401	0.004055622\\
0.4012401	0.004084886\\
0.4013401	0.004114345\\
0.4014401	0.004144008\\
0.4015402	0.004173868\\
0.4016402	0.004203925\\
0.4017402	0.00423419\\
0.4018402	0.004264652\\
0.4019402	0.004295326\\
0.4020402	0.0043262\\
0.4021402	0.004357282\\
0.4022402	0.004388569\\
0.4023402	0.00442007\\
0.4024402	0.004451776\\
0.4025403	0.004483702\\
0.4026403	0.004515841\\
0.4027403	0.004548189\\
0.4028403	0.004580756\\
0.4029403	0.004613536\\
0.4030403	0.00464654\\
0.4031403	0.00467977\\
0.4032403	0.00471321\\
0.4033403	0.004746875\\
0.4034403	0.004780765\\
0.4035404	0.004814876\\
0.4036404	0.00484922\\
0.4037404	0.004883795\\
0.4038404	0.004918599\\
0.4039404	0.004953625\\
0.4040404	0.004988887\\
0.4041404	0.005024385\\
0.4042404	0.005060119\\
0.4043404	0.005096083\\
0.4044404	0.005132291\\
0.4045405	0.005168743\\
0.4046405	0.005205428\\
0.4047405	0.005242356\\
0.4048405	0.00527953\\
0.4049405	0.005316944\\
0.4050405	0.005354615\\
0.4051405	0.005392528\\
0.4052405	0.00543069\\
0.4053405	0.005469106\\
0.4054405	0.005507772\\
0.4055406	0.005546696\\
0.4056406	0.005585871\\
0.4057406	0.005625309\\
0.4058406	0.005664998\\
0.4059406	0.005704954\\
0.4060406	0.00574517\\
0.4061406	0.005785653\\
0.4062406	0.005826393\\
0.4063406	0.005867406\\
0.4064406	0.005908688\\
0.4065407	0.005950237\\
0.4066407	0.005992055\\
0.4067407	0.006034152\\
0.4068407	0.006076523\\
0.4069407	0.006119167\\
0.4070407	0.00616209\\
0.4071407	0.0062053\\
0.4072407	0.006248782\\
0.4073407	0.006292547\\
0.4074407	0.006336603\\
0.4075408	0.006380935\\
0.4076408	0.006425566\\
0.4077408	0.006470483\\
0.4078408	0.006515692\\
0.4079408	0.006561197\\
0.4080408	0.006606996\\
0.4081408	0.00665309\\
0.4082408	0.006699482\\
0.4083408	0.006746171\\
0.4084408	0.006793168\\
0.4085409	0.006840462\\
0.4086409	0.006888065\\
0.4087409	0.006935981\\
0.4088409	0.006984203\\
0.4089409	0.007032727\\
0.4090409	0.007081579\\
0.4091409	0.007130733\\
0.4092409	0.007180211\\
0.4093409	0.007230007\\
0.4094409	0.00728012\\
0.409541	0.007330558\\
0.409641	0.007381322\\
0.409741	0.007432402\\
0.409841	0.007483816\\
0.409941	0.007535557\\
0.410041	0.007587641\\
0.410141	0.007640047\\
0.410241	0.007692787\\
0.410341	0.007745872\\
0.410441	0.007799295\\
0.4105411	0.007853055\\
0.4106411	0.007907161\\
0.4107411	0.007961621\\
0.4108411	0.008016414\\
0.4109411	0.008071563\\
0.4110411	0.008127058\\
0.4111411	0.008182912\\
0.4112411	0.008239125\\
0.4113411	0.008295692\\
0.4114411	0.008352615\\
0.4115412	0.008409904\\
0.4116412	0.008467554\\
0.4117412	0.008525573\\
0.4118412	0.00858395\\
0.4119412	0.008642716\\
0.4120412	0.008701839\\
0.4121412	0.008761339\\
0.4122412	0.008821217\\
0.4123412	0.008881475\\
0.4124412	0.00894211\\
0.4125413	0.009003132\\
0.4126413	0.009064542\\
0.4127413	0.009126338\\
0.4128413	0.009188514\\
0.4129413	0.009251091\\
0.4130413	0.00931406\\
0.4131413	0.009377419\\
0.4132413	0.009441183\\
0.4133413	0.009505352\\
0.4134413	0.009569918\\
0.4135414	0.009634892\\
0.4136414	0.009700272\\
0.4137414	0.009766059\\
0.4138414	0.009832268\\
0.4139414	0.009898886\\
0.4140414	0.009965916\\
0.4141414	0.01003337\\
0.4142414	0.01010125\\
0.4143414	0.01016955\\
0.4144414	0.01023827\\
0.4145415	0.01030743\\
0.4146415	0.01037702\\
0.4147415	0.01044703\\
0.4148415	0.01051749\\
0.4149415	0.01058839\\
0.4150415	0.01065972\\
0.4151415	0.01073151\\
0.4152415	0.01080374\\
0.4153415	0.01087641\\
0.4154415	0.01094954\\
0.4155416	0.01102311\\
0.4156416	0.01109716\\
0.4157416	0.01117165\\
0.4158416	0.01124661\\
0.4159416	0.01132202\\
0.4160416	0.01139791\\
0.4161416	0.01147427\\
0.4162416	0.0115511\\
0.4163416	0.0116284\\
0.4164416	0.01170618\\
0.4165417	0.01178444\\
0.4166417	0.01186318\\
0.4167417	0.01194241\\
0.4168417	0.01202212\\
0.4169417	0.01210233\\
0.4170417	0.01218302\\
0.4171417	0.01226421\\
0.4172417	0.01234591\\
0.4173417	0.0124281\\
0.4174417	0.0125108\\
0.4175418	0.01259401\\
0.4176418	0.01267772\\
0.4177418	0.01276194\\
0.4178418	0.01284668\\
0.4179418	0.01293195\\
0.4180418	0.01301773\\
0.4181418	0.01310403\\
0.4182418	0.01319086\\
0.4183418	0.01327823\\
0.4184418	0.01336613\\
0.4185419	0.01345455\\
0.4186419	0.01354351\\
0.4187419	0.01363303\\
0.4188419	0.01372308\\
0.4189419	0.01381368\\
0.4190419	0.01390483\\
0.4191419	0.01399654\\
0.4192419	0.01408879\\
0.4193419	0.01418162\\
0.4194419	0.014275\\
0.419542	0.01436895\\
0.419642	0.01446347\\
0.419742	0.01455855\\
0.419842	0.01465421\\
0.419942	0.01475045\\
0.420042	0.01484727\\
0.420142	0.01494468\\
0.420242	0.01504267\\
0.420342	0.01514126\\
0.420442	0.01524043\\
0.4205421	0.01534021\\
0.4206421	0.01544058\\
0.4207421	0.01554156\\
0.4208421	0.01564314\\
0.4209421	0.01574534\\
0.4210421	0.01584814\\
0.4211421	0.01595156\\
0.4212421	0.0160556\\
0.4213421	0.01616027\\
0.4214421	0.01626556\\
0.4215422	0.01637147\\
0.4216422	0.01647804\\
0.4217422	0.01658523\\
0.4218422	0.01669305\\
0.4219422	0.01680153\\
0.4220422	0.01691063\\
0.4221422	0.01702042\\
0.4222422	0.01713084\\
0.4223422	0.01724192\\
0.4224422	0.01735366\\
0.4225423	0.01746607\\
0.4226423	0.01757914\\
0.4227423	0.01769288\\
0.4228423	0.0178073\\
0.4229423	0.0179224\\
0.4230423	0.01803817\\
0.4231423	0.01815463\\
0.4232423	0.01827179\\
0.4233423	0.01838963\\
0.4234423	0.01850817\\
0.4235424	0.01862743\\
0.4236424	0.01874737\\
0.4237424	0.01886801\\
0.4238424	0.01898938\\
0.4239424	0.01911146\\
0.4240424	0.01923425\\
0.4241424	0.01935777\\
0.4242424	0.01948202\\
0.4243424	0.01960699\\
0.4244424	0.0197327\\
0.4245425	0.01985916\\
0.4246425	0.01998634\\
0.4247425	0.02011427\\
0.4248425	0.02024296\\
0.4249425	0.02037239\\
0.4250425	0.02050258\\
0.4251425	0.02063354\\
0.4252425	0.02076525\\
0.4253425	0.02089774\\
0.4254425	0.021031\\
0.4255426	0.02116505\\
0.4256426	0.02129986\\
0.4257426	0.02143546\\
0.4258426	0.02157186\\
0.4259426	0.02170904\\
0.4260426	0.02184703\\
0.4261426	0.02198581\\
0.4262426	0.0221254\\
0.4263426	0.02226581\\
0.4264426	0.022407\\
0.4265427	0.02254904\\
0.4266427	0.0226919\\
0.4267427	0.02283557\\
0.4268427	0.02298008\\
0.4269427	0.02312543\\
0.4270427	0.02327161\\
0.4271427	0.02341864\\
0.4272427	0.02356651\\
0.4273427	0.02371525\\
0.4274427	0.02386482\\
0.4275428	0.02401527\\
0.4276428	0.02416658\\
0.4277428	0.02431876\\
0.4278428	0.02447181\\
0.4279428	0.02462574\\
0.4280428	0.02478055\\
0.4281428	0.02493626\\
0.4282428	0.02509285\\
0.4283428	0.02525035\\
0.4284428	0.02540873\\
0.4285429	0.02556804\\
0.4286429	0.02572824\\
0.4287429	0.02588935\\
0.4288429	0.0260514\\
0.4289429	0.02621437\\
0.4290429	0.02637826\\
0.4291429	0.02654309\\
0.4292429	0.02670886\\
0.4293429	0.02687555\\
0.4294429	0.0270432\\
0.429543	0.02721181\\
0.429643	0.02738138\\
0.429743	0.0275519\\
0.429843	0.0277234\\
0.429943	0.02789586\\
0.430043	0.02806931\\
0.430143	0.02824372\\
0.430243	0.02841912\\
0.430343	0.02859552\\
0.430443	0.02877292\\
0.4305431	0.02895131\\
0.4306431	0.02913071\\
0.4307431	0.02931112\\
0.4308431	0.02949255\\
0.4309431	0.029675\\
0.4310431	0.02985848\\
0.4311431	0.03004297\\
0.4312431	0.03022851\\
0.4313431	0.03041508\\
0.4314431	0.03060272\\
0.4315432	0.03079139\\
0.4316432	0.03098113\\
0.4317432	0.03117192\\
0.4318432	0.03136379\\
0.4319432	0.03155673\\
0.4320432	0.03175075\\
0.4321432	0.03194583\\
0.4322432	0.03214202\\
0.4323432	0.0323393\\
0.4324432	0.03253768\\
0.4325433	0.03273716\\
0.4326433	0.03293775\\
0.4327433	0.03313945\\
0.4328433	0.03334228\\
0.4329433	0.03354623\\
0.4330433	0.03375132\\
0.4331433	0.03395753\\
0.4332433	0.0341649\\
0.4333433	0.03437342\\
0.4334433	0.03458308\\
0.4335434	0.03479389\\
0.4336434	0.03500587\\
0.4337434	0.03521903\\
0.4338434	0.03543337\\
0.4339434	0.03564889\\
0.4340434	0.03586559\\
0.4341434	0.03608348\\
0.4342434	0.03630258\\
0.4343434	0.03652288\\
0.4344434	0.03674439\\
0.4345435	0.03696712\\
0.4346435	0.03719105\\
0.4347435	0.03741625\\
0.4348435	0.03764264\\
0.4349435	0.03787029\\
0.4350435	0.03809918\\
0.4351435	0.03832933\\
0.4352435	0.03856072\\
0.4353435	0.0387934\\
0.4354435	0.03902734\\
0.4355436	0.03926255\\
0.4356436	0.03949903\\
0.4357436	0.03973682\\
0.4358436	0.03997589\\
0.4359436	0.04021627\\
0.4360436	0.04045795\\
0.4361436	0.04070094\\
0.4362436	0.04094525\\
0.4363436	0.04119089\\
0.4364436	0.04143784\\
0.4365437	0.04168615\\
0.4366437	0.04193579\\
0.4367437	0.04218681\\
0.4368437	0.04243916\\
0.4369437	0.04269287\\
0.4370437	0.04294795\\
0.4371437	0.04320441\\
0.4372437	0.04346225\\
0.4373437	0.04372148\\
0.4374437	0.0439821\\
0.4375438	0.04424414\\
0.4376438	0.04450756\\
0.4377438	0.04477242\\
0.4378438	0.04503867\\
0.4379438	0.04530637\\
0.4380438	0.04557551\\
0.4381438	0.04584607\\
0.4382438	0.04611809\\
0.4383438	0.04639155\\
0.4384438	0.0466665\\
0.4385439	0.04694291\\
0.4386439	0.04722078\\
0.4387439	0.04750014\\
0.4388439	0.04778098\\
0.4389439	0.04806331\\
0.4390439	0.04834716\\
0.4391439	0.04863252\\
0.4392439	0.04891939\\
0.4393439	0.04920777\\
0.4394439	0.04949768\\
0.439544	0.04978914\\
0.439644	0.05008214\\
0.439744	0.0503767\\
0.439844	0.05067281\\
0.439944	0.05097046\\
0.440044	0.05126972\\
0.440144	0.05157054\\
0.440244	0.05187295\\
0.440344	0.05217695\\
0.440444	0.05248256\\
0.4405441	0.05278979\\
0.4406441	0.05309861\\
0.4407441	0.05340904\\
0.4408441	0.05372113\\
0.4409441	0.05403486\\
0.4410441	0.05435022\\
0.4411441	0.05466723\\
0.4412441	0.05498591\\
0.4413441	0.05530626\\
0.4414441	0.05562827\\
0.4415442	0.05595198\\
0.4416442	0.05627737\\
0.4417442	0.05660446\\
0.4418442	0.05693325\\
0.4419442	0.05726375\\
0.4420442	0.05759597\\
0.4421442	0.05792992\\
0.4422442	0.05826561\\
0.4423442	0.05860305\\
0.4424442	0.05894224\\
0.4425443	0.05928318\\
0.4426443	0.05962588\\
0.4427443	0.05997038\\
0.4428443	0.06031663\\
0.4429443	0.06066468\\
0.4430443	0.06101454\\
0.4431443	0.06136621\\
0.4432443	0.06171968\\
0.4433443	0.06207499\\
0.4434443	0.06243212\\
0.4435444	0.06279107\\
0.4436444	0.0631519\\
0.4437444	0.06351457\\
0.4438444	0.0638791\\
0.4439444	0.06424551\\
0.4440444	0.06461378\\
0.4441444	0.06498393\\
0.4442444	0.065356\\
0.4443444	0.06572998\\
0.4444444	0.06610586\\
0.4445445	0.06648365\\
0.4446445	0.06686339\\
0.4447445	0.06724504\\
0.4448445	0.06762867\\
0.4449445	0.06801424\\
0.4450445	0.06840176\\
0.4451445	0.06879126\\
0.4452445	0.06918274\\
0.4453445	0.0695762\\
0.4454445	0.06997168\\
0.4455446	0.07036915\\
0.4456446	0.07076864\\
0.4457446	0.07117014\\
0.4458446	0.07157368\\
0.4459446	0.07197927\\
0.4460446	0.07238688\\
0.4461446	0.07279657\\
0.4462446	0.07320831\\
0.4463446	0.07362215\\
0.4464446	0.07403807\\
0.4465447	0.07445607\\
0.4466447	0.07487619\\
0.4467447	0.07529842\\
0.4468447	0.07572273\\
0.4469447	0.0761492\\
0.4470447	0.0765778\\
0.4471447	0.07700857\\
0.4472447	0.07744147\\
0.4473447	0.07787654\\
0.4474447	0.0783138\\
0.4475448	0.07875324\\
0.4476448	0.07919486\\
0.4477448	0.07963869\\
0.4478448	0.08008473\\
0.4479448	0.08053301\\
0.4480448	0.08098348\\
0.4481448	0.08143622\\
0.4482448	0.08189119\\
0.4483448	0.08234842\\
0.4484448	0.08280795\\
0.4485449	0.08326972\\
0.4486449	0.08373378\\
0.4487449	0.08420016\\
0.4488449	0.08466884\\
0.4489449	0.08513983\\
0.4490449	0.08561314\\
0.4491449	0.08608879\\
0.4492449	0.08656677\\
0.4493449	0.08704711\\
0.4494449	0.08752984\\
0.449545	0.08801492\\
0.449645	0.08850239\\
0.449745	0.08899225\\
0.449845	0.08948452\\
0.449945	0.08997919\\
0.450045	0.09047629\\
0.450145	0.09097581\\
0.450245	0.09147781\\
0.450345	0.09198224\\
0.450445	0.09248912\\
0.4505451	0.0929985\\
0.4506451	0.09351034\\
0.4507451	0.09402469\\
0.4508451	0.09454153\\
0.4509451	0.09506088\\
0.4510451	0.09558276\\
0.4511451	0.09610719\\
0.4512451	0.09663416\\
0.4513451	0.09716369\\
0.4514451	0.09769574\\
0.4515452	0.09823042\\
0.4516452	0.09876766\\
0.4517452	0.0993075\\
0.4518452	0.09984994\\
0.4519452	0.100395\\
0.4520452	0.1009427\\
0.4521452	0.101493\\
0.4522452	0.102046\\
0.4523452	0.1026016\\
0.4524452	0.10316\\
0.4525453	0.103721\\
0.4526453	0.1042846\\
0.4527453	0.104851\\
0.4528453	0.1054201\\
0.4529453	0.1059919\\
0.4530453	0.1065665\\
0.4531453	0.1071438\\
0.4532453	0.1077238\\
0.4533453	0.1083067\\
0.4534453	0.1088923\\
0.4535454	0.1094807\\
0.4536454	0.1100719\\
0.4537454	0.1106659\\
0.4538454	0.1112627\\
0.4539454	0.1118623\\
0.4540454	0.1124649\\
0.4541454	0.1130702\\
0.4542454	0.1136784\\
0.4543454	0.1142895\\
0.4544454	0.1149035\\
0.4545455	0.1155204\\
0.4546455	0.1161402\\
0.4547455	0.116763\\
0.4548455	0.1173886\\
0.4549455	0.1180173\\
0.4550455	0.1186488\\
0.4551455	0.1192834\\
0.4552455	0.1199209\\
0.4553455	0.1205614\\
0.4554455	0.1212049\\
0.4555456	0.1218514\\
0.4556456	0.122501\\
0.4557456	0.1231536\\
0.4558456	0.1238092\\
0.4559456	0.1244679\\
0.4560456	0.1251297\\
0.4561456	0.1257946\\
0.4562456	0.1264626\\
0.4563456	0.1271336\\
0.4564456	0.1278078\\
0.4565457	0.1284851\\
0.4566457	0.1291656\\
0.4567457	0.1298493\\
0.4568457	0.1305361\\
0.4569457	0.131226\\
0.4570457	0.1319192\\
0.4571457	0.1326156\\
0.4572457	0.1333152\\
0.4573457	0.134018\\
0.4574457	0.134724\\
0.4575458	0.1354334\\
0.4576458	0.1361459\\
0.4577458	0.1368618\\
0.4578458	0.1375809\\
0.4579458	0.1383034\\
0.4580458	0.1390292\\
0.4581458	0.1397583\\
0.4582458	0.1404907\\
0.4583458	0.1412265\\
0.4584458	0.1419656\\
0.4585459	0.1427082\\
0.4586459	0.1434541\\
0.4587459	0.1442034\\
0.4588459	0.1449562\\
0.4589459	0.1457123\\
0.4590459	0.1464719\\
0.4591459	0.147235\\
0.4592459	0.1480015\\
0.4593459	0.1487715\\
0.4594459	0.149545\\
0.459546	0.150322\\
0.459646	0.1511025\\
0.459746	0.1518865\\
0.459846	0.1526741\\
0.459946	0.1534653\\
0.460046	0.15426\\
0.460146	0.1550582\\
0.460246	0.1558601\\
0.460346	0.1566656\\
0.460446	0.1574747\\
0.4605461	0.1582875\\
0.4606461	0.1591039\\
0.4607461	0.1599239\\
0.4608461	0.1607476\\
0.4609461	0.1615751\\
0.4610461	0.1624061\\
0.4611461	0.163241\\
0.4612461	0.1640795\\
0.4613461	0.1649218\\
0.4614461	0.1657679\\
0.4615462	0.1666177\\
0.4616462	0.1674713\\
0.4617462	0.1683287\\
0.4618462	0.1691898\\
0.4619462	0.1700548\\
0.4620462	0.1709237\\
0.4621462	0.1717964\\
0.4622462	0.1726729\\
0.4623462	0.1735533\\
0.4624462	0.1744377\\
0.4625463	0.1753259\\
0.4626463	0.176218\\
0.4627463	0.1771141\\
0.4628463	0.1780141\\
0.4629463	0.178918\\
0.4630463	0.179826\\
0.4631463	0.1807379\\
0.4632463	0.1816538\\
0.4633463	0.1825738\\
0.4634463	0.1834977\\
0.4635464	0.1844257\\
0.4636464	0.1853577\\
0.4637464	0.1862939\\
0.4638464	0.1872341\\
0.4639464	0.1881783\\
0.4640464	0.1891268\\
0.4641464	0.1900793\\
0.4642464	0.191036\\
0.4643464	0.1919968\\
0.4644464	0.1929618\\
0.4645465	0.193931\\
0.4646465	0.1949044\\
0.4647465	0.195882\\
0.4648465	0.1968638\\
0.4649465	0.1978499\\
0.4650465	0.1988401\\
0.4651465	0.1998347\\
0.4652465	0.2008336\\
0.4653465	0.2018367\\
0.4654465	0.2028442\\
0.4655466	0.203856\\
0.4656466	0.2048722\\
0.4657466	0.2058927\\
0.4658466	0.2069175\\
0.4659466	0.2079468\\
0.4660466	0.2089804\\
0.4661466	0.2100185\\
0.4662466	0.211061\\
0.4663466	0.2121079\\
0.4664466	0.2131593\\
0.4665467	0.2142152\\
0.4666467	0.2152756\\
0.4667467	0.2163405\\
0.4668467	0.2174098\\
0.4669467	0.2184837\\
0.4670467	0.2195622\\
0.4671467	0.2206452\\
0.4672467	0.2217329\\
0.4673467	0.2228251\\
0.4674467	0.2239219\\
0.4675468	0.2250233\\
0.4676468	0.2261294\\
0.4677468	0.2272401\\
0.4678468	0.2283555\\
0.4679468	0.2294756\\
0.4680468	0.2306004\\
0.4681468	0.2317299\\
0.4682468	0.2328641\\
0.4683468	0.2340031\\
0.4684468	0.2351468\\
0.4685469	0.2362953\\
0.4686469	0.2374486\\
0.4687469	0.2386067\\
0.4688469	0.2397697\\
0.4689469	0.2409375\\
0.4690469	0.24211\\
0.4691469	0.2432875\\
0.4692469	0.24447\\
0.4693469	0.2456572\\
0.4694469	0.2468494\\
0.469547	0.2480466\\
0.469647	0.2492486\\
0.469747	0.2504557\\
0.469847	0.2516677\\
0.469947	0.2528847\\
0.470047	0.2541067\\
0.470147	0.2553338\\
0.470247	0.2565659\\
0.470347	0.257803\\
0.470447	0.2590452\\
0.4705471	0.2602925\\
0.4706471	0.2615449\\
0.4707471	0.2628025\\
0.4708471	0.2640652\\
0.4709471	0.265333\\
0.4710471	0.266606\\
0.4711471	0.2678842\\
0.4712471	0.2691675\\
0.4713471	0.2704561\\
0.4714471	0.27175\\
0.4715472	0.2730491\\
0.4716472	0.2743534\\
0.4717472	0.275663\\
0.4718472	0.276978\\
0.4719472	0.2782982\\
0.4720472	0.2796238\\
0.4721472	0.2809547\\
0.4722472	0.282291\\
0.4723472	0.2836326\\
0.4724472	0.2849797\\
0.4725473	0.2863321\\
0.4726473	0.28769\\
0.4727473	0.2890534\\
0.4728473	0.2904222\\
0.4729473	0.2917965\\
0.4730473	0.2931762\\
0.4731473	0.2945615\\
0.4732473	0.2959523\\
0.4733473	0.2973487\\
0.4734473	0.2987506\\
0.4735474	0.300158\\
0.4736474	0.3015711\\
0.4737474	0.3029898\\
0.4738474	0.3044141\\
0.4739474	0.3058441\\
0.4740474	0.3072796\\
0.4741474	0.3087209\\
0.4742474	0.3101679\\
0.4743474	0.3116206\\
0.4744474	0.313079\\
0.4745475	0.3145432\\
0.4746475	0.3160131\\
0.4747475	0.3174888\\
0.4748475	0.3189702\\
0.4749475	0.3204575\\
0.4750475	0.3219507\\
0.4751475	0.3234496\\
0.4752475	0.3249545\\
0.4753475	0.3264652\\
0.4754475	0.3279817\\
0.4755476	0.3295043\\
0.4756476	0.3310327\\
0.4757476	0.3325671\\
0.4758476	0.3341074\\
0.4759476	0.3356537\\
0.4760476	0.337206\\
0.4761476	0.3387643\\
0.4762476	0.3403287\\
0.4763476	0.3418991\\
0.4764476	0.3434755\\
0.4765477	0.3450581\\
0.4766477	0.3466467\\
0.4767477	0.3482415\\
0.4768477	0.3498424\\
0.4769477	0.3514495\\
0.4770477	0.3530626\\
0.4771477	0.354682\\
0.4772477	0.3563076\\
0.4773477	0.3579394\\
0.4774477	0.3595774\\
0.4775478	0.3612217\\
0.4776478	0.3628723\\
0.4777478	0.3645291\\
0.4778478	0.3661923\\
0.4779478	0.3678617\\
0.4780478	0.3695376\\
0.4781478	0.3712197\\
0.4782478	0.3729083\\
0.4783478	0.3746032\\
0.4784478	0.3763045\\
0.4785479	0.3780123\\
0.4786479	0.3797265\\
0.4787479	0.3814472\\
0.4788479	0.3831743\\
0.4789479	0.384908\\
0.4790479	0.3866481\\
0.4791479	0.3883948\\
0.4792479	0.390148\\
0.4793479	0.3919078\\
0.4794479	0.3936742\\
0.479548	0.3954472\\
0.479648	0.3972268\\
0.479748	0.399013\\
0.479848	0.4008059\\
0.479948	0.4026054\\
0.480048	0.4044116\\
0.480148	0.4062246\\
0.480248	0.4080442\\
0.480348	0.4098706\\
0.480448	0.4117038\\
0.4805481	0.4135437\\
0.4806481	0.4153905\\
0.4807481	0.417244\\
0.4808481	0.4191043\\
0.4809481	0.4209715\\
0.4810481	0.4228456\\
0.4811481	0.4247266\\
0.4812481	0.4266145\\
0.4813481	0.4285092\\
0.4814481	0.4304109\\
0.4815482	0.4323195\\
0.4816482	0.4342351\\
0.4817482	0.4361578\\
0.4818482	0.4380873\\
0.4819482	0.440024\\
0.4820482	0.4419677\\
0.4821482	0.4439184\\
0.4822482	0.4458762\\
0.4823482	0.4478411\\
0.4824482	0.4498131\\
0.4825483	0.4517922\\
0.4826483	0.4537785\\
0.4827483	0.4557719\\
0.4828483	0.4577725\\
0.4829483	0.4597803\\
0.4830483	0.4617954\\
0.4831483	0.4638176\\
0.4832483	0.4658471\\
0.4833483	0.4678839\\
0.4834483	0.469928\\
0.4835484	0.4719794\\
0.4836484	0.4740381\\
0.4837484	0.4761041\\
0.4838484	0.4781775\\
0.4839484	0.4802583\\
0.4840484	0.4823465\\
0.4841484	0.484442\\
0.4842484	0.4865451\\
0.4843484	0.4886555\\
0.4844484	0.4907734\\
0.4845485	0.4928988\\
0.4846485	0.4950318\\
0.4847485	0.4971722\\
0.4848485	0.4993201\\
0.4849485	0.5014756\\
0.4850485	0.5036387\\
0.4851485	0.5058094\\
0.4852485	0.5079876\\
0.4853485	0.5101735\\
0.4854485	0.5123671\\
0.4855486	0.5145683\\
0.4856486	0.5167772\\
0.4857486	0.5189937\\
0.4858486	0.521218\\
0.4859486	0.52345\\
0.4860486	0.5256898\\
0.4861486	0.5279373\\
0.4862486	0.5301926\\
0.4863486	0.5324557\\
0.4864486	0.5347266\\
0.4865487	0.5370054\\
0.4866487	0.539292\\
0.4867487	0.5415865\\
0.4868487	0.5438889\\
0.4869487	0.5461992\\
0.4870487	0.5485173\\
0.4871487	0.5508435\\
0.4872487	0.5531776\\
0.4873487	0.5555198\\
0.4874487	0.5578698\\
0.4875488	0.5602279\\
0.4876488	0.562594\\
0.4877488	0.5649682\\
0.4878488	0.5673505\\
0.4879488	0.5697408\\
0.4880488	0.5721392\\
0.4881488	0.5745458\\
0.4882488	0.5769604\\
0.4883488	0.5793832\\
0.4884488	0.5818142\\
0.4885489	0.5842534\\
0.4886489	0.5867008\\
0.4887489	0.5891565\\
0.4888489	0.5916203\\
0.4889489	0.5940924\\
0.4890489	0.5965728\\
0.4891489	0.5990615\\
0.4892489	0.6015585\\
0.4893489	0.6040638\\
0.4894489	0.6065775\\
0.489549	0.6090995\\
0.489649	0.6116299\\
0.489749	0.6141687\\
0.489849	0.616716\\
0.489949	0.6192716\\
0.490049	0.6218357\\
0.490149	0.6244083\\
0.490249	0.6269893\\
0.490349	0.6295789\\
0.490449	0.632177\\
0.4905491	0.6347836\\
0.4906491	0.6373988\\
0.4907491	0.6400225\\
0.4908491	0.6426548\\
0.4909491	0.6452957\\
0.4910491	0.6479453\\
0.4911491	0.6506035\\
0.4912491	0.6532703\\
0.4913491	0.6559458\\
0.4914491	0.65863\\
0.4915492	0.6613229\\
0.4916492	0.6640245\\
0.4917492	0.6667349\\
0.4918492	0.669454\\
0.4919492	0.6721819\\
0.4920492	0.6749186\\
0.4921492	0.6776642\\
0.4922492	0.6804185\\
0.4923492	0.6831816\\
0.4924492	0.6859537\\
0.4925493	0.6887346\\
0.4926493	0.6915244\\
0.4927493	0.6943231\\
0.4928493	0.6971307\\
0.4929493	0.6999473\\
0.4930493	0.7027728\\
0.4931493	0.7056074\\
0.4932493	0.7084508\\
0.4933493	0.7113033\\
0.4934493	0.7141649\\
0.4935494	0.7170355\\
0.4936494	0.7199152\\
0.4937494	0.7228039\\
0.4938494	0.7257017\\
0.4939494	0.7286086\\
0.4940494	0.7315247\\
0.4941494	0.7344498\\
0.4942494	0.7373841\\
0.4943494	0.7403277\\
0.4944494	0.7432804\\
0.4945495	0.7462423\\
0.4946495	0.7492134\\
0.4947495	0.7521938\\
0.4948495	0.7551834\\
0.4949495	0.7581824\\
0.4950495	0.7611906\\
0.4951495	0.764208\\
0.4952495	0.7672349\\
0.4953495	0.770271\\
0.4954495	0.7733166\\
0.4955496	0.7763714\\
0.4956496	0.7794357\\
0.4957496	0.7825093\\
0.4958496	0.7855923\\
0.4959496	0.7886849\\
0.4960496	0.7917868\\
0.4961496	0.7948982\\
0.4962496	0.7980191\\
0.4963496	0.8011496\\
0.4964496	0.8042894\\
0.4965497	0.8074388\\
0.4966497	0.8105978\\
0.4967497	0.8137663\\
0.4968497	0.8169444\\
0.4969497	0.820132\\
0.4970497	0.8233294\\
0.4971497	0.8265362\\
0.4972497	0.8297527\\
0.4973497	0.8329789\\
0.4974497	0.8362149\\
0.4975498	0.8394603\\
0.4976498	0.8427155\\
0.4977498	0.8459804\\
0.4978498	0.8492551\\
0.4979498	0.8525395\\
0.4980498	0.8558336\\
0.4981498	0.8591376\\
0.4982498	0.8624513\\
0.4983498	0.8657748\\
0.4984498	0.8691081\\
0.4985499	0.8724512\\
0.4986499	0.8758042\\
0.4987499	0.8791671\\
0.4988499	0.8825399\\
0.4989499	0.8859224\\
0.4990499	0.8893149\\
0.4991499	0.8927173\\
0.4992499	0.8961297\\
0.4993499	0.899552\\
0.4994499	0.9029843\\
0.49955	0.9064265\\
0.49965	0.9098788\\
0.49975	0.913341\\
0.49985	0.9168133\\
0.49995	0.9202955\\
0.50005	0.9237878\\
0.50015	0.9272902\\
0.50025	0.9308027\\
0.50035	0.9343252\\
0.50045	0.9378579\\
0.5005501	0.9414006\\
0.5006501	0.9449535\\
0.5007501	0.9485166\\
0.5008501	0.9520898\\
0.5009501	0.9556731\\
0.5010501	0.9592666\\
0.5011501	0.9628704\\
0.5012501	0.9664843\\
0.5013501	0.9701085\\
0.5014501	0.973743\\
0.5015502	0.9773875\\
0.5016502	0.9810425\\
0.5017502	0.9847077\\
0.5018502	0.9883832\\
0.5019502	0.992069\\
0.5020502	0.9957651\\
0.5021502	0.9994715\\
0.5022502	1.003188\\
0.5023502	1.006915\\
0.5024502	1.010653\\
0.5025503	1.014401\\
0.5026503	1.018159\\
0.5027503	1.021928\\
0.5028503	1.025707\\
0.5029503	1.029496\\
0.5030503	1.033296\\
0.5031503	1.037107\\
0.5032503	1.040928\\
0.5033503	1.044759\\
0.5034503	1.048601\\
0.5035504	1.052453\\
0.5036504	1.056316\\
0.5037504	1.06019\\
0.5038504	1.064074\\
0.5039504	1.067968\\
0.5040504	1.071873\\
0.5041504	1.075789\\
0.5042504	1.079715\\
0.5043504	1.083652\\
0.5044504	1.087599\\
0.5045505	1.091557\\
0.5046505	1.095526\\
0.5047505	1.099505\\
0.5048505	1.103495\\
0.5049505	1.107496\\
0.5050505	1.111507\\
0.5051505	1.115529\\
0.5052505	1.119562\\
0.5053505	1.123605\\
0.5054505	1.127659\\
0.5055506	1.131724\\
0.5056506	1.1358\\
0.5057506	1.139886\\
0.5058506	1.143983\\
0.5059506	1.148091\\
0.5060506	1.15221\\
0.5061506	1.156339\\
0.5062506	1.16048\\
0.5063506	1.164631\\
0.5064506	1.168793\\
0.5065507	1.172965\\
0.5066507	1.177149\\
0.5067507	1.181343\\
0.5068507	1.185549\\
0.5069507	1.189765\\
0.5070507	1.193992\\
0.5071507	1.19823\\
0.5072507	1.202479\\
0.5073507	1.206738\\
0.5074507	1.211009\\
0.5075508	1.215291\\
0.5076508	1.219583\\
0.5077508	1.223887\\
0.5078508	1.228201\\
0.5079508	1.232527\\
0.5080508	1.236863\\
0.5081508	1.24121\\
0.5082508	1.245569\\
0.5083508	1.249938\\
0.5084508	1.254318\\
0.5085509	1.25871\\
0.5086509	1.263112\\
0.5087509	1.267526\\
0.5088509	1.27195\\
0.5089509	1.276386\\
0.5090509	1.280832\\
0.5091509	1.28529\\
0.5092509	1.289759\\
0.5093509	1.294238\\
0.5094509	1.298729\\
0.509551	1.303231\\
0.509651	1.307744\\
0.509751	1.312269\\
0.509851	1.316804\\
0.509951	1.32135\\
0.510051	1.325908\\
0.510151	1.330477\\
0.510251	1.335057\\
0.510351	1.339647\\
0.510451	1.34425\\
0.5105511	1.348863\\
0.5106511	1.353487\\
0.5107511	1.358123\\
0.5108511	1.36277\\
0.5109511	1.367428\\
0.5110511	1.372097\\
0.5111511	1.376778\\
0.5112511	1.381469\\
0.5113511	1.386172\\
0.5114511	1.390886\\
0.5115512	1.395612\\
0.5116512	1.400348\\
0.5117512	1.405096\\
0.5118512	1.409855\\
0.5119512	1.414625\\
0.5120512	1.419407\\
0.5121512	1.4242\\
0.5122512	1.429004\\
0.5123512	1.433819\\
0.5124512	1.438646\\
0.5125513	1.443484\\
0.5126513	1.448333\\
0.5127513	1.453193\\
0.5128513	1.458065\\
0.5129513	1.462948\\
0.5130513	1.467842\\
0.5131513	1.472748\\
0.5132513	1.477665\\
0.5133513	1.482593\\
0.5134513	1.487533\\
0.5135514	1.492483\\
0.5136514	1.497446\\
0.5137514	1.502419\\
0.5138514	1.507404\\
0.5139514	1.5124\\
0.5140514	1.517408\\
0.5141514	1.522426\\
0.5142514	1.527457\\
0.5143514	1.532498\\
0.5144514	1.537551\\
0.5145515	1.542615\\
0.5146515	1.547691\\
0.5147515	1.552778\\
0.5148515	1.557876\\
0.5149515	1.562986\\
0.5150515	1.568107\\
0.5151515	1.573239\\
0.5152515	1.578383\\
0.5153515	1.583538\\
0.5154515	1.588704\\
0.5155516	1.593882\\
0.5156516	1.599071\\
0.5157516	1.604272\\
0.5158516	1.609484\\
0.5159516	1.614707\\
0.5160516	1.619942\\
0.5161516	1.625188\\
0.5162516	1.630445\\
0.5163516	1.635714\\
0.5164516	1.640994\\
0.5165517	1.646286\\
0.5166517	1.651589\\
0.5167517	1.656903\\
0.5168517	1.662229\\
0.5169517	1.667566\\
0.5170517	1.672914\\
0.5171517	1.678274\\
0.5172517	1.683645\\
0.5173517	1.689027\\
0.5174517	1.694421\\
0.5175518	1.699827\\
0.5176518	1.705243\\
0.5177518	1.710671\\
0.5178518	1.716111\\
0.5179518	1.721562\\
0.5180518	1.727024\\
0.5181518	1.732497\\
0.5182518	1.737982\\
0.5183518	1.743478\\
0.5184518	1.748986\\
0.5185519	1.754505\\
0.5186519	1.760035\\
0.5187519	1.765577\\
0.5188519	1.77113\\
0.5189519	1.776694\\
0.5190519	1.78227\\
0.5191519	1.787857\\
0.5192519	1.793455\\
0.5193519	1.799065\\
0.5194519	1.804686\\
0.519552	1.810318\\
0.519652	1.815962\\
0.519752	1.821617\\
0.519852	1.827283\\
0.519952	1.832961\\
0.520052	1.83865\\
0.520152	1.84435\\
0.520252	1.850062\\
0.520352	1.855784\\
0.520452	1.861518\\
0.5205521	1.867264\\
0.5206521	1.873021\\
0.5207521	1.878789\\
0.5208521	1.884568\\
0.5209521	1.890358\\
0.5210521	1.89616\\
0.5211521	1.901973\\
0.5212521	1.907797\\
0.5213521	1.913633\\
0.5214521	1.91948\\
0.5215522	1.925338\\
0.5216522	1.931207\\
0.5217522	1.937088\\
0.5218522	1.942979\\
0.5219522	1.948882\\
0.5220522	1.954796\\
0.5221522	1.960721\\
0.5222522	1.966658\\
0.5223522	1.972606\\
0.5224522	1.978564\\
0.5225523	1.984534\\
0.5226523	1.990516\\
0.5227523	1.996508\\
0.5228523	2.002511\\
0.5229523	2.008526\\
0.5230523	2.014552\\
0.5231523	2.020588\\
0.5232523	2.026636\\
0.5233523	2.032695\\
0.5234523	2.038765\\
0.5235524	2.044847\\
0.5236524	2.050939\\
0.5237524	2.057042\\
0.5238524	2.063157\\
0.5239524	2.069282\\
0.5240524	2.075419\\
0.5241524	2.081566\\
0.5242524	2.087725\\
0.5243524	2.093894\\
0.5244524	2.100075\\
0.5245525	2.106266\\
0.5246525	2.112469\\
0.5247525	2.118682\\
0.5248525	2.124907\\
0.5249525	2.131142\\
0.5250525	2.137389\\
0.5251525	2.143646\\
0.5252525	2.149914\\
0.5253525	2.156193\\
0.5254525	2.162483\\
0.5255526	2.168784\\
0.5256526	2.175096\\
0.5257526	2.181418\\
0.5258526	2.187752\\
0.5259526	2.194096\\
0.5260526	2.200451\\
0.5261526	2.206817\\
0.5262526	2.213194\\
0.5263526	2.219581\\
0.5264526	2.22598\\
0.5265527	2.232389\\
0.5266527	2.238808\\
0.5267527	2.245239\\
0.5268527	2.25168\\
0.5269527	2.258132\\
0.5270527	2.264595\\
0.5271527	2.271068\\
0.5272527	2.277552\\
0.5273527	2.284046\\
0.5274527	2.290551\\
0.5275528	2.297067\\
0.5276528	2.303594\\
0.5277528	2.310131\\
0.5278528	2.316678\\
0.5279528	2.323236\\
0.5280528	2.329805\\
0.5281528	2.336384\\
0.5282528	2.342974\\
0.5283528	2.349574\\
0.5284528	2.356185\\
0.5285529	2.362806\\
0.5286529	2.369438\\
0.5287529	2.37608\\
0.5288529	2.382732\\
0.5289529	2.389395\\
0.5290529	2.396068\\
0.5291529	2.402752\\
0.5292529	2.409446\\
0.5293529	2.41615\\
0.5294529	2.422865\\
0.529553	2.429589\\
0.529653	2.436324\\
0.529753	2.44307\\
0.529853	2.449825\\
0.529953	2.456591\\
0.530053	2.463367\\
0.530153	2.470154\\
0.530253	2.47695\\
0.530353	2.483756\\
0.530453	2.490573\\
0.5305531	2.4974\\
0.5306531	2.504237\\
0.5307531	2.511084\\
0.5308531	2.517941\\
0.5309531	2.524808\\
0.5310531	2.531685\\
0.5311531	2.538572\\
0.5312531	2.545469\\
0.5313531	2.552376\\
0.5314531	2.559293\\
0.5315532	2.56622\\
0.5316532	2.573156\\
0.5317532	2.580103\\
0.5318532	2.587059\\
0.5319532	2.594025\\
0.5320532	2.601001\\
0.5321532	2.607987\\
0.5322532	2.614983\\
0.5323532	2.621988\\
0.5324532	2.629003\\
0.5325533	2.636028\\
0.5326533	2.643062\\
0.5327533	2.650106\\
0.5328533	2.657159\\
0.5329533	2.664223\\
0.5330533	2.671295\\
0.5331533	2.678378\\
0.5332533	2.68547\\
0.5333533	2.692571\\
0.5334533	2.699682\\
0.5335534	2.706802\\
0.5336534	2.713932\\
0.5337534	2.721071\\
0.5338534	2.72822\\
0.5339534	2.735378\\
0.5340534	2.742545\\
0.5341534	2.749721\\
0.5342534	2.756907\\
0.5343534	2.764102\\
0.5344534	2.771307\\
0.5345535	2.77852\\
0.5346535	2.785743\\
0.5347535	2.792975\\
0.5348535	2.800216\\
0.5349535	2.807466\\
0.5350535	2.814725\\
0.5351535	2.821993\\
0.5352535	2.829271\\
0.5353535	2.836557\\
0.5354535	2.843852\\
0.5355536	2.851156\\
0.5356536	2.858469\\
0.5357536	2.865791\\
0.5358536	2.873122\\
0.5359536	2.880462\\
0.5360536	2.88781\\
0.5361536	2.895168\\
0.5362536	2.902534\\
0.5363536	2.909908\\
0.5364536	2.917292\\
0.5365537	2.924684\\
0.5366537	2.932085\\
0.5367537	2.939494\\
0.5368537	2.946912\\
0.5369537	2.954338\\
0.5370537	2.961773\\
0.5371537	2.969217\\
0.5372537	2.976669\\
0.5373537	2.984129\\
0.5374537	2.991598\\
0.5375538	2.999075\\
0.5376538	3.00656\\
0.5377538	3.014054\\
0.5378538	3.021556\\
0.5379538	3.029066\\
0.5380538	3.036585\\
0.5381538	3.044111\\
0.5382538	3.051646\\
0.5383538	3.059189\\
0.5384538	3.06674\\
0.5385539	3.074299\\
0.5386539	3.081866\\
0.5387539	3.089441\\
0.5388539	3.097024\\
0.5389539	3.104615\\
0.5390539	3.112214\\
0.5391539	3.119821\\
0.5392539	3.127435\\
0.5393539	3.135057\\
0.5394539	3.142687\\
0.539554	3.150325\\
0.539654	3.15797\\
0.539754	3.165624\\
0.539854	3.173284\\
0.539954	3.180953\\
0.540054	3.188628\\
0.540154	3.196312\\
0.540254	3.204003\\
0.540354	3.211701\\
0.540454	3.219407\\
0.5405541	3.22712\\
0.5406541	3.23484\\
0.5407541	3.242568\\
0.5408541	3.250303\\
0.5409541	3.258045\\
0.5410541	3.265794\\
0.5411541	3.273551\\
0.5412541	3.281315\\
0.5413541	3.289085\\
0.5414541	3.296863\\
0.5415542	3.304648\\
0.5416542	3.31244\\
0.5417542	3.320239\\
0.5418542	3.328044\\
0.5419542	3.335857\\
0.5420542	3.343676\\
0.5421542	3.351502\\
0.5422542	3.359335\\
0.5423542	3.367175\\
0.5424542	3.375021\\
0.5425543	3.382874\\
0.5426543	3.390733\\
0.5427543	3.398599\\
0.5428543	3.406472\\
0.5429543	3.414351\\
0.5430543	3.422236\\
0.5431543	3.430128\\
0.5432543	3.438027\\
0.5433543	3.445931\\
0.5434543	3.453842\\
0.5435544	3.461759\\
0.5436544	3.469682\\
0.5437544	3.477612\\
0.5438544	3.485547\\
0.5439544	3.493489\\
0.5440544	3.501437\\
0.5441544	3.50939\\
0.5442544	3.51735\\
0.5443544	3.525315\\
0.5444544	3.533287\\
0.5445545	3.541264\\
0.5446545	3.549247\\
0.5447545	3.557236\\
0.5448545	3.56523\\
0.5449545	3.57323\\
0.5450545	3.581236\\
0.5451545	3.589248\\
0.5452545	3.597264\\
0.5453545	3.605287\\
0.5454545	3.613315\\
0.5455546	3.621348\\
0.5456546	3.629386\\
0.5457546	3.63743\\
0.5458546	3.64548\\
0.5459546	3.653534\\
0.5460546	3.661593\\
0.5461546	3.669658\\
0.5462546	3.677728\\
0.5463546	3.685803\\
0.5464546	3.693883\\
0.5465547	3.701968\\
0.5466547	3.710057\\
0.5467547	3.718152\\
0.5468547	3.726252\\
0.5469547	3.734356\\
0.5470547	3.742465\\
0.5471547	3.750578\\
0.5472547	3.758697\\
0.5473547	3.76682\\
0.5474547	3.774947\\
0.5475548	3.783079\\
0.5476548	3.791216\\
0.5477548	3.799356\\
0.5478548	3.807502\\
0.5479548	3.815651\\
0.5480548	3.823805\\
0.5481548	3.831963\\
0.5482548	3.840126\\
0.5483548	3.848292\\
0.5484548	3.856462\\
0.5485549	3.864637\\
0.5486549	3.872815\\
0.5487549	3.880998\\
0.5488549	3.889184\\
0.5489549	3.897374\\
0.5490549	3.905568\\
0.5491549	3.913766\\
0.5492549	3.921968\\
0.5493549	3.930173\\
0.5494549	3.938381\\
0.549555	3.946594\\
0.549655	3.954809\\
0.549755	3.963029\\
0.549855	3.971251\\
0.549955	3.979477\\
0.550055	3.987706\\
0.550155	3.995939\\
0.550255	4.004175\\
0.550355	4.012414\\
0.550455	4.020656\\
0.5505551	4.028901\\
0.5506551	4.037149\\
0.5507551	4.0454\\
0.5508551	4.053654\\
0.5509551	4.06191\\
0.5510551	4.07017\\
0.5511551	4.078432\\
0.5512551	4.086697\\
0.5513551	4.094965\\
0.5514551	4.103235\\
0.5515552	4.111508\\
0.5516552	4.119783\\
0.5517552	4.128061\\
0.5518552	4.136341\\
0.5519552	4.144624\\
0.5520552	4.152909\\
0.5521552	4.161196\\
0.5522552	4.169485\\
0.5523552	4.177776\\
0.5524552	4.186069\\
0.5525553	4.194365\\
0.5526553	4.202662\\
0.5527553	4.210961\\
0.5528553	4.219263\\
0.5529553	4.227566\\
0.5530553	4.23587\\
0.5531553	4.244176\\
0.5532553	4.252485\\
0.5533553	4.260794\\
0.5534553	4.269105\\
0.5535554	4.277418\\
0.5536554	4.285732\\
0.5537554	4.294047\\
0.5538554	4.302364\\
0.5539554	4.310682\\
0.5540554	4.319001\\
0.5541554	4.327322\\
0.5542554	4.335643\\
0.5543554	4.343966\\
0.5544554	4.352289\\
0.5545555	4.360614\\
0.5546555	4.368939\\
0.5547555	4.377265\\
0.5548555	4.385592\\
0.5549555	4.39392\\
0.5550555	4.402248\\
0.5551555	4.410577\\
0.5552555	4.418906\\
0.5553555	4.427236\\
0.5554555	4.435566\\
0.5555556	4.443897\\
0.5556556	4.452228\\
0.5557556	4.460559\\
0.5558556	4.468891\\
0.5559556	4.477223\\
0.5560556	4.485554\\
0.5561556	4.493886\\
0.5562556	4.502218\\
0.5563556	4.510549\\
0.5564556	4.518881\\
0.5565557	4.527212\\
0.5566557	4.535543\\
0.5567557	4.543874\\
0.5568557	4.552204\\
0.5569557	4.560534\\
0.5570557	4.568863\\
0.5571557	4.577192\\
0.5572557	4.58552\\
0.5573557	4.593848\\
0.5574557	4.602175\\
0.5575558	4.610501\\
0.5576558	4.618826\\
0.5577558	4.62715\\
0.5578558	4.635474\\
0.5579558	4.643796\\
0.5580558	4.652117\\
0.5581558	4.660437\\
0.5582558	4.668756\\
0.5583558	4.677074\\
0.5584558	4.68539\\
0.5585559	4.693705\\
0.5586559	4.702019\\
0.5587559	4.710331\\
0.5588559	4.718641\\
0.5589559	4.72695\\
0.5590559	4.735257\\
0.5591559	4.743563\\
0.5592559	4.751867\\
0.5593559	4.760168\\
0.5594559	4.768468\\
0.559556	4.776766\\
0.559656	4.785062\\
0.559756	4.793356\\
0.559856	4.801647\\
0.559956	4.809937\\
0.560056	4.818224\\
0.560156	4.826509\\
0.560256	4.834791\\
0.560356	4.843071\\
0.560456	4.851348\\
0.5605561	4.859623\\
0.5606561	4.867895\\
0.5607561	4.876165\\
0.5608561	4.884431\\
0.5609561	4.892695\\
0.5610561	4.900956\\
0.5611561	4.909214\\
0.5612561	4.917469\\
0.5613561	4.925721\\
0.5614561	4.93397\\
0.5615562	4.942215\\
0.5616562	4.950457\\
0.5617562	4.958696\\
0.5618562	4.966932\\
0.5619562	4.975164\\
0.5620562	4.983392\\
0.5621562	4.991617\\
0.5622562	4.999839\\
0.5623562	5.008056\\
0.5624562	5.01627\\
0.5625563	5.02448\\
0.5626563	5.032686\\
0.5627563	5.040889\\
0.5628563	5.049087\\
0.5629563	5.057281\\
0.5630563	5.065471\\
0.5631563	5.073656\\
0.5632563	5.081838\\
0.5633563	5.090015\\
0.5634563	5.098188\\
0.5635564	5.106356\\
0.5636564	5.11452\\
0.5637564	5.122679\\
0.5638564	5.130833\\
0.5639564	5.138983\\
0.5640564	5.147128\\
0.5641564	5.155268\\
0.5642564	5.163403\\
0.5643564	5.171534\\
0.5644564	5.179659\\
0.5645565	5.187779\\
0.5646565	5.195894\\
0.5647565	5.204004\\
0.5648565	5.212108\\
0.5649565	5.220207\\
0.5650565	5.228301\\
0.5651565	5.236389\\
0.5652565	5.244472\\
0.5653565	5.252549\\
0.5654565	5.26062\\
0.5655566	5.268686\\
0.5656566	5.276746\\
0.5657566	5.2848\\
0.5658566	5.292848\\
0.5659566	5.30089\\
0.5660566	5.308926\\
0.5661566	5.316955\\
0.5662566	5.324979\\
0.5663566	5.332997\\
0.5664566	5.341008\\
0.5665567	5.349012\\
0.5666567	5.35701\\
0.5667567	5.365002\\
0.5668567	5.372987\\
0.5669567	5.380966\\
0.5670567	5.388937\\
0.5671567	5.396902\\
0.5672567	5.404861\\
0.5673567	5.412812\\
0.5674567	5.420756\\
0.5675568	5.428693\\
0.5676568	5.436623\\
0.5677568	5.444547\\
0.5678568	5.452462\\
0.5679568	5.460371\\
0.5680568	5.468272\\
0.5681568	5.476166\\
0.5682568	5.484052\\
0.5683568	5.49193\\
0.5684568	5.499801\\
0.5685569	5.507665\\
0.5686569	5.51552\\
0.5687569	5.523368\\
0.5688569	5.531208\\
0.5689569	5.53904\\
0.5690569	5.546864\\
0.5691569	5.55468\\
0.5692569	5.562487\\
0.5693569	5.570287\\
0.5694569	5.578078\\
0.569557	5.585861\\
0.569657	5.593635\\
0.569757	5.601401\\
0.569857	5.609159\\
0.569957	5.616907\\
0.570057	5.624648\\
0.570157	5.632379\\
0.570257	5.640102\\
0.570357	5.647816\\
0.570457	5.655521\\
0.5705571	5.663216\\
0.5706571	5.670903\\
0.5707571	5.678581\\
0.5708571	5.68625\\
0.5709571	5.693909\\
0.5710571	5.701559\\
0.5711571	5.709199\\
0.5712571	5.716831\\
0.5713571	5.724452\\
0.5714571	5.732064\\
0.5715572	5.739667\\
0.5716572	5.747259\\
0.5717572	5.754842\\
0.5718572	5.762415\\
0.5719572	5.769979\\
0.5720572	5.777532\\
0.5721572	5.785075\\
0.5722572	5.792608\\
0.5723572	5.800131\\
0.5724572	5.807644\\
0.5725573	5.815146\\
0.5726573	5.822638\\
0.5727573	5.83012\\
0.5728573	5.837591\\
0.5729573	5.845051\\
0.5730573	5.852501\\
0.5731573	5.859941\\
0.5732573	5.867369\\
0.5733573	5.874787\\
0.5734573	5.882193\\
0.5735574	5.889589\\
0.5736574	5.896974\\
0.5737574	5.904348\\
0.5738574	5.911711\\
0.5739574	5.919062\\
0.5740574	5.926402\\
0.5741574	5.933731\\
0.5742574	5.941049\\
0.5743574	5.948355\\
0.5744574	5.955649\\
0.5745575	5.962932\\
0.5746575	5.970203\\
0.5747575	5.977463\\
0.5748575	5.984711\\
0.5749575	5.991946\\
0.5750575	5.99917\\
0.5751575	6.006382\\
0.5752575	6.013582\\
0.5753575	6.02077\\
0.5754575	6.027946\\
0.5755576	6.035109\\
0.5756576	6.04226\\
0.5757576	6.049399\\
0.5758576	6.056526\\
0.5759576	6.06364\\
0.5760576	6.070741\\
0.5761576	6.07783\\
0.5762576	6.084906\\
0.5763576	6.091969\\
0.5764576	6.09902\\
0.5765577	6.106057\\
0.5766577	6.113082\\
0.5767577	6.120094\\
0.5768577	6.127093\\
0.5769577	6.134078\\
0.5770577	6.141051\\
0.5771577	6.14801\\
0.5772577	6.154955\\
0.5773577	6.161888\\
0.5774577	6.168807\\
0.5775578	6.175712\\
0.5776578	6.182604\\
0.5777578	6.189483\\
0.5778578	6.196347\\
0.5779578	6.203198\\
0.5780578	6.210035\\
0.5781578	6.216859\\
0.5782578	6.223668\\
0.5783578	6.230463\\
0.5784578	6.237244\\
0.5785579	6.244011\\
0.5786579	6.250764\\
0.5787579	6.257503\\
0.5788579	6.264227\\
0.5789579	6.270937\\
0.5790579	6.277633\\
0.5791579	6.284314\\
0.5792579	6.29098\\
0.5793579	6.297632\\
0.5794579	6.30427\\
0.579558	6.310892\\
0.579658	6.3175\\
0.579758	6.324093\\
0.579858	6.330671\\
0.579958	6.337234\\
0.580058	6.343781\\
0.580158	6.350314\\
0.580258	6.356832\\
0.580358	6.363334\\
0.580458	6.369821\\
0.5805581	6.376293\\
0.5806581	6.382749\\
0.5807581	6.38919\\
0.5808581	6.395616\\
0.5809581	6.402026\\
0.5810581	6.40842\\
0.5811581	6.414798\\
0.5812581	6.421161\\
0.5813581	6.427508\\
0.5814581	6.433839\\
0.5815582	6.440154\\
0.5816582	6.446453\\
0.5817582	6.452736\\
0.5818582	6.459003\\
0.5819582	6.465253\\
0.5820582	6.471488\\
0.5821582	6.477706\\
0.5822582	6.483908\\
0.5823582	6.490093\\
0.5824582	6.496262\\
0.5825583	6.502414\\
0.5826583	6.50855\\
0.5827583	6.514669\\
0.5828583	6.520772\\
0.5829583	6.526857\\
0.5830583	6.532926\\
0.5831583	6.538978\\
0.5832583	6.545013\\
0.5833583	6.551031\\
0.5834583	6.557032\\
0.5835584	6.563016\\
0.5836584	6.568983\\
0.5837584	6.574933\\
0.5838584	6.580865\\
0.5839584	6.58678\\
0.5840584	6.592677\\
0.5841584	6.598558\\
0.5842584	6.60442\\
0.5843584	6.610265\\
0.5844584	6.616093\\
0.5845585	6.621903\\
0.5846585	6.627695\\
0.5847585	6.633469\\
0.5848585	6.639225\\
0.5849585	6.644964\\
0.5850585	6.650685\\
0.5851585	6.656387\\
0.5852585	6.662072\\
0.5853585	6.667738\\
0.5854585	6.673387\\
0.5855586	6.679017\\
0.5856586	6.684628\\
0.5857586	6.690222\\
0.5858586	6.695797\\
0.5859586	6.701354\\
0.5860586	6.706892\\
0.5861586	6.712412\\
0.5862586	6.717913\\
0.5863586	6.723395\\
0.5864586	6.728859\\
0.5865587	6.734304\\
0.5866587	6.73973\\
0.5867587	6.745137\\
0.5868587	6.750526\\
0.5869587	6.755895\\
0.5870587	6.761246\\
0.5871587	6.766577\\
0.5872587	6.77189\\
0.5873587	6.777183\\
0.5874587	6.782457\\
0.5875588	6.787711\\
0.5876588	6.792946\\
0.5877588	6.798162\\
0.5878588	6.803359\\
0.5879588	6.808536\\
0.5880588	6.813693\\
0.5881588	6.818831\\
0.5882588	6.82395\\
0.5883588	6.829048\\
0.5884588	6.834127\\
0.5885589	6.839186\\
0.5886589	6.844225\\
0.5887589	6.849245\\
0.5888589	6.854244\\
0.5889589	6.859224\\
0.5890589	6.864183\\
0.5891589	6.869123\\
0.5892589	6.874042\\
0.5893589	6.878941\\
0.5894589	6.883821\\
0.589559	6.888679\\
0.589659	6.893517\\
0.589759	6.898335\\
0.589859	6.903133\\
0.589959	6.90791\\
0.590059	6.912667\\
0.590159	6.917403\\
0.590259	6.922118\\
0.590359	6.926813\\
0.590459	6.931488\\
0.5905591	6.936141\\
0.5906591	6.940774\\
0.5907591	6.945386\\
0.5908591	6.949977\\
0.5909591	6.954547\\
0.5910591	6.959096\\
0.5911591	6.963624\\
0.5912591	6.968131\\
0.5913591	6.972617\\
0.5914591	6.977082\\
0.5915592	6.981526\\
0.5916592	6.985948\\
0.5917592	6.990349\\
0.5918592	6.994729\\
0.5919592	6.999087\\
0.5920592	7.003425\\
0.5921592	7.00774\\
0.5922592	7.012034\\
0.5923592	7.016307\\
0.5924592	7.020558\\
0.5925593	7.024788\\
0.5926593	7.028995\\
0.5927593	7.033181\\
0.5928593	7.037346\\
0.5929593	7.041488\\
0.5930593	7.045609\\
0.5931593	7.049708\\
0.5932593	7.053785\\
0.5933593	7.05784\\
0.5934593	7.061873\\
0.5935594	7.065884\\
0.5936594	7.069873\\
0.5937594	7.07384\\
0.5938594	7.077784\\
0.5939594	7.081707\\
0.5940594	7.085607\\
0.5941594	7.089485\\
0.5942594	7.093341\\
0.5943594	7.097174\\
0.5944594	7.100985\\
0.5945595	7.104774\\
0.5946595	7.10854\\
0.5947595	7.112283\\
0.5948595	7.116004\\
0.5949595	7.119703\\
0.5950595	7.123379\\
0.5951595	7.127032\\
0.5952595	7.130663\\
0.5953595	7.134271\\
0.5954595	7.137856\\
0.5955596	7.141418\\
0.5956596	7.144957\\
0.5957596	7.148474\\
0.5958596	7.151968\\
0.5959596	7.155438\\
0.5960596	7.158886\\
0.5961596	7.162311\\
0.5962596	7.165713\\
0.5963596	7.169092\\
0.5964596	7.172447\\
0.5965597	7.17578\\
0.5966597	7.179089\\
0.5967597	7.182375\\
0.5968597	7.185638\\
0.5969597	7.188877\\
0.5970597	7.192094\\
0.5971597	7.195286\\
0.5972597	7.198456\\
0.5973597	7.201602\\
0.5974597	7.204725\\
0.5975598	7.207824\\
0.5976598	7.2109\\
0.5977598	7.213952\\
0.5978598	7.216981\\
0.5979598	7.219986\\
0.5980598	7.222967\\
0.5981598	7.225925\\
0.5982598	7.228859\\
0.5983598	7.231769\\
0.5984598	7.234656\\
0.5985599	7.237519\\
0.5986599	7.240358\\
0.5987599	7.243173\\
0.5988599	7.245965\\
0.5989599	7.248732\\
0.5990599	7.251476\\
0.5991599	7.254195\\
0.5992599	7.256891\\
0.5993599	7.259563\\
0.5994599	7.26221\\
0.59956	7.264834\\
0.59966	7.267433\\
0.59976	7.270009\\
0.59986	7.27256\\
0.59996	7.275087\\
0.60006	7.27759\\
0.60016	7.280069\\
0.60026	7.282523\\
0.60036	7.284954\\
0.60046	7.28736\\
0.6005601	7.289741\\
0.6006601	7.292099\\
0.6007601	7.294432\\
0.6008601	7.29674\\
0.6009601	7.299024\\
0.6010601	7.301284\\
0.6011601	7.303519\\
0.6012601	7.30573\\
0.6013601	7.307917\\
0.6014601	7.310078\\
0.6015602	7.312216\\
0.6016602	7.314328\\
0.6017602	7.316416\\
0.6018602	7.31848\\
0.6019602	7.320519\\
0.6020602	7.322533\\
0.6021602	7.324522\\
0.6022602	7.326487\\
0.6023602	7.328427\\
0.6024602	7.330343\\
0.6025603	7.332234\\
0.6026603	7.3341\\
0.6027603	7.335941\\
0.6028603	7.337757\\
0.6029603	7.339548\\
0.6030603	7.341315\\
0.6031603	7.343057\\
0.6032603	7.344774\\
0.6033603	7.346466\\
0.6034603	7.348133\\
0.6035604	7.349775\\
0.6036604	7.351392\\
0.6037604	7.352985\\
0.6038604	7.354552\\
0.6039604	7.356094\\
0.6040604	7.357611\\
0.6041604	7.359104\\
0.6042604	7.360571\\
0.6043604	7.362013\\
0.6044604	7.36343\\
0.6045605	7.364822\\
0.6046605	7.366189\\
0.6047605	7.367531\\
0.6048605	7.368847\\
0.6049605	7.370139\\
0.6050605	7.371405\\
0.6051605	7.372646\\
0.6052605	7.373862\\
0.6053605	7.375053\\
0.6054605	7.376219\\
0.6055606	7.377359\\
0.6056606	7.378474\\
0.6057606	7.379564\\
0.6058606	7.380628\\
0.6059606	7.381668\\
0.6060606	7.382682\\
0.6061606	7.383671\\
0.6062606	7.384634\\
0.6063606	7.385572\\
0.6064606	7.386485\\
0.6065607	7.387373\\
0.6066607	7.388235\\
0.6067607	7.389072\\
0.6068607	7.389883\\
0.6069607	7.390669\\
0.6070607	7.39143\\
0.6071607	7.392165\\
0.6072607	7.392875\\
0.6073607	7.39356\\
0.6074607	7.394219\\
0.6075608	7.394853\\
0.6076608	7.395461\\
0.6077608	7.396044\\
0.6078608	7.396602\\
0.6079608	7.397134\\
0.6080608	7.397641\\
0.6081608	7.398122\\
0.6082608	7.398578\\
0.6083608	7.399008\\
0.6084608	7.399413\\
0.6085609	7.399792\\
0.6086609	7.400146\\
0.6087609	7.400475\\
0.6088609	7.400778\\
0.6089609	7.401055\\
0.6090609	7.401307\\
0.6091609	7.401534\\
0.6092609	7.401735\\
0.6093609	7.401911\\
0.6094609	7.402061\\
0.609561	7.402185\\
0.609661	7.402285\\
0.609761	7.402358\\
0.609861	7.402406\\
0.609961	7.402429\\
0.610061	7.402427\\
0.610161	7.402398\\
0.610261	7.402344\\
0.610361	7.402265\\
0.610461	7.40216\\
0.6105611	7.40203\\
0.6106611	7.401874\\
0.6107611	7.401693\\
0.6108611	7.401486\\
0.6109611	7.401254\\
0.6110611	7.400997\\
0.6111611	7.400713\\
0.6112611	7.400405\\
0.6113611	7.400071\\
0.6114611	7.399711\\
0.6115612	7.399326\\
0.6116612	7.398915\\
0.6117612	7.39848\\
0.6118612	7.398018\\
0.6119612	7.397531\\
0.6120612	7.397019\\
0.6121612	7.396481\\
0.6122612	7.395918\\
0.6123612	7.39533\\
0.6124612	7.394716\\
0.6125613	7.394076\\
0.6126613	7.393412\\
0.6127613	7.392721\\
0.6128613	7.392006\\
0.6129613	7.391265\\
0.6130613	7.390499\\
0.6131613	7.389707\\
0.6132613	7.38889\\
0.6133613	7.388047\\
0.6134613	7.387179\\
0.6135614	7.386286\\
0.6136614	7.385368\\
0.6137614	7.384424\\
0.6138614	7.383455\\
0.6139614	7.382461\\
0.6140614	7.381441\\
0.6141614	7.380396\\
0.6142614	7.379326\\
0.6143614	7.378231\\
0.6144614	7.37711\\
0.6145615	7.375964\\
0.6146615	7.374793\\
0.6147615	7.373597\\
0.6148615	7.372375\\
0.6149615	7.371129\\
0.6150615	7.369857\\
0.6151615	7.36856\\
0.6152615	7.367238\\
0.6153615	7.36589\\
0.6154615	7.364518\\
0.6155616	7.363121\\
0.6156616	7.361698\\
0.6157616	7.36025\\
0.6158616	7.358777\\
0.6159616	7.35728\\
0.6160616	7.355757\\
0.6161616	7.354209\\
0.6162616	7.352637\\
0.6163616	7.351039\\
0.6164616	7.349416\\
0.6165617	7.347768\\
0.6166617	7.346096\\
0.6167617	7.344398\\
0.6168617	7.342676\\
0.6169617	7.340929\\
0.6170617	7.339157\\
0.6171617	7.33736\\
0.6172617	7.335538\\
0.6173617	7.333691\\
0.6174617	7.33182\\
0.6175618	7.329924\\
0.6176618	7.328003\\
0.6177618	7.326057\\
0.6178618	7.324087\\
0.6179618	7.322092\\
0.6180618	7.320072\\
0.6181618	7.318028\\
0.6182618	7.315959\\
0.6183618	7.313866\\
0.6184618	7.311747\\
0.6185619	7.309605\\
0.6186619	7.307438\\
0.6187619	7.305246\\
0.6188619	7.30303\\
0.6189619	7.300789\\
0.6190619	7.298524\\
0.6191619	7.296234\\
0.6192619	7.29392\\
0.6193619	7.291582\\
0.6194619	7.289219\\
0.619562	7.286832\\
0.619662	7.284421\\
0.619762	7.281985\\
0.619862	7.279526\\
0.619962	7.277042\\
0.620062	7.274533\\
0.620162	7.272001\\
0.620262	7.269444\\
0.620362	7.266863\\
0.620462	7.264259\\
0.6205621	7.26163\\
0.6206621	7.258977\\
0.6207621	7.2563\\
0.6208621	7.253599\\
0.6209621	7.250874\\
0.6210621	7.248125\\
0.6211621	7.245353\\
0.6212621	7.242556\\
0.6213621	7.239735\\
0.6214621	7.236891\\
0.6215622	7.234023\\
0.6216622	7.231131\\
0.6217622	7.228216\\
0.6218622	7.225276\\
0.6219622	7.222313\\
0.6220622	7.219327\\
0.6221622	7.216316\\
0.6222622	7.213283\\
0.6223622	7.210225\\
0.6224622	7.207144\\
0.6225623	7.20404\\
0.6226623	7.200912\\
0.6227623	7.197761\\
0.6228623	7.194586\\
0.6229623	7.191388\\
0.6230623	7.188167\\
0.6231623	7.184922\\
0.6232623	7.181654\\
0.6233623	7.178363\\
0.6234623	7.175048\\
0.6235624	7.171711\\
0.6236624	7.16835\\
0.6237624	7.164966\\
0.6238624	7.16156\\
0.6239624	7.15813\\
0.6240624	7.154677\\
0.6241624	7.151201\\
0.6242624	7.147702\\
0.6243624	7.144181\\
0.6244624	7.140636\\
0.6245625	7.137069\\
0.6246625	7.133479\\
0.6247625	7.129866\\
0.6248625	7.12623\\
0.6249625	7.122572\\
0.6250625	7.118891\\
0.6251625	7.115188\\
0.6252625	7.111462\\
0.6253625	7.107713\\
0.6254625	7.103942\\
0.6255626	7.100149\\
0.6256626	7.096333\\
0.6257626	7.092495\\
0.6258626	7.088634\\
0.6259626	7.084751\\
0.6260626	7.080846\\
0.6261626	7.076918\\
0.6262626	7.072969\\
0.6263626	7.068997\\
0.6264626	7.065004\\
0.6265627	7.060988\\
0.6266627	7.05695\\
0.6267627	7.05289\\
0.6268627	7.048808\\
0.6269627	7.044704\\
0.6270627	7.040579\\
0.6271627	7.036431\\
0.6272627	7.032262\\
0.6273627	7.028072\\
0.6274627	7.023859\\
0.6275628	7.019625\\
0.6276628	7.015369\\
0.6277628	7.011092\\
0.6278628	7.006793\\
0.6279628	7.002472\\
0.6280628	6.998131\\
0.6281628	6.993767\\
0.6282628	6.989383\\
0.6283628	6.984977\\
0.6284628	6.98055\\
0.6285629	6.976102\\
0.6286629	6.971632\\
0.6287629	6.967141\\
0.6288629	6.96263\\
0.6289629	6.958097\\
0.6290629	6.953543\\
0.6291629	6.948969\\
0.6292629	6.944373\\
0.6293629	6.939757\\
0.6294629	6.935119\\
0.629563	6.930461\\
0.629663	6.925782\\
0.629763	6.921083\\
0.629863	6.916363\\
0.629963	6.911622\\
0.630063	6.906861\\
0.630163	6.902079\\
0.630263	6.897277\\
0.630363	6.892455\\
0.630463	6.887612\\
0.6305631	6.882749\\
0.6306631	6.877865\\
0.6307631	6.872962\\
0.6308631	6.868038\\
0.6309631	6.863094\\
0.6310631	6.85813\\
0.6311631	6.853146\\
0.6312631	6.848142\\
0.6313631	6.843119\\
0.6314631	6.838075\\
0.6315632	6.833011\\
0.6316632	6.827928\\
0.6317632	6.822825\\
0.6318632	6.817703\\
0.6319632	6.81256\\
0.6320632	6.807399\\
0.6321632	6.802217\\
0.6322632	6.797016\\
0.6323632	6.791796\\
0.6324632	6.786557\\
0.6325633	6.781298\\
0.6326633	6.77602\\
0.6327633	6.770723\\
0.6328633	6.765406\\
0.6329633	6.76007\\
0.6330633	6.754716\\
0.6331633	6.749342\\
0.6332633	6.74395\\
0.6333633	6.738538\\
0.6334633	6.733108\\
0.6335634	6.727659\\
0.6336634	6.722191\\
0.6337634	6.716704\\
0.6338634	6.711199\\
0.6339634	6.705675\\
0.6340634	6.700133\\
0.6341634	6.694572\\
0.6342634	6.688993\\
0.6343634	6.683395\\
0.6344634	6.67778\\
0.6345635	6.672146\\
0.6346635	6.666493\\
0.6347635	6.660823\\
0.6348635	6.655134\\
0.6349635	6.649428\\
0.6350635	6.643703\\
0.6351635	6.637961\\
0.6352635	6.6322\\
0.6353635	6.626422\\
0.6354635	6.620626\\
0.6355636	6.614812\\
0.6356636	6.608981\\
0.6357636	6.603132\\
0.6358636	6.597266\\
0.6359636	6.591382\\
0.6360636	6.58548\\
0.6361636	6.579561\\
0.6362636	6.573625\\
0.6363636	6.567672\\
0.6364636	6.561701\\
0.6365637	6.555714\\
0.6366637	6.549709\\
0.6367637	6.543687\\
0.6368637	6.537648\\
0.6369637	6.531593\\
0.6370637	6.52552\\
0.6371637	6.519431\\
0.6372637	6.513325\\
0.6373637	6.507202\\
0.6374637	6.501062\\
0.6375638	6.494906\\
0.6376638	6.488734\\
0.6377638	6.482545\\
0.6378638	6.476339\\
0.6379638	6.470117\\
0.6380638	6.463879\\
0.6381638	6.457625\\
0.6382638	6.451355\\
0.6383638	6.445068\\
0.6384638	6.438766\\
0.6385639	6.432447\\
0.6386639	6.426113\\
0.6387639	6.419763\\
0.6388639	6.413396\\
0.6389639	6.407014\\
0.6390639	6.400617\\
0.6391639	6.394203\\
0.6392639	6.387775\\
0.6393639	6.38133\\
0.6394639	6.374871\\
0.639564	6.368396\\
0.639664	6.361905\\
0.639764	6.355399\\
0.639864	6.348878\\
0.639964	6.342342\\
0.640064	6.335791\\
0.640164	6.329225\\
0.640264	6.322644\\
0.640364	6.316047\\
0.640464	6.309436\\
0.6405641	6.302811\\
0.6406641	6.29617\\
0.6407641	6.289515\\
0.6408641	6.282845\\
0.6409641	6.276161\\
0.6410641	6.269462\\
0.6411641	6.262749\\
0.6412641	6.256022\\
0.6413641	6.24928\\
0.6414641	6.242524\\
0.6415642	6.235754\\
0.6416642	6.228969\\
0.6417642	6.222171\\
0.6418642	6.215359\\
0.6419642	6.208532\\
0.6420642	6.201692\\
0.6421642	6.194838\\
0.6422642	6.187971\\
0.6423642	6.181089\\
0.6424642	6.174194\\
0.6425643	6.167286\\
0.6426643	6.160364\\
0.6427643	6.153428\\
0.6428643	6.14648\\
0.6429643	6.139518\\
0.6430643	6.132543\\
0.6431643	6.125554\\
0.6432643	6.118552\\
0.6433643	6.111538\\
0.6434643	6.10451\\
0.6435644	6.09747\\
0.6436644	6.090416\\
0.6437644	6.08335\\
0.6438644	6.076271\\
0.6439644	6.06918\\
0.6440644	6.062076\\
0.6441644	6.054959\\
0.6442644	6.04783\\
0.6443644	6.040689\\
0.6444644	6.033534\\
0.6445645	6.026368\\
0.6446645	6.01919\\
0.6447645	6.012\\
0.6448645	6.004797\\
0.6449645	5.997582\\
0.6450645	5.990356\\
0.6451645	5.983117\\
0.6452645	5.975867\\
0.6453645	5.968605\\
0.6454645	5.961331\\
0.6455646	5.954045\\
0.6456646	5.946748\\
0.6457646	5.93944\\
0.6458646	5.93212\\
0.6459646	5.924788\\
0.6460646	5.917446\\
0.6461646	5.910092\\
0.6462646	5.902727\\
0.6463646	5.895351\\
0.6464646	5.887963\\
0.6465647	5.880565\\
0.6466647	5.873156\\
0.6467647	5.865736\\
0.6468647	5.858305\\
0.6469647	5.850863\\
0.6470647	5.843411\\
0.6471647	5.835948\\
0.6472647	5.828475\\
0.6473647	5.820991\\
0.6474647	5.813496\\
0.6475648	5.805992\\
0.6476648	5.798477\\
0.6477648	5.790952\\
0.6478648	5.783416\\
0.6479648	5.775871\\
0.6480648	5.768316\\
0.6481648	5.76075\\
0.6482648	5.753175\\
0.6483648	5.74559\\
0.6484648	5.737995\\
0.6485649	5.73039\\
0.6486649	5.722776\\
0.6487649	5.715153\\
0.6488649	5.707519\\
0.6489649	5.699876\\
0.6490649	5.692225\\
0.6491649	5.684563\\
0.6492649	5.676893\\
0.6493649	5.669213\\
0.6494649	5.661524\\
0.649565	5.653826\\
0.649665	5.646119\\
0.649765	5.638404\\
0.649865	5.630679\\
0.649965	5.622946\\
0.650065	5.615204\\
0.650165	5.607453\\
0.650265	5.599693\\
0.650365	5.591926\\
0.650465	5.584149\\
0.6505651	5.576365\\
0.6506651	5.568571\\
0.6507651	5.56077\\
0.6508651	5.552961\\
0.6509651	5.545143\\
0.6510651	5.537317\\
0.6511651	5.529484\\
0.6512651	5.521642\\
0.6513651	5.513793\\
0.6514651	5.505935\\
0.6515652	5.49807\\
0.6516652	5.490198\\
0.6517652	5.482317\\
0.6518652	5.474429\\
0.6519652	5.466534\\
0.6520652	5.458632\\
0.6521652	5.450721\\
0.6522652	5.442804\\
0.6523652	5.43488\\
0.6524652	5.426948\\
0.6525653	5.419009\\
0.6526653	5.411063\\
0.6527653	5.40311\\
0.6528653	5.395151\\
0.6529653	5.387184\\
0.6530653	5.379211\\
0.6531653	5.371231\\
0.6532653	5.363244\\
0.6533653	5.355251\\
0.6534653	5.347252\\
0.6535654	5.339246\\
0.6536654	5.331233\\
0.6537654	5.323214\\
0.6538654	5.315189\\
0.6539654	5.307158\\
0.6540654	5.299121\\
0.6541654	5.291078\\
0.6542654	5.283028\\
0.6543654	5.274973\\
0.6544654	5.266912\\
0.6545655	5.258845\\
0.6546655	5.250772\\
0.6547655	5.242694\\
0.6548655	5.23461\\
0.6549655	5.226521\\
0.6550655	5.218426\\
0.6551655	5.210326\\
0.6552655	5.20222\\
0.6553655	5.194109\\
0.6554655	5.185993\\
0.6555656	5.177872\\
0.6556656	5.169745\\
0.6557656	5.161614\\
0.6558656	5.153478\\
0.6559656	5.145337\\
0.6560656	5.13719\\
0.6561656	5.12904\\
0.6562656	5.120884\\
0.6563656	5.112724\\
0.6564656	5.104559\\
0.6565657	5.09639\\
0.6566657	5.088216\\
0.6567657	5.080038\\
0.6568657	5.071856\\
0.6569657	5.063669\\
0.6570657	5.055478\\
0.6571657	5.047284\\
0.6572657	5.039084\\
0.6573657	5.030881\\
0.6574657	5.022674\\
0.6575658	5.014463\\
0.6576658	5.006249\\
0.6577658	4.99803\\
0.6578658	4.989808\\
0.6579658	4.981582\\
0.6580658	4.973353\\
0.6581658	4.96512\\
0.6582658	4.956884\\
0.6583658	4.948644\\
0.6584658	4.940401\\
0.6585659	4.932155\\
0.6586659	4.923906\\
0.6587659	4.915653\\
0.6588659	4.907398\\
0.6589659	4.899139\\
0.6590659	4.890877\\
0.6591659	4.882613\\
0.6592659	4.874346\\
0.6593659	4.866076\\
0.6594659	4.857803\\
0.659566	4.849528\\
0.659666	4.84125\\
0.659766	4.832969\\
0.659866	4.824686\\
0.659966	4.816401\\
0.660066	4.808113\\
0.660166	4.799824\\
0.660266	4.791531\\
0.660366	4.783237\\
0.660466	4.774941\\
0.6605661	4.766642\\
0.6606661	4.758342\\
0.6607661	4.75004\\
0.6608661	4.741736\\
0.6609661	4.73343\\
0.6610661	4.725122\\
0.6611661	4.716813\\
0.6612661	4.708502\\
0.6613661	4.70019\\
0.6614661	4.691876\\
0.6615662	4.683561\\
0.6616662	4.675244\\
0.6617662	4.666926\\
0.6618662	4.658607\\
0.6619662	4.650287\\
0.6620662	4.641965\\
0.6621662	4.633643\\
0.6622662	4.625319\\
0.6623662	4.616995\\
0.6624662	4.608669\\
0.6625663	4.600343\\
0.6626663	4.592016\\
0.6627663	4.583688\\
0.6628663	4.57536\\
0.6629663	4.567031\\
0.6630663	4.558701\\
0.6631663	4.550371\\
0.6632663	4.542041\\
0.6633663	4.53371\\
0.6634663	4.525379\\
0.6635664	4.517048\\
0.6636664	4.508716\\
0.6637664	4.500385\\
0.6638664	4.492053\\
0.6639664	4.483721\\
0.6640664	4.47539\\
0.6641664	4.467058\\
0.6642664	4.458727\\
0.6643664	4.450395\\
0.6644664	4.442064\\
0.6645665	4.433734\\
0.6646665	4.425404\\
0.6647665	4.417074\\
0.6648665	4.408744\\
0.6649665	4.400416\\
0.6650665	4.392088\\
0.6651665	4.38376\\
0.6652665	4.375434\\
0.6653665	4.367107\\
0.6654665	4.358782\\
0.6655666	4.350458\\
0.6656666	4.342135\\
0.6657666	4.333813\\
0.6658666	4.325491\\
0.6659666	4.317171\\
0.6660666	4.308852\\
0.6661666	4.300534\\
0.6662666	4.292218\\
0.6663666	4.283903\\
0.6664666	4.275589\\
0.6665667	4.267277\\
0.6666667	4.258966\\
0.6667667	4.250657\\
0.6668667	4.242349\\
0.6669667	4.234043\\
0.6670667	4.225739\\
0.6671667	4.217436\\
0.6672667	4.209135\\
0.6673667	4.200837\\
0.6674667	4.19254\\
0.6675668	4.184245\\
0.6676668	4.175952\\
0.6677668	4.167661\\
0.6678668	4.159372\\
0.6679668	4.151086\\
0.6680668	4.142802\\
0.6681668	4.13452\\
0.6682668	4.12624\\
0.6683668	4.117963\\
0.6684668	4.109688\\
0.6685669	4.101416\\
0.6686669	4.093146\\
0.6687669	4.084879\\
0.6688669	4.076614\\
0.6689669	4.068353\\
0.6690669	4.060094\\
0.6691669	4.051838\\
0.6692669	4.043584\\
0.6693669	4.035334\\
0.6694669	4.027086\\
0.669567	4.018842\\
0.669667	4.010601\\
0.669767	4.002363\\
0.669867	3.994128\\
0.669967	3.985896\\
0.670067	3.977667\\
0.670167	3.969442\\
0.670267	3.96122\\
0.670367	3.953002\\
0.670467	3.944787\\
0.6705671	3.936575\\
0.6706671	3.928367\\
0.6707671	3.920163\\
0.6708671	3.911962\\
0.6709671	3.903765\\
0.6710671	3.895572\\
0.6711671	3.887383\\
0.6712671	3.879197\\
0.6713671	3.871016\\
0.6714671	3.862838\\
0.6715672	3.854665\\
0.6716672	3.846495\\
0.6717672	3.838329\\
0.6718672	3.830168\\
0.6719672	3.822011\\
0.6720672	3.813858\\
0.6721672	3.805709\\
0.6722672	3.797565\\
0.6723672	3.789425\\
0.6724672	3.78129\\
0.6725673	3.773159\\
0.6726673	3.765032\\
0.6727673	3.75691\\
0.6728673	3.748793\\
0.6729673	3.74068\\
0.6730673	3.732572\\
0.6731673	3.724469\\
0.6732673	3.716371\\
0.6733673	3.708277\\
0.6734673	3.700188\\
0.6735674	3.692105\\
0.6736674	3.684026\\
0.6737674	3.675952\\
0.6738674	3.667884\\
0.6739674	3.65982\\
0.6740674	3.651761\\
0.6741674	3.643708\\
0.6742674	3.63566\\
0.6743674	3.627617\\
0.6744674	3.61958\\
0.6745675	3.611548\\
0.6746675	3.603522\\
0.6747675	3.5955\\
0.6748675	3.587485\\
0.6749675	3.579475\\
0.6750675	3.57147\\
0.6751675	3.563471\\
0.6752675	3.555478\\
0.6753675	3.54749\\
0.6754675	3.539509\\
0.6755676	3.531533\\
0.6756676	3.523563\\
0.6757676	3.515598\\
0.6758676	3.50764\\
0.6759676	3.499688\\
0.6760676	3.491741\\
0.6761676	3.483801\\
0.6762676	3.475867\\
0.6763676	3.467939\\
0.6764676	3.460017\\
0.6765677	3.452101\\
0.6766677	3.444192\\
0.6767677	3.436288\\
0.6768677	3.428392\\
0.6769677	3.420501\\
0.6770677	3.412617\\
0.6771677	3.404739\\
0.6772677	3.396868\\
0.6773677	3.389004\\
0.6774677	3.381146\\
0.6775678	3.373294\\
0.6776678	3.365449\\
0.6777678	3.357611\\
0.6778678	3.34978\\
0.6779678	3.341955\\
0.6780678	3.334137\\
0.6781678	3.326326\\
0.6782678	3.318522\\
0.6783678	3.310725\\
0.6784678	3.302935\\
0.6785679	3.295151\\
0.6786679	3.287375\\
0.6787679	3.279606\\
0.6788679	3.271844\\
0.6789679	3.264089\\
0.6790679	3.256341\\
0.6791679	3.2486\\
0.6792679	3.240867\\
0.6793679	3.233141\\
0.6794679	3.225422\\
0.679568	3.217711\\
0.679668	3.210007\\
0.679768	3.20231\\
0.679868	3.194621\\
0.679968	3.186939\\
0.680068	3.179265\\
0.680168	3.171598\\
0.680268	3.163939\\
0.680368	3.156288\\
0.680468	3.148644\\
0.6805681	3.141008\\
0.6806681	3.13338\\
0.6807681	3.125759\\
0.6808681	3.118146\\
0.6809681	3.110541\\
0.6810681	3.102944\\
0.6811681	3.095355\\
0.6812681	3.087774\\
0.6813681	3.080201\\
0.6814681	3.072635\\
0.6815682	3.065078\\
0.6816682	3.057529\\
0.6817682	3.049988\\
0.6818682	3.042455\\
0.6819682	3.03493\\
0.6820682	3.027413\\
0.6821682	3.019905\\
0.6822682	3.012405\\
0.6823682	3.004913\\
0.6824682	2.997429\\
0.6825683	2.989954\\
0.6826683	2.982487\\
0.6827683	2.975029\\
0.6828683	2.967578\\
0.6829683	2.960137\\
0.6830683	2.952704\\
0.6831683	2.945279\\
0.6832683	2.937863\\
0.6833683	2.930456\\
0.6834683	2.923057\\
0.6835684	2.915667\\
0.6836684	2.908285\\
0.6837684	2.900912\\
0.6838684	2.893548\\
0.6839684	2.886193\\
0.6840684	2.878847\\
0.6841684	2.871509\\
0.6842684	2.86418\\
0.6843684	2.85686\\
0.6844684	2.849549\\
0.6845685	2.842246\\
0.6846685	2.834953\\
0.6847685	2.827669\\
0.6848685	2.820394\\
0.6849685	2.813127\\
0.6850685	2.80587\\
0.6851685	2.798622\\
0.6852685	2.791383\\
0.6853685	2.784153\\
0.6854685	2.776932\\
0.6855686	2.769721\\
0.6856686	2.762519\\
0.6857686	2.755325\\
0.6858686	2.748142\\
0.6859686	2.740967\\
0.6860686	2.733802\\
0.6861686	2.726646\\
0.6862686	2.7195\\
0.6863686	2.712363\\
0.6864686	2.705235\\
0.6865687	2.698117\\
0.6866687	2.691008\\
0.6867687	2.683909\\
0.6868687	2.676819\\
0.6869687	2.669739\\
0.6870687	2.662668\\
0.6871687	2.655607\\
0.6872687	2.648555\\
0.6873687	2.641513\\
0.6874687	2.634481\\
0.6875688	2.627459\\
0.6876688	2.620446\\
0.6877688	2.613443\\
0.6878688	2.606449\\
0.6879688	2.599466\\
0.6880688	2.592492\\
0.6881688	2.585528\\
0.6882688	2.578574\\
0.6883688	2.571629\\
0.6884688	2.564695\\
0.6885689	2.55777\\
0.6886689	2.550856\\
0.6887689	2.543951\\
0.6888689	2.537056\\
0.6889689	2.530171\\
0.6890689	2.523296\\
0.6891689	2.516431\\
0.6892689	2.509577\\
0.6893689	2.502732\\
0.6894689	2.495897\\
0.689569	2.489073\\
0.689669	2.482258\\
0.689769	2.475454\\
0.689869	2.46866\\
0.689969	2.461876\\
0.690069	2.455102\\
0.690169	2.448338\\
0.690269	2.441585\\
0.690369	2.434842\\
0.690469	2.428109\\
0.6905691	2.421386\\
0.6906691	2.414674\\
0.6907691	2.407972\\
0.6908691	2.401281\\
0.6909691	2.394599\\
0.6910691	2.387928\\
0.6911691	2.381268\\
0.6912691	2.374617\\
0.6913691	2.367978\\
0.6914691	2.361348\\
0.6915692	2.35473\\
0.6916692	2.348121\\
0.6917692	2.341523\\
0.6918692	2.334936\\
0.6919692	2.328359\\
0.6920692	2.321793\\
0.6921692	2.315237\\
0.6922692	2.308692\\
0.6923692	2.302157\\
0.6924692	2.295633\\
0.6925693	2.289119\\
0.6926693	2.282617\\
0.6927693	2.276124\\
0.6928693	2.269643\\
0.6929693	2.263172\\
0.6930693	2.256711\\
0.6931693	2.250262\\
0.6932693	2.243823\\
0.6933693	2.237395\\
0.6934693	2.230978\\
0.6935694	2.224571\\
0.6936694	2.218175\\
0.6937694	2.21179\\
0.6938694	2.205416\\
0.6939694	2.199052\\
0.6940694	2.1927\\
0.6941694	2.186358\\
0.6942694	2.180027\\
0.6943694	2.173706\\
0.6944694	2.167397\\
0.6945695	2.161098\\
0.6946695	2.154811\\
0.6947695	2.148534\\
0.6948695	2.142268\\
0.6949695	2.136014\\
0.6950695	2.12977\\
0.6951695	2.123537\\
0.6952695	2.117315\\
0.6953695	2.111104\\
0.6954695	2.104904\\
0.6955696	2.098714\\
0.6956696	2.092536\\
0.6957696	2.086369\\
0.6958696	2.080213\\
0.6959696	2.074068\\
0.6960696	2.067934\\
0.6961696	2.061811\\
0.6962696	2.055699\\
0.6963696	2.049598\\
0.6964696	2.043508\\
0.6965697	2.037429\\
0.6966697	2.031361\\
0.6967697	2.025305\\
0.6968697	2.019259\\
0.6969697	2.013225\\
0.6970697	2.007202\\
0.6971697	2.00119\\
0.6972697	1.995188\\
0.6973697	1.989199\\
0.6974697	1.98322\\
0.6975698	1.977252\\
0.6976698	1.971296\\
0.6977698	1.965351\\
0.6978698	1.959417\\
0.6979698	1.953494\\
0.6980698	1.947583\\
0.6981698	1.941682\\
0.6982698	1.935793\\
0.6983698	1.929915\\
0.6984698	1.924048\\
0.6985699	1.918193\\
0.6986699	1.912348\\
0.6987699	1.906515\\
0.6988699	1.900693\\
0.6989699	1.894883\\
0.6990699	1.889083\\
0.6991699	1.883295\\
0.6992699	1.877519\\
0.6993699	1.871753\\
0.6994699	1.865999\\
0.69957	1.860256\\
0.69967	1.854524\\
0.69977	1.848804\\
0.69987	1.843095\\
0.69997	1.837397\\
0.70007	1.831711\\
0.70017	1.826036\\
0.70027	1.820372\\
0.70037	1.814719\\
0.70047	1.809078\\
0.7005701	1.803448\\
0.7006701	1.79783\\
0.7007701	1.792222\\
0.7008701	1.786627\\
0.7009701	1.781042\\
0.7010701	1.775469\\
0.7011701	1.769907\\
0.7012701	1.764357\\
0.7013701	1.758817\\
0.7014701	1.75329\\
0.7015702	1.747773\\
0.7016702	1.742268\\
0.7017702	1.736774\\
0.7018702	1.731292\\
0.7019702	1.725821\\
0.7020702	1.720361\\
0.7021702	1.714913\\
0.7022702	1.709476\\
0.7023702	1.704051\\
0.7024702	1.698637\\
0.7025703	1.693234\\
0.7026703	1.687842\\
0.7027703	1.682462\\
0.7028703	1.677094\\
0.7029703	1.671736\\
0.7030703	1.66639\\
0.7031703	1.661056\\
0.7032703	1.655733\\
0.7033703	1.650421\\
0.7034703	1.645121\\
0.7035704	1.639832\\
0.7036704	1.634554\\
0.7037704	1.629288\\
0.7038704	1.624033\\
0.7039704	1.618789\\
0.7040704	1.613557\\
0.7041704	1.608336\\
0.7042704	1.603127\\
0.7043704	1.597929\\
0.7044704	1.592742\\
0.7045705	1.587567\\
0.7046705	1.582403\\
0.7047705	1.57725\\
0.7048705	1.572109\\
0.7049705	1.566979\\
0.7050705	1.561861\\
0.7051705	1.556753\\
0.7052705	1.551658\\
0.7053705	1.546573\\
0.7054705	1.5415\\
0.7055706	1.536439\\
0.7056706	1.531388\\
0.7057706	1.526349\\
0.7058706	1.521321\\
0.7059706	1.516305\\
0.7060706	1.5113\\
0.7061706	1.506306\\
0.7062706	1.501324\\
0.7063706	1.496353\\
0.7064706	1.491393\\
0.7065707	1.486445\\
0.7066707	1.481508\\
0.7067707	1.476582\\
0.7068707	1.471668\\
0.7069707	1.466765\\
0.7070707	1.461873\\
0.7071707	1.456992\\
0.7072707	1.452123\\
0.7073707	1.447265\\
0.7074707	1.442418\\
0.7075708	1.437583\\
0.7076708	1.432759\\
0.7077708	1.427946\\
0.7078708	1.423144\\
0.7079708	1.418354\\
0.7080708	1.413575\\
0.7081708	1.408807\\
0.7082708	1.404051\\
0.7083708	1.399305\\
0.7084708	1.394571\\
0.7085709	1.389848\\
0.7086709	1.385137\\
0.7087709	1.380436\\
0.7088709	1.375747\\
0.7089709	1.371069\\
0.7090709	1.366402\\
0.7091709	1.361747\\
0.7092709	1.357102\\
0.7093709	1.352469\\
0.7094709	1.347847\\
0.709571	1.343236\\
0.709671	1.338636\\
0.709771	1.334048\\
0.709871	1.329471\\
0.709971	1.324904\\
0.710071	1.320349\\
0.710171	1.315805\\
0.710271	1.311272\\
0.710371	1.306751\\
0.710471	1.30224\\
0.7105711	1.29774\\
0.7106711	1.293252\\
0.7107711	1.288775\\
0.7108711	1.284308\\
0.7109711	1.279853\\
0.7110711	1.275409\\
0.7111711	1.270976\\
0.7112711	1.266554\\
0.7113711	1.262143\\
0.7114711	1.257743\\
0.7115712	1.253354\\
0.7116712	1.248976\\
0.7117712	1.244609\\
0.7118712	1.240253\\
0.7119712	1.235908\\
0.7120712	1.231574\\
0.7121712	1.227251\\
0.7122712	1.222939\\
0.7123712	1.218638\\
0.7124712	1.214348\\
0.7125713	1.210069\\
0.7126713	1.2058\\
0.7127713	1.201543\\
0.7128713	1.197296\\
0.7129713	1.193061\\
0.7130713	1.188836\\
0.7131713	1.184622\\
0.7132713	1.18042\\
0.7133713	1.176228\\
0.7134713	1.172046\\
0.7135714	1.167876\\
0.7136714	1.163716\\
0.7137714	1.159568\\
0.7138714	1.15543\\
0.7139714	1.151303\\
0.7140714	1.147186\\
0.7141714	1.143081\\
0.7142714	1.138986\\
0.7143714	1.134902\\
0.7144714	1.130829\\
0.7145715	1.126766\\
0.7146715	1.122715\\
0.7147715	1.118674\\
0.7148715	1.114643\\
0.7149715	1.110624\\
0.7150715	1.106615\\
0.7151715	1.102616\\
0.7152715	1.098629\\
0.7153715	1.094652\\
0.7154715	1.090686\\
0.7155716	1.08673\\
0.7156716	1.082785\\
0.7157716	1.07885\\
0.7158716	1.074926\\
0.7159716	1.071013\\
0.7160716	1.06711\\
0.7161716	1.063218\\
0.7162716	1.059337\\
0.7163716	1.055465\\
0.7164716	1.051605\\
0.7165717	1.047755\\
0.7166717	1.043915\\
0.7167717	1.040086\\
0.7168717	1.036268\\
0.7169717	1.032459\\
0.7170717	1.028662\\
0.7171717	1.024875\\
0.7172717	1.021098\\
0.7173717	1.017331\\
0.7174717	1.013575\\
0.7175718	1.00983\\
0.7176718	1.006095\\
0.7177718	1.00237\\
0.7178718	0.9986553\\
0.7179718	0.9949511\\
0.7180718	0.9912572\\
0.7181718	0.9875737\\
0.7182718	0.9839004\\
0.7183718	0.9802376\\
0.7184718	0.9765849\\
0.7185719	0.9729425\\
0.7186719	0.9693103\\
0.7187719	0.9656884\\
0.7188719	0.9620767\\
0.7189719	0.9584751\\
0.7190719	0.9548839\\
0.7191719	0.9513028\\
0.7192719	0.9477318\\
0.7193719	0.944171\\
0.7194719	0.9406203\\
0.719572	0.9370798\\
0.719672	0.9335494\\
0.719772	0.9300291\\
0.719872	0.9265189\\
0.719972	0.9230187\\
0.720072	0.9195286\\
0.720172	0.9160485\\
0.720272	0.9125784\\
0.720372	0.9091184\\
0.720472	0.9056683\\
0.7205721	0.9022283\\
0.7206721	0.8987983\\
0.7207721	0.8953782\\
0.7208721	0.891968\\
0.7209721	0.8885678\\
0.7210721	0.8851774\\
0.7211721	0.881797\\
0.7212721	0.8784264\\
0.7213721	0.8750657\\
0.7214721	0.8717149\\
0.7215722	0.8683738\\
0.7216722	0.8650427\\
0.7217722	0.8617214\\
0.7218722	0.8584099\\
0.7219722	0.8551081\\
0.7220722	0.8518161\\
0.7221722	0.8485339\\
0.7222722	0.8452613\\
0.7223722	0.8419986\\
0.7224722	0.8387455\\
0.7225723	0.8355021\\
0.7226723	0.8322684\\
0.7227723	0.8290443\\
0.7228723	0.8258298\\
0.7229723	0.822625\\
0.7230723	0.8194299\\
0.7231723	0.8162444\\
0.7232723	0.8130684\\
0.7233723	0.809902\\
0.7234723	0.8067452\\
0.7235724	0.8035978\\
0.7236724	0.80046\\
0.7237724	0.7973317\\
0.7238724	0.7942129\\
0.7239724	0.7911036\\
0.7240724	0.7880037\\
0.7241724	0.7849133\\
0.7242724	0.7818323\\
0.7243724	0.7787607\\
0.7244724	0.7756985\\
0.7245725	0.7726457\\
0.7246725	0.7696023\\
0.7247725	0.7665682\\
0.7248725	0.7635434\\
0.7249725	0.7605279\\
0.7250725	0.7575218\\
0.7251725	0.7545249\\
0.7252725	0.7515373\\
0.7253725	0.7485589\\
0.7254725	0.7455899\\
0.7255726	0.74263\\
0.7256726	0.7396793\\
0.7257726	0.7367379\\
0.7258726	0.7338055\\
0.7259726	0.7308823\\
0.7260726	0.7279683\\
0.7261726	0.7250634\\
0.7262726	0.7221676\\
0.7263726	0.7192808\\
0.7264726	0.7164032\\
0.7265727	0.7135345\\
0.7266727	0.710675\\
0.7267727	0.7078245\\
0.7268727	0.704983\\
0.7269727	0.7021505\\
0.7270727	0.6993269\\
0.7271727	0.6965123\\
0.7272727	0.6937066\\
0.7273727	0.6909099\\
0.7274727	0.688122\\
0.7275728	0.685343\\
0.7276728	0.682573\\
0.7277728	0.6798118\\
0.7278728	0.6770594\\
0.7279728	0.6743158\\
0.7280728	0.6715811\\
0.7281728	0.6688551\\
0.7282728	0.6661379\\
0.7283728	0.6634294\\
0.7284728	0.6607296\\
0.7285729	0.6580387\\
0.7286729	0.6553564\\
0.7287729	0.6526828\\
0.7288729	0.650018\\
0.7289729	0.6473616\\
0.7290729	0.644714\\
0.7291729	0.642075\\
0.7292729	0.6394445\\
0.7293729	0.6368227\\
0.7294729	0.6342094\\
0.729573	0.6316047\\
0.729673	0.6290085\\
0.729773	0.6264208\\
0.729873	0.6238417\\
0.729973	0.6212709\\
0.730073	0.6187087\\
0.730173	0.6161548\\
0.730273	0.6136094\\
0.730373	0.6110726\\
0.730473	0.608544\\
0.7305731	0.6060238\\
0.7306731	0.603512\\
0.7307731	0.6010084\\
0.7308731	0.5985133\\
0.7309731	0.5960264\\
0.7310731	0.5935478\\
0.7311731	0.5910776\\
0.7312731	0.5886155\\
0.7313731	0.5861617\\
0.7314731	0.5837161\\
0.7315732	0.5812787\\
0.7316732	0.5788495\\
0.7317732	0.5764285\\
0.7318732	0.5740156\\
0.7319732	0.5716109\\
0.7320732	0.5692142\\
0.7321732	0.5668257\\
0.7322732	0.5644452\\
0.7323732	0.5620727\\
0.7324732	0.5597084\\
0.7325733	0.5573521\\
0.7326733	0.5550038\\
0.7327733	0.5526635\\
0.7328733	0.5503311\\
0.7329733	0.5480067\\
0.7330733	0.5456902\\
0.7331733	0.5433817\\
0.7332733	0.5410811\\
0.7333733	0.5387883\\
0.7334733	0.5365034\\
0.7335734	0.5342264\\
0.7336734	0.5319572\\
0.7337734	0.5296958\\
0.7338734	0.5274422\\
0.7339734	0.5251964\\
0.7340734	0.5229583\\
0.7341734	0.520728\\
0.7342734	0.5185054\\
0.7343734	0.5162905\\
0.7344734	0.5140833\\
0.7345735	0.5118838\\
0.7346735	0.509692\\
0.7347735	0.5075078\\
0.7348735	0.5053312\\
0.7349735	0.5031622\\
0.7350735	0.5010007\\
0.7351735	0.4988469\\
0.7352735	0.4967007\\
0.7353735	0.4945619\\
0.7354735	0.4924306\\
0.7355736	0.4903069\\
0.7356736	0.4881906\\
0.7357736	0.4860818\\
0.7358736	0.4839804\\
0.7359736	0.4818864\\
0.7360736	0.4797999\\
0.7361736	0.4777208\\
0.7362736	0.475649\\
0.7363736	0.4735846\\
0.7364736	0.4715275\\
0.7365737	0.4694777\\
0.7366737	0.4674352\\
0.7367737	0.4654\\
0.7368737	0.4633721\\
0.7369737	0.4613515\\
0.7370737	0.4593381\\
0.7371737	0.4573318\\
0.7372737	0.4553327\\
0.7373737	0.4533408\\
0.7374737	0.4513562\\
0.7375738	0.4493786\\
0.7376738	0.4474081\\
0.7377738	0.4454449\\
0.7378738	0.4434886\\
0.7379738	0.4415394\\
0.7380738	0.4395974\\
0.7381738	0.4376622\\
0.7382738	0.4357342\\
0.7383738	0.4338131\\
0.7384738	0.431899\\
0.7385739	0.4299919\\
0.7386739	0.4280917\\
0.7387739	0.4261986\\
0.7388739	0.4243122\\
0.7389739	0.4224328\\
0.7390739	0.4205602\\
0.7391739	0.4186945\\
0.7392739	0.4168356\\
0.7393739	0.4149835\\
0.7394739	0.4131383\\
0.739574	0.4112999\\
0.739674	0.4094682\\
0.739774	0.4076433\\
0.739874	0.4058252\\
0.739974	0.4040137\\
0.740074	0.402209\\
0.740174	0.4004109\\
0.740274	0.3986195\\
0.740374	0.3968347\\
0.740474	0.3950565\\
0.7405741	0.393285\\
0.7406741	0.3915201\\
0.7407741	0.3897617\\
0.7408741	0.38801\\
0.7409741	0.3862647\\
0.7410741	0.384526\\
0.7411741	0.3827938\\
0.7412741	0.3810681\\
0.7413741	0.3793489\\
0.7414741	0.377636\\
0.7415742	0.3759297\\
0.7416742	0.3742298\\
0.7417742	0.3725362\\
0.7418742	0.3708491\\
0.7419742	0.3691683\\
0.7420742	0.3674939\\
0.7421742	0.3658258\\
0.7422742	0.364164\\
0.7423742	0.3625086\\
0.7424742	0.3608595\\
0.7425743	0.3592165\\
0.7426743	0.3575799\\
0.7427743	0.3559494\\
0.7428743	0.3543252\\
0.7429743	0.3527072\\
0.7430743	0.3510953\\
0.7431743	0.3494896\\
0.7432743	0.34789\\
0.7433743	0.3462967\\
0.7434743	0.3447094\\
0.7435744	0.3431282\\
0.7436744	0.3415531\\
0.7437744	0.339984\\
0.7438744	0.338421\\
0.7439744	0.336864\\
0.7440744	0.335313\\
0.7441744	0.333768\\
0.7442744	0.332229\\
0.7443744	0.3306959\\
0.7444744	0.3291688\\
0.7445745	0.3276476\\
0.7446745	0.3261323\\
0.7447745	0.3246229\\
0.7448745	0.3231194\\
0.7449745	0.3216217\\
0.7450745	0.3201298\\
0.7451745	0.3186438\\
0.7452745	0.3171637\\
0.7453745	0.3156892\\
0.7454745	0.3142206\\
0.7455746	0.3127577\\
0.7456746	0.3113005\\
0.7457746	0.3098491\\
0.7458746	0.3084033\\
0.7459746	0.3069633\\
0.7460746	0.305529\\
0.7461746	0.3041003\\
0.7462746	0.3026772\\
0.7463746	0.3012597\\
0.7464746	0.2998479\\
0.7465747	0.2984416\\
0.7466747	0.297041\\
0.7467747	0.2956458\\
0.7468747	0.2942562\\
0.7469747	0.2928722\\
0.7470747	0.2914937\\
0.7471747	0.2901205\\
0.7472747	0.288753\\
0.7473747	0.2873909\\
0.7474747	0.2860342\\
0.7475748	0.2846829\\
0.7476748	0.283337\\
0.7477748	0.2819965\\
0.7478748	0.2806614\\
0.7479748	0.2793316\\
0.7480748	0.2780073\\
0.7481748	0.2766883\\
0.7482748	0.2753745\\
0.7483748	0.274066\\
0.7484748	0.2727628\\
0.7485749	0.2714649\\
0.7486749	0.2701722\\
0.7487749	0.2688847\\
0.7488749	0.2676025\\
0.7489749	0.2663255\\
0.7490749	0.2650536\\
0.7491749	0.2637869\\
0.7492749	0.2625254\\
0.7493749	0.261269\\
0.7494749	0.2600177\\
0.749575	0.2587715\\
0.749675	0.2575304\\
0.749775	0.2562944\\
0.749875	0.2550634\\
0.749975	0.2538375\\
0.750075	0.2526165\\
0.750175	0.2514006\\
0.750275	0.2501897\\
0.750375	0.2489838\\
0.750475	0.2477828\\
0.7505751	0.2465867\\
0.7506751	0.2453956\\
0.7507751	0.2442094\\
0.7508751	0.2430281\\
0.7509751	0.2418517\\
0.7510751	0.2406801\\
0.7511751	0.2395134\\
0.7512751	0.2383515\\
0.7513751	0.2371945\\
0.7514751	0.2360423\\
0.7515752	0.2348948\\
0.7516752	0.2337521\\
0.7517752	0.2326142\\
0.7518752	0.231481\\
0.7519752	0.2303525\\
0.7520752	0.2292288\\
0.7521752	0.2281097\\
0.7522752	0.2269954\\
0.7523752	0.2258857\\
0.7524752	0.2247806\\
0.7525753	0.2236802\\
0.7526753	0.2225844\\
0.7527753	0.2214932\\
0.7528753	0.2204066\\
0.7529753	0.2193245\\
0.7530753	0.2182471\\
0.7531753	0.2171742\\
0.7532753	0.2161058\\
0.7533753	0.2150419\\
0.7534753	0.2139826\\
0.7535754	0.2129276\\
0.7536754	0.2118772\\
0.7537754	0.2108313\\
0.7538754	0.2097898\\
0.7539754	0.2087527\\
0.7540754	0.20772\\
0.7541754	0.2066917\\
0.7542754	0.2056678\\
0.7543754	0.2046482\\
0.7544754	0.2036331\\
0.7545755	0.2026222\\
0.7546755	0.2016157\\
0.7547755	0.2006135\\
0.7548755	0.1996155\\
0.7549755	0.1986219\\
0.7550755	0.1976326\\
0.7551755	0.1966475\\
0.7552755	0.1956666\\
0.7553755	0.1946899\\
0.7554755	0.1937174\\
0.7555756	0.1927492\\
0.7556756	0.1917851\\
0.7557756	0.1908252\\
0.7558756	0.1898694\\
0.7559756	0.1889178\\
0.7560756	0.1879703\\
0.7561756	0.1870269\\
0.7562756	0.1860876\\
0.7563756	0.1851524\\
0.7564756	0.1842212\\
0.7565757	0.1832941\\
0.7566757	0.182371\\
0.7567757	0.1814519\\
0.7568757	0.1805369\\
0.7569757	0.1796259\\
0.7570757	0.1787188\\
0.7571757	0.1778157\\
0.7572757	0.1769166\\
0.7573757	0.1760214\\
0.7574757	0.1751301\\
0.7575758	0.1742428\\
0.7576758	0.1733593\\
0.7577758	0.1724798\\
0.7578758	0.171604\\
0.7579758	0.1707322\\
0.7580758	0.1698642\\
0.7581758	0.169\\
0.7582758	0.1681397\\
0.7583758	0.1672831\\
0.7584758	0.1664304\\
0.7585759	0.1655814\\
0.7586759	0.1647362\\
0.7587759	0.1638947\\
0.7588759	0.163057\\
0.7589759	0.162223\\
0.7590759	0.1613927\\
0.7591759	0.1605661\\
0.7592759	0.1597432\\
0.7593759	0.1589239\\
0.7594759	0.1581084\\
0.759576	0.1572964\\
0.759676	0.1564881\\
0.759776	0.1556834\\
0.759876	0.1548823\\
0.759976	0.1540848\\
0.760076	0.1532909\\
0.760176	0.1525005\\
0.760276	0.1517138\\
0.760376	0.1509305\\
0.760476	0.1501508\\
0.7605761	0.1493746\\
0.7606761	0.1486018\\
0.7607761	0.1478325\\
0.7608761	0.1470668\\
0.7609761	0.1463045\\
0.7610761	0.1455457\\
0.7611761	0.1447902\\
0.7612761	0.1440383\\
0.7613761	0.1432897\\
0.7614761	0.1425445\\
0.7615762	0.1418027\\
0.7616762	0.1410644\\
0.7617762	0.1403293\\
0.7618762	0.1395976\\
0.7619762	0.1388692\\
0.7620762	0.1381442\\
0.7621762	0.1374225\\
0.7622762	0.136704\\
0.7623762	0.1359889\\
0.7624762	0.135277\\
0.7625763	0.1345684\\
0.7626763	0.1338631\\
0.7627763	0.133161\\
0.7628763	0.1324621\\
0.7629763	0.1317664\\
0.7630763	0.131074\\
0.7631763	0.1303847\\
0.7632763	0.1296986\\
0.7633763	0.1290157\\
0.7634763	0.1283359\\
0.7635764	0.1276593\\
0.7636764	0.1269857\\
0.7637764	0.1263153\\
0.7638764	0.125648\\
0.7639764	0.1249838\\
0.7640764	0.1243228\\
0.7641764	0.1236647\\
0.7642764	0.1230097\\
0.7643764	0.1223578\\
0.7644764	0.1217089\\
0.7645765	0.1210631\\
0.7646765	0.1204202\\
0.7647765	0.1197804\\
0.7648765	0.1191435\\
0.7649765	0.1185096\\
0.7650765	0.1178787\\
0.7651765	0.1172507\\
0.7652765	0.1166257\\
0.7653765	0.1160036\\
0.7654765	0.1153845\\
0.7655766	0.1147682\\
0.7656766	0.1141548\\
0.7657766	0.1135444\\
0.7658766	0.1129368\\
0.7659766	0.112332\\
0.7660766	0.1117302\\
0.7661766	0.1111312\\
0.7662766	0.1105349\\
0.7663766	0.1099416\\
0.7664766	0.109351\\
0.7665767	0.1087632\\
0.7666767	0.1081782\\
0.7667767	0.107596\\
0.7668767	0.1070165\\
0.7669767	0.1064399\\
0.7670767	0.1058659\\
0.7671767	0.1052947\\
0.7672767	0.1047262\\
0.7673767	0.1041604\\
0.7674767	0.1035973\\
0.7675768	0.1030369\\
0.7676768	0.1024792\\
0.7677768	0.1019241\\
0.7678768	0.1013717\\
0.7679768	0.100822\\
0.7680768	0.1002749\\
0.7681768	0.09973036\\
0.7682768	0.0991885\\
0.7683768	0.09864924\\
0.7684768	0.09811259\\
0.7685769	0.09757849\\
0.7686769	0.09704697\\
0.7687769	0.096518\\
0.7688769	0.09599159\\
0.7689769	0.09546775\\
0.7690769	0.09494642\\
0.7691769	0.0944276\\
0.7692769	0.09391132\\
0.7693769	0.09339752\\
0.7694769	0.09288621\\
0.769577	0.0923774\\
0.769677	0.09187106\\
0.769777	0.09136716\\
0.769877	0.09086572\\
0.769977	0.09036672\\
0.770077	0.08987015\\
0.770177	0.08937603\\
0.770277	0.08888426\\
0.770377	0.08839492\\
0.770477	0.08790798\\
0.7705771	0.08742342\\
0.7706771	0.08694124\\
0.7707771	0.08646142\\
0.7708771	0.08598395\\
0.7709771	0.08550882\\
0.7710771	0.085036\\
0.7711771	0.08456552\\
0.7712771	0.08409736\\
0.7713771	0.0836315\\
0.7714771	0.08316792\\
0.7715772	0.08270665\\
0.7716772	0.08224766\\
0.7717772	0.08179091\\
0.7718772	0.08133643\\
0.7719772	0.08088419\\
0.7720772	0.0804342\\
0.7721772	0.07998642\\
0.7722772	0.07954085\\
0.7723772	0.0790975\\
0.7724772	0.07865636\\
0.7725773	0.07821741\\
0.7726773	0.07778065\\
0.7727773	0.07734607\\
0.7728773	0.07691364\\
0.7729773	0.07648335\\
0.7730773	0.0760552\\
0.7731773	0.07562918\\
0.7732773	0.07520531\\
0.7733773	0.07478357\\
0.7734773	0.07436391\\
0.7735774	0.07394637\\
0.7736774	0.07353094\\
0.7737774	0.07311757\\
0.7738774	0.07270627\\
0.7739774	0.07229704\\
0.7740774	0.07188986\\
0.7741774	0.07148474\\
0.7742774	0.07108165\\
0.7743774	0.07068057\\
0.7744774	0.07028152\\
0.7745775	0.06988451\\
0.7746775	0.06948947\\
0.7747775	0.06909645\\
0.7748775	0.0687054\\
0.7749775	0.06831632\\
0.7750775	0.06792924\\
0.7751775	0.0675441\\
0.7752775	0.06716092\\
0.7753775	0.0667797\\
0.7754775	0.06640039\\
0.7755776	0.06602299\\
0.7756776	0.06564752\\
0.7757776	0.06527399\\
0.7758776	0.06490234\\
0.7759776	0.06453261\\
0.7760776	0.06416473\\
0.7761776	0.06379873\\
0.7762776	0.06343462\\
0.7763776	0.06307236\\
0.7764776	0.06271195\\
0.7765777	0.0623534\\
0.7766777	0.06199668\\
0.7767777	0.06164177\\
0.7768777	0.06128869\\
0.7769777	0.06093743\\
0.7770777	0.06058795\\
0.7771777	0.06024029\\
0.7772777	0.05989442\\
0.7773777	0.05955031\\
0.7774777	0.05920802\\
0.7775778	0.05886745\\
0.7776778	0.05852865\\
0.7777778	0.05819161\\
0.7778778	0.05785633\\
0.7779778	0.05752276\\
0.7780778	0.05719092\\
0.7781778	0.05686079\\
0.7782778	0.05653235\\
0.7783778	0.05620564\\
0.7784778	0.05588061\\
0.7785779	0.05555728\\
0.7786779	0.05523564\\
0.7787779	0.05491566\\
0.7788779	0.05459733\\
0.7789779	0.05428069\\
0.7790779	0.0539657\\
0.7791779	0.05365234\\
0.7792779	0.0533406\\
0.7793779	0.05303051\\
0.7794779	0.05272203\\
0.779578	0.05241519\\
0.779678	0.05210996\\
0.779778	0.05180629\\
0.779878	0.05150423\\
0.779978	0.05120375\\
0.780078	0.05090484\\
0.780178	0.05060751\\
0.780278	0.05031175\\
0.780378	0.05001754\\
0.780478	0.0497249\\
0.7805781	0.04943379\\
0.7806781	0.04914419\\
0.7807781	0.04885615\\
0.7808781	0.04856961\\
0.7809781	0.04828458\\
0.7810781	0.04800106\\
0.7811781	0.04771907\\
0.7812781	0.04743856\\
0.7813781	0.04715952\\
0.7814781	0.04688197\\
0.7815782	0.04660589\\
0.7816782	0.04633128\\
0.7817782	0.04605814\\
0.7818782	0.04578643\\
0.7819782	0.04551616\\
0.7820782	0.04524736\\
0.7821782	0.04497996\\
0.7822782	0.044714\\
0.7823782	0.04444948\\
0.7824782	0.04418637\\
0.7825783	0.04392464\\
0.7826783	0.04366434\\
0.7827783	0.04340541\\
0.7828783	0.04314787\\
0.7829783	0.04289173\\
0.7830783	0.04263693\\
0.7831783	0.0423835\\
0.7832783	0.04213147\\
0.7833783	0.04188077\\
0.7834783	0.04163141\\
0.7835784	0.0413834\\
0.7836784	0.04113672\\
0.7837784	0.04089139\\
0.7838784	0.04064738\\
0.7839784	0.04040467\\
0.7840784	0.04016327\\
0.7841784	0.03992319\\
0.7842784	0.0396844\\
0.7843784	0.03944689\\
0.7844784	0.03921069\\
0.7845785	0.03897577\\
0.7846785	0.0387421\\
0.7847785	0.03850971\\
0.7848785	0.03827859\\
0.7849785	0.03804871\\
0.7850785	0.03782011\\
0.7851785	0.03759275\\
0.7852785	0.0373666\\
0.7853785	0.03714168\\
0.7854785	0.036918\\
0.7855786	0.03669554\\
0.7856786	0.0364743\\
0.7857786	0.03625427\\
0.7858786	0.03603544\\
0.7859786	0.03581782\\
0.7860786	0.03560139\\
0.7861786	0.03538612\\
0.7862786	0.03517203\\
0.7863786	0.03495914\\
0.7864786	0.03474741\\
0.7865787	0.03453683\\
0.7866787	0.03432743\\
0.7867787	0.03411919\\
0.7868787	0.03391207\\
0.7869787	0.03370611\\
0.7870787	0.03350127\\
0.7871787	0.03329756\\
0.7872787	0.033095\\
0.7873787	0.03289354\\
0.7874787	0.03269318\\
0.7875788	0.03249394\\
0.7876788	0.03229581\\
0.7877788	0.03209878\\
0.7878788	0.03190283\\
0.7879788	0.03170797\\
0.7880788	0.03151418\\
0.7881788	0.03132148\\
0.7882788	0.03112985\\
0.7883788	0.03093929\\
0.7884788	0.03074979\\
0.7885789	0.03056135\\
0.7886789	0.03037396\\
0.7887789	0.03018759\\
0.7888789	0.03000229\\
0.7889789	0.02981802\\
0.7890789	0.02963477\\
0.7891789	0.02945255\\
0.7892789	0.02927135\\
0.7893789	0.02909116\\
0.7894789	0.02891198\\
0.789579	0.02873381\\
0.789679	0.02855663\\
0.789779	0.02838045\\
0.789879	0.02820527\\
0.789979	0.02803107\\
0.790079	0.02785784\\
0.790179	0.02768559\\
0.790279	0.02751431\\
0.790379	0.02734398\\
0.790479	0.02717463\\
0.7905791	0.02700624\\
0.7906791	0.0268388\\
0.7907791	0.02667231\\
0.7908791	0.02650675\\
0.7909791	0.02634211\\
0.7910791	0.02617843\\
0.7911791	0.02601567\\
0.7912791	0.02585383\\
0.7913791	0.02569292\\
0.7914791	0.02553291\\
0.7915792	0.02537381\\
0.7916792	0.02521562\\
0.7917792	0.02505833\\
0.7918792	0.02490193\\
0.7919792	0.02474641\\
0.7920792	0.0245918\\
0.7921792	0.02443806\\
0.7922792	0.0242852\\
0.7923792	0.02413321\\
0.7924792	0.0239821\\
0.7925793	0.02383184\\
0.7926793	0.02368245\\
0.7927793	0.02353391\\
0.7928793	0.02338622\\
0.7929793	0.02323939\\
0.7930793	0.02309339\\
0.7931793	0.02294823\\
0.7932793	0.02280389\\
0.7933793	0.0226604\\
0.7934793	0.02251773\\
0.7935794	0.02237587\\
0.7936794	0.02223483\\
0.7937794	0.0220946\\
0.7938794	0.02195519\\
0.7939794	0.02181659\\
0.7940794	0.0216788\\
0.7941794	0.02154179\\
0.7942794	0.02140558\\
0.7943794	0.02127015\\
0.7944794	0.0211355\\
0.7945795	0.02100162\\
0.7946795	0.02086853\\
0.7947795	0.02073622\\
0.7948795	0.02060465\\
0.7949795	0.02047387\\
0.7950795	0.02034384\\
0.7951795	0.02021457\\
0.7952795	0.02008606\\
0.7953795	0.01995831\\
0.7954795	0.01983127\\
0.7955796	0.01970498\\
0.7956796	0.01957944\\
0.7957796	0.01945463\\
0.7958796	0.01933054\\
0.7959796	0.01920718\\
0.7960796	0.01908454\\
0.7961796	0.01896262\\
0.7962796	0.01884142\\
0.7963796	0.01872093\\
0.7964796	0.01860113\\
0.7965797	0.01848204\\
0.7966797	0.01836365\\
0.7967797	0.01824595\\
0.7968797	0.01812895\\
0.7969797	0.01801264\\
0.7970797	0.01789702\\
0.7971797	0.01778207\\
0.7972797	0.01766779\\
0.7973797	0.0175542\\
0.7974797	0.01744127\\
0.7975798	0.01732902\\
0.7976798	0.01721743\\
0.7977798	0.01710649\\
0.7978798	0.0169962\\
0.7979798	0.01688658\\
0.7980798	0.01677761\\
0.7981798	0.01666928\\
0.7982798	0.0165616\\
0.7983798	0.01645455\\
0.7984798	0.01634813\\
0.7985799	0.01624234\\
0.7986799	0.01613719\\
0.7987799	0.01603266\\
0.7988799	0.01592875\\
0.7989799	0.01582548\\
0.7990799	0.01572281\\
0.7991799	0.01562074\\
0.7992799	0.01551929\\
0.7993799	0.01541845\\
0.7994799	0.01531821\\
0.79958	0.01521856\\
0.79968	0.01511951\\
0.79978	0.01502105\\
0.79988	0.0149232\\
0.79998	0.01482593\\
0.80008	0.01472924\\
};
\addplot [color=mycolor1,solid]
  table[row sep=crcr]{%
0.80008	0.01472924\\
0.80018	0.01463311\\
0.80028	0.01453758\\
0.80038	0.01444262\\
0.80048	0.01434823\\
0.8005801	0.01425441\\
0.8006801	0.01416116\\
0.8007801	0.01406846\\
0.8008801	0.01397632\\
0.8009801	0.01388473\\
0.8010801	0.0137937\\
0.8011801	0.01370321\\
0.8012801	0.01361329\\
0.8013801	0.0135239\\
0.8014801	0.01343504\\
0.8015802	0.01334673\\
0.8016802	0.01325896\\
0.8017802	0.01317172\\
0.8018802	0.01308501\\
0.8019802	0.01299881\\
0.8020802	0.01291315\\
0.8021802	0.012828\\
0.8022802	0.01274338\\
0.8023802	0.01265926\\
0.8024802	0.01257566\\
0.8025803	0.01249255\\
0.8026803	0.01240998\\
0.8027803	0.01232789\\
0.8028803	0.01224631\\
0.8029803	0.01216523\\
0.8030803	0.01208464\\
0.8031803	0.01200453\\
0.8032803	0.01192493\\
0.8033803	0.01184581\\
0.8034803	0.01176718\\
0.8035804	0.01168902\\
0.8036804	0.01161135\\
0.8037804	0.01153415\\
0.8038804	0.01145743\\
0.8039804	0.01138117\\
0.8040804	0.0113054\\
0.8041804	0.01123009\\
0.8042804	0.01115523\\
0.8043804	0.01108083\\
0.8044804	0.01100689\\
0.8045805	0.0109334\\
0.8046805	0.01086037\\
0.8047805	0.0107878\\
0.8048805	0.01071568\\
0.8049805	0.010644\\
0.8050805	0.01057276\\
0.8051805	0.01050197\\
0.8052805	0.0104316\\
0.8053805	0.01036167\\
0.8054805	0.01029218\\
0.8055806	0.01022312\\
0.8056806	0.01015448\\
0.8057806	0.01008628\\
0.8058806	0.01001849\\
0.8059806	0.009951135\\
0.8060806	0.009884192\\
0.8061806	0.009817676\\
0.8062806	0.009751551\\
0.8063806	0.009685849\\
0.8064806	0.009620559\\
0.8065807	0.009555677\\
0.8066807	0.009491195\\
0.8067807	0.009427117\\
0.8068807	0.009363444\\
0.8069807	0.009300164\\
0.8070807	0.009237287\\
0.8071807	0.009174805\\
0.8072807	0.00911271\\
0.8073807	0.009050996\\
0.8074807	0.008989684\\
0.8075808	0.008928745\\
0.8076808	0.008868189\\
0.8077808	0.008808015\\
0.8078808	0.008748225\\
0.8079808	0.008688805\\
0.8080808	0.008629761\\
0.8081808	0.008571079\\
0.8082808	0.008512768\\
0.8083808	0.008454827\\
0.8084808	0.008397262\\
0.8085809	0.008340061\\
0.8086809	0.008283211\\
0.8087809	0.008226725\\
0.8088809	0.008170594\\
0.8089809	0.008114826\\
0.8090809	0.008059397\\
0.8091809	0.008004331\\
0.8092809	0.007949604\\
0.8093809	0.007895235\\
0.8094809	0.0078412\\
0.809581	0.007787519\\
0.809681	0.007734168\\
0.809781	0.007681158\\
0.809881	0.00762848\\
0.809981	0.00757615\\
0.810081	0.007524146\\
0.810181	0.007472471\\
0.810281	0.00742113\\
0.810381	0.007370128\\
0.810481	0.00731944\\
0.8105811	0.007269068\\
0.8106811	0.007219028\\
0.8107811	0.007169294\\
0.8108811	0.007119891\\
0.8109811	0.007070804\\
0.8110811	0.007022037\\
0.8111811	0.006973564\\
0.8112811	0.006925411\\
0.8113811	0.006877568\\
0.8114811	0.006830034\\
0.8115812	0.006782801\\
0.8116812	0.006735871\\
0.8117812	0.006689251\\
0.8118812	0.006642922\\
0.8119812	0.006596898\\
0.8120812	0.006551162\\
0.8121812	0.006505727\\
0.8122812	0.006460573\\
0.8123812	0.006415725\\
0.8124812	0.006371159\\
0.8125813	0.006326882\\
0.8126813	0.006282885\\
0.8127813	0.006239179\\
0.8128813	0.006195758\\
0.8129813	0.00615262\\
0.8130813	0.006109764\\
0.8131813	0.006067178\\
0.8132813	0.006024877\\
0.8133813	0.005982839\\
0.8134813	0.00594108\\
0.8135814	0.005899587\\
0.8136814	0.005858364\\
0.8137814	0.005817405\\
0.8138814	0.00577672\\
0.8139814	0.005736302\\
0.8140814	0.005696142\\
0.8141814	0.005656246\\
0.8142814	0.005616611\\
0.8143814	0.005577236\\
0.8144814	0.005538107\\
0.8145815	0.005499244\\
0.8146815	0.005460628\\
0.8147815	0.005422271\\
0.8148815	0.005384158\\
0.8149815	0.005346307\\
0.8150815	0.005308697\\
0.8151815	0.005271326\\
0.8152815	0.005234205\\
0.8153815	0.005197331\\
0.8154815	0.005160703\\
0.8155816	0.005124311\\
0.8156816	0.005088158\\
0.8157816	0.005052236\\
0.8158816	0.005016556\\
0.8159816	0.004981107\\
0.8160816	0.004945897\\
0.8161816	0.004910918\\
0.8162816	0.004876166\\
0.8163816	0.004841646\\
0.8164816	0.004807353\\
0.8165817	0.004773297\\
0.8166817	0.004739449\\
0.8167817	0.004705835\\
0.8168817	0.004672441\\
0.8169817	0.00463927\\
0.8170817	0.00460631\\
0.8171817	0.004573573\\
0.8172817	0.004541054\\
0.8173817	0.004508747\\
0.8174817	0.004476661\\
0.8175818	0.004444785\\
0.8176818	0.004413124\\
0.8177818	0.004381662\\
0.8178818	0.004350418\\
0.8179818	0.004319387\\
0.8180818	0.004288559\\
0.8181818	0.004257935\\
0.8182818	0.004227511\\
0.8183818	0.004197298\\
0.8184818	0.004167277\\
0.8185819	0.004137465\\
0.8186819	0.004107851\\
0.8187819	0.004078436\\
0.8188819	0.004049212\\
0.8189819	0.004020185\\
0.8190819	0.003991354\\
0.8191819	0.003962713\\
0.8192819	0.003934265\\
0.8193819	0.003906004\\
0.8194819	0.003877947\\
0.819582	0.003850067\\
0.819682	0.003822381\\
0.819782	0.003794877\\
0.819882	0.003767553\\
0.819982	0.003740413\\
0.820082	0.003713459\\
0.820182	0.003686693\\
0.820282	0.003660102\\
0.820382	0.00363369\\
0.820482	0.003607458\\
0.8205821	0.003581404\\
0.8206821	0.003555528\\
0.8207821	0.003529818\\
0.8208821	0.003504285\\
0.8209821	0.003478918\\
0.8210821	0.003453732\\
0.8211821	0.00342871\\
0.8212821	0.003403865\\
0.8213821	0.003379186\\
0.8214821	0.003354675\\
0.8215822	0.00333033\\
0.8216822	0.00330615\\
0.8217822	0.003282138\\
0.8218822	0.003258281\\
0.8219822	0.003234589\\
0.8220822	0.003211063\\
0.8221822	0.003187698\\
0.8222822	0.00316449\\
0.8223822	0.003141435\\
0.8224822	0.003118541\\
0.8225823	0.003095799\\
0.8226823	0.003073219\\
0.8227823	0.00305079\\
0.8228823	0.003028521\\
0.8229823	0.003006393\\
0.8230823	0.002984414\\
0.8231823	0.002962595\\
0.8232823	0.00294092\\
0.8233823	0.002919399\\
0.8234823	0.002898017\\
0.8235824	0.002876793\\
0.8236824	0.002855708\\
0.8237824	0.002834769\\
0.8238824	0.002813974\\
0.8239824	0.00279332\\
0.8240824	0.002772814\\
0.8241824	0.00275244\\
0.8242824	0.002732214\\
0.8243824	0.002712125\\
0.8244824	0.002692172\\
0.8245825	0.002672358\\
0.8246825	0.002652678\\
0.8247825	0.002633141\\
0.8248825	0.002613735\\
0.8249825	0.002594464\\
0.8250825	0.002575323\\
0.8251825	0.002556318\\
0.8252825	0.002537445\\
0.8253825	0.0025187\\
0.8254825	0.002500091\\
0.8255826	0.002481603\\
0.8256826	0.002463244\\
0.8257826	0.002445008\\
0.8258826	0.002426903\\
0.8259826	0.002408927\\
0.8260826	0.002391074\\
0.8261826	0.002373348\\
0.8262826	0.002355741\\
0.8263826	0.002338266\\
0.8264826	0.002320904\\
0.8265827	0.002303664\\
0.8266827	0.002286545\\
0.8267827	0.002269542\\
0.8268827	0.002252657\\
0.8269827	0.002235893\\
0.8270827	0.002219249\\
0.8271827	0.00220272\\
0.8272827	0.002186301\\
0.8273827	0.00217\\
0.8274827	0.002153815\\
0.8275828	0.002137747\\
0.8276828	0.002121786\\
0.8277828	0.002105938\\
0.8278828	0.002090202\\
0.8279828	0.002074573\\
0.8280828	0.002059061\\
0.8281828	0.002043652\\
0.8282828	0.002028356\\
0.8283828	0.002013164\\
0.8284828	0.001998077\\
0.8285829	0.0019831\\
0.8286829	0.001968225\\
0.8287829	0.00195346\\
0.8288829	0.00193879\\
0.8289829	0.001924229\\
0.8290829	0.001909772\\
0.8291829	0.001895416\\
0.8292829	0.001881165\\
0.8293829	0.001867008\\
0.8294829	0.001852959\\
0.829583	0.001839006\\
0.829683	0.00182515\\
0.829783	0.001811395\\
0.829883	0.001797735\\
0.829983	0.001784172\\
0.830083	0.001770697\\
0.830183	0.001757325\\
0.830283	0.001744053\\
0.830383	0.001730873\\
0.830483	0.001717787\\
0.8305831	0.001704791\\
0.8306831	0.001691891\\
0.8307831	0.001679078\\
0.8308831	0.001666355\\
0.8309831	0.00165373\\
0.8310831	0.001641192\\
0.8311831	0.001628745\\
0.8312831	0.001616385\\
0.8313831	0.001604112\\
0.8314831	0.001591932\\
0.8315832	0.001579835\\
0.8316832	0.001567823\\
0.8317832	0.001555897\\
0.8318832	0.001544059\\
0.8319832	0.001532306\\
0.8320832	0.001520635\\
0.8321832	0.001509048\\
0.8322832	0.001497543\\
0.8323832	0.00148612\\
0.8324832	0.001474781\\
0.8325833	0.001463522\\
0.8326833	0.001452348\\
0.8327833	0.001441249\\
0.8328833	0.001430233\\
0.8329833	0.001419295\\
0.8330833	0.001408439\\
0.8331833	0.001397666\\
0.8332833	0.001386963\\
0.8333833	0.001376339\\
0.8334833	0.001365789\\
0.8335834	0.001355318\\
0.8336834	0.001344921\\
0.8337834	0.001334599\\
0.8338834	0.001324352\\
0.8339834	0.00131418\\
0.8340834	0.001304081\\
0.8341834	0.001294056\\
0.8342834	0.001284103\\
0.8343834	0.001274225\\
0.8344834	0.001264414\\
0.8345835	0.001254672\\
0.8346835	0.001245008\\
0.8347835	0.001235412\\
0.8348835	0.001225886\\
0.8349835	0.001216425\\
0.8350835	0.001207035\\
0.8351835	0.001197716\\
0.8352835	0.00118846\\
0.8353835	0.001179277\\
0.8354835	0.001170159\\
0.8355836	0.00116111\\
0.8356836	0.001152125\\
0.8357836	0.001143201\\
0.8358836	0.001134347\\
0.8359836	0.001125555\\
0.8360836	0.001116829\\
0.8361836	0.001108165\\
0.8362836	0.001099564\\
0.8363836	0.00109103\\
0.8364836	0.001082557\\
0.8365837	0.001074146\\
0.8366837	0.001065799\\
0.8367837	0.00105751\\
0.8368837	0.001049285\\
0.8369837	0.001041116\\
0.8370837	0.001033007\\
0.8371837	0.001024962\\
0.8372837	0.001016972\\
0.8373837	0.001009042\\
0.8374837	0.00100117\\
0.8375838	0.0009933561\\
0.8376838	0.0009856052\\
0.8377838	0.0009779069\\
0.8378838	0.0009702656\\
0.8379838	0.0009626771\\
0.8380838	0.0009551466\\
0.8381838	0.0009476741\\
0.8382838	0.0009402556\\
0.8383838	0.0009328935\\
0.8384838	0.0009255877\\
0.8385839	0.0009183306\\
0.8386839	0.0009111276\\
0.8387839	0.0009039765\\
0.8388839	0.0008968837\\
0.8389839	0.0008898428\\
0.8390839	0.0008828501\\
0.8391839	0.0008759129\\
0.8392839	0.0008690267\\
0.8393839	0.0008621928\\
0.8394839	0.0008554103\\
0.839584	0.0008486727\\
0.839684	0.0008419904\\
0.839784	0.000835357\\
0.839884	0.000828771\\
0.839984	0.0008222367\\
0.840084	0.0008157494\\
0.840184	0.0008093126\\
0.840284	0.000802922\\
0.840384	0.0007965766\\
0.840484	0.0007902816\\
0.8405841	0.0007840326\\
0.8406841	0.0007778333\\
0.8407841	0.0007716788\\
0.8408841	0.0007655672\\
0.8409841	0.0007595051\\
0.8410841	0.0007534856\\
0.8411841	0.0007475133\\
0.8412841	0.0007415853\\
0.8413841	0.0007357004\\
0.8414841	0.0007298621\\
0.8415842	0.0007240663\\
0.8416842	0.0007183124\\
0.8417842	0.0007126037\\
0.8418842	0.0007069375\\
0.8419842	0.0007013154\\
0.8420842	0.0006957343\\
0.8421842	0.0006901927\\
0.8422842	0.0006846975\\
0.8423842	0.000679243\\
0.8424842	0.0006738278\\
0.8425843	0.0006684555\\
0.8426843	0.0006631226\\
0.8427843	0.0006578314\\
0.8428843	0.0006525798\\
0.8429843	0.0006473659\\
0.8430843	0.0006421949\\
0.8431843	0.0006370606\\
0.8432843	0.0006319622\\
0.8433843	0.0006269023\\
0.8434843	0.00062188\\
0.8435844	0.0006168991\\
0.8436844	0.0006119579\\
0.8437844	0.0006070536\\
0.8438844	0.0006021863\\
0.8439844	0.0005973549\\
0.8440844	0.000592561\\
0.8441844	0.000587805\\
0.8442844	0.0005830826\\
0.8443844	0.0005783966\\
0.8444844	0.000573748\\
0.8445845	0.0005691305\\
0.8446845	0.0005645519\\
0.8447845	0.000560006\\
0.8448845	0.0005554959\\
0.8449845	0.0005510224\\
0.8450845	0.0005465793\\
0.8451845	0.0005421723\\
0.8452845	0.0005377972\\
0.8453845	0.0005334579\\
0.8454845	0.0005291524\\
0.8455846	0.0005248763\\
0.8456846	0.000520634\\
0.8457846	0.0005164258\\
0.8458846	0.000512249\\
0.8459846	0.0005081064\\
0.8460846	0.0005039931\\
0.8461846	0.0004999118\\
0.8462846	0.0004958643\\
0.8463846	0.0004918439\\
0.8464846	0.0004878549\\
0.8465847	0.0004838976\\
0.8466847	0.000479972\\
0.8467847	0.0004760776\\
0.8468847	0.0004722112\\
0.8469847	0.0004683723\\
0.8470847	0.0004645642\\
0.8471847	0.0004607866\\
0.8472847	0.0004570358\\
0.8473847	0.0004533167\\
0.8474847	0.0004496244\\
0.8475848	0.0004459632\\
0.8476848	0.0004423302\\
0.8477848	0.0004387232\\
0.8478848	0.0004351451\\
0.8479848	0.000431595\\
0.8480848	0.0004280714\\
0.8481848	0.000424577\\
0.8482848	0.000421107\\
0.8483848	0.0004176642\\
0.8484848	0.0004142518\\
0.8485849	0.0004108642\\
0.8486849	0.0004075028\\
0.8487849	0.0004041667\\
0.8488849	0.0004008547\\
0.8489849	0.0003975699\\
0.8490849	0.0003943113\\
0.8491849	0.000391077\\
0.8492849	0.0003878699\\
0.8493849	0.000384687\\
0.8494849	0.00038153\\
0.849585	0.0003783961\\
0.849685	0.0003752849\\
0.849785	0.0003722\\
0.849885	0.0003691403\\
0.849985	0.0003661026\\
0.850085	0.0003630887\\
0.850185	0.000360097\\
0.850285	0.0003571295\\
0.850385	0.0003541871\\
0.850485	0.0003512673\\
0.8505851	0.0003483687\\
0.8506851	0.0003454956\\
0.8507851	0.0003426441\\
0.8508851	0.0003398153\\
0.8509851	0.0003370084\\
0.8510851	0.0003342215\\
0.8511851	0.0003314584\\
0.8512851	0.0003287171\\
0.8513851	0.0003259947\\
0.8514851	0.0003232959\\
0.8515852	0.0003206195\\
0.8516852	0.0003179629\\
0.8517852	0.0003153302\\
0.8518852	0.0003127152\\
0.8519852	0.0003101203\\
0.8520852	0.0003075463\\
0.8521852	0.0003049917\\
0.8522852	0.000302458\\
0.8523852	0.0002999447\\
0.8524852	0.0002974503\\
0.8525853	0.0002949775\\
0.8526853	0.0002925241\\
0.8527853	0.0002900882\\
0.8528853	0.0002876723\\
0.8529853	0.000285276\\
0.8530853	0.0002828986\\
0.8531853	0.0002805418\\
0.8532853	0.0002782021\\
0.8533853	0.0002758795\\
0.8534853	0.0002735772\\
0.8535854	0.0002712923\\
0.8536854	0.0002690261\\
0.8537854	0.0002667786\\
0.8538854	0.0002645466\\
0.8539854	0.0002623339\\
0.8540854	0.0002601397\\
0.8541854	0.0002579617\\
0.8542854	0.0002558014\\
0.8543854	0.0002536589\\
0.8544854	0.0002515331\\
0.8545855	0.0002494256\\
0.8546855	0.0002473324\\
0.8547855	0.0002452544\\
0.8548855	0.0002431963\\
0.8549855	0.0002411545\\
0.8550855	0.0002391283\\
0.8551855	0.0002371185\\
0.8552855	0.0002351241\\
0.8553855	0.0002331454\\
0.8554855	0.0002311853\\
0.8555856	0.0002292394\\
0.8556856	0.0002273092\\
0.8557856	0.0002253949\\
0.8558856	0.0002234951\\
0.8559856	0.0002216109\\
0.8560856	0.00021974\\
0.8561856	0.0002178839\\
0.8562856	0.0002160434\\
0.8563856	0.000214219\\
0.8564856	0.0002124082\\
0.8565857	0.000210613\\
0.8566857	0.0002088318\\
0.8567857	0.0002070645\\
0.8568857	0.0002053131\\
0.8569857	0.0002035756\\
0.8570857	0.0002018509\\
0.8571857	0.0002001406\\
0.8572857	0.0001984439\\
0.8573857	0.0001967609\\
0.8574857	0.0001950935\\
0.8575858	0.000193438\\
0.8576858	0.0001917959\\
0.8577858	0.0001901676\\
0.8578858	0.0001885518\\
0.8579858	0.0001869481\\
0.8580858	0.0001853574\\
0.8581858	0.0001837789\\
0.8582858	0.0001822148\\
0.8583858	0.0001806643\\
0.8584858	0.0001791258\\
0.8585859	0.0001775985\\
0.8586859	0.0001760844\\
0.8587859	0.0001745827\\
0.8588859	0.000173093\\
0.8589859	0.0001716168\\
0.8590859	0.000170151\\
0.8591859	0.0001686973\\
0.8592859	0.0001672574\\
0.8593859	0.0001658272\\
0.8594859	0.0001644076\\
0.859586	0.0001630009\\
0.859686	0.0001616054\\
0.859786	0.0001602216\\
0.859886	0.0001588507\\
0.859986	0.0001574886\\
0.860086	0.0001561383\\
0.860186	0.0001547999\\
0.860286	0.000153472\\
0.860386	0.0001521558\\
0.860486	0.0001508497\\
0.8605861	0.0001495534\\
0.8606861	0.0001482677\\
0.8607861	0.0001469934\\
0.8608861	0.0001457292\\
0.8609861	0.0001444758\\
0.8610861	0.0001432327\\
0.8611861	0.000141999\\
0.8612861	0.0001407753\\
0.8613861	0.0001395631\\
0.8614861	0.000138359\\
0.8615862	0.0001371649\\
0.8616862	0.0001359824\\
0.8617862	0.0001348095\\
0.8618862	0.0001336461\\
0.8619862	0.0001324921\\
0.8620862	0.000131346\\
0.8621862	0.0001302096\\
0.8622862	0.0001290842\\
0.8623862	0.0001279675\\
0.8624862	0.0001268595\\
0.8625863	0.0001257613\\
0.8626863	0.0001246724\\
0.8627863	0.0001235921\\
0.8628863	0.0001225225\\
0.8629863	0.0001214598\\
0.8630863	0.0001204057\\
0.8631863	0.0001193609\\
0.8632863	0.000118325\\
0.8633863	0.0001172966\\
0.8634863	0.0001162768\\
0.8635864	0.0001152657\\
0.8636864	0.0001142628\\
0.8637864	0.0001132698\\
0.8638864	0.000112285\\
0.8639864	0.0001113074\\
0.8640864	0.0001103386\\
0.8641864	0.0001093775\\
0.8642864	0.0001084233\\
0.8643864	0.0001074787\\
0.8644864	0.0001065413\\
0.8645865	0.0001056112\\
0.8646865	0.0001046891\\
0.8647865	0.0001037753\\
0.8648865	0.000102868\\
0.8649865	0.0001019683\\
0.8650865	0.0001010768\\
0.8651865	0.0001001921\\
0.8652865	9.931561e-05\\
0.8653865	9.844765e-05\\
0.8654865	9.758545e-05\\
0.8655866	9.67297e-05\\
0.8656866	9.588199e-05\\
0.8657866	9.504061e-05\\
0.8658866	9.420685e-05\\
0.8659866	9.338125e-05\\
0.8660866	9.256181e-05\\
0.8661866	9.174884e-05\\
0.8662866	9.094319e-05\\
0.8663866	9.01449e-05\\
0.8664866	8.935227e-05\\
0.8665867	8.856699e-05\\
0.8666867	8.778765e-05\\
0.8667867	8.701449e-05\\
0.8668867	8.624927e-05\\
0.8669867	8.548963e-05\\
0.8670867	8.47355e-05\\
0.8671867	8.398812e-05\\
0.8672867	8.324834e-05\\
0.8673867	8.251374e-05\\
0.8674867	8.178575e-05\\
0.8675868	8.106367e-05\\
0.8676868	8.034656e-05\\
0.8677868	7.963549e-05\\
0.8678868	7.893186e-05\\
0.8679868	7.823392e-05\\
0.8680868	7.754218e-05\\
0.8681868	7.685641e-05\\
0.8682868	7.61766e-05\\
0.8683868	7.550174e-05\\
0.8684868	7.483326e-05\\
0.8685869	7.417048e-05\\
0.8686869	7.35124e-05\\
0.8687869	7.286043e-05\\
0.8688869	7.221386e-05\\
0.8689869	7.157331e-05\\
0.8690869	7.093787e-05\\
0.8691869	7.030826e-05\\
0.8692869	6.968269e-05\\
0.8693869	6.906245e-05\\
0.8694869	6.844863e-05\\
0.869587	6.784012e-05\\
0.869687	6.723615e-05\\
0.869787	6.663757e-05\\
0.869887	6.604365e-05\\
0.869987	6.545488e-05\\
0.870087	6.487171e-05\\
0.870187	6.429357e-05\\
0.870287	6.371996e-05\\
0.870387	6.315092e-05\\
0.870487	6.25878e-05\\
0.8705871	6.202927e-05\\
0.8706871	6.147496e-05\\
0.8707871	6.09252e-05\\
0.8708871	6.038038e-05\\
0.8709871	5.983952e-05\\
0.8710871	5.930466e-05\\
0.8711871	5.877425e-05\\
0.8712871	5.824712e-05\\
0.8713871	5.772469e-05\\
0.8714871	5.720725e-05\\
0.8715872	5.66937e-05\\
0.8716872	5.618556e-05\\
0.8717872	5.56826e-05\\
0.8718872	5.518313e-05\\
0.8719872	5.468854e-05\\
0.8720872	5.419758e-05\\
0.8721872	5.371059e-05\\
0.8722872	5.322745e-05\\
0.8723872	5.274859e-05\\
0.8724872	5.22735e-05\\
0.8725873	5.180257e-05\\
0.8726873	5.133612e-05\\
0.8727873	5.087434e-05\\
0.8728873	5.041603e-05\\
0.8729873	4.996143e-05\\
0.8730873	4.951142e-05\\
0.8731873	4.906495e-05\\
0.8732873	4.862205e-05\\
0.8733873	4.818383e-05\\
0.8734873	4.774896e-05\\
0.8735874	4.731675e-05\\
0.8736874	4.688881e-05\\
0.8737874	4.646494e-05\\
0.8738874	4.604498e-05\\
0.8739874	4.562819e-05\\
0.8740874	4.521563e-05\\
0.8741874	4.480601e-05\\
0.8742874	4.439944e-05\\
0.8743874	4.39975e-05\\
0.8744874	4.359883e-05\\
0.8745875	4.320311e-05\\
0.8746875	4.281092e-05\\
0.8747875	4.242267e-05\\
0.8748875	4.203798e-05\\
0.8749875	4.165653e-05\\
0.8750875	4.127873e-05\\
0.8751875	4.09036e-05\\
0.8752875	4.053128e-05\\
0.8753875	4.016268e-05\\
0.8754875	3.979745e-05\\
0.8755876	3.943563e-05\\
0.8756876	3.907646e-05\\
0.8757876	3.872089e-05\\
0.8758876	3.836806e-05\\
0.8759876	3.80178e-05\\
0.8760876	3.767112e-05\\
0.8761876	3.732744e-05\\
0.8762876	3.698638e-05\\
0.8763876	3.664836e-05\\
0.8764876	3.631389e-05\\
0.8765877	3.598216e-05\\
0.8766877	3.565331e-05\\
0.8767877	3.532742e-05\\
0.8768877	3.500431e-05\\
0.8769877	3.468345e-05\\
0.8770877	3.436593e-05\\
0.8771877	3.405162e-05\\
0.8772877	3.374011e-05\\
0.8773877	3.343093e-05\\
0.8774877	3.312464e-05\\
0.8775878	3.282067e-05\\
0.8776878	3.251945e-05\\
0.8777878	3.222146e-05\\
0.8778878	3.19263e-05\\
0.8779878	3.163339e-05\\
0.8780878	3.134245e-05\\
0.8781878	3.105459e-05\\
0.8782878	3.076884e-05\\
0.8783878	3.048523e-05\\
0.8784878	3.020446e-05\\
0.8785879	2.992671e-05\\
0.8786879	2.965055e-05\\
0.8787879	2.937751e-05\\
0.8788879	2.91074e-05\\
0.8789879	2.883974e-05\\
0.8790879	2.857359e-05\\
0.8791879	2.831005e-05\\
0.8792879	2.80489e-05\\
0.8793879	2.778927e-05\\
0.8794879	2.753259e-05\\
0.879588	2.727866e-05\\
0.879688	2.702657e-05\\
0.879788	2.677642e-05\\
0.879888	2.652877e-05\\
0.879988	2.628347e-05\\
0.880088	2.604025e-05\\
0.880188	2.579895e-05\\
0.880288	2.556023e-05\\
0.880388	2.532338e-05\\
0.880488	2.508852e-05\\
0.8805881	2.485562e-05\\
0.8806881	2.462559e-05\\
0.8807881	2.439697e-05\\
0.8808881	2.417026e-05\\
0.8809881	2.394579e-05\\
0.8810881	2.372333e-05\\
0.8811881	2.350284e-05\\
0.8812881	2.328459e-05\\
0.8813881	2.306868e-05\\
0.8814881	2.285414e-05\\
0.8815882	2.26413e-05\\
0.8816882	2.24305e-05\\
0.8817882	2.222169e-05\\
0.8818882	2.20147e-05\\
0.8819882	2.180967e-05\\
0.8820882	2.160647e-05\\
0.8821882	2.140471e-05\\
0.8822882	2.120438e-05\\
0.8823882	2.100621e-05\\
0.8824882	2.081002e-05\\
0.8825883	2.061567e-05\\
0.8826883	2.042253e-05\\
0.8827883	2.023175e-05\\
0.8828883	2.004312e-05\\
0.8829883	1.985561e-05\\
0.8830883	1.967011e-05\\
0.8831883	1.948629e-05\\
0.8832883	1.930373e-05\\
0.8833883	1.912232e-05\\
0.8834883	1.894305e-05\\
0.8835884	1.876567e-05\\
0.8836884	1.858974e-05\\
0.8837884	1.841545e-05\\
0.8838884	1.824285e-05\\
0.8839884	1.807161e-05\\
0.8840884	1.790166e-05\\
0.8841884	1.773344e-05\\
0.8842884	1.756699e-05\\
0.8843884	1.740209e-05\\
0.8844884	1.723836e-05\\
0.8845885	1.707604e-05\\
0.8846885	1.691536e-05\\
0.8847885	1.675621e-05\\
0.8848885	1.659803e-05\\
0.8849885	1.644186e-05\\
0.8850885	1.628693e-05\\
0.8851885	1.613341e-05\\
0.8852885	1.598081e-05\\
0.8853885	1.582994e-05\\
0.8854885	1.568088e-05\\
0.8855886	1.553254e-05\\
0.8856886	1.538554e-05\\
0.8857886	1.524012e-05\\
0.8858886	1.509566e-05\\
0.8859886	1.495259e-05\\
0.8860886	1.48114e-05\\
0.8861886	1.467187e-05\\
0.8862886	1.453328e-05\\
0.8863886	1.439559e-05\\
0.8864886	1.425943e-05\\
0.8865887	1.412443e-05\\
0.8866887	1.399065e-05\\
0.8867887	1.385778e-05\\
0.8868887	1.372642e-05\\
0.8869887	1.359606e-05\\
0.8870887	1.346688e-05\\
0.8871887	1.333874e-05\\
0.8872887	1.321217e-05\\
0.8873887	1.308688e-05\\
0.8874887	1.29622e-05\\
0.8875888	1.283847e-05\\
0.8876888	1.271636e-05\\
0.8877888	1.259518e-05\\
0.8878888	1.247512e-05\\
0.8879888	1.235646e-05\\
0.8880888	1.223926e-05\\
0.8881888	1.212278e-05\\
0.8882888	1.200706e-05\\
0.8883888	1.189254e-05\\
0.8884888	1.177904e-05\\
0.8885889	1.166649e-05\\
0.8886889	1.155492e-05\\
0.8887889	1.144475e-05\\
0.8888889	1.133525e-05\\
0.8889889	1.122671e-05\\
0.8890889	1.111917e-05\\
0.8891889	1.101279e-05\\
0.8892889	1.090763e-05\\
0.8893889	1.080334e-05\\
0.8894889	1.069978e-05\\
0.889589	1.059743e-05\\
0.889689	1.049604e-05\\
0.889789	1.039534e-05\\
0.889889	1.029564e-05\\
0.889989	1.019712e-05\\
0.890089	1.00993e-05\\
0.890189	1.000195e-05\\
0.890289	9.905705e-06\\
0.890389	9.81071e-06\\
0.890489	9.716705e-06\\
0.8905891	9.62347e-06\\
0.8906891	9.531057e-06\\
0.8907891	9.439577e-06\\
0.8908891	9.34866e-06\\
0.8909891	9.258375e-06\\
0.8910891	9.169085e-06\\
0.8911891	9.080753e-06\\
0.8912891	8.993069e-06\\
0.8913891	8.906105e-06\\
0.8914891	8.820032e-06\\
0.8915892	8.734908e-06\\
0.8916892	8.650776e-06\\
0.8917892	8.567144e-06\\
0.8918892	8.484505e-06\\
0.8919892	8.402772e-06\\
0.8920892	8.321606e-06\\
0.8921892	8.241019e-06\\
0.8922892	8.161321e-06\\
0.8923892	8.0825e-06\\
0.8924892	8.004452e-06\\
0.8925893	7.926959e-06\\
0.8926893	7.850343e-06\\
0.8927893	7.774436e-06\\
0.8928893	7.698949e-06\\
0.8929893	7.624156e-06\\
0.8930893	7.550103e-06\\
0.8931893	7.477159e-06\\
0.8932893	7.40469e-06\\
0.8933893	7.332736e-06\\
0.8934893	7.261329e-06\\
0.8935894	7.190835e-06\\
0.8936894	7.120726e-06\\
0.8937894	7.051321e-06\\
0.8938894	6.982742e-06\\
0.8939894	6.914876e-06\\
0.8940894	6.847687e-06\\
0.8941894	6.780923e-06\\
0.8942894	6.714854e-06\\
0.8943894	6.64948e-06\\
0.8944894	6.584877e-06\\
0.8945895	6.52059e-06\\
0.8946895	6.456816e-06\\
0.8947895	6.393698e-06\\
0.8948895	6.331329e-06\\
0.8949895	6.269337e-06\\
0.8950895	6.208152e-06\\
0.8951895	6.147838e-06\\
0.8952895	6.088074e-06\\
0.8953895	6.028515e-06\\
0.8954895	5.969434e-06\\
0.8955896	5.911027e-06\\
0.8956896	5.85319e-06\\
0.8957896	5.795742e-06\\
0.8958896	5.738924e-06\\
0.8959896	5.682796e-06\\
0.8960896	5.627174e-06\\
0.8961896	5.572162e-06\\
0.8962896	5.517343e-06\\
0.8963896	5.463174e-06\\
0.8964896	5.409693e-06\\
0.8965897	5.356632e-06\\
0.8966897	5.303947e-06\\
0.8967897	5.251639e-06\\
0.8968897	5.199951e-06\\
0.8969897	5.148752e-06\\
0.8970897	5.097933e-06\\
0.8971897	5.047722e-06\\
0.8972897	4.998211e-06\\
0.8973897	4.949253e-06\\
0.8974897	4.900441e-06\\
0.8975898	4.852086e-06\\
0.8976898	4.804239e-06\\
0.8977898	4.757017e-06\\
0.8978898	4.710173e-06\\
0.8979898	4.663748e-06\\
0.8980898	4.617851e-06\\
0.8981898	4.572322e-06\\
0.8982898	4.52701e-06\\
0.8983898	4.482039e-06\\
0.8984898	4.437585e-06\\
0.8985899	4.393738e-06\\
0.8986899	4.350235e-06\\
0.8987899	4.307328e-06\\
0.8988899	4.264659e-06\\
0.8989899	4.222501e-06\\
0.8990899	4.180966e-06\\
0.8991899	4.139643e-06\\
0.8992899	4.098732e-06\\
0.8993899	4.058174e-06\\
0.8994899	4.017989e-06\\
0.89959	3.978052e-06\\
0.89969	3.938367e-06\\
0.89979	3.899037e-06\\
0.89989	3.860333e-06\\
0.89999	3.822048e-06\\
0.90009	3.784014e-06\\
0.90019	3.74628e-06\\
0.90029	3.709112e-06\\
0.90039	3.672178e-06\\
0.90049	3.635504e-06\\
0.9005901	3.599071e-06\\
0.9006901	3.563256e-06\\
0.9007901	3.527885e-06\\
0.9008901	3.492886e-06\\
0.9009901	3.458119e-06\\
0.9010901	3.423669e-06\\
0.9011901	3.389682e-06\\
0.9012901	3.355905e-06\\
0.9013901	3.322427e-06\\
0.9014901	3.289303e-06\\
0.9015902	3.256515e-06\\
0.9016902	3.224129e-06\\
0.9017902	3.191884e-06\\
0.9018902	3.159777e-06\\
0.9019902	3.128031e-06\\
0.9020902	3.096731e-06\\
0.9021902	3.065867e-06\\
0.9022902	3.035117e-06\\
0.9023902	3.004611e-06\\
0.9024902	2.974468e-06\\
0.9025903	2.94472e-06\\
0.9026903	2.915073e-06\\
0.9027903	2.885738e-06\\
0.9028903	2.8568e-06\\
0.9029903	2.828269e-06\\
0.9030903	2.799974e-06\\
0.9031903	2.771809e-06\\
0.9032903	2.743837e-06\\
0.9033903	2.716241e-06\\
0.9034903	2.688876e-06\\
0.9035904	2.661883e-06\\
0.9036904	2.635031e-06\\
0.9037904	2.608594e-06\\
0.9038904	2.58242e-06\\
0.9039904	2.556532e-06\\
0.9040904	2.530867e-06\\
0.9041904	2.505307e-06\\
0.9042904	2.480043e-06\\
0.9043904	2.455104e-06\\
0.9044904	2.430427e-06\\
0.9045905	2.405762e-06\\
0.9046905	2.381341e-06\\
0.9047905	2.357254e-06\\
0.9048905	2.333503e-06\\
0.9049905	2.309886e-06\\
0.9050905	2.286459e-06\\
0.9051905	2.263413e-06\\
0.9052905	2.240748e-06\\
0.9053905	2.218131e-06\\
0.9054905	2.195645e-06\\
0.9055906	2.173279e-06\\
0.9056906	2.151205e-06\\
0.9057906	2.129352e-06\\
0.9058906	2.107772e-06\\
0.9059906	2.086351e-06\\
0.9060906	2.06519e-06\\
0.9061906	2.044307e-06\\
0.9062906	2.023568e-06\\
0.9063906	2.002967e-06\\
0.9064906	1.982529e-06\\
0.9065907	1.962302e-06\\
0.9066907	1.942479e-06\\
0.9067907	1.922863e-06\\
0.9068907	1.903356e-06\\
0.9069907	1.883949e-06\\
0.9070907	1.864848e-06\\
0.9071907	1.845917e-06\\
0.9072907	1.827153e-06\\
0.9073907	1.808461e-06\\
0.9074907	1.789983e-06\\
0.9075908	1.771809e-06\\
0.9076908	1.753796e-06\\
0.9077908	1.735869e-06\\
0.9078908	1.718036e-06\\
0.9079908	1.700497e-06\\
0.9080908	1.683154e-06\\
0.9081908	1.666053e-06\\
0.9082908	1.649072e-06\\
0.9083908	1.632265e-06\\
0.9084908	1.61557e-06\\
0.9085909	1.599059e-06\\
0.9086909	1.58274e-06\\
0.9087909	1.566437e-06\\
0.9088909	1.550209e-06\\
0.9089909	1.534247e-06\\
0.9090909	1.518535e-06\\
0.9091909	1.503039e-06\\
0.9092909	1.487658e-06\\
0.9093909	1.472396e-06\\
0.9094909	1.457262e-06\\
0.909591	1.442372e-06\\
0.909691	1.427597e-06\\
0.909791	1.412874e-06\\
0.909891	1.398357e-06\\
0.909991	1.383999e-06\\
0.910091	1.369882e-06\\
0.910191	1.355899e-06\\
0.910291	1.341869e-06\\
0.910391	1.328077e-06\\
0.910491	1.314473e-06\\
0.9105911	1.301073e-06\\
0.9106911	1.287729e-06\\
0.9107911	1.274529e-06\\
0.9108911	1.261378e-06\\
0.9109911	1.248266e-06\\
0.9110911	1.235333e-06\\
0.9111911	1.222474e-06\\
0.9112911	1.209766e-06\\
0.9113911	1.197124e-06\\
0.9114911	1.184755e-06\\
0.9115912	1.172557e-06\\
0.9116912	1.160506e-06\\
0.9117912	1.148508e-06\\
0.9118912	1.136586e-06\\
0.9119912	1.124807e-06\\
0.9120912	1.113232e-06\\
0.9121912	1.101769e-06\\
0.9122912	1.090361e-06\\
0.9123912	1.079049e-06\\
0.9124912	1.06798e-06\\
0.9125913	1.057006e-06\\
0.9126913	1.046075e-06\\
0.9127913	1.035237e-06\\
0.9128913	1.02441e-06\\
0.9129913	1.013716e-06\\
0.9130913	1.003207e-06\\
0.9131913	9.927666e-07\\
0.9132913	9.824614e-07\\
0.9133913	9.722237e-07\\
0.9134913	9.620705e-07\\
0.9135914	9.520872e-07\\
0.9136914	9.422007e-07\\
0.9137914	9.324043e-07\\
0.9138914	9.226995e-07\\
0.9139914	9.131087e-07\\
0.9140914	9.036977e-07\\
0.9141914	8.943756e-07\\
0.9142914	8.850918e-07\\
0.9143914	8.757717e-07\\
0.9144914	8.664967e-07\\
0.9145915	8.574038e-07\\
0.9146915	8.48391e-07\\
0.9147915	8.395599e-07\\
0.9148915	8.307307e-07\\
0.9149915	8.220676e-07\\
0.9150915	8.134971e-07\\
0.9151915	8.050864e-07\\
0.9152915	7.967138e-07\\
0.9153915	7.883987e-07\\
0.9154915	7.801072e-07\\
0.9155916	7.71896e-07\\
0.9156916	7.638626e-07\\
0.9157916	7.558815e-07\\
0.9158916	7.479817e-07\\
0.9159916	7.401522e-07\\
0.9160916	7.323187e-07\\
0.9161916	7.246994e-07\\
0.9162916	7.171214e-07\\
0.9163916	7.096554e-07\\
0.9164916	7.022149e-07\\
0.9165917	6.948202e-07\\
0.9166917	6.875194e-07\\
0.9167917	6.803696e-07\\
0.9168917	6.732517e-07\\
0.9169917	6.661466e-07\\
0.9170917	6.590276e-07\\
0.9171917	6.520533e-07\\
0.9172917	6.451433e-07\\
0.9173917	6.383875e-07\\
0.9174917	6.316622e-07\\
0.9175918	6.250066e-07\\
0.9176918	6.184223e-07\\
0.9177918	6.11919e-07\\
0.9178918	6.054588e-07\\
0.9179918	5.99066e-07\\
0.9180918	5.926979e-07\\
0.9181918	5.863947e-07\\
0.9182918	5.801859e-07\\
0.9183918	5.740736e-07\\
0.9184918	5.681059e-07\\
0.9185919	5.621796e-07\\
0.9186919	5.562461e-07\\
0.9187919	5.503494e-07\\
0.9188919	5.445134e-07\\
0.9189919	5.388054e-07\\
0.9190919	5.331472e-07\\
0.9191919	5.275044e-07\\
0.9192919	5.218633e-07\\
0.9193919	5.16345e-07\\
0.9194919	5.108942e-07\\
0.919592	5.054918e-07\\
0.919692	5.001419e-07\\
0.919792	4.947768e-07\\
0.919892	4.894724e-07\\
0.919992	4.842361e-07\\
0.920092	4.791277e-07\\
0.920192	4.740425e-07\\
0.920292	4.690048e-07\\
0.920392	4.639951e-07\\
0.920492	4.590311e-07\\
0.9205921	4.541276e-07\\
0.9206921	4.492476e-07\\
0.9207921	4.444752e-07\\
0.9208921	4.397252e-07\\
0.9209921	4.349786e-07\\
0.9210921	4.303369e-07\\
0.9211921	4.257996e-07\\
0.9212921	4.213727e-07\\
0.9213921	4.169951e-07\\
0.9214921	4.125861e-07\\
0.9215922	4.081929e-07\\
0.9216922	4.038013e-07\\
0.9217922	3.99489e-07\\
0.9218922	3.952135e-07\\
0.9219922	3.909571e-07\\
0.9220922	3.86744e-07\\
0.9221922	3.825706e-07\\
0.9222922	3.784505e-07\\
0.9223922	3.744299e-07\\
0.9224922	3.704194e-07\\
0.9225923	3.664689e-07\\
0.9226923	3.625087e-07\\
0.9227923	3.58609e-07\\
0.9228923	3.547501e-07\\
0.9229923	3.509822e-07\\
0.9230923	3.472482e-07\\
0.9231923	3.435999e-07\\
0.9232923	3.398815e-07\\
0.9233923	3.36234e-07\\
0.9234923	3.3256e-07\\
0.9235924	3.29022e-07\\
0.9236924	3.254674e-07\\
0.9237924	3.220012e-07\\
0.9238924	3.184581e-07\\
0.9239924	3.150335e-07\\
0.9240924	3.115908e-07\\
0.9241924	3.083271e-07\\
0.9242924	3.050018e-07\\
0.9243924	3.017698e-07\\
0.9244924	2.984723e-07\\
0.9245925	2.952938e-07\\
0.9246925	2.920437e-07\\
0.9247925	2.889606e-07\\
0.9248925	2.858146e-07\\
0.9249925	2.828377e-07\\
0.9250925	2.797542e-07\\
0.9251925	2.767933e-07\\
0.9252925	2.737962e-07\\
0.9253925	2.708966e-07\\
0.9254925	2.679846e-07\\
0.9255926	2.65103e-07\\
0.9256926	2.621906e-07\\
0.9257926	2.592863e-07\\
0.9258926	2.564413e-07\\
0.9259926	2.536032e-07\\
0.9260926	2.508969e-07\\
0.9261926	2.481634e-07\\
0.9262926	2.455092e-07\\
0.9263926	2.427574e-07\\
0.9264926	2.401587e-07\\
0.9265927	2.375121e-07\\
0.9266927	2.349879e-07\\
0.9267927	2.324504e-07\\
0.9268927	2.299612e-07\\
0.9269927	2.274728e-07\\
0.9270927	2.249942e-07\\
0.9271927	2.225912e-07\\
0.9272927	2.201901e-07\\
0.9273927	2.178348e-07\\
0.9274927	2.154335e-07\\
0.9275928	2.131143e-07\\
0.9276928	2.107273e-07\\
0.9277928	2.084491e-07\\
0.9278928	2.061389e-07\\
0.9279928	2.039579e-07\\
0.9280928	2.017425e-07\\
0.9281928	1.99588e-07\\
0.9282928	1.974123e-07\\
0.9283928	1.952294e-07\\
0.9284928	1.931246e-07\\
0.9285929	1.909878e-07\\
0.9286929	1.889358e-07\\
0.9287929	1.868103e-07\\
0.9288929	1.847539e-07\\
0.9289929	1.826434e-07\\
0.9290929	1.806379e-07\\
0.9291929	1.786406e-07\\
0.9292929	1.767117e-07\\
0.9293929	1.748295e-07\\
0.9294929	1.728853e-07\\
0.929593	1.710381e-07\\
0.929693	1.691277e-07\\
0.929793	1.672981e-07\\
0.929893	1.654474e-07\\
0.929993	1.636674e-07\\
0.930093	1.618946e-07\\
0.930193	1.601209e-07\\
0.930293	1.584031e-07\\
0.930393	1.566241e-07\\
0.930493	1.549567e-07\\
0.9305931	1.532354e-07\\
0.9306931	1.515954e-07\\
0.9307931	1.4991e-07\\
0.9308931	1.482391e-07\\
0.9309931	1.465882e-07\\
0.9310931	1.449111e-07\\
0.9311931	1.433532e-07\\
0.9312931	1.417769e-07\\
0.9313931	1.40263e-07\\
0.9314931	1.387105e-07\\
0.9315932	1.371696e-07\\
0.9316932	1.356768e-07\\
0.9317932	1.341204e-07\\
0.9318932	1.326887e-07\\
0.9319932	1.311881e-07\\
0.9320932	1.297591e-07\\
0.9321932	1.283161e-07\\
0.9322932	1.268792e-07\\
0.9323932	1.254919e-07\\
0.9324932	1.24067e-07\\
0.9325933	1.227448e-07\\
0.9326933	1.214138e-07\\
0.9327933	1.201067e-07\\
0.9328933	1.188093e-07\\
0.9329933	1.174571e-07\\
0.9330933	1.161727e-07\\
0.9331933	1.148349e-07\\
0.9332933	1.135542e-07\\
0.9333933	1.123081e-07\\
0.9334933	1.11065e-07\\
0.9335934	1.098631e-07\\
0.9336934	1.086364e-07\\
0.9337934	1.074473e-07\\
0.9338934	1.062795e-07\\
0.9339934	1.051134e-07\\
0.9340934	1.040051e-07\\
0.9341934	1.028337e-07\\
0.9342934	1.017146e-07\\
0.9343934	1.005652e-07\\
0.9344934	9.941645e-08\\
0.9345935	9.830051e-08\\
0.9346935	9.717257e-08\\
0.9347935	9.610506e-08\\
0.9348935	9.502246e-08\\
0.9349935	9.396246e-08\\
0.9350935	9.293184e-08\\
0.9351935	9.183706e-08\\
0.9352935	9.081655e-08\\
0.9353935	8.976944e-08\\
0.9354935	8.875798e-08\\
0.9355936	8.781044e-08\\
0.9356936	8.683319e-08\\
0.9357936	8.588504e-08\\
0.9358936	8.493664e-08\\
0.9359936	8.398116e-08\\
0.9360936	8.308526e-08\\
0.9361936	8.21671e-08\\
0.9362936	8.129114e-08\\
0.9363936	8.042402e-08\\
0.9364936	7.950092e-08\\
0.9365937	7.862396e-08\\
0.9366937	7.770935e-08\\
0.9367937	7.680936e-08\\
0.9368937	7.593449e-08\\
0.9369937	7.503315e-08\\
0.9370937	7.420712e-08\\
0.9371937	7.336364e-08\\
0.9372937	7.253631e-08\\
0.9373937	7.173969e-08\\
0.9374937	7.091629e-08\\
0.9375938	7.011267e-08\\
0.9376938	6.933757e-08\\
0.9377938	6.855065e-08\\
0.9378938	6.780679e-08\\
0.9379938	6.703774e-08\\
0.9380938	6.626149e-08\\
0.9381938	6.552165e-08\\
0.9382938	6.47396e-08\\
0.9383938	6.401164e-08\\
0.9384938	6.331556e-08\\
0.9385939	6.259455e-08\\
0.9386939	6.19277e-08\\
0.9387939	6.124813e-08\\
0.9388939	6.055164e-08\\
0.9389939	5.988167e-08\\
0.9390939	5.919384e-08\\
0.9391939	5.851541e-08\\
0.9392939	5.78842e-08\\
0.9393939	5.721342e-08\\
0.9394939	5.658678e-08\\
0.939594	5.596443e-08\\
0.939694	5.531572e-08\\
0.939794	5.470487e-08\\
0.939894	5.407328e-08\\
0.939994	5.343496e-08\\
0.940094	5.285252e-08\\
0.940194	5.224815e-08\\
0.940294	5.166517e-08\\
0.940394	5.109592e-08\\
0.940494	5.050273e-08\\
0.9405941	4.99064e-08\\
0.9406941	4.934527e-08\\
0.9407941	4.874256e-08\\
0.9408941	4.820706e-08\\
0.9409941	4.767428e-08\\
0.9410941	4.71422e-08\\
0.9411941	4.66327e-08\\
0.9412941	4.612177e-08\\
0.9413941	4.559251e-08\\
0.9414941	4.509428e-08\\
0.9415942	4.458142e-08\\
0.9416942	4.405285e-08\\
0.9417942	4.35821e-08\\
0.9418942	4.307858e-08\\
0.9419942	4.257885e-08\\
0.9420942	4.211778e-08\\
0.9421942	4.162856e-08\\
0.9422942	4.113749e-08\\
0.9423942	4.068914e-08\\
0.9424942	4.02117e-08\\
0.9425943	3.976459e-08\\
0.9426943	3.934597e-08\\
0.9427943	3.891054e-08\\
0.9428943	3.847655e-08\\
0.9429943	3.806112e-08\\
0.9430943	3.760098e-08\\
0.9431943	3.715249e-08\\
0.9432943	3.672757e-08\\
0.9433943	3.628621e-08\\
0.9434943	3.584995e-08\\
0.9435944	3.545736e-08\\
0.9436944	3.503817e-08\\
0.9437944	3.461846e-08\\
0.9438944	3.423283e-08\\
0.9439944	3.382276e-08\\
0.9440944	3.342189e-08\\
0.9441944	3.30484e-08\\
0.9442944	3.267165e-08\\
0.9443944	3.22943e-08\\
0.9444944	3.194542e-08\\
0.9445945	3.160549e-08\\
0.9446945	3.123187e-08\\
0.9447945	3.090383e-08\\
0.9448945	3.054304e-08\\
0.9449945	3.017338e-08\\
0.9450945	2.982821e-08\\
0.9451945	2.949277e-08\\
0.9452945	2.914231e-08\\
0.9453945	2.882608e-08\\
0.9454945	2.851232e-08\\
0.9455946	2.817275e-08\\
0.9456946	2.784795e-08\\
0.9457946	2.752367e-08\\
0.9458946	2.717964e-08\\
0.9459946	2.683785e-08\\
0.9460946	2.652958e-08\\
0.9461946	2.620727e-08\\
0.9462946	2.58804e-08\\
0.9463946	2.559837e-08\\
0.9464946	2.529567e-08\\
0.9465947	2.499568e-08\\
0.9466947	2.470921e-08\\
0.9467947	2.444373e-08\\
0.9468947	2.414337e-08\\
0.9469947	2.386677e-08\\
0.9470947	2.361596e-08\\
0.9471947	2.334031e-08\\
0.9472947	2.308248e-08\\
0.9473947	2.284522e-08\\
0.9474947	2.258203e-08\\
0.9475948	2.23135e-08\\
0.9476948	2.205653e-08\\
0.9477948	2.180034e-08\\
0.9478948	2.152645e-08\\
0.9479948	2.126809e-08\\
0.9480948	2.10347e-08\\
0.9481948	2.07893e-08\\
0.9482948	2.054681e-08\\
0.9483948	2.032848e-08\\
0.9484948	2.009934e-08\\
0.9485949	1.98576e-08\\
0.9486949	1.963271e-08\\
0.9487949	1.942163e-08\\
0.9488949	1.918467e-08\\
0.9489949	1.895354e-08\\
0.9490949	1.875233e-08\\
0.9491949	1.853555e-08\\
0.9492949	1.830992e-08\\
0.9493949	1.810046e-08\\
0.9494949	1.79037e-08\\
0.949595	1.767913e-08\\
0.949695	1.746355e-08\\
0.949795	1.726072e-08\\
0.949895	1.705667e-08\\
0.949995	1.684802e-08\\
0.950095	1.66602e-08\\
0.950195	1.648956e-08\\
0.950295	1.630528e-08\\
0.950395	1.612789e-08\\
0.950495	1.595592e-08\\
0.9505951	1.578314e-08\\
0.9506951	1.559543e-08\\
0.9507951	1.540138e-08\\
0.9508951	1.523396e-08\\
0.9509951	1.505551e-08\\
0.9510951	1.4868e-08\\
0.9511951	1.469419e-08\\
0.9512951	1.453749e-08\\
0.9513951	1.436332e-08\\
0.9514951	1.419367e-08\\
0.9515952	1.40316e-08\\
0.9516952	1.387403e-08\\
0.9517952	1.369976e-08\\
0.9518952	1.352388e-08\\
0.9519952	1.336539e-08\\
0.9520952	1.320401e-08\\
0.9521952	1.303576e-08\\
0.9522952	1.287841e-08\\
0.9523952	1.273625e-08\\
0.9524952	1.259417e-08\\
0.9525953	1.243665e-08\\
0.9526953	1.228617e-08\\
0.9527953	1.214377e-08\\
0.9528953	1.19949e-08\\
0.9529953	1.18339e-08\\
0.9530953	1.168875e-08\\
0.9531953	1.155478e-08\\
0.9532953	1.142109e-08\\
0.9533953	1.128386e-08\\
0.9534953	1.115577e-08\\
0.9535954	1.10454e-08\\
0.9536954	1.092311e-08\\
0.9537954	1.079276e-08\\
0.9538954	1.067201e-08\\
0.9539954	1.055597e-08\\
0.9540954	1.04279e-08\\
0.9541954	1.028626e-08\\
0.9542954	1.015177e-08\\
0.9543954	1.003299e-08\\
0.9544954	9.90069e-09\\
0.9545955	9.770234e-09\\
0.9546955	9.649248e-09\\
0.9547955	9.536845e-09\\
0.9548955	9.427392e-09\\
0.9549955	9.302888e-09\\
0.9550955	9.183578e-09\\
0.9551955	9.080035e-09\\
0.9552955	8.969344e-09\\
0.9553955	8.857141e-09\\
0.9554955	8.743563e-09\\
0.9555956	8.646351e-09\\
0.9556956	8.56048e-09\\
0.9557956	8.463334e-09\\
0.9558956	8.362279e-09\\
0.9559956	8.275045e-09\\
0.9560956	8.194828e-09\\
0.9561956	8.105755e-09\\
0.9562956	8.005204e-09\\
0.9563956	7.912148e-09\\
0.9564956	7.823253e-09\\
0.9565957	7.734699e-09\\
0.9566957	7.630717e-09\\
0.9567957	7.529606e-09\\
0.9568957	7.44239e-09\\
0.9569957	7.361014e-09\\
0.9570957	7.277511e-09\\
0.9571957	7.189316e-09\\
0.9572957	7.108621e-09\\
0.9573957	7.037939e-09\\
0.9574957	6.965293e-09\\
0.9575958	6.877743e-09\\
0.9576958	6.793096e-09\\
0.9577958	6.71391e-09\\
0.9578958	6.639459e-09\\
0.9579958	6.56144e-09\\
0.9580958	6.475157e-09\\
0.9581958	6.39418e-09\\
0.9582958	6.324487e-09\\
0.9583958	6.258801e-09\\
0.9584958	6.182001e-09\\
0.9585959	6.104333e-09\\
0.9586959	6.036762e-09\\
0.9587959	5.973172e-09\\
0.9588959	5.914053e-09\\
0.9589959	5.838099e-09\\
0.9590959	5.764416e-09\\
0.9591959	5.698501e-09\\
0.9592959	5.637715e-09\\
0.9593959	5.570545e-09\\
0.9594959	5.501823e-09\\
0.959596	5.431526e-09\\
0.959696	5.37168e-09\\
0.959796	5.319418e-09\\
0.959896	5.262336e-09\\
0.959996	5.197623e-09\\
0.960096	5.131187e-09\\
0.960196	5.068976e-09\\
0.960296	5.010794e-09\\
0.960396	4.948464e-09\\
0.960496	4.876846e-09\\
0.9605961	4.803477e-09\\
0.9606961	4.738693e-09\\
0.9607961	4.679465e-09\\
0.9608961	4.618121e-09\\
0.9609961	4.550285e-09\\
0.9610961	4.485918e-09\\
0.9611961	4.426679e-09\\
0.9612961	4.376553e-09\\
0.9613961	4.330196e-09\\
0.9614961	4.277084e-09\\
0.9615962	4.221738e-09\\
0.9616962	4.172107e-09\\
0.9617962	4.126922e-09\\
0.9618962	4.079485e-09\\
0.9619962	4.032577e-09\\
0.9620962	3.975695e-09\\
0.9621962	3.919302e-09\\
0.9622962	3.869614e-09\\
0.9623962	3.825886e-09\\
0.9624962	3.780348e-09\\
0.9625963	3.732488e-09\\
0.9626963	3.680056e-09\\
0.9627963	3.631478e-09\\
0.9628963	3.587786e-09\\
0.9629963	3.550616e-09\\
0.9630963	3.507229e-09\\
0.9631963	3.455918e-09\\
0.9632963	3.410364e-09\\
0.9633963	3.364471e-09\\
0.9634963	3.322229e-09\\
0.9635964	3.285276e-09\\
0.9636964	3.244123e-09\\
0.9637964	3.194549e-09\\
0.9638964	3.146877e-09\\
0.9639964	3.108534e-09\\
0.9640964	3.072304e-09\\
0.9641964	3.041206e-09\\
0.9642964	3.008127e-09\\
0.9643964	2.974156e-09\\
0.9644964	2.938972e-09\\
0.9645965	2.909707e-09\\
0.9646965	2.885649e-09\\
0.9647965	2.863755e-09\\
0.9648965	2.835594e-09\\
0.9649965	2.803307e-09\\
0.9650965	2.770812e-09\\
0.9651965	2.734912e-09\\
0.9652965	2.709011e-09\\
0.9653965	2.68292e-09\\
0.9654965	2.654822e-09\\
0.9655966	2.620377e-09\\
0.9656966	2.583652e-09\\
0.9657966	2.550794e-09\\
0.9658966	2.51974e-09\\
0.9659966	2.493158e-09\\
0.9660966	2.470912e-09\\
0.9661966	2.442321e-09\\
0.9662966	2.41225e-09\\
0.9663966	2.382348e-09\\
0.9664966	2.356821e-09\\
0.9665967	2.334227e-09\\
0.9666967	2.312342e-09\\
0.9667967	2.291295e-09\\
0.9668967	2.266511e-09\\
0.9669967	2.23564e-09\\
0.9670967	2.202693e-09\\
0.9671967	2.176562e-09\\
0.9672967	2.151265e-09\\
0.9673967	2.127188e-09\\
0.9674967	2.106029e-09\\
0.9675968	2.081768e-09\\
0.9676968	2.051653e-09\\
0.9677968	2.024763e-09\\
0.9678968	1.998172e-09\\
0.9679968	1.975972e-09\\
0.9680968	1.956845e-09\\
0.9681968	1.936843e-09\\
0.9682968	1.915037e-09\\
0.9683968	1.891122e-09\\
0.9684968	1.86336e-09\\
0.9685969	1.837164e-09\\
0.9686969	1.812049e-09\\
0.9687969	1.792056e-09\\
0.9688969	1.768617e-09\\
0.9689969	1.745027e-09\\
0.9690969	1.718577e-09\\
0.9691969	1.68962e-09\\
0.9692969	1.657156e-09\\
0.9693969	1.625582e-09\\
0.9694969	1.600154e-09\\
0.969597	1.577048e-09\\
0.969697	1.556593e-09\\
0.969797	1.533112e-09\\
0.969897	1.512807e-09\\
0.969997	1.490161e-09\\
0.970097	1.465552e-09\\
0.970197	1.442414e-09\\
0.970297	1.423128e-09\\
0.970397	1.406611e-09\\
0.970497	1.391674e-09\\
0.9705971	1.376351e-09\\
0.9706971	1.361075e-09\\
0.9707971	1.344582e-09\\
0.9708971	1.325359e-09\\
0.9709971	1.303747e-09\\
0.9710971	1.284365e-09\\
0.9711971	1.267705e-09\\
0.9712971	1.252308e-09\\
0.9713971	1.23746e-09\\
0.9714971	1.22272e-09\\
0.9715972	1.205306e-09\\
0.9716972	1.188841e-09\\
0.9717972	1.168943e-09\\
0.9718972	1.148179e-09\\
0.9719972	1.130168e-09\\
0.9720972	1.116472e-09\\
0.9721972	1.102365e-09\\
0.9722972	1.09167e-09\\
0.9723972	1.082108e-09\\
0.9724972	1.073931e-09\\
0.9725973	1.063192e-09\\
0.9726973	1.051476e-09\\
0.9727973	1.038193e-09\\
0.9728973	1.02551e-09\\
0.9729973	1.015803e-09\\
0.9730973	1.004499e-09\\
0.9731973	9.974441e-10\\
0.9732973	9.909004e-10\\
0.9733973	9.877617e-10\\
0.9734973	9.839342e-10\\
0.9735974	9.773432e-10\\
0.9736974	9.69787e-10\\
0.9737974	9.607927e-10\\
0.9738974	9.511377e-10\\
0.9739974	9.401866e-10\\
0.9740974	9.285986e-10\\
0.9741974	9.188679e-10\\
0.9742974	9.110526e-10\\
0.9743974	9.042528e-10\\
0.9744974	8.973885e-10\\
0.9745975	8.904956e-10\\
0.9746975	8.833947e-10\\
0.9747975	8.766893e-10\\
0.9748975	8.667196e-10\\
0.9749975	8.550819e-10\\
0.9750975	8.433489e-10\\
0.9751975	8.310048e-10\\
0.9752975	8.203843e-10\\
0.9753975	8.099341e-10\\
0.9754975	7.990859e-10\\
0.9755976	7.912676e-10\\
0.9756976	7.83473e-10\\
0.9757976	7.770299e-10\\
0.9758976	7.709894e-10\\
0.9759976	7.635715e-10\\
0.9760976	7.552436e-10\\
0.9761976	7.462081e-10\\
0.9762976	7.367897e-10\\
0.9763976	7.251678e-10\\
0.9764976	7.127085e-10\\
0.9765977	6.998026e-10\\
0.9766977	6.880346e-10\\
0.9767977	6.766864e-10\\
0.9768977	6.654847e-10\\
0.9769977	6.540023e-10\\
0.9770977	6.435582e-10\\
0.9771977	6.338033e-10\\
0.9772977	6.24923e-10\\
0.9773977	6.159417e-10\\
0.9774977	6.050473e-10\\
0.9775978	5.947937e-10\\
0.9776978	5.828761e-10\\
0.9777978	5.716964e-10\\
0.9778978	5.595295e-10\\
0.9779978	5.464522e-10\\
0.9780978	5.329953e-10\\
0.9781978	5.199527e-10\\
0.9782978	5.068621e-10\\
0.9783978	4.955452e-10\\
0.9784978	4.846296e-10\\
0.9785979	4.750393e-10\\
0.9786979	4.646354e-10\\
0.9787979	4.569833e-10\\
0.9788979	4.499691e-10\\
0.9789979	4.445345e-10\\
0.9790979	4.397854e-10\\
0.9791979	4.338596e-10\\
0.9792979	4.285044e-10\\
0.9793979	4.235264e-10\\
0.9794979	4.193842e-10\\
0.979598	4.143255e-10\\
0.979698	4.096984e-10\\
0.979798	4.032881e-10\\
0.979898	3.970159e-10\\
0.979998	3.911812e-10\\
0.980098	3.851055e-10\\
0.980198	3.79561e-10\\
0.980298	3.744369e-10\\
0.980398	3.687513e-10\\
0.980498	3.640177e-10\\
0.9805981	3.587499e-10\\
0.9806981	3.560685e-10\\
0.9807981	3.535717e-10\\
0.9808981	3.51006e-10\\
0.9809981	3.49785e-10\\
0.9810981	3.478854e-10\\
0.9811981	3.465315e-10\\
0.9812981	3.466237e-10\\
0.9813981	3.464543e-10\\
0.9814981	3.475055e-10\\
0.9815982	3.471643e-10\\
0.9816982	3.469885e-10\\
0.9817982	3.469106e-10\\
0.9818982	3.470603e-10\\
0.9819982	3.473395e-10\\
0.9820982	3.479323e-10\\
0.9821982	3.478052e-10\\
0.9822982	3.470004e-10\\
0.9823982	3.469282e-10\\
0.9824982	3.457619e-10\\
0.9825983	3.446937e-10\\
0.9826983	3.439025e-10\\
0.9827983	3.436136e-10\\
0.9828983	3.417495e-10\\
0.9829983	3.40133e-10\\
0.9830983	3.372918e-10\\
0.9831983	3.350141e-10\\
0.9832983	3.323976e-10\\
0.9833983	3.298538e-10\\
0.9834983	3.271558e-10\\
0.9835984	3.238103e-10\\
0.9836984	3.205762e-10\\
0.9837984	3.163484e-10\\
0.9838984	3.124903e-10\\
0.9839984	3.080911e-10\\
0.9840984	3.045761e-10\\
0.9841984	3.009024e-10\\
0.9842984	2.974918e-10\\
0.9843984	2.932785e-10\\
0.9844984	2.892689e-10\\
0.9845985	2.843574e-10\\
0.9846985	2.804471e-10\\
0.9847985	2.761187e-10\\
0.9848985	2.726575e-10\\
0.9849985	2.685805e-10\\
0.9850985	2.647023e-10\\
0.9851985	2.603383e-10\\
0.9852985	2.560402e-10\\
0.9853985	2.515027e-10\\
0.9854985	2.462845e-10\\
0.9855986	2.426526e-10\\
0.9856986	2.37672e-10\\
0.9857986	2.336986e-10\\
0.9858986	2.300187e-10\\
0.9859986	2.25384e-10\\
0.9860986	2.206843e-10\\
0.9861986	2.165004e-10\\
0.9862986	2.12141e-10\\
0.9863986	2.080542e-10\\
0.9864986	2.039044e-10\\
0.9865987	2.008127e-10\\
0.9866987	1.976201e-10\\
0.9867987	1.947418e-10\\
0.9868987	1.913436e-10\\
0.9869987	1.884148e-10\\
0.9870987	1.857379e-10\\
0.9871987	1.823862e-10\\
0.9872987	1.797207e-10\\
0.9873987	1.776018e-10\\
0.9874987	1.759773e-10\\
0.9875988	1.734729e-10\\
0.9876988	1.723615e-10\\
0.9877988	1.703646e-10\\
0.9878988	1.677577e-10\\
0.9879988	1.660918e-10\\
0.9880988	1.635677e-10\\
0.9881988	1.613684e-10\\
0.9882988	1.587617e-10\\
0.9883988	1.56629e-10\\
0.9884988	1.546799e-10\\
0.9885989	1.521662e-10\\
0.9886989	1.498408e-10\\
0.9887989	1.471435e-10\\
0.9888989	1.439186e-10\\
0.9889989	1.407902e-10\\
0.9890989	1.371611e-10\\
0.9891989	1.334435e-10\\
0.9892989	1.299497e-10\\
0.9893989	1.267862e-10\\
0.9894989	1.234444e-10\\
0.989599	1.202808e-10\\
0.989699	1.175245e-10\\
0.989799	1.145454e-10\\
0.989899	1.121203e-10\\
0.989999	1.094442e-10\\
0.990099	1.07495e-10\\
0.990199	1.053096e-10\\
0.990299	1.034528e-10\\
0.990399	1.023476e-10\\
0.990499	1.011041e-10\\
0.9905991	1.005625e-10\\
0.9906991	1.002694e-10\\
0.9907991	9.981116e-11\\
0.9908991	9.99627e-11\\
0.9909991	9.951322e-11\\
0.9910991	9.897449e-11\\
0.9911991	9.805011e-11\\
0.9912991	9.685612e-11\\
0.9913991	9.55356e-11\\
0.9914991	9.359924e-11\\
0.9915992	9.15246e-11\\
0.9916992	8.940367e-11\\
0.9917992	8.754734e-11\\
0.9918992	8.573544e-11\\
0.9919992	8.416599e-11\\
0.9920992	8.309778e-11\\
0.9921992	8.223597e-11\\
0.9922992	8.156778e-11\\
0.9923992	8.125266e-11\\
0.9924992	8.125442e-11\\
0.9925993	8.123122e-11\\
0.9926993	8.10784e-11\\
0.9927993	8.086562e-11\\
0.9928993	7.98935e-11\\
0.9929993	7.925768e-11\\
0.9930993	7.777233e-11\\
0.9931993	7.622448e-11\\
0.9932993	7.461509e-11\\
0.9933993	7.258625e-11\\
0.9934993	7.102666e-11\\
0.9935994	6.937069e-11\\
0.9936994	6.79074e-11\\
0.9937994	6.72554e-11\\
0.9938994	6.661438e-11\\
0.9939994	6.615281e-11\\
0.9940994	6.592081e-11\\
0.9941994	6.56418e-11\\
0.9942994	6.5059e-11\\
0.9943994	6.425522e-11\\
0.9944994	6.29384e-11\\
0.9945995	6.11961e-11\\
0.9946995	5.935305e-11\\
0.9947995	5.739155e-11\\
0.9948995	5.560676e-11\\
0.9949995	5.394728e-11\\
0.9950995	5.291748e-11\\
0.9951995	5.213564e-11\\
0.9952995	5.173827e-11\\
0.9953995	5.165547e-11\\
0.9954995	5.134587e-11\\
0.9955996	5.111186e-11\\
0.9956996	5.047385e-11\\
0.9957996	4.957269e-11\\
0.9958996	4.867964e-11\\
0.9959996	4.741164e-11\\
0.9960996	4.642177e-11\\
0.9961996	4.581174e-11\\
0.9962996	4.533563e-11\\
0.9963996	4.540866e-11\\
0.9964996	4.52825e-11\\
0.9965997	4.527493e-11\\
0.9966997	4.517825e-11\\
0.9967997	4.446679e-11\\
0.9968997	4.323423e-11\\
0.9969997	4.202143e-11\\
0.9970997	4.037842e-11\\
0.9971997	3.891031e-11\\
0.9972997	3.774784e-11\\
0.9973997	3.676451e-11\\
0.9974997	3.616561e-11\\
0.9975998	3.558245e-11\\
0.9976998	3.465757e-11\\
0.9977998	3.354356e-11\\
0.9978998	3.210041e-11\\
0.9979998	3.051301e-11\\
0.9980998	2.88882e-11\\
0.9981998	2.744015e-11\\
0.9982998	2.636467e-11\\
0.9983998	2.56116e-11\\
0.9984998	2.52488e-11\\
0.9985999	2.481258e-11\\
0.9986999	2.425222e-11\\
0.9987999	2.368402e-11\\
0.9988999	2.283939e-11\\
0.9989999	2.189448e-11\\
0.9990999	2.111959e-11\\
0.9991999	2.074564e-11\\
0.9992999	2.08783e-11\\
0.9993999	2.102085e-11\\
0.9994999	2.114064e-11\\
0.9996	2.132375e-11\\
0.9997	2.119032e-11\\
0.9998	2.065232e-11\\
0.9999	2.025757e-11\\
1	1.981922e-11\\
};
\addlegendentry{$\text{|}\psi{}_\text{n}\text{(x,t)|}^\text{2}$};

\addplot [color=mycolor2,solid,forget plot]
  table[row sep=crcr]{%
0	-3.787637e-06\\
0.00010001	-3.721108e-06\\
0.00020002	-3.6193e-06\\
0.00030003	-3.492774e-06\\
0.00040004	-3.310344e-06\\
0.00050005	-3.106874e-06\\
0.00060006	-2.919953e-06\\
0.00070007	-2.74047e-06\\
0.00080008	-2.634321e-06\\
0.00090009	-2.506326e-06\\
0.0010001	-2.36282e-06\\
0.00110011	-2.205753e-06\\
0.00120012	-2.00196e-06\\
0.00130013	-1.813745e-06\\
0.00140014	-1.629034e-06\\
0.00150015	-1.513779e-06\\
0.00160016	-1.403393e-06\\
0.00170017	-1.296449e-06\\
0.00180018	-1.146234e-06\\
0.00190019	-9.522884e-07\\
0.0020002	-7.383212e-07\\
0.00210021	-5.511688e-07\\
0.00220022	-4.046892e-07\\
0.00230023	-2.57261e-07\\
0.00240024	-1.192581e-07\\
0.00250025	5.590809e-08\\
0.00260026	2.732271e-07\\
0.00270027	4.958221e-07\\
0.00280028	6.904258e-07\\
0.00290029	8.17273e-07\\
0.0030003	9.153338e-07\\
0.00310031	1.02262e-06\\
0.00320032	1.135745e-06\\
0.00330033	1.289303e-06\\
0.00340034	1.40943e-06\\
0.00350035	1.494615e-06\\
0.00360036	1.520888e-06\\
0.00370037	1.510301e-06\\
0.00380038	1.563583e-06\\
0.00390039	1.656422e-06\\
0.0040004	1.762198e-06\\
0.00410041	1.841707e-06\\
0.00420042	1.902134e-06\\
0.00430043	1.91876e-06\\
0.00440044	2.003883e-06\\
0.00450045	2.112674e-06\\
0.00460046	2.263242e-06\\
0.00470047	2.415075e-06\\
0.00480048	2.497645e-06\\
0.00490049	2.543971e-06\\
0.0050005	2.614252e-06\\
0.00510051	2.727885e-06\\
0.00520052	2.826044e-06\\
0.00530053	2.934795e-06\\
0.00540054	2.971496e-06\\
0.00550055	2.957661e-06\\
0.00560056	2.965273e-06\\
0.00570057	3.000554e-06\\
0.00580058	3.077964e-06\\
0.00590059	3.100639e-06\\
0.0060006	3.087469e-06\\
0.00610061	3.039841e-06\\
0.00620062	3.016411e-06\\
0.00630063	3.015956e-06\\
0.00640064	3.047442e-06\\
0.00650065	3.012794e-06\\
0.00660066	2.927008e-06\\
0.00670067	2.826323e-06\\
0.00680068	2.750326e-06\\
0.00690069	2.703331e-06\\
0.0070007	2.655337e-06\\
0.00710071	2.552178e-06\\
0.00720072	2.417309e-06\\
0.00730073	2.310148e-06\\
0.00740074	2.248545e-06\\
0.00750075	2.254295e-06\\
0.00760076	2.201251e-06\\
0.00770077	2.151394e-06\\
0.00780078	2.053915e-06\\
0.00790079	2.045988e-06\\
0.0080008	2.057342e-06\\
0.00810081	2.051669e-06\\
0.00820082	1.995858e-06\\
0.00830083	1.894697e-06\\
0.00840084	1.819558e-06\\
0.00850085	1.773366e-06\\
0.00860086	1.7273e-06\\
0.00870087	1.617131e-06\\
0.00880088	1.471365e-06\\
0.00890089	1.353863e-06\\
0.0090009	1.30696e-06\\
0.00910091	1.246811e-06\\
0.00920092	1.140188e-06\\
0.00930093	1.02008e-06\\
0.00940094	9.119139e-07\\
0.00950095	8.651254e-07\\
0.00960096	8.203556e-07\\
0.00970097	7.14577e-07\\
0.00980098	5.576429e-07\\
0.00990099	4.461702e-07\\
0.010001	3.863493e-07\\
0.01010101	3.071052e-07\\
0.01020102	1.715061e-07\\
0.01030103	2.95144e-08\\
0.01040104	-7.200793e-08\\
0.01050105	-1.428517e-07\\
0.01060106	-1.958551e-07\\
0.01070107	-3.108413e-07\\
0.01080108	-4.533877e-07\\
0.01090109	-5.181597e-07\\
0.0110011	-5.192525e-07\\
0.01110111	-6.124787e-07\\
0.01120112	-7.086399e-07\\
0.01130113	-8.338723e-07\\
0.01140114	-8.743609e-07\\
0.01150115	-9.082351e-07\\
0.01160116	-1.018012e-06\\
0.01170117	-1.158575e-06\\
0.01180118	-1.248443e-06\\
0.01190119	-1.280368e-06\\
0.0120012	-1.325952e-06\\
0.01210121	-1.387282e-06\\
0.01220122	-1.454847e-06\\
0.01230123	-1.415096e-06\\
0.01240124	-1.339105e-06\\
0.01250125	-1.32815e-06\\
0.01260126	-1.346809e-06\\
0.01270127	-1.339443e-06\\
0.01280128	-1.296362e-06\\
0.01290129	-1.312192e-06\\
0.0130013	-1.415762e-06\\
0.01310131	-1.471625e-06\\
0.01320132	-1.493737e-06\\
0.01330133	-1.488777e-06\\
0.01340134	-1.532255e-06\\
0.01350135	-1.599633e-06\\
0.01360136	-1.587392e-06\\
0.01370137	-1.584742e-06\\
0.01380138	-1.620941e-06\\
0.01390139	-1.708725e-06\\
0.0140014	-1.777351e-06\\
0.01410141	-1.781405e-06\\
0.01420142	-1.827636e-06\\
0.01430143	-1.903248e-06\\
0.01440144	-1.940654e-06\\
0.01450145	-1.887621e-06\\
0.01460146	-1.838768e-06\\
0.01470147	-1.849218e-06\\
0.01480148	-1.812563e-06\\
0.01490149	-1.688759e-06\\
0.0150015	-1.587047e-06\\
0.01510151	-1.524741e-06\\
0.01520152	-1.464088e-06\\
0.01530153	-1.347892e-06\\
0.01540154	-1.250062e-06\\
0.01550155	-1.270331e-06\\
0.01560156	-1.252675e-06\\
0.01570157	-1.207199e-06\\
0.01580158	-1.133681e-06\\
0.01590159	-1.124956e-06\\
0.0160016	-1.080292e-06\\
0.01610161	-9.61754e-07\\
0.01620162	-8.78093e-07\\
0.01630163	-8.714212e-07\\
0.01640164	-8.689207e-07\\
0.01650165	-8.132182e-07\\
0.01660166	-7.670585e-07\\
0.01670167	-7.899955e-07\\
0.01680168	-7.719295e-07\\
0.01690169	-6.906359e-07\\
0.0170017	-6.378708e-07\\
0.01710171	-6.377883e-07\\
0.01720172	-5.737373e-07\\
0.01730173	-4.723078e-07\\
0.01740174	-4.062627e-07\\
0.01750175	-3.94749e-07\\
0.01760176	-3.249007e-07\\
0.01770177	-2.418157e-07\\
0.01780178	-2.659166e-07\\
0.01790179	-2.978497e-07\\
0.0180018	-2.370738e-07\\
0.01810181	-1.644216e-07\\
0.01820182	-1.382173e-07\\
0.01830183	-5.356198e-08\\
0.01840184	7.267926e-08\\
0.01850185	1.681057e-07\\
0.01860186	1.738458e-07\\
0.01870187	2.542808e-07\\
0.01880188	3.542818e-07\\
0.01890189	3.776241e-07\\
0.0190019	3.724503e-07\\
0.01910191	4.479222e-07\\
0.01920192	4.955349e-07\\
0.01930193	4.970411e-07\\
0.01940194	5.38405e-07\\
0.01950195	6.657981e-07\\
0.01960196	7.12514e-07\\
0.01970197	7.090083e-07\\
0.01980198	7.658561e-07\\
0.01990199	8.025598e-07\\
0.020002	7.731948e-07\\
0.02010201	7.728574e-07\\
0.02020202	8.627441e-07\\
0.02030203	9.022258e-07\\
0.02040204	8.96584e-07\\
0.02050205	9.836776e-07\\
0.02060206	1.033761e-06\\
0.02070207	1.03884e-06\\
0.02080208	1.034869e-06\\
0.02090209	1.066505e-06\\
0.0210021	1.012331e-06\\
0.02110211	9.09916e-07\\
0.02120212	8.849961e-07\\
0.02130213	8.696381e-07\\
0.02140214	8.028996e-07\\
0.02150215	8.059241e-07\\
0.02160216	8.735105e-07\\
0.02170217	8.011435e-07\\
0.02180218	7.925528e-07\\
0.02190219	8.321205e-07\\
0.0220022	8.445161e-07\\
0.02210221	7.981801e-07\\
0.02220222	8.390293e-07\\
0.02230223	8.750146e-07\\
0.02240224	8.589288e-07\\
0.02250225	9.393238e-07\\
0.02260226	1.030603e-06\\
0.02270227	1.027493e-06\\
0.02280228	1.012965e-06\\
0.02290229	1.031103e-06\\
0.0230023	9.975696e-07\\
0.02310231	9.238786e-07\\
0.02320232	9.557158e-07\\
0.02330233	9.098643e-07\\
0.02340234	8.436291e-07\\
0.02350235	8.687325e-07\\
0.02360236	8.89721e-07\\
0.02370237	8.100976e-07\\
0.02380238	7.95868e-07\\
0.02390239	7.199153e-07\\
0.0240024	5.959094e-07\\
0.02410241	5.213561e-07\\
0.02420242	4.856724e-07\\
0.02430243	3.896053e-07\\
0.02440244	3.202734e-07\\
0.02450245	3.198287e-07\\
0.02460246	2.598892e-07\\
0.02470247	2.223668e-07\\
0.02480248	2.308061e-07\\
0.02490249	1.524157e-07\\
0.0250025	9.233286e-08\\
0.02510251	1.332618e-07\\
0.02520252	1.439847e-07\\
0.02530253	1.46604e-07\\
0.02540254	2.01717e-07\\
0.02550255	2.20919e-07\\
0.02560256	1.815365e-07\\
0.02570257	2.100047e-07\\
0.02580258	1.507841e-07\\
0.02590259	3.287868e-08\\
0.0260026	6.037139e-09\\
0.02610261	-3.703287e-08\\
0.02620262	-9.36763e-08\\
0.02630263	-6.610816e-08\\
0.02640264	-1.20972e-07\\
0.02650265	-1.629184e-07\\
0.02660266	-1.497993e-07\\
0.02670267	-1.844294e-07\\
0.02680268	-2.532658e-07\\
0.02690269	-2.373752e-07\\
0.0270027	-2.765579e-07\\
0.02710271	-2.645145e-07\\
0.02720272	-2.212522e-07\\
0.02730273	-2.605067e-07\\
0.02740274	-3.116788e-07\\
0.02750275	-3.029631e-07\\
0.02760276	-3.636011e-07\\
0.02770277	-4.545164e-07\\
0.02780278	-4.734925e-07\\
0.02790279	-5.343623e-07\\
0.0280028	-5.044554e-07\\
0.02810281	-4.788718e-07\\
0.02820282	-5.437028e-07\\
0.02830283	-5.408641e-07\\
0.02840284	-5.549341e-07\\
0.02850285	-6.183043e-07\\
0.02860286	-6.461512e-07\\
0.02870287	-6.61206e-07\\
0.02880288	-6.941281e-07\\
0.02890289	-6.276241e-07\\
0.0290029	-6.209655e-07\\
0.02910291	-6.599492e-07\\
0.02920292	-6.175405e-07\\
0.02930293	-6.430448e-07\\
0.02940294	-6.447761e-07\\
0.02950295	-5.995354e-07\\
0.02960296	-6.045114e-07\\
0.02970297	-5.232961e-07\\
0.02980298	-4.310517e-07\\
0.02990299	-4.570615e-07\\
0.030003	-4.085187e-07\\
0.03010301	-4.071167e-07\\
0.03020302	-4.514614e-07\\
0.03030303	-4.129017e-07\\
0.03040304	-4.488362e-07\\
0.03050305	-4.459939e-07\\
0.03060306	-3.720177e-07\\
0.03070307	-3.906562e-07\\
0.03080308	-3.781744e-07\\
0.03090309	-3.942511e-07\\
0.0310031	-4.263386e-07\\
0.03110311	-4.011432e-07\\
0.03120312	-4.151933e-07\\
0.03130313	-4.22703e-07\\
0.03140314	-3.390981e-07\\
0.03150315	-3.963526e-07\\
0.03160316	-4.119431e-07\\
0.03170317	-4.42735e-07\\
0.03180318	-5.066288e-07\\
0.03190319	-4.918624e-07\\
0.0320032	-5.470175e-07\\
0.03210321	-5.088285e-07\\
0.03220322	-3.984457e-07\\
0.03230323	-3.890567e-07\\
0.03240324	-3.111506e-07\\
0.03250325	-2.574704e-07\\
0.03260326	-2.204913e-07\\
0.03270327	-1.355209e-07\\
0.03280328	-1.051928e-07\\
0.03290329	-1.629923e-08\\
0.0330033	6.711793e-08\\
0.03310331	4.073446e-08\\
0.03320332	1.03181e-07\\
0.03330333	6.329607e-08\\
0.03340334	6.428667e-08\\
0.03350335	8.784233e-08\\
0.03360336	5.744514e-08\\
0.03370337	1.523551e-07\\
0.03380338	1.520878e-07\\
0.03390339	1.40584e-07\\
0.0340034	1.583981e-07\\
0.03410341	7.704973e-08\\
0.03420342	7.043343e-08\\
0.03430343	2.369275e-08\\
0.03440344	1.775206e-08\\
0.03450345	8.989503e-08\\
0.03460346	5.898626e-08\\
0.03470347	1.012259e-07\\
0.03480348	1.26648e-07\\
0.03490349	1.343268e-07\\
0.0350035	1.855192e-07\\
0.03510351	1.898361e-07\\
0.03520352	2.910164e-07\\
0.03530353	2.779714e-07\\
0.03540354	2.638154e-07\\
0.03550355	2.711964e-07\\
0.03560356	1.753118e-07\\
0.03570357	1.926873e-07\\
0.03580358	1.578203e-07\\
0.03590359	2.327435e-07\\
0.0360036	2.896105e-07\\
0.03610361	2.78051e-07\\
0.03620362	3.552895e-07\\
0.03630363	3.131295e-07\\
0.03640364	3.298701e-07\\
0.03650365	3.255543e-07\\
0.03660366	3.249124e-07\\
0.03670367	3.733538e-07\\
0.03680368	3.167868e-07\\
0.03690369	3.547213e-07\\
0.0370037	3.073696e-07\\
0.03710371	3.158642e-07\\
0.03720372	3.362892e-07\\
0.03730373	3.588019e-07\\
0.03740374	4.393129e-07\\
0.03750375	4.242126e-07\\
0.03760376	4.63615e-07\\
0.03770377	4.211817e-07\\
0.03780378	3.760991e-07\\
0.03790379	3.817974e-07\\
0.0380038	3.364217e-07\\
0.03810381	3.767953e-07\\
0.03820382	3.230795e-07\\
0.03830383	3.305821e-07\\
0.03840384	2.4963e-07\\
0.03850385	1.912036e-07\\
0.03860386	1.821907e-07\\
0.03870387	1.346871e-07\\
0.03880388	1.509056e-07\\
0.03890389	9.047388e-08\\
0.0390039	1.135723e-07\\
0.03910391	2.503501e-08\\
0.03920392	1.161491e-08\\
0.03930393	3.464169e-08\\
0.03940394	3.466547e-08\\
0.03950395	1.344507e-07\\
0.03960396	1.457458e-07\\
0.03970397	2.08868e-07\\
0.03980398	1.863657e-07\\
0.03990399	2.377245e-07\\
0.040004	2.132105e-07\\
0.04010401	2.398946e-07\\
0.04020402	2.538198e-07\\
0.04030403	1.912025e-07\\
0.04040404	1.836067e-07\\
0.04050405	9.166482e-08\\
0.04060406	1.089842e-07\\
0.04070407	6.585708e-08\\
0.04080408	1.46527e-07\\
0.04090409	1.339634e-07\\
0.0410041	1.53155e-07\\
0.04110411	1.261063e-07\\
0.04120412	5.055049e-08\\
0.04130413	-4.179598e-09\\
0.04140414	-3.189786e-08\\
0.04150415	-1.893114e-08\\
0.04160416	-1.009943e-07\\
0.04170417	-8.490165e-08\\
0.04180418	-1.61143e-07\\
0.04190419	-1.693819e-07\\
0.0420042	-1.979923e-07\\
0.04210421	-1.665448e-07\\
0.04220422	-1.854865e-07\\
0.04230423	-1.994825e-07\\
0.04240424	-2.233014e-07\\
0.04250425	-2.894615e-07\\
0.04260426	-2.195196e-07\\
0.04270427	-2.297012e-07\\
0.04280428	-1.263376e-07\\
0.04290429	-1.031107e-07\\
0.0430043	-4.758348e-08\\
0.04310431	-1.027911e-07\\
0.04320432	-5.277623e-08\\
0.04330433	-1.118327e-07\\
0.04340434	-7.152244e-08\\
0.04350435	-1.013036e-07\\
0.04360436	-9.264474e-08\\
0.04370437	-1.469303e-07\\
0.04380438	-1.549731e-07\\
0.04390439	-1.769999e-07\\
0.0440044	-1.581555e-07\\
0.04410441	-1.016395e-07\\
0.04420442	-1.271452e-07\\
0.04430443	-1.25959e-07\\
0.04440444	-1.752122e-07\\
0.04450445	-1.98792e-07\\
0.04460446	-2.248696e-07\\
0.04470447	-1.875114e-07\\
0.04480448	-2.208279e-07\\
0.04490449	-1.918402e-07\\
0.0450045	-2.527304e-07\\
0.04510451	-1.962941e-07\\
0.04520452	-1.73808e-07\\
0.04530453	-1.175339e-07\\
0.04540454	-1.192112e-07\\
0.04550455	-1.017353e-07\\
0.04560456	-1.890393e-07\\
0.04570457	-1.770109e-07\\
0.04580458	-2.226574e-07\\
0.04590459	-1.893429e-07\\
0.0460046	-1.876474e-07\\
0.04610461	-1.603804e-07\\
0.04620462	-2.333909e-07\\
0.04630463	-2.292432e-07\\
0.04640464	-2.563467e-07\\
0.04650465	-2.434256e-07\\
0.04660466	-2.297212e-07\\
0.04670467	-2.304629e-07\\
0.04680468	-2.321645e-07\\
0.04690469	-1.74152e-07\\
0.0470047	-1.266339e-07\\
0.04710471	-1.465228e-08\\
0.04720472	4.322041e-08\\
0.04730473	7.483068e-08\\
0.04740474	7.067961e-08\\
0.04750475	2.974654e-08\\
0.04760476	3.873563e-08\\
0.04770477	6.868053e-08\\
0.04780478	4.650964e-08\\
0.04790479	5.704276e-08\\
0.0480048	1.837556e-08\\
0.04810481	3.663961e-09\\
0.04820482	1.021011e-08\\
0.04830483	2.838401e-09\\
0.04840484	4.24897e-08\\
0.04850485	-4.430329e-09\\
0.04860486	-4.711177e-08\\
0.04870487	-8.760003e-08\\
0.04880488	-9.254141e-08\\
0.04890489	-7.790492e-08\\
0.0490049	-7.278903e-08\\
0.04910491	-6.728857e-08\\
0.04920492	-9.178895e-08\\
0.04930493	-1.042221e-07\\
0.04940494	-6.77568e-08\\
0.04950495	-5.238463e-08\\
0.04960496	-4.782542e-08\\
0.04970497	-3.854678e-08\\
0.04980498	-4.963744e-08\\
0.04990499	6.419516e-09\\
0.050005	9.922197e-08\\
0.05010501	1.413715e-07\\
0.05020502	2.150728e-07\\
0.05030503	1.771837e-07\\
0.05040504	1.654133e-07\\
0.05050505	1.458237e-07\\
0.05060506	1.489881e-07\\
0.05070507	1.734627e-07\\
0.05080508	1.658739e-07\\
0.05090509	1.609585e-07\\
0.0510051	1.538903e-07\\
0.05110511	1.665548e-07\\
0.05120512	1.84078e-07\\
0.05130513	1.631214e-07\\
0.05140514	1.236419e-07\\
0.05150515	7.73741e-08\\
0.05160516	5.52394e-08\\
0.05170517	6.532422e-08\\
0.05180518	7.451949e-08\\
0.05190519	6.878468e-08\\
0.0520052	6.12384e-08\\
0.05210521	5.117977e-08\\
0.05220522	7.0091e-08\\
0.05230523	8.450641e-08\\
0.05240524	9.869274e-08\\
0.05250525	7.551896e-08\\
0.05260526	5.520211e-08\\
0.05270527	5.243684e-08\\
0.05280528	8.018725e-08\\
0.05290529	1.10022e-07\\
0.0530053	9.029747e-08\\
0.05310531	9.108492e-08\\
0.05320532	1.263385e-08\\
0.05330533	-1.570411e-08\\
0.05340534	2.171162e-08\\
0.05350535	4.547276e-08\\
0.05360536	7.677124e-08\\
0.05370537	8.257332e-08\\
0.05380538	1.454028e-07\\
0.05390539	1.72855e-07\\
0.0540054	1.856027e-07\\
0.05410541	1.704932e-07\\
0.05420542	9.873322e-08\\
0.05430543	6.81901e-08\\
0.05440544	3.538539e-08\\
0.05450545	6.278396e-08\\
0.05460546	5.474942e-08\\
0.05470547	5.034832e-08\\
0.05480548	3.988585e-08\\
0.05490549	4.924143e-08\\
0.0550055	9.553496e-08\\
0.05510551	1.073159e-07\\
0.05520552	9.967875e-08\\
0.05530553	8.950789e-08\\
0.05540554	7.355345e-08\\
0.05550555	8.515384e-08\\
0.05560556	1.041528e-07\\
0.05570557	7.028525e-08\\
0.05580558	1.686308e-08\\
0.05590559	-2.887212e-08\\
0.0560056	-8.48783e-08\\
0.05610561	-7.345239e-08\\
0.05620562	-9.43413e-08\\
0.05630563	-1.10823e-07\\
0.05640564	-1.237434e-07\\
0.05650565	-1.055569e-07\\
0.05660566	-8.761673e-08\\
0.05670567	-1.121458e-07\\
0.05680568	-1.507359e-07\\
0.05690569	-1.691352e-07\\
0.0570057	-1.417116e-07\\
0.05710571	-6.764308e-08\\
0.05720572	-1.494101e-09\\
0.05730573	4.424911e-08\\
0.05740574	7.034509e-08\\
0.05750575	1.018113e-07\\
0.05760576	1.186388e-07\\
0.05770577	1.15621e-07\\
0.05780578	9.898398e-08\\
0.05790579	5.18538e-08\\
0.0580058	8.167713e-09\\
0.05810581	1.509452e-08\\
0.05820582	4.147377e-08\\
0.05830583	2.348268e-08\\
0.05840584	-3.513482e-08\\
0.05850585	-3.253197e-08\\
0.05860586	-6.017343e-08\\
0.05870587	-5.110548e-08\\
0.05880588	-8.524964e-08\\
0.05890589	-1.145182e-07\\
0.0590059	-1.298341e-07\\
0.05910591	-1.388686e-07\\
0.05920592	-1.34085e-07\\
0.05930593	-1.715692e-07\\
0.05940594	-2.040176e-07\\
0.05950595	-2.360034e-07\\
0.05960596	-1.480132e-07\\
0.05970597	-9.158613e-08\\
0.05980598	-4.294559e-08\\
0.05990599	-4.02685e-08\\
0.060006	-1.31726e-08\\
0.06010601	-3.268652e-08\\
0.06020602	-2.904423e-09\\
0.06030603	-3.738244e-08\\
0.06040604	-6.436684e-08\\
0.06050605	-8.066547e-08\\
0.06060606	-5.311582e-08\\
0.06070607	-1.429305e-08\\
0.06080608	-1.38766e-09\\
0.06090609	-2.692805e-08\\
0.0610061	-2.787263e-08\\
0.06110611	-3.602892e-08\\
0.06120612	-1.279103e-09\\
0.06130613	-1.647321e-08\\
0.06140614	-2.954723e-08\\
0.06150615	-8.777738e-09\\
0.06160616	3.935372e-08\\
0.06170617	3.181974e-08\\
0.06180618	-7.985334e-09\\
0.06190619	-5.33028e-08\\
0.0620062	-4.718541e-08\\
0.06210621	-2.731152e-08\\
0.06220622	8.913055e-09\\
0.06230623	-7.682336e-09\\
0.06240624	-3.609746e-08\\
0.06250625	-4.522718e-08\\
0.06260626	-2.979285e-08\\
0.06270627	-4.984438e-08\\
0.06280628	-6.850338e-08\\
0.06290629	-6.617218e-08\\
0.0630063	-4.548614e-08\\
0.06310631	2.157294e-08\\
0.06320632	1.306443e-08\\
0.06330633	8.561672e-09\\
0.06340634	-3.683835e-08\\
0.06350635	-5.497174e-08\\
0.06360636	-4.988912e-08\\
0.06370637	-6.663559e-08\\
0.06380638	-1.013557e-07\\
0.06390639	-9.158579e-08\\
0.0640064	-4.658065e-08\\
0.06410641	-8.717729e-09\\
0.06420642	-2.581964e-08\\
0.06430643	-2.060488e-08\\
0.06440644	2.853185e-08\\
0.06450645	9.969151e-08\\
0.06460646	1.276653e-07\\
0.06470647	1.458605e-07\\
0.06480648	1.017189e-07\\
0.06490649	1.056753e-07\\
0.0650065	8.980017e-08\\
0.06510651	7.783349e-08\\
0.06520652	3.05941e-08\\
0.06530653	5.01924e-08\\
0.06540654	8.669825e-08\\
0.06550655	1.130681e-07\\
0.06560656	1.010172e-07\\
0.06570657	7.225424e-08\\
0.06580658	3.009542e-08\\
0.06590659	2.848907e-08\\
0.0660066	-5.419274e-09\\
0.06610661	-3.88518e-08\\
0.06620662	-4.973479e-08\\
0.06630663	-7.075943e-09\\
0.06640664	-9.149918e-10\\
0.06650665	-2.371143e-08\\
0.06660666	-7.435408e-08\\
0.06670667	-8.859649e-08\\
0.06680668	-6.263959e-08\\
0.06690669	-4.138489e-08\\
0.0670067	-4.780595e-08\\
0.06710671	-6.319075e-08\\
0.06720672	-5.147452e-08\\
0.06730673	1.329505e-08\\
0.06740674	3.255989e-08\\
0.06750675	5.586525e-08\\
0.06760676	8.649213e-08\\
0.06770677	1.534975e-07\\
0.06780678	1.843574e-07\\
0.06790679	1.54533e-07\\
0.0680068	1.197795e-07\\
0.06810681	9.883115e-08\\
0.06820682	1.058808e-07\\
0.06830683	8.288395e-08\\
0.06840684	8.144159e-08\\
0.06850685	9.839606e-08\\
0.06860686	1.086832e-07\\
0.06870687	7.932861e-08\\
0.06880688	2.032613e-08\\
0.06890689	-5.996249e-08\\
0.0690069	-6.998199e-08\\
0.06910691	-5.737809e-08\\
0.06920692	-6.308564e-08\\
0.06930693	-8.374193e-08\\
0.06940694	-6.855976e-08\\
0.06950695	-3.148911e-08\\
0.06960696	1.682524e-09\\
0.06970697	4.721902e-10\\
0.06980698	1.535307e-08\\
0.06990699	3.731147e-08\\
0.070007	8.74103e-08\\
0.07010701	7.291368e-08\\
0.07020702	2.005497e-08\\
0.07030703	-1.452661e-08\\
0.07040704	-1.452452e-08\\
0.07050705	-3.924968e-08\\
0.07060706	-4.882059e-08\\
0.07070707	-4.129289e-08\\
0.07080708	-4.912122e-09\\
0.07090709	-1.644187e-08\\
0.0710071	-5.169699e-08\\
0.07110711	-6.857542e-08\\
0.07120712	-3.190905e-08\\
0.07130713	1.909609e-08\\
0.07140714	2.38675e-08\\
0.07150715	2.813848e-08\\
0.07160716	4.092417e-08\\
0.07170717	8.197824e-08\\
0.07180718	4.388867e-08\\
0.07190719	3.032613e-08\\
0.0720072	6.594911e-10\\
0.07210721	2.648934e-08\\
0.07220722	5.235432e-08\\
0.07230723	1.254722e-08\\
0.07240724	-1.417639e-08\\
0.07250725	-5.720304e-09\\
0.07260726	2.386845e-08\\
0.07270727	4.70419e-08\\
0.07280728	3.636504e-08\\
0.07290729	5.5132e-08\\
0.0730073	3.15819e-08\\
0.07310731	-4.400884e-08\\
0.07320732	-1.158615e-07\\
0.07330733	-1.343683e-07\\
0.07340734	-1.193194e-07\\
0.07350735	-9.610749e-08\\
0.07360736	-1.001072e-07\\
0.07370737	-7.881797e-08\\
0.07380738	-3.881238e-08\\
0.07390739	-4.92292e-08\\
0.0740074	-8.996289e-08\\
0.07410741	-8.526366e-08\\
0.07420742	-6.24728e-08\\
0.07430743	-1.860463e-08\\
0.07440744	-2.968339e-08\\
0.07450745	-4.124562e-08\\
0.07460746	-2.018543e-08\\
0.07470747	1.207136e-08\\
0.07480748	5.321241e-08\\
0.07490749	3.428973e-08\\
0.0750075	6.988152e-08\\
0.07510751	7.71609e-08\\
0.07520752	6.052993e-08\\
0.07530753	2.619568e-08\\
0.07540754	4.709178e-08\\
0.07550755	6.17545e-08\\
0.07560756	7.551064e-08\\
0.07570757	5.438516e-08\\
0.07580758	1.880541e-08\\
0.07590759	1.570204e-08\\
0.0760076	-2.70687e-08\\
0.07610761	-6.382021e-08\\
0.07620762	-6.959256e-08\\
0.07630763	-3.000175e-08\\
0.07640764	-1.308629e-08\\
0.07650765	-6.82803e-08\\
0.07660766	-9.997757e-08\\
0.07670767	-8.617592e-08\\
0.07680768	-1.037102e-07\\
0.07690769	-1.154287e-07\\
0.0770077	-1.09922e-07\\
0.07710771	-7.90751e-08\\
0.07720772	-4.437716e-08\\
0.07730773	-4.072311e-08\\
0.07740774	-2.392344e-08\\
0.07750775	2.953044e-08\\
0.07760776	7.762502e-08\\
0.07770777	9.690504e-08\\
0.07780778	8.383161e-08\\
0.07790779	5.602409e-08\\
0.0780078	1.839133e-08\\
0.07810781	-3.794985e-08\\
0.07820782	-5.43854e-08\\
0.07830783	1.293298e-08\\
0.07840784	4.087476e-08\\
0.07850785	5.284726e-08\\
0.07860786	2.646773e-08\\
0.07870787	6.813582e-08\\
0.07880788	6.961611e-08\\
0.07890789	3.798957e-08\\
0.0790079	1.177477e-08\\
0.07910791	1.954498e-08\\
0.07920792	1.540949e-08\\
0.07930793	-3.105858e-08\\
0.07940794	-5.950611e-08\\
0.07950795	-3.337688e-08\\
0.07960796	2.557263e-09\\
0.07970797	-8.900234e-09\\
0.07980798	-1.568427e-09\\
0.07990799	-1.784209e-09\\
0.080008	-1.592946e-09\\
0.08010801	3.595307e-09\\
0.08020802	2.715763e-08\\
0.08030803	6.323831e-08\\
0.08040804	9.013469e-08\\
0.08050805	1.770283e-08\\
0.08060806	-4.314303e-09\\
0.08070807	-3.750111e-08\\
0.08080808	-5.737714e-08\\
0.08090809	-6.333381e-08\\
0.0810081	-5.459452e-08\\
0.08110811	-6.512733e-09\\
0.08120812	-2.053954e-08\\
0.08130813	-6.847525e-08\\
0.08140814	-7.441557e-08\\
0.08150815	-3.970428e-08\\
0.08160816	-1.361147e-08\\
0.08170817	-3.781877e-08\\
0.08180818	5.406096e-09\\
0.08190819	5.983385e-08\\
0.0820082	8.262284e-08\\
0.08210821	1.004684e-07\\
0.08220822	1.465216e-07\\
0.08230823	1.737928e-07\\
0.08240824	1.473938e-07\\
0.08250825	1.011541e-07\\
0.08260826	5.446353e-08\\
0.08270827	3.15635e-08\\
0.08280828	-5.183291e-09\\
0.08290829	-1.393994e-08\\
0.0830083	4.390744e-08\\
0.08310831	1.004793e-08\\
0.08320832	-5.362518e-08\\
0.08330833	-3.950741e-08\\
0.08340834	-3.096383e-08\\
0.08350835	-2.288013e-08\\
0.08360836	-8.525803e-08\\
0.08370837	-6.686739e-08\\
0.08380838	-6.004008e-08\\
0.08390839	-7.419202e-08\\
0.0840084	-7.949824e-08\\
0.08410841	-3.771885e-08\\
0.08420842	-3.424661e-09\\
0.08430843	-1.311282e-08\\
0.08440844	-3.412684e-08\\
0.08450845	-3.262225e-08\\
0.08460846	7.848314e-10\\
0.08470847	3.23838e-08\\
0.08480848	5.373923e-08\\
0.08490849	6.592794e-08\\
0.0850085	4.79856e-08\\
0.08510851	-8.643638e-09\\
0.08520852	-1.33864e-08\\
0.08530853	1.587402e-08\\
0.08540854	5.093225e-08\\
0.08550855	4.3166e-08\\
0.08560856	9.51752e-08\\
0.08570857	1.180338e-07\\
0.08580858	5.605798e-08\\
0.08590859	4.171234e-08\\
0.0860086	4.543617e-08\\
0.08610861	4.348678e-08\\
0.08620862	-2.281451e-08\\
0.08630863	-6.477512e-08\\
0.08640864	-5.050483e-08\\
0.08650865	-3.694171e-08\\
0.08660866	-5.875092e-08\\
0.08670867	-4.423446e-08\\
0.08680868	-2.987958e-08\\
0.08690869	-5.553748e-08\\
0.0870087	-7.905774e-08\\
0.08710871	-5.198065e-08\\
0.08720872	-1.704137e-08\\
0.08730873	-2.725168e-09\\
0.08740874	4.924475e-09\\
0.08750875	1.149356e-08\\
0.08760876	-5.761155e-09\\
0.08770877	-3.840698e-08\\
0.08780878	2.764068e-09\\
0.08790879	1.676062e-08\\
0.0880088	-5.582811e-09\\
0.08810881	-2.896462e-08\\
0.08820882	-1.41571e-08\\
0.08830883	8.302503e-09\\
0.08840884	-2.450707e-08\\
0.08850885	-1.264595e-09\\
0.08860886	2.897244e-08\\
0.08870887	-6.707355e-09\\
0.08880888	-4.86661e-08\\
0.08890889	-6.931779e-09\\
0.0890089	6.039488e-08\\
0.08910891	5.49904e-08\\
0.08920892	4.72304e-08\\
0.08930893	1.710309e-08\\
0.08940894	-2.429106e-08\\
0.08950895	-7.162683e-08\\
0.08960896	-3.371698e-08\\
0.08970897	1.151205e-08\\
0.08980898	3.604417e-08\\
0.08990899	4.950218e-08\\
0.090009	7.126525e-08\\
0.09010901	6.714808e-08\\
0.09020902	5.753675e-08\\
0.09030903	3.657864e-08\\
0.09040904	3.357935e-08\\
0.09050905	-4.848115e-08\\
0.09060906	-9.695199e-08\\
0.09070907	-6.122878e-08\\
0.09080908	-6.037144e-08\\
0.09090909	-7.0453e-08\\
0.0910091	-9.021031e-08\\
0.09110911	-5.438695e-08\\
0.09120912	-8.301848e-08\\
0.09130913	-7.636118e-08\\
0.09140914	-4.191711e-08\\
0.09150915	-4.131349e-09\\
0.09160916	-2.467437e-08\\
0.09170917	-3.851259e-08\\
0.09180918	-8.0488e-09\\
0.09190919	-5.803076e-09\\
0.0920092	2.257557e-10\\
0.09210921	6.515804e-08\\
0.09220922	6.945955e-08\\
0.09230923	5.438413e-08\\
0.09240924	6.878165e-08\\
0.09250925	1.016777e-07\\
0.09260926	9.201907e-08\\
0.09270927	6.124003e-08\\
0.09280928	6.483949e-08\\
0.09290929	3.187228e-08\\
0.0930093	1.882897e-08\\
0.09310931	3.269604e-08\\
0.09320932	7.735972e-08\\
0.09330933	4.877619e-08\\
0.09340934	-3.40963e-08\\
0.09350935	-6.425637e-08\\
0.09360936	-9.360866e-08\\
0.09370937	-1.196749e-07\\
0.09380938	-9.064115e-08\\
0.09390939	-5.084509e-08\\
0.0940094	-4.698412e-08\\
0.09410941	-6.33226e-08\\
0.09420942	-1.536416e-08\\
0.09430943	-1.591181e-08\\
0.09440944	-3.334814e-08\\
0.09450945	-6.928062e-09\\
0.09460946	-2.1969e-08\\
0.09470947	-1.151988e-08\\
0.09480948	1.908135e-08\\
0.09490949	8.00493e-08\\
0.0950095	7.280235e-08\\
0.09510951	3.072221e-08\\
0.09520952	3.588597e-08\\
0.09530953	3.597295e-08\\
0.09540954	1.190855e-09\\
0.09550955	9.285919e-09\\
0.09560956	1.993118e-08\\
0.09570957	-3.10286e-08\\
0.09580958	-6.109858e-08\\
0.09590959	-3.100978e-08\\
0.0960096	1.680061e-08\\
0.09610961	2.022928e-08\\
0.09620962	4.616522e-08\\
0.09630963	8.157538e-08\\
0.09640964	8.612698e-08\\
0.09650965	5.701174e-08\\
0.09660966	8.764874e-08\\
0.09670967	2.306514e-08\\
0.09680968	-4.124535e-08\\
0.09690969	-5.367726e-08\\
0.0970097	-7.47785e-08\\
0.09710971	-5.040071e-08\\
0.09720972	-6.024125e-09\\
0.09730973	1.051824e-08\\
0.09740974	-3.419802e-08\\
0.09750975	-3.42001e-08\\
0.09760976	-3.475752e-09\\
0.09770977	7.869806e-09\\
0.09780978	4.672597e-09\\
0.09790979	3.032422e-08\\
0.0980098	5.338966e-09\\
0.09810981	-2.706148e-08\\
0.09820982	-2.306645e-08\\
0.09830983	1.480457e-09\\
0.09840984	-3.666464e-08\\
0.09850985	-4.885994e-08\\
0.09860986	-2.143046e-08\\
0.09870987	-1.128465e-08\\
0.09880988	-1.344175e-08\\
0.09890989	-2.511942e-09\\
0.0990099	-7.662405e-09\\
0.09910991	-3.428647e-08\\
0.09920992	5.112864e-09\\
0.09930993	3.24955e-08\\
0.09940994	2.674005e-08\\
0.09950995	5.961886e-08\\
0.09960996	5.162246e-08\\
0.09970997	1.657078e-08\\
0.09980998	3.089459e-08\\
0.09990999	7.115336e-08\\
0.10001	6.442957e-08\\
0.10011	4.724464e-08\\
0.10021	9.006987e-08\\
0.10031	6.778542e-08\\
0.10041	2.844377e-08\\
0.1005101	1.361853e-08\\
0.1006101	-2.281432e-08\\
0.1007101	-1.06218e-07\\
0.1008101	-1.212179e-07\\
0.1009101	-5.710363e-08\\
0.1010101	-3.877093e-08\\
0.1011101	-4.452517e-08\\
0.1012101	-1.867396e-08\\
0.1013101	-5.333292e-08\\
0.1014101	-8.49973e-08\\
0.1015102	-6.009071e-08\\
0.1016102	-5.266348e-08\\
0.1017102	-5.593942e-08\\
0.1018102	-4.077248e-08\\
0.1019102	-2.46762e-08\\
0.1020102	1.062924e-09\\
0.1021102	5.263473e-08\\
0.1022102	6.539406e-08\\
0.1023102	4.586051e-08\\
0.1024102	6.301216e-08\\
0.1025103	9.069492e-08\\
0.1026103	5.614721e-08\\
0.1027103	2.05061e-08\\
0.1028103	3.765795e-08\\
0.1029103	2.073421e-09\\
0.1030103	-4.75717e-09\\
0.1031103	6.156856e-08\\
0.1032103	6.674018e-08\\
0.1033103	2.420519e-08\\
0.1034103	2.335354e-08\\
0.1035104	-1.470869e-08\\
0.1036104	-4.474167e-08\\
0.1037104	-2.903371e-08\\
0.1038104	-5.68424e-08\\
0.1039104	-7.269215e-08\\
0.1040104	-5.638386e-08\\
0.1041104	-2.394473e-08\\
0.1042104	-4.003685e-08\\
0.1043104	-4.360747e-08\\
0.1044104	-3.061679e-08\\
0.1045105	-3.478808e-08\\
0.1046105	1.184567e-08\\
0.1047105	7.977186e-08\\
0.1048105	5.323585e-08\\
0.1049105	2.082375e-08\\
0.1050105	6.636865e-10\\
0.1051105	-2.006971e-08\\
0.1052105	-3.017551e-08\\
0.1053105	-1.425795e-09\\
0.1054105	-2.771337e-08\\
0.1055106	-5.783523e-08\\
0.1056106	-2.348038e-08\\
0.1057106	-7.589685e-09\\
0.1058106	3.155004e-09\\
0.1059106	1.402025e-08\\
0.1060106	9.132036e-09\\
0.1061106	1.990958e-08\\
0.1062106	4.05673e-08\\
0.1063106	3.402341e-08\\
0.1064106	-1.646282e-08\\
0.1065107	-2.555579e-08\\
0.1066107	-1.182094e-08\\
0.1067107	-3.462824e-09\\
0.1068107	5.753695e-08\\
0.1069107	6.894132e-08\\
0.1070107	4.421227e-08\\
0.1071107	4.443748e-08\\
0.1072107	4.514607e-08\\
0.1073107	-3.106062e-09\\
0.1074107	-5.891688e-09\\
0.1075108	1.379162e-08\\
0.1076108	-1.319997e-08\\
0.1077108	4.832731e-09\\
0.1078108	2.288724e-08\\
0.1079108	-2.13374e-08\\
0.1080108	-5.795611e-08\\
0.1081108	-7.58877e-08\\
0.1082108	-6.511551e-08\\
0.1083108	-5.773319e-08\\
0.1084108	-3.939715e-08\\
0.1085109	-6.739849e-08\\
0.1086109	-6.6128e-08\\
0.1087109	-1.099115e-08\\
0.1088109	1.631285e-08\\
0.1089109	2.911488e-08\\
0.1090109	4.738434e-08\\
0.1091109	4.448822e-09\\
0.1092109	5.973483e-09\\
0.1093109	2.014387e-08\\
0.1094109	1.636698e-09\\
0.109511	-2.055315e-09\\
0.109611	2.735383e-08\\
0.109711	2.069948e-08\\
0.109811	3.513534e-08\\
0.109911	5.390782e-08\\
0.110011	-5.961804e-09\\
0.110111	-6.957278e-10\\
0.110211	5.411496e-08\\
0.110311	9.534424e-08\\
0.110411	7.971756e-08\\
0.1105111	9.942276e-08\\
0.1106111	3.43449e-08\\
0.1107111	7.424159e-09\\
0.1108111	-2.75017e-08\\
0.1109111	-6.024178e-08\\
0.1110111	-7.024857e-08\\
0.1111111	-6.718934e-08\\
0.1112111	-5.341186e-08\\
0.1113111	-3.965504e-08\\
0.1114111	-4.427416e-08\\
0.1115112	-8.517161e-08\\
0.1116112	-7.206741e-08\\
0.1117112	-4.177972e-08\\
0.1118112	-2.53517e-08\\
0.1119112	-2.912599e-08\\
0.1120112	-1.770798e-08\\
0.1121112	-3.305292e-08\\
0.1122112	2.368024e-08\\
0.1123112	8.337207e-08\\
0.1124112	9.552553e-08\\
0.1125113	6.716e-08\\
0.1126113	7.029303e-08\\
0.1127113	3.870095e-08\\
0.1128113	3.101851e-08\\
0.1129113	1.853147e-08\\
0.1130113	-4.814391e-08\\
0.1131113	-1.121814e-08\\
0.1132113	-2.805554e-09\\
0.1133113	-1.547414e-09\\
0.1134113	-4.126105e-10\\
0.1135114	-3.136742e-08\\
0.1136114	-4.17092e-08\\
0.1137114	2.36491e-08\\
0.1138114	5.050043e-08\\
0.1139114	2.199162e-08\\
0.1140114	1.953769e-08\\
0.1141114	2.711129e-09\\
0.1142114	-1.617262e-08\\
0.1143114	2.19214e-10\\
0.1144114	-2.301337e-08\\
0.1145115	-3.632664e-08\\
0.1146115	1.987215e-08\\
0.1147115	4.432168e-10\\
0.1148115	-6.575769e-09\\
0.1149115	1.441683e-08\\
0.1150115	-1.341078e-08\\
0.1151115	9.681368e-09\\
0.1152115	3.290251e-08\\
0.1153115	-2.446949e-08\\
0.1154115	-3.056909e-08\\
0.1155116	-4.558714e-08\\
0.1156116	-3.77221e-08\\
0.1157116	3.454654e-09\\
0.1158116	-1.00467e-08\\
0.1159116	-3.818637e-08\\
0.1160116	-3.904298e-08\\
0.1161116	-2.832828e-08\\
0.1162116	-6.142303e-08\\
0.1163116	-3.521373e-08\\
0.1164116	-1.052454e-08\\
0.1165117	3.321018e-08\\
0.1166117	1.16996e-07\\
0.1167117	9.56094e-08\\
0.1168117	5.16669e-08\\
0.1169117	2.041581e-08\\
0.1170117	-1.169741e-08\\
0.1171117	1.85899e-08\\
0.1172117	4.313669e-08\\
0.1173117	1.891673e-08\\
0.1174117	1.616569e-08\\
0.1175118	5.234019e-08\\
0.1176118	4.944342e-08\\
0.1177118	3.816938e-08\\
0.1178118	1.746445e-08\\
0.1179118	-2.095271e-08\\
0.1180118	-1.912693e-08\\
0.1181118	-4.471141e-08\\
0.1182118	-9.89534e-08\\
0.1183118	-8.789736e-08\\
0.1184118	-8.894175e-08\\
0.1185119	-3.470073e-08\\
0.1186119	-2.174571e-08\\
0.1187119	-4.374758e-08\\
0.1188119	-1.874248e-08\\
0.1189119	3.398023e-09\\
0.1190119	3.300713e-09\\
0.1191119	1.307119e-08\\
0.1192119	2.517619e-08\\
0.1193119	-1.08396e-08\\
0.1194119	1.243561e-08\\
0.119512	-1.119964e-08\\
0.119612	-2.083004e-08\\
0.119712	1.648288e-08\\
0.119812	5.397283e-08\\
0.119912	6.033912e-08\\
0.120012	8.184194e-08\\
0.120112	2.291192e-08\\
0.120212	-4.576115e-09\\
0.120312	3.393393e-08\\
0.120412	3.068644e-08\\
0.1205121	3.207619e-08\\
0.1206121	-1.339398e-08\\
0.1207121	-8.948925e-09\\
0.1208121	3.780049e-08\\
0.1209121	2.241455e-08\\
0.1210121	-4.964999e-10\\
0.1211121	-1.282343e-09\\
0.1212121	-4.019161e-08\\
0.1213121	-3.297989e-08\\
0.1214121	-3.510877e-08\\
0.1215122	-7.371008e-08\\
0.1216122	-3.558484e-08\\
0.1217122	4.173231e-09\\
0.1218122	7.284194e-09\\
0.1219122	-7.124874e-09\\
0.1220122	-3.701554e-08\\
0.1221122	-6.186629e-08\\
0.1222122	5.023116e-09\\
0.1223122	1.512641e-08\\
0.1224122	1.715779e-08\\
0.1225123	5.333458e-08\\
0.1226123	1.952601e-08\\
0.1227123	2.119111e-08\\
0.1228123	-1.053148e-08\\
0.1229123	-2.612763e-08\\
0.1230123	5.212859e-09\\
0.1231123	2.796424e-08\\
0.1232123	1.508491e-08\\
0.1233123	1.225128e-08\\
0.1234123	-1.086989e-08\\
0.1235124	3.087388e-08\\
0.1236124	2.683835e-08\\
0.1237124	-8.776654e-09\\
0.1238124	1.001187e-08\\
0.1239124	-3.254225e-08\\
0.1240124	-1.165082e-08\\
0.1241124	-5.288196e-09\\
0.1242124	-4.131894e-08\\
0.1243124	-8.14949e-09\\
0.1244124	3.156541e-08\\
0.1245125	1.396604e-08\\
0.1246125	3.545857e-08\\
0.1247125	6.043332e-08\\
0.1248125	8.261066e-08\\
0.1249125	1.114177e-07\\
0.1250125	7.756271e-08\\
0.1251125	1.483986e-08\\
0.1252125	-6.167842e-08\\
0.1253125	-8.225171e-08\\
0.1254125	-6.66643e-08\\
0.1255126	-7.408738e-08\\
0.1256126	-9.159074e-08\\
0.1257126	-4.77113e-08\\
0.1258126	-4.091798e-08\\
0.1259126	-1.387876e-08\\
0.1260126	-2.809622e-08\\
0.1261126	-4.089054e-08\\
0.1262126	1.767581e-08\\
0.1263126	-1.178224e-10\\
0.1264126	-1.881028e-08\\
0.1265127	1.968289e-08\\
0.1266127	2.984913e-08\\
0.1267127	6.649876e-08\\
0.1268127	8.935956e-08\\
0.1269127	2.013001e-08\\
0.1270127	4.230214e-08\\
0.1271127	2.093295e-08\\
0.1272127	1.839505e-08\\
0.1273127	4.608181e-08\\
0.1274127	2.098948e-08\\
0.1275128	4.994814e-08\\
0.1276128	1.856729e-08\\
0.1277128	-2.127982e-08\\
0.1278128	4.133847e-09\\
0.1279128	1.479014e-08\\
0.1280128	5.09529e-08\\
0.1281128	3.499194e-08\\
0.1282128	-4.373667e-08\\
0.1283128	-5.265761e-08\\
0.1284128	-4.277155e-08\\
0.1285129	-5.317617e-08\\
0.1286129	-2.620945e-08\\
0.1287129	-5.350563e-08\\
0.1288129	-1.146575e-08\\
0.1289129	-1.099808e-08\\
0.1290129	-4.870554e-08\\
0.1291129	-3.015126e-08\\
0.1292129	-1.821957e-08\\
0.1293129	3.014238e-09\\
0.1294129	3.403376e-08\\
0.129513	1.590916e-08\\
0.129613	3.276154e-08\\
0.129713	5.120886e-08\\
0.129813	5.04493e-08\\
0.129913	3.672925e-08\\
0.130013	-2.327007e-08\\
0.130113	-6.203266e-09\\
0.130213	8.530956e-09\\
0.130313	1.446896e-08\\
0.130413	3.22019e-08\\
0.1305131	4.37652e-08\\
0.1306131	4.27899e-08\\
0.1307131	-1.61196e-09\\
0.1308131	-5.440729e-08\\
0.1309131	-1.149003e-08\\
0.1310131	-4.447691e-09\\
0.1311131	1.012104e-08\\
0.1312131	3.425734e-08\\
0.1313131	-6.81148e-09\\
0.1314131	3.867913e-08\\
0.1315132	5.00812e-08\\
0.1316132	-1.379627e-09\\
0.1317132	2.036555e-08\\
0.1318132	-1.875234e-08\\
0.1319132	-1.918374e-08\\
0.1320132	-3.301517e-08\\
0.1321132	-4.986528e-08\\
0.1322132	-9.380462e-09\\
0.1323132	1.497306e-08\\
0.1324132	-1.406378e-10\\
0.1325133	-1.602398e-08\\
0.1326133	9.254846e-10\\
0.1327133	5.675108e-08\\
0.1328133	4.711865e-08\\
0.1329133	5.486867e-10\\
0.1330133	-2.130623e-08\\
0.1331133	-5.935279e-08\\
0.1332133	-7.260609e-09\\
0.1333133	-3.277071e-08\\
0.1334133	-1.164484e-08\\
0.1335134	3.861849e-08\\
0.1336134	1.111267e-08\\
0.1337134	1.609518e-08\\
0.1338134	-3.212044e-08\\
0.1339134	1.480044e-08\\
0.1340134	5.884827e-08\\
0.1341134	2.176763e-08\\
0.1342134	3.89219e-08\\
0.1343134	1.634125e-08\\
0.1344134	1.453326e-08\\
0.1345135	2.945358e-08\\
0.1346135	2.59095e-08\\
0.1347135	2.345722e-08\\
0.1348135	2.373606e-08\\
0.1349135	3.164743e-08\\
0.1350135	5.827394e-08\\
0.1351135	4.399344e-08\\
0.1352135	7.05004e-08\\
0.1353135	2.220513e-08\\
0.1354135	-3.070098e-08\\
0.1355136	-4.072983e-08\\
0.1356136	-6.173698e-08\\
0.1357136	-2.234419e-08\\
0.1358136	-7.008262e-08\\
0.1359136	-7.008688e-08\\
0.1360136	-4.102283e-08\\
0.1361136	-4.872231e-08\\
0.1362136	-1.49053e-08\\
0.1363136	-8.875757e-09\\
0.1364136	6.000025e-09\\
0.1365137	4.680512e-08\\
0.1366137	-8.766371e-09\\
0.1367137	3.477378e-09\\
0.1368137	2.758413e-08\\
0.1369137	5.366423e-08\\
0.1370137	7.845584e-08\\
0.1371137	3.943913e-08\\
0.1372137	5.814465e-08\\
0.1373137	6.839455e-08\\
0.1374137	6.928629e-08\\
0.1375138	1.414548e-08\\
0.1376138	-1.067737e-08\\
0.1377138	3.155321e-08\\
0.1378138	-1.961891e-09\\
0.1379138	4.138239e-08\\
0.1380138	4.652743e-08\\
0.1381138	1.674304e-08\\
0.1382138	5.49505e-09\\
0.1383138	-7.106709e-08\\
0.1384138	-7.576095e-08\\
0.1385139	-5.318477e-08\\
0.1386139	-1.790035e-08\\
0.1387139	2.340427e-08\\
0.1388139	1.777776e-08\\
0.1389139	2.552947e-08\\
0.1390139	3.251492e-08\\
0.1391139	1.244974e-08\\
0.1392139	4.209881e-08\\
0.1393139	2.780378e-08\\
0.1394139	3.151352e-08\\
0.139514	-2.720029e-08\\
0.139614	-1.402726e-08\\
0.139714	5.499949e-09\\
0.139814	-4.795261e-09\\
0.139914	2.720359e-09\\
0.140014	-3.725993e-08\\
0.140114	2.122463e-08\\
0.140214	3.046584e-08\\
0.140314	2.219426e-08\\
0.140414	3.544103e-08\\
0.1405141	6.556785e-09\\
0.1406141	4.111724e-08\\
0.1407141	-4.310647e-09\\
0.1408141	1.382446e-08\\
0.1409141	4.524291e-08\\
0.1410141	3.819415e-08\\
0.1411141	3.402063e-08\\
0.1412141	-2.01021e-08\\
0.1413141	3.059324e-08\\
0.1414141	6.14441e-08\\
0.1415142	3.429642e-08\\
0.1416142	1.574876e-09\\
0.1417142	-1.793824e-08\\
0.1418142	-1.488704e-09\\
0.1419142	-1.530812e-08\\
0.1420142	-3.183687e-09\\
0.1421142	5.811726e-08\\
0.1422142	5.551071e-08\\
0.1423142	8.159315e-08\\
0.1424142	6.436753e-08\\
0.1425143	5.763366e-08\\
0.1426143	8.900947e-09\\
0.1427143	-4.243576e-08\\
0.1428143	-6.117098e-08\\
0.1429143	-7.449473e-08\\
0.1430143	-2.186814e-08\\
0.1431143	-2.647463e-08\\
0.1432143	-1.091481e-08\\
0.1433143	1.541949e-08\\
0.1434143	-1.156033e-08\\
0.1435144	1.941472e-08\\
0.1436144	1.955732e-08\\
0.1437144	7.078341e-08\\
0.1438144	3.684113e-08\\
0.1439144	2.060099e-08\\
0.1440144	1.096931e-08\\
0.1441144	3.094192e-08\\
0.1442144	6.45772e-08\\
0.1443144	2.231682e-08\\
0.1444144	4.658015e-08\\
0.1445145	3.103268e-08\\
0.1446145	3.696946e-08\\
0.1447145	2.619505e-08\\
0.1448145	2.496746e-08\\
0.1449145	8.099194e-08\\
0.1450145	4.560352e-08\\
0.1451145	5.903076e-08\\
0.1452145	4.344378e-08\\
0.1453145	5.864377e-08\\
0.1454145	3.084055e-08\\
0.1455146	-5.501764e-08\\
0.1456146	-1.244109e-08\\
0.1457146	-3.449294e-08\\
0.1458146	2.49137e-08\\
0.1459146	8.67417e-09\\
0.1460146	-1.15548e-08\\
0.1461146	2.76383e-10\\
0.1462146	-3.301989e-08\\
0.1463146	-3.994855e-08\\
0.1464146	-3.803886e-10\\
0.1465147	1.861666e-08\\
0.1466147	1.796107e-08\\
0.1467147	1.016814e-08\\
0.1468147	2.734098e-08\\
0.1469147	3.489253e-08\\
0.1470147	4.658291e-08\\
0.1471147	5.023097e-08\\
0.1472147	9.523733e-08\\
0.1473147	7.636791e-08\\
0.1474147	8.437571e-08\\
0.1475148	6.586257e-08\\
0.1476148	4.083528e-08\\
0.1477148	3.157157e-08\\
0.1478148	-3.229896e-08\\
0.1479148	2.061002e-08\\
0.1480148	2.618009e-08\\
0.1481148	9.957718e-09\\
0.1482148	1.622457e-10\\
0.1483148	5.284109e-09\\
0.1484148	7.071272e-08\\
0.1485149	3.26924e-08\\
0.1486149	2.381093e-08\\
0.1487149	1.609525e-08\\
0.1488149	5.496666e-09\\
0.1489149	-1.092038e-08\\
0.1490149	1.438869e-09\\
0.1491149	6.286229e-08\\
0.1492149	7.726495e-08\\
0.1493149	7.974596e-08\\
0.1494149	3.404039e-08\\
0.149515	5.347577e-08\\
0.149615	3.556945e-08\\
0.149715	1.302172e-08\\
0.149815	-1.23059e-08\\
0.149915	1.900548e-08\\
0.150015	3.964713e-08\\
0.150115	6.875565e-09\\
0.150215	3.427493e-08\\
0.150315	2.194281e-09\\
0.150415	5.33641e-09\\
0.1505151	-1.259671e-08\\
0.1506151	1.716127e-08\\
0.1507151	6.909889e-08\\
0.1508151	1.947306e-08\\
0.1509151	2.03602e-08\\
0.1510151	2.393234e-09\\
0.1511151	5.108469e-08\\
0.1512151	4.832503e-08\\
0.1513151	8.962969e-08\\
0.1514151	9.621689e-08\\
0.1515152	8.323596e-08\\
0.1516152	5.886448e-08\\
0.1517152	5.916475e-08\\
0.1518152	9.333088e-08\\
0.1519152	2.511299e-08\\
0.1520152	1.821113e-08\\
0.1521152	2.618446e-08\\
0.1522152	5.094518e-08\\
0.1523152	1.759013e-08\\
0.1524152	1.791238e-08\\
0.1525153	5.456751e-08\\
0.1526153	4.530064e-08\\
0.1527153	6.487464e-08\\
0.1528153	3.808041e-08\\
0.1529153	3.877911e-08\\
0.1530153	-1.782035e-08\\
0.1531153	-8.55815e-09\\
0.1532153	2.681802e-09\\
0.1533153	4.893439e-08\\
0.1534153	4.85835e-08\\
0.1535154	-3.916786e-09\\
0.1536154	2.369227e-08\\
0.1537154	7.917009e-09\\
0.1538154	3.494588e-08\\
0.1539154	2.596951e-08\\
0.1540154	7.762755e-08\\
0.1541154	7.605072e-08\\
0.1542154	8.41353e-08\\
0.1543154	7.354166e-08\\
0.1544154	7.171656e-08\\
0.1545155	3.580274e-08\\
0.1546155	1.03667e-08\\
0.1547155	5.780652e-08\\
0.1548155	7.754668e-08\\
0.1549155	1.164319e-07\\
0.1550155	8.639459e-08\\
0.1551155	1.07563e-07\\
0.1552155	8.06201e-08\\
0.1553155	8.538958e-08\\
0.1554155	6.329275e-08\\
0.1555156	2.27297e-08\\
0.1556156	1.026857e-08\\
0.1557156	-4.573952e-08\\
0.1558156	1.302705e-08\\
0.1559156	3.36755e-08\\
0.1560156	4.211266e-08\\
0.1561156	1.452006e-08\\
0.1562156	5.426388e-08\\
0.1563156	5.400029e-08\\
0.1564156	6.576669e-08\\
0.1565157	5.228751e-08\\
0.1566157	8.89031e-08\\
0.1567157	6.565746e-08\\
0.1568157	4.541133e-08\\
0.1569157	6.697994e-08\\
0.1570157	4.499796e-08\\
0.1571157	5.299131e-08\\
0.1572157	2.215776e-08\\
0.1573157	9.421305e-08\\
0.1574157	7.840544e-08\\
0.1575158	6.226773e-08\\
0.1576158	2.984889e-08\\
0.1577158	5.190961e-08\\
0.1578158	7.221685e-08\\
0.1579158	7.046137e-08\\
0.1580158	1.20844e-07\\
0.1581158	8.711628e-08\\
0.1582158	8.228245e-08\\
0.1583158	3.692281e-08\\
0.1584158	7.128469e-08\\
0.1585159	3.757603e-08\\
0.1586159	2.667397e-08\\
0.1587159	5.101863e-08\\
0.1588159	8.0163e-08\\
0.1589159	5.992939e-08\\
0.1590159	7.081789e-08\\
0.1591159	8.337293e-08\\
0.1592159	8.230165e-08\\
0.1593159	9.189127e-08\\
0.1594159	9.193626e-08\\
0.159516	1.182956e-07\\
0.159616	5.495063e-08\\
0.159716	8.141279e-08\\
0.159816	1.121564e-07\\
0.159916	1.247623e-07\\
0.160016	5.919689e-08\\
0.160116	3.793153e-08\\
0.160216	7.658044e-09\\
0.160316	8.204087e-09\\
0.160416	7.497644e-09\\
0.1605161	6.619771e-08\\
0.1606161	6.626422e-08\\
0.1607161	4.048742e-08\\
0.1608161	3.995988e-08\\
0.1609161	6.526207e-08\\
0.1610161	8.894138e-08\\
0.1611161	6.783724e-08\\
0.1612161	1.018606e-07\\
0.1613161	1.314236e-07\\
0.1614161	1.586051e-07\\
0.1615162	1.118517e-07\\
0.1616162	1.278998e-07\\
0.1617162	9.019694e-08\\
0.1618162	1.020233e-07\\
0.1619162	9.788555e-08\\
0.1620162	1.163323e-07\\
0.1621162	8.307478e-08\\
0.1622162	5.29074e-08\\
0.1623162	9.546073e-08\\
0.1624162	1.013225e-07\\
0.1625163	1.165735e-07\\
0.1626163	8.555315e-08\\
0.1627163	1.113182e-07\\
0.1628163	6.048054e-08\\
0.1629163	8.628873e-08\\
0.1630163	7.600539e-08\\
0.1631163	4.846364e-08\\
0.1632163	1.0745e-08\\
0.1633163	4.745679e-08\\
0.1634163	8.401491e-08\\
0.1635164	9.563383e-08\\
0.1636164	1.098359e-07\\
0.1637164	1.6091e-07\\
0.1638164	1.259557e-07\\
0.1639164	1.106894e-07\\
0.1640164	6.172913e-08\\
0.1641164	6.709833e-08\\
0.1642164	5.224042e-08\\
0.1643164	9.208176e-08\\
0.1644164	1.378489e-07\\
0.1645165	1.171625e-07\\
0.1646165	1.174581e-07\\
0.1647165	1.103723e-07\\
0.1648165	1.624243e-07\\
0.1649165	1.27194e-07\\
0.1650165	1.264485e-07\\
0.1651165	1.119841e-07\\
0.1652165	1.39635e-07\\
0.1653165	7.998106e-08\\
0.1654165	1.150034e-07\\
0.1655166	1.113505e-07\\
0.1656166	1.10484e-07\\
0.1657166	9.198587e-08\\
0.1658166	1.443488e-07\\
0.1659166	1.169151e-07\\
0.1660166	1.081956e-07\\
0.1661166	8.09158e-08\\
0.1662166	1.365668e-07\\
0.1663166	1.232511e-07\\
0.1664166	1.140483e-07\\
0.1665167	7.485262e-08\\
0.1666167	8.049071e-08\\
0.1667167	9.576705e-08\\
0.1668167	1.585798e-07\\
0.1669167	1.591684e-07\\
0.1670167	1.241644e-07\\
0.1671167	1.149869e-07\\
0.1672167	1.408909e-07\\
0.1673167	1.575534e-07\\
0.1674167	1.326943e-07\\
0.1675168	1.520712e-07\\
0.1676168	1.184694e-07\\
0.1677168	8.901521e-08\\
0.1678168	1.106704e-07\\
0.1679168	1.238771e-07\\
0.1680168	9.615481e-08\\
0.1681168	1.366645e-07\\
0.1682168	1.459655e-07\\
0.1683168	1.560992e-07\\
0.1684168	8.550861e-08\\
0.1685169	1.077051e-07\\
0.1686169	1.505023e-07\\
0.1687169	1.782781e-07\\
0.1688169	1.691801e-07\\
0.1689169	2.09223e-07\\
0.1690169	1.700226e-07\\
0.1691169	1.583477e-07\\
0.1692169	1.579044e-07\\
0.1693169	1.670796e-07\\
0.1694169	1.229413e-07\\
0.169517	1.481061e-07\\
0.169617	1.64625e-07\\
0.169717	2.01477e-07\\
0.169817	1.582282e-07\\
0.169917	1.764739e-07\\
0.170017	1.422079e-07\\
0.170117	1.358992e-07\\
0.170217	9.368633e-08\\
0.170317	9.665783e-08\\
0.170417	1.176218e-07\\
0.1705171	1.605253e-07\\
0.1706171	1.742976e-07\\
0.1707171	1.673564e-07\\
0.1708171	1.303557e-07\\
0.1709171	1.340183e-07\\
0.1710171	1.625395e-07\\
0.1711171	1.588811e-07\\
0.1712171	1.62086e-07\\
0.1713171	1.780493e-07\\
0.1714171	1.455052e-07\\
0.1715172	1.933004e-07\\
0.1716172	2.164487e-07\\
0.1717172	2.196612e-07\\
0.1718172	2.033895e-07\\
0.1719172	2.175997e-07\\
0.1720172	2.197647e-07\\
0.1721172	1.818398e-07\\
0.1722172	1.815717e-07\\
0.1723172	1.866382e-07\\
0.1724172	1.893829e-07\\
0.1725173	2.032499e-07\\
0.1726173	2.140996e-07\\
0.1727173	1.769784e-07\\
0.1728173	1.703864e-07\\
0.1729173	2.007595e-07\\
0.1730173	1.975808e-07\\
0.1731173	1.406142e-07\\
0.1732173	1.244196e-07\\
0.1733173	1.514641e-07\\
0.1734173	1.781364e-07\\
0.1735174	1.852096e-07\\
0.1736174	2.144832e-07\\
0.1737174	1.976872e-07\\
0.1738174	1.954516e-07\\
0.1739174	2.528796e-07\\
0.1740174	2.640792e-07\\
0.1741174	2.550332e-07\\
0.1742174	2.406937e-07\\
0.1743174	2.316343e-07\\
0.1744174	1.874223e-07\\
0.1745175	1.40388e-07\\
0.1746175	1.560356e-07\\
0.1747175	1.783148e-07\\
0.1748175	2.140223e-07\\
0.1749175	2.396197e-07\\
0.1750175	2.389155e-07\\
0.1751175	2.022416e-07\\
0.1752175	2.587635e-07\\
0.1753175	2.591721e-07\\
0.1754175	2.511543e-07\\
0.1755176	2.411566e-07\\
0.1756176	2.493301e-07\\
0.1757176	2.456917e-07\\
0.1758176	2.177836e-07\\
0.1759176	2.456383e-07\\
0.1760176	2.430131e-07\\
0.1761176	2.250132e-07\\
0.1762176	2.343068e-07\\
0.1763176	2.445734e-07\\
0.1764176	2.165213e-07\\
0.1765177	2.492585e-07\\
0.1766177	2.751439e-07\\
0.1767177	2.587172e-07\\
0.1768177	2.435885e-07\\
0.1769177	2.623676e-07\\
0.1770177	2.737498e-07\\
0.1771177	2.725494e-07\\
0.1772177	2.615789e-07\\
0.1773177	2.297619e-07\\
0.1774177	2.102096e-07\\
0.1775178	2.662798e-07\\
0.1776178	3.100271e-07\\
0.1777178	2.659059e-07\\
0.1778178	2.693733e-07\\
0.1779178	2.462676e-07\\
0.1780178	2.269346e-07\\
0.1781178	2.471151e-07\\
0.1782178	2.522262e-07\\
0.1783178	2.680534e-07\\
0.1784178	2.882361e-07\\
0.1785179	3.115289e-07\\
0.1786179	2.974248e-07\\
0.1787179	2.820734e-07\\
0.1788179	2.95112e-07\\
0.1789179	3.261837e-07\\
0.1790179	3.080568e-07\\
0.1791179	2.998894e-07\\
0.1792179	3.182094e-07\\
0.1793179	3.185107e-07\\
0.1794179	3.343728e-07\\
0.179518	3.405132e-07\\
0.179618	2.901748e-07\\
0.179718	3.053288e-07\\
0.179818	3.435368e-07\\
0.179918	3.63354e-07\\
0.180018	3.162618e-07\\
0.180118	3.187286e-07\\
0.180218	2.955926e-07\\
0.180318	2.577185e-07\\
0.180418	2.649499e-07\\
0.1805181	2.952908e-07\\
0.1806181	2.918588e-07\\
0.1807181	3.344834e-07\\
0.1808181	3.274858e-07\\
0.1809181	2.974253e-07\\
0.1810181	3.056156e-07\\
0.1811181	3.73934e-07\\
0.1812181	3.52194e-07\\
0.1813181	3.390077e-07\\
0.1814181	3.479702e-07\\
0.1815182	3.651648e-07\\
0.1816182	3.76064e-07\\
0.1817182	3.877461e-07\\
0.1818182	4.024433e-07\\
0.1819182	3.691485e-07\\
0.1820182	3.643461e-07\\
0.1821182	3.597258e-07\\
0.1822182	3.42048e-07\\
0.1823182	3.581466e-07\\
0.1824182	3.787707e-07\\
0.1825183	3.899346e-07\\
0.1826183	4.080892e-07\\
0.1827183	4.004601e-07\\
0.1828183	3.798308e-07\\
0.1829183	3.840782e-07\\
0.1830183	3.757167e-07\\
0.1831183	3.291118e-07\\
0.1832183	3.543483e-07\\
0.1833183	3.914626e-07\\
0.1834183	4.361271e-07\\
0.1835184	4.108575e-07\\
0.1836184	4.190499e-07\\
0.1837184	3.985448e-07\\
0.1838184	4.046829e-07\\
0.1839184	4.242155e-07\\
0.1840184	4.347627e-07\\
0.1841184	4.018223e-07\\
0.1842184	4.282763e-07\\
0.1843184	4.271569e-07\\
0.1844184	4.027481e-07\\
0.1845185	4.013368e-07\\
0.1846185	4.542745e-07\\
0.1847185	4.391133e-07\\
0.1848185	4.503284e-07\\
0.1849185	4.376485e-07\\
0.1850185	4.306097e-07\\
0.1851185	4.409423e-07\\
0.1852185	4.693163e-07\\
0.1853185	4.467944e-07\\
0.1854185	4.18027e-07\\
0.1855186	4.441394e-07\\
0.1856186	4.445686e-07\\
0.1857186	4.36935e-07\\
0.1858186	4.799102e-07\\
0.1859186	4.638886e-07\\
0.1860186	5.120106e-07\\
0.1861186	5.405722e-07\\
0.1862186	5.370934e-07\\
0.1863186	5.158275e-07\\
0.1864186	5.125908e-07\\
0.1865187	4.992175e-07\\
0.1866187	4.674986e-07\\
0.1867187	4.994287e-07\\
0.1868187	5.093822e-07\\
0.1869187	5.144651e-07\\
0.1870187	4.88306e-07\\
0.1871187	4.757633e-07\\
0.1872187	4.977243e-07\\
0.1873187	5.506347e-07\\
0.1874187	5.649829e-07\\
0.1875188	5.239377e-07\\
0.1876188	4.988903e-07\\
0.1877188	5.188555e-07\\
0.1878188	4.735395e-07\\
0.1879188	4.959559e-07\\
0.1880188	5.398089e-07\\
0.1881188	5.68773e-07\\
0.1882188	5.863785e-07\\
0.1883188	5.84906e-07\\
0.1884188	5.673386e-07\\
0.1885189	5.654925e-07\\
0.1886189	5.695459e-07\\
0.1887189	5.434069e-07\\
0.1888189	5.267263e-07\\
0.1889189	5.876263e-07\\
0.1890189	5.956754e-07\\
0.1891189	6.315318e-07\\
0.1892189	5.928487e-07\\
0.1893189	5.989247e-07\\
0.1894189	6.061216e-07\\
0.189519	6.584735e-07\\
0.189619	6.456579e-07\\
0.189719	6.217423e-07\\
0.189819	6.338227e-07\\
0.189919	6.526886e-07\\
0.190019	6.284571e-07\\
0.190119	6.503879e-07\\
0.190219	6.282382e-07\\
0.190319	6.234898e-07\\
0.190419	6.233362e-07\\
0.1905191	6.050864e-07\\
0.1906191	5.922802e-07\\
0.1907191	6.25648e-07\\
0.1908191	6.532399e-07\\
0.1909191	6.418762e-07\\
0.1910191	6.466521e-07\\
0.1911191	6.692954e-07\\
0.1912191	6.962039e-07\\
0.1913191	7.156962e-07\\
0.1914191	7.17247e-07\\
0.1915192	7.198754e-07\\
0.1916192	7.246012e-07\\
0.1917192	7.33994e-07\\
0.1918192	6.873555e-07\\
0.1919192	6.947272e-07\\
0.1920192	7.316723e-07\\
0.1921192	7.070957e-07\\
0.1922192	7.319298e-07\\
0.1923192	7.40996e-07\\
0.1924192	7.373238e-07\\
0.1925193	7.262065e-07\\
0.1926193	7.276532e-07\\
0.1927193	7.310262e-07\\
0.1928193	7.93344e-07\\
0.1929193	7.954091e-07\\
0.1930193	7.81737e-07\\
0.1931193	7.660464e-07\\
0.1932193	7.771728e-07\\
0.1933193	7.677751e-07\\
0.1934193	7.850173e-07\\
0.1935194	7.79521e-07\\
0.1936194	7.908222e-07\\
0.1937194	8.14969e-07\\
0.1938194	8.32608e-07\\
0.1939194	8.298409e-07\\
0.1940194	8.385893e-07\\
0.1941194	8.76487e-07\\
0.1942194	8.496003e-07\\
0.1943194	8.397556e-07\\
0.1944194	8.491764e-07\\
0.1945195	8.668952e-07\\
0.1946195	8.637071e-07\\
0.1947195	8.264747e-07\\
0.1948195	8.101577e-07\\
0.1949195	8.422259e-07\\
0.1950195	8.60955e-07\\
0.1951195	8.508571e-07\\
0.1952195	8.638816e-07\\
0.1953195	9.225363e-07\\
0.1954195	9.376313e-07\\
0.1955196	9.646742e-07\\
0.1956196	9.417594e-07\\
0.1957196	8.934627e-07\\
0.1958196	9.091816e-07\\
0.1959196	9.021933e-07\\
0.1960196	9.193099e-07\\
0.1961196	9.83695e-07\\
0.1962196	9.995903e-07\\
0.1963196	1.005635e-06\\
0.1964196	9.957084e-07\\
0.1965197	9.996394e-07\\
0.1966197	9.813051e-07\\
0.1967197	1.02083e-06\\
0.1968197	1.001177e-06\\
0.1969197	9.846228e-07\\
0.1970197	1.004876e-06\\
0.1971197	1.012686e-06\\
0.1972197	1.021173e-06\\
0.1973197	1.030795e-06\\
0.1974197	1.063443e-06\\
0.1975198	1.05903e-06\\
0.1976198	1.047632e-06\\
0.1977198	1.047064e-06\\
0.1978198	1.032931e-06\\
0.1979198	1.035953e-06\\
0.1980198	1.044206e-06\\
0.1981198	1.064154e-06\\
0.1982198	1.132811e-06\\
0.1983198	1.101358e-06\\
0.1984198	1.108846e-06\\
0.1985199	1.113358e-06\\
0.1986199	1.12647e-06\\
0.1987199	1.159681e-06\\
0.1988199	1.172829e-06\\
0.1989199	1.150535e-06\\
0.1990199	1.163276e-06\\
0.1991199	1.198688e-06\\
0.1992199	1.173683e-06\\
0.1993199	1.189125e-06\\
0.1994199	1.201266e-06\\
0.19952	1.181676e-06\\
0.19962	1.197931e-06\\
0.19972	1.213825e-06\\
0.19982	1.234892e-06\\
0.19992	1.272616e-06\\
0.20002	1.278601e-06\\
0.20012	1.247818e-06\\
0.20022	1.240996e-06\\
0.20032	1.277218e-06\\
0.20042	1.269502e-06\\
0.2005201	1.268948e-06\\
0.2006201	1.287406e-06\\
0.2007201	1.310193e-06\\
0.2008201	1.340733e-06\\
0.2009201	1.315123e-06\\
0.2010201	1.278911e-06\\
0.2011201	1.327861e-06\\
0.2012201	1.348967e-06\\
0.2013201	1.358449e-06\\
0.2014201	1.377541e-06\\
0.2015202	1.41245e-06\\
0.2016202	1.36542e-06\\
0.2017202	1.388637e-06\\
0.2018202	1.397295e-06\\
0.2019202	1.425571e-06\\
0.2020202	1.45058e-06\\
0.2021202	1.428994e-06\\
0.2022202	1.43515e-06\\
0.2023202	1.482868e-06\\
0.2024202	1.500399e-06\\
0.2025203	1.500094e-06\\
0.2026203	1.482239e-06\\
0.2027203	1.452601e-06\\
0.2028203	1.486598e-06\\
0.2029203	1.493732e-06\\
0.2030203	1.503633e-06\\
0.2031203	1.526876e-06\\
0.2032203	1.552959e-06\\
0.2033203	1.535852e-06\\
0.2034203	1.596522e-06\\
0.2035204	1.596655e-06\\
0.2036204	1.604344e-06\\
0.2037204	1.64244e-06\\
0.2038204	1.640062e-06\\
0.2039204	1.659798e-06\\
0.2040204	1.674236e-06\\
0.2041204	1.64393e-06\\
0.2042204	1.658577e-06\\
0.2043204	1.700377e-06\\
0.2044204	1.700904e-06\\
0.2045205	1.693164e-06\\
0.2046205	1.688361e-06\\
0.2047205	1.661286e-06\\
0.2048205	1.686262e-06\\
0.2049205	1.72317e-06\\
0.2050205	1.730174e-06\\
0.2051205	1.784428e-06\\
0.2052205	1.806028e-06\\
0.2053205	1.766429e-06\\
0.2054205	1.781011e-06\\
0.2055206	1.803416e-06\\
0.2056206	1.825053e-06\\
0.2057206	1.826394e-06\\
0.2058206	1.845527e-06\\
0.2059206	1.904381e-06\\
0.2060206	1.914206e-06\\
0.2061206	1.908488e-06\\
0.2062206	1.916048e-06\\
0.2063206	1.9527e-06\\
0.2064206	1.958643e-06\\
0.2065207	1.961228e-06\\
0.2066207	1.942441e-06\\
0.2067207	1.95567e-06\\
0.2068207	1.992446e-06\\
0.2069207	2.001805e-06\\
0.2070207	2.015749e-06\\
0.2071207	2.053389e-06\\
0.2072207	2.053284e-06\\
0.2073207	2.055116e-06\\
0.2074207	2.101043e-06\\
0.2075208	2.115244e-06\\
0.2076208	2.125401e-06\\
0.2077208	2.093534e-06\\
0.2078208	2.061831e-06\\
0.2079208	2.103233e-06\\
0.2080208	2.151816e-06\\
0.2081208	2.145689e-06\\
0.2082208	2.190769e-06\\
0.2083208	2.227112e-06\\
0.2084208	2.223109e-06\\
0.2085209	2.263487e-06\\
0.2086209	2.280027e-06\\
0.2087209	2.275653e-06\\
0.2088209	2.29384e-06\\
0.2089209	2.311519e-06\\
0.2090209	2.318802e-06\\
0.2091209	2.333902e-06\\
0.2092209	2.311918e-06\\
0.2093209	2.349766e-06\\
0.2094209	2.395055e-06\\
0.209521	2.427312e-06\\
0.209621	2.417816e-06\\
0.209721	2.421657e-06\\
0.209821	2.431117e-06\\
0.209921	2.475577e-06\\
0.210021	2.51429e-06\\
0.210121	2.528834e-06\\
0.210221	2.541798e-06\\
0.210321	2.544876e-06\\
0.210421	2.523005e-06\\
0.2105211	2.564455e-06\\
0.2106211	2.58436e-06\\
0.2107211	2.612037e-06\\
0.2108211	2.619568e-06\\
0.2109211	2.626685e-06\\
0.2110211	2.708868e-06\\
0.2111211	2.706459e-06\\
0.2112211	2.692882e-06\\
0.2113211	2.705321e-06\\
0.2114211	2.729474e-06\\
0.2115212	2.738461e-06\\
0.2116212	2.804498e-06\\
0.2117212	2.822482e-06\\
0.2118212	2.864287e-06\\
0.2119212	2.874333e-06\\
0.2120212	2.840406e-06\\
0.2121212	2.833785e-06\\
0.2122212	2.865266e-06\\
0.2123212	2.873249e-06\\
0.2124212	2.922946e-06\\
0.2125213	2.95825e-06\\
0.2126213	2.998669e-06\\
0.2127213	3.020294e-06\\
0.2128213	2.997964e-06\\
0.2129213	3.06321e-06\\
0.2130213	3.10422e-06\\
0.2131213	3.101982e-06\\
0.2132213	3.089112e-06\\
0.2133213	3.118887e-06\\
0.2134213	3.133917e-06\\
0.2135214	3.159496e-06\\
0.2136214	3.197123e-06\\
0.2137214	3.233774e-06\\
0.2138214	3.278293e-06\\
0.2139214	3.279103e-06\\
0.2140214	3.296989e-06\\
0.2141214	3.345864e-06\\
0.2142214	3.345331e-06\\
0.2143214	3.35961e-06\\
0.2144214	3.411429e-06\\
0.2145215	3.423324e-06\\
0.2146215	3.433525e-06\\
0.2147215	3.431952e-06\\
0.2148215	3.463351e-06\\
0.2149215	3.509677e-06\\
0.2150215	3.512894e-06\\
0.2151215	3.526277e-06\\
0.2152215	3.538226e-06\\
0.2153215	3.56643e-06\\
0.2154215	3.629166e-06\\
0.2155216	3.64699e-06\\
0.2156216	3.671896e-06\\
0.2157216	3.706627e-06\\
0.2158216	3.697874e-06\\
0.2159216	3.772743e-06\\
0.2160216	3.833902e-06\\
0.2161216	3.829394e-06\\
0.2162216	3.823165e-06\\
0.2163216	3.858513e-06\\
0.2164216	3.914468e-06\\
0.2165217	3.963004e-06\\
0.2166217	3.976065e-06\\
0.2167217	4.003753e-06\\
0.2168217	4.03977e-06\\
0.2169217	4.034147e-06\\
0.2170217	4.02259e-06\\
0.2171217	4.061166e-06\\
0.2172217	4.073994e-06\\
0.2173217	4.129083e-06\\
0.2174217	4.174044e-06\\
0.2175218	4.219619e-06\\
0.2176218	4.257551e-06\\
0.2177218	4.241434e-06\\
0.2178218	4.311644e-06\\
0.2179218	4.357367e-06\\
0.2180218	4.366216e-06\\
0.2181218	4.374331e-06\\
0.2182218	4.397779e-06\\
0.2183218	4.453342e-06\\
0.2184218	4.50017e-06\\
0.2185219	4.502573e-06\\
0.2186219	4.551702e-06\\
0.2187219	4.577669e-06\\
0.2188219	4.609376e-06\\
0.2189219	4.656657e-06\\
0.2190219	4.683245e-06\\
0.2191219	4.705591e-06\\
0.2192219	4.727815e-06\\
0.2193219	4.736604e-06\\
0.2194219	4.808217e-06\\
0.219522	4.850379e-06\\
0.219622	4.841345e-06\\
0.219722	4.924035e-06\\
0.219822	4.966895e-06\\
0.219922	4.98402e-06\\
0.220022	5.042194e-06\\
0.220122	5.076364e-06\\
0.220222	5.106085e-06\\
0.220322	5.138478e-06\\
0.220422	5.10867e-06\\
0.2205221	5.159555e-06\\
0.2206221	5.206037e-06\\
0.2207221	5.252709e-06\\
0.2208221	5.359831e-06\\
0.2209221	5.380408e-06\\
0.2210221	5.3952e-06\\
0.2211221	5.414751e-06\\
0.2212221	5.452548e-06\\
0.2213221	5.546524e-06\\
0.2214221	5.546125e-06\\
0.2215222	5.572394e-06\\
0.2216222	5.632217e-06\\
0.2217222	5.650348e-06\\
0.2218222	5.674929e-06\\
0.2219222	5.714133e-06\\
0.2220222	5.7787e-06\\
0.2221222	5.815647e-06\\
0.2222222	5.857053e-06\\
0.2223222	5.905596e-06\\
0.2224222	5.933719e-06\\
0.2225223	5.956326e-06\\
0.2226223	5.998573e-06\\
0.2227223	6.067107e-06\\
0.2228223	6.105245e-06\\
0.2229223	6.151047e-06\\
0.2230223	6.207992e-06\\
0.2231223	6.285102e-06\\
0.2232223	6.313028e-06\\
0.2233223	6.286825e-06\\
0.2234223	6.330498e-06\\
0.2235224	6.412272e-06\\
0.2236224	6.493858e-06\\
0.2237224	6.563653e-06\\
0.2238224	6.570442e-06\\
0.2239224	6.612711e-06\\
0.2240224	6.662217e-06\\
0.2241224	6.697415e-06\\
0.2242224	6.772755e-06\\
0.2243224	6.782874e-06\\
0.2244224	6.824536e-06\\
0.2245225	6.902601e-06\\
0.2246225	6.93869e-06\\
0.2247225	7.000988e-06\\
0.2248225	7.072123e-06\\
0.2249225	7.075073e-06\\
0.2250225	7.155246e-06\\
0.2251225	7.164308e-06\\
0.2252225	7.184156e-06\\
0.2253225	7.260475e-06\\
0.2254225	7.308598e-06\\
0.2255226	7.39632e-06\\
0.2256226	7.451194e-06\\
0.2257226	7.505278e-06\\
0.2258226	7.543051e-06\\
0.2259226	7.585207e-06\\
0.2260226	7.677131e-06\\
0.2261226	7.735716e-06\\
0.2262226	7.764766e-06\\
0.2263226	7.844193e-06\\
0.2264226	7.903917e-06\\
0.2265227	7.963818e-06\\
0.2266227	7.982128e-06\\
0.2267227	8.012182e-06\\
0.2268227	8.09616e-06\\
0.2269227	8.128866e-06\\
0.2270227	8.217932e-06\\
0.2271227	8.299145e-06\\
0.2272227	8.337693e-06\\
0.2273227	8.393396e-06\\
0.2274227	8.449136e-06\\
0.2275228	8.513491e-06\\
0.2276228	8.587788e-06\\
0.2277228	8.615185e-06\\
0.2278228	8.696503e-06\\
0.2279228	8.763203e-06\\
0.2280228	8.758015e-06\\
0.2281228	8.853167e-06\\
0.2282228	8.937274e-06\\
0.2283228	9.023307e-06\\
0.2284228	9.06577e-06\\
0.2285229	9.137623e-06\\
0.2286229	9.21868e-06\\
0.2287229	9.289206e-06\\
0.2288229	9.338568e-06\\
0.2289229	9.435298e-06\\
0.2290229	9.466626e-06\\
0.2291229	9.498241e-06\\
0.2292229	9.573257e-06\\
0.2293229	9.646402e-06\\
0.2294229	9.696319e-06\\
0.229523	9.734947e-06\\
0.229623	9.829928e-06\\
0.229723	9.92897e-06\\
0.229823	1.000764e-05\\
0.229923	1.006448e-05\\
0.230023	1.015744e-05\\
0.230123	1.023468e-05\\
0.230223	1.029693e-05\\
0.230323	1.033078e-05\\
0.230423	1.044083e-05\\
0.2305231	1.048741e-05\\
0.2306231	1.058737e-05\\
0.2307231	1.068865e-05\\
0.2308231	1.073167e-05\\
0.2309231	1.081238e-05\\
0.2310231	1.090596e-05\\
0.2311231	1.097217e-05\\
0.2312231	1.101132e-05\\
0.2313231	1.10505e-05\\
0.2314231	1.11578e-05\\
0.2315232	1.127353e-05\\
0.2316232	1.135314e-05\\
0.2317232	1.147167e-05\\
0.2318232	1.151099e-05\\
0.2319232	1.157503e-05\\
0.2320232	1.165625e-05\\
0.2321232	1.173639e-05\\
0.2322232	1.182743e-05\\
0.2323232	1.189023e-05\\
0.2324232	1.200203e-05\\
0.2325233	1.207759e-05\\
0.2326233	1.211756e-05\\
0.2327233	1.222363e-05\\
0.2328233	1.232781e-05\\
0.2329233	1.242847e-05\\
0.2330233	1.250949e-05\\
0.2331233	1.256552e-05\\
0.2332233	1.269432e-05\\
0.2333233	1.276784e-05\\
0.2334233	1.283577e-05\\
0.2335234	1.295059e-05\\
0.2336234	1.302371e-05\\
0.2337234	1.312494e-05\\
0.2338234	1.322557e-05\\
0.2339234	1.335648e-05\\
0.2340234	1.341881e-05\\
0.2341234	1.347865e-05\\
0.2342234	1.36108e-05\\
0.2343234	1.37156e-05\\
0.2344234	1.37929e-05\\
0.2345235	1.386882e-05\\
0.2346235	1.395015e-05\\
0.2347235	1.405783e-05\\
0.2348235	1.413913e-05\\
0.2349235	1.42379e-05\\
0.2350235	1.437551e-05\\
0.2351235	1.444097e-05\\
0.2352235	1.457086e-05\\
0.2353235	1.465622e-05\\
0.2354235	1.476318e-05\\
0.2355236	1.489587e-05\\
0.2356236	1.496935e-05\\
0.2357236	1.509436e-05\\
0.2358236	1.514929e-05\\
0.2359236	1.525526e-05\\
0.2360236	1.538736e-05\\
0.2361236	1.549956e-05\\
0.2362236	1.560802e-05\\
0.2363236	1.570499e-05\\
0.2364236	1.582425e-05\\
0.2365237	1.594077e-05\\
0.2366237	1.603047e-05\\
0.2367237	1.616007e-05\\
0.2368237	1.624824e-05\\
0.2369237	1.63608e-05\\
0.2370237	1.651501e-05\\
0.2371237	1.658225e-05\\
0.2372237	1.668466e-05\\
0.2373237	1.679273e-05\\
0.2374237	1.695798e-05\\
0.2375238	1.70743e-05\\
0.2376238	1.716919e-05\\
0.2377238	1.731218e-05\\
0.2378238	1.742247e-05\\
0.2379238	1.749842e-05\\
0.2380238	1.761512e-05\\
0.2381238	1.775573e-05\\
0.2382238	1.79011e-05\\
0.2383238	1.803362e-05\\
0.2384238	1.816917e-05\\
0.2385239	1.828182e-05\\
0.2386239	1.837387e-05\\
0.2387239	1.854809e-05\\
0.2388239	1.866122e-05\\
0.2389239	1.876534e-05\\
0.2390239	1.887143e-05\\
0.2391239	1.902405e-05\\
0.2392239	1.918105e-05\\
0.2393239	1.927902e-05\\
0.2394239	1.941604e-05\\
0.239524	1.953118e-05\\
0.239624	1.964824e-05\\
0.239724	1.976579e-05\\
0.239824	1.991292e-05\\
0.239924	2.007723e-05\\
0.240024	2.021511e-05\\
0.240124	2.038038e-05\\
0.240224	2.053455e-05\\
0.240324	2.063511e-05\\
0.240424	2.077632e-05\\
0.2405241	2.092659e-05\\
0.2406241	2.105352e-05\\
0.2407241	2.119506e-05\\
0.2408241	2.130705e-05\\
0.2409241	2.150596e-05\\
0.2410241	2.16438e-05\\
0.2411241	2.178285e-05\\
0.2412241	2.195964e-05\\
0.2413241	2.210903e-05\\
0.2414241	2.224537e-05\\
0.2415242	2.237076e-05\\
0.2416242	2.251867e-05\\
0.2417242	2.263545e-05\\
0.2418242	2.279069e-05\\
0.2419242	2.298317e-05\\
0.2420242	2.314177e-05\\
0.2421242	2.329774e-05\\
0.2422242	2.344776e-05\\
0.2423242	2.35871e-05\\
0.2424242	2.375424e-05\\
0.2425243	2.390882e-05\\
0.2426243	2.411774e-05\\
0.2427243	2.42324e-05\\
0.2428243	2.439025e-05\\
0.2429243	2.455852e-05\\
0.2430243	2.468956e-05\\
0.2431243	2.490692e-05\\
0.2432243	2.506104e-05\\
0.2433243	2.525556e-05\\
0.2434243	2.539501e-05\\
0.2435244	2.557394e-05\\
0.2436244	2.578268e-05\\
0.2437244	2.588593e-05\\
0.2438244	2.60583e-05\\
0.2439244	2.62334e-05\\
0.2440244	2.641079e-05\\
0.2441244	2.65999e-05\\
0.2442244	2.676525e-05\\
0.2443244	2.695828e-05\\
0.2444244	2.715478e-05\\
0.2445245	2.733189e-05\\
0.2446245	2.751524e-05\\
0.2447245	2.765872e-05\\
0.2448245	2.785707e-05\\
0.2449245	2.806102e-05\\
0.2450245	2.823261e-05\\
0.2451245	2.841209e-05\\
0.2452245	2.859139e-05\\
0.2453245	2.87957e-05\\
0.2454245	2.893387e-05\\
0.2455246	2.911358e-05\\
0.2456246	2.933677e-05\\
0.2457246	2.956085e-05\\
0.2458246	2.976967e-05\\
0.2459246	2.994004e-05\\
0.2460246	3.012646e-05\\
0.2461246	3.031463e-05\\
0.2462246	3.055865e-05\\
0.2463246	3.075831e-05\\
0.2464246	3.095124e-05\\
0.2465247	3.113344e-05\\
0.2466247	3.133621e-05\\
0.2467247	3.154409e-05\\
0.2468247	3.168882e-05\\
0.2469247	3.192833e-05\\
0.2470247	3.213096e-05\\
0.2471247	3.233718e-05\\
0.2472247	3.256558e-05\\
0.2473247	3.276134e-05\\
0.2474247	3.29971e-05\\
0.2475248	3.321111e-05\\
0.2476248	3.344162e-05\\
0.2477248	3.366153e-05\\
0.2478248	3.384026e-05\\
0.2479248	3.409918e-05\\
0.2480248	3.428181e-05\\
0.2481248	3.447033e-05\\
0.2482248	3.471058e-05\\
0.2483248	3.496367e-05\\
0.2484248	3.523217e-05\\
0.2485249	3.54101e-05\\
0.2486249	3.56447e-05\\
0.2487249	3.587487e-05\\
0.2488249	3.608168e-05\\
0.2489249	3.632972e-05\\
0.2490249	3.654031e-05\\
0.2491249	3.678232e-05\\
0.2492249	3.699259e-05\\
0.2493249	3.726315e-05\\
0.2494249	3.748776e-05\\
0.249525	3.77255e-05\\
0.249625	3.797034e-05\\
0.249725	3.816879e-05\\
0.249825	3.841232e-05\\
0.249925	3.867958e-05\\
0.250025	3.897102e-05\\
0.250125	3.917548e-05\\
0.250225	3.942037e-05\\
0.250325	3.967222e-05\\
0.250425	3.991049e-05\\
0.2505251	4.017263e-05\\
0.2506251	4.043341e-05\\
0.2507251	4.068451e-05\\
0.2508251	4.093426e-05\\
0.2509251	4.120399e-05\\
0.2510251	4.146766e-05\\
0.2511251	4.169124e-05\\
0.2512251	4.198487e-05\\
0.2513251	4.2271e-05\\
0.2514251	4.251979e-05\\
0.2515252	4.276722e-05\\
0.2516252	4.301107e-05\\
0.2517252	4.328678e-05\\
0.2518252	4.352951e-05\\
0.2519252	4.381908e-05\\
0.2520252	4.410665e-05\\
0.2521252	4.435999e-05\\
0.2522252	4.466143e-05\\
0.2523252	4.490521e-05\\
0.2524252	4.518847e-05\\
0.2525253	4.546561e-05\\
0.2526253	4.578614e-05\\
0.2527253	4.602447e-05\\
0.2528253	4.627821e-05\\
0.2529253	4.660297e-05\\
0.2530253	4.689617e-05\\
0.2531253	4.723168e-05\\
0.2532253	4.747111e-05\\
0.2533253	4.776251e-05\\
0.2534253	4.8042e-05\\
0.2535254	4.83331e-05\\
0.2536254	4.861463e-05\\
0.2537254	4.887149e-05\\
0.2538254	4.921785e-05\\
0.2539254	4.951687e-05\\
0.2540254	4.98163e-05\\
0.2541254	5.011015e-05\\
0.2542254	5.042582e-05\\
0.2543254	5.074287e-05\\
0.2544254	5.101241e-05\\
0.2545255	5.134183e-05\\
0.2546255	5.16471e-05\\
0.2547255	5.194402e-05\\
0.2548255	5.224746e-05\\
0.2549255	5.260541e-05\\
0.2550255	5.288199e-05\\
0.2551255	5.319109e-05\\
0.2552255	5.351456e-05\\
0.2553255	5.37884e-05\\
0.2554255	5.415669e-05\\
0.2555256	5.452483e-05\\
0.2556256	5.485414e-05\\
0.2557256	5.512697e-05\\
0.2558256	5.542775e-05\\
0.2559256	5.575641e-05\\
0.2560256	5.605807e-05\\
0.2561256	5.642796e-05\\
0.2562256	5.677752e-05\\
0.2563256	5.710979e-05\\
0.2564256	5.74512e-05\\
0.2565257	5.778245e-05\\
0.2566257	5.811804e-05\\
0.2567257	5.842948e-05\\
0.2568257	5.877814e-05\\
0.2569257	5.90992e-05\\
0.2570257	5.944573e-05\\
0.2571257	5.980821e-05\\
0.2572257	6.01626e-05\\
0.2573257	6.048256e-05\\
0.2574257	6.083313e-05\\
0.2575258	6.122681e-05\\
0.2576258	6.156307e-05\\
0.2577258	6.190784e-05\\
0.2578258	6.226153e-05\\
0.2579258	6.259944e-05\\
0.2580258	6.296885e-05\\
0.2581258	6.333909e-05\\
0.2582258	6.369671e-05\\
0.2583258	6.404793e-05\\
0.2584258	6.443673e-05\\
0.2585259	6.479469e-05\\
0.2586259	6.515667e-05\\
0.2587259	6.553907e-05\\
0.2588259	6.590164e-05\\
0.2589259	6.626682e-05\\
0.2590259	6.663473e-05\\
0.2591259	6.7044e-05\\
0.2592259	6.739925e-05\\
0.2593259	6.776733e-05\\
0.2594259	6.80896e-05\\
0.259526	6.846331e-05\\
0.259626	6.888541e-05\\
0.259726	6.927932e-05\\
0.259826	6.965918e-05\\
0.259926	7.000797e-05\\
0.260026	7.043903e-05\\
0.260126	7.082563e-05\\
0.260226	7.120233e-05\\
0.260326	7.162666e-05\\
0.260426	7.200361e-05\\
0.2605261	7.240564e-05\\
0.2606261	7.277003e-05\\
0.2607261	7.31751e-05\\
0.2608261	7.356588e-05\\
0.2609261	7.398862e-05\\
0.2610261	7.441733e-05\\
0.2611261	7.479314e-05\\
0.2612261	7.517965e-05\\
0.2613261	7.557578e-05\\
0.2614261	7.598764e-05\\
0.2615262	7.63668e-05\\
0.2616262	7.682199e-05\\
0.2617262	7.722577e-05\\
0.2618262	7.763947e-05\\
0.2619262	7.804567e-05\\
0.2620262	7.845404e-05\\
0.2621262	7.887811e-05\\
0.2622262	7.929918e-05\\
0.2623262	7.971042e-05\\
0.2624262	8.010897e-05\\
0.2625263	8.058151e-05\\
0.2626263	8.100031e-05\\
0.2627263	8.139773e-05\\
0.2628263	8.179431e-05\\
0.2629263	8.223751e-05\\
0.2630263	8.265956e-05\\
0.2631263	8.307691e-05\\
0.2632263	8.356782e-05\\
0.2633263	8.395342e-05\\
0.2634263	8.44182e-05\\
0.2635264	8.483955e-05\\
0.2636264	8.526787e-05\\
0.2637264	8.571349e-05\\
0.2638264	8.615825e-05\\
0.2639264	8.65927e-05\\
0.2640264	8.7016e-05\\
0.2641264	8.746161e-05\\
0.2642264	8.785897e-05\\
0.2643264	8.830137e-05\\
0.2644264	8.875305e-05\\
0.2645265	8.92273e-05\\
0.2646265	8.967261e-05\\
0.2647265	9.012326e-05\\
0.2648265	9.0586e-05\\
0.2649265	9.102645e-05\\
0.2650265	9.152264e-05\\
0.2651265	9.19559e-05\\
0.2652265	9.237901e-05\\
0.2653265	9.28072e-05\\
0.2654265	9.325911e-05\\
0.2655266	9.372451e-05\\
0.2656266	9.419683e-05\\
0.2657266	9.469507e-05\\
0.2658266	9.513049e-05\\
0.2659266	9.557041e-05\\
0.2660266	9.602184e-05\\
0.2661266	9.653691e-05\\
0.2662266	9.695836e-05\\
0.2663266	9.741681e-05\\
0.2664266	9.785359e-05\\
0.2665267	9.833955e-05\\
0.2666267	9.88296e-05\\
0.2667267	9.927847e-05\\
0.2668267	9.973037e-05\\
0.2669267	0.0001002319\\
0.2670267	0.0001007085\\
0.2671267	0.0001011426\\
0.2672267	0.0001016239\\
0.2673267	0.0001020868\\
0.2674267	0.0001025635\\
0.2675268	0.0001030234\\
0.2676268	0.0001035229\\
0.2677268	0.000103996\\
0.2678268	0.0001044772\\
0.2679268	0.0001049685\\
0.2680268	0.0001053986\\
0.2681268	0.0001058874\\
0.2682268	0.0001063361\\
0.2683268	0.000106803\\
0.2684268	0.0001073048\\
0.2685269	0.0001078447\\
0.2686269	0.0001083162\\
0.2687269	0.0001087724\\
0.2688269	0.0001092459\\
0.2689269	0.0001097034\\
0.2690269	0.0001102252\\
0.2691269	0.0001106938\\
0.2692269	0.000111164\\
0.2693269	0.0001116006\\
0.2694269	0.0001121233\\
0.269527	0.000112593\\
0.269627	0.0001130672\\
0.269727	0.0001135511\\
0.269827	0.0001140425\\
0.269927	0.0001145049\\
0.270027	0.0001150157\\
0.270127	0.0001155116\\
0.270227	0.000115963\\
0.270327	0.0001164807\\
0.270427	0.0001169225\\
0.2705271	0.0001174318\\
0.2706271	0.0001179005\\
0.2707271	0.0001183722\\
0.2708271	0.0001188612\\
0.2709271	0.0001193651\\
0.2710271	0.0001198625\\
0.2711271	0.0001203055\\
0.2712271	0.0001208108\\
0.2713271	0.0001212734\\
0.2714271	0.0001217721\\
0.2715272	0.0001222012\\
0.2716272	0.0001227086\\
0.2717272	0.0001231784\\
0.2718272	0.0001236842\\
0.2719272	0.0001241704\\
0.2720272	0.0001246179\\
0.2721272	0.0001251085\\
0.2722272	0.0001255589\\
0.2723272	0.0001260457\\
0.2724272	0.0001265394\\
0.2725273	0.0001270364\\
0.2726273	0.0001274644\\
0.2727273	0.000127945\\
0.2728273	0.0001284331\\
0.2729273	0.0001289158\\
0.2730273	0.0001293698\\
0.2731273	0.0001298554\\
0.2732273	0.0001303409\\
0.2733273	0.0001307842\\
0.2734273	0.0001312379\\
0.2735274	0.0001316993\\
0.2736274	0.000132153\\
0.2737274	0.0001326558\\
0.2738274	0.0001330995\\
0.2739274	0.0001335259\\
0.2740274	0.0001340087\\
0.2741274	0.0001344488\\
0.2742274	0.0001349222\\
0.2743274	0.0001353859\\
0.2744274	0.0001358556\\
0.2745275	0.0001362825\\
0.2746275	0.0001367239\\
0.2747275	0.0001371867\\
0.2748275	0.0001376529\\
0.2749275	0.0001380881\\
0.2750275	0.0001384878\\
0.2751275	0.0001389744\\
0.2752275	0.0001394167\\
0.2753275	0.000139873\\
0.2754275	0.0001402533\\
0.2755276	0.0001406984\\
0.2756276	0.0001411182\\
0.2757276	0.0001415767\\
0.2758276	0.0001420179\\
0.2759276	0.0001424321\\
0.2760276	0.0001428512\\
0.2761276	0.0001432681\\
0.2762276	0.0001436874\\
0.2763276	0.0001440734\\
0.2764276	0.0001445148\\
0.2765277	0.0001448991\\
0.2766277	0.000145322\\
0.2767277	0.0001457451\\
0.2768277	0.0001461543\\
0.2769277	0.0001465211\\
0.2770277	0.0001469165\\
0.2771277	0.0001473079\\
0.2772277	0.0001476858\\
0.2773277	0.0001480428\\
0.2774277	0.0001484366\\
0.2775278	0.0001488534\\
0.2776278	0.0001492679\\
0.2777278	0.0001496448\\
0.2778278	0.0001499706\\
0.2779278	0.0001503548\\
0.2780278	0.0001507045\\
0.2781278	0.000151073\\
0.2782278	0.000151422\\
0.2783278	0.0001517729\\
0.2784278	0.0001520845\\
0.2785279	0.0001524474\\
0.2786279	0.0001527888\\
0.2787279	0.0001531248\\
0.2788279	0.0001534885\\
0.2789279	0.0001538026\\
0.2790279	0.0001541336\\
0.2791279	0.0001544537\\
0.2792279	0.0001547472\\
0.2793279	0.0001550376\\
0.2794279	0.0001553732\\
0.279528	0.000155675\\
0.279628	0.00015597\\
0.279728	0.0001562366\\
0.279828	0.0001565503\\
0.279928	0.0001568232\\
0.280028	0.0001570949\\
0.280128	0.0001573391\\
0.280228	0.0001576448\\
0.280328	0.0001578955\\
0.280428	0.0001581454\\
0.2805281	0.0001583723\\
0.2806281	0.0001586377\\
0.2807281	0.0001588671\\
0.2808281	0.0001590999\\
0.2809281	0.0001593735\\
0.2810281	0.0001595685\\
0.2811281	0.0001597762\\
0.2812281	0.0001599756\\
0.2813281	0.0001602037\\
0.2814281	0.0001603982\\
0.2815282	0.0001605742\\
0.2816282	0.0001607113\\
0.2817282	0.0001609082\\
0.2818282	0.0001610881\\
0.2819282	0.0001612414\\
0.2820282	0.0001614092\\
0.2821282	0.00016157\\
0.2822282	0.0001616918\\
0.2823282	0.0001618345\\
0.2824282	0.0001619659\\
0.2825283	0.0001621103\\
0.2826283	0.0001622231\\
0.2827283	0.0001623124\\
0.2828283	0.0001623891\\
0.2829283	0.0001624865\\
0.2830283	0.0001625782\\
0.2831283	0.0001626241\\
0.2832283	0.0001627463\\
0.2833283	0.0001628103\\
0.2834283	0.0001628266\\
0.2835284	0.0001628369\\
0.2836284	0.0001629129\\
0.2837284	0.0001629063\\
0.2838284	0.0001629319\\
0.2839284	0.0001629474\\
0.2840284	0.0001629456\\
0.2841284	0.0001629388\\
0.2842284	0.0001628983\\
0.2843284	0.0001628135\\
0.2844284	0.0001627932\\
0.2845285	0.0001627584\\
0.2846285	0.0001627218\\
0.2847285	0.0001626905\\
0.2848285	0.0001625922\\
0.2849285	0.0001624921\\
0.2850285	0.0001623985\\
0.2851285	0.0001623096\\
0.2852285	0.0001621462\\
0.2853285	0.0001620336\\
0.2854285	0.0001618574\\
0.2855286	0.0001617409\\
0.2856286	0.0001615667\\
0.2857286	0.0001614013\\
0.2858286	0.0001612181\\
0.2859286	0.0001610434\\
0.2860286	0.0001608203\\
0.2861286	0.0001605826\\
0.2862286	0.0001603394\\
0.2863286	0.0001601399\\
0.2864286	0.0001598693\\
0.2865287	0.0001596144\\
0.2866287	0.0001593228\\
0.2867287	0.0001590292\\
0.2868287	0.0001586949\\
0.2869287	0.0001583854\\
0.2870287	0.0001580759\\
0.2871287	0.0001577329\\
0.2872287	0.0001573561\\
0.2873287	0.0001570337\\
0.2874287	0.0001566839\\
0.2875288	0.0001562395\\
0.2876288	0.0001558219\\
0.2877288	0.0001554031\\
0.2878288	0.0001549688\\
0.2879288	0.0001545066\\
0.2880288	0.0001540383\\
0.2881288	0.0001535524\\
0.2882288	0.0001531168\\
0.2883288	0.0001525713\\
0.2884288	0.0001520549\\
0.2885289	0.0001515074\\
0.2886289	0.0001509638\\
0.2887289	0.0001503708\\
0.2888289	0.000149829\\
0.2889289	0.0001492022\\
0.2890289	0.0001485996\\
0.2891289	0.0001479683\\
0.2892289	0.0001473111\\
0.2893289	0.0001466368\\
0.2894289	0.0001459749\\
0.289529	0.0001452718\\
0.289629	0.0001445826\\
0.289729	0.0001438423\\
0.289829	0.00014305\\
0.289929	0.0001422706\\
0.290029	0.0001415511\\
0.290129	0.0001407329\\
0.290229	0.0001399423\\
0.290329	0.0001391014\\
0.290429	0.0001382607\\
0.2905291	0.0001373668\\
0.2906291	0.0001364711\\
0.2907291	0.0001355621\\
0.2908291	0.0001346277\\
0.2909291	0.0001336929\\
0.2910291	0.0001327162\\
0.2911291	0.0001317459\\
0.2912291	0.0001307402\\
0.2913291	0.0001297011\\
0.2914291	0.0001286781\\
0.2915292	0.0001276233\\
0.2916292	0.0001265128\\
0.2917292	0.0001254281\\
0.2918292	0.0001242865\\
0.2919292	0.0001231597\\
0.2920292	0.0001219586\\
0.2921292	0.0001207988\\
0.2922292	0.0001195916\\
0.2923292	0.0001183858\\
0.2924292	0.0001171064\\
0.2925293	0.0001158576\\
0.2926293	0.0001145512\\
0.2927293	0.0001132164\\
0.2928293	0.0001119032\\
0.2929293	0.0001104939\\
0.2930293	0.0001090848\\
0.2931293	0.0001077379\\
0.2932293	0.0001063109\\
0.2933293	0.0001048996\\
0.2934293	0.0001033803\\
0.2935294	0.000101851\\
0.2936294	0.0001003433\\
0.2937294	9.882103e-05\\
0.2938294	9.7205e-05\\
0.2939294	9.561366e-05\\
0.2940294	9.397927e-05\\
0.2941294	9.231636e-05\\
0.2942294	9.064778e-05\\
0.2943294	8.897445e-05\\
0.2944294	8.723952e-05\\
0.2945295	8.551243e-05\\
0.2946295	8.369529e-05\\
0.2947295	8.189121e-05\\
0.2948295	8.004898e-05\\
0.2949295	7.820217e-05\\
0.2950295	7.635613e-05\\
0.2951295	7.447122e-05\\
0.2952295	7.251223e-05\\
0.2953295	7.053332e-05\\
0.2954295	6.854878e-05\\
0.2955296	6.652814e-05\\
0.2956296	6.447247e-05\\
0.2957296	6.239265e-05\\
0.2958296	6.029462e-05\\
0.2959296	5.819097e-05\\
0.2960296	5.598407e-05\\
0.2961296	5.3764e-05\\
0.2962296	5.156981e-05\\
0.2963296	4.931915e-05\\
0.2964296	4.704304e-05\\
0.2965297	4.474435e-05\\
0.2966297	4.241249e-05\\
0.2967297	4.002598e-05\\
0.2968297	3.762699e-05\\
0.2969297	3.519217e-05\\
0.2970297	3.27431e-05\\
0.2971297	3.02406e-05\\
0.2972297	2.769079e-05\\
0.2973297	2.515026e-05\\
0.2974297	2.256885e-05\\
0.2975298	1.99228e-05\\
0.2976298	1.727931e-05\\
0.2977298	1.459563e-05\\
0.2978298	1.186823e-05\\
0.2979298	9.11509e-06\\
0.2980298	6.362924e-06\\
0.2981298	3.545799e-06\\
0.2982298	6.38918e-07\\
0.2983298	-2.262758e-06\\
0.2984298	-5.166253e-06\\
0.2985299	-8.098221e-06\\
0.2986299	-1.110336e-05\\
0.2987299	-1.411532e-05\\
0.2988299	-1.718025e-05\\
0.2989299	-2.030926e-05\\
0.2990299	-2.341946e-05\\
0.2991299	-2.65999e-05\\
0.2992299	-2.979188e-05\\
0.2993299	-3.301616e-05\\
0.2994299	-3.629806e-05\\
0.29953	-3.961785e-05\\
0.29963	-4.297875e-05\\
0.29973	-4.637134e-05\\
0.29983	-4.978687e-05\\
0.29993	-5.324832e-05\\
0.30003	-5.67656e-05\\
0.30013	-6.028894e-05\\
0.30023	-6.38438e-05\\
0.30033	-6.749062e-05\\
0.30043	-7.112802e-05\\
0.3005301	-7.480492e-05\\
0.3006301	-7.856366e-05\\
0.3007301	-8.2343e-05\\
0.3008301	-8.616724e-05\\
0.3009301	-9.001197e-05\\
0.3010301	-9.389415e-05\\
0.3011301	-9.781628e-05\\
0.3012301	-0.0001017481\\
0.3013301	-0.0001057974\\
0.3014301	-0.000109888\\
0.3015302	-0.0001139652\\
0.3016302	-0.0001180707\\
0.3017302	-0.0001222912\\
0.3018302	-0.0001265289\\
0.3019302	-0.0001307655\\
0.3020302	-0.0001350767\\
0.3021302	-0.0001394053\\
0.3022302	-0.0001438255\\
0.3023302	-0.000148242\\
0.3024302	-0.0001526986\\
0.3025303	-0.0001572194\\
0.3026303	-0.0001617775\\
0.3027303	-0.0001663867\\
0.3028303	-0.0001710807\\
0.3029303	-0.0001757727\\
0.3030303	-0.0001805115\\
0.3031303	-0.0001852944\\
0.3032303	-0.0001901512\\
0.3033303	-0.0001949824\\
0.3034303	-0.0001998852\\
0.3035304	-0.0002048607\\
0.3036304	-0.0002098979\\
0.3037304	-0.0002149439\\
0.3038304	-0.0002200579\\
0.3039304	-0.000225247\\
0.3040304	-0.0002304288\\
0.3041304	-0.0002356595\\
0.3042304	-0.0002409713\\
0.3043304	-0.0002463001\\
0.3044304	-0.0002517267\\
0.3045305	-0.0002571329\\
0.3046305	-0.0002626025\\
0.3047305	-0.0002681112\\
0.3048305	-0.0002737279\\
0.3049305	-0.0002793769\\
0.3050305	-0.0002850565\\
0.3051305	-0.000290731\\
0.3052305	-0.0002965011\\
0.3053305	-0.0003023676\\
0.3054305	-0.00030824\\
0.3055306	-0.0003141808\\
0.3056306	-0.0003201666\\
0.3057306	-0.0003261739\\
0.3058306	-0.0003322543\\
0.3059306	-0.0003383597\\
0.3060306	-0.0003445433\\
0.3061306	-0.0003507891\\
0.3062306	-0.0003570186\\
0.3063306	-0.0003633801\\
0.3064306	-0.0003697678\\
0.3065307	-0.0003762141\\
0.3066307	-0.0003827059\\
0.3067307	-0.0003892427\\
0.3068307	-0.0003958221\\
0.3069307	-0.0004024765\\
0.3070307	-0.000409182\\
0.3071307	-0.0004159083\\
0.3072307	-0.0004226942\\
0.3073307	-0.0004295711\\
0.3074307	-0.000436478\\
0.3075308	-0.0004434265\\
0.3076308	-0.0004504122\\
0.3077308	-0.0004575039\\
0.3078308	-0.0004646204\\
0.3079308	-0.0004718047\\
0.3080308	-0.0004790276\\
0.3081308	-0.0004863229\\
0.3082308	-0.0004936641\\
0.3083308	-0.0005010678\\
0.3084308	-0.0005085034\\
0.3085309	-0.0005160215\\
0.3086309	-0.0005235402\\
0.3087309	-0.0005311702\\
0.3088309	-0.000538845\\
0.3089309	-0.0005465782\\
0.3090309	-0.0005543506\\
0.3091309	-0.0005622078\\
0.3092309	-0.0005700785\\
0.3093309	-0.0005779739\\
0.3094309	-0.0005859716\\
0.309531	-0.0005940823\\
0.309631	-0.0006022036\\
0.309731	-0.0006103338\\
0.309831	-0.0006185571\\
0.309931	-0.0006268649\\
0.310031	-0.0006351828\\
0.310131	-0.0006435971\\
0.310231	-0.0006520631\\
0.310331	-0.0006605608\\
0.310431	-0.0006691399\\
0.3105311	-0.0006777428\\
0.3106311	-0.0006864087\\
0.3107311	-0.000695152\\
0.3108311	-0.0007039388\\
0.3109311	-0.0007128304\\
0.3110311	-0.000721712\\
0.3111311	-0.0007306517\\
0.3112311	-0.0007397202\\
0.3113311	-0.0007487834\\
0.3114311	-0.0007579117\\
0.3115312	-0.0007671064\\
0.3116312	-0.0007763605\\
0.3117312	-0.0007856805\\
0.3118312	-0.0007950636\\
0.3119312	-0.0008044671\\
0.3120312	-0.0008139984\\
0.3121312	-0.0008234867\\
0.3122312	-0.0008330699\\
0.3123312	-0.0008427517\\
0.3124312	-0.0008524748\\
0.3125313	-0.0008622191\\
0.3126313	-0.0008720839\\
0.3127313	-0.0008819192\\
0.3128313	-0.0008918745\\
0.3129313	-0.0009018901\\
0.3130313	-0.0009119669\\
0.3131313	-0.0009220594\\
0.3132313	-0.0009322108\\
0.3133313	-0.0009424558\\
0.3134313	-0.0009527774\\
0.3135314	-0.0009631087\\
0.3136314	-0.0009735183\\
0.3137314	-0.000983963\\
0.3138314	-0.0009944674\\
0.3139314	-0.001005045\\
0.3140314	-0.00101568\\
0.3141314	-0.001026353\\
0.3142314	-0.001037113\\
0.3143314	-0.001047924\\
0.3144314	-0.001058782\\
0.3145315	-0.001069683\\
0.3146315	-0.001080673\\
0.3147315	-0.001091698\\
0.3148315	-0.001102762\\
0.3149315	-0.001113882\\
0.3150315	-0.001125078\\
0.3151315	-0.001136313\\
0.3152315	-0.001147627\\
0.3153315	-0.001159014\\
0.3154315	-0.001170445\\
0.3155316	-0.001181931\\
0.3156316	-0.001193385\\
0.3157316	-0.001204967\\
0.3158316	-0.001216604\\
0.3159316	-0.00122829\\
0.3160316	-0.00124005\\
0.3161316	-0.001251849\\
0.3162316	-0.001263705\\
0.3163316	-0.001275636\\
0.3164316	-0.001287548\\
0.3165317	-0.001299577\\
0.3166317	-0.001311608\\
0.3167317	-0.00132371\\
0.3168317	-0.001335896\\
0.3169317	-0.001348096\\
0.3170317	-0.001360388\\
0.3171317	-0.001372712\\
0.3172317	-0.001385042\\
0.3173317	-0.001397461\\
0.3174317	-0.00140998\\
0.3175318	-0.001422493\\
0.3176318	-0.001435054\\
0.3177318	-0.001447661\\
0.3178318	-0.001460322\\
0.3179318	-0.001473056\\
0.3180318	-0.001485803\\
0.3181318	-0.001498612\\
0.3182318	-0.001511478\\
0.3183318	-0.001524385\\
0.3184318	-0.001537375\\
0.3185319	-0.001550367\\
0.3186319	-0.001563399\\
0.3187319	-0.001576513\\
0.3188319	-0.001589648\\
0.3189319	-0.001602796\\
0.3190319	-0.001616021\\
0.3191319	-0.001629306\\
0.3192319	-0.001642631\\
0.3193319	-0.001655966\\
0.3194319	-0.001669411\\
0.319532	-0.001682828\\
0.319632	-0.001696282\\
0.319732	-0.00170981\\
0.319832	-0.00172338\\
0.319932	-0.001736987\\
0.320032	-0.001750638\\
0.320132	-0.001764286\\
0.320232	-0.001778006\\
0.320332	-0.001791763\\
0.320432	-0.001805593\\
0.3205321	-0.00181945\\
0.3206321	-0.001833242\\
0.3207321	-0.001847163\\
0.3208321	-0.001861104\\
0.3209321	-0.001875078\\
0.3210321	-0.001889081\\
0.3211321	-0.001903122\\
0.3212321	-0.001917186\\
0.3213321	-0.001931296\\
0.3214321	-0.001945404\\
0.3215322	-0.001959555\\
0.3216322	-0.001973753\\
0.3217322	-0.001987991\\
0.3218322	-0.002002238\\
0.3219322	-0.002016499\\
0.3220322	-0.00203078\\
0.3221322	-0.002045122\\
0.3222322	-0.002059457\\
0.3223322	-0.002073833\\
0.3224322	-0.002088249\\
0.3225323	-0.002102652\\
0.3226323	-0.002117099\\
0.3227323	-0.00213154\\
0.3228323	-0.002146033\\
0.3229323	-0.002160548\\
0.3230323	-0.002175056\\
0.3231323	-0.002189642\\
0.3232323	-0.00220419\\
0.3233323	-0.002218727\\
0.3234323	-0.00223331\\
0.3235324	-0.002247859\\
0.3236324	-0.002262496\\
0.3237324	-0.00227712\\
0.3238324	-0.002291777\\
0.3239324	-0.002306418\\
0.3240324	-0.002321041\\
0.3241324	-0.002335729\\
0.3242324	-0.002350432\\
0.3243324	-0.002365067\\
0.3244324	-0.002379772\\
0.3245325	-0.002394452\\
0.3246325	-0.002409137\\
0.3247325	-0.002423819\\
0.3248325	-0.002438508\\
0.3249325	-0.00245319\\
0.3250325	-0.00246784\\
0.3251325	-0.002482548\\
0.3252325	-0.00249724\\
0.3253325	-0.002511899\\
0.3254325	-0.002526596\\
0.3255326	-0.002541265\\
0.3256326	-0.002555938\\
0.3257326	-0.002570579\\
0.3258326	-0.002585175\\
0.3259326	-0.002599808\\
0.3260326	-0.00261441\\
0.3261326	-0.002629042\\
0.3262326	-0.002643625\\
0.3263326	-0.00265815\\
0.3264326	-0.002672705\\
0.3265327	-0.002687265\\
0.3266327	-0.002701762\\
0.3267327	-0.002716241\\
0.3268327	-0.002730682\\
0.3269327	-0.002745118\\
0.3270327	-0.002759532\\
0.3271327	-0.002773912\\
0.3272327	-0.002788251\\
0.3273327	-0.002802573\\
0.3274327	-0.002816881\\
0.3275328	-0.00283115\\
0.3276328	-0.002845353\\
0.3277328	-0.002859546\\
0.3278328	-0.002873686\\
0.3279328	-0.002887758\\
0.3280328	-0.002901864\\
0.3281328	-0.002915904\\
0.3282328	-0.0029299\\
0.3283328	-0.00294379\\
0.3284328	-0.002957678\\
0.3285329	-0.002971549\\
0.3286329	-0.002985332\\
0.3287329	-0.002999072\\
0.3288329	-0.003012759\\
0.3289329	-0.003026383\\
0.3290329	-0.003039961\\
0.3291329	-0.003053486\\
0.3292329	-0.003066916\\
0.3293329	-0.003080343\\
0.3294329	-0.003093686\\
0.329533	-0.003106936\\
0.329633	-0.003120105\\
0.329733	-0.003133251\\
0.329833	-0.003146305\\
0.329933	-0.003159283\\
0.330033	-0.003172203\\
0.330133	-0.00318506\\
0.330233	-0.003197793\\
0.330333	-0.003210504\\
0.330433	-0.003223106\\
0.3305331	-0.00323561\\
0.3306331	-0.003248034\\
0.3307331	-0.003260371\\
0.3308331	-0.003272653\\
0.3309331	-0.003284795\\
0.3310331	-0.003296894\\
0.3311331	-0.003308853\\
0.3312331	-0.003320696\\
0.3313331	-0.003332511\\
0.3314331	-0.003344219\\
0.3315332	-0.003355761\\
0.3316332	-0.003367237\\
0.3317332	-0.003378615\\
0.3318332	-0.003389895\\
0.3319332	-0.003400994\\
0.3320332	-0.003412019\\
0.3321332	-0.003422944\\
0.3322332	-0.003433755\\
0.3323332	-0.003444458\\
0.3324332	-0.003455029\\
0.3325333	-0.003465442\\
0.3326333	-0.003475739\\
0.3327333	-0.003485895\\
0.3328333	-0.003495959\\
0.3329333	-0.003505881\\
0.3330333	-0.003515681\\
0.3331333	-0.003525352\\
0.3332333	-0.003534828\\
0.3333333	-0.003544213\\
0.3334333	-0.003553402\\
0.3335334	-0.003562462\\
0.3336334	-0.003571438\\
0.3337334	-0.003580213\\
0.3338334	-0.003588818\\
0.3339334	-0.003597272\\
0.3340334	-0.003605583\\
0.3341334	-0.003613746\\
0.3342334	-0.003621705\\
0.3343334	-0.003629555\\
0.3344334	-0.003637192\\
0.3345335	-0.003644658\\
0.3346335	-0.003652007\\
0.3347335	-0.003659116\\
0.3348335	-0.00366609\\
0.3349335	-0.003672879\\
0.3350335	-0.003679469\\
0.3351335	-0.003685877\\
0.3352335	-0.00369207\\
0.3353335	-0.003698112\\
0.3354335	-0.003703959\\
0.3355336	-0.003709602\\
0.3356336	-0.003715083\\
0.3357336	-0.003720315\\
0.3358336	-0.003725342\\
0.3359336	-0.003730197\\
0.3360336	-0.00373481\\
0.3361336	-0.003739244\\
0.3362336	-0.003743441\\
0.3363336	-0.003747436\\
0.3364336	-0.003751203\\
0.3365337	-0.003754777\\
0.3366337	-0.003758104\\
0.3367337	-0.003761176\\
0.3368337	-0.003764069\\
0.3369337	-0.003766719\\
0.3370337	-0.00376908\\
0.3371337	-0.003771251\\
0.3372337	-0.003773192\\
0.3373337	-0.003774887\\
0.3374337	-0.003776299\\
0.3375338	-0.003777482\\
0.3376338	-0.00377843\\
0.3377338	-0.003779089\\
0.3378338	-0.003779541\\
0.3379338	-0.003779692\\
0.3380338	-0.003779588\\
0.3381338	-0.003779222\\
0.3382338	-0.003778556\\
0.3383338	-0.003777681\\
0.3384338	-0.00377654\\
0.3385339	-0.00377506\\
0.3386339	-0.003773314\\
0.3387339	-0.00377127\\
0.3388339	-0.003769007\\
0.3389339	-0.003766406\\
0.3390339	-0.00376352\\
0.3391339	-0.003760356\\
0.3392339	-0.003756866\\
0.3393339	-0.003753066\\
0.3394339	-0.003748965\\
0.339534	-0.003744582\\
0.339634	-0.003739901\\
0.339734	-0.003734892\\
0.339834	-0.003729577\\
0.339934	-0.003723932\\
0.340034	-0.003717962\\
0.340134	-0.003711699\\
0.340234	-0.003705057\\
0.340334	-0.003698138\\
0.340434	-0.003690852\\
0.3405341	-0.003683223\\
0.3406341	-0.003675278\\
0.3407341	-0.003667002\\
0.3408341	-0.00365835\\
0.3409341	-0.003649346\\
0.3410341	-0.003640056\\
0.3411341	-0.003630357\\
0.3412341	-0.003620298\\
0.3413341	-0.003609875\\
0.3414341	-0.003599086\\
0.3415342	-0.003587971\\
0.3416342	-0.003576491\\
0.3417342	-0.003564599\\
0.3418342	-0.003552337\\
0.3419342	-0.003539734\\
0.3420342	-0.003526751\\
0.3421342	-0.003513344\\
0.3422342	-0.003499551\\
0.3423342	-0.003485365\\
0.3424342	-0.003470773\\
0.3425343	-0.003455835\\
0.3426343	-0.00344047\\
0.3427343	-0.003424738\\
0.3428343	-0.003408567\\
0.3429343	-0.003392013\\
0.3430343	-0.003375055\\
0.3431343	-0.003357633\\
0.3432343	-0.003339838\\
0.3433343	-0.003321604\\
0.3434343	-0.003302922\\
0.3435344	-0.003283858\\
0.3436344	-0.003264341\\
0.3437344	-0.003244411\\
0.3438344	-0.003224072\\
0.3439344	-0.003203239\\
0.3440344	-0.003181983\\
0.3441344	-0.003160256\\
0.3442344	-0.003138172\\
0.3443344	-0.003115579\\
0.3444344	-0.003092532\\
0.3445345	-0.003069075\\
0.3446345	-0.003045094\\
0.3447345	-0.003020699\\
0.3448345	-0.002995819\\
0.3449345	-0.002970453\\
0.3450345	-0.002944684\\
0.3451345	-0.002918399\\
0.3452345	-0.00289165\\
0.3453345	-0.002864428\\
0.3454345	-0.002836714\\
0.3455346	-0.002808535\\
0.3456346	-0.002779841\\
0.3457346	-0.00275073\\
0.3458346	-0.002721058\\
0.3459346	-0.002690887\\
0.3460346	-0.00266026\\
0.3461346	-0.002629113\\
0.3462346	-0.002597485\\
0.3463346	-0.002565333\\
0.3464346	-0.00253269\\
0.3465347	-0.002499546\\
0.3466347	-0.002465886\\
0.3467347	-0.002431671\\
0.3468347	-0.002396972\\
0.3469347	-0.00236179\\
0.3470347	-0.002326078\\
0.3471347	-0.002289754\\
0.3472347	-0.002252972\\
0.3473347	-0.002215672\\
0.3474347	-0.002177818\\
0.3475348	-0.002139424\\
0.3476348	-0.002100507\\
0.3477348	-0.002061071\\
0.3478348	-0.002021079\\
0.3479348	-0.00198059\\
0.3480348	-0.001939482\\
0.3481348	-0.001897859\\
0.3482348	-0.001855636\\
0.3483348	-0.001812932\\
0.3484348	-0.001769647\\
0.3485349	-0.001725822\\
0.3486349	-0.001681447\\
0.3487349	-0.001636485\\
0.3488349	-0.001591002\\
0.3489349	-0.001544914\\
0.3490349	-0.001498248\\
0.3491349	-0.001451057\\
0.3492349	-0.00140329\\
0.3493349	-0.00135496\\
0.3494349	-0.001306025\\
0.349535	-0.001256527\\
0.349635	-0.001206479\\
0.349735	-0.001155802\\
0.349835	-0.001104612\\
0.349935	-0.001052785\\
0.350035	-0.00100039\\
0.350135	-0.000947419\\
0.350235	-0.0008938617\\
0.350335	-0.0008397642\\
0.350435	-0.0007850216\\
0.3505351	-0.0007296733\\
0.3506351	-0.0006737605\\
0.3507351	-0.0006172808\\
0.3508351	-0.0005601668\\
0.3509351	-0.0005024812\\
0.3510351	-0.0004442588\\
0.3511351	-0.0003853294\\
0.3512351	-0.0003258249\\
0.3513351	-0.0002657449\\
0.3514351	-0.000205089\\
0.3515352	-0.0001438106\\
0.3516352	-8.19287e-05\\
0.3517352	-1.943015e-05\\
0.3518352	4.366728e-05\\
0.3519352	0.0001073304\\
0.3520352	0.0001716203\\
0.3521352	0.0002365552\\
0.3522352	0.0003020295\\
0.3523352	0.0003681466\\
0.3524352	0.0004348802\\
0.3525353	0.000502171\\
0.3526353	0.0005701333\\
0.3527353	0.0006386681\\
0.3528353	0.0007078304\\
0.3529353	0.0007775508\\
0.3530353	0.0008479181\\
0.3531353	0.0009188951\\
0.3532353	0.0009904741\\
0.3533353	0.001062672\\
0.3534353	0.00113546\\
0.3535354	0.001208885\\
0.3536354	0.00128289\\
0.3537354	0.001357496\\
0.3538354	0.001432729\\
0.3539354	0.00150857\\
0.3540354	0.001584986\\
0.3541354	0.001662023\\
0.3542354	0.001739708\\
0.3543354	0.001817971\\
0.3544354	0.001896794\\
0.3545355	0.001976253\\
0.3546355	0.002056327\\
0.3547355	0.002137008\\
0.3548355	0.002218243\\
0.3549355	0.002300087\\
0.3550355	0.002382538\\
0.3551355	0.002465595\\
0.3552355	0.002549291\\
0.3553355	0.002633514\\
0.3554355	0.002718382\\
0.3555356	0.002803754\\
0.3556356	0.002889765\\
0.3557356	0.002976377\\
0.3558356	0.00306355\\
0.3559356	0.003151343\\
0.3560356	0.003239661\\
0.3561356	0.003328612\\
0.3562356	0.003418102\\
0.3563356	0.003508193\\
0.3564356	0.003598857\\
0.3565357	0.003690092\\
0.3566357	0.003781881\\
0.3567357	0.003874219\\
0.3568357	0.003967158\\
0.3569357	0.004060674\\
0.3570357	0.004154713\\
0.3571357	0.004249276\\
0.3572357	0.004344426\\
0.3573357	0.004440145\\
0.3574357	0.004536378\\
0.3575358	0.004633171\\
0.3576358	0.004730516\\
0.3577358	0.004828359\\
0.3578358	0.004926767\\
0.3579358	0.005025672\\
0.3580358	0.00512513\\
0.3581358	0.005225101\\
0.3582358	0.005325607\\
0.3583358	0.005426658\\
0.3584358	0.005528131\\
0.3585359	0.005630165\\
0.3586359	0.005732673\\
0.3587359	0.005835715\\
0.3588359	0.005939229\\
0.3589359	0.006043215\\
0.3590359	0.006147712\\
0.3591359	0.006252691\\
0.3592359	0.006358137\\
0.3593359	0.006464035\\
0.3594359	0.006570428\\
0.359536	0.006677235\\
0.359636	0.006784573\\
0.359736	0.006892342\\
0.359836	0.007000512\\
0.359936	0.007109128\\
0.360036	0.007218177\\
0.360136	0.007327679\\
0.360236	0.007437582\\
0.360336	0.007547924\\
0.360436	0.007658663\\
0.3605361	0.007769788\\
0.3606361	0.00788133\\
0.3607361	0.007993229\\
0.3608361	0.008105559\\
0.3609361	0.008218285\\
0.3610361	0.008331346\\
0.3611361	0.008444761\\
0.3612361	0.00855854\\
0.3613361	0.008672683\\
0.3614361	0.008787165\\
0.3615362	0.00890199\\
0.3616362	0.009017147\\
0.3617362	0.009132621\\
0.3618362	0.009248406\\
0.3619362	0.009364486\\
0.3620362	0.009480905\\
0.3621362	0.009597568\\
0.3622362	0.009714545\\
0.3623362	0.009831816\\
0.3624362	0.0099493\\
0.3625363	0.01006707\\
0.3626363	0.01018507\\
0.3627363	0.01030341\\
0.3628363	0.01042189\\
0.3629363	0.01054059\\
0.3630363	0.01065954\\
0.3631363	0.01077867\\
0.3632363	0.01089801\\
0.3633363	0.01101752\\
0.3634363	0.01113727\\
0.3635364	0.01125715\\
0.3636364	0.01137719\\
0.3637364	0.01149737\\
0.3638364	0.01161775\\
0.3639364	0.01173822\\
0.3640364	0.01185877\\
0.3641364	0.01197947\\
0.3642364	0.01210025\\
0.3643364	0.01222117\\
0.3644364	0.01234211\\
0.3645365	0.01246316\\
0.3646365	0.01258425\\
0.3647365	0.01270541\\
0.3648365	0.0128266\\
0.3649365	0.01294776\\
0.3650365	0.01306899\\
0.3651365	0.01319021\\
0.3652365	0.01331145\\
0.3653365	0.01343264\\
0.3654365	0.0135538\\
0.3655366	0.01367486\\
0.3656366	0.01379593\\
0.3657366	0.01391695\\
0.3658366	0.01403782\\
0.3659366	0.01415868\\
0.3660366	0.01427941\\
0.3661366	0.01440002\\
0.3662366	0.01452049\\
0.3663366	0.01464077\\
0.3664366	0.01476093\\
0.3665367	0.01488094\\
0.3666367	0.01500075\\
0.3667367	0.01512037\\
0.3668367	0.0152398\\
0.3669367	0.015359\\
0.3670367	0.01547794\\
0.3671367	0.01559665\\
0.3672367	0.01571512\\
0.3673367	0.0158333\\
0.3674367	0.01595117\\
0.3675368	0.01606872\\
0.3676368	0.01618596\\
0.3677368	0.0163029\\
0.3678368	0.0164195\\
0.3679368	0.01653567\\
0.3680368	0.01665151\\
0.3681368	0.01676695\\
0.3682368	0.01688199\\
0.3683368	0.0169966\\
0.3684368	0.01711074\\
0.3685369	0.01722446\\
0.3686369	0.01733773\\
0.3687369	0.01745049\\
0.3688369	0.01756276\\
0.3689369	0.01767453\\
0.3690369	0.01778574\\
0.3691369	0.01789639\\
0.3692369	0.0180065\\
0.3693369	0.01811604\\
0.3694369	0.01822497\\
0.369537	0.01833329\\
0.369637	0.01844101\\
0.369737	0.01854804\\
0.369837	0.01865443\\
0.369937	0.01876015\\
0.370037	0.01886516\\
0.370137	0.01896945\\
0.370237	0.019073\\
0.370337	0.01917586\\
0.370437	0.01927792\\
0.3705371	0.01937917\\
0.3706371	0.01947964\\
0.3707371	0.01957935\\
0.3708371	0.01967818\\
0.3709371	0.01977616\\
0.3710371	0.01987328\\
0.3711371	0.01996948\\
0.3712371	0.02006481\\
0.3713371	0.02015918\\
0.3714371	0.02025264\\
0.3715372	0.02034515\\
0.3716372	0.02043666\\
0.3717372	0.0205272\\
0.3718372	0.0206167\\
0.3719372	0.02070514\\
0.3720372	0.02079255\\
0.3721372	0.02087895\\
0.3722372	0.02096418\\
0.3723372	0.02104832\\
0.3724372	0.02113135\\
0.3725373	0.02121322\\
0.3726373	0.02129392\\
0.3727373	0.02137346\\
0.3728373	0.02145182\\
0.3729373	0.02152888\\
0.3730373	0.02160473\\
0.3731373	0.0216793\\
0.3732373	0.02175263\\
0.3733373	0.02182465\\
0.3734373	0.02189531\\
0.3735374	0.02196468\\
0.3736374	0.02203267\\
0.3737374	0.02209926\\
0.3738374	0.02216444\\
0.3739374	0.02222828\\
0.3740374	0.02229061\\
0.3741374	0.02235151\\
0.3742374	0.02241096\\
0.3743374	0.02246883\\
0.3744374	0.02252523\\
0.3745375	0.02258007\\
0.3746375	0.02263337\\
0.3747375	0.02268507\\
0.3748375	0.0227352\\
0.3749375	0.02278369\\
0.3750375	0.02283056\\
0.3751375	0.02287574\\
0.3752375	0.02291924\\
0.3753375	0.02296105\\
0.3754375	0.02300114\\
0.3755376	0.02303951\\
0.3756376	0.02307608\\
0.3757376	0.0231109\\
0.3758376	0.02314389\\
0.3759376	0.02317507\\
0.3760376	0.02320443\\
0.3761376	0.02323187\\
0.3762376	0.02325744\\
0.3763376	0.02328112\\
0.3764376	0.02330289\\
0.3765377	0.02332271\\
0.3766377	0.02334058\\
0.3767377	0.02335641\\
0.3768377	0.02337026\\
0.3769377	0.02338209\\
0.3770377	0.0233919\\
0.3771377	0.02339963\\
0.3772377	0.02340518\\
0.3773377	0.02340873\\
0.3774377	0.02341014\\
0.3775378	0.02340939\\
0.3776378	0.02340645\\
0.3777378	0.02340135\\
0.3778378	0.02339401\\
0.3779378	0.02338448\\
0.3780378	0.02337267\\
0.3781378	0.02335861\\
0.3782378	0.0233423\\
0.3783378	0.02332363\\
0.3784378	0.02330266\\
0.3785379	0.02327932\\
0.3786379	0.02325362\\
0.3787379	0.02322555\\
0.3788379	0.02319509\\
0.3789379	0.02316217\\
0.3790379	0.02312683\\
0.3791379	0.02308904\\
0.3792379	0.02304873\\
0.3793379	0.02300593\\
0.3794379	0.02296064\\
0.379538	0.02291282\\
0.379638	0.0228624\\
0.379738	0.0228094\\
0.379838	0.02275386\\
0.379938	0.02269572\\
0.380038	0.02263489\\
0.380138	0.02257144\\
0.380238	0.02250535\\
0.380338	0.02243653\\
0.380438	0.02236502\\
0.3805381	0.02229077\\
0.3806381	0.02221383\\
0.3807381	0.02213414\\
0.3808381	0.02205169\\
0.3809381	0.02196641\\
0.3810381	0.02187833\\
0.3811381	0.02178744\\
0.3812381	0.02169371\\
0.3813381	0.02159716\\
0.3814381	0.02149769\\
0.3815382	0.02139535\\
0.3816382	0.02129012\\
0.3817382	0.02118201\\
0.3818382	0.02107088\\
0.3819382	0.02095687\\
0.3820382	0.02083987\\
0.3821382	0.02071987\\
0.3822382	0.02059693\\
0.3823382	0.02047095\\
0.3824382	0.02034197\\
0.3825383	0.02020991\\
0.3826383	0.02007484\\
0.3827383	0.01993668\\
0.3828383	0.01979547\\
0.3829383	0.01965115\\
0.3830383	0.01950376\\
0.3831383	0.01935325\\
0.3832383	0.01919954\\
0.3833383	0.01904278\\
0.3834383	0.01888278\\
0.3835384	0.01871968\\
0.3836384	0.01855339\\
0.3837384	0.01838395\\
0.3838384	0.01821124\\
0.3839384	0.01803535\\
0.3840384	0.01785623\\
0.3841384	0.01767389\\
0.3842384	0.01748833\\
0.3843384	0.01729946\\
0.3844384	0.01710741\\
0.3845385	0.01691204\\
0.3846385	0.0167134\\
0.3847385	0.01651149\\
0.3848385	0.01630628\\
0.3849385	0.01609774\\
0.3850385	0.01588592\\
0.3851385	0.0156708\\
0.3852385	0.01545232\\
0.3853385	0.01523052\\
0.3854385	0.01500538\\
0.3855386	0.01477691\\
0.3856386	0.01454508\\
0.3857386	0.01430988\\
0.3858386	0.01407135\\
0.3859386	0.01382943\\
0.3860386	0.01358416\\
0.3861386	0.01333557\\
0.3862386	0.01308352\\
0.3863386	0.01282811\\
0.3864386	0.01256935\\
0.3865387	0.01230718\\
0.3866387	0.01204163\\
0.3867387	0.01177267\\
0.3868387	0.01150037\\
0.3869387	0.01122464\\
0.3870387	0.01094554\\
0.3871387	0.01066301\\
0.3872387	0.01037715\\
0.3873387	0.01008788\\
0.3874387	0.009795217\\
0.3875388	0.009499146\\
0.3876388	0.009199699\\
0.3877388	0.008896878\\
0.3878388	0.008590664\\
0.3879388	0.008281096\\
0.3880388	0.007968098\\
0.3881388	0.007651785\\
0.3882388	0.007332087\\
0.3883388	0.007009057\\
0.3884388	0.006682657\\
0.3885389	0.00635291\\
0.3886389	0.006019788\\
0.3887389	0.005683358\\
0.3888389	0.005343585\\
0.3889389	0.005000506\\
0.3890389	0.004654095\\
0.3891389	0.00430436\\
0.3892389	0.003951377\\
0.3893389	0.003595101\\
0.3894389	0.003235529\\
0.389539	0.00287271\\
0.389639	0.002506629\\
0.389739	0.002137277\\
0.389839	0.001764751\\
0.389939	0.001388947\\
0.390039	0.00100996\\
0.390139	0.0006278121\\
0.390239	0.0002424753\\
0.390339	-0.0001460601\\
0.390439	-0.0005377501\\
0.3905391	-0.000932502\\
0.3906391	-0.001330443\\
0.3907391	-0.001731439\\
0.3908391	-0.002135552\\
0.3909391	-0.002542668\\
0.3910391	-0.002952889\\
0.3911391	-0.003366138\\
0.3912391	-0.003782377\\
0.3913391	-0.004201606\\
0.3914391	-0.004623773\\
0.3915392	-0.005048909\\
0.3916392	-0.005476971\\
0.3917392	-0.005907959\\
0.3918392	-0.006341806\\
0.3919392	-0.00677852\\
0.3920392	-0.007218035\\
0.3921392	-0.007660375\\
0.3922392	-0.008105508\\
0.3923392	-0.008553407\\
0.3924392	-0.009004025\\
0.3925393	-0.009457348\\
0.3926393	-0.009913309\\
0.3927393	-0.01037198\\
0.3928393	-0.01083326\\
0.3929393	-0.0112971\\
0.3930393	-0.01176351\\
0.3931393	-0.01223247\\
0.3932393	-0.01270394\\
0.3933393	-0.01317786\\
0.3934393	-0.01365422\\
0.3935394	-0.01413299\\
0.3936394	-0.01461412\\
0.3937394	-0.01509756\\
0.3938394	-0.01558336\\
0.3939394	-0.0160714\\
0.3940394	-0.01656167\\
0.3941394	-0.01705417\\
0.3942394	-0.01754877\\
0.3943394	-0.0180455\\
0.3944394	-0.01854432\\
0.3945395	-0.01904519\\
0.3946395	-0.01954805\\
0.3947395	-0.02005294\\
0.3948395	-0.02055969\\
0.3949395	-0.02106833\\
0.3950395	-0.02157883\\
0.3951395	-0.02209114\\
0.3952395	-0.02260516\\
0.3953395	-0.02312088\\
0.3954395	-0.02363828\\
0.3955396	-0.02415735\\
0.3956396	-0.02467793\\
0.3957396	-0.02520009\\
0.3958396	-0.02572373\\
0.3959396	-0.02624876\\
0.3960396	-0.02677522\\
0.3961396	-0.027303\\
0.3962396	-0.02783207\\
0.3963396	-0.02836235\\
0.3964396	-0.02889388\\
0.3965397	-0.02942649\\
0.3966397	-0.02996021\\
0.3967397	-0.03049496\\
0.3968397	-0.03103071\\
0.3969397	-0.03156738\\
0.3970397	-0.03210488\\
0.3971397	-0.03264321\\
0.3972397	-0.03318229\\
0.3973397	-0.03372207\\
0.3974397	-0.03426258\\
0.3975398	-0.03480362\\
0.3976398	-0.03534518\\
0.3977398	-0.03588721\\
0.3978398	-0.03642967\\
0.3979398	-0.0369725\\
0.3980398	-0.03751559\\
0.3981398	-0.03805895\\
0.3982398	-0.03860245\\
0.3983398	-0.03914608\\
0.3984398	-0.03968971\\
0.3985399	-0.04023335\\
0.3986399	-0.04077694\\
0.3987399	-0.04132036\\
0.3988399	-0.04186358\\
0.3989399	-0.04240648\\
0.3990399	-0.04294908\\
0.3991399	-0.04349126\\
0.3992399	-0.04403294\\
0.3993399	-0.04457409\\
0.3994399	-0.04511463\\
0.39954	-0.04565451\\
0.39964	-0.0461936\\
0.39974	-0.04673187\\
0.39984	-0.04726927\\
0.39994	-0.04780567\\
0.40004	-0.04834107\\
};
\addplot [color=mycolor2,solid,forget plot]
  table[row sep=crcr]{%
0.40004	-0.04834107\\
0.40014	-0.04887533\\
0.40024	-0.04940844\\
0.40034	-0.04994024\\
0.40044	-0.05047077\\
0.4005401	-0.05099988\\
0.4006401	-0.05152752\\
0.4007401	-0.05205355\\
0.4008401	-0.05257796\\
0.4009401	-0.05310075\\
0.4010401	-0.05362169\\
0.4011401	-0.05414078\\
0.4012401	-0.05465792\\
0.4013401	-0.05517303\\
0.4014401	-0.05568608\\
0.4015402	-0.05619698\\
0.4016402	-0.05670555\\
0.4017402	-0.05721183\\
0.4018402	-0.05771569\\
0.4019402	-0.05821709\\
0.4020402	-0.05871587\\
0.4021402	-0.059212\\
0.4022402	-0.05970536\\
0.4023402	-0.06019594\\
0.4024402	-0.06068357\\
0.4025403	-0.06116826\\
0.4026403	-0.06164988\\
0.4027403	-0.0621283\\
0.4028403	-0.06260351\\
0.4029403	-0.06307535\\
0.4030403	-0.06354382\\
0.4031403	-0.06400884\\
0.4032403	-0.0644702\\
0.4033403	-0.06492791\\
0.4034403	-0.06538187\\
0.4035404	-0.06583195\\
0.4036404	-0.06627815\\
0.4037404	-0.06672036\\
0.4038404	-0.06715847\\
0.4039404	-0.06759231\\
0.4040404	-0.06802191\\
0.4041404	-0.06844716\\
0.4042404	-0.06886796\\
0.4043404	-0.06928416\\
0.4044404	-0.06969578\\
0.4045405	-0.07010272\\
0.4046405	-0.07050478\\
0.4047405	-0.07090195\\
0.4048405	-0.07129415\\
0.4049405	-0.07168123\\
0.4050405	-0.07206321\\
0.4051405	-0.0724399\\
0.4052405	-0.07281123\\
0.4053405	-0.07317714\\
0.4054405	-0.0735375\\
0.4055406	-0.07389228\\
0.4056406	-0.07424131\\
0.4057406	-0.07458457\\
0.4058406	-0.0749219\\
0.4059406	-0.07525328\\
0.4060406	-0.07557858\\
0.4061406	-0.07589776\\
0.4062406	-0.07621061\\
0.4063406	-0.07651717\\
0.4064406	-0.07681729\\
0.4065407	-0.07711088\\
0.4066407	-0.07739784\\
0.4067407	-0.07767813\\
0.4068407	-0.07795162\\
0.4069407	-0.0782182\\
0.4070407	-0.07847782\\
0.4071407	-0.07873041\\
0.4072407	-0.07897579\\
0.4073407	-0.07921393\\
0.4074407	-0.07944478\\
0.4075408	-0.07966814\\
0.4076408	-0.07988405\\
0.4077408	-0.08009234\\
0.4078408	-0.08029294\\
0.4079408	-0.08048579\\
0.4080408	-0.08067076\\
0.4081408	-0.08084778\\
0.4082408	-0.08101673\\
0.4083408	-0.08117755\\
0.4084408	-0.08133019\\
0.4085409	-0.08147449\\
0.4086409	-0.08161042\\
0.4087409	-0.08173791\\
0.4088409	-0.08185683\\
0.4089409	-0.08196704\\
0.4090409	-0.08206862\\
0.4091409	-0.08216131\\
0.4092409	-0.08224514\\
0.4093409	-0.08232\\
0.4094409	-0.08238577\\
0.409541	-0.08244241\\
0.409641	-0.08248984\\
0.409741	-0.0825279\\
0.409841	-0.08255661\\
0.409941	-0.08257584\\
0.410041	-0.08258558\\
0.410141	-0.08258561\\
0.410241	-0.08257596\\
0.410341	-0.08255654\\
0.410441	-0.08252725\\
0.4105411	-0.08248802\\
0.4106411	-0.08243879\\
0.4107411	-0.08237949\\
0.4108411	-0.08230997\\
0.4109411	-0.08223022\\
0.4110411	-0.08214017\\
0.4111411	-0.08203976\\
0.4112411	-0.08192891\\
0.4113411	-0.08180751\\
0.4114411	-0.0816755\\
0.4115412	-0.08153284\\
0.4116412	-0.08137946\\
0.4117412	-0.08121528\\
0.4118412	-0.08104017\\
0.4119412	-0.08085424\\
0.4120412	-0.08065725\\
0.4121412	-0.08044923\\
0.4122412	-0.08023008\\
0.4123412	-0.07999973\\
0.4124412	-0.07975815\\
0.4125413	-0.07950529\\
0.4126413	-0.07924105\\
0.4127413	-0.07896541\\
0.4128413	-0.07867823\\
0.4129413	-0.07837959\\
0.4130413	-0.07806934\\
0.4131413	-0.07774744\\
0.4132413	-0.07741386\\
0.4133413	-0.07706857\\
0.4134413	-0.07671147\\
0.4135414	-0.07634255\\
0.4136414	-0.07596171\\
0.4137414	-0.07556896\\
0.4138414	-0.07516425\\
0.4139414	-0.07474751\\
0.4140414	-0.07431868\\
0.4141414	-0.07387777\\
0.4142414	-0.07342473\\
0.4143414	-0.07295951\\
0.4144414	-0.07248205\\
0.4145415	-0.07199239\\
0.4146415	-0.07149042\\
0.4147415	-0.0709761\\
0.4148415	-0.07044947\\
0.4149415	-0.06991043\\
0.4150415	-0.06935901\\
0.4151415	-0.06879518\\
0.4152415	-0.06821887\\
0.4153415	-0.06763008\\
0.4154415	-0.06702876\\
0.4155416	-0.06641492\\
0.4156416	-0.06578862\\
0.4157416	-0.06514967\\
0.4158416	-0.06449818\\
0.4159416	-0.06383406\\
0.4160416	-0.06315739\\
0.4161416	-0.06246812\\
0.4162416	-0.06176619\\
0.4163416	-0.0610516\\
0.4164416	-0.06032444\\
0.4165417	-0.05958461\\
0.4166417	-0.05883216\\
0.4167417	-0.05806707\\
0.4168417	-0.05728933\\
0.4169417	-0.05649891\\
0.4170417	-0.0556959\\
0.4171417	-0.05488027\\
0.4172417	-0.05405203\\
0.4173417	-0.05321116\\
0.4174417	-0.05235774\\
0.4175418	-0.05149172\\
0.4176418	-0.05061313\\
0.4177418	-0.04972198\\
0.4178418	-0.04881831\\
0.4179418	-0.04790215\\
0.4180418	-0.04697356\\
0.4181418	-0.04603244\\
0.4182418	-0.04507895\\
0.4183418	-0.04411304\\
0.4184418	-0.04313476\\
0.4185419	-0.04214416\\
0.4186419	-0.04114128\\
0.4187419	-0.04012617\\
0.4188419	-0.0390988\\
0.4189419	-0.03805926\\
0.4190419	-0.03700764\\
0.4191419	-0.0359439\\
0.4192419	-0.03486813\\
0.4193419	-0.03378041\\
0.4194419	-0.03268073\\
0.419542	-0.03156921\\
0.419642	-0.03044585\\
0.419742	-0.02931074\\
0.419842	-0.02816394\\
0.419942	-0.02700551\\
0.420042	-0.02583552\\
0.420142	-0.02465401\\
0.420242	-0.02346112\\
0.420342	-0.02225683\\
0.420442	-0.02104129\\
0.4205421	-0.01981458\\
0.4206421	-0.01857673\\
0.4207421	-0.0173278\\
0.4208421	-0.01606794\\
0.4209421	-0.01479723\\
0.4210421	-0.01351572\\
0.4211421	-0.01222353\\
0.4212421	-0.01092075\\
0.4213421	-0.009607485\\
0.4214421	-0.008283789\\
0.4215422	-0.006949795\\
0.4216422	-0.005605651\\
0.4217422	-0.004251395\\
0.4218422	-0.002887115\\
0.4219422	-0.001512975\\
0.4220422	-0.0001290534\\
0.4221422	0.001264455\\
0.4222422	0.002667515\\
0.4223422	0.004080022\\
0.4224422	0.005501789\\
0.4225423	0.006932745\\
0.4226423	0.008372718\\
0.4227423	0.009821631\\
0.4228423	0.01127931\\
0.4229423	0.01274563\\
0.4230423	0.01422049\\
0.4231423	0.01570368\\
0.4232423	0.01719515\\
0.4233423	0.01869468\\
0.4234423	0.02020213\\
0.4235424	0.02171742\\
0.4236424	0.02324035\\
0.4237424	0.02477077\\
0.4238424	0.02630858\\
0.4239424	0.02785348\\
0.4240424	0.02940545\\
0.4241424	0.03096428\\
0.4242424	0.0325298\\
0.4243424	0.03410183\\
0.4244424	0.03568025\\
0.4245425	0.0372649\\
0.4246425	0.0388555\\
0.4247425	0.04045194\\
0.4248425	0.0420541\\
0.4249425	0.0436617\\
0.4250425	0.0452746\\
0.4251425	0.04689261\\
0.4252425	0.04851553\\
0.4253425	0.05014323\\
0.4254425	0.05177543\\
0.4255426	0.05341203\\
0.4256426	0.05505275\\
0.4257426	0.05669745\\
0.4258426	0.05834589\\
0.4259426	0.05999784\\
0.4260426	0.06165316\\
0.4261426	0.06331166\\
0.4262426	0.06497308\\
0.4263426	0.06663719\\
0.4264426	0.06830377\\
0.4265427	0.06997267\\
0.4266427	0.07164364\\
0.4267427	0.07331647\\
0.4268427	0.07499088\\
0.4269427	0.07666671\\
0.4270427	0.07834373\\
0.4271427	0.08002172\\
0.4272427	0.08170036\\
0.4273427	0.08337952\\
0.4274427	0.08505891\\
0.4275428	0.08673832\\
0.4276428	0.0884175\\
0.4277428	0.09009624\\
0.4278428	0.0917742\\
0.4279428	0.09345128\\
0.4280428	0.0951271\\
0.4281428	0.0968015\\
0.4282428	0.0984742\\
0.4283428	0.1001449\\
0.4284428	0.1018135\\
0.4285429	0.1034796\\
0.4286429	0.105143\\
0.4287429	0.1068033\\
0.4288429	0.1084605\\
0.4289429	0.1101142\\
0.4290429	0.1117641\\
0.4291429	0.11341\\
0.4292429	0.1150517\\
0.4293429	0.1166887\\
0.4294429	0.1183209\\
0.429543	0.1199481\\
0.429643	0.1215698\\
0.429743	0.123186\\
0.429843	0.1247962\\
0.429943	0.1264002\\
0.430043	0.1279978\\
0.430143	0.1295886\\
0.430243	0.1311724\\
0.430343	0.1327488\\
0.430443	0.1343178\\
0.4305431	0.1358788\\
0.4306431	0.1374316\\
0.4307431	0.1389761\\
0.4308431	0.1405117\\
0.4309431	0.1420384\\
0.4310431	0.1435559\\
0.4311431	0.1450636\\
0.4312431	0.1465615\\
0.4313431	0.1480492\\
0.4314431	0.1495266\\
0.4315432	0.1509931\\
0.4316432	0.1524485\\
0.4317432	0.1538926\\
0.4318432	0.1553251\\
0.4319432	0.1567457\\
0.4320432	0.1581541\\
0.4321432	0.1595498\\
0.4322432	0.1609329\\
0.4323432	0.1623028\\
0.4324432	0.1636592\\
0.4325433	0.165002\\
0.4326433	0.1663308\\
0.4327433	0.1676453\\
0.4328433	0.1689451\\
0.4329433	0.1702301\\
0.4330433	0.1714999\\
0.4331433	0.1727541\\
0.4332433	0.1739927\\
0.4333433	0.1752151\\
0.4334433	0.1764212\\
0.4335434	0.1776105\\
0.4336434	0.1787829\\
0.4337434	0.179938\\
0.4338434	0.1810755\\
0.4339434	0.1821952\\
0.4340434	0.1832967\\
0.4341434	0.1843797\\
0.4342434	0.1854441\\
0.4343434	0.1864893\\
0.4344434	0.1875153\\
0.4345435	0.1885216\\
0.4346435	0.1895079\\
0.4347435	0.1904741\\
0.4348435	0.1914197\\
0.4349435	0.1923446\\
0.4350435	0.1932484\\
0.4351435	0.1941309\\
0.4352435	0.1949916\\
0.4353435	0.1958305\\
0.4354435	0.1966472\\
0.4355436	0.1974413\\
0.4356436	0.1982126\\
0.4357436	0.198961\\
0.4358436	0.1996859\\
0.4359436	0.2003873\\
0.4360436	0.2010647\\
0.4361436	0.201718\\
0.4362436	0.2023469\\
0.4363436	0.2029511\\
0.4364436	0.2035303\\
0.4365437	0.2040843\\
0.4366437	0.2046127\\
0.4367437	0.2051155\\
0.4368437	0.2055922\\
0.4369437	0.2060426\\
0.4370437	0.2064665\\
0.4371437	0.2068635\\
0.4372437	0.2072336\\
0.4373437	0.2075764\\
0.4374437	0.2078916\\
0.4375438	0.2081791\\
0.4376438	0.2084386\\
0.4377438	0.2086698\\
0.4378438	0.2088725\\
0.4379438	0.2090464\\
0.4380438	0.2091915\\
0.4381438	0.2093073\\
0.4382438	0.2093938\\
0.4383438	0.2094506\\
0.4384438	0.2094776\\
0.4385439	0.2094745\\
0.4386439	0.2094411\\
0.4387439	0.2093773\\
0.4388439	0.2092827\\
0.4389439	0.2091573\\
0.4390439	0.2090008\\
0.4391439	0.208813\\
0.4392439	0.2085937\\
0.4393439	0.2083428\\
0.4394439	0.20806\\
0.439544	0.2077451\\
0.439644	0.2073981\\
0.439744	0.2070187\\
0.439844	0.2066068\\
0.439944	0.206162\\
0.440044	0.2056845\\
0.440144	0.2051739\\
0.440244	0.2046302\\
0.440344	0.204053\\
0.440444	0.2034425\\
0.4405441	0.2027983\\
0.4406441	0.2021203\\
0.4407441	0.2014084\\
0.4408441	0.2006626\\
0.4409441	0.1998826\\
0.4410441	0.1990683\\
0.4411441	0.1982197\\
0.4412441	0.1973366\\
0.4413441	0.196419\\
0.4414441	0.1954666\\
0.4415442	0.1944795\\
0.4416442	0.1934576\\
0.4417442	0.1924007\\
0.4418442	0.1913087\\
0.4419442	0.1901817\\
0.4420442	0.1890195\\
0.4421442	0.1878221\\
0.4422442	0.1865894\\
0.4423442	0.1853214\\
0.4424442	0.1840181\\
0.4425443	0.1826792\\
0.4426443	0.181305\\
0.4427443	0.1798952\\
0.4428443	0.1784499\\
0.4429443	0.1769692\\
0.4430443	0.1754528\\
0.4431443	0.173901\\
0.4432443	0.1723135\\
0.4433443	0.1706906\\
0.4434443	0.1690321\\
0.4435444	0.167338\\
0.4436444	0.1656085\\
0.4437444	0.1638435\\
0.4438444	0.1620431\\
0.4439444	0.1602073\\
0.4440444	0.1583361\\
0.4441444	0.1564297\\
0.4442444	0.154488\\
0.4443444	0.1525111\\
0.4444444	0.1504992\\
0.4445445	0.1484522\\
0.4446445	0.1463704\\
0.4447445	0.1442536\\
0.4448445	0.1421022\\
0.4449445	0.1399161\\
0.4450445	0.1376955\\
0.4451445	0.1354405\\
0.4452445	0.1331511\\
0.4453445	0.1308276\\
0.4454445	0.1284702\\
0.4455446	0.1260788\\
0.4456446	0.1236537\\
0.4457446	0.1211951\\
0.4458446	0.118703\\
0.4459446	0.1161778\\
0.4460446	0.1136195\\
0.4461446	0.1110283\\
0.4462446	0.1084045\\
0.4463446	0.1057482\\
0.4464446	0.1030596\\
0.4465447	0.100339\\
0.4466447	0.09758656\\
0.4467447	0.09480251\\
0.4468447	0.09198708\\
0.4469447	0.08914053\\
0.4470447	0.08626308\\
0.4471447	0.08335506\\
0.4472447	0.08041666\\
0.4473447	0.07744816\\
0.4474447	0.07444985\\
0.4475448	0.07142199\\
0.4476448	0.0683649\\
0.4477448	0.06527887\\
0.4478448	0.06216424\\
0.4479448	0.05902126\\
0.4480448	0.05585025\\
0.4481448	0.05265153\\
0.4482448	0.04942553\\
0.4483448	0.04617254\\
0.4484448	0.04289285\\
0.4485449	0.03958686\\
0.4486449	0.03625496\\
0.4487449	0.03289745\\
0.4488449	0.02951477\\
0.4489449	0.02610728\\
0.4490449	0.02267536\\
0.4491449	0.0192194\\
0.4492449	0.01573984\\
0.4493449	0.0122371\\
0.4494449	0.008711544\\
0.449545	0.005163638\\
0.449645	0.001593761\\
0.449745	-0.001997578\\
0.449845	-0.005609951\\
0.449945	-0.009242931\\
0.450045	-0.01289605\\
0.450145	-0.01656883\\
0.450245	-0.02026081\\
0.450345	-0.02397145\\
0.450445	-0.02770034\\
0.4505451	-0.03144697\\
0.4506451	-0.03521088\\
0.4507451	-0.03899144\\
0.4508451	-0.04278825\\
0.4509451	-0.04660077\\
0.4510451	-0.05042837\\
0.4511451	-0.05427064\\
0.4512451	-0.05812701\\
0.4513451	-0.06199694\\
0.4514451	-0.06587983\\
0.4515452	-0.06977518\\
0.4516452	-0.07368237\\
0.4517452	-0.07760089\\
0.4518452	-0.08153011\\
0.4519452	-0.08546945\\
0.4520452	-0.08941838\\
0.4521452	-0.09337626\\
0.4522452	-0.09734247\\
0.4523452	-0.1013165\\
0.4524452	-0.1052975\\
0.4525453	-0.1092851\\
0.4526453	-0.1132787\\
0.4527453	-0.1172775\\
0.4528453	-0.1212809\\
0.4529453	-0.1252883\\
0.4530453	-0.129299\\
0.4531453	-0.1333124\\
0.4532453	-0.1373279\\
0.4533453	-0.1413448\\
0.4534453	-0.1453624\\
0.4535454	-0.14938\\
0.4536454	-0.153397\\
0.4537454	-0.1574127\\
0.4538454	-0.1614264\\
0.4539454	-0.1654373\\
0.4540454	-0.169445\\
0.4541454	-0.1734485\\
0.4542454	-0.1774473\\
0.4543454	-0.1814406\\
0.4544454	-0.1854277\\
0.4545455	-0.1894079\\
0.4546455	-0.1933804\\
0.4547455	-0.1973446\\
0.4548455	-0.2012997\\
0.4549455	-0.2052451\\
0.4550455	-0.2091799\\
0.4551455	-0.2131034\\
0.4552455	-0.2170149\\
0.4553455	-0.2209137\\
0.4554455	-0.224799\\
0.4555456	-0.22867\\
0.4556456	-0.232526\\
0.4557456	-0.2363663\\
0.4558456	-0.24019\\
0.4559456	-0.2439966\\
0.4560456	-0.247785\\
0.4561456	-0.2515547\\
0.4562456	-0.2553049\\
0.4563456	-0.2590346\\
0.4564456	-0.2627434\\
0.4565457	-0.2664302\\
0.4566457	-0.2700945\\
0.4567457	-0.2737353\\
0.4568457	-0.277352\\
0.4569457	-0.2809437\\
0.4570457	-0.2845097\\
0.4571457	-0.2880492\\
0.4572457	-0.2915614\\
0.4573457	-0.2950455\\
0.4574457	-0.2985008\\
0.4575458	-0.3019265\\
0.4576458	-0.3053218\\
0.4577458	-0.3086859\\
0.4578458	-0.312018\\
0.4579458	-0.3153174\\
0.4580458	-0.3185833\\
0.4581458	-0.3218149\\
0.4582458	-0.3250115\\
0.4583458	-0.3281722\\
0.4584458	-0.3312962\\
0.4585459	-0.3343829\\
0.4586459	-0.3374315\\
0.4587459	-0.340441\\
0.4588459	-0.3434109\\
0.4589459	-0.3463403\\
0.4590459	-0.3492285\\
0.4591459	-0.3520747\\
0.4592459	-0.3548781\\
0.4593459	-0.3576381\\
0.4594459	-0.3603537\\
0.459546	-0.3630243\\
0.459646	-0.3656492\\
0.459746	-0.3682275\\
0.459846	-0.3707586\\
0.459946	-0.3732416\\
0.460046	-0.375676\\
0.460146	-0.3780608\\
0.460246	-0.3803955\\
0.460346	-0.3826792\\
0.460446	-0.3849113\\
0.4605461	-0.3870911\\
0.4606461	-0.3892178\\
0.4607461	-0.3912907\\
0.4608461	-0.3933091\\
0.4609461	-0.3952724\\
0.4610461	-0.3971798\\
0.4611461	-0.3990307\\
0.4612461	-0.4008244\\
0.4613461	-0.4025602\\
0.4614461	-0.4042374\\
0.4615462	-0.4058554\\
0.4616462	-0.4074135\\
0.4617462	-0.4089112\\
0.4618462	-0.4103476\\
0.4619462	-0.4117223\\
0.4620462	-0.4130346\\
0.4621462	-0.4142838\\
0.4622462	-0.4154693\\
0.4623462	-0.4165906\\
0.4624462	-0.4176471\\
0.4625463	-0.4186381\\
0.4626463	-0.4195631\\
0.4627463	-0.4204215\\
0.4628463	-0.4212128\\
0.4629463	-0.4219363\\
0.4630463	-0.4225916\\
0.4631463	-0.4231781\\
0.4632463	-0.4236953\\
0.4633463	-0.4241427\\
0.4634463	-0.4245196\\
0.4635464	-0.4248258\\
0.4636464	-0.4250606\\
0.4637464	-0.4252237\\
0.4638464	-0.4253144\\
0.4639464	-0.4253323\\
0.4640464	-0.4252771\\
0.4641464	-0.4251483\\
0.4642464	-0.4249454\\
0.4643464	-0.4246681\\
0.4644464	-0.4243158\\
0.4645465	-0.4238883\\
0.4646465	-0.4233851\\
0.4647465	-0.422806\\
0.4648465	-0.4221504\\
0.4649465	-0.4214182\\
0.4650465	-0.4206088\\
0.4651465	-0.419722\\
0.4652465	-0.4187577\\
0.4653465	-0.4177152\\
0.4654465	-0.4165946\\
0.4655466	-0.4153954\\
0.4656466	-0.4141174\\
0.4657466	-0.4127603\\
0.4658466	-0.411324\\
0.4659466	-0.4098081\\
0.4660466	-0.4082126\\
0.4661466	-0.4065372\\
0.4662466	-0.4047817\\
0.4663466	-0.402946\\
0.4664466	-0.4010299\\
0.4665467	-0.3990332\\
0.4666467	-0.396956\\
0.4667467	-0.394798\\
0.4668467	-0.3925592\\
0.4669467	-0.3902394\\
0.4670467	-0.3878387\\
0.4671467	-0.385357\\
0.4672467	-0.3827943\\
0.4673467	-0.3801504\\
0.4674467	-0.3774255\\
0.4675468	-0.3746196\\
0.4676468	-0.3717326\\
0.4677468	-0.3687647\\
0.4678468	-0.3657159\\
0.4679468	-0.3625862\\
0.4680468	-0.3593758\\
0.4681468	-0.3560848\\
0.4682468	-0.3527132\\
0.4683468	-0.3492614\\
0.4684468	-0.3457293\\
0.4685469	-0.3421172\\
0.4686469	-0.3384252\\
0.4687469	-0.3346537\\
0.4688469	-0.3308027\\
0.4689469	-0.3268726\\
0.4690469	-0.3228636\\
0.4691469	-0.318776\\
0.4692469	-0.31461\\
0.4693469	-0.310366\\
0.4694469	-0.3060444\\
0.469547	-0.3016454\\
0.469647	-0.2971694\\
0.469747	-0.2926168\\
0.469847	-0.287988\\
0.469947	-0.2832834\\
0.470047	-0.2785034\\
0.470147	-0.2736486\\
0.470247	-0.2687193\\
0.470347	-0.263716\\
0.470447	-0.2586392\\
0.4705471	-0.2534895\\
0.4706471	-0.2482674\\
0.4707471	-0.2429734\\
0.4708471	-0.2376081\\
0.4709471	-0.2321722\\
0.4710471	-0.2266661\\
0.4711471	-0.2210906\\
0.4712471	-0.2154463\\
0.4713471	-0.2097337\\
0.4714471	-0.2039538\\
0.4715472	-0.198107\\
0.4716472	-0.192194\\
0.4717472	-0.1862158\\
0.4718472	-0.180173\\
0.4719472	-0.1740663\\
0.4720472	-0.1678965\\
0.4721472	-0.1616645\\
0.4722472	-0.155371\\
0.4723472	-0.1490169\\
0.4724472	-0.1426031\\
0.4725473	-0.1361304\\
0.4726473	-0.1295997\\
0.4727473	-0.1230118\\
0.4728473	-0.1163678\\
0.4729473	-0.1096686\\
0.4730473	-0.102915\\
0.4731473	-0.09610812\\
0.4732473	-0.08924891\\
0.4733473	-0.08233836\\
0.4734473	-0.07537754\\
0.4735474	-0.06836739\\
0.4736474	-0.06130906\\
0.4737474	-0.05420358\\
0.4738474	-0.04705206\\
0.4739474	-0.03985553\\
0.4740474	-0.03261522\\
0.4741474	-0.02533213\\
0.4742474	-0.01800752\\
0.4743474	-0.01064252\\
0.4744474	-0.003238291\\
0.4745475	0.004203904\\
0.4746475	0.01168296\\
0.4747475	0.01919757\\
0.4748475	0.02674654\\
0.4749475	0.03432858\\
0.4750475	0.04194244\\
0.4751475	0.04958679\\
0.4752475	0.05726041\\
0.4753475	0.06496194\\
0.4754475	0.07269\\
0.4755476	0.08044339\\
0.4756476	0.0882206\\
0.4757476	0.09602039\\
0.4758476	0.1038413\\
0.4759476	0.111682\\
0.4760476	0.119541\\
0.4761476	0.1274169\\
0.4762476	0.1353084\\
0.4763476	0.1432139\\
0.4764476	0.151132\\
0.4765477	0.1590612\\
0.4766477	0.167\\
0.4767477	0.1749471\\
0.4768477	0.1829008\\
0.4769477	0.1908596\\
0.4770477	0.1988221\\
0.4771477	0.2067866\\
0.4772477	0.2147518\\
0.4773477	0.2227159\\
0.4774477	0.2306774\\
0.4775478	0.2386349\\
0.4776478	0.2465866\\
0.4777478	0.254531\\
0.4778478	0.2624666\\
0.4779478	0.2703916\\
0.4780478	0.2783046\\
0.4781478	0.2862038\\
0.4782478	0.2940876\\
0.4783478	0.3019544\\
0.4784478	0.3098027\\
0.4785479	0.3176306\\
0.4786479	0.3254366\\
0.4787479	0.333219\\
0.4788479	0.3409762\\
0.4789479	0.3487064\\
0.4790479	0.356408\\
0.4791479	0.3640794\\
0.4792479	0.3717188\\
0.4793479	0.3793245\\
0.4794479	0.3868949\\
0.479548	0.3944282\\
0.479648	0.4019228\\
0.479748	0.4093771\\
0.479848	0.4167891\\
0.479948	0.4241573\\
0.480048	0.43148\\
0.480148	0.4387554\\
0.480248	0.4459819\\
0.480348	0.4531578\\
0.480448	0.4602812\\
0.4805481	0.4673506\\
0.4806481	0.4743642\\
0.4807481	0.4813204\\
0.4808481	0.4882173\\
0.4809481	0.4950533\\
0.4810481	0.5018268\\
0.4811481	0.508536\\
0.4812481	0.5151793\\
0.4813481	0.5217548\\
0.4814481	0.528261\\
0.4815482	0.5346961\\
0.4816482	0.5410586\\
0.4817482	0.5473467\\
0.4818482	0.5535587\\
0.4819482	0.559693\\
0.4820482	0.5657479\\
0.4821482	0.5717218\\
0.4822482	0.5776131\\
0.4823482	0.5834201\\
0.4824482	0.5891411\\
0.4825483	0.5947746\\
0.4826483	0.6003189\\
0.4827483	0.6057724\\
0.4828483	0.6111337\\
0.4829483	0.616401\\
0.4830483	0.6215728\\
0.4831483	0.6266475\\
0.4832483	0.6316235\\
0.4833483	0.6364995\\
0.4834483	0.6412738\\
0.4835484	0.6459448\\
0.4836484	0.6505111\\
0.4837484	0.6549713\\
0.4838484	0.6593237\\
0.4839484	0.6635671\\
0.4840484	0.6676998\\
0.4841484	0.6717205\\
0.4842484	0.6756278\\
0.4843484	0.6794202\\
0.4844484	0.6830964\\
0.4845485	0.686655\\
0.4846485	0.6900946\\
0.4847485	0.6934139\\
0.4848485	0.6966116\\
0.4849485	0.6996864\\
0.4850485	0.7026369\\
0.4851485	0.705462\\
0.4852485	0.7081603\\
0.4853485	0.7107308\\
0.4854485	0.713172\\
0.4855486	0.7154829\\
0.4856486	0.7176623\\
0.4857486	0.719709\\
0.4858486	0.721622\\
0.4859486	0.7234\\
0.4860486	0.7250421\\
0.4861486	0.7265472\\
0.4862486	0.7279142\\
0.4863486	0.7291421\\
0.4864486	0.7302301\\
0.4865487	0.7311769\\
0.4866487	0.7319818\\
0.4867487	0.7326438\\
0.4868487	0.7331621\\
0.4869487	0.7335358\\
0.4870487	0.7337639\\
0.4871487	0.7338458\\
0.4872487	0.7337806\\
0.4873487	0.7335677\\
0.4874487	0.7332061\\
0.4875488	0.7326953\\
0.4876488	0.7320346\\
0.4877488	0.7312233\\
0.4878488	0.7302608\\
0.4879488	0.7291465\\
0.4880488	0.7278799\\
0.4881488	0.7264605\\
0.4882488	0.7248876\\
0.4883488	0.7231608\\
0.4884488	0.7212799\\
0.4885489	0.7192442\\
0.4886489	0.7170534\\
0.4887489	0.7147073\\
0.4888489	0.7122054\\
0.4889489	0.7095474\\
0.4890489	0.7067332\\
0.4891489	0.7037626\\
0.4892489	0.7006352\\
0.4893489	0.6973511\\
0.4894489	0.6939099\\
0.489549	0.6903118\\
0.489649	0.6865565\\
0.489749	0.6826442\\
0.489849	0.6785748\\
0.489949	0.6743484\\
0.490049	0.6699649\\
0.490149	0.6654247\\
0.490249	0.6607278\\
0.490349	0.6558743\\
0.490449	0.6508647\\
0.4905491	0.6456989\\
0.4906491	0.6403775\\
0.4907491	0.6349008\\
0.4908491	0.6292689\\
0.4909491	0.6234825\\
0.4910491	0.6175419\\
0.4911491	0.6114476\\
0.4912491	0.6052001\\
0.4913491	0.5987999\\
0.4914491	0.5922477\\
0.4915492	0.5855441\\
0.4916492	0.5786897\\
0.4917492	0.5716852\\
0.4918492	0.5645314\\
0.4919492	0.5572289\\
0.4920492	0.5497788\\
0.4921492	0.5421817\\
0.4922492	0.5344385\\
0.4923492	0.5265502\\
0.4924492	0.5185177\\
0.4925493	0.5103421\\
0.4926493	0.5020243\\
0.4927493	0.4935654\\
0.4928493	0.4849665\\
0.4929493	0.4762288\\
0.4930493	0.4673534\\
0.4931493	0.4583416\\
0.4932493	0.4491946\\
0.4933493	0.4399137\\
0.4934493	0.4305002\\
0.4935494	0.4209555\\
0.4936494	0.4112811\\
0.4937494	0.4014782\\
0.4938494	0.3915485\\
0.4939494	0.3814933\\
0.4940494	0.3713143\\
0.4941494	0.3610131\\
0.4942494	0.3505913\\
0.4943494	0.3400505\\
0.4944494	0.3293923\\
0.4945495	0.3186187\\
0.4946495	0.3077312\\
0.4947495	0.2967318\\
0.4948495	0.2856222\\
0.4949495	0.2744043\\
0.4950495	0.26308\\
0.4951495	0.2516512\\
0.4952495	0.24012\\
0.4953495	0.2284883\\
0.4954495	0.2167582\\
0.4955496	0.2049317\\
0.4956496	0.193011\\
0.4957496	0.1809981\\
0.4958496	0.1688953\\
0.4959496	0.1567048\\
0.4960496	0.1444287\\
0.4961496	0.1320695\\
0.4962496	0.1196292\\
0.4963496	0.1071104\\
0.4964496	0.09451531\\
0.4965497	0.08184641\\
0.4966497	0.06910599\\
0.4967497	0.05629663\\
0.4968497	0.04342082\\
0.4969497	0.03048096\\
0.4970497	0.0174796\\
0.4971497	0.004419388\\
0.4972497	-0.008697131\\
0.4973497	-0.02186736\\
0.4974497	-0.03508863\\
0.4975498	-0.04835826\\
0.4976498	-0.06167355\\
0.4977498	-0.07503182\\
0.4978498	-0.0884303\\
0.4979498	-0.1018662\\
0.4980498	-0.1153367\\
0.4981498	-0.1288391\\
0.4982498	-0.1423705\\
0.4983498	-0.155928\\
0.4984498	-0.1695087\\
0.4985499	-0.1831097\\
0.4986499	-0.1967282\\
0.4987499	-0.2103612\\
0.4988499	-0.2240056\\
0.4989499	-0.2376586\\
0.4990499	-0.251317\\
0.4991499	-0.264978\\
0.4992499	-0.2786384\\
0.4993499	-0.2922953\\
0.4994499	-0.3059454\\
0.49955	-0.3195859\\
0.49965	-0.3332135\\
0.49975	-0.3468252\\
0.49985	-0.3604177\\
0.49995	-0.3739881\\
0.50005	-0.3875331\\
0.50015	-0.4010497\\
0.50025	-0.4145345\\
0.50035	-0.4279845\\
0.50045	-0.4413964\\
0.5005501	-0.454767\\
0.5006501	-0.4680933\\
0.5007501	-0.4813719\\
0.5008501	-0.4945996\\
0.5009501	-0.5077732\\
0.5010501	-0.5208894\\
0.5011501	-0.5339452\\
0.5012501	-0.5469371\\
0.5013501	-0.559862\\
0.5014501	-0.5727167\\
0.5015502	-0.5854978\\
0.5016502	-0.5982022\\
0.5017502	-0.6108266\\
0.5018502	-0.6233678\\
0.5019502	-0.6358226\\
0.5020502	-0.6481877\\
0.5021502	-0.6604598\\
0.5022502	-0.6726359\\
0.5023502	-0.6847126\\
0.5024502	-0.6966867\\
0.5025503	-0.7085551\\
0.5026503	-0.7203145\\
0.5027503	-0.7319617\\
0.5028503	-0.7434937\\
0.5029503	-0.7549071\\
0.5030503	-0.7661989\\
0.5031503	-0.7773659\\
0.5032503	-0.788405\\
0.5033503	-0.799313\\
0.5034503	-0.8100868\\
0.5035504	-0.8207234\\
0.5036504	-0.8312198\\
0.5037504	-0.8415727\\
0.5038504	-0.8517793\\
0.5039504	-0.8618364\\
0.5040504	-0.8717411\\
0.5041504	-0.8814904\\
0.5042504	-0.8910813\\
0.5043504	-0.900511\\
0.5044504	-0.9097764\\
0.5045505	-0.9188748\\
0.5046505	-0.9278031\\
0.5047505	-0.9365587\\
0.5048505	-0.9451386\\
0.5049505	-0.9535402\\
0.5050505	-0.9617606\\
0.5051505	-0.9697971\\
0.5052505	-0.977647\\
0.5053505	-0.9853076\\
0.5054505	-0.9927764\\
0.5055506	-1.000051\\
0.5056506	-1.007128\\
0.5057506	-1.014005\\
0.5058506	-1.020681\\
0.5059506	-1.027152\\
0.5060506	-1.033415\\
0.5061506	-1.03947\\
0.5062506	-1.045312\\
0.5063506	-1.050941\\
0.5064506	-1.056353\\
0.5065507	-1.061546\\
0.5066507	-1.066519\\
0.5067507	-1.071269\\
0.5068507	-1.075793\\
0.5069507	-1.08009\\
0.5070507	-1.084159\\
0.5071507	-1.087996\\
0.5072507	-1.091599\\
0.5073507	-1.094968\\
0.5074507	-1.0981\\
0.5075508	-1.100993\\
0.5076508	-1.103646\\
0.5077508	-1.106057\\
0.5078508	-1.108224\\
0.5079508	-1.110145\\
0.5080508	-1.11182\\
0.5081508	-1.113246\\
0.5082508	-1.114423\\
0.5083508	-1.115348\\
0.5084508	-1.11602\\
0.5085509	-1.116439\\
0.5086509	-1.116603\\
0.5087509	-1.11651\\
0.5088509	-1.11616\\
0.5089509	-1.115551\\
0.5090509	-1.114683\\
0.5091509	-1.113554\\
0.5092509	-1.112165\\
0.5093509	-1.110513\\
0.5094509	-1.108598\\
0.509551	-1.106419\\
0.509651	-1.103977\\
0.509751	-1.101269\\
0.509851	-1.098296\\
0.509951	-1.095058\\
0.510051	-1.091553\\
0.510151	-1.087781\\
0.510251	-1.083743\\
0.510351	-1.079437\\
0.510451	-1.074865\\
0.5105511	-1.070025\\
0.5106511	-1.064918\\
0.5107511	-1.059544\\
0.5108511	-1.053902\\
0.5109511	-1.047994\\
0.5110511	-1.041819\\
0.5111511	-1.035377\\
0.5112511	-1.02867\\
0.5113511	-1.021697\\
0.5114511	-1.014459\\
0.5115512	-1.006956\\
0.5116512	-0.9991902\\
0.5117512	-0.9911609\\
0.5118512	-0.9828693\\
0.5119512	-0.9743163\\
0.5120512	-0.9655028\\
0.5121512	-0.9564298\\
0.5122512	-0.9470984\\
0.5123512	-0.9375097\\
0.5124512	-0.9276649\\
0.5125513	-0.9175653\\
0.5126513	-0.9072122\\
0.5127513	-0.896607\\
0.5128513	-0.8857512\\
0.5129513	-0.8746464\\
0.5130513	-0.8632942\\
0.5131513	-0.8516962\\
0.5132513	-0.8398543\\
0.5133513	-0.8277702\\
0.5134513	-0.8154459\\
0.5135514	-0.8028833\\
0.5136514	-0.7900844\\
0.5137514	-0.7770514\\
0.5138514	-0.7637864\\
0.5139514	-0.7502916\\
0.5140514	-0.7365695\\
0.5141514	-0.7226222\\
0.5142514	-0.7084524\\
0.5143514	-0.6940625\\
0.5144514	-0.6794551\\
0.5145515	-0.6646328\\
0.5146515	-0.6495983\\
0.5147515	-0.6343544\\
0.5148515	-0.618904\\
0.5149515	-0.60325\\
0.5150515	-0.5873953\\
0.5151515	-0.571343\\
0.5152515	-0.5550962\\
0.5153515	-0.538658\\
0.5154515	-0.5220318\\
0.5155516	-0.5052206\\
0.5156516	-0.4882281\\
0.5157516	-0.4710575\\
0.5158516	-0.4537123\\
0.5159516	-0.436196\\
0.5160516	-0.4185124\\
0.5161516	-0.400665\\
0.5162516	-0.3826575\\
0.5163516	-0.3644937\\
0.5164516	-0.3461775\\
0.5165517	-0.3277127\\
0.5166517	-0.3091033\\
0.5167517	-0.2903534\\
0.5168517	-0.2714668\\
0.5169517	-0.2524479\\
0.5170517	-0.2333006\\
0.5171517	-0.2140293\\
0.5172517	-0.1946381\\
0.5173517	-0.1751315\\
0.5174517	-0.1555137\\
0.5175518	-0.1357893\\
0.5176518	-0.1159627\\
0.5177518	-0.09603827\\
0.5178518	-0.07602075\\
0.5179518	-0.05591461\\
0.5180518	-0.03572456\\
0.5181518	-0.01545523\\
0.5182518	0.00488853\\
0.5183518	0.02530203\\
0.5184518	0.04578045\\
0.5185519	0.06631893\\
0.5186519	0.0869126\\
0.5187519	0.1075565\\
0.5188519	0.1282457\\
0.5189519	0.1489752\\
0.5190519	0.1697399\\
0.5191519	0.1905348\\
0.5192519	0.2113549\\
0.5193519	0.2321948\\
0.5194519	0.2530496\\
0.519552	0.2739141\\
0.519652	0.294783\\
0.519752	0.315651\\
0.519852	0.336513\\
0.519952	0.3573637\\
0.520052	0.3781977\\
0.520152	0.3990098\\
0.520252	0.4197946\\
0.520352	0.4405466\\
0.520452	0.4612607\\
0.5205521	0.4819313\\
0.5206521	0.5025532\\
0.5207521	0.5231207\\
0.5208521	0.5436286\\
0.5209521	0.5640714\\
0.5210521	0.5844437\\
0.5211521	0.6047399\\
0.5212521	0.6249548\\
0.5213521	0.6450827\\
0.5214521	0.6651182\\
0.5215522	0.685056\\
0.5216522	0.7048904\\
0.5217522	0.7246161\\
0.5218522	0.7442276\\
0.5219522	0.7637195\\
0.5220522	0.7830862\\
0.5221522	0.8023224\\
0.5222522	0.8214227\\
0.5223522	0.8403816\\
0.5224522	0.8591937\\
0.5225523	0.8778538\\
0.5226523	0.8963562\\
0.5227523	0.9146957\\
0.5228523	0.9328671\\
0.5229523	0.9508649\\
0.5230523	0.9686838\\
0.5231523	0.9863187\\
0.5232523	1.003764\\
0.5233523	1.021015\\
0.5234523	1.038066\\
0.5235524	1.054912\\
0.5236524	1.071548\\
0.5237524	1.087969\\
0.5238524	1.10417\\
0.5239524	1.120146\\
0.5240524	1.135891\\
0.5241524	1.151401\\
0.5242524	1.16667\\
0.5243524	1.181695\\
0.5244524	1.19647\\
0.5245525	1.21099\\
0.5246525	1.225251\\
0.5247525	1.239248\\
0.5248525	1.252976\\
0.5249525	1.266431\\
0.5250525	1.279608\\
0.5251525	1.292503\\
0.5252525	1.305111\\
0.5253525	1.317427\\
0.5254525	1.329449\\
0.5255526	1.34117\\
0.5256526	1.352588\\
0.5257526	1.363697\\
0.5258526	1.374495\\
0.5259526	1.384976\\
0.5260526	1.395138\\
0.5261526	1.404975\\
0.5262526	1.414484\\
0.5263526	1.423663\\
0.5264526	1.432506\\
0.5265527	1.44101\\
0.5266527	1.449172\\
0.5267527	1.456988\\
0.5268527	1.464456\\
0.5269527	1.471571\\
0.5270527	1.478331\\
0.5271527	1.484732\\
0.5272527	1.490771\\
0.5273527	1.496446\\
0.5274527	1.501753\\
0.5275528	1.506691\\
0.5276528	1.511255\\
0.5277528	1.515444\\
0.5278528	1.519255\\
0.5279528	1.522685\\
0.5280528	1.525733\\
0.5281528	1.528395\\
0.5282528	1.530671\\
0.5283528	1.532557\\
0.5284528	1.534052\\
0.5285529	1.535155\\
0.5286529	1.535862\\
0.5287529	1.536173\\
0.5288529	1.536087\\
0.5289529	1.535601\\
0.5290529	1.534714\\
0.5291529	1.533425\\
0.5292529	1.531733\\
0.5293529	1.529637\\
0.5294529	1.527136\\
0.529553	1.524229\\
0.529653	1.520916\\
0.529753	1.517195\\
0.529853	1.513066\\
0.529953	1.508529\\
0.530053	1.503584\\
0.530153	1.498231\\
0.530253	1.492468\\
0.530353	1.486297\\
0.530453	1.479718\\
0.5305531	1.472731\\
0.5306531	1.465335\\
0.5307531	1.457533\\
0.5308531	1.449323\\
0.5309531	1.440708\\
0.5310531	1.431688\\
0.5311531	1.422264\\
0.5312531	1.412436\\
0.5313531	1.402207\\
0.5314531	1.391577\\
0.5315532	1.380547\\
0.5316532	1.369121\\
0.5317532	1.357298\\
0.5318532	1.345081\\
0.5319532	1.332471\\
0.5320532	1.319472\\
0.5321532	1.306084\\
0.5322532	1.29231\\
0.5323532	1.278152\\
0.5324532	1.263613\\
0.5325533	1.248695\\
0.5326533	1.233402\\
0.5327533	1.217735\\
0.5328533	1.201698\\
0.5329533	1.185294\\
0.5330533	1.168526\\
0.5331533	1.151397\\
0.5332533	1.13391\\
0.5333533	1.11607\\
0.5334533	1.097879\\
0.5335534	1.079342\\
0.5336534	1.060462\\
0.5337534	1.041243\\
0.5338534	1.021689\\
0.5339534	1.001805\\
0.5340534	0.981594\\
0.5341534	0.9610612\\
0.5342534	0.9402109\\
0.5343534	0.9190476\\
0.5344534	0.8975761\\
0.5345535	0.8758012\\
0.5346535	0.8537276\\
0.5347535	0.8313606\\
0.5348535	0.808705\\
0.5349535	0.7857663\\
0.5350535	0.7625496\\
0.5351535	0.7390605\\
0.5352535	0.7153044\\
0.5353535	0.6912868\\
0.5354535	0.6670136\\
0.5355536	0.6424905\\
0.5356536	0.6177234\\
0.5357536	0.5927182\\
0.5358536	0.5674811\\
0.5359536	0.5420183\\
0.5360536	0.5163358\\
0.5361536	0.4904402\\
0.5362536	0.4643379\\
0.5363536	0.4380352\\
0.5364536	0.4115389\\
0.5365537	0.3848556\\
0.5366537	0.357992\\
0.5367537	0.330955\\
0.5368537	0.3037514\\
0.5369537	0.2763883\\
0.5370537	0.2488728\\
0.5371537	0.2212118\\
0.5372537	0.1934127\\
0.5373537	0.1654826\\
0.5374537	0.137429\\
0.5375538	0.1092591\\
0.5376538	0.08098051\\
0.5377538	0.05260066\\
0.5378538	0.02412712\\
0.5379538	-0.004432452\\
0.5380538	-0.03307036\\
0.5381538	-0.06177893\\
0.5382538	-0.09055035\\
0.5383538	-0.1193768\\
0.5384538	-0.1482503\\
0.5385539	-0.1771631\\
0.5386539	-0.2061072\\
0.5387539	-0.2350744\\
0.5388539	-0.2640569\\
0.5389539	-0.2930464\\
0.5390539	-0.3220351\\
0.5391539	-0.3510145\\
0.5392539	-0.3799767\\
0.5393539	-0.4089133\\
0.5394539	-0.4378162\\
0.539554	-0.4666772\\
0.539654	-0.4954878\\
0.539754	-0.52424\\
0.539854	-0.5529253\\
0.539954	-0.5815355\\
0.540054	-0.6100623\\
0.540154	-0.6384973\\
0.540254	-0.6668321\\
0.540354	-0.6950585\\
0.540454	-0.723168\\
0.5405541	-0.7511525\\
0.5406541	-0.7790034\\
0.5407541	-0.8067126\\
0.5408541	-0.8342716\\
0.5409541	-0.8616723\\
0.5410541	-0.8889062\\
0.5411541	-0.9159651\\
0.5412541	-0.9428408\\
0.5413541	-0.969525\\
0.5414541	-0.9960095\\
0.5415542	-1.022286\\
0.5416542	-1.048347\\
0.5417542	-1.074183\\
0.5418542	-1.099787\\
0.5419542	-1.125151\\
0.5420542	-1.150267\\
0.5421542	-1.175126\\
0.5422542	-1.199722\\
0.5423542	-1.224045\\
0.5424542	-1.248088\\
0.5425543	-1.271843\\
0.5426543	-1.295303\\
0.5427543	-1.318461\\
0.5428543	-1.341307\\
0.5429543	-1.363835\\
0.5430543	-1.386038\\
0.5431543	-1.407907\\
0.5432543	-1.429436\\
0.5433543	-1.450617\\
0.5434543	-1.471444\\
0.5435544	-1.491908\\
0.5436544	-1.512003\\
0.5437544	-1.531723\\
0.5438544	-1.551059\\
0.5439544	-1.570006\\
0.5440544	-1.588556\\
0.5441544	-1.606704\\
0.5442544	-1.624442\\
0.5443544	-1.641764\\
0.5444544	-1.658664\\
0.5445545	-1.675136\\
0.5446545	-1.691173\\
0.5447545	-1.706769\\
0.5448545	-1.72192\\
0.5449545	-1.736618\\
0.5450545	-1.750858\\
0.5451545	-1.764635\\
0.5452545	-1.777943\\
0.5453545	-1.790777\\
0.5454545	-1.803131\\
0.5455546	-1.815001\\
0.5456546	-1.826382\\
0.5457546	-1.837268\\
0.5458546	-1.847656\\
0.5459546	-1.85754\\
0.5460546	-1.866916\\
0.5461546	-1.87578\\
0.5462546	-1.884128\\
0.5463546	-1.891955\\
0.5464546	-1.899258\\
0.5465547	-1.906033\\
0.5466547	-1.912277\\
0.5467547	-1.917985\\
0.5468547	-1.923155\\
0.5469547	-1.927784\\
0.5470547	-1.931868\\
0.5471547	-1.935405\\
0.5472547	-1.938392\\
0.5473547	-1.940826\\
0.5474547	-1.942705\\
0.5475548	-1.944027\\
0.5476548	-1.94479\\
0.5477548	-1.944992\\
0.5478548	-1.94463\\
0.5479548	-1.943705\\
0.5480548	-1.942213\\
0.5481548	-1.940154\\
0.5482548	-1.937527\\
0.5483548	-1.934331\\
0.5484548	-1.930564\\
0.5485549	-1.926227\\
0.5486549	-1.92132\\
0.5487549	-1.915841\\
0.5488549	-1.909791\\
0.5489549	-1.903169\\
0.5490549	-1.895977\\
0.5491549	-1.888214\\
0.5492549	-1.879882\\
0.5493549	-1.87098\\
0.5494549	-1.86151\\
0.549555	-1.851474\\
0.549655	-1.840872\\
0.549755	-1.829705\\
0.549855	-1.817976\\
0.549955	-1.805687\\
0.550055	-1.792839\\
0.550155	-1.779435\\
0.550255	-1.765476\\
0.550355	-1.750966\\
0.550455	-1.735908\\
0.5505551	-1.720304\\
0.5506551	-1.704156\\
0.5507551	-1.68747\\
0.5508551	-1.670247\\
0.5509551	-1.652492\\
0.5510551	-1.634209\\
0.5511551	-1.6154\\
0.5512551	-1.596071\\
0.5513551	-1.576226\\
0.5514551	-1.555869\\
0.5515552	-1.535005\\
0.5516552	-1.513638\\
0.5517552	-1.491775\\
0.5518552	-1.469419\\
0.5519552	-1.446576\\
0.5520552	-1.423252\\
0.5521552	-1.399452\\
0.5522552	-1.375182\\
0.5523552	-1.350449\\
0.5524552	-1.325258\\
0.5525553	-1.299615\\
0.5526553	-1.273527\\
0.5527553	-1.247001\\
0.5528553	-1.220044\\
0.5529553	-1.192662\\
0.5530553	-1.164863\\
0.5531553	-1.136653\\
0.5532553	-1.108041\\
0.5533553	-1.079034\\
0.5534553	-1.049639\\
0.5535554	-1.019864\\
0.5536554	-0.9897175\\
0.5537554	-0.9592075\\
0.5538554	-0.9283422\\
0.5539554	-0.8971301\\
0.5540554	-0.8655795\\
0.5541554	-0.8336994\\
0.5542554	-0.8014984\\
0.5543554	-0.7689855\\
0.5544554	-0.7361697\\
0.5545555	-0.7030602\\
0.5546555	-0.6696665\\
0.5547555	-0.6359978\\
0.5548555	-0.6020637\\
0.5549555	-0.5678739\\
0.5550555	-0.5334382\\
0.5551555	-0.4987664\\
0.5552555	-0.4638685\\
0.5553555	-0.4287546\\
0.5554555	-0.3934349\\
0.5555556	-0.3579198\\
0.5556556	-0.3222194\\
0.5557556	-0.2863444\\
0.5558556	-0.2503054\\
0.5559556	-0.2141129\\
0.5560556	-0.1777776\\
0.5561556	-0.1413104\\
0.5562556	-0.1047222\\
0.5563556	-0.06802401\\
0.5564556	-0.0312267\\
0.5565557	0.005658474\\
0.5566557	0.04262052\\
0.5567557	0.07964808\\
0.5568557	0.1167299\\
0.5569557	0.1538547\\
0.5570557	0.1910111\\
0.5571557	0.2281876\\
0.5572557	0.2653728\\
0.5573557	0.3025552\\
0.5574557	0.3397232\\
0.5575558	0.3768652\\
0.5576558	0.4139697\\
0.5577558	0.451025\\
0.5578558	0.4880195\\
0.5579558	0.5249414\\
0.5580558	0.5617792\\
0.5581558	0.598521\\
0.5582558	0.6351553\\
0.5583558	0.6716704\\
0.5584558	0.7080543\\
0.5585559	0.7442956\\
0.5586559	0.7803825\\
0.5587559	0.8163034\\
0.5588559	0.8520465\\
0.5589559	0.8876003\\
0.5590559	0.9229531\\
0.5591559	0.9580933\\
0.5592559	0.9930094\\
0.5593559	1.02769\\
0.5594559	1.062123\\
0.559556	1.096298\\
0.559656	1.130203\\
0.559756	1.163826\\
0.559856	1.197157\\
0.559956	1.230184\\
0.560056	1.262896\\
0.560156	1.295281\\
0.560256	1.32733\\
0.560356	1.359031\\
0.560456	1.390372\\
0.5605561	1.421344\\
0.5606561	1.451935\\
0.5607561	1.482135\\
0.5608561	1.511934\\
0.5609561	1.54132\\
0.5610561	1.570284\\
0.5611561	1.598815\\
0.5612561	1.626904\\
0.5613561	1.65454\\
0.5614561	1.681713\\
0.5615562	1.708413\\
0.5616562	1.734631\\
0.5617562	1.760358\\
0.5618562	1.785584\\
0.5619562	1.810299\\
0.5620562	1.834495\\
0.5621562	1.858163\\
0.5622562	1.881293\\
0.5623562	1.903877\\
0.5624562	1.925907\\
0.5625563	1.947374\\
0.5626563	1.968269\\
0.5627563	1.988586\\
0.5628563	2.008315\\
0.5629563	2.02745\\
0.5630563	2.045982\\
0.5631563	2.063904\\
0.5632563	2.081209\\
0.5633563	2.097889\\
0.5634563	2.113939\\
0.5635564	2.129352\\
0.5636564	2.14412\\
0.5637564	2.158237\\
0.5638564	2.171698\\
0.5639564	2.184497\\
0.5640564	2.196627\\
0.5641564	2.208083\\
0.5642564	2.218861\\
0.5643564	2.228954\\
0.5644564	2.238359\\
0.5645565	2.24707\\
0.5646565	2.255082\\
0.5647565	2.262392\\
0.5648565	2.268996\\
0.5649565	2.274889\\
0.5650565	2.280069\\
0.5651565	2.284531\\
0.5652565	2.288273\\
0.5653565	2.291292\\
0.5654565	2.293584\\
0.5655566	2.295149\\
0.5656566	2.295982\\
0.5657566	2.296083\\
0.5658566	2.29545\\
0.5659566	2.29408\\
0.5660566	2.291974\\
0.5661566	2.289129\\
0.5662566	2.285545\\
0.5663566	2.281222\\
0.5664566	2.276158\\
0.5665567	2.270355\\
0.5666567	2.263812\\
0.5667567	2.256529\\
0.5668567	2.248508\\
0.5669567	2.239749\\
0.5670567	2.230254\\
0.5671567	2.220023\\
0.5672567	2.209058\\
0.5673567	2.197362\\
0.5674567	2.184936\\
0.5675568	2.171784\\
0.5676568	2.157906\\
0.5677568	2.143307\\
0.5678568	2.12799\\
0.5679568	2.111958\\
0.5680568	2.095214\\
0.5681568	2.077763\\
0.5682568	2.059608\\
0.5683568	2.040754\\
0.5684568	2.021206\\
0.5685569	2.000968\\
0.5686569	1.980046\\
0.5687569	1.958445\\
0.5688569	1.936171\\
0.5689569	1.913229\\
0.5690569	1.889625\\
0.5691569	1.865366\\
0.5692569	1.840459\\
0.5693569	1.814909\\
0.5694569	1.788725\\
0.569557	1.761913\\
0.569657	1.734481\\
0.569757	1.706437\\
0.569857	1.677788\\
0.569957	1.648543\\
0.570057	1.61871\\
0.570157	1.588298\\
0.570257	1.557315\\
0.570357	1.525771\\
0.570457	1.493675\\
0.5705571	1.461036\\
0.5706571	1.427864\\
0.5707571	1.394169\\
0.5708571	1.359961\\
0.5709571	1.325251\\
0.5710571	1.290048\\
0.5711571	1.254363\\
0.5712571	1.218208\\
0.5713571	1.181593\\
0.5714571	1.14453\\
0.5715572	1.10703\\
0.5716572	1.069104\\
0.5717572	1.030765\\
0.5718572	0.9920246\\
0.5719572	0.9528944\\
0.5720572	0.913387\\
0.5721572	0.8735149\\
0.5722572	0.8332907\\
0.5723572	0.792727\\
0.5724572	0.7518369\\
0.5725573	0.7106334\\
0.5726573	0.6691296\\
0.5727573	0.6273389\\
0.5728573	0.5852748\\
0.5729573	0.5429507\\
0.5730573	0.5003806\\
0.5731573	0.4575783\\
0.5732573	0.4145575\\
0.5733573	0.3713324\\
0.5734573	0.3279173\\
0.5735574	0.2843263\\
0.5736574	0.2405738\\
0.5737574	0.1966743\\
0.5738574	0.1526423\\
0.5739574	0.1084925\\
0.5740574	0.06423957\\
0.5741574	0.0198983\\
0.5742574	-0.02451645\\
0.5743574	-0.06898984\\
0.5744574	-0.1135068\\
0.5745575	-0.1580524\\
0.5746575	-0.2026115\\
0.5747575	-0.2471689\\
0.5748575	-0.2917096\\
0.5749575	-0.3362184\\
0.5750575	-0.38068\\
0.5751575	-0.4250792\\
0.5752575	-0.4694008\\
0.5753575	-0.5136295\\
0.5754575	-0.5577499\\
0.5755576	-0.6017469\\
0.5756576	-0.6456052\\
0.5757576	-0.6893095\\
0.5758576	-0.7328445\\
0.5759576	-0.7761951\\
0.5760576	-0.819346\\
0.5761576	-0.8622821\\
0.5762576	-0.9049883\\
0.5763576	-0.9474495\\
0.5764576	-0.9896505\\
0.5765577	-1.031577\\
0.5766577	-1.073213\\
0.5767577	-1.114544\\
0.5768577	-1.155555\\
0.5769577	-1.196232\\
0.5770577	-1.236561\\
0.5771577	-1.276525\\
0.5772577	-1.316112\\
0.5773577	-1.355306\\
0.5774577	-1.394094\\
0.5775578	-1.432461\\
0.5776578	-1.470392\\
0.5777578	-1.507876\\
0.5778578	-1.544896\\
0.5779578	-1.581441\\
0.5780578	-1.617495\\
0.5781578	-1.653047\\
0.5782578	-1.688082\\
0.5783578	-1.722587\\
0.5784578	-1.75655\\
0.5785579	-1.789958\\
0.5786579	-1.822797\\
0.5787579	-1.855057\\
0.5788579	-1.886723\\
0.5789579	-1.917785\\
0.5790579	-1.94823\\
0.5791579	-1.978047\\
0.5792579	-2.007224\\
0.5793579	-2.035749\\
0.5794579	-2.063612\\
0.579558	-2.090802\\
0.579658	-2.117307\\
0.579758	-2.143118\\
0.579858	-2.168224\\
0.579958	-2.192615\\
0.580058	-2.216281\\
0.580158	-2.239213\\
0.580258	-2.261401\\
0.580358	-2.282836\\
0.580458	-2.30351\\
0.5805581	-2.323413\\
0.5806581	-2.342537\\
0.5807581	-2.360875\\
0.5808581	-2.378418\\
0.5809581	-2.395159\\
0.5810581	-2.41109\\
0.5811581	-2.426204\\
0.5812581	-2.440496\\
0.5813581	-2.453957\\
0.5814581	-2.466582\\
0.5815582	-2.478365\\
0.5816582	-2.489301\\
0.5817582	-2.499383\\
0.5818582	-2.508608\\
0.5819582	-2.51697\\
0.5820582	-2.524464\\
0.5821582	-2.531087\\
0.5822582	-2.536834\\
0.5823582	-2.541703\\
0.5824582	-2.545689\\
0.5825583	-2.548791\\
0.5826583	-2.551004\\
0.5827583	-2.552328\\
0.5828583	-2.552759\\
0.5829583	-2.552297\\
0.5830583	-2.550939\\
0.5831583	-2.548686\\
0.5832583	-2.545535\\
0.5833583	-2.541487\\
0.5834583	-2.536542\\
0.5835584	-2.530699\\
0.5836584	-2.52396\\
0.5837584	-2.516325\\
0.5838584	-2.507795\\
0.5839584	-2.498373\\
0.5840584	-2.488059\\
0.5841584	-2.476856\\
0.5842584	-2.464767\\
0.5843584	-2.451794\\
0.5844584	-2.43794\\
0.5845585	-2.423209\\
0.5846585	-2.407606\\
0.5847585	-2.391132\\
0.5848585	-2.373795\\
0.5849585	-2.355597\\
0.5850585	-2.336545\\
0.5851585	-2.316643\\
0.5852585	-2.295898\\
0.5853585	-2.274315\\
0.5854585	-2.251901\\
0.5855586	-2.228663\\
0.5856586	-2.204607\\
0.5857586	-2.179741\\
0.5858586	-2.154072\\
0.5859586	-2.127609\\
0.5860586	-2.10036\\
0.5861586	-2.072332\\
0.5862586	-2.043536\\
0.5863586	-2.01398\\
0.5864586	-1.983674\\
0.5865587	-1.952627\\
0.5866587	-1.92085\\
0.5867587	-1.888352\\
0.5868587	-1.855145\\
0.5869587	-1.821239\\
0.5870587	-1.786645\\
0.5871587	-1.751376\\
0.5872587	-1.715442\\
0.5873587	-1.678855\\
0.5874587	-1.641629\\
0.5875588	-1.603775\\
0.5876588	-1.565305\\
0.5877588	-1.526235\\
0.5878588	-1.486575\\
0.5879588	-1.44634\\
0.5880588	-1.405544\\
0.5881588	-1.364201\\
0.5882588	-1.322324\\
0.5883588	-1.279929\\
0.5884588	-1.23703\\
0.5885589	-1.193641\\
0.5886589	-1.149779\\
0.5887589	-1.105458\\
0.5888589	-1.060693\\
0.5889589	-1.015501\\
0.5890589	-0.9698967\\
0.5891589	-0.923897\\
0.5892589	-0.877518\\
0.5893589	-0.8307759\\
0.5894589	-0.7836873\\
0.589559	-0.7362691\\
0.589659	-0.688538\\
0.589759	-0.6405113\\
0.589859	-0.5922059\\
0.589959	-0.5436394\\
0.590059	-0.494829\\
0.590159	-0.4457925\\
0.590259	-0.3965475\\
0.590359	-0.3471119\\
0.590459	-0.2975035\\
0.5905591	-0.2477404\\
0.5906591	-0.1978407\\
0.5907591	-0.1478226\\
0.5908591	-0.09770433\\
0.5909591	-0.04750434\\
0.5910591	0.002759024\\
0.5911591	0.05306725\\
0.5912591	0.1034017\\
0.5913591	0.153744\\
0.5914591	0.2040755\\
0.5915592	0.2543775\\
0.5916592	0.3046312\\
0.5917592	0.3548181\\
0.5918592	0.4049195\\
0.5919592	0.4549165\\
0.5920592	0.5047906\\
0.5921592	0.5545231\\
0.5922592	0.6040951\\
0.5923592	0.6534883\\
0.5924592	0.7026838\\
0.5925593	0.7516632\\
0.5926593	0.8004079\\
0.5927593	0.8488994\\
0.5928593	0.8971193\\
0.5929593	0.9450493\\
0.5930593	0.9926711\\
0.5931593	1.039967\\
0.5932593	1.086918\\
0.5933593	1.133506\\
0.5934593	1.179714\\
0.5935594	1.225524\\
0.5936594	1.270918\\
0.5937594	1.315879\\
0.5938594	1.360389\\
0.5939594	1.40443\\
0.5940594	1.447987\\
0.5941594	1.491041\\
0.5942594	1.533577\\
0.5943594	1.575577\\
0.5944594	1.617024\\
0.5945595	1.657904\\
0.5946595	1.698198\\
0.5947595	1.737893\\
0.5948595	1.776971\\
0.5949595	1.815417\\
0.5950595	1.853217\\
0.5951595	1.890354\\
0.5952595	1.926815\\
0.5953595	1.962584\\
0.5954595	1.997648\\
0.5955596	2.031991\\
0.5956596	2.0656\\
0.5957596	2.098462\\
0.5958596	2.130563\\
0.5959596	2.16189\\
0.5960596	2.19243\\
0.5961596	2.222171\\
0.5962596	2.2511\\
0.5963596	2.279206\\
0.5964596	2.306476\\
0.5965597	2.332899\\
0.5966597	2.358465\\
0.5967597	2.383163\\
0.5968597	2.406981\\
0.5969597	2.42991\\
0.5970597	2.451941\\
0.5971597	2.473062\\
0.5972597	2.493267\\
0.5973597	2.512545\\
0.5974597	2.530889\\
0.5975598	2.548289\\
0.5976598	2.564739\\
0.5977598	2.580231\\
0.5978598	2.594758\\
0.5979598	2.608313\\
0.5980598	2.620889\\
0.5981598	2.632482\\
0.5982598	2.643085\\
0.5983598	2.652693\\
0.5984598	2.661302\\
0.5985599	2.668906\\
0.5986599	2.675502\\
0.5987599	2.681086\\
0.5988599	2.685655\\
0.5989599	2.689206\\
0.5990599	2.691736\\
0.5991599	2.693243\\
0.5992599	2.693725\\
0.5993599	2.693182\\
0.5994599	2.691611\\
0.59956	2.689013\\
0.59966	2.685388\\
0.59976	2.680735\\
0.59986	2.675056\\
0.59996	2.668351\\
0.60006	2.660622\\
0.60016	2.651871\\
0.60026	2.642099\\
0.60036	2.63131\\
0.60046	2.619507\\
0.6005601	2.606693\\
0.6006601	2.592872\\
0.6007601	2.578048\\
0.6008601	2.562227\\
0.6009601	2.545412\\
0.6010601	2.527609\\
0.6011601	2.508825\\
0.6012601	2.489066\\
0.6013601	2.468338\\
0.6014601	2.446648\\
0.6015602	2.424004\\
0.6016602	2.400413\\
0.6017602	2.375884\\
0.6018602	2.350425\\
0.6019602	2.324046\\
0.6020602	2.296755\\
0.6021602	2.268562\\
0.6022602	2.239478\\
0.6023602	2.209513\\
0.6024602	2.178677\\
0.6025603	2.146982\\
0.6026603	2.114439\\
0.6027603	2.08106\\
0.6028603	2.046858\\
0.6029603	2.011845\\
0.6030603	1.976034\\
0.6031603	1.939438\\
0.6032603	1.90207\\
0.6033603	1.863946\\
0.6034603	1.825078\\
0.6035604	1.785481\\
0.6036604	1.745171\\
0.6037604	1.704162\\
0.6038604	1.66247\\
0.6039604	1.620111\\
0.6040604	1.5771\\
0.6041604	1.533454\\
0.6042604	1.48919\\
0.6043604	1.444324\\
0.6044604	1.398874\\
0.6045605	1.352857\\
0.6046605	1.30629\\
0.6047605	1.259192\\
0.6048605	1.21158\\
0.6049605	1.163473\\
0.6050605	1.11489\\
0.6051605	1.065849\\
0.6052605	1.01637\\
0.6053605	0.9664708\\
0.6054605	0.9161718\\
0.6055606	0.8654922\\
0.6056606	0.814452\\
0.6057606	0.763071\\
0.6058606	0.7113693\\
0.6059606	0.6593673\\
0.6060606	0.6070852\\
0.6061606	0.5545437\\
0.6062606	0.5017635\\
0.6063606	0.4487652\\
0.6064606	0.39557\\
0.6065607	0.3421988\\
0.6066607	0.2886727\\
0.6067607	0.2350129\\
0.6068607	0.1812408\\
0.6069607	0.1273778\\
0.6070607	0.07344522\\
0.6071607	0.01946463\\
0.6072607	-0.03454243\\
0.6073607	-0.08855438\\
0.6074607	-0.1425496\\
0.6075608	-0.1965064\\
0.6076608	-0.2504032\\
0.6077608	-0.3042183\\
0.6078608	-0.35793\\
0.6079608	-0.4115168\\
0.6080608	-0.464957\\
0.6081608	-0.518229\\
0.6082608	-0.5713113\\
0.6083608	-0.6241823\\
0.6084608	-0.6768208\\
0.6085609	-0.7292051\\
0.6086609	-0.7813142\\
0.6087609	-0.8331266\\
0.6088609	-0.8846215\\
0.6089609	-0.9357777\\
0.6090609	-0.9865743\\
0.6091609	-1.036991\\
0.6092609	-1.087006\\
0.6093609	-1.136599\\
0.6094609	-1.185751\\
0.609561	-1.234441\\
0.609661	-1.282649\\
0.609761	-1.330354\\
0.609861	-1.377538\\
0.609961	-1.424181\\
0.610061	-1.470263\\
0.610161	-1.515765\\
0.610261	-1.560669\\
0.610361	-1.604955\\
0.610461	-1.648606\\
0.6105611	-1.691603\\
0.6106611	-1.733929\\
0.6107611	-1.775564\\
0.6108611	-1.816493\\
0.6109611	-1.856698\\
0.6110611	-1.896162\\
0.6111611	-1.934868\\
0.6112611	-1.9728\\
0.6113611	-2.009942\\
0.6114611	-2.046279\\
0.6115612	-2.081794\\
0.6116612	-2.116474\\
0.6117612	-2.150302\\
0.6118612	-2.183265\\
0.6119612	-2.215349\\
0.6120612	-2.24654\\
0.6121612	-2.276825\\
0.6122612	-2.30619\\
0.6123612	-2.334623\\
0.6124612	-2.362112\\
0.6125613	-2.388644\\
0.6126613	-2.414209\\
0.6127613	-2.438794\\
0.6128613	-2.462391\\
0.6129613	-2.484987\\
0.6130613	-2.506573\\
0.6131613	-2.52714\\
0.6132613	-2.546678\\
0.6133613	-2.56518\\
0.6134613	-2.582635\\
0.6135614	-2.599038\\
0.6136614	-2.61438\\
0.6137614	-2.628654\\
0.6138614	-2.641853\\
0.6139614	-2.653973\\
0.6140614	-2.665006\\
0.6141614	-2.674948\\
0.6142614	-2.683795\\
0.6143614	-2.691541\\
0.6144614	-2.698183\\
0.6145615	-2.703717\\
0.6146615	-2.708141\\
0.6147615	-2.711452\\
0.6148615	-2.713648\\
0.6149615	-2.714728\\
0.6150615	-2.71469\\
0.6151615	-2.713534\\
0.6152615	-2.711259\\
0.6153615	-2.707867\\
0.6154615	-2.703357\\
0.6155616	-2.697732\\
0.6156616	-2.690992\\
0.6157616	-2.683141\\
0.6158616	-2.67418\\
0.6159616	-2.664113\\
0.6160616	-2.652943\\
0.6161616	-2.640675\\
0.6162616	-2.627314\\
0.6163616	-2.612863\\
0.6164616	-2.597329\\
0.6165617	-2.580717\\
0.6166617	-2.563035\\
0.6167617	-2.544288\\
0.6168617	-2.524484\\
0.6169617	-2.503632\\
0.6170617	-2.481738\\
0.6171617	-2.458813\\
0.6172617	-2.434864\\
0.6173617	-2.409902\\
0.6174617	-2.383936\\
0.6175618	-2.356977\\
0.6176618	-2.329037\\
0.6177618	-2.300125\\
0.6178618	-2.270254\\
0.6179618	-2.239436\\
0.6180618	-2.207683\\
0.6181618	-2.175009\\
0.6182618	-2.141427\\
0.6183618	-2.106951\\
0.6184618	-2.071594\\
0.6185619	-2.035371\\
0.6186619	-1.998298\\
0.6187619	-1.960389\\
0.6188619	-1.92166\\
0.6189619	-1.882128\\
0.6190619	-1.841808\\
0.6191619	-1.800718\\
0.6192619	-1.758874\\
0.6193619	-1.716294\\
0.6194619	-1.672995\\
0.619562	-1.628997\\
0.619662	-1.584316\\
0.619762	-1.538972\\
0.619862	-1.492984\\
0.619962	-1.446372\\
0.620062	-1.399153\\
0.620162	-1.35135\\
0.620262	-1.302981\\
0.620362	-1.254067\\
0.620462	-1.204629\\
0.6205621	-1.154687\\
0.6206621	-1.104263\\
0.6207621	-1.053379\\
0.6208621	-1.002054\\
0.6209621	-0.9503125\\
0.6210621	-0.8981749\\
0.6211621	-0.8456638\\
0.6212621	-0.7928014\\
0.6213621	-0.7396103\\
0.6214621	-0.6861132\\
0.6215622	-0.6323329\\
0.6216622	-0.5782923\\
0.6217622	-0.5240144\\
0.6218622	-0.4695226\\
0.6219622	-0.41484\\
0.6220622	-0.3599901\\
0.6221622	-0.3049965\\
0.6222622	-0.2498826\\
0.6223622	-0.1946722\\
0.6224622	-0.139389\\
0.6225623	-0.08405668\\
0.6226623	-0.0286991\\
0.6227623	0.02665985\\
0.6228623	0.08199645\\
0.6229623	0.1372867\\
0.6230623	0.1925069\\
0.6231623	0.2476331\\
0.6232623	0.3026415\\
0.6233623	0.3575082\\
0.6234623	0.4122096\\
0.6235624	0.4667221\\
0.6236624	0.5210219\\
0.6237624	0.5750854\\
0.6238624	0.6288894\\
0.6239624	0.6824103\\
0.6240624	0.7356249\\
0.6241624	0.7885101\\
0.6242624	0.8410428\\
0.6243624	0.8932003\\
0.6244624	0.9449597\\
0.6245625	0.9962985\\
0.6246625	1.047194\\
0.6247625	1.097625\\
0.6248625	1.147568\\
0.6249625	1.197003\\
0.6250625	1.245906\\
0.6251625	1.294258\\
0.6252625	1.342036\\
0.6253625	1.38922\\
0.6254625	1.435789\\
0.6255626	1.481722\\
0.6256626	1.527\\
0.6257626	1.571603\\
0.6258626	1.615511\\
0.6259626	1.658704\\
0.6260626	1.701164\\
0.6261626	1.742872\\
0.6262626	1.78381\\
0.6263626	1.823959\\
0.6264626	1.863302\\
0.6265627	1.901821\\
0.6266627	1.9395\\
0.6267627	1.976321\\
0.6268627	2.012269\\
0.6269627	2.047328\\
0.6270627	2.081482\\
0.6271627	2.114715\\
0.6272627	2.147014\\
0.6273627	2.178364\\
0.6274627	2.20875\\
0.6275628	2.238161\\
0.6276628	2.266581\\
0.6277628	2.293999\\
0.6278628	2.320403\\
0.6279628	2.345781\\
0.6280628	2.370121\\
0.6281628	2.393413\\
0.6282628	2.415647\\
0.6283628	2.436811\\
0.6284628	2.456898\\
0.6285629	2.475898\\
0.6286629	2.493802\\
0.6287629	2.510602\\
0.6288629	2.526292\\
0.6289629	2.540863\\
0.6290629	2.55431\\
0.6291629	2.566626\\
0.6292629	2.577806\\
0.6293629	2.587845\\
0.6294629	2.596738\\
0.629563	2.604481\\
0.629663	2.611071\\
0.629763	2.616506\\
0.629863	2.620781\\
0.629963	2.623896\\
0.630063	2.625849\\
0.630163	2.626639\\
0.630263	2.626266\\
0.630363	2.62473\\
0.630463	2.622032\\
0.6305631	2.618172\\
0.6306631	2.613153\\
0.6307631	2.606976\\
0.6308631	2.599645\\
0.6309631	2.591163\\
0.6310631	2.581534\\
0.6311631	2.570762\\
0.6312631	2.558851\\
0.6313631	2.545808\\
0.6314631	2.531638\\
0.6315632	2.516347\\
0.6316632	2.499943\\
0.6317632	2.482432\\
0.6318632	2.463823\\
0.6319632	2.444124\\
0.6320632	2.423345\\
0.6321632	2.401493\\
0.6322632	2.37858\\
0.6323632	2.354616\\
0.6324632	2.329611\\
0.6325633	2.303577\\
0.6326633	2.276526\\
0.6327633	2.24847\\
0.6328633	2.219422\\
0.6329633	2.189395\\
0.6330633	2.158402\\
0.6331633	2.126458\\
0.6332633	2.093577\\
0.6333633	2.059775\\
0.6334633	2.025066\\
0.6335634	1.989466\\
0.6336634	1.952992\\
0.6337634	1.91566\\
0.6338634	1.877488\\
0.6339634	1.838492\\
0.6340634	1.798691\\
0.6341634	1.758102\\
0.6342634	1.716745\\
0.6343634	1.674637\\
0.6344634	1.631799\\
0.6345635	1.58825\\
0.6346635	1.54401\\
0.6347635	1.499099\\
0.6348635	1.453538\\
0.6349635	1.407347\\
0.6350635	1.360548\\
0.6351635	1.313162\\
0.6352635	1.265211\\
0.6353635	1.216717\\
0.6354635	1.167702\\
0.6355636	1.118188\\
0.6356636	1.068199\\
0.6357636	1.017757\\
0.6358636	0.9668857\\
0.6359636	0.9156081\\
0.6360636	0.8639479\\
0.6361636	0.8119287\\
0.6362636	0.7595747\\
0.6363636	0.7069098\\
0.6364636	0.6539583\\
0.6365637	0.6007446\\
0.6366637	0.5472931\\
0.6367637	0.4936284\\
0.6368637	0.4397754\\
0.6369637	0.3857588\\
0.6370637	0.3316034\\
0.6371637	0.2773343\\
0.6372637	0.2229765\\
0.6373637	0.1685551\\
0.6374637	0.1140951\\
0.6375638	0.05962165\\
0.6376638	0.005159925\\
0.6377638	-0.04926487\\
0.6378638	-0.1036277\\
0.6379638	-0.1579034\\
0.6380638	-0.212067\\
0.6381638	-0.2660935\\
0.6382638	-0.3199577\\
0.6383638	-0.3736349\\
0.6384638	-0.4271003\\
0.6385639	-0.4803291\\
0.6386639	-0.5332968\\
0.6387639	-0.5859788\\
0.6388639	-0.6383508\\
0.6389639	-0.6903886\\
0.6390639	-0.7420681\\
0.6391639	-0.7933654\\
0.6392639	-0.8442568\\
0.6393639	-0.8947187\\
0.6394639	-0.9447279\\
0.639564	-0.994261\\
0.639664	-1.043295\\
0.639764	-1.091808\\
0.639864	-1.139777\\
0.639964	-1.18718\\
0.640064	-1.233994\\
0.640164	-1.280199\\
0.640264	-1.325773\\
0.640364	-1.370695\\
0.640464	-1.414943\\
0.6405641	-1.458499\\
0.6406641	-1.50134\\
0.6407641	-1.543449\\
0.6408641	-1.584805\\
0.6409641	-1.625389\\
0.6410641	-1.665182\\
0.6411641	-1.704166\\
0.6412641	-1.742324\\
0.6413641	-1.779637\\
0.6414641	-1.816088\\
0.6415642	-1.851661\\
0.6416642	-1.886339\\
0.6417642	-1.920106\\
0.6418642	-1.952947\\
0.6419642	-1.984847\\
0.6420642	-2.01579\\
0.6421642	-2.045763\\
0.6422642	-2.074753\\
0.6423642	-2.102745\\
0.6424642	-2.129727\\
0.6425643	-2.155687\\
0.6426643	-2.180613\\
0.6427643	-2.204493\\
0.6428643	-2.227318\\
0.6429643	-2.249075\\
0.6430643	-2.269757\\
0.6431643	-2.289352\\
0.6432643	-2.307854\\
0.6433643	-2.325252\\
0.6434643	-2.34154\\
0.6435644	-2.356711\\
0.6436644	-2.370757\\
0.6437644	-2.383672\\
0.6438644	-2.395452\\
0.6439644	-2.406091\\
0.6440644	-2.415583\\
0.6441644	-2.423927\\
0.6442644	-2.431117\\
0.6443644	-2.437151\\
0.6444644	-2.442028\\
0.6445645	-2.445744\\
0.6446645	-2.448299\\
0.6447645	-2.449693\\
0.6448645	-2.449925\\
0.6449645	-2.448996\\
0.6450645	-2.446906\\
0.6451645	-2.443658\\
0.6452645	-2.439253\\
0.6453645	-2.433694\\
0.6454645	-2.426985\\
0.6455646	-2.419129\\
0.6456646	-2.410131\\
0.6457646	-2.399996\\
0.6458646	-2.388729\\
0.6459646	-2.376335\\
0.6460646	-2.362823\\
0.6461646	-2.348198\\
0.6462646	-2.332468\\
0.6463646	-2.315643\\
0.6464646	-2.297729\\
0.6465647	-2.278737\\
0.6466647	-2.258676\\
0.6467647	-2.237557\\
0.6468647	-2.21539\\
0.6469647	-2.192187\\
0.6470647	-2.167959\\
0.6471647	-2.142718\\
0.6472647	-2.116478\\
0.6473647	-2.089252\\
0.6474647	-2.061053\\
0.6475648	-2.031895\\
0.6476648	-2.001794\\
0.6477648	-1.970764\\
0.6478648	-1.938821\\
0.6479648	-1.905981\\
0.6480648	-1.87226\\
0.6481648	-1.837676\\
0.6482648	-1.802245\\
0.6483648	-1.765985\\
0.6484648	-1.728915\\
0.6485649	-1.691052\\
0.6486649	-1.652416\\
0.6487649	-1.613026\\
0.6488649	-1.572902\\
0.6489649	-1.532063\\
0.6490649	-1.490529\\
0.6491649	-1.448323\\
0.6492649	-1.405463\\
0.6493649	-1.361973\\
0.6494649	-1.317872\\
0.649565	-1.273184\\
0.649665	-1.22793\\
0.649765	-1.182134\\
0.649865	-1.135816\\
0.649965	-1.089001\\
0.650065	-1.041712\\
0.650165	-0.9939715\\
0.650265	-0.9458039\\
0.650365	-0.8972328\\
0.650465	-0.8482823\\
0.6505651	-0.7989766\\
0.6506651	-0.7493401\\
0.6507651	-0.6993972\\
0.6508651	-0.6491728\\
0.6509651	-0.5986915\\
0.6510651	-0.5479784\\
0.6511651	-0.4970584\\
0.6512651	-0.4459566\\
0.6513651	-0.3946984\\
0.6514651	-0.3433091\\
0.6515652	-0.2918138\\
0.6516652	-0.2402381\\
0.6517652	-0.1886073\\
0.6518652	-0.1369469\\
0.6519652	-0.0852824\\
0.6520652	-0.03363909\\
0.6521652	0.01795744\\
0.6522652	0.06948191\\
0.6523652	0.1209089\\
0.6524652	0.1722132\\
0.6525653	0.2233696\\
0.6526653	0.2743529\\
0.6527653	0.3251381\\
0.6528653	0.3757002\\
0.6529653	0.4260144\\
0.6530653	0.4760561\\
0.6531653	0.5258006\\
0.6532653	0.5752237\\
0.6533653	0.6243011\\
0.6534653	0.6730088\\
0.6535654	0.7213229\\
0.6536654	0.7692198\\
0.6537654	0.8166761\\
0.6538654	0.8636686\\
0.6539654	0.9101743\\
0.6540654	0.9561707\\
0.6541654	1.001635\\
0.6542654	1.046546\\
0.6543654	1.09088\\
0.6544654	1.134617\\
0.6545655	1.177736\\
0.6546655	1.220214\\
0.6547655	1.262033\\
0.6548655	1.303171\\
0.6549655	1.343608\\
0.6550655	1.383325\\
0.6551655	1.422303\\
0.6552655	1.460524\\
0.6553655	1.497967\\
0.6554655	1.534616\\
0.6555656	1.570453\\
0.6556656	1.605461\\
0.6557656	1.639623\\
0.6558656	1.672922\\
0.6559656	1.705343\\
0.6560656	1.73687\\
0.6561656	1.767488\\
0.6562656	1.797183\\
0.6563656	1.825941\\
0.6564656	1.853748\\
0.6565657	1.88059\\
0.6566657	1.906457\\
0.6567657	1.931334\\
0.6568657	1.955211\\
0.6569657	1.978077\\
0.6570657	1.999921\\
0.6571657	2.020734\\
0.6572657	2.040504\\
0.6573657	2.059225\\
0.6574657	2.076886\\
0.6575658	2.093481\\
0.6576658	2.109002\\
0.6577658	2.123442\\
0.6578658	2.136795\\
0.6579658	2.149055\\
0.6580658	2.160217\\
0.6581658	2.170277\\
0.6582658	2.179231\\
0.6583658	2.187074\\
0.6584658	2.193805\\
0.6585659	2.19942\\
0.6586659	2.203919\\
0.6587659	2.2073\\
0.6588659	2.209563\\
0.6589659	2.210706\\
0.6590659	2.210732\\
0.6591659	2.209641\\
0.6592659	2.207435\\
0.6593659	2.204116\\
0.6594659	2.199687\\
0.659566	2.194152\\
0.659666	2.187513\\
0.659766	2.179777\\
0.659866	2.170947\\
0.659966	2.16103\\
0.660066	2.150032\\
0.660166	2.13796\\
0.660266	2.12482\\
0.660366	2.110621\\
0.660466	2.095371\\
0.6605661	2.079079\\
0.6606661	2.061754\\
0.6607661	2.043407\\
0.6608661	2.024048\\
0.6609661	2.003688\\
0.6610661	1.982338\\
0.6611661	1.960011\\
0.6612661	1.936718\\
0.6613661	1.912474\\
0.6614661	1.887291\\
0.6615662	1.861184\\
0.6616662	1.834167\\
0.6617662	1.806254\\
0.6618662	1.777461\\
0.6619662	1.747805\\
0.6620662	1.717301\\
0.6621662	1.685965\\
0.6622662	1.653815\\
0.6623662	1.620869\\
0.6624662	1.587144\\
0.6625663	1.552659\\
0.6626663	1.517432\\
0.6627663	1.481482\\
0.6628663	1.444828\\
0.6629663	1.407491\\
0.6630663	1.369491\\
0.6631663	1.330847\\
0.6632663	1.291582\\
0.6633663	1.251714\\
0.6634663	1.211267\\
0.6635664	1.170262\\
0.6636664	1.12872\\
0.6637664	1.086663\\
0.6638664	1.044115\\
0.6639664	1.001098\\
0.6640664	0.9576342\\
0.6641664	0.9137474\\
0.6642664	0.8694606\\
0.6643664	0.8247974\\
0.6644664	0.7797816\\
0.6645665	0.7344367\\
0.6646665	0.688787\\
0.6647665	0.6428565\\
0.6648665	0.5966696\\
0.6649665	0.5502505\\
0.6650665	0.5036238\\
0.6651665	0.4568141\\
0.6652665	0.4098459\\
0.6653665	0.362744\\
0.6654665	0.3155331\\
0.6655666	0.2682382\\
0.6656666	0.2208838\\
0.6657666	0.173495\\
0.6658666	0.1260965\\
0.6659666	0.07871319\\
0.6660666	0.03136983\\
0.6661666	-0.01590885\\
0.6662666	-0.06309822\\
0.6663666	-0.1101735\\
0.6664666	-0.1571103\\
0.6665667	-0.2038842\\
0.6666667	-0.2504707\\
0.6667667	-0.2968457\\
0.6668667	-0.342985\\
0.6669667	-0.3888647\\
0.6670667	-0.4344612\\
0.6671667	-0.4797507\\
0.6672667	-0.5247097\\
0.6673667	-0.5693151\\
0.6674667	-0.6135439\\
0.6675668	-0.6573731\\
0.6676668	-0.7007804\\
0.6677668	-0.7437433\\
0.6678668	-0.7862398\\
0.6679668	-0.828248\\
0.6680668	-0.8697464\\
0.6681668	-0.9107138\\
0.6682668	-0.9511292\\
0.6683668	-0.9909721\\
0.6684668	-1.030222\\
0.6685669	-1.068859\\
0.6686669	-1.106864\\
0.6687669	-1.144217\\
0.6688669	-1.180899\\
0.6689669	-1.216892\\
0.6690669	-1.252178\\
0.6691669	-1.286739\\
0.6692669	-1.320558\\
0.6693669	-1.353617\\
0.6694669	-1.3859\\
0.669567	-1.417391\\
0.669667	-1.448075\\
0.669767	-1.477937\\
0.669867	-1.50696\\
0.669967	-1.535132\\
0.670067	-1.562439\\
0.670167	-1.588866\\
0.670267	-1.614402\\
0.670367	-1.639034\\
0.670467	-1.66275\\
0.6705671	-1.685538\\
0.6706671	-1.707389\\
0.6707671	-1.728291\\
0.6708671	-1.748235\\
0.6709671	-1.767211\\
0.6710671	-1.785211\\
0.6711671	-1.802227\\
0.6712671	-1.818251\\
0.6713671	-1.833275\\
0.6714671	-1.847293\\
0.6715672	-1.860299\\
0.6716672	-1.872288\\
0.6717672	-1.883255\\
0.6718672	-1.893195\\
0.6719672	-1.902104\\
0.6720672	-1.90998\\
0.6721672	-1.916819\\
0.6722672	-1.922619\\
0.6723672	-1.927379\\
0.6724672	-1.931098\\
0.6725673	-1.933776\\
0.6726673	-1.935411\\
0.6727673	-1.936006\\
0.6728673	-1.935562\\
0.6729673	-1.934079\\
0.6730673	-1.931562\\
0.6731673	-1.928011\\
0.6732673	-1.923432\\
0.6733673	-1.917828\\
0.6734673	-1.911203\\
0.6735674	-1.903564\\
0.6736674	-1.894915\\
0.6737674	-1.885262\\
0.6738674	-1.874613\\
0.6739674	-1.862975\\
0.6740674	-1.850356\\
0.6741674	-1.836764\\
0.6742674	-1.822207\\
0.6743674	-1.806696\\
0.6744674	-1.79024\\
0.6745675	-1.77285\\
0.6746675	-1.754537\\
0.6747675	-1.735311\\
0.6748675	-1.715186\\
0.6749675	-1.694172\\
0.6750675	-1.672284\\
0.6751675	-1.649534\\
0.6752675	-1.625937\\
0.6753675	-1.601505\\
0.6754675	-1.576255\\
0.6755676	-1.550201\\
0.6756676	-1.523359\\
0.6757676	-1.495744\\
0.6758676	-1.467373\\
0.6759676	-1.438263\\
0.6760676	-1.40843\\
0.6761676	-1.377892\\
0.6762676	-1.346668\\
0.6763676	-1.314775\\
0.6764676	-1.282232\\
0.6765677	-1.249057\\
0.6766677	-1.21527\\
0.6767677	-1.180891\\
0.6768677	-1.145939\\
0.6769677	-1.110434\\
0.6770677	-1.074397\\
0.6771677	-1.037849\\
0.6772677	-1.00081\\
0.6773677	-0.9633015\\
0.6774677	-0.925345\\
0.6775678	-0.8869621\\
0.6776678	-0.8481745\\
0.6777678	-0.8090044\\
0.6778678	-0.769474\\
0.6779678	-0.7296056\\
0.6780678	-0.6894218\\
0.6781678	-0.6489452\\
0.6782678	-0.6081987\\
0.6783678	-0.5672053\\
0.6784678	-0.5259878\\
0.6785679	-0.4845697\\
0.6786679	-0.4429741\\
0.6787679	-0.4012242\\
0.6788679	-0.3593433\\
0.6789679	-0.317355\\
0.6790679	-0.2752826\\
0.6791679	-0.2331495\\
0.6792679	-0.1909792\\
0.6793679	-0.1487952\\
0.6794679	-0.1066208\\
0.679568	-0.06447942\\
0.679668	-0.02239431\\
0.679768	0.01961115\\
0.679868	0.06151387\\
0.679968	0.1032907\\
0.680068	0.1449187\\
0.680168	0.1863749\\
0.680268	0.2276367\\
0.680368	0.2686813\\
0.680468	0.3094864\\
0.6805681	0.3500296\\
0.6806681	0.3902888\\
0.6807681	0.4302421\\
0.6808681	0.4698677\\
0.6809681	0.5091442\\
0.6810681	0.5480502\\
0.6811681	0.5865648\\
0.6812681	0.624667\\
0.6813681	0.6623364\\
0.6814681	0.6995527\\
0.6815682	0.7362958\\
0.6816682	0.7725462\\
0.6817682	0.8082843\\
0.6818682	0.8434912\\
0.6819682	0.8781481\\
0.6820682	0.9122366\\
0.6821682	0.9457386\\
0.6822682	0.9786363\\
0.6823682	1.010913\\
0.6824682	1.04255\\
0.6825683	1.073533\\
0.6826683	1.103844\\
0.6827683	1.133469\\
0.6828683	1.16239\\
0.6829683	1.190595\\
0.6830683	1.218067\\
0.6831683	1.244794\\
0.6832683	1.270761\\
0.6833683	1.295955\\
0.6834683	1.320363\\
0.6835684	1.343974\\
0.6836684	1.366775\\
0.6837684	1.388754\\
0.6838684	1.409902\\
0.6839684	1.430208\\
0.6840684	1.449662\\
0.6841684	1.468254\\
0.6842684	1.485975\\
0.6843684	1.502818\\
0.6844684	1.518775\\
0.6845685	1.533838\\
0.6846685	1.548\\
0.6847685	1.561255\\
0.6848685	1.573597\\
0.6849685	1.585021\\
0.6850685	1.595523\\
0.6851685	1.605098\\
0.6852685	1.613743\\
0.6853685	1.621454\\
0.6854685	1.628229\\
0.6855686	1.634065\\
0.6856686	1.638963\\
0.6857686	1.642919\\
0.6858686	1.645935\\
0.6859686	1.64801\\
0.6860686	1.649145\\
0.6861686	1.649342\\
0.6862686	1.648601\\
0.6863686	1.646925\\
0.6864686	1.644317\\
0.6865687	1.640781\\
0.6866687	1.63632\\
0.6867687	1.630938\\
0.6868687	1.624641\\
0.6869687	1.617434\\
0.6870687	1.609323\\
0.6871687	1.600315\\
0.6872687	1.590416\\
0.6873687	1.579634\\
0.6874687	1.567976\\
0.6875688	1.555452\\
0.6876688	1.542071\\
0.6877688	1.527841\\
0.6878688	1.512773\\
0.6879688	1.496876\\
0.6880688	1.480163\\
0.6881688	1.462643\\
0.6882688	1.44433\\
0.6883688	1.425234\\
0.6884688	1.40537\\
0.6885689	1.384748\\
0.6886689	1.363384\\
0.6887689	1.341291\\
0.6888689	1.318483\\
0.6889689	1.294976\\
0.6890689	1.270782\\
0.6891689	1.24592\\
0.6892689	1.220403\\
0.6893689	1.194248\\
0.6894689	1.167472\\
0.689569	1.140092\\
0.689669	1.112124\\
0.689769	1.083586\\
0.689869	1.054496\\
0.689969	1.024871\\
0.690069	0.9947304\\
0.690169	0.9640924\\
0.690269	0.9329758\\
0.690369	0.9013996\\
0.690469	0.8693832\\
0.6905691	0.8369462\\
0.6906691	0.8041082\\
0.6907691	0.7708892\\
0.6908691	0.7373093\\
0.6909691	0.7033889\\
0.6910691	0.6691484\\
0.6911691	0.6346084\\
0.6912691	0.5997897\\
0.6913691	0.5647132\\
0.6914691	0.5293998\\
0.6915692	0.4938707\\
0.6916692	0.4581472\\
0.6917692	0.4222504\\
0.6918692	0.3862016\\
0.6919692	0.3500223\\
0.6920692	0.3137339\\
0.6921692	0.2773579\\
0.6922692	0.2409156\\
0.6923692	0.2044286\\
0.6924692	0.1679183\\
0.6925693	0.1314061\\
0.6926693	0.0949135\\
0.6927693	0.05846173\\
0.6928693	0.02207208\\
0.6929693	-0.01423421\\
0.6930693	-0.05043612\\
0.6931693	-0.08651253\\
0.6932693	-0.1224426\\
0.6933693	-0.1582054\\
0.6934693	-0.1937806\\
0.6935694	-0.2291475\\
0.6936694	-0.2642858\\
0.6937694	-0.2991753\\
0.6938694	-0.3337961\\
0.6939694	-0.3681284\\
0.6940694	-0.4021527\\
0.6941694	-0.4358495\\
0.6942694	-0.4691998\\
0.6943694	-0.5021847\\
0.6944694	-0.5347855\\
0.6945695	-0.5669839\\
0.6946695	-0.5987619\\
0.6947695	-0.6301015\\
0.6948695	-0.6609852\\
0.6949695	-0.691396\\
0.6950695	-0.7213168\\
0.6951695	-0.7507311\\
0.6952695	-0.7796226\\
0.6953695	-0.8079755\\
0.6954695	-0.8357742\\
0.6955696	-0.8630035\\
0.6956696	-0.8896485\\
0.6957696	-0.9156949\\
0.6958696	-0.9411287\\
0.6959696	-0.965936\\
0.6960696	-0.9901036\\
0.6961696	-1.013619\\
0.6962696	-1.036469\\
0.6963696	-1.058642\\
0.6964696	-1.080127\\
0.6965697	-1.100911\\
0.6966697	-1.120986\\
0.6967697	-1.140339\\
0.6968697	-1.158962\\
0.6969697	-1.176845\\
0.6970697	-1.19398\\
0.6971697	-1.210356\\
0.6972697	-1.225968\\
0.6973697	-1.240807\\
0.6974697	-1.254866\\
0.6975698	-1.268139\\
0.6976698	-1.280619\\
0.6977698	-1.292301\\
0.6978698	-1.303181\\
0.6979698	-1.313253\\
0.6980698	-1.322513\\
0.6981698	-1.330957\\
0.6982698	-1.338584\\
0.6983698	-1.345389\\
0.6984698	-1.351372\\
0.6985699	-1.35653\\
0.6986699	-1.360862\\
0.6987699	-1.364368\\
0.6988699	-1.367047\\
0.6989699	-1.3689\\
0.6990699	-1.369928\\
0.6991699	-1.370133\\
0.6992699	-1.369516\\
0.6993699	-1.368079\\
0.6994699	-1.365826\\
0.69957	-1.362759\\
0.69967	-1.358884\\
0.69977	-1.354203\\
0.69987	-1.348722\\
0.69997	-1.342446\\
0.70007	-1.335381\\
0.70017	-1.327533\\
0.70027	-1.318909\\
0.70037	-1.309516\\
0.70047	-1.299362\\
0.7005701	-1.288454\\
0.7006701	-1.276801\\
0.7007701	-1.264412\\
0.7008701	-1.251296\\
0.7009701	-1.237464\\
0.7010701	-1.222924\\
0.7011701	-1.207689\\
0.7012701	-1.191768\\
0.7013701	-1.175174\\
0.7014701	-1.157918\\
0.7015702	-1.140012\\
0.7016702	-1.121468\\
0.7017702	-1.102301\\
0.7018702	-1.082522\\
0.7019702	-1.062145\\
0.7020702	-1.041185\\
0.7021702	-1.019656\\
0.7022702	-0.9975718\\
0.7023702	-0.9749476\\
0.7024702	-0.9517986\\
0.7025703	-0.9281404\\
0.7026703	-0.9039887\\
0.7027703	-0.8793593\\
0.7028703	-0.8542687\\
0.7029703	-0.8287333\\
0.7030703	-0.8027699\\
0.7031703	-0.7763955\\
0.7032703	-0.7496273\\
0.7033703	-0.7224826\\
0.7034703	-0.6949791\\
0.7035704	-0.6671345\\
0.7036704	-0.6389667\\
0.7037704	-0.6104939\\
0.7038704	-0.5817344\\
0.7039704	-0.5527064\\
0.7040704	-0.5234286\\
0.7041704	-0.4939195\\
0.7042704	-0.464198\\
0.7043704	-0.4342826\\
0.7044704	-0.4041925\\
0.7045705	-0.3739465\\
0.7046705	-0.3435637\\
0.7047705	-0.313063\\
0.7048705	-0.2824637\\
0.7049705	-0.2517847\\
0.7050705	-0.2210452\\
0.7051705	-0.1902643\\
0.7052705	-0.1594611\\
0.7053705	-0.1286546\\
0.7054705	-0.09786396\\
0.7055706	-0.06710798\\
0.7056706	-0.03640564\\
0.7057706	-0.005775736\\
0.7058706	0.02476302\\
0.7059706	0.05519202\\
0.7060706	0.08549263\\
0.7061706	0.1156465\\
0.7062706	0.1456354\\
0.7063706	0.1754413\\
0.7064706	0.205046\\
0.7065707	0.2344318\\
0.7066707	0.2635811\\
0.7067707	0.2924765\\
0.7068707	0.3211008\\
0.7069707	0.3494368\\
0.7070707	0.3774678\\
0.7071707	0.4051772\\
0.7072707	0.4325486\\
0.7073707	0.4595661\\
0.7074707	0.4862137\\
0.7075708	0.5124759\\
0.7076708	0.5383373\\
0.7077708	0.5637831\\
0.7078708	0.5887985\\
0.7079708	0.613369\\
0.7080708	0.6374807\\
0.7081708	0.6611197\\
0.7082708	0.6842726\\
0.7083708	0.7069263\\
0.7084708	0.729068\\
0.7085709	0.7506853\\
0.7086709	0.7717663\\
0.7087709	0.7922991\\
0.7088709	0.8122725\\
0.7089709	0.8316756\\
0.7090709	0.8504977\\
0.7091709	0.8687288\\
0.7092709	0.886359\\
0.7093709	0.9033792\\
0.7094709	0.9197801\\
0.709571	0.9355534\\
0.709671	0.9506909\\
0.709771	0.9651849\\
0.709871	0.9790281\\
0.709971	0.9922137\\
0.710071	1.004735\\
0.710171	1.016587\\
0.710271	1.027763\\
0.710371	1.038258\\
0.710471	1.048067\\
0.7105711	1.057188\\
0.7106711	1.065614\\
0.7107711	1.073345\\
0.7108711	1.080375\\
0.7109711	1.086704\\
0.7110711	1.092328\\
0.7111711	1.097247\\
0.7112711	1.101459\\
0.7113711	1.104965\\
0.7114711	1.107763\\
0.7115712	1.109853\\
0.7116712	1.111237\\
0.7117712	1.111917\\
0.7118712	1.111892\\
0.7119712	1.111166\\
0.7120712	1.109741\\
0.7121712	1.107619\\
0.7122712	1.104805\\
0.7123712	1.101302\\
0.7124712	1.097115\\
0.7125713	1.092247\\
0.7126713	1.086704\\
0.7127713	1.080492\\
0.7128713	1.073616\\
0.7129713	1.066083\\
0.7130713	1.0579\\
0.7131713	1.049073\\
0.7132713	1.03961\\
0.7133713	1.029519\\
0.7134713	1.018809\\
0.7135714	1.007487\\
0.7136714	0.9955639\\
0.7137714	0.9830478\\
0.7138714	0.9699487\\
0.7139714	0.956277\\
0.7140714	0.9420428\\
0.7141714	0.927257\\
0.7142714	0.9119308\\
0.7143714	0.8960755\\
0.7144714	0.8797028\\
0.7145715	0.8628248\\
0.7146715	0.8454539\\
0.7147715	0.8276024\\
0.7148715	0.8092834\\
0.7149715	0.79051\\
0.7150715	0.7712956\\
0.7151715	0.7516539\\
0.7152715	0.7315987\\
0.7153715	0.7111442\\
0.7154715	0.6903047\\
0.7155716	0.6690948\\
0.7156716	0.6475291\\
0.7157716	0.6256227\\
0.7158716	0.6033907\\
0.7159716	0.5808484\\
0.7160716	0.5580112\\
0.7161716	0.5348946\\
0.7162716	0.5115146\\
0.7163716	0.4878869\\
0.7164716	0.4640275\\
0.7165717	0.4399525\\
0.7166717	0.4156782\\
0.7167717	0.3912207\\
0.7168717	0.3665964\\
0.7169717	0.3418217\\
0.7170717	0.3169132\\
0.7171717	0.2918874\\
0.7172717	0.2667608\\
0.7173717	0.2415498\\
0.7174717	0.2162712\\
0.7175718	0.1909414\\
0.7176718	0.1655772\\
0.7177718	0.1401949\\
0.7178718	0.1148111\\
0.7179718	0.08944226\\
0.7180718	0.06410478\\
0.7181718	0.03881503\\
0.7182718	0.01358926\\
0.7183718	-0.01155638\\
0.7184718	-0.03660569\\
0.7185719	-0.06154284\\
0.7186719	-0.08635188\\
0.7187719	-0.1110172\\
0.7188719	-0.135523\\
0.7189719	-0.1598539\\
0.7190719	-0.1839947\\
0.7191719	-0.2079302\\
0.7192719	-0.2316456\\
0.7193719	-0.2551258\\
0.7194719	-0.2783566\\
0.719572	-0.3013234\\
0.719672	-0.3240121\\
0.719772	-0.3464088\\
0.719872	-0.3684998\\
0.719972	-0.3902717\\
0.720072	-0.4117111\\
0.720172	-0.4328051\\
0.720272	-0.4535411\\
0.720372	-0.4739067\\
0.720472	-0.4938897\\
0.7205721	-0.5134781\\
0.7206721	-0.5326605\\
0.7207721	-0.5514256\\
0.7208721	-0.5697623\\
0.7209721	-0.5876601\\
0.7210721	-0.6051087\\
0.7211721	-0.622098\\
0.7212721	-0.6386183\\
0.7213721	-0.6546604\\
0.7214721	-0.6702152\\
0.7215722	-0.685274\\
0.7216722	-0.6998287\\
0.7217722	-0.7138711\\
0.7218722	-0.7273938\\
0.7219722	-0.7403896\\
0.7220722	-0.7528515\\
0.7221722	-0.7647732\\
0.7222722	-0.7761484\\
0.7223722	-0.7869716\\
0.7224722	-0.7972374\\
0.7225723	-0.8069408\\
0.7226723	-0.8160773\\
0.7227723	-0.8246426\\
0.7228723	-0.832633\\
0.7229723	-0.8400451\\
0.7230723	-0.8468761\\
0.7231723	-0.853123\\
0.7232723	-0.858784\\
0.7233723	-0.863857\\
0.7234723	-0.8683406\\
0.7235724	-0.872234\\
0.7236724	-0.8755363\\
0.7237724	-0.8782475\\
0.7238724	-0.8803677\\
0.7239724	-0.8818974\\
0.7240724	-0.8828375\\
0.7241724	-0.8831895\\
0.7242724	-0.8829551\\
0.7243724	-0.8821363\\
0.7244724	-0.8807357\\
0.7245725	-0.8787562\\
0.7246725	-0.8762009\\
0.7247725	-0.8730737\\
0.7248725	-0.8693783\\
0.7249725	-0.8651192\\
0.7250725	-0.8603012\\
0.7251725	-0.8549294\\
0.7252725	-0.8490091\\
0.7253725	-0.8425463\\
0.7254725	-0.835547\\
0.7255726	-0.8280178\\
0.7256726	-0.8199654\\
0.7257726	-0.8113971\\
0.7258726	-0.8023202\\
0.7259726	-0.7927426\\
0.7260726	-0.7826725\\
0.7261726	-0.7721182\\
0.7262726	-0.7610885\\
0.7263726	-0.7495923\\
0.7264726	-0.737639\\
0.7265727	-0.7252381\\
0.7266727	-0.7123995\\
0.7267727	-0.6991334\\
0.7268727	-0.68545\\
0.7269727	-0.67136\\
0.7270727	-0.6568743\\
0.7271727	-0.6420039\\
0.7272727	-0.6267601\\
0.7273727	-0.6111546\\
0.7274727	-0.5951991\\
0.7275728	-0.5789055\\
0.7276728	-0.5622859\\
0.7277728	-0.5453528\\
0.7278728	-0.5281185\\
0.7279728	-0.5105957\\
0.7280728	-0.4927973\\
0.7281728	-0.4747362\\
0.7282728	-0.4564255\\
0.7283728	-0.4378785\\
0.7284728	-0.4191085\\
0.7285729	-0.4001289\\
0.7286729	-0.3809534\\
0.7287729	-0.3615956\\
0.7288729	-0.3420692\\
0.7289729	-0.322388\\
0.7290729	-0.3025659\\
0.7291729	-0.2826168\\
0.7292729	-0.2625547\\
0.7293729	-0.2423935\\
0.7294729	-0.2221473\\
0.729573	-0.2018302\\
0.729673	-0.181456\\
0.729773	-0.1610389\\
0.729873	-0.140593\\
0.729973	-0.1201322\\
0.730073	-0.09967038\\
0.730173	-0.07922152\\
0.730273	-0.05879948\\
0.730373	-0.03841807\\
0.730473	-0.01809102\\
0.7305731	0.002167989\\
0.7306731	0.02234547\\
0.7307731	0.04242797\\
0.7308731	0.06240206\\
0.7309731	0.08225456\\
0.7310731	0.1019724\\
0.7311731	0.1215425\\
0.7312731	0.1409521\\
0.7313731	0.1601885\\
0.7314731	0.1792392\\
0.7315732	0.1980918\\
0.7316732	0.2167341\\
0.7317732	0.2351543\\
0.7318732	0.2533404\\
0.7319732	0.2712808\\
0.7320732	0.2889642\\
0.7321732	0.3063793\\
0.7322732	0.3235152\\
0.7323732	0.3403612\\
0.7324732	0.3569068\\
0.7325733	0.3731417\\
0.7326733	0.3890558\\
0.7327733	0.4046394\\
0.7328733	0.4198831\\
0.7329733	0.4347776\\
0.7330733	0.4493139\\
0.7331733	0.4634833\\
0.7332733	0.4772775\\
0.7333733	0.4906882\\
0.7334733	0.5037077\\
0.7335734	0.5163285\\
0.7336734	0.5285432\\
0.7337734	0.5403451\\
0.7338734	0.5517274\\
0.7339734	0.5626838\\
0.7340734	0.5732084\\
0.7341734	0.5832955\\
0.7342734	0.5929398\\
0.7343734	0.6021361\\
0.7344734	0.6108798\\
0.7345735	0.6191667\\
0.7346735	0.6269925\\
0.7347735	0.6343536\\
0.7348735	0.6412466\\
0.7349735	0.6476685\\
0.7350735	0.6536166\\
0.7351735	0.6590885\\
0.7352735	0.6640823\\
0.7353735	0.6685962\\
0.7354735	0.6726289\\
0.7355736	0.6761794\\
0.7356736	0.679247\\
0.7357736	0.6818315\\
0.7358736	0.6839327\\
0.7359736	0.6855511\\
0.7360736	0.6866875\\
0.7361736	0.6873427\\
0.7362736	0.6875181\\
0.7363736	0.6872155\\
0.7364736	0.6864368\\
0.7365737	0.6851844\\
0.7366737	0.6834609\\
0.7367737	0.6812694\\
0.7368737	0.6786131\\
0.7369737	0.6754956\\
0.7370737	0.6719209\\
0.7371737	0.6678932\\
0.7372737	0.6634169\\
0.7373737	0.6584971\\
0.7374737	0.6531387\\
0.7375738	0.6473473\\
0.7376738	0.6411284\\
0.7377738	0.6344882\\
0.7378738	0.6274328\\
0.7379738	0.6199687\\
0.7380738	0.6121028\\
0.7381738	0.6038421\\
0.7382738	0.595194\\
0.7383738	0.586166\\
0.7384738	0.5767658\\
0.7385739	0.5670015\\
0.7386739	0.5568814\\
0.7387739	0.5464139\\
0.7388739	0.5356078\\
0.7389739	0.5244719\\
0.7390739	0.5130153\\
0.7391739	0.5012474\\
0.7392739	0.4891776\\
0.7393739	0.4768156\\
0.7394739	0.4641713\\
0.739574	0.4512546\\
0.739674	0.4380757\\
0.739774	0.424645\\
0.739874	0.4109728\\
0.739974	0.3970698\\
0.740074	0.3829466\\
0.740174	0.3686142\\
0.740274	0.3540835\\
0.740374	0.3393654\\
0.740474	0.3244711\\
0.7405741	0.309412\\
0.7406741	0.2941992\\
0.7407741	0.2788441\\
0.7408741	0.2633583\\
0.7409741	0.247753\\
0.7410741	0.23204\\
0.7411741	0.2162307\\
0.7412741	0.2003368\\
0.7413741	0.1843698\\
0.7414741	0.1683414\\
0.7415742	0.1522631\\
0.7416742	0.1361467\\
0.7417742	0.1200037\\
0.7418742	0.1038457\\
0.7419742	0.08768421\\
0.7420742	0.0715308\\
0.7421742	0.05539694\\
0.7422742	0.03929392\\
0.7423742	0.02323323\\
0.7424742	0.00722612\\
0.7425743	-0.008716298\\
0.7426743	-0.02458287\\
0.7427743	-0.04036257\\
0.7428743	-0.05604452\\
0.7429743	-0.07161783\\
0.7430743	-0.08707194\\
0.7431743	-0.1023962\\
0.7432743	-0.1175802\\
0.7433743	-0.1326138\\
0.7434743	-0.1474865\\
0.7435744	-0.1621886\\
0.7436744	-0.1767102\\
0.7437744	-0.1910416\\
0.7438744	-0.2051733\\
0.7439744	-0.219096\\
0.7440744	-0.2328006\\
0.7441744	-0.2462781\\
0.7442744	-0.2595197\\
0.7443744	-0.272517\\
0.7444744	-0.2852615\\
0.7445745	-0.2977452\\
0.7446745	-0.3099602\\
0.7447745	-0.3218987\\
0.7448745	-0.3335533\\
0.7449745	-0.3449167\\
0.7450745	-0.3559821\\
0.7451745	-0.3667427\\
0.7452745	-0.3771918\\
0.7453745	-0.3873233\\
0.7454745	-0.3971312\\
0.7455746	-0.4066097\\
0.7456746	-0.4157534\\
0.7457746	-0.424557\\
0.7458746	-0.4330155\\
0.7459746	-0.4411244\\
0.7460746	-0.4488791\\
0.7461746	-0.4562755\\
0.7462746	-0.4633098\\
0.7463746	-0.4699782\\
0.7464746	-0.4762776\\
0.7465747	-0.482205\\
0.7466747	-0.4877574\\
0.7467747	-0.4929326\\
0.7468747	-0.4977282\\
0.7469747	-0.5021424\\
0.7470747	-0.5061736\\
0.7471747	-0.5098203\\
0.7472747	-0.5130817\\
0.7473747	-0.5159569\\
0.7474747	-0.5184455\\
0.7475748	-0.5205471\\
0.7476748	-0.5222619\\
0.7477748	-0.5235904\\
0.7478748	-0.524533\\
0.7479748	-0.5250909\\
0.7480748	-0.5252651\\
0.7481748	-0.5250572\\
0.7482748	-0.5244688\\
0.7483748	-0.523502\\
0.7484748	-0.5221591\\
0.7485749	-0.5204427\\
0.7486749	-0.5183555\\
0.7487749	-0.5159007\\
0.7488749	-0.5130816\\
0.7489749	-0.5099018\\
0.7490749	-0.5063651\\
0.7491749	-0.5024756\\
0.7492749	-0.4982377\\
0.7493749	-0.493656\\
0.7494749	-0.4887351\\
0.749575	-0.4834802\\
0.749675	-0.4778965\\
0.749775	-0.4719895\\
0.749875	-0.4657649\\
0.749975	-0.4592287\\
0.750075	-0.4523868\\
0.750175	-0.4452457\\
0.750275	-0.4378118\\
0.750375	-0.4300919\\
0.750475	-0.4220929\\
0.7505751	-0.4138218\\
0.7506751	-0.4052857\\
0.7507751	-0.3964923\\
0.7508751	-0.3874489\\
0.7509751	-0.3781634\\
0.7510751	-0.3686436\\
0.7511751	-0.3588975\\
0.7512751	-0.3489333\\
0.7513751	-0.3387592\\
0.7514751	-0.3283837\\
0.7515752	-0.3178152\\
0.7516752	-0.3070624\\
0.7517752	-0.2961339\\
0.7518752	-0.2850387\\
0.7519752	-0.2737857\\
0.7520752	-0.2623837\\
0.7521752	-0.2508419\\
0.7522752	-0.2391695\\
0.7523752	-0.2273756\\
0.7524752	-0.2154694\\
0.7525753	-0.2034603\\
0.7526753	-0.1913576\\
0.7527753	-0.1791707\\
0.7528753	-0.1669089\\
0.7529753	-0.1545817\\
0.7530753	-0.1421984\\
0.7531753	-0.1297686\\
0.7532753	-0.1173016\\
0.7533753	-0.1048069\\
0.7534753	-0.09229379\\
0.7535754	-0.07977169\\
0.7536754	-0.06724996\\
0.7537754	-0.05473791\\
0.7538754	-0.04224477\\
0.7539754	-0.02977973\\
0.7540754	-0.01735195\\
0.7541754	-0.004970562\\
0.7542754	0.007355437\\
0.7543754	0.01961719\\
0.7544754	0.03180575\\
0.7545755	0.04391235\\
0.7546755	0.05592843\\
0.7547755	0.06784536\\
0.7548755	0.07965467\\
0.7549755	0.09134805\\
0.7550755	0.1029173\\
0.7551755	0.1143542\\
0.7552755	0.125651\\
0.7553755	0.1367997\\
0.7554755	0.1477927\\
0.7555756	0.1586225\\
0.7556756	0.1692816\\
0.7557756	0.1797629\\
0.7558756	0.1900594\\
0.7559756	0.200164\\
0.7560756	0.2100702\\
0.7561756	0.2197713\\
0.7562756	0.229261\\
0.7563756	0.2385331\\
0.7564756	0.2475816\\
0.7565757	0.2564008\\
0.7566757	0.2649851\\
0.7567757	0.273329\\
0.7568757	0.2814274\\
0.7569757	0.2892752\\
0.7570757	0.2968678\\
0.7571757	0.3042005\\
0.7572757	0.3112689\\
0.7573757	0.318069\\
0.7574757	0.324597\\
0.7575758	0.3308489\\
0.7576758	0.3368214\\
0.7577758	0.3425113\\
0.7578758	0.3479154\\
0.7579758	0.3530312\\
0.7580758	0.3578559\\
0.7581758	0.3623873\\
0.7582758	0.3666233\\
0.7583758	0.370562\\
0.7584758	0.3742018\\
0.7585759	0.3775413\\
0.7586759	0.3805793\\
0.7587759	0.3833149\\
0.7588759	0.3857475\\
0.7589759	0.3878765\\
0.7590759	0.3897018\\
0.7591759	0.3912233\\
0.7592759	0.3924413\\
0.7593759	0.3933562\\
0.7594759	0.3939687\\
0.759576	0.3942798\\
0.759676	0.3942906\\
0.759776	0.3940025\\
0.759876	0.393417\\
0.759976	0.392536\\
0.760076	0.3913615\\
0.760176	0.3898958\\
0.760276	0.3881413\\
0.760376	0.3861007\\
0.760476	0.3837768\\
0.7605761	0.3811728\\
0.7606761	0.3782918\\
0.7607761	0.3751374\\
0.7608761	0.3717133\\
0.7609761	0.3680233\\
0.7610761	0.3640715\\
0.7611761	0.3598621\\
0.7612761	0.3553995\\
0.7613761	0.3506884\\
0.7614761	0.3457334\\
0.7615762	0.3405395\\
0.7616762	0.3351117\\
0.7617762	0.3294554\\
0.7618762	0.3235758\\
0.7619762	0.3174786\\
0.7620762	0.3111695\\
0.7621762	0.3046542\\
0.7622762	0.2979387\\
0.7623762	0.2910292\\
0.7624762	0.2839318\\
0.7625763	0.2766529\\
0.7626763	0.2691988\\
0.7627763	0.2615762\\
0.7628763	0.2537917\\
0.7629763	0.245852\\
0.7630763	0.2377639\\
0.7631763	0.2295345\\
0.7632763	0.2211706\\
0.7633763	0.2126794\\
0.7634763	0.204068\\
0.7635764	0.1953436\\
0.7636764	0.1865135\\
0.7637764	0.1775849\\
0.7638764	0.1685652\\
0.7639764	0.1594618\\
0.7640764	0.1502821\\
0.7641764	0.1410336\\
0.7642764	0.1317238\\
0.7643764	0.12236\\
0.7644764	0.1129499\\
0.7645765	0.1035009\\
0.7646765	0.09402048\\
0.7647765	0.08451617\\
0.7648765	0.07499539\\
0.7649765	0.06546561\\
0.7650765	0.0559343\\
0.7651765	0.04640875\\
0.7652765	0.03689641\\
0.7653765	0.02740456\\
0.7654765	0.01794043\\
0.7655766	0.00851125\\
0.7656766	-0.0008757898\\
0.7657766	-0.0102137\\
0.7658766	-0.01949539\\
0.7659766	-0.02871392\\
0.7660766	-0.03786242\\
0.7661766	-0.04693409\\
0.7662766	-0.05592234\\
0.7663766	-0.06482056\\
0.7664766	-0.07362217\\
0.7665767	-0.08232083\\
0.7666767	-0.09091029\\
0.7667767	-0.09938438\\
0.7668767	-0.1077371\\
0.7669767	-0.1159624\\
0.7670767	-0.1240545\\
0.7671767	-0.1320078\\
0.7672767	-0.1398168\\
0.7673767	-0.1474759\\
0.7674767	-0.1549802\\
0.7675768	-0.1623243\\
0.7676768	-0.1695033\\
0.7677768	-0.1765124\\
0.7678768	-0.1833469\\
0.7679768	-0.1900025\\
0.7680768	-0.1964746\\
0.7681768	-0.2027592\\
0.7682768	-0.2088523\\
0.7683768	-0.21475\\
0.7684768	-0.2204485\\
0.7685769	-0.2259446\\
0.7686769	-0.2312347\\
0.7687769	-0.2363158\\
0.7688769	-0.241185\\
0.7689769	-0.2458395\\
0.7690769	-0.2502765\\
0.7691769	-0.2544938\\
0.7692769	-0.258489\\
0.7693769	-0.2622602\\
0.7694769	-0.2658055\\
0.769577	-0.2691232\\
0.769677	-0.2722119\\
0.769777	-0.2750701\\
0.769877	-0.2776968\\
0.769977	-0.2800911\\
0.770077	-0.2822522\\
0.770177	-0.2841797\\
0.770277	-0.285873\\
0.770377	-0.2873321\\
0.770477	-0.2885571\\
0.7705771	-0.2895479\\
0.7706771	-0.2903052\\
0.7707771	-0.2908293\\
0.7708771	-0.2911211\\
0.7709771	-0.2911815\\
0.7710771	-0.2910115\\
0.7711771	-0.2906125\\
0.7712771	-0.289986\\
0.7713771	-0.2891336\\
0.7714771	-0.288057\\
0.7715772	-0.2867583\\
0.7716772	-0.2852396\\
0.7717772	-0.2835032\\
0.7718772	-0.2815516\\
0.7719772	-0.2793873\\
0.7720772	-0.2770133\\
0.7721772	-0.2744324\\
0.7722772	-0.2716477\\
0.7723772	-0.2686624\\
0.7724772	-0.2654801\\
0.7725773	-0.2621041\\
0.7726773	-0.2585381\\
0.7727773	-0.2547861\\
0.7728773	-0.2508517\\
0.7729773	-0.2467392\\
0.7730773	-0.2424527\\
0.7731773	-0.2379964\\
0.7732773	-0.2333749\\
0.7733773	-0.2285926\\
0.7734773	-0.2236542\\
0.7735774	-0.2185644\\
0.7736774	-0.2133281\\
0.7737774	-0.2079502\\
0.7738774	-0.2024356\\
0.7739774	-0.1967896\\
0.7740774	-0.1910172\\
0.7741774	-0.1851237\\
0.7742774	-0.1791146\\
0.7743774	-0.1729951\\
0.7744774	-0.1667708\\
0.7745775	-0.1604471\\
0.7746775	-0.1540297\\
0.7747775	-0.1475242\\
0.7748775	-0.1409362\\
0.7749775	-0.1342715\\
0.7750775	-0.1275358\\
0.7751775	-0.1207349\\
0.7752775	-0.1138745\\
0.7753775	-0.1069605\\
0.7754775	-0.09999879\\
0.7755776	-0.0929951\\
0.7756776	-0.08595528\\
0.7757776	-0.07888529\\
0.7758776	-0.07179091\\
0.7759776	-0.06467801\\
0.7760776	-0.05755238\\
0.7761776	-0.05041983\\
0.7762776	-0.04328615\\
0.7763776	-0.03615714\\
0.7764776	-0.02903843\\
0.7765777	-0.02193582\\
0.7766777	-0.01485496\\
0.7767777	-0.007801367\\
0.7768777	-0.000780646\\
0.7769777	0.006201713\\
0.7770777	0.01314024\\
0.7771777	0.02002958\\
0.7772777	0.02686442\\
0.7773777	0.03363949\\
0.7774777	0.04034961\\
0.7775778	0.04698966\\
0.7776778	0.05355471\\
0.7777778	0.06003979\\
0.7778778	0.06644001\\
0.7779778	0.07275063\\
0.7780778	0.07896708\\
0.7781778	0.08508472\\
0.7782778	0.09109914\\
0.7783778	0.09700606\\
0.7784778	0.1028011\\
0.7785779	0.1084803\\
0.7786779	0.1140396\\
0.7787779	0.119475\\
0.7788779	0.1247828\\
0.7789779	0.1299595\\
0.7790779	0.1350013\\
0.7791779	0.139905\\
0.7792779	0.1446672\\
0.7793779	0.149285\\
0.7794779	0.1537551\\
0.779578	0.1580749\\
0.779678	0.1622414\\
0.779778	0.1662522\\
0.779878	0.1701049\\
0.779978	0.173797\\
0.780078	0.1773264\\
0.780178	0.1806912\\
0.780278	0.1838894\\
0.780378	0.1869194\\
0.780478	0.1897794\\
0.7805781	0.1924681\\
0.7806781	0.1949842\\
0.7807781	0.1973266\\
0.7808781	0.1994942\\
0.7809781	0.2014862\\
0.7810781	0.203302\\
0.7811781	0.204941\\
0.7812781	0.2064027\\
0.7813781	0.2076869\\
0.7814781	0.2087936\\
0.7815782	0.2097227\\
0.7816782	0.2104745\\
0.7817782	0.2110493\\
0.7818782	0.2114474\\
0.7819782	0.2116697\\
0.7820782	0.2117169\\
0.7821782	0.2115897\\
0.7822782	0.2112894\\
0.7823782	0.2108171\\
0.7824782	0.2101741\\
0.7825783	0.2093617\\
0.7826783	0.2083818\\
0.7827783	0.2072358\\
0.7828783	0.2059257\\
0.7829783	0.2044535\\
0.7830783	0.2028212\\
0.7831783	0.2010311\\
0.7832783	0.1990855\\
0.7833783	0.1969868\\
0.7834783	0.1947376\\
0.7835784	0.1923405\\
0.7836784	0.1897985\\
0.7837784	0.1871144\\
0.7838784	0.1842911\\
0.7839784	0.1813318\\
0.7840784	0.1782396\\
0.7841784	0.1750178\\
0.7842784	0.1716699\\
0.7843784	0.1681992\\
0.7844784	0.1646094\\
0.7845785	0.160904\\
0.7846785	0.1570868\\
0.7847785	0.1531616\\
0.7848785	0.1491322\\
0.7849785	0.1450025\\
0.7850785	0.1407765\\
0.7851785	0.1364583\\
0.7852785	0.132052\\
0.7853785	0.1275616\\
0.7854785	0.1229915\\
0.7855786	0.1183458\\
0.7856786	0.1136288\\
0.7857786	0.1088449\\
0.7858786	0.1039985\\
0.7859786	0.0990939\\
0.7860786	0.09413551\\
0.7861786	0.08912779\\
0.7862786	0.08407515\\
0.7863786	0.07898213\\
0.7864786	0.07385318\\
0.7865787	0.06869274\\
0.7866787	0.06350539\\
0.7867787	0.05829558\\
0.7868787	0.05306778\\
0.7869787	0.04782648\\
0.7870787	0.04257614\\
0.7871787	0.03732117\\
0.7872787	0.03206605\\
0.7873787	0.02681525\\
0.7874787	0.02157303\\
0.7875788	0.01634372\\
0.7876788	0.01113174\\
0.7877788	0.005941366\\
0.7878788	0.0007767761\\
0.7879788	-0.004357814\\
0.7880788	-0.009458269\\
0.7881788	-0.01452051\\
0.7882788	-0.01954043\\
0.7883788	-0.02451407\\
0.7884788	-0.02943755\\
0.7885789	-0.03430691\\
0.7886789	-0.0391184\\
0.7887789	-0.04386832\\
0.7888789	-0.04855304\\
0.7889789	-0.05316888\\
0.7890789	-0.05771236\\
0.7891789	-0.0621801\\
0.7892789	-0.06656873\\
0.7893789	-0.07087501\\
0.7894789	-0.07509576\\
0.789579	-0.07922784\\
0.789679	-0.08326834\\
0.789779	-0.08721434\\
0.789879	-0.09106304\\
0.789979	-0.09481166\\
0.790079	-0.09845768\\
0.790179	-0.1019986\\
0.790279	-0.1054319\\
0.790379	-0.1087554\\
0.790479	-0.111967\\
0.7905791	-0.1150644\\
0.7906791	-0.1180457\\
0.7907791	-0.120909\\
0.7908791	-0.1236526\\
0.7909791	-0.1262748\\
0.7910791	-0.1287741\\
0.7911791	-0.1311492\\
0.7912791	-0.1333985\\
0.7913791	-0.1355211\\
0.7914791	-0.1375157\\
0.7915792	-0.1393814\\
0.7916792	-0.1411174\\
0.7917792	-0.1427229\\
0.7918792	-0.1441974\\
0.7919792	-0.1455402\\
0.7920792	-0.1467511\\
0.7921792	-0.1478297\\
0.7922792	-0.1487758\\
0.7923792	-0.1495896\\
0.7924792	-0.1502709\\
0.7925793	-0.1508199\\
0.7926793	-0.1512371\\
0.7927793	-0.1515227\\
0.7928793	-0.1516773\\
0.7929793	-0.1517015\\
0.7930793	-0.151596\\
0.7931793	-0.1513617\\
0.7932793	-0.1509995\\
0.7933793	-0.1505105\\
0.7934793	-0.1498958\\
0.7935794	-0.1491567\\
0.7936794	-0.1482944\\
0.7937794	-0.1473106\\
0.7938794	-0.1462066\\
0.7939794	-0.1449843\\
0.7940794	-0.1436453\\
0.7941794	-0.1421914\\
0.7942794	-0.1406245\\
0.7943794	-0.1389467\\
0.7944794	-0.13716\\
0.7945795	-0.1352665\\
0.7946795	-0.1332686\\
0.7947795	-0.1311687\\
0.7948795	-0.1289689\\
0.7949795	-0.126672\\
0.7950795	-0.1242803\\
0.7951795	-0.1217965\\
0.7952795	-0.1192233\\
0.7953795	-0.1165635\\
0.7954795	-0.1138198\\
0.7955796	-0.110995\\
0.7956796	-0.1080921\\
0.7957796	-0.1051142\\
0.7958796	-0.1020641\\
0.7959796	-0.09894483\\
0.7960796	-0.09575966\\
0.7961796	-0.09251165\\
0.7962796	-0.089204\\
0.7963796	-0.08583991\\
0.7964796	-0.0824226\\
0.7965797	-0.07895535\\
0.7966797	-0.07544148\\
0.7967797	-0.07188428\\
0.7968797	-0.0682871\\
0.7969797	-0.06465329\\
0.7970797	-0.06098627\\
0.7971797	-0.05728931\\
0.7972797	-0.05356586\\
0.7973797	-0.04981932\\
0.7974797	-0.04605302\\
0.7975798	-0.04227041\\
0.7976798	-0.03847488\\
0.7977798	-0.03466974\\
0.7978798	-0.03085838\\
0.7979798	-0.02704413\\
0.7980798	-0.0232304\\
0.7981798	-0.01942047\\
0.7982798	-0.01561767\\
0.7983798	-0.01182516\\
0.7984798	-0.008046298\\
0.7985799	-0.004284179\\
0.7986799	-0.000542111\\
0.7987799	0.003176888\\
0.7988799	0.006869651\\
0.7989799	0.01053309\\
0.7990799	0.01416414\\
0.7991799	0.01775993\\
0.7992799	0.0213175\\
0.7993799	0.02483393\\
0.7994799	0.02830639\\
0.79958	0.0317322\\
0.79968	0.03510853\\
0.79978	0.03843282\\
0.79988	0.04170243\\
0.79998	0.04491481\\
0.80008	0.0480674\\
};
\addplot [color=mycolor2,solid]
  table[row sep=crcr]{%
0.80008	0.0480674\\
0.80018	0.05115798\\
0.80028	0.0541841\\
0.80038	0.05714357\\
0.80048	0.06003409\\
0.8005801	0.06285363\\
0.8006801	0.06560012\\
0.8007801	0.06827152\\
0.8008801	0.070866\\
0.8009801	0.07338179\\
0.8010801	0.07581709\\
0.8011801	0.07817028\\
0.8012801	0.08043979\\
0.8013801	0.08262407\\
0.8014801	0.08472172\\
0.8015802	0.08673149\\
0.8016802	0.08865216\\
0.8017802	0.09048247\\
0.8018802	0.0922214\\
0.8019802	0.09386797\\
0.8020802	0.09542131\\
0.8021802	0.09688058\\
0.8022802	0.09824517\\
0.8023802	0.0995143\\
0.8024802	0.1006875\\
0.8025803	0.1017643\\
0.8026803	0.1027445\\
0.8027803	0.1036275\\
0.8028803	0.1044134\\
0.8029803	0.105102\\
0.8030803	0.1056932\\
0.8031803	0.1061872\\
0.8032803	0.1065841\\
0.8033803	0.1068841\\
0.8034803	0.1070877\\
0.8035804	0.107195\\
0.8036804	0.1072069\\
0.8037804	0.1071236\\
0.8038804	0.106946\\
0.8039804	0.1066747\\
0.8040804	0.1063106\\
0.8041804	0.1058546\\
0.8042804	0.1053076\\
0.8043804	0.1046707\\
0.8044804	0.1039449\\
0.8045805	0.1031316\\
0.8046805	0.1022319\\
0.8047805	0.1012473\\
0.8048805	0.100179\\
0.8049805	0.0990286\\
0.8050805	0.09779761\\
0.8051805	0.09648764\\
0.8052805	0.09510024\\
0.8053805	0.09363721\\
0.8054805	0.09210034\\
0.8055806	0.09049148\\
0.8056806	0.0888124\\
0.8057806	0.08706517\\
0.8058806	0.08525166\\
0.8059806	0.08337399\\
0.8060806	0.08143422\\
0.8061806	0.07943448\\
0.8062806	0.07737685\\
0.8063806	0.07526356\\
0.8064806	0.07309692\\
0.8065807	0.07087916\\
0.8066807	0.06861258\\
0.8067807	0.06629948\\
0.8068807	0.06394234\\
0.8069807	0.0615434\\
0.8070807	0.05910521\\
0.8071807	0.05663013\\
0.8072807	0.05412064\\
0.8073807	0.05157915\\
0.8074807	0.04900823\\
0.8075808	0.04641033\\
0.8076808	0.04378792\\
0.8077808	0.04114354\\
0.8078808	0.03847973\\
0.8079808	0.03579905\\
0.8080808	0.0331039\\
0.8081808	0.03039687\\
0.8082808	0.02768043\\
0.8083808	0.02495712\\
0.8084808	0.02222941\\
0.8085809	0.01949988\\
0.8086809	0.01677094\\
0.8087809	0.01404505\\
0.8088809	0.01132456\\
0.8089809	0.00861207\\
0.8090809	0.005909842\\
0.8091809	0.003220334\\
0.8092809	0.0005458197\\
0.8093809	-0.002111311\\
0.8094809	-0.004748803\\
0.809581	-0.007364406\\
0.809681	-0.009955815\\
0.809781	-0.01252091\\
0.809881	-0.01505756\\
0.809981	-0.01756363\\
0.810081	-0.02003695\\
0.810181	-0.02247562\\
0.810281	-0.02487757\\
0.810381	-0.02724088\\
0.810481	-0.02956357\\
0.8105811	-0.0318439\\
0.8106811	-0.03408005\\
0.8107811	-0.03627023\\
0.8108811	-0.03841283\\
0.8109811	-0.04050612\\
0.8110811	-0.04254853\\
0.8111811	-0.04453845\\
0.8112811	-0.04647461\\
0.8113811	-0.04835542\\
0.8114811	-0.05017962\\
0.8115812	-0.05194581\\
0.8116812	-0.05365286\\
0.8117812	-0.05529952\\
0.8118812	-0.05688468\\
0.8119812	-0.05840733\\
0.8120812	-0.05986645\\
0.8121812	-0.06126114\\
0.8122812	-0.0625905\\
0.8123812	-0.06385383\\
0.8124812	-0.06505029\\
0.8125813	-0.06617927\\
0.8126813	-0.06724015\\
0.8127813	-0.0682325\\
0.8128813	-0.06915573\\
0.8129813	-0.07000952\\
0.8130813	-0.07079351\\
0.8131813	-0.07150738\\
0.8132813	-0.07215101\\
0.8133813	-0.07272418\\
0.8134813	-0.07322692\\
0.8135814	-0.07365914\\
0.8136814	-0.07402094\\
0.8137814	-0.07431242\\
0.8138814	-0.07453386\\
0.8139814	-0.07468544\\
0.8140814	-0.07476745\\
0.8141814	-0.07478032\\
0.8142814	-0.07472446\\
0.8143814	-0.07460039\\
0.8144814	-0.07440857\\
0.8145815	-0.07414978\\
0.8146815	-0.07382457\\
0.8147815	-0.07343375\\
0.8148815	-0.07297803\\
0.8149815	-0.07245839\\
0.8150815	-0.07187559\\
0.8151815	-0.07123058\\
0.8152815	-0.07052443\\
0.8153815	-0.06975818\\
0.8154815	-0.06893297\\
0.8155816	-0.06804986\\
0.8156816	-0.06711008\\
0.8157816	-0.06611484\\
0.8158816	-0.06506547\\
0.8159816	-0.06396327\\
0.8160816	-0.06280964\\
0.8161816	-0.06160597\\
0.8162816	-0.06035366\\
0.8163816	-0.05905423\\
0.8164816	-0.05770919\\
0.8165817	-0.05632016\\
0.8166817	-0.05488853\\
0.8167817	-0.05341607\\
0.8168817	-0.05190436\\
0.8169817	-0.05035507\\
0.8170817	-0.04876985\\
0.8171817	-0.04715042\\
0.8172817	-0.04549856\\
0.8173817	-0.04381592\\
0.8174817	-0.04210439\\
0.8175818	-0.04036568\\
0.8176818	-0.03860163\\
0.8177818	-0.03681393\\
0.8178818	-0.0350045\\
0.8179818	-0.03317521\\
0.8180818	-0.03132783\\
0.8181818	-0.02946421\\
0.8182818	-0.02758614\\
0.8183818	-0.02569559\\
0.8184818	-0.02379427\\
0.8185819	-0.02188408\\
0.8186819	-0.0199669\\
0.8187819	-0.01804455\\
0.8188819	-0.01611884\\
0.8189819	-0.01419154\\
0.8190819	-0.01226458\\
0.8191819	-0.01033962\\
0.8192819	-0.00841853\\
0.8193819	-0.006503009\\
0.8194819	-0.004594921\\
0.819582	-0.002695966\\
0.819682	-0.0008078343\\
0.819782	0.001067775\\
0.819882	0.002929197\\
0.819982	0.004774766\\
0.820082	0.006602948\\
0.820182	0.008412026\\
0.820282	0.01020046\\
0.820382	0.01196671\\
0.820482	0.01370929\\
0.8205821	0.01542677\\
0.8206821	0.01711754\\
0.8207821	0.01878036\\
0.8208821	0.02041378\\
0.8209821	0.02201652\\
0.8210821	0.02358726\\
0.8211821	0.0251247\\
0.8212821	0.02662771\\
0.8213821	0.02809497\\
0.8214821	0.02952546\\
0.8215822	0.030918\\
0.8216822	0.0322716\\
0.8217822	0.03358517\\
0.8218822	0.03485779\\
0.8219822	0.03608858\\
0.8220822	0.03727665\\
0.8221822	0.03842109\\
0.8222822	0.03952114\\
0.8223822	0.04057612\\
0.8224822	0.04158529\\
0.8225823	0.04254802\\
0.8226823	0.04346379\\
0.8227823	0.04433192\\
0.8228823	0.04515204\\
0.8229823	0.04592353\\
0.8230823	0.04664614\\
0.8231823	0.04731953\\
0.8232823	0.04794332\\
0.8233823	0.04851731\\
0.8234823	0.04904123\\
0.8235824	0.04951505\\
0.8236824	0.04993852\\
0.8237824	0.05031165\\
0.8238824	0.05063443\\
0.8239824	0.05090689\\
0.8240824	0.05112916\\
0.8241824	0.05130126\\
0.8242824	0.0514235\\
0.8243824	0.05149603\\
0.8244824	0.05151912\\
0.8245825	0.05149312\\
0.8246825	0.05141836\\
0.8247825	0.05129533\\
0.8248825	0.05112437\\
0.8249825	0.05090604\\
0.8250825	0.05064082\\
0.8251825	0.05032934\\
0.8252825	0.04997221\\
0.8253825	0.04957003\\
0.8254825	0.0491236\\
0.8255826	0.04863352\\
0.8256826	0.04810063\\
0.8257826	0.04752567\\
0.8258826	0.04690956\\
0.8259826	0.04625321\\
0.8260826	0.04555743\\
0.8261826	0.04482323\\
0.8262826	0.0440515\\
0.8263826	0.04324337\\
0.8264826	0.04239972\\
0.8265827	0.04152164\\
0.8266827	0.04061026\\
0.8267827	0.03966662\\
0.8268827	0.03869186\\
0.8269827	0.03768717\\
0.8270827	0.03665374\\
0.8271827	0.03559276\\
0.8272827	0.0345053\\
0.8273827	0.03339272\\
0.8274827	0.03225625\\
0.8275828	0.03109716\\
0.8276828	0.02991671\\
0.8277828	0.02871611\\
0.8278828	0.02749678\\
0.8279828	0.02625989\\
0.8280828	0.02500687\\
0.8281828	0.02373895\\
0.8282828	0.02245752\\
0.8283828	0.02116392\\
0.8284828	0.01985934\\
0.8285829	0.01854531\\
0.8286829	0.01722301\\
0.8287829	0.01589386\\
0.8288829	0.01455916\\
0.8289829	0.01322018\\
0.8290829	0.01187835\\
0.8291829	0.01053492\\
0.8292829	0.009191246\\
0.8293829	0.007848548\\
0.8294829	0.006508191\\
0.829583	0.005171438\\
0.829683	0.003839552\\
0.829783	0.002513789\\
0.829883	0.001195395\\
0.829983	-0.0001144381\\
0.830083	-0.001414507\\
0.830183	-0.002703665\\
0.830283	-0.003980624\\
0.830383	-0.005244283\\
0.830483	-0.006493485\\
0.8305831	-0.007727119\\
0.8306831	-0.0089442\\
0.8307831	-0.0101435\\
0.8308831	-0.01132416\\
0.8309831	-0.01248508\\
0.8310831	-0.01362524\\
0.8311831	-0.0147437\\
0.8312831	-0.01583952\\
0.8313831	-0.01691187\\
0.8314831	-0.01795976\\
0.8315832	-0.0189824\\
0.8316832	-0.01997897\\
0.8317832	-0.0209487\\
0.8318832	-0.02189088\\
0.8319832	-0.02280465\\
0.8320832	-0.02368944\\
0.8321832	-0.02454455\\
0.8322832	-0.02536936\\
0.8323832	-0.0261633\\
0.8324832	-0.02692581\\
0.8325833	-0.02765638\\
0.8326833	-0.02835454\\
0.8327833	-0.02901973\\
0.8328833	-0.02965169\\
0.8329833	-0.03024994\\
0.8330833	-0.03081423\\
0.8331833	-0.03134417\\
0.8332833	-0.03183941\\
0.8333833	-0.03229988\\
0.8334833	-0.03272523\\
0.8335834	-0.03311543\\
0.8336834	-0.03347027\\
0.8337834	-0.03378968\\
0.8338834	-0.03407362\\
0.8339834	-0.03432202\\
0.8340834	-0.03453494\\
0.8341834	-0.03471242\\
0.8342834	-0.0348545\\
0.8343834	-0.03496137\\
0.8344834	-0.03503306\\
0.8345835	-0.03506981\\
0.8346835	-0.03507194\\
0.8347835	-0.03503958\\
0.8348835	-0.03497305\\
0.8349835	-0.03487259\\
0.8350835	-0.03473863\\
0.8351835	-0.03457157\\
0.8352835	-0.03437168\\
0.8353835	-0.03413957\\
0.8354835	-0.03387558\\
0.8355836	-0.03358028\\
0.8356836	-0.03325412\\
0.8357836	-0.03289759\\
0.8358836	-0.03251139\\
0.8359836	-0.03209606\\
0.8360836	-0.03165224\\
0.8361836	-0.03118054\\
0.8362836	-0.03068163\\
0.8363836	-0.03015628\\
0.8364836	-0.02960514\\
0.8365837	-0.02902894\\
0.8366837	-0.02842847\\
0.8367837	-0.02780443\\
0.8368837	-0.02715771\\
0.8369837	-0.026489\\
0.8370837	-0.02579916\\
0.8371837	-0.02508912\\
0.8372837	-0.02435961\\
0.8373837	-0.02361155\\
0.8374837	-0.02284581\\
0.8375838	-0.02206324\\
0.8376838	-0.02126489\\
0.8377838	-0.02045155\\
0.8378838	-0.01962411\\
0.8379838	-0.01878351\\
0.8380838	-0.01793068\\
0.8381838	-0.01706664\\
0.8382838	-0.01619224\\
0.8383838	-0.01530846\\
0.8384838	-0.01441634\\
0.8385839	-0.01351667\\
0.8386839	-0.01261045\\
0.8387839	-0.01169863\\
0.8388839	-0.01078219\\
0.8389839	-0.009862104\\
0.8390839	-0.008939242\\
0.8391839	-0.008014579\\
0.8392839	-0.007089068\\
0.8393839	-0.006163614\\
0.8394839	-0.00523919\\
0.839584	-0.004316646\\
0.839684	-0.003396884\\
0.839784	-0.002480888\\
0.839884	-0.001569437\\
0.839984	-0.0006634952\\
0.840084	0.0002361231\\
0.840184	0.001128532\\
0.840284	0.002012861\\
0.840384	0.002888397\\
0.840484	0.003754274\\
0.8405841	0.004609659\\
0.8406841	0.005453835\\
0.8407841	0.006285922\\
0.8408841	0.007105285\\
0.8409841	0.007911211\\
0.8410841	0.008702891\\
0.8411841	0.00947975\\
0.8412841	0.01024104\\
0.8413841	0.01098612\\
0.8414841	0.01171443\\
0.8415842	0.01242522\\
0.8416842	0.01311805\\
0.8417842	0.01379234\\
0.8418842	0.01444748\\
0.8419842	0.01508301\\
0.8420842	0.01569838\\
0.8421842	0.01629315\\
0.8422842	0.01686698\\
0.8423842	0.01741928\\
0.8424842	0.01794971\\
0.8425843	0.01845795\\
0.8426843	0.01894361\\
0.8427843	0.01940641\\
0.8428843	0.01984596\\
0.8429843	0.02026207\\
0.8430843	0.02065452\\
0.8431843	0.02102299\\
0.8432843	0.02136731\\
0.8433843	0.02168736\\
0.8434843	0.02198302\\
0.8435844	0.0222542\\
0.8436844	0.02250077\\
0.8437844	0.02272265\\
0.8438844	0.02291979\\
0.8439844	0.02309216\\
0.8440844	0.02323984\\
0.8441844	0.02336284\\
0.8442844	0.02346116\\
0.8443844	0.02353497\\
0.8444844	0.02358436\\
0.8445845	0.02360938\\
0.8446845	0.02361034\\
0.8447845	0.0235873\\
0.8448845	0.02354053\\
0.8449845	0.02347028\\
0.8450845	0.02337668\\
0.8451845	0.02326013\\
0.8452845	0.02312083\\
0.8453845	0.02295918\\
0.8454845	0.02277549\\
0.8455846	0.02257002\\
0.8456846	0.02234324\\
0.8457846	0.02209555\\
0.8458846	0.02182732\\
0.8459846	0.02153904\\
0.8460846	0.02123105\\
0.8461846	0.0209039\\
0.8462846	0.02055808\\
0.8463846	0.020194\\
0.8464846	0.0198122\\
0.8465847	0.01941324\\
0.8466847	0.01899768\\
0.8467847	0.01856606\\
0.8468847	0.01811891\\
0.8469847	0.01765675\\
0.8470847	0.01718025\\
0.8471847	0.01669002\\
0.8472847	0.01618657\\
0.8473847	0.01567064\\
0.8474847	0.01514276\\
0.8475848	0.01460363\\
0.8476848	0.01405391\\
0.8477848	0.01349414\\
0.8478848	0.012925\\
0.8479848	0.01234722\\
0.8480848	0.01176139\\
0.8481848	0.01116822\\
0.8482848	0.0105683\\
0.8483848	0.009962321\\
0.8484848	0.009351047\\
0.8485849	0.008735062\\
0.8486849	0.008115083\\
0.8487849	0.007491739\\
0.8488849	0.006865677\\
0.8489849	0.006237551\\
0.8490849	0.005608102\\
0.8491849	0.004977904\\
0.8492849	0.004347695\\
0.8493849	0.003718092\\
0.8494849	0.003089726\\
0.849585	0.002463272\\
0.849685	0.001839315\\
0.849785	0.001218434\\
0.849885	0.0006013656\\
0.849985	-1.131125e-05\\
0.850085	-0.0006190789\\
0.850185	-0.001221269\\
0.850285	-0.00181742\\
0.850385	-0.002406829\\
0.850485	-0.002988947\\
0.8505851	-0.003563311\\
0.8506851	-0.00412934\\
0.8507851	-0.004686506\\
0.8508851	-0.005234301\\
0.8509851	-0.005772205\\
0.8510851	-0.0062998\\
0.8511851	-0.00681661\\
0.8512851	-0.007322142\\
0.8513851	-0.00781597\\
0.8514851	-0.008297747\\
0.8515852	-0.008767023\\
0.8516852	-0.009223383\\
0.8517852	-0.009666452\\
0.8518852	-0.01009581\\
0.8519852	-0.01051123\\
0.8520852	-0.01091236\\
0.8521852	-0.01129888\\
0.8522852	-0.01167057\\
0.8523852	-0.01202705\\
0.8524852	-0.01236813\\
0.8525853	-0.01269362\\
0.8526853	-0.01300321\\
0.8527853	-0.01329671\\
0.8528853	-0.01357401\\
0.8529853	-0.01383491\\
0.8530853	-0.01407928\\
0.8531853	-0.01430701\\
0.8532853	-0.0145179\\
0.8533853	-0.01471192\\
0.8534853	-0.01488907\\
0.8535854	-0.01504921\\
0.8536854	-0.01519234\\
0.8537854	-0.01531846\\
0.8538854	-0.01542748\\
0.8539854	-0.01551956\\
0.8540854	-0.01559467\\
0.8541854	-0.01565284\\
0.8542854	-0.01569419\\
0.8543854	-0.01571879\\
0.8544854	-0.01572674\\
0.8545855	-0.01571821\\
0.8546855	-0.01569321\\
0.8547855	-0.01565195\\
0.8548855	-0.01559474\\
0.8549855	-0.01552166\\
0.8550855	-0.01543289\\
0.8551855	-0.01532868\\
0.8552855	-0.01520924\\
0.8553855	-0.01507481\\
0.8554855	-0.01492576\\
0.8555856	-0.01476224\\
0.8556856	-0.01458457\\
0.8557856	-0.01439307\\
0.8558856	-0.01418801\\
0.8559856	-0.01396975\\
0.8560856	-0.01373856\\
0.8561856	-0.01349482\\
0.8562856	-0.0132389\\
0.8563856	-0.01297123\\
0.8564856	-0.01269209\\
0.8565857	-0.01240193\\
0.8566857	-0.0121011\\
0.8567857	-0.01179001\\
0.8568857	-0.0114691\\
0.8569857	-0.01113882\\
0.8570857	-0.01079948\\
0.8571857	-0.01045156\\
0.8572857	-0.01009552\\
0.8573857	-0.009731769\\
0.8574857	-0.009360819\\
0.8575858	-0.008983088\\
0.8576858	-0.008598992\\
0.8577858	-0.008209003\\
0.8578858	-0.007813639\\
0.8579858	-0.007413239\\
0.8580858	-0.007008372\\
0.8581858	-0.006599413\\
0.8582858	-0.006186872\\
0.8583858	-0.00577127\\
0.8584858	-0.005353101\\
0.8585859	-0.004932699\\
0.8586859	-0.004510559\\
0.8587859	-0.004087205\\
0.8588859	-0.003663033\\
0.8589859	-0.00323856\\
0.8590859	-0.002814219\\
0.8591859	-0.002390426\\
0.8592859	-0.001967685\\
0.8593859	-0.001546429\\
0.8594859	-0.001126992\\
0.859586	-0.0007099133\\
0.859686	-0.0002955651\\
0.859786	0.0001156477\\
0.859886	0.0005232172\\
0.859986	0.000926789\\
0.860086	0.001326026\\
0.860186	0.001720471\\
0.860286	0.002109766\\
0.860386	0.002493507\\
0.860486	0.002871341\\
0.8605861	0.003242916\\
0.8606861	0.003607935\\
0.8607861	0.003966035\\
0.8608861	0.00431684\\
0.8609861	0.004660108\\
0.8610861	0.004995483\\
0.8611861	0.005322643\\
0.8612861	0.005641387\\
0.8613861	0.005951411\\
0.8614861	0.006252409\\
0.8615862	0.006544209\\
0.8616862	0.006826582\\
0.8617862	0.007099197\\
0.8618862	0.007361918\\
0.8619862	0.007614492\\
0.8620862	0.007856712\\
0.8621862	0.008088531\\
0.8622862	0.008309734\\
0.8623862	0.008520124\\
0.8624862	0.008719578\\
0.8625863	0.008907993\\
0.8626863	0.009085259\\
0.8627863	0.009251262\\
0.8628863	0.009405943\\
0.8629863	0.009549133\\
0.8630863	0.00968084\\
0.8631863	0.009801048\\
0.8632863	0.009909685\\
0.8633863	0.01000671\\
0.8634863	0.01009214\\
0.8635864	0.010166\\
0.8636864	0.01022827\\
0.8637864	0.01027908\\
0.8638864	0.01031835\\
0.8639864	0.01034612\\
0.8640864	0.01036254\\
0.8641864	0.01036763\\
0.8642864	0.01036144\\
0.8643864	0.01034421\\
0.8644864	0.01031592\\
0.8645865	0.01027672\\
0.8646865	0.01022677\\
0.8647865	0.01016623\\
0.8648865	0.01009517\\
0.8649865	0.01001381\\
0.8650865	0.009922352\\
0.8651865	0.009820914\\
0.8652865	0.009709757\\
0.8653865	0.009589112\\
0.8654865	0.009459069\\
0.8655866	0.009319852\\
0.8656866	0.009171775\\
0.8657866	0.009014994\\
0.8658866	0.008849815\\
0.8659866	0.008676516\\
0.8660866	0.008495292\\
0.8661866	0.008306369\\
0.8662866	0.008110085\\
0.8663866	0.007906732\\
0.8664866	0.007696519\\
0.8665867	0.00747977\\
0.8666867	0.007256762\\
0.8667867	0.007027768\\
0.8668867	0.006793146\\
0.8669867	0.006553186\\
0.8670867	0.006308111\\
0.8671867	0.006058291\\
0.8672867	0.005804107\\
0.8673867	0.005545769\\
0.8674867	0.005283644\\
0.8675868	0.005018064\\
0.8676868	0.004749259\\
0.8677868	0.004477564\\
0.8678868	0.004203383\\
0.8679868	0.003927024\\
0.8680868	0.00364877\\
0.8681868	0.003368941\\
0.8682868	0.003087936\\
0.8683868	0.002805957\\
0.8684868	0.002523366\\
0.8685869	0.002240522\\
0.8686869	0.001957669\\
0.8687869	0.001675097\\
0.8688869	0.001393187\\
0.8689869	0.001112248\\
0.8690869	0.0008324827\\
0.8691869	0.0005543114\\
0.8692869	0.0002779616\\
0.8693869	3.653627e-06\\
0.8694869	-0.0002682388\\
0.869587	-0.0005374359\\
0.869687	-0.0008037026\\
0.869787	-0.001066746\\
0.869887	-0.001326307\\
0.869987	-0.00158217\\
0.870087	-0.001834077\\
0.870187	-0.002081708\\
0.870287	-0.00232488\\
0.870387	-0.002563389\\
0.870487	-0.002797002\\
0.8705871	-0.00302544\\
0.8706871	-0.003248553\\
0.8707871	-0.003466124\\
0.8708871	-0.00367795\\
0.8709871	-0.003883909\\
0.8710871	-0.004083806\\
0.8711871	-0.004277386\\
0.8712871	-0.004464513\\
0.8713871	-0.00464511\\
0.8714871	-0.004819016\\
0.8715872	-0.004986062\\
0.8716872	-0.005146208\\
0.8717872	-0.005299251\\
0.8718872	-0.005445012\\
0.8719872	-0.00558354\\
0.8720872	-0.005714624\\
0.8721872	-0.005838196\\
0.8722872	-0.005954222\\
0.8723872	-0.006062661\\
0.8724872	-0.006163412\\
0.8725873	-0.006256461\\
0.8726873	-0.006341807\\
0.8727873	-0.006419401\\
0.8728873	-0.006489147\\
0.8729873	-0.006551079\\
0.8730873	-0.006605245\\
0.8731873	-0.006651578\\
0.8732873	-0.006690114\\
0.8733873	-0.006720944\\
0.8734873	-0.006743995\\
0.8735874	-0.006759287\\
0.8736874	-0.006766989\\
0.8737874	-0.006767128\\
0.8738874	-0.006759761\\
0.8739874	-0.006744916\\
0.8740874	-0.006722748\\
0.8741874	-0.006693255\\
0.8742874	-0.006656541\\
0.8743874	-0.006612829\\
0.8744874	-0.006562132\\
0.8745875	-0.006504542\\
0.8746875	-0.006440221\\
0.8747875	-0.006369339\\
0.8748875	-0.006292008\\
0.8749875	-0.006208341\\
0.8750875	-0.00611853\\
0.8751875	-0.006022671\\
0.8752875	-0.005920902\\
0.8753875	-0.005813449\\
0.8754875	-0.005700491\\
0.8755876	-0.005582208\\
0.8756876	-0.005458695\\
0.8757876	-0.005330242\\
0.8758876	-0.00519696\\
0.8759876	-0.005059017\\
0.8760876	-0.004916674\\
0.8761876	-0.00477013\\
0.8762876	-0.004619532\\
0.8763876	-0.004465069\\
0.8764876	-0.004307076\\
0.8765877	-0.004145668\\
0.8766877	-0.003981028\\
0.8767877	-0.003813443\\
0.8768877	-0.003643103\\
0.8769877	-0.003470149\\
0.8770877	-0.003294859\\
0.8771877	-0.003117474\\
0.8772877	-0.00293824\\
0.8773877	-0.002757271\\
0.8774877	-0.002574872\\
0.8775878	-0.002391192\\
0.8776878	-0.002206483\\
0.8777878	-0.002020925\\
0.8778878	-0.001834843\\
0.8779878	-0.001648429\\
0.8780878	-0.001461772\\
0.8781878	-0.001275173\\
0.8782878	-0.001088816\\
0.8783878	-0.000902906\\
0.8784878	-0.0007176433\\
0.8785879	-0.0005332739\\
0.8786879	-0.0003499429\\
0.8787879	-0.0001678464\\
0.8788879	1.277188e-05\\
0.8789879	0.0001916535\\
0.8790879	0.0003687088\\
0.8791879	0.0005437512\\
0.8792879	0.0007166023\\
0.8793879	0.0008870631\\
0.8794879	0.00105503\\
0.879588	0.001220266\\
0.879688	0.001382561\\
0.879788	0.001541822\\
0.879888	0.00169794\\
0.879988	0.001850724\\
0.880088	0.001999959\\
0.880188	0.002145606\\
0.880288	0.002287498\\
0.880388	0.002425476\\
0.880488	0.002559451\\
0.8805881	0.002689332\\
0.8806881	0.002814994\\
0.8807881	0.002936251\\
0.8808881	0.003053081\\
0.8809881	0.003165411\\
0.8810881	0.003273105\\
0.8811881	0.003376098\\
0.8812881	0.003474365\\
0.8813881	0.003567767\\
0.8814881	0.003656189\\
0.8815882	0.003739644\\
0.8816882	0.003818123\\
0.8817882	0.003891531\\
0.8818882	0.003959819\\
0.8819882	0.004022989\\
0.8820882	0.004080995\\
0.8821882	0.004133765\\
0.8822882	0.004181328\\
0.8823882	0.004223733\\
0.8824882	0.004260952\\
0.8825883	0.004292966\\
0.8826883	0.004319732\\
0.8827883	0.004341397\\
0.8828883	0.004357945\\
0.8829883	0.00436929\\
0.8830883	0.004375576\\
0.8831883	0.004376806\\
0.8832883	0.004372972\\
0.8833883	0.004364126\\
0.8834883	0.004350427\\
0.8835884	0.004331906\\
0.8836884	0.004308579\\
0.8837884	0.004280541\\
0.8838884	0.004247874\\
0.8839884	0.004210623\\
0.8840884	0.00416886\\
0.8841884	0.00412272\\
0.8842884	0.004072312\\
0.8843884	0.00401771\\
0.8844884	0.003958973\\
0.8845885	0.003896217\\
0.8846885	0.003829595\\
0.8847885	0.003759203\\
0.8848885	0.003685087\\
0.8849885	0.003607482\\
0.8850885	0.003526449\\
0.8851885	0.003442129\\
0.8852885	0.003354545\\
0.8853885	0.003263945\\
0.8854885	0.003170519\\
0.8855886	0.003074238\\
0.8856886	0.002975296\\
0.8857886	0.002873873\\
0.8858886	0.002770041\\
0.8859886	0.00266397\\
0.8860886	0.002555837\\
0.8861886	0.002445832\\
0.8862886	0.00233404\\
0.8863886	0.002220564\\
0.8864886	0.002105583\\
0.8865887	0.001989278\\
0.8866887	0.001871746\\
0.8867887	0.001753107\\
0.8868887	0.001633582\\
0.8869887	0.001513284\\
0.8870887	0.001392346\\
0.8871887	0.001270896\\
0.8872887	0.001149139\\
0.8873887	0.001027243\\
0.8874887	0.0009052652\\
0.8875888	0.0007833613\\
0.8876888	0.0006616655\\
0.8877888	0.0005403498\\
0.8878888	0.0004195478\\
0.8879888	0.0002993438\\
0.8880888	0.0001800241\\
0.8881888	6.162198e-05\\
0.8882888	-5.576061e-05\\
0.8883888	-0.0001719935\\
0.8884888	-0.0002869218\\
0.8885889	-0.0004004435\\
0.8886889	-0.0005124346\\
0.8887889	-0.000622804\\
0.8888889	-0.0007313612\\
0.8889889	-0.0008380549\\
0.8890889	-0.0009428065\\
0.8891889	-0.001045497\\
0.8892889	-0.001145984\\
0.8893889	-0.001244164\\
0.8894889	-0.001339978\\
0.889589	-0.001433341\\
0.889689	-0.001524135\\
0.889789	-0.001612262\\
0.889889	-0.001697741\\
0.889989	-0.001780401\\
0.890089	-0.001860182\\
0.890189	-0.001936995\\
0.890289	-0.002010904\\
0.890389	-0.002081802\\
0.890489	-0.002149564\\
0.8905891	-0.00221417\\
0.8906891	-0.002275571\\
0.8907891	-0.002333702\\
0.8908891	-0.002388511\\
0.8909891	-0.002440023\\
0.8910891	-0.00248822\\
0.8911891	-0.002533053\\
0.8912891	-0.002574463\\
0.8913891	-0.002612466\\
0.8914891	-0.002647062\\
0.8915892	-0.00267826\\
0.8916892	-0.00270603\\
0.8917892	-0.002730329\\
0.8918892	-0.002751221\\
0.8919892	-0.00276871\\
0.8920892	-0.002782737\\
0.8921892	-0.002793336\\
0.8922892	-0.002800592\\
0.8923892	-0.002804508\\
0.8924892	-0.002805091\\
0.8925893	-0.002802337\\
0.8926893	-0.002796347\\
0.8927893	-0.002787123\\
0.8928893	-0.002774658\\
0.8929893	-0.002759045\\
0.8930893	-0.002740338\\
0.8931893	-0.002718659\\
0.8932893	-0.002693952\\
0.8933893	-0.002666283\\
0.8934893	-0.002635705\\
0.8935894	-0.002602362\\
0.8936894	-0.002566215\\
0.8937894	-0.002527392\\
0.8938894	-0.002485981\\
0.8939894	-0.002442052\\
0.8940894	-0.002395679\\
0.8941894	-0.002346874\\
0.8942894	-0.002295776\\
0.8943894	-0.002242454\\
0.8944894	-0.002187048\\
0.8945895	-0.00212956\\
0.8946895	-0.002070056\\
0.8947895	-0.002008673\\
0.8948895	-0.00194553\\
0.8949895	-0.001880662\\
0.8950895	-0.001814196\\
0.8951895	-0.001746265\\
0.8952895	-0.00167697\\
0.8953895	-0.001606332\\
0.8954895	-0.001534427\\
0.8955896	-0.001461371\\
0.8956896	-0.001387333\\
0.8957896	-0.001312327\\
0.8958896	-0.001236466\\
0.8959896	-0.001159896\\
0.8960896	-0.001082713\\
0.8961896	-0.001005004\\
0.8962896	-0.0009268008\\
0.8963896	-0.0008482553\\
0.8964896	-0.0007695017\\
0.8965897	-0.0006905727\\
0.8966897	-0.0006116319\\
0.8967897	-0.0005326494\\
0.8968897	-0.0004538109\\
0.8969897	-0.0003752039\\
0.8970897	-0.0002969088\\
0.8971897	-0.0002189686\\
0.8972897	-0.0001415388\\
0.8973897	-6.474435e-05\\
0.8974897	1.135553e-05\\
0.8975898	8.673673e-05\\
0.8976898	0.0001613356\\
0.8977898	0.0002349721\\
0.8978898	0.0003076033\\
0.8979898	0.0003791612\\
0.8980898	0.0004495831\\
0.8981898	0.0005187365\\
0.8982898	0.0005865731\\
0.8983898	0.0006530895\\
0.8984898	0.0007182412\\
0.8985899	0.0007818732\\
0.8986899	0.0008439506\\
0.8987899	0.0009044012\\
0.8988899	0.0009632091\\
0.8989899	0.001020284\\
0.8990899	0.001075585\\
0.8991899	0.001129013\\
0.8992899	0.001180565\\
0.8993899	0.001230203\\
0.8994899	0.001277867\\
0.89959	0.001323474\\
0.89969	0.001367072\\
0.89979	0.001408638\\
0.89989	0.001448135\\
0.89999	0.001485496\\
0.90009	0.001520683\\
0.90019	0.001553687\\
0.90029	0.001584547\\
0.90039	0.00161315\\
0.90049	0.001639522\\
0.9005901	0.001663655\\
0.9006901	0.00168562\\
0.9007901	0.001705354\\
0.9008901	0.001722822\\
0.9009901	0.001738022\\
0.9010901	0.001750975\\
0.9011901	0.001761725\\
0.9012901	0.001770205\\
0.9013901	0.001776459\\
0.9014901	0.001780522\\
0.9015902	0.001782404\\
0.9016902	0.001782129\\
0.9017902	0.001779662\\
0.9018902	0.001775022\\
0.9019902	0.0017683\\
0.9020902	0.001759545\\
0.9021902	0.001748786\\
0.9022902	0.001735968\\
0.9023902	0.001721156\\
0.9024902	0.001704429\\
0.9025903	0.001685834\\
0.9026903	0.001665325\\
0.9027903	0.001643008\\
0.9028903	0.001618938\\
0.9029903	0.001593189\\
0.9030903	0.001565763\\
0.9031903	0.001536677\\
0.9032903	0.001505975\\
0.9033903	0.001473777\\
0.9034903	0.001440098\\
0.9035904	0.001405034\\
0.9036904	0.001368567\\
0.9037904	0.001330843\\
0.9038904	0.001291863\\
0.9039904	0.001251738\\
0.9040904	0.001210496\\
0.9041904	0.001168143\\
0.9042904	0.001124786\\
0.9043904	0.001080529\\
0.9044904	0.001035408\\
0.9045905	0.0009894248\\
0.9046905	0.0009426631\\
0.9047905	0.0008952401\\
0.9048905	0.0008472355\\
0.9049905	0.0007986446\\
0.9050905	0.000749528\\
0.9051905	0.0007000204\\
0.9052905	0.0006501788\\
0.9053905	0.0006001026\\
0.9054905	0.0005497509\\
0.9055906	0.0004992052\\
0.9056906	0.0004485384\\
0.9057906	0.0003978007\\
0.9058906	0.0003471261\\
0.9059906	0.000296517\\
0.9060906	0.0002460507\\
0.9061906	0.0001957869\\
0.9062906	0.0001458279\\
0.9063906	9.617033e-05\\
0.9064906	4.688602e-05\\
0.9065907	-2.018335e-06\\
0.9066907	-5.042824e-05\\
0.9067907	-9.824881e-05\\
0.9068907	-0.0001454181\\
0.9069907	-0.0001919652\\
0.9070907	-0.0002378445\\
0.9071907	-0.0002829564\\
0.9072907	-0.0003272178\\
0.9073907	-0.0003706642\\
0.9074907	-0.00041324\\
0.9075908	-0.0004548881\\
0.9076908	-0.0004955404\\
0.9077908	-0.0005351302\\
0.9078908	-0.0005737015\\
0.9079908	-0.0006112001\\
0.9080908	-0.0006475903\\
0.9081908	-0.0006828073\\
0.9082908	-0.0007167913\\
0.9083908	-0.0007495777\\
0.9084908	-0.0007810742\\
0.9085909	-0.0008112999\\
0.9086909	-0.0008402135\\
0.9087909	-0.0008677686\\
0.9088909	-0.0008939607\\
0.9089909	-0.0009188671\\
0.9090909	-0.0009424065\\
0.9091909	-0.0009645379\\
0.9092909	-0.0009852206\\
0.9093909	-0.001004479\\
0.9094909	-0.001022314\\
0.909591	-0.001038717\\
0.909691	-0.001053653\\
0.909791	-0.001067104\\
0.909891	-0.001079134\\
0.909991	-0.001089712\\
0.910091	-0.001098864\\
0.910191	-0.001106549\\
0.910291	-0.001112719\\
0.910391	-0.001117493\\
0.910491	-0.001120858\\
0.9105911	-0.001122823\\
0.9106911	-0.001123337\\
0.9107911	-0.001122461\\
0.9108911	-0.001120168\\
0.9109911	-0.001116472\\
0.9110911	-0.001111452\\
0.9111911	-0.001105079\\
0.9112911	-0.001097408\\
0.9113911	-0.00108842\\
0.9114911	-0.001078232\\
0.9115912	-0.001066826\\
0.9116912	-0.001054232\\
0.9117912	-0.001040426\\
0.9118912	-0.001025437\\
0.9119912	-0.001009329\\
0.9120912	-0.0009921704\\
0.9121912	-0.0009739627\\
0.9122912	-0.0009546863\\
0.9123912	-0.0009344138\\
0.9124912	-0.0009132283\\
0.9125913	-0.0008911373\\
0.9126913	-0.0008681498\\
0.9127913	-0.0008443038\\
0.9128913	-0.0008196012\\
0.9129913	-0.0007941195\\
0.9130913	-0.0007679464\\
0.9131913	-0.0007410481\\
0.9132913	-0.0007135474\\
0.9133913	-0.000685403\\
0.9134913	-0.0006566829\\
0.9135914	-0.000627446\\
0.9136914	-0.0005977322\\
0.9137914	-0.0005675818\\
0.9138914	-0.000537021\\
0.9139914	-0.0005060834\\
0.9140914	-0.0004748754\\
0.9141914	-0.0004434219\\
0.9142914	-0.0004117687\\
0.9143914	-0.0003798596\\
0.9144914	-0.0003477802\\
0.9145915	-0.0003155631\\
0.9146915	-0.0002832883\\
0.9147915	-0.0002510144\\
0.9148915	-0.0002187426\\
0.9149915	-0.0001865109\\
0.9150915	-0.0001543733\\
0.9151915	-0.0001224323\\
0.9152915	-9.06996e-05\\
0.9153915	-5.912436e-05\\
0.9154915	-2.782199e-05\\
0.9155916	3.20525e-06\\
0.9156916	3.393477e-05\\
0.9157916	6.427759e-05\\
0.9158916	9.4204e-05\\
0.9159916	0.0001237246\\
0.9160916	0.0001527832\\
0.9161916	0.0001813798\\
0.9162916	0.0002094373\\
0.9163916	0.0002369113\\
0.9164916	0.0002638041\\
0.9165917	0.0002900753\\
0.9166917	0.0003157594\\
0.9167917	0.0003407506\\
0.9168917	0.0003650209\\
0.9169917	0.0003885546\\
0.9170917	0.0004113641\\
0.9171917	0.0004334635\\
0.9172917	0.0004547956\\
0.9173917	0.0004753378\\
0.9174917	0.0004950563\\
0.9175918	0.0005139297\\
0.9176918	0.0005319685\\
0.9177918	0.0005491445\\
0.9178918	0.0005654374\\
0.9179918	0.0005808239\\
0.9180918	0.0005953205\\
0.9181918	0.000608899\\
0.9182918	0.0006216369\\
0.9183918	0.0006334381\\
0.9184918	0.0006443541\\
0.9185919	0.0006543051\\
0.9186919	0.0006632712\\
0.9187919	0.0006713088\\
0.9188919	0.0006784228\\
0.9189919	0.0006846236\\
0.9190919	0.000689896\\
0.9191919	0.0006942003\\
0.9192919	0.0006975456\\
0.9193919	0.0007000165\\
0.9194919	0.0007015824\\
0.919592	0.0007022269\\
0.919692	0.0007019713\\
0.919792	0.0007007764\\
0.919892	0.0006987108\\
0.919992	0.000695785\\
0.920092	0.0006920557\\
0.920192	0.0006874626\\
0.920292	0.0006820375\\
0.920392	0.000675787\\
0.920492	0.0006687319\\
0.9205921	0.000660907\\
0.9206921	0.0006522891\\
0.9207921	0.0006429838\\
0.9208921	0.0006329239\\
0.9209921	0.0006221329\\
0.9210921	0.0006106702\\
0.9211921	0.000598596\\
0.9212921	0.0005859286\\
0.9213921	0.0005726826\\
0.9214921	0.0005587918\\
0.9215922	0.0005443371\\
0.9216922	0.0005292694\\
0.9217922	0.0005137157\\
0.9218922	0.0004976634\\
0.9219922	0.0004811471\\
0.9220922	0.0004641534\\
0.9221922	0.0004467598\\
0.9222922	0.0004289197\\
0.9223922	0.0004108218\\
0.9224922	0.0003923323\\
0.9225923	0.0003735958\\
0.9226923	0.0003544931\\
0.9227923	0.0003351965\\
0.9228923	0.0003155917\\
0.9229923	0.0002958719\\
0.9230923	0.0002759479\\
0.9231923	0.0002559562\\
0.9232923	0.0002357965\\
0.9233923	0.0002155902\\
0.9234923	0.0001952129\\
0.9235924	0.0001748689\\
0.9236924	0.0001544899\\
0.9237924	0.0001341886\\
0.9238924	0.0001138542\\
0.9239924	9.360446e-05\\
0.9240924	7.339741e-05\\
0.9241924	5.336952e-05\\
0.9242924	3.347187e-05\\
0.9243924	1.380624e-05\\
0.9244924	-5.705731e-06\\
0.9245925	-2.502708e-05\\
0.9246925	-4.41101e-05\\
0.9247925	-6.296059e-05\\
0.9248925	-8.146371e-05\\
0.9249925	-9.973799e-05\\
0.9250925	-0.0001175978\\
0.9251925	-0.0001352033\\
0.9252925	-0.0001523406\\
0.9253925	-0.000169197\\
0.9254925	-0.0001855519\\
0.9255926	-0.0002015417\\
0.9256926	-0.0002170045\\
0.9257926	-0.0002321319\\
0.9258926	-0.0002467368\\
0.9259926	-0.000260973\\
0.9260926	-0.0002746912\\
0.9261926	-0.0002878969\\
0.9262926	-0.0003005653\\
0.9263926	-0.0003126748\\
0.9264926	-0.0003243342\\
0.9265927	-0.0003354049\\
0.9266927	-0.0003459982\\
0.9267927	-0.0003559411\\
0.9268927	-0.0003653831\\
0.9269927	-0.0003741713\\
0.9270927	-0.0003824175\\
0.9271927	-0.000390074\\
0.9272927	-0.0003971703\\
0.9273927	-0.0004036408\\
0.9274927	-0.0004094718\\
0.9275928	-0.0004147501\\
0.9276928	-0.0004193626\\
0.9277928	-0.0004234712\\
0.9278928	-0.0004269205\\
0.9279928	-0.0004299012\\
0.9280928	-0.0004321939\\
0.9281928	-0.0004339569\\
0.9282928	-0.0004350709\\
0.9283928	-0.0004355919\\
0.9284928	-0.0004355909\\
0.9285929	-0.0004349681\\
0.9286929	-0.0004338519\\
0.9287929	-0.0004320739\\
0.9288929	-0.0004298075\\
0.9289929	-0.0004269159\\
0.9290929	-0.0004235853\\
0.9291929	-0.000419723\\
0.9292929	-0.0004153998\\
0.9293929	-0.0004106213\\
0.9294929	-0.0004052568\\
0.929593	-0.0003995153\\
0.929693	-0.0003932144\\
0.929793	-0.0003865214\\
0.929893	-0.0003793602\\
0.929993	-0.0003718212\\
0.930093	-0.0003638935\\
0.930193	-0.0003555227\\
0.930293	-0.0003468541\\
0.930393	-0.0003377084\\
0.930493	-0.0003283203\\
0.9305931	-0.0003185514\\
0.9306931	-0.000308509\\
0.9307931	-0.0002981711\\
0.9308931	-0.0002874864\\
0.9309931	-0.000276578\\
0.9310931	-0.0002653216\\
0.9311931	-0.0002539351\\
0.9312931	-0.0002423414\\
0.9313931	-0.0002305501\\
0.9314931	-0.0002186422\\
0.9315932	-0.0002064982\\
0.9316932	-0.0001942742\\
0.9317932	-0.0001818309\\
0.9318932	-0.0001693583\\
0.9319932	-0.0001567761\\
0.9320932	-0.0001441014\\
0.9321932	-0.0001314408\\
0.9322932	-0.0001186477\\
0.9323932	-0.00010591\\
0.9324932	-9.311439e-05\\
0.9325933	-8.032483e-05\\
0.9326933	-6.765279e-05\\
0.9327933	-5.496892e-05\\
0.9328933	-4.248491e-05\\
0.9329933	-2.999055e-05\\
0.9330933	-1.759902e-05\\
0.9331933	-5.3102e-06\\
0.9332933	6.930076e-06\\
0.9333933	1.895222e-05\\
0.9334933	3.086799e-05\\
0.9335934	4.256124e-05\\
0.9336934	5.407764e-05\\
0.9337934	6.545003e-05\\
0.9338934	7.652773e-05\\
0.9339934	8.747752e-05\\
0.9340934	9.809134e-05\\
0.9341934	0.0001084531\\
0.9342934	0.0001185794\\
0.9343934	0.0001283649\\
0.9344934	0.0001379756\\
0.9345935	0.0001472617\\
0.9346935	0.0001562467\\
0.9347935	0.0001650161\\
0.9348935	0.0001733529\\
0.9349935	0.000181449\\
0.9350935	0.000189165\\
0.9351935	0.0001965033\\
0.9352935	0.0002036057\\
0.9353935	0.0002102807\\
0.9354935	0.0002166831\\
0.9355936	0.0002227469\\
0.9356936	0.0002283885\\
0.9357936	0.0002337136\\
0.9358936	0.0002385904\\
0.9359936	0.0002431411\\
0.9360936	0.0002473594\\
0.9361936	0.0002511362\\
0.9362936	0.0002546154\\
0.9363936	0.0002576656\\
0.9364936	0.000260255\\
0.9365937	0.0002625313\\
0.9366937	0.00026434\\
0.9367937	0.0002658273\\
0.9368937	0.0002669488\\
0.9369937	0.0002676514\\
0.9370937	0.000268108\\
0.9371937	0.0002681388\\
0.9372937	0.0002678305\\
0.9373937	0.0002672021\\
0.9374937	0.0002661556\\
0.9375938	0.0002647864\\
0.9376938	0.0002631143\\
0.9377938	0.0002610668\\
0.9378938	0.0002587582\\
0.9379938	0.0002560758\\
0.9380938	0.0002530401\\
0.9381938	0.0002497553\\
0.9382938	0.0002460771\\
0.9383938	0.0002421646\\
0.9384938	0.0002380378\\
0.9385939	0.0002335641\\
0.9386939	0.00022892\\
0.9387939	0.0002240042\\
0.9388939	0.0002187862\\
0.9389939	0.0002133686\\
0.9390939	0.0002076902\\
0.9391939	0.0002017558\\
0.9392939	0.0001957224\\
0.9393939	0.0001894006\\
0.9394939	0.0001829393\\
0.939594	0.0001763469\\
0.939694	0.0001695281\\
0.939794	0.00016258\\
0.939894	0.0001554901\\
0.939994	0.0001481912\\
0.940094	0.0001408447\\
0.940194	0.0001334014\\
0.940294	0.0001258209\\
0.940394	0.0001182143\\
0.940494	0.0001105449\\
0.9405941	0.0001027199\\
0.9406941	9.489772e-05\\
0.9407941	8.69385e-05\\
0.9408941	7.89834e-05\\
0.9409941	7.107983e-05\\
0.9410941	6.312442e-05\\
0.9411941	5.51878e-05\\
0.9412941	4.732414e-05\\
0.9413941	3.945756e-05\\
0.9414941	3.158188e-05\\
0.9415942	2.38499e-05\\
0.9416942	1.60596e-05\\
0.9417942	8.385843e-06\\
0.9418942	7.992824e-07\\
0.9419942	-6.732947e-06\\
0.9420942	-1.419395e-05\\
0.9421942	-2.146385e-05\\
0.9422942	-2.872428e-05\\
0.9423942	-3.583204e-05\\
0.9424942	-4.278757e-05\\
0.9425943	-4.969905e-05\\
0.9426943	-5.640583e-05\\
0.9427943	-6.291067e-05\\
0.9428943	-6.929023e-05\\
0.9429943	-7.545666e-05\\
0.9430943	-8.138888e-05\\
0.9431943	-8.72419e-05\\
0.9432943	-9.290154e-05\\
0.9433943	-9.833666e-05\\
0.9434943	-0.0001036482\\
0.9435944	-0.0001087679\\
0.9436944	-0.0001136068\\
0.9437944	-0.0001182757\\
0.9438944	-0.0001227694\\
0.9439944	-0.0001269541\\
0.9440944	-0.0001309999\\
0.9441944	-0.0001348377\\
0.9442944	-0.0001384129\\
0.9443944	-0.0001417856\\
0.9444944	-0.0001449593\\
0.9445945	-0.0001478704\\
0.9446945	-0.0001504844\\
0.9447945	-0.0001529579\\
0.9448945	-0.0001550786\\
0.9449945	-0.0001569587\\
0.9450945	-0.0001586761\\
0.9451945	-0.0001601499\\
0.9452945	-0.0001613475\\
0.9453945	-0.0001624047\\
0.9454945	-0.0001631993\\
0.9455946	-0.0001636772\\
0.9456946	-0.0001639735\\
0.9457946	-0.0001640277\\
0.9458946	-0.0001637896\\
0.9459946	-0.0001633325\\
0.9460946	-0.0001627456\\
0.9461946	-0.0001618854\\
0.9462946	-0.0001607852\\
0.9463946	-0.0001596003\\
0.9464946	-0.0001581381\\
0.9465947	-0.0001564663\\
0.9466947	-0.0001546273\\
0.9467947	-0.0001526588\\
0.9468947	-0.0001503809\\
0.9469947	-0.0001479641\\
0.9470947	-0.0001454632\\
0.9471947	-0.0001427123\\
0.9472947	-0.00013981\\
0.9473947	-0.0001368222\\
0.9474947	-0.0001336336\\
0.9475948	-0.0001302567\\
0.9476948	-0.0001267444\\
0.9477948	-0.0001231174\\
0.9478948	-0.0001192919\\
0.9479948	-0.0001153148\\
0.9480948	-0.0001113375\\
0.9481948	-0.000107213\\
0.9482948	-0.0001029586\\
0.9483948	-9.867525e-05\\
0.9484948	-9.432337e-05\\
0.9485949	-8.983136e-05\\
0.9486949	-8.524388e-05\\
0.9487949	-8.069629e-05\\
0.9488949	-7.60269e-05\\
0.9489949	-7.124848e-05\\
0.9490949	-6.649597e-05\\
0.9491949	-6.171493e-05\\
0.9492949	-5.687401e-05\\
0.9493949	-5.196427e-05\\
0.9494949	-4.715294e-05\\
0.949595	-4.229532e-05\\
0.949695	-3.734785e-05\\
0.949795	-3.243197e-05\\
0.949895	-2.755955e-05\\
0.949995	-2.265126e-05\\
0.950095	-1.773553e-05\\
0.950195	-1.289864e-05\\
0.950295	-8.188434e-06\\
0.950395	-3.446604e-06\\
0.950495	1.230378e-06\\
0.9505951	5.766967e-06\\
0.9506951	1.026212e-05\\
0.9507951	1.476685e-05\\
0.9508951	1.919735e-05\\
0.9509951	2.347395e-05\\
0.9510951	2.77225e-05\\
0.9511951	3.193604e-05\\
0.9512951	3.601548e-05\\
0.9513951	3.99475e-05\\
0.9514951	4.3824e-05\\
0.9515952	4.760736e-05\\
0.9516952	5.124005e-05\\
0.9517952	5.470021e-05\\
0.9518952	5.813353e-05\\
0.9519952	6.146561e-05\\
0.9520952	6.460671e-05\\
0.9521952	6.764865e-05\\
0.9522952	7.060929e-05\\
0.9523952	7.346371e-05\\
0.9524952	7.612565e-05\\
0.9525953	7.858601e-05\\
0.9526953	8.099377e-05\\
0.9527953	8.32349e-05\\
0.9528953	8.528857e-05\\
0.9529953	8.718766e-05\\
0.9530953	8.902914e-05\\
0.9531953	9.074315e-05\\
0.9532953	9.227543e-05\\
0.9533953	9.364806e-05\\
0.9534953	9.493856e-05\\
0.9535954	9.610858e-05\\
0.9536954	9.704832e-05\\
0.9537954	9.782886e-05\\
0.9538954	9.849503e-05\\
0.9539954	9.902544e-05\\
0.9540954	9.934129e-05\\
0.9541954	9.945619e-05\\
0.9542954	9.947124e-05\\
0.9543954	9.941097e-05\\
0.9544954	9.913495e-05\\
0.9545955	9.872681e-05\\
0.9546955	9.822477e-05\\
0.9547955	9.762207e-05\\
0.9548955	9.689861e-05\\
0.9549955	9.596548e-05\\
0.9550955	9.49163e-05\\
0.9551955	9.381759e-05\\
0.9552955	9.256579e-05\\
0.9553955	9.117382e-05\\
0.9554955	8.964401e-05\\
0.9555956	8.806848e-05\\
0.9556956	8.64552e-05\\
0.9557956	8.468604e-05\\
0.9558956	8.278122e-05\\
0.9559956	8.082752e-05\\
0.9560956	7.884607e-05\\
0.9561956	7.674538e-05\\
0.9562956	7.4496e-05\\
0.9563956	7.217751e-05\\
0.9564956	6.980062e-05\\
0.9565957	6.738894e-05\\
0.9566957	6.483205e-05\\
0.9567957	6.215213e-05\\
0.9568957	5.946293e-05\\
0.9569957	5.677456e-05\\
0.9570957	5.403885e-05\\
0.9571957	5.120436e-05\\
0.9572957	4.835169e-05\\
0.9573957	4.548377e-05\\
0.9574957	4.269035e-05\\
0.9575958	3.978656e-05\\
0.9576958	3.67996e-05\\
0.9577958	3.382195e-05\\
0.9578958	3.087048e-05\\
0.9579958	2.789854e-05\\
0.9580958	2.490683e-05\\
0.9581958	2.185622e-05\\
0.9582958	1.8827e-05\\
0.9583958	1.587296e-05\\
0.9584958	1.292687e-05\\
0.9585959	9.954611e-06\\
0.9586959	6.977141e-06\\
0.9587959	4.049226e-06\\
0.9588959	1.222672e-06\\
0.9589959	-1.564407e-06\\
0.9590959	-4.390084e-06\\
0.9591959	-7.190421e-06\\
0.9592959	-9.912304e-06\\
0.9593959	-1.257897e-05\\
0.9594959	-1.517577e-05\\
0.959596	-1.777719e-05\\
0.959696	-2.035993e-05\\
0.959796	-2.284458e-05\\
0.959896	-2.519064e-05\\
0.959996	-2.747229e-05\\
0.960096	-2.973041e-05\\
0.960196	-3.192292e-05\\
0.960296	-3.401484e-05\\
0.960396	-3.600887e-05\\
0.960496	-3.790456e-05\\
0.9605961	-3.974617e-05\\
0.9606961	-4.160901e-05\\
0.9607961	-4.33374e-05\\
0.9608961	-4.497646e-05\\
0.9609961	-4.649172e-05\\
0.9610961	-4.79811e-05\\
0.9611961	-4.941674e-05\\
0.9612961	-5.081098e-05\\
0.9613961	-5.208677e-05\\
0.9614961	-5.321001e-05\\
0.9615962	-5.425316e-05\\
0.9616962	-5.524477e-05\\
0.9617962	-5.61538e-05\\
0.9618962	-5.695076e-05\\
0.9619962	-5.763279e-05\\
0.9620962	-5.815572e-05\\
0.9621962	-5.861486e-05\\
0.9622962	-5.90413e-05\\
0.9623962	-5.939856e-05\\
0.9624962	-5.963995e-05\\
0.9625963	-5.977153e-05\\
0.9626963	-5.97808e-05\\
0.9627963	-5.973001e-05\\
0.9628963	-5.963206e-05\\
0.9629963	-5.949151e-05\\
0.9630963	-5.920993e-05\\
0.9631963	-5.877825e-05\\
0.9632963	-5.831001e-05\\
0.9633963	-5.775135e-05\\
0.9634963	-5.713989e-05\\
0.9635964	-5.649434e-05\\
0.9636964	-5.574416e-05\\
0.9637964	-5.484352e-05\\
0.9638964	-5.386443e-05\\
0.9639964	-5.287404e-05\\
0.9640964	-5.183578e-05\\
0.9641964	-5.078164e-05\\
0.9642964	-4.964698e-05\\
0.9643964	-4.844686e-05\\
0.9644964	-4.716199e-05\\
0.9645965	-4.58358e-05\\
0.9646965	-4.452233e-05\\
0.9647965	-4.32216e-05\\
0.9648965	-4.182424e-05\\
0.9649965	-4.034757e-05\\
0.9650965	-3.881545e-05\\
0.9651965	-3.721019e-05\\
0.9652965	-3.560712e-05\\
0.9653965	-3.400325e-05\\
0.9654965	-3.239069e-05\\
0.9655966	-3.071029e-05\\
0.9656966	-2.89524e-05\\
0.9657966	-2.716492e-05\\
0.9658966	-2.535055e-05\\
0.9659966	-2.355234e-05\\
0.9660966	-2.178276e-05\\
0.9661966	-1.999491e-05\\
0.9662966	-1.820072e-05\\
0.9663966	-1.634873e-05\\
0.9664966	-1.44878e-05\\
0.9665967	-1.267005e-05\\
0.9666967	-1.086765e-05\\
0.9667967	-9.116258e-06\\
0.9668967	-7.382748e-06\\
0.9669967	-5.659736e-06\\
0.9670967	-3.856207e-06\\
0.9671967	-2.073757e-06\\
0.9672967	-3.015401e-07\\
0.9673967	1.409321e-06\\
0.9674967	3.103487e-06\\
0.9675968	4.747888e-06\\
0.9676968	6.329283e-06\\
0.9677968	7.960719e-06\\
0.9678968	9.584266e-06\\
0.9679968	1.117892e-05\\
0.9680968	1.270922e-05\\
0.9681968	1.417182e-05\\
0.9682968	1.558414e-05\\
0.9683968	1.692438e-05\\
0.9684968	1.821714e-05\\
0.9685969	1.95297e-05\\
0.9686969	2.079675e-05\\
0.9687969	2.204048e-05\\
0.9688969	2.316983e-05\\
0.9689969	2.423547e-05\\
0.9690969	2.525051e-05\\
0.9691969	2.61951e-05\\
0.9692969	2.707143e-05\\
0.9693969	2.795738e-05\\
0.9694969	2.885743e-05\\
0.969597	2.970463e-05\\
0.969697	3.048967e-05\\
0.969797	3.119347e-05\\
0.969897	3.185721e-05\\
0.969997	3.243469e-05\\
0.970097	3.293685e-05\\
0.970197	3.342019e-05\\
0.970297	3.389139e-05\\
0.970397	3.432793e-05\\
0.970497	3.469515e-05\\
0.9705971	3.500712e-05\\
0.9706971	3.524295e-05\\
0.9707971	3.541222e-05\\
0.9708971	3.548261e-05\\
0.9709971	3.547501e-05\\
0.9710971	3.544786e-05\\
0.9711971	3.539405e-05\\
0.9712971	3.530154e-05\\
0.9713971	3.516066e-05\\
0.9714971	3.496646e-05\\
0.9715972	3.468305e-05\\
0.9716972	3.436324e-05\\
0.9717972	3.394177e-05\\
0.9718972	3.344927e-05\\
0.9719972	3.293901e-05\\
0.9720972	3.243696e-05\\
0.9721972	3.187585e-05\\
0.9722972	3.131617e-05\\
0.9723972	3.072806e-05\\
0.9724972	3.012896e-05\\
0.9725973	2.947319e-05\\
0.9726973	2.875138e-05\\
0.9727973	2.796218e-05\\
0.9728973	2.71364e-05\\
0.9729973	2.630675e-05\\
0.9730973	2.541512e-05\\
0.9731973	2.453763e-05\\
0.9732973	2.363046e-05\\
0.9733973	2.277139e-05\\
0.9734973	2.190464e-05\\
0.9735974	2.100529e-05\\
0.9736974	2.006182e-05\\
0.9737974	1.909365e-05\\
0.9738974	1.810304e-05\\
0.9739974	1.707768e-05\\
0.9740974	1.600988e-05\\
0.9741974	1.491123e-05\\
0.9742974	1.382756e-05\\
0.9743974	1.274805e-05\\
0.9744974	1.169408e-05\\
0.9745975	1.064227e-05\\
0.9746975	9.587441e-06\\
0.9747975	8.539392e-06\\
0.9748975	7.527297e-06\\
0.9749975	6.474443e-06\\
0.9750975	5.413615e-06\\
0.9751975	4.317206e-06\\
0.9752975	3.235656e-06\\
0.9753975	2.150266e-06\\
0.9754975	1.066664e-06\\
0.9755976	-9.143546e-09\\
0.9756976	-1.090642e-06\\
0.9757976	-2.078352e-06\\
0.9758976	-3.070814e-06\\
0.9759976	-4.017893e-06\\
0.9760976	-4.928184e-06\\
0.9761976	-5.830648e-06\\
0.9762976	-6.706207e-06\\
0.9763976	-7.533788e-06\\
0.9764976	-8.338337e-06\\
0.9765977	-9.158341e-06\\
0.9766977	-9.983971e-06\\
0.9767977	-1.07706e-05\\
0.9768977	-1.153433e-05\\
0.9769977	-1.227609e-05\\
0.9770977	-1.300618e-05\\
0.9771977	-1.369363e-05\\
0.9772977	-1.436207e-05\\
0.9773977	-1.497996e-05\\
0.9774977	-1.553013e-05\\
0.9775978	-1.606819e-05\\
0.9776978	-1.653213e-05\\
0.9777978	-1.700189e-05\\
0.9778978	-1.741156e-05\\
0.9779978	-1.776975e-05\\
0.9780978	-1.810745e-05\\
0.9781978	-1.841557e-05\\
0.9782978	-1.871242e-05\\
0.9783978	-1.900276e-05\\
0.9784978	-1.924503e-05\\
0.9785979	-1.948526e-05\\
0.9786979	-1.967116e-05\\
0.9787979	-1.987227e-05\\
0.9788979	-2.004331e-05\\
0.9789979	-2.02009e-05\\
0.9790979	-2.032573e-05\\
0.9791979	-2.03864e-05\\
0.9792979	-2.041784e-05\\
0.9793979	-2.042169e-05\\
0.9794979	-2.040766e-05\\
0.979598	-2.033617e-05\\
0.979698	-2.024083e-05\\
0.979798	-2.006912e-05\\
0.979898	-1.986755e-05\\
0.979998	-1.964229e-05\\
0.980098	-1.937528e-05\\
0.980198	-1.909046e-05\\
0.980298	-1.878518e-05\\
0.980398	-1.842746e-05\\
0.980498	-1.80565e-05\\
0.9805981	-1.763383e-05\\
0.9806981	-1.724502e-05\\
0.9807981	-1.683584e-05\\
0.9808981	-1.6385e-05\\
0.9809981	-1.596304e-05\\
0.9810981	-1.547204e-05\\
0.9811981	-1.498452e-05\\
0.9812981	-1.44976e-05\\
0.9813981	-1.400851e-05\\
0.9814981	-1.351183e-05\\
0.9815982	-1.298781e-05\\
0.9816982	-1.246232e-05\\
0.9817982	-1.190807e-05\\
0.9818982	-1.134721e-05\\
0.9819982	-1.077537e-05\\
0.9820982	-1.022783e-05\\
0.9821982	-9.655368e-06\\
0.9822982	-9.071752e-06\\
0.9823982	-8.484171e-06\\
0.9824982	-7.881858e-06\\
0.9825983	-7.276432e-06\\
0.9826983	-6.658654e-06\\
0.9827983	-6.067627e-06\\
0.9828983	-5.492133e-06\\
0.9829983	-4.875683e-06\\
0.9830983	-4.282738e-06\\
0.9831983	-3.670925e-06\\
0.9832983	-3.046225e-06\\
0.9833983	-2.453479e-06\\
0.9834983	-1.850567e-06\\
0.9835984	-1.281635e-06\\
0.9836984	-6.912609e-07\\
0.9837984	-1.183728e-07\\
0.9838984	4.668229e-07\\
0.9839984	1.056571e-06\\
0.9840984	1.612449e-06\\
0.9841984	2.169751e-06\\
0.9842984	2.726979e-06\\
0.9843984	3.245234e-06\\
0.9844984	3.741393e-06\\
0.9845985	4.248198e-06\\
0.9846985	4.746528e-06\\
0.9847985	5.231483e-06\\
0.9848985	5.709952e-06\\
0.9849985	6.139684e-06\\
0.9850985	6.573356e-06\\
0.9851985	6.957878e-06\\
0.9852985	7.345594e-06\\
0.9853985	7.707198e-06\\
0.9854985	8.055832e-06\\
0.9855986	8.414134e-06\\
0.9856986	8.729e-06\\
0.9857986	9.029948e-06\\
0.9858986	9.330287e-06\\
0.9859986	9.587514e-06\\
0.9860986	9.825116e-06\\
0.9861986	1.004561e-05\\
0.9862986	1.027597e-05\\
0.9863986	1.049137e-05\\
0.9864986	1.068633e-05\\
0.9865987	1.089098e-05\\
0.9866987	1.107203e-05\\
0.9867987	1.124136e-05\\
0.9868987	1.137866e-05\\
0.9869987	1.152471e-05\\
0.9870987	1.164912e-05\\
0.9871987	1.176088e-05\\
0.9872987	1.186812e-05\\
0.9873987	1.199244e-05\\
0.9874987	1.210923e-05\\
0.9875988	1.218475e-05\\
0.9876988	1.229025e-05\\
0.9877988	1.235595e-05\\
0.9878988	1.238252e-05\\
0.9879988	1.243813e-05\\
0.9880988	1.244469e-05\\
0.9881988	1.24512e-05\\
0.9882988	1.243066e-05\\
0.9883988	1.241267e-05\\
0.9884988	1.238319e-05\\
0.9885989	1.231537e-05\\
0.9886989	1.223805e-05\\
0.9887989	1.212926e-05\\
0.9888989	1.198192e-05\\
0.9889989	1.18233e-05\\
0.9890989	1.162605e-05\\
0.9891989	1.140879e-05\\
0.9892989	1.118537e-05\\
0.9893989	1.096727e-05\\
0.9894989	1.072243e-05\\
0.989599	1.048219e-05\\
0.989699	1.025531e-05\\
0.989799	1.000924e-05\\
0.989899	9.79714e-06\\
0.989999	9.560506e-06\\
0.990099	9.354509e-06\\
0.990199	9.133763e-06\\
0.990299	8.925867e-06\\
0.990399	8.72674e-06\\
0.990499	8.520189e-06\\
0.9905991	8.333168e-06\\
0.9906991	8.142065e-06\\
0.9907991	7.925299e-06\\
0.9908991	7.726678e-06\\
0.9909991	7.480151e-06\\
0.9910991	7.235724e-06\\
0.9911991	6.953408e-06\\
0.9912991	6.652782e-06\\
0.9913991	6.342556e-06\\
0.9914991	6.002799e-06\\
0.9915992	5.658538e-06\\
0.9916992	5.320891e-06\\
0.9917992	5.005881e-06\\
0.9918992	4.674688e-06\\
0.9919992	4.383751e-06\\
0.9920992	4.117325e-06\\
0.9921992	3.869091e-06\\
0.9922992	3.602089e-06\\
0.9923992	3.352316e-06\\
0.9924992	3.115965e-06\\
0.9925993	2.837794e-06\\
0.9926993	2.527517e-06\\
0.9927993	2.248237e-06\\
0.9928993	1.895538e-06\\
0.9929993	1.549321e-06\\
0.9930993	1.194875e-06\\
0.9931993	8.399186e-07\\
0.9932993	5.143535e-07\\
0.9933993	1.937372e-07\\
0.9934993	-8.005678e-08\\
0.9935994	-3.380139e-07\\
0.9936994	-5.501023e-07\\
0.9937994	-7.841862e-07\\
0.9938994	-9.906148e-07\\
0.9939994	-1.200922e-06\\
0.9940994	-1.413063e-06\\
0.9941994	-1.661347e-06\\
0.9942994	-1.906948e-06\\
0.9943994	-2.151627e-06\\
0.9944994	-2.379361e-06\\
0.9945995	-2.579213e-06\\
0.9946995	-2.758155e-06\\
0.9947995	-2.9097e-06\\
0.9948995	-3.035487e-06\\
0.9949995	-3.13818e-06\\
0.9950995	-3.238311e-06\\
0.9951995	-3.345716e-06\\
0.9952995	-3.508143e-06\\
0.9953995	-3.659764e-06\\
0.9954995	-3.859306e-06\\
0.9955996	-4.033122e-06\\
0.9956996	-4.221493e-06\\
0.9957996	-4.383472e-06\\
0.9958996	-4.509436e-06\\
0.9959996	-4.614007e-06\\
0.9960996	-4.695386e-06\\
0.9961996	-4.782394e-06\\
0.9962996	-4.866546e-06\\
0.9963996	-4.997252e-06\\
0.9964996	-5.115425e-06\\
0.9965997	-5.266193e-06\\
0.9966997	-5.415849e-06\\
0.9967997	-5.529452e-06\\
0.9968997	-5.587387e-06\\
0.9969997	-5.649737e-06\\
0.9970997	-5.639734e-06\\
0.9971997	-5.618793e-06\\
0.9972997	-5.606409e-06\\
0.9973997	-5.610519e-06\\
0.9974997	-5.631819e-06\\
0.9975998	-5.660778e-06\\
0.9976998	-5.662149e-06\\
0.9977998	-5.629635e-06\\
0.9978998	-5.564114e-06\\
0.9979998	-5.458977e-06\\
0.9980998	-5.339135e-06\\
0.9981998	-5.217497e-06\\
0.9982998	-5.123966e-06\\
0.9983998	-5.05694e-06\\
0.9984998	-5.024595e-06\\
0.9985999	-4.979703e-06\\
0.9986999	-4.914009e-06\\
0.9987999	-4.842129e-06\\
0.9988999	-4.730069e-06\\
0.9989999	-4.609987e-06\\
0.9990999	-4.50583e-06\\
0.9991999	-4.446144e-06\\
0.9992999	-4.43541e-06\\
0.9993999	-4.412105e-06\\
0.9994999	-4.376313e-06\\
0.9996	-4.323155e-06\\
0.9997	-4.232815e-06\\
0.9998	-4.092071e-06\\
0.9999	-3.978604e-06\\
1	-3.859643e-06\\
};
\addlegendentry{$\text{imag(}\psi{}_\text{n}\text{(x,t))}$};

\addplot [color=red,mark size=0.7pt,only marks,mark=*,mark options={solid}]
  table[row sep=crcr]{%
0	1.12280550358738e-27\\
0.0167016701670167	3.5711682614031e-26\\
0.0334033403340334	1.03182681356081e-24\\
0.0501050105010501	2.70828286273305e-23\\
0.0668066806680668	6.45761203611166e-22\\
0.0835083508350835	1.39875159122457e-20\\
0.1002100210021	2.75232651094492e-19\\
0.116911691169117	4.91982923643958e-18\\
0.133613361336134	7.98897055527809e-17\\
0.15031503150315	1.17848030216476e-15\\
0.167016701670167	1.57922696574414e-14\\
0.183718371837184	1.92246078723463e-13\\
0.2004200420042	2.12598964626336e-12\\
0.217121712171217	2.13577508524757e-11\\
0.233823382338234	1.94912908752564e-10\\
0.250525052505251	1.61590715972605e-09\\
0.267226722672267	1.21697855013256e-08\\
0.283928392839284	8.32607191905476e-08\\
0.300630063006301	5.17473492549773e-07\\
0.317331733173317	2.92164048634288e-06\\
0.334033403340334	1.49849795972466e-05\\
0.350735073507351	6.98194227281616e-05\\
0.367436743674367	0.000295520128764376\\
0.384138413841384	0.0011362884619439\\
0.400840084008401	0.0039689976534691\\
0.417541754175418	0.0125940033170714\\
0.434243424342434	0.0363025791034471\\
0.450945094509451	0.0950608894170288\\
0.467646764676468	0.226129374883182\\
0.484348434843484	0.488655517497859\\
0.501050105010501	0.95926665303725\\
0.517751775177518	1.71067143587644\\
0.534453445344534	2.77130659976884\\
0.551155115511551	4.07843236130027\\
0.567856785678568	5.45246225894796\\
0.584558455845585	6.6219025920082\\
0.601260126012601	7.3057300993023\\
0.617961796179618	7.32209189799364\\
0.634663466346635	6.6664932355315\\
0.651365136513651	5.51379259867068\\
0.668066806680668	4.14280150430213\\
0.684768476847685	2.82766880251364\\
0.701470147014701	1.75328965865538\\
0.718171817181718	0.987573705149207\\
0.734873487348735	0.505331162207536\\
0.751575157515752	0.234894776477497\\
0.768276827682768	0.099188507067937\\
0.784978497849785	0.0380487211009603\\
0.801680168016802	0.013258960899664\\
0.818381838183818	0.00419729633274309\\
0.835083508350835	0.0012070367407305\\
0.851785178517852	0.000315327656668106\\
0.868486848684868	7.48332062273012e-05\\
0.885188518851885	1.61330826382423e-05\\
0.901890189018902	3.15959253948534e-06\\
0.918591859185919	5.62128386357581e-07\\
0.935293529352935	9.08512003874884e-08\\
0.951995199519952	1.33387931788587e-08\\
0.968696869686969	1.77907035711915e-09\\
0.985398539853985	2.15556152971362e-10\\
};
\addlegendentry{$\text{|}\psi{}_\text{a}\text{(x,t)|}^\text{2}$};

\addplot [color=black,mark size=0.7pt,only marks,mark=*,mark options={solid}]
  table[row sep=crcr]{%
0	-2.74993547506715e-14\\
0.0167016701670167	-9.67447892751169e-14\\
0.0334033403340334	8.35499497674032e-13\\
0.0501050105010501	3.77218611404016e-12\\
0.0668066806680668	-1.02657276538842e-11\\
0.0835083508350835	-1.18006895886979e-10\\
0.1002100210021	-3.32844962412669e-10\\
0.116911691169117	2.75421716014112e-10\\
0.133613361336134	6.39918969411723e-09\\
0.15031503150315	3.33642848196626e-08\\
0.167016701670167	1.24077013033903e-07\\
0.183718371837184	4.00734989464281e-07\\
0.2004200420042	1.25199074335831e-06\\
0.217121712171217	4.01023213491137e-06\\
0.233823382338234	1.30471096076136e-05\\
0.250525052505251	4.01019602772927e-05\\
0.267226722672267	0.000103103588004439\\
0.283928392839284	0.000171390422589169\\
0.300630063006301	-5.2819879722987e-05\\
0.317331733173317	-0.00136122042288101\\
0.334033403340334	-0.00365388012999901\\
0.350735073507351	-0.000916281500205732\\
0.367436743674367	0.015710562377742\\
0.384138413841384	0.0186938291712979\\
0.400840084008401	-0.0512970308159267\\
0.417541754175418	-0.055041154090313\\
0.434243424342434	0.183752775586047\\
0.450945094509451	-0.035619910409676\\
0.467646764676468	-0.382147873038368\\
0.484348434843484	0.673072775974356\\
0.501050105010501	-0.490751761967334\\
0.517751775177518	-0.142843533426192\\
0.534453445344534	0.947371534755941\\
0.551155115511551	-1.65790411864739\\
0.567856785678568	2.16115621339672\\
0.584558455845585	-2.45276037841647\\
0.601260126012601	2.52531849185241\\
0.617961796179618	-2.29256382646278\\
0.634663466346635	1.61736832907994\\
0.651365136513651	-0.477612660244532\\
0.668066806680668	-0.803065887338974\\
0.684768476847685	1.53780382812928\\
0.701470147014701	-1.18024714771184\\
0.718171817181718	0.0744690318706889\\
0.734873487348735	0.629808676711546\\
0.751575157515752	-0.330757104468726\\
0.768276827682768	-0.200247530159866\\
0.784978497849785	0.149596771252783\\
0.801680168016802	0.0859546276108174\\
0.818381838183818	-0.0278158654961181\\
0.835083508350835	-0.0347346332651556\\
0.851785178517852	-0.00912490313010176\\
0.868486848684868	0.00281908403346439\\
0.885188518851885	0.00351421289288571\\
0.901890189018902	0.00177040786945668\\
0.918591859185919	0.000640647425171167\\
0.935293529352935	0.0001955811108108\\
0.951995199519952	5.79830181181226e-05\\
0.968696869686969	1.96311328407086e-05\\
0.985398539853985	8.05289586178963e-06\\
};
\addlegendentry{$\text{imag(}\psi{}_\text{a}\text{(x,t))}$};

\end{axis}
\end{tikzpicture}%
		\caption{The numerical solution ($\psi_n$) plotted against the analytical solution ($\psi_a$).}
		\label{fig:smallErrorPlot}
	\end{subfigure}
	\begin{subfigure}{.9\linewidth}
		\setlength\figureheight{.5\linewidth}
		\setlength\figurewidth{.9\linewidth}
		% This file was created by matlab2tikz.
% Minimal pgfplots version: 1.3
%
%The latest updates can be retrieved from
%  http://www.mathworks.com/matlabcentral/fileexchange/22022-matlab2tikz
%where you can also make suggestions and rate matlab2tikz.
%
\definecolor{mycolor1}{rgb}{0.00000,0.44700,0.74100}%
%
\begin{tikzpicture}

\begin{axis}[%
width=0.95092\figurewidth,
height=\figureheight,
at={(0\figurewidth,0\figureheight)},
scale only axis,
xmin=0,
xmax=1,
xlabel={Position},
ymin=-5e-06,
ymax=5e-06,
ylabel={Magnitude},
title style={font=\bfseries},
title={$\text{Error of |}\psi{}_\text{n}\text{(x,t)|}^\text{2}\text{, dt = 2e-06, dx = 0.0001}$},
title style={font=\small},ticklabel style={font=\tiny}
]
\addplot [color=mycolor1,solid,forget plot]
  table[row sep=crcr]{%
0	-3.105907e-11\\
0.00010001	-3.064461e-11\\
0.00020002	-3.023558e-11\\
0.00030003	-2.983191e-11\\
0.00040004	-2.943352e-11\\
0.00050005	-2.904036e-11\\
0.00060006	-2.865235e-11\\
0.00070007	-2.826942e-11\\
0.00080008	-2.789152e-11\\
0.00090009	-2.751858e-11\\
0.0010001	-2.715052e-11\\
0.00110011	-2.67873e-11\\
0.00120012	-2.642885e-11\\
0.00130013	-2.60751e-11\\
0.00140014	-2.5726e-11\\
0.00150015	-2.538148e-11\\
0.00160016	-2.50415e-11\\
0.00170017	-2.470598e-11\\
0.00180018	-2.437487e-11\\
0.00190019	-2.404812e-11\\
0.0020002	-2.372567e-11\\
0.00210021	-2.340746e-11\\
0.00220022	-2.309344e-11\\
0.00230023	-2.278355e-11\\
0.00240024	-2.247774e-11\\
0.00250025	-2.217596e-11\\
0.00260026	-2.187816e-11\\
0.00270027	-2.158428e-11\\
0.00280028	-2.129428e-11\\
0.00290029	-2.10081e-11\\
0.0030003	-2.07257e-11\\
0.00310031	-2.044702e-11\\
0.00320032	-2.017202e-11\\
0.00330033	-1.990065e-11\\
0.00340034	-1.963286e-11\\
0.00350035	-1.936861e-11\\
0.00360036	-1.910785e-11\\
0.00370037	-1.885053e-11\\
0.00380038	-1.859662e-11\\
0.00390039	-1.834606e-11\\
0.0040004	-1.809882e-11\\
0.00410041	-1.785485e-11\\
0.00420042	-1.76141e-11\\
0.00430043	-1.737655e-11\\
0.00440044	-1.714213e-11\\
0.00450045	-1.691082e-11\\
0.00460046	-1.668258e-11\\
0.00470047	-1.645736e-11\\
0.00480048	-1.623512e-11\\
0.00490049	-1.601583e-11\\
0.0050005	-1.579945e-11\\
0.00510051	-1.558593e-11\\
0.00520052	-1.537525e-11\\
0.00530053	-1.516737e-11\\
0.00540054	-1.496224e-11\\
0.00550055	-1.475984e-11\\
0.00560056	-1.456012e-11\\
0.00570057	-1.436306e-11\\
0.00580058	-1.416862e-11\\
0.00590059	-1.397676e-11\\
0.0060006	-1.378745e-11\\
0.00610061	-1.360066e-11\\
0.00620062	-1.341635e-11\\
0.00630063	-1.323449e-11\\
0.00640064	-1.305506e-11\\
0.00650065	-1.287801e-11\\
0.00660066	-1.270332e-11\\
0.00670067	-1.253096e-11\\
0.00680068	-1.236089e-11\\
0.00690069	-1.219309e-11\\
0.0070007	-1.202753e-11\\
0.00710071	-1.186417e-11\\
0.00720072	-1.170299e-11\\
0.00730073	-1.154396e-11\\
0.00740074	-1.138706e-11\\
0.00750075	-1.123224e-11\\
0.00760076	-1.10795e-11\\
0.00770077	-1.092879e-11\\
0.00780078	-1.07801e-11\\
0.00790079	-1.063339e-11\\
0.0080008	-1.048864e-11\\
0.00810081	-1.034583e-11\\
0.00820082	-1.020493e-11\\
0.00830083	-1.006591e-11\\
0.00840084	-9.92875099999999e-12\\
0.00850085	-9.79342699999999e-12\\
0.00860086	-9.65991399999999e-12\\
0.00870087	-9.52818899999999e-12\\
0.00880088	-9.39822799999999e-12\\
0.00890089	-9.27000699999999e-12\\
0.0090009	-9.14350399999999e-12\\
0.00910091	-9.01869699999999e-12\\
0.00920092	-8.89556199999999e-12\\
0.00930093	-8.77407899999999e-12\\
0.00940094	-8.65422499999999e-12\\
0.00950095	-8.53597799999999e-12\\
0.00960096	-8.41931799999999e-12\\
0.00970097	-8.30422399999999e-12\\
0.00980098	-8.19067499999999e-12\\
0.00990099	-8.07865099999999e-12\\
0.010001	-7.96813199999999e-12\\
0.01010101	-7.85909699999999e-12\\
0.01020102	-7.75152799999999e-12\\
0.01030103	-7.64540499999999e-12\\
0.01040104	-7.54070899999999e-12\\
0.01050105	-7.43741999999999e-12\\
0.01060106	-7.33552199999999e-12\\
0.01070107	-7.23499399999999e-12\\
0.01080108	-7.13581999999999e-12\\
0.01090109	-7.03798099999999e-12\\
0.0110011	-6.94145899999999e-12\\
0.01110111	-6.84623799999999e-12\\
0.01120112	-6.75229899999999e-12\\
0.01130113	-6.65962699999999e-12\\
0.01140114	-6.56820299999999e-12\\
0.01150115	-6.47801299999999e-12\\
0.01160116	-6.38903899999999e-12\\
0.01170117	-6.30126499999999e-12\\
0.01180118	-6.21467599999999e-12\\
0.01190119	-6.12925599999999e-12\\
0.0120012	-6.04498799999999e-12\\
0.01210121	-5.96185899999999e-12\\
0.01220122	-5.87985299999999e-12\\
0.01230123	-5.79895499999999e-12\\
0.01240124	-5.71914999999999e-12\\
0.01250125	-5.64042399999998e-12\\
0.01260126	-5.56276199999999e-12\\
0.01270127	-5.48615099999998e-12\\
0.01280128	-5.41057599999998e-12\\
0.01290129	-5.33602499999998e-12\\
0.0130013	-5.26248199999998e-12\\
0.01310131	-5.18993499999998e-12\\
0.01320132	-5.11836999999998e-12\\
0.01330133	-5.04777499999998e-12\\
0.01340134	-4.97813599999998e-12\\
0.01350135	-4.90944099999998e-12\\
0.01360136	-4.84167799999998e-12\\
0.01370137	-4.77483299999998e-12\\
0.01380138	-4.70889499999998e-12\\
0.01390139	-4.64385199999998e-12\\
0.0140014	-4.57969099999998e-12\\
0.01410141	-4.51640099999998e-12\\
0.01420142	-4.45396999999998e-12\\
0.01430143	-4.39238799999998e-12\\
0.01440144	-4.33164199999998e-12\\
0.01450145	-4.27172099999998e-12\\
0.01460146	-4.21261399999998e-12\\
0.01470147	-4.15431199999998e-12\\
0.01480148	-4.09680199999998e-12\\
0.01490149	-4.04007399999997e-12\\
0.0150015	-3.98411799999998e-12\\
0.01510151	-3.92892299999997e-12\\
0.01520152	-3.87447999999997e-12\\
0.01530153	-3.82077799999997e-12\\
0.01540154	-3.76780699999997e-12\\
0.01550155	-3.71555799999997e-12\\
0.01560156	-3.66402099999997e-12\\
0.01570157	-3.61318699999997e-12\\
0.01580158	-3.56304499999997e-12\\
0.01590159	-3.51358699999997e-12\\
0.0160016	-3.46480399999997e-12\\
0.01610161	-3.41668599999997e-12\\
0.01620162	-3.36922499999997e-12\\
0.01630163	-3.32241099999997e-12\\
0.01640164	-3.27623699999997e-12\\
0.01650165	-3.23069299999997e-12\\
0.01660166	-3.18577199999996e-12\\
0.01670167	-3.14146399999996e-12\\
0.01680168	-3.09776199999996e-12\\
0.01690169	-3.05465799999996e-12\\
0.0170017	-3.01214199999996e-12\\
0.01710171	-2.97020899999996e-12\\
0.01720172	-2.92884899999996e-12\\
0.01730173	-2.88805499999996e-12\\
0.01740174	-2.84781899999996e-12\\
0.01750175	-2.80813399999996e-12\\
0.01760176	-2.76899299999996e-12\\
0.01770177	-2.73038799999996e-12\\
0.01780178	-2.69231199999995e-12\\
0.01790179	-2.65475799999995e-12\\
0.0180018	-2.61771799999995e-12\\
0.01810181	-2.58118699999995e-12\\
0.01820182	-2.54515599999995e-12\\
0.01830183	-2.50961999999995e-12\\
0.01840184	-2.47457099999995e-12\\
0.01850185	-2.44000399999995e-12\\
0.01860186	-2.40591099999995e-12\\
0.01870187	-2.37228599999995e-12\\
0.01880188	-2.33912299999995e-12\\
0.01890189	-2.30641599999994e-12\\
0.0190019	-2.27415799999994e-12\\
0.01910191	-2.24234399999994e-12\\
0.01920192	-2.21096699999994e-12\\
0.01930193	-2.18002199999994e-12\\
0.01940194	-2.14950299999994e-12\\
0.01950195	-2.11940299999994e-12\\
0.01960196	-2.08971799999994e-12\\
0.01970197	-2.06044099999993e-12\\
0.01980198	-2.03156799999993e-12\\
0.01990199	-2.00309199999993e-12\\
0.020002	-1.97500899999993e-12\\
0.02010201	-1.94731299999993e-12\\
0.02020202	-1.91999799999993e-12\\
0.02030203	-1.89305999999993e-12\\
0.02040204	-1.86649399999992e-12\\
0.02050205	-1.84029399999992e-12\\
0.02060206	-1.81445499999992e-12\\
0.02070207	-1.78897399999992e-12\\
0.02080208	-1.76384299999992e-12\\
0.02090209	-1.73906099999992e-12\\
0.0210021	-1.71461999999991e-12\\
0.02110211	-1.69051699999991e-12\\
0.02120212	-1.66674699999991e-12\\
0.02130213	-1.64330599999991e-12\\
0.02140214	-1.62018799999991e-12\\
0.02150215	-1.59739099999991e-12\\
0.02160216	-1.5749089999999e-12\\
0.02170217	-1.5527369999999e-12\\
0.02180218	-1.5308729999999e-12\\
0.02190219	-1.5093119999999e-12\\
0.0220022	-1.48804899999989e-12\\
0.02210221	-1.46707999999989e-12\\
0.02220222	-1.44640199999989e-12\\
0.02230223	-1.42600999999989e-12\\
0.02240224	-1.40590199999989e-12\\
0.02250225	-1.38607199999988e-12\\
0.02260226	-1.36651599999988e-12\\
0.02270227	-1.34723299999988e-12\\
0.02280228	-1.32821599999988e-12\\
0.02290229	-1.30946399999987e-12\\
0.0230023	-1.29097199999987e-12\\
0.02310231	-1.27273699999987e-12\\
0.02320232	-1.25475499999987e-12\\
0.02330233	-1.23702199999986e-12\\
0.02340234	-1.21953599999986e-12\\
0.02350235	-1.20229399999986e-12\\
0.02360236	-1.18528999999986e-12\\
0.02370237	-1.16852399999985e-12\\
0.02380238	-1.15198999999985e-12\\
0.02390239	-1.13568699999985e-12\\
0.0240024	-1.11960999999984e-12\\
0.02410241	-1.10375699999984e-12\\
0.02420242	-1.08812499999984e-12\\
0.02430243	-1.07271099999983e-12\\
0.02440244	-1.05751099999983e-12\\
0.02450245	-1.04252299999983e-12\\
0.02460246	-1.02774399999982e-12\\
0.02470247	-1.01317099999982e-12\\
0.02480248	-9.98801599999815e-13\\
0.02490249	-9.84632199999812e-13\\
0.0250025	-9.70660399999808e-13\\
0.02510251	-9.56883599999804e-13\\
0.02520252	-9.432991999998e-13\\
0.02530253	-9.29904299999796e-13\\
0.02540254	-9.16696599999792e-13\\
0.02550255	-9.03673199999787e-13\\
0.02560256	-8.90831899999783e-13\\
0.02570257	-8.78169999999779e-13\\
0.02580258	-8.65685099999774e-13\\
0.02590259	-8.53374699999769e-13\\
0.0260026	-8.41236499999765e-13\\
0.02610261	-8.2926809999976e-13\\
0.02620262	-8.17467199999755e-13\\
0.02630263	-8.0583139999975e-13\\
0.02640264	-7.94358399999745e-13\\
0.02650265	-7.8304619999974e-13\\
0.02660266	-7.71892399999735e-13\\
0.02670267	-7.60894799999729e-13\\
0.02680268	-7.50051399999724e-13\\
0.02690269	-7.39359899999718e-13\\
0.0270027	-7.28818299999712e-13\\
0.02710271	-7.18424499999707e-13\\
0.02720272	-7.08176499999701e-13\\
0.02730273	-6.98072299999695e-13\\
0.02740274	-6.88109899999688e-13\\
0.02750275	-6.78287299999682e-13\\
0.02760276	-6.68602699999676e-13\\
0.02770277	-6.59053999999669e-13\\
0.02780278	-6.49639499999662e-13\\
0.02790279	-6.40357299999656e-13\\
0.0280028	-6.31205499999649e-13\\
0.02810281	-6.22182399999641e-13\\
0.02820282	-6.13286099999634e-13\\
0.02830283	-6.04514999999627e-13\\
0.02840284	-5.95867199999619e-13\\
0.02850285	-5.87341199999612e-13\\
0.02860286	-5.78935199999604e-13\\
0.02870287	-5.70647399999596e-13\\
0.02880288	-5.62476399999588e-13\\
0.02890289	-5.54420499999579e-13\\
0.0290029	-5.46478099999571e-13\\
0.02910291	-5.38647599999562e-13\\
0.02920292	-5.30927499999553e-13\\
0.02930293	-5.23316299999544e-13\\
0.02940294	-5.15812299999535e-13\\
0.02950295	-5.08414299999526e-13\\
0.02960296	-5.01120599999516e-13\\
0.02970297	-4.93929799999506e-13\\
0.02980298	-4.86840599999496e-13\\
0.02990299	-4.79851399999486e-13\\
0.030003	-4.72960999999476e-13\\
0.03010301	-4.66167899999465e-13\\
0.03020302	-4.59470799999454e-13\\
0.03030303	-4.52868399999443e-13\\
0.03040304	-4.46359199999432e-13\\
0.03050305	-4.39942199999421e-13\\
0.03060306	-4.33615899999409e-13\\
0.03070307	-4.27379099999397e-13\\
0.03080308	-4.21230499999385e-13\\
0.03090309	-4.15168999999373e-13\\
0.0310031	-4.0919329999936e-13\\
0.03110311	-4.03302299999347e-13\\
0.03120312	-3.97494599999334e-13\\
0.03130313	-3.9176929999932e-13\\
0.03140314	-3.86125099999307e-13\\
0.03150315	-3.80560899999293e-13\\
0.03160316	-3.75075599999279e-13\\
0.03170317	-3.69668099999264e-13\\
0.03180318	-3.64337299999249e-13\\
0.03190319	-3.59082099999234e-13\\
0.0320032	-3.53901599999219e-13\\
0.03210321	-3.48794499999203e-13\\
0.03220322	-3.43759999999187e-13\\
0.03230323	-3.38796999999171e-13\\
0.03240324	-3.33904499999154e-13\\
0.03250325	-3.29081499999137e-13\\
0.03260326	-3.2432709999912e-13\\
0.03270327	-3.19640199999102e-13\\
0.03280328	-3.15020099999084e-13\\
0.03290329	-3.10465599999066e-13\\
0.0330033	-3.05975899999047e-13\\
0.03310331	-3.01550099999028e-13\\
0.03320332	-2.97187299999008e-13\\
0.03330333	-2.92886599998988e-13\\
0.03340334	-2.88647199998968e-13\\
0.03350335	-2.84468099998948e-13\\
0.03360336	-2.80348599998926e-13\\
0.03370337	-2.76287799998905e-13\\
0.03380338	-2.72284799998883e-13\\
0.03390339	-2.68338999998861e-13\\
0.0340034	-2.64449399998838e-13\\
0.03410341	-2.60615299998814e-13\\
0.03420342	-2.56835899998791e-13\\
0.03430343	-2.53110399998766e-13\\
0.03440344	-2.49438199998742e-13\\
0.03450345	-2.45818299998717e-13\\
0.03460346	-2.42250199998691e-13\\
0.03470347	-2.38732999998665e-13\\
0.03480348	-2.35265999998638e-13\\
0.03490349	-2.31848699998611e-13\\
0.0350035	-2.28480099998583e-13\\
0.03510351	-2.25159799998555e-13\\
0.03520352	-2.21886899998526e-13\\
0.03530353	-2.18660799998496e-13\\
0.03540354	-2.15480999998466e-13\\
0.03550355	-2.12346599998436e-13\\
0.03560356	-2.09257099998404e-13\\
0.03570357	-2.06211799998372e-13\\
0.03580358	-2.0321009999834e-13\\
0.03590359	-2.00251499998307e-13\\
0.0360036	-1.97335199998273e-13\\
0.03610361	-1.94460699998239e-13\\
0.03620362	-1.91627499998203e-13\\
0.03630363	-1.88834799998168e-13\\
0.03640364	-1.86082299998131e-13\\
0.03650365	-1.83369199998094e-13\\
0.03660366	-1.80694999998056e-13\\
0.03670367	-1.78059199998017e-13\\
0.03680368	-1.75461299997977e-13\\
0.03690369	-1.72900699997937e-13\\
0.0370037	-1.70376799997896e-13\\
0.03710371	-1.67889199997854e-13\\
0.03720372	-1.65437399997811e-13\\
0.03730373	-1.63020799997768e-13\\
0.03740374	-1.60638899997723e-13\\
0.03750375	-1.58291299997678e-13\\
0.03760376	-1.55977499997632e-13\\
0.03770377	-1.53696899997585e-13\\
0.03780378	-1.51449199997537e-13\\
0.03790379	-1.49233799997488e-13\\
0.0380038	-1.47050399997438e-13\\
0.03810381	-1.44898299997387e-13\\
0.03820382	-1.42777299997335e-13\\
0.03830383	-1.40686799997282e-13\\
0.03840384	-1.38626499997228e-13\\
0.03850385	-1.36595799997172e-13\\
0.03860386	-1.34594499997116e-13\\
0.03870387	-1.32621999997059e-13\\
0.03880388	-1.30677899997001e-13\\
0.03890389	-1.28761899996941e-13\\
0.0390039	-1.2687359999688e-13\\
0.03910391	-1.25012499996818e-13\\
0.03920392	-1.23178299996755e-13\\
0.03930393	-1.21370599996691e-13\\
0.03940394	-1.19588999996625e-13\\
0.03950395	-1.17833099996558e-13\\
0.03960396	-1.1610259999649e-13\\
0.03970397	-1.1439719999642e-13\\
0.03980398	-1.12716399996349e-13\\
0.03990399	-1.11059899996277e-13\\
0.040004	-1.09427399996203e-13\\
0.04010401	-1.07818499996128e-13\\
0.04020402	-1.06232899996051e-13\\
0.04030403	-1.04670199995973e-13\\
0.04040404	-1.03130199995893e-13\\
0.04050405	-1.01612499995812e-13\\
0.04060406	-1.00116799995729e-13\\
0.04070407	-9.86427399956443e-14\\
0.04080408	-9.71900599955581e-14\\
0.04090409	-9.57584499954702e-14\\
0.0410041	-9.43475999953805e-14\\
0.04110411	-9.29572299952892e-14\\
0.04120412	-9.1587019995196e-14\\
0.04130413	-9.0236699995101e-14\\
0.04140414	-8.89059999950041e-14\\
0.04150415	-8.75945999949053e-14\\
0.04160416	-8.63022499948046e-14\\
0.04170417	-8.5028689994702e-14\\
0.04180418	-8.37736299945973e-14\\
0.04190419	-8.25368199944905e-14\\
0.0420042	-8.13179799943817e-14\\
0.04210421	-8.01168799942708e-14\\
0.04220422	-7.89332299941576e-14\\
0.04230423	-7.77668199940423e-14\\
0.04240424	-7.66173899939247e-14\\
0.04250425	-7.54846699938048e-14\\
0.04260426	-7.43684599936826e-14\\
0.04270427	-7.3268509993558e-14\\
0.04280428	-7.21845699934309e-14\\
0.04290429	-7.11164299933014e-14\\
0.0430043	-7.00638599931693e-14\\
0.04310431	-6.90266299930346e-14\\
0.04320432	-6.80045299928973e-14\\
0.04330433	-6.69973399927574e-14\\
0.04340434	-6.60048399926147e-14\\
0.04350435	-6.50268199924692e-14\\
0.04360436	-6.40630699923209e-14\\
0.04370437	-6.31133999921696e-14\\
0.04380438	-6.21775899920155e-14\\
0.04390439	-6.12554599918583e-14\\
0.0440044	-6.03467899916981e-14\\
0.04410441	-5.94513999915347e-14\\
0.04420442	-5.85690999913681e-14\\
0.04430443	-5.76996999911983e-14\\
0.04440444	-5.68430099910252e-14\\
0.04450445	-5.59988499908487e-14\\
0.04460446	-5.51670499906688e-14\\
0.04470447	-5.43474099904854e-14\\
0.04480448	-5.35397699902984e-14\\
0.04490449	-5.27439599901078e-14\\
0.0450045	-5.19597899899134e-14\\
0.04510451	-5.11871199897153e-14\\
0.04520452	-5.04257699895133e-14\\
0.04530453	-4.96755699893074e-14\\
0.04540454	-4.89363599890975e-14\\
0.04550455	-4.82079899888835e-14\\
0.04560456	-4.74902999886653e-14\\
0.04570457	-4.67831399884429e-14\\
0.04580458	-4.60863499882162e-14\\
0.04590459	-4.5399779987985e-14\\
0.0460046	-4.47232899877494e-14\\
0.04610461	-4.40567299875092e-14\\
0.04620462	-4.33999499872643e-14\\
0.04630463	-4.27528299870147e-14\\
0.04640464	-4.21151999867602e-14\\
0.04650465	-4.14869499865008e-14\\
0.04660466	-4.08679199862363e-14\\
0.04670467	-4.02579999859667e-14\\
0.04680468	-3.96570399856919e-14\\
0.04690469	-3.90649199854117e-14\\
0.0470047	-3.84814999851262e-14\\
0.04710471	-3.7906669984835e-14\\
0.04720472	-3.73402999845383e-14\\
0.04730473	-3.67822599842357e-14\\
0.04740474	-3.62324399839274e-14\\
0.04750475	-3.5690709983613e-14\\
0.04760476	-3.51569499832925e-14\\
0.04770477	-3.46310599829659e-14\\
0.04780478	-3.41129199826329e-14\\
0.04790479	-3.36024099822935e-14\\
0.0480048	-3.30994199819475e-14\\
0.04810481	-3.26038499815947e-14\\
0.04820482	-3.21155899812352e-14\\
0.04830483	-3.16345199808687e-14\\
0.04840484	-3.11605599804951e-14\\
0.04850485	-3.06935799801143e-14\\
0.04860486	-3.02334899797262e-14\\
0.04870487	-2.97802099793305e-14\\
0.04880488	-2.93336099789272e-14\\
0.04890489	-2.8893599978516e-14\\
0.0490049	-2.8460089978097e-14\\
0.04910491	-2.80329799776698e-14\\
0.04920492	-2.76121899772344e-14\\
0.04930493	-2.71976199767905e-14\\
0.04940494	-2.67891699763381e-14\\
0.04950495	-2.6386759975877e-14\\
0.04960496	-2.5990299975407e-14\\
0.04970497	-2.55997099749279e-14\\
0.04980498	-2.52148999744395e-14\\
0.04990499	-2.48357699739417e-14\\
0.050005	-2.44622699734343e-14\\
0.05010501	-2.40942899729172e-14\\
0.05020502	-2.373176997239e-14\\
0.05030503	-2.33746099718527e-14\\
0.05040504	-2.30227399713051e-14\\
0.05050505	-2.26760899707469e-14\\
0.05060506	-2.23345699701779e-14\\
0.05070507	-2.19981299695979e-14\\
0.05080508	-2.16666599690068e-14\\
0.05090509	-2.13401199684043e-14\\
0.0510051	-2.10184099677903e-14\\
0.05110511	-2.07014899671643e-14\\
0.05120512	-2.03892699665264e-14\\
0.05130513	-2.00816899658761e-14\\
0.05140514	-1.97786699652134e-14\\
0.05150515	-1.94801599645379e-14\\
0.05160516	-1.91860799638494e-14\\
0.05170517	-1.88963699631477e-14\\
0.05180518	-1.86109699624324e-14\\
0.05190519	-1.83298199617035e-14\\
0.0520052	-1.80528399609605e-14\\
0.05210521	-1.77799899602032e-14\\
0.05220522	-1.75111999594314e-14\\
0.05230523	-1.72464099586448e-14\\
0.05240524	-1.6985569957843e-14\\
0.05250525	-1.67286099570259e-14\\
0.05260526	-1.64754799561931e-14\\
0.05270527	-1.62261199553442e-14\\
0.05280528	-1.59804799544791e-14\\
0.05290529	-1.57385099535974e-14\\
0.0530053	-1.55001399526988e-14\\
0.05310531	-1.5265329951783e-14\\
0.05320532	-1.50340299508495e-14\\
0.05330533	-1.48061799498982e-14\\
0.05340534	-1.45817299489287e-14\\
0.05350535	-1.43606399479405e-14\\
0.05360536	-1.41428499469335e-14\\
0.05370537	-1.39283199459071e-14\\
0.05380538	-1.37169899448611e-14\\
0.05390539	-1.3508829943795e-14\\
0.0540054	-1.33037799427085e-14\\
0.05410541	-1.31017999416013e-14\\
0.05420542	-1.29028499404728e-14\\
0.05430543	-1.27068699393227e-14\\
0.05440544	-1.25138299381506e-14\\
0.05450545	-1.23236799369561e-14\\
0.05460546	-1.21363799357388e-14\\
0.05470547	-1.19518999344981e-14\\
0.05480548	-1.17701699332338e-14\\
0.05490549	-1.15911799319453e-14\\
0.0550055	-1.14148699306321e-14\\
0.05510551	-1.12412099292939e-14\\
0.05520552	-1.10701599279301e-14\\
0.05530553	-1.09016699265402e-14\\
0.05540554	-1.07357199251238e-14\\
0.05550555	-1.05722599236803e-14\\
0.05560556	-1.04112599222093e-14\\
0.05570557	-1.02526799207102e-14\\
0.05580558	-1.00964799191825e-14\\
0.05590559	-9.94264191762562e-15\\
0.0560056	-9.79111091603905e-15\\
0.05610561	-9.64185991442222e-15\\
0.05620562	-9.49486291277456e-15\\
0.05630563	-9.35007091109548e-15\\
0.05640564	-9.20746590938439e-15\\
0.05650565	-9.06700890764068e-15\\
0.05660566	-8.92866990586375e-15\\
0.05670567	-8.79241090405296e-15\\
0.05680568	-8.65821190220767e-15\\
0.05690569	-8.52603490032724e-15\\
0.0570057	-8.395851898411e-15\\
0.05710571	-8.26763089645827e-15\\
0.05720572	-8.14134789446838e-15\\
0.05730573	-8.01697089244061e-15\\
0.05740574	-7.89446889037426e-15\\
0.05750575	-7.77381688826859e-15\\
0.05760576	-7.65498688612287e-15\\
0.05770577	-7.53795088393634e-15\\
0.05780578	-7.42268588170824e-15\\
0.05790579	-7.30915887943778e-15\\
0.0580058	-7.19734687712416e-15\\
0.05810581	-7.08722687476658e-15\\
0.05820582	-6.9787698723642e-15\\
0.05830583	-6.87194986991619e-15\\
0.05840584	-6.76674786742168e-15\\
0.05850585	-6.66313686487979e-15\\
0.05860586	-6.56109086228965e-15\\
0.05870587	-6.46058485965034e-15\\
0.05880588	-6.36160285696094e-15\\
0.05890589	-6.26411485422051e-15\\
0.0590059	-6.16810185142808e-15\\
0.05910591	-6.07354184858269e-15\\
0.05920592	-5.98041084568334e-15\\
0.05930593	-5.888690842729e-15\\
0.05940594	-5.79835483971867e-15\\
0.05950595	-5.70938683665127e-15\\
0.05960596	-5.62176483352574e-15\\
0.05970597	-5.535467830341e-15\\
0.05980598	-5.45047882709592e-15\\
0.05990599	-5.36677082378938e-15\\
0.060006	-5.28433282042023e-15\\
0.06010601	-5.20314081698729e-15\\
0.06020602	-5.12317681348936e-15\\
0.06030603	-5.04442380992524e-15\\
0.06040604	-4.96686280629367e-15\\
0.06050605	-4.8904738025934e-15\\
0.06060606	-4.81524179882313e-15\\
0.06070607	-4.74114579498157e-15\\
0.06080608	-4.66817479106737e-15\\
0.06090609	-4.59630378707917e-15\\
0.0610061	-4.52552278301558e-15\\
0.06110611	-4.45581277887521e-15\\
0.06120612	-4.38715677465661e-15\\
0.06130613	-4.31954077035832e-15\\
0.06140614	-4.25294676597884e-15\\
0.06150615	-4.18736276151667e-15\\
0.06160616	-4.12276975697025e-15\\
0.06170617	-4.05915475233801e-15\\
0.06180618	-3.99650374761835e-15\\
0.06190619	-3.93480074280964e-15\\
0.0620062	-3.8740307379102e-15\\
0.06210621	-3.81418273291835e-15\\
0.06220622	-3.75524072783236e-15\\
0.06230623	-3.69718972265047e-15\\
0.06240624	-3.6400207173709e-15\\
0.06250625	-3.58371771199182e-15\\
0.06260626	-3.52826770651137e-15\\
0.06270627	-3.47365670092767e-15\\
0.06280628	-3.41987569523878e-15\\
0.06290629	-3.36690868944275e-15\\
0.0630063	-3.31474668353758e-15\\
0.06310631	-3.26337467752124e-15\\
0.06320632	-3.21278267139164e-15\\
0.06330633	-3.1629596651467e-15\\
0.06340634	-3.11389165878424e-15\\
0.06350635	-3.0655696523021e-15\\
0.06360636	-3.01798364569803e-15\\
0.06370637	-2.97111763896976e-15\\
0.06380638	-2.924966632115e-15\\
0.06390639	-2.87951662513137e-15\\
0.0640064	-2.83475961801649e-15\\
0.06410641	-2.79068361076791e-15\\
0.06420642	-2.74727860338315e-15\\
0.06430643	-2.70453559585967e-15\\
0.06440644	-2.6624435881949e-15\\
0.06450645	-2.6209965803862e-15\\
0.06460646	-2.5801805724309e-15\\
0.06470647	-2.53998756432629e-15\\
0.06480648	-2.50040855606958e-15\\
0.06490649	-2.46143654765795e-15\\
0.0650065	-2.42306053908852e-15\\
0.06510651	-2.38527153035836e-15\\
0.06520652	-2.3480615214645e-15\\
0.06530653	-2.31142251240388e-15\\
0.06540654	-2.27534550317343e-15\\
0.06550655	-2.23982349376997e-15\\
0.06560656	-2.20484648419032e-15\\
0.06570657	-2.17040747443119e-15\\
0.06580658	-2.13649846448927e-15\\
0.06590659	-2.10311245436115e-15\\
0.0660066	-2.0702414440434e-15\\
0.06610661	-2.0378764335325e-15\\
0.06620662	-2.00601042282487e-15\\
0.06630663	-1.97463841191686e-15\\
0.06640664	-1.94375040080476e-15\\
0.06650665	-1.9133413894848e-15\\
0.06660666	-1.88340237795312e-15\\
0.06670667	-1.85392936620581e-15\\
0.06680668	-1.82491135423888e-15\\
0.06690669	-1.79634534204826e-15\\
0.0670067	-1.76822432962981e-15\\
0.06710671	-1.74053931697933e-15\\
0.06720672	-1.71328430409252e-15\\
0.06730673	-1.68645529096501e-15\\
0.06740674	-1.66004427759235e-15\\
0.06750675	-1.63404426397001e-15\\
0.06760676	-1.60845225009338e-15\\
0.06770677	-1.58325823595776e-15\\
0.06780678	-1.55845922155836e-15\\
0.06790679	-1.53404820689033e-15\\
0.0680068	-1.51001819194868e-15\\
0.06810681	-1.48636617672839e-15\\
0.06820682	-1.46308416122429e-15\\
0.06830683	-1.44016814543116e-15\\
0.06840684	-1.41761112934366e-15\\
0.06850685	-1.39540811295636e-15\\
0.06860686	-1.37355509626374e-15\\
0.06870687	-1.35204407926017e-15\\
0.06880688	-1.3308710619399e-15\\
0.06890689	-1.31003304429711e-15\\
0.0690069	-1.28952202632585e-15\\
0.06910691	-1.26933400802008e-15\\
0.06920692	-1.24946298937362e-15\\
0.06930693	-1.2299049703802e-15\\
0.06940694	-1.21065595103345e-15\\
0.06950695	-1.19170993132684e-15\\
0.06960696	-1.17306091125376e-15\\
0.06970697	-1.15470789080746e-15\\
0.06980698	-1.13664286998108e-15\\
0.06990699	-1.11886084876762e-15\\
0.070007	-1.10136082715997e-15\\
0.07010701	-1.08413680515088e-15\\
0.07020702	-1.06718178273297e-15\\
0.07030703	-1.05049375989871e-15\\
0.07040704	-1.03406973664047e-15\\
0.07050705	-1.01790271295045e-15\\
0.07060706	-1.00198968882072e-15\\
0.07070707	-9.86326264243202e-16\\
0.07080708	-9.70908639209675e-16\\
0.07090709	-9.55733413711766e-16\\
0.0710071	-9.40794987740952e-16\\
0.07110711	-9.26091261288553e-16\\
0.07120712	-9.11616434345733e-16\\
0.07130713	-8.97368306903496e-16\\
0.07140714	-8.8334287895268e-16\\
0.07150715	-8.69535650483961e-16\\
0.07160716	-8.55942821487841e-16\\
0.07170717	-8.42562591954654e-16\\
0.07180718	-8.29390161874557e-16\\
0.07190719	-8.16420931237528e-16\\
0.0720072	-8.03654100033365e-16\\
0.07210721	-7.91083868251678e-16\\
0.07220722	-7.78708035881893e-16\\
0.07230723	-7.66523802913242e-16\\
0.07240724	-7.54527169334761e-16\\
0.07250725	-7.42713835135289e-16\\
0.07260726	-7.31083300303465e-16\\
0.07270727	-7.19631364827717e-16\\
0.07280728	-7.08353628696269e-16\\
0.07290729	-6.97249791897128e-16\\
0.0730073	-6.86314054418087e-16\\
0.07310731	-6.75545916246716e-16\\
0.07320732	-6.64941877370362e-16\\
0.07330733	-6.54498237776142e-16\\
0.07340734	-6.4421319745094e-16\\
0.07350735	-6.34084256381405e-16\\
0.07360736	-6.24108614553943e-16\\
0.07370737	-6.14283371954713e-16\\
0.07380738	-6.04606628569629e-16\\
0.07390739	-5.95075784384345e-16\\
0.0740074	-5.8568823938426e-16\\
0.07410741	-5.76441493554507e-16\\
0.07420742	-5.67333646879952e-16\\
0.07430743	-5.58362699345189e-16\\
0.07440744	-5.49525450934531e-16\\
0.07450745	-5.40821001632011e-16\\
0.07460746	-5.32246151421372e-16\\
0.07470747	-5.23799600286066e-16\\
0.07480748	-5.15479448209246e-16\\
0.07490749	-5.0728289517376e-16\\
0.0750075	-4.99209141162148e-16\\
0.07510751	-4.91255286156637e-16\\
0.07520752	-4.83420130139131e-16\\
0.07530753	-4.75702173091211e-16\\
0.07540754	-4.68098314994124e-16\\
0.07550755	-4.60608355828782e-16\\
0.07560756	-4.5322979557575e-16\\
0.07570757	-4.45961834215248e-16\\
0.07580758	-4.38802271727136e-16\\
0.07590759	-4.31749008090914e-16\\
0.0760076	-4.24801343285713e-16\\
0.07610761	-4.17957977290289e-16\\
0.07620762	-4.11217610083015e-16\\
0.07630763	-4.04577841641877e-16\\
0.07640764	-3.98037971944466e-16\\
0.07650765	-3.91596800967967e-16\\
0.07660766	-3.85252528689159e-16\\
0.07670767	-3.79004355084403e-16\\
0.07680768	-3.72851080129634e-16\\
0.07690769	-3.66791203800358e-16\\
0.0770077	-3.60823526071639e-16\\
0.07710771	-3.54946946918094e-16\\
0.07720772	-3.49160766313887e-16\\
0.07730773	-3.43463484232716e-16\\
0.07740774	-3.37853800647808e-16\\
0.07750775	-3.32330915531911e-16\\
0.07760776	-3.26893728857284e-16\\
0.07770777	-3.21541540595689e-16\\
0.07780778	-3.16273650718382e-16\\
0.07790779	-3.11087059196105e-16\\
0.0780078	-3.05983265999074e-16\\
0.07810781	-3.00959871096974e-16\\
0.07820782	-2.96016874458946e-16\\
0.07830783	-2.91152376053579e-16\\
0.07840784	-2.86366375848901e-16\\
0.07850785	-2.81657273812366e-16\\
0.07860786	-2.77024169910848e-16\\
0.07870787	-2.72466864110627e-16\\
0.07880788	-2.67984256377381e-16\\
0.07890789	-2.63575046676174e-16\\
0.0790079	-2.59238834971447e-16\\
0.07910791	-2.54974221227003e-16\\
0.07920792	-2.50781105406001e-16\\
0.07930793	-2.46657787470941e-16\\
0.07940794	-2.42604567383653e-16\\
0.07950795	-2.38619645105286e-16\\
0.07960796	-2.34702120596295e-16\\
0.07970797	-2.30851993816429e-16\\
0.07980798	-2.27068064724722e-16\\
0.07990799	-2.23348733279472e-16\\
0.080008	-2.19694599438237e-16\\
0.08010801	-2.16103463157817e-16\\
0.08020802	-2.12575824394242e-16\\
0.08030803	-2.09109483102759e-16\\
0.08040804	-2.05704739237816e-16\\
0.08050805	-2.02360492753049e-16\\
0.08060806	-1.99074843601271e-16\\
0.08070807	-1.95848491734451e-16\\
0.08080808	-1.92679837103706e-16\\
0.08090809	-1.89567979659281e-16\\
0.0810081	-1.86512119350536e-16\\
0.08110811	-1.83511356125928e-16\\
0.08120812	-1.80565389933e-16\\
0.08130813	-1.7767252071836e-16\\
0.08140814	-1.74832348427667e-16\\
0.08150815	-1.72044473005613e-16\\
0.08160816	-1.69306894395908e-16\\
0.08170817	-1.66619612541261e-16\\
0.08180818	-1.63981327383362e-16\\
0.08190819	-1.61391538862866e-16\\
0.0820082	-1.58848846919375e-16\\
0.08210821	-1.56352751491416e-16\\
0.08220822	-1.53902752516426e-16\\
0.08230823	-1.51497049930731e-16\\
0.08240824	-1.49135143669527e-16\\
0.08250825	-1.4681623366686e-16\\
0.08260826	-1.44539519855605e-16\\
0.08270827	-1.42304002167449e-16\\
0.08280828	-1.40109080532865e-16\\
0.08290829	-1.37953054881095e-16\\
0.0830083	-1.35836025140124e-16\\
0.08310831	-1.33756391236664e-16\\
0.08320832	-1.31713853096126e-16\\
0.08330833	-1.29707110642601e-16\\
0.08340834	-1.27735563798834e-16\\
0.08350835	-1.25797912486205e-16\\
0.08360836	-1.23893756624699e-16\\
0.08370837	-1.22022596132885e-16\\
0.08380838	-1.20182630927891e-16\\
0.08390839	-1.18373560925381e-16\\
0.0840084	-1.16594886039524e-16\\
0.08410841	-1.14844706182973e-16\\
0.08420842	-1.13123921266835e-16\\
0.08430843	-1.11430231200645e-16\\
0.08440844	-1.09763335892342e-16\\
0.08450845	-1.08122535248235e-16\\
0.08460846	-1.06507229172978e-16\\
0.08470847	-1.04916517569542e-16\\
0.08480848	-1.03349800339184e-16\\
0.08490849	-1.01805877381418e-16\\
0.0850085	-1.00284848593981e-16\\
0.08510851	-9.87850538728077e-17\\
0.08520852	-9.73069931119967e-17\\
0.08530853	-9.58489062037769e-17\\
0.08540854	-9.44109530384765e-17\\
0.08550855	-9.29921835044894e-17\\
0.08560856	-9.1591927488242e-17\\
0.08570857	-9.02099048741582e-17\\
0.08580858	-8.8845415544625e-17\\
0.08590859	-8.74979593799569e-17\\
0.0860086	-8.61668262583596e-17\\
0.08610861	-8.48515760558938e-17\\
0.08620862	-8.35523086464374e-17\\
0.08630863	-8.22677839016475e-17\\
0.08640864	-8.09976016909222e-17\\
0.08650865	-7.97420618813608e-17\\
0.08660866	-7.85003243377243e-17\\
0.08670867	-7.72716789223944e-17\\
0.08680868	-7.60564454953321e-17\\
0.08690869	-7.4854033914036e-17\\
0.0870087	-7.36641340334991e-17\\
0.08710871	-7.24864957061657e-17\\
0.08720872	-7.13206687818867e-17\\
0.08730873	-7.01666931078747e-17\\
0.08740874	-6.90240485286586e-17\\
0.08750875	-6.78929648860362e-17\\
0.08760876	-6.67728120190277e-17\\
0.08770877	-6.56639197638269e-17\\
0.08780878	-6.45656379537524e-17\\
0.08790879	-6.34780064191976e-17\\
0.0880088	-6.24012049875803e-17\\
0.08810881	-6.13348534832907e-17\\
0.08820882	-6.02791017276393e-17\\
0.08830883	-5.92338695388031e-17\\
0.08840884	-5.81990067317719e-17\\
0.08850885	-5.71745831182925e-17\\
0.08860886	-5.61605785068128e-17\\
0.08870887	-5.51570727024251e-17\\
0.08880888	-5.41641055068072e-17\\
0.08890889	-5.31816267181638e-17\\
0.0890089	-5.22097161311663e-17\\
0.08910891	-5.1248553536892e-17\\
0.08920892	-5.02982387227614e-17\\
0.08930893	-4.93588114724751e-17\\
0.08940894	-4.84300715659498e-17\\
0.08950895	-4.75128087792528e-17\\
0.08960896	-4.66066728845352e-17\\
0.08970897	-4.57118136499645e-17\\
0.08980898	-4.48284508396557e-17\\
0.08990899	-4.39568542136013e-17\\
0.090009	-4.30968535276004e-17\\
0.09010901	-4.22489685331857e-17\\
0.09020902	-4.14131489775507e-17\\
0.09030903	-4.05895646034739e-17\\
0.09040904	-3.97784351492434e-17\\
0.09050905	-3.89799403485791e-17\\
0.09060906	-3.81939699305541e-17\\
0.09070907	-3.74209436195143e-17\\
0.09080908	-3.66611111349972e-17\\
0.09090909	-3.59143221916489e-17\\
0.0910091	-3.51810564991399e-17\\
0.09110911	-3.44610837620795e-17\\
0.09120912	-3.37546736799282e-17\\
0.09130913	-3.30622359469097e-17\\
0.09140914	-3.23834802519201e-17\\
0.09150915	-3.17186862784367e-17\\
0.09160916	-3.10680937044245e-17\\
0.09170917	-3.04316222022415e-17\\
0.09180918	-2.98093814385421e-17\\
0.09190919	-2.92015510741793e-17\\
0.0920092	-2.86080207641046e-17\\
0.09210921	-2.80290501572669e-17\\
0.09220922	-2.74644088965091e-17\\
0.09230923	-2.69143666184634e-17\\
0.09240924	-2.63788529534444e-17\\
0.09250925	-2.58578075253408e-17\\
0.09260926	-2.53513499515048e-17\\
0.09270927	-2.48593098426404e-17\\
0.09280928	-2.43817068026886e-17\\
0.09290929	-2.39184404287119e-17\\
0.0930093	-2.34694803107762e-17\\
0.09310931	-2.30348160318304e-17\\
0.09320932	-2.26139871675851e-17\\
0.09330933	-2.22074332863877e-17\\
0.09340934	-2.18144439490968e-17\\
0.09350935	-2.14353387089534e-17\\
0.09360936	-2.10697871114509e-17\\
0.09370937	-2.07173086942018e-17\\
0.09380938	-2.03781329868032e-17\\
0.09390939	-2.00518295106993e-17\\
0.0940094	-1.9738267779042e-17\\
0.09410941	-1.94372172965488e-17\\
0.09420942	-1.91481675593586e-17\\
0.09430943	-1.8871148054885e-17\\
0.09440944	-1.86056982616669e-17\\
0.09450945	-1.83516076492166e-17\\
0.09460946	-1.81085356778661e-17\\
0.09470947	-1.78762917986093e-17\\
0.09480948	-1.76542254529433e-17\\
0.09490949	-1.74421960727055e-17\\
0.0950095	-1.72399030799089e-17\\
0.09510951	-1.7046895886574e-17\\
0.09520952	-1.68627838945588e-17\\
0.09530953	-1.66871964953846e-17\\
0.09540954	-1.65197630700601e-17\\
0.09550955	-1.63601129889017e-17\\
0.09560956	-1.62079456113513e-17\\
0.09570957	-1.60624602857907e-17\\
0.09580958	-1.59236363493531e-17\\
0.09590959	-1.5790873127731e-17\\
0.0960096	-1.56637699349818e-17\\
0.09610961	-1.5542076073329e-17\\
0.09620962	-1.54249808329606e-17\\
0.09630963	-1.53125134918245e-17\\
0.09640964	-1.52039333154195e-17\\
0.09650965	-1.50989695565838e-17\\
0.09660966	-1.49971314552791e-17\\
0.09670967	-1.48980582383717e-17\\
0.09680968	-1.48012791194097e-17\\
0.09690969	-1.47063532983962e-17\\
0.0970097	-1.46130099615589e-17\\
0.09710971	-1.45207082811162e-17\\
0.09720972	-1.44292474150388e-17\\
0.09730973	-1.43379765068074e-17\\
0.09740974	-1.42466446851667e-17\\
0.09750975	-1.41550510638749e-17\\
0.09760976	-1.40626747414491e-17\\
0.09770977	-1.39691648009065e-17\\
0.09780978	-1.38742903095015e-17\\
0.09790979	-1.37776303184575e-17\\
0.0980098	-1.36789138626955e-17\\
0.09810981	-1.3577869960557e-17\\
0.09820982	-1.34741276135228e-17\\
0.09830983	-1.33676658059274e-17\\
0.09840984	-1.32580235046679e-17\\
0.09850985	-1.31450296589083e-17\\
0.09860986	-1.30284531997792e-17\\
0.09870987	-1.29080830400721e-17\\
0.09880988	-1.27838280739287e-17\\
0.09890989	-1.26555071765249e-17\\
0.0990099	-1.25228392037497e-17\\
0.09910991	-1.23858529918787e-17\\
0.09920992	-1.2244407357242e-17\\
0.09930993	-1.20983310958866e-17\\
0.09940994	-1.19476229832333e-17\\
0.09950995	-1.17922717737278e-17\\
0.09960996	-1.16321462004858e-17\\
0.09970997	-1.14671849749326e-17\\
0.09980998	-1.12975967864359e-17\\
0.09990999	-1.11232603019336e-17\\
0.10001	-1.09441741655542e-17\\
0.10011	-1.0760517472999e-17\\
0.10021	-1.0572268363475e-17\\
0.10031	-1.0379635410777e-17\\
0.10041	-1.01825171643204e-17\\
0.1005101	-9.98131705978667e-18\\
0.1006101	-9.77587368544598e-18\\
0.1007101	-9.56660451362578e-18\\
0.1008101	-9.35354299201234e-18\\
0.1009101	-9.13700854178471e-18\\
0.1010101	-8.91697555716782e-18\\
0.1011101	-8.69394640497809e-18\\
0.1012101	-8.46793042416159e-18\\
0.1013101	-8.23919692532442e-18\\
0.1014101	-8.00812519025533e-18\\
0.1015102	-7.77491341958738e-18\\
0.1016102	-7.53981883405588e-18\\
0.1017102	-7.3030996780379e-18\\
0.1018102	-7.06521111308709e-18\\
0.1019102	-6.82627326943062e-18\\
0.1020102	-6.58664524544215e-18\\
0.1021102	-6.34674810710604e-18\\
0.1022102	-6.10671088747275e-18\\
0.1023102	-5.86706958610512e-18\\
0.1024102	-5.62805616851552e-18\\
0.1025103	-5.3900173711374e-18\\
0.1026103	-5.15323235321368e-18\\
0.1027103	-4.91810790337424e-18\\
0.1028103	-4.68502184507505e-18\\
0.1029103	-4.45420896476347e-18\\
0.1030103	-4.22606901125704e-18\\
0.1031103	-4.00088969511192e-18\\
0.1032103	-3.77914168798096e-18\\
0.1033103	-3.56093662196098e-18\\
0.1034103	-3.34673708892945e-18\\
0.1035104	-3.1367900900626e-18\\
0.1036104	-2.9314050857125e-18\\
0.1037104	-2.73097413931275e-18\\
0.1038104	-2.53567867496344e-18\\
0.1039104	-2.3458610730592e-18\\
0.1040104	-2.16177766955709e-18\\
0.1041104	-1.98371075523244e-18\\
0.1042104	-1.81180457492244e-18\\
0.1043104	-1.6464983267572e-18\\
0.1044104	-1.4878621613781e-18\\
0.1045105	-1.3361170241901e-18\\
0.1046105	-1.19151510611369e-18\\
0.1047105	-1.05417842678212e-18\\
0.1048105	-9.24357938753036e-19\\
0.1049105	-8.02145542988502e-19\\
0.1050105	-6.87664087992671e-19\\
0.1051105	-5.8110336893525e-19\\
0.1052105	-4.82503126760543e-19\\
0.1053105	-3.91969047281775e-19\\
0.1054105	-3.09585660260509e-19\\
0.1055106	-2.35358876471337e-19\\
0.1056106	-1.69376725865179e-19\\
0.1057106	-1.11638607682641e-19\\
0.1058106	-6.21247179054101e-20\\
0.1059106	-2.08319916271377e-20\\
0.1060106	1.22761979619634e-20\\
0.1061106	3.72734406071537e-20\\
0.1062106	5.41994900382532e-20\\
0.1063106	6.31938650359491e-20\\
0.1064106	6.43222505155578e-20\\
0.1065107	5.77402179986077e-20\\
0.1066107	4.35519967216886e-20\\
0.1067107	2.19594528132354e-20\\
0.1068107	-6.88885300249115e-21\\
0.1069107	-4.27999881731249e-20\\
0.1070107	-8.55320471231212e-20\\
0.1071107	-1.34918150038817e-19\\
0.1072107	-1.90692441633853e-19\\
0.1073107	-2.5262448989413e-19\\
0.1074107	-3.2040758480228e-19\\
0.1075108	-3.93765943760228e-19\\
0.1076108	-4.72483390317955e-19\\
0.1077108	-5.56170267386978e-19\\
0.1078108	-6.44558539673691e-19\\
0.1079108	-7.37364087181433e-19\\
0.1080108	-8.34181703804105e-19\\
0.1081108	-9.3473209589701e-19\\
0.1082108	-1.03868288082399e-18\\
0.1083108	-1.14559058548102e-18\\
0.1084108	-1.25516864479554e-18\\
0.1085109	-1.36706324072281e-18\\
0.1086109	-1.48081859164014e-18\\
0.1087109	-1.59619602890972e-18\\
0.1088109	-1.71273660398219e-18\\
0.1089109	-1.83012226845842e-18\\
0.1090109	-1.94787387243545e-18\\
0.1091109	-2.06568916282537e-18\\
0.1092109	-2.18323078164674e-18\\
0.1093109	-2.30007226428814e-18\\
0.1094109	-2.4158480377434e-18\\
0.109511	-2.53019551154349e-18\\
0.109611	-2.64273129440473e-18\\
0.109711	-2.75322397364254e-18\\
0.109811	-2.86117252432341e-18\\
0.109911	-2.96637880385143e-18\\
0.110011	-3.06831655002349e-18\\
0.110111	-3.16687537905277e-18\\
0.110211	-3.26163578356002e-18\\
0.110311	-3.35222513053203e-18\\
0.110411	-3.43850565924697e-18\\
0.1105111	-3.52002413722738e-18\\
0.1106111	-3.59669974005983e-18\\
0.1107111	-3.66819544673577e-18\\
0.1108111	-3.73428196412991e-18\\
0.1109111	-3.79472586051457e-18\\
0.1110111	-3.84930456327377e-18\\
0.1111111	-3.89783835658033e-18\\
0.1112111	-3.94012737903498e-18\\
0.1113111	-3.97612762126723e-18\\
0.1114111	-4.00543492349724e-18\\
0.1115112	-4.02813738555339e-18\\
0.1116112	-4.04413313986066e-18\\
0.1117112	-4.05325953768422e-18\\
0.1118112	-4.05536279226124e-18\\
0.1119112	-4.0505829536189e-18\\
0.1120112	-4.03867890588882e-18\\
0.1121112	-4.01971736457768e-18\\
0.1122112	-3.99384887379382e-18\\
0.1123112	-3.96081880342873e-18\\
0.1124112	-3.92083834629283e-18\\
0.1125113	-3.87383972514201e-18\\
0.1126113	-3.81998967077456e-18\\
0.1127113	-3.75932070478794e-18\\
0.1128113	-3.6921002818959e-18\\
0.1129113	-3.61818666469529e-18\\
0.1130113	-3.5378899205121e-18\\
0.1131113	-3.45130991819584e-18\\
0.1132113	-3.35867732486278e-18\\
0.1133113	-3.26017060258559e-18\\
0.1134113	-3.15594300503025e-18\\
0.1135114	-3.04615545241923e-18\\
0.1136114	-2.93120521186042e-18\\
0.1137114	-2.81125355862147e-18\\
0.1138114	-2.68659087778098e-18\\
0.1139114	-2.55741932835633e-18\\
0.1140114	-2.42406383960016e-18\\
0.1141114	-2.28685210723757e-18\\
0.1142114	-2.14608858964218e-18\\
0.1143114	-2.00201250395061e-18\\
0.1144114	-1.85496982211422e-18\\
0.1145115	-1.70530148226947e-18\\
0.1146115	-1.55338557486927e-18\\
0.1147115	-1.39953445970405e-18\\
0.1148115	-1.24414608694798e-18\\
0.1149115	-1.08746014086828e-18\\
0.1150115	-9.2988403547857e-19\\
0.1151115	-7.71836910123057e-19\\
0.1152115	-6.13628624989124e-19\\
0.1153115	-4.5562575654762e-19\\
0.1154115	-2.98169592919676e-19\\
0.1155116	-1.41589111924421e-19\\
0.1156116	1.36360745889917e-20\\
0.1157116	1.67217488237855e-19\\
0.1158116	3.18816042449795e-19\\
0.1159116	4.68088963315199e-19\\
0.1160116	6.14670794686274e-19\\
0.1161116	7.58273403357583e-19\\
0.1162116	8.98396984330388e-19\\
0.1163116	1.03493306616112e-18\\
0.1164116	1.16754351639621e-18\\
0.1165117	1.29589764672835e-18\\
0.1166117	1.41962914212226e-18\\
0.1167117	1.53854372026138e-18\\
0.1168117	1.65223266112962e-18\\
0.1169117	1.76063461221711e-18\\
0.1170117	1.86339159450022e-18\\
0.1171117	1.96033900851689e-18\\
0.1172117	2.05117164054006e-18\\
0.1173117	2.13575666884905e-18\\
0.1174117	2.2138936701027e-18\\
0.1175118	2.28549298694996e-18\\
0.1176118	2.3502378502588e-18\\
0.1177118	2.40807589805369e-18\\
0.1178118	2.45883136673081e-18\\
0.1179118	2.50254692450002e-18\\
0.1180118	2.53895767839432e-18\\
0.1181118	2.56815018139287e-18\\
0.1182118	2.58993343965646e-18\\
0.1183118	2.60445091987994e-18\\
0.1184118	2.61147455676176e-18\\
0.1185119	2.61128294941658e-18\\
0.1186119	2.60362645437566e-18\\
0.1187119	2.58864283524041e-18\\
0.1188119	2.5664299763387e-18\\
0.1189119	2.53697026918708e-18\\
0.1190119	2.50043162070604e-18\\
0.1191119	2.45690046156612e-18\\
0.1192119	2.40655675466745e-18\\
0.1193119	2.34944400375426e-18\\
0.1194119	2.28574526216694e-18\\
0.119512	2.21592017917969e-18\\
0.119612	2.13967202985202e-18\\
0.119712	2.0574834731274e-18\\
0.119812	1.96949785166975e-18\\
0.119912	1.87603410364809e-18\\
0.120012	1.77722977236052e-18\\
0.120112	1.67343301601078e-18\\
0.120212	1.56491961764079e-18\\
0.120312	1.45186299522134e-18\\
0.120412	1.33467421190339e-18\\
0.1205121	1.21377842749378e-18\\
0.1206121	1.08911070367091e-18\\
0.1207121	9.61313026862856e-19\\
0.1208121	8.30544147448494e-19\\
0.1209121	6.97276514753234e-19\\
0.1210121	5.61645288318849e-19\\
0.1211121	4.2420234935053e-19\\
0.1212121	2.85211312349107e-19\\
0.1213121	1.45171536924786e-19\\
0.1214121	4.19713980056016e-21\\
0.1215122	-1.37064966343209e-19\\
0.1216122	-2.78573151533885e-19\\
0.1217122	-4.19750892812992e-19\\
0.1218122	-5.60427920310431e-19\\
0.1219122	-7.00123144098045e-19\\
0.1220122	-8.38494640998197e-19\\
0.1221122	-9.75079641182275e-19\\
0.1222122	-1.10971451455729e-18\\
0.1223122	-1.24198475693736e-18\\
0.1224122	-1.37153497599425e-18\\
0.1225123	-1.49798634697481e-18\\
0.1226123	-1.62128316578243e-18\\
0.1227123	-1.74081825297891e-18\\
0.1228123	-1.85647251773107e-18\\
0.1229123	-1.96794590740007e-18\\
0.1230123	-2.07477739210465e-18\\
0.1231123	-2.17697494904043e-18\\
0.1232123	-2.27406554655105e-18\\
0.1233123	-2.36585512794762e-18\\
0.1234123	-2.45223859507261e-18\\
0.1235124	-2.53257897727073e-18\\
0.1236124	-2.6073364882476e-18\\
0.1237124	-2.67578610450971e-18\\
0.1238124	-2.73809839128872e-18\\
0.1239124	-2.79396278617048e-18\\
0.1240124	-2.84328758103948e-18\\
0.1241124	-2.88585990373654e-18\\
0.1242124	-2.92159569942795e-18\\
0.1243124	-2.95041971168118e-18\\
0.1244124	-2.97250546324105e-18\\
0.1245125	-2.9871693446262e-18\\
0.1246125	-2.99510923689757e-18\\
0.1247125	-2.99589383406725e-18\\
0.1248125	-2.98966157591742e-18\\
0.1249125	-2.97647958063173e-18\\
0.1250125	-2.95621362367992e-18\\
0.1251125	-2.92907811637344e-18\\
0.1252125	-2.89491608408161e-18\\
0.1253125	-2.85402914410557e-18\\
0.1254125	-2.80639748320254e-18\\
0.1255126	-2.75192088545467e-18\\
0.1256126	-2.69134871065796e-18\\
0.1257126	-2.62435346687407e-18\\
0.1258126	-2.55121938526151e-18\\
0.1259126	-2.47241914862759e-18\\
0.1260126	-2.38762386675081e-18\\
0.1261126	-2.29749305131312e-18\\
0.1262126	-2.2016945904415e-18\\
0.1263126	-2.10115472284648e-18\\
0.1264126	-1.99571801155824e-18\\
0.1265127	-1.88534594003697e-18\\
0.1266127	-1.77092557599793e-18\\
0.1267127	-1.65272162288149e-18\\
0.1268127	-1.53064166820319e-18\\
0.1269127	-1.40524148613142e-18\\
0.1270127	-1.27660500865269e-18\\
0.1271127	-1.14533429628334e-18\\
0.1272127	-1.011589508324e-18\\
0.1273127	-8.75668872647439e-19\\
0.1274127	-7.3795865501407e-19\\
0.1275128	-5.98318323333351e-19\\
0.1276128	-4.58151716615138e-19\\
0.1277128	-3.17072107595972e-19\\
0.1278128	-1.7561759491242e-19\\
0.1279128	-3.41141542785455e-20\\
0.1280128	1.0723439519759e-19\\
0.1281128	2.47666425217907e-19\\
0.1282128	3.87642533036833e-19\\
0.1283128	5.25975576743403e-19\\
0.1284128	6.62640711097302e-19\\
0.1285129	7.98146562853221e-19\\
0.1286129	9.30480137242474e-19\\
0.1287129	1.06040556685439e-18\\
0.1288129	1.18732756526208e-18\\
0.1289129	1.31113333048778e-18\\
0.1290129	1.43121258432683e-18\\
0.1291129	1.54754761228551e-18\\
0.1292129	1.65986330413965e-18\\
0.1293129	1.76770719512777e-18\\
0.1294129	1.87100950778148e-18\\
0.129513	1.97005580762292e-18\\
0.129613	2.0632776694594e-18\\
0.129713	2.15120535246528e-18\\
0.129813	2.23357026275288e-18\\
0.129913	2.31014671291799e-18\\
0.130013	2.38085196793746e-18\\
0.130113	2.44548629177781e-18\\
0.130213	2.50394299473075e-18\\
0.130313	2.55585848148513e-18\\
0.130413	2.60158229994354e-18\\
0.1305131	2.64128903107003e-18\\
0.1306131	2.67361398940518e-18\\
0.1307131	2.6992854930142e-18\\
0.1308131	2.71807616955901e-18\\
0.1309131	2.73000204567214e-18\\
0.1310131	2.73513260050489e-18\\
0.1311131	2.73349082010698e-18\\
0.1312131	2.72499325264726e-18\\
0.1313131	2.70968006449425e-18\\
0.1314131	2.68758509716185e-18\\
0.1315132	2.65979780158148e-18\\
0.1316132	2.62434107031395e-18\\
0.1317132	2.58267696525619e-18\\
0.1318132	2.53432461978927e-18\\
0.1319132	2.47992714075652e-18\\
0.1320132	2.41950167091287e-18\\
0.1321132	2.35265945234771e-18\\
0.1322132	2.28060589088087e-18\\
0.1323132	2.20260062145818e-18\\
0.1324132	2.11947757455917e-18\\
0.1325133	2.03206133912599e-18\\
0.1326133	1.93871998530853e-18\\
0.1327133	1.8405693877343e-18\\
0.1328133	1.73801134847192e-18\\
0.1329133	1.63148231295387e-18\\
0.1330133	1.52080344278289e-18\\
0.1331133	1.40649068966487e-18\\
0.1332133	1.28894487048298e-18\\
0.1333133	1.16813174353025e-18\\
0.1334133	1.04477208592073e-18\\
0.1335134	9.20211046902469e-19\\
0.1336134	7.92302176656073e-19\\
0.1337134	6.62550351088097e-19\\
0.1338134	5.31067226214802e-19\\
0.1339134	3.98619882279982e-19\\
0.1340134	2.65380908608877e-19\\
0.1341134	1.3164848976052e-19\\
0.1342134	-2.60350697962647e-21\\
0.1343134	-1.35975442773511e-19\\
0.1344134	-2.69131817021397e-19\\
0.1345135	-4.00195480728698e-19\\
0.1346135	-5.31575633219357e-19\\
0.1347135	-6.61227232140032e-19\\
0.1348135	-7.89320393507691e-19\\
0.1349135	-9.15148899235408e-19\\
0.1350135	-1.03870009828668e-18\\
0.1351135	-1.15937480630559e-18\\
0.1352135	-1.27729720369986e-18\\
0.1353135	-1.39173473217108e-18\\
0.1354135	-1.50224798964328e-18\\
0.1355136	-1.60770936600259e-18\\
0.1356136	-1.71059901176862e-18\\
0.1357136	-1.80920357845539e-18\\
0.1358136	-1.9031551982678e-18\\
0.1359136	-1.99237860950821e-18\\
0.1360136	-2.0765910414568e-18\\
0.1361136	-2.1554020974951e-18\\
0.1362136	-2.22911363643031e-18\\
0.1363136	-2.29711965199942e-18\\
0.1364136	-2.35930615054774e-18\\
0.1365137	-2.41355840794764e-18\\
0.1366137	-2.46429737873762e-18\\
0.1367137	-2.50842513255408e-18\\
0.1368137	-2.54639445632056e-18\\
0.1369137	-2.57814950907396e-18\\
0.1370137	-2.60372568796101e-18\\
0.1371137	-2.62244949217475e-18\\
0.1372137	-2.63513838481256e-18\\
0.1373137	-2.64120065263617e-18\\
0.1374137	-2.64103526366955e-18\\
0.1375138	-2.6315741744006e-18\\
0.1376138	-2.61807260408996e-18\\
0.1377138	-2.5986822771217e-18\\
0.1378138	-2.57236340868084e-18\\
0.1379138	-2.54026614981673e-18\\
0.1380138	-2.5017304314938e-18\\
0.1381138	-2.45728580627041e-18\\
0.1382138	-2.40715128756451e-18\\
0.1383138	-2.3507351864576e-18\\
0.1384138	-2.28913494601814e-18\\
0.1385139	-2.21875186251368e-18\\
0.1386139	-2.14618476796526e-18\\
0.1387139	-2.06855821847646e-18\\
0.1388139	-1.98582446562236e-18\\
0.1389139	-1.89882402578467e-18\\
0.1390139	-1.80678549874304e-18\\
0.1391139	-1.71082538351043e-18\\
0.1392139	-1.61084789136162e-18\\
0.1393139	-1.5069447560426e-18\\
0.1394139	-1.40009504105623e-18\\
0.139514	-1.28617906550829e-18\\
0.139614	-1.1730671641631e-18\\
0.139714	-1.057167014282e-18\\
0.139814	-9.38604998679251e-19\\
0.139914	-8.18293821428932e-19\\
0.140014	-6.96132296943404e-19\\
0.140114	-5.72805135803455e-19\\
0.140214	-4.48382727388991e-19\\
0.140314	-3.22820919188525e-19\\
0.140414	-1.96860792744671e-19\\
0.1405141	-6.66562576613478e-20\\
0.1406141	5.93013265315602e-20\\
0.1407141	1.84625883949401e-19\\
0.1408141	3.0973772164876e-19\\
0.1409141	4.33473085030099e-19\\
0.1410141	5.55984403020622e-19\\
0.1411141	6.76840536977342e-19\\
0.1412141	7.95827033379205e-19\\
0.1413141	9.12346380309488e-19\\
0.1414141	1.02581826784546e-18\\
0.1415142	1.14233822934715e-18\\
0.1416142	1.25011913045825e-18\\
0.1417142	1.35501869955057e-18\\
0.1418142	1.45622813651477e-18\\
0.1419142	1.55375720566846e-18\\
0.1420142	1.6475345206495e-18\\
0.1421142	1.73670783364059e-18\\
0.1422142	1.82084432894679e-18\\
0.1423142	1.90073092098151e-18\\
0.1424142	1.97557455679999e-18\\
0.1425143	2.05136373422047e-18\\
0.1426143	2.11571138313291e-18\\
0.1427143	2.1753615861757e-18\\
0.1428143	2.22940455971572e-18\\
0.1429143	2.27725213541675e-18\\
0.1430143	2.32053809119081e-18\\
0.1431143	2.3570184870907e-18\\
0.1432143	2.3879720062987e-18\\
0.1433143	2.41260030120652e-18\\
0.1434143	2.43152834477043e-18\\
0.1435144	2.45070496064002e-18\\
0.1436144	2.45730470723088e-18\\
0.1437144	2.45772424693372e-18\\
0.1438144	2.45278547990256e-18\\
0.1439144	2.44123546469451e-18\\
0.1440144	2.42374680247692e-18\\
0.1441144	2.40011802707956e-18\\
0.1442144	2.3706740008169e-18\\
0.1443144	2.33536631636569e-18\\
0.1444144	2.29477370455985e-18\\
0.1445145	2.2558004061636e-18\\
0.1446145	2.20420424409085e-18\\
0.1447145	2.14722822417988e-18\\
0.1448145	2.08446386138579e-18\\
0.1449145	2.01743194217688e-18\\
0.1450145	1.94558297054988e-18\\
0.1451145	1.86819762069207e-18\\
0.1452145	1.78748719636875e-18\\
0.1453145	1.70199409730335e-18\\
0.1454145	1.6125922924695e-18\\
0.1455146	1.52816877400867e-18\\
0.1456146	1.43103976787602e-18\\
0.1457146	1.33132040663829e-18\\
0.1458146	1.22761584642641e-18\\
0.1459146	1.12256528808652e-18\\
0.1460146	1.01404249461519e-18\\
0.1461146	9.02856316319983e-19\\
0.1462146	7.90451223856042e-19\\
0.1463146	6.75807849181893e-19\\
0.1464146	5.60543534661832e-19\\
0.1465147	4.53792835022763e-19\\
0.1466147	3.36051390416593e-19\\
0.1467147	2.18269459998823e-19\\
0.1468147	9.96169234629909e-20\\
0.1469147	-1.95967577574431e-20\\
0.1470147	-1.36921940355351e-19\\
0.1471147	-2.54068190051059e-19\\
0.1472147	-3.71603663483999e-19\\
0.1473147	-4.8605448083521e-19\\
0.1474147	-6.00804089082495e-19\\
0.1475148	-7.01262015111618e-19\\
0.1476148	-8.11395179663616e-19\\
0.1477148	-9.19111955717318e-19\\
0.1478148	-1.02461849295798e-18\\
0.1479148	-1.12677493853528e-18\\
0.1480148	-1.22689474130203e-18\\
0.1481148	-1.32364394571296e-18\\
0.1482148	-1.41564047520668e-18\\
0.1483148	-1.50575340494728e-18\\
0.1484148	-1.59110222371812e-18\\
0.1485149	-1.65848160800656e-18\\
0.1486149	-1.73503628266082e-18\\
0.1487149	-1.80787678829749e-18\\
0.1488149	-1.8755166126971e-18\\
0.1489149	-1.93871579831128e-18\\
0.1490149	-1.99648013597011e-18\\
0.1491149	-2.05016034652498e-18\\
0.1492149	-2.09845125042703e-18\\
0.1493149	-2.14109092497419e-18\\
0.1494149	-2.1788598490049e-18\\
0.149515	-2.19472019467813e-18\\
0.149615	-2.22039493122631e-18\\
0.149715	-2.24168199138243e-18\\
0.149815	-2.25652258972346e-18\\
0.149915	-2.26579587134943e-18\\
0.150015	-2.27131797765064e-18\\
0.150115	-2.26984109829846e-18\\
0.150215	-2.26305250933972e-18\\
0.150315	-2.25157359684926e-18\\
0.150415	-2.23295886631113e-18\\
0.1505151	-2.19145217701281e-18\\
0.1506151	-2.16265421476523e-18\\
0.1507151	-2.1289678520194e-18\\
0.1508151	-2.09066979736833e-18\\
0.1509151	-2.04596470042301e-18\\
0.1510151	-1.99698406986356e-18\\
0.1511151	-1.94378517573661e-18\\
0.1512151	-1.88634993536506e-18\\
0.1513151	-1.82358378298232e-18\\
0.1514151	-1.75831452260067e-18\\
0.1515152	-1.66480281830501e-18\\
0.1516152	-1.58934161562955e-18\\
0.1517152	-1.51137775623785e-18\\
0.1518152	-1.43041662717908e-18\\
0.1519152	-1.34487998983064e-18\\
0.1520152	-1.25710472746638e-18\\
0.1521152	-1.16534157450861e-18\\
0.1522152	-1.07375382699435e-18\\
0.1523152	-9.7741603409443e-19\\
0.1524152	-8.80312670331704e-19\\
0.1525153	-7.54064613386686e-19\\
0.1526153	-6.53605270295954e-19\\
0.1527153	-5.5077341860529e-19\\
0.1528153	-4.47179465778023e-19\\
0.1529153	-3.42336805808611e-19\\
0.1530153	-2.3866037006138e-19\\
0.1531153	-1.33465156593259e-19\\
0.1532153	-2.9964738161021e-20\\
0.1533153	7.57302519051513e-20\\
0.1534153	1.79613655967983e-19\\
0.1535154	3.11467297764001e-19\\
0.1536154	4.13613915620046e-19\\
0.1537154	5.1458040280899e-19\\
0.1538154	6.13794434132255e-19\\
0.1539154	7.1079618769263e-19\\
0.1540154	8.05240021175567e-19\\
0.1541154	8.98896173071543e-19\\
0.1542154	9.88652488567323e-19\\
0.1543154	1.07651617067298e-18\\
0.1544154	1.15961555715577e-18\\
0.1545155	1.27602131172878e-18\\
0.1546155	1.35402879718037e-18\\
0.1547155	1.42841053611026e-18\\
0.1548155	1.49979782869799e-18\\
0.1549155	1.56495239369889e-18\\
0.1550155	1.62776830683865e-18\\
0.1551155	1.68627396771685e-18\\
0.1552155	1.73763409567977e-18\\
0.1553155	1.78715175430369e-18\\
0.1554155	1.83127040596364e-18\\
0.1555156	1.91237844891371e-18\\
0.1556156	1.94825135703628e-18\\
0.1557156	1.97692898487904e-18\\
0.1558156	2.0024376892489e-18\\
0.1559156	2.02195495818014e-18\\
0.1560156	2.03681165164963e-18\\
0.1561156	2.04549427506732e-18\\
0.1562156	2.04964728604847e-18\\
0.1563156	2.04907543443859e-18\\
0.1564156	2.04174613676909e-18\\
0.1565157	2.07955974625649e-18\\
0.1566157	2.0630369391496e-18\\
0.1567157	2.04067075567807e-18\\
0.1568157	2.01510389440284e-18\\
0.1569157	1.98415412300413e-18\\
0.1570157	1.94881686763271e-18\\
0.1571157	1.90626783971225e-18\\
0.1572157	1.85986570112934e-18\\
0.1573157	1.81115476793577e-18\\
0.1574157	1.75486775344522e-18\\
0.1575158	1.75180257885587e-18\\
0.1576158	1.68920918224507e-18\\
0.1577158	1.6224096735999e-18\\
0.1578158	1.55091879222795e-18\\
0.1579158	1.47745400186518e-18\\
0.1580158	1.39893848162082e-18\\
0.1581158	1.31750415956519e-18\\
0.1582158	1.2314947914764e-18\\
0.1583158	1.14346908296815e-18\\
0.1584158	1.0522038578208e-18\\
0.1585159	1.02800170740927e-18\\
0.1586159	9.30500183378316e-19\\
0.1587159	8.36446258417815e-19\\
0.1588159	7.37522067556057e-19\\
0.1589159	6.35644408131805e-19\\
0.1590159	5.33968192783668e-19\\
0.1591159	4.30889953075917e-19\\
0.1592159	3.28051392573891e-19\\
0.1593159	2.22342992572444e-19\\
0.1594159	1.18907668836241e-19\\
0.159516	8.8416010006086e-20\\
0.159616	-1.58258139951088e-20\\
0.159716	-1.19795781188175e-19\\
0.159816	-2.23303327743493e-19\\
0.159916	-3.26886357736426e-19\\
0.160016	-4.27807257458295e-19\\
0.160116	-5.30048852248425e-19\\
0.160216	-6.29310305324963e-19\\
0.160316	-7.2700295699941e-19\\
0.160416	-8.2324610442271e-19\\
0.1605161	-8.27841536232509e-19\\
0.1606161	-9.19970470217137e-19\\
0.1607161	-1.00819147266203e-18\\
0.1608161	-1.09341192201735e-18\\
0.1609161	-1.17722511708062e-18\\
0.1610161	-1.25890567879878e-18\\
0.1611161	-1.33440488536545e-18\\
0.1612161	-1.40834594057667e-18\\
0.1613161	-1.4790191742389e-18\\
0.1614161	-1.54737717342369e-18\\
0.1615162	-1.50519196283855e-18\\
0.1616162	-1.56479407795166e-18\\
0.1617162	-1.61883825267904e-18\\
0.1618162	-1.67087564161975e-18\\
0.1619162	-1.71609423890332e-18\\
0.1620162	-1.75931357485294e-18\\
0.1621162	-1.79897933534952e-18\\
0.1622162	-1.83415790617019e-18\\
0.1623162	-1.86353083732906e-18\\
0.1624162	-1.89138922883982e-18\\
0.1625163	-1.79157740336423e-18\\
0.1626163	-1.80582263100405e-18\\
0.1627163	-1.81799844158089e-18\\
0.1628163	-1.82477594928566e-18\\
0.1629163	-1.82840651042859e-18\\
0.1630163	-1.82871560983021e-18\\
0.1631163	-1.82309665848453e-18\\
0.1632163	-1.81550470270724e-18\\
0.1633163	-1.80145004431573e-18\\
0.1634163	-1.78499176789141e-18\\
0.1635164	-1.6246913579114e-18\\
0.1636164	-1.59559754841582e-18\\
0.1637164	-1.5684713895012e-18\\
0.1638164	-1.53650002999669e-18\\
0.1639164	-1.50138562269393e-18\\
0.1640164	-1.45933827716639e-18\\
0.1641164	-1.41806891292629e-18\\
0.1642164	-1.3697820098893e-18\\
0.1643164	-1.32316825646279e-18\\
0.1644164	-1.27239709106563e-18\\
0.1645165	-1.06586409120994e-18\\
0.1646165	-1.00464743496411e-18\\
0.1647165	-9.48540294856891e-19\\
0.1648165	-8.89549833778313e-19\\
0.1649165	-8.2912304620277e-19\\
0.1650165	-7.58138639441318e-19\\
0.1651165	-6.9689880033737e-19\\
0.1652165	-6.35120845350483e-19\\
0.1653165	-5.71928749789153e-19\\
0.1654165	-5.0584455916524e-19\\
0.1655166	-2.42605488265166e-19\\
0.1656166	-1.70932565352778e-19\\
0.1657166	-1.08190706773071e-19\\
0.1658166	-4.04052281194663e-20\\
0.1659166	2.70453102659577e-20\\
0.1660166	8.94384515524921e-20\\
0.1661166	1.62717324263376e-19\\
0.1662166	2.13500257478893e-19\\
0.1663166	2.89090533970497e-19\\
0.1664166	3.47486279756183e-19\\
0.1665167	6.20804247736966e-19\\
0.1666167	6.85020885559205e-19\\
0.1667167	7.40357966678696e-19\\
0.1668167	7.87695474589153e-19\\
0.1669167	8.48659842188621e-19\\
0.1670167	8.95634716114669e-19\\
0.1671167	9.41771868422624e-19\\
0.1672167	9.81002267192637e-19\\
0.1673167	1.01804730080959e-18\\
0.1674167	1.04843016231185e-18\\
0.1675168	1.33812822164609e-18\\
0.1676168	1.36894714850888e-18\\
0.1677168	1.39465900009174e-18\\
0.1678168	1.41411213650296e-18\\
0.1679168	1.43701611484113e-18\\
0.1680168	1.44395407309116e-18\\
0.1681168	1.45639528914324e-18\\
0.1682168	1.45670791666689e-18\\
0.1683168	1.4581719020645e-18\\
0.1684168	1.45499208124694e-18\\
0.1685169	1.74394824152174e-18\\
0.1686169	1.73241147302941e-18\\
0.1687169	1.71459606178601e-18\\
0.1688169	1.6985421032202e-18\\
0.1689169	1.67328291060179e-18\\
0.1690169	1.64885925854623e-18\\
0.1691169	1.60633383128552e-18\\
0.1692169	1.56780586998198e-18\\
0.1693169	1.51642602966615e-18\\
0.1694169	1.46641144315738e-18\\
0.169517	1.76336396869417e-18\\
0.169617	1.70834546314742e-18\\
0.169717	1.6439743895209e-18\\
0.169817	1.57888985559558e-18\\
0.169917	1.51287602332592e-18\\
0.170017	1.43687848874135e-18\\
0.170117	1.35302089237776e-18\\
0.170217	1.27462176436716e-18\\
0.170317	1.18621160208757e-18\\
0.170417	1.09355019256759e-18\\
0.1705171	1.41032299148529e-18\\
0.1706171	1.30754929094309e-18\\
0.1707171	1.21544649442883e-18\\
0.1708171	1.11487095709934e-18\\
0.1709171	1.00799865099259e-18\\
0.1710171	8.98343993711652e-19\\
0.1711171	7.80778939299049e-19\\
0.1712171	6.71552338777524e-19\\
0.1713171	5.48309570377563e-19\\
0.1714171	4.40112443737693e-19\\
0.1715172	7.89421874926601e-19\\
0.1716172	6.61337408772625e-19\\
0.1717172	5.44290034092815e-19\\
0.1718172	4.23213473888949e-19\\
0.1719172	2.94561682334842e-19\\
0.1720172	1.66330479021616e-19\\
0.1721172	4.80794858875362e-20\\
0.1722172	-8.90456311255461e-20\\
0.1723172	-2.12290608670519e-19\\
0.1724172	-3.47269719988097e-19\\
0.1725173	6.95920547760692e-20\\
0.1726173	-6.08745153925748e-20\\
0.1727173	-1.89775480698457e-19\\
0.1728173	-3.15962394994242e-19\\
0.1729173	-4.36536113434072e-19\\
0.1730173	-5.66821941300414e-19\\
0.1731173	-6.90344442800282e-19\\
0.1732173	-8.18801892752723e-19\\
0.1733173	-9.52040377484708e-19\\
0.1734173	-1.06802753465682e-18\\
0.1735174	-5.67840302353347e-19\\
0.1736174	-6.86114697532118e-19\\
0.1737174	-7.99363036339479e-19\\
0.1738174	-9.17767160331513e-19\\
0.1739174	-1.02949354791379e-18\\
0.1740174	-1.14066478234591e-18\\
0.1741174	-1.25533062952323e-18\\
0.1742174	-1.36543870525235e-18\\
0.1743174	-1.47080474489762e-18\\
0.1744174	-1.57908245611496e-18\\
0.1745175	-9.49586286397274e-19\\
0.1746175	-1.04689949475436e-18\\
0.1747175	-1.14664538627466e-18\\
0.1748175	-1.24751732796662e-18\\
0.1749175	-1.32588950317492e-18\\
0.1750175	-1.41578416574794e-18\\
0.1751175	-1.51883844279911e-18\\
0.1752175	-1.60427068071427e-18\\
0.1753175	-1.68884631837082e-18\\
0.1754175	-1.77684329531377e-18\\
0.1755176	-1.00689140241814e-18\\
0.1756176	-1.08178203887499e-18\\
0.1757176	-1.14746495330662e-18\\
0.1758176	-1.22790981390594e-18\\
0.1759176	-1.28441846682584e-18\\
0.1760176	-1.35558736741703e-18\\
0.1761176	-1.42726949718973e-18\\
0.1762176	-1.49253575402194e-18\\
0.1763176	-1.57163580563391e-18\\
0.1764176	-1.63195840573383e-18\\
0.1765177	-7.09643554132376e-19\\
0.1766177	-7.58304082556292e-19\\
0.1767177	-8.12569901711519e-19\\
0.1768177	-8.77975163521889e-19\\
0.1769177	-9.16986280906361e-19\\
0.1770177	-9.78958847897499e-19\\
0.1771177	-1.03009395685917e-18\\
0.1772177	-1.09339392714558e-18\\
0.1773177	-1.14861741710362e-18\\
0.1774177	-1.21223391568255e-18\\
0.1775178	-1.32775334870426e-19\\
0.1776178	-1.82298258783922e-19\\
0.1777178	-2.2917775729976e-19\\
0.1778178	-2.76255457244252e-19\\
0.1779178	-3.22846677245541e-19\\
0.1780178	-3.94691046777337e-19\\
0.1781178	-4.43902433944092e-19\\
0.1782178	-5.08918187326894e-19\\
0.1783178	-5.84447684693725e-19\\
0.1784178	-6.51420173675086e-19\\
0.1785179	5.9816495279021e-19\\
0.1786179	5.4042246455345e-19\\
0.1787179	4.81945143863779e-19\\
0.1788179	4.07464343871257e-19\\
0.1789179	3.35763335369674e-19\\
0.1790179	2.49733902505157e-19\\
0.1791179	1.76433719215478e-19\\
0.1792179	8.71445149918369e-20\\
0.1793179	7.4310264092519e-21\\
0.1794179	-7.27992296547527e-20\\
0.179518	1.33428215039024e-18\\
0.179618	1.20294937329691e-18\\
0.179718	1.16781460029875e-18\\
0.179818	1.05677524045728e-18\\
0.179918	9.02382856640505e-19\\
0.180018	8.41908000024496e-19\\
0.180118	7.174059412367e-19\\
0.180218	5.75783286880935e-19\\
0.180318	3.68865515500733e-19\\
0.180418	2.53465427453079e-19\\
0.1805181	1.86555202864119e-18\\
0.1806181	1.74992868639404e-18\\
0.1807181	1.62725881384398e-18\\
0.1808181	1.47501321683385e-18\\
0.1809181	1.27600662653004e-18\\
0.1810181	1.11847194845704e-18\\
0.1811181	9.96135525947491e-19\\
0.1812181	8.08293415241963e-19\\
0.1813181	5.59888721059428e-19\\
0.1814181	3.61589962266791e-19\\
0.1815182	2.18157727506152e-18\\
0.1816182	2.0687982277071e-18\\
0.1817182	1.77371435888973e-18\\
0.1818182	1.63070725166369e-18\\
0.1819182	1.38029214422726e-18\\
0.1820182	1.16920293219374e-18\\
0.1821182	9.50478306302283e-19\\
0.1822182	6.8354907562383e-19\\
0.1823182	4.34326659233131e-19\\
0.1824182	1.75292795380677e-19\\
0.1825183	2.25751399775046e-18\\
0.1826183	1.95764024190357e-18\\
0.1827183	1.70623372129401e-18\\
0.1828183	1.50298603051107e-18\\
0.1829183	1.15462626607717e-18\\
0.1830183	8.75018277743327e-19\\
0.1831183	5.85259243790384e-19\\
0.1832183	3.13779562103616e-19\\
0.1833183	-3.55588266923744e-21\\
0.1834183	-3.23345093600873e-19\\
0.1835184	2.04661354869623e-18\\
0.1836184	1.72268235332893e-18\\
0.1837184	1.47334382565017e-18\\
0.1838184	1.07321071184368e-18\\
0.1839184	8.04967309604305e-19\\
0.1840184	4.59480694525079e-19\\
0.1841184	3.59134722010967e-20\\
0.1842184	-2.58161963838521e-19\\
0.1843184	-6.06647626105762e-19\\
0.1844184	-1.0848025669474e-18\\
0.1845185	1.70763987759582e-18\\
0.1846185	1.42550816219758e-18\\
0.1847185	1.04164127500927e-18\\
0.1848185	6.1656686440513e-19\\
0.1849185	2.20066680113412e-19\\
0.1850185	-1.68696246585217e-19\\
0.1851185	-5.61048685131338e-19\\
0.1852185	-9.58676583750445e-19\\
0.1853185	-1.35349267150672e-18\\
0.1854185	-1.82750226471467e-18\\
0.1855186	1.40450345075169e-18\\
0.1856186	1.01936216841403e-18\\
0.1857186	5.70578536184795e-19\\
0.1858186	2.17165579687681e-19\\
0.1859186	-2.71257685221851e-19\\
0.1860186	-7.14320891006537e-19\\
0.1861186	-1.12075500183427e-18\\
0.1862186	-1.58824299908943e-18\\
0.1863186	-2.00326858247438e-18\\
0.1864186	-2.54096283552903e-18\\
0.1865187	1.25709601241943e-18\\
0.1866187	7.55850797034682e-19\\
0.1867187	3.77089552905144e-19\\
0.1868187	-1.07321379496453e-19\\
0.1869187	-6.13364808091451e-19\\
0.1870187	-1.04470659974094e-18\\
0.1871187	-1.49252736874606e-18\\
0.1872187	-2.0353519030429e-18\\
0.1873187	-2.53887631878102e-18\\
0.1874187	-3.05579290652375e-18\\
0.1875188	1.34684946768503e-18\\
0.1876188	8.68189778850618e-19\\
0.1877188	2.98865239184828e-19\\
0.1878188	-1.59988335634035e-19\\
0.1879188	-6.93319553461943e-19\\
0.1880188	-1.17197172254604e-18\\
0.1881188	-1.65249067050404e-18\\
0.1882188	-2.27692991223318e-18\\
0.1883188	-2.77265329373819e-18\\
0.1884188	-3.35213496019403e-18\\
0.1885189	1.70835361099825e-18\\
0.1886189	1.16530933130473e-18\\
0.1887189	5.93604201025521e-19\\
0.1888189	4.23927750109728e-20\\
0.1889189	-5.23239661055327e-19\\
0.1890189	-1.02206052865887e-18\\
0.1891189	-1.65647051547323e-18\\
0.1892189	-2.21228120968277e-18\\
0.1893189	-2.85848977521378e-18\\
0.1894189	-3.54705064060275e-18\\
0.189519	2.36987404975087e-18\\
0.189619	1.70302988788497e-18\\
0.189719	1.14386511851368e-18\\
0.189819	5.10958907783725e-19\\
0.189919	-5.88786453195874e-20\\
0.190019	-7.10359381083301e-19\\
0.190119	-1.36946663101603e-18\\
0.190219	-2.04320150036113e-18\\
0.190319	-2.81932572508903e-18\\
0.190419	-3.46610116702578e-18\\
0.1905191	3.2412905759081e-18\\
0.1906191	2.53464568187915e-18\\
0.1907191	1.87372268770303e-18\\
0.1908191	1.17095499054767e-18\\
0.1909191	4.59631477650938e-19\\
0.1910191	-3.05821511228885e-19\\
0.1911191	-1.04955453042381e-18\\
0.1912191	-1.87400555863287e-18\\
0.1913191	-2.65960674266631e-18\\
0.1914191	-3.46448721456315e-18\\
0.1915192	4.2883624404495e-18\\
0.1916192	3.48414223972822e-18\\
0.1917192	2.72482650720484e-18\\
0.1918192	1.84457273558109e-18\\
0.1919192	1.00138749063776e-18\\
0.1920192	1.77447922580166e-19\\
0.1921192	-7.2057248064691e-19\\
0.1922192	-1.66117382747965e-18\\
0.1923192	-2.68769545548481e-18\\
0.1924192	-3.61797707450797e-18\\
0.1925193	5.27792323459116e-18\\
0.1926193	4.43153721689288e-18\\
0.1927193	3.48629939529735e-18\\
0.1928193	2.52987392087899e-18\\
0.1929193	1.4771871836221e-18\\
0.1930193	4.70794280421238e-19\\
0.1931193	-6.18749664920904e-19\\
0.1932193	-1.69251328881769e-18\\
0.1933193	-2.72280783188883e-18\\
0.1934193	-3.9528009900219e-18\\
0.1935194	6.33029117803537e-18\\
0.1936194	5.25215563388942e-18\\
0.1937194	4.12587988485972e-18\\
0.1938194	2.92886557330194e-18\\
0.1939194	1.86966658757649e-18\\
0.1940194	5.88406694899768e-19\\
0.1941194	-6.42797382342812e-19\\
0.1942194	-1.91940727920606e-18\\
0.1943194	-3.20402873039365e-18\\
0.1944194	-4.52597163711124e-18\\
0.1945195	7.17361026801137e-18\\
0.1946195	5.91080058548229e-18\\
0.1947195	4.62959231010768e-18\\
0.1948195	3.33841043576367e-18\\
0.1949195	1.98126406503166e-18\\
0.1950195	5.38222159277298e-19\\
0.1951195	-6.74104572327856e-19\\
0.1952195	-1.90207545844406e-18\\
0.1953195	-4.35452526086145e-18\\
0.1954195	-5.20226309039785e-18\\
0.1955196	7.56086796288061e-18\\
0.1956196	6.77862123422725e-18\\
0.1957196	5.32488871436396e-18\\
0.1958196	3.18614711684815e-18\\
0.1959196	2.3895045728205e-18\\
0.1960196	3.24228748778253e-21\\
0.1961196	-1.86263677102736e-18\\
0.1962196	-3.05585203802259e-18\\
0.1963196	-5.38128220074824e-18\\
0.1964196	-6.60039480302032e-18\\
0.1965197	8.98492294499239e-18\\
0.1966197	7.11598060160104e-18\\
0.1967197	5.3386317341873e-18\\
0.1968197	4.07095403236104e-18\\
0.1969197	1.77740255095083e-18\\
0.1970197	-3.05738139965556e-20\\
0.1971197	-1.7939077627152e-18\\
0.1972197	-3.90528104781294e-18\\
0.1973197	-5.70848355129494e-18\\
0.1974197	-8.49776418579422e-18\\
0.1975198	9.51109676466787e-18\\
0.1976198	7.34987291327322e-18\\
0.1977198	5.62757350369859e-18\\
0.1978198	3.25482821653944e-18\\
0.1979198	1.19518293115723e-18\\
0.1980198	-5.34198820834448e-19\\
0.1981198	-2.86182540859985e-18\\
0.1982198	-4.66115733285512e-18\\
0.1983198	-7.74987822232345e-18\\
0.1984198	-9.88915630725296e-18\\
0.1985199	1.12419393640695e-17\\
0.1986199	8.27065198279746e-18\\
0.1987199	6.31636655400031e-18\\
0.1988199	3.85161425488139e-18\\
0.1989199	1.40928080390477e-18\\
0.1990199	-1.41659586296468e-18\\
0.1991199	-3.97001750330093e-18\\
0.1992199	-6.53220710404362e-18\\
0.1993199	-8.32077983837262e-18\\
0.1994199	-1.14889032571482e-17\\
0.19952	1.23360217020814e-17\\
0.19962	9.58137352989006e-18\\
0.19972	7.3879979678011e-18\\
0.19982	3.86920289472583e-18\\
0.19992	1.20710892779801e-18\\
0.20002	-1.3464435797409e-18\\
0.20012	-4.46897598626313e-18\\
0.20022	-6.76644082966687e-18\\
0.20032	-9.77227947040429e-18\\
0.20042	-1.29464674365422e-17\\
0.2005201	1.37235482332465e-17\\
0.2006201	1.05554541995775e-17\\
0.2007201	8.29541499152743e-18\\
0.2008201	4.78719007822141e-18\\
0.2009201	1.95296614922935e-18\\
0.2010201	-1.20561217638744e-18\\
0.2011201	-4.60639859931489e-18\\
0.2012201	-8.08568759499708e-18\\
0.2013201	-1.13971435855202e-17\\
0.2014201	-1.52107162794168e-17\\
0.2015202	1.57979144384e-17\\
0.2016202	1.27963213994957e-17\\
0.2017202	9.80273935481175e-18\\
0.2018202	5.49329409478153e-18\\
0.2019202	2.63346306084576e-18\\
0.2020202	-9.20753885752002e-19\\
0.2021202	-5.2216484618524e-18\\
0.2022202	-9.22860298839267e-18\\
0.2023202	-1.28068738116363e-17\\
0.2024202	-1.67263594793759e-17\\
0.2025203	1.84159488254862e-17\\
0.2026203	1.54481836378275e-17\\
0.2027203	1.1421787276678e-17\\
0.2028203	6.96103252583582e-18\\
0.2029203	2.79195187878016e-18\\
0.2030203	-1.25633266527012e-18\\
0.2031203	-5.25026302788618e-18\\
0.2032203	-1.01504813490432e-17\\
0.2033203	-1.48104485792251e-17\\
0.2034203	-1.89750457954813e-17\\
0.2035204	2.1712601075587e-17\\
0.2036204	1.73821392770539e-17\\
0.2037204	1.29867880271331e-17\\
0.2038204	8.2304500118452e-18\\
0.2039204	3.9328758001159e-18\\
0.2040204	-9.68826824352771e-19\\
0.2041204	-6.41866425863962e-18\\
0.2042204	-1.12402091191206e-17\\
0.2043204	-1.61350324079754e-17\\
0.2044204	-2.16811160244027e-17\\
0.2045205	2.54306288906959e-17\\
0.2046205	2.00789502998456e-17\\
0.2047205	1.51096495560461e-17\\
0.2048205	1.04555169953722e-17\\
0.2049205	5.18118014803251e-18\\
0.2050205	-5.15183587864369e-19\\
0.2051205	-6.30011241134279e-18\\
0.2052205	-1.27031039686525e-17\\
0.2053205	-1.81148362104397e-17\\
0.2054205	-2.47853664808452e-17\\
0.2055206	2.96850532528324e-17\\
0.2056206	2.41516244682248e-17\\
0.2057206	1.74825639596352e-17\\
0.2058206	1.20089202313723e-17\\
0.2059206	5.2117192887019e-18\\
0.2060206	-2.76092864883263e-19\\
0.2061206	-7.66768293275293e-18\\
0.2062206	-1.40203380771327e-17\\
0.2063206	-2.12334488422992e-17\\
0.2064206	-2.80464659354928e-17\\
0.2065207	3.4328743945088e-17\\
0.2066207	2.7700453854611e-17\\
0.2067207	2.02471972983999e-17\\
0.2068207	1.38904012733334e-17\\
0.2069207	6.72204265143868e-18\\
0.2070207	-9.93149223697235e-19\\
0.2071207	-8.81546185896023e-18\\
0.2072207	-1.61279410984439e-17\\
0.2073207	-2.41341042028552e-17\\
0.2074207	-3.28556235490043e-17\\
0.2075208	3.83627756029402e-17\\
0.2076208	3.09737009194197e-17\\
0.2077208	2.29335533733837e-17\\
0.2078208	1.49719801531739e-17\\
0.2079208	7.01250044146829e-18\\
0.2080208	-8.24998144974421e-19\\
0.2081208	-1.02217316529358e-17\\
0.2082208	-1.86574601136037e-17\\
0.2083208	-2.84078953705228e-17\\
0.2084208	-3.75420768285581e-17\\
0.2085209	4.41468379611893e-17\\
0.2086209	3.51224217168864e-17\\
0.2087209	2.67905941447254e-17\\
0.2088209	1.69361354909426e-17\\
0.2089209	7.56412682831187e-18\\
0.2090209	-2.09722155727245e-18\\
0.2091209	-1.15937062907224e-17\\
0.2092209	-2.22422308699981e-17\\
0.2093209	-3.21278691760864e-17\\
0.2094209	-4.31008929178838e-17\\
0.209521	5.0515111567196e-17\\
0.209621	4.01921346228708e-17\\
0.209721	2.96890597085917e-17\\
0.209821	1.91255559386119e-17\\
0.209921	8.87153090253044e-18\\
0.210021	-2.44966538746948e-18\\
0.210121	-1.3957997542413e-17\\
0.210221	-2.55134603645524e-17\\
0.210321	-3.77127540498248e-17\\
0.210421	-4.98859172966483e-17\\
0.2105211	5.72962114117781e-17\\
0.2106211	4.58918127898951e-17\\
0.2107211	3.4179326212298e-17\\
0.2108211	2.19301549545741e-17\\
0.2109211	1.01998410563479e-17\\
0.2110211	-2.66830780261928e-18\\
0.2111211	-1.60395465756663e-17\\
0.2112211	-2.89839756606568e-17\\
0.2113211	-4.32727761361718e-17\\
0.2114211	-5.73743997059832e-17\\
0.2115212	6.5177169599084e-17\\
0.2116212	5.51271675172531e-17\\
0.2117212	4.27388097298654e-17\\
0.2118212	2.97939486287974e-17\\
0.2119212	1.13969476034918e-17\\
0.2120212	-7.02121542414728e-18\\
0.2121212	-1.96987921111891e-17\\
0.2122212	-3.05390573907148e-17\\
0.2123212	-5.31060501072012e-17\\
0.2124212	-6.06202613054833e-17\\
0.2125213	7.03466673590032e-17\\
0.2126213	5.6786150664115e-17\\
0.2127213	4.07501713209962e-17\\
0.2128213	3.04369920095249e-17\\
0.2129213	1.44108030210399e-17\\
0.2130213	-8.39363489365628e-18\\
0.2131213	-1.8666290328035e-17\\
0.2132213	-3.67171028236867e-17\\
0.2133213	-5.24711650052913e-17\\
0.2134213	-7.54638456252921e-17\\
0.2135214	8.39081690127435e-17\\
0.2136214	7.18264466002155e-17\\
0.2137214	4.63198888009965e-17\\
0.2138214	2.94624217656409e-17\\
0.2139214	1.3743005625556e-17\\
0.2140214	-7.9291159656378e-18\\
0.2141214	-2.22191004257995e-17\\
0.2142214	-4.53611256425763e-17\\
0.2143214	-6.31529436475214e-17\\
0.2144214	-8.09503654553808e-17\\
0.2145215	1.00676327350996e-16\\
0.2146215	8.13445785032775e-17\\
0.2147215	5.86838356128594e-17\\
0.2148215	3.91630299089338e-17\\
0.2149215	1.97216461130811e-17\\
0.2150215	-2.22434330631196e-18\\
0.2151215	-2.87764701855546e-17\\
0.2152215	-5.15476896383746e-17\\
0.2153215	-7.16562230228913e-17\\
0.2154215	-9.97193265380296e-17\\
0.2155216	1.07667838969763e-16\\
0.2156216	8.70120438419461e-17\\
0.2157216	6.96603643596346e-17\\
0.2158216	4.70638022190741e-17\\
0.2159216	2.1206658926029e-17\\
0.2160216	-5.38675805320061e-18\\
0.2161216	-2.96453486593625e-17\\
0.2162216	-5.79443473380247e-17\\
0.2163216	-8.60983691241844e-17\\
0.2164216	-1.09354367769013e-16\\
0.2165217	1.24379600986881e-16\\
0.2166217	1.01054461629834e-16\\
0.2167217	8.04109519080288e-17\\
0.2168217	4.95455093991517e-17\\
0.2169217	2.61587577849841e-17\\
0.2170217	-1.43691818307036e-18\\
0.2171217	-3.43097068194986e-17\\
0.2172217	-6.29006085181234e-17\\
0.2173217	-9.70155804390407e-17\\
0.2174217	-1.25817586510823e-16\\
0.2175218	1.46822198830342e-16\\
0.2176218	1.178348218127e-16\\
0.2177218	9.06636131688494e-17\\
0.2178218	5.87997383766822e-17\\
0.2179218	2.64186010033889e-17\\
0.2180218	-1.1611604749692e-17\\
0.2181218	-3.97212382179557e-17\\
0.2182218	-7.1630452039655e-17\\
0.2183218	-1.10340322817938e-16\\
0.2184218	-1.48123875797613e-16\\
0.2185219	1.71256276242897e-16\\
0.2186219	1.36092513589959e-16\\
0.2187219	9.95560565171999e-17\\
0.2188219	6.23788210070151e-17\\
0.2189219	2.60673294052499e-17\\
0.2190219	-7.08763758304882e-18\\
0.2191219	-4.40012729432713e-17\\
0.2192219	-8.07848548284539e-17\\
0.2193219	-1.22735739911692e-16\\
0.2194219	-1.6432723349463e-16\\
0.219522	1.87675097794007e-16\\
0.219622	1.51997903592019e-16\\
0.219722	1.18376123338408e-16\\
0.219822	7.57372964769145e-17\\
0.219922	3.38855419784429e-17\\
0.220022	-6.48755309718889e-18\\
0.220122	-5.37919115423722e-17\\
0.220222	-9.55278145040579e-17\\
0.220322	-1.38274609289488e-16\\
0.220422	-1.87679285802986e-16\\
0.2205221	2.14301955917949e-16\\
0.2206221	1.74421314393098e-16\\
0.2207221	1.26730016858138e-16\\
0.2208221	7.94285353535872e-17\\
0.2209221	4.17089471940871e-17\\
0.2210221	-6.23278608263941e-18\\
0.2211221	-6.31843912675713e-17\\
0.2212221	-1.06904704583589e-16\\
0.2213221	-1.64110941274084e-16\\
0.2214221	-2.10465814016649e-16\\
0.2215222	2.45740660337322e-16\\
0.2216222	1.95200134309812e-16\\
0.2217222	1.45069161947189e-16\\
0.2218222	9.40361264243461e-17\\
0.2219222	4.19107125979625e-17\\
0.2220222	-1.03622512956323e-17\\
0.2221222	-7.06887846640097e-17\\
0.2222222	-1.25811572148444e-16\\
0.2223222	-1.81295619572623e-16\\
0.2224222	-2.41513729550739e-16\\
0.2225223	2.78947557873111e-16\\
0.2226223	2.28735024416416e-16\\
0.2227223	1.68070994249533e-16\\
0.2228223	1.07501467961908e-16\\
0.2229223	4.88399298005596e-17\\
0.2230223	-1.48170530671167e-17\\
0.2231223	-7.90740432569803e-17\\
0.2232223	-1.48220758402743e-16\\
0.2233223	-2.15215901986495e-16\\
0.2234223	-2.81670802868035e-16\\
0.2235224	3.22902431447934e-16\\
0.2236224	2.5697402636693e-16\\
0.2237224	1.95115928234645e-16\\
0.2238224	1.21275297015731e-16\\
0.2239224	5.08314779416337e-17\\
0.2240224	-1.93864278061916e-17\\
0.2241224	-9.10817810667864e-17\\
0.2242224	-1.64472408157616e-16\\
0.2243224	-2.38272392192981e-16\\
0.2244224	-3.1967364704711e-16\\
0.2245225	3.59763262363736e-16\\
0.2246225	2.87772066102368e-16\\
0.2247225	2.16052752715225e-16\\
0.2248225	1.43692410154764e-16\\
0.2249225	6.13958011050961e-17\\
0.2250225	-1.84948463232652e-17\\
0.2251225	-1.01979818183179e-16\\
0.2252225	-1.83381654060602e-16\\
0.2253225	-2.75324645161936e-16\\
0.2254225	-3.58714088194727e-16\\
0.2255226	4.17418430335524e-16\\
0.2256226	3.36765329991798e-16\\
0.2257226	2.42620310519548e-16\\
0.2258226	1.61168307262421e-16\\
0.2259226	7.04207615631562e-17\\
0.2260226	-1.97621061631728e-17\\
0.2261226	-1.176487632568e-16\\
0.2262226	-2.09613562432147e-16\\
0.2263226	-3.10113664643736e-16\\
0.2264226	-4.11665696951029e-16\\
0.2265227	4.75723973064125e-16\\
0.2266227	3.75300053889683e-16\\
0.2267227	2.78319658434259e-16\\
0.2268227	1.8026941628031e-16\\
0.2269227	7.86974745757789e-17\\
0.2270227	-2.67614473630922e-17\\
0.2271227	-1.34360701166122e-16\\
0.2272227	-2.40216054984103e-16\\
0.2273227	-3.58279738304639e-16\\
0.2274227	-4.70314175177738e-16\\
0.2275228	5.41360267706396e-16\\
0.2276228	4.29630839622105e-16\\
0.2277228	3.22235656006632e-16\\
0.2278228	2.06442735229348e-16\\
0.2279228	9.18459829004601e-17\\
0.2280228	-2.96066303866845e-17\\
0.2281228	-1.53584547786139e-16\\
0.2282228	-2.73345781163471e-16\\
0.2283228	-3.99707717732144e-16\\
0.2284228	-5.31017582915789e-16\\
0.2285229	6.09191870174375e-16\\
0.2286229	4.8958505888045e-16\\
0.2287229	3.45320731151707e-16\\
0.2288229	1.98237277751889e-16\\
0.2289229	9.2796259297271e-17\\
0.2290229	-5.38859138464476e-17\\
0.2291229	-1.2200592278701e-16\\
0.2292229	-2.89041106564866e-16\\
0.2293229	-4.29716649374298e-16\\
0.2294229	-6.15972377253575e-16\\
0.229523	7.41299550787409e-16\\
0.229623	5.78210708425911e-16\\
0.229723	3.7101028237291e-16\\
0.229823	2.59242670972384e-16\\
0.229923	8.54096015288716e-17\\
0.230023	-4.99426236150258e-18\\
0.230123	-1.63445299806929e-16\\
0.230223	-3.38354478921124e-16\\
0.230323	-4.75030494397343e-16\\
0.230423	-7.15642451335052e-16\\
0.2305231	7.58511281149002e-16\\
0.2306231	6.17622229026024e-16\\
0.2307231	4.65776169938613e-16\\
0.2308231	2.73747502280987e-16\\
0.2309231	1.15643422098732e-16\\
0.2310231	-3.10560743395886e-17\\
0.2311231	-1.8545772894504e-16\\
0.2312231	-3.63214073602111e-16\\
0.2313231	-5.76482020524973e-16\\
0.2314231	-8.33880954831954e-16\\
0.2315232	9.43329744392242e-16\\
0.2316232	7.10373288680158e-16\\
0.2317232	5.29383304963872e-16\\
0.2318232	3.06323106061667e-16\\
0.2319232	1.50910520735595e-16\\
0.2320232	-2.33371875043942e-17\\
0.2321232	-2.99057880513916e-16\\
0.2322232	-4.54998573175766e-16\\
0.2323232	-6.65968927254275e-16\\
0.2324232	-9.02794217249128e-16\\
0.2325233	1.00768218708889e-15\\
0.2326233	8.50262137994675e-16\\
0.2327233	5.79239906034493e-16\\
0.2328233	4.40213487039797e-16\\
0.2329233	1.83008875010442e-16\\
0.2330233	-3.8269519311572e-17\\
0.2331233	-2.65187854957425e-16\\
0.2332233	-5.34931274815436e-16\\
0.2333233	-7.80251701920542e-16\\
0.2334233	-1.02941504011425e-15\\
0.2335234	1.12395371069119e-15\\
0.2336234	9.31564047765766e-16\\
0.2337234	6.78374936379617e-16\\
0.2338234	4.54609611551843e-16\\
0.2339234	1.55250696102329e-16\\
0.2340234	-1.99032256442915e-17\\
0.2341234	-3.66180375383673e-16\\
0.2342234	-5.7397789027179e-16\\
0.2343234	-8.28703265867552e-16\\
0.2344234	-1.11071511349268e-15\\
0.2345235	1.36336126636819e-15\\
0.2346235	1.04134073084199e-15\\
0.2347235	7.82291605101521e-16\\
0.2348235	5.26662271979153e-16\\
0.2349235	2.20256646612222e-16\\
0.2350235	-8.57023671915843e-17\\
0.2351235	-3.3450867921442e-16\\
0.2352235	-6.63908101909719e-16\\
0.2353235	-1.00603265970049e-15\\
0.2354235	-1.28733416687932e-15\\
0.2355236	1.50194281695365e-15\\
0.2356236	1.22575953474124e-15\\
0.2357236	8.66294902162686e-16\\
0.2358236	6.20503650858101e-16\\
0.2359236	2.91364585018362e-16\\
0.2360236	-1.12048275617672e-16\\
0.2361236	-3.74493731411758e-16\\
0.2362236	-7.74490611827586e-16\\
0.2363236	-1.08424415557221e-15\\
0.2364236	-1.46957153117867e-15\\
0.2365237	1.66133995106518e-15\\
0.2366237	1.3983314429924e-15\\
0.2367237	1.00194065521623e-15\\
0.2368237	6.32671317919501e-16\\
0.2369237	2.57800642887542e-16\\
0.2370237	-4.8540928637132e-17\\
0.2371237	-5.05289085019893e-16\\
0.2372237	-8.24364063032574e-16\\
0.2373237	-1.31058814691588e-15\\
0.2374237	-1.66160222237857e-15\\
0.2375238	1.9592151915533e-15\\
0.2376238	1.56572014265804e-15\\
0.2377238	1.1590778749222e-15\\
0.2378238	7.71234380449589e-16\\
0.2379238	3.41748908863325e-16\\
0.2380238	-8.21167151671435e-17\\
0.2381238	-5.45307744841735e-16\\
0.2382238	-9.84885126681756e-16\\
0.2383238	-1.42993307799775e-15\\
0.2384238	-1.90146562082508e-15\\
0.2385239	2.15264711916283e-15\\
0.2386239	1.75536480465275e-15\\
0.2387239	1.31863578218402e-15\\
0.2388239	8.54691527116919e-16\\
0.2389239	3.84317379341443e-16\\
0.2390239	-1.6304741401572e-16\\
0.2391239	-6.49208530067367e-16\\
0.2392239	-1.1271142398831e-15\\
0.2393239	-1.64075190993298e-15\\
0.2394239	-2.12504337559887e-15\\
0.239524	2.4672747360552e-15\\
0.239624	1.93868528162432e-15\\
0.239724	1.49093203952709e-15\\
0.239824	9.26435516878547e-16\\
0.239924	4.57223253099351e-16\\
0.240024	-9.49581588877754e-17\\
0.240124	-6.98529847777751e-16\\
0.240224	-1.31196610760046e-15\\
0.240324	-1.88367858566245e-15\\
0.240424	-2.45189912113043e-15\\
0.2405241	2.81541359974426e-15\\
0.2406241	2.25419753969578e-15\\
0.2407241	1.64494643001248e-15\\
0.2408241	1.09135649120769e-15\\
0.2409241	5.07909743954127e-16\\
0.2410241	-1.80000677849173e-16\\
0.2411241	-7.3594379836193e-16\\
0.2412241	-1.41232266616182e-15\\
0.2413241	-2.05024471500667e-15\\
0.2414241	-2.77939074681695e-15\\
0.2415242	3.11794588155671e-15\\
0.2416242	2.54656576518183e-15\\
0.2417242	1.86582343920741e-15\\
0.2418242	1.19309714735829e-15\\
0.2419242	5.57869705946877e-16\\
0.2420242	-1.98131330404006e-16\\
0.2421242	-9.20791924689677e-16\\
0.2422242	-1.64346821255612e-15\\
0.2423242	-2.38684157109393e-15\\
0.2424242	-3.158772023512e-15\\
0.2425243	3.55761100151372e-15\\
0.2426243	2.85057327719126e-15\\
0.2427243	2.17095516411555e-15\\
0.2428243	1.36369443579717e-15\\
0.2429243	5.87308296427428e-16\\
0.2430243	-1.85949919529934e-16\\
0.2431243	-9.69932248716289e-16\\
0.2432243	-1.86443584662498e-15\\
0.2433243	-2.65504096688346e-15\\
0.2434243	-3.5129470524327e-15\\
0.2435244	4.00551965570031e-15\\
0.2436244	3.26333543836262e-15\\
0.2437244	2.42948327213017e-15\\
0.2438244	1.59197754870394e-15\\
0.2439244	6.5406083052892e-16\\
0.2440244	-2.65621001752086e-16\\
0.2441244	-1.13284160907951e-15\\
0.2442244	-2.09761511188781e-15\\
0.2443244	-2.99401499612177e-15\\
0.2444244	-4.03999099872968e-15\\
0.2445245	4.57842268344421e-15\\
0.2446245	3.66394542877079e-15\\
0.2447245	2.74608721468805e-15\\
0.2448245	1.77328258770532e-15\\
0.2449245	8.11036657681403e-16\\
0.2450245	-2.57879167349887e-16\\
0.2451245	-1.33322934798745e-15\\
0.2452245	-2.39711408806603e-15\\
0.2453245	-3.41376710188932e-15\\
0.2454245	-4.52935100775702e-15\\
0.2455246	5.20163905445792e-15\\
0.2456246	4.15729233192496e-15\\
0.2457246	3.0876517506797e-15\\
0.2458246	2.02094931476343e-15\\
0.2459246	9.04545592361974e-16\\
0.2460246	-2.94851645127099e-16\\
0.2461246	-1.49096652499702e-15\\
0.2462246	-2.67773160311186e-15\\
0.2463246	-3.9290619791826e-15\\
0.2464246	-5.198626897764e-15\\
0.2465247	5.871753809107e-15\\
0.2466247	4.63819041770274e-15\\
0.2467247	3.25100100331453e-15\\
0.2468247	2.53985603225745e-15\\
0.2469247	7.55852522896139e-16\\
0.2470247	-2.28241967805732e-16\\
0.2471247	-1.51774264287357e-15\\
0.2472247	-3.19579810321329e-15\\
0.2473247	-4.32313815099591e-15\\
0.2474247	-5.93781892820211e-15\\
0.2475248	6.83441869752917e-15\\
0.2476248	5.39339297888307e-15\\
0.2477248	3.49153546004629e-15\\
0.2478248	2.18411471668435e-15\\
0.2479248	1.55039072732145e-15\\
0.2480248	-3.06112843936088e-16\\
0.2481248	-2.25733338263874e-15\\
0.2482248	-3.15039110316468e-15\\
0.2483248	-4.80730762442694e-15\\
0.2484248	-7.02472159206354e-15\\
0.2485249	7.11602925603186e-15\\
0.2486249	5.68273834985174e-15\\
0.2487249	4.45651797971145e-15\\
0.2488249	2.74519556423894e-15\\
0.2489249	8.83451536207951e-16\\
0.2490249	-7.66876934637339e-16\\
0.2491249	-1.81648895355715e-15\\
0.2492249	-3.84830948435118e-15\\
0.2493249	-5.41717555679936e-15\\
0.2494249	-7.04951904779954e-15\\
0.249525	8.47392872894848e-15\\
0.249625	6.46680926791035e-15\\
0.249725	4.99314779621814e-15\\
0.249825	3.64340522370021e-15\\
0.249925	1.03808691723538e-15\\
0.250025	-1.71918691269455e-16\\
0.250125	-2.30499772094228e-15\\
0.250225	-4.64846482457574e-15\\
0.250325	-6.45821334429668e-15\\
0.250425	-7.9583616355112e-15\\
0.2505251	9.65834580154144e-15\\
0.2506251	7.47732503089523e-15\\
0.2507251	6.13068393341337e-15\\
0.2508251	3.52505184384138e-15\\
0.2509251	1.60066039917722e-15\\
0.2510251	-6.6827898527102e-16\\
0.2511251	-2.27319359956761e-15\\
0.2512251	-5.17076375682229e-15\\
0.2513251	-7.28253291509356e-15\\
0.2514251	-9.49451358864923e-15\\
0.2515252	1.09107688343977e-14\\
0.2516252	8.25829078283108e-15\\
0.2517252	6.06799678729538e-15\\
0.2518252	3.60006312454089e-15\\
0.2519252	2.15223316169176e-15\\
0.2520252	-9.39762045831858e-16\\
0.2521252	-3.30177878513043e-15\\
0.2522252	-5.52083094033641e-15\\
0.2523252	-8.14465572969458e-15\\
0.2524252	-1.06812746435644e-14\\
0.2525253	1.18581833516219e-14\\
0.2526253	9.44099227249384e-15\\
0.2527253	6.83333339272493e-15\\
0.2528253	4.69057601136157e-15\\
0.2529253	1.71007305407623e-15\\
0.2530253	-3.68370595428228e-16\\
0.2531253	-3.76202470673492e-15\\
0.2532253	-6.64475524235283e-15\\
0.2533253	-9.14654070423015e-15\\
0.2534253	-1.23529832382606e-14\\
0.2535254	1.34008640738307e-14\\
0.2536254	1.10545609980107e-14\\
0.2537254	8.02209401821816e-15\\
0.2538254	5.40040789399399e-15\\
0.2539254	2.33334876071983e-15\\
0.2540254	-9.87814526023338e-16\\
0.2541254	-4.32386337498052e-15\\
0.2542254	-7.38709689139303e-15\\
0.2543254	-9.84079349781143e-15\\
0.2544254	-1.32986667636273e-14\\
0.2545255	1.50335598509208e-14\\
0.2546255	1.23163897221141e-14\\
0.2547255	9.06138473987758e-15\\
0.2548255	5.85870199650279e-15\\
0.2549255	2.35087370737599e-15\\
0.2550255	-7.66612781746551e-16\\
0.2551255	-4.74472846384279e-15\\
0.2552255	-7.78031008717996e-15\\
0.2553255	-1.2015461111692e-14\\
0.2554255	-1.55369462462035e-14\\
0.2555256	1.70863109521348e-14\\
0.2556256	1.33935271985187e-14\\
0.2557256	1.0497411038872e-14\\
0.2558256	6.53879564919861e-15\\
0.2559256	2.71697927466471e-15\\
0.2560256	-7.09629355317657e-16\\
0.2561256	-4.42285889568017e-15\\
0.2562256	-9.0441161174681e-15\\
0.2563256	-1.31337196719492e-14\\
0.2564256	-1.71902266849794e-14\\
0.2565257	1.94222010842524e-14\\
0.2566257	1.56777328682927e-14\\
0.2567257	1.18538231142818e-14\\
0.2568257	7.70590362895023e-15\\
0.2569257	3.05464438974194e-15\\
0.2570257	-1.21332891373565e-15\\
0.2571257	-5.14470969565579e-15\\
0.2572257	-9.71877747490357e-15\\
0.2573257	-1.48466572177451e-14\\
0.2574257	-1.93705707133171e-14\\
0.2575258	2.21855105828988e-14\\
0.2576258	1.7173572593245e-14\\
0.2577258	1.26663650395698e-14\\
0.2578258	8.10499734146089e-15\\
0.2579258	4.00334677486066e-15\\
0.2580258	-1.05114390426791e-15\\
0.2581258	-6.39666063006669e-15\\
0.2582258	-1.12962032263925e-14\\
0.2583258	-1.69367623519919e-14\\
0.2584258	-2.14284875884758e-14\\
0.2585259	2.425548577939e-14\\
0.2586259	1.96612718757027e-14\\
0.2587259	1.43621068326051e-14\\
0.2588259	9.56378269982322e-15\\
0.2589259	3.55322691459209e-15\\
0.2590259	-1.30061160038853e-15\\
0.2591259	-6.54586690895773e-15\\
0.2592259	-1.26468511946198e-14\\
0.2593259	-1.89831402164242e-14\\
0.2594259	-2.48486492087463e-14\\
0.259526	2.81287951024489e-14\\
0.259626	2.23645388437635e-14\\
0.259726	1.67206811865791e-14\\
0.259826	1.02554765477274e-14\\
0.259926	4.11760956729901e-15\\
0.260026	-1.45282077899525e-15\\
0.260126	-8.1233066112613e-15\\
0.260226	-1.44679264138601e-14\\
0.260326	-2.09663293336557e-14\\
0.260426	-2.80027091255206e-14\\
0.2605261	3.10282731631602e-14\\
0.2606261	2.49288015732899e-14\\
0.2607261	1.8730885955615e-14\\
0.2608261	1.14427329655427e-14\\
0.2609261	5.17330046436897e-15\\
0.2610261	-1.86661019426265e-15\\
0.2611261	-9.36325004838454e-15\\
0.2612261	-1.58988071998718e-14\\
0.2613261	-2.39502793779603e-14\\
0.2614261	-3.08883351824709e-14\\
0.2615262	3.5117380836099e-14\\
0.2616262	2.7597895529798e-14\\
0.2617262	2.07393763307929e-14\\
0.2618262	1.36082131393672e-14\\
0.2619262	5.38300255765441e-15\\
0.2620262	-1.64424000112235e-15\\
0.2621262	-1.00668552499602e-14\\
0.2622262	-1.83623021204695e-14\\
0.2623262	-2.68909074688826e-14\\
0.2624262	-3.48946019346724e-14\\
0.2625263	3.97895993952166e-14\\
0.2626263	3.15678637049586e-14\\
0.2627263	2.2879641526572e-14\\
0.2628263	1.49696132263672e-14\\
0.2629263	6.20737551489482e-15\\
0.2630263	-1.91121382678097e-15\\
0.2631263	-1.17626772952314e-14\\
0.2632263	-2.0594544737192e-14\\
0.2633263	-2.95239661680629e-14\\
0.2634263	-3.95363102553605e-14\\
0.2635264	4.41004644610825e-14\\
0.2636264	3.55959299866038e-14\\
0.2637264	2.58697770537893e-14\\
0.2638264	1.64782094407995e-14\\
0.2639264	7.11644292512709e-15\\
0.2640264	-2.37980379715697e-15\\
0.2641264	-1.30328027165657e-14\\
0.2642264	-2.27212941210538e-14\\
0.2643264	-3.31789487785001e-14\\
0.2644264	-4.39928144432828e-14\\
0.2645265	4.95179158329853e-14\\
0.2646265	4.0051672854034e-14\\
0.2647265	2.93775343993573e-14\\
0.2648265	1.85109615519119e-14\\
0.2649265	7.62205346142522e-15\\
0.2650265	-2.96279980945545e-15\\
0.2651265	-1.37592478894217e-14\\
0.2652265	-2.61232926914136e-14\\
0.2653265	-3.72495845877085e-14\\
0.2654265	-5.01697003091008e-14\\
0.2655266	5.62849544330938e-14\\
0.2656266	4.87168242223283e-14\\
0.2657266	3.12716208047001e-14\\
0.2658266	2.05878091460847e-14\\
0.2659266	1.24758086255514e-14\\
0.2660266	-7.08017585953164e-15\\
0.2661266	-1.19203042189915e-14\\
0.2662266	-2.57072323609622e-14\\
0.2663266	-4.19242240141387e-14\\
0.2664266	-5.3873243844491e-14\\
0.2665267	5.7818182364805e-14\\
0.2666267	4.67708847347012e-14\\
0.2667267	4.12101784010877e-14\\
0.2668267	2.85789547360429e-14\\
0.2669267	6.51124051143952e-15\\
0.2670267	-7.16577359954268e-15\\
0.2671267	-1.4429685190743e-14\\
0.2672267	-3.70608103682343e-14\\
0.2673267	-4.66400931117293e-14\\
0.2674267	-6.4546993461414e-14\\
0.2675268	6.67076647909689e-14\\
0.2676268	6.0542895175596e-14\\
0.2677268	4.31618198994487e-14\\
0.2678268	2.40126813382593e-14\\
0.2679268	1.27560961043216e-14\\
0.2680268	-7.32700256399819e-16\\
0.2681268	-1.63615854431046e-14\\
0.2682268	-3.38192595994293e-14\\
0.2683268	-5.25729316710765e-14\\
0.2684268	-7.18659813186015e-14\\
0.2685269	7.60387812722067e-14\\
0.2686269	6.07663325409144e-14\\
0.2687269	4.86126348866748e-14\\
0.2688269	3.12537152263356e-14\\
0.2689269	1.06014818345806e-14\\
0.2690269	-1.19379195363703e-15\\
0.2691269	-2.17409579988773e-14\\
0.2692269	-3.84054595167802e-14\\
0.2693269	-5.83067700038616e-14\\
0.2694269	-7.83158072926101e-14\\
0.269527	9.19265037613522e-14\\
0.269627	7.42453918013034e-14\\
0.269727	4.73854114310016e-14\\
0.269827	3.54965196935568e-14\\
0.269927	1.29905619130595e-14\\
0.270027	-5.45598019744716e-15\\
0.270127	-2.48992141549226e-14\\
0.270227	-5.01250322466574e-14\\
0.270327	-6.56462805975127e-14\\
0.270427	-8.56998986067038e-14\\
0.2705271	9.53381260073844e-14\\
0.2706271	7.70281556015442e-14\\
0.2707271	6.31862275688265e-14\\
0.2708271	4.07085032791242e-14\\
0.2709271	1.67817945310054e-14\\
0.2710271	-1.11339914671445e-15\\
0.2711271	-2.51998218333291e-14\\
0.2712271	-4.74003617941312e-14\\
0.2713271	-7.93349210777659e-14\\
0.2714271	-1.02317253208663e-13\\
0.2715272	1.07483246650668e-13\\
0.2716272	9.2386732125221e-14\\
0.2717272	6.41999560607786e-14\\
0.2718272	4.286633961422e-14\\
0.2719272	1.86517489842246e-14\\
0.2720272	-7.85214702671003e-15\\
0.2721272	-3.57244864616815e-14\\
0.2722272	-5.37117880472591e-14\\
0.2723272	-8.02244925457181e-14\\
0.2724272	-1.13333468225267e-13\\
0.2725273	1.22271256454031e-13\\
0.2726273	9.61278424717944e-14\\
0.2727273	7.12816015059302e-14\\
0.2728273	5.10559577126705e-14\\
0.2729273	1.91319136912577e-14\\
0.2730273	-1.04482401211569e-14\\
0.2731273	-3.32771791050207e-14\\
0.2732273	-6.45787609142857e-14\\
0.2733273	-9.92042035682805e-14\\
0.2734273	-1.21628226111402e-13\\
0.2735274	1.38578312758184e-13\\
0.2736274	1.12001188462632e-13\\
0.2737274	8.65363706975409e-14\\
0.2738274	4.92549025671248e-14\\
0.2739274	1.76242130909172e-14\\
0.2740274	-1.04877869810421e-14\\
0.2741274	-3.68085672205967e-14\\
0.2742274	-7.26568006331437e-14\\
0.2743274	-1.08938142677755e-13\\
0.2744274	-1.36140971035315e-13\\
0.2745275	1.55329506093153e-13\\
0.2746275	1.20238962796193e-13\\
0.2747275	9.53497302164867e-14\\
0.2748275	6.18864497375441e-14\\
0.2749275	2.1512996389051e-14\\
0.2750275	-1.36630001311952e-14\\
0.2751275	-4.10855873421855e-14\\
0.2752275	-7.7745880518923e-14\\
0.2753275	-1.20177403485611e-13\\
0.2754275	-1.54451385633836e-13\\
0.2755276	1.72689528410064e-13\\
0.2756276	1.42549947727094e-13\\
0.2757276	1.05221303494493e-13\\
0.2758276	6.65299972843981e-14\\
0.2759276	3.2788954886112e-14\\
0.2760276	-9.19738613204936e-15\\
0.2761276	-5.21280373315946e-14\\
0.2762276	-8.82003713296403e-14\\
0.2763276	-1.29104983987978e-13\\
0.2764276	-1.76020504195893e-13\\
0.2765277	1.92968604477622e-13\\
0.2766277	1.57207819223005e-13\\
0.2767277	1.15075351118191e-13\\
0.2768277	7.74938915721594e-14\\
0.2769277	3.59248246264145e-14\\
0.2770277	-7.62626913447753e-15\\
0.2771277	-6.06040562185652e-14\\
0.2772277	-9.98978379587407e-14\\
0.2773277	-1.51835885151757e-13\\
0.2774277	-1.92179713790926e-13\\
0.2775278	2.15191941496401e-13\\
0.2776278	1.78217676638489e-13\\
0.2777278	1.29067941205639e-13\\
0.2778278	8.4307180859525e-14\\
0.2779278	3.1096065950501e-14\\
0.2780278	-1.28024420066847e-14\\
0.2781278	-5.90169636435951e-14\\
0.2782278	-1.1856151558223e-13\\
0.2783278	-1.61829263541054e-13\\
0.2784278	-2.18586214267668e-13\\
0.2785279	2.47652778179093e-13\\
0.2786279	1.92919834093994e-13\\
0.2787279	1.39289107976599e-13\\
0.2788279	8.95682062167528e-14\\
0.2789279	3.72243748275185e-14\\
0.2790279	-1.36088117344077e-14\\
0.2791279	-6.81247002187887e-14\\
0.2792279	-1.30836741816318e-13\\
0.2793279	-1.85571606694209e-13\\
0.2794279	-2.45462236740542e-13\\
0.279528	2.73173793430331e-13\\
0.279628	2.1279799042261e-13\\
0.279728	1.62169857507465e-13\\
0.279828	1.01002968866132e-13\\
0.279928	3.97404148017244e-14\\
0.280028	-2.04378301456711e-14\\
0.280128	-7.76074713516398e-14\\
0.280228	-1.39092384945176e-13\\
0.280328	-2.11457034862147e-13\\
0.280428	-2.70498818768088e-13\\
0.2805281	3.02241812140695e-13\\
0.2806281	2.42702804429746e-13\\
0.2807281	1.83851496773269e-13\\
0.2808281	1.13037613000574e-13\\
0.2809281	4.8417355396263e-14\\
0.2810281	-2.10384829667279e-14\\
0.2811281	-8.55363303605867e-14\\
0.2812281	-1.54451567587101e-13\\
0.2813281	-2.26320178704775e-13\\
0.2814281	-3.08830325439665e-13\\
0.2815282	3.39659337247752e-13\\
0.2816282	2.74184474803973e-13\\
0.2817282	2.00261396654738e-13\\
0.2818282	1.23680047849142e-13\\
0.2819282	5.11215853853637e-14\\
0.2820282	-1.98326986114358e-14\\
0.2821282	-1.00692358238162e-13\\
0.2822282	-1.8204869612112e-13\\
0.2823282	-2.53565577125085e-13\\
0.2824282	-3.43970148776007e-13\\
0.2825283	3.80876488070028e-13\\
0.2826283	2.99152323495981e-13\\
0.2827283	2.18168037423566e-13\\
0.2828283	1.43038873097377e-13\\
0.2829283	5.98645315781824e-14\\
0.2830283	-2.42610090281217e-14\\
0.2831283	-1.11239573613266e-13\\
0.2832283	-2.01958778855289e-13\\
0.2833283	-2.86281927867439e-13\\
0.2834283	-3.83037802974224e-13\\
0.2835284	4.24731777640153e-13\\
0.2836284	3.40648877757674e-13\\
0.2837284	2.44062398943596e-13\\
0.2838284	1.60386720566369e-13\\
0.2839284	6.61232541476217e-14\\
0.2840284	-3.11287563466848e-14\\
0.2841284	-1.22661325135127e-13\\
0.2842284	-2.1864671674583e-13\\
0.2843284	-3.18126331233952e-13\\
0.2844284	-4.18999483148991e-13\\
0.2845285	4.69943321515875e-13\\
0.2846285	3.7814417123602e-13\\
0.2847285	2.78336051430569e-13\\
0.2848285	1.77303536566311e-13\\
0.2849285	7.30310247258546e-14\\
0.2850285	-2.52853897766401e-14\\
0.2851285	-1.37225927212708e-13\\
0.2852285	-2.41135019487735e-13\\
0.2853285	-3.54109090300068e-13\\
0.2854285	-4.71984226557332e-13\\
0.2855286	5.23357380092756e-13\\
0.2856286	4.45426693945253e-13\\
0.2857286	3.12902484746817e-13\\
0.2858286	2.25117023989738e-13\\
0.2859286	1.12726376646866e-13\\
0.2860286	7.72345053477011e-15\\
0.2861286	-1.56548831845058e-13\\
0.2862286	-2.45384232119853e-13\\
0.2863286	-4.22699773975293e-13\\
0.2864286	-5.51022191588519e-13\\
0.2865287	5.4868259087348e-13\\
0.2866287	4.4989475244178e-13\\
0.2867287	3.21141443675807e-13\\
0.2868287	2.05596616685076e-13\\
0.2869287	4.78942196688768e-14\\
0.2870287	-5.85746699516873e-15\\
0.2871287	-2.08061365750448e-13\\
0.2872287	-3.09616987410623e-13\\
0.2873287	-4.59905696276083e-13\\
0.2874287	-6.0677582270336e-13\\
0.2875288	6.85220242883111e-13\\
0.2876288	5.22840281235359e-13\\
0.2877288	4.29840657854134e-13\\
0.2878288	2.64658403293919e-13\\
0.2879288	8.73401255220592e-14\\
0.2880288	-4.04422497582071e-14\\
0.2881288	-1.5537559228231e-13\\
0.2882288	-2.92489534407262e-13\\
0.2883288	-4.85140280182748e-13\\
0.2884288	-6.64994271281483e-13\\
0.2885289	6.76948089421133e-13\\
0.2886289	5.51171079757735e-13\\
0.2887289	4.5367012156861e-13\\
0.2888289	2.59706032997197e-13\\
0.2889289	1.46313418840194e-13\\
0.2890289	-7.6820614741775e-15\\
0.2891289	-2.21646253217932e-13\\
0.2892289	-4.13118912641972e-13\\
0.2893289	-5.97795941185517e-13\\
0.2894289	-6.89511461639323e-13\\
0.289529	8.1323241550205e-13\\
0.289629	6.91941962118614e-13\\
0.289729	4.3365594015282e-13\\
0.289829	3.32171924725712e-13\\
0.289929	8.32414554071472e-14\\
0.290029	-1.54110057455295e-14\\
0.290129	-2.64068953860018e-13\\
0.290229	-4.61004537030781e-13\\
0.290329	-6.02459066327304e-13\\
0.290429	-7.82623359820043e-13\\
0.2905291	9.13435705333857e-13\\
0.2906291	7.02065769307959e-13\\
0.2907291	4.82129127130231e-13\\
0.2908291	3.67839394226649e-13\\
0.2909291	1.75561790101941e-13\\
0.2910291	-7.61660796403142e-14\\
0.2911291	-2.66613255677462e-13\\
0.2912291	-4.72834252005815e-13\\
0.2913291	-6.69647698875406e-13\\
0.2914291	-9.29614777818416e-13\\
0.2915292	9.98760862770632e-13\\
0.2916292	8.26657242276924e-13\\
0.2917292	5.87715232091319e-13\\
0.2918292	4.18625395666813e-13\\
0.2919292	1.58446583156823e-13\\
0.2920292	-5.13712875483104e-14\\
0.2921292	-2.66964245802013e-13\\
0.2922292	-5.42031062849566e-13\\
0.2923292	-7.2780984575586e-13\\
0.2924292	-9.73054398953103e-13\\
0.2925293	1.13583354393549e-12\\
0.2926293	8.60632365836692e-13\\
0.2927293	6.95109467791725e-13\\
0.2928293	4.0069147788275e-13\\
0.2929293	1.41410787588684e-13\\
0.2930293	-1.16069498706488e-13\\
0.2931293	-3.02430369059822e-13\\
0.2932293	-5.45671488166483e-13\\
0.2933293	-8.71085553884266e-13\\
0.2934293	-1.10123240069137e-12\\
0.2935294	1.26777814740078e-12\\
0.2936294	9.99449541796666e-13\\
0.2937294	6.75655429449243e-13\\
0.2938294	4.85033874924533e-13\\
0.2939294	2.19088843592099e-13\\
0.2940294	-1.27782459639454e-13\\
0.2941294	-3.5826201008269e-13\\
0.2942294	-6.72083123845433e-13\\
0.2943294	-9.66002370643609e-13\\
0.2944294	-1.23377122595947e-12\\
0.2945295	1.34989714058073e-12\\
0.2946295	1.09624342149832e-12\\
0.2947295	8.06120549264681e-13\\
0.2948295	4.98089201946195e-13\\
0.2949295	1.93860808838303e-13\\
0.2950295	-8.1672503002272e-14\\
0.2951295	-4.00407726954087e-13\\
0.2952295	-7.31000496630725e-13\\
0.2953295	-1.03883434424152e-12\\
0.2954295	-1.38598967000814e-12\\
0.2955296	1.50852491872276e-12\\
0.2956296	1.24407493887047e-12\\
0.2957296	8.74545277824428e-13\\
0.2958296	5.5138741962223e-13\\
0.2959296	2.29515398356244e-13\\
0.2960296	-1.32661444748631e-13\\
0.2961296	-4.73205420175706e-13\\
0.2962296	-8.26617151413438e-13\\
0.2963296	-1.12380192176976e-12\\
0.2964296	-1.49203574824383e-12\\
0.2965297	1.64320389436073e-12\\
0.2966297	1.30360951168896e-12\\
0.2967297	1.03364348562457e-12\\
0.2968297	6.20896320867128e-13\\
0.2969297	2.56762216919934e-13\\
0.2970297	-6.35251243922317e-14\\
0.2971297	-5.40856349462724e-13\\
0.2972297	-8.72209983901348e-13\\
0.2973297	-1.25061563829777e-12\\
0.2974297	-1.66511688435746e-12\\
0.2975298	1.89403021381238e-12\\
0.2976298	1.49824786585614e-12\\
0.2977298	1.12395691482306e-12\\
0.2978298	6.98440059653254e-13\\
0.2979298	2.53156756064288e-13\\
0.2980298	-7.62176206157728e-14\\
0.2981298	-5.49752250469643e-13\\
0.2982298	-1.02322100577711e-12\\
0.2983298	-1.44806222151665e-12\\
0.2984298	-1.87133811073467e-12\\
0.2985299	2.09784789769851e-12\\
0.2986299	1.700578959486e-12\\
0.2987299	1.19082530324558e-12\\
0.2988299	7.394469055752e-13\\
0.2989299	3.21888329244259e-13\\
0.2990299	-1.81778485344992e-13\\
0.2991299	-5.86811005997159e-13\\
0.2992299	-1.1037525591052e-12\\
0.2993299	-1.63838836652474e-12\\
0.2994299	-2.09170119305259e-12\\
0.29953	2.35894252598978e-12\\
0.29963	1.83601150243242e-12\\
0.29973	1.32123547841601e-12\\
0.29983	8.33297928120295e-13\\
0.29993	3.95912542320068e-13\\
0.30003	-1.62130003310278e-13\\
0.30013	-7.06914874694648e-13\\
0.30023	-1.19935548512497e-12\\
0.30033	-1.79514547383213e-12\\
0.30043	-2.34471026680494e-12\\
0.3005301	2.56200838348487e-12\\
0.3006301	2.03157889784744e-12\\
0.3007301	1.52876683622254e-12\\
0.3008301	9.24718129954706e-13\\
0.3009301	3.96095665125792e-13\\
0.3010301	-1.74869637711427e-13\\
0.3011301	-8.00327329784736e-13\\
0.3012301	-1.38675545169051e-12\\
0.3013301	-1.93490810394442e-12\\
0.3014301	-2.53976258026713e-12\\
0.3015302	2.85744661208508e-12\\
0.3016302	2.24021270582782e-12\\
0.3017302	1.61716759027548e-12\\
0.3018302	1.01698552408444e-12\\
0.3019302	4.74389139243312e-13\\
0.3020302	-1.69794833566739e-13\\
0.3021302	-8.68579324662555e-13\\
0.3022302	-1.46876020416043e-12\\
0.3023302	-2.21085907603165e-12\\
0.3024302	-2.82906555261285e-12\\
0.3025303	3.15150696435493e-12\\
0.3026303	2.47325260975989e-12\\
0.3027303	1.85558697271524e-12\\
0.3028303	1.19024234838386e-12\\
0.3029303	4.75579339436455e-13\\
0.3030303	-1.83352340853022e-13\\
0.3031303	-9.74752511626926e-13\\
0.3032303	-1.6800086643591e-12\\
0.3033303	-2.37363351556072e-12\\
0.3034303	-3.12320206796774e-12\\
0.3035304	3.53642377144599e-12\\
0.3036304	2.77661778306797e-12\\
0.3037304	2.00116451884466e-12\\
0.3038304	1.27088776131009e-12\\
0.3039304	5.53872212160583e-13\\
0.3040304	-2.74470182206691e-13\\
0.3041304	-1.03133332983942e-12\\
0.3042304	-1.8264495331499e-12\\
0.3043304	-2.66202139152083e-12\\
0.3044304	-3.5326531743175e-12\\
0.3045305	3.89852966998288e-12\\
0.3046305	3.08891767452066e-12\\
0.3047305	2.30755103771153e-12\\
0.3048305	1.39092529981039e-12\\
0.3049305	5.8348670978742e-13\\
0.3050305	-2.62295478653174e-13\\
0.3051305	-1.18585606651117e-12\\
0.3052305	-2.01846038020915e-12\\
0.3053305	-2.98313005441278e-12\\
0.3054305	-3.89456821621434e-12\\
0.3055306	4.24537116702499e-12\\
0.3056306	3.42284216443906e-12\\
0.3057306	2.46101936595945e-12\\
0.3058306	1.57924361483866e-12\\
0.3059306	7.05558294890171e-13\\
0.3060306	-3.23211926273985e-13\\
0.3061306	-1.26138159810508e-12\\
0.3062306	-2.25432437573775e-12\\
0.3063306	-3.23839218884744e-12\\
0.3064306	-4.34083369585144e-12\\
0.3065307	4.69611144443068e-12\\
0.3066307	3.71445542645408e-12\\
0.3067307	2.78913034635893e-12\\
0.3068307	1.73013936437121e-12\\
0.3069307	2.57007243593361e-13\\
0.3070307	-8.01133878051123e-13\\
0.3071307	-1.80545794294622e-12\\
0.3072307	-2.70735766725814e-12\\
0.3073307	-3.44835655662718e-12\\
0.3074307	-4.96002009119091e-12\\
0.3075308	4.98300210272976e-12\\
0.3076308	4.28853808179949e-12\\
0.3077308	3.0904642970724e-12\\
0.3078308	1.49796074863325e-12\\
0.3079308	6.30620956633239e-13\\
0.3080308	-3.81454659339645e-13\\
0.3081308	-1.39756451133998e-12\\
0.3082308	-2.26631087375847e-12\\
0.3083308	-3.82550404093631e-12\\
0.3084308	-4.90206567511736e-12\\
0.3085309	6.01545150954438e-12\\
0.3086309	4.59189431181607e-12\\
0.3087309	3.23772609992541e-12\\
0.3088309	2.17057396290244e-12\\
0.3089309	6.19449320083773e-13\\
0.3090309	-1.75150334901026e-13\\
0.3091309	-1.9611389710212e-12\\
0.3092309	-3.47473859632104e-12\\
0.3093309	-4.44037499229787e-12\\
0.3094309	-5.57057258885474e-12\\
0.309531	6.06222821372465e-12\\
0.309631	4.65057861533324e-12\\
0.309731	4.01059765727691e-12\\
0.309831	2.47844818746209e-12\\
0.309931	1.40273373149444e-12\\
0.310031	-8.55390860419908e-13\\
0.310131	-1.92210794840392e-12\\
0.310231	-3.41082461334347e-12\\
0.310331	-4.92205924345699e-12\\
0.310431	-6.04332720967852e-12\\
0.3105311	6.71163020446901e-12\\
0.3106311	5.81131821228825e-12\\
0.3107311	4.61906202750807e-12\\
0.3108311	2.60053492153734e-12\\
0.3109311	1.23499968101925e-12\\
0.3110311	-9.84571101346338e-13\\
0.3111311	-2.55137380754034e-12\\
0.3112311	-3.94465154548536e-12\\
0.3113311	-5.62957079558237e-12\\
0.3114311	-7.05709717403141e-12\\
0.3115312	7.97404103971984e-12\\
0.3116312	5.93202827112587e-12\\
0.3117312	4.88257568289972e-12\\
0.3118312	2.43280023875206e-12\\
0.3119312	1.20465734047457e-12\\
0.3120312	-1.64928543310919e-13\\
0.3121312	-2.02393204459473e-12\\
0.3122312	-3.70509404775542e-12\\
0.3123312	-5.52578771526071e-12\\
0.3124312	-7.78788337699101e-12\\
0.3125313	8.59620029984034e-12\\
0.3126313	6.79125597353289e-12\\
0.3127313	4.73543690014735e-12\\
0.3128313	3.19027371375147e-12\\
0.3129313	9.3349252900872e-13\\
0.3130313	-2.4084312370201e-13\\
0.3131313	-2.52218908236072e-12\\
0.3132313	-4.08337629448068e-12\\
0.3133313	-7.08046521278916e-12\\
0.3134313	-8.65259911407175e-12\\
0.3135314	9.36175640843848e-12\\
0.3136314	7.49166475374897e-12\\
0.3137314	5.73400833209909e-12\\
0.3138314	4.01882201791069e-12\\
0.3139314	1.2938101293416e-12\\
0.3140314	-4.75499437442112e-13\\
0.3141314	-3.30559973026789e-12\\
0.3142314	-5.19484806759272e-12\\
0.3143314	-7.1233079779015e-12\\
0.3144314	-1.00525899453599e-11\\
0.3145315	1.04582646359942e-11\\
0.3146315	7.93830101623538e-12\\
0.3147315	5.64711941750755e-12\\
0.3148315	3.69856030982721e-12\\
0.3149315	1.22573384149605e-12\\
0.3150315	-6.18812836445558e-13\\
0.3151315	-2.66292692564562e-12\\
0.3152315	-5.71467978097305e-12\\
0.3153315	-7.56219550675347e-12\\
0.3154315	-1.09734780521545e-11\\
0.3155316	1.19967614850539e-11\\
0.3156316	9.47839273050244e-12\\
0.3157316	7.21702025999679e-12\\
0.3158316	4.52689902055408e-12\\
0.3159316	1.74329018805787e-12\\
0.3160316	-7.77357268504341e-13\\
0.3161316	-3.65722364849267e-12\\
0.3162316	-6.4969338634125e-12\\
0.3163316	-8.87537153248022e-12\\
0.3164316	-1.13494913684528e-11\\
0.3165317	1.27764101636043e-11\\
0.3166317	9.79583359313728e-12\\
0.3167317	7.18470359576634e-12\\
0.3168317	4.47570982571355e-12\\
0.3169317	2.2244318728979e-12\\
0.3170317	-9.90463889443354e-13\\
0.3171317	-3.5670258695729e-12\\
0.3172317	-6.87981711200385e-12\\
0.3173317	-9.2797135908373e-12\\
0.3174317	-1.30937008900728e-11\\
0.3175318	1.43934923946639e-11\\
0.3176318	1.11603294774305e-11\\
0.3177318	8.68027773970873e-12\\
0.3178318	4.72400453266892e-12\\
0.3179318	2.0871094612739e-12\\
0.3180318	-1.40966217004716e-12\\
0.3181318	-3.92020427621501e-12\\
0.3182318	-7.57283310173518e-12\\
0.3183318	-1.04700685487575e-11\\
0.3184318	-1.36884138633644e-11\\
0.3185319	1.5801241290925e-11\\
0.3186319	1.21384490983357e-11\\
0.3187319	9.08638419395297e-12\\
0.3188319	5.6748765851216e-12\\
0.3189319	1.96090196506157e-12\\
0.3190319	-9.71187025538314e-13\\
0.3191319	-5.0094277235393e-12\\
0.3192319	-8.01401253741803e-12\\
0.3193319	-1.18170520103491e-11\\
0.3194319	-1.52223361487856e-11\\
0.319532	1.74346928483129e-11\\
0.319632	1.38815510515323e-11\\
0.319732	9.49043136075407e-12\\
0.319832	6.57328129585085e-12\\
0.319932	2.47159183056615e-12\\
0.320032	-1.44335214065345e-12\\
0.320132	-4.77021935457901e-12\\
0.320232	-9.07737789930085e-12\\
0.320332	-1.29026384997917e-11\\
0.320432	-1.67529961706667e-11\\
0.3205321	1.90229040981612e-11\\
0.3206321	1.51137797023811e-11\\
0.3207321	1.08496868614198e-11\\
0.3208321	6.84955211530813e-12\\
0.3209321	2.76444187450421e-12\\
0.3210321	-1.72216660725112e-12\\
0.3211321	-5.89411188245729e-12\\
0.3212321	-1.00022729771911e-11\\
0.3213321	-1.42642917851362e-11\\
0.3214321	-1.88642931621576e-11\\
0.3215322	2.02195679336554e-11\\
0.3216322	1.59520576611884e-11\\
0.3217322	1.20020587825192e-11\\
0.3218322	7.32239163085609e-12\\
0.3219322	2.90082646437802e-12\\
0.3220322	-1.23962331200579e-12\\
0.3221322	-6.04040541441972e-12\\
0.3222322	-1.14071307213043e-11\\
0.3223322	-1.62092730010834e-11\\
0.3224322	-2.12798660426995e-11\\
0.3225323	2.31922762397395e-11\\
0.3226323	1.76992787449248e-11\\
0.3227323	1.36647283530581e-11\\
0.3228323	8.40446365947485e-12\\
0.3229323	3.27231319467811e-12\\
0.3230323	-1.3395873833485e-12\\
0.3231323	-7.00047552681916e-12\\
0.3232323	-1.22406395127711e-11\\
0.3233323	-1.75510936130553e-11\\
0.3234323	-2.33832508037153e-11\\
0.3235324	2.53204691795414e-11\\
0.3236324	1.97617502221663e-11\\
0.3237324	1.45726991222802e-11\\
0.3238324	9.46367432898656e-12\\
0.3239324	4.18631148592829e-12\\
0.3240324	-1.46613352968589e-12\\
0.3241324	-7.6584389278536e-12\\
0.3242324	-1.35130686268174e-11\\
0.3243324	-1.9109821209209e-11\\
0.3244324	-2.54854760867407e-11\\
0.3245325	2.7162487453124e-11\\
0.3246325	2.18523727835057e-11\\
0.3247325	1.59198292010398e-11\\
0.3248325	1.05037924455744e-11\\
0.3249325	3.7880286006794e-12\\
0.3250325	-1.99849512108705e-12\\
0.3251325	-8.58123692919633e-12\\
0.3252325	-1.46397040430435e-11\\
0.3253325	-2.08070732444067e-11\\
0.3254325	-2.76698087422892e-11\\
0.3255326	2.98631455198503e-11\\
0.3256326	2.36521107866509e-11\\
0.3257326	1.72760404724929e-11\\
0.3258326	1.13392517336902e-11\\
0.3259326	4.49473100510016e-12\\
0.3260326	-2.55546579617995e-12\\
0.3261326	-9.0598096262113e-12\\
0.3262326	-1.62168918608501e-11\\
0.3263326	-2.31750146518692e-11\\
0.3264326	-3.00317779669272e-11\\
0.3265327	3.31848931226471e-11\\
0.3266327	2.61140229271367e-11\\
0.3267327	1.916618393179e-11\\
0.3268327	1.24508293029114e-11\\
0.3269327	5.13022703762671e-12\\
0.3270327	-2.58010797514301e-12\\
0.3271327	-1.04109781539285e-11\\
0.3272327	-1.80390644167528e-11\\
0.3273327	-2.50864841154833e-11\\
0.3274327	-3.31203450760044e-11\\
0.3275328	3.6357964263643e-11\\
0.3276328	2.86065230528041e-11\\
0.3277328	2.15085092562063e-11\\
0.3278328	1.37214341654874e-11\\
0.3279328	4.96009860076897e-12\\
0.3280328	-3.00294067972942e-12\\
0.3281328	-1.13369007700281e-11\\
0.3282328	-1.91522993651775e-11\\
0.3283328	-2.75004772761904e-11\\
0.3284328	-3.63731177321973e-11\\
0.3285329	3.99579523318433e-11\\
0.3286329	3.21056545488329e-11\\
0.3287329	2.3123873734942e-11\\
0.3288329	1.42645003644584e-11\\
0.3289329	5.84154234620877e-12\\
0.3290329	-2.76837175586079e-12\\
0.3291329	-1.21254859485628e-11\\
0.3292329	-2.17264058323684e-11\\
0.3293329	-3.10035840045669e-11\\
0.3294329	-4.03248015304992e-11\\
0.329533	4.40330326962766e-11\\
0.329633	3.46600193343186e-11\\
0.329733	2.5540759331899e-11\\
0.329833	1.85714888546105e-11\\
0.329933	3.71576801581419e-12\\
0.330033	-4.9949763377251e-12\\
0.330133	-9.46090457360068e-12\\
0.330233	-2.15132131774459e-11\\
0.330333	-3.29135801068852e-11\\
0.330433	-4.53536058569201e-11\\
0.3305331	5.27176074722433e-11\\
0.3306331	3.43666354021257e-11\\
0.3307331	3.03358788369485e-11\\
0.3308331	1.92197939590758e-11\\
0.3309331	9.68577332020307e-12\\
0.3310331	4.74730658904239e-13\\
0.3311331	-9.5983105005557e-12\\
0.3312331	-2.16436230120561e-11\\
0.3313331	-3.66961786021095e-11\\
0.3314331	-4.57150455423388e-11\\
0.3315332	5.358450656253e-11\\
0.3316332	4.51114379650065e-11\\
0.3317332	3.02656758992471e-11\\
0.3318332	1.83979577839127e-11\\
0.3319332	8.93800458924804e-12\\
0.3320332	-8.60484946168497e-12\\
0.3321332	-1.46410224337323e-11\\
0.3322332	-2.9500045879401e-11\\
0.3323332	-4.34299205400449e-11\\
0.3324332	-5.65964679906817e-11\\
0.3325333	5.50040738121955e-11\\
0.3326333	4.43444195446575e-11\\
0.3327333	3.44603890897326e-11\\
0.3328333	2.55215462749859e-11\\
0.3329333	7.78295071958619e-12\\
0.3330333	-8.4141638934923e-12\\
0.3331333	-1.26417069561536e-11\\
0.3332333	-3.43840428931153e-11\\
0.3333333	-4.30372979295961e-11\\
0.3334333	-5.79086617694743e-11\\
0.3335334	6.77955301642432e-11\\
0.3336334	5.41765566548178e-11\\
0.3337334	3.69695930905198e-11\\
0.3338334	2.7230373509228e-11\\
0.3339334	6.10713837662059e-12\\
0.3340334	-5.15863608137345e-12\\
0.3341334	-1.5231502359023e-11\\
0.3342334	-3.26813019226361e-11\\
0.3343334	-4.59824187393535e-11\\
0.3344334	-6.35130281836449e-11\\
0.3345335	6.54739733160466e-11\\
0.3346335	5.61037289217338e-11\\
0.3347335	3.79662776098331e-11\\
0.3348335	2.30760013453636e-11\\
0.3349335	1.35473317066921e-11\\
0.3350335	-8.40446570167129e-12\\
0.3351335	-2.04625006751445e-11\\
0.3352335	-4.02074617598545e-11\\
0.3353335	-5.51168143785898e-11\\
0.3354335	-7.25639925276971e-11\\
0.3355336	7.34152699875432e-11\\
0.3356336	6.06818307184547e-11\\
0.3357336	4.39356178967966e-11\\
0.3358336	2.62276288909996e-11\\
0.3359336	1.0717022663867e-11\\
0.3360336	-9.32803659347333e-12\\
0.3361336	-2.05295295906572e-11\\
0.3362336	-3.93987252727873e-11\\
0.3363336	-6.23353192319626e-11\\
0.3364336	-7.5626564167671e-11\\
0.3365337	8.32809662466123e-11\\
0.3366337	6.24972387718319e-11\\
0.3367337	5.31946508360254e-11\\
0.3368337	2.95446481983366e-11\\
0.3369337	1.58355586198718e-11\\
0.3370337	-3.52650179362619e-12\\
0.3371337	-2.4016714818601e-11\\
0.3372337	-4.09906401830913e-11\\
0.3373337	-5.96832892149298e-11\\
0.3374337	-8.52081907229451e-11\\
0.3375338	9.30662039132831e-11\\
0.3376338	7.07977697226149e-11\\
0.3377338	5.71101417770766e-11\\
0.3378338	3.73853134796767e-11\\
0.3379338	1.71315288852462e-11\\
0.3380338	-8.01574211940401e-12\\
0.3381338	-3.22928229939772e-11\\
0.3382338	-4.9806842637311e-11\\
0.3383338	-7.45347393728984e-11\\
0.3384338	-9.03222596948484e-11\\
0.3385339	1.03155528325701e-10\\
0.3386339	7.61712574627235e-11\\
0.3387339	5.74040332581877e-11\\
0.3388339	3.35431832625881e-11\\
0.3389339	1.14143487650804e-11\\
0.3390339	-1.20194689874023e-11\\
0.3391339	-2.96568555371618e-11\\
0.3392339	-5.42569223022907e-11\\
0.3393339	-7.84382375776372e-11\\
0.3394339	-1.04677751530121e-10\\
0.339534	1.08794431552843e-10\\
0.339634	8.36819511017414e-11\\
0.339734	5.99590377517543e-11\\
0.339834	3.57262772493418e-11\\
0.339934	1.92313705752064e-11\\
0.340034	-1.1129742638708e-11\\
0.340134	-3.68117544063127e-11\\
0.340234	-5.91188495422214e-11\\
0.340334	-7.92035589234307e-11\\
0.340434	-1.08065605685455e-10\\
0.3405341	1.1940852503962e-10\\
0.3406341	9.28990927356855e-11\\
0.3407341	6.55612168015945e-11\\
0.3408341	4.70182746056325e-11\\
0.3409341	1.7052349532827e-11\\
0.3410341	-4.39456619626632e-12\\
0.3411341	-3.72193576211378e-11\\
0.3412341	-6.11565668407214e-11\\
0.3413341	-9.57771625376168e-11\\
0.3414341	-1.20487302369266e-10\\
0.3415342	1.35227411604858e-10\\
0.3416342	1.05277325770339e-10\\
0.3417342	7.80417835322361e-11\\
0.3418342	4.47865991255042e-11\\
0.3419342	1.69487221701953e-11\\
0.3420342	-1.38624723272683e-11\\
0.3421342	-3.58638830506614e-11\\
0.3422342	-6.70973734608277e-11\\
0.3423342	-9.54284538692038e-11\\
0.3424342	-1.28544953408851e-10\\
0.3425343	1.41696649754591e-10\\
0.3426343	1.09375023164685e-10\\
0.3427343	8.03069301799609e-11\\
0.3428343	4.75289074658197e-11\\
0.3429343	2.42619498913555e-11\\
0.3430343	-6.0871058898699e-12\\
0.3431343	-3.99241874301614e-11\\
0.3432343	-7.34665846021288e-11\\
0.3433343	-1.12741536450274e-10\\
0.3434343	-1.43584809420374e-10\\
0.3435344	1.52188710804857e-10\\
0.3436344	1.24424250592882e-10\\
0.3437344	8.87722220155296e-11\\
0.3438344	6.01773672329032e-11\\
0.3439344	2.37831618412959e-11\\
0.3440344	-1.50667019550681e-11\\
0.3441344	-5.08268249807976e-11\\
0.3442344	-8.7748596739191e-11\\
0.3443344	-1.19878683680643e-10\\
0.3444344	-1.61057506151412e-10\\
0.3445345	1.69552533634248e-10\\
0.3446345	1.32794911196076e-10\\
0.3447345	9.67470533007786e-11\\
0.3448345	5.84091795858221e-11\\
0.3449345	2.49955311372343e-11\\
0.3450345	-1.60640426877462e-11\\
0.3451345	-4.71224871731631e-11\\
0.3452345	-9.03139332593329e-11\\
0.3453345	-1.27552078240647e-10\\
0.3454345	-1.70528555855462e-10\\
0.3455346	1.87071223102357e-10\\
0.3456346	1.41925674215933e-10\\
0.3457346	1.07344845211581e-10\\
0.3458346	7.25421146356752e-11\\
0.3459346	2.69612463593954e-11\\
0.3460346	-9.721911989368e-12\\
0.3461346	-5.75977212159771e-11\\
0.3462346	-9.65210268007448e-11\\
0.3463346	-1.4610942841622e-10\\
0.3464346	-1.8574153476267e-10\\
0.3465347	1.99428622074183e-10\\
0.3466347	1.5632560970393e-10\\
0.3467347	1.16526104886635e-10\\
0.3468347	7.16253909820359e-11\\
0.3469347	3.34666435460474e-11\\
0.3470347	-1.58572525295613e-11\\
0.3471347	-6.40018753056068e-11\\
0.3472347	-1.08369424680699e-10\\
0.3473347	-1.56106869272888e-10\\
0.3474347	-2.04104080973888e-10\\
0.3475348	2.24317945667844e-10\\
0.3476348	1.70283030616325e-10\\
0.3477348	1.26914178446354e-10\\
0.3478348	7.83690751928064e-11\\
0.3479348	2.90720213405745e-11\\
0.3480348	-1.62841262623206e-11\\
0.3481348	-6.27359991484545e-11\\
0.3482348	-1.15047750928016e-10\\
0.3483348	-1.67709075285223e-10\\
0.3484348	-2.24933211962516e-10\\
0.3485349	2.35359404559513e-10\\
0.3486349	1.91951919155831e-10\\
0.3487349	1.43057530387492e-10\\
0.3488349	8.55893038801997e-11\\
0.3489349	3.67469306800918e-11\\
0.3490349	-1.59811971734434e-11\\
0.3491349	-7.4815898552902e-11\\
0.3492349	-1.31685101598068e-10\\
0.3493349	-1.88221726019847e-10\\
0.3494349	-2.4576155302177e-10\\
0.349535	2.57028794921994e-10\\
0.349635	2.09521483293001e-10\\
0.349735	1.48847687254645e-10\\
0.349835	9.48820686055103e-11\\
0.349935	3.78072943232281e-11\\
0.350035	-2.18837489497853e-11\\
0.350135	-8.3385724025534e-11\\
0.350235	-1.35578597082312e-10\\
0.350335	-1.97025380019767e-10\\
0.350435	-2.65969852817683e-10\\
0.3505351	2.82335374955016e-10\\
0.3506351	2.3021087847705e-10\\
0.3507351	1.67838597322579e-10\\
0.3508351	1.08275048960895e-10\\
0.3509351	4.49075843108075e-11\\
0.3510351	-1.85432488320533e-11\\
0.3511351	-8.80213238044604e-11\\
0.3512351	-1.49132538620773e-10\\
0.3513351	-2.17142404466855e-10\\
0.3514351	-2.86973617178582e-10\\
0.3515352	3.14025024692887e-10\\
0.3516352	2.42868922537129e-10\\
0.3517352	1.77251090866491e-10\\
0.3518352	1.13645043625221e-10\\
0.3519352	4.88794918575937e-11\\
0.3520352	-1.98591336924387e-11\\
0.3521352	-9.50241424581928e-11\\
0.3522352	-1.68706026606362e-10\\
0.3523352	-2.42629888389293e-10\\
0.3524352	-3.18152848616487e-10\\
0.3525353	3.40123666020932e-10\\
0.3526353	2.6499776067349e-10\\
0.3527353	1.96485377691432e-10\\
0.3528353	1.24727933588516e-10\\
0.3529353	5.02480292075523e-11\\
0.3530353	-2.60478552247499e-11\\
0.3531353	-1.02866653847841e-10\\
0.3532353	-1.78525987043618e-10\\
0.3533353	-2.60951418026151e-10\\
0.3534353	-3.37673692162705e-10\\
0.3535354	3.64676235904823e-10\\
0.3536354	2.95059864453655e-10\\
0.3537354	2.11005085000199e-10\\
0.3538354	1.36588806806887e-10\\
0.3539354	5.62968359944923e-11\\
0.3540354	-2.49732543807857e-11\\
0.3541354	-1.11909237803767e-10\\
0.3542354	-1.98781327945892e-10\\
0.3543354	-2.79439253366483e-10\\
0.3544354	-3.67309316805491e-10\\
0.3545355	3.94578101813839e-10\\
0.3546355	3.20969499554582e-10\\
0.3547355	2.32498209676605e-10\\
0.3548355	1.27461883997243e-10\\
0.3549355	5.45966009625172e-11\\
0.3550355	-5.6920073838204e-11\\
0.3551355	-1.47466013087749e-10\\
0.3552355	-2.56971432671736e-10\\
0.3553355	-3.24915774705314e-10\\
0.3554355	-3.90324575571675e-10\\
0.3555356	4.43438738989077e-10\\
0.3556356	3.7571283009654e-10\\
0.3557356	2.96263537235219e-10\\
0.3558356	1.67913170835091e-10\\
0.3559356	5.39539261250159e-11\\
0.3560356	-8.18488650472385e-11\\
0.3561356	-1.75253453215491e-10\\
0.3562356	-2.6153837781465e-10\\
0.3563356	-3.75499150965185e-10\\
0.3564356	-4.51444917947265e-10\\
0.3565357	4.93461664013831e-10\\
0.3566357	4.01085150362373e-10\\
0.3567357	2.46815508203363e-10\\
0.3568357	1.98323350770286e-10\\
0.3569357	2.37826656426191e-11\\
0.3570357	-8.12572477023894e-12\\
0.3571357	-1.28210673143708e-10\\
0.3572357	-2.66767200506317e-10\\
0.3573357	-3.53572968266828e-10\\
0.3574357	-5.17884723359686e-10\\
0.3575358	5.16370071972517e-10\\
0.3576358	4.20579303724239e-10\\
0.3577358	2.6274621969238e-10\\
0.3578358	2.15734216840562e-10\\
0.3579358	5.29457003955493e-11\\
0.3580358	-5.1674234281731e-11\\
0.3581358	-1.23634088860722e-10\\
0.3582358	-2.87892233327442e-10\\
0.3583358	-3.68853151364382e-10\\
0.3584358	-4.90363662435794e-10\\
0.3585359	5.2445428333275e-10\\
0.3586359	4.62506119530302e-10\\
0.3587359	2.91930142901206e-10\\
0.3588359	1.9114900303872e-10\\
0.3589359	3.91622565268628e-11\\
0.3590359	-8.44497164255789e-11\\
0.3591359	-1.99521770063776e-10\\
0.3592359	-3.25300023978339e-10\\
0.3593359	-4.80437872211331e-10\\
0.3594359	-5.82991969641251e-10\\
0.359536	6.52864971882171e-10\\
0.359636	5.1447569151381e-10\\
0.359736	3.78205226034614e-10\\
0.359836	2.28425141621897e-10\\
0.359936	5.01241973003674e-11\\
0.360036	-7.10874922837425e-11\\
0.360136	-1.48974268088676e-10\\
0.360236	-2.96670708561497e-10\\
0.360336	-4.26677392094225e-10\\
0.360436	-6.50856629062866e-10\\
0.3605361	6.34325109609814e-10\\
0.3606361	5.00417400768095e-10\\
0.3607361	3.40713268416701e-10\\
0.3608361	2.45947311616149e-10\\
0.3609361	1.07514121818744e-10\\
0.3610361	-8.25272915597978e-11\\
0.3611361	-2.31449056019859e-10\\
0.3612361	-3.45849952355145e-10\\
0.3613361	-5.31650917869964e-10\\
0.3614361	-6.94090511411602e-10\\
0.3615362	6.9748366419509e-10\\
0.3616362	5.81366761182771e-10\\
0.3617362	3.77132712239174e-10\\
0.3618362	2.82319020293792e-10\\
0.3619362	9.51686315745024e-11\\
0.3620362	-8.53653669291316e-11\\
0.3621362	-2.59615024465233e-10\\
0.3622362	-4.27192775701328e-10\\
0.3623362	-5.86986658715136e-10\\
0.3624362	-7.3715550747951e-10\\
0.3625363	7.90185152445104e-10\\
0.3626363	5.81298376648625e-10\\
0.3627363	4.92088307114166e-10\\
0.3628363	3.27362340062304e-10\\
0.3629363	9.26815540546605e-11\\
0.3630363	-1.05634305203875e-10\\
0.3631363	-2.60501813723795e-10\\
0.3632363	-4.64068830211646e-10\\
0.3633363	-6.07709420925775e-10\\
0.3634363	-7.82018760105184e-10\\
0.3635364	8.28981192610261e-10\\
0.3636364	6.39336393017369e-10\\
0.3637364	4.52074303908473e-10\\
0.3638364	2.79767734207102e-10\\
0.3639364	1.35794348340859e-10\\
0.3640364	-6.56580484155022e-11\\
0.3641364	-2.09586198033187e-10\\
0.3642364	-4.80166042126848e-10\\
0.3643364	-6.60747343476801e-10\\
0.3644364	-8.33848280408587e-10\\
0.3645365	8.76263115859961e-10\\
0.3646365	6.89726833454215e-10\\
0.3647365	5.68280892964757e-10\\
0.3648365	3.32789032523586e-10\\
0.3649365	1.04974105579955e-10\\
0.3650365	-9.2576327842802e-11\\
0.3651365	-2.36404361039257e-10\\
0.3652365	-5.02176077360039e-10\\
0.3653365	-6.6467585683452e-10\\
0.3654365	-8.97800635809957e-10\\
0.3655366	9.46448332737621e-10\\
0.3656366	7.71009933252676e-10\\
0.3657366	6.08767424165914e-10\\
0.3658366	3.89432514014132e-10\\
0.3659366	1.43633518761514e-10\\
0.3660366	-9.70787120938184e-11\\
0.3661366	-3.00224825089488e-10\\
0.3662366	-5.32390963779716e-10\\
0.3663366	-7.592227349194e-10\\
0.3664366	-9.45419132683447e-10\\
0.3665367	1.0427075148949e-09\\
0.3666367	8.65884977596914e-10\\
0.3667367	6.41480377605759e-10\\
0.3668367	4.08643533399665e-10\\
0.3669367	1.07501758796896e-10\\
0.3670367	-1.20833858001899e-10\\
0.3671367	-3.34262136741405e-10\\
0.3672367	-5.89685450859759e-10\\
0.3673367	-8.43003342750493e-10\\
0.3674367	-1.04910608802675e-09\\
0.3675368	1.12577450128975e-09\\
0.3676368	8.73315085341258e-10\\
0.3677368	7.09669049057688e-10\\
0.3678368	3.84049190344481e-10\\
0.3679368	1.46710249634627e-10\\
0.3680368	-5.10444307265032e-11\\
0.3681368	-3.56856232064384e-10\\
0.3682368	-6.17304529867101e-10\\
0.3683368	-8.77899935418976e-10\\
0.3684368	-1.18307750282629e-09\\
0.3685369	1.21643616073043e-09\\
0.3686369	1.014474708069e-09\\
0.3687369	7.41307261398465e-10\\
0.3688369	4.56870882597181e-10\\
0.3689369	2.22212767446699e-10\\
0.3690369	-1.00502715308023e-10\\
0.3691369	-3.47986955061014e-10\\
0.3692369	-6.55819993146747e-10\\
0.3693369	-9.58443368995286e-10\\
0.3694369	-1.28915294032254e-09\\
0.369537	1.33335500138772e-09\\
0.369637	1.07419819279876e-09\\
0.369737	7.94185021294868e-10\\
0.369837	4.64676262800032e-10\\
0.369937	1.58214929276302e-10\\
0.370037	-1.51466281357593e-10\\
0.370137	-3.89437525684464e-10\\
0.370237	-7.79564295983002e-10\\
0.370337	-1.04449985731956e-09\\
0.370437	-1.40567766487073e-09\\
0.3705371	1.46793301217816e-09\\
0.3706371	1.17964347892772e-09\\
0.3707371	8.38389765923063e-10\\
0.3708371	5.27695770760617e-10\\
0.3709371	2.32343839263668e-10\\
0.3710371	-1.61617385652842e-10\\
0.3711371	-4.66865063861004e-10\\
0.3712371	-7.94794223382515e-10\\
0.3713371	-1.15550977494066e-09\\
0.3714371	-1.45781849222523e-09\\
0.3715372	1.59798087744989e-09\\
0.3716372	1.21794062281268e-09\\
0.3717372	8.77952434710783e-10\\
0.3718372	5.74484566851202e-10\\
0.3719372	2.05344213529725e-10\\
0.3720372	-1.30314215961071e-10\\
0.3721372	-5.31980738269471e-10\\
0.3722372	-8.97781432583709e-10\\
0.3723372	-1.2244700026056e-09\\
0.3724372	-1.60741931072052e-09\\
0.3725373	1.74201212190059e-09\\
0.3726373	1.39465020050156e-09\\
0.3727373	1.01348894494856e-09\\
0.3728373	6.08765924117998e-10\\
0.3729373	1.92142609711354e-10\\
0.3730373	-1.2328687631338e-10\\
0.3731373	-5.22986958373297e-10\\
0.3732373	-8.90971768368817e-10\\
0.3733373	-1.30979625414954e-09\\
0.3734373	-1.76054723436455e-09\\
0.3735374	1.85603670637986e-09\\
0.3736374	1.43491162474165e-09\\
0.3737374	1.04772255303011e-09\\
0.3738374	6.19347022452754e-10\\
0.3739374	2.76176112361248e-10\\
0.3740374	-1.53876314576591e-10\\
0.3741374	-5.41364351242489e-10\\
0.3742374	-9.55300697379723e-10\\
0.3743374	-1.46314729613812e-09\\
0.3744374	-1.83080590466546e-09\\
0.3745375	1.97478296367314e-09\\
0.3746375	1.52921560516491e-09\\
0.3747375	1.13582672098007e-09\\
0.3748375	7.35052160523268e-10\\
0.3749375	2.6893585015203e-10\\
0.3750375	-1.188604723543e-10\\
0.3751375	-5.83047490387671e-10\\
0.3752375	-1.07669846595351e-09\\
0.3753375	-1.55123935810346e-09\\
0.3754375	-2.05643889753231e-09\\
0.3755376	2.09933019709382e-09\\
0.3756376	1.72578839641255e-09\\
0.3757376	1.28258140566495e-09\\
0.3758376	7.26671521675444e-10\\
0.3759376	3.1672874403494e-10\\
0.3760376	-1.86858981846699e-10\\
0.3761376	-6.21975460826825e-10\\
0.3762376	-1.12476588654854e-09\\
0.3763376	-1.62962643282093e-09\\
0.3764376	-2.16919379323038e-09\\
0.3765377	2.33318661927665e-09\\
0.3766377	1.87162765234034e-09\\
0.3767377	1.28838773504059e-09\\
0.3768377	8.57975636175089e-10\\
0.3769377	3.5671274983308e-10\\
0.3770377	-2.37256111892264e-10\\
0.3771377	-7.43951860944329e-10\\
0.3772377	-1.28155025068584e-09\\
0.3773377	-1.76637089449507e-09\\
0.3774377	-2.31286626309335e-09\\
0.3775378	2.46889826074556e-09\\
0.3776378	2.00428053331609e-09\\
0.3777378	1.44623671231078e-09\\
0.3778378	8.87896107657458e-10\\
0.3779378	3.24311117457649e-10\\
0.3780378	-2.47531422225764e-10\\
0.3781378	-8.28698820285953e-10\\
0.3782378	-1.31830106650955e-09\\
0.3783378	-1.9134792833336e-09\\
0.3784378	-2.50939415330064e-09\\
0.3785379	2.72732370379131e-09\\
0.3786379	2.0965037541296e-09\\
0.3787379	1.59177642323604e-09\\
0.3788379	9.26021920277277e-10\\
0.3789379	4.14159737971778e-10\\
0.3790379	-2.2683940012827e-10\\
0.3791379	-8.77941532949518e-10\\
0.3792379	-1.41803740407939e-09\\
0.3793379	-2.12393032675284e-09\\
0.3794379	-2.7703240077382e-09\\
0.379538	2.95175689728373e-09\\
0.379638	2.2559924028713e-09\\
0.379738	1.70866921284617e-09\\
0.379838	1.04360728891105e-09\\
0.379938	3.96788070065532e-10\\
0.380038	-2.93632981102597e-10\\
0.380138	-8.8731376423918e-10\\
0.380238	-1.54171285666246e-09\\
0.380338	-2.212076785418e-09\\
0.380438	-2.95142719790829e-09\\
0.3805381	3.15912522216074e-09\\
0.3806381	2.48240912526475e-09\\
0.3807381	1.79147466587877e-09\\
0.3808381	1.14233125606336e-09\\
0.3809381	3.93278177493635e-10\\
0.3810381	-2.9508223989283e-10\\
0.3811381	-9.59831375079052e-10\\
0.3812381	-1.73572097460998e-09\\
0.3813381	-2.45515980375893e-09\\
0.3814381	-3.14820016030729e-09\\
0.3815382	3.35053243743361e-09\\
0.3816382	2.68398729641201e-09\\
0.3817382	1.96833618750902e-09\\
0.3818382	1.18309081368478e-09\\
0.3819382	5.10187579249449e-10\\
0.3820382	-2.65998570826015e-10\\
0.3821382	-1.0586402424172e-09\\
0.3822382	-1.87844360248837e-09\\
0.3823382	-2.63363434511131e-09\\
0.3824382	-3.42994357308696e-09\\
0.3825383	3.58345549587857e-09\\
0.3826383	2.75602103205069e-09\\
0.3827383	2.38578765012105e-09\\
0.3828383	8.77147715355395e-10\\
0.3829383	6.37059785566635e-10\\
0.3830383	7.50631716072564e-11\\
0.3831383	-1.39670748577903e-09\\
0.3832383	-2.36350754484298e-09\\
0.3833383	-2.40796762282329e-09\\
0.3834383	-4.11007880746043e-09\\
0.3835384	3.40794030538952e-09\\
0.3836384	2.72359042523956e-09\\
0.3837384	2.65803195252656e-09\\
0.3838384	1.64198416777346e-09\\
0.3839384	1.08880986475143e-10\\
0.3840384	-5.05113773724419e-10\\
0.3841384	-7.61091114610812e-10\\
0.3842384	-2.21738147049462e-09\\
0.3843384	-3.42953925631592e-09\\
0.3844384	-3.95032737098878e-09\\
0.3845385	3.76917378775224e-09\\
0.3846385	3.05090245276453e-09\\
0.3847385	2.38307479760898e-09\\
0.3848385	1.22425362187283e-09\\
0.3849385	1.03587198801297e-09\\
0.3850385	-7.17750834678191e-10\\
0.3851385	-1.56939330454853e-09\\
0.3852385	-2.04891551511613e-09\\
0.3853385	-3.68324300594781e-09\\
0.3854385	-3.99635048859738e-09\\
0.3855386	4.27883772098799e-09\\
0.3856386	3.11965552349247e-09\\
0.3857386	2.72812431022576e-09\\
0.3858386	1.592239781429e-09\\
0.3859386	2.03030921068492e-10\\
0.3860386	5.45766958102295e-11\\
0.3861386	-1.35597712767367e-09\\
0.3862386	-2.52840112453627e-09\\
0.3863386	-3.95936526099983e-09\\
0.3864386	-4.14242186429561e-09\\
0.3865387	4.95767342022918e-09\\
0.3866387	3.87885890464415e-09\\
0.3867387	2.58668965153613e-09\\
0.3868387	1.60025888156987e-09\\
0.3869387	4.41863771700329e-10\\
0.3870387	-3.62977058999958e-10\\
0.3871387	-1.28550595233749e-09\\
0.3872387	-2.79370857198848e-09\\
0.3873387	-3.35229616288402e-09\\
0.3874387	-5.42268774003131e-09\\
0.3875388	4.85170479919833e-09\\
0.3876388	4.46855599528159e-09\\
0.3877388	3.20982916353178e-09\\
0.3878388	1.62745735876428e-09\\
0.3879388	2.76756160740368e-10\\
0.3880388	-2.83557986986704e-10\\
0.3881388	-1.49134946167591e-09\\
0.3882388	-2.78104492776668e-09\\
0.3883388	-3.58361482524251e-09\\
0.3884388	-5.32655473569656e-09\\
0.3885389	5.72458031954277e-09\\
0.3886389	4.91991307026082e-09\\
0.3887389	2.91401055954425e-09\\
0.3888389	2.29346815887668e-09\\
0.3889389	6.48450491002803e-10\\
0.3890389	-4.27289398091321e-10\\
0.3891389	-1.33639088207085e-09\\
0.3892389	-3.47786639678244e-09\\
0.3893389	-4.24708209855081e-09\\
0.3894389	-6.03573837546119e-09\\
0.389539	5.82850138443856e-09\\
0.389639	4.93414129020854e-09\\
0.389739	3.86805125212816e-09\\
0.389839	2.25339404427705e-09\\
0.389939	7.17096859106578e-10\\
0.390039	-1.1012872146339e-10\\
0.390139	-1.59376669847666e-09\\
0.390239	-3.09547642900199e-09\\
0.390339	-4.97307243367377e-09\\
0.390439	-6.58050404618582e-09\\
0.3905391	6.75620511988474e-09\\
0.3906391	4.74271969050019e-09\\
0.3907391	3.96160608354471e-09\\
0.3908391	2.07458471820736e-09\\
0.3909391	7.47344174392706e-10\\
0.3910391	-3.50437937782916e-10\\
0.3911391	-1.54507400667095e-09\\
0.3912391	-3.15884543565861e-09\\
0.3913391	-5.5099814398607e-09\\
0.3914391	-6.91263789742366e-09\\
0.3915392	7.37646314689921e-09\\
0.3916392	5.05137952213491e-09\\
0.3917392	3.75766182839771e-09\\
0.3918392	2.19766582709724e-09\\
0.3919392	1.07792809487389e-09\\
0.3920392	-8.90812208720337e-10\\
0.3921392	-1.99359144427078e-09\\
0.3922392	-3.51119963225982e-09\\
0.3923392	-5.72015833961645e-09\\
0.3924392	-6.89269857596236e-09\\
0.3925393	7.85554348870302e-09\\
0.3926393	6.07029551060942e-09\\
0.3927393	4.53148190889333e-09\\
0.3928393	2.98426141307276e-09\\
0.3929393	1.17819537633068e-09\\
0.3930393	-1.13272957940475e-09\\
0.3931393	-2.19007871386789e-09\\
0.3932393	-4.23094639748559e-09\\
0.3933393	-5.4879330259848e-09\\
0.3934393	-7.18912202251482e-09\\
0.3935394	7.76706101995525e-09\\
0.3936394	6.63290649662029e-09\\
0.3937394	4.39842117285275e-09\\
0.3938394	2.85382669058809e-09\\
0.3939394	7.93978453730165e-10\\
0.3940394	-9.81610752317369e-10\\
0.3941394	-2.66874703456002e-09\\
0.3942394	-4.45853154699624e-09\\
0.3943394	-6.53733660217412e-09\\
0.3944394	-8.08678161110313e-09\\
0.3945395	8.29261051927405e-09\\
0.3946395	6.40575606144014e-09\\
0.3947395	4.53299759013703e-09\\
0.3948395	3.51197422749153e-09\\
0.3949395	1.18519958430818e-09\\
0.3950395	-5.99913653909906e-10\\
0.3951395	-2.99102902558482e-09\\
0.3952395	-5.13086146876687e-09\\
0.3953395	-7.15715247050913e-09\\
0.3954395	-8.20264501070686e-09\\
0.3955396	9.51553653753823e-09\\
0.3956396	7.19170273073139e-09\\
0.3957396	5.48156581800865e-09\\
0.3958396	3.27263383450085e-09\\
0.3959396	1.45753985400163e-09\\
0.3960396	-1.06593243997047e-09\\
0.3961396	-3.39482308453834e-09\\
0.3962396	-4.62096998990122e-09\\
0.3963396	-6.8309831085904e-09\\
0.3964396	-9.10621833364114e-09\\
0.3965397	9.81014403045322e-09\\
0.3966397	7.32879825490057e-09\\
0.3967397	5.57088542310624e-09\\
0.3968397	3.47633306125791e-09\\
0.3969397	9.90454241626276e-10\\
0.3970397	-9.36025769913973e-10\\
0.3971397	-3.34694282315071e-09\\
0.3972397	-5.28066705553915e-09\\
0.3973397	-7.77007597022386e-09\\
0.3974397	-9.84252743066807e-09\\
0.3975398	1.03285934063314e-08\\
0.3976398	8.18707080125006e-09\\
0.3977398	6.41481398555577e-09\\
0.3978398	4.00682168390579e-09\\
0.3979398	9.63748933088815e-10\\
0.3980398	-7.08065164660432e-10\\
0.3981398	-2.9965698069602e-09\\
0.3982398	-5.88397454110573e-09\\
0.3983398	-8.34672136008213e-09\\
0.3984398	-1.03554566479973e-08\\
0.3985399	1.15876518987784e-08\\
0.3986399	8.76537734756377e-09\\
0.3987399	6.52128118866502e-09\\
0.3988399	3.90818831377121e-09\\
0.3989399	1.98486113352561e-09\\
0.3990399	-1.1839715857026e-09\\
0.3991399	-3.52758604575329e-09\\
0.3992399	-5.96923442446051e-09\\
0.3993399	-8.42611588662556e-09\\
0.3994399	-1.18093474851463e-08\\
0.39954	1.21573900499668e-08\\
0.39964	9.3904049219655e-09\\
0.39974	7.00161472376448e-09\\
0.39984	4.10453061686425e-09\\
0.39994	1.81889320185341e-09\\
0.40004	-7.29297771479009e-10\\
};
\addplot [color=mycolor1,solid,forget plot]
  table[row sep=crcr]{%
0.40004	-7.29297771479009e-10\\
0.40014	-3.40775346049232e-09\\
0.40024	-7.07786597950802e-09\\
0.40034	-9.5946782788757e-09\\
0.40044	-1.18068539537233e-08\\
0.4005401	1.245381626774e-08\\
0.4006401	1.05198034892548e-08\\
0.4007401	7.38419022108805e-09\\
0.4008401	4.22414101960977e-09\\
0.4009401	2.22335250297384e-09\\
0.4010401	-1.42791586540958e-09\\
0.4011401	-4.53281068322348e-09\\
0.4012401	-6.88785363640865e-09\\
0.4013401	-1.02829102524571e-08\\
0.4014401	-1.35011586233774e-08\\
0.4015402	1.36373340953178e-08\\
0.4016402	1.06513234673228e-08\\
0.4017402	7.53426652750494e-09\\
0.4018402	4.53005696843256e-09\\
0.4019402	1.88943410830283e-09\\
0.4020402	-1.12998518875251e-09\\
0.4021402	-4.26367437710018e-09\\
0.4022402	-7.24016505429526e-09\\
0.4023402	-1.07810146973003e-08\\
0.4024402	-1.36007742151945e-08\\
0.4025403	1.46187843618611e-08\\
0.4026403	1.13397070943336e-08\\
0.4027403	8.68174241264869e-09\\
0.4028403	4.95869904772878e-09\\
0.4029403	1.49155602105489e-09\\
0.4030403	-1.39150427291845e-09\\
0.4031403	-4.35506290492188e-09\\
0.4032403	-8.05643103618525e-09\\
0.4033403	-1.11456165529517e-08\\
0.4034403	-1.52652905377124e-08\\
0.4035404	1.61746981745814e-08\\
0.4036404	1.23229451634219e-08\\
0.4037404	8.55843160056657e-09\\
0.4038404	5.26817873439805e-09\\
0.4039404	1.84671395103003e-09\\
0.4040404	-1.30389490957811e-09\\
0.4041404	-4.77400536203626e-09\\
0.4042404	-8.14636571741523e-09\\
0.4043404	-1.19960806002281e-08\\
0.4044404	-1.58905762584829e-08\\
0.4045405	1.71732353849638e-08\\
0.4046405	1.27593576771876e-08\\
0.4047405	9.64629860624594e-09\\
0.4048405	6.29770210546637e-09\\
0.4049405	2.18506548792241e-09\\
0.4050405	-1.21222513364133e-09\\
0.4051405	-5.40685949138031e-09\\
0.4052405	-8.90356746750115e-09\\
0.4053405	-1.31990834668222e-08\\
0.4054405	-1.67821106187085e-08\\
0.4055406	1.79121162477078e-08\\
0.4056406	1.45767887959822e-08\\
0.4057406	1.05379279495207e-08\\
0.4058406	6.33932425130046e-09\\
0.4059406	2.53298031074184e-09\\
0.4060406	-1.32085273881616e-09\\
0.4061406	-5.65363855731249e-09\\
0.4062406	-9.88851902878479e-09\\
0.4063406	-1.44402772092253e-08\\
0.4064406	-1.87153005679758e-08\\
0.4065407	1.95696712420104e-08\\
0.4066407	1.49350215875033e-08\\
0.4067407	1.14103129713813e-08\\
0.4068407	6.62312102557572e-09\\
0.4069407	2.20960354180288e-09\\
0.4070407	-2.18546182886858e-09\\
0.4071407	-5.90863984389206e-09\\
0.4072407	-1.02977998836518e-08\\
0.4073407	-1.46820777636308e-08\\
0.4074407	-1.9381837797032e-08\\
0.4075408	2.07699342438458e-08\\
0.4076408	1.58023171787508e-08\\
0.4077408	1.16128477892913e-08\\
0.4078408	6.91664032992367e-09\\
0.4079408	2.43777286204616e-09\\
0.4080408	-2.09067389875944e-09\\
0.4081408	-6.9265776525898e-09\\
0.4082408	-1.131873559896e-08\\
0.4083408	-1.55068251069673e-08\\
0.4084408	-2.07213646120225e-08\\
0.4085409	2.22624876190097e-08\\
0.4086409	1.76467639034436e-08\\
0.4087409	1.23701850663835e-08\\
0.4088409	7.23927281748998e-09\\
0.4089409	3.06990600908857e-09\\
0.4090409	-2.31263955902117e-09\\
0.4091409	-7.07365097659535e-09\\
0.4092409	-1.23689380552086e-08\\
0.4093409	-1.73447931512208e-08\\
0.4094409	-2.21379505907504e-08\\
0.409541	2.37175326522551e-08\\
0.409641	1.82428856456904e-08\\
0.409741	1.3600384004761e-08\\
0.409841	8.69194136197265e-09\\
0.409941	3.42923322724231e-09\\
0.410041	-2.26626202786545e-09\\
0.410141	-7.46322189205256e-09\\
0.410241	-1.32204384210854e-08\\
0.410341	-1.85867768068976e-08\\
0.410441	-2.3601133752188e-08\\
0.4105411	2.46364056776666e-08\\
0.4106411	1.95937030472021e-08\\
0.4107411	1.3848516201842e-08\\
0.4108411	8.40225027748387e-09\\
0.4109411	3.2664885223771e-09\\
0.4110411	-2.53696557511729e-09\\
0.4111411	-7.97604583595291e-09\\
0.4112411	-1.40083810020591e-08\\
0.4113411	-1.95812521319033e-08\\
0.4114411	-2.56315498183696e-08\\
0.4115412	2.63774880774476e-08\\
0.4116412	2.09681680057167e-08\\
0.4117412	1.5335620599799e-08\\
0.4118412	9.58495687333749e-09\\
0.4119412	3.83189338246104e-09\\
0.4120412	-2.79720411590356e-09\\
0.4121412	-8.16527731192784e-09\\
0.4122412	-1.4124531747331e-08\\
0.4123412	-2.05163929872798e-08\\
0.4124412	-2.71714629823672e-08\\
0.4125413	2.82971667209253e-08\\
0.4126413	2.2053597146382e-08\\
0.4127413	1.61226034692957e-08\\
0.4128413	9.71733047326617e-09\\
0.4129413	4.06196754010324e-09\\
0.4130413	-2.60820667841166e-09\\
0.4131413	-9.04677979231272e-09\\
0.4132413	-1.4996160844355e-08\\
0.4133413	-2.21875354651513e-08\\
0.4134413	-2.83408210857705e-08\\
0.4135414	3.00051898015352e-08\\
0.4136414	2.32224175791829e-08\\
0.4137414	1.73883634470295e-08\\
0.4138414	1.0828646713687e-08\\
0.4139414	3.88038132355384e-09\\
0.4140414	-3.10777816849017e-09\\
0.4141414	-7.77559112409565e-09\\
0.4142414	-1.87511848392247e-08\\
0.4143414	-2.56510086834444e-08\\
0.4144414	-2.60797880947267e-08\\
0.4145415	3.17334310773187e-08\\
0.4146415	2.19122030398894e-08\\
0.4147415	1.58210863229757e-08\\
0.4148415	1.59027262712497e-08\\
0.4149415	4.61172426809153e-09\\
0.4150415	-5.58531574809673e-09\\
0.4151415	-1.2209740246627e-08\\
0.4152415	-1.27707993138881e-08\\
0.4153415	-2.47655995572194e-08\\
0.4154415	-3.56790569814441e-08\\
0.4155416	3.08167232253298e-08\\
0.4156416	3.01176252843344e-08\\
0.4157416	1.81212965778832e-08\\
0.4158416	7.39206929174785e-09\\
0.4159416	5.06703566258149e-10\\
0.4160416	5.44350814424277e-11\\
0.4161416	-1.13629767799911e-08\\
0.4162416	-2.11312011829529e-08\\
0.4163416	-2.66232876144218e-08\\
0.4164416	-3.51996177171132e-08\\
0.4165417	3.42840470266698e-08\\
0.4166417	2.79925684663201e-08\\
0.4167417	1.66148757339141e-08\\
0.4168417	1.28417552494597e-08\\
0.4169417	9.37690381591638e-09\\
0.4170417	-1.06302274274772e-09\\
0.4171417	-1.57483603466929e-08\\
0.4172417	-2.19363879058049e-08\\
0.4173417	-2.68712781296976e-08\\
0.4174417	-3.77840482564812e-08\\
0.4175418	4.15579682680467e-08\\
0.4176418	2.75602579664219e-08\\
0.4177418	2.59738574068941e-08\\
0.4178418	9.62087843593695e-09\\
0.4179418	1.33683600127277e-09\\
0.4180418	-6.02930211660024e-09\\
0.4181418	-9.61506553716351e-09\\
0.4182418	-1.65444309328799e-08\\
0.4183418	-3.39277722114645e-08\\
0.4184418	-3.88618101969385e-08\\
0.4185419	4.02597979849706e-08\\
0.4186419	2.95288363934276e-08\\
0.4187419	2.00423888718076e-08\\
0.4188419	1.47588646911417e-08\\
0.4189419	6.65057861559681e-09\\
0.4190419	-1.29619812452852e-09\\
0.4191419	-1.60811873609057e-08\\
0.4192419	-2.46900524868837e-08\\
0.4193419	-3.40943477842442e-08\\
0.4194419	-4.12514671501579e-08\\
0.419542	4.1117512781727e-08\\
0.419642	3.82093783887066e-08\\
0.419742	2.67648059392583e-08\\
0.419842	1.98835708540429e-08\\
0.419942	6.79866033262422e-10\\
0.420042	-7.71764633276939e-09\\
0.420142	-1.21657831274324e-08\\
0.420242	-1.95067881864769e-08\\
0.420342	-3.6568280474894e-08\\
0.420442	-4.01632020939624e-08\\
0.4205421	4.29730125581229e-08\\
0.4206421	3.65328745571131e-08\\
0.4207421	2.32123033676312e-08\\
0.4208421	1.6257601941766e-08\\
0.4209421	8.93001175089614e-09\\
0.4210421	-5.49423465751042e-09\\
0.4211421	-1.37238607526846e-08\\
0.4212421	-2.24524937773862e-08\\
0.4213421	-3.83586113671119e-08\\
0.4214421	-4.81054889291122e-08\\
0.4215422	4.78848108456664e-08\\
0.4216422	4.11622300515446e-08\\
0.4217422	3.07041971570055e-08\\
0.4218422	1.9908791993456e-08\\
0.4219422	2.18956243761004e-09\\
0.4220422	-9.02442216441091e-09\\
0.4221422	-2.02885174667433e-08\\
0.4222422	-2.81424507829475e-08\\
0.4223422	-3.91102671255805e-08\\
0.4224422	-4.97002757386678e-08\\
0.4225423	5.63217485462231e-08\\
0.4226423	3.76948174098113e-08\\
0.4227423	3.00192666835564e-08\\
0.4228423	1.68502849971774e-08\\
0.4229423	1.75906663887893e-09\\
0.4230423	-1.66713451377198e-09\\
0.4231423	-1.98249505457826e-08\\
0.4232423	-2.90948454911844e-08\\
0.4233423	-3.58410605291737e-08\\
0.4234423	-5.64115599718806e-08\\
0.4235424	5.2442797791874e-08\\
0.4236424	4.59505962099871e-08\\
0.4237424	2.6692134801054e-08\\
0.4238424	1.83851197009532e-08\\
0.4239424	4.76380768166251e-09\\
0.4240424	-4.20939390982644e-10\\
0.4241424	-1.34015995766068e-08\\
0.4242424	-3.03939356356697e-08\\
0.4243424	-4.75969404618859e-08\\
0.4244424	-5.11927814791924e-08\\
0.4245425	5.9457474447383e-08\\
0.4246425	4.534026270403e-08\\
0.4247425	3.63890933982591e-08\\
0.4248425	1.6489669397296e-08\\
0.4249425	9.54479576972811e-09\\
0.4250425	-1.05255653741476e-08\\
0.4251425	-1.97842389242087e-08\\
0.4252425	-3.42767814097422e-08\\
0.4253425	-5.00314253359946e-08\\
0.4254425	-6.30590233398343e-08\\
0.4255426	6.50608106139705e-08\\
0.4256426	5.03073222767048e-08\\
0.4257426	3.03572090451509e-08\\
0.4258426	1.92697132486708e-08\\
0.4259426	1.11217364623384e-08\\
0.4260426	-9.99210458471622e-09\\
0.4261426	-1.99594216848675e-08\\
0.4262426	-3.46499992726068e-08\\
0.4263426	-4.99157383829218e-08\\
0.4264426	-6.15905998112287e-08\\
0.4265427	6.69362491684256e-08\\
0.4266427	5.58373870092332e-08\\
0.4267427	4.09399508322084e-08\\
0.4268427	2.64823204684794e-08\\
0.4269427	6.72109729329962e-09\\
0.4270427	-4.0688389359167e-09\\
0.4271427	-2.15942737476216e-08\\
0.4272427	-3.154360136981e-08\\
0.4273427	-4.95867679521644e-08\\
0.4274427	-7.13752154359037e-08\\
0.4275428	6.83192131409327e-08\\
0.4276428	5.30273899165823e-08\\
0.4277428	3.71519693828648e-08\\
0.4278428	2.51161144312895e-08\\
0.4279428	1.13617761102192e-08\\
0.4280428	-9.65024881185994e-09\\
0.4281428	-2.34402614852913e-08\\
0.4282428	-3.55096038164027e-08\\
0.4283428	-5.13406015027751e-08\\
0.4284428	-6.63965073061101e-08\\
0.4285429	7.36134753882844e-08\\
0.4286429	5.47067757067365e-08\\
0.4287429	4.03050233951929e-08\\
0.4288429	2.50218483120734e-08\\
0.4289429	1.34902387073643e-08\\
0.4290429	-9.63740117917999e-09\\
0.4291429	-1.96891943755884e-08\\
0.4292429	-4.19737337352399e-08\\
0.4293429	-6.1780024641217e-08\\
0.4294429	-7.43774284904519e-08\\
0.429543	7.40518039457627e-08\\
0.429643	6.11020793336858e-08\\
0.429743	3.96763985867354e-08\\
0.429843	2.45845714098858e-08\\
0.429943	1.06563385407488e-08\\
0.430043	-7.25857116426476e-09\\
0.430143	-2.42903828717722e-08\\
0.430243	-4.5549218907065e-08\\
0.430343	-5.61250411805492e-08\\
0.430443	-8.10875938617173e-08\\
0.4305431	8.33916949938285e-08\\
0.4306431	6.55356358521264e-08\\
0.4307431	4.82104951758655e-08\\
0.4308431	2.64279940084822e-08\\
0.4309431	1.52203579809052e-08\\
0.4310431	-1.03596253936411e-08\\
0.4311431	-2.52385488723472e-08\\
0.4312431	-4.43223284046779e-08\\
0.4313431	-6.24961458348117e-08\\
0.4314431	-8.46243911889477e-08\\
0.4315432	8.36363211603741e-08\\
0.4316432	7.0148480139226e-08\\
0.4317432	4.82435727651442e-08\\
0.4318432	3.3140970162604e-08\\
0.4319432	1.00811218747832e-08\\
0.4320432	-5.67438699500267e-09\\
0.4321432	-2.88427779229772e-08\\
0.4322432	-4.41200220663296e-08\\
0.4323432	-6.61807827428973e-08\\
0.4324432	-8.96783583156924e-08\\
0.4325433	9.07701237229386e-08\\
0.4326433	7.1636794767016e-08\\
0.4327433	4.72409402221818e-08\\
0.4328433	3.30153156272162e-08\\
0.4329433	1.44143278285602e-08\\
0.4330433	-1.30859078573131e-08\\
0.4331433	-3.39875110832999e-08\\
0.4332433	-5.27707786149301e-08\\
0.4333433	-7.38941276987037e-08\\
0.4334433	-9.17940389255634e-08\\
0.4335434	1.00497764110585e-07\\
0.4336434	7.6990549151279e-08\\
0.4337434	5.35351611791191e-08\\
0.4338434	3.57834582728755e-08\\
0.4339434	9.40951994099493e-09\\
0.4340434	-9.89029573517097e-09\\
0.4341434	-2.639729603654e-08\\
0.4342434	-5.43703960256825e-08\\
0.4343434	-7.80460622240575e-08\\
0.4344434	-1.01638255936565e-07\\
0.4345435	1.03974422231701e-07\\
0.4346435	7.92222431614253e-08\\
0.4347435	6.20531156389492e-08\\
0.4348435	3.83436880510568e-08\\
0.4349435	1.39933967260863e-08\\
0.4350435	-5.07547742722991e-09\\
0.4351435	-3.29177530511426e-08\\
0.4352435	-5.35652916466844e-08\\
0.4353435	-8.1026941399498e-08\\
0.4354435	-9.92884812561767e-08\\
0.4355436	1.03514688966355e-07\\
0.4356436	8.30729296186217e-08\\
0.4357436	6.00190159655289e-08\\
0.4358436	4.04600306036507e-08\\
0.4359436	1.05264053792964e-08\\
0.4360436	-1.36280226895047e-08\\
0.4361436	-3.5825955792268e-08\\
0.4362436	-5.98665797837961e-08\\
0.4363436	-8.9525508235877e-08\\
0.4364436	-1.08554727089316e-07\\
0.4365437	1.18266551558943e-07\\
0.4366437	9.06819368901024e-08\\
0.4367437	6.26197240033433e-08\\
0.4368437	4.04230146786033e-08\\
0.4369437	1.04588146962459e-08\\
0.4370437	-1.08819114552627e-08\\
0.4371437	-3.7184185008865e-08\\
0.4372437	-6.2008958794102e-08\\
0.4373437	-8.88930617379269e-08\\
0.4374437	-1.11349144550632e-07\\
0.4375438	1.19836208650748e-07\\
0.4376438	9.72059546952564e-08\\
0.4377438	6.86179559075684e-08\\
0.4378438	4.06568441274624e-08\\
0.4379438	1.99317012339328e-08\\
0.4380438	-6.92388638784402e-09\\
0.4381438	-4.32517737389282e-08\\
0.4382438	-6.23692039333767e-08\\
0.4383438	-9.75687537083103e-08\\
0.4384438	-1.22118279891059e-07\\
0.4385439	1.2784869033583e-07\\
0.4386439	9.63723786537307e-08\\
0.4387439	7.58981930359948e-08\\
0.4388439	4.3257715194589e-08\\
0.4389439	1.53075137426795e-08\\
0.4390439	-1.10708028647211e-08\\
0.4391439	-3.89705338693647e-08\\
0.4392439	-7.1459833050902e-08\\
0.4393439	-1.01581656343497e-07\\
0.4394439	-1.32353709110522e-07\\
0.439544	1.35428484183964e-07\\
0.439644	1.05951039365559e-07\\
0.439744	7.69293467747789e-08\\
0.439844	4.54472510388881e-08\\
0.439944	1.86141085461999e-08\\
0.440044	-1.64351616344094e-08\\
0.440144	-4.25400254813746e-08\\
0.440244	-7.25142832239123e-08\\
0.440344	-1.09146017751582e-07\\
0.440444	-1.35197543105203e-07\\
0.4405441	1.44583595422521e-07\\
0.4406441	1.1312766545063e-07\\
0.4407441	8.4127451734084e-08\\
0.4408441	4.4924249056888e-08\\
0.4409441	1.28853749684077e-08\\
0.4410441	-1.45957805008923e-08\\
0.4411441	-5.00997047875318e-08\\
0.4412441	-7.6180712349494e-08\\
0.4413441	-1.15366895120372e-07\\
0.4414441	-1.40160073359985e-07\\
0.4415442	1.51475697332537e-07\\
0.4416442	1.19761660970219e-07\\
0.4417442	8.51008868998915e-08\\
0.4418442	5.50971637272579e-08\\
0.4419442	1.73807971903805e-08\\
0.4420442	-1.03913396359845e-08\\
0.4421442	-5.05357590149802e-08\\
0.4422442	-8.53423097746386e-08\\
0.4423442	-1.17074129515893e-07\\
0.4424442	-1.4796759677943e-07\\
0.4425443	1.51558254016249e-07\\
0.4426443	1.17510000563481e-07\\
0.4427443	8.77598546680791e-08\\
0.4428443	5.01789778964423e-08\\
0.4429443	2.26655269691678e-08\\
0.4430443	-1.68552976506642e-08\\
0.4431443	-5.04312063984158e-08\\
0.4432443	-9.0082773865352e-08\\
0.4433443	-1.17803392381266e-07\\
0.4434443	-1.55559227062174e-07\\
0.4435444	1.64563731336331e-07\\
0.4436444	1.32798283106195e-07\\
0.4437444	9.52710788870714e-08\\
0.4438444	6.01253489912468e-08\\
0.4439444	1.55317785854869e-08\\
0.4440444	-2.03114483565692e-08\\
0.4441444	-4.91786032313923e-08\\
0.4442444	-9.28163693958428e-08\\
0.4443444	-1.3294379785922e-07\\
0.4444444	-1.71252263675425e-07\\
0.4445445	1.69320252110805e-07\\
0.4446445	1.31619223187052e-07\\
0.4447445	1.00837928923569e-07\\
0.4448445	5.53961603116937e-08\\
0.4449445	2.37416000192781e-08\\
0.4450445	-1.56501354514615e-08\\
0.4451445	-5.42754523963351e-08\\
0.4452445	-9.36027394066175e-08\\
0.4453445	-1.35072324533581e-07\\
0.4454445	-1.80096434501675e-07\\
0.4455446	1.78377300083521e-07\\
0.4456446	1.44138112709302e-07\\
0.4457446	1.0228542632329e-07\\
0.4458446	6.15198580550258e-08\\
0.4459446	2.05703301753291e-08\\
0.4460446	-2.18058906209517e-08\\
0.4461446	-5.68231525305274e-08\\
0.4462446	-1.0566737994322e-07\\
0.4463446	-1.39496033807474e-07\\
0.4464446	-1.89438073311399e-07\\
0.4465447	1.92419344471895e-07\\
0.4466447	1.49084048634007e-07\\
0.4467447	1.06428856935792e-07\\
0.4468447	6.34392300219444e-08\\
0.4469447	1.91293204054865e-08\\
0.4470447	-1.74579895817484e-08\\
0.4471447	-6.72510521160241e-08\\
0.4472447	-1.11149415726874e-07\\
0.4473447	-1.50023790043696e-07\\
0.4474447	-1.9471600937182e-07\\
0.4475448	2.04445530602171e-07\\
0.4476448	1.5790500471069e-07\\
0.4477448	1.1320399743997e-07\\
0.4478448	6.96165582403374e-08\\
0.4479448	2.64457882120928e-08\\
0.4480448	-1.69761267404533e-08\\
0.4481448	-6.1287882144323e-08\\
0.4482448	-1.07099020443813e-07\\
0.4483448	-1.54989897971358e-07\\
0.4484448	-2.05511652626167e-07\\
0.4485449	2.13692888509365e-07\\
0.4486449	1.68664127150886e-07\\
0.4487449	1.19536264442388e-07\\
0.4488449	7.58753905261145e-08\\
0.4489449	2.72769742487577e-08\\
0.4490449	-1.66341055513053e-08\\
0.4491449	-6.62035302689823e-08\\
0.4492449	-1.21747509684855e-07\\
0.4493449	-1.6355275307689e-07\\
0.4494449	-2.11876439296099e-07\\
0.449545	2.19279825811047e-07\\
0.449645	1.69653956702542e-07\\
0.449745	1.22925427947784e-07\\
0.449845	7.89555014840682e-08\\
0.449945	2.76351127675989e-08\\
0.450045	-2.11151020002021e-08\\
0.450145	-7.73447780266778e-08\\
0.450245	-1.21073796766069e-07\\
0.450345	-1.72292257030748e-07\\
0.450445	-2.30960450806061e-07\\
0.4505451	2.33546030112275e-07\\
0.4506451	1.82704635837139e-07\\
0.4507451	1.34721876571264e-07\\
0.4508451	7.97569734145576e-08\\
0.4509451	2.79990799573193e-08\\
0.4510451	-2.03326946807136e-08\\
0.4511451	-7.49892607732061e-08\\
0.4512451	-1.25691527977367e-07\\
0.4513451	-1.82130379236534e-07\\
0.4514451	-2.33966650567408e-07\\
0.4515452	2.45064269363549e-07\\
0.4516452	1.86161534618257e-07\\
0.4517452	1.33069816651843e-07\\
0.4518452	8.62487144343005e-08\\
0.4519452	3.6187978544322e-08\\
0.4520452	-2.65924673809037e-08\\
0.4521452	-5.15425080116483e-08\\
0.4522452	-1.28081817360459e-07\\
0.4523452	-1.45599839379562e-07\\
0.4524452	-2.93455770863926e-07\\
0.4525453	3.0129901106013e-07\\
0.4526453	2.27507150799e-07\\
0.4527453	1.55492199852003e-07\\
0.4528453	9.60161554569572e-08\\
0.4529453	5.98713471855472e-08\\
0.4530453	-4.21195471972391e-08\\
0.4531453	-9.91034879160857e-08\\
0.4532453	-1.00197056332463e-07\\
0.4533453	-2.3448644410895e-07\\
0.4534453	-2.91027435414537e-07\\
0.4535454	2.3088621285694e-07\\
0.4536454	1.65601594756537e-07\\
0.4537454	1.11091634336091e-07\\
0.4538454	7.84223282979113e-08\\
0.4539454	7.86901417776376e-08\\
0.4540454	-7.69779812948013e-08\\
0.4541454	-7.74246074336515e-08\\
0.4542454	-1.11461803051216e-07\\
0.4543454	-1.67871126680041e-07\\
0.4544454	-2.35403616835606e-07\\
0.4545455	3.15508139989151e-07\\
0.4546455	2.62515454166334e-07\\
0.4547455	1.32341036560013e-07\\
0.4548455	1.36356034602381e-07\\
0.4549455	-1.40378461910284e-08\\
0.4550455	-7.40833734635959e-09\\
0.4551455	-1.32292600651951e-07\\
0.4552455	-1.77197223089065e-07\\
0.4553455	-2.3059821380178e-07\\
0.4554455	-2.80940996494339e-07\\
0.4555456	3.31336596895437e-07\\
0.4556456	2.24928393355306e-07\\
0.4557456	1.5643732409476e-07\\
0.4558456	1.37540363237543e-07\\
0.4559456	7.99450808469659e-08\\
0.4560456	-4.61035229437634e-09\\
0.4561456	-1.04357166658131e-07\\
0.4562456	-2.07495993664875e-07\\
0.4563456	-2.02196864723669e-07\\
0.4564456	-2.76599213616668e-07\\
0.4565457	2.60017548908698e-07\\
0.4566457	2.65066418508875e-07\\
0.4567457	1.26191050969782e-07\\
0.4568457	1.5537441735991e-07\\
0.4569457	6.46300740425598e-08\\
0.4570457	-3.39978401542318e-08\\
0.4571457	-1.2843460969747e-07\\
0.4572457	-2.06574947286819e-07\\
0.4573457	-2.5628299885061e-07\\
0.4574457	-3.65392352041916e-07\\
0.4575458	2.89169749145834e-07\\
0.4576458	3.0115038954226e-07\\
0.4577458	1.90424527701794e-07\\
0.4578458	6.92807719526733e-08\\
0.4579458	5.00382487111928e-08\\
0.4580458	-5.49534045490851e-08\\
0.4581458	-1.3331405135375e-07\\
0.4582458	-1.72633066786165e-07\\
0.4583458	-2.60469348256631e-07\\
0.4584458	-2.84351329621346e-07\\
0.4585459	3.12369680954827e-07\\
0.4586459	2.57328069352969e-07\\
0.4587459	2.03850697350072e-07\\
0.4588459	1.64530889856795e-07\\
0.4589459	5.19923674946465e-08\\
0.4590459	-2.11107701320046e-08\\
0.4591459	-1.42094058741549e-07\\
0.4592459	-1.98242687071692e-07\\
0.4593459	-2.76811515459441e-07\\
0.4594459	-3.65025091431104e-07\\
0.459546	3.28594890269818e-07\\
0.459646	2.63061755018112e-07\\
0.459746	2.26404785252932e-07\\
0.459846	1.31520514673467e-07\\
0.459946	9.13356887899397e-08\\
0.460046	-8.11927584609684e-08\\
0.460146	-7.30777250657066e-08\\
0.460246	-1.71301964613235e-07\\
0.460346	-2.62818111551599e-07\\
0.460446	-3.34548702851656e-07\\
0.4605461	3.41097601996854e-07\\
0.4606461	2.51943379991904e-07\\
0.4607461	2.22000603511407e-07\\
0.4608461	1.64466890967274e-07\\
0.4609461	9.25698257492336e-08\\
0.4610461	-8.04330744508253e-08\\
0.4611461	-1.41254383995149e-07\\
0.4612461	-1.76576800042794e-07\\
0.4613461	-2.7305317329418e-07\\
0.4614461	-4.17306537686679e-07\\
0.4615462	3.55680412650328e-07\\
0.4616462	2.59908866456771e-07\\
0.4617462	2.56683281518733e-07\\
0.4618462	1.5949959275563e-07\\
0.4619462	8.18833870153135e-08\\
0.4620462	-6.26101337719565e-08\\
0.4621462	-1.60396188469969e-07\\
0.4622462	-1.97860451106591e-07\\
0.4623462	-2.61359089981372e-07\\
0.4624462	-4.37218804494321e-07\\
0.4625463	3.78345821394621e-07\\
0.4626463	3.22823493908997e-07\\
0.4627463	1.96150190451272e-07\\
0.4628463	1.12116722322808e-07\\
0.4629463	8.45431700202681e-08\\
0.4630463	-7.27211597884025e-08\\
0.4631463	-1.45797777267864e-07\\
0.4632463	-2.20779050957853e-07\\
0.4633463	-2.83728254446558e-07\\
0.4634463	-4.20679609369756e-07\\
0.4635464	4.12291404433551e-07\\
0.4636464	3.73410099313576e-07\\
0.4637464	2.02611888683313e-07\\
0.4638464	1.13978318727614e-07\\
0.4639464	2.16197509894656e-08\\
0.4640464	-6.03246872177277e-08\\
0.4641464	-1.17687153855517e-07\\
0.4642464	-2.36271141790256e-07\\
0.4643464	-3.01851530354336e-07\\
0.4644464	-4.00174639064721e-07\\
0.4645465	4.54222572027208e-07\\
0.4646465	3.37492063795519e-07\\
0.4647465	2.30965058245491e-07\\
0.4648465	1.49008968525211e-07\\
0.4649465	6.01949348699726e-09\\
0.4650465	-8.35794370102061e-08\\
0.4651465	-1.05335721228839e-07\\
0.4652465	-2.44769144203127e-07\\
0.4653465	-3.87371435511819e-07\\
0.4654465	-4.18606330127025e-07\\
0.4655466	4.89955064131298e-07\\
0.4656466	3.2952326553537e-07\\
0.4657466	2.2424964446488e-07\\
0.4658466	1.88781855009479e-07\\
0.4659466	3.77952328323161e-08\\
0.4660466	-1.40072691778848e-08\\
0.4661466	-1.51895145600323e-07\\
0.4662466	-2.61110407739906e-07\\
0.4663466	-3.26867647060247e-07\\
0.4664466	-4.3435410562731e-07\\
0.4665467	4.8927963308687e-07\\
0.4666467	3.47377493575829e-07\\
0.4667467	3.08362723744349e-07\\
0.4668467	1.87156811826128e-07\\
0.4669467	9.87082440506715e-08\\
0.4670467	-4.20075693119415e-08\\
0.4671467	-1.19988366520385e-07\\
0.4672467	-2.20205111228777e-07\\
0.4673467	-4.27602064839716e-07\\
0.4674467	-5.2709686429453e-07\\
0.4675468	5.0006178423101e-07\\
0.4676468	3.66370223436885e-07\\
0.4677468	2.86002052790613e-07\\
0.4678468	1.74145379200352e-07\\
0.4679468	4.60145416492086e-08\\
0.4680468	-8.31499717413031e-08\\
0.4681468	-1.98081606145317e-07\\
0.4682468	-2.83487821273187e-07\\
0.4683468	-4.24050178970958e-07\\
0.4684468	-5.0442442350862e-07\\
0.4685469	5.4154991882549e-07\\
0.4686469	4.32486167656343e-07\\
0.4687469	3.29812763655291e-07\\
0.4688469	1.48976383673727e-07\\
0.4689469	1.05449086101883e-07\\
0.4690469	-8.52717825794436e-08\\
0.4691469	-2.07663677309622e-07\\
0.4692469	-2.46178946006204e-07\\
0.4693469	-3.85244924683281e-07\\
0.4694469	-5.09264026449241e-07\\
0.469547	4.96865832977367e-07\\
0.469647	4.54786609566105e-07\\
0.469747	2.74711282721718e-07\\
0.469847	1.72336191928135e-07\\
0.469947	6.33821192752571e-08\\
0.470047	-3.64058124757705e-08\\
0.470147	-2.1125823962409e-07\\
0.470247	-3.45381655664223e-07\\
0.470347	-4.22958517320993e-07\\
0.470447	-5.28147341083418e-07\\
0.4705471	6.04652226621205e-07\\
0.4706471	3.96971258864731e-07\\
0.4707471	3.09361286365828e-07\\
0.4708471	1.57758522245643e-07\\
0.4709471	5.81225901297167e-08\\
0.4710471	-7.35635812265656e-08\\
0.4711471	-2.21293868341021e-07\\
0.4712471	-2.69039062583687e-07\\
0.4713471	-4.00746979156708e-07\\
0.4714471	-6.00342573486845e-07\\
0.4715472	5.4985267666785e-07\\
0.4716472	4.68070165349221e-07\\
0.4717472	3.66777321847422e-07\\
0.4718472	1.62139534631844e-07\\
0.4719472	7.03444817395926e-08\\
0.4720472	-9.23979823763688e-08\\
0.4721472	-2.09855948063442e-07\\
0.4722472	-3.65775568111903e-07\\
0.4723472	-4.43881177425443e-07\\
0.4724472	-6.27875416880919e-07\\
0.4725473	6.5360045831131e-07\\
0.4726473	5.12242172234778e-07\\
0.4727473	3.14032664761044e-07\\
0.4728473	1.75354875231459e-07\\
0.4729473	1.12612818514712e-07\\
0.4730473	-5.77685387859184e-08\\
0.4731473	-2.193434068265e-07\\
0.4732473	-3.55645301486351e-07\\
0.4733473	-4.5018717315104e-07\\
0.4734473	-5.86461539942018e-07\\
0.4735474	6.62193614275886e-07\\
0.4736474	4.97657420384101e-07\\
0.4737474	3.41049048169229e-07\\
0.4738474	2.08956412617312e-07\\
0.4739474	1.17987199121927e-07\\
0.4740474	-1.15231279074202e-07\\
0.4741474	-1.74052213752773e-07\\
0.4742474	-3.4180944019635e-07\\
0.4743474	-5.01817580045305e-07\\
0.4744474	-6.37372176870432e-07\\
0.4745475	6.3513482612132e-07\\
0.4746475	5.04442578985742e-07\\
0.4747475	3.4844704638326e-07\\
0.4748475	1.83927582664278e-07\\
0.4749475	1.2768190166268e-07\\
0.4750475	-1.03474071089238e-07\\
0.4751475	-1.92706342161397e-07\\
0.4752475	-4.23163001528337e-07\\
0.4753475	-4.77974370083345e-07\\
0.4754475	-6.40253149108982e-07\\
0.4755476	7.3221590340955e-07\\
0.4756476	5.11668742764204e-07\\
0.4757476	4.34437203988836e-07\\
0.4758476	2.17477551855971e-07\\
0.4759476	7.77629055037643e-08\\
0.4760476	-6.77169236240616e-08\\
0.4761476	-2.01955581702329e-07\\
0.4762476	-4.0793033706521e-07\\
0.4763476	-5.68602235073712e-07\\
0.4764476	-6.66916260994022e-07\\
0.4765477	6.99628069034119e-07\\
0.4766477	5.83363336470466e-07\\
0.4767477	3.80733374660824e-07\\
0.4768477	2.08855833183641e-07\\
0.4769477	8.48636059758157e-08\\
0.4770477	-7.40953362554642e-08\\
0.4771477	-2.50858113282071e-07\\
0.4772477	-4.28247103334556e-07\\
0.4773477	-5.89070111356449e-07\\
0.4774477	-7.16120542421095e-07\\
0.4775478	7.55080747871784e-07\\
0.4776478	5.53528864910557e-07\\
0.4777478	4.37471289693736e-07\\
0.4778478	2.24170511597954e-07\\
0.4779478	1.30902553296952e-07\\
0.4780478	-1.25043205734254e-07\\
0.4781478	-2.26364209487961e-07\\
0.4782478	-4.55744899818811e-07\\
0.4783478	-5.95856898766289e-07\\
0.4784478	-7.29359190432266e-07\\
0.4785479	7.71913151731418e-07\\
0.4786479	6.10152996349811e-07\\
0.4787479	4.07116993450529e-07\\
0.4788479	2.80194898727437e-07\\
0.4789479	4.67881870469533e-08\\
0.4790479	-7.56901344378669e-08\\
0.4791479	-2.6981571799034e-07\\
0.4792479	-4.1815305668047e-07\\
0.4793479	-6.03255682685155e-07\\
0.4794479	-8.07666353574987e-07\\
0.479548	7.62184714342418e-07\\
0.479648	5.78196196920899e-07\\
0.479748	4.273515601394e-07\\
0.479848	2.27149213161137e-07\\
0.479948	9.50973664259891e-08\\
0.480048	-5.12861691448485e-08\\
0.480148	-2.94474175877024e-07\\
0.480248	-4.16930237057134e-07\\
0.480348	-6.01108926912008e-07\\
0.480448	-8.29456020534636e-07\\
0.4805481	8.58732302155829e-07\\
0.4806481	6.01545030898265e-07\\
0.4807481	4.52920389470979e-07\\
0.4808481	3.30445779239685e-07\\
0.4809481	5.17163752933314e-08\\
0.4810481	-6.56650767649758e-08\\
0.4811481	-3.04088472735664e-07\\
0.4812481	-4.45936562287752e-07\\
0.4813481	-6.73585155819278e-07\\
0.4814481	-8.69403338321639e-07\\
0.4815482	8.96184063237104e-07\\
0.4816482	7.23921220913137e-07\\
0.4817482	4.36437392337119e-07\\
0.4818482	2.51388252936469e-07\\
0.4819482	8.64351220042359e-08\\
0.4820482	-1.40755250732649e-07\\
0.4821482	-3.12510909095121e-07\\
0.4822482	-5.11154911675415e-07\\
0.4823482	-7.1900554793336e-07\\
0.4824482	-9.18376568626034e-07\\
0.4825483	8.90921949325829e-07\\
0.4826483	6.68738680986714e-07\\
0.4827483	5.08136488053257e-07\\
0.4828483	3.26817538420787e-07\\
0.4829483	1.42487415200065e-07\\
0.4830483	-1.27145119044325e-07\\
0.4831483	-2.64368350322286e-07\\
0.4832483	-5.51467844389641e-07\\
0.4833483	-6.70726683860146e-07\\
0.4834483	-9.04425694614464e-07\\
0.4835484	9.1998710610719e-07\\
0.4836484	7.17902986113739e-07\\
0.4837484	5.54565106780647e-07\\
0.4838484	3.47699295299986e-07\\
0.4839484	1.15032451397656e-07\\
0.4840484	-1.25707698861444e-07\\
0.4841484	-3.56792840994213e-07\\
0.4842484	-5.60494313406323e-07\\
0.4843484	-7.19083356026662e-07\\
0.4844484	-9.14831347675893e-07\\
0.4845485	9.98924709949556e-07\\
0.4846485	7.89550057389565e-07\\
0.4847485	4.96216735423882e-07\\
0.4848485	3.36650330667876e-07\\
0.4849485	1.28575054048063e-07\\
0.4850485	-1.10286516141755e-07\\
0.4851485	-3.62213680493717e-07\\
0.4852485	-5.09487874444936e-07\\
0.4853485	-7.34392923806837e-07\\
0.4854485	-1.01921529338522e-06\\
0.4855486	9.585672605672e-07\\
0.4856486	8.14723363773417e-07\\
0.4857486	5.64102030975455e-07\\
0.4858486	3.24403655049821e-07\\
0.4859486	1.13324691630723e-07\\
0.4860486	-1.51442593465667e-07\\
0.4861486	-3.5221039096367e-07\\
0.4862486	-5.71295611262457e-07\\
0.4863486	-7.91020145718413e-07\\
0.4864486	-9.93711137931008e-07\\
0.4865487	1.02075793162459e-06\\
0.4866487	8.12991475473979e-07\\
0.4867487	5.75260632018626e-07\\
0.4868487	3.25214587704181e-07\\
0.4869487	8.04959355837909e-08\\
0.4870487	-1.41259591801557e-07\\
0.4871487	-3.22423396537097e-07\\
0.4872487	-5.45374288285494e-07\\
0.4873487	-7.92498759172311e-07\\
0.4874487	-1.0461912636206e-06\\
0.4875488	1.07301824359141e-06\\
0.4876488	8.67010521266032e-07\\
0.4877488	6.07218466508463e-07\\
0.4878488	3.11213008252764e-07\\
0.4879488	9.65557205301337e-08\\
0.4880488	-1.1920145248645e-07\\
0.4881488	-3.18516486408349e-07\\
0.4882488	-5.83857555302991e-07\\
0.4883488	-7.97703319066478e-07\\
0.4884488	-1.14254321270657e-06\\
0.4885489	1.142167124768e-06\\
0.4886489	8.96038898190454e-07\\
0.4887489	5.71398214299634e-07\\
0.4888489	3.85709663763834e-07\\
0.4889489	1.56425622899192e-07\\
0.4890489	-9.90140451850863e-08\\
0.4891489	-3.63182273588336e-07\\
0.4892489	-6.18665091933757e-07\\
0.4893489	-8.48061921243648e-07\\
0.4894489	-1.1339858642101e-06\\
0.489549	1.16690130758101e-06\\
0.489649	9.28414471523809e-07\\
0.489749	6.85493448004948e-07\\
0.489849	3.55467394186348e-07\\
0.489949	1.55650284772157e-07\\
0.490049	-9.66593896034595e-08\\
0.490149	-3.84178921231104e-07\\
0.490249	-6.89641693085719e-07\\
0.490349	-8.95797482747263e-07\\
0.490449	-1.18541276594719e-06\\
0.4905491	1.16934999971896e-06\\
0.4906491	8.73007515833635e-07\\
0.4907491	6.44820001616253e-07\\
0.4908491	4.01951085970964e-07\\
0.4909491	1.61546150456715e-07\\
0.4910491	-1.59267980692768e-07\\
0.4911491	-4.43383349257864e-07\\
0.4912491	-6.73711178289516e-07\\
0.4913491	-9.33182190077098e-07\\
0.4914491	-1.20474691311934e-06\\
0.4915492	1.2256198577143e-06\\
0.4916492	9.8966572659176e-07\\
0.4917492	6.9266122104672e-07\\
0.4918492	4.5157336692192e-07\\
0.4919492	1.83347783844212e-07\\
0.4920492	-9.50916293485093e-08\\
0.4921492	-3.66843025734198e-07\\
0.4922492	-7.15026925335316e-07\\
0.4923492	-9.22786541956988e-07\\
0.4924492	-1.27328810839433e-06\\
0.4925493	1.23534944329595e-06\\
0.4926493	9.5867082183787e-07\\
0.4927493	6.89626927785625e-07\\
0.4928493	4.449561621378e-07\\
0.4929493	1.41372267448148e-07\\
0.4930493	-1.04435996850505e-07\\
0.4931493	-3.75805196850543e-07\\
0.4932493	-7.56097543685996e-07\\
0.4933493	-1.02870123419585e-06\\
0.4934493	-1.27703078312003e-06\\
0.4935494	1.29029560669291e-06\\
0.4936494	1.04922948329644e-06\\
0.4937494	6.82049189126843e-07\\
0.4938494	4.05231555866692e-07\\
0.4939494	1.35225410202722e-07\\
0.4940494	-1.11548755787894e-07\\
0.4941494	-4.18699139581591e-07\\
0.4942494	-7.698629509445e-07\\
0.4943494	-1.04870676720381e-06\\
0.4944494	-1.33892687059767e-06\\
0.4945495	1.34197756007648e-06\\
0.4946495	1.0870247556527e-06\\
0.4947495	7.69441336312404e-07\\
0.4948495	4.05408725367984e-07\\
0.4949495	1.11076906894603e-07\\
0.4950495	-1.97435911397648e-07\\
0.4951495	-4.04043649981922e-07\\
0.4952495	-7.9269271402449e-07\\
0.4953495	-1.04736233008662e-06\\
0.4954495	-1.35206490781137e-06\\
0.4955496	1.36840779374481e-06\\
0.4956496	1.12085738934731e-06\\
0.4957496	7.71050063907985e-07\\
0.4958496	4.3483713063619e-07\\
0.4959496	1.28034953750245e-07\\
0.4960496	-1.33575407956776e-07\\
0.4961496	-4.34248552227956e-07\\
0.4962496	-7.58275099266292e-07\\
0.4963496	-1.08998204251254e-06\\
0.4964496	-1.41373311013382e-06\\
0.4965497	1.43994241585954e-06\\
0.4966497	1.08841006907223e-06\\
0.4967497	7.9153390197817e-07\\
0.4968497	4.64799603694388e-07\\
0.4969497	1.23654318673339e-07\\
0.4970497	-1.16493713675503e-07\\
0.4971497	-4.4027552503767e-07\\
0.4972497	-8.32361785385594e-07\\
0.4973497	-1.17746316197209e-06\\
0.4974497	-1.46033069126084e-06\\
0.4975498	1.48428705470938e-06\\
0.4976498	1.18117098624193e-06\\
0.4977498	7.85804875946816e-07\\
0.4978498	5.13272488955607e-07\\
0.4979498	1.78615382373515e-07\\
0.4980498	-2.03167470180787e-07\\
0.4981498	-5.17120050669284e-07\\
0.4982498	-8.48329667513248e-07\\
0.4983498	-1.18192733233879e-06\\
0.4984498	-1.50308812785216e-06\\
0.4985499	1.55069791651918e-06\\
0.4986499	1.20855675456166e-06\\
0.4987499	8.2307369142498e-07\\
0.4988499	5.08893546191658e-07\\
0.4989499	1.8061519324597e-07\\
0.4990499	-1.47208826350997e-07\\
0.4991499	-5.60072680455548e-07\\
0.4992499	-8.43517623638768e-07\\
0.4993499	-1.18313238339773e-06\\
0.4994499	-1.56455352851026e-06\\
0.49955	1.57342123530135e-06\\
0.49965	1.16127820637413e-06\\
0.49975	8.50142372055274e-07\\
0.49985	5.54181901146755e-07\\
0.49995	1.87515208027733e-07\\
0.50005	-1.35789435806721e-07\\
0.50015	-5.01714287892696e-07\\
0.50025	-8.96292518204334e-07\\
0.50035	-1.2056085942902e-06\\
0.50045	-1.61579866830852e-06\\
0.5005501	1.63441790634877e-06\\
0.5006501	1.27399626981273e-06\\
0.5007501	8.53991878990712e-07\\
0.5008501	4.88057912972906e-07\\
0.5009501	1.89793930238658e-07\\
0.5010501	-2.27254522533471e-07\\
0.5011501	-5.49596317678258e-07\\
0.5012501	-8.63795108441145e-07\\
0.5013501	-1.25646974002969e-06\\
0.5014501	-1.61429463474416e-06\\
0.5015502	1.62542370552021e-06\\
0.5016502	1.28731867021781e-06\\
0.5017502	9.23715831602223e-07\\
0.5018502	5.47714472221017e-07\\
0.5019502	1.72356324767975e-07\\
0.5020502	-1.89374815606769e-07\\
0.5021502	-5.24553489755775e-07\\
0.5022502	-6.20312968768744e-07\\
0.5023502	-8.63845652121498e-07\\
0.5024502	-1.74240345973686e-06\\
0.5025503	1.50939911569559e-06\\
0.5026503	1.4091918081327e-06\\
0.5027503	7.11855038026954e-07\\
0.5028503	4.29894754683247e-07\\
0.5029503	5.7575539713639e-07\\
0.5030503	1.61819487809467e-07\\
0.5031503	-7.99592752676048e-07\\
0.5032503	-1.29622383338024e-06\\
0.5033503	-1.3158793596979e-06\\
0.5034503	-1.84642845457716e-06\\
0.5035504	1.98142631724352e-06\\
0.5036504	1.47574582531895e-06\\
0.5037504	4.95191207372514e-07\\
0.5038504	1.05163521313401e-06\\
0.5039504	1.56885066493473e-07\\
0.5040504	-1.77317935223797e-07\\
0.5041504	-9.3929882205579e-07\\
0.5042504	-1.11744935815672e-06\\
0.5043504	-1.70022844026008e-06\\
0.5044504	-1.67616250856817e-06\\
0.5045505	1.92911459651057e-06\\
0.5046505	1.21165429267656e-06\\
0.5047505	1.13506163090804e-06\\
0.5048505	7.10535891901287e-07\\
0.5049505	-5.07932083060325e-08\\
0.5050505	-1.37865924676461e-07\\
0.5051505	-5.39692895706878e-07\\
0.5052505	-1.24535553602811e-06\\
0.5053505	-1.24400646606126e-06\\
0.5054505	-1.52486989368761e-06\\
0.5055506	1.99257875532766e-06\\
0.5056506	1.19007368959423e-06\\
0.5057506	1.13726035277928e-06\\
0.5058506	8.44623983642379e-07\\
0.5059506	3.22576194422552e-07\\
0.5060506	-4.18545453406693e-07\\
0.5061506	-3.68477850676285e-07\\
0.5062506	-1.51703275363957e-06\\
0.5063506	-1.85409718733709e-06\\
0.5064506	-2.36963386091027e-06\\
0.5065507	2.124058849029e-06\\
0.5066507	1.29223110012511e-06\\
0.5067507	1.31159457916752e-06\\
0.5068507	1.91879687028162e-07\\
0.5069507	-5.72609077931219e-08\\
0.5070507	-4.26252674934346e-07\\
0.5071507	-9.05599628842069e-07\\
0.5072507	-1.48588474235822e-06\\
0.5073507	-1.15777036624465e-06\\
0.5074507	-1.91199861876434e-06\\
0.5075508	1.54725271706013e-06\\
0.5076508	1.66673212986801e-06\\
0.5077508	7.31168747547528e-07\\
0.5078508	7.49497150120959e-07\\
0.5079508	-2.69429923482889e-07\\
0.5080508	-3.16841966885306e-07\\
0.5081508	-3.84051126012253e-07\\
0.5082508	-1.46245261234412e-06\\
0.5083508	-1.54352511216871e-06\\
0.5084508	-1.61883119309536e-06\\
0.5085509	1.71643642854491e-06\\
0.5086509	1.68866438410298e-06\\
0.5087509	6.9147177650386e-07\\
0.5088509	7.32956309423827e-07\\
0.5089509	-1.78870253719765e-07\\
0.5090509	-3.60824969902751e-08\\
0.5091509	-8.30841764942747e-07\\
0.5092509	-1.55539656643278e-06\\
0.5093509	-1.20208299114744e-06\\
0.5094509	-1.76332512658028e-06\\
0.509551	2.27545072539748e-06\\
0.509651	1.91857492981562e-06\\
0.509751	6.69346412296434e-07\\
0.509851	5.34984951672257e-07\\
0.509951	5.22620270837493e-07\\
0.510051	-3.60708343860594e-07\\
0.510151	-1.10805246444912e-06\\
0.510251	-7.12554933279819e-07\\
0.510351	-1.16745026579324e-06\\
0.510451	-2.46606506770775e-06\\
0.5105511	2.0166349994355e-06\\
0.5106511	2.06140363356866e-06\\
0.5107511	1.28192273951555e-06\\
0.5108511	6.84493196123626e-07\\
0.5109511	2.75321752063462e-07\\
0.5110511	6.05206227355382e-08\\
0.5111511	-9.53892917321397e-07\\
0.5112511	-7.61996947096222e-07\\
0.5113511	-1.35796529709964e-06\\
0.5114511	-1.73606795961234e-06\\
0.5115512	1.83979310297566e-06\\
0.5116512	1.92545690791412e-06\\
0.5117512	1.24560075942526e-06\\
0.5118512	8.05565850692247e-07\\
0.5119512	6.10595189964869e-07\\
0.5120512	-3.34166828652727e-07\\
0.5121512	-1.02367479803966e-06\\
0.5122512	-1.45298272236083e-06\\
0.5123512	-1.61724441238853e-06\\
0.5124512	-2.51171389575866e-06\\
0.5125513	1.71127853820785e-06\\
0.5126513	1.38150974504647e-06\\
0.5127513	1.33517016220885e-06\\
0.5128513	5.76600784230052e-07\\
0.5129513	1.10040375034259e-07\\
0.5130513	-6.03749263916598e-08\\
0.5131513	-9.30612007232767e-07\\
0.5132513	-1.49674118299359e-06\\
0.5133513	-1.7549365949332e-06\\
0.5134513	-2.70147660885733e-06\\
0.5135514	2.62328897382424e-06\\
0.5136514	1.32212623826788e-06\\
0.5137514	1.34315779320993e-06\\
0.5138514	6.8968427191507e-07\\
0.5139514	3.64900092586495e-07\\
0.5140514	-6.28106939792872e-07\\
0.5141514	-2.8635604021332e-07\\
0.5142514	-1.60697381446084e-06\\
0.5143514	-1.58719467280299e-06\\
0.5144514	-2.22436122032654e-06\\
0.5145515	2.55346309008253e-06\\
0.5146515	1.62129183722826e-06\\
0.5147515	1.03949626240052e-06\\
0.5148515	8.10296943098976e-07\\
0.5149515	-6.4195696092284e-08\\
0.5150515	-5.81981782987739e-07\\
0.5151515	-7.41172378582178e-07\\
0.5152515	-1.53998987473258e-06\\
0.5153515	-1.97676837787597e-06\\
0.5154515	-2.04995409891495e-06\\
0.5155516	2.4248747627631e-06\\
0.5156516	2.0944526066824e-06\\
0.5157516	1.13160925918621e-06\\
0.5158516	5.37446109838413e-07\\
0.5159516	3.12950509728793e-07\\
0.5160516	-5.41004598675343e-07\\
0.5161516	-1.02366109810959e-06\\
0.5162516	-1.13437603954658e-06\\
0.5163516	-1.87262204098815e-06\\
0.5164516	-2.23798767651218e-06\\
0.5165517	2.06652230549942e-06\\
0.5166517	1.45905940573954e-06\\
0.5167517	1.22501166521793e-06\\
0.5168517	3.64322811696738e-07\\
0.5169517	-1.23181272071449e-07\\
0.5170517	-2.37792930235514e-07\\
0.5171517	-9.79923095112412e-07\\
0.5172517	-1.35010167090499e-06\\
0.5173517	-2.3489779030772e-06\\
0.5174517	-1.97732075468693e-06\\
0.5175518	2.17441220939918e-06\\
0.5176518	2.29571659771111e-06\\
0.5177518	1.78452427723741e-06\\
0.5178518	6.39583588535331e-07\\
0.5179518	-1.40478697119306e-07\\
0.5180518	-5.57157738390046e-07\\
0.5181518	-6.12070998551317e-07\\
0.5182518	-1.30695861888519e-06\\
0.5183518	-1.64368375865642e-06\\
0.5184518	-2.62423297736625e-06\\
0.5185519	2.27333889024628e-06\\
0.5186519	2.01003710897929e-06\\
0.5187519	1.09620513621955e-06\\
0.5188519	5.29359094336712e-07\\
0.5189519	3.06889891055206e-07\\
0.5190519	-5.73937120051937e-07\\
0.5191519	-1.11598249974421e-06\\
0.5192519	-1.32223308146884e-06\\
0.5193519	-2.19580232285743e-06\\
0.5194519	-2.73993066057798e-06\\
0.519552	2.67946362964011e-06\\
0.519652	1.79530690047613e-06\\
0.519752	1.23010068686646e-06\\
0.519852	9.80092997204807e-07\\
0.519952	4.14031067208498e-08\\
0.520052	-5.89978783605005e-07\\
0.520152	-9.18191899179632e-07\\
0.520252	-9.47505222059775e-07\\
0.520352	-1.6823178663472e-06\\
0.520452	-2.12715938041619e-06\\
0.5205521	2.46379706392297e-06\\
0.5206521	1.59606371141763e-06\\
0.5207521	1.00392172441843e-06\\
0.5208521	6.82316193101329e-07\\
0.5209521	6.2606008088828e-07\\
0.5210521	-1.70166132962279e-07\\
0.5211521	-7.11814785336884e-07\\
0.5212521	-1.00447136230919e-06\\
0.5213521	-2.05385482909826e-06\\
0.5214521	-2.86581795494456e-06\\
0.5215522	2.41669112588383e-06\\
0.5216522	2.07269569996171e-06\\
0.5217522	1.9477498487408e-06\\
0.5218522	1.03546216312367e-06\\
0.5219522	3.29305794855017e-07\\
0.5220522	-1.77381867416671e-07\\
0.5221522	-4.91399506197254e-07\\
0.5222522	-1.61968220679931e-06\\
0.5223522	-2.56930178643699e-06\\
0.5224522	-2.34746707872269e-06\\
0.5225523	3.01344682296367e-06\\
0.5226523	2.56716671964696e-06\\
0.5227523	1.26988135851214e-06\\
0.5228523	1.11383049672398e-06\\
0.5229523	9.11152819860206e-08\\
0.5230523	-8.06302018929728e-07\\
0.5231523	-5.86598359664237e-07\\
0.5232523	-1.25809020046219e-06\\
0.5233523	-1.82923382485711e-06\\
0.5234523	-2.30862562711209e-06\\
0.5235524	2.38114529604871e-06\\
0.5236524	2.06997561313926e-06\\
0.5237524	1.82391576153407e-06\\
0.5238524	6.33805742111804e-07\\
0.5239524	4.90343948111871e-07\\
0.5240524	-6.15913121837508e-07\\
0.5241524	-6.94551158364476e-07\\
0.5242524	-1.75529832979748e-06\\
0.5243524	-1.80802556437598e-06\\
0.5244524	-2.86274685823429e-06\\
0.5245525	3.26680958373515e-06\\
0.5246525	2.18845693833458e-06\\
0.5247525	1.07719661546213e-06\\
0.5248525	9.22439457706759e-07\\
0.5249525	7.13451823397548e-07\\
0.5250525	-5.60644637825902e-07\\
0.5251525	-9.10873295900672e-07\\
0.5252525	-1.34840280496817e-06\\
0.5253525	-1.88454736571586e-06\\
0.5254525	-2.53076700884591e-06\\
0.5255526	3.00700493482964e-06\\
0.5256526	2.11653104509324e-06\\
0.5257526	1.08071240623175e-06\\
0.5258526	8.87502888069491e-07\\
0.5259526	5.2470921740877e-07\\
0.5260526	-2.00092826752041e-08\\
0.5261526	-7.59140951878123e-07\\
0.5262526	-1.70532204624152e-06\\
0.5263526	-1.87133699114383e-06\\
0.5264526	-3.27011862788495e-06\\
0.5265527	2.49898492610257e-06\\
0.5266527	2.60601065393828e-06\\
0.5267527	1.44056466355735e-06\\
0.5268527	9.89117789895744e-07\\
0.5269527	2.37991216511801e-07\\
0.5270527	-8.2664378364683e-07\\
0.5271527	-1.21876604852744e-06\\
0.5272527	-1.95250479828601e-06\\
0.5273527	-2.04213985011137e-06\\
0.5274527	-2.5021018648097e-06\\
0.5275528	3.17349051037752e-06\\
0.5276528	1.93957102334608e-06\\
0.5277528	1.2911103133284e-06\\
0.5278528	1.21307192424425e-06\\
0.5279528	6.90267430769609e-07\\
0.5280528	-2.9264378520466e-07\\
0.5281528	-7.51154761502448e-07\\
0.5282528	-1.70091115991866e-06\\
0.5283528	-2.15771149392197e-06\\
0.5284528	-3.13750733793583e-06\\
0.5285529	2.96930784227101e-06\\
0.5286529	1.90548942757829e-06\\
0.5287529	1.2698838438574e-06\\
0.5288529	1.0459250341377e-06\\
0.5289529	2.1689284457338e-07\\
0.5290529	-2.34087156325558e-07\\
0.5291529	-1.32404392605423e-06\\
0.5292529	-2.07016109055047e-06\\
0.5293529	-2.48977716665877e-06\\
0.5294529	-2.60038574495525e-06\\
0.529553	3.30968995543302e-06\\
0.529653	2.77425891503924e-06\\
0.529753	1.49440419017211e-06\\
0.529853	1.45200983592275e-06\\
0.529953	6.2880390494513e-07\\
0.530053	6.35826102879378e-09\\
0.530153	-1.43391159879158e-06\\
0.530253	-1.71074670962668e-06\\
0.530353	-2.84304483066933e-06\\
0.530453	-2.84986058796122e-06\\
0.5305531	3.08074552446413e-06\\
0.5306531	2.27717937306693e-06\\
0.5307531	1.54096910831569e-06\\
0.5308531	8.5243075309549e-07\\
0.5309531	1.91722623554824e-07\\
0.5310531	-4.61154820552423e-07\\
0.5311531	-1.12635912419279e-06\\
0.5312531	-1.8242060235707e-06\\
0.5313531	-2.57516954338399e-06\\
0.5314531	-3.39988217712417e-06\\
0.5315532	2.61189640893278e-06\\
0.5316532	2.58702744693196e-06\\
0.5317532	1.42553846060878e-06\\
0.5318532	1.10616151793863e-06\\
0.5319532	6.07469511582082e-07\\
0.5320532	-9.21239653450812e-08\\
0.5321532	-1.01436474908567e-06\\
0.5322532	-2.18115823980014e-06\\
0.5323532	-2.61456952088679e-06\\
0.5324532	-3.33682348951925e-06\\
0.5325533	2.65850326730543e-06\\
0.5326533	2.30090427777441e-06\\
0.5327533	1.58680505801811e-06\\
0.5328533	1.49334011734226e-06\\
0.5329533	-2.51642973125854e-09\\
0.5330533	7.60490248552514e-08\\
0.5331533	-1.2943105098806e-06\\
0.5332533	-2.13710276630863e-06\\
0.5333533	-2.4759962684584e-06\\
0.5334533	-3.33482046599443e-06\\
0.5335534	3.38675614841577e-06\\
0.5336534	2.42535786298959e-06\\
0.5337534	1.87154967479231e-06\\
0.5338534	7.00857380131481e-07\\
0.5339534	-1.11354621967763e-07\\
0.5340534	-5.89883380364142e-07\\
0.5341534	-7.59687497353667e-07\\
0.5342534	-1.64588718476466e-06\\
0.5343534	-2.27376435590898e-06\\
0.5344534	-3.66876269675132e-06\\
0.5345535	3.3609241949506e-06\\
0.5346535	2.36387357821499e-06\\
0.5347535	1.52237340644845e-06\\
0.5348535	8.10332060563468e-07\\
0.5349535	2.01495823137776e-07\\
0.5350535	-3.30551183758132e-07\\
0.5351535	-8.12387050519447e-07\\
0.5352535	-2.27075212411165e-06\\
0.5353535	-2.73254910876375e-06\\
0.5354535	-3.22484303127979e-06\\
0.5355536	3.53305510802926e-06\\
0.5356536	2.90682361159256e-06\\
0.5357536	2.16789779150517e-06\\
0.5358536	1.28856253667209e-06\\
0.5359536	2.40940203344309e-07\\
0.5360536	-3.00945313114198e-09\\
0.5361536	-1.47148930995655e-06\\
0.5362536	-2.19286488567505e-06\\
0.5363536	-2.19566436987506e-06\\
0.5364536	-3.50857865294429e-06\\
0.5365537	3.23521196543552e-06\\
0.5366537	2.22396201143837e-06\\
0.5367537	1.81551902045385e-06\\
0.5368537	9.80541023043457e-07\\
0.5369537	6.89523297126016e-07\\
0.5370537	-8.72016348019145e-08\\
0.5371537	-1.37946401945754e-06\\
0.5372537	-2.21725684523832e-06\\
0.5373537	-2.63073585804463e-06\\
0.5374537	-3.65021956705291e-06\\
0.5375538	3.1743305606291e-06\\
0.5376538	2.8595485543903e-06\\
0.5377538	1.84679914383423e-06\\
0.5378538	1.10511282924719e-06\\
0.5379538	6.03357426687268e-07\\
0.5380538	-6.89761868066796e-07\\
0.5381538	-8.05702478956505e-07\\
0.5382538	-1.77608441287447e-06\\
0.5383538	-2.63269022848789e-06\\
0.5384538	-3.40746499682965e-06\\
0.5385539	3.42977681144774e-06\\
0.5386539	2.73018030272354e-06\\
0.5387539	2.01557242629491e-06\\
0.5388539	1.25335830514217e-06\\
0.5389539	4.10780772952535e-07\\
0.5390539	-5.45079574365559e-07\\
0.5391539	-1.64730432183902e-06\\
0.5392539	-1.92913717977916e-06\\
0.5393539	-2.42398390559728e-06\\
0.5394539	-3.16541227940448e-06\\
0.539554	3.45367867415902e-06\\
0.539654	3.12540487001911e-06\\
0.539754	1.44884072916796e-06\\
0.539854	1.38977105512694e-06\\
0.539954	-8.61809636987232e-08\\
0.540054	-1.35537083778559e-08\\
0.540154	-1.42704698191665e-06\\
0.540254	-2.36152197574668e-06\\
0.540354	-2.85200114813478e-06\\
0.540454	-3.93366813566942e-06\\
0.5405541	3.07410271638275e-06\\
0.5406541	2.7111850742223e-06\\
0.5407541	1.65052685519029e-06\\
0.5408541	8.5630054291741e-07\\
0.5409541	2.92517964517458e-07\\
0.5410541	-7.69696044677914e-08\\
0.5411541	-1.28847131897203e-06\\
0.5412541	-2.37845664674552e-06\\
0.5413541	-2.38355524384559e-06\\
0.5414541	-3.34055682493783e-06\\
0.5415542	3.5011397763185e-06\\
0.5416542	2.53627882340624e-06\\
0.5417542	1.50814926103138e-06\\
0.5418542	1.37932226129678e-06\\
0.5419542	1.1220960871583e-07\\
0.5420542	-3.30936190007236e-07\\
0.5421542	-9.88021704007025e-07\\
0.5422542	-1.89711246623148e-06\\
0.5423542	-3.09643280482774e-06\\
0.5424542	-3.62436569734115e-06\\
0.5425543	3.33595914581153e-06\\
0.5426543	3.04159315955843e-06\\
0.5427543	2.30247649124493e-06\\
0.5428543	1.07959304607164e-06\\
0.5429543	3.33768917837318e-07\\
0.5430543	2.56725543046343e-08\\
0.5431543	-8.8418507049326e-07\\
0.5432543	-2.43545025435665e-06\\
0.5433543	-2.66792640601921e-06\\
0.5434543	-3.6215738492551e-06\\
0.5435544	3.58288448909505e-06\\
0.5436544	3.07256503351283e-06\\
0.5437544	1.72021325539617e-06\\
0.5438544	1.48524295351393e-06\\
0.5439544	3.26911996140211e-07\\
0.5440544	-7.95677461695732e-07\\
0.5441544	-9.23578810763104e-07\\
0.5442544	-2.09800072203947e-06\\
0.5443544	-2.36030695832667e-06\\
0.5444544	-3.7520161546567e-06\\
0.5445545	3.66453904865338e-06\\
0.5446545	2.89461576530314e-06\\
0.5447545	1.86976456451049e-06\\
0.5448545	1.54784942951736e-06\\
0.5449545	-1.13419389879965e-07\\
0.5450545	-1.56485137292606e-07\\
0.5451545	-1.62394430791579e-06\\
0.5452545	-1.55854639993791e-06\\
0.5453545	-3.00319367818602e-06\\
0.5454545	-4.00094090791114e-06\\
0.5455546	3.44010091879454e-06\\
0.5456546	3.21171928385766e-06\\
0.5457546	2.30011728907797e-06\\
0.5458546	1.66163254311513e-06\\
0.5459546	2.524514410851e-07\\
0.5460546	2.86094374857271e-08\\
0.5461546	-1.05400866790006e-06\\
0.5462546	-2.03966849854353e-06\\
0.5463546	-2.97278575978055e-06\\
0.5464546	-3.89792600063998e-06\\
0.5465547	3.22670295149763e-06\\
0.5466547	3.18811812105579e-06\\
0.5467547	2.02287246198196e-06\\
0.5468547	6.85803850597466e-07\\
0.5469547	1.31601784580937e-07\\
0.5470547	-6.85192305560633e-07\\
0.5471547	-8.1018475661665e-07\\
0.5472547	-2.28912935984482e-06\\
0.5473547	-3.16792705357116e-06\\
0.5474547	-3.49262560828123e-06\\
0.5475548	3.8240046533744e-06\\
0.5476548	2.47321366053654e-06\\
0.5477548	1.537454143552e-06\\
0.5478548	9.70094075292138e-07\\
0.5479548	7.24356208792187e-07\\
0.5480548	-2.46681572946272e-07\\
0.5481548	-9.90085919916339e-07\\
0.5482548	-2.55306771679287e-06\\
0.5483548	-2.98298166878652e-06\\
0.5484548	-3.32732600583086e-06\\
0.5485549	3.54195650853484e-06\\
0.5486549	3.2296506020657e-06\\
0.5487549	1.85951408671059e-06\\
0.5488549	1.38347805389571e-06\\
0.5489549	7.53331900860132e-07\\
0.5490549	-7.92763543699948e-08\\
0.5491549	-1.16283968543129e-06\\
0.5492549	-2.54599169213776e-06\\
0.5493549	-3.27750621975653e-06\\
0.5494549	-3.40629698269979e-06\\
0.549555	3.23176990546514e-06\\
0.549655	3.1646077722769e-06\\
0.549755	1.5525434720054e-06\\
0.549855	1.34610752633435e-06\\
0.549955	4.95692525692704e-07\\
0.550055	-4.84464708350174e-08\\
0.550155	-1.33619153253051e-06\\
0.550255	-2.41756144969685e-06\\
0.550355	-3.34271134772024e-06\\
0.550455	-4.16193224328509e-06\\
0.5505551	3.32009803916122e-06\\
0.5506551	2.56430012246511e-06\\
0.5507551	1.76269609752921e-06\\
0.5508551	8.6445533398205e-07\\
0.5509551	8.18613411723845e-07\\
0.5510551	-4.25927468405973e-07\\
0.5511551	-9.20398049331084e-07\\
0.5512551	-1.716161587062e-06\\
0.5513551	-2.86471340427141e-06\\
0.5514551	-3.41768045508672e-06\\
0.5515552	3.84642586848116e-06\\
0.5516552	3.33168914856685e-06\\
0.5517552	2.25681896637298e-06\\
0.5518552	1.56966622100185e-06\\
0.5519552	2.17952441872171e-07\\
0.5520552	-8.50729752421842e-07\\
0.5521552	-1.68891618113065e-06\\
0.5522552	-2.34927063313961e-06\\
0.5523552	-2.8845844086689e-06\\
0.5524552	-3.34777584765078e-06\\
0.5525553	3.50365885815052e-06\\
0.5526553	3.02739063773316e-06\\
0.5527553	2.46367900835764e-06\\
0.5528553	7.5910221308817e-07\\
0.5529553	8.60113839706855e-07\\
0.5530553	-2.86956614026224e-07\\
0.5531553	-1.73590328156337e-06\\
0.5532553	-2.54064343074845e-06\\
0.5533553	-2.75521695058956e-06\\
0.5534553	-3.43378585032639e-06\\
0.5535554	3.68189285460119e-06\\
0.5536554	2.91376127226073e-06\\
0.5537554	2.51836606857125e-06\\
0.5538554	1.44106174015945e-06\\
0.5539554	6.27083238669002e-07\\
0.5540554	2.15464934782972e-08\\
0.5541554	-1.43055107670875e-06\\
0.5542554	-1.78433000996847e-06\\
0.5543554	-3.09502827722241e-06\\
0.5544554	-3.4180007846274e-06\\
0.5545555	3.51533856246533e-06\\
0.5546555	3.00213685910933e-06\\
0.5547555	2.30984451476957e-06\\
0.5548555	1.3826443554521e-06\\
0.5549555	1.6460501850446e-07\\
0.5550555	-4.00318492133067e-07\\
0.5551555	-1.36828422014901e-06\\
0.5552555	-1.79556271895365e-06\\
0.5553555	-2.73853649179756e-06\\
0.5554555	-4.25369941492448e-06\\
0.5555556	3.9323667344604e-06\\
0.5556556	3.10318727603232e-06\\
0.5557556	2.53161758934084e-06\\
0.5558556	1.16072389300115e-06\\
0.5559556	-6.65360939677839e-08\\
0.5560556	-2.07312581324004e-07\\
0.5561556	-1.31886309606699e-06\\
0.5562556	-2.45855193004019e-06\\
0.5563556	-2.68384950441458e-06\\
0.5564556	-4.05233180078568e-06\\
0.5565557	3.70863105292329e-06\\
0.5566557	2.88034449447139e-06\\
0.5567557	1.73545996240421e-06\\
0.5568557	1.21598516589216e-06\\
0.5569557	2.63825275403917e-07\\
0.5570557	-1.79216475437727e-07\\
0.5571557	-1.17143815003828e-06\\
0.5572557	-1.77123849187666e-06\\
0.5573557	-3.03711630778025e-06\\
0.5574557	-4.0276698642927e-06\\
0.5575558	3.52321815455525e-06\\
0.5576558	2.90625210386253e-06\\
0.5577558	2.38816624875682e-06\\
0.5578558	9.09970768070423e-07\\
0.5579558	4.12579567843352e-07\\
0.5580558	-1.63189079138704e-07\\
0.5581558	-8.76611896849511e-07\\
0.5582558	-1.78705994979822e-06\\
0.5583558	-2.95399800087637e-06\\
0.5584558	-3.43698384419611e-06\\
0.5585559	4.01776517389152e-06\\
0.5586559	2.72217606145375e-06\\
0.5587559	1.93125418324058e-06\\
0.5588559	1.58507604641756e-06\\
0.5589559	6.23628436891011e-07\\
0.5590559	-1.31909265732588e-08\\
0.5591559	-1.38557275075613e-06\\
0.5592559	-2.55379543645518e-06\\
0.5593559	-2.57822444371669e-06\\
0.5594559	-3.51931158881769e-06\\
0.559556	3.85847751438462e-06\\
0.559656	2.90030837213351e-06\\
0.559756	1.84355350185683e-06\\
0.559856	1.62742238796199e-06\\
0.559956	1.91041574559847e-07\\
0.560056	-5.26544619461333e-07\\
0.560156	-1.58637341129975e-06\\
0.560256	-2.0495628429984e-06\\
0.560356	-2.97731109721866e-06\\
0.560456	-3.43089577548028e-06\\
0.5605561	3.80097064311968e-06\\
0.5606561	3.10888676668952e-06\\
0.5607561	1.70660267873046e-06\\
0.5608561	1.53253011081489e-06\\
0.5609561	5.25004961282605e-07\\
0.5610561	-3.77712026100596e-07\\
0.5611561	-1.23743448821045e-06\\
0.5612561	-2.11604977540247e-06\\
0.5613561	-3.07551817702034e-06\\
0.5614561	-4.17787219753052e-06\\
0.5615562	3.75785196471412e-06\\
0.5616562	3.18004565524888e-06\\
0.5617562	2.27276792141851e-06\\
0.5618562	9.73704662676766e-07\\
0.5619562	2.20473253875753e-07\\
0.5620562	-4.93767231546371e-08\\
0.5621562	-8.98362723766866e-07\\
0.5622562	-2.38906846394826e-06\\
0.5623562	-2.58414321052669e-06\\
0.5624562	-3.54630099774766e-06\\
0.5625563	3.86895029436118e-06\\
0.5626563	3.18030485413345e-06\\
0.5627563	1.53599028873685e-06\\
0.5628563	8.7304101814567e-07\\
0.5629563	1.28430492019049e-07\\
0.5630563	-7.60928025478336e-07\\
0.5631563	-8.58180675855635e-07\\
0.5632563	-2.2265322758841e-06\\
0.5633563	-2.92924548705997e-06\\
0.5634563	-4.02964006340767e-06\\
0.5635564	3.57409254991836e-06\\
0.5636564	2.48359527788011e-06\\
0.5637564	1.80505915281515e-06\\
0.5638564	1.47494382662927e-06\\
0.5639564	4.2965580782095e-07\\
0.5640564	-3.94450767160492e-07\\
0.5641564	-1.06107334296723e-06\\
0.5642564	-1.63396014141171e-06\\
0.5643564	-3.17690937290394e-06\\
0.5644564	-3.75376841255104e-06\\
0.5645565	3.68831951025328e-06\\
0.5646565	2.84671220107668e-06\\
0.5647565	1.77930186051611e-06\\
0.5648565	1.42205237452231e-06\\
0.5649565	7.10882483545561e-07\\
0.5650565	-4.18333393703563e-07\\
0.5651565	-1.02976437243996e-06\\
0.5652565	-2.18762225578928e-06\\
0.5653565	-2.95616074907912e-06\\
0.5654565	-3.39967460671886e-06\\
0.5655566	3.47942433887027e-06\\
0.5656566	2.4870794410603e-06\\
0.5657566	1.62657235680541e-06\\
0.5658566	8.334522734188e-07\\
0.5659566	4.3231482393935e-08\\
0.5660566	-8.0861383899844e-07\\
0.5661566	-7.86642764261103e-07\\
0.5662566	-1.95544879222354e-06\\
0.5663566	-2.3796590093994e-06\\
0.5664566	-4.123933250888e-06\\
0.5665567	3.74769055433433e-06\\
0.5666567	3.16270011335718e-06\\
0.5667567	2.06341334596516e-06\\
0.5668567	1.38504783020466e-06\\
0.5669567	6.27926448615312e-08\\
0.5670567	3.18092006068582e-08\\
0.5671567	-7.72767910106609e-07\\
0.5672567	-2.41583006310719e-06\\
0.5673567	-2.96229371610224e-06\\
0.5674567	-3.47709960824005e-06\\
0.5675568	3.90769819080816e-06\\
0.5676568	2.25416217691787e-06\\
0.5677568	1.43725987822307e-06\\
0.5678568	1.39196219439697e-06\\
0.5679568	5.32200523650772e-08\\
0.5680568	-6.44034641616997e-07\\
0.5681568	-7.64888207882564e-07\\
0.5682568	-2.37444427586553e-06\\
0.5683568	-2.53782293668792e-06\\
0.5684568	-3.32015984927381e-06\\
0.5685569	3.07206051175513e-06\\
0.5686569	2.84856024901359e-06\\
0.5687569	1.81054483050502e-06\\
0.5688569	8.92825016585164e-07\\
0.5689569	3.02004101726538e-08\\
0.5690569	-8.42539715328883e-07\\
0.5691569	-7.90615505508185e-07\\
0.5692569	-1.87925562755709e-06\\
0.5693569	-3.17369639812881e-06\\
0.5694569	-3.73918091511882e-06\\
0.569557	3.13694599718417e-06\\
0.569657	2.8251879564678e-06\\
0.569757	2.04656002367187e-06\\
0.569857	7.35800977480494e-07\\
0.569957	8.27647325785108e-07\\
0.570057	-7.43165868399842e-07\\
0.570157	-1.04190400840309e-06\\
0.570257	-2.13383210567741e-06\\
0.570357	-3.08421388606206e-06\\
0.570457	-3.95831088706444e-06\\
0.5705571	3.86923465178768e-06\\
0.5706571	2.94284907287334e-06\\
0.5707571	1.89692066143721e-06\\
0.5708571	6.6620572880538e-07\\
0.5709571	1.85467293079e-07\\
0.5710571	-6.10523992250478e-07\\
0.5711571	-7.86988936241073e-07\\
0.5712571	-2.40913889903993e-06\\
0.5713571	-2.54217488393493e-06\\
0.5714571	-3.25128661060603e-06\\
0.5715572	2.99515079138502e-06\\
0.5716572	2.92862900508339e-06\\
0.5717572	2.09047391042105e-06\\
0.5718572	1.41555003274618e-06\\
0.5719572	-1.61262253328687e-07\\
0.5720572	-7.05065819062156e-07\\
0.5721572	-1.28094584272986e-06\\
0.5722572	-1.95396887825439e-06\\
0.5723572	-2.7891819671666e-06\\
0.5724572	-3.85161169980108e-06\\
0.5725573	3.29020911404854e-06\\
0.5726573	2.56796303155227e-06\\
0.5727573	1.42348753051635e-06\\
0.5728573	7.91847091896614e-07\\
0.5729573	6.08131250245947e-07\\
0.5730573	-1.92544444388432e-07\\
0.5731573	-6.75037516550958e-07\\
0.5732573	-1.90417763956674e-06\\
0.5733573	-2.94476570505964e-06\\
0.5734573	-3.86157288723155e-06\\
0.5735574	3.67030495951326e-06\\
0.5736574	2.79583147655416e-06\\
0.5737574	1.85094826576915e-06\\
0.5738574	7.71012356182155e-07\\
0.5739574	4.91415127257255e-07\\
0.5740574	-5.24167118598484e-08\\
0.5741574	-9.2502024973129e-07\\
0.5742574	-2.19089538244788e-06\\
0.5743574	-2.91450388889558e-06\\
0.5744574	-3.16026852242857e-06\\
0.5745575	3.28377509983824e-06\\
0.5746575	2.7889077509613e-06\\
0.5747575	1.57879536200056e-06\\
0.5748575	5.89180766397135e-07\\
0.5749575	7.55850507516698e-07\\
0.5750575	1.46358098973565e-08\\
0.5751575	-6.98586505087917e-07\\
0.5752575	-1.44789308187399e-06\\
0.5753575	-2.29731307577197e-06\\
0.5754575	-3.31082723370457e-06\\
0.5755576	3.60425007794873e-06\\
0.5756576	2.05847988965502e-06\\
0.5757576	2.15690943239366e-06\\
0.5758576	8.35761091444454e-07\\
0.5759576	3.13103685201099e-08\\
0.5760576	-3.20113166196734e-07\\
0.5761576	-1.28212493155644e-06\\
0.5762576	-1.91828442996922e-06\\
0.5763576	-2.29209426372279e-06\\
0.5764576	-3.46699920061866e-06\\
0.5765577	3.52411553183885e-06\\
0.5766577	2.54396257570733e-06\\
0.5767577	1.57267330447297e-06\\
0.5768577	5.47043717880058e-07\\
0.5769577	4.03932356896064e-07\\
0.5770577	8.02612243333556e-08\\
0.5771577	-1.4869832583031e-06\\
0.5772577	-1.36074929635299e-06\\
0.5773577	-2.60391881301558e-06\\
0.5774577	-3.27930648325037e-06\\
0.5775578	3.44839568811039e-06\\
0.5776578	2.7068123626961e-06\\
0.5777578	1.3449171429869e-06\\
0.5778578	1.30017379973424e-06\\
0.5779578	5.10118045582431e-07\\
0.5780578	-8.76415002792896e-08\\
0.5781578	-1.55542240865003e-06\\
0.5782578	-1.95546748660291e-06\\
0.5783578	-2.34994381731468e-06\\
0.5784578	-2.80094184734736e-06\\
0.5785579	3.38886902540025e-06\\
0.5786579	2.62465611111651e-06\\
0.5787579	1.61805764165024e-06\\
0.5788579	1.30729920932993e-06\\
0.5789579	6.30687783775841e-07\\
0.5790579	-4.73387407495807e-07\\
0.5791579	-1.06645390562932e-06\\
0.5792579	-2.20995511224231e-06\\
0.5793579	-1.96524933482323e-06\\
0.5794579	-3.39360885526219e-06\\
0.579558	3.05822348956042e-06\\
0.579658	2.085439385624e-06\\
0.579758	1.25623765256933e-06\\
0.579858	5.0969954568103e-07\\
0.579958	-2.15002993186886e-07\\
0.580058	2.13936095505574e-08\\
0.580158	-8.41754510538806e-07\\
0.580258	-1.86499772691917e-06\\
0.580358	-2.10879200146508e-06\\
0.580458	-2.63349793705459e-06\\
0.5805581	2.96405591715398e-06\\
0.5806581	2.68139949710644e-06\\
0.5807581	1.93727099873087e-06\\
0.5808581	6.71700600918257e-07\\
0.5809581	-1.75181502193311e-07\\
0.5810581	-6.63144217760703e-07\\
0.5811581	-8.51854606409574e-07\\
0.5812581	-1.80087695511588e-06\\
0.5813581	-2.56967187617363e-06\\
0.5814581	-3.21759536259236e-06\\
0.5815582	2.50251891475983e-06\\
0.5816582	1.90266627786428e-06\\
0.5817582	1.24619497743339e-06\\
0.5818582	4.74176775178137e-07\\
0.5819582	5.2779264070324e-07\\
0.5820582	-6.51666333340017e-07\\
0.5821582	-1.12279801367521e-06\\
0.5822582	-1.94408831077908e-06\\
0.5823582	-2.17391026779978e-06\\
0.5824582	-2.87052315250946e-06\\
0.5825583	3.05141680101428e-06\\
0.5826583	2.23029197954361e-06\\
0.5827583	1.76823238273727e-06\\
0.5828583	6.07443087830006e-07\\
0.5829583	6.90247496493157e-07\\
0.5830583	-4.0911762866358e-08\\
0.5831583	-6.43471907757487e-07\\
0.5832583	-1.17474912730842e-06\\
0.5833583	-1.69193767440845e-06\\
0.5834583	-2.25210894644334e-06\\
0.5835584	3.0625517934979e-06\\
0.5836584	2.2285115566234e-06\\
0.5837584	1.1809637756599e-06\\
0.5838584	8.63337517031937e-07\\
0.5839584	2.1918917614272e-07\\
0.5840584	-8.07796635804436e-07\\
0.5841584	-1.27380620096318e-06\\
0.5842584	-1.23489579806346e-06\\
0.5843584	-1.74699081867402e-06\\
0.5844584	-2.86588487963257e-06\\
0.5845585	2.15312108320376e-06\\
0.5846585	1.63603386216238e-06\\
0.5847585	1.34551090003043e-06\\
0.5848585	1.22629465426627e-06\\
0.5849585	2.23263769250082e-07\\
0.5850585	-7.18566068869109e-07\\
0.5851585	-6.54041210523815e-07\\
0.5852585	-1.63786919937792e-06\\
0.5853585	-1.72461789649248e-06\\
0.5854585	-2.96871459326553e-06\\
0.5855586	2.19596609607464e-06\\
0.5856586	1.45616525148995e-06\\
0.5857586	1.39653260688988e-06\\
0.5858586	9.63211928883823e-07\\
0.5859586	1.02491865661136e-07\\
0.5860586	-2.3919320035759e-07\\
0.5861586	-1.11526225676783e-06\\
0.5862586	-1.57898683283975e-06\\
0.5863586	-1.68349014195712e-06\\
0.5864586	-2.48174621653163e-06\\
0.5865587	2.40847583032888e-06\\
0.5866587	2.04556750205143e-06\\
0.5867587	8.30835416465447e-07\\
0.5868587	7.11910868034238e-07\\
0.5869587	6.36578604940041e-07\\
0.5870587	-4.47222403110459e-07\\
0.5871587	-5.91398078064742e-07\\
0.5872587	-1.84769840050336e-06\\
0.5873587	-2.26771658873304e-06\\
0.5874587	-2.90288825954832e-06\\
0.5875588	2.43994788640123e-06\\
0.5876588	2.20145390716908e-06\\
0.5877588	1.59440266145339e-06\\
0.5878588	5.67997407330267e-07\\
0.5879588	7.16031465231026e-08\\
0.5880588	5.47474447998297e-08\\
0.5881588	-5.32878749126553e-07\\
0.5882588	-1.74142027820068e-06\\
0.5883588	-1.62085699262349e-06\\
0.5884588	-2.22100289182237e-06\\
0.5885589	2.45721255431874e-06\\
0.5886589	1.24702905779372e-06\\
0.5887589	1.16764816393555e-06\\
0.5888589	1.16992741983069e-06\\
0.5889589	2.04894230471098e-07\\
0.5890589	-7.76253346046474e-07\\
0.5891589	-8.22145816314901e-07\\
0.5892589	-9.81241427666646e-07\\
0.5893589	-1.30182538882195e-06\\
0.5894589	-1.83200908843162e-06\\
0.589559	2.22832811935803e-06\\
0.589659	2.11498085800343e-06\\
0.589759	1.6487029990131e-06\\
0.589859	7.82086471140531e-07\\
0.589959	4.67900941991672e-07\\
0.590059	-3.4090541412013e-07\\
0.590159	-6.91205142011597e-07\\
0.590259	-1.62969074057884e-06\\
0.590359	-1.20287388494233e-06\\
0.590459	-2.45708466906081e-06\\
0.5905591	2.20415844687238e-06\\
0.5906591	1.42883611786715e-06\\
0.5907591	8.34547246419959e-07\\
0.5908591	3.75695827337097e-07\\
0.5909591	6.87121648468292e-09\\
0.5910591	-3.17151116924208e-07\\
0.5911591	-6.41408845147851e-07\\
0.5912591	-1.01075206160317e-06\\
0.5913591	-1.46984249571602e-06\\
0.5914591	-2.0631528130366e-06\\
0.5915592	1.59764772789117e-06\\
0.5916592	1.5819950096585e-06\\
0.5917592	1.29980237062455e-06\\
0.5918592	7.07360963581039e-07\\
0.5919592	-2.38845352384942e-07\\
0.5920592	-5.82139279892147e-07\\
0.5921592	-3.65649358791131e-07\\
0.5922592	-6.32309260950592e-07\\
0.5923592	-1.42485707321782e-06\\
0.5924592	-1.78583459220505e-06\\
0.5925593	1.46061134831399e-06\\
0.5926593	1.81426376499871e-06\\
0.5927593	1.47300402275619e-06\\
0.5928593	3.95082722981499e-07\\
0.5929593	5.38950247097603e-07\\
0.5930593	-1.36742538181522e-07\\
0.5931593	-6.73143589757785e-07\\
0.5932593	-1.1111990216861e-06\\
0.5933593	-1.49165241136728e-06\\
0.5934593	-1.85504412275606e-06\\
0.5935594	1.75786664602384e-06\\
0.5936594	1.28571002466771e-06\\
0.5937594	7.10180191099141e-07\\
0.5938594	9.91556563967322e-07\\
0.5939594	9.03251251571646e-08\\
0.5940594	-3.28209246447386e-08\\
0.5941594	-4.16980522999211e-07\\
0.5942594	-1.10104406303435e-06\\
0.5943594	-1.12369277083246e-06\\
0.5944594	-1.52339804415647e-06\\
0.5945595	1.43853258016691e-06\\
0.5946595	1.14767015269024e-06\\
0.5947595	1.36556459739978e-06\\
0.5948595	5.45904255133678e-08\\
0.5949595	1.77335156337222e-07\\
0.5950595	-3.03400055301495e-07\\
0.5951595	-4.24599784665247e-07\\
0.5952595	-1.22303374361366e-06\\
0.5953595	-1.73525613966774e-06\\
0.5954595	-1.99760507957336e-06\\
0.5955596	1.50433294088259e-06\\
0.5956596	1.61074405813366e-06\\
0.5957596	8.59281490406261e-07\\
0.5958596	2.14477866578022e-07\\
0.5959596	-3.58915077214306e-07\\
0.5960596	1.04074123541409e-07\\
0.5961596	-4.31362764174992e-07\\
0.5962596	-9.99813081214995e-07\\
0.5963596	-1.63564268351735e-06\\
0.5964596	-1.37299540003966e-06\\
0.5965597	1.07474424471121e-06\\
0.5966597	1.00961903282126e-06\\
0.5967597	7.41843967588807e-07\\
0.5968597	2.38169383770526e-07\\
0.5969597	4.65570511920532e-07\\
0.5970597	-6.08751965280874e-07\\
0.5971597	-1.73713639028961e-08\\
0.5972597	-7.92634453183894e-07\\
0.5973597	-9.66660883783277e-07\\
0.5974597	-1.57134269063164e-06\\
0.5975598	1.44883588060196e-06\\
0.5976598	8.64571105374523e-07\\
0.5977598	7.55315014977498e-07\\
0.5978598	9.00917438428905e-08\\
0.5979598	-1.61844289969792e-07\\
0.5980598	-3.10078522858248e-08\\
0.5981598	-5.47682350138246e-07\\
0.5982598	-7.41919364166677e-07\\
0.5983598	-6.43538106359642e-07\\
0.5984598	-1.28212493422097e-06\\
0.5985599	1.16365747615532e-06\\
0.5986599	9.3949944179883e-07\\
0.5987599	8.9098765876372e-07\\
0.5988599	-1.0527562821494e-08\\
0.5989599	2.065394077988e-07\\
0.5990599	-4.8598997448579e-07\\
0.5991599	-1.16057969457017e-07\\
0.5992599	-7.11370072181694e-07\\
0.5993599	-1.29939454751593e-06\\
0.5994599	-9.07361944690876e-07\\
0.59956	1.04903599318362e-06\\
0.59966	1.29635719492427e-06\\
0.59976	4.43448223919063e-07\\
0.59986	4.6403416043006e-07\\
0.59996	3.32080072773522e-07\\
0.60006	2.17914157829568e-08\\
0.60016	-4.92385472838919e-07\\
0.60026	-2.35762956712904e-07\\
0.60036	-1.23341170432667e-06\\
0.60046	-1.5101602031109e-06\\
0.6005601	1.2786530350084e-06\\
0.6006601	3.4585009434096e-07\\
0.6007601	6.08950321279167e-08\\
0.6008601	3.99932371664136e-07\\
0.6009601	3.393508736238e-07\\
0.6010601	-1.44216082631488e-07\\
0.6011601	-7.38900860497438e-08\\
0.6012601	-4.72547307950322e-07\\
0.6013601	-3.62818116883545e-07\\
0.6014601	-7.67086667075034e-07\\
0.6015602	1.41728136782149e-06\\
0.6016602	8.94282599972485e-07\\
0.6017602	7.91593040005978e-07\\
0.6018602	8.78173258556103e-08\\
0.6019602	-2.38191825374656e-07\\
0.6020602	-2.07333293644751e-07\\
0.6021602	1.59742846506106e-07\\
0.6022602	-1.57364355324319e-07\\
0.6023602	-1.17880636540946e-06\\
0.6024602	-9.24484806930082e-07\\
0.6025603	4.64068465433343e-07\\
0.6026603	1.86440340499416e-07\\
0.6027603	1.26354206209101e-07\\
0.6028603	2.64911362179987e-07\\
0.6029603	-4.16535421265962e-07\\
0.6030603	6.36184553926e-08\\
0.6031603	-3.12770294108589e-07\\
0.6032603	-5.63592529090329e-07\\
0.6033603	-7.06486366297554e-07\\
0.6034603	-7.58836876890712e-07\\
0.6035604	8.91763851207372e-07\\
0.6036604	9.4440659559325e-07\\
0.6037604	3.69421409018855e-08\\
0.6038604	1.53000782709967e-07\\
0.6039604	2.76467279292092e-07\\
0.6040604	3.91481160377793e-07\\
0.6041604	-5.17563004898136e-07\\
0.6042604	-4.66015347200255e-07\\
0.6043604	-4.68970436706684e-07\\
0.6044604	-5.41267008991042e-07\\
0.6045605	6.81796734625095e-07\\
0.6046605	4.02220200257375e-07\\
0.6047605	1.03094892622835e-08\\
0.6048605	4.92251914785413e-07\\
0.6049605	-1.65508145855142e-07\\
0.6050605	2.37309611961223e-08\\
0.6051605	4.69284087145638e-08\\
0.6052605	-1.0869889521814e-07\\
0.6053605	-4.55676074651024e-07\\
0.6054605	-1.00627009214094e-06\\
0.6055606	3.55119894379641e-07\\
0.6056606	3.36291903657582e-07\\
0.6057606	7.85859572971503e-08\\
0.6058606	5.70770119168174e-07\\
0.6059606	-1.9812836971056e-07\\
0.6060606	-2.3882291166899e-07\\
0.6061606	-5.61767329543272e-07\\
0.6062606	-1.77155738967372e-07\\
0.6063606	-9.49223180057857e-08\\
0.6064606	-3.24741185941946e-07\\
0.6065607	-1.25396582006942e-09\\
0.6066607	9.15062985384907e-08\\
0.6067607	-1.55254700651142e-07\\
0.6068607	2.49831122012267e-07\\
0.6069607	2.98392742514864e-07\\
0.6070607	-1.76798922169041e-08\\
0.6071607	2.93764350622894e-07\\
0.6072607	2.25137862130964e-07\\
0.6073607	-2.30885518703872e-07\\
0.6074607	-8.13704374991175e-08\\
0.6075608	2.87912839347371e-07\\
0.6076608	6.029467041202e-07\\
0.6077608	5.03889417302616e-07\\
0.6078608	-1.52760577520894e-08\\
0.6079608	3.96953456771598e-08\\
0.6080608	-3.36689097046872e-07\\
0.6081608	-1.4965981787185e-07\\
0.6082608	-4.04184849145395e-07\\
0.6083608	-1.04969766212548e-07\\
0.6084608	-2.56457580505298e-07\\
0.6085609	5.03823460284991e-07\\
0.6086609	4.131907500593e-07\\
0.6087609	-1.39901347395721e-07\\
0.6088609	-1.58844628117549e-07\\
0.6089609	3.53231977889834e-07\\
0.6090609	3.93462504355568e-07\\
0.6091609	-4.07560243331773e-08\\
0.6092609	4.8236453942252e-08\\
0.6093609	-3.41636913425702e-07\\
0.6094609	-2.12189871895418e-07\\
0.609561	5.46919894262032e-07\\
0.609661	-3.14864919914726e-07\\
0.609761	3.38807554456366e-07\\
0.609861	-4.92823437170387e-07\\
0.609961	1.89744599587982e-07\\
0.610061	3.86277367425691e-07\\
0.610161	9.68038342819e-08\\
0.610261	3.21616193232899e-07\\
0.610361	6.12698665136691e-08\\
0.610461	3.16583512294244e-07\\
0.6105611	-5.43428946286895e-08\\
0.6106611	2.10318336968385e-07\\
0.6107611	-5.32357979921017e-09\\
0.6108611	3.00602986946785e-07\\
0.6109611	1.30232776562877e-07\\
0.6110611	4.85963599139438e-07\\
0.6111611	3.70456295861743e-07\\
0.6112611	-2.13365314749581e-07\\
0.6113611	-2.62314478582937e-07\\
0.6114611	2.27058454704832e-07\\
0.6115612	-1.39243060637284e-07\\
0.6116612	-5.87277728669733e-07\\
0.6117612	-4.85060353661027e-07\\
0.6118612	1.71909829482786e-07\\
0.6119612	3.88396158790272e-07\\
0.6120612	1.69424388296591e-07\\
0.6121612	5.20282672944461e-07\\
0.6122612	4.46521427832636e-07\\
0.6123612	-4.60467450835722e-08\\
0.6124612	4.86528408671916e-08\\
0.6125613	8.49314334416817e-08\\
0.6126613	-6.51958007580333e-07\\
0.6127613	2.18222782066846e-07\\
0.6128613	-2.97404000981771e-07\\
0.6129613	-1.91454665454671e-07\\
0.6130613	5.43715881207163e-07\\
0.6131613	-8.39860101464751e-08\\
0.6132613	-6.63928796384994e-08\\
0.6133613	6.04923729774498e-07\\
0.6134613	-6.13469097743291e-08\\
0.6135614	3.80763331975231e-08\\
0.6136614	-3.01573096272989e-07\\
0.6137614	4.88311711066558e-08\\
0.6138614	9.90204735984435e-08\\
0.6139614	-1.41013797083644e-07\\
0.6140614	3.38979615932544e-07\\
0.6141614	5.49511662839564e-07\\
0.6142614	5.01352813486733e-07\\
0.6143614	2.0553286539382e-07\\
0.6144614	6.73340763590602e-07\\
0.6145615	-2.42053735277636e-07\\
0.6146615	-2.37294779381614e-07\\
0.6147615	-4.33476083472328e-07\\
0.6148615	1.81726110604075e-07\\
0.6149615	-3.79106286807485e-07\\
0.6150615	-1.03133475093387e-07\\
0.6151615	2.27420606790929e-08\\
0.6152615	1.18752776501196e-08\\
0.6153615	8.77878381544406e-07\\
0.6154615	6.34620571204891e-07\\
0.6155616	-1.11366591593054e-06\\
0.6156616	-5.57884598961778e-07\\
0.6157616	-6.82022065490173e-08\\
0.6158616	-6.29723698075679e-07\\
0.6159616	-2.27298253108188e-07\\
0.6160616	1.54480439995552e-07\\
0.6161616	5.3127394838981e-07\\
0.6162616	-8.10011826501977e-08\\
0.6163616	3.33826291942785e-07\\
0.6164616	7.92182037301359e-07\\
0.6165617	-3.49212301742341e-07\\
0.6166617	-7.78421011204955e-07\\
0.6167617	-1.13283395108965e-07\\
0.6168617	-3.36358699115635e-07\\
0.6169617	-4.29953176173115e-07\\
0.6170617	-3.76120408596137e-07\\
0.6171617	-1.56661609906905e-07\\
0.6172617	2.46874046538892e-07\\
0.6173617	8.53189143334987e-07\\
0.6174617	6.81237661837031e-07\\
0.6175618	-1.15809176293169e-06\\
0.6176618	-8.53440162096319e-07\\
0.6177618	-2.68668092218149e-07\\
0.6178618	-3.83818934857061e-07\\
0.6179618	-1.7868629775819e-07\\
0.6180618	3.67185597482944e-07\\
0.6181618	2.7450160278164e-07\\
0.6182618	5.64215258513912e-07\\
0.6183618	1.25752847335292e-06\\
0.6184618	1.37589113169412e-06\\
0.6185619	-1.21371576877749e-06\\
0.6186619	-1.20443473416998e-06\\
0.6187619	-7.0424807230296e-07\\
0.6188619	-6.90717424589593e-07\\
0.6189619	-1.41158317923384e-07\\
0.6190619	-3.26405533712659e-08\\
0.6191619	6.58011402698833e-07\\
0.6192619	9.5421802015494e-07\\
0.6193619	8.79644304063731e-07\\
0.6194619	1.4581993967866e-06\\
0.619562	-6.84873384670937e-07\\
0.619662	-7.51644773089311e-07\\
0.619762	-9.20744502863613e-08\\
0.619862	-6.81280596381839e-07\\
0.619962	-4.94139349882516e-07\\
0.620062	4.94714791798856e-07\\
0.620162	3.10888512267127e-07\\
0.620262	9.80229276414946e-07\\
0.620362	5.28824855017263e-07\\
0.620462	9.83002907517516e-07\\
0.6205621	-1.27132018512555e-06\\
0.6206621	-9.50064492855063e-07\\
0.6207621	-6.42781424531336e-07\\
0.6208621	-3.22188085988273e-07\\
0.6209621	3.92360188783414e-08\\
0.6210621	4.69248500500896e-07\\
0.6211621	-4.15634460182446e-09\\
0.6212621	6.47251948926453e-07\\
0.6213621	1.45193956324619e-06\\
0.6214621	1.43860789680872e-06\\
0.6215622	-1.24350603591949e-06\\
0.6216622	-8.2957883051904e-07\\
0.6217622	-1.14613071122704e-06\\
0.6218622	-1.63524052965158e-07\\
0.6219622	1.48111512388027e-07\\
0.6220622	-1.81121381004345e-07\\
0.6221622	8.79111621721052e-07\\
0.6222622	3.59376105762976e-07\\
0.6223622	1.29046834551616e-06\\
0.6224622	1.70341482252923e-06\\
0.6225623	-1.48634697350047e-06\\
0.6226623	-1.03913407567546e-06\\
0.6227623	-1.0155798646494e-06\\
0.6228623	-3.83742189136171e-07\\
0.6229623	-1.11451377016181e-07\\
0.6230623	-1.66310780258527e-07\\
0.6231623	4.84302691283744e-07\\
0.6232623	8.7323799924377e-07\\
0.6233623	1.03356943625954e-06\\
0.6234623	1.99859607796782e-06\\
0.6235624	-1.54693679821349e-06\\
0.6236624	-8.94837384635139e-07\\
0.6237624	-3.36768517072983e-07\\
0.6238624	-8.38537488334623e-07\\
0.6239624	-3.65729649232094e-07\\
0.6240624	1.16290982177247e-07\\
0.6241624	6.42381150051108e-07\\
0.6242624	1.24761779307647e-06\\
0.6243624	9.6729741105861e-07\\
0.6244624	1.83693552013864e-06\\
0.6245625	-1.68608522432834e-06\\
0.6246625	-1.43187090984043e-06\\
0.6247625	-9.19807327193212e-07\\
0.6248625	-1.13508772159321e-07\\
0.6249625	2.36264172670531e-08\\
0.6250625	5.28415275091731e-07\\
0.6251625	4.37889577931116e-07\\
0.6252625	7.89295219405517e-07\\
0.6253625	1.62009159954124e-06\\
0.6254625	1.96795098261759e-06\\
0.6255626	-1.93356037048176e-06\\
0.6256626	-1.46009767920674e-06\\
0.6257626	-1.35524588795732e-06\\
0.6258626	-5.80487376389272e-07\\
0.6259626	-9.70948450529363e-08\\
0.6260626	1.33868041096719e-07\\
0.6261626	1.51545997972846e-07\\
0.6262626	9.95291473238069e-07\\
0.6263626	1.70466397975133e-06\\
0.6264626	2.31942946271602e-06\\
0.6265627	-2.14690521893601e-06\\
0.6266627	-1.62322793340053e-06\\
0.6267627	-1.07358755840892e-06\\
0.6268627	-4.57398963149558e-07\\
0.6269627	2.66126038539483e-07\\
0.6270627	1.3797803077864e-07\\
0.6271627	1.99349289964346e-07\\
0.6272627	1.49163310680933e-06\\
0.6273627	1.05642311254428e-06\\
0.6274627	1.93551257954283e-06\\
0.6275628	-2.07368951343767e-06\\
0.6276628	-1.46140660195471e-06\\
0.6277628	-1.40821107397215e-06\\
0.6278628	-8.71517900513652e-07\\
0.6279628	1.91454105014088e-07\\
0.6280628	-1.76318456723834e-07\\
0.6281628	1.0683357318797e-06\\
0.6282628	9.68781988852641e-07\\
0.6283628	1.56857891742135e-06\\
0.6284628	1.91147767125699e-06\\
0.6285629	-2.41705117787205e-06\\
0.6286629	-1.47707533315611e-06\\
0.6287629	-6.61551652569869e-07\\
0.6288629	-9.25965806253259e-07\\
0.6289629	-2.25614553706066e-07\\
0.6290629	4.84393515520765e-07\\
0.6291629	2.49137261043586e-07\\
0.6292629	1.11388220691566e-06\\
0.6293629	2.1240798453448e-06\\
0.6294629	2.32536685906837e-06\\
0.629563	-1.90437399538723e-06\\
0.629663	-1.20395514091598e-06\\
0.629763	-1.17438693170158e-06\\
0.629863	-7.69299245639843e-07\\
0.629963	5.78594203659577e-08\\
0.630063	3.5382126473138e-07\\
0.630163	1.16549832984703e-06\\
0.630263	1.53998176877224e-06\\
0.630363	1.52454110491362e-06\\
0.630463	2.16662338026197e-06\\
0.6305631	-2.35894193689745e-06\\
0.6306631	-1.27898925850189e-06\\
0.6307631	-1.39806677612597e-06\\
0.6308631	-6.68024775940523e-07\\
0.6309631	-4.05399180891663e-08\\
0.6310631	5.32884008919154e-07\\
0.6311631	1.10091528693346e-06\\
0.6312631	1.71239347235286e-06\\
0.6313631	1.416328605508e-06\\
0.6314631	2.26190039143148e-06\\
0.6315632	-1.77440390203287e-06\\
0.6316632	-1.51707959261671e-06\\
0.6317632	-9.69519661886409e-07\\
0.6318632	-1.0818735436402e-06\\
0.6319632	1.95874968689225e-07\\
0.6320632	-8.60930819968075e-08\\
0.6321632	1.12256737860861e-06\\
0.6322632	8.7236461343565e-07\\
0.6323632	2.21396927102546e-06\\
0.6324632	2.19821357205774e-06\\
0.6325633	-2.39187743922287e-06\\
0.6326633	-1.98844466048342e-06\\
0.6327633	-7.88863085965374e-07\\
0.6328633	-7.41662371517293e-07\\
0.6329633	2.04785273183461e-07\\
0.6330633	1.02264207768599e-07\\
0.6331633	1.00271454606116e-06\\
0.6332633	9.58231311187774e-07\\
0.6333633	2.02106360802645e-06\\
0.6334633	2.24361375344984e-06\\
0.6335634	-2.77951439997537e-06\\
0.6336634	-2.09842147480543e-06\\
0.6337634	-1.09943987514072e-06\\
0.6338634	-7.29562669832262e-07\\
0.6339634	6.436605293203e-08\\
0.6340634	3.35650329752468e-07\\
0.6341634	1.137741502788e-06\\
0.6342634	1.52423730881424e-06\\
0.6343634	1.54888105274864e-06\\
0.6344634	2.26556073545936e-06\\
0.6345635	-2.91434706856109e-06\\
0.6346635	-1.6695296167768e-06\\
0.6347635	-1.57009825851873e-06\\
0.6348635	-5.61594704073798e-07\\
0.6349635	-5.89420291241538e-07\\
0.6350635	4.01163132046634e-07\\
0.6351635	4.65032354846073e-07\\
0.6352635	1.65720189926333e-06\\
0.6353635	2.03282316491027e-06\\
0.6354635	2.64718354170412e-06\\
0.6355636	-2.26622917676167e-06\\
0.6356636	-2.02561125206557e-06\\
0.6357636	-1.37952913004113e-06\\
0.6358636	-1.2721601132526e-06\\
0.6359636	-6.47549956767079e-07\\
0.6360636	5.5038630986104e-07\\
0.6361636	1.37786343756829e-06\\
0.6362636	1.89122511429218e-06\\
0.6363636	2.14694302513863e-06\\
0.6364636	3.20161599187685e-06\\
0.6365637	-2.88368232492786e-06\\
0.6366637	-2.07785769923419e-06\\
0.6367637	-1.30247067531286e-06\\
0.6368637	-5.00422745020046e-07\\
0.6369637	-6.14492765116381e-07\\
0.6370637	4.12662169324562e-07\\
0.6371637	6.38505785133248e-07\\
0.6372637	1.12062175627869e-06\\
0.6373637	1.91671278670924e-06\\
0.6374637	3.08459971432029e-06\\
0.6375638	-2.48145585057813e-06\\
0.6376638	-2.41252978305795e-06\\
0.6377638	-1.79758741758462e-06\\
0.6378638	-5.78344377721862e-07\\
0.6379638	-6.96402705280263e-07\\
0.6380638	-9.32517796314869e-08\\
0.6381638	1.2897307666293e-06\\
0.6382638	1.51127814707053e-06\\
0.6383638	2.63023350743907e-06\\
0.6384638	2.70554897774389e-06\\
0.6385639	-2.52960499036448e-06\\
0.6386639	-2.38017717713035e-06\\
0.6387639	-1.09682864035676e-06\\
0.6388639	-6.20180271582171e-07\\
0.6389639	1.09251409696753e-07\\
0.6390639	1.51053391128642e-07\\
0.6391639	5.64915209722017e-07\\
0.6392639	1.41062805614212e-06\\
0.6393639	2.74808382183522e-06\\
0.6394639	2.63727419191895e-06\\
0.639564	-2.34389097464316e-06\\
0.639664	-2.1861596577466e-06\\
0.639764	-1.29606889842648e-06\\
0.639864	-6.13237025604008e-07\\
0.639964	-7.71872397109519e-08\\
0.640064	3.72651477675845e-07\\
0.640164	7.96943445457998e-07\\
0.640264	1.25644531578928e-06\\
0.640364	2.81200519225422e-06\\
0.640464	3.52456164964821e-06\\
0.6405641	-3.1773045794381e-06\\
0.6406641	-1.98227216952063e-06\\
0.6407641	-1.44683259062361e-06\\
0.6408641	-5.0969454967742e-07\\
0.6409641	-1.09480931875794e-07\\
0.6410641	8.15270245979605e-07\\
0.6411641	1.32610492364904e-06\\
0.6412641	1.48465204841131e-06\\
0.6413641	2.35262264070002e-06\\
0.6414641	2.99180885132699e-06\\
0.6415642	-3.31251456131554e-06\\
0.6416642	-1.9592747317887e-06\\
0.6417642	-1.64890332232659e-06\\
0.6418642	-1.31929295310584e-06\\
0.6419642	9.17402038780324e-08\\
0.6420642	6.46455479369479e-07\\
0.6421642	1.407186759117e-06\\
0.6422642	1.43634155769234e-06\\
0.6423642	2.79640005285131e-06\\
0.6424642	3.54991419460049e-06\\
0.6425643	-3.15504151071622e-06\\
0.6426643	-2.44012787664616e-06\\
0.6427643	-1.14361898440052e-06\\
0.6428643	-1.20268529268941e-06\\
0.6429643	-5.54430201304967e-07\\
0.6430643	8.64108955944687e-07\\
0.6431643	1.11596000706982e-06\\
0.6432643	2.26421498794593e-06\\
0.6433643	2.37202920505553e-06\\
0.6434643	3.50262029513715e-06\\
0.6435644	-3.32695838345387e-06\\
0.6436644	-1.97373495502973e-06\\
0.6437644	-1.40765379263996e-06\\
0.6438644	-5.65257311180289e-07\\
0.6439644	-3.83030324790923e-07\\
0.6440644	2.02599053977792e-07\\
0.6441644	1.25525842609164e-06\\
0.6442644	1.83863020808417e-06\\
0.6443644	3.01645065192702e-06\\
0.6444644	2.85250892240185e-06\\
0.6445645	-2.76091944506618e-06\\
0.6446645	-2.42899305114008e-06\\
0.6447645	-1.24708926207973e-06\\
0.6448645	-1.1512165958294e-06\\
0.6449645	-7.733537543686e-08\\
0.6450645	3.86413674391406e-08\\
0.6451645	1.26084692020356e-06\\
0.6452645	1.65345998137667e-06\\
0.6453645	2.28070368901712e-06\\
0.6454645	3.20684471510901e-06\\
0.6455646	-2.79432108651889e-06\\
0.6456646	-2.08895625330996e-06\\
0.6457646	-1.89157793784034e-06\\
0.6458646	-1.13775472510014e-06\\
0.6459646	2.36983645152122e-07\\
0.6460646	2.97145314576142e-07\\
0.6461646	1.10727542068645e-06\\
0.6462646	1.73195510466684e-06\\
0.6463646	2.23580061931017e-06\\
0.6464646	3.68346239376649e-06\\
0.6465647	-3.26339941025822e-06\\
0.6466647	-2.74491625695816e-06\\
0.6467647	-2.08840515547593e-06\\
0.6468647	-1.22908831734492e-06\\
0.6469647	-1.02158444370559e-07\\
0.6470647	3.57220349300746e-07\\
0.6471647	1.21391159169093e-06\\
0.6472647	1.53280554293644e-06\\
0.6473647	2.37881824016739e-06\\
0.6474647	3.81689061690338e-06\\
0.6475648	-3.59707235109141e-06\\
0.6476648	-2.79020848825695e-06\\
0.6477648	-2.19625721786798e-06\\
0.6478648	-7.50187378883993e-07\\
0.6479648	-3.86947553110417e-07\\
0.6480648	9.58532996087058e-07\\
0.6481648	1.35134341228138e-06\\
0.6482648	1.85659032858609e-06\\
0.6483648	2.53939693450889e-06\\
0.6484648	3.46490207991934e-06\\
0.6485649	-2.910336479367e-06\\
0.6486649	-2.31355408253364e-06\\
0.6487649	-2.27850670864171e-06\\
0.6488649	-7.4000207028746e-07\\
0.6489649	-6.32836808733828e-07\\
0.6490649	1.08202590887174e-07\\
0.6491649	1.54833888821315e-06\\
0.6492649	1.75280320302562e-06\\
0.6493649	2.78683406396141e-06\\
0.6494649	3.71567652468485e-06\\
0.649565	-3.0970325450852e-06\\
0.649665	-2.19175220017576e-06\\
0.649765	-2.19582975624633e-06\\
0.649865	-1.04400313460928e-06\\
0.649965	-6.71008249852889e-07\\
0.650065	9.88420079117702e-07\\
0.650165	9.99547228097697e-07\\
0.650265	2.42763788538269e-06\\
0.650365	2.33795512816215e-06\\
0.650465	3.79575953957101e-06\\
0.6505651	-2.92179716776531e-06\\
0.6506651	-2.18154043984953e-06\\
0.6507651	-1.69797343740896e-06\\
0.6508651	-1.40585449059927e-06\\
0.6509651	-2.39948888314245e-07\\
0.6510651	8.64970255420872e-07\\
0.6511651	9.74121063812561e-07\\
0.6512651	2.15271208947598e-06\\
0.6513651	2.46594134978295e-06\\
0.6514651	3.97899548953973e-06\\
0.6515652	-3.1110226546005e-06\\
0.6516652	-3.0104459582958e-06\\
0.6517652	-1.51449638696732e-06\\
0.6518652	-1.55804161217077e-06\\
0.6519652	-7.59650973236603e-08\\
0.6520652	9.96833050770363e-07\\
0.6521652	1.72543521870239e-06\\
0.6522652	2.17490539977661e-06\\
0.6523652	2.41028832981272e-06\\
0.6524652	3.49660862308809e-06\\
0.6525653	-3.44265171658975e-06\\
0.6526653	-2.46645423551684e-06\\
0.6527653	-1.44430998805234e-06\\
0.6528653	-1.31128351732457e-06\\
0.6529653	-2.46379450175027e-09\\
0.6530653	5.47034905373778e-07\\
0.6531653	1.40207215970634e-06\\
0.6532653	1.62748054677309e-06\\
0.6533653	3.28806480620614e-06\\
0.6534653	3.44860094880772e-06\\
0.6535654	-3.83463858355526e-06\\
0.6536654	-2.48632788046876e-06\\
0.6537654	-1.44385327693897e-06\\
0.6538654	-6.42562056896168e-07\\
0.6539654	-1.7834456578214e-08\\
0.6540654	4.94915513016281e-07\\
0.6541654	9.60239258773754e-07\\
0.6542654	2.4426527112098e-06\\
0.6543654	3.00663554586578e-06\\
0.6544654	3.71663032527891e-06\\
0.6545655	-3.43191413065824e-06\\
0.6546655	-2.24241340518461e-06\\
0.6547655	-1.71373971191713e-06\\
0.6548655	-7.81607258915074e-07\\
0.6549655	-3.81771497970362e-07\\
0.6550655	5.49970057051041e-07\\
0.6551655	1.07777701607858e-06\\
0.6552655	2.26576528206124e-06\\
0.6553655	3.17800624127784e-06\\
0.6554655	3.87852597771854e-06\\
0.6555656	-3.69169789848911e-06\\
0.6556656	-2.22777959546505e-06\\
0.6557656	-1.78372050196884e-06\\
0.6558656	-1.29568399387381e-06\\
0.6559656	-6.99882776622474e-07\\
0.6560656	6.74202622619191e-08\\
0.6561656	1.06991128578926e-06\\
0.6562656	2.37122470991835e-06\\
0.6563656	3.03494240139202e-06\\
0.6564656	4.12459289567124e-06\\
0.6565657	-3.46700672171352e-06\\
0.6566657	-2.33953775996554e-06\\
0.6567657	-1.5958149601758e-06\\
0.6568657	-1.17253105358373e-06\\
0.6569657	-6.43604103345297e-09\\
0.6570657	9.65662036556125e-07\\
0.6571657	1.80689633477726e-06\\
0.6572657	2.58034040623301e-06\\
0.6573657	3.34900745002642e-06\\
0.6574657	4.17584955059169e-06\\
0.6575658	-3.08821564143358e-06\\
0.6576658	-2.96020025736254e-06\\
0.6577658	-1.58546530570902e-06\\
0.6578658	-9.01310937884148e-07\\
0.6579658	1.54897742632443e-07\\
0.6580658	6.45729964610098e-07\\
0.6581658	1.63368848848933e-06\\
0.6582658	2.18120887929274e-06\\
0.6583658	3.35065876022611e-06\\
0.6584658	4.20433705450307e-06\\
0.6585659	-3.44253424611196e-06\\
0.6586659	-3.03694110037611e-06\\
0.6587659	-1.76058026735149e-06\\
0.6588659	-1.55143505331523e-06\\
0.6589659	-3.47561166513799e-07\\
0.6590659	9.1291258907944e-07\\
0.6591659	1.29178355567916e-06\\
0.6592659	1.85077451231308e-06\\
0.6593659	2.65153294254361e-06\\
0.6594659	3.75563031340675e-06\\
0.659566	-4.05126806857226e-06\\
0.659666	-3.15862903921271e-06\\
0.659766	-1.77833858217014e-06\\
0.659866	-8.49136507774517e-07\\
0.659966	-3.09842219436973e-07\\
0.660066	9.00644629986402e-07\\
0.660166	1.84334337216541e-06\\
0.660266	2.57919165402853e-06\\
0.660366	3.16904474484403e-06\\
0.660466	3.67367482478898e-06\\
0.6605661	-3.14474263607423e-06\\
0.6606661	-2.63051142024295e-06\\
0.6607661	-2.0196315677623e-06\\
0.6608661	-1.25167020037509e-06\\
0.6609661	-2.6628095284309e-07\\
0.6610661	9.9679534582009e-07\\
0.6611661	1.59772999008823e-06\\
0.6612661	2.59660573753706e-06\\
0.6613661	3.0534161163942e-06\\
0.6614661	4.02806478394524e-06\\
0.6615662	-3.73477481119266e-06\\
0.6616662	-2.54643425456891e-06\\
0.6617662	-1.66103082843705e-06\\
0.6618662	-1.01902714266799e-06\\
0.6619662	-5.60979000141515e-07\\
0.6620662	7.72463963905068e-07\\
0.6621662	1.0405576382766e-06\\
0.6622662	2.3024627626711e-06\\
0.6623662	2.61724430838939e-06\\
0.6624662	4.04387082131308e-06\\
0.6625663	-3.68458429100826e-06\\
0.6626663	-2.85849216563605e-06\\
0.6627663	-1.74417566256579e-06\\
0.6628663	-1.28305834490305e-06\\
0.6629663	-4.16663365143677e-07\\
0.6630663	9.13385933998256e-07\\
0.6631663	1.7653653747729e-06\\
0.6632663	2.19744934160104e-06\\
0.6633663	3.26771017089555e-06\\
0.6634663	4.0341175138181e-06\\
0.6635664	-3.77604565127143e-06\\
0.6636664	-2.44400936022515e-06\\
0.6637664	-2.24248205249467e-06\\
0.6638664	-1.11391092261215e-06\\
0.6639664	-8.48833003885829e-10\\
0.6640664	1.54045065592356e-07\\
0.6641664	1.4080047279208e-06\\
0.6642664	1.81815669009211e-06\\
0.6643664	3.44151940545601e-06\\
0.6644664	4.33500270613507e-06\\
0.6645665	-3.77418982555611e-06\\
0.6646665	-3.1697613493975e-06\\
0.6647665	-2.12508383334864e-06\\
0.6648665	-1.58368786262741e-06\\
0.6649665	-4.89215522314623e-07\\
0.6650665	2.14579046087238e-07\\
0.6651665	1.58382906789711e-06\\
0.6652665	2.67455457247934e-06\\
0.6653665	3.54266185986774e-06\\
0.6654665	4.24394290643448e-06\\
0.6655666	-3.48887158185818e-06\\
0.6656666	-2.95335478828207e-06\\
0.6657666	-2.41792414801267e-06\\
0.6658666	-8.27250439705551e-07\\
0.6659666	-1.26121486587749e-07\\
0.6660666	7.40557352330029e-07\\
0.6661666	1.82776261592466e-06\\
0.6662666	2.19035225690334e-06\\
0.6663666	2.88306506490699e-06\\
0.6664666	3.96052016338189e-06\\
0.6665667	-3.83352829835815e-06\\
0.6666667	-2.82169168297486e-06\\
0.6667667	-2.26192436247885e-06\\
0.6668667	-1.10009113107878e-06\\
0.6669667	-2.82179041910524e-07\\
0.6670667	2.45702120871272e-07\\
0.6671667	1.53731934293688e-06\\
0.6672667	2.64631587132413e-06\\
0.6673667	2.6262107191144e-06\\
0.6674667	3.53039818179468e-06\\
0.6675668	-3.88096338710398e-06\\
0.6676668	-2.9664518743644e-06\\
0.6677668	-1.96816511088826e-06\\
0.6678668	-8.33212674855588e-07\\
0.6679668	-5.08831271339716e-07\\
0.6680668	5.76147716202513e-08\\
0.6681668	9.18633060287277e-07\\
0.6682668	2.12660264686804e-06\\
0.6683668	2.73377359505389e-06\\
0.6684668	3.79226647861941e-06\\
0.6685669	-3.91609600569609e-06\\
0.6686669	-2.79653658274981e-06\\
0.6687669	-2.07002320262717e-06\\
0.6688669	-6.84957814023335e-07\\
0.6689669	-5.89874114886868e-07\\
0.6690669	2.6656202489761e-07\\
0.6691669	9.35552148639829e-07\\
0.6692669	2.46816474813727e-06\\
0.6693669	2.91533484642059e-06\\
0.6694669	4.32786356530102e-06\\
0.669567	-3.48562734764357e-06\\
0.669667	-2.98742394644336e-06\\
0.669767	-2.37221587529035e-06\\
0.669867	-1.58974197894679e-06\\
0.669967	-5.89877098722269e-07\\
0.670067	6.77367514079208e-07\\
0.670167	1.26184380144778e-06\\
0.670267	2.2132664740937e-06\\
0.670367	2.58121262541877e-06\\
0.670467	3.4151213608169e-06\\
0.6705671	-3.44458167145234e-06\\
0.6706671	-2.52739558481707e-06\\
0.6707671	-1.99671273515278e-06\\
0.6708671	-8.03650154423963e-07\\
0.6709671	1.00535197056217e-07\\
0.6710671	7.64446005607766e-07\\
0.6711671	1.23654426209541e-06\\
0.6712671	2.56515086993758e-06\\
0.6713671	2.79844529949003e-06\\
0.6714671	3.98446520932794e-06\\
0.6715672	-3.9996890235372e-06\\
0.6716672	-2.76060201098716e-06\\
0.6717672	-1.42548347836779e-06\\
0.6718672	-9.46866742346231e-07\\
0.6719672	-2.77428696904281e-07\\
0.6720672	6.30009822977229e-07\\
0.6721672	1.82248370483862e-06\\
0.6722672	2.34688322686694e-06\\
0.6723672	3.24995370304393e-06\\
0.6724672	3.57829519392894e-06\\
0.6725673	-3.74958424442795e-06\\
0.6726673	-2.42694224228401e-06\\
0.6727673	-1.54005800645507e-06\\
0.6728673	-1.04291604774431e-06\\
0.6729673	1.10352217586041e-07\\
0.6730673	9.65468192504204e-07\\
0.6731673	1.56800571504689e-06\\
0.6732673	1.96339080860142e-06\\
0.6733673	3.19690135697215e-06\\
0.6734673	3.31366681605161e-06\\
0.6735674	-3.72180554641943e-06\\
0.6736674	-2.69874170122364e-06\\
0.6737674	-1.65788357575636e-06\\
0.6738674	-1.55469855833701e-06\\
0.6739674	-3.4480393473757e-07\\
0.6740674	1.01603283875207e-06\\
0.6741674	1.57189384530909e-06\\
0.6742674	2.36671045383829e-06\\
0.6743674	3.44426301612799e-06\\
0.6744674	3.8481806163837e-06\\
0.6745675	-3.40658331721144e-06\\
0.6746675	-2.21421992963045e-06\\
0.6747675	-1.56547137741825e-06\\
0.6748675	-1.41731641400966e-06\\
0.6749675	-7.26886378643599e-07\\
0.6750675	5.48534610622653e-07\\
0.6751675	1.451509341166e-06\\
0.6752675	2.02444731067786e-06\\
0.6753675	3.30960446559558e-06\\
0.6754675	3.34908297716652e-06\\
0.6755676	-3.78741825723949e-06\\
0.6756676	-3.10774788880863e-06\\
0.6757676	-1.54833310306302e-06\\
0.6758676	-1.06768932051082e-06\\
0.6759676	-6.24486871991792e-07\\
0.6760676	8.22448764647987e-07\\
0.6761676	1.31413677273429e-06\\
0.6762676	1.89144076223613e-06\\
0.6763676	2.59506857736014e-06\\
0.6764676	3.46557208485976e-06\\
0.6765677	-3.36845506687666e-06\\
0.6766677	-3.03690188463435e-06\\
0.6767677	-1.41771526918077e-06\\
0.6768677	-1.47096940006364e-06\\
0.6769677	-1.56895421543624e-07\\
0.6770677	5.64118377344158e-07\\
0.6771677	1.7315264035922e-06\\
0.6772677	2.38462553747709e-06\\
0.6773677	2.56255498687352e-06\\
0.6774677	3.30429609673999e-06\\
0.6775678	-3.1986658997063e-06\\
0.6776678	-2.20632909897489e-06\\
0.6777678	-1.53414699344268e-06\\
0.6778678	-1.14377149706968e-06\\
0.6779678	2.98680191690437e-09\\
0.6780678	9.4415847540219e-07\\
0.6781678	1.71761511280621e-06\\
0.6782678	2.36106916462475e-06\\
0.6783678	2.91207380120895e-06\\
0.6784678	3.40802277021268e-06\\
0.6785679	-2.89286465271488e-06\\
0.6786679	-2.38844642375113e-06\\
0.6787679	-1.82782372215584e-06\\
0.6788679	-1.17424202761285e-06\\
0.6789679	-3.91106912189798e-07\\
0.6790679	5.58015829987824e-07\\
0.6791679	1.70940006016806e-06\\
0.6792679	2.09915921045578e-06\\
0.6793679	2.76324611103718e-06\\
0.6794679	3.73745291337713e-06\\
0.679568	-3.64958065324217e-06\\
0.679668	-2.94100194997426e-06\\
0.679768	-1.81585683156626e-06\\
0.679868	-1.23899711912401e-06\\
0.679968	-1.75435819649294e-07\\
0.680068	4.09652753496914e-07\\
0.680168	1.55093290787889e-06\\
0.680268	2.28290747106286e-06\\
0.680368	2.639917714653e-06\\
0.680468	3.65614326325314e-06\\
0.6805681	-3.26582674414766e-06\\
0.6806681	-2.82153468589286e-06\\
0.6807681	-1.61642546903096e-06\\
0.6808681	-6.16966956190623e-07\\
0.6809681	2.10210969520119e-07\\
0.6810681	8.98316355524287e-07\\
0.6811681	1.48039510294851e-06\\
0.6812681	1.98933094841536e-06\\
0.6813681	2.45784534858018e-06\\
0.6814681	3.91849746605999e-06\\
0.6815682	-3.14880364049586e-06\\
0.6816682	-2.59877520170093e-06\\
0.6817682	-1.95987386586438e-06\\
0.6818682	-1.20019029381524e-06\\
0.6819682	-2.87977697688291e-07\\
0.6820682	8.0834816662545e-07\\
0.6821682	1.12020895315368e-06\\
0.6822682	1.67886368895864e-06\\
0.6823682	2.51540876439194e-06\\
0.6824682	3.66077789148989e-06\\
0.6825683	-3.32458874474639e-06\\
0.6826683	-2.46103567835831e-06\\
0.6827683	-2.19680326951632e-06\\
0.6828683	-5.01608822922606e-07\\
0.6829683	-3.45332398676845e-07\\
0.6830683	3.01983195782896e-07\\
0.6831683	1.47013238649407e-06\\
0.6832683	2.18874684865256e-06\\
0.6833683	2.48729550911264e-06\\
0.6834683	3.39508452462667e-06\\
0.6835684	-3.44386330564106e-06\\
0.6836684	-2.22164424013016e-06\\
0.6837684	-1.30321402469491e-06\\
0.6838684	-6.59917548606614e-07\\
0.6839684	-2.63262422972588e-07\\
0.6840684	9.15081092678349e-07\\
0.6841684	9.03280085395863e-07\\
0.6842684	1.72933903774819e-06\\
0.6843684	2.42109982995231e-06\\
0.6844684	3.00624179550013e-06\\
0.6845685	-2.78473816361924e-06\\
0.6846685	-2.32148401568821e-06\\
0.6847685	-1.88275856505271e-06\\
0.6848685	-1.44153249381773e-06\\
0.6849685	2.90611623654513e-08\\
0.6850685	5.55727059925459e-07\\
0.6851685	1.16500759750338e-06\\
0.6852685	1.8832829744575e-06\\
0.6853685	2.73677123141525e-06\\
0.6854685	3.75152834530823e-06\\
0.6855686	-3.25274305135181e-06\\
0.6856686	-2.82870267342972e-06\\
0.6857686	-1.16617197010171e-06\\
0.6858686	-1.23974286081463e-06\\
0.6859686	-2.41689996904881e-08\\
0.6860686	5.05634263792132e-07\\
0.6861686	1.37458997695461e-06\\
0.6862686	1.60745966626408e-06\\
0.6863686	2.22884342004903e-06\\
0.6864686	3.26317996046654e-06\\
0.6865687	-3.37805028793881e-06\\
0.6866687	-2.43566360635938e-06\\
0.6867687	-2.00795132876408e-06\\
0.6868687	-1.07111941760607e-06\\
0.6869687	-6.01534726030195e-07\\
0.6870687	4.24275093280357e-07\\
0.6871687	1.02962169634679e-06\\
0.6872687	2.23765615414706e-06\\
0.6873687	3.07136905020045e-06\\
0.6874687	3.55359061909866e-06\\
0.6875688	-3.31000782871627e-06\\
0.6876688	-2.45321428815615e-06\\
0.6877688	-1.88036023862637e-06\\
0.6878688	-5.69255953930536e-07\\
0.6879688	-4.97871499138824e-07\\
0.6880688	3.55663424933539e-07\\
0.6881688	1.01305956468067e-06\\
0.6882688	1.49586826747594e-06\\
0.6883688	2.82548161134599e-06\\
0.6884688	3.0231325385266e-06\\
0.6885689	-2.80906175964546e-06\\
0.6886689	-1.802352269209e-06\\
0.6887689	-1.86483914488278e-06\\
0.6888689	-9.75924761004165e-07\\
0.6889689	-1.15169927372705e-07\\
0.6890689	7.37706265407923e-07\\
0.6891689	1.60282659456001e-06\\
0.6892689	1.50015585731467e-06\\
0.6893689	2.44950103001074e-06\\
0.6894689	3.47051141380561e-06\\
0.689569	-3.23615191710758e-06\\
0.689669	-2.00337291955677e-06\\
0.689769	-1.64090607324141e-06\\
0.689869	-1.12973127119886e-06\\
0.689969	-4.5098526557652e-07\\
0.690069	4.14038550111684e-07\\
0.690169	1.48389025156348e-06\\
0.690269	1.7769636011522e-06\\
0.690369	2.31149620910642e-06\\
0.690469	3.10556972049625e-06\\
0.6905691	-2.53966865093602e-06\\
0.6906691	-2.1625872732578e-06\\
0.6907691	-1.47263649674656e-06\\
0.6908691	-4.52356855529246e-07\\
0.6909691	-8.44439260738739e-08\\
0.6910691	6.48251871293581e-07\\
0.6911691	7.62725474778847e-07\\
0.6912691	2.27581736611526e-06\\
0.6913691	2.20421377905566e-06\\
0.6914691	3.56444689231239e-06\\
0.6915692	-3.24006175222991e-06\\
0.6916692	-1.95670136493575e-06\\
0.6917692	-1.19281399602045e-06\\
0.6918692	-9.32481766913185e-07\\
0.6919692	-1.59939807176102e-07\\
0.6920692	1.40423940475642e-07\\
0.6921692	9.84068979636987e-07\\
0.6922692	1.38630241908189e-06\\
0.6923692	2.36227923222643e-06\\
0.6924692	2.92700246529165e-06\\
0.6925693	-2.41219619523392e-06\\
0.6926693	-2.61495195141848e-06\\
0.6927693	-1.18484696898236e-06\\
0.6928693	-1.10748379578141e-06\\
0.6929693	-3.68615765022895e-07\\
0.6930693	4.58532407598966e-08\\
0.6931693	1.14986901467162e-06\\
0.6932693	1.95722728024705e-06\\
0.6933693	2.48157390503323e-06\\
0.6934693	2.73640516956775e-06\\
0.6935694	-2.66555164474269e-06\\
0.6936694	-1.8990958783327e-06\\
0.6937694	-1.36254883686959e-06\\
0.6938694	-1.04301034431487e-06\\
0.6939694	7.22714008638548e-08\\
0.6940694	-4.09990441596619e-09\\
0.6941694	7.40331563164887e-07\\
0.6942694	1.31787404233918e-06\\
0.6943694	2.74068840866448e-06\\
0.6944694	3.02078845226816e-06\\
0.6945695	-3.12236567001634e-06\\
0.6946695	-2.08135325463488e-06\\
0.6947695	-1.14788218308348e-06\\
0.6948695	-1.31052453822278e-06\\
0.6949695	-5.57998158789985e-07\\
0.6950695	1.20833616445992e-07\\
0.6951695	7.36962239233918e-07\\
0.6952695	1.30123419994632e-06\\
0.6953695	1.82435131712566e-06\\
0.6954695	3.3168710174003e-06\\
0.6955696	-2.39382222799733e-06\\
0.6956696	-1.92040611679545e-06\\
0.6957696	-1.44676869595628e-06\\
0.6958696	-9.62927475534059e-07\\
0.6959696	-4.59042952538624e-07\\
0.6960696	7.45816701730462e-08\\
0.6961696	6.47500770334375e-07\\
0.6962696	1.26912662112844e-06\\
0.6963696	1.94872964209836e-06\\
0.6964696	2.69543872200018e-06\\
0.6965697	-2.55438914420481e-06\\
0.6966697	-1.63555665455561e-06\\
0.6967697	-1.62306910933552e-06\\
0.6968697	-5.08360943118902e-07\\
0.6969697	-2.83006668766461e-07\\
0.6970697	6.12794517351745e-08\\
0.6971697	5.32643723349935e-07\\
0.6972697	1.13909329990669e-06\\
0.6973697	1.88849650517575e-06\\
0.6974697	2.78858313684793e-06\\
0.6975698	-2.1144096529202e-06\\
0.6976698	-1.879149923667e-06\\
0.6977698	-1.47083818657912e-06\\
0.6978698	-8.82295780213482e-07\\
0.6979698	-1.06481002903891e-07\\
0.6980698	8.63511198323508e-07\\
0.6981698	1.03444954890897e-06\\
0.6982698	1.41296675049141e-06\\
0.6983698	2.00555984508632e-06\\
0.6984698	2.81859048212318e-06\\
0.6985699	-2.99105403378519e-06\\
0.6986699	-1.70736742499678e-06\\
0.6987699	-1.18496137457846e-06\\
0.6988699	-4.18012564784576e-07\\
0.6989699	-4.00831432711968e-07\\
0.6990699	8.72138187002491e-07\\
0.6991699	1.4063193833902e-06\\
0.6992699	1.20700250483274e-06\\
0.6993699	2.27934549190678e-06\\
0.6994699	2.62837421383111e-06\\
0.69957	-2.47773855899247e-06\\
0.69967	-1.54949868691467e-06\\
0.69977	-1.33027997062385e-06\\
0.69987	-8.15581586710579e-07\\
0.69997	-1.03308539500802e-09\\
0.70007	1.17605940319621e-07\\
0.70017	5.44446215000605e-07\\
0.70027	1.28346911143495e-06\\
0.70037	2.33852699560444e-06\\
0.70047	2.71334360002484e-06\\
0.7005701	-2.21211898732143e-06\\
0.7006701	-2.17579815808833e-06\\
0.7007701	-8.09312614924451e-07\\
0.7008701	-1.10944989439865e-06\\
0.7009701	-7.31244140794729e-08\\
0.7010701	3.02622841008571e-07\\
0.7011701	1.02062470297426e-06\\
0.7012701	1.08358816564014e-06\\
0.7013701	2.49409473895845e-06\\
0.7014701	2.25460081959561e-06\\
0.7015702	-2.14276654064705e-06\\
0.7016702	-1.66403536039361e-06\\
0.7017702	-8.28680610132082e-07\\
0.7018702	-6.34742909122465e-07\\
0.7019702	-8.0386190148829e-08\\
0.7020702	8.36102660439764e-07\\
0.7021702	1.11631416999103e-06\\
0.7022702	1.76171664145741e-06\\
0.7023702	1.7736565283144e-06\\
0.7024702	2.15335878683831e-06\\
0.7025703	-2.49463369739189e-06\\
0.7026703	-1.36484558654892e-06\\
0.7027703	-8.64344633200176e-07\\
0.7028703	-9.92388243803433e-07\\
0.7029703	2.51646535387451e-07\\
0.7030703	8.68263412501236e-07\\
0.7031703	8.57847194613015e-07\\
0.7032703	1.22066418106748e-06\\
0.7033703	1.95686253290361e-06\\
0.7034703	2.06647265499349e-06\\
0.7035704	-2.73341720924414e-06\\
0.7036704	-1.86598850904218e-06\\
0.7037704	-1.62575998818859e-06\\
0.7038704	-1.01316861722545e-06\\
0.7039704	-2.87672328180832e-08\\
0.7040704	3.26775849268657e-07\\
0.7041704	1.05267720496016e-06\\
0.7042704	1.1480386978846e-06\\
0.7043704	1.61184785385515e-06\\
0.7044704	2.44297824125717e-06\\
0.7045705	-2.52892462304644e-06\\
0.7046705	-1.9556208954441e-06\\
0.7047705	-1.01905658089763e-06\\
0.7048705	-7.20810165777408e-07\\
0.7049705	-6.25721587876882e-08\\
0.7050705	-4.61446827415557e-08\\
0.7051705	1.32655887896682e-06\\
0.7052705	1.05351429890987e-06\\
0.7053705	2.13258688486206e-06\\
0.7054705	2.56153187616626e-06\\
0.7055706	-1.71755008837593e-06\\
0.7056706	-1.58468910904652e-06\\
0.7057706	-1.10934914299143e-06\\
0.7058706	-2.94211423002011e-07\\
0.7059706	-1.42065285313464e-07\\
0.7060706	3.44192215395722e-07\\
0.7061706	1.16155670992768e-06\\
0.7062706	1.30691690780615e-06\\
0.7063706	1.77705498916048e-06\\
0.7064706	2.56864700709514e-06\\
0.7065707	-2.26396360347714e-06\\
0.7066707	-1.82854540242694e-06\\
0.7067707	-1.08228160278223e-06\\
0.7068707	-1.02891680131556e-06\\
0.7069707	3.27700269808773e-07\\
0.7070707	-1.63828683952261e-08\\
0.7071707	9.34777982353197e-07\\
0.7072707	1.17702406443598e-06\\
0.7073707	1.70609408889355e-06\\
0.7074707	2.5176246292169e-06\\
0.7075708	-2.22211683587048e-06\\
0.7076708	-1.84789097845695e-06\\
0.7077708	-1.20490841060139e-06\\
0.7078708	-2.97937184035035e-07\\
0.7079708	-1.31845482531645e-07\\
0.7080708	2.88398795067479e-07\\
0.7081708	9.57728430783789e-07\\
0.7082708	8.70977293843112e-07\\
0.7083708	2.02288074580004e-06\\
0.7084708	2.40807604545701e-06\\
0.7085709	-1.69566654162345e-06\\
0.7086709	-1.84914669931757e-06\\
0.7087709	-7.86013504727734e-07\\
0.7088709	-5.12018149878557e-07\\
0.7089709	-3.30079055199661e-08\\
0.7090709	6.4507429464733e-07\\
0.7091709	5.16190246990789e-07\\
0.7092709	1.57420691837551e-06\\
0.7093709	1.81289680756258e-06\\
0.7094709	2.22593838539176e-06\\
0.709571	-1.7979151405445e-06\\
0.709671	-1.04435075121145e-06\\
0.709771	-1.13596605078037e-06\\
0.709871	-1.07945473604154e-06\\
0.709971	1.18397508019896e-07\\
0.710071	4.50713409039238e-07\\
0.710171	9.10524516228861e-07\\
0.710271	1.49077161260003e-06\\
0.710371	1.18430512174861e-06\\
0.710471	1.98388551631012e-06\\
0.7105711	-1.611365531895e-06\\
0.7106711	-1.61068013126986e-06\\
0.7107711	-5.26209569695624e-07\\
0.7108711	-3.65549159386802e-07\\
0.7109711	-1.36382112048139e-07\\
0.7110711	1.53520872459012e-07\\
0.7111711	4.96302013930361e-07\\
0.7112711	8.84016861046888e-07\\
0.7113711	1.30863471814457e-06\\
0.7114711	1.76203902779726e-06\\
0.7115712	-2.14698516765033e-06\\
0.7116712	-1.64969082661059e-06\\
0.7117712	-1.14847908982973e-06\\
0.7118712	-6.51805867724065e-07\\
0.7119712	-1.68210879003539e-07\\
0.7120712	2.93682779650695e-07\\
0.7121712	7.25169034598849e-07\\
0.7122712	1.11745925557294e-06\\
0.7123712	1.46168265691138e-06\\
0.7124712	1.74888671433671e-06\\
0.7125713	-2.30327187566992e-06\\
0.7126713	-1.14636875148122e-06\\
0.7127713	-1.07383812220085e-06\\
0.7128713	-9.49554539442232e-08\\
0.7129713	-2.19075899421384e-07\\
0.7130713	5.44366111743244e-07\\
0.7131713	1.18585728015042e-06\\
0.7132713	6.95805843387021e-07\\
0.7133713	1.06454200698991e-06\\
0.7134713	2.28231832211101e-06\\
0.7135714	-1.82521116331991e-06\\
0.7136714	-9.28081049789142e-07\\
0.7137714	-1.21162486621529e-06\\
0.7138714	-6.85896593299873e-07\\
0.7139714	-3.61025791706382e-07\\
0.7140714	7.5278278632851e-07\\
0.7141714	6.45249616848531e-07\\
0.7142714	1.30602081171993e-06\\
0.7143714	1.72466852732533e-06\\
0.7144714	1.89069137301878e-06\\
0.7145715	-1.2632121402234e-06\\
0.7146715	-1.62351455901266e-06\\
0.7147715	-1.2683939845104e-06\\
0.7148715	-2.08641992127667e-07\\
0.7149715	-4.55121666353975e-07\\
0.7150715	-1.87671949092305e-08\\
0.7151715	1.08941653920169e-06\\
0.7152715	8.583543635865e-07\\
0.7153715	1.2769012203151e-06\\
0.7154715	2.33384258541136e-06\\
0.7155716	-1.93210601007188e-06\\
0.7156716	-1.62168505934979e-06\\
0.7157716	-7.06937463013091e-07\\
0.7158716	-1.99351679608384e-07\\
0.7159716	-1.10483606041711e-07\\
0.7160716	5.48043801007481e-07\\
0.7161716	7.64540943665182e-07\\
0.7162716	1.52725198732107e-06\\
0.7163716	1.8243552584174e-06\\
0.7164716	1.64396364987773e-06\\
0.7165717	-1.87028828957381e-06\\
0.7166717	-1.03109819793445e-06\\
0.7167717	-7.05464997974303e-07\\
0.7168717	-9.05533574568906e-07\\
0.7169717	3.56487765085234e-07\\
0.7170717	6.83276866197957e-08\\
0.7171717	2.17652235789956e-07\\
0.7172717	7.92065230958983e-07\\
0.7173717	1.7791086686092e-06\\
0.7174717	2.16626311488888e-06\\
0.7175718	-1.79908244790639e-06\\
0.7176718	-1.63913942641258e-06\\
0.7177718	-1.11702098037902e-06\\
0.7178718	-4.45488316680454e-07\\
0.7179718	-1.37362083152404e-07\\
0.7180718	2.9447804428262e-07\\
0.7181718	6.37093735655903e-07\\
0.7182718	9.77488416320504e-07\\
0.7183718	1.40260765713052e-06\\
0.7184718	1.69933957394619e-06\\
0.7185719	-1.58239966940865e-06\\
0.7186719	-1.1717673417122e-06\\
0.7187719	-8.29212115083244e-07\\
0.7188719	-4.68071828740868e-07\\
0.7189719	-1.01739811797508e-07\\
0.7190719	2.56335505155469e-07\\
0.7191719	5.92650981956488e-07\\
0.7192719	9.93649159841681e-07\\
0.7193719	1.34571865162414e-06\\
0.7194719	1.63519452722749e-06\\
0.719572	-1.48676580469242e-06\\
0.719672	-1.15358113583763e-06\\
0.719772	-8.24316059189201e-07\\
0.719872	-4.1284582308343e-07\\
0.719972	-1.33097279975125e-07\\
0.720072	2.00951502216107e-07\\
0.720172	5.75271639546848e-07\\
0.720272	9.75783792211971e-07\\
0.720372	1.28835856927356e-06\\
0.720472	1.59881690131591e-06\\
0.7205721	-1.44178340100165e-06\\
0.7206721	-1.16832972818504e-06\\
0.7207721	-7.3983827009183e-07\\
0.7208721	-4.70683015674211e-07\\
0.7209721	-7.52857153285547e-08\\
0.7210721	2.31884493273604e-07\\
0.7211721	5.36311470455964e-07\\
0.7212721	9.23432450616879e-07\\
0.7213721	1.27863841858922e-06\\
0.7214721	1.58727449317553e-06\\
0.7215722	-1.40109297463997e-06\\
0.7216722	-1.11992634677094e-06\\
0.7217722	-7.29580530101082e-07\\
0.7218722	-4.44890170037127e-07\\
0.7219722	-8.07339037844912e-08\\
0.7220722	2.47966011501788e-07\\
0.7221722	5.26244083753902e-07\\
0.7222722	9.39091934215597e-07\\
0.7223722	1.17145868971935e-06\\
0.7224722	1.50825134537236e-06\\
0.7225723	-1.40389435443478e-06\\
0.7226723	-1.09402908909928e-06\\
0.7227723	-7.25281262603517e-07\\
0.7228723	-4.12908754410246e-07\\
0.7229723	-7.22097277527922e-08\\
0.7230723	1.81477733618429e-07\\
0.7231723	5.3277599199042e-07\\
0.7232723	8.66268216492827e-07\\
0.7233723	1.16649875281372e-06\\
0.7234723	1.51797348479921e-06\\
0.7235724	-1.33708519234332e-06\\
0.7236724	-1.02024412296231e-06\\
0.7237724	-6.98883987348964e-07\\
0.7238724	-3.88649169402733e-07\\
0.7239724	-1.0522070925667e-07\\
0.7240724	2.35684057159347e-07\\
0.7241724	5.18311849884157e-07\\
0.7242724	8.26873818082774e-07\\
0.7243724	1.14554587604676e-06\\
0.7244724	1.45846907173564e-06\\
0.7245725	-1.29807022475337e-06\\
0.7246725	-1.03500486814401e-06\\
0.7247725	-7.25486874908476e-07\\
0.7248725	-3.85511169098685e-07\\
0.7249725	-1.31105774237028e-07\\
0.7250725	2.21668535860253e-07\\
0.7251725	5.56718585231764e-07\\
0.7252725	8.57919145280661e-07\\
0.7253725	1.10911328510266e-06\\
0.7254725	1.39411271737888e-06\\
0.7255726	-1.25829129249411e-06\\
0.7256726	-9.45176637623391e-07\\
0.7257726	-6.47022521560459e-07\\
0.7258726	-3.80139226985321e-07\\
0.7259726	-6.08666659296375e-08\\
0.7260726	1.94425960442501e-07\\
0.7261726	4.69340504682947e-07\\
0.7262726	7.47450212501555e-07\\
0.7263726	1.11230007138197e-06\\
0.7264726	1.34740713386705e-06\\
0.7265727	-1.22752439613016e-06\\
0.7266727	-8.92432853283331e-07\\
0.7267727	-6.36713703650038e-07\\
0.7268727	-3.76958244396874e-07\\
0.7269727	-2.9784027577584e-08\\
0.7270727	1.88165488301806e-07\\
0.7271727	4.60221266207483e-07\\
0.7272727	7.69689025514886e-07\\
0.7273727	9.9984956825061e-07\\
0.7274727	1.33395911372514e-06\\
0.7275728	-1.21898665550457e-06\\
0.7276728	-9.18445164144188e-07\\
0.7277728	-6.64347060586046e-07\\
0.7278728	-3.73531201858235e-07\\
0.7279728	-6.28593962437307e-08\\
0.7280728	1.50783927788112e-07\\
0.7281728	4.50492043779782e-07\\
0.7282728	7.1933624501419e-07\\
0.7283728	1.04036617909475e-06\\
0.7284728	1.29661015335625e-06\\
0.7285729	-1.11529227830864e-06\\
0.7286729	-8.30926345840588e-07\\
0.7287729	-5.62403294379621e-07\\
0.7288729	-3.26776980474897e-07\\
0.7289729	-4.11209930550882e-08\\
0.7290729	1.77471641826799e-07\\
0.7291729	4.11888788187653e-07\\
0.7292729	7.44999510526867e-07\\
0.7293729	9.59654387067133e-07\\
0.7294729	1.23868581980435e-06\\
0.729573	-1.13529330614259e-06\\
0.729673	-8.70561334465059e-07\\
0.729773	-5.9307878530479e-07\\
0.729873	-3.2008288064489e-07\\
0.729973	-6.88274754967466e-08\\
0.730073	1.43417246634137e-07\\
0.730173	3.99365080894221e-07\\
0.730273	6.81714101058439e-07\\
0.730373	9.73146966010141e-07\\
0.730473	1.15633122477199e-06\\
0.7305731	-1.10183721058199e-06\\
0.7306731	-7.78858143091199e-07\\
0.7307731	-6.1623011904377e-07\\
0.7308731	-3.31343071202639e-07\\
0.7309731	-4.16005461190139e-08\\
0.7310731	1.35580577986794e-07\\
0.7311731	3.82770393825282e-07\\
0.7312731	6.82526258066751e-07\\
0.7313731	9.17393083943452e-07\\
0.7314731	1.16990363152603e-06\\
0.7315732	-1.01047018097322e-06\\
0.7316732	-7.66947147234909e-07\\
0.7317732	-5.58269627859431e-07\\
0.7318732	-3.01950585868127e-07\\
0.7319732	-1.55136908874809e-08\\
0.7320732	1.83506963402635e-07\\
0.7321732	3.77567146325752e-07\\
0.7322732	6.49112770645388e-07\\
0.7323732	8.80580179551949e-07\\
0.7324732	1.15439642034421e-06\\
0.7325733	-9.99111350430404e-07\\
0.7326733	-7.85353672427824e-07\\
0.7327733	-4.82036549565024e-07\\
0.7328733	-3.06767305779942e-07\\
0.7329733	-7.71611637917502e-08\\
0.7330733	1.89159038432862e-07\\
0.7331733	3.74563106753101e-07\\
0.7332733	6.61413776414932e-07\\
0.7333733	8.3206697942817e-07\\
0.7334733	1.0688721164831e-06\\
0.7335734	-1.0187200963907e-06\\
0.7336734	-6.94765669884134e-07\\
0.7337734	-5.57665265188412e-07\\
0.7338734	-3.25092918584957e-07\\
0.7339734	-1.4727855490726e-08\\
0.7340734	1.55745771945881e-07\\
0.7341734	3.68639153425221e-07\\
0.7342734	6.06259078272942e-07\\
0.7343734	8.50908200566991e-07\\
0.7344734	1.08488529682038e-06\\
0.7345735	-9.0497512661436e-07\\
0.7346735	-7.37814889961719e-07\\
0.7347735	-5.34468711399505e-07\\
0.7348735	-3.12650727418706e-07\\
0.7349735	-9.00776627732469e-08\\
0.7350735	1.15531425759485e-07\\
0.7351735	3.86455408618591e-07\\
0.7352735	6.04971324480452e-07\\
0.7353735	7.53354638949322e-07\\
0.7354735	1.01387949336118e-06\\
0.7355736	-9.50981070446666e-07\\
0.7356736	-7.11886947890594e-07\\
0.7357736	-5.13852045302965e-07\\
0.7358736	-2.74605036765241e-07\\
0.7359736	-1.18746825150673e-08\\
0.7360736	1.56610412804792e-07\\
0.7361736	3.13122043738012e-07\\
0.7362736	5.39932644427754e-07\\
0.7363736	8.19315531530584e-07\\
0.7364736	9.33545143633197e-07\\
0.7365737	-8.81015066622748e-07\\
0.7366737	-6.42972483089199e-07\\
0.7367737	-4.23268468829896e-07\\
0.7368737	-2.39621723185923e-07\\
0.7369737	-9.74863817448934e-09\\
0.7370737	1.48636929153678e-07\\
0.7371737	3.17823898698943e-07\\
0.7372737	5.80104185077257e-07\\
0.7373737	7.1777293875197e-07\\
0.7374737	9.13128768931237e-07\\
0.7375738	-8.25323020814395e-07\\
0.7376738	-6.6056878328391e-07\\
0.7377738	-4.91226280929791e-07\\
0.7378738	-2.34980786217864e-07\\
0.7379738	-9.51297907114679e-09\\
0.7380738	1.67501276460147e-07\\
0.7381738	3.78391149635782e-07\\
0.7382738	5.0549106889175e-07\\
0.7383738	7.31140936127694e-07\\
0.7384738	9.37686349566924e-07\\
0.7385739	-7.95972174505444e-07\\
0.7386739	-5.73638234524498e-07\\
0.7387739	-4.23353251011171e-07\\
0.7388739	-2.62746676749082e-07\\
0.7389739	-9.44119360468676e-09\\
0.7390739	1.18947500793443e-07\\
0.7391739	3.04810902962416e-07\\
0.7392739	5.30547918098367e-07\\
0.7393739	6.78565062028902e-07\\
0.7394739	8.31276672719206e-07\\
0.739574	-7.63763966848963e-07\\
0.739674	-5.4762733059377e-07\\
0.739774	-3.79521770343949e-07\\
0.739874	-2.76999595261174e-07\\
0.739974	-5.76042641364793e-08\\
0.740074	1.61129810438698e-07\\
0.740174	3.61677463200927e-07\\
0.740274	5.26522979382715e-07\\
0.740374	6.38160287724787e-07\\
0.740474	8.79093162764466e-07\\
0.7405741	-7.36208465301758e-07\\
0.7406741	-5.82547191319005e-07\\
0.7407741	-3.52035470163603e-07\\
0.7408741	-2.62128190042787e-07\\
0.7409741	-3.0269424355911e-08\\
0.7410741	1.26107760234362e-07\\
0.7411741	2.89581493173596e-07\\
0.7412741	4.42741286243997e-07\\
0.7413741	6.68188231023326e-07\\
0.7414741	8.48535173181286e-07\\
0.7415742	-7.36559037595175e-07\\
0.7416742	-5.92114114073006e-07\\
0.7417742	-3.44875583535575e-07\\
0.7418742	-2.12181694214664e-07\\
0.7419742	-1.13580201355212e-08\\
0.7420742	1.40282726046603e-07\\
0.7421742	3.25440869863858e-07\\
0.7422742	4.26829956656771e-07\\
0.7423742	6.27176925616091e-07\\
0.7424742	8.09222288622458e-07\\
0.7425743	-6.8390359347914e-07\\
0.7426743	-4.83946142715297e-07\\
0.7427743	-3.54002792524177e-07\\
0.7428743	-2.1127698474066e-07\\
0.7429743	-7.29577225300204e-08\\
0.7430743	1.43780600259102e-07\\
0.7431743	3.21778368839176e-07\\
0.7432743	4.43890914192391e-07\\
0.7433743	5.92988684600115e-07\\
0.7434743	7.51957405453751e-07\\
0.7435744	-6.7430628170051e-07\\
0.7436744	-5.40809390714436e-07\\
0.7437744	-3.48708101949313e-07\\
0.7438744	-2.1505391112342e-07\\
0.7439744	-5.68822137725356e-08\\
0.7440744	1.08787845454206e-07\\
0.7441744	2.64953539774737e-07\\
0.7442744	3.94628715094658e-07\\
0.7443744	5.80843947328713e-07\\
0.7444744	7.06646699266678e-07\\
0.7445745	-6.62991432642812e-07\\
0.7446745	-5.02905661781128e-07\\
0.7447745	-3.54004011438036e-07\\
0.7448745	-2.33169916430409e-07\\
0.7449745	-5.72691599698416e-08\\
0.7450745	1.56850277377441e-07\\
0.7451745	2.92358364128997e-07\\
0.7452745	4.32443168885133e-07\\
0.7453745	5.60311001984726e-07\\
0.7454745	6.59186572327108e-07\\
0.7455746	-6.47559341426795e-07\\
0.7456746	-4.51190755978992e-07\\
0.7457746	-3.34045238248315e-07\\
0.7458746	-2.12823059841938e-07\\
0.7459746	-4.20538248668834e-09\\
0.7460746	7.51458759662427e-08\\
0.7461746	2.08588183558955e-07\\
0.7462746	3.79498532598888e-07\\
0.7463746	5.71273576888132e-07\\
0.7464746	6.67329766390612e-07\\
0.7465747	-5.52221614547044e-07\\
0.7466747	-4.91711019068308e-07\\
0.7467747	-2.76566310519755e-07\\
0.7468747	-2.23290488543526e-07\\
0.7469747	-4.8366088634122e-08\\
0.7470747	1.31744946840762e-07\\
0.7471747	2.00601398270273e-07\\
0.7472747	3.41782897317344e-07\\
0.7473747	5.38890054135877e-07\\
0.7474747	6.75544582884324e-07\\
0.7475748	-6.13041870323183e-07\\
0.7476748	-4.40943715540598e-07\\
0.7477748	-2.78321464985964e-07\\
0.7478748	-1.41467716541221e-07\\
0.7479748	-4.66533461862539e-08\\
0.7480748	8.98726177922526e-08\\
0.7481748	2.51883105928385e-07\\
0.7482748	3.23173132976695e-07\\
0.7483748	4.8755991377547e-07\\
0.7484748	6.28882979802903e-07\\
0.7485749	-5.64165935634797e-07\\
0.7486749	-4.12125231386007e-07\\
0.7487749	-3.31510669804302e-07\\
0.7488749	-1.38392277970745e-07\\
0.7489749	-4.88171937318427e-08\\
0.7490749	1.21190442736729e-07\\
0.7491749	2.55629601708662e-07\\
0.7492749	3.38522472775882e-07\\
0.7493749	4.5391457248467e-07\\
0.7494749	5.85874851299906e-07\\
0.749575	-5.25023995889828e-07\\
0.749675	-4.02549389877382e-07\\
0.749775	-3.11173920686247e-07\\
0.749875	-1.6673381336707e-07\\
0.749975	1.49586666942092e-08\\
0.750075	1.18115315306255e-07\\
0.750175	2.26972095673794e-07\\
0.750275	3.25789230370965e-07\\
0.750375	4.98851306540704e-07\\
0.750475	5.30467372722443e-07\\
0.7505751	-4.88485768224134e-07\\
0.7506751	-3.81816815275826e-07\\
0.7507751	-2.6353458790207e-07\\
0.7508751	-1.49231197510424e-07\\
0.7509751	-5.44738145424883e-08\\
0.7510751	1.05195424843219e-07\\
0.7511751	2.14259511965853e-07\\
0.7512751	3.57226655672838e-07\\
0.7513751	4.18630374532381e-07\\
0.7514751	5.83029584860784e-07\\
0.7515752	-5.0994828668105e-07\\
0.7516752	-3.81014808731317e-07\\
0.7517752	-2.9527116460204e-07\\
0.7518752	-1.68055936972111e-07\\
0.7519752	-1.4681874827005e-08\\
0.7520752	4.95641909736833e-08\\
0.7521752	2.09421435165558e-07\\
0.7522752	3.49655114362246e-07\\
0.7523752	4.55056654358188e-07\\
0.7524752	5.10443727896881e-07\\
0.7525753	-4.97334765758861e-07\\
0.7526753	-3.82806074200603e-07\\
0.7527753	-2.63695627639171e-07\\
0.7528753	-1.55079939445635e-07\\
0.7529753	-7.20088798034091e-08\\
0.7530753	7.04943978746542e-08\\
0.7531753	1.57433533043427e-07\\
0.7532753	2.73839033032575e-07\\
0.7533753	4.047683448416e-07\\
0.7534753	5.35305929139085e-07\\
0.7535754	-4.01981771730453e-07\\
0.7536754	-3.12403416913298e-07\\
0.7537754	-2.67815431165275e-07\\
0.7538754	-8.30245612815084e-08\\
0.7539754	-7.28101891278587e-08\\
0.7540754	4.80757361975837e-08\\
0.7541754	1.64908764893612e-07\\
0.7542754	2.62992016053509e-07\\
0.7543754	4.27656238005669e-07\\
0.7544754	4.44259875331809e-07\\
0.7545755	-4.10390976535169e-07\\
0.7546755	-3.29406621341066e-07\\
0.7547755	-2.4025524539284e-07\\
0.7548755	-1.57466961164809e-07\\
0.7549755	4.45612791111216e-09\\
0.7550755	3.1039986747583e-08\\
0.7551755	2.07838705995611e-07\\
0.7552755	2.20434562187011e-07\\
0.7553755	3.54438074084973e-07\\
0.7554755	4.95488060664462e-07\\
0.7555756	-4.36820865507004e-07\\
0.7556756	-3.20476737258879e-07\\
0.7557756	-2.40013586727539e-07\\
0.7558756	-1.09678815862768e-07\\
0.7559756	-4.36912542534085e-08\\
0.7560756	4.37588924262489e-08\\
0.7561756	1.3851009494914e-07\\
0.7562756	2.2642954919383e-07\\
0.7563756	3.93413230853623e-07\\
0.7564756	4.25385938640499e-07\\
0.7565757	-4.16692922722284e-07\\
0.7566757	-2.9282230015748e-07\\
0.7567757	-2.46028791506259e-07\\
0.7568757	-9.02717855189028e-08\\
0.7569757	-3.94816125626196e-08\\
0.7570757	9.24405034186115e-08\\
0.7571757	1.91622488793231e-07\\
0.7572757	2.44221462830652e-07\\
0.7573757	3.36423781194251e-07\\
0.7574757	4.54445077013865e-07\\
0.7575758	-4.00786209137083e-07\\
0.7576758	-2.68471263331627e-07\\
0.7577758	-1.51528045805316e-07\\
0.7578758	-6.36233769280015e-08\\
0.7579758	-1.83946006759861e-08\\
0.7580758	7.05504539233459e-08\\
0.7581758	1.89633509534159e-07\\
0.7582758	2.25305875484194e-07\\
0.7583758	3.64048488077362e-07\\
0.7584758	3.92371938495728e-07\\
0.7585759	-3.50193580328506e-07\\
0.7586759	-2.79301893252315e-07\\
0.7587759	-1.59132686916585e-07\\
0.7588759	-1.03056370687193e-07\\
0.7589759	-2.44135285598635e-08\\
0.7590759	6.34851061165342e-08\\
0.7591759	1.4735868514304e-07\\
0.7592759	2.13956273170668e-07\\
0.7593759	3.50056877429505e-07\\
0.7594759	3.42469474751139e-07\\
0.759576	-3.32012385528824e-07\\
0.759676	-2.62805394030119e-07\\
0.759776	-1.76692470404971e-07\\
0.759876	-8.67444572283382e-08\\
0.759976	-6.00209681955732e-09\\
0.760076	5.25239949866751e-08\\
0.760176	1.75853351130906e-07\\
0.760276	2.51035676357425e-07\\
0.760376	2.65150869732667e-07\\
0.760476	4.05309048601143e-07\\
0.7605761	-3.15742030132515e-07\\
0.7606761	-2.58552622145469e-07\\
0.7607761	-2.13821161415551e-07\\
0.7608761	-9.4316433441044e-08\\
0.7609761	-1.27769058666516e-08\\
0.7610761	1.80892843759661e-08\\
0.7611761	8.56043517460936e-08\\
0.7612761	1.77120878130932e-07\\
0.7613761	2.80021829335286e-07\\
0.7614761	3.81720569542221e-07\\
0.7615762	-2.70360553333848e-07\\
0.7616762	-2.05337015085627e-07\\
0.7617762	-1.79105534153789e-07\\
0.7618762	-1.04130976452677e-07\\
0.7619762	7.15225978442646e-09\\
0.7620762	4.23402561899877e-08\\
0.7621762	8.90595852898191e-08\\
0.7622762	2.34967321099955e-07\\
0.7623762	2.6775104974841e-07\\
0.7624762	3.75128878488118e-07\\
0.7625763	-2.6205211178798e-07\\
0.7626763	-2.38965610965813e-07\\
0.7627763	-1.77959644243053e-07\\
0.7628763	-9.119390173562e-08\\
0.7629763	9.20248588442973e-09\\
0.7630763	1.11309583217256e-08\\
0.7631763	1.02523525585818e-07\\
0.7632763	1.71342777077754e-07\\
0.7633763	3.05581884046946e-07\\
0.7634763	2.93264605294441e-07\\
0.7635764	-2.5255719079964e-07\\
0.7636764	-1.90669972754964e-07\\
0.7637764	-1.11096182225934e-07\\
0.7638764	-1.25689654295602e-07\\
0.7639764	-4.62736255979479e-08\\
0.7640764	1.53592671081615e-08\\
0.7641764	1.47446986106803e-07\\
0.7642764	1.38258093140831e-07\\
0.7643764	2.7609174875487e-07\\
0.7644764	3.49277710184892e-07\\
0.7645765	-2.98117281319343e-07\\
0.7646765	-1.86108753116865e-07\\
0.7647765	-1.7358664804179e-07\\
0.7648765	-7.20988127117517e-08\\
0.7649765	6.83748868635092e-09\\
0.7650765	5.17355695506261e-08\\
0.7651765	1.51139319073956e-07\\
0.7652765	1.93623192576431e-07\\
0.7653765	2.67792207839235e-07\\
0.7654765	2.62281938764874e-07\\
0.7655766	-2.48985843914751e-07\\
0.7656766	-2.44933735585318e-07\\
0.7657766	-1.54482184211435e-07\\
0.7658766	-8.88734358467369e-08\\
0.7659766	-5.93192200915027e-08\\
0.7660766	2.29992406231005e-08\\
0.7661766	1.46931221578495e-07\\
0.7662766	2.01356485082793e-07\\
0.7663766	1.75185270109846e-07\\
0.7664766	2.57358280950082e-07\\
0.7665767	-2.49477454097868e-07\\
0.7666767	-1.80891253601101e-07\\
0.7667767	-1.36966993496968e-07\\
0.7668767	-2.8642198585116e-08\\
0.7669767	3.31760088706501e-08\\
0.7670767	3.7610895675666e-08\\
0.7671767	7.38161017449501e-08\\
0.7672767	1.30975629064611e-07\\
0.7673767	1.98303821516421e-07\\
0.7674767	2.65045352595972e-07\\
0.7675768	-2.38527245871589e-07\\
0.7676768	-2.02431133675129e-07\\
0.7677768	-9.90165030662071e-08\\
0.7678768	-3.89175022208477e-08\\
0.7679768	-3.27380346581974e-08\\
0.7680768	8.94822131669581e-09\\
0.7681768	9.55977940975128e-08\\
0.7682768	1.46697400046913e-07\\
0.7683768	2.01763923429743e-07\\
0.7684768	2.50344397087487e-07\\
0.7685769	-2.20732982267324e-07\\
0.7686769	-1.73795512578567e-07\\
0.7687769	-1.14532658371247e-07\\
0.7688769	-6.32769711211845e-08\\
0.7689769	-1.03309600318902e-08\\
0.7690769	4.40328861012196e-08\\
0.7691769	8.95720751320495e-08\\
0.7692769	1.4607409122569e-07\\
0.7693769	1.93356367583775e-07\\
0.7694769	2.41266264855433e-07\\
0.769577	-2.17852417730424e-07\\
0.769677	-1.665602399753e-07\\
0.769777	-1.14929090813676e-07\\
0.769877	-6.2992117494387e-08\\
0.769977	-1.0752667062297e-08\\
0.770077	4.18156895221067e-08\\
0.770177	8.47691303401676e-08\\
0.770277	1.38193555779265e-07\\
0.770377	1.8220456100837e-07\\
0.770477	2.26947408379918e-07\\
0.7705771	-2.0072897921608e-07\\
0.7706771	-1.51601911996257e-07\\
0.7707771	-1.11139243391678e-07\\
0.7708771	-5.90772955139096e-08\\
0.7709771	-1.5122866209305e-08\\
0.7710771	4.10467406591897e-08\\
0.7711771	7.97836856580858e-08\\
0.7712771	1.31469562167363e-07\\
0.7713771	1.76515368613406e-07\\
0.7714771	2.25361473371599e-07\\
0.7715772	-1.91619194511716e-07\\
0.7716772	-1.51464031802884e-07\\
0.7717772	-1.06021005114032e-07\\
0.7718772	-5.47325445382496e-08\\
0.7719772	-7.01187040375029e-09\\
0.7720772	3.77569762677332e-08\\
0.7721772	8.02190997795105e-08\\
0.7722772	1.21048715662209e-07\\
0.7723772	1.70949117681007e-07\\
0.7724772	2.10652642315678e-07\\
0.7725773	-1.86895796547293e-07\\
0.7726773	-1.43096261298603e-07\\
0.7727773	-9.71318468018234e-08\\
0.7728773	-4.81543631636328e-08\\
0.7729773	-5.28675299371617e-09\\
0.7730773	3.23768721832707e-08\\
0.7731773	7.57711959564755e-08\\
0.7732773	1.15859659755024e-07\\
0.7733773	1.6363442598144e-07\\
0.7734773	2.00116342247147e-07\\
0.7735774	-1.80100912572367e-07\\
0.7736774	-1.30941230067072e-07\\
0.7737774	-8.98490098016902e-08\\
0.7738774	-4.56890192623849e-08\\
0.7739774	-7.29753103778652e-09\\
0.7740774	2.65176352820529e-08\\
0.7741774	6.69770761213551e-08\\
0.7742774	1.15329763641925e-07\\
0.7743774	1.5285300630774e-07\\
0.7744774	1.90852408835096e-07\\
0.7745775	-1.65324334097661e-07\\
0.7746775	-1.30343956405987e-07\\
0.7747775	-9.08098270330226e-08\\
0.7748775	-4.53035333414231e-08\\
0.7749775	-1.2378570279914e-08\\
0.7750775	2.94396181738099e-08\\
0.7751775	7.16535970834986e-08\\
0.7752775	1.0579389861376e-07\\
0.7753775	1.434189798033e-07\\
0.7754775	1.76115180772207e-07\\
0.7755776	-1.60882424049547e-07\\
0.7756776	-1.21259495819426e-07\\
0.7757776	-8.16478793602693e-08\\
0.7758776	-5.03501071114076e-08\\
0.7759776	-5.64104558897771e-09\\
0.7760776	2.42320626897552e-08\\
0.7761776	6.10495524688837e-08\\
0.7762776	9.66192928580911e-08\\
0.7763776	1.32776642833288e-07\\
0.7764776	1.71384406180697e-07\\
0.7765777	-1.53273873253201e-07\\
0.7766777	-1.12235095259261e-07\\
0.7767777	-8.30013154456966e-08\\
0.7768777	-4.36003752546377e-08\\
0.7769777	-1.20328978725492e-08\\
0.7770777	2.37276651174345e-08\\
0.7771777	6.57349876753699e-08\\
0.7772777	9.60698230897949e-08\\
0.7773777	1.26839957176328e-07\\
0.7774777	1.60180161266898e-07\\
0.7775778	-1.41389143171189e-07\\
0.7776778	-1.04643755695355e-07\\
0.7777778	-7.87703107324456e-08\\
0.7778778	-4.15265387701536e-08\\
0.7779778	-1.06434220326079e-08\\
0.7780778	2.61747591656647e-08\\
0.7781778	6.1250376139077e-08\\
0.7782778	8.69324049551645e-08\\
0.7783778	1.25596377457959e-07\\
0.7784778	1.59644332697884e-07\\
0.7785779	-1.40951142452161e-07\\
0.7786779	-1.07146820069182e-07\\
0.7787779	-7.0601349343169e-08\\
0.7788779	-3.88071260959411e-08\\
0.7789779	-9.23028516813806e-09\\
0.7790779	2.06892504170386e-08\\
0.7791779	5.35377191052122e-08\\
0.7792779	8.19274713487195e-08\\
0.7793779	1.18496918867506e-07\\
0.7794779	1.45910484868061e-07\\
0.779578	-1.29165358686567e-07\\
0.779678	-1.00363880276011e-07\\
0.779778	-6.25766970602237e-08\\
0.779878	-3.30358214672977e-08\\
0.779978	-8.94750858820847e-09\\
0.780078	2.25076923834888e-08\\
0.780178	5.41748878674086e-08\\
0.780278	7.89247867546972e-08\\
0.780378	1.09653649771335e-07\\
0.780478	1.39283237395038e-07\\
0.7805781	-1.19558417457921e-07\\
0.7806781	-9.17290470525689e-08\\
0.7807781	-6.60881799841029e-08\\
0.7808781	-3.9612551280821e-08\\
0.7809781	-9.25365726406335e-09\\
0.7810781	1.80621926235136e-08\\
0.7811781	4.54338223085382e-08\\
0.7812781	7.59851369117048e-08\\
0.7813781	1.02865069281921e-07\\
0.7814781	1.29247527520804e-07\\
0.7815782	-1.16984752232541e-07\\
0.7816782	-9.05133426196136e-08\\
0.7817782	-6.48751046911089e-08\\
0.7818782	-3.67967384062862e-08\\
0.7819782	-2.98023589834218e-09\\
0.7820782	1.98970649192831e-08\\
0.7821782	4.5182426550161e-08\\
0.7822782	7.62476587978789e-08\\
0.7823782	9.64890630589244e-08\\
0.7824782	1.29327379458366e-07\\
0.7825783	-1.12781845230947e-07\\
0.7826783	-8.29937509758127e-08\\
0.7827783	-6.02069303370922e-08\\
0.7828783	-3.09034090678617e-08\\
0.7829783	-1.54104622118734e-09\\
0.7830783	2.14464105779966e-08\\
0.7831783	4.16492705671012e-08\\
0.7832783	7.26818448273936e-08\\
0.7833783	9.81823912588053e-08\\
0.7834783	1.21813059736353e-07\\
0.7835784	-1.00055239447838e-07\\
0.7836784	-7.77502000906538e-08\\
0.7837784	-5.61920443301056e-08\\
0.7838784	-3.16235832728551e-08\\
0.7839784	-2.64012721407081e-10\\
0.7840784	2.16910314917995e-08\\
0.7841784	3.80694173685026e-08\\
0.7842784	6.27224623905853e-08\\
0.7843784	8.95248761512146e-08\\
0.7844784	1.12374705610319e-07\\
0.7845785	-9.90753015323032e-08\\
0.7846785	-7.10725409289181e-08\\
0.7847785	-5.51947815022324e-08\\
0.7848785	-2.74511637865515e-08\\
0.7849785	-3.82777203683515e-09\\
0.7850785	1.9712310631026e-08\\
0.7851785	3.72289456485042e-08\\
0.7852785	6.28048828707484e-08\\
0.7853785	8.05457041411217e-08\\
0.7854785	1.04579766797841e-07\\
0.7855786	-9.27684887033764e-08\\
0.7856786	-7.24603412757818e-08\\
0.7857786	-5.33435836544416e-08\\
0.7858786	-3.11992986029574e-08\\
0.7859786	-1.78607596768021e-09\\
0.7860786	1.91599304477652e-08\\
0.7861786	3.59249462816846e-08\\
0.7862786	6.28175194986036e-08\\
0.7863786	8.41684645522256e-08\\
0.7864786	1.04330804259756e-07\\
0.7865787	-9.22866346952933e-08\\
0.7866787	-7.0199098423096e-08\\
0.7867787	-4.61139838328628e-08\\
0.7868787	-2.55899789275427e-08\\
0.7869787	-4.16384834250882e-09\\
0.7870787	1.26495107860558e-08\\
0.7871787	3.93570110113206e-08\\
0.7872787	6.04873179599341e-08\\
0.7873787	8.05907929835037e-08\\
0.7874787	9.4239435956267e-08\\
0.7875788	-8.26385604182911e-08\\
0.7876788	-6.6997247402667e-08\\
0.7877788	-4.3970859689435e-08\\
0.7878788	-1.89013937279547e-08\\
0.7879788	-7.10949355636759e-09\\
0.7880788	1.61054920580495e-08\\
0.7881788	3.54654520196007e-08\\
0.7882788	5.57134559216188e-08\\
0.7883788	7.16136961086133e-08\\
0.7884788	8.79514316723395e-08\\
0.7885789	-7.8369068120393e-08\\
0.7886789	-5.56691010233923e-08\\
0.7887789	-3.8055182928487e-08\\
0.7888789	-2.06583448646158e-08\\
0.7889789	-8.58883816498568e-09\\
0.7890789	1.30638068175648e-08\\
0.7891789	2.92307241051004e-08\\
0.7892789	5.48636545881498e-08\\
0.7893789	6.49348885981005e-08\\
0.7894789	8.44372085896661e-08\\
0.789579	-7.92711572390747e-08\\
0.789679	-5.48496379877694e-08\\
0.789779	-3.59012430166394e-08\\
0.789879	-1.73517753811192e-08\\
0.789979	-4.10683294663206e-09\\
0.790079	8.94813418847451e-09\\
0.790179	2.69477663447648e-08\\
0.790279	4.50467367088536e-08\\
0.790379	6.84196942851711e-08\\
0.790479	8.22612067630235e-08\\
0.7905791	-6.6118046176955e-08\\
0.7906791	-5.47277962462123e-08\\
0.7907791	-3.71710880478493e-08\\
0.7908791	-1.81742338609125e-08\\
0.7909791	-2.44391418560697e-09\\
0.7910791	1.53327657619051e-08\\
0.7911791	3.04882176692389e-08\\
0.7912791	4.83743139122195e-08\\
0.7913791	6.4362329729728e-08\\
0.7914791	7.3842886572778e-08\\
0.7915792	-6.64023441267203e-08\\
0.7916792	-5.27856921818393e-08\\
0.7917792	-3.9393672060517e-08\\
0.7918792	-2.07588302277018e-08\\
0.7919792	-1.39465237969372e-09\\
0.7920792	1.42043792249191e-08\\
0.7921792	3.15627291247456e-08\\
0.7922792	4.6223751548613e-08\\
0.7923792	5.37496335911714e-08\\
0.7924792	6.97213381230244e-08\\
0.7925793	-6.00704445768985e-08\\
0.7926793	-4.95319085400125e-08\\
0.7927793	-3.369692666913e-08\\
0.7928793	-1.69099814463503e-08\\
0.7929793	-3.49706330213251e-09\\
0.7930793	1.22342732576664e-08\\
0.7931793	2.59948520382747e-08\\
0.7932793	4.35138194851414e-08\\
0.7933793	5.05385876828357e-08\\
0.7934793	7.28347780386274e-08\\
0.7935794	-5.52409486626049e-08\\
0.7936794	-4.42177327288074e-08\\
0.7937794	-3.05217098470201e-08\\
0.7938794	-1.83149674419514e-08\\
0.7939794	-1.74166548544408e-09\\
0.7940794	1.50719073818695e-08\\
0.7941794	2.80172775084242e-08\\
0.7942794	3.30037302730124e-08\\
0.7943794	5.59582539784442e-08\\
0.7944794	6.28254837771458e-08\\
0.7945795	-5.38966282888353e-08\\
0.7946795	-4.05259245953027e-08\\
0.7947795	-2.53072814128397e-08\\
0.7948795	-1.22260176947664e-08\\
0.7949795	-5.25008501603419e-09\\
0.7950795	1.16698768981571e-08\\
0.7951795	2.46004848679371e-08\\
0.7952795	3.96255559059755e-08\\
0.7953795	5.28460522841723e-08\\
0.7954795	6.03800258992815e-08\\
0.7955796	-5.75401462137026e-08\\
0.7956796	-4.22223006384115e-08\\
0.7957796	-2.41386608063054e-08\\
0.7958796	-1.71033529353104e-08\\
0.7959796	-4.91369019431365e-09\\
0.7960796	8.64977240874754e-09\\
0.7961796	1.98231828209272e-08\\
0.7962796	3.48593369829675e-08\\
0.7963796	5.00276240746056e-08\\
0.7964796	6.16139714873509e-08\\
0.7965797	-5.28044217483481e-08\\
0.7966797	-3.87610896261659e-08\\
0.7967797	-2.93468890993909e-08\\
0.7968797	-1.82102703159825e-08\\
0.7969797	1.01658169729402e-09\\
0.7970797	4.71769253523879e-09\\
0.7971797	1.92932438879134e-08\\
0.7972797	3.11595199052606e-08\\
0.7973797	4.67488521366244e-08\\
0.7974797	5.25095658308883e-08\\
0.7975798	-4.70092028741209e-08\\
0.7976798	-3.08356866603188e-08\\
0.7977798	-2.50538873987349e-08\\
0.7978798	-1.31520295244825e-08\\
0.7979798	1.39738673063383e-09\\
0.7980798	5.13753143499551e-09\\
0.7981798	1.46271913747942e-08\\
0.7982798	2.64407165155578e-08\\
0.7983798	3.71679665482627e-08\\
0.7984798	5.34142574545871e-08\\
0.7985799	-4.36560439291689e-08\\
0.7986799	-3.58670668200445e-08\\
0.7987799	-2.26542361293769e-08\\
0.7988799	-1.73509274205286e-08\\
0.7989799	-3.27532558713761e-09\\
0.7990799	6.26952224858091e-09\\
0.7991799	1.79956536842246e-08\\
0.7992799	2.86301389106763e-08\\
0.7993799	3.49150273727672e-08\\
0.7994799	4.36072956095518e-08\\
0.79958	-3.78545184902962e-08\\
0.79968	-3.34227527338388e-08\\
0.79978	-2.62268310992353e-08\\
0.79988	-9.45057787929415e-09\\
0.79998	-6.26315223989371e-09\\
0.80008	1.01809001576414e-08\\
};
\addplot [color=mycolor1,solid,forget plot]
  table[row sep=crcr]{%
0.80008	1.01809001576414e-08\\
0.80018	1.67415951865535e-08\\
0.80028	3.02934583858855e-08\\
0.80038	3.77254731186855e-08\\
0.80048	4.59410289463391e-08\\
0.8005801	-4.16731039849794e-08\\
0.8006801	-3.05598276059693e-08\\
0.8007801	-1.78699662307241e-08\\
0.8008801	-6.64300624404757e-09\\
0.8009801	9.57149984270877e-11\\
0.8010801	9.33495821094088e-09\\
0.8011801	1.80775309634867e-08\\
0.8012801	2.33402368196578e-08\\
0.8013801	3.21538247277653e-08\\
0.8014801	4.15629377772259e-08\\
0.8015802	-3.9408756570547e-08\\
0.8016802	-2.70860737016515e-08\\
0.8017802	-2.29539829938591e-08\\
0.8018802	-9.91275555216331e-09\\
0.8019802	-8.49020532697753e-10\\
0.8020802	1.13641846280488e-08\\
0.8021802	1.38673643679255e-08\\
0.8022802	2.38145148254154e-08\\
0.8023802	2.83730740534283e-08\\
0.8024802	4.47238721092147e-08\\
0.8025803	-3.27697669139093e-08\\
0.8026803	-3.07338801769363e-08\\
0.8027803	-1.52861807933835e-08\\
0.8028803	-9.19275310561074e-09\\
0.8029803	-5.20653681453098e-09\\
0.8030803	3.93262376309489e-09\\
0.8031803	1.54979305673486e-08\\
0.8032803	2.67755829036487e-08\\
0.8033803	2.50647283726868e-08\\
0.8034803	3.76774144904279e-08\\
0.8035804	-3.59670241870808e-08\\
0.8036804	-2.22421017940372e-08\\
0.8037804	-1.21832285128465e-08\\
0.8038804	-8.42723168158199e-09\\
0.8039804	-3.59828115047145e-09\\
0.8040804	9.69206196549388e-09\\
0.8041804	8.84479764949597e-09\\
0.8042804	2.1273438890676e-08\\
0.8043804	2.44039638306465e-08\\
0.8044804	3.56747679973457e-08\\
0.8045805	-3.07093460816948e-08\\
0.8046805	-2.03415767251192e-08\\
0.8047805	-1.94468314316382e-08\\
0.8048805	-1.05375081577153e-08\\
0.8049805	-6.1138247069159e-09\\
0.8050805	1.33613427960588e-09\\
0.8051805	9.33637018390998e-09\\
0.8052805	1.54229235892894e-08\\
0.8053805	2.7143827437473e-08\\
0.8054805	3.20590605468879e-08\\
0.8055806	-3.10990269508649e-08\\
0.8056806	-2.66405046946139e-08\\
0.8057806	-1.62389066811486e-08\\
0.8058806	-1.22869242644846e-08\\
0.8059806	-1.16553506429584e-09\\
0.8060806	5.75595140474283e-09\\
0.8061806	1.21198472785577e-08\\
0.8062806	1.85800408704362e-08\\
0.8063806	2.48019509079683e-08\\
0.8064806	3.14624809122366e-08\\
0.8065807	-2.7424240673507e-08\\
0.8066807	-2.04063691187917e-08\\
0.8067807	-1.38529958615408e-08\\
0.8068807	-8.04172058110897e-09\\
0.8069807	-1.23888466325994e-09\\
0.8070807	5.30038408823319e-09\\
0.8071807	1.1332126528818e-08\\
0.8072807	1.76235075040332e-08\\
0.8073807	2.29527711854277e-08\\
0.8074807	2.91091966260482e-08\\
0.8075808	-2.58453881404191e-08\\
0.8076808	-1.92415371344512e-08\\
0.8077808	-1.33790933540268e-08\\
0.8078808	-7.42506990961123e-09\\
0.8079808	-1.53566595550991e-09\\
0.8080808	5.14368989762426e-09\\
0.8081808	1.04782953806726e-08\\
0.8082808	1.63441313035667e-08\\
0.8083808	2.16278180035362e-08\\
0.8084808	2.72265719396952e-08\\
0.8085809	-2.39729187178489e-08\\
0.8086809	-1.76501096238457e-08\\
0.8087809	-1.22595048088897e-08\\
0.8088809	-6.86192413307252e-09\\
0.8089809	-1.50780639709969e-09\\
0.8090809	4.76274791307019e-09\\
0.8091809	9.91993407209457e-09\\
0.8092809	1.49442006634054e-08\\
0.8093809	2.08262072587395e-08\\
0.8094809	2.55667821497108e-08\\
0.809581	-2.23345649093265e-08\\
0.809681	-1.64940004293568e-08\\
0.809781	-1.17337546458665e-08\\
0.809881	-6.01273421243359e-09\\
0.809981	-1.27988648672417e-09\\
0.810081	4.52575889871992e-09\\
0.810181	9.47504878463096e-09\\
0.810281	1.46486649242952e-08\\
0.810381	1.91370829539031e-08\\
0.810481	2.40405312592826e-08\\
0.8105811	-2.07303258404831e-08\\
0.8106811	-1.63366187114558e-08\\
0.8107811	-1.11709544930311e-08\\
0.8108811	-6.09449263798412e-09\\
0.8109811	-9.58843502860129e-10\\
0.8110811	4.39389119272909e-09\\
0.8111811	9.1310781722212e-09\\
0.8112811	1.34295123832215e-08\\
0.8113811	1.84753769583504e-08\\
0.8114811	2.24642033707506e-08\\
0.8115812	-1.94739024029708e-08\\
0.8116812	-1.46728927142084e-08\\
0.8117812	-1.02884068183332e-08\\
0.8118812	-5.08790793263963e-09\\
0.8119812	-8.29708940441876e-10\\
0.8120812	3.7369883940791e-09\\
0.8121812	7.8720542185523e-09\\
0.8122812	1.2844391839191e-08\\
0.8123812	1.69318985004593e-08\\
0.8124812	2.14214268294197e-08\\
0.8125813	-1.8519640475681e-08\\
0.8126813	-1.40447259291271e-08\\
0.8127813	-9.25556788242826e-09\\
0.8128813	-4.82986905591537e-09\\
0.8129813	-4.36568971078077e-10\\
0.8130813	3.264117936011e-09\\
0.8131813	7.62066437211273e-09\\
0.8132813	1.19901922895557e-08\\
0.8133813	1.57384352078571e-08\\
0.8134813	2.02397005405475e-08\\
0.8135814	-1.74743649532932e-08\\
0.8136814	-1.30412992618498e-08\\
0.8137814	-8.68281573116608e-09\\
0.8138814	-4.99067669221948e-09\\
0.8139814	-5.48256968470329e-10\\
0.8140814	3.06941921966036e-09\\
0.8141814	7.29564144600486e-09\\
0.8142814	1.15719764361144e-08\\
0.8143814	1.53482313421105e-08\\
0.8144814	1.90824173927628e-08\\
0.8145815	-1.64937983921851e-08\\
0.8146815	-1.21839443031208e-08\\
0.8147815	-8.49477761917516e-09\\
0.8148815	-4.93582518433511e-09\\
0.8149815	-1.00859055742614e-09\\
0.8150815	2.79341038321446e-09\\
0.8151815	6.98461369904452e-09\\
0.8152815	1.00873718683614e-08\\
0.8153815	1.36319183510758e-08\\
0.8154815	1.7156332232475e-08\\
0.8155816	-1.50635471015195e-08\\
0.8156816	-1.16955962596935e-08\\
0.8157816	-7.6882529977812e-09\\
0.8158816	-4.47239884016071e-09\\
0.8159816	-4.71244576347174e-10\\
0.8160816	2.8996351006047e-09\\
0.8161816	6.23226704705299e-09\\
0.8162816	1.01262452407644e-08\\
0.8163816	1.31886966616909e-08\\
0.8164816	1.60342468035618e-08\\
0.8165817	-1.46649057229317e-08\\
0.8166817	-1.11551056907258e-08\\
0.8167817	-6.9750377314215e-09\\
0.8168817	-3.48041371157209e-09\\
0.8169817	-1.0196161866205e-09\\
0.8170817	3.06626825232609e-09\\
0.8171817	5.44341555250988e-09\\
0.8172817	8.7852309047301e-09\\
0.8173817	1.27723155931778e-08\\
0.8174817	1.50924339319472e-08\\
0.8175818	-1.33259850793044e-08\\
0.8176818	-1.00369363584263e-08\\
0.8177818	-7.31019913092351e-09\\
0.8178818	-3.42967826274382e-09\\
0.8179818	-6.72279174170065e-10\\
0.8180818	2.69205970536424e-09\\
0.8181818	5.4003347080625e-09\\
0.8182818	8.19644476856424e-09\\
0.8183818	1.18311593248072e-08\\
0.8184818	1.40620865177882e-08\\
0.8185819	-1.20588975489194e-08\\
0.8186819	-9.13700647694332e-09\\
0.8187819	-6.3065833744505e-09\\
0.8188819	-3.78297623521845e-09\\
0.8189819	-7.74852434819173e-10\\
0.8190819	2.51576982596841e-09\\
0.8191819	4.89349049476173e-09\\
0.8192819	8.1694965679488e-09\\
0.8193819	1.11615311316343e-08\\
0.8194819	1.36938625504117e-08\\
0.819582	-1.11839518965318e-08\\
0.819682	-8.88564566240216e-09\\
0.819782	-5.53654570757331e-09\\
0.819882	-3.28658238037821e-09\\
0.819982	-2.79313321603636e-10\\
0.820082	2.34804635153643e-09\\
0.820182	4.46459397056287e-09\\
0.820282	7.94570933961708e-09\\
0.820382	9.67302490103433e-09\\
0.820482	1.2534395940015e-08\\
0.8205821	-1.05420045271379e-08\\
0.8206821	-8.54883313227903e-09\\
0.8207821	-5.72208945026978e-09\\
0.8208821	-3.14931587709275e-09\\
0.8209821	-9.11979108550454e-10\\
0.8210821	1.91450087293904e-09\\
0.8211821	4.26072160842825e-09\\
0.8212821	7.06326941986996e-09\\
0.8213821	9.26469057678167e-09\\
0.8214821	1.18134626360747e-08\\
0.8215822	-1.05962972372226e-08\\
0.8216822	-7.31906482599395e-09\\
0.8217822	-4.81477737193647e-09\\
0.8218822	-3.11151105835544e-09\\
0.8219822	-2.31552619526371e-10\\
0.8220822	1.8085726206582e-09\\
0.8221822	3.9980728077163e-09\\
0.8222822	6.33186172894692e-09\\
0.8223822	8.81053108204785e-09\\
0.8224822	1.14403228253582e-08\\
0.8225823	-9.42527434528648e-09\\
0.8226823	-7.2973575931945e-09\\
0.8227823	-4.9669396182811e-09\\
0.8228823	-2.40544592317732e-09\\
0.8229823	-5.78788330197338e-10\\
0.8230823	1.55260811585817e-09\\
0.8231823	4.0337780958348e-09\\
0.8232823	5.91518930278395e-09\\
0.8233823	8.25271576640821e-09\\
0.8234823	1.01076112254267e-08\\
0.8235824	-8.60802346546938e-09\\
0.8236824	-6.36804904424401e-09\\
0.8237824	-4.39585056419625e-09\\
0.8238824	-2.60891397114299e-09\\
0.8239824	8.05230261809176e-11\\
0.8240824	1.76544504709719e-09\\
0.8241824	3.54403311211837e-09\\
0.8242824	5.51963888614435e-09\\
0.8243824	7.80075898884972e-09\\
0.8244824	9.50100943926391e-09\\
0.8245825	-8.0041170080758e-09\\
0.8246825	-5.96818280559908e-09\\
0.8247825	-4.14267682571218e-09\\
0.8248825	-2.39375427032312e-09\\
0.8249825	-5.82577361785824e-10\\
0.8250825	1.4346596423874e-09\\
0.8251825	3.80670569502181e-09\\
0.8252825	5.68722799106428e-09\\
0.8253825	7.23478741368813e-09\\
0.8254825	8.61281393427227e-09\\
0.8255826	-7.4297537393242e-09\\
0.8256826	-5.75338965271704e-09\\
0.8257826	-3.72811826679018e-09\\
0.8258826	-2.17127247691823e-09\\
0.8259826	1.04562428594052e-10\\
0.8260826	1.2915247578002e-09\\
0.8261826	3.58645249262657e-09\\
0.8262826	5.19085950655487e-09\\
0.8263826	6.31091185343199e-09\\
0.8264826	8.15740413715946e-09\\
0.8265827	-7.23220504360647e-09\\
0.8266827	-5.16227936092628e-09\\
0.8267827	-3.70676642592188e-09\\
0.8268827	-1.63658709761288e-09\\
0.8269827	-7.1815032219591e-10\\
0.8270827	1.28662381356642e-09\\
0.8271827	2.62028110797549e-09\\
0.8272827	4.52981047853809e-09\\
0.8273827	6.26662108879975e-09\\
0.8274827	8.08651974808922e-09\\
0.8275828	-6.76466608423576e-09\\
0.8276828	-4.88145201083673e-09\\
0.8277828	-3.12229789332236e-09\\
0.8278828	-2.21402689671441e-09\\
0.8279828	1.2082348793338e-10\\
0.8280828	1.16397935907842e-09\\
0.8281828	3.20140835933028e-09\\
0.8282828	4.5232977571863e-09\\
0.8283828	5.42403262365418e-09\\
0.8284828	7.20217408151438e-09\\
0.8285829	-5.76369267027074e-09\\
0.8286829	-4.21331716390916e-09\\
0.8287829	-2.86569839863179e-09\\
0.8288829	-1.40578253248272e-09\\
0.8289829	-5.1444704767846e-10\\
0.8290829	1.13147807356134e-09\\
0.8291829	2.85918923042736e-09\\
0.8292829	3.9998881711141e-09\\
0.8293829	5.88876113887318e-09\\
0.8294829	6.8649579059582e-09\\
0.829583	-5.63147564019463e-09\\
0.829683	-4.34897922188537e-09\\
0.829783	-2.93878224793404e-09\\
0.829883	-2.0460832594188e-09\\
0.829983	-3.12220143402347e-10\\
0.830083	6.25309615068584e-10\\
0.830183	2.13282870401357e-09\\
0.830283	3.58045982771192e-09\\
0.830383	5.34210568424627e-09\\
0.830483	6.7954289831914e-09\\
0.8305831	-5.62526353471184e-09\\
0.8306831	-4.54850476749348e-09\\
0.8307831	-2.62532801314358e-09\\
0.8308831	-1.46322472346049e-09\\
0.8309831	-6.66024911003676e-10\\
0.8310831	1.16608346069358e-09\\
0.8311831	2.43653507764861e-09\\
0.8312831	3.55236798072096e-09\\
0.8313831	4.92420444480821e-09\\
0.8314831	5.96623177403187e-09\\
0.8315832	-4.95629270155611e-09\\
0.8316832	-4.23093575647494e-09\\
0.8317832	-2.57219889159867e-09\\
0.8318832	-1.55182046141129e-09\\
0.8319832	2.61931956376318e-10\\
0.8320832	1.30424291137042e-09\\
0.8321832	2.01373056315894e-09\\
0.8322832	2.83242824593993e-09\\
0.8323832	4.2057660958484e-09\\
0.8324832	5.58255277140034e-09\\
0.8325833	-4.80062385872261e-09\\
0.8326833	-3.97645067481273e-09\\
0.8327833	-2.78285267421054e-09\\
0.8328833	-1.75768502920851e-09\\
0.8329833	-4.35514587717942e-10\\
0.8330833	6.52362302849471e-10\\
0.8331833	1.97790208647172e-09\\
0.8332833	3.01629635613039e-09\\
0.8333833	4.24595430039212e-09\\
0.8334833	5.14848524533693e-09\\
0.8335834	-4.22431288039926e-09\\
0.8336834	-3.4431040867814e-09\\
0.8337834	-2.52567461823255e-09\\
0.8338834	-9.77784804512272e-10\\
0.8339834	-3.02081063505502e-10\\
0.8340834	1.00188702126247e-09\\
0.8341834	1.43764982947839e-09\\
0.8342834	2.51180081855401e-09\\
0.8343834	3.73397976666072e-09\\
0.8344834	4.61685604894249e-09\\
0.8345835	-4.02536586051674e-09\\
0.8346835	-3.20059620364903e-09\\
0.8347835	-2.15958611887376e-09\\
0.8348835	-1.37771030543112e-09\\
0.8349835	-3.27396233604574e-10\\
0.8350835	5.2185962148843e-10\\
0.8351835	1.70347559647586e-09\\
0.8352835	2.75376875664458e-09\\
0.8353835	3.21193887090586e-09\\
0.8354835	4.6200524476004e-09\\
0.8355836	-3.49494063789195e-09\\
0.8356836	-2.48353558703583e-09\\
0.8357836	-1.87939333084856e-09\\
0.8358836	-1.12913550861723e-09\\
0.8359836	3.23404340850722e-10\\
0.8360836	1.03716532259303e-09\\
0.8361836	1.57384368450024e-09\\
0.8362836	2.49787750054332e-09\\
0.8363836	3.37643137212316e-09\\
0.8364836	3.77938120816462e-09\\
0.8365837	-3.10026348950693e-09\\
0.8366837	-2.86666388196744e-09\\
0.8367837	-1.38334456997041e-09\\
0.8368837	-1.06973249876241e-09\\
0.8369837	-3.42618337997522e-10\\
0.8370837	3.83828733770478e-10\\
0.8371837	1.69804630621197e-09\\
0.8372837	2.19106408298926e-09\\
0.8373837	3.45648930744763e-09\\
0.8374837	4.09049223226016e-09\\
0.8375838	-3.19172766081316e-09\\
0.8376838	-2.4645087312071e-09\\
0.8377838	-1.66536219797092e-09\\
0.8378838	-8.88006941742424e-10\\
0.8379838	-1.23670353254987e-10\\
0.8380838	6.38897551725505e-10\\
0.8381838	1.4134101272191e-09\\
0.8382838	2.11603008975613e-09\\
0.8383838	2.86535565868946e-09\\
0.8384838	3.48240670111286e-09\\
0.8385839	-3.03663330852161e-09\\
0.8386839	-2.2579233347397e-09\\
0.8387839	-1.53440625342335e-09\\
0.8388839	-8.35510185451423e-10\\
0.8389839	-1.28309761036546e-10\\
0.8390839	6.22460432481674e-10\\
0.8391839	1.25439250024499e-09\\
0.8392839	1.90739191704547e-09\\
0.8393839	2.62366430534358e-09\\
0.8394839	3.24770225014308e-09\\
0.839584	-2.78201410281072e-09\\
0.839684	-2.14995625718705e-09\\
0.839784	-1.46341168046418e-09\\
0.839884	-7.68868145474728e-10\\
0.839984	-1.10591384140657e-10\\
0.840084	5.69362115505727e-10\\
0.840184	1.13113241570571e-09\\
0.840284	1.83704341398094e-09\\
0.840384	2.45159027182761e-09\\
0.840484	3.04142686459018e-09\\
0.8405841	-2.54897723002585e-09\\
0.8406841	-1.95347597971816e-09\\
0.8407841	-1.37022143588059e-09\\
0.8408841	-7.24046932978027e-10\\
0.8409841	-1.37688883515216e-10\\
0.8410841	4.6820106255549e-10\\
0.8411841	1.07504388637686e-09\\
0.8412841	1.66632116405888e-09\\
0.8413841	2.22756306954806e-09\\
0.8414841	2.8463364424745e-09\\
0.8415842	-2.36095881896419e-09\\
0.8416842	-1.81294956792203e-09\\
0.8417842	-1.23291213197964e-09\\
0.8418842	-6.25249139030344e-10\\
0.8419842	-9.2385318362434e-11\\
0.8420842	4.65220913658734e-10\\
0.8421842	1.04906593080458e-09\\
0.8422842	1.56258946016827e-09\\
0.8423842	2.11116315705293e-09\\
0.8424842	2.60207926754998e-09\\
0.8425843	-2.20826859901289e-09\\
0.8426843	-1.66274625459321e-09\\
0.8427843	-1.14187984859727e-09\\
0.8428843	-6.30805197248015e-10\\
0.8429843	-1.127933913804e-10\\
0.8430843	4.30738190016107e-10\\
0.8431843	9.20214938103514e-10\\
0.8432843	1.4778941203426e-09\\
0.8433843	1.92785404177012e-09\\
0.8434843	2.395983275498e-09\\
0.8435844	-2.05126459459741e-09\\
0.8436844	-1.524303017684e-09\\
0.8437844	-1.09101968269205e-09\\
0.8438844	-5.18389905762758e-10\\
0.8439844	-7.16318345785061e-11\\
0.8440844	3.85783112378212e-10\\
0.8441844	8.92119836554475e-10\\
0.8442844	1.38736922167607e-09\\
0.8443844	1.81323784719661e-09\\
0.8444844	2.21313773839278e-09\\
0.8445845	-1.86446215931502e-09\\
0.8446845	-1.44445296164624e-09\\
0.8447845	-1.01029599150568e-09\\
0.8448845	-5.11857065521766e-10\\
0.8449845	-9.73469707621355e-11\\
0.8450845	3.86668631792084e-10\\
0.8451845	7.95259333863027e-10\\
0.8452845	1.2851201544985e-09\\
0.8453845	1.61456177735168e-09\\
0.8454845	2.04350082927234e-09\\
0.8455846	-1.72384206909015e-09\\
0.8456846	-1.37717671164564e-09\\
0.8457846	-9.41959324559857e-10\\
0.8458846	-4.51944096404623e-10\\
0.8459846	-3.93271104700763e-11\\
0.8460846	3.65244243161721e-10\\
0.8461846	7.32661901742433e-10\\
0.8462846	1.13534780795538e-09\\
0.8463846	1.54724466203647e-09\\
0.8464846	1.94380667318432e-09\\
0.8465847	-1.63957934181487e-09\\
0.8466847	-1.21099438186697e-09\\
0.8467847	-8.62623238573508e-10\\
0.8468847	-4.13055339823325e-10\\
0.8469847	-7.94139418938342e-11\\
0.8470847	3.22634906638197e-10\\
0.8471847	6.78873509870389e-10\\
0.8472847	1.0765236780892e-09\\
0.8473847	1.40423794917381e-09\\
0.8474847	1.75209084339514e-09\\
0.8475848	-1.5363695353909e-09\\
0.8476848	-1.14416998347588e-09\\
0.8477848	-7.53368050343713e-10\\
0.8478848	-3.68286357736291e-10\\
0.8479848	-9.18685532920287e-11\\
0.8480848	2.74312227663623e-10\\
0.8481848	6.30044948179455e-10\\
0.8482848	9.76472295437323e-10\\
0.8483848	1.31608235334063e-09\\
0.8484848	1.65270033097333e-09\\
0.8485849	-1.38348129084592e-09\\
0.8486849	-1.00985231895473e-09\\
0.8487849	-7.19990814310092e-10\\
0.8488849	-4.04803724991364e-10\\
0.8489849	-5.39016194959357e-11\\
0.8490849	2.44393279105835e-10\\
0.8491849	6.03039132518468e-10\\
0.8492849	9.36266559994989e-10\\
0.8493849	1.25957074533899e-09\\
0.8494849	1.48970360103629e-09\\
0.849585	-1.27661479347799e-09\\
0.849685	-9.53226901437895e-10\\
0.849785	-6.65470942836765e-10\\
0.849885	-3.91644454226289e-10\\
0.849985	-8.82682609729818e-12\\
0.850085	2.07113105257697e-10\\
0.850185	5.81509511767345e-10\\
0.850285	8.40892066698504e-10\\
0.850385	1.11297848541525e-09\\
0.850485	1.42666709153881e-09\\
0.8505851	-1.17359434350555e-09\\
0.8506851	-8.62701938966065e-10\\
0.8507851	-6.16669307306784e-10\\
0.8508851	-3.01947924996604e-10\\
0.8509851	-8.38451684925974e-11\\
0.8510851	2.7346851904286e-10\\
0.8511851	5.06952465900365e-10\\
0.8512851	7.54688686110255e-10\\
0.8513851	1.05587481499022e-09\\
0.8514851	1.35081708188892e-09\\
0.8515852	-1.08587102813163e-09\\
0.8516852	-8.57099067500672e-10\\
0.8517852	-6.07226186325437e-10\\
0.8518852	-2.91580125007847e-10\\
0.8519852	-6.44145609247915e-11\\
0.8520852	2.2108410187207e-10\\
0.8521852	5.12789980020457e-10\\
0.8522852	7.59630997535822e-10\\
0.8523852	1.01158220770217e-09\\
0.8524852	1.21965915134397e-09\\
0.8525853	-1.02775767090628e-09\\
0.8526853	-7.30765571481645e-10\\
0.8527853	-5.18569387734577e-10\\
0.8528853	-2.36056251477247e-10\\
0.8529853	-2.71054490701486e-11\\
0.8530853	1.65405156368412e-10\\
0.8531853	3.99592753851097e-10\\
0.8532853	6.34563224611705e-10\\
0.8533853	9.30404840302455e-10\\
0.8534853	1.14818199143264e-09\\
0.8535854	-9.25264001104976e-10\\
0.8536854	-7.58470588744196e-10\\
0.8537854	-4.80889553980575e-10\\
0.8538854	-2.27613124311951e-10\\
0.8539854	-3.27882562680543e-11\\
0.8540854	1.70377300538326e-10\\
0.8541854	4.49608978140507e-10\\
0.8542854	5.73559395724427e-10\\
0.8543854	8.1180239950001e-10\\
0.8544854	1.03482713657886e-09\\
0.8545855	-8.86344766895877e-10\\
0.8546855	-6.61892463689646e-10\\
0.8547855	-4.35884269162686e-10\\
0.8548855	-2.3422751493354e-10\\
0.8549855	1.80566417937525e-11\\
0.8550855	1.96827494109092e-10\\
0.8551855	3.7881910116915e-10\\
0.8552855	5.41634616497977e-10\\
0.8553855	7.63740674002145e-10\\
0.8554855	9.24461788858692e-10\\
0.8555856	-8.34318220955567e-10\\
0.8556856	-6.39449139255496e-10\\
0.8557856	-3.63017318408178e-10\\
0.8558856	-2.22321167091212e-10\\
0.8559856	-3.38287011250604e-11\\
0.8560856	1.86817057971304e-10\\
0.8561856	3.24792707621447e-10\\
0.8562856	5.66089130242921e-10\\
0.8563856	6.97506182017115e-10\\
0.8564856	9.06647416675345e-10\\
0.8565857	-7.06156603189675e-10\\
0.8566857	-5.61167621648383e-10\\
0.8567857	-3.70983539623847e-10\\
0.8568857	-2.44837387939744e-10\\
0.8569857	8.81557186190735e-12\\
0.8570857	1.82292748360539e-10\\
0.8571857	2.68679157427228e-10\\
0.8572857	4.61822393071605e-10\\
0.8573857	6.56327610911844e-10\\
0.8574857	8.4755256147249e-10\\
0.8575858	-7.17294647452609e-10\\
0.8576858	-5.30233027786191e-10\\
0.8577858	-3.56014240115131e-10\\
0.8578858	-1.96318612516629e-10\\
0.8579858	-5.20983295295829e-11\\
0.8580858	1.76417768658595e-10\\
0.8581858	2.89719407481608e-10\\
0.8582858	4.89010135227945e-10\\
0.8583858	5.76202606244677e-10\\
0.8584858	7.53913888864579e-10\\
0.8585859	-5.94546587228415e-10\\
0.8586859	-5.12802061149739e-10\\
0.8587859	-3.28602459424756e-10\\
0.8588859	-1.36558709354579e-10\\
0.8589859	-3.06003616272043e-11\\
0.8590859	9.60198875163236e-11\\
0.8591859	2.50721696930332e-10\\
0.8592859	4.41592610026485e-10\\
0.8593859	5.77383614199954e-10\\
0.8594859	6.67504718084586e-10\\
0.859586	-5.7866680205714e-10\\
0.859686	-4.37610668649379e-10\\
0.859786	-3.10172363189804e-10\\
0.859886	-1.84348786209194e-10\\
0.859986	-4.74995154157738e-11\\
0.860086	1.13648938187844e-10\\
0.860186	2.12998896677915e-10\\
0.860286	3.65077300936703e-10\\
0.860386	4.85031531513016e-10\\
0.860486	5.88625246824735e-10\\
0.8605861	-4.98034547630199e-10\\
0.8606861	-3.66850763986295e-10\\
0.8607861	-3.01162171589819e-10\\
0.8608861	-1.82780486492756e-10\\
0.8609861	7.07842898878264e-12\\
0.8610861	8.77905566981901e-11\\
0.8611861	1.79319739043509e-10\\
0.8612861	3.02213713066326e-10\\
0.8613861	4.77600180676671e-10\\
0.8614861	6.27182885677311e-10\\
0.8615862	-5.14888833419479e-10\\
0.8616862	-3.39736765548335e-10\\
0.8617862	-2.21936031808171e-10\\
0.8618862	-1.3751680173822e-10\\
0.8619862	-6.19524108120932e-11\\
0.8620862	1.29836859606757e-10\\
0.8621862	1.63480039808107e-10\\
0.8622862	3.65151742474706e-10\\
0.8623862	4.61568460481589e-10\\
0.8624862	5.79984885909924e-10\\
0.8625863	-4.45486047654991e-10\\
0.8626863	-3.90129193327763e-10\\
0.8627863	-2.2788671871863e-10\\
0.8628863	-1.29386968587136e-10\\
0.8629863	3.52618409596222e-11\\
0.8630863	9.64680761894555e-11\\
0.8631863	1.85153161858941e-10\\
0.8632863	3.32748076610338e-10\\
0.8633863	3.71189871455893e-10\\
0.8634863	5.32918213966044e-10\\
0.8635864	-4.55500024688655e-10\\
0.8636864	-3.39529802141419e-10\\
0.8637864	-2.00038462479083e-10\\
0.8638864	-1.02609573425895e-10\\
0.8639864	-1.23410689897548e-11\\
0.8640864	1.06151409941268e-10\\
0.8641864	1.88731208246762e-10\\
0.8642864	2.71737362985054e-10\\
0.8643864	3.91981333052989e-10\\
0.8644864	4.8674374838846e-10\\
0.8645865	-4.31932421809098e-10\\
0.8646865	-2.66710973361524e-10\\
0.8647865	-2.12407471575663e-10\\
0.8648865	-1.29896250738093e-10\\
0.8649865	2.04015698513629e-11\\
0.8650865	7.8514957312396e-11\\
0.8651865	1.8491985307107e-10\\
0.8652865	2.50536082263791e-10\\
0.8653865	3.36724280335337e-10\\
0.8654865	4.15282842077154e-10\\
0.8655866	-3.52750216108541e-10\\
0.8656866	-2.65191138932317e-10\\
0.8657866	-1.77326893742295e-10\\
0.8658866	-9.56377709367346e-11\\
0.8659866	-6.18128386475035e-12\\
0.8660866	7.54048961129367e-11\\
0.8661866	1.53900014328062e-10\\
0.8662866	2.34497333457075e-10\\
0.8663866	3.02801250328007e-10\\
0.8664866	3.84824429117939e-10\\
0.8665867	-3.2541727152573e-10\\
0.8666867	-2.39718243180503e-10\\
0.8667867	-1.59893310411005e-10\\
0.8668867	-8.83246304464154e-11\\
0.8669867	-7.00015102849706e-12\\
0.8670867	6.24836317940098e-11\\
0.8671867	1.38918934718796e-10\\
0.8672867	2.1148396280187e-10\\
0.8673867	2.79740205853156e-10\\
0.8674867	3.53629748300519e-10\\
0.8675868	-2.95437995695793e-10\\
0.8676868	-2.22874452766408e-10\\
0.8677868	-1.52643147037154e-10\\
0.8678868	-8.33057785080971e-11\\
0.8679868	-1.30566373413975e-11\\
0.8680868	6.0274807633107e-11\\
0.8681868	1.29221343152762e-10\\
0.8682868	1.96675454012542e-10\\
0.8683868	2.55886788009239e-10\\
0.8684868	3.2045963696244e-10\\
0.8685869	-2.75983469272809e-10\\
0.8686869	-2.02867781885882e-10\\
0.8687869	-1.41519985658099e-10\\
0.8688869	-7.69417566633254e-11\\
0.8689869	-3.7924908439374e-12\\
0.8690869	5.36082707901491e-11\\
0.8691869	1.21278434072735e-10\\
0.8692869	1.75570958338704e-10\\
0.8693869	2.4317147891118e-10\\
0.8694869	3.01095944075895e-10\\
0.869587	-2.49623848528517e-10\\
0.869687	-1.83529851675835e-10\\
0.869787	-1.2414741925996e-10\\
0.869887	-6.31623443055496e-11\\
0.869987	-1.19416907325099e-11\\
0.870087	4.84639339293561e-11\\
0.870187	1.07318124913515e-10\\
0.870287	1.6419643424049e-10\\
0.870387	2.18984145825885e-10\\
0.870487	2.71874062135859e-10\\
0.8705871	-2.33145604439187e-10\\
0.8706871	-1.77493890394262e-10\\
0.8707871	-1.21377109643488e-10\\
0.8708871	-6.33939277008594e-11\\
0.8709871	-1.18463407892911e-11\\
0.8710871	4.52581960759058e-11\\
0.8711871	1.00204655323669e-10\\
0.8712871	1.45568338036795e-10\\
0.8713871	2.04212788111689e-10\\
0.8714871	2.49287721142014e-10\\
0.8715872	-2.06388079769645e-10\\
0.8716872	-1.5348311335116e-10\\
0.8717872	-1.03037868046259e-10\\
0.8718872	-6.07781125232763e-11\\
0.8719872	-2.15360138034436e-12\\
0.8720872	3.76599334128615e-11\\
0.8721872	9.37588040449892e-11\\
0.8722872	1.41509404237919e-10\\
0.8723872	1.86546254321346e-10\\
0.8724872	2.34770065015008e-10\\
0.8725873	-1.9599112313824e-10\\
0.8726873	-1.48597057124118e-10\\
0.8727873	-9.87691402325403e-11\\
0.8728873	-4.95607789765986e-11\\
0.8729873	-3.76869419189702e-12\\
0.8730873	3.60652155451742e-11\\
0.8731873	8.76520239649994e-11\\
0.8732873	1.28953938727337e-10\\
0.8733873	1.68182476029555e-10\\
0.8734873	2.13796646971473e-10\\
0.8735874	-1.74904098142355e-10\\
0.8736874	-1.36442062760123e-10\\
0.8737874	-8.47810329698741e-11\\
0.8738874	-5.04896716002254e-11\\
0.8739874	-3.89803574347199e-12\\
0.8740874	3.49006819876986e-11\\
0.8741874	7.60484232485421e-11\\
0.8742874	1.1992054511277e-10\\
0.8743874	1.57124114520885e-10\\
0.8744874	1.98496210448789e-10\\
0.8745875	-1.58466404827026e-10\\
0.8746875	-1.21912139807107e-10\\
0.8747875	-7.73431847199738e-11\\
0.8748875	-4.30189107806286e-11\\
0.8749875	-6.97698616139138e-12\\
0.8750875	3.29649978850015e-11\\
0.8751875	6.92079092602313e-11\\
0.8752875	1.04369469196405e-10\\
0.8753875	1.41282657102582e-10\\
0.8754875	1.72994126056809e-10\\
0.8755876	-1.47830853820515e-10\\
0.8756876	-1.13387707568731e-10\\
0.8757876	-7.37661779668395e-11\\
0.8758876	-3.50803397425716e-11\\
0.8759876	-3.23835680086969e-12\\
0.8760876	2.60559963972192e-11\\
0.8761876	6.73018242630902e-11\\
0.8762876	9.51996100291672e-11\\
0.8763876	1.34649726705159e-10\\
0.8764876	1.60750955637543e-10\\
0.8765877	-1.31463692959032e-10\\
0.8766877	-1.03082361175872e-10\\
0.8767877	-7.16019832468673e-11\\
0.8768877	-3.1144546251867e-11\\
0.8769877	-5.64087755590978e-12\\
0.8770877	3.11679332652215e-11\\
0.8771877	5.57291259445153e-11\\
0.8772877	8.46768679547733e-11\\
0.8773877	1.14830863662762e-10\\
0.8774877	1.53194972861231e-10\\
0.8775878	-1.2541892496071e-10\\
0.8776878	-9.62317406547086e-11\\
0.8777878	-6.67539662462706e-11\\
0.8778878	-2.92585679601974e-11\\
0.8779878	-5.84112437102468e-12\\
0.8780878	2.15788484560979e-11\\
0.8781878	5.12565812141102e-11\\
0.8782878	8.16207433061776e-11\\
0.8783878	1.11272145370786e-10\\
0.8784878	1.38982450788777e-10\\
0.8785879	-1.13049644627239e-10\\
0.8786879	-8.97838183669202e-11\\
0.8787879	-6.115238589287e-11\\
0.8788879	-2.77128246769349e-11\\
0.8789879	1.41926973781315e-13\\
0.8790879	2.21822312376622e-11\\
0.8791879	4.8340524850854e-11\\
0.8792879	7.87100989176363e-11\\
0.8793879	1.03543888740898e-10\\
0.8794879	1.232532711655e-10\\
0.879588	-1.04785563774225e-10\\
0.879688	-8.12164872030299e-11\\
0.879788	-5.06179788025991e-11\\
0.879888	-3.19565986018721e-11\\
0.879988	-4.04635278462937e-12\\
0.880088	2.44501530736422e-11\\
0.880188	4.50205693851204e-11\\
0.880288	6.93016657955629e-11\\
0.880388	8.90782026185264e-11\\
0.880488	1.16281809563055e-10\\
0.8805881	-9.85735680601241e-11\\
0.8806881	-6.80761322828646e-11\\
0.8807881	-4.35041719953845e-11\\
0.8808881	-2.23494886707448e-11\\
0.8809881	-1.96250221951608e-12\\
0.8810881	2.04466746742886e-11\\
0.8811881	3.78071689962332e-11\\
0.8812881	6.31862889681272e-11\\
0.8813881	7.97884716941823e-11\\
0.8814881	1.0095423898365e-10\\
0.8815882	-9.15471391388692e-11\\
0.8816882	-6.87999866428343e-11\\
0.8817882	-4.06777009164417e-11\\
0.8818882	-2.3305587287906e-11\\
0.8819882	-2.6779794947882e-12\\
0.8820882	1.53407592754168e-11\\
0.8821882	3.50152424755308e-11\\
0.8822882	6.07380739297691e-11\\
0.8823882	7.70288678524913e-11\\
0.8824882	9.85332762986333e-11\\
0.8825883	-8.34606688705146e-11\\
0.8826883	-5.53553861074225e-11\\
0.8827883	-3.73678503369284e-11\\
0.8828883	-2.43577262275718e-11\\
0.8829883	-1.06340202485691e-12\\
0.8830883	1.78970780083637e-11\\
0.8831883	3.80250892046145e-11\\
0.8832883	5.49405068761923e-11\\
0.8833883	7.43807949294426e-11\\
0.8834883	8.22001002929605e-11\\
0.8835884	-7.23961516041103e-11\\
0.8836884	-5.82006210600125e-11\\
0.8837884	-3.73862378738371e-11\\
0.8838884	-2.36409370371778e-11\\
0.8839884	-5.4040263949384e-13\\
0.8840884	1.84510647083561e-11\\
0.8841884	2.997968703394e-11\\
0.8842884	5.08013541071506e-11\\
0.8843884	6.77807746995301e-11\\
0.8844884	7.78906341111589e-11\\
0.8845885	-6.32223636337886e-11\\
0.8846885	-5.40448229653092e-11\\
0.8847885	-3.01919195922634e-11\\
0.8848885	-1.42672628709363e-11\\
0.8849885	1.22939065001835e-12\\
0.8850885	1.39013309056243e-11\\
0.8851885	3.14540931346222e-11\\
0.8852885	4.16946630250414e-11\\
0.8853885	6.2530687127137e-11\\
0.8854885	7.19696891905502e-11\\
0.8855886	-5.92604715943061e-11\\
0.8856886	-4.68582736292702e-11\\
0.8857886	-3.12506627258462e-11\\
0.8858886	-1.40382328699985e-11\\
0.8859886	3.27446463295315e-12\\
0.8860886	9.27817153008738e-12\\
0.8861886	2.26581773363297e-11\\
0.8862886	4.2193580594558e-11\\
0.8863886	5.67565546935546e-11\\
0.8864886	6.53116196379106e-11\\
0.8865887	-5.75849630685062e-11\\
0.8866887	-4.25595876103773e-11\\
0.8867887	-2.61124057387996e-11\\
0.8868887	-1.8916601442952e-11\\
0.8869887	-1.55657642299993e-12\\
0.8870887	1.54713529483639e-11\\
0.8871887	2.17582632632869e-11\\
0.8872887	3.69819378689718e-11\\
0.8873887	5.09061840583616e-11\\
0.8874887	6.33801567493172e-11\\
0.8875888	-4.83648843664522e-11\\
0.8876888	-3.77824258204496e-11\\
0.8877888	-2.86072435322528e-11\\
0.8878888	-1.06553804221952e-11\\
0.8879888	-3.66084362632836e-12\\
0.8880888	1.27237484498423e-11\\
0.8881888	1.89265231057814e-11\\
0.8882888	3.54557153222582e-11\\
0.8883888	4.28990343783939e-11\\
0.8884888	5.19230351480471e-11\\
0.8885889	-4.86275428306988e-11\\
0.8886889	-3.31016669136212e-11\\
0.8887889	-2.35376506578234e-11\\
0.8888889	-8.95967518350861e-12\\
0.8889889	1.68384909660123e-12\\
0.8890889	9.51968098940976e-12\\
0.8891889	1.57491517148846e-11\\
0.8892889	3.16475726121922e-11\\
0.8893889	3.85636472790939e-11\\
0.8894889	4.79188874876331e-11\\
0.889589	-4.08050558792938e-11\\
0.889689	-3.10773070011329e-11\\
0.889789	-2.42231061784278e-11\\
0.889889	-8.53538754736936e-12\\
0.889989	-2.23713127185763e-12\\
0.890089	6.51807860479635e-12\\
0.890189	1.9645500305945e-11\\
0.890289	2.81286845080437e-11\\
0.890389	3.70189286753045e-11\\
0.890489	4.54347356382257e-11\\
0.8905891	-3.84033981881465e-11\\
0.8906891	-2.84537142145622e-11\\
0.8907891	-1.9232733105629e-11\\
0.8908891	-1.03584013772423e-11\\
0.8909891	-1.38410837018056e-12\\
0.8910891	8.2007959586539e-12\\
0.8911891	1.69704872967209e-11\\
0.8912891	2.55621569952033e-11\\
0.8913891	3.36755059245981e-11\\
0.8914891	4.20722425505755e-11\\
0.8915892	-3.51138410255774e-11\\
0.8916892	-2.58322145497422e-11\\
0.8917892	-1.76220235828523e-11\\
0.8918892	-9.47841884325765e-12\\
0.8919892	-1.33699925554344e-12\\
0.8920892	6.92570742542745e-12\\
0.8921892	1.54917685803039e-11\\
0.8922892	2.26013739550222e-11\\
0.8923892	3.05523662704648e-11\\
0.8924892	3.76997757589993e-11\\
0.8925893	-3.16679030122889e-11\\
0.8926893	-2.41163006583925e-11\\
0.8927893	-1.59711185266097e-11\\
0.8928893	-8.65310819484909e-12\\
0.8929893	-5.28110244947468e-13\\
0.8930893	6.09250085446919e-12\\
0.8931893	1.3951375229134e-11\\
0.8932893	2.08447495138461e-11\\
0.8933893	2.76220118758624e-11\\
0.8934893	3.41852709232978e-11\\
0.8935894	-2.87190710708218e-11\\
0.8936894	-2.20109868764011e-11\\
0.8937894	-1.45044426417277e-11\\
0.8938894	-7.09064825304544e-12\\
0.8939894	-6.10202726631444e-13\\
0.8940894	6.14649404049678e-12\\
0.8941894	1.24388317842509e-11\\
0.8942894	1.85755843275202e-11\\
0.8943894	2.49145067725672e-11\\
0.8944894	3.18619357590747e-11\\
0.8945895	-2.60175998685597e-11\\
0.8946895	-1.98409719109863e-11\\
0.8947895	-1.35547156274713e-11\\
0.8948895	-6.56205714749382e-12\\
0.8949895	-2.19594215688623e-13\\
0.8950895	5.16232202062936e-12\\
0.8951895	1.13192083566369e-11\\
0.8952895	1.70320737968486e-11\\
0.8953895	2.31270466204126e-11\\
0.8954895	2.84750043063762e-11\\
0.8955896	-2.41286014538117e-11\\
0.8956896	-1.79362581085679e-11\\
0.8957896	-1.16184347519465e-11\\
0.8958896	-6.12885725938245e-12\\
0.8959896	-3.78310493791745e-13\\
0.8960896	4.76500880006376e-12\\
0.8961896	1.04751410817951e-11\\
0.8962896	1.59680164801986e-11\\
0.8963896	2.05011096320439e-11\\
0.8964896	2.63730973120745e-11\\
0.8965897	-2.19283835499936e-11\\
0.8966897	-1.58187046047739e-11\\
0.8967897	-1.12348324147346e-11\\
0.8968897	-5.71661353898043e-12\\
0.8969897	-7.64365429301993e-13\\
0.8970897	4.16079680034913e-12\\
0.8971897	9.63663605886561e-12\\
0.8972897	1.42794716472371e-11\\
0.8973897	1.87438599560388e-11\\
0.8974897	2.37222777669567e-11\\
0.8975898	-2.00995318428403e-11\\
0.8976898	-1.44086064439989e-11\\
0.8977898	-9.90600635712908e-12\\
0.8978898	-4.75065496534003e-12\\
0.8979898	-6.51024585057573e-14\\
0.8980898	4.0641718125056e-12\\
0.8981898	8.58645969453855e-12\\
0.8982898	1.34865264987436e-11\\
0.8983898	1.77843160738692e-11\\
0.8984898	2.15346578680194e-11\\
0.8985899	-1.78315459559229e-11\\
0.8986899	-1.34567824158755e-11\\
0.8987899	-9.26238405144284e-12\\
0.8988899	-5.05682812687776e-12\\
0.8989899	-6.151366935492e-13\\
0.8990899	3.32084414051456e-12\\
0.8991899	8.04216701981203e-12\\
0.8992899	1.18725084398638e-11\\
0.8993899	1.61678959422065e-11\\
0.8994899	1.93164375367514e-11\\
0.89959	-1.5920883315755e-11\\
0.89969	-1.23947692091088e-11\\
0.89979	-8.66487048991905e-12\\
0.89989	-4.21739039785357e-12\\
0.89999	-5.07773470927837e-13\\
0.90009	3.03903641237074e-12\\
0.90019	7.02833993219246e-12\\
0.90029	1.10954285466976e-11\\
0.90039	1.39053322024462e-11\\
0.90049	1.81525695349906e-11\\
0.9005901	-1.44520221286442e-11\\
0.9006901	-1.07835468006827e-11\\
0.9007901	-7.42292442002486e-12\\
0.9008901	-3.56011390750506e-12\\
0.9009901	-3.56806311435637e-13\\
0.9010901	3.05333690392407e-12\\
0.9011901	6.56444739273229e-12\\
0.9012901	1.00982153822635e-11\\
0.9013901	1.26036569867894e-11\\
0.9014901	1.60568834386446e-11\\
0.9015902	-1.32295824834069e-11\\
0.9016902	-1.05297130183305e-11\\
0.9017902	-6.79636620800697e-12\\
0.9018902	-3.94729616382349e-12\\
0.9019902	1.25711082389195e-13\\
0.9020902	2.55661753144229e-12\\
0.9021902	5.50491498791996e-12\\
0.9022902	9.15540867927879e-12\\
0.9023902	1.17180023140376e-11\\
0.9024902	1.44274851808194e-11\\
0.9025903	-1.21206232421034e-11\\
0.9026903	-9.02716782330185e-12\\
0.9027903	-5.93792666446342e-12\\
0.9028903	-3.52064064708019e-12\\
0.9029903	-4.19205315212433e-13\\
0.9030903	2.74612642856968e-12\\
0.9031903	5.37854281876181e-12\\
0.9032903	7.90447356286439e-12\\
0.9033903	1.07733917044668e-11\\
0.9034903	1.34576170796006e-11\\
0.9035904	-1.14559269527776e-11\\
0.9036904	-8.37212087020722e-12\\
0.9037904	-5.92335802176075e-12\\
0.9038904	-2.54784316288146e-12\\
0.9039904	-6.61893553865095e-13\\
0.9040904	2.33987382980183e-12\\
0.9041904	5.0843565994077e-12\\
0.9042904	7.2197827279347e-12\\
0.9043904	9.4155280038708e-12\\
0.9044904	1.23619345910074e-11\\
0.9045905	-9.62936767168632e-12\\
0.9046905	-7.7895781340644e-12\\
0.9047905	-5.00632973526307e-12\\
0.9048905	-2.50717713310664e-12\\
0.9049905	-4.99593063468963e-13\\
0.9050905	1.82885917925451e-12\\
0.9051905	4.31035456520325e-12\\
0.9052905	6.79663652756066e-12\\
0.9053905	9.15884855122836e-12\\
0.9054905	1.12873671747749e-11\\
0.9055906	-9.02530490816557e-12\\
0.9056906	-6.40036488350502e-12\\
0.9057906	-4.22628404178239e-12\\
0.9058906	-2.53737850238142e-12\\
0.9059906	-3.4954645938384e-13\\
0.9060906	2.33957276549326e-12\\
0.9061906	3.55044102418148e-12\\
0.9062906	6.32146489059859e-12\\
0.9063906	7.70884048030901e-12\\
0.9064906	9.78639967644284e-12\\
0.9065907	-8.39532579138339e-12\\
0.9066907	-6.44917383106047e-12\\
0.9067907	-4.48891434006587e-12\\
0.9068907	-2.37166963653235e-12\\
0.9069907	6.23228317979158e-14\\
0.9070907	1.98956392077615e-12\\
0.9071907	3.60314754280999e-12\\
0.9072907	5.11261663375364e-12\\
0.9073907	6.74382023303868e-12\\
0.9074907	8.738771768921e-12\\
0.9075908	-7.79755058095868e-12\\
0.9076908	-5.10606980069607e-12\\
0.9077908	-3.23091660279793e-12\\
0.9078908	-1.86679838974356e-12\\
0.9079908	3.0705046108977e-13\\
0.9080908	1.62673370964886e-12\\
0.9081908	3.44355955408997e-12\\
0.9082908	5.12390798019899e-12\\
0.9083908	7.04909920370509e-12\\
0.9084908	8.61526330973729e-12\\
0.9085909	-7.20408971633367e-12\\
0.9086909	-4.94630238541035e-12\\
0.9087909	-3.77313454888246e-12\\
0.9088909	-2.23048568894621e-12\\
0.9089909	1.49918155773837e-13\\
0.9090909	1.85040038182086e-12\\
0.9091909	3.36721064463792e-12\\
0.9092909	4.21040287727047e-12\\
0.9093909	5.90371425748779e-12\\
0.9094909	6.98444517651182e-12\\
0.909591	-5.87507731804428e-12\\
0.909691	-4.20615946603319e-12\\
0.909791	-3.4591430926118e-12\\
0.909891	-2.04364229184844e-12\\
0.909991	-3.56293387458054e-13\\
0.910091	1.21913095920957e-12\\
0.910191	2.3116085394047e-12\\
0.910291	3.56275532998789e-12\\
0.910391	5.62671420894325e-12\\
0.910491	7.17004452424974e-12\\
0.9105911	-5.59097648896489e-12\\
0.9106911	-3.90590733605907e-12\\
0.9107911	-2.6696705221275e-12\\
0.9108911	-1.16711426802496e-12\\
0.9109911	3.28791367131779e-13\\
0.9110911	1.55684955258542e-12\\
0.9111911	2.26753256653071e-12\\
0.9112911	3.2228786357876e-12\\
0.9113911	5.19638956210112e-12\\
0.9114911	5.97292927292246e-12\\
0.9115912	-4.82853883655887e-12\\
0.9116912	-3.92459038328127e-12\\
0.9117912	-2.79702682350344e-12\\
0.9118912	-1.61652265780125e-12\\
0.9119912	-5.4288510607972e-13\\
0.9120912	1.27484945394192e-12\\
0.9121912	1.69831982758916e-12\\
0.9122912	3.59974515863284e-12\\
0.9123912	4.86183089607401e-12\\
0.9124912	5.37767554045657e-12\\
0.9125913	-4.95987767064764e-12\\
0.9126913	-3.1055785715064e-12\\
0.9127913	-2.25785865533264e-12\\
0.9128913	-1.48295401900795e-12\\
0.9129913	1.62837653280292e-13\\
0.9130913	6.33068469557517e-13\\
0.9131913	1.9910525170182e-12\\
0.9132913	3.00977867295107e-12\\
0.9133913	3.97182426417745e-12\\
0.9134913	4.96926941399515e-12\\
0.9135914	-4.0485652103354e-12\\
0.9136914	-3.06623752360723e-12\\
0.9137914	-2.01707405525454e-12\\
0.9138914	-1.07181639742714e-12\\
0.9139914	-9.2120880299753e-14\\
0.9140914	8.69360065211467e-13\\
0.9141914	1.76889721367493e-12\\
0.9142914	2.67160456876135e-12\\
0.9143914	3.55136005027057e-12\\
0.9144914	4.49072685187135e-12\\
0.9145915	-3.7114641136131e-12\\
0.9146915	-2.77993040176386e-12\\
0.9147915	-1.79064805202528e-12\\
0.9148915	-9.27077408621539e-13\\
0.9149915	-6.4376827296822e-14\\
0.9150915	8.30522642773946e-13\\
0.9151915	1.59884319014391e-12\\
0.9152915	2.48988687782972e-12\\
0.9153915	3.26096278610765e-12\\
0.9154915	4.07731489981683e-12\\
0.9155916	-3.31013781113851e-12\\
0.9156916	-2.49372700355181e-12\\
0.9157916	-1.69021862671058e-12\\
0.9158916	-9.03330869909055e-13\\
0.9159916	-2.91987199638538e-14\\
0.9160916	7.43557492146661e-13\\
0.9161916	1.43376415618332e-12\\
0.9162916	2.16762704009361e-12\\
0.9163916	2.97866451418491e-12\\
0.9164916	3.60764135519769e-12\\
0.9165917	-3.0311304693154e-12\\
0.9166917	-2.24067203961901e-12\\
0.9167917	-1.46788929017802e-12\\
0.9168917	-7.43675493577092e-13\\
0.9169917	-9.20000955190569e-14\\
0.9170917	6.70028438808768e-13\\
0.9171917	1.33210031735558e-12\\
0.9172917	1.99064268753821e-12\\
0.9173917	2.64875837112053e-12\\
0.9174917	3.31616519319436e-12\\
0.9175918	-2.71015075691941e-12\\
0.9176918	-2.00123701618286e-12\\
0.9177918	-1.31545541552826e-12\\
0.9178918	-7.17218308508288e-13\\
0.9179918	-6.46188287391382e-14\\
0.9180918	5.90511545030584e-13\\
0.9181918	1.20254586129646e-12\\
0.9182918	1.83200509667831e-12\\
0.9183918	2.34550203296997e-12\\
0.9184918	3.01568561750973e-12\\
0.9185919	-2.45121589186062e-12\\
0.9186919	-1.86462889802004e-12\\
0.9187919	-1.16834708117513e-12\\
0.9188919	-6.66117696373382e-13\\
0.9189919	-5.59228440778534e-14\\
0.9190919	5.69967722489085e-13\\
0.9191919	1.12494429779448e-12\\
0.9192919	1.62800532400192e-12\\
0.9193919	2.20370602636417e-12\\
0.9194919	2.68210738618984e-12\\
0.919592	-2.18809757321523e-12\\
0.919692	-1.63693982085917e-12\\
0.919792	-1.0609177782313e-12\\
0.919892	-6.08442820865945e-13\\
0.919992	-2.26688277769896e-14\\
0.920092	4.5845950718852e-13\\
0.920192	1.00215873633851e-12\\
0.920292	1.48075914821857e-12\\
0.920392	1.97165770091652e-12\\
0.920492	2.45727133251812e-12\\
0.9205921	-1.93194496627801e-12\\
0.9206921	-1.43967934083731e-12\\
0.9207921	-9.76288849186986e-13\\
0.9208921	-5.39730360507973e-13\\
0.9209921	-2.31680782605241e-14\\
0.9210921	3.84982192105353e-13\\
0.9211921	9.01009178329257e-13\\
0.9212921	1.34586264779642e-12\\
0.9213921	1.74511030584864e-12\\
0.9214921	2.12889508956341e-12\\
0.9215922	-1.74572580292209e-12\\
0.9216922	-1.33901925241039e-12\\
0.9217922	-9.3077124053045e-13\\
0.9218922	-4.72953310401016e-13\\
0.9219922	-1.31698291472303e-14\\
0.9220922	4.05301479129788e-13\\
0.9221922	7.4346940024369e-13\\
0.9222922	1.16658943677584e-12\\
0.9223922	1.54412436978396e-12\\
0.9224922	1.94970516268679e-12\\
0.9225923	-1.58311261141406e-12\\
0.9226923	-1.24308808921217e-12\\
0.9227923	-8.29913476001219e-13\\
0.9228923	-4.53665072908137e-13\\
0.9229923	-2.04413295961182e-14\\
0.9230923	3.67600090610739e-13\\
0.9231923	7.12205592566849e-13\\
0.9232923	1.11898925732865e-12\\
0.9233923	1.39739675429425e-12\\
0.9234923	1.76066962055167e-12\\
0.9235924	-1.42663225369922e-12\\
0.9236924	-1.10186353153134e-12\\
0.9237924	-7.30456273303433e-13\\
0.9238924	-3.84333095723037e-13\\
0.9239924	-3.17948426482055e-14\\
0.9240924	2.62445566750439e-13\\
0.9241924	6.37229526375381e-13\\
0.9242924	9.34919485022076e-13\\
0.9243924	1.30136596779239e-12\\
0.9244924	1.58587489072832e-12\\
0.9245925	-1.25728223339669e-12\\
0.9246925	-9.41614868917988e-13\\
0.9247925	-6.398962933016e-13\\
0.9248925	-2.89316741395675e-13\\
0.9249925	-2.37701642613277e-14\\
0.9250925	3.26114798340294e-13\\
0.9251925	5.32944138416517e-13\\
0.9252925	8.72528065737108e-13\\
0.9253925	1.12385086936856e-12\\
0.9254925	1.46904105074547e-12\\
0.9255926	-1.18538336897529e-12\\
0.9256926	-8.63431309977928e-13\\
0.9257926	-5.82962824674104e-13\\
0.9258926	-3.49562871123649e-13\\
0.9259926	-6.58175750182622e-14\\
0.9260926	2.686574903922e-13\\
0.9261926	5.57189259356479e-13\\
0.9262926	8.06019397543032e-13\\
0.9263926	1.02427677469468e-12\\
0.9264926	1.32395017814262e-12\\
0.9265927	-1.07017533671041e-12\\
0.9266927	-8.33125289994271e-13\\
0.9267927	-5.62080664430353e-13\\
0.9268927	-2.33878162835607e-13\\
0.9269927	-2.26273464030406e-14\\
0.9270927	2.00263536921708e-13\\
0.9271927	4.66061947195078e-13\\
0.9272927	7.08685690376856e-13\\
0.9273927	9.64677776364937e-13\\
0.9274927	1.17318151396594e-12\\
0.9275928	-9.53557409819796e-13\\
0.9276928	-6.8771109006591e-13\\
0.9277928	-4.36819066427366e-13\\
0.9278928	-2.51578949994592e-13\\
0.9279928	1.97907318214337e-14\\
0.9280928	2.3152694583132e-13\\
0.9281928	4.40298801706194e-13\\
0.9282928	6.05184379429683e-13\\
0.9283928	8.87647798848463e-13\\
0.9284928	1.05151648358145e-12\\
0.9285929	-8.31422347591522e-13\\
0.9286929	-6.81719127142735e-13\\
0.9287929	-4.45400871523816e-13\\
0.9288929	-2.49407901292169e-13\\
0.9289929	-1.84278493242055e-14\\
0.9290929	2.25082820574335e-13\\
0.9291929	3.60877496099034e-13\\
0.9292929	5.70898290328523e-13\\
0.9293929	7.39255112487288e-13\\
0.9294929	9.52204928198946e-13\\
0.929593	-7.84216330426472e-13\\
0.929693	-5.94804669955035e-13\\
0.929793	-3.89557734066863e-13\\
0.929893	-1.73831898604821e-13\\
0.929993	-5.09373878007087e-14\\
0.930093	1.77842106292787e-13\\
0.930193	3.13229976721574e-13\\
0.930293	5.57937445067567e-13\\
0.930393	7.16644461712443e-13\\
0.930493	7.95980797262439e-13\\
0.9305931	-6.86673830988757e-13\\
0.9306931	-5.20438330610361e-13\\
0.9307931	-3.02358471870938e-13\\
0.9308931	-2.1818899618182e-13\\
0.9309931	-5.1826822449679e-14\\
0.9310931	1.146710642777e-13\\
0.9311931	3.01069880780628e-13\\
0.9312931	4.28939491719435e-13\\
0.9313931	6.21637001919242e-13\\
0.9314931	8.04289522377767e-13\\
0.9315932	-6.15114106012763e-13\\
0.9316932	-4.53915204490318e-13\\
0.9317932	-3.17098751215522e-13\\
0.9318932	-1.72625144635032e-13\\
0.9319932	1.32313690025241e-14\\
0.9320932	1.75866396649361e-13\\
0.9321932	2.5232924217074e-13\\
0.9322932	3.81306912309098e-13\\
0.9323932	5.03108249141278e-13\\
0.9324932	6.5964823220519e-13\\
0.9325933	-5.6923808711718e-13\\
0.9326933	-3.96482686152565e-13\\
0.9327933	-2.53914367160965e-13\\
0.9328933	-9.33462472626922e-14\\
0.9329933	3.49383021941447e-14\\
0.9330933	8.21708019104648e-14\\
0.9331933	2.01082957653673e-13\\
0.9332933	3.45892080112324e-13\\
0.9333933	4.72286644477496e-13\\
0.9334933	6.37411987132213e-13\\
0.9335934	-5.24022914314084e-13\\
0.9336934	-3.91055543115707e-13\\
0.9337934	-2.39459147900638e-13\\
0.9338934	-1.06399960673355e-13\\
0.9339934	-2.7656917506037e-14\\
0.9340934	6.23648200681441e-14\\
0.9341934	2.30620468800041e-13\\
0.9342934	3.45412351845634e-13\\
0.9343934	4.76376756740609e-13\\
0.9344934	5.44470904699366e-13\\
0.9345935	-4.4606904605272e-13\\
0.9346935	-3.33754279375569e-13\\
0.9347935	-2.23765409624747e-13\\
0.9348935	-1.09988008516839e-13\\
0.9349935	-5.05003821156383e-15\\
0.9350935	9.96658512384228e-14\\
0.9351935	2.04010124821901e-13\\
0.9352935	2.99054285851258e-13\\
0.9353935	3.97078913702929e-13\\
0.9354935	4.91561810293235e-13\\
0.9355936	-3.97605357054269e-13\\
0.9356936	-2.94362002844342e-13\\
0.9357936	-1.97272903348216e-13\\
0.9358936	-9.81865167325471e-14\\
0.9359936	-7.81169342136344e-15\\
0.9360936	8.42711260881794e-14\\
0.9361936	1.79598820742543e-13\\
0.9362936	2.70814600399853e-13\\
0.9363936	3.5165712404713e-13\\
0.9364936	4.36949720809493e-13\\
0.9365937	-3.60294084805672e-13\\
0.9366937	-2.67739889038998e-13\\
0.9367937	-1.7996537336373e-13\\
0.9368937	-8.7913279114081e-14\\
0.9369937	-1.49408138919208e-15\\
0.9370937	8.04038257889018e-14\\
0.9371937	1.59904046226445e-13\\
0.9372937	2.40132189944078e-13\\
0.9373937	3.15205976698778e-13\\
0.9374937	3.90225434455839e-13\\
0.9375938	-3.19994489464471e-13\\
0.9376938	-2.37265982387104e-13\\
0.9377938	-1.63586038891155e-13\\
0.9378938	-8.00206330863523e-14\\
0.9379938	-6.70107170891792e-15\\
0.9380938	6.71667489622185e-14\\
0.9381938	1.43293174821281e-13\\
0.9382938	2.14295713041006e-13\\
0.9383938	2.83690036853813e-13\\
0.9384938	3.55881076568574e-13\\
0.9385939	-2.92735605009091e-13\\
0.9386939	-2.10460778770616e-13\\
0.9387939	-1.47009267497201e-13\\
0.9388939	-7.45056621428702e-14\\
0.9389939	-4.22859090033297e-15\\
0.9390939	6.33808719615445e-14\\
0.9391939	1.28710872018269e-13\\
0.9392939	1.92970526421849e-13\\
0.9393939	2.58181750846794e-13\\
0.9394939	3.17171166484167e-13\\
0.939594	-2.61309580735649e-13\\
0.939694	-1.96130860280231e-13\\
0.939794	-1.2402005285634e-13\\
0.939894	-6.90103458379229e-14\\
0.939994	-4.36955064263901e-15\\
0.940094	5.73922599536482e-14\\
0.940194	1.14515183097116e-13\\
0.940294	1.75981994901737e-13\\
0.940394	2.31510727307667e-13\\
0.940494	2.81547318293276e-13\\
0.9405941	-2.31122734416466e-13\\
0.9406941	-1.71593614637271e-13\\
0.9407941	-1.11990890917656e-13\\
0.9408941	-5.90284021750382e-14\\
0.9409941	1.27220797094918e-15\\
0.9410941	5.35745566958327e-14\\
0.9411941	1.03220650497628e-13\\
0.9412941	1.56224083959266e-13\\
0.9413941	2.0926330302106e-13\\
0.9414941	2.49674932061768e-13\\
0.9415942	-2.03227063753438e-13\\
0.9416942	-1.57837508664366e-13\\
0.9417942	-9.924276840596e-14\\
0.9418942	-4.75383167863185e-14\\
0.9419942	-2.1938620922543e-15\\
0.9420942	4.79403591019959e-14\\
0.9421942	9.46273482788767e-14\\
0.9422942	1.40237171526408e-13\\
0.9423942	1.87740846240136e-13\\
0.9424942	2.30704287620316e-13\\
0.9425943	-1.81796234893932e-13\\
0.9426943	-1.39818959987042e-13\\
0.9427943	-8.82288538051084e-14\\
0.9428943	-5.11392202275137e-14\\
0.9429943	-2.0978732018549e-15\\
0.9430943	3.59071529052441e-14\\
0.9431943	8.04419511190521e-14\\
0.9432943	1.19621138493095e-13\\
0.9433943	1.62102313167957e-13\\
0.9434943	2.07080563285349e-13\\
0.9435944	-1.62703006295871e-13\\
0.9436944	-1.18494764085413e-13\\
0.9437944	-8.1082323624494e-14\\
0.9438944	-3.91739754016475e-14\\
0.9439944	-9.67189461114104e-16\\
0.9440944	3.58462062906507e-14\\
0.9441944	7.40748997457481e-14\\
0.9442944	1.07023016337962e-13\\
0.9443944	1.48485092461026e-13\\
0.9444944	1.82741095712906e-13\\
0.9445945	-1.49295835651466e-13\\
0.9446945	-1.10634404028723e-13\\
0.9447945	-7.35196669489054e-14\\
0.9448945	-4.17782843020924e-14\\
0.9449945	1.22433974358478e-15\\
0.9450945	3.2579363293403e-14\\
0.9451945	6.9829858863392e-14\\
0.9452945	1.00966213555526e-13\\
0.9453945	1.34421569951912e-13\\
0.9454945	1.5906731591629e-13\\
0.9455946	-1.30957621431617e-13\\
0.9456946	-1.01946028548954e-13\\
0.9457946	-6.25844993137116e-14\\
0.9458946	-3.22924877329898e-14\\
0.9459946	-7.31101301386503e-17\\
0.9460946	2.54826022045093e-14\\
0.9461946	5.61914921675298e-14\\
0.9462946	8.42741085540524e-14\\
0.9463946	1.12350577430215e-13\\
0.9464946	1.43436514951162e-13\\
0.9465947	-1.19555769112511e-13\\
0.9466947	-8.8580060607967e-14\\
0.9467947	-6.32509953409508e-14\\
0.9468947	-2.90106912127825e-14\\
0.9469947	-9.25619701262321e-16\\
0.9470947	2.63095431629866e-14\\
0.9471947	4.83681037854247e-14\\
0.9472947	8.12875746398781e-14\\
0.9473947	1.01465938185605e-13\\
0.9474947	1.25657943452994e-13\\
0.9475948	-1.08457090399307e-13\\
0.9476948	-8.16423633531647e-14\\
0.9477948	-4.84776311806144e-14\\
0.9478948	-3.0817017503916e-14\\
0.9479948	-1.75846170667251e-16\\
0.9480948	2.22658736138009e-14\\
0.9481948	4.56599948089421e-14\\
0.9482948	6.94868150857909e-14\\
0.9483948	9.35516930063187e-14\\
0.9484948	1.17981697918791e-13\\
0.9485949	-9.83805178004329e-14\\
0.9486949	-6.8951717674929e-14\\
0.9487949	-4.69080942303615e-14\\
0.9488949	-2.08682682317448e-14\\
0.9489949	8.54589311633144e-16\\
0.9490949	2.02496025635416e-14\\
0.9491949	3.96050668148213e-14\\
0.9492949	6.15053558172017e-14\\
0.9493949	7.88278609504576e-14\\
0.9494949	1.04739958817625e-13\\
0.949595	-8.39926101440767e-14\\
0.949695	-5.7911495038064e-14\\
0.949795	-4.20544423810938e-14\\
0.949895	-2.21235665100823e-14\\
0.949995	-3.54570644654311e-15\\
0.950095	1.85247193902591e-14\\
0.950195	3.92028881466918e-14\\
0.950295	5.38707754597881e-14\\
0.950395	7.81743860927101e-14\\
0.950495	8.80210125861326e-14\\
0.9505951	-7.48128954685001e-14\\
0.9506951	-5.30303370246789e-14\\
0.9507951	-3.64653623015949e-14\\
0.9508951	-1.81921477357566e-14\\
0.9509951	-1.03688559518034e-15\\
0.9510951	1.2419636017689e-14\\
0.9511951	2.98394781025516e-14\\
0.9512951	4.91250218552958e-14\\
0.9513951	6.84164651329605e-14\\
0.9514951	8.60893443198825e-14\\
0.9515952	-6.36858921786803e-14\\
0.9516952	-5.13191094535287e-14\\
0.9517952	-3.40774852696359e-14\\
0.9518952	-1.26681957548743e-14\\
0.9519952	2.42489603120106e-15\\
0.9520952	1.09389070590084e-14\\
0.9521952	3.28296257669901e-14\\
0.9522952	4.82692264514638e-14\\
0.9523952	5.7644006718472e-14\\
0.9524952	7.15521475019651e-14\\
0.9525953	-5.5792086054958e-14\\
0.9526953	-4.85092866977513e-14\\
0.9527953	-3.36680194033532e-14\\
0.9528953	-9.84431537891524e-15\\
0.9529953	-5.41318823191822e-15\\
0.9530953	1.1449258716645e-14\\
0.9531953	2.27637464478776e-14\\
0.9532953	4.07457554456916e-14\\
0.9533953	5.78034818482707e-14\\
0.9534953	6.65358139109321e-14\\
0.9535954	-5.09091811304178e-14\\
0.9536954	-3.87792184246213e-14\\
0.9537954	-2.60705049115808e-14\\
0.9538954	-9.44102556899504e-15\\
0.9539954	-5.36788755497722e-15\\
0.9540954	9.85077760668244e-15\\
0.9541954	2.00939290681571e-14\\
0.9542954	3.94157528054035e-14\\
0.9543954	5.20438158497237e-14\\
0.9544954	5.83772392507068e-14\\
0.9545955	-4.73957935022555e-14\\
0.9546955	-3.54376651064138e-14\\
0.9547955	-2.3580493664274e-14\\
0.9548955	-1.17566776351345e-14\\
0.9549955	-7.35922194368884e-16\\
0.9550955	1.08730428292037e-14\\
0.9551955	2.16207622948545e-14\\
0.9552955	3.22153729187576e-14\\
0.9553955	4.15209371059681e-14\\
0.9554955	5.1555793919488e-14\\
0.9555956	-4.21506573035587e-14\\
0.9556956	-3.07968070397141e-14\\
0.9557956	-2.07626771707145e-14\\
0.9558956	-1.04288288358495e-14\\
0.9559956	-2.95432055635846e-17\\
0.9560956	9.34562833772674e-15\\
0.9561956	1.87503300253988e-14\\
0.9562956	2.83798824951367e-14\\
0.9563956	3.75697799675404e-14\\
0.9564956	4.57942026448304e-14\\
0.9565957	-3.75998210762757e-14\\
0.9566957	-2.72681059128246e-14\\
0.9567957	-1.86739109726322e-14\\
0.9568957	-8.80275473592939e-15\\
0.9569957	-5.0868388036913e-16\\
0.9570957	8.48432835545659e-15\\
0.9571957	1.65809980817432e-14\\
0.9572957	2.53133611123923e-14\\
0.9573957	3.33394171627901e-14\\
0.9574957	4.14417883125124e-14\\
0.9575958	-3.35807213927659e-14\\
0.9576958	-2.45324318240255e-14\\
0.9577958	-1.63362297541925e-14\\
0.9578958	-8.72402834969807e-15\\
0.9579958	-3.09623758486124e-16\\
0.9580958	7.41004372365512e-15\\
0.9581958	1.50536425095141e-14\\
0.9582958	2.23542137379568e-14\\
0.9583958	2.91579499925286e-14\\
0.9584958	3.6422985239935e-14\\
0.9585959	-2.9824096783286e-14\\
0.9586959	-2.1469439206182e-14\\
0.9587959	-1.41289995166493e-14\\
0.9588959	-7.40861846168119e-15\\
0.9589959	1.91940871172698e-16\\
0.9590959	6.27785673352506e-15\\
0.9591959	1.35581239122057e-14\\
0.9592959	1.98444409249589e-14\\
0.9593959	2.60501086320077e-14\\
0.9594959	3.21889400735975e-14\\
0.959596	-2.65811656278825e-14\\
0.959696	-1.93790085345916e-14\\
0.959796	-1.26185483951884e-14\\
0.959896	-6.89450262933167e-15\\
0.959996	-7.06359980194823e-16\\
0.960096	5.54059572996396e-15\\
0.960196	1.15342724224758e-14\\
0.960296	1.80547673660885e-14\\
0.960396	2.29733745961793e-14\\
0.960496	2.92516033169618e-14\\
0.9605961	-2.38031871725372e-14\\
0.9606961	-1.68885308020871e-14\\
0.9607961	-1.12058300610358e-14\\
0.9608961	-5.44224464417286e-15\\
0.9609961	-1.9947784230208e-16\\
0.9610961	5.00530143688129e-15\\
0.9611961	1.07385463494678e-14\\
0.9612961	1.5649429176747e-14\\
0.9613961	2.04689480036816e-14\\
0.9614961	2.60090425895802e-14\\
0.9615962	-2.11523676664888e-14\\
0.9616962	-1.58128083219314e-14\\
0.9617962	-9.84681779548283e-15\\
0.9618962	-5.12726963592845e-15\\
0.9619962	-4.50380140092933e-16\\
0.9620962	4.46346183566506e-15\\
0.9621962	8.96887344457275e-15\\
0.9622962	1.34946651798705e-14\\
0.9623962	1.85430369451005e-14\\
0.9624962	2.26887819922182e-14\\
0.9625963	-1.80060682564783e-14\\
0.9626963	-1.41174103928399e-14\\
0.9627963	-8.98544783580082e-15\\
0.9628963	-4.75273534262968e-15\\
0.9629963	-4.93090032761499e-16\\
0.9630963	3.7876619455937e-15\\
0.9631963	8.15094639069016e-15\\
0.9632963	1.27247098324062e-14\\
0.9633963	1.67026960251678e-14\\
0.9634963	2.03437299604866e-14\\
0.9635964	-1.65095056476354e-14\\
0.9636964	-1.20305254115064e-14\\
0.9637964	-7.73409975971933e-15\\
0.9638964	-4.10802586306173e-15\\
0.9639964	-5.78488155065571e-16\\
0.9640964	3.48926987955693e-15\\
0.9641964	7.79028093758652e-15\\
0.9642964	1.10791963106302e-14\\
0.9643964	1.4169635277093e-14\\
0.9644964	1.79335409875778e-14\\
0.9645965	-1.46347493108715e-14\\
0.9646965	-1.12518632916089e-14\\
0.9647965	-7.24100229267208e-15\\
0.9648965	-3.50317480798782e-15\\
0.9649965	1.15815765722195e-16\\
0.9650965	2.82476680172865e-15\\
0.9651965	6.88647913414063e-15\\
0.9652965	9.61716558854387e-15\\
0.9653965	1.3385866065925e-14\\
0.9654965	1.56138687463199e-14\\
0.9655966	-1.31150417083111e-14\\
0.9656966	-9.11920127284094e-15\\
0.9657966	-6.09680109144292e-15\\
0.9658966	-3.4231763242671e-15\\
0.9659966	-4.24217526153491e-16\\
0.9660966	2.62308650956835e-15\\
0.9661966	5.49011165020508e-15\\
0.9662966	8.99606703326662e-15\\
0.9663966	1.10074693091938e-14\\
0.9664966	1.44376227426955e-14\\
0.9665967	-1.10425637374967e-14\\
0.9666967	-8.51301386447406e-15\\
0.9667967	-5.55312116391063e-15\\
0.9668967	-3.06747850846805e-15\\
0.9669967	8.35896771831815e-17\\
0.9670967	2.08353948946424e-15\\
0.9671967	5.15912514093559e-15\\
0.9672967	7.57992161249487e-15\\
0.9673967	1.06578523669113e-14\\
0.9674967	1.27467221710197e-14\\
0.9675968	-9.8440874385441e-15\\
0.9676968	-7.20663396971472e-15\\
0.9677968	-5.25364001119986e-15\\
0.9678968	-2.46830872091988e-15\\
0.9679968	-2.94224511735042e-16\\
0.9680968	1.86420893402575e-15\\
0.9681968	4.64133540533416e-15\\
0.9682968	6.70981745644722e-15\\
0.9683968	8.78021235693654e-15\\
0.9684968	1.16005526190768e-14\\
0.9685969	-9.2818743312778e-15\\
0.9686969	-6.30293718842813e-15\\
0.9687969	-4.11235757797346e-15\\
0.9688969	-1.8162721536545e-15\\
0.9689969	-4.85373705714259e-16\\
0.9690969	1.8446957746017e-15\\
0.9691969	4.17295598757923e-15\\
0.9692969	6.53270366292409e-15\\
0.9693969	7.9911318968103e-15\\
0.9694969	9.64895361215664e-15\\
0.969597	-8.0661410832953e-15\\
0.969697	-6.33807988334016e-15\\
0.969797	-3.91385683925444e-15\\
0.969897	-1.56233257024991e-15\\
0.969997	-2.06708468745964e-17\\
0.970097	2.0053086113032e-15\\
0.970197	3.84078107909063e-15\\
0.970297	5.84157162761332e-15\\
0.970397	7.39381360086209e-15\\
0.970497	8.91361077488199e-15\\
0.9705971	-6.60989508607506e-15\\
0.9706971	-5.57782711506611e-15\\
0.9707971	-3.1558749865779e-15\\
0.9708971	-1.81133029026383e-15\\
0.9709971	1.68504455935052e-17\\
0.9710971	1.91772927700315e-15\\
0.9711971	3.50807387967105e-15\\
0.9712971	4.43204777606452e-15\\
0.9713971	6.36090414397908e-15\\
0.9714971	7.99268281289595e-15\\
0.9715972	-6.40647432008597e-15\\
0.9716972	-4.98201904267371e-15\\
0.9717972	-2.60499316265235e-15\\
0.9718972	-1.47314920349514e-15\\
0.9719972	2.41080423026208e-16\\
0.9720972	1.39029974335876e-15\\
0.9721972	2.85186849065646e-15\\
0.9722972	4.52762452103539e-15\\
0.9723972	5.34360921608017e-15\\
0.9724972	7.24979587295427e-15\\
0.9725973	-5.46426306194351e-15\\
0.9726973	-4.26729909565046e-15\\
0.9727973	-2.99126197164252e-15\\
0.9728973	-1.59308536264361e-15\\
0.9729973	-7.08553338097333e-18\\
0.9730973	8.54784294233544e-16\\
0.9731973	2.10268244668669e-15\\
0.9732973	3.86862993028047e-15\\
0.9733973	4.90626442004583e-15\\
0.9734973	6.09059681064396e-15\\
0.9735974	-4.89140277865436e-15\\
0.9736974	-3.65697547113295e-15\\
0.9737974	-2.4267242668249e-15\\
0.9738974	-1.24250612434502e-15\\
0.9739974	-2.59825943125385e-17\\
0.9740974	1.12115231928131e-15\\
0.9741974	2.21694681014783e-15\\
0.9742974	3.29896837272143e-15\\
0.9743974	4.32408337596874e-15\\
0.9744974	5.36823902939616e-15\\
0.9745975	-4.28550192664138e-15\\
0.9746975	-3.16943631297114e-15\\
0.9747975	-2.09570680166515e-15\\
0.9748975	-1.01415746812805e-15\\
0.9749975	-5.66065101056527e-17\\
0.9750975	9.62949697253627e-16\\
0.9751975	1.94813479984715e-15\\
0.9752975	2.91999290742279e-15\\
0.9753975	3.81679112978651e-15\\
0.9754975	4.79382440208151e-15\\
0.9755976	-3.8490354757526e-15\\
0.9756976	-2.86255895634949e-15\\
0.9757976	-1.83110241264071e-15\\
0.9758976	-9.33153081497267e-16\\
0.9759976	-3.11148973242964e-17\\
0.9760976	8.28508827009157e-16\\
0.9761976	1.71493475684131e-15\\
0.9762976	2.61292086805371e-15\\
0.9763976	3.42258975312395e-15\\
0.9764976	4.25925389286201e-15\\
0.9765977	-3.32000230225411e-15\\
0.9766977	-2.52073339731169e-15\\
0.9767977	-1.66034385924096e-15\\
0.9768977	-8.64456307135649e-16\\
0.9769977	-4.43487869603866e-17\\
0.9770977	8.02881777464565e-16\\
0.9771977	1.49415793462481e-15\\
0.9772977	2.26026181934888e-15\\
0.9773977	3.04567711125048e-15\\
0.9774977	3.70843263417033e-15\\
0.9775978	-2.97919337352154e-15\\
0.9776978	-2.24318193153444e-15\\
0.9777978	-1.47710598183703e-15\\
0.9778978	-7.70272118347874e-16\\
0.9779978	8.01986035263158e-19\\
0.9780978	6.72241413153093e-16\\
0.9781978	1.3926693949565e-15\\
0.9782978	2.02306432046396e-15\\
0.9783978	2.63661835834648e-15\\
0.9784978	3.3185976124688e-15\\
0.9785979	-2.66985644218757e-15\\
0.9786979	-1.96693991365657e-15\\
0.9787979	-1.26970471954955e-15\\
0.9788979	-6.45944885393322e-16\\
0.9789979	-5.20567437670089e-17\\
0.9790979	5.66830337111536e-16\\
0.9791979	1.17672464972125e-15\\
0.9792979	1.75464432191231e-15\\
0.9793979	2.38849092751444e-15\\
0.9794979	2.87692443490551e-15\\
0.979598	-2.34238443707008e-15\\
0.979698	-1.73493355353771e-15\\
0.979798	-1.11445915835182e-15\\
0.979898	-5.40479745654109e-16\\
0.979998	3.76401327312253e-17\\
0.980098	5.80572962211392e-16\\
0.980198	1.05891275268285e-15\\
0.980298	1.55306078258673e-15\\
0.980398	2.05311261496897e-15\\
0.980498	2.55874639597289e-15\\
0.9805981	-2.01568933528675e-15\\
0.9806981	-1.49883275412846e-15\\
0.9807981	-1.02173035347508e-15\\
0.9808981	-5.47456853234365e-16\\
0.9809981	-3.00445388982413e-17\\
0.9810981	4.85412323675131e-16\\
0.9811981	9.6265426644083e-16\\
0.9812981	1.37415454606186e-15\\
0.9813981	1.80101820924303e-15\\
0.9814981	2.23288233018754e-15\\
0.9815982	-1.82794995510307e-15\\
0.9816982	-1.32757285178102e-15\\
0.9817982	-9.03754475376562e-16\\
0.9818982	-4.33697290338819e-16\\
0.9819982	1.34457956797396e-17\\
0.9820982	3.76478160086848e-16\\
0.9821982	8.02067307619498e-16\\
0.9822982	1.24465369418171e-15\\
0.9823982	1.66636061829846e-15\\
0.9824982	2.0369051073456e-15\\
0.9825983	-1.63226138339822e-15\\
0.9826983	-1.17545012445675e-15\\
0.9827983	-8.16568309999518e-16\\
0.9828983	-3.56387935117344e-16\\
0.9829983	1.14819603380833e-17\\
0.9830983	4.00512058944482e-16\\
0.9831983	7.31170462216012e-16\\
0.9832983	1.13084133115859e-15\\
0.9833983	1.43374442847537e-15\\
0.9834983	1.78085558068922e-15\\
0.9835984	-1.37719748044031e-15\\
0.9836984	-1.04833559775735e-15\\
0.9837984	-7.13119689562355e-16\\
0.9838984	-3.04320776009049e-16\\
0.9839984	-4.83382725923814e-17\\
0.9840984	3.34725737869503e-16\\
0.9841984	6.30993049933921e-16\\
0.9842984	9.32736735955767e-16\\
0.9843984	1.23830936451892e-15\\
0.9844984	1.55207202894022e-15\\
0.9845985	-1.19827661212437e-15\\
0.9846985	-8.92672059714557e-16\\
0.9847985	-6.30915830659185e-16\\
0.9848985	-2.85300721669752e-16\\
0.9849985	-2.2454002938661e-17\\
0.9850985	2.96596326151591e-16\\
0.9851985	6.16356023259786e-16\\
0.9852985	8.86799841887112e-16\\
0.9853985	1.06330750289316e-15\\
0.9854985	1.40660039557747e-15\\
0.9855986	-1.13365466426478e-15\\
0.9856986	-8.19381466786991e-16\\
0.9857986	-5.25152081019436e-16\\
0.9858986	-2.69492022376612e-16\\
0.9859986	3.41089823190265e-17\\
0.9860986	2.771379083514e-16\\
0.9861986	4.56000030015886e-16\\
0.9862986	7.7196115873242e-16\\
0.9863986	9.31090568394068e-16\\
0.9864986	1.24420456155302e-15\\
0.9865987	-9.659337710365e-16\\
0.9866987	-6.6349082925385e-16\\
0.9867987	-4.47059909104707e-16\\
0.9868987	-1.87383621796342e-16\\
0.9869987	-5.07302266596473e-17\\
0.9870987	2.01053699356616e-16\\
0.9871987	4.10491197804172e-16\\
0.9872987	6.24423464437119e-16\\
0.9873987	8.93958970225117e-16\\
0.9874987	1.07442316961313e-15\\
0.9875988	-8.81654169982294e-16\\
0.9876988	-6.70069257388948e-16\\
0.9877988	-4.57101994315213e-16\\
0.9878988	-1.71046480340222e-16\\
0.9879988	-3.62228157495298e-17\\
0.9880988	2.2697596997202e-16\\
0.9881988	4.02037539846683e-16\\
0.9882988	5.76284384742757e-16\\
0.9883988	7.40828504606258e-16\\
0.9884988	8.90526605728326e-16\\
0.9885989	-7.30726646342412e-16\\
0.9886989	-5.87854491204673e-16\\
0.9887989	-3.53522726361314e-16\\
0.9888989	-2.18330092507568e-16\\
0.9889989	3.06529350726108e-17\\
0.9890989	2.09842354657143e-16\\
0.9891989	3.39099319571509e-16\\
0.9892989	5.41689305678409e-16\\
0.9893989	6.44241747886584e-16\\
0.9894989	7.7671013988937e-16\\
0.989599	-6.5966786156736e-16\\
0.989699	-4.43638759470962e-16\\
0.989799	-2.88539686272847e-16\\
0.989899	-1.51507604362058e-16\\
0.989999	1.3451864910689e-17\\
0.990099	1.55427232316576e-16\\
0.990199	3.26564355198794e-16\\
0.990299	4.82030096714917e-16\\
0.990399	5.79976381135925e-16\\
0.990499	6.81504683815718e-16\\
0.9905991	-5.8493486921548e-16\\
0.9906991	-4.62996238566503e-16\\
0.9907991	-2.37051717556047e-16\\
0.9908991	-1.34545834229701e-16\\
0.9909991	1.98549048936075e-17\\
0.9910991	1.04228896393467e-16\\
0.9911991	2.99366710877155e-16\\
0.9912991	3.88738742045859e-16\\
0.9913991	5.18463232234175e-16\\
0.9914991	6.37274671073253e-16\\
0.9915992	-5.05838814960281e-16\\
0.9916992	-3.76188880512203e-16\\
0.9917992	-2.46063940570047e-16\\
0.9918992	-1.16570782555271e-16\\
0.9919992	3.64744054211748e-18\\
0.9920992	1.18381746764244e-16\\
0.9921992	2.33828213798706e-16\\
0.9922992	3.48559197624052e-16\\
0.9923992	4.53494897982358e-16\\
0.9924992	5.61875240213999e-16\\
0.9925993	-4.40379813324969e-16\\
0.9926993	-3.32013742453361e-16\\
0.9927993	-2.17019925478044e-16\\
0.9928993	-1.03151910764365e-16\\
0.9929993	4.02075145648678e-18\\
0.9930993	1.01086314733174e-16\\
0.9931993	2.06764945755982e-16\\
0.9932993	3.0188313471445e-16\\
0.9933993	3.99348409809398e-16\\
0.9934993	4.94124340278315e-16\\
0.9935994	-3.91820447271422e-16\\
0.9936994	-2.86888143743227e-16\\
0.9937994	-1.87821808305208e-16\\
0.9938994	-9.16758798766323e-17\\
0.9939994	-3.56942139665406e-18\\
0.9940994	9.32905948648913e-17\\
0.9941994	1.7758627247157e-16\\
0.9942994	2.69866045687359e-16\\
0.9943994	3.525222079576e-16\\
0.9944994	4.29768691571783e-16\\
0.9945995	-3.38844718419597e-16\\
0.9946995	-2.51550917625835e-16\\
0.9947995	-1.66452958835511e-16\\
0.9948995	-8.22655737112991e-17\\
0.9949995	4.01091635655171e-18\\
0.9950995	7.70699047613612e-17\\
0.9951995	1.63278054898912e-16\\
0.9952995	2.30655090458849e-16\\
0.9953995	3.08853826319983e-16\\
0.9954995	3.79140432306351e-16\\
0.9955996	-2.97679450216933e-16\\
0.9956996	-2.23335392353575e-16\\
0.9957996	-1.43741932617933e-16\\
0.9958996	-7.13706872547744e-17\\
0.9959996	2.82482374349785e-18\\
0.9960996	6.93907989934537e-17\\
0.9961996	1.40354994282518e-16\\
0.9962996	2.09208772745976e-16\\
0.9963996	2.70889369990475e-16\\
0.9964996	3.317623700071e-16\\
0.9965997	-2.60537040939547e-16\\
0.9966997	-1.97994362529457e-16\\
0.9967997	-1.28873393677529e-16\\
0.9968997	-6.12649697603942e-17\\
0.9969997	-1.91616618311282e-18\\
0.9970997	6.37533803443012e-17\\
0.9971997	1.2163534827658e-16\\
0.9972997	1.78916796357809e-16\\
0.9973997	2.34064418925546e-16\\
0.9974997	2.96808986874765e-16\\
0.9975998	-2.31128326083684e-16\\
0.9976998	-1.71485685179508e-16\\
0.9977998	-1.17718126461217e-16\\
0.9978998	-5.51893081408364e-17\\
0.9979998	1.92610466123742e-18\\
0.9980998	5.06279563686943e-17\\
0.9981998	1.09076289334618e-16\\
0.9982998	1.56577203367534e-16\\
0.9983998	2.03568882752572e-16\\
0.9984998	2.6160778761704e-16\\
0.9985999	-2.04751888954428e-16\\
0.9986999	-1.48919434213002e-16\\
0.9987999	-1.00880956379835e-16\\
0.9988999	-4.47406394448414e-17\\
0.9989999	-3.55101129658721e-18\\
0.9990999	5.06742078236784e-17\\
0.9991999	9.69474064844696e-17\\
0.9992999	1.35294539099555e-16\\
0.9993999	1.86742730012554e-16\\
0.9994999	2.23308026246231e-16\\
0.9996	-1.77555470540306e-16\\
0.9997	-1.34988167175258e-16\\
0.9998	-8.55169907301101e-17\\
0.9999	-4.32888453425701e-17\\
1	-1.52080219417315e-18\\
};
\end{axis}
\end{tikzpicture}%
		\caption{The error is defined as the difference between the analytical solution and numerical solution.}
		\label{fig:smallErrorError}
	\end{subfigure}
	\caption{Here we see the error using smaller timesteps and spacial steps to get good resolution and precision on how the function will look in the end.}
	\label{fig:smallError}
\end{figure}
\fi

\newpage
\subsection{Behavior at the end of the Solution Space}
looking at figure \ref{fig:periodicInf} we can clearly see the periodic behavior of the numerical solution. This is in a stark contrast against the numerical solution. This probably lies in the fast fourier transform package used. This because the discrete fourier transform can only be done on periodic problems. Thus as soon we are outside of our spacial solutions space of $x=[0,1]$ we will get significant error. This is highlighted in figure \ref{fig:periodicInfError}.
\iftikz
\begin{figure}[H]
	\centering
	\begin{subfigure}{.9\linewidth}
		\setlength\figureheight{.5\linewidth}
		\setlength\figurewidth{.9\linewidth}
		% This file was created by matlab2tikz.
% Minimal pgfplots version: 1.3
%
%The latest updates can be retrieved from
%  http://www.mathworks.com/matlabcentral/fileexchange/22022-matlab2tikz
%where you can also make suggestions and rate matlab2tikz.
%
\definecolor{mycolor1}{rgb}{0.00000,0.44700,0.74100}%
\definecolor{mycolor2}{rgb}{0.85000,0.32500,0.09800}%
%
\begin{tikzpicture}

\begin{axis}[%
width=0.95092\figurewidth,
height=\figureheight,
at={(0\figurewidth,0\figureheight)},
scale only axis,
xmin=0,
xmax=1,
xlabel={Position},
ymin=-3,
ymax=6,
title style={font=\bfseries},
title={Analytical Solution over Numerical Solution},
legend style={at={(0.03,0.97)},anchor=north west,legend cell align=left,align=left,draw=white!15!black},
title style={font=\small},ticklabel style={font=\tiny}
]
\addplot [color=mycolor1,solid,forget plot]
  table[row sep=crcr]{%
0	0.03279891\\
0.00010001	0.0326747\\
0.00020002	0.03186465\\
0.00030003	0.03087951\\
0.00040004	0.03040553\\
0.00050005	0.0300395\\
0.00060006	0.02995223\\
0.00070007	0.03017902\\
0.00080008	0.03050235\\
0.00090009	0.03055689\\
0.0010001	0.03044676\\
0.00110011	0.03006607\\
0.00120012	0.02965036\\
0.00130013	0.02982389\\
0.00140014	0.02958991\\
0.00150015	0.03005099\\
0.00160016	0.03027945\\
0.00170017	0.03061407\\
0.00180018	0.02989003\\
0.00190019	0.0295733\\
0.0020002	0.02899653\\
0.00210021	0.02903299\\
0.00220022	0.02924911\\
0.00230023	0.02902951\\
0.00240024	0.02921216\\
0.00250025	0.02825344\\
0.00260026	0.02756002\\
0.00270027	0.02747151\\
0.00280028	0.02778646\\
0.00290029	0.02774148\\
0.0030003	0.02800545\\
0.00310031	0.02795833\\
0.00320032	0.02792743\\
0.00330033	0.02714662\\
0.00340034	0.02759028\\
0.00350035	0.02813572\\
0.00360036	0.02849558\\
0.00370037	0.02847096\\
0.00380038	0.02843458\\
0.00390039	0.02867554\\
0.0040004	0.02826214\\
0.00410041	0.0277668\\
0.00420042	0.02778857\\
0.00430043	0.02731028\\
0.00440044	0.02749018\\
0.00450045	0.02714522\\
0.00460046	0.02627582\\
0.00470047	0.0257735\\
0.00480048	0.02510986\\
0.00490049	0.02574393\\
0.0050005	0.02594624\\
0.00510051	0.0259037\\
0.00520052	0.02527817\\
0.00530053	0.02531767\\
0.00540054	0.02539622\\
0.00550055	0.02551154\\
0.00560056	0.0258043\\
0.00570057	0.02536188\\
0.00580058	0.02516997\\
0.00590059	0.0246524\\
0.0060006	0.02464822\\
0.00610061	0.02497073\\
0.00620062	0.02504789\\
0.00630063	0.02473015\\
0.00640064	0.02398102\\
0.00650065	0.02297684\\
0.00660066	0.02354314\\
0.00670067	0.02356506\\
0.00680068	0.02328437\\
0.00690069	0.02316501\\
0.0070007	0.02315845\\
0.00710071	0.02345513\\
0.00720072	0.02380649\\
0.00730073	0.02427267\\
0.00740074	0.02428627\\
0.00750075	0.02451394\\
0.00760076	0.02453857\\
0.00770077	0.02509476\\
0.00780078	0.02548144\\
0.00790079	0.02510888\\
0.0080008	0.02451986\\
0.00810081	0.02378216\\
0.00820082	0.02357047\\
0.00830083	0.02378276\\
0.00840084	0.02306014\\
0.00850085	0.02206686\\
0.00860086	0.02150218\\
0.00870087	0.02125111\\
0.00880088	0.02139687\\
0.00890089	0.02140333\\
0.0090009	0.0217792\\
0.00910091	0.02191427\\
0.00920092	0.02128772\\
0.00930093	0.02224904\\
0.00940094	0.02238147\\
0.00950095	0.02224286\\
0.00960096	0.02157667\\
0.00970097	0.02127288\\
0.00980098	0.02116126\\
0.00990099	0.02105036\\
0.010001	0.02104549\\
0.01010101	0.02040838\\
0.01020102	0.01995105\\
0.01030103	0.020191\\
0.01040104	0.02085803\\
0.01050105	0.020893\\
0.01060106	0.02099419\\
0.01070107	0.02101218\\
0.01080108	0.02129707\\
0.01090109	0.02145562\\
0.0110011	0.02115158\\
0.01110111	0.02039286\\
0.01120112	0.02025078\\
0.01130113	0.02008614\\
0.01140114	0.02007754\\
0.01150115	0.01990756\\
0.01160116	0.01932657\\
0.01170117	0.01901076\\
0.01180118	0.01903419\\
0.01190119	0.01933833\\
0.0120012	0.01993611\\
0.01210121	0.01956339\\
0.01220122	0.01920894\\
0.01230123	0.02004775\\
0.01240124	0.02065795\\
0.01250125	0.01988511\\
0.01260126	0.01951036\\
0.01270127	0.0195763\\
0.01280128	0.02023035\\
0.01290129	0.01972612\\
0.0130013	0.01912858\\
0.01310131	0.0188694\\
0.01320132	0.01928717\\
0.01330133	0.01890234\\
0.01340134	0.01838385\\
0.01350135	0.01793528\\
0.01360136	0.01816563\\
0.01370137	0.01884729\\
0.01380138	0.01809102\\
0.01390139	0.01761618\\
0.0140014	0.01759306\\
0.01410141	0.01772421\\
0.01420142	0.01692827\\
0.01430143	0.01640072\\
0.01440144	0.016409\\
0.01450145	0.0168541\\
0.01460146	0.01687614\\
0.01470147	0.01680702\\
0.01480148	0.01678405\\
0.01490149	0.01751277\\
0.0150015	0.01820367\\
0.01510151	0.01763165\\
0.01520152	0.01732458\\
0.01530153	0.01755883\\
0.01540154	0.01785804\\
0.01550155	0.01744876\\
0.01560156	0.01674341\\
0.01570157	0.01720526\\
0.01580158	0.01725453\\
0.01590159	0.01647132\\
0.0160016	0.01651581\\
0.01610161	0.01654435\\
0.01620162	0.01658319\\
0.01630163	0.01675631\\
0.01640164	0.01696304\\
0.01650165	0.01727014\\
0.01660166	0.01696546\\
0.01670167	0.01617824\\
0.01680168	0.01558916\\
0.01690169	0.01619746\\
0.0170017	0.01634499\\
0.01710171	0.01612091\\
0.01720172	0.0159139\\
0.01730173	0.01637823\\
0.01740174	0.01614508\\
0.01750175	0.01561616\\
0.01760176	0.01534785\\
0.01770177	0.01555142\\
0.01780178	0.01525876\\
0.01790179	0.01470636\\
0.0180018	0.01507191\\
0.01810181	0.01440759\\
0.01820182	0.01389743\\
0.01830183	0.01355486\\
0.01840184	0.01385405\\
0.01850185	0.01459324\\
0.01860186	0.01479545\\
0.01870187	0.01467257\\
0.01880188	0.01500455\\
0.01890189	0.01525513\\
0.0190019	0.01491749\\
0.01910191	0.01474385\\
0.01920192	0.01487441\\
0.01930193	0.01485295\\
0.01940194	0.01488229\\
0.01950195	0.01498013\\
0.01960196	0.01461288\\
0.01970197	0.01426432\\
0.01980198	0.01434327\\
0.01990199	0.01435977\\
0.020002	0.01385353\\
0.02010201	0.01386864\\
0.02020202	0.01359846\\
0.02030203	0.01345535\\
0.02040204	0.01316322\\
0.02050205	0.01263971\\
0.02060206	0.01340157\\
0.02070207	0.01379289\\
0.02080208	0.01391897\\
0.02090209	0.01377972\\
0.0210021	0.01396197\\
0.02110211	0.01418684\\
0.02120212	0.01433694\\
0.02130213	0.01448349\\
0.02140214	0.01378539\\
0.02150215	0.01319255\\
0.02160216	0.01313312\\
0.02170217	0.01385992\\
0.02180218	0.01321704\\
0.02190219	0.01310436\\
0.0220022	0.01329737\\
0.02210221	0.01265199\\
0.02220222	0.01197205\\
0.02230223	0.0118698\\
0.02240224	0.01231787\\
0.02250225	0.01173885\\
0.02260226	0.01169175\\
0.02270227	0.01185941\\
0.02280228	0.0117783\\
0.02290229	0.01201141\\
0.0230023	0.01230409\\
0.02310231	0.01238037\\
0.02320232	0.01184208\\
0.02330233	0.01185691\\
0.02340234	0.0120895\\
0.02350235	0.01176416\\
0.02360236	0.01193758\\
0.02370237	0.01185876\\
0.02380238	0.01128476\\
0.02390239	0.01200747\\
0.0240024	0.01211972\\
0.02410241	0.01217124\\
0.02420242	0.01226636\\
0.02430243	0.01224921\\
0.02440244	0.01191926\\
0.02450245	0.01190428\\
0.02460246	0.01233165\\
0.02470247	0.01245038\\
0.02480248	0.01196744\\
0.02490249	0.01184733\\
0.0250025	0.01208723\\
0.02510251	0.01159301\\
0.02520252	0.01147872\\
0.02530253	0.01118573\\
0.02540254	0.01096636\\
0.02550255	0.01047533\\
0.02560256	0.01038058\\
0.02570257	0.01037632\\
0.02580258	0.01036965\\
0.02590259	0.01047896\\
0.0260026	0.01027745\\
0.02610261	0.01090067\\
0.02620262	0.01101785\\
0.02630263	0.01068623\\
0.02640264	0.01057648\\
0.02650265	0.01076076\\
0.02660266	0.01076977\\
0.02670267	0.01055801\\
0.02680268	0.01072351\\
0.02690269	0.01018356\\
0.0270027	0.01020656\\
0.02710271	0.01000544\\
0.02720272	0.009601827\\
0.02730273	0.009584042\\
0.02740274	0.009581976\\
0.02750275	0.009736833\\
0.02760276	0.01033626\\
0.02770277	0.009997812\\
0.02780278	0.009611788\\
0.02790279	0.01011936\\
0.0280028	0.01068211\\
0.02810281	0.01017663\\
0.02820282	0.01002172\\
0.02830283	0.009922288\\
0.02840284	0.009839875\\
0.02850285	0.01002707\\
0.02860286	0.009809572\\
0.02870287	0.009605796\\
0.02880288	0.009941631\\
0.02890289	0.009701868\\
0.0290029	0.009577669\\
0.02910291	0.00954319\\
0.02920292	0.009149161\\
0.02930293	0.009879691\\
0.02940294	0.009581105\\
0.02950295	0.0093875\\
0.02960296	0.009314919\\
0.02970297	0.009365687\\
0.02980298	0.008981108\\
0.02990299	0.009002584\\
0.030003	0.008668719\\
0.03010301	0.008782852\\
0.03020302	0.008990538\\
0.03030303	0.009209409\\
0.03040304	0.009155987\\
0.03050305	0.009201231\\
0.03060306	0.008981072\\
0.03070307	0.009134416\\
0.03080308	0.009125746\\
0.03090309	0.008506506\\
0.0310031	0.008647017\\
0.03110311	0.008573618\\
0.03120312	0.00841567\\
0.03130313	0.008184082\\
0.03140314	0.00796471\\
0.03150315	0.00782538\\
0.03160316	0.007801551\\
0.03170317	0.007257149\\
0.03180318	0.007601144\\
0.03190319	0.007692192\\
0.0320032	0.007754905\\
0.03210321	0.008490003\\
0.03220322	0.008390323\\
0.03230323	0.008426989\\
0.03240324	0.008650309\\
0.03250325	0.008150308\\
0.03260326	0.00862822\\
0.03270327	0.008631255\\
0.03280328	0.0088332\\
0.03290329	0.008684659\\
0.0330033	0.008555928\\
0.03310331	0.008735619\\
0.03320332	0.008732209\\
0.03330333	0.008046671\\
0.03340334	0.007917161\\
0.03350335	0.007824257\\
0.03360336	0.007737371\\
0.03370337	0.008019023\\
0.03380338	0.007685393\\
0.03390339	0.007447921\\
0.0340034	0.007173883\\
0.03410341	0.006986045\\
0.03420342	0.006991961\\
0.03430343	0.00691781\\
0.03440344	0.007182163\\
0.03450345	0.007127342\\
0.03460346	0.007268524\\
0.03470347	0.007188024\\
0.03480348	0.007007627\\
0.03490349	0.007305441\\
0.0350035	0.00744545\\
0.03510351	0.007233016\\
0.03520352	0.007477751\\
0.03530353	0.00738384\\
0.03540354	0.007183854\\
0.03550355	0.006832617\\
0.03560356	0.006798891\\
0.03570357	0.007083925\\
0.03580358	0.00652053\\
0.03590359	0.006520855\\
0.0360036	0.006728152\\
0.03610361	0.006783296\\
0.03620362	0.006660594\\
0.03630363	0.00680416\\
0.03640364	0.006986361\\
0.03650365	0.007069497\\
0.03660366	0.006912253\\
0.03670367	0.007204746\\
0.03680368	0.006947075\\
0.03690369	0.00677217\\
0.0370037	0.006517618\\
0.03710371	0.006341969\\
0.03720372	0.006378411\\
0.03730373	0.006344911\\
0.03740374	0.006591903\\
0.03750375	0.006772859\\
0.03760376	0.00686254\\
0.03770377	0.006703892\\
0.03780378	0.006414949\\
0.03790379	0.00635618\\
0.0380038	0.00631953\\
0.03810381	0.006518432\\
0.03820382	0.006450342\\
0.03830383	0.006369636\\
0.03840384	0.006477444\\
0.03850385	0.006308992\\
0.03860386	0.006166603\\
0.03870387	0.006092823\\
0.03880388	0.006379064\\
0.03890389	0.006104841\\
0.0390039	0.00614191\\
0.03910391	0.005830091\\
0.03920392	0.005623284\\
0.03930393	0.005418884\\
0.03940394	0.005247497\\
0.03950395	0.005443117\\
0.03960396	0.005417847\\
0.03970397	0.00585313\\
0.03980398	0.005511527\\
0.03990399	0.005508006\\
0.040004	0.005774232\\
0.04010401	0.005645905\\
0.04020402	0.005851713\\
0.04030403	0.005834255\\
0.04040404	0.005926942\\
0.04050405	0.00570357\\
0.04060406	0.005489222\\
0.04070407	0.005346348\\
0.04080408	0.005455909\\
0.04090409	0.00538873\\
0.0410041	0.005298873\\
0.04110411	0.005276357\\
0.04120412	0.005433797\\
0.04130413	0.005808833\\
0.04140414	0.005355826\\
0.04150415	0.005776454\\
0.04160416	0.005629201\\
0.04170417	0.005896409\\
0.04180418	0.005840726\\
0.04190419	0.005834513\\
0.0420042	0.005834301\\
0.04210421	0.005499234\\
0.04220422	0.005812607\\
0.04230423	0.00554965\\
0.04240424	0.005600339\\
0.04250425	0.005243153\\
0.04260426	0.00487147\\
0.04270427	0.00456888\\
0.04280428	0.004692562\\
0.04290429	0.004165969\\
0.0430043	0.004243096\\
0.04310431	0.004443942\\
0.04320432	0.00436417\\
0.04330433	0.004573209\\
0.04340434	0.004757453\\
0.04350435	0.005045549\\
0.04360436	0.004795178\\
0.04370437	0.005276419\\
0.04380438	0.004908096\\
0.04390439	0.004719769\\
0.0440044	0.004553008\\
0.04410441	0.004461041\\
0.04420442	0.004707908\\
0.04430443	0.005034878\\
0.04440444	0.004942825\\
0.04450445	0.005094678\\
0.04460446	0.004783427\\
0.04470447	0.004703993\\
0.04480448	0.004687545\\
0.04490449	0.004632402\\
0.0450045	0.004646111\\
0.04510451	0.004729542\\
0.04520452	0.004456228\\
0.04530453	0.004382034\\
0.04540454	0.004240763\\
0.04550455	0.004239705\\
0.04560456	0.004193251\\
0.04570457	0.004158962\\
0.04580458	0.004232914\\
0.04590459	0.004123133\\
0.0460046	0.004463883\\
0.04610461	0.004266157\\
0.04620462	0.004517729\\
0.04630463	0.004678357\\
0.04640464	0.004698267\\
0.04650465	0.004877839\\
0.04660466	0.004923176\\
0.04670467	0.004709238\\
0.04680468	0.004600252\\
0.04690469	0.00440147\\
0.0470047	0.004230036\\
0.04710471	0.004059681\\
0.04720472	0.003808869\\
0.04730473	0.003552785\\
0.04740474	0.003389714\\
0.04750475	0.003472435\\
0.04760476	0.003835628\\
0.04770477	0.003966962\\
0.04780478	0.00393837\\
0.04790479	0.003950855\\
0.0480048	0.003899902\\
0.04810481	0.003840653\\
0.04820482	0.003751005\\
0.04830483	0.003753817\\
0.04840484	0.003703659\\
0.04850485	0.003787015\\
0.04860486	0.003843523\\
0.04870487	0.003756728\\
0.04880488	0.003628575\\
0.04890489	0.003634349\\
0.0490049	0.003552841\\
0.04910491	0.003679132\\
0.04920492	0.003855274\\
0.04930493	0.003780428\\
0.04940494	0.003806095\\
0.04950495	0.003832593\\
0.04960496	0.003773936\\
0.04970497	0.003708721\\
0.04980498	0.003824191\\
0.04990499	0.004005917\\
0.050005	0.003876358\\
0.05010501	0.003728915\\
0.05020502	0.003794168\\
0.05030503	0.003770764\\
0.05040504	0.003668214\\
0.05050505	0.003640178\\
0.05060506	0.003782546\\
0.05070507	0.003707704\\
0.05080508	0.003566656\\
0.05090509	0.003621068\\
0.0510051	0.003439766\\
0.05110511	0.003342905\\
0.05120512	0.003102974\\
0.05130513	0.00291753\\
0.05140514	0.002845985\\
0.05150515	0.003000869\\
0.05160516	0.003121632\\
0.05170517	0.003363007\\
0.05180518	0.003297123\\
0.05190519	0.003224094\\
0.0520052	0.003052827\\
0.05210521	0.003070491\\
0.05220522	0.003157289\\
0.05230523	0.003177816\\
0.05240524	0.003123892\\
0.05250525	0.003002546\\
0.05260526	0.003105801\\
0.05270527	0.003168233\\
0.05280528	0.003116044\\
0.05290529	0.00310376\\
0.0530053	0.002859377\\
0.05310531	0.003134774\\
0.05320532	0.003103138\\
0.05330533	0.003021663\\
0.05340534	0.003211921\\
0.05350535	0.003228424\\
0.05360536	0.003255859\\
0.05370537	0.003219102\\
0.05380538	0.003427985\\
0.05390539	0.003480507\\
0.0540054	0.003281313\\
0.05410541	0.003014299\\
0.05420542	0.002632718\\
0.05430543	0.002662939\\
0.05440544	0.002635283\\
0.05450545	0.002733913\\
0.05460546	0.002704099\\
0.05470547	0.002768946\\
0.05480548	0.00273391\\
0.05490549	0.002644037\\
0.0550055	0.002843187\\
0.05510551	0.002810305\\
0.05520552	0.002834158\\
0.05530553	0.002965283\\
0.05540554	0.002952573\\
0.05550555	0.003231828\\
0.05560556	0.003229889\\
0.05570557	0.003046842\\
0.05580558	0.002803214\\
0.05590559	0.002886716\\
0.0560056	0.002699803\\
0.05610561	0.00259251\\
0.05620562	0.002497497\\
0.05630563	0.00236598\\
0.05640564	0.00234427\\
0.05650565	0.002274305\\
0.05660566	0.00224567\\
0.05670567	0.002264785\\
0.05680568	0.002259639\\
0.05690569	0.002094986\\
0.0570057	0.002118752\\
0.05710571	0.002234975\\
0.05720572	0.002504049\\
0.05730573	0.002664596\\
0.05740574	0.002505782\\
0.05750575	0.00250284\\
0.05760576	0.002670157\\
0.05770577	0.002863322\\
0.05780578	0.002758817\\
0.05790579	0.002728493\\
0.0580058	0.002748634\\
0.05810581	0.002815257\\
0.05820582	0.002671019\\
0.05830583	0.002550152\\
0.05840584	0.002488231\\
0.05850585	0.002283651\\
0.05860586	0.002256381\\
0.05870587	0.002389128\\
0.05880588	0.0023761\\
0.05890589	0.002368781\\
0.0590059	0.002382495\\
0.05910591	0.002372575\\
0.05920592	0.002410086\\
0.05930593	0.002426543\\
0.05940594	0.002248589\\
0.05950595	0.002155983\\
0.05960596	0.001984698\\
0.05970597	0.001998551\\
0.05980598	0.002206848\\
0.05990599	0.002199414\\
0.060006	0.002093164\\
0.06010601	0.002218783\\
0.06020602	0.002242471\\
0.06030603	0.002223206\\
0.06040604	0.002018011\\
0.06050605	0.00193467\\
0.06060606	0.001796589\\
0.06070607	0.001996639\\
0.06080608	0.002041163\\
0.06090609	0.001991367\\
0.0610061	0.001959344\\
0.06110611	0.001932947\\
0.06120612	0.002032873\\
0.06130613	0.001997857\\
0.06140614	0.001978261\\
0.06150615	0.002059095\\
0.06160616	0.001963452\\
0.06170617	0.00207171\\
0.06180618	0.002361791\\
0.06190619	0.002163765\\
0.0620062	0.002104117\\
0.06210621	0.002106156\\
0.06220622	0.002053985\\
0.06230623	0.002076208\\
0.06240624	0.002171138\\
0.06250625	0.002168706\\
0.06260626	0.002083409\\
0.06270627	0.001992857\\
0.06280628	0.001830893\\
0.06290629	0.001701871\\
0.0630063	0.001644819\\
0.06310631	0.001581419\\
0.06320632	0.001784408\\
0.06330633	0.001803032\\
0.06340634	0.001932399\\
0.06350635	0.001920145\\
0.06360636	0.00196842\\
0.06370637	0.002053729\\
0.06380638	0.002240904\\
0.06390639	0.002041307\\
0.0640064	0.001892587\\
0.06410641	0.001637451\\
0.06420642	0.001718805\\
0.06430643	0.001781795\\
0.06440644	0.001669173\\
0.06450645	0.001589179\\
0.06460646	0.00156879\\
0.06470647	0.00158998\\
0.06480648	0.001616001\\
0.06490649	0.001575271\\
0.0650065	0.001712378\\
0.06510651	0.001759014\\
0.06520652	0.001742739\\
0.06530653	0.001651579\\
0.06540654	0.001536079\\
0.06550655	0.001455519\\
0.06560656	0.001484478\\
0.06570657	0.001442868\\
0.06580658	0.001561539\\
0.06590659	0.001443403\\
0.0660066	0.001502358\\
0.06610661	0.001558735\\
0.06620662	0.001633075\\
0.06630663	0.00157995\\
0.06640664	0.001534228\\
0.06650665	0.00167251\\
0.06660666	0.001778973\\
0.06670667	0.001838002\\
0.06680668	0.001932532\\
0.06690669	0.001795086\\
0.0670067	0.001806466\\
0.06710671	0.001764944\\
0.06720672	0.001772846\\
0.06730673	0.001680197\\
0.06740674	0.001709618\\
0.06750675	0.001740619\\
0.06760676	0.001569594\\
0.06770677	0.001394041\\
0.06780678	0.001328509\\
0.06790679	0.001496255\\
0.0680068	0.0014177\\
0.06810681	0.001411976\\
0.06820682	0.001284749\\
0.06830683	0.001248329\\
0.06840684	0.001201505\\
0.06850685	0.001197113\\
0.06860686	0.001067194\\
0.06870687	0.0009809874\\
0.06880688	0.001180799\\
0.06890689	0.001501283\\
0.0690069	0.001503094\\
0.06910691	0.00130136\\
0.06920692	0.001515732\\
0.06930693	0.001614412\\
0.06940694	0.001653411\\
0.06950695	0.001532822\\
0.06960696	0.00156897\\
0.06970697	0.001594978\\
0.06980698	0.001327413\\
0.06990699	0.001117064\\
0.070007	0.0010896\\
0.07010701	0.001151534\\
0.07020702	0.00109173\\
0.07030703	0.001236969\\
0.07040704	0.001220059\\
0.07050705	0.001164747\\
0.07060706	0.001343049\\
0.07070707	0.001316647\\
0.07080708	0.001350395\\
0.07090709	0.001350037\\
0.0710071	0.001304153\\
0.07110711	0.00140101\\
0.07120712	0.001492532\\
0.07130713	0.001541976\\
0.07140714	0.001369302\\
0.07150715	0.001307207\\
0.07160716	0.001307626\\
0.07170717	0.001320234\\
0.07180718	0.001130304\\
0.07190719	0.00115291\\
0.0720072	0.001065816\\
0.07210721	0.001136576\\
0.07220722	0.001127326\\
0.07230723	0.001103261\\
0.07240724	0.001045486\\
0.07250725	0.001078626\\
0.07260726	0.001156773\\
0.07270727	0.001209118\\
0.07280728	0.001127894\\
0.07290729	0.001183338\\
0.0730073	0.001161001\\
0.07310731	0.001101548\\
0.07320732	0.001119629\\
0.07330733	0.001125708\\
0.07340734	0.001116986\\
0.07350735	0.001023456\\
0.07360736	0.001008934\\
0.07370737	0.0009649225\\
0.07380738	0.0009787418\\
0.07390739	0.0009369257\\
0.0740074	0.000954921\\
0.07410741	0.0009730922\\
0.07420742	0.001053994\\
0.07430743	0.001048891\\
0.07440744	0.0009349881\\
0.07450745	0.0009462137\\
0.07460746	0.001014707\\
0.07470747	0.001047297\\
0.07480748	0.001070986\\
0.07490749	0.001058692\\
0.0750075	0.001104565\\
0.07510751	0.001143096\\
0.07520752	0.001010357\\
0.07530753	0.0009884484\\
0.07540754	0.001141689\\
0.07550755	0.001244522\\
0.07560756	0.001184422\\
0.07570757	0.001071182\\
0.07580758	0.001222946\\
0.07590759	0.001316661\\
0.0760076	0.001161674\\
0.07610761	0.0008768073\\
0.07620762	0.0008953024\\
0.07630763	0.001093369\\
0.07640764	0.001124688\\
0.07650765	0.0009000488\\
0.07660766	0.0008052823\\
0.07670767	0.0007670848\\
0.07680768	0.0006888846\\
0.07690769	0.0006612707\\
0.0770077	0.0007361883\\
0.07710771	0.0007919448\\
0.07720772	0.0007025787\\
0.07730773	0.000780165\\
0.07740774	0.0008421752\\
0.07750775	0.000884049\\
0.07760776	0.0009612682\\
0.07770777	0.0009393991\\
0.07780778	0.000860494\\
0.07790779	0.0008838822\\
0.0780078	0.000909071\\
0.07810781	0.0008710655\\
0.07820782	0.0008558885\\
0.07830783	0.0008842407\\
0.07840784	0.0008439743\\
0.07850785	0.0009203671\\
0.07860786	0.0007930433\\
0.07870787	0.0007780339\\
0.07880788	0.0008299308\\
0.07890789	0.0008496115\\
0.0790079	0.000914926\\
0.07910791	0.0009977589\\
0.07920792	0.0009973923\\
0.07930793	0.0008498898\\
0.07940794	0.000767312\\
0.07950795	0.0008104733\\
0.07960796	0.0007855899\\
0.07970797	0.0008058593\\
0.07980798	0.00070093\\
0.07990799	0.000731751\\
0.080008	0.0007555885\\
0.08010801	0.0006497741\\
0.08020802	0.0006434299\\
0.08030803	0.0007946998\\
0.08040804	0.000849674\\
0.08050805	0.0008899058\\
0.08060806	0.0008955598\\
0.08070807	0.0008998896\\
0.08080808	0.000951929\\
0.08090809	0.0008426006\\
0.0810081	0.0007626101\\
0.08110811	0.00076264\\
0.08120812	0.0007941963\\
0.08130813	0.0007449969\\
0.08140814	0.0007650779\\
0.08150815	0.0006685281\\
0.08160816	0.0006538237\\
0.08170817	0.0006036514\\
0.08180818	0.000519616\\
0.08190819	0.0005027574\\
0.0820082	0.0004996426\\
0.08210821	0.0004505498\\
0.08220822	0.000545713\\
0.08230823	0.0006691671\\
0.08240824	0.0006869525\\
0.08250825	0.0007145338\\
0.08260826	0.0007585865\\
0.08270827	0.000780003\\
0.08280828	0.0007659221\\
0.08290829	0.0007639468\\
0.0830083	0.0006509054\\
0.08310831	0.000688231\\
0.08320832	0.0007541235\\
0.08330833	0.0005745899\\
0.08340834	0.0006132521\\
0.08350835	0.0006459514\\
0.08360836	0.0007151584\\
0.08370837	0.0006645439\\
0.08380838	0.0006943674\\
0.08390839	0.0007576039\\
0.0840084	0.0006941542\\
0.08410841	0.0006549954\\
0.08420842	0.000645115\\
0.08430843	0.00066934\\
0.08440844	0.0006277432\\
0.08450845	0.000604214\\
0.08460846	0.000658198\\
0.08470847	0.0007231711\\
0.08480848	0.000632125\\
0.08490849	0.0006047122\\
0.0850085	0.000683354\\
0.08510851	0.0008013122\\
0.08520852	0.0007461035\\
0.08530853	0.0006150955\\
0.08540854	0.0005491165\\
0.08550855	0.00057412\\
0.08560856	0.0003892717\\
0.08570857	0.0003200231\\
0.08580858	0.0003467168\\
0.08590859	0.0003345067\\
0.0860086	0.0004127892\\
0.08610861	0.000473253\\
0.08620862	0.0005146369\\
0.08630863	0.000528923\\
0.08640864	0.0005773587\\
0.08650865	0.0005668871\\
0.08660866	0.000595173\\
0.08670867	0.0005497511\\
0.08680868	0.0005783996\\
0.08690869	0.0007476953\\
0.0870087	0.0007845195\\
0.08710871	0.000610536\\
0.08720872	0.0005402758\\
0.08730873	0.0006052737\\
0.08740874	0.0005603943\\
0.08750875	0.0004601692\\
0.08760876	0.0004876186\\
0.08770877	0.0005179415\\
0.08780878	0.0005749436\\
0.08790879	0.0004721204\\
0.0880088	0.0005120626\\
0.08810881	0.0005044295\\
0.08820882	0.0004846912\\
0.08830883	0.0004321573\\
0.08840884	0.0005069835\\
0.08850885	0.0004854225\\
0.08860886	0.0005593259\\
0.08870887	0.0005067298\\
0.08880888	0.0004042348\\
0.08890889	0.0004340468\\
0.0890089	0.0004644995\\
0.08910891	0.000501016\\
0.08920892	0.0005613647\\
0.08930893	0.000552089\\
0.08940894	0.0005083832\\
0.08950895	0.00058583\\
0.08960896	0.0005243337\\
0.08970897	0.0004772152\\
0.08980898	0.0004888977\\
0.08990899	0.0005427072\\
0.090009	0.0004315433\\
0.09010901	0.0003588812\\
0.09020902	0.0003575113\\
0.09030903	0.0003906272\\
0.09040904	0.0003271767\\
0.09050905	0.0003514294\\
0.09060906	0.0004022206\\
0.09070907	0.0004200896\\
0.09080908	0.0004031773\\
0.09090909	0.0003930643\\
0.0910091	0.0003162622\\
0.09110911	0.0003284846\\
0.09120912	0.0003383815\\
0.09130913	0.0004171842\\
0.09140914	0.000437808\\
0.09150915	0.0003382735\\
0.09160916	0.0004053413\\
0.09170917	0.0004455895\\
0.09180918	0.000471399\\
0.09190919	0.0004616603\\
0.0920092	0.0004927528\\
0.09210921	0.0005897982\\
0.09220922	0.0005516271\\
0.09230923	0.0004770465\\
0.09240924	0.0004845091\\
0.09250925	0.000453494\\
0.09260926	0.0003859922\\
0.09270927	0.0003930486\\
0.09280928	0.0004008152\\
0.09290929	0.0003913126\\
0.0930093	0.0004719218\\
0.09310931	0.0005049672\\
0.09320932	0.0005113087\\
0.09330933	0.0004558629\\
0.09340934	0.0003526419\\
0.09350935	0.000304072\\
0.09360936	0.0002491351\\
0.09370937	0.0002541888\\
0.09380938	0.0002787425\\
0.09390939	0.0002920951\\
0.0940094	0.0003171202\\
0.09410941	0.0003003937\\
0.09420942	0.0003240092\\
0.09430943	0.0003420471\\
0.09440944	0.0002867048\\
0.09450945	0.0003247349\\
0.09460946	0.0003307265\\
0.09470947	0.0003597022\\
0.09480948	0.0003689019\\
0.09490949	0.0003701658\\
0.0950095	0.0003974529\\
0.09510951	0.0003596663\\
0.09520952	0.0002874466\\
0.09530953	0.0003004486\\
0.09540954	0.0003457463\\
0.09550955	0.0002943489\\
0.09560956	0.0002649512\\
0.09570957	0.0002193849\\
0.09580958	0.0002428233\\
0.09590959	0.0003007657\\
0.0960096	0.000371183\\
0.09610961	0.0004130805\\
0.09620962	0.0004870237\\
0.09630963	0.0004630716\\
0.09640964	0.0003672031\\
0.09650965	0.0004552271\\
0.09660966	0.0005675541\\
0.09670967	0.0004266081\\
0.09680968	0.0003030398\\
0.09690969	0.0002970065\\
0.0970097	0.0002712729\\
0.09710971	0.0002318649\\
0.09720972	0.0002499221\\
0.09730973	0.0002834431\\
0.09740974	0.000226265\\
0.09750975	0.000208654\\
0.09760976	0.0002658437\\
0.09770977	0.0002430822\\
0.09780978	0.0002474214\\
0.09790979	0.0002856024\\
0.0980098	0.0003257631\\
0.09810981	0.0003562947\\
0.09820982	0.0003894195\\
0.09830983	0.0003869254\\
0.09840984	0.0003333395\\
0.09850985	0.0003349105\\
0.09860986	0.0003331662\\
0.09870987	0.0002950022\\
0.09880988	0.0002528589\\
0.09890989	0.0003055484\\
0.0990099	0.0002917841\\
0.09910991	0.0001645628\\
0.09920992	0.0001441605\\
0.09930993	0.0002159751\\
0.09940994	0.0002246502\\
0.09950995	0.0002375859\\
0.09960996	0.000174235\\
0.09970997	0.000147128\\
0.09980998	0.0001516428\\
0.09990999	0.0001829799\\
0.10001	0.0002457809\\
0.10011	0.0002410658\\
0.10021	0.000244388\\
0.10031	0.0003404709\\
0.10041	0.0003870653\\
0.1005101	0.0003757715\\
0.1006101	0.000372733\\
0.1007101	0.0003688378\\
0.1008101	0.0003236815\\
0.1009101	0.0003047345\\
0.1010101	0.0003089025\\
0.1011101	0.0003046025\\
0.1012101	0.0002900064\\
0.1013101	0.0003127349\\
0.1014101	0.0002770593\\
0.1015102	0.0002479533\\
0.1016102	0.0002974073\\
0.1017102	0.0002183094\\
0.1018102	0.0001744761\\
0.1019102	0.0002340673\\
0.1020102	0.0001869042\\
0.1021102	0.0002259741\\
0.1022102	0.0002424325\\
0.1023102	0.0001787233\\
0.1024102	0.0001723447\\
0.1025103	0.000171769\\
0.1026103	0.0002151309\\
0.1027103	0.0002688395\\
0.1028103	0.0001997164\\
0.1029103	0.0001614063\\
0.1030103	0.0001263092\\
0.1031103	0.0001148933\\
0.1032103	0.0001331078\\
0.1033103	0.0001764134\\
0.1034103	0.0001783901\\
0.1035104	0.0002371345\\
0.1036104	0.0002469402\\
0.1037104	0.0002341161\\
0.1038104	0.0002808939\\
0.1039104	0.0002624409\\
0.1040104	0.0002633967\\
0.1041104	0.000211065\\
0.1042104	0.0002693361\\
0.1043104	0.0002474548\\
0.1044104	0.0002382144\\
0.1045105	0.0001972028\\
0.1046105	0.0001547556\\
0.1047105	0.0001339477\\
0.1048105	0.0001143501\\
0.1049105	0.0001287976\\
0.1050105	0.0002049354\\
0.1051105	0.0002325712\\
0.1052105	0.0002594732\\
0.1053105	0.0003049342\\
0.1054105	0.000285484\\
0.1055106	0.0002407369\\
0.1056106	0.0002237744\\
0.1057106	0.0002477049\\
0.1058106	0.0001611484\\
0.1059106	0.0001452653\\
0.1060106	0.0001312221\\
0.1061106	0.0001513431\\
0.1062106	0.0001528723\\
0.1063106	0.0001794423\\
0.1064106	0.0002238068\\
0.1065107	0.0002239951\\
0.1066107	0.0002190533\\
0.1067107	0.0002126014\\
0.1068107	0.0001513278\\
0.1069107	0.0001447735\\
0.1070107	0.0001748897\\
0.1071107	0.0001640082\\
0.1072107	0.0001572347\\
0.1073107	0.0001151167\\
0.1074107	0.0001155071\\
0.1075108	0.0001165962\\
0.1076108	0.0001812576\\
0.1077108	0.0001322886\\
0.1078108	0.0001378733\\
0.1079108	0.0001527227\\
0.1080108	0.0001502729\\
0.1081108	0.0001745085\\
0.1082108	0.0001634341\\
0.1083108	0.0001288394\\
0.1084108	0.0001389737\\
0.1085109	0.0001674897\\
0.1086109	0.0001853646\\
0.1087109	0.0001211291\\
0.1088109	0.0001553647\\
0.1089109	0.0001551714\\
0.1090109	0.0001110637\\
0.1091109	0.0001226049\\
0.1092109	0.0001416696\\
0.1093109	0.0001455569\\
0.1094109	0.0001339651\\
0.109511	0.0002172473\\
0.109611	0.000257557\\
0.109711	0.0002368339\\
0.109811	0.0002453466\\
0.109911	0.0002098713\\
0.110011	0.000179587\\
0.110111	0.0002318867\\
0.110211	0.0002158151\\
0.110311	0.0002470337\\
0.110411	0.0001963563\\
0.1105111	0.0001830456\\
0.1106111	0.0001388239\\
0.1107111	0.0001590598\\
0.1108111	0.0001179628\\
0.1109111	7.628033e-05\\
0.1110111	6.835693e-05\\
0.1111111	6.969415e-05\\
0.1112111	7.308747e-05\\
0.1113111	7.369802e-05\\
0.1114111	3.498144e-05\\
0.1115112	4.994931e-05\\
0.1116112	8.533238e-05\\
0.1117112	9.913957e-05\\
0.1118112	0.000146959\\
0.1119112	0.0001839695\\
0.1120112	0.0001839826\\
0.1121112	0.0002011051\\
0.1122112	0.0002586874\\
0.1123112	0.0002249501\\
0.1124112	0.0002364518\\
0.1125113	0.0001596547\\
0.1126113	0.0001154637\\
0.1127113	0.0001244012\\
0.1128113	0.0001057602\\
0.1129113	8.181185e-05\\
0.1130113	8.125224e-05\\
0.1131113	9.815883e-05\\
0.1132113	0.0001001123\\
0.1133113	0.0001167727\\
0.1134113	0.0001362699\\
0.1135114	0.0001308674\\
0.1136114	0.0001536051\\
0.1137114	0.0001731659\\
0.1138114	0.0001316843\\
0.1139114	9.506546e-05\\
0.1140114	0.0001383467\\
0.1141114	0.0001411878\\
0.1142114	0.0001286728\\
0.1143114	0.000103764\\
0.1144114	8.336658e-05\\
0.1145115	0.0001245693\\
0.1146115	0.0001599251\\
0.1147115	0.0001056959\\
0.1148115	0.0001220673\\
0.1149115	0.0001002202\\
0.1150115	0.000100592\\
0.1151115	0.000115213\\
0.1152115	0.0001488677\\
0.1153115	0.0001320547\\
0.1154115	9.312994e-05\\
0.1155116	8.785683e-05\\
0.1156116	7.798844e-05\\
0.1157116	0.0001396018\\
0.1158116	0.0001068272\\
0.1159116	7.270833e-05\\
0.1160116	5.638938e-05\\
0.1161116	5.253479e-05\\
0.1162116	6.344179e-05\\
0.1163116	7.768159e-05\\
0.1164116	6.663263e-05\\
0.1165117	8.907191e-05\\
0.1166117	0.0001247516\\
0.1167117	0.0001215134\\
0.1168117	9.00876e-05\\
0.1169117	9.271846e-05\\
0.1170117	8.012573e-05\\
0.1171117	5.778744e-05\\
0.1172117	5.851585e-05\\
0.1173117	9.079327e-05\\
0.1174117	9.291804e-05\\
0.1175118	8.914647e-05\\
0.1176118	9.156941e-05\\
0.1177118	0.0001548325\\
0.1178118	0.0001484153\\
0.1179118	0.0001245496\\
0.1180118	0.0001641983\\
0.1181118	0.0001177192\\
0.1182118	0.0001050112\\
0.1183118	0.000130316\\
0.1184118	0.000126778\\
0.1185119	0.0001192886\\
0.1186119	0.0001226804\\
0.1187119	7.962574e-05\\
0.1188119	5.610059e-05\\
0.1189119	6.028737e-05\\
0.1190119	9.540916e-05\\
0.1191119	0.0001174459\\
0.1192119	0.0001210442\\
0.1193119	8.892503e-05\\
0.1194119	0.0001003403\\
0.119512	8.125999e-05\\
0.119612	6.45152e-05\\
0.119712	6.217849e-05\\
0.119812	5.534132e-05\\
0.119912	5.651687e-05\\
0.120012	7.249881e-05\\
0.120112	5.670191e-05\\
0.120212	4.006544e-05\\
0.120312	3.405915e-05\\
0.120412	2.788833e-05\\
0.1205121	4.385601e-05\\
0.1206121	6.537113e-05\\
0.1207121	5.264492e-05\\
0.1208121	6.859674e-05\\
0.1209121	9.778951e-05\\
0.1210121	8.88784e-05\\
0.1211121	0.0001410804\\
0.1212121	0.0001227997\\
0.1213121	0.0001294422\\
0.1214121	0.0001401431\\
0.1215122	0.0001190495\\
0.1216122	8.747754e-05\\
0.1217122	0.0001038017\\
0.1218122	7.43547e-05\\
0.1219122	7.824951e-05\\
0.1220122	8.587082e-05\\
0.1221122	9.539535e-05\\
0.1222122	8.090865e-05\\
0.1223122	4.355359e-05\\
0.1224122	5.532059e-05\\
0.1225123	4.476742e-05\\
0.1226123	4.707219e-05\\
0.1227123	9.119312e-05\\
0.1228123	8.753455e-05\\
0.1229123	7.375932e-05\\
0.1230123	6.568932e-05\\
0.1231123	5.544791e-05\\
0.1232123	4.748162e-05\\
0.1233123	5.441328e-05\\
0.1234123	3.711952e-05\\
0.1235124	2.24949e-05\\
0.1236124	7.374288e-05\\
0.1237124	0.0001222923\\
0.1238124	0.0001089178\\
0.1239124	0.0001506718\\
0.1240124	0.000103669\\
0.1241124	0.0001051823\\
0.1242124	0.000114503\\
0.1243124	0.0001162666\\
0.1244124	8.389533e-05\\
0.1245125	4.86197e-05\\
0.1246125	3.231958e-05\\
0.1247125	4.005187e-05\\
0.1248125	1.81724e-05\\
0.1249125	1.362834e-06\\
0.1250125	1.017259e-05\\
0.1251125	1.959949e-05\\
0.1252125	6.052862e-05\\
0.1253125	6.036962e-05\\
0.1254125	6.473057e-05\\
0.1255126	8.041232e-05\\
0.1256126	8.230674e-05\\
0.1257126	7.834958e-05\\
0.1258126	8.303722e-05\\
0.1259126	9.38621e-05\\
0.1260126	0.0001152791\\
0.1261126	0.0001040963\\
0.1262126	5.559311e-05\\
0.1263126	6.424034e-05\\
0.1264126	7.404275e-05\\
0.1265127	6.495282e-05\\
0.1266127	7.154888e-05\\
0.1267127	4.664892e-05\\
0.1268127	6.482145e-05\\
0.1269127	8.540007e-05\\
0.1270127	6.515724e-05\\
0.1271127	3.745181e-05\\
0.1272127	3.984255e-05\\
0.1273127	5.848135e-05\\
0.1274127	6.787543e-05\\
0.1275128	6.271341e-05\\
0.1276128	4.955301e-05\\
0.1277128	5.493338e-05\\
0.1278128	8.120271e-05\\
0.1279128	8.386168e-05\\
0.1280128	8.247244e-05\\
0.1281128	5.94698e-05\\
0.1282128	4.926985e-05\\
0.1283128	3.041778e-05\\
0.1284128	1.805968e-05\\
0.1285129	2.609003e-05\\
0.1286129	1.634611e-05\\
0.1287129	1.730418e-05\\
0.1288129	1.88455e-05\\
0.1289129	8.540364e-06\\
0.1290129	2.123154e-05\\
0.1291129	4.45542e-05\\
0.1292129	3.987005e-05\\
0.1293129	5.48769e-05\\
0.1294129	6.89635e-05\\
0.129513	9.104843e-05\\
0.129613	9.901879e-05\\
0.129713	2.944111e-05\\
0.129813	5.659922e-05\\
0.129913	4.893707e-05\\
0.130013	5.568408e-05\\
0.130113	5.126538e-05\\
0.130213	3.882629e-05\\
0.130313	6.27159e-05\\
0.130413	4.485289e-05\\
0.1305131	3.430638e-05\\
0.1306131	4.632272e-05\\
0.1307131	3.640517e-05\\
0.1308131	5.277837e-05\\
0.1309131	7.660333e-05\\
0.1310131	6.008988e-05\\
0.1311131	5.16342e-05\\
0.1312131	4.946807e-05\\
0.1313131	4.287452e-05\\
0.1314131	7.109099e-05\\
0.1315132	8.263312e-05\\
0.1316132	2.999292e-05\\
0.1317132	2.479314e-05\\
0.1318132	1.552334e-05\\
0.1319132	1.24898e-05\\
0.1320132	7.672551e-06\\
0.1321132	2.246115e-05\\
0.1322132	3.730875e-05\\
0.1323132	6.983097e-05\\
0.1324132	6.665789e-05\\
0.1325133	4.504008e-05\\
0.1326133	6.305406e-05\\
0.1327133	6.580732e-05\\
0.1328133	4.497887e-05\\
0.1329133	3.908706e-05\\
0.1330133	3.282982e-05\\
0.1331133	2.633638e-05\\
0.1332133	2.923598e-05\\
0.1333133	2.281597e-05\\
0.1334133	2.932609e-05\\
0.1335134	5.277032e-05\\
0.1336134	4.724905e-05\\
0.1337134	1.665515e-05\\
0.1338134	9.325223e-06\\
0.1339134	1.658138e-05\\
0.1340134	2.146583e-05\\
0.1341134	9.566872e-06\\
0.1342134	1.502397e-05\\
0.1343134	2.19449e-05\\
0.1344134	3.839058e-05\\
0.1345135	1.59577e-05\\
0.1346135	3.060848e-05\\
0.1347135	2.184014e-05\\
0.1348135	6.438725e-05\\
0.1349135	0.000112057\\
0.1350135	9.806113e-05\\
0.1351135	7.300279e-05\\
0.1352135	9.672767e-05\\
0.1353135	8.994574e-05\\
0.1354135	5.978721e-05\\
0.1355136	5.981295e-05\\
0.1356136	6.7338e-05\\
0.1357136	9.157563e-05\\
0.1358136	8.239082e-05\\
0.1359136	4.157133e-05\\
0.1360136	3.1021e-05\\
0.1361136	1.784793e-05\\
0.1362136	1.248765e-05\\
0.1363136	8.552034e-06\\
0.1364136	1.687492e-05\\
0.1365137	2.337505e-05\\
0.1366137	8.355474e-06\\
0.1367137	1.539424e-05\\
0.1368137	9.070945e-06\\
0.1369137	1.938996e-05\\
0.1370137	3.368342e-05\\
0.1371137	1.28399e-05\\
0.1372137	3.958045e-05\\
0.1373137	1.719606e-05\\
0.1374137	1.703235e-05\\
0.1375138	4.373212e-05\\
0.1376138	1.978962e-05\\
0.1377138	2.468422e-05\\
0.1378138	1.85639e-05\\
0.1379138	2.834662e-05\\
0.1380138	2.679653e-05\\
0.1381138	4.157191e-05\\
0.1382138	4.69349e-05\\
0.1383138	2.154599e-05\\
0.1384138	1.454827e-05\\
0.1385139	2.932877e-05\\
0.1386139	2.51049e-05\\
0.1387139	2.202689e-05\\
0.1388139	1.16651e-05\\
0.1389139	3.081799e-05\\
0.1390139	4.444628e-05\\
0.1391139	1.578595e-05\\
0.1392139	3.031491e-05\\
0.1393139	4.273242e-05\\
0.1394139	6.084221e-05\\
0.139514	7.16272e-05\\
0.139614	4.370832e-05\\
0.139714	5.765393e-05\\
0.139814	4.436943e-05\\
0.139914	4.233289e-05\\
0.140014	3.955066e-05\\
0.140114	1.66152e-05\\
0.140214	3.263687e-05\\
0.140314	1.867677e-05\\
0.140414	1.608403e-05\\
0.1405141	1.88219e-05\\
0.1406141	1.050177e-05\\
0.1407141	9.319622e-06\\
0.1408141	1.435135e-05\\
0.1409141	1.749842e-05\\
0.1410141	3.232712e-05\\
0.1411141	3.461699e-05\\
0.1412141	2.838136e-05\\
0.1413141	1.967699e-05\\
0.1414141	3.932322e-05\\
0.1415142	4.249414e-05\\
0.1416142	5.373531e-05\\
0.1417142	5.542926e-05\\
0.1418142	4.414993e-05\\
0.1419142	4.761814e-05\\
0.1420142	1.920378e-05\\
0.1421142	4.293381e-06\\
0.1422142	2.081403e-06\\
0.1423142	2.035077e-06\\
0.1424142	1.042924e-06\\
0.1425143	6.099915e-06\\
0.1426143	1.683691e-05\\
0.1427143	3.770773e-05\\
0.1428143	4.583501e-05\\
0.1429143	2.959687e-05\\
0.1430143	3.192156e-05\\
0.1431143	1.906834e-05\\
0.1432143	2.689203e-05\\
0.1433143	2.232485e-05\\
0.1434143	2.132136e-05\\
0.1435144	2.245204e-05\\
0.1436144	2.492182e-05\\
0.1437144	1.710094e-05\\
0.1438144	1.573705e-05\\
0.1439144	4.353425e-05\\
0.1440144	7.013507e-05\\
0.1441144	5.017187e-05\\
0.1442144	2.123202e-05\\
0.1443144	2.701175e-05\\
0.1444144	3.679467e-05\\
0.1445145	6.519417e-05\\
0.1446145	4.111604e-05\\
0.1447145	2.247069e-05\\
0.1448145	1.870848e-05\\
0.1449145	2.658247e-05\\
0.1450145	1.642048e-05\\
0.1451145	2.2773e-05\\
0.1452145	1.371928e-05\\
0.1453145	1.757634e-05\\
0.1454145	5.742714e-06\\
0.1455146	3.411322e-06\\
0.1456146	1.435115e-05\\
0.1457146	2.160121e-06\\
0.1458146	1.540189e-06\\
0.1459146	2.043444e-06\\
0.1460146	8.49611e-06\\
0.1461146	8.747587e-06\\
0.1462146	1.199001e-05\\
0.1463146	2.69062e-05\\
0.1464146	1.26386e-05\\
0.1465147	1.211818e-05\\
0.1466147	3.472231e-05\\
0.1467147	4.695762e-05\\
0.1468147	4.31591e-05\\
0.1469147	3.364523e-05\\
0.1470147	3.278253e-05\\
0.1471147	7.256825e-05\\
0.1472147	4.644113e-05\\
0.1473147	2.989724e-05\\
0.1474147	2.264943e-05\\
0.1475148	5.689271e-06\\
0.1476148	7.209976e-06\\
0.1477148	3.00246e-06\\
0.1478148	9.810733e-06\\
0.1479148	9.927061e-06\\
0.1480148	1.994639e-05\\
0.1481148	1.329523e-05\\
0.1482148	1.2801e-05\\
0.1483148	3.754819e-05\\
0.1484148	3.235224e-05\\
0.1485149	2.086617e-05\\
0.1486149	5.505694e-06\\
0.1487149	4.988223e-06\\
0.1488149	1.697511e-05\\
0.1489149	1.989736e-05\\
0.1490149	4.270563e-05\\
0.1491149	2.901529e-05\\
0.1492149	2.910145e-05\\
0.1493149	2.075372e-05\\
0.1494149	1.85306e-05\\
0.149515	8.780812e-06\\
0.149615	7.337665e-06\\
0.149715	1.283733e-05\\
0.149815	1.33132e-05\\
0.149915	2.528943e-05\\
0.150015	9.976535e-06\\
0.150115	8.469798e-06\\
0.150215	1.845968e-05\\
0.150315	1.077445e-06\\
0.150415	7.376223e-06\\
0.1505151	1.587964e-05\\
0.1506151	1.242777e-05\\
0.1507151	9.714349e-06\\
0.1508151	1.039811e-05\\
0.1509151	1.41667e-05\\
0.1510151	2.372321e-05\\
0.1511151	1.772001e-05\\
0.1512151	6.548972e-06\\
0.1513151	6.784766e-06\\
0.1514151	9.926895e-06\\
0.1515152	1.734637e-05\\
0.1516152	8.704619e-06\\
0.1517152	1.640694e-05\\
0.1518152	1.141154e-05\\
0.1519152	8.651711e-06\\
0.1520152	4.916692e-06\\
0.1521152	1.568304e-05\\
0.1522152	1.351931e-05\\
0.1523152	6.115417e-06\\
0.1524152	1.386184e-05\\
0.1525153	2.733501e-05\\
0.1526153	3.273933e-05\\
0.1527153	3.460528e-05\\
0.1528153	3.138748e-05\\
0.1529153	2.570238e-05\\
0.1530153	3.023515e-05\\
0.1531153	2.147599e-05\\
0.1532153	3.787033e-05\\
0.1533153	3.099707e-05\\
0.1534153	3.218713e-05\\
0.1535154	1.462253e-05\\
0.1536154	6.705979e-06\\
0.1537154	6.437613e-06\\
0.1538154	1.405958e-05\\
0.1539154	1.09025e-05\\
0.1540154	2.082857e-05\\
0.1541154	8.780554e-06\\
0.1542154	1.20217e-06\\
0.1543154	4.879859e-06\\
0.1544154	7.863813e-07\\
0.1545155	5.299166e-06\\
0.1546155	1.113452e-05\\
0.1547155	1.187893e-06\\
0.1548155	1.258288e-05\\
0.1549155	1.136017e-05\\
0.1550155	1.639878e-05\\
0.1551155	3.376625e-05\\
0.1552155	4.807668e-05\\
0.1553155	4.311208e-05\\
0.1554155	3.821409e-05\\
0.1555156	4.211487e-05\\
0.1556156	2.401741e-05\\
0.1557156	3.948814e-06\\
0.1558156	9.216019e-06\\
0.1559156	1.284061e-05\\
0.1560156	1.121274e-05\\
0.1561156	9.274215e-06\\
0.1562156	1.080507e-05\\
0.1563156	9.344274e-06\\
0.1564156	2.239928e-06\\
0.1565157	7.790583e-06\\
0.1566157	1.358575e-05\\
0.1567157	1.16639e-05\\
0.1568157	5.695764e-06\\
0.1569157	1.982656e-05\\
0.1570157	2.21456e-05\\
0.1571157	7.905044e-06\\
0.1572157	8.331455e-06\\
0.1573157	7.351566e-06\\
0.1574157	2.46273e-05\\
0.1575158	2.747052e-05\\
0.1576158	2.052715e-05\\
0.1577158	1.11136e-05\\
0.1578158	3.432038e-06\\
0.1579158	1.052208e-06\\
0.1580158	7.26592e-06\\
0.1581158	4.355595e-06\\
0.1582158	7.817031e-06\\
0.1583158	6.643731e-06\\
0.1584158	1.557985e-05\\
0.1585159	3.052433e-05\\
0.1586159	4.36234e-05\\
0.1587159	2.379989e-05\\
0.1588159	2.155913e-05\\
0.1589159	2.15254e-05\\
0.1590159	1.763843e-05\\
0.1591159	1.551033e-05\\
0.1592159	6.38167e-06\\
0.1593159	1.385997e-07\\
0.1594159	3.822373e-06\\
0.159516	7.936648e-07\\
0.159616	2.684902e-06\\
0.159716	5.501016e-06\\
0.159816	3.600809e-06\\
0.159916	1.235101e-05\\
0.160016	1.730321e-05\\
0.160116	1.255093e-05\\
0.160216	2.232275e-05\\
0.160316	2.933421e-05\\
0.160416	2.290341e-05\\
0.1605161	1.564367e-05\\
0.1606161	3.386553e-05\\
0.1607161	2.534242e-05\\
0.1608161	2.498682e-05\\
0.1609161	7.426672e-06\\
0.1610161	1.698255e-05\\
0.1611161	2.194162e-05\\
0.1612161	4.430575e-06\\
0.1613161	8.331721e-06\\
0.1614161	9.460835e-06\\
0.1615162	1.816892e-05\\
0.1616162	2.585135e-06\\
0.1617162	8.414561e-06\\
0.1618162	5.455315e-06\\
0.1619162	8.397832e-06\\
0.1620162	1.54125e-05\\
0.1621162	2.211571e-05\\
0.1622162	1.186808e-05\\
0.1623162	1.086876e-06\\
0.1624162	3.787196e-06\\
0.1625163	5.550525e-06\\
0.1626163	1.038254e-05\\
0.1627163	5.576548e-06\\
0.1628163	4.284876e-06\\
0.1629163	1.633014e-05\\
0.1630163	1.38048e-05\\
0.1631163	5.811688e-06\\
0.1632163	2.606711e-06\\
0.1633163	4.931207e-07\\
0.1634163	5.932372e-06\\
0.1635164	3.976849e-07\\
0.1636164	1.669281e-06\\
0.1637164	1.16776e-07\\
0.1638164	3.098302e-06\\
0.1639164	9.569829e-06\\
0.1640164	2.464476e-05\\
0.1641164	1.811825e-05\\
0.1642164	2.466827e-05\\
0.1643164	2.288473e-05\\
0.1644164	5.915264e-06\\
0.1645165	5.324474e-06\\
0.1646165	1.988886e-05\\
0.1647165	3.205995e-05\\
0.1648165	1.705214e-05\\
0.1649165	2.607231e-05\\
0.1650165	8.331408e-06\\
0.1651165	1.285723e-05\\
0.1652165	6.433162e-06\\
0.1653165	4.428372e-06\\
0.1654165	2.482237e-06\\
0.1655166	5.75223e-06\\
0.1656166	5.029594e-06\\
0.1657166	8.39169e-06\\
0.1658166	1.104396e-05\\
0.1659166	1.397306e-06\\
0.1660166	2.673587e-06\\
0.1661166	1.523598e-06\\
0.1662166	7.035948e-07\\
0.1663166	1.536085e-05\\
0.1664166	4.774271e-06\\
0.1665167	9.552619e-06\\
0.1666167	1.005925e-05\\
0.1667167	1.720085e-05\\
0.1668167	1.84123e-05\\
0.1669167	1.345842e-05\\
0.1670167	2.346116e-05\\
0.1671167	1.432271e-05\\
0.1672167	1.841779e-05\\
0.1673167	1.118895e-05\\
0.1674167	8.900175e-06\\
0.1675168	9.108163e-06\\
0.1676168	1.323253e-05\\
0.1677168	1.47583e-05\\
0.1678168	4.350143e-06\\
0.1679168	8.888029e-07\\
0.1680168	4.803618e-07\\
0.1681168	7.211359e-08\\
0.1682168	2.263754e-06\\
0.1683168	2.048983e-06\\
0.1684168	2.094972e-06\\
0.1685169	5.315963e-06\\
0.1686169	1.81908e-05\\
0.1687169	7.819015e-06\\
0.1688169	3.378089e-06\\
0.1689169	1.652272e-06\\
0.1690169	6.048121e-06\\
0.1691169	4.138479e-06\\
0.1692169	2.47376e-06\\
0.1693169	3.350669e-06\\
0.1694169	1.196443e-05\\
0.169517	2.054327e-05\\
0.169617	1.654585e-05\\
0.169717	1.747949e-05\\
0.169817	3.216791e-05\\
0.169917	4.086344e-05\\
0.170017	1.198455e-05\\
0.170117	6.704065e-06\\
0.170217	7.087142e-06\\
0.170317	1.03868e-05\\
0.170417	4.149349e-06\\
0.1705171	9.964637e-06\\
0.1706171	8.589264e-06\\
0.1707171	3.825977e-06\\
0.1708171	5.316156e-07\\
0.1709171	7.56572e-07\\
0.1710171	1.178239e-05\\
0.1711171	1.070953e-05\\
0.1712171	5.519318e-06\\
0.1713171	8.84371e-06\\
0.1714171	2.542415e-06\\
0.1715172	3.140655e-06\\
0.1716172	9.900116e-07\\
0.1717172	3.143819e-06\\
0.1718172	1.005432e-06\\
0.1719172	3.560003e-07\\
0.1720172	4.110607e-07\\
0.1721172	1.09009e-06\\
0.1722172	5.666376e-06\\
0.1723172	3.786221e-06\\
0.1724172	5.006286e-06\\
0.1725173	1.580674e-05\\
0.1726173	2.050725e-05\\
0.1727173	1.408461e-05\\
0.1728173	1.283266e-05\\
0.1729173	3.358197e-06\\
0.1730173	1.006396e-05\\
0.1731173	4.431904e-06\\
0.1732173	6.161579e-06\\
0.1733173	8.682991e-06\\
0.1734173	1.065308e-06\\
0.1735174	2.816104e-07\\
0.1736174	1.225619e-07\\
0.1737174	5.575618e-07\\
0.1738174	1.909088e-06\\
0.1739174	1.702396e-05\\
0.1740174	1.569191e-05\\
0.1741174	1.330527e-05\\
0.1742174	4.688563e-05\\
0.1743174	3.814458e-05\\
0.1744174	3.417294e-05\\
0.1745175	1.561334e-05\\
0.1746175	8.131716e-06\\
0.1747175	8.93369e-06\\
0.1748175	5.521464e-06\\
0.1749175	6.290771e-06\\
0.1750175	8.15337e-06\\
0.1751175	5.631445e-06\\
0.1752175	8.06377e-08\\
0.1753175	1.777784e-06\\
0.1754175	2.315996e-07\\
0.1755176	6.105012e-06\\
0.1756176	8.695867e-06\\
0.1757176	1.036725e-05\\
0.1758176	9.725974e-06\\
0.1759176	1.179795e-05\\
0.1760176	8.46075e-06\\
0.1761176	1.447601e-05\\
0.1762176	1.26757e-05\\
0.1763176	9.254071e-06\\
0.1764176	2.090151e-05\\
0.1765177	8.262169e-06\\
0.1766177	3.424462e-06\\
0.1767177	1.021545e-06\\
0.1768177	2.249777e-06\\
0.1769177	2.794265e-06\\
0.1770177	4.389538e-06\\
0.1771177	3.708379e-06\\
0.1772177	2.569222e-06\\
0.1773177	2.679994e-06\\
0.1774177	1.114319e-06\\
0.1775178	8.535435e-06\\
0.1776178	1.725048e-05\\
0.1777178	9.159333e-06\\
0.1778178	1.478969e-05\\
0.1779178	2.443531e-05\\
0.1780178	3.055679e-05\\
0.1781178	1.394861e-05\\
0.1782178	9.276628e-06\\
0.1783178	3.332739e-06\\
0.1784178	8.451432e-07\\
0.1785179	6.853856e-06\\
0.1786179	1.4915e-05\\
0.1787179	2.155806e-05\\
0.1788179	1.884345e-05\\
0.1789179	2.327863e-05\\
0.1790179	4.617765e-05\\
0.1791179	2.942195e-05\\
0.1792179	1.850922e-05\\
0.1793179	9.422829e-06\\
0.1794179	1.566313e-05\\
0.179518	7.214541e-06\\
0.179618	4.064771e-07\\
0.179718	4.779524e-06\\
0.179818	8.868335e-07\\
0.179918	3.435782e-06\\
0.180018	1.308478e-06\\
0.180118	1.844332e-06\\
0.180218	4.631207e-07\\
0.180318	3.602013e-06\\
0.180418	2.916341e-06\\
0.1805181	9.749575e-07\\
0.1806181	1.316953e-06\\
0.1807181	2.191622e-07\\
0.1808181	1.08734e-06\\
0.1809181	9.3268e-06\\
0.1810181	6.364453e-06\\
0.1811181	7.621833e-07\\
0.1812181	4.860571e-06\\
0.1813181	8.125827e-06\\
0.1814181	1.298023e-05\\
0.1815182	5.705956e-06\\
0.1816182	6.282706e-06\\
0.1817182	6.084519e-07\\
0.1818182	3.813629e-08\\
0.1819182	1.155164e-06\\
0.1820182	7.0554e-07\\
0.1821182	4.887872e-07\\
0.1822182	6.888411e-06\\
0.1823182	1.215969e-05\\
0.1824182	3.425963e-06\\
0.1825183	3.811831e-06\\
0.1826183	1.049274e-05\\
0.1827183	4.784078e-06\\
0.1828183	5.279986e-06\\
0.1829183	3.486717e-06\\
0.1830183	1.301618e-06\\
0.1831183	3.585037e-06\\
0.1832183	1.212756e-05\\
0.1833183	8.88634e-06\\
0.1834183	2.491571e-06\\
0.1835184	9.360353e-07\\
0.1836184	1.033303e-06\\
0.1837184	2.251307e-07\\
0.1838184	5.085932e-06\\
0.1839184	1.186681e-06\\
0.1840184	1.196791e-06\\
0.1841184	1.249869e-07\\
0.1842184	4.524102e-07\\
0.1843184	1.791586e-06\\
0.1844184	7.759115e-06\\
0.1845185	7.914658e-06\\
0.1846185	2.960667e-06\\
0.1847185	3.597528e-06\\
0.1848185	5.479353e-06\\
0.1849185	1.652528e-06\\
0.1850185	3.040732e-06\\
0.1851185	1.403253e-05\\
0.1852185	3.601744e-06\\
0.1853185	2.215309e-06\\
0.1854185	3.829541e-06\\
0.1855186	7.341721e-06\\
0.1856186	2.940946e-06\\
0.1857186	2.589446e-07\\
0.1858186	3.70062e-07\\
0.1859186	1.117833e-05\\
0.1860186	3.943868e-06\\
0.1861186	3.358025e-06\\
0.1862186	4.517641e-06\\
0.1863186	3.579101e-06\\
0.1864186	2.388391e-06\\
0.1865187	1.335671e-06\\
0.1866187	8.499277e-06\\
0.1867187	4.53704e-06\\
0.1868187	2.88894e-06\\
0.1869187	3.227653e-06\\
0.1870187	1.533678e-06\\
0.1871187	2.362583e-06\\
0.1872187	5.795187e-06\\
0.1873187	2.268301e-05\\
0.1874187	1.34145e-05\\
0.1875188	7.578667e-06\\
0.1876188	7.467323e-07\\
0.1877188	7.53943e-07\\
0.1878188	4.351943e-06\\
0.1879188	6.405753e-06\\
0.1880188	7.604861e-06\\
0.1881188	1.200139e-05\\
0.1882188	8.323163e-06\\
0.1883188	1.405075e-05\\
0.1884188	1.129206e-05\\
0.1885189	1.132475e-05\\
0.1886189	1.291124e-05\\
0.1887189	2.12876e-05\\
0.1888189	1.192625e-05\\
0.1889189	6.730274e-06\\
0.1890189	1.24942e-05\\
0.1891189	3.170435e-06\\
0.1892189	9.531027e-06\\
0.1893189	1.073712e-05\\
0.1894189	1.442159e-05\\
0.189519	1.069829e-05\\
0.189619	9.828158e-06\\
0.189719	2.015212e-05\\
0.189819	1.758618e-05\\
0.189919	3.132878e-05\\
0.190019	2.882406e-05\\
0.190119	3.909712e-05\\
0.190219	2.767994e-05\\
0.190319	2.896743e-05\\
0.190419	1.31292e-05\\
0.1905191	8.261747e-06\\
0.1906191	4.981766e-06\\
0.1907191	8.168648e-06\\
0.1908191	7.327697e-06\\
0.1909191	5.145339e-06\\
0.1910191	5.067084e-06\\
0.1911191	2.344899e-06\\
0.1912191	3.637105e-06\\
0.1913191	4.222966e-06\\
0.1914191	1.597356e-06\\
0.1915192	5.234222e-06\\
0.1916192	1.001706e-05\\
0.1917192	8.650248e-06\\
0.1918192	9.781435e-07\\
0.1919192	3.035706e-06\\
0.1920192	4.310069e-06\\
0.1921192	1.813474e-06\\
0.1922192	5.270448e-06\\
0.1923192	1.442873e-05\\
0.1924192	8.187811e-06\\
0.1925193	4.422589e-06\\
0.1926193	2.937396e-06\\
0.1927193	4.574662e-07\\
0.1928193	6.358802e-06\\
0.1929193	1.211302e-05\\
0.1930193	3.316397e-06\\
0.1931193	8.45876e-06\\
0.1932193	1.603781e-05\\
0.1933193	9.95332e-07\\
0.1934193	1.876652e-07\\
0.1935194	6.502469e-06\\
0.1936194	3.080863e-06\\
0.1937194	3.486383e-06\\
0.1938194	5.875669e-06\\
0.1939194	6.484535e-06\\
0.1940194	1.613868e-06\\
0.1941194	2.341985e-06\\
0.1942194	5.344825e-06\\
0.1943194	1.390745e-05\\
0.1944194	9.53606e-06\\
0.1945195	5.639633e-06\\
0.1946195	4.20508e-06\\
0.1947195	3.095016e-06\\
0.1948195	7.031893e-06\\
0.1949195	1.509816e-06\\
0.1950195	1.276271e-07\\
0.1951195	7.438058e-07\\
0.1952195	4.10561e-06\\
0.1953195	1.180015e-06\\
0.1954195	2.660389e-06\\
0.1955196	7.551806e-06\\
0.1956196	8.068705e-06\\
0.1957196	1.438735e-05\\
0.1958196	2.074933e-06\\
0.1959196	3.31482e-06\\
0.1960196	2.116074e-06\\
0.1961196	7.875546e-07\\
0.1962196	5.9174e-06\\
0.1963196	3.566788e-06\\
0.1964196	1.480189e-05\\
0.1965197	1.190881e-05\\
0.1966197	3.505452e-06\\
0.1967197	5.160484e-06\\
0.1968197	2.939068e-06\\
0.1969197	8.876951e-06\\
0.1970197	7.987406e-06\\
0.1971197	2.599605e-06\\
0.1972197	9.228209e-06\\
0.1973197	6.618509e-06\\
0.1974197	5.761864e-06\\
0.1975198	5.614772e-06\\
0.1976198	2.136761e-06\\
0.1977198	4.245531e-07\\
0.1978198	1.390202e-06\\
0.1979198	4.986043e-06\\
0.1980198	5.219279e-08\\
0.1981198	2.435841e-06\\
0.1982198	1.752929e-06\\
0.1983198	4.587296e-06\\
0.1984198	9.538239e-08\\
0.1985199	4.070501e-07\\
0.1986199	1.976398e-07\\
0.1987199	1.349076e-06\\
0.1988199	9.235402e-06\\
0.1989199	7.336296e-06\\
0.1990199	1.549714e-05\\
0.1991199	3.263894e-06\\
0.1992199	1.238386e-06\\
0.1993199	2.68553e-07\\
0.1994199	2.987399e-06\\
0.19952	5.111251e-06\\
0.19962	9.476188e-06\\
0.19972	8.596943e-06\\
0.19982	1.733185e-06\\
0.19992	2.054313e-06\\
0.20002	1.106923e-07\\
0.20012	1.107316e-06\\
0.20022	4.025679e-06\\
0.20032	1.390377e-06\\
0.20042	7.469073e-06\\
0.2005201	4.000079e-06\\
0.2006201	9.734987e-07\\
0.2007201	7.668334e-07\\
0.2008201	5.27048e-06\\
0.2009201	7.825876e-06\\
0.2010201	1.418847e-07\\
0.2011201	3.557703e-08\\
0.2012201	5.437593e-06\\
0.2013201	2.5786e-06\\
0.2014201	6.500111e-06\\
0.2015202	3.121182e-07\\
0.2016202	1.166438e-05\\
0.2017202	1.506948e-05\\
0.2018202	9.963464e-06\\
0.2019202	1.019536e-05\\
0.2020202	6.148639e-06\\
0.2021202	6.948643e-06\\
0.2022202	1.343164e-05\\
0.2023202	2.758789e-05\\
0.2024202	1.095375e-05\\
0.2025203	3.183205e-06\\
0.2026203	6.873993e-06\\
0.2027203	9.652304e-06\\
0.2028203	8.942227e-06\\
0.2029203	1.116267e-05\\
0.2030203	3.511813e-06\\
0.2031203	2.042714e-08\\
0.2032203	7.541768e-07\\
0.2033203	1.037813e-05\\
0.2034203	9.613053e-06\\
0.2035204	1.378426e-05\\
0.2036204	7.720107e-06\\
0.2037204	8.051921e-06\\
0.2038204	2.468577e-05\\
0.2039204	2.063761e-05\\
0.2040204	9.36254e-06\\
0.2041204	6.21978e-06\\
0.2042204	1.643227e-05\\
0.2043204	1.993456e-05\\
0.2044204	1.124442e-05\\
0.2045205	5.58429e-06\\
0.2046205	1.456453e-06\\
0.2047205	1.330527e-06\\
0.2048205	2.353447e-06\\
0.2049205	2.211219e-06\\
0.2050205	2.475559e-07\\
0.2051205	1.131977e-05\\
0.2052205	8.405898e-06\\
0.2053205	1.136602e-06\\
0.2054205	7.015147e-07\\
0.2055206	2.464857e-06\\
0.2056206	3.490198e-06\\
0.2057206	3.1855e-06\\
0.2058206	5.119643e-06\\
0.2059206	1.50642e-05\\
0.2060206	8.972126e-06\\
0.2061206	1.286434e-05\\
0.2062206	3.435027e-06\\
0.2063206	6.484751e-06\\
0.2064206	6.02114e-06\\
0.2065207	4.77219e-06\\
0.2066207	1.288929e-05\\
0.2067207	9.7965e-06\\
0.2068207	6.190453e-06\\
0.2069207	1.256863e-05\\
0.2070207	1.601976e-05\\
0.2071207	7.114035e-06\\
0.2072207	7.783265e-06\\
0.2073207	2.334068e-05\\
0.2074207	2.785251e-05\\
0.2075208	1.193156e-05\\
0.2076208	5.319181e-06\\
0.2077208	2.861846e-06\\
0.2078208	2.944022e-06\\
0.2079208	2.851519e-06\\
0.2080208	7.288677e-06\\
0.2081208	4.760574e-07\\
0.2082208	1.511943e-06\\
0.2083208	8.87692e-07\\
0.2084208	6.287835e-07\\
0.2085209	1.147271e-06\\
0.2086209	1.416715e-05\\
0.2087209	1.067684e-05\\
0.2088209	1.228774e-05\\
0.2089209	1.9369e-07\\
0.2090209	4.852476e-06\\
0.2091209	4.442133e-06\\
0.2092209	7.892893e-06\\
0.2093209	8.447814e-07\\
0.2094209	1.200311e-06\\
0.209521	6.567255e-07\\
0.209621	3.152603e-06\\
0.209721	4.83703e-06\\
0.209821	8.268516e-07\\
0.209921	8.561238e-07\\
0.210021	3.575275e-06\\
0.210121	2.559247e-06\\
0.210221	3.492935e-06\\
0.210321	1.00891e-05\\
0.210421	6.276525e-06\\
0.2105211	6.32355e-06\\
0.2106211	4.73633e-06\\
0.2107211	2.603728e-06\\
0.2108211	3.307138e-06\\
0.2109211	3.763468e-06\\
0.2110211	1.38886e-06\\
0.2111211	8.003403e-06\\
0.2112211	5.857904e-06\\
0.2113211	8.674787e-06\\
0.2114211	5.743025e-06\\
0.2115212	8.173603e-07\\
0.2116212	3.481103e-06\\
0.2117212	6.867822e-06\\
0.2118212	1.108206e-05\\
0.2119212	1.08751e-05\\
0.2120212	5.083675e-06\\
0.2121212	1.0682e-08\\
0.2122212	4.115457e-06\\
0.2123212	2.911209e-06\\
0.2124212	8.138734e-07\\
0.2125213	5.812703e-06\\
0.2126213	8.562461e-06\\
0.2127213	3.123421e-06\\
0.2128213	1.44385e-06\\
0.2129213	5.592149e-06\\
0.2130213	9.091747e-06\\
0.2131213	2.403359e-06\\
0.2132213	6.398814e-06\\
0.2133213	5.855542e-06\\
0.2134213	6.28473e-06\\
0.2135214	1.206731e-05\\
0.2136214	1.292267e-05\\
0.2137214	4.433104e-06\\
0.2138214	7.660194e-06\\
0.2139214	2.211755e-05\\
0.2140214	1.243477e-05\\
0.2141214	3.553959e-06\\
0.2142214	8.063316e-06\\
0.2143214	6.816349e-06\\
0.2144214	6.280086e-06\\
0.2145215	1.23713e-05\\
0.2146215	3.640055e-06\\
0.2147215	3.698356e-06\\
0.2148215	1.201262e-05\\
0.2149215	8.685438e-06\\
0.2150215	2.490302e-06\\
0.2151215	1.670314e-06\\
0.2152215	6.18199e-06\\
0.2153215	4.403971e-06\\
0.2154215	1.850473e-06\\
0.2155216	2.696518e-06\\
0.2156216	1.96812e-06\\
0.2157216	4.249454e-07\\
0.2158216	3.024324e-06\\
0.2159216	1.009457e-06\\
0.2160216	2.203129e-06\\
0.2161216	2.662354e-06\\
0.2162216	1.170349e-05\\
0.2163216	6.038316e-06\\
0.2164216	4.285271e-06\\
0.2165217	6.746007e-07\\
0.2166217	1.658637e-06\\
0.2167217	3.591395e-06\\
0.2168217	9.048614e-06\\
0.2169217	9.679503e-07\\
0.2170217	1.20031e-06\\
0.2171217	4.142171e-06\\
0.2172217	7.852978e-06\\
0.2173217	3.1925e-06\\
0.2174217	3.772245e-06\\
0.2175218	9.813409e-07\\
0.2176218	1.141903e-05\\
0.2177218	1.421176e-05\\
0.2178218	7.44241e-06\\
0.2179218	7.267193e-06\\
0.2180218	2.953589e-06\\
0.2181218	2.489828e-07\\
0.2182218	2.723712e-06\\
0.2183218	6.780412e-06\\
0.2184218	3.219634e-06\\
0.2185219	2.579299e-06\\
0.2186219	4.565792e-06\\
0.2187219	1.102439e-07\\
0.2188219	3.241872e-06\\
0.2189219	1.063185e-06\\
0.2190219	5.373094e-06\\
0.2191219	1.539045e-06\\
0.2192219	1.382629e-06\\
0.2193219	2.382225e-06\\
0.2194219	5.077725e-07\\
0.219522	1.304895e-06\\
0.219622	1.064217e-05\\
0.219722	6.999812e-06\\
0.219822	7.747248e-06\\
0.219922	2.601447e-06\\
0.220022	5.153962e-06\\
0.220122	9.175537e-06\\
0.220222	1.47712e-05\\
0.220322	1.458055e-05\\
0.220422	1.050968e-05\\
0.2205221	1.646083e-05\\
0.2206221	3.805095e-06\\
0.2207221	1.934997e-06\\
0.2208221	4.917977e-07\\
0.2209221	9.20482e-07\\
0.2210221	9.534655e-07\\
0.2211221	1.636288e-06\\
0.2212221	8.807642e-07\\
0.2213221	8.398882e-07\\
0.2214221	9.518651e-06\\
0.2215222	1.716913e-05\\
0.2216222	1.196533e-05\\
0.2217222	1.085197e-05\\
0.2218222	4.880131e-06\\
0.2219222	6.698028e-07\\
0.2220222	1.050118e-05\\
0.2221222	8.197142e-06\\
0.2222222	2.183155e-07\\
0.2223222	8.360846e-06\\
0.2224222	1.31949e-05\\
0.2225223	1.010737e-05\\
0.2226223	3.246258e-06\\
0.2227223	2.890044e-06\\
0.2228223	5.026479e-06\\
0.2229223	2.862911e-07\\
0.2230223	1.030568e-06\\
0.2231223	2.65451e-06\\
0.2232223	1.773542e-06\\
0.2233223	5.525713e-06\\
0.2234223	3.818831e-06\\
0.2235224	4.453706e-06\\
0.2236224	1.263245e-06\\
0.2237224	5.170618e-06\\
0.2238224	9.050143e-06\\
0.2239224	5.509791e-06\\
0.2240224	5.062719e-06\\
0.2241224	1.724778e-05\\
0.2242224	4.523545e-06\\
0.2243224	6.718252e-07\\
0.2244224	5.829849e-06\\
0.2245225	1.101296e-05\\
0.2246225	1.173054e-05\\
0.2247225	1.269675e-05\\
0.2248225	1.507902e-05\\
0.2249225	2.6622e-06\\
0.2250225	9.241489e-06\\
0.2251225	1.16552e-06\\
0.2252225	3.088142e-07\\
0.2253225	1.560693e-06\\
0.2254225	1.958214e-06\\
0.2255226	1.772264e-06\\
0.2256226	5.963656e-06\\
0.2257226	1.350632e-06\\
0.2258226	6.779741e-07\\
0.2259226	2.268636e-06\\
0.2260226	2.614414e-06\\
0.2261226	1.11749e-06\\
0.2262226	1.997672e-06\\
0.2263226	4.509409e-06\\
0.2264226	1.749235e-06\\
0.2265227	3.029845e-06\\
0.2266227	1.049129e-06\\
0.2267227	7.87398e-07\\
0.2268227	1.444778e-07\\
0.2269227	1.973034e-06\\
0.2270227	1.270488e-06\\
0.2271227	2.664876e-07\\
0.2272227	3.860468e-06\\
0.2273227	1.277203e-07\\
0.2274227	2.171966e-06\\
0.2275228	3.773601e-06\\
0.2276228	1.261194e-05\\
0.2277228	9.120117e-06\\
0.2278228	2.078003e-05\\
0.2279228	1.177846e-05\\
0.2280228	2.864316e-05\\
0.2281228	2.05832e-05\\
0.2282228	1.955726e-05\\
0.2283228	6.859123e-06\\
0.2284228	6.368072e-06\\
0.2285229	7.075242e-06\\
0.2286229	9.509198e-06\\
0.2287229	7.025859e-06\\
0.2288229	1.201601e-05\\
0.2289229	1.419947e-05\\
0.2290229	3.49981e-06\\
0.2291229	3.449904e-06\\
0.2292229	7.600965e-06\\
0.2293229	1.840179e-06\\
0.2294229	2.72715e-06\\
0.229523	4.326745e-07\\
0.229623	5.918834e-07\\
0.229723	8.921898e-07\\
0.229823	4.420639e-06\\
0.229923	1.99085e-07\\
0.230023	5.067211e-07\\
0.230123	1.699172e-07\\
0.230223	1.653477e-06\\
0.230323	5.955693e-07\\
0.230423	5.396388e-07\\
0.2305231	8.287224e-07\\
0.2306231	3.857933e-06\\
0.2307231	1.021099e-05\\
0.2308231	1.951588e-08\\
0.2309231	4.065905e-06\\
0.2310231	6.277478e-06\\
0.2311231	1.195807e-05\\
0.2312231	1.253073e-05\\
0.2313231	2.434275e-05\\
0.2314231	3.15698e-05\\
0.2315232	8.664198e-06\\
0.2316232	6.886218e-06\\
0.2317232	3.827142e-07\\
0.2318232	3.681241e-06\\
0.2319232	6.13438e-06\\
0.2320232	1.11485e-05\\
0.2321232	1.302192e-05\\
0.2322232	5.538685e-06\\
0.2323232	1.019238e-05\\
0.2324232	1.326982e-05\\
0.2325233	4.188681e-06\\
0.2326233	5.331731e-06\\
0.2327233	6.725352e-06\\
0.2328233	4.612039e-06\\
0.2329233	1.697385e-05\\
0.2330233	1.905465e-05\\
0.2331233	2.15373e-06\\
0.2332233	6.505924e-06\\
0.2333233	2.852029e-06\\
0.2334233	3.901199e-06\\
0.2335234	6.049072e-06\\
0.2336234	9.75441e-06\\
0.2337234	5.680873e-06\\
0.2338234	5.12299e-06\\
0.2339234	3.355831e-06\\
0.2340234	4.317703e-06\\
0.2341234	8.985439e-06\\
0.2342234	3.000799e-06\\
0.2343234	6.028079e-06\\
0.2344234	4.542752e-06\\
0.2345235	7.68341e-08\\
0.2346235	5.016554e-07\\
0.2347235	2.51341e-06\\
0.2348235	1.59998e-07\\
0.2349235	2.435975e-06\\
0.2350235	1.14857e-06\\
0.2351235	2.637721e-07\\
0.2352235	1.297367e-07\\
0.2353235	2.094469e-06\\
0.2354235	6.123659e-06\\
0.2355236	1.871664e-06\\
0.2356236	5.167416e-06\\
0.2357236	7.004855e-07\\
0.2358236	8.770495e-06\\
0.2359236	9.863253e-06\\
0.2360236	1.209733e-06\\
0.2361236	3.375315e-06\\
0.2362236	3.175421e-06\\
0.2363236	4.251982e-06\\
0.2364236	2.169516e-06\\
0.2365237	1.013528e-06\\
0.2366237	4.766752e-06\\
0.2367237	7.451753e-09\\
0.2368237	1.207412e-07\\
0.2369237	2.667672e-06\\
0.2370237	1.343474e-05\\
0.2371237	9.058274e-06\\
0.2372237	4.151555e-06\\
0.2373237	2.364204e-06\\
0.2374237	7.820094e-06\\
0.2375238	2.880765e-06\\
0.2376238	3.836834e-06\\
0.2377238	5.101259e-06\\
0.2378238	7.704372e-06\\
0.2379238	1.058564e-05\\
0.2380238	1.915327e-05\\
0.2381238	7.291493e-06\\
0.2382238	9.19826e-06\\
0.2383238	3.672957e-06\\
0.2384238	7.254546e-07\\
0.2385239	7.716687e-07\\
0.2386239	5.514039e-06\\
0.2387239	3.70319e-06\\
0.2388239	1.254125e-05\\
0.2389239	1.312378e-06\\
0.2390239	4.049556e-06\\
0.2391239	1.009851e-05\\
0.2392239	3.498813e-06\\
0.2393239	1.029251e-05\\
0.2394239	1.157892e-05\\
0.239524	5.307354e-06\\
0.239624	3.159482e-07\\
0.239724	3.734287e-06\\
0.239824	7.699801e-07\\
0.239924	8.474659e-08\\
0.240024	1.912529e-06\\
0.240124	9.413616e-06\\
0.240224	2.949438e-06\\
0.240324	2.100703e-06\\
0.240424	8.034682e-08\\
0.2405241	4.769463e-06\\
0.2406241	5.110121e-06\\
0.2407241	1.428221e-06\\
0.2408241	1.157794e-06\\
0.2409241	4.778208e-06\\
0.2410241	1.592342e-06\\
0.2411241	3.427028e-07\\
0.2412241	1.320531e-06\\
0.2413241	2.176557e-06\\
0.2414241	3.25376e-06\\
0.2415242	1.481107e-05\\
0.2416242	6.58704e-06\\
0.2417242	1.605375e-05\\
0.2418242	2.790897e-06\\
0.2419242	4.956479e-06\\
0.2420242	3.227341e-06\\
0.2421242	1.05625e-05\\
0.2422242	1.376907e-06\\
0.2423242	5.342279e-07\\
0.2424242	2.106689e-06\\
0.2425243	1.990915e-06\\
0.2426243	7.14129e-07\\
0.2427243	4.038234e-06\\
0.2428243	3.063229e-07\\
0.2429243	1.725832e-07\\
0.2430243	1.65363e-06\\
0.2431243	3.094185e-07\\
0.2432243	6.09125e-08\\
0.2433243	1.99227e-06\\
0.2434243	9.081146e-06\\
0.2435244	1.36617e-05\\
0.2436244	2.140972e-05\\
0.2437244	1.433469e-05\\
0.2438244	1.055389e-05\\
0.2439244	9.360795e-06\\
0.2440244	8.407475e-06\\
0.2441244	1.480516e-06\\
0.2442244	5.139067e-06\\
0.2443244	6.673191e-06\\
0.2444244	9.786361e-06\\
0.2445245	5.19498e-06\\
0.2446245	1.955474e-06\\
0.2447245	7.134944e-06\\
0.2448245	6.052282e-06\\
0.2449245	1.425498e-05\\
0.2450245	2.171946e-05\\
0.2451245	1.406123e-05\\
0.2452245	5.805926e-06\\
0.2453245	1.577771e-05\\
0.2454245	1.206904e-05\\
0.2455246	1.736404e-05\\
0.2456246	1.074458e-05\\
0.2457246	6.196303e-06\\
0.2458246	6.406558e-06\\
0.2459246	1.230364e-05\\
0.2460246	8.911506e-06\\
0.2461246	1.439385e-05\\
0.2462246	1.043926e-05\\
0.2463246	9.209947e-06\\
0.2464246	9.413018e-06\\
0.2465247	9.517808e-06\\
0.2466247	7.286069e-06\\
0.2467247	6.351319e-06\\
0.2468247	4.047492e-06\\
0.2469247	8.6962e-06\\
0.2470247	1.429729e-05\\
0.2471247	8.738925e-07\\
0.2472247	2.409782e-06\\
0.2473247	3.124676e-06\\
0.2474247	7.452986e-06\\
0.2475248	5.024198e-06\\
0.2476248	3.938986e-06\\
0.2477248	5.417e-06\\
0.2478248	1.083967e-07\\
0.2479248	6.891005e-06\\
0.2480248	7.91424e-06\\
0.2481248	9.486057e-06\\
0.2482248	8.413044e-06\\
0.2483248	1.859628e-06\\
0.2484248	1.902221e-06\\
0.2485249	9.188872e-06\\
0.2486249	4.95116e-06\\
0.2487249	5.662373e-07\\
0.2488249	2.361242e-06\\
0.2489249	3.006344e-06\\
0.2490249	8.547229e-07\\
0.2491249	5.1555e-07\\
0.2492249	9.91668e-07\\
0.2493249	7.033594e-07\\
0.2494249	1.001666e-06\\
0.249525	4.68542e-06\\
0.249625	3.306061e-06\\
0.249725	5.161157e-06\\
0.249825	5.547617e-06\\
0.249925	1.537688e-06\\
0.250025	1.321642e-05\\
0.250125	5.259172e-06\\
0.250225	5.987893e-06\\
0.250325	4.279877e-07\\
0.250425	3.748018e-06\\
0.2505251	7.897702e-06\\
0.2506251	6.218001e-06\\
0.2507251	6.847021e-06\\
0.2508251	2.487278e-06\\
0.2509251	1.050512e-06\\
0.2510251	2.845994e-07\\
0.2511251	4.788909e-06\\
0.2512251	4.214003e-06\\
0.2513251	6.19013e-06\\
0.2514251	3.044298e-06\\
0.2515252	3.190924e-06\\
0.2516252	5.133442e-06\\
0.2517252	2.208649e-06\\
0.2518252	1.156703e-06\\
0.2519252	4.71755e-06\\
0.2520252	2.149723e-06\\
0.2521252	3.253682e-06\\
0.2522252	8.73538e-07\\
0.2523252	1.734159e-07\\
0.2524252	4.270506e-06\\
0.2525253	2.840233e-06\\
0.2526253	2.351095e-06\\
0.2527253	5.728175e-06\\
0.2528253	6.730344e-06\\
0.2529253	5.191527e-08\\
0.2530253	2.426357e-06\\
0.2531253	7.137553e-06\\
0.2532253	7.536372e-06\\
0.2533253	4.568205e-06\\
0.2534253	4.18612e-06\\
0.2535254	8.386637e-06\\
0.2536254	4.410232e-06\\
0.2537254	1.178113e-05\\
0.2538254	3.130821e-06\\
0.2539254	2.249506e-06\\
0.2540254	6.380815e-06\\
0.2541254	2.93957e-06\\
0.2542254	4.832619e-06\\
0.2543254	1.966121e-06\\
0.2544254	9.545657e-07\\
0.2545255	4.477898e-06\\
0.2546255	6.394152e-06\\
0.2547255	3.918539e-06\\
0.2548255	2.323097e-07\\
0.2549255	5.877669e-06\\
0.2550255	1.661534e-07\\
0.2551255	8.804337e-07\\
0.2552255	9.661041e-06\\
0.2553255	5.53488e-06\\
0.2554255	1.191317e-06\\
0.2555256	1.664065e-06\\
0.2556256	2.538292e-06\\
0.2557256	1.580291e-05\\
0.2558256	1.217369e-05\\
0.2559256	3.413119e-06\\
0.2560256	8.818793e-06\\
0.2561256	9.598016e-06\\
0.2562256	5.894388e-06\\
0.2563256	6.081449e-06\\
0.2564256	4.481558e-06\\
0.2565257	1.02268e-05\\
0.2566257	1.09338e-05\\
0.2567257	1.505128e-06\\
0.2568257	5.118368e-06\\
0.2569257	1.324831e-06\\
0.2570257	9.321259e-06\\
0.2571257	6.18159e-06\\
0.2572257	7.979237e-06\\
0.2573257	1.571813e-05\\
0.2574257	1.52636e-05\\
0.2575258	5.045183e-06\\
0.2576258	4.953265e-06\\
0.2577258	1.052601e-05\\
0.2578258	6.64529e-06\\
0.2579258	5.463153e-07\\
0.2580258	2.870147e-06\\
0.2581258	3.837445e-06\\
0.2582258	8.447618e-06\\
0.2583258	1.717944e-05\\
0.2584258	2.669597e-06\\
0.2585259	4.797368e-06\\
0.2586259	3.154255e-06\\
0.2587259	1.856617e-06\\
0.2588259	3.849895e-06\\
0.2589259	6.153606e-06\\
0.2590259	5.612928e-06\\
0.2591259	5.022377e-06\\
0.2592259	4.290164e-06\\
0.2593259	4.099166e-06\\
0.2594259	5.527955e-06\\
0.259526	8.670231e-06\\
0.259626	7.330376e-06\\
0.259726	5.079889e-07\\
0.259826	2.446553e-06\\
0.259926	1.122693e-06\\
0.260026	2.384297e-06\\
0.260126	1.149879e-06\\
0.260226	7.379685e-07\\
0.260326	5.788829e-06\\
0.260426	4.950272e-06\\
0.2605261	2.996912e-06\\
0.2606261	6.407764e-06\\
0.2607261	1.607627e-06\\
0.2608261	1.083425e-05\\
0.2609261	1.129584e-05\\
0.2610261	6.983832e-06\\
0.2611261	6.403558e-06\\
0.2612261	4.610291e-06\\
0.2613261	1.28713e-05\\
0.2614261	7.321934e-06\\
0.2615262	2.088265e-06\\
0.2616262	1.518459e-06\\
0.2617262	2.101871e-06\\
0.2618262	8.289366e-07\\
0.2619262	5.602461e-07\\
0.2620262	1.40402e-06\\
0.2621262	1.654375e-06\\
0.2622262	4.29423e-06\\
0.2623262	4.530978e-06\\
0.2624262	5.324021e-06\\
0.2625263	7.740339e-06\\
0.2626263	3.022187e-06\\
0.2627263	3.714214e-06\\
0.2628263	1.227461e-05\\
0.2629263	2.395567e-06\\
0.2630263	1.00267e-05\\
0.2631263	1.461514e-05\\
0.2632263	5.887549e-06\\
0.2633263	7.140528e-06\\
0.2634263	3.059967e-06\\
0.2635264	5.068145e-06\\
0.2636264	4.1875e-06\\
0.2637264	8.306508e-06\\
0.2638264	4.589326e-06\\
0.2639264	1.5024e-06\\
0.2640264	7.051176e-06\\
0.2641264	4.658274e-06\\
0.2642264	1.662507e-07\\
0.2643264	5.061596e-06\\
0.2644264	1.994942e-06\\
0.2645265	4.234645e-07\\
0.2646265	7.799586e-07\\
0.2647265	5.519613e-07\\
0.2648265	1.088096e-06\\
0.2649265	2.213113e-06\\
0.2650265	1.391709e-05\\
0.2651265	1.33829e-05\\
0.2652265	8.226762e-06\\
0.2653265	4.698872e-06\\
0.2654265	4.053221e-06\\
0.2655266	5.90354e-06\\
0.2656266	1.06026e-06\\
0.2657266	1.573245e-06\\
0.2658266	1.704861e-06\\
0.2659266	4.220734e-07\\
0.2660266	1.091701e-06\\
0.2661266	8.198112e-07\\
0.2662266	3.824531e-06\\
0.2663266	6.434559e-06\\
0.2664266	1.05575e-05\\
0.2665267	2.687351e-06\\
0.2666267	5.729369e-08\\
0.2667267	2.687215e-06\\
0.2668267	1.093954e-05\\
0.2669267	1.269955e-05\\
0.2670267	1.426562e-06\\
0.2671267	8.703938e-06\\
0.2672267	4.616653e-06\\
0.2673267	8.615017e-07\\
0.2674267	2.676444e-06\\
0.2675268	1.476049e-06\\
0.2676268	1.114845e-05\\
0.2677268	8.569e-06\\
0.2678268	1.425189e-05\\
0.2679268	1.42753e-05\\
0.2680268	6.524726e-06\\
0.2681268	9.259967e-06\\
0.2682268	1.938415e-05\\
0.2683268	1.666689e-05\\
0.2684268	5.78195e-06\\
0.2685269	6.656919e-07\\
0.2686269	5.016402e-06\\
0.2687269	1.279349e-06\\
0.2688269	4.585178e-06\\
0.2689269	9.052259e-06\\
0.2690269	3.323777e-06\\
0.2691269	3.046057e-06\\
0.2692269	2.299719e-06\\
0.2693269	4.709114e-06\\
0.2694269	1.356649e-05\\
0.269527	7.424596e-07\\
0.269627	4.206735e-06\\
0.269727	8.584251e-07\\
0.269827	2.53886e-06\\
0.269927	2.991265e-06\\
0.270027	2.139537e-07\\
0.270127	1.01951e-06\\
0.270227	2.042318e-06\\
0.270327	3.626735e-07\\
0.270427	1.509177e-06\\
0.2705271	2.220547e-07\\
0.2706271	4.718468e-06\\
0.2707271	1.56049e-05\\
0.2708271	1.31008e-05\\
0.2709271	3.485396e-06\\
0.2710271	8.704097e-07\\
0.2711271	3.232601e-06\\
0.2712271	4.254758e-06\\
0.2713271	3.456841e-06\\
0.2714271	3.181814e-06\\
0.2715272	1.383805e-06\\
0.2716272	2.64178e-06\\
0.2717272	7.388236e-06\\
0.2718272	7.121533e-06\\
0.2719272	1.369279e-05\\
0.2720272	2.632036e-06\\
0.2721272	3.633857e-06\\
0.2722272	4.241511e-07\\
0.2723272	1.666504e-08\\
0.2724272	3.38811e-06\\
0.2725273	6.393726e-06\\
0.2726273	3.164707e-06\\
0.2727273	5.915992e-06\\
0.2728273	1.236344e-05\\
0.2729273	1.375484e-05\\
0.2730273	4.067821e-06\\
0.2731273	6.711692e-06\\
0.2732273	2.913061e-07\\
0.2733273	1.850981e-06\\
0.2734273	3.357209e-07\\
0.2735274	1.718019e-06\\
0.2736274	2.112767e-06\\
0.2737274	3.524563e-06\\
0.2738274	1.198395e-06\\
0.2739274	9.68727e-07\\
0.2740274	3.675301e-06\\
0.2741274	1.870674e-05\\
0.2742274	8.946314e-07\\
0.2743274	1.906417e-06\\
0.2744274	4.031908e-07\\
0.2745275	6.969911e-07\\
0.2746275	5.701679e-06\\
0.2747275	1.819044e-06\\
0.2748275	6.020947e-06\\
0.2749275	3.414397e-06\\
0.2750275	2.105363e-06\\
0.2751275	5.397756e-06\\
0.2752275	2.291351e-06\\
0.2753275	1.417389e-05\\
0.2754275	3.949932e-06\\
0.2755276	4.796449e-06\\
0.2756276	8.159476e-06\\
0.2757276	3.083578e-06\\
0.2758276	4.510272e-06\\
0.2759276	3.584893e-07\\
0.2760276	2.486669e-06\\
0.2761276	1.116781e-06\\
0.2762276	3.443842e-06\\
0.2763276	1.369548e-05\\
0.2764276	1.020195e-05\\
0.2765277	2.537841e-06\\
0.2766277	6.361919e-06\\
0.2767277	1.187567e-05\\
0.2768277	1.612669e-05\\
0.2769277	1.168898e-05\\
0.2770277	4.69712e-06\\
0.2771277	9.671143e-06\\
0.2772277	5.392988e-06\\
0.2773277	1.785093e-05\\
0.2774277	4.842723e-06\\
0.2775278	2.898423e-07\\
0.2776278	3.105087e-06\\
0.2777278	1.748444e-07\\
0.2778278	8.660323e-07\\
0.2779278	3.788462e-06\\
0.2780278	2.965076e-06\\
0.2781278	1.366406e-06\\
0.2782278	1.762276e-06\\
0.2783278	6.041998e-06\\
0.2784278	1.170138e-06\\
0.2785279	1.264052e-06\\
0.2786279	4.956513e-06\\
0.2787279	3.577609e-06\\
0.2788279	1.141072e-06\\
0.2789279	2.928421e-06\\
0.2790279	3.459209e-06\\
0.2791279	4.430438e-06\\
0.2792279	1.035917e-06\\
0.2793279	3.926728e-06\\
0.2794279	7.359265e-07\\
0.279528	6.838806e-06\\
0.279628	1.099077e-05\\
0.279728	1.180294e-05\\
0.279828	1.103144e-05\\
0.279928	8.677058e-06\\
0.280028	5.601428e-06\\
0.280128	7.170073e-06\\
0.280228	8.090143e-07\\
0.280328	6.309424e-06\\
0.280428	8.305637e-06\\
0.2805281	1.603224e-05\\
0.2806281	2.352668e-05\\
0.2807281	3.185491e-05\\
0.2808281	3.091616e-06\\
0.2809281	4.400822e-07\\
0.2810281	1.649992e-06\\
0.2811281	1.630279e-06\\
0.2812281	1.015363e-05\\
0.2813281	7.873987e-06\\
0.2814281	9.016701e-06\\
0.2815282	3.645643e-07\\
0.2816282	4.475162e-06\\
0.2817282	1.831706e-06\\
0.2818282	3.124792e-06\\
0.2819282	6.356691e-06\\
0.2820282	5.225011e-06\\
0.2821282	5.659368e-06\\
0.2822282	8.16739e-06\\
0.2823282	5.114854e-06\\
0.2824282	8.614794e-06\\
0.2825283	1.18971e-05\\
0.2826283	2.092518e-07\\
0.2827283	2.117567e-06\\
0.2828283	6.827441e-06\\
0.2829283	7.894746e-08\\
0.2830283	8.979657e-06\\
0.2831283	1.009511e-05\\
0.2832283	9.196206e-07\\
0.2833283	9.006065e-07\\
0.2834283	1.221937e-05\\
0.2835284	7.677948e-06\\
0.2836284	3.1241e-06\\
0.2837284	3.666417e-06\\
0.2838284	4.340138e-07\\
0.2839284	1.760666e-08\\
0.2840284	2.9274e-06\\
0.2841284	3.689647e-06\\
0.2842284	5.684673e-06\\
0.2843284	1.069669e-06\\
0.2844284	1.729179e-07\\
0.2845285	6.413626e-06\\
0.2846285	9.911472e-06\\
0.2847285	5.67666e-06\\
0.2848285	5.383834e-06\\
0.2849285	2.183604e-06\\
0.2850285	3.195416e-06\\
0.2851285	2.638332e-06\\
0.2852285	2.345953e-06\\
0.2853285	3.398391e-06\\
0.2854285	5.206285e-06\\
0.2855286	2.008702e-06\\
0.2856286	3.196683e-07\\
0.2857286	5.092216e-06\\
0.2858286	1.106565e-06\\
0.2859286	8.331178e-06\\
0.2860286	5.63698e-06\\
0.2861286	1.466319e-07\\
0.2862286	2.051947e-06\\
0.2863286	8.488801e-06\\
0.2864286	1.028671e-06\\
0.2865287	2.8077e-07\\
0.2866287	2.048599e-06\\
0.2867287	1.118652e-06\\
0.2868287	2.325867e-06\\
0.2869287	2.155397e-06\\
0.2870287	6.557822e-07\\
0.2871287	2.454406e-06\\
0.2872287	9.100617e-06\\
0.2873287	7.225012e-06\\
0.2874287	2.219937e-06\\
0.2875288	6.123568e-07\\
0.2876288	1.021921e-05\\
0.2877288	5.899093e-06\\
0.2878288	2.199982e-06\\
0.2879288	4.204805e-07\\
0.2880288	2.407983e-06\\
0.2881288	3.575172e-07\\
0.2882288	2.520136e-06\\
0.2883288	3.250416e-09\\
0.2884288	2.641036e-06\\
0.2885289	5.345846e-06\\
0.2886289	3.296914e-06\\
0.2887289	4.925621e-06\\
0.2888289	2.549942e-06\\
0.2889289	2.752249e-06\\
0.2890289	7.616087e-07\\
0.2891289	1.88618e-06\\
0.2892289	2.491293e-06\\
0.2893289	6.071481e-06\\
0.2894289	5.885455e-06\\
0.289529	1.148524e-05\\
0.289629	2.551656e-05\\
0.289729	2.177975e-05\\
0.289829	5.529257e-06\\
0.289929	7.440268e-06\\
0.290029	5.128538e-07\\
0.290129	1.451572e-06\\
0.290229	8.73392e-07\\
0.290329	5.815855e-06\\
0.290429	1.05832e-08\\
0.2905291	1.498713e-06\\
0.2906291	2.183061e-06\\
0.2907291	2.858184e-06\\
0.2908291	2.448265e-06\\
0.2909291	3.640765e-06\\
0.2910291	9.681144e-06\\
0.2911291	3.692435e-07\\
0.2912291	2.929481e-07\\
0.2913291	4.725468e-06\\
0.2914291	3.65971e-06\\
0.2915292	2.868921e-07\\
0.2916292	2.090792e-06\\
0.2917292	2.352292e-07\\
0.2918292	4.593397e-07\\
0.2919292	3.715533e-06\\
0.2920292	6.398614e-07\\
0.2921292	2.141196e-06\\
0.2922292	6.491662e-06\\
0.2923292	6.689698e-06\\
0.2924292	9.907856e-06\\
0.2925293	7.443613e-06\\
0.2926293	8.275318e-06\\
0.2927293	1.036775e-05\\
0.2928293	2.59436e-05\\
0.2929293	2.811635e-05\\
0.2930293	2.696972e-05\\
0.2931293	1.438424e-05\\
0.2932293	9.111948e-06\\
0.2933293	1.712131e-06\\
0.2934293	6.948406e-07\\
0.2935294	6.413234e-06\\
0.2936294	2.503264e-06\\
0.2937294	1.637069e-06\\
0.2938294	3.306768e-06\\
0.2939294	2.58896e-06\\
0.2940294	2.84827e-07\\
0.2941294	4.813127e-06\\
0.2942294	1.824941e-06\\
0.2943294	4.22111e-06\\
0.2944294	6.684243e-07\\
0.2945295	1.107442e-05\\
0.2946295	2.113424e-06\\
0.2947295	2.62778e-06\\
0.2948295	1.102188e-05\\
0.2949295	8.273387e-06\\
0.2950295	1.298535e-07\\
0.2951295	1.456329e-06\\
0.2952295	6.859594e-06\\
0.2953295	7.76837e-06\\
0.2954295	3.478474e-06\\
0.2955296	8.475912e-07\\
0.2956296	3.28493e-06\\
0.2957296	5.543062e-06\\
0.2958296	2.565215e-06\\
0.2959296	5.694028e-06\\
0.2960296	6.952348e-06\\
0.2961296	6.214635e-06\\
0.2962296	4.9442e-06\\
0.2963296	4.7845e-06\\
0.2964296	5.806301e-06\\
0.2965297	7.224973e-06\\
0.2966297	5.499842e-06\\
0.2967297	1.345901e-06\\
0.2968297	4.008788e-07\\
0.2969297	3.625813e-06\\
0.2970297	4.572387e-06\\
0.2971297	6.685736e-06\\
0.2972297	3.693823e-06\\
0.2973297	8.80024e-07\\
0.2974297	2.254324e-06\\
0.2975298	2.130528e-06\\
0.2976298	6.594941e-06\\
0.2977298	2.343125e-06\\
0.2978298	1.279626e-06\\
0.2979298	3.219571e-07\\
0.2980298	6.175995e-06\\
0.2981298	3.676694e-06\\
0.2982298	1.948531e-06\\
0.2983298	1.119743e-06\\
0.2984298	1.902996e-07\\
0.2985299	6.722459e-07\\
0.2986299	8.969251e-07\\
0.2987299	3.466734e-07\\
0.2988299	3.349369e-06\\
0.2989299	1.363734e-05\\
0.2990299	1.571789e-06\\
0.2991299	1.967149e-06\\
0.2992299	1.027197e-05\\
0.2993299	1.089971e-05\\
0.2994299	7.370242e-06\\
0.29953	1.16593e-05\\
0.29963	4.752617e-06\\
0.29973	9.782695e-07\\
0.29983	4.163553e-06\\
0.29993	1.520213e-05\\
0.30003	1.184176e-05\\
0.30013	1.286871e-05\\
0.30023	7.947793e-06\\
0.30033	2.589351e-06\\
0.30043	1.536168e-06\\
0.3005301	1.141965e-06\\
0.3006301	1.737113e-06\\
0.3007301	1.079735e-05\\
0.3008301	1.339768e-05\\
0.3009301	1.235557e-05\\
0.3010301	9.790661e-06\\
0.3011301	1.012526e-06\\
0.3012301	3.583552e-07\\
0.3013301	5.588525e-07\\
0.3014301	9.501715e-07\\
0.3015302	1.244732e-06\\
0.3016302	1.534694e-06\\
0.3017302	2.07037e-08\\
0.3018302	5.351932e-06\\
0.3019302	1.076628e-06\\
0.3020302	4.424719e-07\\
0.3021302	2.727001e-06\\
0.3022302	1.372147e-06\\
0.3023302	7.136708e-06\\
0.3024302	3.676984e-06\\
0.3025303	6.792512e-06\\
0.3026303	2.146766e-06\\
0.3027303	6.040303e-06\\
0.3028303	8.302018e-06\\
0.3029303	4.47284e-06\\
0.3030303	1.820253e-05\\
0.3031303	1.74644e-05\\
0.3032303	1.133387e-05\\
0.3033303	1.558262e-06\\
0.3034303	1.043456e-06\\
0.3035304	2.371441e-06\\
0.3036304	4.126913e-06\\
0.3037304	4.890952e-07\\
0.3038304	1.112299e-06\\
0.3039304	3.944559e-07\\
0.3040304	4.046264e-06\\
0.3041304	1.445351e-06\\
0.3042304	2.45075e-06\\
0.3043304	2.112007e-06\\
0.3044304	6.770189e-08\\
0.3045305	1.034495e-06\\
0.3046305	1.570493e-07\\
0.3047305	3.019752e-06\\
0.3048305	9.916169e-07\\
0.3049305	3.778147e-06\\
0.3050305	1.006028e-07\\
0.3051305	1.173146e-06\\
0.3052305	2.766651e-07\\
0.3053305	1.402851e-06\\
0.3054305	1.128012e-06\\
0.3055306	8.758792e-06\\
0.3056306	5.308228e-06\\
0.3057306	1.581758e-05\\
0.3058306	9.351406e-06\\
0.3059306	1.326012e-06\\
0.3060306	4.375601e-06\\
0.3061306	7.361679e-06\\
0.3062306	8.743955e-06\\
0.3063306	1.737371e-06\\
0.3064306	5.987129e-06\\
0.3065307	4.118072e-06\\
0.3066307	1.020903e-05\\
0.3067307	9.050025e-06\\
0.3068307	1.061084e-05\\
0.3069307	6.791458e-06\\
0.3070307	3.165242e-06\\
0.3071307	4.226566e-08\\
0.3072307	1.443689e-06\\
0.3073307	9.767403e-06\\
0.3074307	1.637966e-05\\
0.3075308	7.584306e-06\\
0.3076308	3.082833e-06\\
0.3077308	1.572244e-06\\
0.3078308	7.575462e-08\\
0.3079308	1.053261e-06\\
0.3080308	2.596434e-07\\
0.3081308	6.071621e-07\\
0.3082308	3.856969e-06\\
0.3083308	4.684216e-06\\
0.3084308	1.018474e-06\\
0.3085309	2.239924e-06\\
0.3086309	7.654207e-06\\
0.3087309	3.791559e-06\\
0.3088309	5.280689e-06\\
0.3089309	1.961528e-06\\
0.3090309	1.186224e-05\\
0.3091309	8.98444e-06\\
0.3092309	3.102528e-06\\
0.3093309	7.127465e-09\\
0.3094309	4.900871e-06\\
0.309531	1.612705e-06\\
0.309631	3.077993e-06\\
0.309731	9.34914e-06\\
0.309831	3.762614e-06\\
0.309931	7.261407e-06\\
0.310031	1.35598e-05\\
0.310131	8.220592e-06\\
0.310231	3.081698e-06\\
0.310331	1.409994e-05\\
0.310431	8.546875e-06\\
0.3105311	1.196128e-05\\
0.3106311	1.338109e-05\\
0.3107311	2.004609e-05\\
0.3108311	7.457152e-06\\
0.3109311	4.968035e-06\\
0.3110311	1.703063e-06\\
0.3111311	3.465614e-06\\
0.3112311	6.562247e-06\\
0.3113311	3.765079e-06\\
0.3114311	7.221519e-06\\
0.3115312	9.337605e-06\\
0.3116312	6.750666e-06\\
0.3117312	3.62537e-07\\
0.3118312	1.07188e-06\\
0.3119312	1.466694e-06\\
0.3120312	4.460911e-06\\
0.3121312	2.531476e-06\\
0.3122312	1.496129e-06\\
0.3123312	5.325095e-07\\
0.3124312	5.886155e-06\\
0.3125313	1.280197e-06\\
0.3126313	2.33296e-06\\
0.3127313	2.371108e-06\\
0.3128313	3.507823e-06\\
0.3129313	1.770987e-06\\
0.3130313	5.665654e-07\\
0.3131313	1.956372e-06\\
0.3132313	2.780902e-06\\
0.3133313	3.144105e-06\\
0.3134313	8.141882e-06\\
0.3135314	6.991717e-06\\
0.3136314	3.90121e-06\\
0.3137314	5.818464e-06\\
0.3138314	6.852348e-07\\
0.3139314	1.284528e-06\\
0.3140314	4.866648e-07\\
0.3141314	3.663732e-06\\
0.3142314	2.401102e-06\\
0.3143314	6.313064e-06\\
0.3144314	7.581986e-06\\
0.3145315	1.531984e-05\\
0.3146315	3.577392e-06\\
0.3147315	1.614882e-06\\
0.3148315	3.988977e-07\\
0.3149315	5.298195e-07\\
0.3150315	2.961774e-06\\
0.3151315	3.90816e-06\\
0.3152315	2.410519e-06\\
0.3153315	9.344411e-06\\
0.3154315	1.993899e-05\\
0.3155316	2.878766e-05\\
0.3156316	1.042068e-05\\
0.3157316	1.24218e-06\\
0.3158316	1.167003e-06\\
0.3159316	4.936881e-08\\
0.3160316	5.029463e-07\\
0.3161316	1.616894e-06\\
0.3162316	4.923216e-07\\
0.3163316	1.219589e-06\\
0.3164316	1.367693e-07\\
0.3165317	4.729192e-07\\
0.3166317	2.961752e-07\\
0.3167317	1.532415e-06\\
0.3168317	6.70291e-07\\
0.3169317	3.187178e-06\\
0.3170317	2.445474e-06\\
0.3171317	5.19766e-06\\
0.3172317	9.433083e-06\\
0.3173317	5.51639e-06\\
0.3174317	4.726241e-06\\
0.3175318	5.585313e-06\\
0.3176318	1.397972e-05\\
0.3177318	4.050914e-06\\
0.3178318	6.211873e-06\\
0.3179318	4.927442e-06\\
0.3180318	1.381161e-06\\
0.3181318	5.184651e-06\\
0.3182318	2.85256e-06\\
0.3183318	3.922846e-07\\
0.3184318	1.505876e-06\\
0.3185319	3.626385e-06\\
0.3186319	5.402479e-06\\
0.3187319	4.354905e-07\\
0.3188319	2.104624e-06\\
0.3189319	5.840167e-06\\
0.3190319	3.50873e-06\\
0.3191319	3.425769e-06\\
0.3192319	9.248891e-08\\
0.3193319	7.497222e-07\\
0.3194319	3.802647e-06\\
0.319532	1.841772e-06\\
0.319632	1.821802e-06\\
0.319732	8.317228e-07\\
0.319832	3.801442e-06\\
0.319932	2.890151e-06\\
0.320032	8.088203e-07\\
0.320132	2.230702e-06\\
0.320232	8.091834e-07\\
0.320332	9.166526e-07\\
0.320432	1.2227e-06\\
0.3205321	1.509335e-06\\
0.3206321	5.251959e-07\\
0.3207321	2.853676e-06\\
0.3208321	6.90359e-10\\
0.3209321	6.426159e-07\\
0.3210321	3.839709e-06\\
0.3211321	1.083001e-06\\
0.3212321	2.273131e-06\\
0.3213321	1.412626e-06\\
0.3214321	4.41118e-06\\
0.3215322	1.139282e-06\\
0.3216322	2.537226e-06\\
0.3217322	7.666019e-07\\
0.3218322	3.948928e-07\\
0.3219322	5.716197e-06\\
0.3220322	5.571471e-06\\
0.3221322	4.067474e-06\\
0.3222322	4.070962e-06\\
0.3223322	1.504866e-06\\
0.3224322	1.863292e-06\\
0.3225323	4.966993e-06\\
0.3226323	1.258714e-05\\
0.3227323	1.3647e-05\\
0.3228323	2.755402e-06\\
0.3229323	1.164893e-06\\
0.3230323	5.31239e-06\\
0.3231323	2.760456e-05\\
0.3232323	4.413991e-05\\
0.3233323	3.677175e-05\\
0.3234323	2.515344e-05\\
0.3235324	2.874293e-06\\
0.3236324	2.307241e-06\\
0.3237324	1.179959e-06\\
0.3238324	5.638863e-06\\
0.3239324	7.300939e-06\\
0.3240324	7.394196e-06\\
0.3241324	3.614553e-06\\
0.3242324	1.735253e-06\\
0.3243324	1.133085e-05\\
0.3244324	8.51716e-06\\
0.3245325	1.105247e-05\\
0.3246325	1.432022e-05\\
0.3247325	1.814519e-05\\
0.3248325	5.57656e-06\\
0.3249325	1.019247e-05\\
0.3250325	9.936105e-07\\
0.3251325	8.600257e-06\\
0.3252325	4.151411e-06\\
0.3253325	1.793777e-06\\
0.3254325	1.059197e-05\\
0.3255326	7.866098e-06\\
0.3256326	3.255716e-06\\
0.3257326	7.513754e-06\\
0.3258326	7.888529e-06\\
0.3259326	8.940162e-06\\
0.3260326	5.996323e-07\\
0.3261326	5.847016e-06\\
0.3262326	1.677525e-06\\
0.3263326	1.909418e-06\\
0.3264326	2.641612e-06\\
0.3265327	2.558798e-06\\
0.3266327	4.818166e-06\\
0.3267327	5.48492e-06\\
0.3268327	3.060885e-06\\
0.3269327	8.89605e-06\\
0.3270327	2.750963e-06\\
0.3271327	1.462234e-06\\
0.3272327	3.803715e-06\\
0.3273327	6.514493e-06\\
0.3274327	3.00871e-06\\
0.3275328	1.129905e-06\\
0.3276328	1.360086e-06\\
0.3277328	4.355917e-06\\
0.3278328	2.482573e-06\\
0.3279328	6.043297e-06\\
0.3280328	8.637432e-06\\
0.3281328	4.110987e-06\\
0.3282328	9.522861e-06\\
0.3283328	5.957075e-06\\
0.3284328	2.815048e-06\\
0.3285329	9.10571e-07\\
0.3286329	7.954371e-07\\
0.3287329	1.144881e-06\\
0.3288329	4.479675e-06\\
0.3289329	2.124952e-06\\
0.3290329	7.176767e-06\\
0.3291329	5.083172e-07\\
0.3292329	1.487932e-06\\
0.3293329	3.961513e-06\\
0.3294329	7.294059e-06\\
0.329533	2.883074e-06\\
0.329633	6.425036e-07\\
0.329733	6.280349e-06\\
0.329833	3.911327e-06\\
0.329933	2.218925e-06\\
0.330033	5.681682e-07\\
0.330133	3.757142e-07\\
0.330233	1.242855e-06\\
0.330333	1.063353e-06\\
0.330433	4.666359e-06\\
0.3305331	6.495618e-07\\
0.3306331	6.058597e-08\\
0.3307331	7.405695e-07\\
0.3308331	2.373196e-06\\
0.3309331	7.92916e-07\\
0.3310331	1.58522e-07\\
0.3311331	1.377898e-07\\
0.3312331	2.572104e-06\\
0.3313331	2.96523e-06\\
0.3314331	1.91675e-06\\
0.3315332	1.386988e-06\\
0.3316332	2.165182e-07\\
0.3317332	2.39306e-06\\
0.3318332	4.704112e-07\\
0.3319332	5.27121e-06\\
0.3320332	4.348868e-06\\
0.3321332	3.73682e-07\\
0.3322332	2.710325e-06\\
0.3323332	3.466594e-06\\
0.3324332	6.04325e-06\\
0.3325333	1.211103e-06\\
0.3326333	1.449788e-05\\
0.3327333	3.518685e-06\\
0.3328333	1.915587e-07\\
0.3329333	2.029479e-06\\
0.3330333	3.610083e-06\\
0.3331333	1.438486e-06\\
0.3332333	2.642716e-06\\
0.3333333	2.04289e-06\\
0.3334333	2.236181e-06\\
0.3335334	1.096821e-06\\
0.3336334	2.272244e-06\\
0.3337334	3.933896e-06\\
0.3338334	4.515216e-06\\
0.3339334	5.829385e-06\\
0.3340334	2.22854e-06\\
0.3341334	4.255665e-06\\
0.3342334	2.78883e-06\\
0.3343334	2.001091e-06\\
0.3344334	7.043375e-06\\
0.3345335	1.425424e-05\\
0.3346335	6.230253e-06\\
0.3347335	7.104208e-06\\
0.3348335	4.125647e-07\\
0.3349335	3.004372e-06\\
0.3350335	6.541803e-06\\
0.3351335	6.439512e-06\\
0.3352335	6.821933e-06\\
0.3353335	1.670337e-06\\
0.3354335	1.419979e-06\\
0.3355336	1.019557e-05\\
0.3356336	1.46752e-05\\
0.3357336	1.9885e-05\\
0.3358336	6.151439e-06\\
0.3359336	3.547077e-06\\
0.3360336	2.918163e-06\\
0.3361336	4.337327e-06\\
0.3362336	3.070737e-06\\
0.3363336	6.040499e-06\\
0.3364336	3.505052e-06\\
0.3365337	1.218648e-06\\
0.3366337	4.320334e-06\\
0.3367337	5.846966e-07\\
0.3368337	3.023884e-06\\
0.3369337	5.141862e-06\\
0.3370337	1.926083e-06\\
0.3371337	1.935794e-07\\
0.3372337	7.319978e-06\\
0.3373337	2.643505e-06\\
0.3374337	1.566987e-08\\
0.3375338	4.186334e-06\\
0.3376338	2.88183e-06\\
0.3377338	4.324271e-06\\
0.3378338	1.216271e-05\\
0.3379338	2.225534e-05\\
0.3380338	2.013959e-05\\
0.3381338	4.517722e-06\\
0.3382338	1.316019e-06\\
0.3383338	1.470598e-06\\
0.3384338	4.648166e-07\\
0.3385339	5.886641e-06\\
0.3386339	1.395417e-05\\
0.3387339	1.21027e-05\\
0.3388339	1.64045e-05\\
0.3389339	1.063973e-05\\
0.3390339	3.060446e-05\\
0.3391339	1.856509e-05\\
0.3392339	1.252188e-05\\
0.3393339	3.167695e-06\\
0.3394339	3.60673e-06\\
0.339534	3.574949e-06\\
0.339634	2.222813e-06\\
0.339734	8.011689e-07\\
0.339834	2.422642e-06\\
0.339934	1.89564e-06\\
0.340034	4.492043e-06\\
0.340134	3.847188e-06\\
0.340234	1.800008e-06\\
0.340334	5.173319e-07\\
0.340434	3.308498e-06\\
0.3405341	3.87527e-06\\
0.3406341	1.101353e-05\\
0.3407341	1.827278e-05\\
0.3408341	4.11076e-06\\
0.3409341	1.177898e-06\\
0.3410341	9.859756e-06\\
0.3411341	8.107579e-07\\
0.3412341	1.405224e-06\\
0.3413341	1.664875e-06\\
0.3414341	7.206221e-06\\
0.3415342	1.219652e-06\\
0.3416342	1.164392e-05\\
0.3417342	3.163761e-06\\
0.3418342	7.748003e-06\\
0.3419342	1.017586e-06\\
0.3420342	6.82515e-06\\
0.3421342	4.425583e-06\\
0.3422342	5.919441e-06\\
0.3423342	6.460657e-06\\
0.3424342	4.842688e-06\\
0.3425343	1.323332e-05\\
0.3426343	1.505614e-06\\
0.3427343	3.16132e-06\\
0.3428343	9.648545e-07\\
0.3429343	1.439435e-06\\
0.3430343	2.756558e-07\\
0.3431343	1.942571e-06\\
0.3432343	9.210779e-06\\
0.3433343	6.377639e-06\\
0.3434343	6.152053e-06\\
0.3435344	1.003666e-05\\
0.3436344	1.127723e-05\\
0.3437344	2.443839e-06\\
0.3438344	8.051848e-06\\
0.3439344	8.255457e-06\\
0.3440344	4.717314e-06\\
0.3441344	5.484562e-06\\
0.3442344	4.288029e-07\\
0.3443344	2.628615e-07\\
0.3444344	8.169631e-07\\
0.3445345	1.855867e-06\\
0.3446345	3.360871e-06\\
0.3447345	7.71724e-07\\
0.3448345	1.992124e-06\\
0.3449345	5.064084e-06\\
0.3450345	1.821836e-06\\
0.3451345	3.491287e-06\\
0.3452345	4.86021e-06\\
0.3453345	2.108115e-06\\
0.3454345	2.903937e-07\\
0.3455346	5.49264e-06\\
0.3456346	1.456138e-05\\
0.3457346	2.590228e-05\\
0.3458346	1.404903e-05\\
0.3459346	2.259082e-06\\
0.3460346	4.512713e-06\\
0.3461346	9.45466e-07\\
0.3462346	4.520375e-07\\
0.3463346	5.119666e-06\\
0.3464346	4.851676e-06\\
0.3465347	4.39869e-06\\
0.3466347	6.914256e-06\\
0.3467347	8.593321e-08\\
0.3468347	1.152803e-06\\
0.3469347	3.836221e-06\\
0.3470347	4.366783e-07\\
0.3471347	7.784628e-08\\
0.3472347	3.72793e-07\\
0.3473347	5.156907e-07\\
0.3474347	5.889934e-06\\
0.3475348	5.970453e-07\\
0.3476348	6.042411e-07\\
0.3477348	1.729356e-06\\
0.3478348	7.633947e-06\\
0.3479348	1.256115e-05\\
0.3480348	1.401841e-05\\
0.3481348	7.655384e-06\\
0.3482348	1.27012e-06\\
0.3483348	5.079631e-06\\
0.3484348	1.946308e-05\\
0.3485349	1.584941e-05\\
0.3486349	4.532317e-06\\
0.3487349	2.917127e-07\\
0.3488349	4.621705e-06\\
0.3489349	1.67907e-06\\
0.3490349	4.737563e-06\\
0.3491349	1.670818e-06\\
0.3492349	2.353809e-06\\
0.3493349	9.533778e-06\\
0.3494349	7.38181e-06\\
0.349535	8.141703e-06\\
0.349635	3.239187e-06\\
0.349735	1.615355e-06\\
0.349835	6.678595e-06\\
0.349935	8.159658e-06\\
0.350035	1.644851e-05\\
0.350135	2.149528e-05\\
0.350235	5.742906e-06\\
0.350335	1.206259e-05\\
0.350435	2.93981e-06\\
0.3505351	7.980228e-06\\
0.3506351	4.185563e-06\\
0.3507351	5.267397e-06\\
0.3508351	3.026555e-06\\
0.3509351	4.494977e-08\\
0.3510351	4.184565e-06\\
0.3511351	8.552922e-06\\
0.3512351	4.701318e-07\\
0.3513351	4.00011e-06\\
0.3514351	1.757973e-06\\
0.3515352	1.590798e-06\\
0.3516352	1.209359e-06\\
0.3517352	1.093584e-06\\
0.3518352	2.087646e-06\\
0.3519352	1.218324e-06\\
0.3520352	1.637874e-06\\
0.3521352	8.855962e-07\\
0.3522352	2.737295e-06\\
0.3523352	5.878507e-06\\
0.3524352	4.722947e-06\\
0.3525353	4.61249e-06\\
0.3526353	3.119598e-07\\
0.3527353	4.96136e-06\\
0.3528353	7.922507e-08\\
0.3529353	2.857691e-06\\
0.3530353	5.771321e-06\\
0.3531353	4.943163e-06\\
0.3532353	3.288471e-06\\
0.3533353	9.936377e-06\\
0.3534353	5.069263e-06\\
0.3535354	1.910885e-06\\
0.3536354	1.224545e-06\\
0.3537354	1.778403e-05\\
0.3538354	1.285393e-05\\
0.3539354	6.135752e-06\\
0.3540354	1.447031e-05\\
0.3541354	1.097755e-05\\
0.3542354	1.340482e-05\\
0.3543354	1.592139e-05\\
0.3544354	6.784259e-06\\
0.3545355	3.67507e-06\\
0.3546355	8.2982e-06\\
0.3547355	8.288564e-06\\
0.3548355	6.359908e-06\\
0.3549355	6.531601e-06\\
0.3550355	1.258746e-05\\
0.3551355	1.035155e-05\\
0.3552355	5.635059e-06\\
0.3553355	5.411421e-06\\
0.3554355	6.930542e-07\\
0.3555356	1.2711e-06\\
0.3556356	6.402248e-06\\
0.3557356	3.708227e-06\\
0.3558356	2.233907e-07\\
0.3559356	4.12563e-06\\
0.3560356	5.645516e-06\\
0.3561356	1.153405e-06\\
0.3562356	6.563576e-07\\
0.3563356	1.404923e-06\\
0.3564356	3.70179e-06\\
0.3565357	5.761652e-06\\
0.3566357	5.199811e-06\\
0.3567357	7.316024e-06\\
0.3568357	7.267144e-06\\
0.3569357	4.512235e-06\\
0.3570357	1.152842e-06\\
0.3571357	1.892495e-07\\
0.3572357	6.593825e-06\\
0.3573357	5.314716e-06\\
0.3574357	8.321338e-06\\
0.3575358	6.316066e-06\\
0.3576358	5.091265e-06\\
0.3577358	8.375772e-07\\
0.3578358	1.300137e-05\\
0.3579358	1.807973e-05\\
0.3580358	1.46039e-05\\
0.3581358	2.07295e-05\\
0.3582358	3.350984e-05\\
0.3583358	2.717782e-05\\
0.3584358	5.492461e-06\\
0.3585359	2.488971e-06\\
0.3586359	3.774781e-06\\
0.3587359	2.91188e-06\\
0.3588359	7.918208e-06\\
0.3589359	7.524091e-06\\
0.3590359	4.375943e-06\\
0.3591359	4.359389e-06\\
0.3592359	2.699412e-06\\
0.3593359	3.854755e-06\\
0.3594359	8.767061e-07\\
0.359536	6.22318e-06\\
0.359636	1.050882e-07\\
0.359736	7.428666e-06\\
0.359836	3.729092e-06\\
0.359936	1.31636e-06\\
0.360036	5.642119e-06\\
0.360136	7.178168e-07\\
0.360236	4.398949e-06\\
0.360336	7.592734e-06\\
0.360436	8.702426e-06\\
0.3605361	6.975274e-06\\
0.3606361	1.004405e-05\\
0.3607361	1.155541e-05\\
0.3608361	2.897589e-05\\
0.3609361	1.046063e-05\\
0.3610361	9.76199e-06\\
0.3611361	7.502536e-06\\
0.3612361	8.792865e-06\\
0.3613361	4.514764e-07\\
0.3614361	1.517547e-06\\
0.3615362	5.054286e-06\\
0.3616362	3.974258e-06\\
0.3617362	3.239679e-06\\
0.3618362	1.50975e-06\\
0.3619362	2.806863e-06\\
0.3620362	6.062448e-06\\
0.3621362	4.515776e-06\\
0.3622362	3.992236e-06\\
0.3623362	1.924356e-06\\
0.3624362	1.374178e-05\\
0.3625363	1.03866e-05\\
0.3626363	1.289651e-05\\
0.3627363	2.072062e-06\\
0.3628363	3.283209e-06\\
0.3629363	3.365236e-06\\
0.3630363	1.425781e-06\\
0.3631363	1.484113e-06\\
0.3632363	2.472332e-06\\
0.3633363	4.474568e-07\\
0.3634363	9.697893e-07\\
0.3635364	1.196806e-06\\
0.3636364	3.532893e-06\\
0.3637364	7.850757e-06\\
0.3638364	7.207896e-07\\
0.3639364	2.737147e-08\\
0.3640364	1.391385e-06\\
0.3641364	8.295449e-06\\
0.3642364	1.430028e-05\\
0.3643364	1.336239e-05\\
0.3644364	3.032211e-06\\
0.3645365	2.064244e-06\\
0.3646365	1.412246e-05\\
0.3647365	2.302787e-05\\
0.3648365	1.4954e-05\\
0.3649365	8.086588e-06\\
0.3650365	5.378667e-06\\
0.3651365	3.501446e-06\\
0.3652365	5.108396e-06\\
0.3653365	4.315942e-06\\
0.3654365	1.166825e-06\\
0.3655366	3.150866e-06\\
0.3656366	3.017646e-06\\
0.3657366	2.341452e-06\\
0.3658366	2.52194e-05\\
0.3659366	1.517183e-05\\
0.3660366	1.680341e-05\\
0.3661366	7.504146e-06\\
0.3662366	1.461607e-05\\
0.3663366	1.720599e-05\\
0.3664366	1.349952e-05\\
0.3665367	4.425692e-06\\
0.3666367	2.859136e-06\\
0.3667367	1.529326e-07\\
0.3668367	1.068017e-06\\
0.3669367	2.938839e-07\\
0.3670367	2.596393e-06\\
0.3671367	2.13339e-06\\
0.3672367	5.306084e-07\\
0.3673367	3.192186e-06\\
0.3674367	1.092154e-06\\
0.3675368	5.709498e-06\\
0.3676368	1.052167e-05\\
0.3677368	6.233912e-06\\
0.3678368	3.458321e-06\\
0.3679368	3.704228e-06\\
0.3680368	1.972457e-05\\
0.3681368	3.479016e-05\\
0.3682368	1.397851e-05\\
0.3683368	1.352394e-05\\
0.3684368	1.601289e-05\\
0.3685369	4.245213e-06\\
0.3686369	1.359786e-05\\
0.3687369	1.751288e-05\\
0.3688369	1.585933e-05\\
0.3689369	5.291549e-06\\
0.3690369	3.38636e-06\\
0.3691369	5.945864e-06\\
0.3692369	1.243968e-06\\
0.3693369	7.637345e-07\\
0.3694369	4.132435e-06\\
0.369537	4.210006e-06\\
0.369637	2.141464e-06\\
0.369737	8.050282e-06\\
0.369837	4.174741e-06\\
0.369937	2.405132e-06\\
0.370037	1.468356e-05\\
0.370137	7.9016e-06\\
0.370237	5.41885e-06\\
0.370337	5.953235e-06\\
0.370437	4.060861e-06\\
0.3705371	3.307262e-06\\
0.3706371	4.744996e-06\\
0.3707371	3.856933e-07\\
0.3708371	2.165416e-06\\
0.3709371	1.641465e-05\\
0.3710371	1.339357e-05\\
0.3711371	1.011826e-05\\
0.3712371	5.181644e-06\\
0.3713371	1.189007e-05\\
0.3714371	2.007958e-05\\
0.3715372	2.097293e-05\\
0.3716372	9.871643e-06\\
0.3717372	1.507216e-05\\
0.3718372	2.154855e-05\\
0.3719372	4.632793e-06\\
0.3720372	5.260435e-06\\
0.3721372	1.118474e-05\\
0.3722372	5.770416e-07\\
0.3723372	2.101307e-06\\
0.3724372	5.750951e-06\\
0.3725373	2.637363e-05\\
0.3726373	1.948868e-05\\
0.3727373	1.005105e-05\\
0.3728373	3.01779e-06\\
0.3729373	1.185029e-05\\
0.3730373	1.335478e-05\\
0.3731373	1.579278e-05\\
0.3732373	1.414989e-05\\
0.3733373	1.557777e-05\\
0.3734373	1.111694e-05\\
0.3735374	1.154023e-05\\
0.3736374	5.325839e-06\\
0.3737374	1.838202e-06\\
0.3738374	1.417434e-06\\
0.3739374	2.751937e-07\\
0.3740374	1.564859e-05\\
0.3741374	1.379988e-05\\
0.3742374	3.731015e-06\\
0.3743374	3.358619e-06\\
0.3744374	4.722925e-06\\
0.3745375	1.64291e-06\\
0.3746375	9.670807e-06\\
0.3747375	1.906522e-05\\
0.3748375	2.245585e-05\\
0.3749375	2.200184e-05\\
0.3750375	1.837291e-05\\
0.3751375	6.399793e-06\\
0.3752375	1.511178e-05\\
0.3753375	1.393129e-05\\
0.3754375	7.496206e-06\\
0.3755376	1.14299e-05\\
0.3756376	1.905388e-05\\
0.3757376	5.603168e-06\\
0.3758376	1.948203e-06\\
0.3759376	8.740539e-06\\
0.3760376	1.044279e-05\\
0.3761376	1.977486e-05\\
0.3762376	8.237066e-06\\
0.3763376	1.481043e-05\\
0.3764376	1.21183e-05\\
0.3765377	1.015893e-05\\
0.3766377	1.914594e-05\\
0.3767377	1.121348e-05\\
0.3768377	7.82534e-06\\
0.3769377	1.401772e-05\\
0.3770377	1.526034e-05\\
0.3771377	7.556411e-06\\
0.3772377	2.406602e-06\\
0.3773377	1.853071e-06\\
0.3774377	1.182498e-05\\
0.3775378	1.588295e-05\\
0.3776378	5.895196e-06\\
0.3777378	5.019683e-06\\
0.3778378	1.602938e-06\\
0.3779378	3.648535e-06\\
0.3780378	3.972845e-06\\
0.3781378	8.575784e-06\\
0.3782378	4.932599e-06\\
0.3783378	2.194003e-05\\
0.3784378	2.432484e-05\\
0.3785379	1.760935e-05\\
0.3786379	2.743317e-05\\
0.3787379	3.06197e-05\\
0.3788379	2.776468e-05\\
0.3789379	8.947282e-06\\
0.3790379	1.974488e-05\\
0.3791379	1.774024e-05\\
0.3792379	1.901912e-05\\
0.3793379	1.024834e-05\\
0.3794379	1.35809e-05\\
0.379538	1.311369e-05\\
0.379638	1.016431e-05\\
0.379738	2.70923e-06\\
0.379838	3.254723e-06\\
0.379938	9.982678e-06\\
0.380038	2.48724e-06\\
0.380138	1.1304e-05\\
0.380238	1.106948e-05\\
0.380338	5.758106e-06\\
0.380438	2.018072e-06\\
0.3805381	6.088979e-06\\
0.3806381	1.040869e-05\\
0.3807381	3.514543e-05\\
0.3808381	2.687312e-05\\
0.3809381	2.059008e-05\\
0.3810381	2.398144e-05\\
0.3811381	1.133465e-05\\
0.3812381	1.994681e-05\\
0.3813381	7.38635e-06\\
0.3814381	1.20716e-06\\
0.3815382	5.090481e-06\\
0.3816382	5.0081e-07\\
0.3817382	8.139504e-06\\
0.3818382	2.456164e-05\\
0.3819382	1.865418e-05\\
0.3820382	1.782767e-05\\
0.3821382	1.674867e-05\\
0.3822382	2.004949e-05\\
0.3823382	2.285026e-05\\
0.3824382	2.400762e-05\\
0.3825383	5.045205e-06\\
0.3826383	5.064586e-06\\
0.3827383	1.255833e-05\\
0.3828383	9.237218e-06\\
0.3829383	5.098617e-06\\
0.3830383	1.06395e-05\\
0.3831383	2.009682e-05\\
0.3832383	1.193817e-05\\
0.3833383	3.102735e-06\\
0.3834383	1.194597e-05\\
0.3835384	1.610413e-05\\
0.3836384	2.656376e-05\\
0.3837384	2.750846e-05\\
0.3838384	2.223611e-05\\
0.3839384	4.260587e-05\\
0.3840384	2.5816e-05\\
0.3841384	1.636669e-05\\
0.3842384	6.218724e-06\\
0.3843384	6.607299e-06\\
0.3844384	1.61298e-05\\
0.3845385	2.047635e-05\\
0.3846385	1.222534e-05\\
0.3847385	8.664142e-06\\
0.3848385	4.565199e-06\\
0.3849385	6.835623e-06\\
0.3850385	4.461899e-06\\
0.3851385	4.052834e-06\\
0.3852385	1.112456e-05\\
0.3853385	1.394867e-05\\
0.3854385	1.748107e-05\\
0.3855386	2.23301e-05\\
0.3856386	2.425184e-05\\
0.3857386	9.573725e-06\\
0.3858386	2.546403e-05\\
0.3859386	1.842598e-05\\
0.3860386	2.04014e-05\\
0.3861386	3.395045e-05\\
0.3862386	1.73217e-05\\
0.3863386	1.295973e-05\\
0.3864386	1.635203e-05\\
0.3865387	1.806516e-05\\
0.3866387	1.589279e-05\\
0.3867387	1.692318e-05\\
0.3868387	2.012286e-05\\
0.3869387	1.656656e-05\\
0.3870387	6.353234e-06\\
0.3871387	1.083777e-05\\
0.3872387	1.300178e-05\\
0.3873387	1.298382e-05\\
0.3874387	3.245156e-05\\
0.3875388	4.155999e-05\\
0.3876388	3.326238e-05\\
0.3877388	2.856397e-05\\
0.3878388	2.461159e-05\\
0.3879388	2.295844e-05\\
0.3880388	1.079064e-05\\
0.3881388	5.327506e-06\\
0.3882388	9.094202e-07\\
0.3883388	1.270762e-06\\
0.3884388	5.176975e-06\\
0.3885389	1.310261e-05\\
0.3886389	2.2121e-05\\
0.3887389	4.61564e-05\\
0.3888389	2.967204e-05\\
0.3889389	2.648641e-05\\
0.3890389	2.368163e-05\\
0.3891389	4.554897e-05\\
0.3892389	3.342395e-05\\
0.3893389	2.82053e-05\\
0.3894389	2.888815e-05\\
0.389539	2.99263e-05\\
0.389639	2.522146e-05\\
0.389739	1.364485e-05\\
0.389839	1.46935e-05\\
0.389939	7.14355e-06\\
0.390039	1.268873e-05\\
0.390139	5.125605e-06\\
0.390239	1.876562e-05\\
0.390339	1.20295e-05\\
0.390439	9.632021e-06\\
0.3905391	2.800457e-05\\
0.3906391	1.623085e-05\\
0.3907391	2.625478e-05\\
0.3908391	4.757157e-05\\
0.3909391	4.177455e-05\\
0.3910391	1.79051e-05\\
0.3911391	1.831168e-05\\
0.3912391	9.607161e-06\\
0.3913391	1.736503e-05\\
0.3914391	2.503219e-05\\
0.3915392	1.823781e-05\\
0.3916392	9.971385e-06\\
0.3917392	1.852352e-05\\
0.3918392	2.647955e-05\\
0.3919392	3.836495e-05\\
0.3920392	4.320162e-05\\
0.3921392	3.90736e-05\\
0.3922392	4.392545e-05\\
0.3923392	2.580695e-05\\
0.3924392	3.181323e-05\\
0.3925393	1.161491e-05\\
0.3926393	1.997493e-06\\
0.3927393	1.083796e-05\\
0.3928393	2.212301e-05\\
0.3929393	1.693559e-05\\
0.3930393	6.674722e-06\\
0.3931393	1.427603e-05\\
0.3932393	1.382518e-05\\
0.3933393	2.578854e-05\\
0.3934393	4.619637e-05\\
0.3935394	5.348185e-05\\
0.3936394	6.960047e-05\\
0.3937394	5.81893e-05\\
0.3938394	4.565119e-05\\
0.3939394	3.703492e-05\\
0.3940394	2.97544e-05\\
0.3941394	2.812792e-05\\
0.3942394	1.40618e-05\\
0.3943394	1.1889e-05\\
0.3944394	1.830585e-05\\
0.3945395	3.199532e-05\\
0.3946395	3.42452e-05\\
0.3947395	2.403513e-05\\
0.3948395	3.143241e-05\\
0.3949395	3.342721e-05\\
0.3950395	1.59672e-05\\
0.3951395	6.084851e-06\\
0.3952395	3.266574e-05\\
0.3953395	3.417694e-05\\
0.3954395	3.834814e-05\\
0.3955396	1.068026e-05\\
0.3956396	1.450297e-05\\
0.3957396	9.095168e-06\\
0.3958396	4.159859e-06\\
0.3959396	7.909361e-06\\
0.3960396	2.176395e-05\\
0.3961396	3.776765e-05\\
0.3962396	4.958427e-05\\
0.3963396	6.178421e-05\\
0.3964396	6.142913e-05\\
0.3965397	5.048266e-05\\
0.3966397	5.897825e-05\\
0.3967397	3.038512e-05\\
0.3968397	2.296173e-05\\
0.3969397	1.360562e-05\\
0.3970397	1.689241e-05\\
0.3971397	2.640192e-05\\
0.3972397	3.228302e-05\\
0.3973397	3.819321e-05\\
0.3974397	3.940809e-05\\
0.3975398	3.984588e-05\\
0.3976398	1.917402e-05\\
0.3977398	3.071838e-05\\
0.3978398	3.308229e-05\\
0.3979398	2.391534e-05\\
0.3980398	3.658944e-05\\
0.3981398	3.430638e-05\\
0.3982398	2.468104e-05\\
0.3983398	2.942267e-05\\
0.3984398	2.827941e-05\\
0.3985399	4.775442e-05\\
0.3986399	4.042897e-05\\
0.3987399	2.90193e-05\\
0.3988399	4.658316e-05\\
0.3989399	3.293135e-05\\
0.3990399	2.040647e-05\\
0.3991399	2.326088e-05\\
0.3992399	6.407716e-06\\
0.3993399	1.421637e-05\\
0.3994399	4.293336e-05\\
0.39954	4.263155e-05\\
0.39964	4.82489e-05\\
0.39974	5.088337e-05\\
0.39984	3.429025e-05\\
0.39994	3.87348e-05\\
0.40004	3.131967e-05\\
};
\addplot [color=mycolor1,solid,forget plot]
  table[row sep=crcr]{%
0.40004	3.131967e-05\\
0.40014	2.028094e-05\\
0.40024	4.105511e-05\\
0.40034	6.541444e-05\\
0.40044	4.077597e-05\\
0.4005401	2.471025e-05\\
0.4006401	2.739853e-05\\
0.4007401	4.002566e-05\\
0.4008401	3.19007e-05\\
0.4009401	1.251966e-05\\
0.4010401	1.38584e-05\\
0.4011401	3.131108e-05\\
0.4012401	4.851751e-05\\
0.4013401	7.767788e-05\\
0.4014401	4.700643e-05\\
0.4015402	3.832384e-05\\
0.4016402	8.038985e-05\\
0.4017402	5.858949e-05\\
0.4018402	5.920385e-05\\
0.4019402	6.330061e-05\\
0.4020402	6.725136e-05\\
0.4021402	3.458492e-05\\
0.4022402	4.304391e-05\\
0.4023402	4.670745e-05\\
0.4024402	3.187074e-05\\
0.4025403	1.410997e-05\\
0.4026403	3.403731e-05\\
0.4027403	4.859369e-05\\
0.4028403	2.57822e-05\\
0.4029403	1.045571e-05\\
0.4030403	2.733252e-05\\
0.4031403	1.663096e-05\\
0.4032403	2.559506e-05\\
0.4033403	2.424068e-05\\
0.4034403	4.113302e-05\\
0.4035404	5.475949e-05\\
0.4036404	5.882046e-05\\
0.4037404	5.711921e-05\\
0.4038404	5.105233e-05\\
0.4039404	8.46406e-05\\
0.4040404	9.677193e-05\\
0.4041404	7.606394e-05\\
0.4042404	4.248153e-05\\
0.4043404	8.705622e-05\\
0.4044404	4.794799e-05\\
0.4045405	4.585662e-05\\
0.4046405	5.929561e-05\\
0.4047405	6.128144e-05\\
0.4048405	6.305063e-05\\
0.4049405	6.775536e-05\\
0.4050405	1.669385e-05\\
0.4051405	1.356519e-05\\
0.4052405	3.48901e-05\\
0.4053405	5.519978e-05\\
0.4054405	4.732745e-05\\
0.4055406	3.018507e-05\\
0.4056406	5.360759e-05\\
0.4057406	4.752957e-05\\
0.4058406	3.867393e-05\\
0.4059406	3.439483e-05\\
0.4060406	6.64244e-05\\
0.4061406	6.163865e-05\\
0.4062406	6.636252e-05\\
0.4063406	5.810888e-05\\
0.4064406	3.801697e-05\\
0.4065407	4.51024e-05\\
0.4066407	8.656727e-05\\
0.4067407	6.805117e-05\\
0.4068407	6.365775e-05\\
0.4069407	6.097403e-05\\
0.4070407	4.745311e-05\\
0.4071407	6.006939e-05\\
0.4072407	3.716086e-05\\
0.4073407	1.225095e-05\\
0.4074407	1.028097e-05\\
0.4075408	4.529431e-05\\
0.4076408	6.169749e-05\\
0.4077408	7.665785e-05\\
0.4078408	8.567033e-05\\
0.4079408	8.642397e-05\\
0.4080408	0.0001020176\\
0.4081408	9.130902e-05\\
0.4082408	8.141145e-05\\
0.4083408	7.444698e-05\\
0.4084408	6.752938e-05\\
0.4085409	5.292067e-05\\
0.4086409	3.90453e-05\\
0.4087409	3.309972e-05\\
0.4088409	4.314512e-05\\
0.4089409	6.872543e-05\\
0.4090409	3.27797e-05\\
0.4091409	5.952393e-05\\
0.4092409	9.378124e-05\\
0.4093409	8.478761e-05\\
0.4094409	6.85763e-05\\
0.409541	8.293683e-05\\
0.409641	6.08038e-05\\
0.409741	6.20212e-05\\
0.409841	0.0001008692\\
0.409941	9.308643e-05\\
0.410041	3.633727e-05\\
0.410141	2.317689e-05\\
0.410241	4.184159e-05\\
0.410341	3.749634e-05\\
0.410441	4.111339e-05\\
0.4105411	7.196789e-05\\
0.4106411	7.477749e-05\\
0.4107411	4.736652e-05\\
0.4108411	5.100037e-05\\
0.4109411	6.396115e-05\\
0.4110411	6.838777e-05\\
0.4111411	9.686667e-05\\
0.4112411	0.0001105041\\
0.4113411	8.605228e-05\\
0.4114411	0.0001001049\\
0.4115412	8.655205e-05\\
0.4116412	8.176533e-05\\
0.4117412	5.597321e-05\\
0.4118412	5.263962e-05\\
0.4119412	4.341789e-05\\
0.4120412	0.000127119\\
0.4121412	0.0001572886\\
0.4122412	9.034743e-05\\
0.4123412	8.987995e-05\\
0.4124412	9.438336e-05\\
0.4125413	0.000130819\\
0.4126413	7.780268e-05\\
0.4127413	6.727317e-05\\
0.4128413	7.577315e-05\\
0.4129413	5.561761e-05\\
0.4130413	5.243261e-05\\
0.4131413	7.052224e-05\\
0.4132413	7.353629e-05\\
0.4133413	4.552421e-05\\
0.4134413	5.657172e-05\\
0.4135414	6.387636e-05\\
0.4136414	7.613158e-05\\
0.4137414	7.807244e-05\\
0.4138414	6.635914e-05\\
0.4139414	6.514304e-05\\
0.4140414	0.0001056208\\
0.4141414	6.161237e-05\\
0.4142414	8.212646e-05\\
0.4143414	0.0001030951\\
0.4144414	0.0001079414\\
0.4145415	0.0001170389\\
0.4146415	0.0001099119\\
0.4147415	0.0001456714\\
0.4148415	0.0001349051\\
0.4149415	0.000104813\\
0.4150415	9.09941e-05\\
0.4151415	8.568923e-05\\
0.4152415	7.606441e-05\\
0.4153415	7.161205e-05\\
0.4154415	7.948421e-05\\
0.4155416	9.1566e-05\\
0.4156416	4.808736e-05\\
0.4157416	6.982834e-05\\
0.4158416	6.586169e-05\\
0.4159416	7.81606e-05\\
0.4160416	0.0001127454\\
0.4161416	0.0001042485\\
0.4162416	8.721969e-05\\
0.4163416	0.0001111197\\
0.4164416	0.0001819319\\
0.4165417	0.0001575713\\
0.4166417	0.0001299887\\
0.4167417	0.0001321209\\
0.4168417	0.0001031439\\
0.4169417	7.404489e-05\\
0.4170417	6.097214e-05\\
0.4171417	8.136796e-05\\
0.4172417	7.48921e-05\\
0.4173417	6.738947e-05\\
0.4174417	7.054593e-05\\
0.4175418	8.441393e-05\\
0.4176418	7.557922e-05\\
0.4177418	0.0001162125\\
0.4178418	0.000110381\\
0.4179418	7.57243e-05\\
0.4180418	0.0001063965\\
0.4181418	0.0001418003\\
0.4182418	0.0001162269\\
0.4183418	0.0001224724\\
0.4184418	0.000116293\\
0.4185419	0.0001639271\\
0.4186419	0.0001591714\\
0.4187419	0.0001122492\\
0.4188419	0.0001364733\\
0.4189419	0.0001334527\\
0.4190419	9.83388e-05\\
0.4191419	9.933334e-05\\
0.4192419	9.655659e-05\\
0.4193419	0.0001090223\\
0.4194419	0.0001010533\\
0.419542	8.217646e-05\\
0.419642	0.0001124769\\
0.419742	0.0001323988\\
0.419842	0.0001733654\\
0.419942	0.0001512082\\
0.420042	0.0001306081\\
0.420142	9.360846e-05\\
0.420242	0.0001131844\\
0.420342	0.0001461146\\
0.420442	0.0001550217\\
0.4205421	0.0001608097\\
0.4206421	0.0001136806\\
0.4207421	0.0001016064\\
0.4208421	0.0001121864\\
0.4209421	6.534996e-05\\
0.4210421	7.287681e-05\\
0.4211421	8.493885e-05\\
0.4212421	9.414303e-05\\
0.4213421	0.0001122449\\
0.4214421	0.0001301322\\
0.4215422	0.0001693256\\
0.4216422	0.0001480065\\
0.4217422	0.0001590606\\
0.4218422	0.0001770962\\
0.4219422	0.0001853826\\
0.4220422	0.0002027289\\
0.4221422	0.0001257514\\
0.4222422	0.0001085975\\
0.4223422	0.0001271135\\
0.4224422	0.0001465654\\
0.4225423	0.0001455057\\
0.4226423	0.0001521568\\
0.4227423	0.0001396901\\
0.4228423	0.0001562638\\
0.4229423	0.000187086\\
0.4230423	0.0001869857\\
0.4231423	0.0001169029\\
0.4232423	0.0001481795\\
0.4233423	0.0001524867\\
0.4234423	0.0001341171\\
0.4235424	0.0001271116\\
0.4236424	0.000105025\\
0.4237424	0.0001387782\\
0.4238424	0.000137201\\
0.4239424	0.0001348238\\
0.4240424	0.000121418\\
0.4241424	0.0001431222\\
0.4242424	0.0001750371\\
0.4243424	0.0001879828\\
0.4244424	0.0002035286\\
0.4245425	0.0001607081\\
0.4246425	0.0001520758\\
0.4247425	0.0001445843\\
0.4248425	0.0001241596\\
0.4249425	0.0001600565\\
0.4250425	0.0001452469\\
0.4251425	0.0001558197\\
0.4252425	0.0002183655\\
0.4253425	0.0001634633\\
0.4254425	0.0001694825\\
0.4255426	0.0001460209\\
0.4256426	0.0001717704\\
0.4257426	0.0001836489\\
0.4258426	0.000202515\\
0.4259426	0.0002174973\\
0.4260426	0.0001949627\\
0.4261426	0.000174986\\
0.4262426	0.000115011\\
0.4263426	0.0001153076\\
0.4264426	0.0002308622\\
0.4265427	0.0001996975\\
0.4266427	0.0001426886\\
0.4267427	0.0001421759\\
0.4268427	0.0001605817\\
0.4269427	0.0001599778\\
0.4270427	0.0002052454\\
0.4271427	0.0002061709\\
0.4272427	0.0002231178\\
0.4273427	0.000219343\\
0.4274427	0.0002866179\\
0.4275428	0.0002778465\\
0.4276428	0.0002247264\\
0.4277428	0.0002006\\
0.4278428	0.0001819519\\
0.4279428	0.0001813429\\
0.4280428	0.0001373801\\
0.4281428	0.0001307764\\
0.4282428	0.0001371318\\
0.4283428	0.0001413833\\
0.4284428	0.0001929383\\
0.4285429	0.0002291109\\
0.4286429	0.0001942777\\
0.4287429	0.0002464958\\
0.4288429	0.000173451\\
0.4289429	0.0002181441\\
0.4290429	0.0002071617\\
0.4291429	0.0002431859\\
0.4292429	0.0002309239\\
0.4293429	0.0002307707\\
0.4294429	0.0002345976\\
0.429543	0.0002683273\\
0.429643	0.0002026161\\
0.429743	0.0001992324\\
0.429843	0.0001818055\\
0.429943	0.0001789123\\
0.430043	0.000156547\\
0.430143	0.0001920751\\
0.430243	0.0002612251\\
0.430343	0.0002528677\\
0.430443	0.0002646316\\
0.4305431	0.000253647\\
0.4306431	0.0002331243\\
0.4307431	0.0002997227\\
0.4308431	0.0003122449\\
0.4309431	0.0002578775\\
0.4310431	0.0002135602\\
0.4311431	0.0002241311\\
0.4312431	0.0002129952\\
0.4313431	0.0001448875\\
0.4314431	0.0001590912\\
0.4315432	0.0001948305\\
0.4316432	0.0001778752\\
0.4317432	0.0002434969\\
0.4318432	0.0002607518\\
0.4319432	0.0002810819\\
0.4320432	0.0002675996\\
0.4321432	0.0002480022\\
0.4322432	0.0002928123\\
0.4323432	0.0002411509\\
0.4324432	0.0002337602\\
0.4325433	0.0002733543\\
0.4326433	0.0002914701\\
0.4327433	0.0002908906\\
0.4328433	0.000232506\\
0.4329433	0.0002757916\\
0.4330433	0.0003092529\\
0.4331433	0.000288732\\
0.4332433	0.0002868845\\
0.4333433	0.0002501643\\
0.4334433	0.0002555161\\
0.4335434	0.0002160778\\
0.4336434	0.0002047698\\
0.4337434	0.0002490298\\
0.4338434	0.0002448451\\
0.4339434	0.0002523559\\
0.4340434	0.0002887985\\
0.4341434	0.0003188139\\
0.4342434	0.0003543973\\
0.4343434	0.0002662361\\
0.4344434	0.0003126929\\
0.4345435	0.000320316\\
0.4346435	0.000309461\\
0.4347435	0.0002277933\\
0.4348435	0.0002942063\\
0.4349435	0.0002559989\\
0.4350435	0.0002773601\\
0.4351435	0.0002339129\\
0.4352435	0.0002971298\\
0.4353435	0.0003385144\\
0.4354435	0.0003578592\\
0.4355436	0.0003645208\\
0.4356436	0.0003568999\\
0.4357436	0.0003147016\\
0.4358436	0.0003252526\\
0.4359436	0.0003168429\\
0.4360436	0.0002739342\\
0.4361436	0.0002452517\\
0.4362436	0.0002628727\\
0.4363436	0.0002984438\\
0.4364436	0.0003394116\\
0.4365437	0.000286753\\
0.4366437	0.0002671459\\
0.4367437	0.0002626827\\
0.4368437	0.0002295019\\
0.4369437	0.0002746783\\
0.4370437	0.0003403751\\
0.4371437	0.0003658576\\
0.4372437	0.0003772206\\
0.4373437	0.0003835039\\
0.4374437	0.0004584115\\
0.4375438	0.0004687518\\
0.4376438	0.0003958275\\
0.4377438	0.0003778426\\
0.4378438	0.0003377219\\
0.4379438	0.0003452941\\
0.4380438	0.000332046\\
0.4381438	0.0003148791\\
0.4382438	0.0003995332\\
0.4383438	0.0004126719\\
0.4384438	0.0003554655\\
0.4385439	0.0003488391\\
0.4386439	0.0003050942\\
0.4387439	0.0003123767\\
0.4388439	0.0003239561\\
0.4389439	0.0002862217\\
0.4390439	0.0003114564\\
0.4391439	0.0003396829\\
0.4392439	0.0003956918\\
0.4393439	0.0004120828\\
0.4394439	0.0003763826\\
0.439544	0.0003794131\\
0.439644	0.0003197307\\
0.439744	0.0003590246\\
0.439844	0.000385561\\
0.439944	0.0004045989\\
0.440044	0.0004436284\\
0.440144	0.0003723254\\
0.440244	0.0004678223\\
0.440344	0.0004692408\\
0.440444	0.0004386266\\
0.4405441	0.0004264706\\
0.4406441	0.0004244792\\
0.4407441	0.0004662796\\
0.4408441	0.0003930091\\
0.4409441	0.0003461962\\
0.4410441	0.0003905332\\
0.4411441	0.0003277753\\
0.4412441	0.0003761963\\
0.4413441	0.000367432\\
0.4414441	0.0003909189\\
0.4415442	0.0004017145\\
0.4416442	0.0003673552\\
0.4417442	0.0004482216\\
0.4418442	0.0005905026\\
0.4419442	0.0006169264\\
0.4420442	0.0005836475\\
0.4421442	0.0004783782\\
0.4422442	0.0004517613\\
0.4423442	0.0004300426\\
0.4424442	0.0003891855\\
0.4425443	0.0004038751\\
0.4426443	0.0003661319\\
0.4427443	0.0004403943\\
0.4428443	0.00047024\\
0.4429443	0.0003750957\\
0.4430443	0.0003826572\\
0.4431443	0.0004373761\\
0.4432443	0.0004962367\\
0.4433443	0.00048704\\
0.4434443	0.0004973388\\
0.4435444	0.0006106633\\
0.4436444	0.0005604564\\
0.4437444	0.0005152899\\
0.4438444	0.0004410146\\
0.4439444	0.0004783178\\
0.4440444	0.000457152\\
0.4441444	0.0004446465\\
0.4442444	0.0004662558\\
0.4443444	0.0004624159\\
0.4444444	0.0003761841\\
0.4445445	0.0004346662\\
0.4446445	0.000452662\\
0.4447445	0.0005444553\\
0.4448445	0.0006346161\\
0.4449445	0.0005933329\\
0.4450445	0.0006184321\\
0.4451445	0.0006581002\\
0.4452445	0.0006654218\\
0.4453445	0.0005898409\\
0.4454445	0.0004946493\\
0.4455446	0.0004518563\\
0.4456446	0.0005129356\\
0.4457446	0.000568644\\
0.4458446	0.0005880607\\
0.4459446	0.0004839092\\
0.4460446	0.000626661\\
0.4461446	0.0005641228\\
0.4462446	0.0005748751\\
0.4463446	0.0005415761\\
0.4464446	0.0005570739\\
0.4465447	0.0005167148\\
0.4466447	0.0004711142\\
0.4467447	0.0004172915\\
0.4468447	0.0004630028\\
0.4469447	0.0005461979\\
0.4470447	0.0005150078\\
0.4471447	0.0005207002\\
0.4472447	0.0005255276\\
0.4473447	0.0006337718\\
0.4474447	0.0005914419\\
0.4475448	0.0005907083\\
0.4476448	0.0007117677\\
0.4477448	0.0008072882\\
0.4478448	0.0008129181\\
0.4479448	0.0006922719\\
0.4480448	0.000632294\\
0.4481448	0.0006547323\\
0.4482448	0.0007167224\\
0.4483448	0.0006421963\\
0.4484448	0.0005367892\\
0.4485449	0.0005917691\\
0.4486449	0.0006268746\\
0.4487449	0.0005642764\\
0.4488449	0.0005503103\\
0.4489449	0.0006109882\\
0.4490449	0.0006354881\\
0.4491449	0.0007237502\\
0.4492449	0.0007091287\\
0.4493449	0.0006717591\\
0.4494449	0.0006715045\\
0.449545	0.0005964735\\
0.449645	0.0006464543\\
0.449745	0.0006217391\\
0.449845	0.0005936714\\
0.449945	0.0006945775\\
0.450045	0.0007764023\\
0.450145	0.0008040657\\
0.450245	0.0007225745\\
0.450345	0.0006927711\\
0.450445	0.0007370154\\
0.4505451	0.0007101099\\
0.4506451	0.0007143342\\
0.4507451	0.0006002464\\
0.4508451	0.0006785076\\
0.4509451	0.000695396\\
0.4510451	0.0007029022\\
0.4511451	0.0007001306\\
0.4512451	0.0006813439\\
0.4513451	0.0007214285\\
0.4514451	0.0007497251\\
0.4515452	0.0007352803\\
0.4516452	0.0007560172\\
0.4517452	0.000831881\\
0.4518452	0.0008502925\\
0.4519452	0.000733151\\
0.4520452	0.0007318036\\
0.4521452	0.0008743856\\
0.4522452	0.0008204526\\
0.4523452	0.0008584035\\
0.4524452	0.0007704377\\
0.4525453	0.0008182677\\
0.4526453	0.0008784933\\
0.4527453	0.0008552697\\
0.4528453	0.000804406\\
0.4529453	0.000797859\\
0.4530453	0.0008057772\\
0.4531453	0.0007772179\\
0.4532453	0.000774957\\
0.4533453	0.000898939\\
0.4534453	0.0008160864\\
0.4535454	0.0007552686\\
0.4536454	0.0007681903\\
0.4537454	0.0007902487\\
0.4538454	0.0007796803\\
0.4539454	0.0006893397\\
0.4540454	0.0007802532\\
0.4541454	0.0008589218\\
0.4542454	0.0008816979\\
0.4543454	0.0009145115\\
0.4544454	0.0009857992\\
0.4545455	0.001067395\\
0.4546455	0.000929715\\
0.4547455	0.0008402581\\
0.4548455	0.000859173\\
0.4549455	0.0008907547\\
0.4550455	0.0009180331\\
0.4551455	0.0009453097\\
0.4552455	0.0008076035\\
0.4553455	0.0008032152\\
0.4554455	0.0009201626\\
0.4555456	0.001012241\\
0.4556456	0.001036373\\
0.4557456	0.001044863\\
0.4558456	0.001057086\\
0.4559456	0.0009963704\\
0.4560456	0.0009455024\\
0.4561456	0.001018381\\
0.4562456	0.001036\\
0.4563456	0.0009363568\\
0.4564456	0.001013312\\
0.4565457	0.00108555\\
0.4566457	0.001046963\\
0.4567457	0.0009157915\\
0.4568457	0.0008605273\\
0.4569457	0.0009135827\\
0.4570457	0.0009431873\\
0.4571457	0.0009428514\\
0.4572457	0.0009807576\\
0.4573457	0.0008929126\\
0.4574457	0.0009420069\\
0.4575458	0.0009113309\\
0.4576458	0.001018963\\
0.4577458	0.001103489\\
0.4578458	0.001138698\\
0.4579458	0.001144901\\
0.4580458	0.001058214\\
0.4581458	0.001152456\\
0.4582458	0.001214004\\
0.4583458	0.001309887\\
0.4584458	0.001237799\\
0.4585459	0.001142444\\
0.4586459	0.001224605\\
0.4587459	0.001019092\\
0.4588459	0.001023559\\
0.4589459	0.001096297\\
0.4590459	0.001008728\\
0.4591459	0.0009565091\\
0.4592459	0.001016484\\
0.4593459	0.001036063\\
0.4594459	0.00121625\\
0.459546	0.001251675\\
0.459646	0.001159686\\
0.459746	0.001221762\\
0.459846	0.001319924\\
0.459946	0.001321511\\
0.460046	0.001321867\\
0.460146	0.001325401\\
0.460246	0.001220789\\
0.460346	0.001244795\\
0.460446	0.001161452\\
0.4605461	0.001110332\\
0.4606461	0.001095727\\
0.4607461	0.001031134\\
0.4608461	0.001121119\\
0.4609461	0.001265289\\
0.4610461	0.001311196\\
0.4611461	0.001264952\\
0.4612461	0.001125335\\
0.4613461	0.00123239\\
0.4614461	0.001391695\\
0.4615462	0.001482227\\
0.4616462	0.00138382\\
0.4617462	0.001403407\\
0.4618462	0.001383451\\
0.4619462	0.001280761\\
0.4620462	0.0011695\\
0.4621462	0.001230501\\
0.4622462	0.001390749\\
0.4623462	0.001433982\\
0.4624462	0.001330002\\
0.4625463	0.001360847\\
0.4626463	0.00144205\\
0.4627463	0.001342016\\
0.4628463	0.001268581\\
0.4629463	0.001351874\\
0.4630463	0.00147618\\
0.4631463	0.001464812\\
0.4632463	0.00152462\\
0.4633463	0.001429852\\
0.4634463	0.001483188\\
0.4635464	0.001579004\\
0.4636464	0.001457732\\
0.4637464	0.001510466\\
0.4638464	0.001500177\\
0.4639464	0.001348764\\
0.4640464	0.001409332\\
0.4641464	0.001480013\\
0.4642464	0.001326139\\
0.4643464	0.001401954\\
0.4644464	0.00151298\\
0.4645465	0.001625706\\
0.4646465	0.001468001\\
0.4647465	0.001378039\\
0.4648465	0.001569204\\
0.4649465	0.001573752\\
0.4650465	0.001599366\\
0.4651465	0.001642667\\
0.4652465	0.001676057\\
0.4653465	0.001510481\\
0.4654465	0.001705917\\
0.4655466	0.001619958\\
0.4656466	0.001404942\\
0.4657466	0.001501645\\
0.4658466	0.001599989\\
0.4659466	0.001645867\\
0.4660466	0.001606798\\
0.4661466	0.001809332\\
0.4662466	0.001825436\\
0.4663466	0.001794714\\
0.4664466	0.00169463\\
0.4665467	0.001635106\\
0.4666467	0.00166118\\
0.4667467	0.001612485\\
0.4668467	0.001689047\\
0.4669467	0.001865283\\
0.4670467	0.001887834\\
0.4671467	0.001815502\\
0.4672467	0.001818017\\
0.4673467	0.001671338\\
0.4674467	0.00165663\\
0.4675468	0.001675017\\
0.4676468	0.001705223\\
0.4677468	0.001669977\\
0.4678468	0.001663086\\
0.4679468	0.001769618\\
0.4680468	0.00176661\\
0.4681468	0.00188842\\
0.4682468	0.001954134\\
0.4683468	0.001976946\\
0.4684468	0.001905504\\
0.4685469	0.001895701\\
0.4686469	0.001883657\\
0.4687469	0.001997104\\
0.4688469	0.001914612\\
0.4689469	0.00195957\\
0.4690469	0.00189386\\
0.4691469	0.002021815\\
0.4692469	0.002028163\\
0.4693469	0.001930715\\
0.4694469	0.001879601\\
0.469547	0.001939787\\
0.469647	0.001927207\\
0.469747	0.001855183\\
0.469847	0.001948519\\
0.469947	0.002007359\\
0.470047	0.002069558\\
0.470147	0.002097246\\
0.470247	0.002038345\\
0.470347	0.001969212\\
0.470447	0.002074747\\
0.4705471	0.00195105\\
0.4706471	0.002126826\\
0.4707471	0.002085472\\
0.4708471	0.002072218\\
0.4709471	0.002117018\\
0.4710471	0.002302198\\
0.4711471	0.002374832\\
0.4712471	0.002307948\\
0.4713471	0.002356754\\
0.4714471	0.002327915\\
0.4715472	0.002324915\\
0.4716472	0.002302721\\
0.4717472	0.002071711\\
0.4718472	0.00216029\\
0.4719472	0.002032002\\
0.4720472	0.002150715\\
0.4721472	0.002138388\\
0.4722472	0.002097976\\
0.4723472	0.002313401\\
0.4724472	0.002229894\\
0.4725473	0.002275767\\
0.4726473	0.002306186\\
0.4727473	0.002349951\\
0.4728473	0.002369865\\
0.4729473	0.002452967\\
0.4730473	0.00238007\\
0.4731473	0.002418152\\
0.4732473	0.002458201\\
0.4733473	0.002409498\\
0.4734473	0.002228443\\
0.4735474	0.002250249\\
0.4736474	0.002327477\\
0.4737474	0.002557853\\
0.4738474	0.002597855\\
0.4739474	0.002710832\\
0.4740474	0.00255883\\
0.4741474	0.002675243\\
0.4742474	0.00287694\\
0.4743474	0.00267003\\
0.4744474	0.002595671\\
0.4745475	0.002349103\\
0.4746475	0.002362446\\
0.4747475	0.002506457\\
0.4748475	0.002544954\\
0.4749475	0.002579694\\
0.4750475	0.002690201\\
0.4751475	0.002548166\\
0.4752475	0.002504585\\
0.4753475	0.002650438\\
0.4754475	0.0026175\\
0.4755476	0.002775252\\
0.4756476	0.002654069\\
0.4757476	0.002699388\\
0.4758476	0.002851199\\
0.4759476	0.002870905\\
0.4760476	0.002939042\\
0.4761476	0.002913343\\
0.4762476	0.002928902\\
0.4763476	0.002891518\\
0.4764476	0.002988216\\
0.4765477	0.002750796\\
0.4766477	0.002676456\\
0.4767477	0.002419684\\
0.4768477	0.002495846\\
0.4769477	0.002712797\\
0.4770477	0.002884711\\
0.4771477	0.003151713\\
0.4772477	0.002991782\\
0.4773477	0.003213448\\
0.4774477	0.003300293\\
0.4775478	0.003112174\\
0.4776478	0.003001227\\
0.4777478	0.003082949\\
0.4778478	0.003106303\\
0.4779478	0.003057572\\
0.4780478	0.002916781\\
0.4781478	0.002680711\\
0.4782478	0.00300466\\
0.4783478	0.003228918\\
0.4784478	0.003034591\\
0.4785479	0.002975456\\
0.4786479	0.003273843\\
0.4787479	0.00362176\\
0.4788479	0.003393063\\
0.4789479	0.003321667\\
0.4790479	0.003351317\\
0.4791479	0.003388673\\
0.4792479	0.003306118\\
0.4793479	0.003450913\\
0.4794479	0.003327208\\
0.479548	0.002993746\\
0.479648	0.003155349\\
0.479748	0.003290257\\
0.479848	0.003165505\\
0.479948	0.003350047\\
0.480048	0.003320663\\
0.480148	0.003112665\\
0.480248	0.003054759\\
0.480348	0.00335564\\
0.480448	0.003668548\\
0.4805481	0.003748392\\
0.4806481	0.003493825\\
0.4807481	0.003454105\\
0.4808481	0.003589585\\
0.4809481	0.003527292\\
0.4810481	0.003302478\\
0.4811481	0.003352957\\
0.4812481	0.00370258\\
0.4813481	0.003846912\\
0.4814481	0.003827059\\
0.4815482	0.004035007\\
0.4816482	0.00422027\\
0.4817482	0.004018706\\
0.4818482	0.003783053\\
0.4819482	0.003912253\\
0.4820482	0.003952913\\
0.4821482	0.003675173\\
0.4822482	0.003748043\\
0.4823482	0.003657418\\
0.4824482	0.003580676\\
0.4825483	0.003401949\\
0.4826483	0.003610704\\
0.4827483	0.003600634\\
0.4828483	0.003793001\\
0.4829483	0.003929747\\
0.4830483	0.003969883\\
0.4831483	0.004008457\\
0.4832483	0.003948796\\
0.4833483	0.004047823\\
0.4834483	0.004032749\\
0.4835484	0.004133497\\
0.4836484	0.004277509\\
0.4837484	0.004189791\\
0.4838484	0.004221586\\
0.4839484	0.004383014\\
0.4840484	0.004082533\\
0.4841484	0.00407683\\
0.4842484	0.004258952\\
0.4843484	0.004322573\\
0.4844484	0.003988606\\
0.4845485	0.004200265\\
0.4846485	0.004295494\\
0.4847485	0.004082913\\
0.4848485	0.004207849\\
0.4849485	0.004362834\\
0.4850485	0.004248953\\
0.4851485	0.004375019\\
0.4852485	0.004319571\\
0.4853485	0.004373037\\
0.4854485	0.004580765\\
0.4855486	0.004733413\\
0.4856486	0.004954649\\
0.4857486	0.004727256\\
0.4858486	0.004948613\\
0.4859486	0.004704855\\
0.4860486	0.0046089\\
0.4861486	0.004569641\\
0.4862486	0.004605763\\
0.4863486	0.004437644\\
0.4864486	0.004341497\\
0.4865487	0.004519451\\
0.4866487	0.004439916\\
0.4867487	0.004398793\\
0.4868487	0.004591341\\
0.4869487	0.004896996\\
0.4870487	0.004877218\\
0.4871487	0.004955971\\
0.4872487	0.00533243\\
0.4873487	0.005105121\\
0.4874487	0.004817865\\
0.4875488	0.004750349\\
0.4876488	0.004934386\\
0.4877488	0.005021655\\
0.4878488	0.004828932\\
0.4879488	0.004774604\\
0.4880488	0.005035163\\
0.4881488	0.005195652\\
0.4882488	0.00519769\\
0.4883488	0.005222244\\
0.4884488	0.005150517\\
0.4885489	0.005170999\\
0.4886489	0.005099494\\
0.4887489	0.005413285\\
0.4888489	0.005566906\\
0.4889489	0.005444772\\
0.4890489	0.005258478\\
0.4891489	0.005534228\\
0.4892489	0.005622213\\
0.4893489	0.005536298\\
0.4894489	0.005722483\\
0.489549	0.00547655\\
0.489649	0.005067017\\
0.489749	0.005507256\\
0.489849	0.005716474\\
0.489949	0.005550532\\
0.490049	0.005109614\\
0.490149	0.005404712\\
0.490249	0.005711987\\
0.490349	0.005619499\\
0.490449	0.005529244\\
0.4905491	0.005400029\\
0.4906491	0.005608097\\
0.4907491	0.005758263\\
0.4908491	0.005899573\\
0.4909491	0.005763354\\
0.4910491	0.005922383\\
0.4911491	0.006081179\\
0.4912491	0.00589583\\
0.4913491	0.00596544\\
0.4914491	0.006490471\\
0.4915492	0.006166909\\
0.4916492	0.00591222\\
0.4917492	0.005846905\\
0.4918492	0.005827432\\
0.4919492	0.005959681\\
0.4920492	0.006129385\\
0.4921492	0.006493798\\
0.4922492	0.00653285\\
0.4923492	0.006778223\\
0.4924492	0.006659844\\
0.4925493	0.006641735\\
0.4926493	0.006192009\\
0.4927493	0.006020197\\
0.4928493	0.006253567\\
0.4929493	0.006429185\\
0.4930493	0.006354228\\
0.4931493	0.00637863\\
0.4932493	0.00634865\\
0.4933493	0.006277455\\
0.4934493	0.006429837\\
0.4935494	0.006551081\\
0.4936494	0.006736832\\
0.4937494	0.006712473\\
0.4938494	0.006724685\\
0.4939494	0.006608187\\
0.4940494	0.007099258\\
0.4941494	0.007234683\\
0.4942494	0.006889992\\
0.4943494	0.006856018\\
0.4944494	0.006898273\\
0.4945495	0.007241447\\
0.4946495	0.007023417\\
0.4947495	0.006664043\\
0.4948495	0.006851049\\
0.4949495	0.006727496\\
0.4950495	0.007031112\\
0.4951495	0.007103277\\
0.4952495	0.007148256\\
0.4953495	0.007077475\\
0.4954495	0.007445103\\
0.4955496	0.007511633\\
0.4956496	0.007653357\\
0.4957496	0.007496475\\
0.4958496	0.007397402\\
0.4959496	0.007540961\\
0.4960496	0.007786501\\
0.4961496	0.007516847\\
0.4962496	0.00762105\\
0.4963496	0.007568184\\
0.4964496	0.007188054\\
0.4965497	0.007492354\\
0.4966497	0.007610497\\
0.4967497	0.007986836\\
0.4968497	0.007969096\\
0.4969497	0.007698221\\
0.4970497	0.00774832\\
0.4971497	0.007625749\\
0.4972497	0.007832936\\
0.4973497	0.00815561\\
0.4974497	0.007861228\\
0.4975498	0.00793742\\
0.4976498	0.008174138\\
0.4977498	0.008195592\\
0.4978498	0.007933528\\
0.4979498	0.008352071\\
0.4980498	0.008248278\\
0.4981498	0.00859916\\
0.4982498	0.008670628\\
0.4983498	0.008389313\\
0.4984498	0.008330265\\
0.4985499	0.008094371\\
0.4986499	0.007872119\\
0.4987499	0.008135042\\
0.4988499	0.008432257\\
0.4989499	0.008499153\\
0.4990499	0.008650618\\
0.4991499	0.008943719\\
0.4992499	0.008850925\\
0.4993499	0.008737606\\
0.4994499	0.008918299\\
0.49955	0.009033562\\
0.49965	0.009021799\\
0.49975	0.008949471\\
0.49985	0.009009688\\
0.49995	0.008947399\\
0.50005	0.009155313\\
0.50015	0.009213411\\
0.50025	0.009345245\\
0.50035	0.009236517\\
0.50045	0.009087522\\
0.5005501	0.00911854\\
0.5006501	0.009149282\\
0.5007501	0.009230069\\
0.5008501	0.00913978\\
0.5009501	0.009225558\\
0.5010501	0.009317322\\
0.5011501	0.009595349\\
0.5012501	0.009630063\\
0.5013501	0.009777305\\
0.5014501	0.009900513\\
0.5015502	0.009505482\\
0.5016502	0.009624662\\
0.5017502	0.009660942\\
0.5018502	0.009760637\\
0.5019502	0.009822771\\
0.5020502	0.01007069\\
0.5021502	0.01014866\\
0.5022502	0.01019404\\
0.5023502	0.009888185\\
0.5024502	0.01037665\\
0.5025503	0.01046227\\
0.5026503	0.01014418\\
0.5027503	0.009849868\\
0.5028503	0.01043199\\
0.5029503	0.01067255\\
0.5030503	0.01014014\\
0.5031503	0.01036622\\
0.5032503	0.01048609\\
0.5033503	0.01064202\\
0.5034503	0.01047516\\
0.5035504	0.01065908\\
0.5036504	0.01072488\\
0.5037504	0.01098054\\
0.5038504	0.0109544\\
0.5039504	0.01044764\\
0.5040504	0.0105382\\
0.5041504	0.01074175\\
0.5042504	0.01114183\\
0.5043504	0.0108639\\
0.5044504	0.01097883\\
0.5045505	0.01129112\\
0.5046505	0.01099391\\
0.5047505	0.01135359\\
0.5048505	0.01163166\\
0.5049505	0.01186458\\
0.5050505	0.01172482\\
0.5051505	0.01193807\\
0.5052505	0.01171734\\
0.5053505	0.01197837\\
0.5054505	0.01208118\\
0.5055506	0.01157008\\
0.5056506	0.01143548\\
0.5057506	0.01161354\\
0.5058506	0.01172633\\
0.5059506	0.01131697\\
0.5060506	0.01091779\\
0.5061506	0.0113937\\
0.5062506	0.01183263\\
0.5063506	0.0116188\\
0.5064506	0.01197831\\
0.5065507	0.01223936\\
0.5066507	0.01246206\\
0.5067507	0.01297372\\
0.5068507	0.01331844\\
0.5069507	0.01303273\\
0.5070507	0.01310068\\
0.5071507	0.0126983\\
0.5072507	0.01265269\\
0.5073507	0.01235784\\
0.5074507	0.01218552\\
0.5075508	0.0126348\\
0.5076508	0.01288284\\
0.5077508	0.01322251\\
0.5078508	0.01308554\\
0.5079508	0.01318154\\
0.5080508	0.01324204\\
0.5081508	0.01340212\\
0.5082508	0.01324636\\
0.5083508	0.01334435\\
0.5084508	0.01316432\\
0.5085509	0.01347983\\
0.5086509	0.01361146\\
0.5087509	0.01335804\\
0.5088509	0.01333189\\
0.5089509	0.01317265\\
0.5090509	0.01351155\\
0.5091509	0.01366815\\
0.5092509	0.01396391\\
0.5093509	0.01387317\\
0.5094509	0.01423777\\
0.509551	0.01392\\
0.509651	0.01392731\\
0.509751	0.01423306\\
0.509851	0.01425157\\
0.509951	0.01477771\\
0.510051	0.0144282\\
0.510151	0.01435922\\
0.510251	0.01464864\\
0.510351	0.01463726\\
0.510451	0.01467309\\
0.5105511	0.01421916\\
0.5106511	0.01395411\\
0.5107511	0.01443417\\
0.5108511	0.01512011\\
0.5109511	0.01541926\\
0.5110511	0.01528367\\
0.5111511	0.015353\\
0.5112511	0.01541803\\
0.5113511	0.01538943\\
0.5114511	0.01541966\\
0.5115512	0.01560315\\
0.5116512	0.01569458\\
0.5117512	0.01534265\\
0.5118512	0.01521469\\
0.5119512	0.01541291\\
0.5120512	0.01593843\\
0.5121512	0.01553578\\
0.5122512	0.0158755\\
0.5123512	0.01602742\\
0.5124512	0.01562088\\
0.5125513	0.01580826\\
0.5126513	0.01628183\\
0.5127513	0.01646148\\
0.5128513	0.01643719\\
0.5129513	0.01669327\\
0.5130513	0.01723998\\
0.5131513	0.01688239\\
0.5132513	0.01640141\\
0.5133513	0.0163313\\
0.5134513	0.01649363\\
0.5135514	0.01698633\\
0.5136514	0.01685629\\
0.5137514	0.01613646\\
0.5138514	0.01586846\\
0.5139514	0.01606489\\
0.5140514	0.01667178\\
0.5141514	0.0172548\\
0.5142514	0.01751283\\
0.5143514	0.01794344\\
0.5144514	0.01848066\\
0.5145515	0.018161\\
0.5146515	0.01795214\\
0.5147515	0.01811335\\
0.5148515	0.01805408\\
0.5149515	0.01830009\\
0.5150515	0.0180589\\
0.5151515	0.01863615\\
0.5152515	0.01844424\\
0.5153515	0.01822081\\
0.5154515	0.0184488\\
0.5155516	0.01825643\\
0.5156516	0.01827963\\
0.5157516	0.01844559\\
0.5158516	0.01850983\\
0.5159516	0.01884974\\
0.5160516	0.01823624\\
0.5161516	0.01861567\\
0.5162516	0.018617\\
0.5163516	0.01888072\\
0.5164516	0.01891015\\
0.5165517	0.01847776\\
0.5166517	0.01888476\\
0.5167517	0.01944783\\
0.5168517	0.01953094\\
0.5169517	0.01945501\\
0.5170517	0.01974139\\
0.5171517	0.02019705\\
0.5172517	0.02020451\\
0.5173517	0.02009462\\
0.5174517	0.02023852\\
0.5175518	0.02020433\\
0.5176518	0.02075489\\
0.5177518	0.02005284\\
0.5178518	0.02000179\\
0.5179518	0.02050629\\
0.5180518	0.02055078\\
0.5181518	0.02051199\\
0.5182518	0.02052591\\
0.5183518	0.02087579\\
0.5184518	0.02148867\\
0.5185519	0.02171671\\
0.5186519	0.0211073\\
0.5187519	0.02099006\\
0.5188519	0.02162311\\
0.5189519	0.02099753\\
0.5190519	0.02099223\\
0.5191519	0.02119316\\
0.5192519	0.02109024\\
0.5193519	0.02151253\\
0.5194519	0.02239425\\
0.519552	0.02263682\\
0.519652	0.0220723\\
0.519752	0.02187216\\
0.519852	0.02246897\\
0.519952	0.0230352\\
0.520052	0.02253634\\
0.520152	0.02222494\\
0.520252	0.02260038\\
0.520352	0.02252044\\
0.520452	0.02241908\\
0.5205521	0.02226085\\
0.5206521	0.02204503\\
0.5207521	0.02301164\\
0.5208521	0.02330849\\
0.5209521	0.0232292\\
0.5210521	0.02306282\\
0.5211521	0.02363738\\
0.5212521	0.02422842\\
0.5213521	0.02423699\\
0.5214521	0.02422433\\
0.5215522	0.02397115\\
0.5216522	0.02465751\\
0.5217522	0.02465823\\
0.5218522	0.02414963\\
0.5219522	0.02423277\\
0.5220522	0.02496511\\
0.5221522	0.02460083\\
0.5222522	0.02459528\\
0.5223522	0.02434254\\
0.5224522	0.02490581\\
0.5225523	0.024481\\
0.5226523	0.02453407\\
0.5227523	0.02499808\\
0.5228523	0.02556827\\
0.5229523	0.02615646\\
0.5230523	0.02535862\\
0.5231523	0.02543725\\
0.5232523	0.02553768\\
0.5233523	0.02614129\\
0.5234523	0.02652944\\
0.5235524	0.02661781\\
0.5236524	0.02649856\\
0.5237524	0.02640007\\
0.5238524	0.02619562\\
0.5239524	0.02653575\\
0.5240524	0.02626371\\
0.5241524	0.02629248\\
0.5242524	0.02609728\\
0.5243524	0.02657674\\
0.5244524	0.02694489\\
0.5245525	0.02706737\\
0.5246525	0.02657979\\
0.5247525	0.02707142\\
0.5248525	0.02787296\\
0.5249525	0.02843184\\
0.5250525	0.02872942\\
0.5251525	0.02909788\\
0.5252525	0.02855504\\
0.5253525	0.0289149\\
0.5254525	0.02893257\\
0.5255526	0.02833285\\
0.5256526	0.02861007\\
0.5257526	0.02776159\\
0.5258526	0.02811165\\
0.5259526	0.02827584\\
0.5260526	0.0292082\\
0.5261526	0.02908715\\
0.5262526	0.0290326\\
0.5263526	0.02962143\\
0.5264526	0.03016191\\
0.5265527	0.03022669\\
0.5266527	0.02966134\\
0.5267527	0.02973049\\
0.5268527	0.03051553\\
0.5269527	0.02966349\\
0.5270527	0.03008452\\
0.5271527	0.03050294\\
0.5272527	0.03001557\\
0.5273527	0.03091019\\
0.5274527	0.03050069\\
0.5275528	0.03077923\\
0.5276528	0.0313766\\
0.5277528	0.03119438\\
0.5278528	0.03104723\\
0.5279528	0.03141678\\
0.5280528	0.03181603\\
0.5281528	0.03207626\\
0.5282528	0.0321546\\
0.5283528	0.03235945\\
0.5284528	0.03184807\\
0.5285529	0.03215235\\
0.5286529	0.0323189\\
0.5287529	0.0327411\\
0.5288529	0.0334414\\
0.5289529	0.03313407\\
0.5290529	0.03297651\\
0.5291529	0.03347636\\
0.5292529	0.03327595\\
0.5293529	0.03304634\\
0.5294529	0.03320006\\
0.529553	0.03408912\\
0.529653	0.03365764\\
0.529753	0.03368131\\
0.529853	0.03405197\\
0.529953	0.03355483\\
0.530053	0.03435926\\
0.530153	0.03441717\\
0.530253	0.03429002\\
0.530353	0.0346018\\
0.530453	0.03500629\\
0.5305531	0.03521775\\
0.5306531	0.03586315\\
0.5307531	0.03544989\\
0.5308531	0.03610359\\
0.5309531	0.036477\\
0.5310531	0.03646354\\
0.5311531	0.03654321\\
0.5312531	0.03626331\\
0.5313531	0.03686383\\
0.5314531	0.03630447\\
0.5315532	0.03540812\\
0.5316532	0.03599641\\
0.5317532	0.03622338\\
0.5318532	0.03729852\\
0.5319532	0.03779091\\
0.5320532	0.0372871\\
0.5321532	0.03739053\\
0.5322532	0.03735344\\
0.5323532	0.03768818\\
0.5324532	0.0383501\\
0.5325533	0.03830032\\
0.5326533	0.03851459\\
0.5327533	0.03865531\\
0.5328533	0.03888769\\
0.5329533	0.03862226\\
0.5330533	0.03890874\\
0.5331533	0.03905725\\
0.5332533	0.03908864\\
0.5333533	0.03957341\\
0.5334533	0.03931285\\
0.5335534	0.03995847\\
0.5336534	0.03936091\\
0.5337534	0.04015043\\
0.5338534	0.04078902\\
0.5339534	0.04102766\\
0.5340534	0.04137291\\
0.5341534	0.04194618\\
0.5342534	0.04145153\\
0.5343534	0.04112573\\
0.5344534	0.04096386\\
0.5345535	0.04102397\\
0.5346535	0.04147078\\
0.5347535	0.041489\\
0.5348535	0.0414124\\
0.5349535	0.04217747\\
0.5350535	0.04224598\\
0.5351535	0.04242245\\
0.5352535	0.04273739\\
0.5353535	0.04300402\\
0.5354535	0.04255735\\
0.5355536	0.04277807\\
0.5356536	0.04333525\\
0.5357536	0.04318971\\
0.5358536	0.04372023\\
0.5359536	0.04375226\\
0.5360536	0.04394735\\
0.5361536	0.04441322\\
0.5362536	0.04498625\\
0.5363536	0.04596818\\
0.5364536	0.04593455\\
0.5365537	0.04600166\\
0.5366537	0.04536353\\
0.5367537	0.04494585\\
0.5368537	0.04551404\\
0.5369537	0.04491146\\
0.5370537	0.04503648\\
0.5371537	0.0452876\\
0.5372537	0.04617911\\
0.5373537	0.04565487\\
0.5374537	0.04620235\\
0.5375538	0.04743583\\
0.5376538	0.04787723\\
0.5377538	0.04740744\\
0.5378538	0.04777313\\
0.5379538	0.04776306\\
0.5380538	0.04827374\\
0.5381538	0.04900488\\
0.5382538	0.04811281\\
0.5383538	0.04861375\\
0.5384538	0.04924641\\
0.5385539	0.0488927\\
0.5386539	0.04825308\\
0.5387539	0.04871488\\
0.5388539	0.04961255\\
0.5389539	0.05076925\\
0.5390539	0.05059654\\
0.5391539	0.05016249\\
0.5392539	0.05053003\\
0.5393539	0.05074199\\
0.5394539	0.05021331\\
0.539554	0.04965676\\
0.539654	0.05010974\\
0.539754	0.04989783\\
0.539854	0.05053769\\
0.539954	0.05084937\\
0.540054	0.05081262\\
0.540154	0.05159508\\
0.540254	0.05221044\\
0.540354	0.05228407\\
0.540454	0.05339094\\
0.5405541	0.05451845\\
0.5406541	0.05493616\\
0.5407541	0.05451722\\
0.5408541	0.05478492\\
0.5409541	0.05524422\\
0.5410541	0.05437599\\
0.5411541	0.05381077\\
0.5412541	0.05350453\\
0.5413541	0.05486058\\
0.5414541	0.05497088\\
0.5415542	0.05467826\\
0.5416542	0.05494415\\
0.5417542	0.05535285\\
0.5418542	0.05492369\\
0.5419542	0.0552103\\
0.5420542	0.05597699\\
0.5421542	0.05704333\\
0.5422542	0.0575701\\
0.5423542	0.05801181\\
0.5424542	0.05730855\\
0.5425543	0.05639691\\
0.5426543	0.05655114\\
0.5427543	0.05758821\\
0.5428543	0.0575716\\
0.5429543	0.05790873\\
0.5430543	0.05770322\\
0.5431543	0.05824508\\
0.5432543	0.05864743\\
0.5433543	0.05919396\\
0.5434543	0.05963498\\
0.5435544	0.0601568\\
0.5436544	0.061047\\
0.5437544	0.060664\\
0.5438544	0.06079703\\
0.5439544	0.06060492\\
0.5440544	0.06020558\\
0.5441544	0.06089257\\
0.5442544	0.06179448\\
0.5443544	0.06220159\\
0.5444544	0.06187543\\
0.5445545	0.06158383\\
0.5446545	0.06174941\\
0.5447545	0.06338605\\
0.5448545	0.06395691\\
0.5449545	0.0637278\\
0.5450545	0.06306281\\
0.5451545	0.06291205\\
0.5452545	0.06372639\\
0.5453545	0.06309095\\
0.5454545	0.0639614\\
0.5455546	0.06482499\\
0.5456546	0.0652087\\
0.5457546	0.06541917\\
0.5458546	0.06574604\\
0.5459546	0.06574014\\
0.5460546	0.06542345\\
0.5461546	0.0658914\\
0.5462546	0.06633405\\
0.5463546	0.0654029\\
0.5464546	0.06561435\\
0.5465547	0.06726101\\
0.5466547	0.06780156\\
0.5467547	0.06805312\\
0.5468547	0.06722037\\
0.5469547	0.06816018\\
0.5470547	0.06879405\\
0.5471547	0.0690133\\
0.5472547	0.06931064\\
0.5473547	0.07038943\\
0.5474547	0.07002029\\
0.5475548	0.06952341\\
0.5476548	0.06962056\\
0.5477548	0.06996142\\
0.5478548	0.06985196\\
0.5479548	0.07026357\\
0.5480548	0.07074329\\
0.5481548	0.07203345\\
0.5482548	0.07308447\\
0.5483548	0.071936\\
0.5484548	0.07211707\\
0.5485549	0.07206892\\
0.5486549	0.07303562\\
0.5487549	0.07434574\\
0.5488549	0.07518176\\
0.5489549	0.0742338\\
0.5490549	0.07418737\\
0.5491549	0.07412587\\
0.5492549	0.07479269\\
0.5493549	0.07463342\\
0.5494549	0.07451053\\
0.549555	0.0752754\\
0.549655	0.07521869\\
0.549755	0.07423333\\
0.549855	0.07553655\\
0.549955	0.07618892\\
0.550055	0.07634626\\
0.550155	0.07670144\\
0.550255	0.07634559\\
0.550355	0.07755124\\
0.550455	0.07875283\\
0.5505551	0.07880328\\
0.5506551	0.07852245\\
0.5507551	0.08007283\\
0.5508551	0.08026126\\
0.5509551	0.08036652\\
0.5510551	0.0807324\\
0.5511551	0.08066351\\
0.5512551	0.08059113\\
0.5513551	0.08192279\\
0.5514551	0.08115475\\
0.5515552	0.08114911\\
0.5516552	0.08093646\\
0.5517552	0.08149602\\
0.5518552	0.08151354\\
0.5519552	0.0811291\\
0.5520552	0.08292534\\
0.5521552	0.08424127\\
0.5522552	0.08363539\\
0.5523552	0.08371631\\
0.5524552	0.08461016\\
0.5525553	0.08565333\\
0.5526553	0.08549148\\
0.5527553	0.08681234\\
0.5528553	0.08623708\\
0.5529553	0.08524189\\
0.5530553	0.08632289\\
0.5531553	0.08654447\\
0.5532553	0.08618201\\
0.5533553	0.08617581\\
0.5534553	0.08658772\\
0.5535554	0.08705782\\
0.5536554	0.08704395\\
0.5537554	0.08761731\\
0.5538554	0.08772897\\
0.5539554	0.08916262\\
0.5540554	0.08989124\\
0.5541554	0.09117192\\
0.5542554	0.09200531\\
0.5543554	0.09115909\\
0.5544554	0.09234336\\
0.5545555	0.09215821\\
0.5546555	0.09111886\\
0.5547555	0.09268684\\
0.5548555	0.0923941\\
0.5549555	0.09246142\\
0.5550555	0.0926575\\
0.5551555	0.09292763\\
0.5552555	0.09272624\\
0.5553555	0.09368442\\
0.5554555	0.09338437\\
0.5555556	0.09453743\\
0.5556556	0.09411319\\
0.5557556	0.09542694\\
0.5558556	0.09668171\\
0.5559556	0.0975318\\
0.5560556	0.09817547\\
0.5561556	0.09740064\\
0.5562556	0.09728414\\
0.5563556	0.09691834\\
0.5564556	0.09782678\\
0.5565557	0.09752466\\
0.5566557	0.09740687\\
0.5567557	0.09726453\\
0.5568557	0.09888513\\
0.5569557	0.09972605\\
0.5570557	0.1009841\\
0.5571557	0.1012669\\
0.5572557	0.1024722\\
0.5573557	0.1022976\\
0.5574557	0.1034994\\
0.5575558	0.1025663\\
0.5576558	0.1018113\\
0.5577558	0.1021107\\
0.5578558	0.1037056\\
0.5579558	0.1043135\\
0.5580558	0.1036781\\
0.5581558	0.1033437\\
0.5582558	0.1039405\\
0.5583558	0.1042098\\
0.5584558	0.1055726\\
0.5585559	0.1060113\\
0.5586559	0.1063903\\
0.5587559	0.1062332\\
0.5588559	0.1075576\\
0.5589559	0.1059791\\
0.5590559	0.1080851\\
0.5591559	0.1089028\\
0.5592559	0.1086841\\
0.5593559	0.1094435\\
0.5594559	0.1100637\\
0.559556	0.1099473\\
0.559656	0.1096137\\
0.559756	0.1104971\\
0.559856	0.1114514\\
0.559956	0.111861\\
0.560056	0.112649\\
0.560156	0.1134491\\
0.560256	0.1134351\\
0.560356	0.1142428\\
0.560456	0.1138487\\
0.5605561	0.114838\\
0.5606561	0.1147383\\
0.5607561	0.1145901\\
0.5608561	0.1148575\\
0.5609561	0.1138967\\
0.5610561	0.1149231\\
0.5611561	0.1151924\\
0.5612561	0.115143\\
0.5613561	0.1164775\\
0.5614561	0.1180263\\
0.5615562	0.1185599\\
0.5616562	0.1191123\\
0.5617562	0.1205811\\
0.5618562	0.1206808\\
0.5619562	0.1209244\\
0.5620562	0.1213511\\
0.5621562	0.1225572\\
0.5622562	0.1223189\\
0.5623562	0.1213518\\
0.5624562	0.1208145\\
0.5625563	0.1216788\\
0.5626563	0.1230796\\
0.5627563	0.123381\\
0.5628563	0.1228234\\
0.5629563	0.1235664\\
0.5630563	0.1254965\\
0.5631563	0.1251457\\
0.5632563	0.1245532\\
0.5633563	0.1252312\\
0.5634563	0.1282127\\
0.5635564	0.1273523\\
0.5636564	0.1272093\\
0.5637564	0.1281119\\
0.5638564	0.1286566\\
0.5639564	0.1287013\\
0.5640564	0.1305847\\
0.5641564	0.1301183\\
0.5642564	0.1316286\\
0.5643564	0.1328785\\
0.5644564	0.1324086\\
0.5645565	0.1310121\\
0.5646565	0.1318967\\
0.5647565	0.1335866\\
0.5648565	0.1339935\\
0.5649565	0.1335733\\
0.5650565	0.1329967\\
0.5651565	0.1337947\\
0.5652565	0.1348455\\
0.5653565	0.1364488\\
0.5654565	0.1354507\\
0.5655566	0.1367133\\
0.5656566	0.1374627\\
0.5657566	0.1363492\\
0.5658566	0.1356025\\
0.5659566	0.1375468\\
0.5660566	0.1393449\\
0.5661566	0.1390579\\
0.5662566	0.1392686\\
0.5663566	0.1419216\\
0.5664566	0.1419115\\
0.5665567	0.1421505\\
0.5666567	0.1426233\\
0.5667567	0.143483\\
0.5668567	0.1424732\\
0.5669567	0.1435651\\
0.5670567	0.1455433\\
0.5671567	0.1451184\\
0.5672567	0.1445946\\
0.5673567	0.1445388\\
0.5674567	0.1454138\\
0.5675568	0.1479981\\
0.5676568	0.1495813\\
0.5677568	0.1492108\\
0.5678568	0.1485162\\
0.5679568	0.1494212\\
0.5680568	0.1506293\\
0.5681568	0.1503922\\
0.5682568	0.1496197\\
0.5683568	0.1499733\\
0.5684568	0.150413\\
0.5685569	0.1501095\\
0.5686569	0.1525013\\
0.5687569	0.1517104\\
0.5688569	0.1526974\\
0.5689569	0.1536476\\
0.5690569	0.153637\\
0.5691569	0.153862\\
0.5692569	0.1552856\\
0.5693569	0.1565065\\
0.5694569	0.1574385\\
0.569557	0.1580395\\
0.569657	0.1582386\\
0.569757	0.1596585\\
0.569857	0.1607801\\
0.569957	0.1599455\\
0.570057	0.1596824\\
0.570157	0.1610542\\
0.570257	0.1616709\\
0.570357	0.1619416\\
0.570457	0.1627871\\
0.5705571	0.1642432\\
0.5706571	0.1638593\\
0.5707571	0.163538\\
0.5708571	0.1643256\\
0.5709571	0.1660461\\
0.5710571	0.1663257\\
0.5711571	0.1658307\\
0.5712571	0.1659683\\
0.5713571	0.1674275\\
0.5714571	0.1671291\\
0.5715572	0.1691046\\
0.5716572	0.1701615\\
0.5717572	0.1695906\\
0.5718572	0.1703494\\
0.5719572	0.1705388\\
0.5720572	0.1746531\\
0.5721572	0.1743864\\
0.5722572	0.1723615\\
0.5723572	0.1715534\\
0.5724572	0.173629\\
0.5725573	0.1744867\\
0.5726573	0.1751784\\
0.5727573	0.1751185\\
0.5728573	0.1759763\\
0.5729573	0.1781377\\
0.5730573	0.1781088\\
0.5731573	0.1774865\\
0.5732573	0.177556\\
0.5733573	0.1787612\\
0.5734573	0.1801669\\
0.5735574	0.1822179\\
0.5736574	0.1823576\\
0.5737574	0.180641\\
0.5738574	0.1818122\\
0.5739574	0.1825895\\
0.5740574	0.1833948\\
0.5741574	0.1855868\\
0.5742574	0.1877752\\
0.5743574	0.1881674\\
0.5744574	0.1872467\\
0.5745575	0.1877518\\
0.5746575	0.1888443\\
0.5747575	0.188575\\
0.5748575	0.1898625\\
0.5749575	0.1906252\\
0.5750575	0.1905325\\
0.5751575	0.1902596\\
0.5752575	0.1905926\\
0.5753575	0.1918063\\
0.5754575	0.1913457\\
0.5755576	0.191294\\
0.5756576	0.1929805\\
0.5757576	0.1956429\\
0.5758576	0.1950421\\
0.5759576	0.1948967\\
0.5760576	0.1972625\\
0.5761576	0.1991386\\
0.5762576	0.1997398\\
0.5763576	0.1987182\\
0.5764576	0.1990969\\
0.5765577	0.201016\\
0.5766577	0.2018206\\
0.5767577	0.2022144\\
0.5768577	0.201361\\
0.5769577	0.2030776\\
0.5770577	0.204893\\
0.5771577	0.2046224\\
0.5772577	0.203766\\
0.5773577	0.2063394\\
0.5774577	0.2063614\\
0.5775578	0.2058704\\
0.5776578	0.2068466\\
0.5777578	0.2081561\\
0.5778578	0.2091303\\
0.5779578	0.2112712\\
0.5780578	0.2118104\\
0.5781578	0.2121002\\
0.5782578	0.2133904\\
0.5783578	0.213041\\
0.5784578	0.2153462\\
0.5785579	0.2148442\\
0.5786579	0.2155064\\
0.5787579	0.2171157\\
0.5788579	0.2159746\\
0.5789579	0.2168084\\
0.5790579	0.216295\\
0.5791579	0.2163727\\
0.5792579	0.2165855\\
0.5793579	0.2195705\\
0.5794579	0.2200933\\
0.579558	0.2210532\\
0.579658	0.2207702\\
0.579758	0.2221621\\
0.579858	0.2242105\\
0.579958	0.2248977\\
0.580058	0.2276462\\
0.580158	0.2292992\\
0.580258	0.2290455\\
0.580358	0.2286418\\
0.580458	0.2297913\\
0.5805581	0.2278179\\
0.5806581	0.2273688\\
0.5807581	0.228679\\
0.5808581	0.2290667\\
0.5809581	0.2307986\\
0.5810581	0.2337605\\
0.5811581	0.2336956\\
0.5812581	0.2352793\\
0.5813581	0.2369933\\
0.5814581	0.2379383\\
0.5815582	0.2373818\\
0.5816582	0.2379362\\
0.5817582	0.238653\\
0.5818582	0.2391624\\
0.5819582	0.2388814\\
0.5820582	0.2402373\\
0.5821582	0.2407369\\
0.5822582	0.2402041\\
0.5823582	0.2427387\\
0.5824582	0.2430765\\
0.5825583	0.2434032\\
0.5826583	0.2464865\\
0.5827583	0.2452025\\
0.5828583	0.2453242\\
0.5829583	0.2466889\\
0.5830583	0.2484468\\
0.5831583	0.2508555\\
0.5832583	0.2524104\\
0.5833583	0.2524078\\
0.5834583	0.2538954\\
0.5835584	0.2555852\\
0.5836584	0.2550206\\
0.5837584	0.2551593\\
0.5838584	0.254306\\
0.5839584	0.2561555\\
0.5840584	0.2579639\\
0.5841584	0.2581653\\
0.5842584	0.2569595\\
0.5843584	0.257754\\
0.5844584	0.2571758\\
0.5845585	0.258361\\
0.5846585	0.259889\\
0.5847585	0.2608869\\
0.5848585	0.2648419\\
0.5849585	0.2671879\\
0.5850585	0.266305\\
0.5851585	0.2668187\\
0.5852585	0.2659576\\
0.5853585	0.2668599\\
0.5854585	0.2698013\\
0.5855586	0.2705262\\
0.5856586	0.2716926\\
0.5857586	0.2727884\\
0.5858586	0.2731246\\
0.5859586	0.2729226\\
0.5860586	0.2735001\\
0.5861586	0.2744205\\
0.5862586	0.2771499\\
0.5863586	0.277226\\
0.5864586	0.2772458\\
0.5865587	0.2786016\\
0.5866587	0.2786156\\
0.5867587	0.2802042\\
0.5868587	0.2806559\\
0.5869587	0.2825191\\
0.5870587	0.282393\\
0.5871587	0.2836942\\
0.5872587	0.2854755\\
0.5873587	0.2856299\\
0.5874587	0.2872603\\
0.5875588	0.286528\\
0.5876588	0.2885035\\
0.5877588	0.288996\\
0.5878588	0.2907477\\
0.5879588	0.2898741\\
0.5880588	0.2914996\\
0.5881588	0.2933489\\
0.5882588	0.2947927\\
0.5883588	0.2974796\\
0.5884588	0.2959604\\
0.5885589	0.2986561\\
0.5886589	0.2989971\\
0.5887589	0.2972409\\
0.5888589	0.2972618\\
0.5889589	0.2983813\\
0.5890589	0.3015588\\
0.5891589	0.3019917\\
0.5892589	0.3016552\\
0.5893589	0.3024059\\
0.5894589	0.3055678\\
0.589559	0.3064204\\
0.589659	0.3085692\\
0.589759	0.3104791\\
0.589859	0.3118574\\
0.589959	0.3122745\\
0.590059	0.311997\\
0.590159	0.3124906\\
0.590259	0.3137746\\
0.590359	0.3144304\\
0.590459	0.3139771\\
0.5905591	0.3154026\\
0.5906591	0.3154327\\
0.5907591	0.3167094\\
0.5908591	0.3185626\\
0.5909591	0.3209891\\
0.5910591	0.320813\\
0.5911591	0.3224142\\
0.5912591	0.3234849\\
0.5913591	0.3271834\\
0.5914591	0.3296277\\
0.5915592	0.3282395\\
0.5916592	0.3280893\\
0.5917592	0.3282906\\
0.5918592	0.3272597\\
0.5919592	0.3275074\\
0.5920592	0.3284011\\
0.5921592	0.3288412\\
0.5922592	0.3299991\\
0.5923592	0.3304157\\
0.5924592	0.3333503\\
0.5925593	0.3359863\\
0.5926593	0.3388965\\
0.5927593	0.3410386\\
0.5928593	0.3408568\\
0.5929593	0.3433127\\
0.5930593	0.3446573\\
0.5931593	0.3447421\\
0.5932593	0.3425977\\
0.5933593	0.3451932\\
0.5934593	0.3487139\\
0.5935594	0.3501281\\
0.5936594	0.34979\\
0.5937594	0.3492887\\
0.5938594	0.349792\\
0.5939594	0.3523429\\
0.5940594	0.3530272\\
0.5941594	0.3537861\\
0.5942594	0.3549828\\
0.5943594	0.3565438\\
0.5944594	0.3537373\\
0.5945595	0.3547111\\
0.5946595	0.3556834\\
0.5947595	0.3574686\\
0.5948595	0.3613071\\
0.5949595	0.3619623\\
0.5950595	0.3627666\\
0.5951595	0.3640136\\
0.5952595	0.3667419\\
0.5953595	0.3679337\\
0.5954595	0.3699607\\
0.5955596	0.3681684\\
0.5956596	0.371383\\
0.5957596	0.3731426\\
0.5958596	0.3750261\\
0.5959596	0.3728945\\
0.5960596	0.3733728\\
0.5961596	0.3735964\\
0.5962596	0.3741319\\
0.5963596	0.3783981\\
0.5964596	0.3794404\\
0.5965597	0.3815361\\
0.5966597	0.3815495\\
0.5967597	0.3827075\\
0.5968597	0.3860374\\
0.5969597	0.3876954\\
0.5970597	0.388413\\
0.5971597	0.3863478\\
0.5972597	0.3875884\\
0.5973597	0.3905752\\
0.5974597	0.3922799\\
0.5975598	0.3916368\\
0.5976598	0.3904366\\
0.5977598	0.3940628\\
0.5978598	0.3954587\\
0.5979598	0.3970848\\
0.5980598	0.3993251\\
0.5981598	0.4002932\\
0.5982598	0.4010213\\
0.5983598	0.4023211\\
0.5984598	0.4013846\\
0.5985599	0.4041111\\
0.5986599	0.4063708\\
0.5987599	0.4051943\\
0.5988599	0.4068858\\
0.5989599	0.4071614\\
0.5990599	0.4078322\\
0.5991599	0.4091903\\
0.5992599	0.4130353\\
0.5993599	0.4135898\\
0.5994599	0.4149827\\
0.59956	0.4182396\\
0.59966	0.4191609\\
0.59976	0.4190599\\
0.59986	0.4198402\\
0.59996	0.4214513\\
0.60006	0.4218214\\
0.60016	0.4251204\\
0.60026	0.4294779\\
0.60036	0.4273294\\
0.60046	0.4284806\\
0.6005601	0.4287131\\
0.6006601	0.4310598\\
0.6007601	0.4310312\\
0.6008601	0.4309644\\
0.6009601	0.4336937\\
0.6010601	0.4361545\\
0.6011601	0.4352562\\
0.6012601	0.4360881\\
0.6013601	0.4394345\\
0.6014601	0.4383113\\
0.6015602	0.4405269\\
0.6016602	0.4410148\\
0.6017602	0.4453187\\
0.6018602	0.4470446\\
0.6019602	0.4490782\\
0.6020602	0.4503316\\
0.6021602	0.4525213\\
0.6022602	0.4517762\\
0.6023602	0.4538125\\
0.6024602	0.4542551\\
0.6025603	0.4505587\\
0.6026603	0.4536188\\
0.6027603	0.4537202\\
0.6028603	0.455771\\
0.6029603	0.4608379\\
0.6030603	0.461986\\
0.6031603	0.4653235\\
0.6032603	0.4658273\\
0.6033603	0.4674059\\
0.6034603	0.4680029\\
0.6035604	0.470178\\
0.6036604	0.4698062\\
0.6037604	0.470336\\
0.6038604	0.4722769\\
0.6039604	0.4743479\\
0.6040604	0.4763194\\
0.6041604	0.4745732\\
0.6042604	0.478651\\
0.6043604	0.4783162\\
0.6044604	0.480164\\
0.6045605	0.4829299\\
0.6046605	0.484632\\
0.6047605	0.4849673\\
0.6048605	0.486489\\
0.6049605	0.4882522\\
0.6050605	0.488936\\
0.6051605	0.4906078\\
0.6052605	0.4916274\\
0.6053605	0.4946293\\
0.6054605	0.4957843\\
0.6055606	0.4967548\\
0.6056606	0.4973392\\
0.6057606	0.4976786\\
0.6058606	0.4984673\\
0.6059606	0.4982986\\
0.6060606	0.501913\\
0.6061606	0.5049065\\
0.6062606	0.5070805\\
0.6063606	0.5079544\\
0.6064606	0.5106219\\
0.6065607	0.5122527\\
0.6066607	0.5157345\\
0.6067607	0.5155011\\
0.6068607	0.5150648\\
0.6069607	0.5164189\\
0.6070607	0.5189478\\
0.6071607	0.5211866\\
0.6072607	0.5196428\\
0.6073607	0.5218598\\
0.6074607	0.5242468\\
0.6075608	0.5261172\\
0.6076608	0.5258614\\
0.6077608	0.5273029\\
0.6078608	0.5297922\\
0.6079608	0.5321719\\
0.6080608	0.5338505\\
0.6081608	0.5344538\\
0.6082608	0.5359706\\
0.6083608	0.5350363\\
0.6084608	0.5361131\\
0.6085609	0.5378841\\
0.6086609	0.5399574\\
0.6087609	0.5428884\\
0.6088609	0.5403341\\
0.6089609	0.5464615\\
0.6090609	0.5513723\\
0.6091609	0.5538692\\
0.6092609	0.555533\\
0.6093609	0.5562239\\
0.6094609	0.5585045\\
0.609561	0.5608679\\
0.609661	0.5599002\\
0.609761	0.5593806\\
0.609861	0.5599392\\
0.609961	0.5617993\\
0.610061	0.5607576\\
0.610161	0.5602498\\
0.610261	0.5649728\\
0.610361	0.566638\\
0.610461	0.5680881\\
0.6105611	0.5704398\\
0.6106611	0.5733313\\
0.6107611	0.5765167\\
0.6108611	0.5794731\\
0.6109611	0.5774257\\
0.6110611	0.5810686\\
0.6111611	0.582909\\
0.6112611	0.5840909\\
0.6113611	0.5863308\\
0.6114611	0.5840756\\
0.6115612	0.5836447\\
0.6116612	0.5841973\\
0.6117612	0.5890423\\
0.6118612	0.5919241\\
0.6119612	0.5953416\\
0.6120612	0.5990706\\
0.6121612	0.5991379\\
0.6122612	0.6007072\\
0.6123612	0.6030655\\
0.6124612	0.6079114\\
0.6125613	0.6077384\\
0.6126613	0.6061135\\
0.6127613	0.6055741\\
0.6128613	0.6091056\\
0.6129613	0.612098\\
0.6130613	0.612399\\
0.6131613	0.6125746\\
0.6132613	0.6126209\\
0.6133613	0.6164725\\
0.6134613	0.620524\\
0.6135614	0.6214916\\
0.6136614	0.6260904\\
0.6137614	0.6269057\\
0.6138614	0.626878\\
0.6139614	0.6254797\\
0.6140614	0.6263956\\
0.6141614	0.6303621\\
0.6142614	0.6311912\\
0.6143614	0.6325075\\
0.6144614	0.6365033\\
0.6145615	0.638142\\
0.6146615	0.6397098\\
0.6147615	0.64546\\
0.6148615	0.6474711\\
0.6149615	0.6455811\\
0.6150615	0.6454254\\
0.6151615	0.648525\\
0.6152615	0.6512508\\
0.6153615	0.6514203\\
0.6154615	0.6527776\\
0.6155616	0.656374\\
0.6156616	0.6612039\\
0.6157616	0.6587885\\
0.6158616	0.6581154\\
0.6159616	0.6611223\\
0.6160616	0.6600678\\
0.6161616	0.6642052\\
0.6162616	0.6719789\\
0.6163616	0.6732919\\
0.6164616	0.6723338\\
0.6165617	0.6733966\\
0.6166617	0.6762648\\
0.6167617	0.6788671\\
0.6168617	0.6799859\\
0.6169617	0.684654\\
0.6170617	0.6873329\\
0.6171617	0.6879287\\
0.6172617	0.6864046\\
0.6173617	0.6877236\\
0.6174617	0.6914454\\
0.6175618	0.6927663\\
0.6176618	0.6925591\\
0.6177618	0.6934544\\
0.6178618	0.6937142\\
0.6179618	0.6966676\\
0.6180618	0.6957091\\
0.6181618	0.699505\\
0.6182618	0.6991675\\
0.6183618	0.7023573\\
0.6184618	0.7090754\\
0.6185619	0.7101581\\
0.6186619	0.7151859\\
0.6187619	0.7171848\\
0.6188619	0.7221992\\
0.6189619	0.7231637\\
0.6190619	0.7232251\\
0.6191619	0.7238962\\
0.6192619	0.7251656\\
0.6193619	0.7266781\\
0.6194619	0.7253797\\
0.619562	0.7303269\\
0.619662	0.7336881\\
0.619762	0.7335224\\
0.619862	0.7318327\\
0.619962	0.734333\\
0.620062	0.7371969\\
0.620162	0.7437995\\
0.620262	0.7448231\\
0.620362	0.7445305\\
0.620462	0.7470534\\
0.6205621	0.7498991\\
0.6206621	0.7523076\\
0.6207621	0.7536368\\
0.6208621	0.754964\\
0.6209621	0.7537855\\
0.6210621	0.7579336\\
0.6211621	0.7614312\\
0.6212621	0.7618663\\
0.6213621	0.7641744\\
0.6214621	0.7675847\\
0.6215622	0.769038\\
0.6216622	0.7710721\\
0.6217622	0.7724514\\
0.6218622	0.7726785\\
0.6219622	0.7731904\\
0.6220622	0.7767659\\
0.6221622	0.7786539\\
0.6222622	0.783858\\
0.6223622	0.7849474\\
0.6224622	0.7877789\\
0.6225623	0.7890131\\
0.6226623	0.793223\\
0.6227623	0.80058\\
0.6228623	0.7984403\\
0.6229623	0.7959668\\
0.6230623	0.799836\\
0.6231623	0.8019043\\
0.6232623	0.804494\\
0.6233623	0.8036682\\
0.6234623	0.8052843\\
0.6235624	0.8087257\\
0.6236624	0.8098889\\
0.6237624	0.8118305\\
0.6238624	0.8152902\\
0.6239624	0.8169234\\
0.6240624	0.8202058\\
0.6241624	0.8243645\\
0.6242624	0.8256853\\
0.6243624	0.8277529\\
0.6244624	0.8258771\\
0.6245625	0.8290799\\
0.6246625	0.8347456\\
0.6247625	0.834713\\
0.6248625	0.8340118\\
0.6249625	0.8402933\\
0.6250625	0.837793\\
0.6251625	0.8400968\\
0.6252625	0.8423771\\
0.6253625	0.8460131\\
0.6254625	0.8500135\\
0.6255626	0.8524916\\
0.6256626	0.8538036\\
0.6257626	0.8557226\\
0.6258626	0.8609498\\
0.6259626	0.8625332\\
0.6260626	0.8652237\\
0.6261626	0.8692786\\
0.6262626	0.8726172\\
0.6263626	0.8746104\\
0.6264626	0.8726782\\
0.6265627	0.8702722\\
0.6266627	0.8719888\\
0.6267627	0.8726023\\
0.6268627	0.874941\\
0.6269627	0.8751574\\
0.6270627	0.8815938\\
0.6271627	0.8844385\\
0.6272627	0.8877513\\
0.6273627	0.8958192\\
0.6274627	0.8953512\\
0.6275628	0.8987687\\
0.6276628	0.9025187\\
0.6277628	0.8994506\\
0.6278628	0.9025819\\
0.6279628	0.903953\\
0.6280628	0.9058501\\
0.6281628	0.908712\\
0.6282628	0.9084736\\
0.6283628	0.9151278\\
0.6284628	0.9154739\\
0.6285629	0.9170567\\
0.6286629	0.9205277\\
0.6287629	0.9223621\\
0.6288629	0.9250273\\
0.6289629	0.9263578\\
0.6290629	0.9320212\\
0.6291629	0.9362934\\
0.6292629	0.9388703\\
0.6293629	0.9397292\\
0.6294629	0.935361\\
0.629563	0.935172\\
0.629663	0.9385315\\
0.629763	0.9395702\\
0.629863	0.9450162\\
0.629963	0.9499597\\
0.630063	0.9533573\\
0.630163	0.9557829\\
0.630263	0.9573282\\
0.630363	0.9628072\\
0.630463	0.9647518\\
0.6305631	0.9657557\\
0.6306631	0.9668938\\
0.6307631	0.9706516\\
0.6308631	0.9752255\\
0.6309631	0.9735948\\
0.6310631	0.9754078\\
0.6311631	0.9786246\\
0.6312631	0.9796448\\
0.6313631	0.979448\\
0.6314631	0.9791316\\
0.6315632	0.9852423\\
0.6316632	0.9911837\\
0.6317632	0.9948244\\
0.6318632	0.9951822\\
0.6319632	0.9967861\\
0.6320632	1.000088\\
0.6321632	1.000586\\
0.6322632	1.004606\\
0.6323632	1.008058\\
0.6324632	1.01195\\
0.6325633	1.013178\\
0.6326633	1.019457\\
0.6327633	1.022575\\
0.6328633	1.018634\\
0.6329633	1.019984\\
0.6330633	1.026487\\
0.6331633	1.026842\\
0.6332633	1.026293\\
0.6333633	1.033008\\
0.6334633	1.034582\\
0.6335634	1.035462\\
0.6336634	1.034742\\
0.6337634	1.037243\\
0.6338634	1.039317\\
0.6339634	1.043643\\
0.6340634	1.05222\\
0.6341634	1.053199\\
0.6342634	1.056887\\
0.6343634	1.058425\\
0.6344634	1.057911\\
0.6345635	1.061687\\
0.6346635	1.063208\\
0.6347635	1.065102\\
0.6348635	1.062727\\
0.6349635	1.069718\\
0.6350635	1.07316\\
0.6351635	1.074262\\
0.6352635	1.081174\\
0.6353635	1.086771\\
0.6354635	1.084299\\
0.6355636	1.088533\\
0.6356636	1.094989\\
0.6357636	1.098835\\
0.6358636	1.097082\\
0.6359636	1.096745\\
0.6360636	1.101489\\
0.6361636	1.103749\\
0.6362636	1.106193\\
0.6363636	1.105162\\
0.6364636	1.104685\\
0.6365637	1.110702\\
0.6366637	1.116568\\
0.6367637	1.118275\\
0.6368637	1.11721\\
0.6369637	1.120525\\
0.6370637	1.122456\\
0.6371637	1.122251\\
0.6372637	1.129046\\
0.6373637	1.132749\\
0.6374637	1.134383\\
0.6375638	1.139913\\
0.6376638	1.141652\\
0.6377638	1.143597\\
0.6378638	1.14735\\
0.6379638	1.148988\\
0.6380638	1.15198\\
0.6381638	1.155487\\
0.6382638	1.160705\\
0.6383638	1.162596\\
0.6384638	1.168545\\
0.6385639	1.166433\\
0.6386639	1.167008\\
0.6387639	1.170744\\
0.6388639	1.17497\\
0.6389639	1.177276\\
0.6390639	1.1808\\
0.6391639	1.178002\\
0.6392639	1.183601\\
0.6393639	1.184971\\
0.6394639	1.186734\\
0.639564	1.196045\\
0.639664	1.195893\\
0.639764	1.196899\\
0.639864	1.201681\\
0.639964	1.20048\\
0.640064	1.20044\\
0.640164	1.208793\\
0.640264	1.212931\\
0.640364	1.210939\\
0.640464	1.21225\\
0.6405641	1.223043\\
0.6406641	1.226087\\
0.6407641	1.226842\\
0.6408641	1.228641\\
0.6409641	1.234942\\
0.6410641	1.235418\\
0.6411641	1.234064\\
0.6412641	1.237757\\
0.6413641	1.240643\\
0.6414641	1.239245\\
0.6415642	1.244326\\
0.6416642	1.246598\\
0.6417642	1.252473\\
0.6418642	1.258797\\
0.6419642	1.255644\\
0.6420642	1.260094\\
0.6421642	1.264388\\
0.6422642	1.270926\\
0.6423642	1.271047\\
0.6424642	1.278677\\
0.6425643	1.285134\\
0.6426643	1.282331\\
0.6427643	1.283689\\
0.6428643	1.287034\\
0.6429643	1.287151\\
0.6430643	1.28906\\
0.6431643	1.287289\\
0.6432643	1.290873\\
0.6433643	1.298517\\
0.6434643	1.300011\\
0.6435644	1.304974\\
0.6436644	1.300294\\
0.6437644	1.304234\\
0.6438644	1.308172\\
0.6439644	1.314979\\
0.6440644	1.318475\\
0.6441644	1.321846\\
0.6442644	1.327304\\
0.6443644	1.331628\\
0.6444644	1.334404\\
0.6445645	1.338273\\
0.6446645	1.339431\\
0.6447645	1.3422\\
0.6448645	1.343273\\
0.6449645	1.345215\\
0.6450645	1.346887\\
0.6451645	1.341214\\
0.6452645	1.345695\\
0.6453645	1.352811\\
0.6454645	1.3595\\
0.6455646	1.367978\\
0.6456646	1.370368\\
0.6457646	1.373031\\
0.6458646	1.37664\\
0.6459646	1.37345\\
0.6460646	1.377417\\
0.6461646	1.381115\\
0.6462646	1.386168\\
0.6463646	1.384899\\
0.6464646	1.388601\\
0.6465647	1.390667\\
0.6466647	1.393583\\
0.6467647	1.398833\\
0.6468647	1.40202\\
0.6469647	1.402547\\
0.6470647	1.407902\\
0.6471647	1.410765\\
0.6472647	1.414605\\
0.6473647	1.414938\\
0.6474647	1.418843\\
0.6475648	1.423106\\
0.6476648	1.425492\\
0.6477648	1.432763\\
0.6478648	1.430713\\
0.6479648	1.434619\\
0.6480648	1.438657\\
0.6481648	1.441811\\
0.6482648	1.445582\\
0.6483648	1.445658\\
0.6484648	1.447431\\
0.6485649	1.454751\\
0.6486649	1.456227\\
0.6487649	1.462616\\
0.6488649	1.463274\\
0.6489649	1.462136\\
0.6490649	1.468558\\
0.6491649	1.470416\\
0.6492649	1.472145\\
0.6493649	1.481467\\
0.6494649	1.488677\\
0.649565	1.488904\\
0.649665	1.491429\\
0.649765	1.491644\\
0.649865	1.49484\\
0.649965	1.495623\\
0.650065	1.496674\\
0.650165	1.502989\\
0.650265	1.50491\\
0.650365	1.509529\\
0.650465	1.50866\\
0.6505651	1.512396\\
0.6506651	1.51686\\
0.6507651	1.516598\\
0.6508651	1.52191\\
0.6509651	1.527272\\
0.6510651	1.533833\\
0.6511651	1.537956\\
0.6512651	1.541497\\
0.6513651	1.543357\\
0.6514651	1.547509\\
0.6515652	1.549977\\
0.6516652	1.553699\\
0.6517652	1.555376\\
0.6518652	1.559895\\
0.6519652	1.563897\\
0.6520652	1.565165\\
0.6521652	1.569538\\
0.6522652	1.575191\\
0.6523652	1.575186\\
0.6524652	1.575791\\
0.6525653	1.582912\\
0.6526653	1.585984\\
0.6527653	1.580459\\
0.6528653	1.578614\\
0.6529653	1.584906\\
0.6530653	1.589976\\
0.6531653	1.599257\\
0.6532653	1.600574\\
0.6533653	1.603179\\
0.6534653	1.608515\\
0.6535654	1.612705\\
0.6536654	1.615802\\
0.6537654	1.620579\\
0.6538654	1.630872\\
0.6539654	1.631306\\
0.6540654	1.632317\\
0.6541654	1.635899\\
0.6542654	1.638569\\
0.6543654	1.639168\\
0.6544654	1.642523\\
0.6545655	1.647704\\
0.6546655	1.64901\\
0.6547655	1.649724\\
0.6548655	1.652813\\
0.6549655	1.656356\\
0.6550655	1.66187\\
0.6551655	1.672949\\
0.6552655	1.672319\\
0.6553655	1.66972\\
0.6554655	1.67298\\
0.6555656	1.67995\\
0.6556656	1.683521\\
0.6557656	1.683562\\
0.6558656	1.689276\\
0.6559656	1.689101\\
0.6560656	1.694634\\
0.6561656	1.702828\\
0.6562656	1.707206\\
0.6563656	1.706408\\
0.6564656	1.707976\\
0.6565657	1.710064\\
0.6566657	1.71428\\
0.6567657	1.714973\\
0.6568657	1.721083\\
0.6569657	1.725746\\
0.6570657	1.728244\\
0.6571657	1.730933\\
0.6572657	1.740104\\
0.6573657	1.743045\\
0.6574657	1.74504\\
0.6575658	1.749122\\
0.6576658	1.751874\\
0.6577658	1.757228\\
0.6578658	1.761521\\
0.6579658	1.758988\\
0.6580658	1.765457\\
0.6581658	1.769562\\
0.6582658	1.770213\\
0.6583658	1.777383\\
0.6584658	1.777269\\
0.6585659	1.781571\\
0.6586659	1.780762\\
0.6587659	1.784657\\
0.6588659	1.790778\\
0.6589659	1.80006\\
0.6590659	1.804042\\
0.6591659	1.801639\\
0.6592659	1.799848\\
0.6593659	1.806958\\
0.6594659	1.816728\\
0.659566	1.822249\\
0.659666	1.824168\\
0.659766	1.826836\\
0.659866	1.827339\\
0.659966	1.828437\\
0.660066	1.828643\\
0.660166	1.833633\\
0.660266	1.839423\\
0.660366	1.841745\\
0.660466	1.84623\\
0.6605661	1.848856\\
0.6606661	1.851892\\
0.6607661	1.853728\\
0.6608661	1.853184\\
0.6609661	1.86884\\
0.6610661	1.877789\\
0.6611661	1.877076\\
0.6612661	1.878475\\
0.6613661	1.8851\\
0.6614661	1.889467\\
0.6615662	1.893014\\
0.6616662	1.895536\\
0.6617662	1.900991\\
0.6618662	1.904628\\
0.6619662	1.901039\\
0.6620662	1.902126\\
0.6621662	1.902501\\
0.6622662	1.905255\\
0.6623662	1.912712\\
0.6624662	1.919324\\
0.6625663	1.918649\\
0.6626663	1.926437\\
0.6627663	1.932236\\
0.6628663	1.937382\\
0.6629663	1.939085\\
0.6630663	1.939045\\
0.6631663	1.945383\\
0.6632663	1.944583\\
0.6633663	1.949056\\
0.6634663	1.958791\\
0.6635664	1.957946\\
0.6636664	1.96331\\
0.6637664	1.96822\\
0.6638664	1.969858\\
0.6639664	1.973898\\
0.6640664	1.980267\\
0.6641664	1.985336\\
0.6642664	1.984362\\
0.6643664	1.986946\\
0.6644664	1.993923\\
0.6645665	1.996484\\
0.6646665	2.000357\\
0.6647665	2.001524\\
0.6648665	2.007344\\
0.6649665	2.009899\\
0.6650665	2.01887\\
0.6651665	2.020106\\
0.6652665	2.02136\\
0.6653665	2.02939\\
0.6654665	2.033696\\
0.6655666	2.039059\\
0.6656666	2.03807\\
0.6657666	2.036687\\
0.6658666	2.040247\\
0.6659666	2.042481\\
0.6660666	2.046845\\
0.6661666	2.05494\\
0.6662666	2.061395\\
0.6663666	2.063807\\
0.6664666	2.067463\\
0.6665667	2.076831\\
0.6666667	2.074505\\
0.6667667	2.076645\\
0.6668667	2.077201\\
0.6669667	2.082242\\
0.6670667	2.086164\\
0.6671667	2.088942\\
0.6672667	2.092037\\
0.6673667	2.101193\\
0.6674667	2.104867\\
0.6675668	2.115604\\
0.6676668	2.117581\\
0.6677668	2.119068\\
0.6678668	2.120738\\
0.6679668	2.124418\\
0.6680668	2.12631\\
0.6681668	2.133432\\
0.6682668	2.135118\\
0.6683668	2.135081\\
0.6684668	2.141765\\
0.6685669	2.146648\\
0.6686669	2.155283\\
0.6687669	2.161018\\
0.6688669	2.15758\\
0.6689669	2.157092\\
0.6690669	2.16136\\
0.6691669	2.164641\\
0.6692669	2.16969\\
0.6693669	2.171842\\
0.6694669	2.181819\\
0.669567	2.183877\\
0.669667	2.182413\\
0.669767	2.188059\\
0.669867	2.195389\\
0.669967	2.198457\\
0.670067	2.200783\\
0.670167	2.209688\\
0.670267	2.214192\\
0.670367	2.218177\\
0.670467	2.218768\\
0.6705671	2.224895\\
0.6706671	2.23409\\
0.6707671	2.240277\\
0.6708671	2.240062\\
0.6709671	2.240895\\
0.6710671	2.238976\\
0.6711671	2.245131\\
0.6712671	2.246148\\
0.6713671	2.246777\\
0.6714671	2.253584\\
0.6715672	2.263597\\
0.6716672	2.26732\\
0.6717672	2.273847\\
0.6718672	2.272113\\
0.6719672	2.276611\\
0.6720672	2.278507\\
0.6721672	2.280277\\
0.6722672	2.283259\\
0.6723672	2.290077\\
0.6724672	2.290877\\
0.6725673	2.298823\\
0.6726673	2.302283\\
0.6727673	2.307291\\
0.6728673	2.315466\\
0.6729673	2.321266\\
0.6730673	2.323796\\
0.6731673	2.330912\\
0.6732673	2.330431\\
0.6733673	2.330921\\
0.6734673	2.340774\\
0.6735674	2.34784\\
0.6736674	2.3496\\
0.6737674	2.350858\\
0.6738674	2.343806\\
0.6739674	2.351457\\
0.6740674	2.350734\\
0.6741674	2.356764\\
0.6742674	2.363358\\
0.6743674	2.371544\\
0.6744674	2.37553\\
0.6745675	2.380849\\
0.6746675	2.381688\\
0.6747675	2.385592\\
0.6748675	2.396957\\
0.6749675	2.400322\\
0.6750675	2.405665\\
0.6751675	2.410777\\
0.6752675	2.409811\\
0.6753675	2.418024\\
0.6754675	2.417071\\
0.6755676	2.409673\\
0.6756676	2.41895\\
0.6757676	2.423863\\
0.6758676	2.431255\\
0.6759676	2.435264\\
0.6760676	2.432903\\
0.6761676	2.435523\\
0.6762676	2.444578\\
0.6763676	2.447783\\
0.6764676	2.449317\\
0.6765677	2.459021\\
0.6766677	2.468129\\
0.6767677	2.472667\\
0.6768677	2.471701\\
0.6769677	2.47934\\
0.6770677	2.48493\\
0.6771677	2.483097\\
0.6772677	2.488039\\
0.6773677	2.494644\\
0.6774677	2.496537\\
0.6775678	2.498168\\
0.6776678	2.504278\\
0.6777678	2.506609\\
0.6778678	2.513317\\
0.6779678	2.521652\\
0.6780678	2.52687\\
0.6781678	2.527346\\
0.6782678	2.527178\\
0.6783678	2.534853\\
0.6784678	2.543041\\
0.6785679	2.537853\\
0.6786679	2.530013\\
0.6787679	2.5382\\
0.6788679	2.550283\\
0.6789679	2.553325\\
0.6790679	2.555038\\
0.6791679	2.559589\\
0.6792679	2.565366\\
0.6793679	2.572564\\
0.6794679	2.582902\\
0.679568	2.583856\\
0.679668	2.588046\\
0.679768	2.589456\\
0.679868	2.592637\\
0.679968	2.596194\\
0.680068	2.603992\\
0.680168	2.60548\\
0.680268	2.61046\\
0.680368	2.61266\\
0.680468	2.620923\\
0.6805681	2.621931\\
0.6806681	2.626404\\
0.6807681	2.624971\\
0.6808681	2.636161\\
0.6809681	2.637828\\
0.6810681	2.653733\\
0.6811681	2.656183\\
0.6812681	2.649654\\
0.6813681	2.651221\\
0.6814681	2.653848\\
0.6815682	2.661036\\
0.6816682	2.66996\\
0.6817682	2.672774\\
0.6818682	2.671829\\
0.6819682	2.680144\\
0.6820682	2.679733\\
0.6821682	2.6842\\
0.6822682	2.692285\\
0.6823682	2.689899\\
0.6824682	2.693128\\
0.6825683	2.70655\\
0.6826683	2.707038\\
0.6827683	2.712803\\
0.6828683	2.715039\\
0.6829683	2.717013\\
0.6830683	2.723013\\
0.6831683	2.731616\\
0.6832683	2.737173\\
0.6833683	2.740617\\
0.6834683	2.74217\\
0.6835684	2.749237\\
0.6836684	2.755103\\
0.6837684	2.758606\\
0.6838684	2.758841\\
0.6839684	2.759951\\
0.6840684	2.759749\\
0.6841684	2.763635\\
0.6842684	2.769985\\
0.6843684	2.782369\\
0.6844684	2.785989\\
0.6845685	2.78609\\
0.6846685	2.794155\\
0.6847685	2.800867\\
0.6848685	2.806485\\
0.6849685	2.805845\\
0.6850685	2.810578\\
0.6851685	2.812041\\
0.6852685	2.812924\\
0.6853685	2.815509\\
0.6854685	2.824055\\
0.6855686	2.821785\\
0.6856686	2.827612\\
0.6857686	2.836015\\
0.6858686	2.84248\\
0.6859686	2.842104\\
0.6860686	2.843209\\
0.6861686	2.846554\\
0.6862686	2.859653\\
0.6863686	2.867037\\
0.6864686	2.866262\\
0.6865687	2.867134\\
0.6866687	2.876346\\
0.6867687	2.881284\\
0.6868687	2.886249\\
0.6869687	2.898284\\
0.6870687	2.904524\\
0.6871687	2.906835\\
0.6872687	2.907161\\
0.6873687	2.907331\\
0.6874687	2.908005\\
0.6875688	2.906205\\
0.6876688	2.901333\\
0.6877688	2.908952\\
0.6878688	2.913638\\
0.6879688	2.923922\\
0.6880688	2.931362\\
0.6881688	2.938255\\
0.6882688	2.941324\\
0.6883688	2.949123\\
0.6884688	2.956614\\
0.6885689	2.957874\\
0.6886689	2.963809\\
0.6887689	2.968467\\
0.6888689	2.969688\\
0.6889689	2.96991\\
0.6890689	2.974808\\
0.6891689	2.978068\\
0.6892689	2.978948\\
0.6893689	2.98086\\
0.6894689	2.991175\\
0.689569	3.001309\\
0.689669	2.999749\\
0.689769	3.002584\\
0.689869	3.010331\\
0.689969	3.016021\\
0.690069	3.023769\\
0.690169	3.024128\\
0.690269	3.027758\\
0.690369	3.032932\\
0.690469	3.040417\\
0.6905691	3.040564\\
0.6906691	3.040836\\
0.6907691	3.052204\\
0.6908691	3.049577\\
0.6909691	3.045079\\
0.6910691	3.045301\\
0.6911691	3.053076\\
0.6912691	3.062563\\
0.6913691	3.079737\\
0.6914691	3.08656\\
0.6915692	3.090356\\
0.6916692	3.090879\\
0.6917692	3.091283\\
0.6918692	3.094169\\
0.6919692	3.095306\\
0.6920692	3.096065\\
0.6921692	3.103683\\
0.6922692	3.111749\\
0.6923692	3.114857\\
0.6924692	3.120224\\
0.6925693	3.121105\\
0.6926693	3.128245\\
0.6927693	3.126937\\
0.6928693	3.131771\\
0.6929693	3.139133\\
0.6930693	3.148425\\
0.6931693	3.147954\\
0.6932693	3.153951\\
0.6933693	3.15971\\
0.6934693	3.161608\\
0.6935694	3.1683\\
0.6936694	3.173391\\
0.6937694	3.17883\\
0.6938694	3.176694\\
0.6939694	3.184094\\
0.6940694	3.188822\\
0.6941694	3.191282\\
0.6942694	3.184089\\
0.6943694	3.196098\\
0.6944694	3.211739\\
0.6945695	3.217495\\
0.6946695	3.216897\\
0.6947695	3.215685\\
0.6948695	3.223894\\
0.6949695	3.222284\\
0.6950695	3.226393\\
0.6951695	3.227938\\
0.6952695	3.234047\\
0.6953695	3.232894\\
0.6954695	3.233627\\
0.6955696	3.23804\\
0.6956696	3.242919\\
0.6957696	3.255584\\
0.6958696	3.268663\\
0.6959696	3.269932\\
0.6960696	3.271973\\
0.6961696	3.275804\\
0.6962696	3.288854\\
0.6963696	3.292365\\
0.6964696	3.303931\\
0.6965697	3.308692\\
0.6966697	3.304201\\
0.6967697	3.307856\\
0.6968697	3.303636\\
0.6969697	3.308256\\
0.6970697	3.310052\\
0.6971697	3.31498\\
0.6972697	3.318448\\
0.6973697	3.323922\\
0.6974697	3.324397\\
0.6975698	3.322485\\
0.6976698	3.328662\\
0.6977698	3.337795\\
0.6978698	3.351184\\
0.6979698	3.354814\\
0.6980698	3.355443\\
0.6981698	3.360294\\
0.6982698	3.366853\\
0.6983698	3.360445\\
0.6984698	3.372694\\
0.6985699	3.377407\\
0.6986699	3.383752\\
0.6987699	3.390297\\
0.6988699	3.392205\\
0.6989699	3.40345\\
0.6990699	3.402461\\
0.6991699	3.400261\\
0.6992699	3.400546\\
0.6993699	3.405567\\
0.6994699	3.405961\\
0.69957	3.411483\\
0.69967	3.419622\\
0.69977	3.428069\\
0.69987	3.433974\\
0.69997	3.440389\\
0.70007	3.444596\\
0.70017	3.455974\\
0.70027	3.455529\\
0.70037	3.458542\\
0.70047	3.466673\\
0.7005701	3.463773\\
0.7006701	3.463033\\
0.7007701	3.462667\\
0.7008701	3.467144\\
0.7009701	3.473838\\
0.7010701	3.48108\\
0.7011701	3.481827\\
0.7012701	3.484372\\
0.7013701	3.498422\\
0.7014701	3.491743\\
0.7015702	3.495262\\
0.7016702	3.504402\\
0.7017702	3.513732\\
0.7018702	3.516454\\
0.7019702	3.526755\\
0.7020702	3.526243\\
0.7021702	3.527153\\
0.7022702	3.535931\\
0.7023702	3.538981\\
0.7024702	3.538102\\
0.7025703	3.54054\\
0.7026703	3.55024\\
0.7027703	3.548614\\
0.7028703	3.556334\\
0.7029703	3.56425\\
0.7030703	3.56864\\
0.7031703	3.573059\\
0.7032703	3.572138\\
0.7033703	3.581633\\
0.7034703	3.584104\\
0.7035704	3.588953\\
0.7036704	3.596533\\
0.7037704	3.601529\\
0.7038704	3.602485\\
0.7039704	3.603796\\
0.7040704	3.615602\\
0.7041704	3.611599\\
0.7042704	3.613122\\
0.7043704	3.609038\\
0.7044704	3.61325\\
0.7045705	3.614637\\
0.7046705	3.620535\\
0.7047705	3.629636\\
0.7048705	3.635141\\
0.7049705	3.644469\\
0.7050705	3.657455\\
0.7051705	3.65999\\
0.7052705	3.662017\\
0.7053705	3.66801\\
0.7054705	3.671457\\
0.7055706	3.681969\\
0.7056706	3.680869\\
0.7057706	3.679225\\
0.7058706	3.678027\\
0.7059706	3.680509\\
0.7060706	3.688058\\
0.7061706	3.688783\\
0.7062706	3.694125\\
0.7063706	3.701732\\
0.7064706	3.706145\\
0.7065707	3.709727\\
0.7066707	3.716241\\
0.7067707	3.724209\\
0.7068707	3.729872\\
0.7069707	3.732823\\
0.7070707	3.734471\\
0.7071707	3.731298\\
0.7072707	3.727945\\
0.7073707	3.732746\\
0.7074707	3.736659\\
0.7075708	3.744498\\
0.7076708	3.760082\\
0.7077708	3.769355\\
0.7078708	3.774711\\
0.7079708	3.773062\\
0.7080708	3.77559\\
0.7081708	3.777722\\
0.7082708	3.779434\\
0.7083708	3.789163\\
0.7084708	3.792355\\
0.7085709	3.794566\\
0.7086709	3.794257\\
0.7087709	3.793339\\
0.7088709	3.797322\\
0.7089709	3.80858\\
0.7090709	3.808082\\
0.7091709	3.814715\\
0.7092709	3.814408\\
0.7093709	3.821848\\
0.7094709	3.832006\\
0.709571	3.831527\\
0.709671	3.841931\\
0.709771	3.846898\\
0.709871	3.850497\\
0.709971	3.857081\\
0.710071	3.872352\\
0.710171	3.868413\\
0.710271	3.860444\\
0.710371	3.866165\\
0.710471	3.868903\\
0.7105711	3.866214\\
0.7106711	3.869934\\
0.7107711	3.874954\\
0.7108711	3.873902\\
0.7109711	3.878902\\
0.7110711	3.887632\\
0.7111711	3.894058\\
0.7112711	3.907716\\
0.7113711	3.90979\\
0.7114711	3.913279\\
0.7115712	3.908833\\
0.7116712	3.915527\\
0.7117712	3.925838\\
0.7118712	3.927797\\
0.7119712	3.933193\\
0.7120712	3.935313\\
0.7121712	3.939878\\
0.7122712	3.942073\\
0.7123712	3.951787\\
0.7124712	3.949958\\
0.7125713	3.955865\\
0.7126713	3.960949\\
0.7127713	3.9587\\
0.7128713	3.965962\\
0.7129713	3.972427\\
0.7130713	3.979254\\
0.7131713	3.973047\\
0.7132713	3.986061\\
0.7133713	3.993579\\
0.7134713	3.994156\\
0.7135714	3.99425\\
0.7136714	3.993077\\
0.7137714	3.992902\\
0.7138714	3.997324\\
0.7139714	4.00405\\
0.7140714	4.005084\\
0.7141714	4.007856\\
0.7142714	4.02442\\
0.7143714	4.026682\\
0.7144714	4.025951\\
0.7145715	4.032113\\
0.7146715	4.03315\\
0.7147715	4.031118\\
0.7148715	4.038341\\
0.7149715	4.048256\\
0.7150715	4.052428\\
0.7151715	4.062398\\
0.7152715	4.06702\\
0.7153715	4.073351\\
0.7154715	4.071795\\
0.7155716	4.084405\\
0.7156716	4.082629\\
0.7157716	4.083712\\
0.7158716	4.08753\\
0.7159716	4.086497\\
0.7160716	4.086582\\
0.7161716	4.08857\\
0.7162716	4.091445\\
0.7163716	4.097023\\
0.7164716	4.101882\\
0.7165717	4.112757\\
0.7166717	4.112224\\
0.7167717	4.111157\\
0.7168717	4.11417\\
0.7169717	4.128081\\
0.7170717	4.129694\\
0.7171717	4.137812\\
0.7172717	4.146861\\
0.7173717	4.143764\\
0.7174717	4.144893\\
0.7175718	4.148178\\
0.7176718	4.147534\\
0.7177718	4.145022\\
0.7178718	4.151171\\
0.7179718	4.156509\\
0.7180718	4.170223\\
0.7181718	4.16654\\
0.7182718	4.174426\\
0.7183718	4.183859\\
0.7184718	4.182965\\
0.7185719	4.192799\\
0.7186719	4.192711\\
0.7187719	4.194626\\
0.7188719	4.20114\\
0.7189719	4.199427\\
0.7190719	4.205299\\
0.7191719	4.211901\\
0.7192719	4.217804\\
0.7193719	4.216304\\
0.7194719	4.215097\\
0.719572	4.222041\\
0.719672	4.231116\\
0.719772	4.23464\\
0.719872	4.235656\\
0.719972	4.236727\\
0.720072	4.241793\\
0.720172	4.248598\\
0.720272	4.247484\\
0.720372	4.247523\\
0.720472	4.253821\\
0.7205721	4.259463\\
0.7206721	4.260893\\
0.7207721	4.272177\\
0.7208721	4.265342\\
0.7209721	4.269274\\
0.7210721	4.272726\\
0.7211721	4.277269\\
0.7212721	4.284142\\
0.7213721	4.281224\\
0.7214721	4.294254\\
0.7215722	4.294571\\
0.7216722	4.306357\\
0.7217722	4.31426\\
0.7218722	4.318981\\
0.7219722	4.323799\\
0.7220722	4.325999\\
0.7221722	4.329605\\
0.7222722	4.327027\\
0.7223722	4.32515\\
0.7224722	4.324818\\
0.7225723	4.328346\\
0.7226723	4.332929\\
0.7227723	4.328021\\
0.7228723	4.327678\\
0.7229723	4.336435\\
0.7230723	4.343794\\
0.7231723	4.358998\\
0.7232723	4.368964\\
0.7233723	4.373275\\
0.7234723	4.375416\\
0.7235724	4.370537\\
0.7236724	4.371995\\
0.7237724	4.371693\\
0.7238724	4.37957\\
0.7239724	4.380789\\
0.7240724	4.37923\\
0.7241724	4.389596\\
0.7242724	4.390798\\
0.7243724	4.391144\\
0.7244724	4.396548\\
0.7245725	4.404595\\
0.7246725	4.408173\\
0.7247725	4.417417\\
0.7248725	4.42665\\
0.7249725	4.427045\\
0.7250725	4.428842\\
0.7251725	4.424862\\
0.7252725	4.426837\\
0.7253725	4.426398\\
0.7254725	4.433006\\
0.7255726	4.435724\\
0.7256726	4.439642\\
0.7257726	4.446229\\
0.7258726	4.441925\\
0.7259726	4.449183\\
0.7260726	4.454163\\
0.7261726	4.459974\\
0.7262726	4.470058\\
0.7263726	4.469874\\
0.7264726	4.476016\\
0.7265727	4.47232\\
0.7266727	4.468977\\
0.7267727	4.480086\\
0.7268727	4.486579\\
0.7269727	4.492383\\
0.7270727	4.486837\\
0.7271727	4.493658\\
0.7272727	4.499609\\
0.7273727	4.502795\\
0.7274727	4.498335\\
0.7275728	4.49468\\
0.7276728	4.500747\\
0.7277728	4.516261\\
0.7278728	4.516913\\
0.7279728	4.513772\\
0.7280728	4.516756\\
0.7281728	4.526758\\
0.7282728	4.541643\\
0.7283728	4.5339\\
0.7284728	4.539281\\
0.7285729	4.542127\\
0.7286729	4.546469\\
0.7287729	4.545235\\
0.7288729	4.557422\\
0.7289729	4.569187\\
0.7290729	4.566056\\
0.7291729	4.562386\\
0.7292729	4.559464\\
0.7293729	4.565201\\
0.7294729	4.568215\\
0.729573	4.569879\\
0.729673	4.575018\\
0.729773	4.570596\\
0.729873	4.578779\\
0.729973	4.583879\\
0.730073	4.581514\\
0.730173	4.579564\\
0.730273	4.579817\\
0.730373	4.583121\\
0.730473	4.590531\\
0.7305731	4.610335\\
0.7306731	4.615174\\
0.7307731	4.617308\\
0.7308731	4.622607\\
0.7309731	4.621947\\
0.7310731	4.621003\\
0.7311731	4.631185\\
0.7312731	4.633984\\
0.7313731	4.639384\\
0.7314731	4.638166\\
0.7315732	4.640614\\
0.7316732	4.641786\\
0.7317732	4.635516\\
0.7318732	4.646382\\
0.7319732	4.64746\\
0.7320732	4.650187\\
0.7321732	4.661859\\
0.7322732	4.661374\\
0.7323732	4.658513\\
0.7324732	4.651562\\
0.7325733	4.660008\\
0.7326733	4.667672\\
0.7327733	4.675111\\
0.7328733	4.674344\\
0.7329733	4.66912\\
0.7330733	4.672419\\
0.7331733	4.684625\\
0.7332733	4.686282\\
0.7333733	4.691675\\
0.7334733	4.699868\\
0.7335734	4.697566\\
0.7336734	4.708287\\
0.7337734	4.705133\\
0.7338734	4.706059\\
0.7339734	4.71246\\
0.7340734	4.72082\\
0.7341734	4.714048\\
0.7342734	4.710091\\
0.7343734	4.715895\\
0.7344734	4.717295\\
0.7345735	4.721965\\
0.7346735	4.726518\\
0.7347735	4.729288\\
0.7348735	4.738807\\
0.7349735	4.74503\\
0.7350735	4.744575\\
0.7351735	4.754957\\
0.7352735	4.752439\\
0.7353735	4.762927\\
0.7354735	4.758663\\
0.7355736	4.75344\\
0.7356736	4.761734\\
0.7357736	4.765539\\
0.7358736	4.759725\\
0.7359736	4.757429\\
0.7360736	4.763031\\
0.7361736	4.774348\\
0.7362736	4.76594\\
0.7363736	4.768535\\
0.7364736	4.776087\\
0.7365737	4.773951\\
0.7366737	4.787548\\
0.7367737	4.789952\\
0.7368737	4.796718\\
0.7369737	4.791336\\
0.7370737	4.798711\\
0.7371737	4.813297\\
0.7372737	4.810041\\
0.7373737	4.810808\\
0.7374737	4.8105\\
0.7375738	4.80507\\
0.7376738	4.810031\\
0.7377738	4.809729\\
0.7378738	4.820243\\
0.7379738	4.81994\\
0.7380738	4.825928\\
0.7381738	4.836352\\
0.7382738	4.833796\\
0.7383738	4.841231\\
0.7384738	4.840637\\
0.7385739	4.83505\\
0.7386739	4.849559\\
0.7387739	4.847601\\
0.7388739	4.853646\\
0.7389739	4.851517\\
0.7390739	4.84782\\
0.7391739	4.849446\\
0.7392739	4.849924\\
0.7393739	4.852732\\
0.7394739	4.847632\\
0.739574	4.850399\\
0.739674	4.857447\\
0.739774	4.862271\\
0.739874	4.86834\\
0.739974	4.879731\\
0.740074	4.878824\\
0.740174	4.890949\\
0.740274	4.896448\\
0.740374	4.901688\\
0.740474	4.899769\\
0.7405741	4.890252\\
0.7406741	4.885781\\
0.7407741	4.884348\\
0.7408741	4.895983\\
0.7409741	4.9018\\
0.7410741	4.902435\\
0.7411741	4.913114\\
0.7412741	4.905208\\
0.7413741	4.906121\\
0.7414741	4.915163\\
0.7415742	4.915103\\
0.7416742	4.921077\\
0.7417742	4.91705\\
0.7418742	4.918798\\
0.7419742	4.920627\\
0.7420742	4.914549\\
0.7421742	4.924022\\
0.7422742	4.927615\\
0.7423742	4.93123\\
0.7424742	4.94756\\
0.7425743	4.946246\\
0.7426743	4.945402\\
0.7427743	4.951839\\
0.7428743	4.947426\\
0.7429743	4.953293\\
0.7430743	4.950142\\
0.7431743	4.953605\\
0.7432743	4.956637\\
0.7433743	4.95568\\
0.7434743	4.95839\\
0.7435744	4.946186\\
0.7436744	4.951733\\
0.7437744	4.96024\\
0.7438744	4.969379\\
0.7439744	4.96674\\
0.7440744	4.972082\\
0.7441744	4.983544\\
0.7442744	4.992637\\
0.7443744	4.989121\\
0.7444744	4.987443\\
0.7445745	4.99358\\
0.7446745	4.990733\\
0.7447745	4.987318\\
0.7448745	4.987054\\
0.7449745	4.993548\\
0.7450745	5.001074\\
0.7451745	5.002235\\
0.7452745	5.000871\\
0.7453745	5.00547\\
0.7454745	5.00648\\
0.7455746	5.000255\\
0.7456746	5.000417\\
0.7457746	5.004178\\
0.7458746	5.003756\\
0.7459746	5.011121\\
0.7460746	5.013501\\
0.7461746	5.022127\\
0.7462746	5.020599\\
0.7463746	5.018252\\
0.7464746	5.022114\\
0.7465747	5.026198\\
0.7466747	5.03191\\
0.7467747	5.036556\\
0.7468747	5.042319\\
0.7469747	5.038394\\
0.7470747	5.0329\\
0.7471747	5.035029\\
0.7472747	5.045218\\
0.7473747	5.051327\\
0.7474747	5.053994\\
0.7475748	5.055995\\
0.7476748	5.056655\\
0.7477748	5.056856\\
0.7478748	5.058495\\
0.7479748	5.060116\\
0.7480748	5.047642\\
0.7481748	5.051609\\
0.7482748	5.063069\\
0.7483748	5.066757\\
0.7484748	5.061536\\
0.7485749	5.059942\\
0.7486749	5.068923\\
0.7487749	5.077774\\
0.7488749	5.070724\\
0.7489749	5.070305\\
0.7490749	5.076809\\
0.7491749	5.071319\\
0.7492749	5.072598\\
0.7493749	5.071239\\
0.7494749	5.075203\\
0.749575	5.083594\\
0.749675	5.087298\\
0.749775	5.089448\\
0.749875	5.086873\\
0.749975	5.096038\\
0.750075	5.104466\\
0.750175	5.103355\\
0.750275	5.101627\\
0.750375	5.10748\\
0.750475	5.111573\\
0.7505751	5.117973\\
0.7506751	5.116394\\
0.7507751	5.112401\\
0.7508751	5.107166\\
0.7509751	5.109323\\
0.7510751	5.108583\\
0.7511751	5.099656\\
0.7512751	5.103828\\
0.7513751	5.109516\\
0.7514751	5.111215\\
0.7515752	5.103358\\
0.7516752	5.123746\\
0.7517752	5.130376\\
0.7518752	5.130084\\
0.7519752	5.134414\\
0.7520752	5.134838\\
0.7521752	5.136482\\
0.7522752	5.137357\\
0.7523752	5.131252\\
0.7524752	5.13708\\
0.7525753	5.134633\\
0.7526753	5.132821\\
0.7527753	5.138434\\
0.7528753	5.134486\\
0.7529753	5.130698\\
0.7530753	5.132672\\
0.7531753	5.136106\\
0.7532753	5.14139\\
0.7533753	5.149225\\
0.7534753	5.154754\\
0.7535754	5.16199\\
0.7536754	5.15798\\
0.7537754	5.159874\\
0.7538754	5.16429\\
0.7539754	5.163387\\
0.7540754	5.165892\\
0.7541754	5.169126\\
0.7542754	5.165012\\
0.7543754	5.172222\\
0.7544754	5.163855\\
0.7545755	5.168116\\
0.7546755	5.165271\\
0.7547755	5.161595\\
0.7548755	5.162235\\
0.7549755	5.164464\\
0.7550755	5.16241\\
0.7551755	5.166684\\
0.7552755	5.177057\\
0.7553755	5.176695\\
0.7554755	5.18099\\
0.7555756	5.176685\\
0.7556756	5.180903\\
0.7557756	5.183138\\
0.7558756	5.182544\\
0.7559756	5.177115\\
0.7560756	5.186979\\
0.7561756	5.17418\\
0.7562756	5.183817\\
0.7563756	5.194008\\
0.7564756	5.200644\\
0.7565757	5.213717\\
0.7566757	5.209905\\
0.7567757	5.205485\\
0.7568757	5.19879\\
0.7569757	5.19272\\
0.7570757	5.191683\\
0.7571757	5.192527\\
0.7572757	5.192933\\
0.7573757	5.205352\\
0.7574757	5.202451\\
0.7575758	5.203743\\
0.7576758	5.20444\\
0.7577758	5.202561\\
0.7578758	5.209427\\
0.7579758	5.210614\\
0.7580758	5.211975\\
0.7581758	5.212509\\
0.7582758	5.210332\\
0.7583758	5.213524\\
0.7584758	5.210281\\
0.7585759	5.208312\\
0.7586759	5.208817\\
0.7587759	5.208912\\
0.7588759	5.216676\\
0.7589759	5.218911\\
0.7590759	5.224156\\
0.7591759	5.218855\\
0.7592759	5.221847\\
0.7593759	5.22587\\
0.7594759	5.233633\\
0.759576	5.236051\\
0.759676	5.234096\\
0.759776	5.237146\\
0.759876	5.233038\\
0.759976	5.226958\\
0.760076	5.219289\\
0.760176	5.226951\\
0.760276	5.227862\\
0.760376	5.224107\\
0.760476	5.222347\\
0.7605761	5.225768\\
0.7606761	5.231453\\
0.7607761	5.241784\\
0.7608761	5.244479\\
0.7609761	5.24863\\
0.7610761	5.245136\\
0.7611761	5.236143\\
0.7612761	5.239609\\
0.7613761	5.239444\\
0.7614761	5.235409\\
0.7615762	5.239572\\
0.7616762	5.241534\\
0.7617762	5.246833\\
0.7618762	5.246929\\
0.7619762	5.239241\\
0.7620762	5.247464\\
0.7621762	5.247244\\
0.7622762	5.246477\\
0.7623762	5.248073\\
0.7624762	5.248654\\
0.7625763	5.240325\\
0.7626763	5.234622\\
0.7627763	5.238939\\
0.7628763	5.245291\\
0.7629763	5.251876\\
0.7630763	5.254658\\
0.7631763	5.264998\\
0.7632763	5.267495\\
0.7633763	5.255121\\
0.7634763	5.255176\\
0.7635764	5.26134\\
0.7636764	5.262342\\
0.7637764	5.263916\\
0.7638764	5.273375\\
0.7639764	5.275075\\
0.7640764	5.263073\\
0.7641764	5.254869\\
0.7642764	5.255018\\
0.7643764	5.261169\\
0.7644764	5.246561\\
0.7645765	5.238948\\
0.7646765	5.248919\\
0.7647765	5.252782\\
0.7648765	5.25049\\
0.7649765	5.252916\\
0.7650765	5.252105\\
0.7651765	5.266381\\
0.7652765	5.264939\\
0.7653765	5.261749\\
0.7654765	5.267578\\
0.7655766	5.274122\\
0.7656766	5.279559\\
0.7657766	5.276886\\
0.7658766	5.273054\\
0.7659766	5.279433\\
0.7660766	5.28338\\
0.7661766	5.267586\\
0.7662766	5.262684\\
0.7663766	5.268408\\
0.7664766	5.263054\\
0.7665767	5.261104\\
0.7666767	5.265709\\
0.7667767	5.260311\\
0.7668767	5.275259\\
0.7669767	5.276259\\
0.7670767	5.276802\\
0.7671767	5.270525\\
0.7672767	5.273614\\
0.7673767	5.274142\\
0.7674767	5.277801\\
0.7675768	5.276064\\
0.7676768	5.271909\\
0.7677768	5.265684\\
0.7678768	5.263947\\
0.7679768	5.264832\\
0.7680768	5.269703\\
0.7681768	5.270134\\
0.7682768	5.268352\\
0.7683768	5.277053\\
0.7684768	5.270372\\
0.7685769	5.271904\\
0.7686769	5.269004\\
0.7687769	5.273757\\
0.7688769	5.280713\\
0.7689769	5.284274\\
0.7690769	5.279162\\
0.7691769	5.28001\\
0.7692769	5.276794\\
0.7693769	5.271539\\
0.7694769	5.273153\\
0.769577	5.283211\\
0.769677	5.271867\\
0.769777	5.270712\\
0.769877	5.274681\\
0.769977	5.275899\\
0.770077	5.282004\\
0.770177	5.274838\\
0.770277	5.275913\\
0.770377	5.275825\\
0.770477	5.27885\\
0.7705771	5.280536\\
0.7706771	5.279096\\
0.7707771	5.278351\\
0.7708771	5.271515\\
0.7709771	5.271909\\
0.7710771	5.266672\\
0.7711771	5.264609\\
0.7712771	5.270506\\
0.7713771	5.263414\\
0.7714771	5.265679\\
0.7715772	5.27203\\
0.7716772	5.275669\\
0.7717772	5.275668\\
0.7718772	5.274891\\
0.7719772	5.274666\\
0.7720772	5.272458\\
0.7721772	5.272323\\
0.7722772	5.262293\\
0.7723772	5.266128\\
0.7724772	5.271836\\
0.7725773	5.272327\\
0.7726773	5.28545\\
0.7727773	5.283177\\
0.7728773	5.283307\\
0.7729773	5.275508\\
0.7730773	5.274131\\
0.7731773	5.275056\\
0.7732773	5.272164\\
0.7733773	5.268817\\
0.7734773	5.270656\\
0.7735774	5.273544\\
0.7736774	5.268932\\
0.7737774	5.259036\\
0.7738774	5.25027\\
0.7739774	5.253862\\
0.7740774	5.248965\\
0.7741774	5.251698\\
0.7742774	5.254614\\
0.7743774	5.263863\\
0.7744774	5.27163\\
0.7745775	5.269878\\
0.7746775	5.266745\\
0.7747775	5.2694\\
0.7748775	5.269385\\
0.7749775	5.268276\\
0.7750775	5.270831\\
0.7751775	5.277993\\
0.7752775	5.269765\\
0.7753775	5.260007\\
0.7754775	5.248362\\
0.7755776	5.246035\\
0.7756776	5.262647\\
0.7757776	5.265207\\
0.7758776	5.26015\\
0.7759776	5.25925\\
0.7760776	5.257671\\
0.7761776	5.260306\\
0.7762776	5.263929\\
0.7763776	5.251806\\
0.7764776	5.251655\\
0.7765777	5.250323\\
0.7766777	5.24188\\
0.7767777	5.237221\\
0.7768777	5.245403\\
0.7769777	5.253491\\
0.7770777	5.25422\\
0.7771777	5.255973\\
0.7772777	5.263664\\
0.7773777	5.261947\\
0.7774777	5.256894\\
0.7775778	5.246512\\
0.7776778	5.2465\\
0.7777778	5.251678\\
0.7778778	5.249365\\
0.7779778	5.239115\\
0.7780778	5.226839\\
0.7781778	5.229427\\
0.7782778	5.22886\\
0.7783778	5.239372\\
0.7784778	5.242652\\
0.7785779	5.249793\\
0.7786779	5.247694\\
0.7787779	5.2455\\
0.7788779	5.245172\\
0.7789779	5.241283\\
0.7790779	5.242563\\
0.7791779	5.247965\\
0.7792779	5.243907\\
0.7793779	5.234908\\
0.7794779	5.234457\\
0.779578	5.239866\\
0.779678	5.230377\\
0.779778	5.223437\\
0.779878	5.225455\\
0.779978	5.227877\\
0.780078	5.22238\\
0.780178	5.224458\\
0.780278	5.22749\\
0.780378	5.228883\\
0.780478	5.221107\\
0.7805781	5.221741\\
0.7806781	5.223622\\
0.7807781	5.222569\\
0.7808781	5.221561\\
0.7809781	5.212653\\
0.7810781	5.21498\\
0.7811781	5.220723\\
0.7812781	5.215162\\
0.7813781	5.20785\\
0.7814781	5.209793\\
0.7815782	5.211907\\
0.7816782	5.214852\\
0.7817782	5.214557\\
0.7818782	5.20833\\
0.7819782	5.206845\\
0.7820782	5.214805\\
0.7821782	5.214087\\
0.7822782	5.214269\\
0.7823782	5.218426\\
0.7824782	5.222288\\
0.7825783	5.211195\\
0.7826783	5.204207\\
0.7827783	5.194067\\
0.7828783	5.199141\\
0.7829783	5.197535\\
0.7830783	5.193047\\
0.7831783	5.185519\\
0.7832783	5.176452\\
0.7833783	5.180227\\
0.7834783	5.186291\\
0.7835784	5.186918\\
0.7836784	5.18642\\
0.7837784	5.188798\\
0.7838784	5.187682\\
0.7839784	5.186093\\
0.7840784	5.181061\\
0.7841784	5.179374\\
0.7842784	5.180587\\
0.7843784	5.180032\\
0.7844784	5.182236\\
0.7845785	5.186204\\
0.7846785	5.174153\\
0.7847785	5.171293\\
0.7848785	5.178777\\
0.7849785	5.169408\\
0.7850785	5.179074\\
0.7851785	5.175207\\
0.7852785	5.164074\\
0.7853785	5.158738\\
0.7854785	5.165118\\
0.7855786	5.161791\\
0.7856786	5.164959\\
0.7857786	5.16608\\
0.7858786	5.157327\\
0.7859786	5.161387\\
0.7860786	5.156178\\
0.7861786	5.157412\\
0.7862786	5.148716\\
0.7863786	5.14441\\
0.7864786	5.144836\\
0.7865787	5.148139\\
0.7866787	5.14986\\
0.7867787	5.155281\\
0.7868787	5.148569\\
0.7869787	5.143958\\
0.7870787	5.134239\\
0.7871787	5.136281\\
0.7872787	5.14071\\
0.7873787	5.127053\\
0.7874787	5.123994\\
0.7875788	5.130697\\
0.7876788	5.129363\\
0.7877788	5.130539\\
0.7878788	5.125267\\
0.7879788	5.121777\\
0.7880788	5.120881\\
0.7881788	5.125044\\
0.7882788	5.127434\\
0.7883788	5.132119\\
0.7884788	5.128978\\
0.7885789	5.12837\\
0.7886789	5.129996\\
0.7887789	5.122441\\
0.7888789	5.120479\\
0.7889789	5.110227\\
0.7890789	5.110756\\
0.7891789	5.109255\\
0.7892789	5.100824\\
0.7893789	5.088099\\
0.7894789	5.096179\\
0.789579	5.101177\\
0.789679	5.097591\\
0.789779	5.093754\\
0.789879	5.104802\\
0.789979	5.09777\\
0.790079	5.092387\\
0.790179	5.081072\\
0.790279	5.076655\\
0.790379	5.072761\\
0.790479	5.080632\\
0.7905791	5.081428\\
0.7906791	5.07552\\
0.7907791	5.096139\\
0.7908791	5.091008\\
0.7909791	5.073558\\
0.7910791	5.070804\\
0.7911791	5.082318\\
0.7912791	5.089214\\
0.7913791	5.076487\\
0.7914791	5.07329\\
0.7915792	5.06629\\
0.7916792	5.066346\\
0.7917792	5.064942\\
0.7918792	5.058299\\
0.7919792	5.048516\\
0.7920792	5.051519\\
0.7921792	5.054131\\
0.7922792	5.04333\\
0.7923792	5.043635\\
0.7924792	5.053214\\
0.7925793	5.050573\\
0.7926793	5.049444\\
0.7927793	5.043116\\
0.7928793	5.034127\\
0.7929793	5.032113\\
0.7930793	5.033317\\
0.7931793	5.024874\\
0.7932793	5.016177\\
0.7933793	5.022173\\
0.7934793	5.021684\\
0.7935794	5.020764\\
0.7936794	5.018004\\
0.7937794	5.022247\\
0.7938794	5.023074\\
0.7939794	5.029676\\
0.7940794	5.033528\\
0.7941794	5.034739\\
0.7942794	5.020663\\
0.7943794	5.011629\\
0.7944794	5.008159\\
0.7945795	4.990606\\
0.7946795	4.99366\\
0.7947795	4.991526\\
0.7948795	4.986215\\
0.7949795	4.984698\\
0.7950795	4.990356\\
0.7951795	4.987581\\
0.7952795	4.985184\\
0.7953795	4.987134\\
0.7954795	4.984527\\
0.7955796	4.98971\\
0.7956796	4.987204\\
0.7957796	4.982591\\
0.7958796	4.979442\\
0.7959796	4.969281\\
0.7960796	4.963403\\
0.7961796	4.961303\\
0.7962796	4.964858\\
0.7963796	4.9643\\
0.7964796	4.95604\\
0.7965797	4.950017\\
0.7966797	4.954711\\
0.7967797	4.951219\\
0.7968797	4.940408\\
0.7969797	4.937931\\
0.7970797	4.945693\\
0.7971797	4.941993\\
0.7972797	4.943857\\
0.7973797	4.940986\\
0.7974797	4.935172\\
0.7975798	4.932199\\
0.7976798	4.936708\\
0.7977798	4.942373\\
0.7978798	4.928418\\
0.7979798	4.927252\\
0.7980798	4.933146\\
0.7981798	4.92472\\
0.7982798	4.918255\\
0.7983798	4.923241\\
0.7984798	4.919042\\
0.7985799	4.915661\\
0.7986799	4.908262\\
0.7987799	4.899206\\
0.7988799	4.890598\\
0.7989799	4.895001\\
0.7990799	4.897711\\
0.7991799	4.893785\\
0.7992799	4.893384\\
0.7993799	4.884705\\
0.7994799	4.888211\\
0.79958	4.878143\\
0.79968	4.870626\\
0.79978	4.868787\\
0.79988	4.873916\\
0.79998	4.876792\\
0.80008	4.878103\\
};
\addplot [color=mycolor1,solid]
  table[row sep=crcr]{%
0.80008	4.878103\\
0.80018	4.873618\\
0.80028	4.874526\\
0.80038	4.871356\\
0.80048	4.875637\\
0.8005801	4.864094\\
0.8006801	4.870974\\
0.8007801	4.86083\\
0.8008801	4.858067\\
0.8009801	4.851789\\
0.8010801	4.851776\\
0.8011801	4.84409\\
0.8012801	4.841027\\
0.8013801	4.836053\\
0.8014801	4.836606\\
0.8015802	4.823674\\
0.8016802	4.823836\\
0.8017802	4.834706\\
0.8018802	4.838027\\
0.8019802	4.830105\\
0.8020802	4.824568\\
0.8021802	4.819908\\
0.8022802	4.815425\\
0.8023802	4.811427\\
0.8024802	4.810978\\
0.8025803	4.796663\\
0.8026803	4.803308\\
0.8027803	4.804256\\
0.8028803	4.800649\\
0.8029803	4.801146\\
0.8030803	4.794329\\
0.8031803	4.796141\\
0.8032803	4.789794\\
0.8033803	4.786587\\
0.8034803	4.775432\\
0.8035804	4.778514\\
0.8036804	4.77687\\
0.8037804	4.779925\\
0.8038804	4.776208\\
0.8039804	4.76617\\
0.8040804	4.764479\\
0.8041804	4.770375\\
0.8042804	4.767296\\
0.8043804	4.768165\\
0.8044804	4.755043\\
0.8045805	4.75314\\
0.8046805	4.742291\\
0.8047805	4.743092\\
0.8048805	4.739957\\
0.8049805	4.731822\\
0.8050805	4.739598\\
0.8051805	4.747065\\
0.8052805	4.742538\\
0.8053805	4.735934\\
0.8054805	4.731642\\
0.8055806	4.72211\\
0.8056806	4.705352\\
0.8057806	4.701381\\
0.8058806	4.702959\\
0.8059806	4.699802\\
0.8060806	4.70134\\
0.8061806	4.706942\\
0.8062806	4.701519\\
0.8063806	4.697845\\
0.8064806	4.694479\\
0.8065807	4.68258\\
0.8066807	4.684798\\
0.8067807	4.681518\\
0.8068807	4.691723\\
0.8069807	4.67812\\
0.8070807	4.67444\\
0.8071807	4.677032\\
0.8072807	4.679046\\
0.8073807	4.679081\\
0.8074807	4.668965\\
0.8075808	4.672206\\
0.8076808	4.670552\\
0.8077808	4.664511\\
0.8078808	4.647533\\
0.8079808	4.650224\\
0.8080808	4.65179\\
0.8081808	4.64503\\
0.8082808	4.626654\\
0.8083808	4.619654\\
0.8084808	4.613345\\
0.8085809	4.611182\\
0.8086809	4.610572\\
0.8087809	4.613476\\
0.8088809	4.612522\\
0.8089809	4.612629\\
0.8090809	4.613532\\
0.8091809	4.617177\\
0.8092809	4.61561\\
0.8093809	4.618137\\
0.8094809	4.615001\\
0.809581	4.609225\\
0.809681	4.604371\\
0.809781	4.593267\\
0.809881	4.580121\\
0.809981	4.57879\\
0.810081	4.584546\\
0.810181	4.57647\\
0.810281	4.57653\\
0.810381	4.575714\\
0.810481	4.57408\\
0.8105811	4.575027\\
0.8106811	4.572612\\
0.8107811	4.564837\\
0.8108811	4.551203\\
0.8109811	4.552328\\
0.8110811	4.547364\\
0.8111811	4.543144\\
0.8112811	4.544531\\
0.8113811	4.532618\\
0.8114811	4.529358\\
0.8115812	4.537885\\
0.8116812	4.529953\\
0.8117812	4.525544\\
0.8118812	4.530755\\
0.8119812	4.527093\\
0.8120812	4.523653\\
0.8121812	4.514404\\
0.8122812	4.514388\\
0.8123812	4.510954\\
0.8124812	4.503273\\
0.8125813	4.501843\\
0.8126813	4.508354\\
0.8127813	4.492769\\
0.8128813	4.482177\\
0.8129813	4.478635\\
0.8130813	4.477191\\
0.8131813	4.480196\\
0.8132813	4.487443\\
0.8133813	4.488246\\
0.8134813	4.478671\\
0.8135814	4.480203\\
0.8136814	4.475641\\
0.8137814	4.475924\\
0.8138814	4.457948\\
0.8139814	4.455375\\
0.8140814	4.453981\\
0.8141814	4.450111\\
0.8142814	4.446436\\
0.8143814	4.44239\\
0.8144814	4.441948\\
0.8145815	4.433042\\
0.8146815	4.434859\\
0.8147815	4.427751\\
0.8148815	4.417084\\
0.8149815	4.408793\\
0.8150815	4.414931\\
0.8151815	4.409974\\
0.8152815	4.406228\\
0.8153815	4.400054\\
0.8154815	4.39551\\
0.8155816	4.405015\\
0.8156816	4.394336\\
0.8157816	4.395609\\
0.8158816	4.396588\\
0.8159816	4.389096\\
0.8160816	4.382223\\
0.8161816	4.378181\\
0.8162816	4.379161\\
0.8163816	4.372965\\
0.8164816	4.368213\\
0.8165817	4.362915\\
0.8166817	4.365446\\
0.8167817	4.357964\\
0.8168817	4.356794\\
0.8169817	4.355907\\
0.8170817	4.352059\\
0.8171817	4.346632\\
0.8172817	4.342344\\
0.8173817	4.340168\\
0.8174817	4.328898\\
0.8175818	4.317519\\
0.8176818	4.31\\
0.8177818	4.31264\\
0.8178818	4.306801\\
0.8179818	4.302254\\
0.8180818	4.29936\\
0.8181818	4.300037\\
0.8182818	4.294979\\
0.8183818	4.298907\\
0.8184818	4.300905\\
0.8185819	4.287262\\
0.8186819	4.286963\\
0.8187819	4.285293\\
0.8188819	4.279449\\
0.8189819	4.275777\\
0.8190819	4.274104\\
0.8191819	4.274925\\
0.8192819	4.275304\\
0.8193819	4.268248\\
0.8194819	4.253091\\
0.819582	4.247425\\
0.819682	4.247745\\
0.819782	4.242469\\
0.819882	4.235025\\
0.819982	4.232933\\
0.820082	4.229545\\
0.820182	4.221194\\
0.820282	4.228131\\
0.820382	4.224373\\
0.820482	4.219203\\
0.8205821	4.20917\\
0.8206821	4.2142\\
0.8207821	4.221778\\
0.8208821	4.209327\\
0.8209821	4.207256\\
0.8210821	4.198095\\
0.8211821	4.194136\\
0.8212821	4.184299\\
0.8213821	4.185509\\
0.8214821	4.187957\\
0.8215822	4.175741\\
0.8216822	4.170787\\
0.8217822	4.163868\\
0.8218822	4.166048\\
0.8219822	4.15779\\
0.8220822	4.146683\\
0.8221822	4.150372\\
0.8222822	4.150719\\
0.8223822	4.149603\\
0.8224822	4.15069\\
0.8225823	4.154679\\
0.8226823	4.141663\\
0.8227823	4.139334\\
0.8228823	4.136442\\
0.8229823	4.13799\\
0.8230823	4.126764\\
0.8231823	4.127003\\
0.8232823	4.123665\\
0.8233823	4.116099\\
0.8234823	4.112761\\
0.8235824	4.108855\\
0.8236824	4.104544\\
0.8237824	4.094178\\
0.8238824	4.083632\\
0.8239824	4.086537\\
0.8240824	4.087061\\
0.8241824	4.088561\\
0.8242824	4.075168\\
0.8243824	4.061765\\
0.8244824	4.064005\\
0.8245825	4.070352\\
0.8246825	4.064376\\
0.8247825	4.054147\\
0.8248825	4.055713\\
0.8249825	4.048797\\
0.8250825	4.048338\\
0.8251825	4.042935\\
0.8252825	4.039315\\
0.8253825	4.039162\\
0.8254825	4.0309\\
0.8255826	4.031783\\
0.8256826	4.036314\\
0.8257826	4.037809\\
0.8258826	4.032251\\
0.8259826	4.022963\\
0.8260826	4.014775\\
0.8261826	4.010283\\
0.8262826	4.000323\\
0.8263826	3.98522\\
0.8264826	3.982833\\
0.8265827	3.989582\\
0.8266827	3.988936\\
0.8267827	3.984166\\
0.8268827	3.974711\\
0.8269827	3.970713\\
0.8270827	3.969138\\
0.8271827	3.962844\\
0.8272827	3.959843\\
0.8273827	3.95478\\
0.8274827	3.950741\\
0.8275828	3.94533\\
0.8276828	3.940043\\
0.8277828	3.929888\\
0.8278828	3.931298\\
0.8279828	3.93064\\
0.8280828	3.924908\\
0.8281828	3.922695\\
0.8282828	3.918296\\
0.8283828	3.911898\\
0.8284828	3.911754\\
0.8285829	3.915997\\
0.8286829	3.913999\\
0.8287829	3.906387\\
0.8288829	3.904229\\
0.8289829	3.895823\\
0.8290829	3.882671\\
0.8291829	3.878801\\
0.8292829	3.870963\\
0.8293829	3.872118\\
0.8294829	3.868671\\
0.829583	3.855577\\
0.829683	3.862895\\
0.829783	3.854484\\
0.829883	3.854663\\
0.829983	3.845576\\
0.830083	3.846869\\
0.830183	3.847141\\
0.830283	3.840363\\
0.830383	3.833777\\
0.830483	3.821594\\
0.8305831	3.819673\\
0.8306831	3.816888\\
0.8307831	3.817719\\
0.8308831	3.81306\\
0.8309831	3.802498\\
0.8310831	3.798656\\
0.8311831	3.799318\\
0.8312831	3.78799\\
0.8313831	3.785122\\
0.8314831	3.775114\\
0.8315832	3.773168\\
0.8316832	3.774879\\
0.8317832	3.774698\\
0.8318832	3.778347\\
0.8319832	3.771858\\
0.8320832	3.768886\\
0.8321832	3.768546\\
0.8322832	3.763321\\
0.8323832	3.761539\\
0.8324832	3.754017\\
0.8325833	3.751557\\
0.8326833	3.742116\\
0.8327833	3.740628\\
0.8328833	3.740841\\
0.8329833	3.734344\\
0.8330833	3.720077\\
0.8331833	3.707525\\
0.8332833	3.704369\\
0.8333833	3.699602\\
0.8334833	3.693047\\
0.8335834	3.695847\\
0.8336834	3.701851\\
0.8337834	3.696768\\
0.8338834	3.696163\\
0.8339834	3.680309\\
0.8340834	3.685184\\
0.8341834	3.680746\\
0.8342834	3.677208\\
0.8343834	3.671694\\
0.8344834	3.670013\\
0.8345835	3.675449\\
0.8346835	3.668924\\
0.8347835	3.658408\\
0.8348835	3.653471\\
0.8349835	3.651567\\
0.8350835	3.649777\\
0.8351835	3.646601\\
0.8352835	3.646401\\
0.8353835	3.639028\\
0.8354835	3.630239\\
0.8355836	3.622452\\
0.8356836	3.61516\\
0.8357836	3.612431\\
0.8358836	3.614376\\
0.8359836	3.60784\\
0.8360836	3.605224\\
0.8361836	3.602458\\
0.8362836	3.599142\\
0.8363836	3.599653\\
0.8364836	3.589186\\
0.8365837	3.578038\\
0.8366837	3.571776\\
0.8367837	3.571275\\
0.8368837	3.560349\\
0.8369837	3.555522\\
0.8370837	3.57167\\
0.8371837	3.561449\\
0.8372837	3.557187\\
0.8373837	3.553379\\
0.8374837	3.550301\\
0.8375838	3.550233\\
0.8376838	3.538939\\
0.8377838	3.53461\\
0.8378838	3.530581\\
0.8379838	3.523601\\
0.8380838	3.517375\\
0.8381838	3.517938\\
0.8382838	3.511303\\
0.8383838	3.510529\\
0.8384838	3.500612\\
0.8385839	3.499788\\
0.8386839	3.489481\\
0.8387839	3.491642\\
0.8388839	3.48737\\
0.8389839	3.481693\\
0.8390839	3.478377\\
0.8391839	3.473451\\
0.8392839	3.47433\\
0.8393839	3.462365\\
0.8394839	3.452017\\
0.839584	3.453971\\
0.839684	3.450445\\
0.839784	3.443733\\
0.839884	3.43572\\
0.839984	3.428434\\
0.840084	3.429884\\
0.840184	3.427557\\
0.840284	3.424442\\
0.840384	3.41293\\
0.840484	3.414645\\
0.8405841	3.412145\\
0.8406841	3.397344\\
0.8407841	3.394051\\
0.8408841	3.393978\\
0.8409841	3.391167\\
0.8410841	3.387095\\
0.8411841	3.381119\\
0.8412841	3.377579\\
0.8413841	3.376414\\
0.8414841	3.371917\\
0.8415842	3.365417\\
0.8416842	3.363939\\
0.8417842	3.36852\\
0.8418842	3.361416\\
0.8419842	3.35735\\
0.8420842	3.35551\\
0.8421842	3.348348\\
0.8422842	3.341196\\
0.8423842	3.331511\\
0.8424842	3.324088\\
0.8425843	3.317853\\
0.8426843	3.310048\\
0.8427843	3.310164\\
0.8428843	3.302688\\
0.8429843	3.296773\\
0.8430843	3.289137\\
0.8431843	3.282474\\
0.8432843	3.283477\\
0.8433843	3.282543\\
0.8434843	3.285193\\
0.8435844	3.296127\\
0.8436844	3.290084\\
0.8437844	3.282544\\
0.8438844	3.280046\\
0.8439844	3.270203\\
0.8440844	3.273066\\
0.8441844	3.261679\\
0.8442844	3.253952\\
0.8443844	3.251637\\
0.8444844	3.241164\\
0.8445845	3.238763\\
0.8446845	3.229697\\
0.8447845	3.231682\\
0.8448845	3.239057\\
0.8449845	3.234622\\
0.8450845	3.226544\\
0.8451845	3.220142\\
0.8452845	3.215996\\
0.8453845	3.211614\\
0.8454845	3.205707\\
0.8455846	3.19743\\
0.8456846	3.19081\\
0.8457846	3.194564\\
0.8458846	3.19371\\
0.8459846	3.188239\\
0.8460846	3.192855\\
0.8461846	3.182555\\
0.8462846	3.178053\\
0.8463846	3.169414\\
0.8464846	3.165369\\
0.8465847	3.16167\\
0.8466847	3.151719\\
0.8467847	3.152114\\
0.8468847	3.145183\\
0.8469847	3.138132\\
0.8470847	3.143564\\
0.8471847	3.139572\\
0.8472847	3.135945\\
0.8473847	3.13502\\
0.8474847	3.128467\\
0.8475848	3.13333\\
0.8476848	3.129076\\
0.8477848	3.114756\\
0.8478848	3.10924\\
0.8479848	3.104429\\
0.8480848	3.103714\\
0.8481848	3.094404\\
0.8482848	3.088269\\
0.8483848	3.093604\\
0.8484848	3.091753\\
0.8485849	3.084613\\
0.8486849	3.073309\\
0.8487849	3.068749\\
0.8488849	3.064725\\
0.8489849	3.062407\\
0.8490849	3.055982\\
0.8491849	3.054896\\
0.8492849	3.05651\\
0.8493849	3.046572\\
0.8494849	3.029152\\
0.849585	3.020453\\
0.849685	3.022565\\
0.849785	3.025691\\
0.849885	3.021365\\
0.849985	3.011507\\
0.850085	3.007059\\
0.850185	3.005393\\
0.850285	3.000109\\
0.850385	2.998759\\
0.850485	2.997553\\
0.8505851	3.001427\\
0.8506851	2.998779\\
0.8507851	2.984896\\
0.8508851	2.980527\\
0.8509851	2.975715\\
0.8510851	2.971071\\
0.8511851	2.967703\\
0.8512851	2.963637\\
0.8513851	2.956763\\
0.8514851	2.951823\\
0.8515852	2.943596\\
0.8516852	2.942221\\
0.8517852	2.939379\\
0.8518852	2.926331\\
0.8519852	2.915204\\
0.8520852	2.911664\\
0.8521852	2.905952\\
0.8522852	2.900858\\
0.8523852	2.900376\\
0.8524852	2.90139\\
0.8525853	2.899185\\
0.8526853	2.899472\\
0.8527853	2.891262\\
0.8528853	2.886866\\
0.8529853	2.880101\\
0.8530853	2.881196\\
0.8531853	2.88436\\
0.8532853	2.869487\\
0.8533853	2.868128\\
0.8534853	2.868423\\
0.8535854	2.864109\\
0.8536854	2.85777\\
0.8537854	2.856812\\
0.8538854	2.844608\\
0.8539854	2.840859\\
0.8540854	2.8387\\
0.8541854	2.832533\\
0.8542854	2.828811\\
0.8543854	2.828826\\
0.8544854	2.821981\\
0.8545855	2.809347\\
0.8546855	2.804533\\
0.8547855	2.798234\\
0.8548855	2.79916\\
0.8549855	2.798337\\
0.8550855	2.796641\\
0.8551855	2.798683\\
0.8552855	2.796768\\
0.8553855	2.788783\\
0.8554855	2.786918\\
0.8555856	2.782488\\
0.8556856	2.780721\\
0.8557856	2.778804\\
0.8558856	2.767967\\
0.8559856	2.756803\\
0.8560856	2.751234\\
0.8561856	2.741417\\
0.8562856	2.740304\\
0.8563856	2.739503\\
0.8564856	2.738617\\
0.8565857	2.737679\\
0.8566857	2.737075\\
0.8567857	2.738527\\
0.8568857	2.737614\\
0.8569857	2.7381\\
0.8570857	2.729461\\
0.8571857	2.724889\\
0.8572857	2.722148\\
0.8573857	2.717997\\
0.8574857	2.714114\\
0.8575858	2.70491\\
0.8576858	2.699932\\
0.8577858	2.699018\\
0.8578858	2.687425\\
0.8579858	2.68185\\
0.8580858	2.675258\\
0.8581858	2.673003\\
0.8582858	2.667207\\
0.8583858	2.671942\\
0.8584858	2.666914\\
0.8585859	2.65943\\
0.8586859	2.66044\\
0.8587859	2.653272\\
0.8588859	2.650784\\
0.8589859	2.653817\\
0.8590859	2.645521\\
0.8591859	2.63689\\
0.8592859	2.629207\\
0.8593859	2.625091\\
0.8594859	2.622815\\
0.859586	2.619454\\
0.859686	2.61861\\
0.859786	2.614652\\
0.859886	2.615955\\
0.859986	2.611954\\
0.860086	2.606836\\
0.860186	2.605143\\
0.860286	2.603377\\
0.860386	2.596855\\
0.860486	2.595165\\
0.8605861	2.583291\\
0.8606861	2.568657\\
0.8607861	2.565094\\
0.8608861	2.560488\\
0.8609861	2.556028\\
0.8610861	2.554485\\
0.8611861	2.55113\\
0.8612861	2.545421\\
0.8613861	2.542437\\
0.8614861	2.536971\\
0.8615862	2.525265\\
0.8616862	2.523942\\
0.8617862	2.523563\\
0.8618862	2.521335\\
0.8619862	2.520997\\
0.8620862	2.515803\\
0.8621862	2.507956\\
0.8622862	2.49954\\
0.8623862	2.495969\\
0.8624862	2.494555\\
0.8625863	2.492426\\
0.8626863	2.49512\\
0.8627863	2.488797\\
0.8628863	2.481195\\
0.8629863	2.47895\\
0.8630863	2.473189\\
0.8631863	2.465607\\
0.8632863	2.457537\\
0.8633863	2.449626\\
0.8634863	2.439332\\
0.8635864	2.435651\\
0.8636864	2.434823\\
0.8637864	2.436766\\
0.8638864	2.434638\\
0.8639864	2.435256\\
0.8640864	2.431997\\
0.8641864	2.422584\\
0.8642864	2.414528\\
0.8643864	2.416145\\
0.8644864	2.41527\\
0.8645865	2.405897\\
0.8646865	2.403108\\
0.8647865	2.398449\\
0.8648865	2.395118\\
0.8649865	2.381681\\
0.8650865	2.377395\\
0.8651865	2.370823\\
0.8652865	2.370452\\
0.8653865	2.37672\\
0.8654865	2.371683\\
0.8655866	2.362861\\
0.8656866	2.360522\\
0.8657866	2.355718\\
0.8658866	2.349789\\
0.8659866	2.353098\\
0.8660866	2.349729\\
0.8661866	2.339945\\
0.8662866	2.340912\\
0.8663866	2.339246\\
0.8664866	2.335343\\
0.8665867	2.337949\\
0.8666867	2.337117\\
0.8667867	2.32667\\
0.8668867	2.324371\\
0.8669867	2.327657\\
0.8670867	2.323847\\
0.8671867	2.308767\\
0.8672867	2.300541\\
0.8673867	2.297343\\
0.8674867	2.292114\\
0.8675868	2.291773\\
0.8676868	2.28758\\
0.8677868	2.282991\\
0.8678868	2.272177\\
0.8679868	2.270235\\
0.8680868	2.269827\\
0.8681868	2.272382\\
0.8682868	2.269096\\
0.8683868	2.263196\\
0.8684868	2.266911\\
0.8685869	2.26819\\
0.8686869	2.265887\\
0.8687869	2.260409\\
0.8688869	2.249544\\
0.8689869	2.248864\\
0.8690869	2.249319\\
0.8691869	2.250039\\
0.8692869	2.244212\\
0.8693869	2.231029\\
0.8694869	2.228346\\
0.869587	2.227964\\
0.869687	2.224018\\
0.869787	2.216826\\
0.869887	2.210264\\
0.869987	2.208549\\
0.870087	2.205151\\
0.870187	2.200889\\
0.870287	2.198248\\
0.870387	2.195015\\
0.870487	2.196989\\
0.8705871	2.191526\\
0.8706871	2.186179\\
0.8707871	2.181774\\
0.8708871	2.178378\\
0.8709871	2.168371\\
0.8710871	2.161407\\
0.8711871	2.157672\\
0.8712871	2.154619\\
0.8713871	2.149966\\
0.8714871	2.145201\\
0.8715872	2.138645\\
0.8716872	2.140174\\
0.8717872	2.143108\\
0.8718872	2.139761\\
0.8719872	2.141375\\
0.8720872	2.133677\\
0.8721872	2.131441\\
0.8722872	2.126201\\
0.8723872	2.119572\\
0.8724872	2.11044\\
0.8725873	2.104339\\
0.8726873	2.091362\\
0.8727873	2.089321\\
0.8728873	2.090929\\
0.8729873	2.086601\\
0.8730873	2.079437\\
0.8731873	2.074735\\
0.8732873	2.075033\\
0.8733873	2.074836\\
0.8734873	2.07289\\
0.8735874	2.067436\\
0.8736874	2.064957\\
0.8737874	2.058614\\
0.8738874	2.053761\\
0.8739874	2.0421\\
0.8740874	2.040891\\
0.8741874	2.033306\\
0.8742874	2.027216\\
0.8743874	2.023673\\
0.8744874	2.017367\\
0.8745875	2.013911\\
0.8746875	2.0074\\
0.8747875	2.000788\\
0.8748875	1.999154\\
0.8749875	1.998941\\
0.8750875	2.003445\\
0.8751875	2.003199\\
0.8752875	1.99806\\
0.8753875	1.995033\\
0.8754875	1.985907\\
0.8755876	1.985114\\
0.8756876	1.979775\\
0.8757876	1.978245\\
0.8758876	1.976562\\
0.8759876	1.97077\\
0.8760876	1.963243\\
0.8761876	1.962225\\
0.8762876	1.959447\\
0.8763876	1.950462\\
0.8764876	1.943445\\
0.8765877	1.939932\\
0.8766877	1.940005\\
0.8767877	1.932583\\
0.8768877	1.929635\\
0.8769877	1.927057\\
0.8770877	1.920178\\
0.8771877	1.913627\\
0.8772877	1.916839\\
0.8773877	1.914373\\
0.8774877	1.907785\\
0.8775878	1.907515\\
0.8776878	1.906175\\
0.8777878	1.903687\\
0.8778878	1.902144\\
0.8779878	1.904139\\
0.8780878	1.901388\\
0.8781878	1.90517\\
0.8782878	1.902346\\
0.8783878	1.895882\\
0.8784878	1.885476\\
0.8785879	1.882971\\
0.8786879	1.879242\\
0.8787879	1.873136\\
0.8788879	1.865173\\
0.8789879	1.866133\\
0.8790879	1.861782\\
0.8791879	1.861333\\
0.8792879	1.864839\\
0.8793879	1.861546\\
0.8794879	1.856366\\
0.879588	1.85374\\
0.879688	1.843076\\
0.879788	1.837638\\
0.879888	1.83909\\
0.879988	1.83588\\
0.880088	1.839976\\
0.880188	1.833028\\
0.880288	1.829203\\
0.880388	1.828244\\
0.880488	1.826555\\
0.8805881	1.817755\\
0.8806881	1.81855\\
0.8807881	1.818161\\
0.8808881	1.815706\\
0.8809881	1.807618\\
0.8810881	1.802524\\
0.8811881	1.796976\\
0.8812881	1.792746\\
0.8813881	1.788162\\
0.8814881	1.7924\\
0.8815882	1.794524\\
0.8816882	1.789343\\
0.8817882	1.781764\\
0.8818882	1.779194\\
0.8819882	1.780578\\
0.8820882	1.775712\\
0.8821882	1.770839\\
0.8822882	1.769153\\
0.8823882	1.761009\\
0.8824882	1.757169\\
0.8825883	1.755462\\
0.8826883	1.747069\\
0.8827883	1.74139\\
0.8828883	1.734497\\
0.8829883	1.730402\\
0.8830883	1.728681\\
0.8831883	1.72447\\
0.8832883	1.722909\\
0.8833883	1.723046\\
0.8834883	1.719771\\
0.8835884	1.714471\\
0.8836884	1.706806\\
0.8837884	1.705263\\
0.8838884	1.700475\\
0.8839884	1.693521\\
0.8840884	1.692169\\
0.8841884	1.687292\\
0.8842884	1.679367\\
0.8843884	1.677452\\
0.8844884	1.679191\\
0.8845885	1.672329\\
0.8846885	1.668251\\
0.8847885	1.671253\\
0.8848885	1.664509\\
0.8849885	1.656586\\
0.8850885	1.653565\\
0.8851885	1.658909\\
0.8852885	1.65085\\
0.8853885	1.642095\\
0.8854885	1.643423\\
0.8855886	1.639663\\
0.8856886	1.63566\\
0.8857886	1.629969\\
0.8858886	1.618665\\
0.8859886	1.612399\\
0.8860886	1.60667\\
0.8861886	1.599122\\
0.8862886	1.593206\\
0.8863886	1.593614\\
0.8864886	1.587861\\
0.8865887	1.586465\\
0.8866887	1.589285\\
0.8867887	1.58837\\
0.8868887	1.589099\\
0.8869887	1.589535\\
0.8870887	1.588722\\
0.8871887	1.585588\\
0.8872887	1.578387\\
0.8873887	1.57433\\
0.8874887	1.567882\\
0.8875888	1.560081\\
0.8876888	1.560421\\
0.8877888	1.559398\\
0.8878888	1.553131\\
0.8879888	1.541302\\
0.8880888	1.540159\\
0.8881888	1.539265\\
0.8882888	1.541722\\
0.8883888	1.542628\\
0.8884888	1.540518\\
0.8885889	1.535673\\
0.8886889	1.533156\\
0.8887889	1.525621\\
0.8888889	1.525911\\
0.8889889	1.525045\\
0.8890889	1.519704\\
0.8891889	1.51523\\
0.8892889	1.513014\\
0.8893889	1.511925\\
0.8894889	1.509991\\
0.889589	1.508138\\
0.889689	1.505904\\
0.889789	1.502484\\
0.889889	1.495429\\
0.889989	1.496269\\
0.890089	1.493355\\
0.890189	1.492089\\
0.890289	1.491491\\
0.890389	1.49073\\
0.890489	1.48847\\
0.8905891	1.485113\\
0.8906891	1.479569\\
0.8907891	1.474569\\
0.8908891	1.471505\\
0.8909891	1.472703\\
0.8910891	1.473286\\
0.8911891	1.474613\\
0.8912891	1.46939\\
0.8913891	1.468292\\
0.8914891	1.469368\\
0.8915892	1.465388\\
0.8916892	1.460502\\
0.8917892	1.458115\\
0.8918892	1.453444\\
0.8919892	1.44241\\
0.8920892	1.442992\\
0.8921892	1.446967\\
0.8922892	1.445479\\
0.8923892	1.434092\\
0.8924892	1.427135\\
0.8925893	1.426686\\
0.8926893	1.425794\\
0.8927893	1.422154\\
0.8928893	1.41917\\
0.8929893	1.417936\\
0.8930893	1.414363\\
0.8931893	1.41073\\
0.8932893	1.407481\\
0.8933893	1.410388\\
0.8934893	1.407095\\
0.8935894	1.404567\\
0.8936894	1.401161\\
0.8937894	1.39762\\
0.8938894	1.395747\\
0.8939894	1.389108\\
0.8940894	1.384511\\
0.8941894	1.385356\\
0.8942894	1.385274\\
0.8943894	1.376639\\
0.8944894	1.367855\\
0.8945895	1.368286\\
0.8946895	1.365335\\
0.8947895	1.35981\\
0.8948895	1.356733\\
0.8949895	1.355559\\
0.8950895	1.3511\\
0.8951895	1.343391\\
0.8952895	1.335283\\
0.8953895	1.332139\\
0.8954895	1.330325\\
0.8955896	1.327585\\
0.8956896	1.323125\\
0.8957896	1.319955\\
0.8958896	1.314535\\
0.8959896	1.310078\\
0.8960896	1.309509\\
0.8961896	1.305247\\
0.8962896	1.303639\\
0.8963896	1.300436\\
0.8964896	1.29597\\
0.8965897	1.291464\\
0.8966897	1.293713\\
0.8967897	1.290788\\
0.8968897	1.287791\\
0.8969897	1.285375\\
0.8970897	1.284398\\
0.8971897	1.277984\\
0.8972897	1.267439\\
0.8973897	1.263099\\
0.8974897	1.26174\\
0.8975898	1.256862\\
0.8976898	1.251724\\
0.8977898	1.24865\\
0.8978898	1.242536\\
0.8979898	1.240578\\
0.8980898	1.238516\\
0.8981898	1.23367\\
0.8982898	1.236753\\
0.8983898	1.23624\\
0.8984898	1.237433\\
0.8985899	1.234912\\
0.8986899	1.230843\\
0.8987899	1.227476\\
0.8988899	1.220383\\
0.8989899	1.216013\\
0.8990899	1.208903\\
0.8991899	1.207426\\
0.8992899	1.205099\\
0.8993899	1.200968\\
0.8994899	1.194863\\
0.89959	1.195005\\
0.89969	1.197629\\
0.89979	1.197801\\
0.89989	1.198098\\
0.89999	1.196465\\
0.90009	1.193345\\
0.90019	1.192487\\
0.90029	1.189705\\
0.90039	1.188076\\
0.90049	1.185967\\
0.9005901	1.187357\\
0.9006901	1.186517\\
0.9007901	1.184437\\
0.9008901	1.182052\\
0.9009901	1.175888\\
0.9010901	1.174679\\
0.9011901	1.168025\\
0.9012901	1.164134\\
0.9013901	1.161318\\
0.9014901	1.158636\\
0.9015902	1.155852\\
0.9016902	1.154001\\
0.9017902	1.155741\\
0.9018902	1.156954\\
0.9019902	1.157711\\
0.9020902	1.153711\\
0.9021902	1.149799\\
0.9022902	1.14968\\
0.9023902	1.147171\\
0.9024902	1.146877\\
0.9025903	1.142783\\
0.9026903	1.14316\\
0.9027903	1.142778\\
0.9028903	1.140303\\
0.9029903	1.13954\\
0.9030903	1.135704\\
0.9031903	1.132121\\
0.9032903	1.131426\\
0.9033903	1.134469\\
0.9034903	1.132132\\
0.9035904	1.127568\\
0.9036904	1.127514\\
0.9037904	1.127653\\
0.9038904	1.123065\\
0.9039904	1.114085\\
0.9040904	1.107296\\
0.9041904	1.106514\\
0.9042904	1.105783\\
0.9043904	1.098819\\
0.9044904	1.095903\\
0.9045905	1.090377\\
0.9046905	1.091064\\
0.9047905	1.091333\\
0.9048905	1.090808\\
0.9049905	1.091595\\
0.9050905	1.089178\\
0.9051905	1.085263\\
0.9052905	1.080193\\
0.9053905	1.075966\\
0.9054905	1.0692\\
0.9055906	1.067221\\
0.9056906	1.067276\\
0.9057906	1.062643\\
0.9058906	1.060491\\
0.9059906	1.059324\\
0.9060906	1.057284\\
0.9061906	1.054738\\
0.9062906	1.050237\\
0.9063906	1.045532\\
0.9064906	1.0434\\
0.9065907	1.042503\\
0.9066907	1.032956\\
0.9067907	1.029719\\
0.9068907	1.026323\\
0.9069907	1.021034\\
0.9070907	1.019883\\
0.9071907	1.016911\\
0.9072907	1.009954\\
0.9073907	1.010011\\
0.9074907	1.005156\\
0.9075908	1.004816\\
0.9076908	1.000245\\
0.9077908	0.9945903\\
0.9078908	0.9912406\\
0.9079908	0.9898854\\
0.9080908	0.9878864\\
0.9081908	0.9867836\\
0.9082908	0.9807536\\
0.9083908	0.9755691\\
0.9084908	0.973499\\
0.9085909	0.9688547\\
0.9086909	0.9624268\\
0.9087909	0.9595387\\
0.9088909	0.9609088\\
0.9089909	0.9595458\\
0.9090909	0.9577978\\
0.9091909	0.9531198\\
0.9092909	0.9524953\\
0.9093909	0.948349\\
0.9094909	0.9472844\\
0.909591	0.94462\\
0.909691	0.9435143\\
0.909791	0.9405657\\
0.909891	0.9374142\\
0.909991	0.9364592\\
0.910091	0.9352735\\
0.910191	0.9318819\\
0.910291	0.9331659\\
0.910391	0.92722\\
0.910491	0.925131\\
0.9105911	0.9219627\\
0.9106911	0.9194646\\
0.9107911	0.9117012\\
0.9108911	0.9077448\\
0.9109911	0.9067853\\
0.9110911	0.9027292\\
0.9111911	0.9020051\\
0.9112911	0.9003534\\
0.9113911	0.9013911\\
0.9114911	0.8998968\\
0.9115912	0.9040879\\
0.9116912	0.9061214\\
0.9117912	0.9065747\\
0.9118912	0.9044433\\
0.9119912	0.9017262\\
0.9120912	0.9031054\\
0.9121912	0.9025872\\
0.9122912	0.898853\\
0.9123912	0.8967449\\
0.9124912	0.8922316\\
0.9125913	0.8920865\\
0.9126913	0.8896661\\
0.9127913	0.8847014\\
0.9128913	0.8817354\\
0.9129913	0.8859034\\
0.9130913	0.8851776\\
0.9131913	0.8839448\\
0.9132913	0.8800861\\
0.9133913	0.8780604\\
0.9134913	0.8811145\\
0.9135914	0.8798129\\
0.9136914	0.8778526\\
0.9137914	0.8773712\\
0.9138914	0.8762835\\
0.9139914	0.8762829\\
0.9140914	0.8732021\\
0.9141914	0.8701059\\
0.9142914	0.8631304\\
0.9143914	0.8607146\\
0.9144914	0.8602493\\
0.9145915	0.8595734\\
0.9146915	0.8625142\\
0.9147915	0.8625064\\
0.9148915	0.8616707\\
0.9149915	0.8604549\\
0.9150915	0.860195\\
0.9151915	0.8618426\\
0.9152915	0.8605466\\
0.9153915	0.853913\\
0.9154915	0.8444105\\
0.9155916	0.8432818\\
0.9156916	0.8414161\\
0.9157916	0.8428642\\
0.9158916	0.8382978\\
0.9159916	0.834535\\
0.9160916	0.8367003\\
0.9161916	0.8346081\\
0.9162916	0.834699\\
0.9163916	0.8285285\\
0.9164916	0.8231619\\
0.9165917	0.8224158\\
0.9166917	0.8255011\\
0.9167917	0.8195849\\
0.9168917	0.8158392\\
0.9169917	0.8110457\\
0.9170917	0.807081\\
0.9171917	0.8030814\\
0.9172917	0.8025199\\
0.9173917	0.8023338\\
0.9174917	0.7960923\\
0.9175918	0.7926747\\
0.9176918	0.787444\\
0.9177918	0.7850118\\
0.9178918	0.7840566\\
0.9179918	0.7813161\\
0.9180918	0.7778155\\
0.9181918	0.7738422\\
0.9182918	0.7714893\\
0.9183918	0.7710177\\
0.9184918	0.7688755\\
0.9185919	0.7661189\\
0.9186919	0.7614792\\
0.9187919	0.7631521\\
0.9188919	0.7608378\\
0.9189919	0.755225\\
0.9190919	0.7521316\\
0.9191919	0.7485428\\
0.9192919	0.7448688\\
0.9193919	0.7404739\\
0.9194919	0.737249\\
0.919592	0.7331814\\
0.919692	0.730196\\
0.919792	0.7288401\\
0.919892	0.7260848\\
0.919992	0.7203431\\
0.920092	0.7191087\\
0.920192	0.715072\\
0.920292	0.7110085\\
0.920392	0.7083879\\
0.920492	0.7054636\\
0.9205921	0.7027121\\
0.9206921	0.7025542\\
0.9207921	0.6987854\\
0.9208921	0.6978227\\
0.9209921	0.6955385\\
0.9210921	0.6990048\\
0.9211921	0.7004395\\
0.9212921	0.7004649\\
0.9213921	0.6989538\\
0.9214921	0.6967444\\
0.9215922	0.6910843\\
0.9216922	0.6883035\\
0.9217922	0.6869932\\
0.9218922	0.6840687\\
0.9219922	0.682182\\
0.9220922	0.6782555\\
0.9221922	0.6763654\\
0.9222922	0.6745844\\
0.9223922	0.6719879\\
0.9224922	0.6740088\\
0.9225923	0.6752104\\
0.9226923	0.6719914\\
0.9227923	0.6711129\\
0.9228923	0.6699702\\
0.9229923	0.6695076\\
0.9230923	0.6674304\\
0.9231923	0.6692125\\
0.9232923	0.6656391\\
0.9233923	0.6661772\\
0.9234923	0.666625\\
0.9235924	0.6684744\\
0.9236924	0.6655045\\
0.9237924	0.6636805\\
0.9238924	0.6572737\\
0.9239924	0.6606526\\
0.9240924	0.6594688\\
0.9241924	0.6628893\\
0.9242924	0.6623614\\
0.9243924	0.6624084\\
0.9244924	0.6609757\\
0.9245925	0.6582273\\
0.9246925	0.6520768\\
0.9247925	0.6564745\\
0.9248925	0.6544435\\
0.9249925	0.659531\\
0.9250925	0.6588029\\
0.9251925	0.6593138\\
0.9252925	0.6571462\\
0.9253925	0.6578084\\
0.9254925	0.6583894\\
0.9255926	0.6535852\\
0.9256926	0.6530781\\
0.9257926	0.6495955\\
0.9258926	0.6454609\\
0.9259926	0.6430216\\
0.9260926	0.6435603\\
0.9261926	0.6417555\\
0.9262926	0.6427309\\
0.9263926	0.6401473\\
0.9264926	0.6400162\\
0.9265927	0.6382794\\
0.9266927	0.6365296\\
0.9267927	0.635958\\
0.9268927	0.6370497\\
0.9269927	0.6341302\\
0.9270927	0.628104\\
0.9271927	0.6273794\\
0.9272927	0.6266595\\
0.9273927	0.6287023\\
0.9274927	0.6252952\\
0.9275928	0.6242825\\
0.9276928	0.6207759\\
0.9277928	0.6192804\\
0.9278928	0.6156726\\
0.9279928	0.6143126\\
0.9280928	0.6132907\\
0.9281928	0.612046\\
0.9282928	0.6127267\\
0.9283928	0.6075273\\
0.9284928	0.6068595\\
0.9285929	0.6042836\\
0.9286929	0.6014484\\
0.9287929	0.5955554\\
0.9288929	0.5907412\\
0.9289929	0.5850719\\
0.9290929	0.5806338\\
0.9291929	0.5779078\\
0.9292929	0.5729734\\
0.9293929	0.5742192\\
0.9294929	0.57121\\
0.929593	0.5690388\\
0.929693	0.5642491\\
0.929793	0.5627527\\
0.929893	0.5623196\\
0.929993	0.5615407\\
0.930093	0.5620702\\
0.930193	0.5605443\\
0.930293	0.5565652\\
0.930393	0.5510691\\
0.930493	0.5489211\\
0.9305931	0.5457418\\
0.9306931	0.5443257\\
0.9307931	0.5438068\\
0.9308931	0.5399315\\
0.9309931	0.5341045\\
0.9310931	0.5287837\\
0.9311931	0.5281897\\
0.9312931	0.5248616\\
0.9313931	0.5244337\\
0.9314931	0.5233698\\
0.9315932	0.5184968\\
0.9316932	0.5202877\\
0.9317932	0.5149979\\
0.9318932	0.512671\\
0.9319932	0.5088422\\
0.9320932	0.5070566\\
0.9321932	0.5060338\\
0.9322932	0.5033908\\
0.9323932	0.5017204\\
0.9324932	0.4975911\\
0.9325933	0.498514\\
0.9326933	0.5015552\\
0.9327933	0.4993848\\
0.9328933	0.4989799\\
0.9329933	0.497389\\
0.9330933	0.4952613\\
0.9331933	0.4936974\\
0.9332933	0.4890258\\
0.9333933	0.4904566\\
0.9334933	0.4867633\\
0.9335934	0.4853237\\
0.9336934	0.4881743\\
0.9337934	0.4857009\\
0.9338934	0.4851034\\
0.9339934	0.4838583\\
0.9340934	0.4871672\\
0.9341934	0.4871813\\
0.9342934	0.4854482\\
0.9343934	0.4878924\\
0.9344934	0.4862761\\
0.9345935	0.4868361\\
0.9346935	0.4842048\\
0.9347935	0.4843374\\
0.9348935	0.4855246\\
0.9349935	0.4840251\\
0.9350935	0.4834753\\
0.9351935	0.4813068\\
0.9352935	0.4802627\\
0.9353935	0.4804027\\
0.9354935	0.4782085\\
0.9355936	0.4761696\\
0.9356936	0.4771457\\
0.9357936	0.4752974\\
0.9358936	0.4785598\\
0.9359936	0.4759882\\
0.9360936	0.4786191\\
0.9361936	0.4795808\\
0.9362936	0.4787256\\
0.9363936	0.4854477\\
0.9364936	0.4848735\\
0.9365937	0.4854967\\
0.9366937	0.4849902\\
0.9367937	0.4808094\\
0.9368937	0.4802721\\
0.9369937	0.4780952\\
0.9370937	0.477346\\
0.9371937	0.4769411\\
0.9372937	0.4766874\\
0.9373937	0.4782207\\
0.9374937	0.4799293\\
0.9375938	0.4765222\\
0.9376938	0.4774359\\
0.9377938	0.4756087\\
0.9378938	0.472834\\
0.9379938	0.4716955\\
0.9380938	0.4696217\\
0.9381938	0.4701853\\
0.9382938	0.4658481\\
0.9383938	0.4629425\\
0.9384938	0.461478\\
0.9385939	0.4597336\\
0.9386939	0.4581291\\
0.9387939	0.4579399\\
0.9388939	0.4577537\\
0.9389939	0.4556971\\
0.9390939	0.458208\\
0.9391939	0.4545267\\
0.9392939	0.4533385\\
0.9393939	0.4525304\\
0.9394939	0.4490049\\
0.939594	0.4482888\\
0.939694	0.4465179\\
0.939794	0.4453875\\
0.939894	0.44286\\
0.939994	0.439315\\
0.940094	0.4342113\\
0.940194	0.4317932\\
0.940294	0.4311459\\
0.940394	0.4309788\\
0.940494	0.4283165\\
0.9405941	0.4208315\\
0.9406941	0.4210788\\
0.9407941	0.4203039\\
0.9408941	0.4138451\\
0.9409941	0.4081136\\
0.9410941	0.4069022\\
0.9411941	0.4024185\\
0.9412941	0.4026566\\
0.9413941	0.4039594\\
0.9414941	0.4017518\\
0.9415942	0.4006024\\
0.9416942	0.3976865\\
0.9417942	0.3952857\\
0.9418942	0.3908739\\
0.9419942	0.387136\\
0.9420942	0.3850362\\
0.9421942	0.3823094\\
0.9422942	0.3794415\\
0.9423942	0.3749022\\
0.9424942	0.373489\\
0.9425943	0.3704665\\
0.9426943	0.3668873\\
0.9427943	0.3681155\\
0.9428943	0.3644839\\
0.9429943	0.3658893\\
0.9430943	0.3651883\\
0.9431943	0.36368\\
0.9432943	0.3620988\\
0.9433943	0.3624226\\
0.9434943	0.3586696\\
0.9435944	0.3574526\\
0.9436944	0.3551845\\
0.9437944	0.3522959\\
0.9438944	0.3514212\\
0.9439944	0.3506852\\
0.9440944	0.3495434\\
0.9441944	0.3456356\\
0.9442944	0.3462289\\
0.9443944	0.343873\\
0.9444944	0.3400614\\
0.9445945	0.3417397\\
0.9446945	0.3412286\\
0.9447945	0.3418411\\
0.9448945	0.3404805\\
0.9449945	0.3397698\\
0.9450945	0.3374538\\
0.9451945	0.3376751\\
0.9452945	0.3364405\\
0.9453945	0.3360859\\
0.9454945	0.3372973\\
0.9455946	0.3395803\\
0.9456946	0.3403736\\
0.9457946	0.3401955\\
0.9458946	0.343481\\
0.9459946	0.34277\\
0.9460946	0.344533\\
0.9461946	0.3447063\\
0.9462946	0.3439077\\
0.9463946	0.3429254\\
0.9464946	0.3424712\\
0.9465947	0.3430086\\
0.9466947	0.3407054\\
0.9467947	0.3441766\\
0.9468947	0.3443989\\
0.9469947	0.3408149\\
0.9470947	0.3398746\\
0.9471947	0.3385337\\
0.9472947	0.3402778\\
0.9473947	0.3415908\\
0.9474947	0.3416195\\
0.9475948	0.3440878\\
0.9476948	0.3447154\\
0.9477948	0.3461921\\
0.9478948	0.3486087\\
0.9479948	0.3461867\\
0.9480948	0.3465007\\
0.9481948	0.3462456\\
0.9482948	0.3451811\\
0.9483948	0.3455872\\
0.9484948	0.3448051\\
0.9485949	0.3452147\\
0.9486949	0.3451354\\
0.9487949	0.3458614\\
0.9488949	0.3469765\\
0.9489949	0.3462784\\
0.9490949	0.3463053\\
0.9491949	0.3464262\\
0.9492949	0.3462018\\
0.9493949	0.3437038\\
0.9494949	0.3420335\\
0.949595	0.3429781\\
0.949695	0.3427204\\
0.949795	0.3403293\\
0.949895	0.3383276\\
0.949995	0.3396988\\
0.950095	0.3377848\\
0.950195	0.3342906\\
0.950295	0.333073\\
0.950395	0.3329109\\
0.950495	0.3295497\\
0.9505951	0.3268124\\
0.9506951	0.3268078\\
0.9507951	0.3244106\\
0.9508951	0.3238101\\
0.9509951	0.3252751\\
0.9510951	0.323995\\
0.9511951	0.3194223\\
0.9512951	0.3163753\\
0.9513951	0.3147915\\
0.9514951	0.312688\\
0.9515952	0.3079559\\
0.9516952	0.3055872\\
0.9517952	0.3063795\\
0.9518952	0.3042056\\
0.9519952	0.3015702\\
0.9520952	0.2981272\\
0.9521952	0.2950479\\
0.9522952	0.2932661\\
0.9523952	0.2907167\\
0.9524952	0.2904458\\
0.9525953	0.2913603\\
0.9526953	0.2883635\\
0.9527953	0.2843196\\
0.9528953	0.2836621\\
0.9529953	0.2830705\\
0.9530953	0.2798711\\
0.9531953	0.2754049\\
0.9532953	0.2729127\\
0.9533953	0.271261\\
0.9534953	0.2651062\\
0.9535954	0.2600804\\
0.9536954	0.2578859\\
0.9537954	0.2554107\\
0.9538954	0.2533769\\
0.9539954	0.2503741\\
0.9540954	0.2493731\\
0.9541954	0.2503319\\
0.9542954	0.2490191\\
0.9543954	0.2465965\\
0.9544954	0.2448809\\
0.9545955	0.245686\\
0.9546955	0.2454803\\
0.9547955	0.2425237\\
0.9548955	0.2397343\\
0.9549955	0.2417408\\
0.9550955	0.2408434\\
0.9551955	0.2387447\\
0.9552955	0.2363666\\
0.9553955	0.2368603\\
0.9554955	0.2362989\\
0.9555956	0.2346835\\
0.9556956	0.2330681\\
0.9557956	0.2314616\\
0.9558956	0.2315889\\
0.9559956	0.2288953\\
0.9560956	0.2274045\\
0.9561956	0.2251701\\
0.9562956	0.2277858\\
0.9563956	0.228458\\
0.9564956	0.2293095\\
0.9565957	0.228594\\
0.9566957	0.2295311\\
0.9567957	0.228874\\
0.9568957	0.2287519\\
0.9569957	0.228696\\
0.9570957	0.2287617\\
0.9571957	0.2309013\\
0.9572957	0.2311142\\
0.9573957	0.2291113\\
0.9574957	0.2304871\\
0.9575958	0.231586\\
0.9576958	0.2328919\\
0.9577958	0.2353585\\
0.9578958	0.2355978\\
0.9579958	0.2357219\\
0.9580958	0.2386371\\
0.9581958	0.2404175\\
0.9582958	0.2398432\\
0.9583958	0.2391968\\
0.9584958	0.2394328\\
0.9585959	0.2400968\\
0.9586959	0.2412187\\
0.9587959	0.2412456\\
0.9588959	0.2397077\\
0.9589959	0.2413919\\
0.9590959	0.2430388\\
0.9591959	0.2472134\\
0.9592959	0.2463478\\
0.9593959	0.2464438\\
0.9594959	0.2489333\\
0.959596	0.2494933\\
0.959696	0.2486058\\
0.959796	0.2475355\\
0.959896	0.2485614\\
0.959996	0.2480175\\
0.960096	0.2470782\\
0.960196	0.2470833\\
0.960296	0.2463797\\
0.960396	0.24376\\
0.960496	0.2446979\\
0.9605961	0.2470396\\
0.9606961	0.2477621\\
0.9607961	0.2492803\\
0.9608961	0.2478224\\
0.9609961	0.2482253\\
0.9610961	0.2493452\\
0.9611961	0.2495631\\
0.9612961	0.2479361\\
0.9613961	0.2457635\\
0.9614961	0.2448755\\
0.9615962	0.2435771\\
0.9616962	0.2436303\\
0.9617962	0.2429555\\
0.9618962	0.2391529\\
0.9619962	0.2376332\\
0.9620962	0.2380938\\
0.9621962	0.2388104\\
0.9622962	0.2355296\\
0.9623962	0.2330147\\
0.9624962	0.2311213\\
0.9625963	0.2270384\\
0.9626963	0.2255186\\
0.9627963	0.2245333\\
0.9628963	0.2240249\\
0.9629963	0.2205359\\
0.9630963	0.2179531\\
0.9631963	0.2132496\\
0.9632963	0.2112859\\
0.9633963	0.2107229\\
0.9634963	0.2079908\\
0.9635964	0.2030493\\
0.9636964	0.2019689\\
0.9637964	0.2027998\\
0.9638964	0.2020975\\
0.9639964	0.2016117\\
0.9640964	0.1999805\\
0.9641964	0.1982395\\
0.9642964	0.1954228\\
0.9643964	0.1936725\\
0.9644964	0.1915057\\
0.9645965	0.187928\\
0.9646965	0.185312\\
0.9647965	0.1833995\\
0.9648965	0.1791527\\
0.9649965	0.1750512\\
0.9650965	0.1715494\\
0.9651965	0.1714251\\
0.9652965	0.1685794\\
0.9653965	0.1666872\\
0.9654965	0.1640971\\
0.9655966	0.1623572\\
0.9656966	0.1607545\\
0.9657966	0.1600102\\
0.9658966	0.1609432\\
0.9659966	0.1610811\\
0.9660966	0.1576532\\
0.9661966	0.1554837\\
0.9662966	0.1547706\\
0.9663966	0.1532205\\
0.9664966	0.1526304\\
0.9665967	0.1519413\\
0.9666967	0.1510877\\
0.9667967	0.1485323\\
0.9668967	0.1473617\\
0.9669967	0.1465891\\
0.9670967	0.1464349\\
0.9671967	0.1463428\\
0.9672967	0.1472373\\
0.9673967	0.1481721\\
0.9674967	0.1465469\\
0.9675968	0.1448816\\
0.9676968	0.1458921\\
0.9677968	0.1467011\\
0.9678968	0.1480129\\
0.9679968	0.1489546\\
0.9680968	0.1500836\\
0.9681968	0.148998\\
0.9682968	0.1491973\\
0.9683968	0.1487751\\
0.9684968	0.149701\\
0.9685969	0.1504195\\
0.9686969	0.1509242\\
0.9687969	0.1511326\\
0.9688969	0.1530967\\
0.9689969	0.1525522\\
0.9690969	0.1520664\\
0.9691969	0.1534442\\
0.9692969	0.1552586\\
0.9693969	0.1568414\\
0.9694969	0.1582413\\
0.969597	0.1600486\\
0.969697	0.1612742\\
0.969797	0.1628609\\
0.969897	0.1618468\\
0.969997	0.1637276\\
0.970097	0.1659045\\
0.970197	0.1675823\\
0.970297	0.1692643\\
0.970397	0.1711105\\
0.970497	0.1738588\\
0.9705971	0.1746105\\
0.9706971	0.1737982\\
0.9707971	0.1746653\\
0.9708971	0.1746907\\
0.9709971	0.1762525\\
0.9710971	0.1772781\\
0.9711971	0.1775149\\
0.9712971	0.180242\\
0.9713971	0.1807373\\
0.9714971	0.1798766\\
0.9715972	0.180247\\
0.9716972	0.179226\\
0.9717972	0.1793241\\
0.9718972	0.1781555\\
0.9719972	0.1766126\\
0.9720972	0.1782937\\
0.9721972	0.1800448\\
0.9722972	0.178543\\
0.9723972	0.1781261\\
0.9724972	0.1786646\\
0.9725973	0.1788297\\
0.9726973	0.1788935\\
0.9727973	0.1785005\\
0.9728973	0.1780998\\
0.9729973	0.1776745\\
0.9730973	0.1776183\\
0.9731973	0.1772348\\
0.9732973	0.1779841\\
0.9733973	0.1772374\\
0.9734973	0.1753225\\
0.9735974	0.1738702\\
0.9736974	0.1715273\\
0.9737974	0.1676834\\
0.9738974	0.1644419\\
0.9739974	0.1629188\\
0.9740974	0.1629137\\
0.9741974	0.1607435\\
0.9742974	0.1585257\\
0.9743974	0.1591427\\
0.9744974	0.1569218\\
0.9745975	0.1545549\\
0.9746975	0.1520486\\
0.9747975	0.1491659\\
0.9748975	0.1462168\\
0.9749975	0.1444318\\
0.9750975	0.1412816\\
0.9751975	0.1380387\\
0.9752975	0.1364628\\
0.9753975	0.1344342\\
0.9754975	0.133525\\
0.9755976	0.1318575\\
0.9756976	0.1302281\\
0.9757976	0.1279328\\
0.9758976	0.126549\\
0.9759976	0.1252211\\
0.9760976	0.1223971\\
0.9761976	0.1197009\\
0.9762976	0.1163382\\
0.9763976	0.112496\\
0.9764976	0.1099891\\
0.9765977	0.1080144\\
0.9766977	0.1065687\\
0.9767977	0.1040058\\
0.9768977	0.1030872\\
0.9769977	0.1024929\\
0.9770977	0.1016184\\
0.9771977	0.1006989\\
0.9772977	0.09969075\\
0.9773977	0.09908296\\
0.9774977	0.09729033\\
0.9775978	0.09595868\\
0.9776978	0.09463953\\
0.9777978	0.09270602\\
0.9778978	0.09022318\\
0.9779978	0.08884746\\
0.9780978	0.08756641\\
0.9781978	0.08662138\\
0.9782978	0.08564624\\
0.9783978	0.08455533\\
0.9784978	0.08440477\\
0.9785979	0.08395598\\
0.9786979	0.08489617\\
0.9787979	0.0846849\\
0.9788979	0.08585329\\
0.9789979	0.08628419\\
0.9790979	0.08656906\\
0.9791979	0.08813179\\
0.9792979	0.08928918\\
0.9793979	0.08993976\\
0.9794979	0.08935775\\
0.979598	0.09085021\\
0.979698	0.09181444\\
0.979798	0.09294306\\
0.979898	0.0932424\\
0.979998	0.09409501\\
0.980098	0.09493776\\
0.980198	0.09581242\\
0.980298	0.09549563\\
0.980398	0.09654003\\
0.980498	0.09844028\\
0.9805981	0.1004191\\
0.9806981	0.1011535\\
0.9807981	0.1021603\\
0.9808981	0.1045116\\
0.9809981	0.1046854\\
0.9810981	0.1065464\\
0.9811981	0.1086026\\
0.9812981	0.1103606\\
0.9813981	0.1109999\\
0.9814981	0.1124229\\
0.9815982	0.1145485\\
0.9816982	0.1149848\\
0.9817982	0.1173526\\
0.9818982	0.1196865\\
0.9819982	0.1218392\\
0.9820982	0.1229184\\
0.9821982	0.1245977\\
0.9822982	0.1263099\\
0.9823982	0.1282369\\
0.9824982	0.1293231\\
0.9825983	0.1298028\\
0.9826983	0.1305781\\
0.9827983	0.1313713\\
0.9828983	0.1319097\\
0.9829983	0.1331825\\
0.9830983	0.1340347\\
0.9831983	0.1343423\\
0.9832983	0.1349217\\
0.9833983	0.1348261\\
0.9834983	0.1350205\\
0.9835984	0.1343863\\
0.9836984	0.1343211\\
0.9837984	0.1338821\\
0.9838984	0.1344695\\
0.9839984	0.1357064\\
0.9840984	0.1361315\\
0.9841984	0.1353406\\
0.9842984	0.1352257\\
0.9843984	0.1346178\\
0.9844984	0.1338388\\
0.9845985	0.1348415\\
0.9846985	0.1348647\\
0.9847985	0.1334617\\
0.9848985	0.1324907\\
0.9849985	0.1307586\\
0.9850985	0.1306021\\
0.9851985	0.1297994\\
0.9852985	0.1288487\\
0.9853985	0.1279404\\
0.9854985	0.1279756\\
0.9855986	0.1287686\\
0.9856986	0.1287174\\
0.9857986	0.1277705\\
0.9858986	0.1263297\\
0.9859986	0.1258216\\
0.9860986	0.1258809\\
0.9861986	0.1260837\\
0.9862986	0.1259342\\
0.9863986	0.1268857\\
0.9864986	0.1272023\\
0.9865987	0.1274872\\
0.9866987	0.1274662\\
0.9867987	0.1278216\\
0.9868987	0.1279911\\
0.9869987	0.1281659\\
0.9870987	0.1270913\\
0.9871987	0.1280616\\
0.9872987	0.1288196\\
0.9873987	0.1287312\\
0.9874987	0.1300582\\
0.9875988	0.1292708\\
0.9876988	0.1296362\\
0.9877988	0.1294739\\
0.9878988	0.1295512\\
0.9879988	0.1291705\\
0.9880988	0.1290139\\
0.9881988	0.1297047\\
0.9882988	0.1306901\\
0.9883988	0.1303628\\
0.9884988	0.1307456\\
0.9885989	0.1301117\\
0.9886989	0.1291086\\
0.9887989	0.1274045\\
0.9888989	0.1271537\\
0.9889989	0.1268138\\
0.9890989	0.1253566\\
0.9891989	0.1237835\\
0.9892989	0.1223863\\
0.9893989	0.1217012\\
0.9894989	0.1210426\\
0.989599	0.1215262\\
0.989699	0.121196\\
0.989799	0.1210106\\
0.989899	0.1197669\\
0.989999	0.1194221\\
0.990099	0.119079\\
0.990199	0.1183595\\
0.990299	0.1184255\\
0.990399	0.1194306\\
0.990499	0.1203822\\
0.9905991	0.1207705\\
0.9906991	0.1216113\\
0.9907991	0.1204397\\
0.9908991	0.1211595\\
0.9909991	0.1205558\\
0.9910991	0.1194763\\
0.9911991	0.1174895\\
0.9912991	0.1156342\\
0.9913991	0.1146506\\
0.9914991	0.1137386\\
0.9915992	0.1117798\\
0.9916992	0.1099285\\
0.9917992	0.1097512\\
0.9918992	0.1085644\\
0.9919992	0.1081617\\
0.9920992	0.1083895\\
0.9921992	0.1080801\\
0.9922992	0.1082185\\
0.9923992	0.1077733\\
0.9924992	0.1065615\\
0.9925993	0.1055003\\
0.9926993	0.1040426\\
0.9927993	0.1017291\\
0.9928993	0.100786\\
0.9929993	0.09929942\\
0.9930993	0.09695137\\
0.9931993	0.09542868\\
0.9932993	0.09301954\\
0.9933993	0.09157389\\
0.9934993	0.09140489\\
0.9935994	0.08968543\\
0.9936994	0.08902935\\
0.9937994	0.08941585\\
0.9938994	0.08840217\\
0.9939994	0.08766632\\
0.9940994	0.08629635\\
0.9941994	0.0846184\\
0.9942994	0.08292933\\
0.9943994	0.08201693\\
0.9944994	0.07928949\\
0.9945995	0.07756109\\
0.9946995	0.07588494\\
0.9947995	0.07528941\\
0.9948995	0.07473441\\
0.9949995	0.07358542\\
0.9950995	0.07206776\\
0.9951995	0.07134671\\
0.9952995	0.07035614\\
0.9953995	0.06976495\\
0.9954995	0.06841557\\
0.9955996	0.06707341\\
0.9956996	0.06548726\\
0.9957996	0.06462551\\
0.9958996	0.06400444\\
0.9959996	0.06210639\\
0.9960996	0.06235607\\
0.9961996	0.06226705\\
0.9962996	0.06116866\\
0.9963996	0.05958042\\
0.9964996	0.05876268\\
0.9965997	0.05701796\\
0.9966997	0.05505632\\
0.9967997	0.05283456\\
0.9968997	0.05149237\\
0.9969997	0.04978405\\
0.9970997	0.04870378\\
0.9971997	0.04738763\\
0.9972997	0.04638117\\
0.9973997	0.04488127\\
0.9974997	0.04478742\\
0.9975998	0.04388335\\
0.9976998	0.04177614\\
0.9977998	0.04079268\\
0.9978998	0.03926787\\
0.9979998	0.03859873\\
0.9980998	0.03818853\\
0.9981998	0.03789537\\
0.9982998	0.0378193\\
0.9983998	0.03747809\\
0.9984998	0.03697124\\
0.9985999	0.03683473\\
0.9986999	0.03554119\\
0.9987999	0.03535857\\
0.9988999	0.03527208\\
0.9989999	0.03523584\\
0.9990999	0.03512575\\
0.9991999	0.0355604\\
0.9992999	0.03528455\\
0.9993999	0.03552341\\
0.9994999	0.03455031\\
0.9996	0.03394304\\
0.9997	0.03377166\\
0.9998	0.03290953\\
0.9999	0.03312837\\
1	0.03294526\\
};
\addlegendentry{$\text{|}\psi{}_\text{n}\text{(x,t)|}^\text{2}$};

\addplot [color=mycolor2,solid,forget plot]
  table[row sep=crcr]{%
0	-0.1784732\\
0.00010001	-0.1791198\\
0.00020002	-0.1773611\\
0.00030003	-0.1750989\\
0.00040004	-0.1740345\\
0.00050005	-0.1731944\\
0.00060006	-0.1730486\\
0.00070007	-0.1737014\\
0.00080008	-0.1745322\\
0.00090009	-0.1743524\\
0.0010001	-0.1734143\\
0.00110011	-0.1718576\\
0.00120012	-0.17024\\
0.00130013	-0.1700123\\
0.00140014	-0.1687604\\
0.00150015	-0.1686385\\
0.00160016	-0.1677291\\
0.00170017	-0.1667247\\
0.00180018	-0.1633206\\
0.00190019	-0.1604784\\
0.0020002	-0.1572148\\
0.00210021	-0.1556874\\
0.00220022	-0.154226\\
0.00230023	-0.1512206\\
0.00240024	-0.1482253\\
0.00250025	-0.1435527\\
0.00260026	-0.1406803\\
0.00270027	-0.1380398\\
0.00280028	-0.1364835\\
0.00290029	-0.1345096\\
0.0030003	-0.1320232\\
0.00310031	-0.1289661\\
0.00320032	-0.1251674\\
0.00330033	-0.1217024\\
0.00340034	-0.1184506\\
0.00350035	-0.115984\\
0.00360036	-0.1135771\\
0.00370037	-0.1093029\\
0.00380038	-0.1031468\\
0.00390039	-0.09934604\\
0.0040004	-0.09574646\\
0.00410041	-0.09100841\\
0.00420042	-0.08607548\\
0.00430043	-0.08064029\\
0.00440044	-0.07624818\\
0.00450045	-0.07084748\\
0.00460046	-0.0643907\\
0.00470047	-0.05833303\\
0.00480048	-0.05576972\\
0.00490049	-0.05533438\\
0.0050005	-0.05096563\\
0.00510051	-0.04425949\\
0.00520052	-0.04102225\\
0.00530053	-0.03703001\\
0.00540054	-0.03225542\\
0.00550055	-0.02879644\\
0.00560056	-0.02131519\\
0.00570057	-0.01674746\\
0.00580058	-0.01135262\\
0.00590059	-0.008482901\\
0.0060006	-0.005018666\\
0.00610061	-8.431048e-05\\
0.00620062	0.004413624\\
0.00630063	0.01132841\\
0.00640064	0.01591903\\
0.00650065	0.01825227\\
0.00660066	0.02210799\\
0.00670067	0.02489278\\
0.00680068	0.02974965\\
0.00690069	0.03301377\\
0.0070007	0.03617218\\
0.00710071	0.03965353\\
0.00720072	0.04316366\\
0.00730073	0.04812455\\
0.00740074	0.05325768\\
0.00750075	0.0591663\\
0.00760076	0.062457\\
0.00770077	0.06812949\\
0.00780078	0.07364136\\
0.00790079	0.08134486\\
0.0080008	0.08559597\\
0.00810081	0.08816733\\
0.00820082	0.09168981\\
0.00830083	0.09432769\\
0.00840084	0.09593771\\
0.00850085	0.09850347\\
0.00860086	0.1004359\\
0.00870087	0.1014034\\
0.00880088	0.1032666\\
0.00890089	0.1057338\\
0.0090009	0.1094957\\
0.00910091	0.1132227\\
0.00920092	0.1143402\\
0.00930093	0.1185877\\
0.00940094	0.1217079\\
0.00950095	0.1241333\\
0.00960096	0.1253038\\
0.00970097	0.126805\\
0.00980098	0.1290263\\
0.00990099	0.1291122\\
0.010001	0.1312362\\
0.01010101	0.1314821\\
0.01020102	0.1308309\\
0.01030103	0.1317471\\
0.01040104	0.1353964\\
0.01050105	0.1378566\\
0.01060106	0.1398002\\
0.01070107	0.1405616\\
0.01080108	0.1427318\\
0.01090109	0.1442444\\
0.0110011	0.1439499\\
0.01110111	0.1420262\\
0.01120112	0.1419399\\
0.01130113	0.1415191\\
0.01140114	0.1416257\\
0.01150115	0.1410911\\
0.01160116	0.1390072\\
0.01170117	0.1377958\\
0.01180118	0.1376648\\
0.01190119	0.1382389\\
0.0120012	0.1399211\\
0.01210121	0.138015\\
0.01220122	0.1364121\\
0.01230123	0.1382969\\
0.01240124	0.1392624\\
0.01250125	0.1352045\\
0.01260126	0.1331379\\
0.01270127	0.1315859\\
0.01280128	0.132091\\
0.01290129	0.1287466\\
0.0130013	0.1245158\\
0.01310131	0.1229271\\
0.01320132	0.1215301\\
0.01330133	0.1168859\\
0.01340134	0.1140689\\
0.01350135	0.1110201\\
0.01360136	0.1098253\\
0.01370137	0.1091045\\
0.01380138	0.1033697\\
0.01390139	0.1003919\\
0.0140014	0.09815081\\
0.01410141	0.09533814\\
0.01420142	0.09019303\\
0.01430143	0.08768825\\
0.01440144	0.08639717\\
0.01450145	0.08544233\\
0.01460146	0.08337283\\
0.01470147	0.07981117\\
0.01480148	0.07750778\\
0.01490149	0.07629643\\
0.0150015	0.07258344\\
0.01510151	0.0677979\\
0.01520152	0.06354448\\
0.01530153	0.06120211\\
0.01540154	0.05741877\\
0.01550155	0.05303548\\
0.01560156	0.04828765\\
0.01570157	0.04552996\\
0.01580158	0.04088677\\
0.01590159	0.03576585\\
0.0160016	0.0334715\\
0.01610161	0.03111921\\
0.01620162	0.02683768\\
0.01630163	0.02334701\\
0.01640164	0.02033156\\
0.01650165	0.01505191\\
0.01660166	0.0101311\\
0.01670167	0.004520644\\
0.01680168	0.001718279\\
0.01690169	-0.001163227\\
0.0170017	-0.004681667\\
0.01710171	-0.008952129\\
0.01720172	-0.01153174\\
0.01730173	-0.01522815\\
0.01740174	-0.02044039\\
0.01750175	-0.02533239\\
0.01760176	-0.02841276\\
0.01770177	-0.03171307\\
0.01780178	-0.03552816\\
0.01790179	-0.03841937\\
0.0180018	-0.04004397\\
0.01810181	-0.04331819\\
0.01820182	-0.04593727\\
0.01830183	-0.04474421\\
0.01840184	-0.04667591\\
0.01850185	-0.05276672\\
0.01860186	-0.05631153\\
0.01870187	-0.05854873\\
0.01880188	-0.06311616\\
0.01890189	-0.06781841\\
0.0190019	-0.06964474\\
0.01910191	-0.07078413\\
0.01920192	-0.07428616\\
0.01930193	-0.07794442\\
0.01940194	-0.08218223\\
0.01950195	-0.08418649\\
0.01960196	-0.08635292\\
0.01970197	-0.08872375\\
0.01980198	-0.09123925\\
0.01990199	-0.094013\\
0.020002	-0.09476174\\
0.02010201	-0.09716567\\
0.02020202	-0.09710739\\
0.02030203	-0.0979378\\
0.02040204	-0.09856713\\
0.02050205	-0.09629236\\
0.02060206	-0.1000979\\
0.02070207	-0.1040499\\
0.02080208	-0.1064849\\
0.02090209	-0.1070974\\
0.0210021	-0.1090994\\
0.02110211	-0.1127275\\
0.02120212	-0.1142986\\
0.02130213	-0.1159033\\
0.02140214	-0.1142347\\
0.02150215	-0.1124883\\
0.02160216	-0.1128263\\
0.02170217	-0.1165865\\
0.02180218	-0.1142225\\
0.02190219	-0.114174\\
0.0220022	-0.1152734\\
0.02210221	-0.1124714\\
0.02220222	-0.1094055\\
0.02230223	-0.108904\\
0.02240224	-0.1107193\\
0.02250225	-0.1078624\\
0.02260226	-0.1075665\\
0.02270227	-0.1078969\\
0.02280228	-0.1072149\\
0.02290229	-0.108194\\
0.0230023	-0.1087294\\
0.02310231	-0.1081165\\
0.02320232	-0.1049453\\
0.02330233	-0.1041871\\
0.02340234	-0.1037544\\
0.02350235	-0.1012913\\
0.02360236	-0.1006215\\
0.02370237	-0.09936609\\
0.02380238	-0.09606537\\
0.02390239	-0.09854762\\
0.0240024	-0.09685415\\
0.02410241	-0.09451528\\
0.02420242	-0.09380667\\
0.02430243	-0.09169599\\
0.02440244	-0.08806021\\
0.02450245	-0.08642259\\
0.02460246	-0.08637829\\
0.02470247	-0.08403363\\
0.02480248	-0.07864427\\
0.02490249	-0.07687939\\
0.0250025	-0.07424804\\
0.02510251	-0.0688634\\
0.02520252	-0.06475167\\
0.02530253	-0.06065223\\
0.02540254	-0.05949109\\
0.02550255	-0.05674271\\
0.02560256	-0.05352333\\
0.02570257	-0.05227655\\
0.02580258	-0.05060867\\
0.02590259	-0.0481792\\
0.0260026	-0.0463017\\
0.02610261	-0.04502705\\
0.02620262	-0.04176905\\
0.02630263	-0.03844182\\
0.02640264	-0.0363441\\
0.02650265	-0.03225201\\
0.02660266	-0.02858011\\
0.02670267	-0.02509591\\
0.02680268	-0.01997514\\
0.02690269	-0.01870322\\
0.0270027	-0.01603965\\
0.02710271	-0.00972087\\
0.02720272	-0.006757423\\
0.02730273	-0.00601204\\
0.02740274	-0.005112096\\
0.02750275	-0.005643121\\
0.02760276	-0.004815959\\
0.02770277	5.56237e-05\\
0.02780278	0.004897445\\
0.02790279	0.007267329\\
0.0280028	0.01247844\\
0.02810281	0.0169916\\
0.02820282	0.02146909\\
0.02830283	0.02402071\\
0.02840284	0.02310589\\
0.02850285	0.02723219\\
0.02860286	0.03025358\\
0.02870287	0.0330405\\
0.02880288	0.03643788\\
0.02890289	0.03863932\\
0.0290029	0.0405907\\
0.02910291	0.04331745\\
0.02920292	0.04500852\\
0.02930293	0.04754126\\
0.02940294	0.05212636\\
0.02950295	0.05451571\\
0.02960296	0.0559105\\
0.02970297	0.05860553\\
0.02980298	0.06024206\\
0.02990299	0.06237742\\
0.030003	0.06336634\\
0.03010301	0.06418606\\
0.03020302	0.06559361\\
0.03030303	0.06932832\\
0.03040304	0.07147265\\
0.03050305	0.07358131\\
0.03060306	0.07403415\\
0.03070307	0.07700523\\
0.03080308	0.07995892\\
0.03090309	0.07840772\\
0.0310031	0.08071351\\
0.03110311	0.08230459\\
0.03120312	0.08293823\\
0.03130313	0.08250393\\
0.03140314	0.08212286\\
0.03150315	0.0822777\\
0.03160316	0.08332974\\
0.03170317	0.08038686\\
0.03180318	0.08152541\\
0.03190319	0.08257187\\
0.0320032	0.08383988\\
0.03210321	0.0886912\\
0.03220322	0.08919168\\
0.03230323	0.09009843\\
0.03240324	0.09196821\\
0.03250325	0.08953468\\
0.03260326	0.09232507\\
0.03270327	0.09275951\\
0.03280328	0.09393568\\
0.03290329	0.09319031\\
0.0330033	0.0923603\\
0.03310331	0.09295938\\
0.03320332	0.09236728\\
0.03330333	0.08838213\\
0.03340334	0.08742366\\
0.03350335	0.08575717\\
0.03360336	0.08481433\\
0.03370337	0.08539112\\
0.03380338	0.08282356\\
0.03390339	0.08124639\\
0.0340034	0.07874598\\
0.03410341	0.07802166\\
0.03420342	0.07656102\\
0.03430343	0.07490028\\
0.03440344	0.07596969\\
0.03450345	0.07490541\\
0.03460346	0.07420521\\
0.03470347	0.07134698\\
0.03480348	0.07059452\\
0.03490349	0.07083929\\
0.0350035	0.0685534\\
0.03510351	0.06684929\\
0.03520352	0.06478493\\
0.03530353	0.06241222\\
0.03540354	0.05992096\\
0.03550355	0.05663955\\
0.03560356	0.05563308\\
0.03570357	0.05435889\\
0.03580358	0.05191744\\
0.03590359	0.05083271\\
0.0360036	0.04919623\\
0.03610361	0.04769363\\
0.03620362	0.04435216\\
0.03630363	0.04360585\\
0.03640364	0.04192765\\
0.03650365	0.03908215\\
0.03660366	0.03741782\\
0.03670367	0.03302746\\
0.03680368	0.03117577\\
0.03690369	0.02776329\\
0.0370037	0.02291905\\
0.03710371	0.0231228\\
0.03720372	0.02030011\\
0.03730373	0.02043595\\
0.03740374	0.01841131\\
0.03750375	0.01378515\\
0.03760376	0.01249955\\
0.03770377	0.008410198\\
0.03780378	0.006321688\\
0.03790379	0.002696408\\
0.0380038	0.002827333\\
0.03810381	0.0009015981\\
0.03820382	-0.003011427\\
0.03830383	-0.005427831\\
0.03840384	-0.008835746\\
0.03850385	-0.01153974\\
0.03860386	-0.01470306\\
0.03870387	-0.01625738\\
0.03880388	-0.02014559\\
0.03890389	-0.02215937\\
0.0390039	-0.02282441\\
0.03910391	-0.02640033\\
0.03920392	-0.02779707\\
0.03930393	-0.02837352\\
0.03940394	-0.02792616\\
0.03950395	-0.02928564\\
0.03960396	-0.02932024\\
0.03970397	-0.03394228\\
0.03980398	-0.03547702\\
0.03990399	-0.03720897\\
0.040004	-0.04078204\\
0.04010401	-0.04154665\\
0.04020402	-0.04552428\\
0.04030403	-0.04733092\\
0.04040404	-0.05011716\\
0.04050405	-0.05112978\\
0.04060406	-0.05040409\\
0.04070407	-0.05135172\\
0.04080408	-0.05303585\\
0.04090409	-0.05391018\\
0.0410041	-0.05396468\\
0.04110411	-0.05648432\\
0.04120412	-0.05890184\\
0.04130413	-0.06193053\\
0.04140414	-0.06058214\\
0.04150415	-0.06320821\\
0.04160416	-0.0649502\\
0.04170417	-0.06765436\\
0.04180418	-0.0686263\\
0.04190419	-0.06997249\\
0.0420042	-0.07113096\\
0.04210421	-0.07005626\\
0.04220422	-0.07271902\\
0.04230423	-0.07205072\\
0.04240424	-0.07337028\\
0.04250425	-0.07151783\\
0.04260426	-0.06918077\\
0.04270427	-0.06727479\\
0.04280428	-0.06810175\\
0.04290429	-0.06425878\\
0.0430043	-0.06482672\\
0.04310431	-0.06657166\\
0.04320432	-0.06592581\\
0.04330433	-0.06762032\\
0.04340434	-0.06897323\\
0.04350435	-0.07102459\\
0.04360436	-0.06920248\\
0.04370437	-0.07229384\\
0.04380438	-0.06906449\\
0.04390439	-0.06753714\\
0.0440044	-0.066529\\
0.04410441	-0.06514304\\
0.04420442	-0.06639197\\
0.04430443	-0.06800862\\
0.04440444	-0.06662439\\
0.04450445	-0.06719995\\
0.04460446	-0.06291771\\
0.04470447	-0.06214345\\
0.04480448	-0.06053961\\
0.04490449	-0.0595035\\
0.0450045	-0.05887928\\
0.04510451	-0.05844176\\
0.04520452	-0.05494067\\
0.04530453	-0.05520517\\
0.04540454	-0.05233786\\
0.04550455	-0.05132344\\
0.04560456	-0.05048623\\
0.04570457	-0.04866819\\
0.04580458	-0.04735799\\
0.04590459	-0.04647429\\
0.0460046	-0.04713287\\
0.04610461	-0.04433747\\
0.04620462	-0.04460825\\
0.04630463	-0.04327483\\
0.04640464	-0.04082057\\
0.04650465	-0.03962222\\
0.04660466	-0.03631582\\
0.04670467	-0.03494665\\
0.04680468	-0.03165671\\
0.04690469	-0.02591893\\
0.0470047	-0.02351838\\
0.04710471	-0.02132667\\
0.04720472	-0.01888687\\
0.04730473	-0.0183637\\
0.04740474	-0.01783603\\
0.04750475	-0.0177333\\
0.04760476	-0.01765326\\
0.04770477	-0.01484551\\
0.04780478	-0.01563567\\
0.04790479	-0.01135496\\
0.0480048	-0.00970951\\
0.04810481	-0.007711804\\
0.04820482	-0.003893395\\
0.04830483	-0.003137666\\
0.04840484	-1.086657e-05\\
0.04850485	0.0002372787\\
0.04860486	0.001315171\\
0.04870487	0.002325827\\
0.04880488	0.00500459\\
0.04890489	0.00702721\\
0.0490049	0.008485616\\
0.04910491	0.009604571\\
0.04920492	0.009750197\\
0.04930493	0.01284496\\
0.04940494	0.0152364\\
0.04950495	0.01868043\\
0.04960496	0.01877152\\
0.04970497	0.02189274\\
0.04980498	0.02354759\\
0.04990499	0.02667703\\
0.050005	0.0283616\\
0.05010501	0.02957494\\
0.05020502	0.03234265\\
0.05030503	0.03190053\\
0.05040504	0.03317559\\
0.05050505	0.03549447\\
0.05060506	0.03808268\\
0.05070507	0.04065325\\
0.05080508	0.04089003\\
0.05090509	0.04199066\\
0.0510051	0.04218183\\
0.05110511	0.04340443\\
0.05120512	0.04367862\\
0.05130513	0.04232836\\
0.05140514	0.04240435\\
0.05150515	0.04417268\\
0.05160516	0.04495163\\
0.05170517	0.04673139\\
0.05180518	0.04909222\\
0.05190519	0.04945762\\
0.0520052	0.04783407\\
0.05210521	0.04810908\\
0.05220522	0.05025043\\
0.05230523	0.0514657\\
0.05240524	0.05247228\\
0.05250525	0.05195905\\
0.05260526	0.05233952\\
0.05270527	0.05423597\\
0.05280528	0.05387954\\
0.05290529	0.05429243\\
0.0530053	0.05215315\\
0.05310531	0.05442737\\
0.05320532	0.05474014\\
0.05330533	0.05458563\\
0.05340534	0.05653501\\
0.05350535	0.05672162\\
0.05360536	0.05704048\\
0.05370537	0.056736\\
0.05380538	0.05848789\\
0.05390539	0.05861187\\
0.0540054	0.05627531\\
0.05410541	0.05353223\\
0.05420542	0.05013097\\
0.05430543	0.05043155\\
0.05440544	0.04998873\\
0.05450545	0.05055124\\
0.05460546	0.04967685\\
0.05470547	0.04961074\\
0.05480548	0.04935713\\
0.05490549	0.04810098\\
0.0550055	0.04908213\\
0.05510551	0.04792568\\
0.05520552	0.04788201\\
0.05530553	0.04882512\\
0.05540554	0.04742946\\
0.05550555	0.04793835\\
0.05560556	0.04583284\\
0.05570557	0.04275178\\
0.05580558	0.04030478\\
0.05590559	0.03964935\\
0.0560056	0.03652161\\
0.05610561	0.03432274\\
0.05620562	0.03182972\\
0.05630563	0.02993116\\
0.05640564	0.03043522\\
0.05650565	0.0295819\\
0.05660566	0.02798182\\
0.05670567	0.02685817\\
0.05680568	0.02718521\\
0.05690569	0.02760648\\
0.0570057	0.02659968\\
0.05710571	0.0259541\\
0.05720572	0.02682298\\
0.05730573	0.02662055\\
0.05740574	0.0239607\\
0.05750575	0.02223825\\
0.05760576	0.02034563\\
0.05770577	0.01718926\\
0.05780578	0.01503301\\
0.05790579	0.01368498\\
0.0580058	0.01141288\\
0.05810581	0.009450221\\
0.05820582	0.008237489\\
0.05830583	0.005596995\\
0.05840584	0.004576484\\
0.05850585	0.003959781\\
0.05860586	0.00154075\\
0.05870587	0.0008838225\\
0.05880588	-0.001620288\\
0.05890589	-0.003534929\\
0.0590059	-0.004661072\\
0.05910591	-0.006889861\\
0.05920592	-0.007609789\\
0.05930593	-0.01152122\\
0.05940594	-0.01263014\\
0.05950595	-0.01532876\\
0.05960596	-0.01239348\\
0.05970597	-0.01266795\\
0.05980598	-0.01310584\\
0.05990599	-0.01576504\\
0.060006	-0.01614588\\
0.06010601	-0.01847327\\
0.06020602	-0.02073057\\
0.06030603	-0.02436295\\
0.06040604	-0.02290182\\
0.06050605	-0.02159578\\
0.06060606	-0.02240361\\
0.06070607	-0.0250799\\
0.06080608	-0.02731127\\
0.06090609	-0.02797343\\
0.0610061	-0.02855243\\
0.06110611	-0.02856199\\
0.06120612	-0.02971267\\
0.06130613	-0.03012353\\
0.06140614	-0.03105161\\
0.06150615	-0.03256968\\
0.06160616	-0.03238544\\
0.06170617	-0.03498193\\
0.06180618	-0.03810678\\
0.06190619	-0.03728747\\
0.0620062	-0.03800694\\
0.06210621	-0.03925509\\
0.06220622	-0.0394937\\
0.06230623	-0.04067501\\
0.06240624	-0.04275407\\
0.06250625	-0.04363483\\
0.06260626	-0.04326313\\
0.06270627	-0.04273223\\
0.06280628	-0.04138333\\
0.06290629	-0.04019809\\
0.0630063	-0.03923106\\
0.06310631	-0.03836992\\
0.06320632	-0.04065039\\
0.06330633	-0.04126967\\
0.06340634	-0.04316424\\
0.06350635	-0.04341464\\
0.06360636	-0.04417875\\
0.06370637	-0.04530127\\
0.06380638	-0.04730741\\
0.06390639	-0.04511555\\
0.0640064	-0.04335128\\
0.06410641	-0.04017735\\
0.06420642	-0.04100841\\
0.06430643	-0.04124671\\
0.06440644	-0.03971614\\
0.06450645	-0.0388373\\
0.06460646	-0.03846366\\
0.06470647	-0.03835447\\
0.06480648	-0.03822401\\
0.06490649	-0.037401\\
0.0650065	-0.03898423\\
0.06510651	-0.03892308\\
0.06520652	-0.03816699\\
0.06530653	-0.03596121\\
0.06540654	-0.03425717\\
0.06550655	-0.03387415\\
0.06560656	-0.03372875\\
0.06570657	-0.03173746\\
0.06580658	-0.0328893\\
0.06590659	-0.03143701\\
0.0660066	-0.03111011\\
0.06610661	-0.03235536\\
0.06620662	-0.03138758\\
0.06630663	-0.02989081\\
0.06640664	-0.02937527\\
0.06650665	-0.02929468\\
0.06660666	-0.02948129\\
0.06670667	-0.02949603\\
0.06680668	-0.02738975\\
0.06690669	-0.02548693\\
0.0670067	-0.02347098\\
0.06710671	-0.02252487\\
0.06720672	-0.02111182\\
0.06730673	-0.0175603\\
0.06740674	-0.0159488\\
0.06750675	-0.01364921\\
0.06760676	-0.01092101\\
0.06770677	-0.009905357\\
0.06780678	-0.01007792\\
0.06790679	-0.01099183\\
0.0680068	-0.008764689\\
0.06810681	-0.006604942\\
0.06820682	-0.005114084\\
0.06830683	-0.005379974\\
0.06840684	-0.00386372\\
0.06850685	-0.00185283\\
0.06860686	-0.002226752\\
0.06870687	-0.002927545\\
0.06880688	-0.00534917\\
0.06890689	-0.00542673\\
0.0690069	-0.003183973\\
0.06910691	-0.001066282\\
0.06920692	0.0009379306\\
0.06930693	0.00273031\\
0.06940694	0.005229169\\
0.06950695	0.007115165\\
0.06960696	0.01066999\\
0.06970697	0.01211214\\
0.06980698	0.01328618\\
0.06990699	0.01400176\\
0.070007	0.01324447\\
0.07010701	0.0124428\\
0.07020702	0.01071786\\
0.07030703	0.01287871\\
0.07040704	0.01443326\\
0.07050705	0.01492055\\
0.07060706	0.01685495\\
0.07070707	0.01718983\\
0.07080708	0.02009459\\
0.07090709	0.02078159\\
0.0710071	0.02045481\\
0.07110711	0.02099339\\
0.07120712	0.02424436\\
0.07130713	0.02788959\\
0.07140714	0.02790845\\
0.07150715	0.0274479\\
0.07160716	0.02781334\\
0.07170717	0.02949628\\
0.07180718	0.02690255\\
0.07190719	0.02787888\\
0.0720072	0.02714577\\
0.07210721	0.02882705\\
0.07220722	0.02960338\\
0.07230723	0.02856871\\
0.07240724	0.02875548\\
0.07250725	0.03006659\\
0.07260726	0.03162799\\
0.07270727	0.0322334\\
0.07280728	0.03142513\\
0.07290729	0.03263295\\
0.0730073	0.03293295\\
0.07310731	0.0324031\\
0.07320732	0.03257003\\
0.07330733	0.03224566\\
0.07340734	0.03273363\\
0.07350735	0.0318241\\
0.07360736	0.03166282\\
0.07370737	0.03103971\\
0.07380738	0.03128422\\
0.07390739	0.0304168\\
0.0740074	0.03087632\\
0.07410741	0.03119068\\
0.07420742	0.0324251\\
0.07430743	0.03227314\\
0.07440744	0.03046526\\
0.07450745	0.03062738\\
0.07460746	0.03166714\\
0.07470747	0.03195768\\
0.07480748	0.03220969\\
0.07490749	0.0316303\\
0.0750075	0.03218975\\
0.07510751	0.03159147\\
0.07520752	0.02953389\\
0.07530753	0.02900866\\
0.07540754	0.03100105\\
0.07550755	0.03165133\\
0.07560756	0.02952487\\
0.07570757	0.02734946\\
0.07580758	0.02868511\\
0.07590759	0.02864897\\
0.0760076	0.02496044\\
0.07610761	0.02111603\\
0.07620762	0.02081262\\
0.07630763	0.02337618\\
0.07640764	0.02239615\\
0.07650765	0.01886922\\
0.07660766	0.01595191\\
0.07670767	0.01607135\\
0.07680768	0.01359176\\
0.07690769	0.01459829\\
0.0770077	0.01452813\\
0.07710771	0.01518518\\
0.07720772	0.01475603\\
0.07730773	0.0159071\\
0.07740774	0.01743529\\
0.07750775	0.01601338\\
0.07760776	0.01406064\\
0.07770777	0.01255768\\
0.07780778	0.01102145\\
0.07790779	0.008700246\\
0.0780078	0.007867096\\
0.07810781	0.008426615\\
0.07820782	0.008251193\\
0.07830783	0.00715011\\
0.07840784	0.005834719\\
0.07850785	0.005229754\\
0.07860786	0.004124016\\
0.07870787	0.002378536\\
0.07880788	0.001820158\\
0.07890789	0.001558729\\
0.0790079	-1.255383e-05\\
0.07910791	-0.001456229\\
0.07920792	-0.003411982\\
0.07930793	-0.004309612\\
0.07940794	-0.004826715\\
0.07950795	-0.006341637\\
0.07960796	-0.007377062\\
0.07970797	-0.009140446\\
0.07980798	-0.008216576\\
0.07990799	-0.008110727\\
0.080008	-0.00980725\\
0.08010801	-0.00927047\\
0.08020802	-0.007542012\\
0.08030803	-0.007843787\\
0.08040804	-0.008764332\\
0.08050805	-0.01335136\\
0.08060806	-0.01534959\\
0.08070807	-0.01534902\\
0.08080808	-0.01625497\\
0.08090809	-0.01818831\\
0.0810081	-0.01862694\\
0.08110811	-0.01914868\\
0.08120812	-0.02048788\\
0.08130813	-0.02158355\\
0.08140814	-0.02138488\\
0.08150815	-0.02071588\\
0.08160816	-0.02035827\\
0.08170817	-0.01988686\\
0.08180818	-0.01851875\\
0.08190819	-0.01813291\\
0.0820082	-0.01873893\\
0.08210821	-0.01771352\\
0.08220822	-0.01760937\\
0.08230823	-0.01928704\\
0.08240824	-0.02145215\\
0.08250825	-0.02334729\\
0.08260826	-0.0248535\\
0.08270827	-0.0255425\\
0.08280828	-0.02532265\\
0.08290829	-0.02581223\\
0.0830083	-0.02447387\\
0.08310831	-0.02494661\\
0.08320832	-0.02666146\\
0.08330833	-0.02354313\\
0.08340834	-0.02458443\\
0.08350835	-0.02507041\\
0.08360836	-0.02633253\\
0.08370837	-0.02568911\\
0.08380838	-0.02633654\\
0.08390839	-0.02752192\\
0.0840084	-0.02620962\\
0.08410841	-0.02546339\\
0.08420842	-0.02534699\\
0.08430843	-0.02564007\\
0.08440844	-0.02487598\\
0.08450845	-0.02442788\\
0.08460846	-0.02522616\\
0.08470847	-0.02558496\\
0.08480848	-0.02406637\\
0.08490849	-0.02358596\\
0.0850085	-0.02436037\\
0.08510851	-0.02508327\\
0.08520852	-0.02279299\\
0.08530853	-0.0203642\\
0.08540854	-0.01809455\\
0.08550855	-0.01823771\\
0.08560856	-0.01536409\\
0.08570857	-0.01414038\\
0.08580858	-0.01560318\\
0.08590859	-0.01536903\\
0.0860086	-0.01736578\\
0.08610861	-0.0184058\\
0.08620862	-0.01912229\\
0.08630863	-0.01909734\\
0.08640864	-0.01937691\\
0.08650865	-0.0181688\\
0.08660866	-0.01789494\\
0.08670867	-0.0166009\\
0.08680868	-0.01707755\\
0.08690869	-0.01767403\\
0.0870087	-0.01483572\\
0.08710871	-0.01088558\\
0.08720872	-0.009783414\\
0.08730873	-0.01100308\\
0.08740874	-0.009379968\\
0.08750875	-0.008023644\\
0.08760876	-0.008922386\\
0.08770877	-0.008853345\\
0.08780878	-0.007306042\\
0.08790879	-0.006796764\\
0.0880088	-0.006230583\\
0.08810881	-0.005562056\\
0.08820882	-0.005771714\\
0.08830883	-0.003816852\\
0.08840884	-0.003020157\\
0.08850885	-0.002347909\\
0.08860886	-0.00059104\\
0.08870887	1.78139e-05\\
0.08880888	0.000762334\\
0.08890889	0.0007332713\\
0.0890089	0.001426321\\
0.08910891	0.00161314\\
0.08920892	0.001190531\\
0.08930893	0.0005824538\\
0.08940894	0.001698636\\
0.08950895	0.002931095\\
0.08960896	0.005319562\\
0.08970897	0.007609883\\
0.08980898	0.008389694\\
0.08990899	0.009770561\\
0.090009	0.01073855\\
0.09010901	0.01113743\\
0.09020902	0.01064404\\
0.09030903	0.009571467\\
0.09040904	0.008732977\\
0.09050905	0.007960501\\
0.09060906	0.0105978\\
0.09070907	0.01169222\\
0.09080908	0.01158518\\
0.09090909	0.01184701\\
0.0910091	0.010253\\
0.09110911	0.01056672\\
0.09120912	0.009502023\\
0.09130913	0.01286015\\
0.09140914	0.01433517\\
0.09150915	0.01209355\\
0.09160916	0.01413023\\
0.09170917	0.01640391\\
0.09180918	0.01605138\\
0.09190919	0.01468179\\
0.0920092	0.0168389\\
0.09210921	0.02111515\\
0.09220922	0.02100676\\
0.09230923	0.01915579\\
0.09240924	0.01992696\\
0.09250925	0.01983613\\
0.09260926	0.01860573\\
0.09270927	0.01885995\\
0.09280928	0.01952437\\
0.09290929	0.01922221\\
0.0930093	0.02122098\\
0.09310931	0.02219518\\
0.09320932	0.02254989\\
0.09330933	0.02135018\\
0.09340934	0.01876597\\
0.09350935	0.01743764\\
0.09360936	0.01577651\\
0.09370937	0.01592902\\
0.09380938	0.01669529\\
0.09390939	0.01703894\\
0.0940094	0.01777246\\
0.09410941	0.01730457\\
0.09420942	0.01786301\\
0.09430943	0.01820074\\
0.09440944	0.01674705\\
0.09450945	0.01781533\\
0.09460946	0.01787624\\
0.09470947	0.01853852\\
0.09480948	0.01859323\\
0.09490949	0.01821477\\
0.0950095	0.01785691\\
0.09510951	0.01706115\\
0.09520952	0.01512826\\
0.09530953	0.0151576\\
0.09540954	0.01606279\\
0.09550955	0.01466308\\
0.09560956	0.01366313\\
0.09570957	0.01240243\\
0.09580958	0.01405632\\
0.09590959	0.0151955\\
0.0960096	0.0169634\\
0.09610961	0.01724693\\
0.09620962	0.01782775\\
0.09630963	0.01509419\\
0.09640964	0.0133627\\
0.09650965	0.01275596\\
0.09660966	0.01299094\\
0.09670967	0.00993605\\
0.09680968	0.007013594\\
0.09690969	0.007064539\\
0.0970097	0.005669072\\
0.09710971	0.005393736\\
0.09720972	0.005102639\\
0.09730973	0.005595443\\
0.09740974	0.005278921\\
0.09750975	0.006241486\\
0.09760976	0.00727175\\
0.09770977	0.006904193\\
0.09780978	0.005759879\\
0.09790979	0.007123543\\
0.0980098	0.005276273\\
0.09810981	0.004777994\\
0.09820982	0.004501624\\
0.09830983	0.003017669\\
0.09840984	-0.0007485647\\
0.09850985	-0.00181459\\
0.09860986	-0.001428311\\
0.09870987	-0.002540984\\
0.09880988	-0.001416866\\
0.09890989	-0.002587194\\
0.0990099	-0.004771637\\
0.09910991	-0.003760599\\
0.09920992	-0.002931016\\
0.09930993	-0.003742714\\
0.09940994	-0.004380697\\
0.09950995	-0.001956908\\
0.09960996	-0.003355996\\
0.09970997	-0.004385533\\
0.09980998	-0.003072239\\
0.09990999	-0.001610172\\
0.10001	-0.003109152\\
0.10011	-0.005383281\\
0.10021	-0.004528272\\
0.10031	-0.006571203\\
0.10041	-0.007359723\\
0.1005101	-0.007776381\\
0.1006101	-0.009597252\\
0.1007101	-0.01123952\\
0.1008101	-0.01044907\\
0.1009101	-0.01078248\\
0.1010101	-0.01208155\\
0.1011101	-0.0123275\\
0.1012101	-0.01337973\\
0.1013101	-0.01628347\\
0.1014101	-0.01513492\\
0.1015102	-0.01320087\\
0.1016102	-0.01467816\\
0.1017102	-0.01350696\\
0.1018102	-0.01175547\\
0.1019102	-0.01370899\\
0.1020102	-0.01254071\\
0.1021102	-0.01364572\\
0.1022102	-0.01418018\\
0.1023102	-0.01281761\\
0.1024102	-0.01270873\\
0.1025103	-0.01184411\\
0.1026103	-0.01361177\\
0.1027103	-0.01594652\\
0.1028103	-0.01402297\\
0.1029103	-0.01266759\\
0.1030103	-0.01119063\\
0.1031103	-0.01033807\\
0.1032103	-0.01109259\\
0.1033103	-0.01285871\\
0.1034103	-0.01293296\\
0.1035104	-0.0153594\\
0.1036104	-0.01569918\\
0.1037104	-0.01529202\\
0.1038104	-0.01670097\\
0.1039104	-0.01610553\\
0.1040104	-0.01612332\\
0.1041104	-0.01450397\\
0.1042104	-0.01611789\\
0.1043104	-0.0150544\\
0.1044104	-0.01415996\\
0.1045105	-0.01245054\\
0.1046105	-0.01171111\\
0.1047105	-0.01098817\\
0.1048105	-0.01016763\\
0.1049105	-0.01081286\\
0.1050105	-0.01303148\\
0.1051105	-0.01383283\\
0.1052105	-0.01447303\\
0.1053105	-0.01527804\\
0.1054105	-0.01459342\\
0.1055106	-0.01273126\\
0.1056106	-0.01057018\\
0.1057106	-0.01058781\\
0.1058106	-0.008157185\\
0.1059106	-0.007597113\\
0.1060106	-0.007159798\\
0.1061106	-0.006300569\\
0.1062106	-0.008437759\\
0.1063106	-0.01047406\\
0.1064106	-0.009678208\\
0.1065107	-0.007948812\\
0.1066107	-0.006947955\\
0.1067107	-0.00551925\\
0.1068107	-0.00427707\\
0.1069107	-0.005621825\\
0.1070107	-0.006349453\\
0.1071107	-0.004540397\\
0.1072107	-0.004215253\\
0.1073107	-0.004820977\\
0.1074107	-0.003363729\\
0.1075108	-0.001930777\\
0.1076108	-0.001975423\\
0.1077108	-0.00218255\\
0.1078108	-0.002836447\\
0.1079108	-0.002896024\\
0.1080108	-0.00234683\\
0.1081108	-0.002558801\\
0.1082108	-0.001826645\\
0.1083108	-0.0002688672\\
0.1084108	-0.0007489465\\
0.1085109	-0.001463453\\
0.1086109	0.0005797952\\
0.1087109	0.0002578831\\
0.1088109	0.001860438\\
0.1089109	0.001789783\\
0.1090109	0.002594728\\
0.1091109	0.002979324\\
0.1092109	0.001673436\\
0.1093109	0.0008318078\\
0.1094109	0.001009891\\
0.109511	0.003059723\\
0.109611	0.004010644\\
0.109711	0.003717168\\
0.109811	0.005571772\\
0.109911	0.006110169\\
0.110011	0.006763947\\
0.110111	0.009685184\\
0.110211	0.009418036\\
0.110311	0.01175453\\
0.110411	0.01147796\\
0.1105111	0.01164186\\
0.1106111	0.009844498\\
0.1107111	0.01121876\\
0.1108111	0.009719719\\
0.1109111	0.007985638\\
0.1110111	0.007095038\\
0.1111111	0.006829085\\
0.1112111	0.007670552\\
0.1113111	0.007644624\\
0.1114111	0.005136604\\
0.1115112	0.003293855\\
0.1116112	0.005324395\\
0.1117112	0.006538952\\
0.1118112	0.008244079\\
0.1119112	0.008927717\\
0.1120112	0.0105355\\
0.1121112	0.01192901\\
0.1122112	0.01429799\\
0.1123112	0.01448753\\
0.1124112	0.01522356\\
0.1125113	0.01263163\\
0.1126113	0.0106871\\
0.1127113	0.01112291\\
0.1128113	0.01028092\\
0.1129113	0.008996325\\
0.1130113	0.008931404\\
0.1131113	0.009876721\\
0.1132113	0.009989983\\
0.1133113	0.01078234\\
0.1134113	0.01165632\\
0.1135114	0.01141371\\
0.1136114	0.01225572\\
0.1137114	0.01315899\\
0.1138114	0.01137617\\
0.1139114	0.009538778\\
0.1140114	0.0107666\\
0.1141114	0.01123859\\
0.1142114	0.01059431\\
0.1143114	0.009045374\\
0.1144114	0.008213733\\
0.1145115	0.01048775\\
0.1146115	0.01102179\\
0.1147115	0.008225928\\
0.1148115	0.00821483\\
0.1149115	0.00821724\\
0.1150115	0.008532815\\
0.1151115	0.009144158\\
0.1152115	0.008272372\\
0.1153115	0.007724836\\
0.1154115	0.007439563\\
0.1155116	0.006409112\\
0.1156116	0.006946267\\
0.1157116	0.008995164\\
0.1158116	0.005503727\\
0.1159116	0.004528187\\
0.1160116	0.003600492\\
0.1161116	0.002754178\\
0.1162116	0.003664232\\
0.1163116	0.003596085\\
0.1164116	0.005350875\\
0.1165117	0.005382127\\
0.1166117	0.006844015\\
0.1167117	0.005306301\\
0.1168117	0.004683491\\
0.1169117	0.002998374\\
0.1170117	0.003020253\\
0.1171117	0.003475375\\
0.1172117	0.002931971\\
0.1173117	0.004420412\\
0.1174117	0.003876887\\
0.1175118	0.002900873\\
0.1176118	0.003098843\\
0.1177118	0.003596409\\
0.1178118	0.002310026\\
0.1179118	0.001238931\\
0.1180118	-0.0003425202\\
0.1181118	-0.001464809\\
0.1182118	-0.001530992\\
0.1183118	-0.00188581\\
0.1184118	-0.003114384\\
0.1185119	-0.002059374\\
0.1186119	-0.002583257\\
0.1187119	-0.003929632\\
0.1188119	-0.004250944\\
0.1189119	-0.002584914\\
0.1190119	-0.001656424\\
0.1191119	-0.002886213\\
0.1192119	-0.004893069\\
0.1193119	-0.006009233\\
0.1194119	-0.006861099\\
0.119512	-0.006475281\\
0.119612	-0.005659378\\
0.119712	-0.00576811\\
0.119812	-0.004297784\\
0.119912	-0.003850239\\
0.120012	-0.005154458\\
0.120112	-0.006519781\\
0.120212	-0.005374743\\
0.120312	-0.003772176\\
0.120412	-0.004049395\\
0.1205121	-0.003509346\\
0.1206121	-0.003690123\\
0.1207121	-0.00159522\\
0.1208121	-0.002763918\\
0.1209121	-0.006000591\\
0.1210121	-0.006499002\\
0.1211121	-0.008758987\\
0.1212121	-0.009015115\\
0.1213121	-0.009091536\\
0.1214121	-0.01118021\\
0.1215122	-0.01056023\\
0.1216122	-0.009222585\\
0.1217122	-0.009246756\\
0.1218122	-0.007864754\\
0.1219122	-0.008644104\\
0.1220122	-0.009155897\\
0.1221122	-0.009619902\\
0.1222122	-0.008967288\\
0.1223122	-0.006500907\\
0.1224122	-0.007435435\\
0.1225123	-0.006569445\\
0.1226123	-0.006825547\\
0.1227123	-0.009549455\\
0.1228123	-0.009347551\\
0.1229123	-0.008444644\\
0.1230123	-0.00804505\\
0.1231123	-0.007355035\\
0.1232123	-0.006820348\\
0.1233123	-0.007283356\\
0.1234123	-0.005671114\\
0.1235124	-0.004738717\\
0.1236124	-0.008585988\\
0.1237124	-0.01085453\\
0.1238124	-0.009980286\\
0.1239124	-0.01112333\\
0.1240124	-0.008627882\\
0.1241124	-0.008734988\\
0.1242124	-0.008407321\\
0.1243124	-0.006833361\\
0.1244124	-0.005916557\\
0.1245125	-0.003823966\\
0.1246125	-0.002443942\\
0.1247125	-0.001791975\\
0.1248125	-0.0002454469\\
0.1249125	-0.001066308\\
0.1250125	-0.003166502\\
0.1251125	-0.004383243\\
0.1252125	-0.00773179\\
0.1253125	-0.007163615\\
0.1254125	-0.006749512\\
0.1255126	-0.007437727\\
0.1256126	-0.006364976\\
0.1257126	-0.00702305\\
0.1258126	-0.00666902\\
0.1259126	-0.005742053\\
0.1260126	-0.006310811\\
0.1261126	-0.004884521\\
0.1262126	-0.002080543\\
0.1263126	-0.001555706\\
0.1264126	-0.003344612\\
0.1265127	-0.001954552\\
0.1266127	-0.001004163\\
0.1267127	-0.0009936044\\
0.1268127	-0.00244968\\
0.1269127	-0.003033493\\
0.1270127	-0.00151749\\
0.1271127	-0.0001871888\\
0.1272127	-0.0002935627\\
0.1273127	-0.0009604826\\
0.1274127	-0.0009715048\\
0.1275128	0.001965749\\
0.1276128	0.001487367\\
0.1277128	0.0003783221\\
0.1278128	0.002772482\\
0.1279128	0.002150845\\
0.1280128	0.005020404\\
0.1281128	0.003677382\\
0.1282128	0.003256723\\
0.1283128	0.002545258\\
0.1284128	0.002946798\\
0.1285129	0.003985993\\
0.1286129	0.001814626\\
0.1287129	-0.000591549\\
0.1288129	-0.0005543072\\
0.1289129	-2.834861e-05\\
0.1290129	0.0002829251\\
0.1291129	0.0008504773\\
0.1292129	0.001600483\\
0.1293129	0.003170989\\
0.1294129	0.003456924\\
0.129513	0.005493122\\
0.129613	0.00573633\\
0.129713	0.003196881\\
0.129813	0.005448776\\
0.129913	0.004839108\\
0.130013	0.003922589\\
0.130113	0.004914403\\
0.130213	0.003570424\\
0.130313	0.006587276\\
0.130413	0.006107693\\
0.1305131	0.005018643\\
0.1306131	0.00601649\\
0.1307131	0.005293899\\
0.1308131	0.005792227\\
0.1309131	0.007838501\\
0.1310131	0.007559528\\
0.1311131	0.006668643\\
0.1312131	0.006477838\\
0.1313131	0.00602342\\
0.1314131	0.008400657\\
0.1315132	0.009076549\\
0.1316132	0.005476359\\
0.1317132	0.004916504\\
0.1318132	0.003905632\\
0.1319132	0.003350459\\
0.1320132	0.002670761\\
0.1321132	0.004598745\\
0.1322132	0.005833628\\
0.1323132	0.008016956\\
0.1324132	0.008147044\\
0.1325133	0.00662176\\
0.1326133	0.00787103\\
0.1327133	0.007533053\\
0.1328133	0.005764285\\
0.1329133	0.005252399\\
0.1330133	0.004752123\\
0.1331133	0.005072541\\
0.1332133	0.005259056\\
0.1333133	0.004648049\\
0.1334133	0.005026757\\
0.1335134	0.006698834\\
0.1336134	0.005579421\\
0.1337134	0.002237011\\
0.1338134	0.001107559\\
0.1339134	0.003373213\\
0.1340134	0.003644273\\
0.1341134	0.002557001\\
0.1342134	0.003152297\\
0.1343134	0.004595455\\
0.1344134	0.006127384\\
0.1345135	0.003222348\\
0.1346135	0.005028077\\
0.1347135	0.004659076\\
0.1348135	0.007303454\\
0.1349135	0.00876019\\
0.1350135	0.007585468\\
0.1351135	0.006961036\\
0.1352135	0.006553091\\
0.1353135	0.003651309\\
0.1354135	0.001673477\\
0.1355136	0.0006338206\\
0.1356136	0.001898517\\
0.1357136	0.002550827\\
0.1358136	0.00103447\\
0.1359136	-0.0009193744\\
0.1360136	-0.001417908\\
0.1361136	-0.0007963781\\
0.1362136	-0.001493725\\
0.1363136	-0.001731796\\
0.1364136	-0.001459554\\
0.1365137	-0.0003106431\\
0.1366137	0.0005964352\\
0.1367137	-0.0003168654\\
0.1368137	0.001191486\\
0.1369137	0.001680708\\
0.1370137	0.0005255067\\
0.1371137	0.001067601\\
0.1372137	-0.0009094623\\
0.1373137	-0.0005817741\\
0.1374137	-9.449187e-05\\
0.1375138	-0.001995466\\
0.1376138	-0.0007647829\\
0.1377138	0.0002135728\\
0.1378138	-0.0002293007\\
0.1379138	0.00140803\\
0.1380138	-0.0002369368\\
0.1381138	-0.0003519484\\
0.1382138	-0.002622244\\
0.1383138	-0.002714722\\
0.1384138	-0.003416034\\
0.1385139	-0.004108154\\
0.1386139	-0.002942803\\
0.1387139	-0.003208268\\
0.1388139	-0.00188164\\
0.1389139	-0.002448473\\
0.1390139	-0.002647167\\
0.1391139	-0.002534838\\
0.1392139	-0.001562042\\
0.1393139	-0.0004194555\\
0.1394139	-0.00290467\\
0.139514	-0.006859209\\
0.139614	-0.005696951\\
0.139714	-0.006096098\\
0.139814	-0.005836684\\
0.139914	-0.00630505\\
0.140014	-0.006249804\\
0.140114	-0.004074972\\
0.140214	-0.005702086\\
0.140314	-0.004315887\\
0.140414	-0.003929936\\
0.1405141	-0.004239153\\
0.1406141	-0.002697748\\
0.1407141	-0.002630279\\
0.1408141	-0.001062603\\
0.1409141	-0.00165108\\
0.1410141	-0.004166495\\
0.1411141	-0.005669584\\
0.1412141	-0.005309355\\
0.1413141	-0.004063678\\
0.1414141	-0.006263586\\
0.1415142	-0.006518687\\
0.1416142	-0.007317928\\
0.1417142	-0.007334219\\
0.1418142	-0.006329047\\
0.1419142	-0.005331982\\
0.1420142	-0.001907862\\
0.1421142	-0.0002772545\\
0.1422142	-0.0007938264\\
0.1423142	-0.0009626779\\
0.1424142	0.0009408894\\
0.1425143	-0.001170595\\
0.1426143	-0.003509715\\
0.1427143	-0.00613415\\
0.1428143	-0.006692504\\
0.1429143	-0.005344667\\
0.1430143	-0.005641025\\
0.1431143	-0.004282061\\
0.1432143	-0.005134131\\
0.1433143	-0.004680221\\
0.1434143	-0.004511351\\
0.1435144	-0.004562743\\
0.1436144	-0.003022746\\
0.1437144	-0.002792783\\
0.1438144	-0.00341976\\
0.1439144	-0.003357926\\
0.1440144	-0.005837524\\
0.1441144	-0.005151754\\
0.1442144	-0.003777573\\
0.1443144	-0.003107227\\
0.1444144	-0.00200001\\
0.1445145	-0.005144245\\
0.1446145	-0.002481087\\
0.1447145	-0.0006245195\\
0.1448145	-0.001376883\\
0.1449145	0.0001516928\\
0.1450145	0.0004069178\\
0.1451145	0.002060898\\
0.1452145	0.001674167\\
0.1453145	0.002011968\\
0.1454145	0.0009945166\\
0.1455146	-0.0002624394\\
0.1456146	0.001625202\\
0.1457146	0.0004105846\\
0.1458146	0.0004284644\\
0.1459146	-0.0007071464\\
0.1460146	-0.002076699\\
0.1461146	-0.002484855\\
0.1462146	-0.002146452\\
0.1463146	-0.003938187\\
0.1464146	-0.002115491\\
0.1465147	-0.002976959\\
0.1466147	-0.003479285\\
0.1467147	-0.000232528\\
0.1468147	-9.351901e-05\\
0.1469147	0.001726377\\
0.1470147	0.00110298\\
0.1471147	0.001791239\\
0.1472147	0.004575124\\
0.1473147	0.003998802\\
0.1474147	0.003803187\\
0.1475148	0.002266697\\
0.1476148	0.002681143\\
0.1477148	0.0006734077\\
0.1478148	-0.001301419\\
0.1479148	0.0005625798\\
0.1480148	0.001988028\\
0.1481148	0.00206473\\
0.1482148	0.001490907\\
0.1483148	0.00312068\\
0.1484148	0.003804113\\
0.1485149	0.003013968\\
0.1486149	0.002077839\\
0.1487149	0.001934369\\
0.1488149	0.001534122\\
0.1489149	0.001918304\\
0.1490149	0.004863268\\
0.1491149	0.004900897\\
0.1492149	0.004854526\\
0.1493149	0.004351911\\
0.1494149	0.003955397\\
0.149515	0.002724868\\
0.149615	0.002673878\\
0.149715	0.003571746\\
0.149815	0.003639889\\
0.149915	0.005015722\\
0.150015	0.003103714\\
0.150115	0.002353445\\
0.150215	0.00340467\\
0.150315	0.000640645\\
0.150415	0.0002746153\\
0.1505151	0.0008605176\\
0.1506151	0.002950685\\
0.1507151	0.003088338\\
0.1508151	0.002800114\\
0.1509151	0.003348406\\
0.1510151	0.004625886\\
0.1511151	0.004208872\\
0.1512151	0.00248256\\
0.1513151	0.002603508\\
0.1514151	0.003054606\\
0.1515152	0.003566632\\
0.1516152	0.002229194\\
0.1517152	0.003731172\\
0.1518152	0.00229308\\
0.1519152	0.002853974\\
0.1520152	0.002187356\\
0.1521152	0.003593173\\
0.1522152	0.003666836\\
0.1523152	0.002449968\\
0.1524152	0.003665361\\
0.1525153	0.005228095\\
0.1526153	0.005591912\\
0.1527153	0.005139767\\
0.1528153	0.004847484\\
0.1529153	0.002792014\\
0.1530153	0.002384723\\
0.1531153	0.002891866\\
0.1532153	0.003393365\\
0.1533153	0.00252004\\
0.1534153	-0.0001341044\\
0.1535154	0.0002246664\\
0.1536154	0.000255826\\
0.1537154	0.0005533063\\
0.1538154	-0.0008657223\\
0.1539154	0.000319153\\
0.1540154	0.0004585162\\
0.1541154	-9.020927e-06\\
0.1542154	-0.00106966\\
0.1543154	-0.002140155\\
0.1544154	-0.0003268033\\
0.1545155	-0.002254377\\
0.1546155	-0.0002971431\\
0.1547155	0.001081545\\
0.1548155	0.003536097\\
0.1549155	0.003300884\\
0.1550155	0.003959531\\
0.1551155	0.005187328\\
0.1552155	0.004088574\\
0.1553155	0.001823904\\
0.1554155	0.000303723\\
0.1555156	0.000388038\\
0.1556156	-0.0007149051\\
0.1557156	-0.001326236\\
0.1558156	-0.001857255\\
0.1559156	6.59543e-05\\
0.1560156	2.429958e-05\\
0.1561156	-0.001591816\\
0.1562156	-0.001563224\\
0.1563156	-0.002858217\\
0.1564156	-0.001117206\\
0.1565157	-0.002229443\\
0.1566157	0.0008639226\\
0.1567157	0.0001471482\\
0.1568157	0.0008527034\\
0.1569157	-0.002165524\\
0.1570157	-0.003032819\\
0.1571157	-0.001210206\\
0.1572157	-0.001392272\\
0.1573157	-0.0005389004\\
0.1574157	-0.003832213\\
0.1575158	-0.00391282\\
0.1576158	-0.004241399\\
0.1577158	-0.003275546\\
0.1578158	-0.001457689\\
0.1579158	0.0001272624\\
0.1580158	0.0006258002\\
0.1581158	0.0005035479\\
0.1582158	0.0006747591\\
0.1583158	-0.001284184\\
0.1584158	-0.002470834\\
0.1585159	-0.004402784\\
0.1586159	-0.006521286\\
0.1587159	-0.00482807\\
0.1588159	-0.004580564\\
0.1589159	-0.004367156\\
0.1590159	-0.003654462\\
0.1591159	-0.003884708\\
0.1592159	-0.002032802\\
0.1593159	-0.000369457\\
0.1594159	0.001383468\\
0.159516	0.0008760122\\
0.159616	0.0007027859\\
0.159716	0.002143807\\
0.159816	0.001896617\\
0.159916	-0.001443223\\
0.160016	-0.001264196\\
0.160116	-0.002068634\\
0.160216	-0.004402844\\
0.160316	-0.005361849\\
0.160416	-0.00466344\\
0.1605161	-0.003122354\\
0.1606161	-0.005634077\\
0.1607161	-0.005029023\\
0.1608161	-0.004858746\\
0.1609161	-0.002137119\\
0.1610161	-0.003603971\\
0.1611161	-0.002005342\\
0.1612161	-0.001969509\\
0.1613161	-0.002690573\\
0.1614161	-0.001490652\\
0.1615162	-0.002622947\\
0.1616162	0.0001985305\\
0.1617162	-0.001652862\\
0.1618162	-0.002041237\\
0.1619162	-0.002695189\\
0.1620162	-0.00334772\\
0.1621162	-0.004149672\\
0.1622162	-0.002636779\\
0.1623162	-0.0007789697\\
0.1624162	0.00022326\\
0.1625163	0.0005297959\\
0.1626163	0.001724864\\
0.1627163	0.001583564\\
0.1628163	-0.0003648636\\
0.1629163	-0.001225688\\
0.1630163	-0.0009077479\\
0.1631163	-0.002095741\\
0.1632163	-0.0007450674\\
0.1633163	0.0001970233\\
0.1634163	0.002428047\\
0.1635164	7.938488e-05\\
0.1636164	-0.0008408771\\
0.1637164	-1.757629e-05\\
0.1638164	-0.001325132\\
0.1639164	-0.002072043\\
0.1640164	-0.004943303\\
0.1641164	-0.003559089\\
0.1642164	-0.003382717\\
0.1643164	-0.001513915\\
0.1644164	-0.0004402856\\
0.1645165	-0.0008454223\\
0.1646165	-0.001336448\\
0.1647165	-0.0004292867\\
0.1648165	0.0009322368\\
0.1649165	0.002535114\\
0.1650165	0.001077692\\
0.1651165	0.001332373\\
0.1652165	0.0005280828\\
0.1653165	0.0009661835\\
0.1654165	0.001186532\\
0.1655166	0.001380219\\
0.1656166	0.0005947954\\
0.1657166	0.002409039\\
0.1658166	0.00330346\\
0.1659166	0.0008609645\\
0.1660166	0.001527372\\
0.1661166	0.0009991277\\
0.1662166	0.0008273606\\
0.1663166	-0.0009769594\\
0.1664166	-0.0006097279\\
0.1665167	-0.002115301\\
0.1666167	-0.001174919\\
0.1667167	0.001898223\\
0.1668167	0.003716358\\
0.1669167	0.001356196\\
0.1670167	0.002179287\\
0.1671167	0.003254659\\
0.1672167	0.004196654\\
0.1673167	0.003341764\\
0.1674167	0.00281159\\
0.1675168	0.0024503\\
0.1676168	0.002694816\\
0.1677168	0.001360779\\
0.1678168	0.00190352\\
0.1679168	3.528997e-05\\
0.1680168	-0.0006460257\\
0.1681168	-0.0002556295\\
0.1682168	-0.0003188461\\
0.1683168	0.0006358008\\
0.1684168	0.0009969649\\
0.1685169	0.0003564473\\
0.1686169	0.0008038345\\
0.1687169	0.0003335928\\
0.1688169	0.0006259953\\
0.1689169	0.001208889\\
0.1690169	0.002291517\\
0.1691169	0.001994207\\
0.1692169	0.001542406\\
0.1693169	0.001602351\\
0.1694169	0.002788716\\
0.169517	0.002724461\\
0.169617	0.003922862\\
0.169717	0.004167846\\
0.169817	0.005388437\\
0.169917	0.00554397\\
0.170017	0.002418417\\
0.170117	0.00195726\\
0.170217	0.001221132\\
0.170317	8.605004e-05\\
0.170417	0.001898823\\
0.1705171	0.002947605\\
0.1706171	0.001801433\\
0.1707171	0.0007356065\\
0.1708171	-0.0007156768\\
0.1709171	0.0003941458\\
0.1710171	0.002072124\\
0.1711171	0.0006477351\\
0.1712171	-0.0006609831\\
0.1713171	-0.0003017127\\
0.1714171	-0.001531734\\
0.1715172	-0.001330335\\
0.1716172	-0.0007041143\\
0.1717172	0.00118399\\
0.1718172	0.0007202189\\
0.1719172	0.0005613346\\
0.1720172	2.247118e-05\\
0.1721172	0.0008976092\\
0.1722172	0.002089587\\
0.1723172	0.001715185\\
0.1724172	0.002107149\\
0.1725173	0.003710057\\
0.1726173	0.004374217\\
0.1727173	0.00189321\\
0.1728173	0.0002233017\\
0.1729173	0.0005895296\\
0.1730173	-0.0001593006\\
0.1731173	-0.0008650975\\
0.1732173	-0.001828492\\
0.1733173	-0.001042325\\
0.1734173	-0.0009122719\\
0.1735174	-0.000205453\\
0.1736174	7.377059e-05\\
0.1737174	0.0007201481\\
0.1738174	0.0009636585\\
0.1739174	0.003807662\\
0.1740174	0.002929539\\
0.1741174	0.001964599\\
0.1742174	0.001223213\\
0.1743174	-0.0004820576\\
0.1744174	-0.003280121\\
0.1745175	-0.003671064\\
0.1746175	-0.00270919\\
0.1747175	-0.002975066\\
0.1748175	-0.002343341\\
0.1749175	-0.002507527\\
0.1750175	-0.002854407\\
0.1751175	-0.002208983\\
0.1752175	-3.878669e-05\\
0.1753175	0.001056576\\
0.1754175	0.0004804061\\
0.1755176	0.002338432\\
0.1756176	0.002884503\\
0.1757176	0.001435736\\
0.1758176	-0.0001829088\\
0.1759176	-0.0007147084\\
0.1760176	-0.002814917\\
0.1761176	-0.003804537\\
0.1762176	-0.002987131\\
0.1763176	-0.002482822\\
0.1764176	-0.004554612\\
0.1765177	-0.00281418\\
0.1766177	-0.001835097\\
0.1767177	-0.0009511032\\
0.1768177	-0.001213568\\
0.1769177	0.0006707857\\
0.1770177	0.0002094259\\
0.1771177	0.001921811\\
0.1772177	0.001555207\\
0.1773177	-0.0008095438\\
0.1774177	0.0009861858\\
0.1775178	0.002410714\\
0.1776178	0.002473939\\
0.1777178	-0.002013392\\
0.1778178	-0.001813895\\
0.1779178	-0.002969298\\
0.1780178	-0.004817465\\
0.1781178	-0.003072532\\
0.1782178	-0.002585723\\
0.1783178	-0.001441365\\
0.1784178	-0.000495867\\
0.1785179	-0.0004091027\\
0.1786179	-0.003847491\\
0.1787179	-0.003887629\\
0.1788179	-0.003321058\\
0.1789179	-0.004318499\\
0.1790179	-0.005938744\\
0.1791179	-0.002862138\\
0.1792179	-0.0004657696\\
0.1793179	-0.0005773477\\
0.1794179	-0.0001042768\\
0.179518	0.00153418\\
0.179618	0.0005441544\\
0.179718	0.002166187\\
0.179818	0.000728536\\
0.179918	0.001360592\\
0.180018	0.001057743\\
0.180118	0.001225564\\
0.180218	0.0005180653\\
0.180318	-0.001843404\\
0.180418	-0.001701471\\
0.1805181	-0.0009419026\\
0.1806181	-0.0003086622\\
0.1807181	0.0003676372\\
0.1808181	-0.0004732075\\
0.1809181	-0.002793388\\
0.1810181	-0.002461226\\
0.1811181	-0.0006221405\\
0.1812181	-0.002200379\\
0.1813181	-0.002449704\\
0.1814181	-0.002235944\\
0.1815182	-0.0007145324\\
0.1816182	0.001862687\\
0.1817182	0.0001814793\\
0.1818182	-6.27023e-05\\
0.1819182	-0.001073006\\
0.1820182	-0.0007254852\\
0.1821182	-0.0006656787\\
0.1822182	-0.001028992\\
0.1823182	-0.002442245\\
0.1824182	-0.0004454093\\
0.1825183	0.001657975\\
0.1826183	0.002812665\\
0.1827183	-4.730181e-05\\
0.1828183	-0.0005253527\\
0.1829183	-0.0002078626\\
0.1830183	-0.0002119992\\
0.1831183	-0.001071809\\
0.1832183	-7.550165e-05\\
0.1833183	0.0008006963\\
0.1834183	0.001477221\\
0.1835184	0.0009397915\\
0.1836184	0.001016394\\
0.1837184	-0.0003521804\\
0.1838184	0.001142194\\
0.1839184	0.001054822\\
0.1840184	0.0009949411\\
0.1841184	-2.058587e-05\\
0.1842184	0.000113548\\
0.1843184	9.96268e-05\\
0.1844184	0.001264821\\
0.1845185	0.001491529\\
0.1846185	0.001130579\\
0.1847185	0.001726598\\
0.1848185	0.002326184\\
0.1849185	0.001139522\\
0.1850185	0.001153374\\
0.1851185	0.002316881\\
0.1852185	0.001026141\\
0.1853185	-0.0006979319\\
0.1854185	-0.001583646\\
0.1855186	-0.002381433\\
0.1856186	-0.001578419\\
0.1857186	0.0002661371\\
0.1858186	0.0005438259\\
0.1859186	5.448989e-06\\
0.1860186	0.001109491\\
0.1861186	0.001831828\\
0.1862186	0.001583387\\
0.1863186	0.001466851\\
0.1864186	0.0004734211\\
0.1865187	0.001048968\\
0.1866187	-0.001482982\\
0.1867187	-0.0004398782\\
0.1868187	0.001401885\\
0.1869187	0.00168467\\
0.1870187	0.001195903\\
0.1871187	0.001270866\\
0.1872187	0.001592489\\
0.1873187	0.004639796\\
0.1874187	0.003590634\\
0.1875188	0.002694186\\
0.1876188	0.0008237635\\
0.1877188	0.0006761105\\
0.1878188	-0.0007881697\\
0.1879188	0.001077609\\
0.1880188	0.002629927\\
0.1881188	0.003368878\\
0.1882188	0.00266646\\
0.1883188	0.00255359\\
0.1884188	0.002339889\\
0.1885189	0.001019989\\
0.1886189	-0.001144152\\
0.1887189	-0.003993535\\
0.1888189	-0.003378674\\
0.1889189	-0.002291528\\
0.1890189	-0.003222224\\
0.1891189	-0.001719767\\
0.1892189	-0.002257491\\
0.1893189	-0.001833389\\
0.1894189	0.001091582\\
0.189519	0.0008594095\\
0.189619	0.0005444162\\
0.189719	0.003447932\\
0.189819	0.003527051\\
0.189919	0.005564993\\
0.190019	0.005036718\\
0.190119	0.005026928\\
0.190219	0.001954089\\
0.190319	0.0005866155\\
0.190419	-0.0006093805\\
0.1905191	-0.002273053\\
0.1906191	-0.001637027\\
0.1907191	-0.001869371\\
0.1908191	-0.001883903\\
0.1909191	-0.002226704\\
0.1910191	-0.001184259\\
0.1911191	-9.365499e-05\\
0.1912191	0.00122881\\
0.1913191	0.001205517\\
0.1914191	0.001256725\\
0.1915192	0.001936691\\
0.1916192	0.002275483\\
0.1917192	0.001282952\\
0.1918192	-0.0006045151\\
0.1919192	-0.0006535804\\
0.1920192	0.0006891222\\
0.1921192	-0.001031851\\
0.1922192	-0.001697907\\
0.1923192	-0.001749983\\
0.1924192	-0.000981364\\
0.1925193	-0.000862345\\
0.1926193	-0.001659489\\
0.1927193	-0.0006662874\\
0.1928193	0.002510838\\
0.1929193	0.002308174\\
0.1930193	-0.0007736076\\
0.1931193	-0.00263513\\
0.1932193	-0.002859054\\
0.1933193	-0.0005468382\\
0.1934193	0.0003207565\\
0.1935194	-0.002549558\\
0.1936194	-0.001465776\\
0.1937194	-0.0009251579\\
0.1938194	-0.002390601\\
0.1939194	-0.002122131\\
0.1940194	-0.0009989574\\
0.1941194	0.001525452\\
0.1942194	0.002261624\\
0.1943194	0.003458622\\
0.1944194	0.001452779\\
0.1945195	0.002279749\\
0.1946195	0.001132872\\
0.1947195	-0.0009863548\\
0.1948195	-0.002424865\\
0.1949195	-0.001090535\\
0.1950195	-0.0001699144\\
0.1951195	-0.0008535325\\
0.1952195	-0.001974234\\
0.1953195	-0.0003133633\\
0.1954195	-9.914302e-05\\
0.1955196	-0.0006304274\\
0.1956196	-0.001195671\\
0.1957196	-0.00184425\\
0.1958196	-0.0005637949\\
0.1959196	-0.001807044\\
0.1960196	-0.001439008\\
0.1961196	-0.0008828719\\
0.1962196	-0.002417242\\
0.1963196	-0.001832228\\
0.1964196	-0.002086971\\
0.1965197	-0.002607299\\
0.1966197	-0.001595324\\
0.1967197	-0.002239746\\
0.1968197	-0.001703685\\
0.1969197	0.0004217366\\
0.1970197	0.0005587028\\
0.1971197	-0.0008338749\\
0.1972197	0.001213551\\
0.1973197	0.001403723\\
0.1974197	0.002014904\\
0.1975198	0.002351623\\
0.1976198	0.0008372588\\
0.1977198	-0.0002511064\\
0.1978198	-0.0008862805\\
0.1979198	0.0002322029\\
0.1980198	-0.0002073874\\
0.1981198	-0.00103404\\
0.1982198	0.000796566\\
0.1983198	-0.0005124131\\
0.1984198	-0.0001412455\\
0.1985199	-0.0004102691\\
0.1986199	-0.0004442189\\
0.1987199	0.0004783706\\
0.1988199	-0.001454203\\
0.1989199	-0.002514393\\
0.1990199	-0.002802756\\
0.1991199	-0.001303431\\
0.1992199	-0.001101341\\
0.1993199	-8.391323e-05\\
0.1994199	-0.001720949\\
0.19952	-0.001973722\\
0.19962	-0.001033857\\
0.19972	-0.0001323427\\
0.19982	0.001149799\\
0.19992	0.00104587\\
0.20002	0.0003027992\\
0.20012	0.0008660448\\
0.20022	0.001797225\\
0.20032	0.0006451922\\
0.20042	-0.0004802411\\
0.2005201	0.0005832864\\
0.2006201	0.0005883432\\
0.2007201	0.0007725425\\
0.2008201	0.0008375568\\
0.2009201	0.0005333004\\
0.2010201	-0.0002358774\\
0.2011201	-3.995476e-05\\
0.2012201	-0.001319826\\
0.2013201	-0.001337417\\
0.2014201	-0.002524224\\
0.2015202	-0.0004340628\\
0.2016202	-0.0006100729\\
0.2017202	-0.001401966\\
0.2018202	0.0009731295\\
0.2019202	0.00265594\\
0.2020202	0.00246161\\
0.2021202	0.002212361\\
0.2022202	0.003660361\\
0.2023202	0.004285048\\
0.2024202	0.002310933\\
0.2025203	0.0008859091\\
0.2026203	0.0007076773\\
0.2027203	-0.000657669\\
0.2028203	-0.0004957923\\
0.2029203	-0.003064757\\
0.2030203	-0.001760481\\
0.2031203	-0.0001065179\\
0.2032203	-0.0007552059\\
0.2033203	-0.002551372\\
0.2034203	-0.002539952\\
0.2035204	-0.003214353\\
0.2036204	-0.002273045\\
0.2037204	-0.0003207364\\
0.2038204	0.0008459021\\
0.2039204	0.002261377\\
0.2040204	0.002723699\\
0.2041204	0.001904314\\
0.2042204	0.00353606\\
0.2043204	0.00444044\\
0.2044204	0.002018916\\
0.2045205	0.0006455437\\
0.2046205	-1.289652e-05\\
0.2047205	0.0003354565\\
0.2048205	0.0007103843\\
0.2049205	0.00136253\\
0.2050205	0.0004561528\\
0.2051205	0.001127423\\
0.2052205	0.002102478\\
0.2053205	0.001056453\\
0.2054205	-0.0005620119\\
0.2055206	-0.001486935\\
0.2056206	-0.0006811257\\
0.2057206	-0.0003286123\\
0.2058206	0.001776393\\
0.2059206	0.003854927\\
0.2060206	0.000985244\\
0.2061206	0.0003577264\\
0.2062206	0.0002810282\\
0.2063206	-0.001570533\\
0.2064206	-0.002367338\\
0.2065207	-0.001295857\\
0.2066207	-0.00222221\\
0.2067207	-0.003073153\\
0.2068207	-0.00208331\\
0.2069207	-0.001750999\\
0.2070207	0.0009920229\\
0.2071207	0.00136061\\
0.2072207	0.001187078\\
0.2073207	0.004208629\\
0.2074207	0.005269918\\
0.2075208	0.003359777\\
0.2076208	0.001557405\\
0.2077208	-0.0005531864\\
0.2078208	-0.001014689\\
0.2079208	-0.001377908\\
0.2080208	-0.001893387\\
0.2081208	-0.0001816026\\
0.2082208	-0.0002537151\\
0.2083208	-0.0004569991\\
0.2084208	-0.000211845\\
0.2085209	0.001040776\\
0.2086209	0.003762345\\
0.2087209	0.00288153\\
0.2088209	0.001033613\\
0.2089209	9.133956e-05\\
0.2090209	-0.00159153\\
0.2091209	-0.00143554\\
0.2092209	-0.001920658\\
0.2093209	-0.000253177\\
0.2094209	-0.0005685546\\
0.209521	-0.0007847903\\
0.209621	-0.001501092\\
0.209721	-0.002195993\\
0.209821	-0.0003670358\\
0.209921	0.0008558934\\
0.210021	0.001779864\\
0.210121	0.0015909\\
0.210221	0.0008269105\\
0.210321	0.001609546\\
0.210421	0.0005553724\\
0.2105211	-0.0008221706\\
0.2106211	-0.002032082\\
0.2107211	-0.001493516\\
0.2108211	-0.0009856136\\
0.2109211	-0.0007646031\\
0.2110211	-0.0005866419\\
0.2111211	-0.00223354\\
0.2112211	-0.002229405\\
0.2113211	-0.002743697\\
0.2114211	-0.001745531\\
0.2115212	-0.0009003351\\
0.2116212	0.001860113\\
0.2117212	0.002550453\\
0.2118212	0.003029453\\
0.2119212	0.002817692\\
0.2120212	0.001541736\\
0.2121212	-9.842506e-05\\
0.2122212	-0.001882633\\
0.2123212	0.0001634253\\
0.2124212	-0.0006217487\\
0.2125213	0.0008992771\\
0.2126213	0.001720851\\
0.2127213	0.0004191458\\
0.2128213	-5.970464e-05\\
0.2129213	0.0005956537\\
0.2130213	0.000146117\\
0.2131213	-0.0003621664\\
0.2132213	-0.001403231\\
0.2133213	-0.002231866\\
0.2134213	-0.002394299\\
0.2135214	-0.003467263\\
0.2136214	-0.003443256\\
0.2137214	-0.001988169\\
0.2138214	-0.002211756\\
0.2139214	-0.003436492\\
0.2140214	-0.001468677\\
0.2141214	-0.00126323\\
0.2142214	0.000131527\\
0.2143214	6.157229e-05\\
0.2144214	0.001648821\\
0.2145215	0.00350192\\
0.2146215	0.001842324\\
0.2147215	0.001505517\\
0.2148215	0.003465915\\
0.2149215	0.00277039\\
0.2150215	0.0009051415\\
0.2151215	-0.0001185797\\
0.2152215	-0.001647897\\
0.2153215	-0.0010293\\
0.2154215	-0.0008721187\\
0.2155216	-0.001635518\\
0.2156216	-0.001340388\\
0.2157216	-0.0006351808\\
0.2158216	-0.001735637\\
0.2159216	0.0007844202\\
0.2160216	0.0007764856\\
0.2161216	-0.0009273096\\
0.2162216	-0.002379738\\
0.2163216	-0.00139439\\
0.2164216	0.001379718\\
0.2165217	0.000817505\\
0.2166217	0.001251522\\
0.2167217	0.001365895\\
0.2168217	-0.0004040765\\
0.2169217	0.000205602\\
0.2170217	-8.728325e-05\\
0.2171217	-0.001559702\\
0.2172217	-0.002589062\\
0.2173217	-0.001785533\\
0.2174217	-0.0004172733\\
0.2175218	0.0006632596\\
0.2176218	-0.001158935\\
0.2177218	-0.001154029\\
0.2178218	0.001038979\\
0.2179218	-0.000373802\\
0.2180218	-0.0001148966\\
0.2181218	-0.0002033285\\
0.2182218	0.001587986\\
0.2183218	0.002462874\\
0.2184218	0.001786642\\
0.2185219	0.0008005494\\
0.2186219	0.002101333\\
0.2187219	-0.0002750239\\
0.2188219	-0.001773722\\
0.2189219	-0.0001490183\\
0.2190219	5.642273e-05\\
0.2191219	-0.001226506\\
0.2192219	-0.001157646\\
0.2193219	-0.0008272075\\
0.2194219	0.0005162221\\
0.219522	0.0008607081\\
0.219622	-0.001004138\\
0.219722	0.001748135\\
0.219822	0.002779537\\
0.219922	0.001605754\\
0.220022	0.002134162\\
0.220122	0.002952162\\
0.220222	0.002116797\\
0.220322	0.0003754184\\
0.220422	-0.003004059\\
0.2205221	-0.003562019\\
0.2206221	-0.001704722\\
0.2207221	-0.001247947\\
0.2208221	-0.0004960549\\
0.2209221	-0.0009575668\\
0.2210221	-0.0009761461\\
0.2211221	-0.001052009\\
0.2212221	-0.000910057\\
0.2213221	0.0002194795\\
0.2214221	-0.001059391\\
0.2215222	8.384058e-05\\
0.2216222	0.0005313135\\
0.2217222	0.001521091\\
0.2218222	0.001480538\\
0.2219222	0.0005639029\\
0.2220222	0.002902885\\
0.2221222	0.002835563\\
0.2222222	0.0002885955\\
0.2223222	0.002299655\\
0.2224222	-0.0001685256\\
0.2225223	0.0002435475\\
0.2226223	0.001053302\\
0.2227223	0.001315119\\
0.2228223	0.001930425\\
0.2229223	0.0004753145\\
0.2230223	0.0008871332\\
0.2231223	0.001612101\\
0.2232223	0.0001194033\\
0.2233223	-0.0004737706\\
0.2234223	-0.0004373596\\
0.2235224	-0.00122373\\
0.2236224	-0.0009471333\\
0.2237224	-0.002020491\\
0.2238224	-0.002888654\\
0.2239224	-0.002230489\\
0.2240224	-0.002203523\\
0.2241224	-0.003398193\\
0.2242224	-0.001332881\\
0.2243224	-7.496695e-05\\
0.2244224	0.001886836\\
0.2245225	0.002112052\\
0.2246225	0.003381171\\
0.2247225	0.003553636\\
0.2248225	0.003700863\\
0.2249225	0.001192799\\
0.2250225	0.0004093498\\
0.2251225	0.0004885439\\
0.2252225	1.37124e-05\\
0.2253225	-0.0001122789\\
0.2254225	-0.001399263\\
0.2255226	-0.0007510417\\
0.2256226	-0.002441163\\
0.2257226	-0.001161953\\
0.2258226	-0.0007851118\\
0.2259226	-0.0007195869\\
0.2260226	0.001457142\\
0.2261226	0.0007514673\\
0.2262226	0.001369\\
0.2263226	0.001993891\\
0.2264226	0.0001088954\\
0.2265227	0.0006827203\\
0.2266227	0.0001739406\\
0.2267227	-0.0008234683\\
0.2268227	7.604361e-05\\
0.2269227	-0.001024304\\
0.2270227	0.0001084464\\
0.2271227	-0.0004680174\\
0.2272227	-0.001764055\\
0.2273227	2.92788e-05\\
0.2274227	0.0008745793\\
0.2275228	0.001601539\\
0.2276228	0.003493956\\
0.2277228	0.001882316\\
0.2278228	0.0003140678\\
0.2279228	-0.003014\\
0.2280228	-0.005187966\\
0.2281228	-0.003761201\\
0.2282228	-0.003698596\\
0.2283228	-0.001462345\\
0.2284228	-0.001798623\\
0.2285229	0.0004897395\\
0.2286229	0.0001255203\\
0.2287229	0.001812735\\
0.2288229	0.003402532\\
0.2289229	0.003677457\\
0.2290229	0.00118234\\
0.2291229	0.0002431565\\
0.2292229	0.0008688039\\
0.2293229	-3.86706e-05\\
0.2294229	-0.001292147\\
0.229523	-0.0006495243\\
0.229623	0.0007655376\\
0.229723	-0.0009356017\\
0.229823	0.001533686\\
0.229923	-0.0003177789\\
0.230023	-0.000708393\\
0.230123	-0.0003987042\\
0.230223	-0.0008484147\\
0.230323	0.0005878796\\
0.230423	0.0001969138\\
0.2305231	-0.0007567344\\
0.2306231	0.001955811\\
0.2307231	0.002252127\\
0.2308231	0.0001294474\\
0.2309231	-0.001443583\\
0.2310231	-0.0007873145\\
0.2311231	-0.00119825\\
0.2312231	-0.003019461\\
0.2313231	-0.004455934\\
0.2314231	-0.003994267\\
0.2315232	-0.002616246\\
0.2316232	-0.002007276\\
0.2317232	0.0005298648\\
0.2318232	0.001698643\\
0.2319232	0.0005179168\\
0.2320232	0.0004020955\\
0.2321232	0.001216884\\
0.2322232	0.001665962\\
0.2323232	0.002268089\\
0.2324232	0.003174518\\
0.2325233	0.001874137\\
0.2326233	0.0002900939\\
0.2327233	0.0009736716\\
0.2328233	0.001390603\\
0.2329233	0.001164269\\
0.2330233	-0.001729182\\
0.2331233	-0.0009187303\\
0.2332233	0.0004259662\\
0.2333233	0.0008834084\\
0.2334233	-0.001782005\\
0.2335234	-0.002333113\\
0.2336234	-0.001774204\\
0.2337234	-0.001850344\\
0.2338234	-0.001662204\\
0.2339234	-0.001312403\\
0.2340234	-0.001386435\\
0.2341234	0.0008434827\\
0.2342234	0.001678711\\
0.2343234	0.001649304\\
0.2344234	0.002130089\\
0.2345235	-0.0001621263\\
0.2346235	-0.0006963649\\
0.2347235	-0.0001818749\\
0.2348235	0.0002967759\\
0.2349235	-0.0006885367\\
0.2350235	-0.0002141046\\
0.2351235	-0.0001734948\\
0.2352235	0.0003305517\\
0.2353235	-0.0008799461\\
0.2354235	-0.001587743\\
0.2355236	0.0009539597\\
0.2356236	0.002164461\\
0.2357236	-0.0003108278\\
0.2358236	-0.001889399\\
0.2359236	0.0001024141\\
0.2360236	0.001075541\\
0.2361236	0.00156154\\
0.2362236	-0.001292868\\
0.2363236	-0.002059595\\
0.2364236	0.000415195\\
0.2365237	5.035785e-05\\
0.2366237	0.0002042022\\
0.2367237	-6.705518e-05\\
0.2368237	-2.302669e-05\\
0.2369237	0.001426147\\
0.2370237	0.002034499\\
0.2371237	0.002838011\\
0.2372237	0.001240157\\
0.2373237	0.001114735\\
0.2374237	0.002785291\\
0.2375238	0.001256141\\
0.2376238	0.001372281\\
0.2377238	1.206498e-05\\
0.2378238	-0.00131515\\
0.2379238	-0.003250396\\
0.2380238	-0.004357749\\
0.2381238	-0.002257159\\
0.2382238	-0.002960978\\
0.2383238	-0.00150749\\
0.2384238	-0.0004994301\\
0.2385239	-0.0006221617\\
0.2386239	0.0001280868\\
0.2387239	0.0001562563\\
0.2388239	0.0009933118\\
0.2389239	-0.0002586496\\
0.2390239	0.0009222755\\
0.2391239	0.001436734\\
0.2392239	0.001848105\\
0.2393239	0.003182107\\
0.2394239	0.003396673\\
0.239524	0.0002299653\\
0.239624	-0.0003970944\\
0.239724	-0.0007676582\\
0.239824	-0.0007137598\\
0.239924	-0.0001984528\\
0.240024	-0.000788494\\
0.240124	0.002643046\\
0.240224	0.001384476\\
0.240324	0.0001925755\\
0.240424	6.729669e-05\\
0.2405241	0.001129035\\
0.2406241	0.001994225\\
0.2407241	-0.0008770685\\
0.2408241	-0.001004314\\
0.2409241	-0.001399957\\
0.2410241	-0.0004389508\\
0.2411241	0.0005063852\\
0.2412241	-0.0009000089\\
0.2413241	-0.0002152734\\
0.2414241	-0.00080141\\
0.2415242	-0.001593677\\
0.2416242	-0.002408385\\
0.2417242	-0.003356553\\
0.2418242	-0.0003902098\\
0.2419242	0.00138837\\
0.2420242	0.0007019232\\
0.2421242	0.002912809\\
0.2422242	0.001012156\\
0.2423242	0.0002904007\\
0.2424242	0.0008096154\\
0.2425243	-0.0001822554\\
0.2426243	0.0007758951\\
0.2427243	-0.0003132802\\
0.2428243	0.000439115\\
0.2429243	0.0002542852\\
0.2430243	-0.001203154\\
0.2431243	0.0003692891\\
0.2432243	-0.0001471127\\
0.2433243	0.001290946\\
0.2434243	0.002994214\\
0.2435244	0.003092807\\
0.2436244	0.001919666\\
0.2437244	-0.001143855\\
0.2438244	-0.002692478\\
0.2439244	-0.003026699\\
0.2440244	-0.002838347\\
0.2441244	-0.0009629185\\
0.2442244	-0.0009449401\\
0.2443244	-0.001712459\\
0.2444244	-0.0003762403\\
0.2445245	0.0004476783\\
0.2446245	0.0005996841\\
0.2447245	0.002569721\\
0.2448245	0.001951847\\
0.2449245	0.003547675\\
0.2450245	0.003356594\\
0.2451245	0.002521247\\
0.2452245	0.0005915974\\
0.2453245	-0.002421857\\
0.2454245	-0.002489331\\
0.2455246	-0.004040297\\
0.2456246	-0.002980261\\
0.2457246	-0.002282543\\
0.2458246	-0.002482152\\
0.2459246	-0.002556822\\
0.2460246	-0.001945102\\
0.2461246	-0.001075184\\
0.2462246	0.001173755\\
0.2463246	0.0006772029\\
0.2464246	0.002285209\\
0.2465247	0.002257579\\
0.2466247	0.002654576\\
0.2467247	0.002349679\\
0.2468247	0.0009905337\\
0.2469247	0.002155985\\
0.2470247	0.0003234449\\
0.2471247	-0.0009165821\\
0.2472247	-0.001175667\\
0.2473247	-0.0006132573\\
0.2474247	0.0009488939\\
0.2475248	0.0009876515\\
0.2476248	0.001984547\\
0.2477248	0.0005358468\\
0.2478248	-0.0001442152\\
0.2479248	0.001130554\\
0.2480248	-0.001928852\\
0.2481248	-0.00279143\\
0.2482248	-0.002009548\\
0.2483248	-0.0009412398\\
0.2484248	-0.001253542\\
0.2485249	-0.002274398\\
0.2486249	-0.001073339\\
0.2487249	0.0001527323\\
0.2488249	-0.001486054\\
0.2489249	-0.001722357\\
0.2490249	-0.0008492891\\
0.2491249	0.0005970937\\
0.2492249	0.0009853035\\
0.2493249	0.0006932373\\
0.2494249	0.0007210453\\
0.249525	0.001616882\\
0.249625	0.0007606142\\
0.249725	3.639786e-07\\
0.249825	-0.0004525142\\
0.249925	0.0008824529\\
0.250025	0.00363285\\
0.250125	0.001859801\\
0.250225	0.001858903\\
0.250325	0.0002502849\\
0.250425	-0.000962136\\
0.2505251	-0.001726465\\
0.2506251	-0.001759969\\
0.2507251	-0.001997433\\
0.2508251	-0.001536878\\
0.2509251	0.0009677916\\
0.2510251	-0.0005271457\\
0.2511251	-0.0007586574\\
0.2512251	0.0008147309\\
0.2513251	0.002444378\\
0.2514251	0.00173525\\
0.2515252	-0.001283248\\
0.2516252	-0.002022951\\
0.2517252	-0.001208473\\
0.2518252	0.0009420282\\
0.2519252	0.00195221\\
0.2520252	0.0003398277\\
0.2521252	-0.001763984\\
0.2522252	0.0009261444\\
0.2523252	0.000408711\\
0.2524252	-0.0008823193\\
0.2525253	-0.001628098\\
0.2526253	-0.001533327\\
0.2527253	-0.002267596\\
0.2528253	-0.001880465\\
0.2529253	9.321714e-05\\
0.2530253	0.001131325\\
0.2531253	0.002330765\\
0.2532253	0.002354986\\
0.2533253	0.002014792\\
0.2534253	0.002027143\\
0.2535254	0.002821959\\
0.2536254	-0.0004902862\\
0.2537254	-0.003315525\\
0.2538254	-0.001466144\\
0.2539254	-0.001022539\\
0.2540254	-0.002408894\\
0.2541254	-0.001460907\\
0.2542254	0.0005868816\\
0.2543254	0.001202237\\
0.2544254	0.0007611881\\
0.2545255	0.002107041\\
0.2546255	0.002495361\\
0.2547255	0.001975603\\
0.2548255	0.0001071276\\
0.2549255	0.001941476\\
0.2550255	-0.000349757\\
0.2551255	0.0006419391\\
0.2552255	0.001434012\\
0.2553255	-0.001301649\\
0.2554255	-0.0007761312\\
0.2555256	0.0005030974\\
0.2556256	-0.0005572612\\
0.2557256	-0.00342031\\
0.2558256	-0.003121402\\
0.2559256	-0.001436419\\
0.2560256	-0.002647719\\
0.2561256	-0.0009022266\\
0.2562256	-0.0006305481\\
0.2563256	0.0007024008\\
0.2564256	0.001088212\\
0.2565257	0.0004729741\\
0.2566257	-0.001521576\\
0.2567257	-0.0001152808\\
0.2568257	0.002159057\\
0.2569257	0.0009179619\\
0.2570257	0.002333008\\
0.2571257	0.001339464\\
0.2572257	0.002518252\\
0.2573257	0.0008167053\\
0.2574257	0.00201085\\
0.2575258	0.001473159\\
0.2576258	0.001296561\\
0.2577258	0.0009043335\\
0.2578258	0.001375748\\
0.2579258	0.0007088135\\
0.2580258	-0.0009813908\\
0.2581258	-0.0009594621\\
0.2582258	-0.00286537\\
0.2583258	-0.003248088\\
0.2584258	-0.001629293\\
0.2585259	-0.001902659\\
0.2586259	-0.001448968\\
0.2587259	-0.001240628\\
0.2588259	-0.001018787\\
0.2589259	-0.0009940997\\
0.2590259	0.0009135079\\
0.2591259	0.002106182\\
0.2592259	0.001541666\\
0.2593259	0.001927423\\
0.2594259	0.0005149736\\
0.259526	0.001182216\\
0.259626	0.001823821\\
0.259726	0.0003209115\\
0.259826	-0.0007758637\\
0.259926	-0.0008726287\\
0.260026	-0.0009751452\\
0.260126	-0.001062981\\
0.260226	-0.0007439466\\
0.260326	-0.0007168441\\
0.260426	-0.001134601\\
0.2605261	-0.001409467\\
0.2606261	-0.0005708684\\
0.2607261	0.001017365\\
0.2608261	0.003277632\\
0.2609261	0.003352668\\
0.2610261	0.00220702\\
0.2611261	0.001408133\\
0.2612261	-0.0004451142\\
0.2613261	-0.001978115\\
0.2614261	-0.00229583\\
0.2615262	-0.000857283\\
0.2616262	-0.0007077487\\
0.2617262	-0.0007569275\\
0.2618262	-0.00087995\\
0.2619262	-0.0001540014\\
0.2620262	0.0006589712\\
0.2621262	0.001266901\\
0.2622262	0.001797027\\
0.2623262	0.0006422874\\
0.2624262	0.002218887\\
0.2625263	0.002007973\\
0.2626263	-0.0002597078\\
0.2627263	-0.0007286727\\
0.2628263	-0.00221054\\
0.2629263	-0.001373113\\
0.2630263	-0.003166043\\
0.2631263	-0.003166137\\
0.2632263	-0.002298875\\
0.2633263	-0.002612052\\
0.2634263	0.0003501922\\
0.2635264	0.0002059214\\
0.2636264	0.0007176609\\
0.2637264	0.002879447\\
0.2638264	0.002142241\\
0.2639264	0.001194985\\
0.2640264	0.001788293\\
0.2641264	0.001468275\\
0.2642264	0.0003752139\\
0.2643264	-0.002124063\\
0.2644264	-0.0005253256\\
0.2645265	0.0006430619\\
0.2646265	-0.0001038945\\
0.2647265	-0.0004590687\\
0.2648265	-0.0007538474\\
0.2649265	0.00147942\\
0.2650265	0.002359196\\
0.2651265	0.002808783\\
0.2652265	0.000631621\\
0.2653265	0.0007698917\\
0.2654265	-0.001312005\\
0.2655266	-0.002399118\\
0.2656266	-0.000761615\\
0.2657266	-0.001226285\\
0.2658266	0.0005361997\\
0.2659266	-0.0005544369\\
0.2660266	-0.0001629642\\
0.2661266	-0.0006987812\\
0.2662266	-0.001936714\\
0.2663266	-0.002533716\\
0.2664266	-0.003245479\\
0.2665267	-0.001356619\\
0.2666267	-0.0001834954\\
0.2667267	0.001170235\\
0.2668267	-0.0007818913\\
0.2669267	0.0004669843\\
0.2670267	0.0005036356\\
0.2671267	0.002554717\\
0.2672267	0.001651467\\
0.2673267	0.0007578209\\
0.2674267	0.001620237\\
0.2675268	0.0008340872\\
0.2676268	0.003297007\\
0.2677268	0.002468315\\
0.2678268	0.002659394\\
0.2679268	0.0002040514\\
0.2680268	0.0002322088\\
0.2681268	-0.001249505\\
0.2682268	-0.003559278\\
0.2683268	-0.004020802\\
0.2684268	-0.002285659\\
0.2685269	-0.0007210167\\
0.2686269	-0.001891754\\
0.2687269	-0.000429861\\
0.2688269	2.326858e-05\\
0.2689269	0.0009939682\\
0.2690269	0.0008420014\\
0.2691269	0.001235086\\
0.2692269	0.001506906\\
0.2693269	0.001498181\\
0.2694269	0.001262033\\
0.269527	-7.162033e-05\\
0.269627	0.002017895\\
0.269727	-0.0001055586\\
0.269827	-0.001553494\\
0.269927	-0.001717732\\
0.270027	-0.0003317458\\
0.270127	-0.0002803295\\
0.270227	-0.001343111\\
0.270327	0.0003651027\\
0.270427	-0.0001331007\\
0.2705271	-2.722943e-05\\
0.2706271	-0.001889634\\
0.2707271	-0.000760784\\
0.2708271	-0.0006257916\\
0.2709271	0.0001783639\\
0.2710271	0.0008272112\\
0.2711271	-0.000699361\\
0.2712271	0.00108646\\
0.2713271	0.0012152\\
0.2714271	0.001584577\\
0.2715272	-0.0003828794\\
0.2716272	0.0001222871\\
0.2717272	0.0002726675\\
0.2718272	0.001226813\\
0.2719272	0.002772262\\
0.2720272	0.00130929\\
0.2721272	0.001089788\\
0.2722272	0.0005254804\\
0.2723272	9.085463e-05\\
0.2724272	0.001838914\\
0.2725273	7.613e-05\\
0.2726273	-0.001248555\\
0.2727273	-0.002178977\\
0.2728273	-0.002066386\\
0.2729273	-0.00334365\\
0.2730273	-0.002014582\\
0.2731273	-0.001510396\\
0.2732273	-0.0004506573\\
0.2733273	0.001242993\\
0.2734273	-0.0004034082\\
0.2735274	0.001186381\\
0.2736274	0.001191729\\
0.2737274	0.001651414\\
0.2738274	0.0001917306\\
0.2739274	1.09861e-05\\
0.2740274	-0.001650709\\
0.2741274	-0.003304307\\
0.2742274	-0.0003455276\\
0.2743274	0.0007746113\\
0.2744274	0.0001036849\\
0.2745275	0.0002907737\\
0.2746275	0.00234632\\
0.2747275	0.001233454\\
0.2748275	0.002093498\\
0.2749275	0.001847192\\
0.2750275	0.0009537757\\
0.2751275	0.002320842\\
0.2752275	0.00072747\\
0.2753275	0.0008443229\\
0.2754275	-2.852249e-05\\
0.2755276	-0.0009434757\\
0.2756276	-0.001697789\\
0.2757276	-0.0009857275\\
0.2758276	-0.001534056\\
0.2759276	0.0004829269\\
0.2760276	-0.001052853\\
0.2761276	-0.0009631136\\
0.2762276	-0.001684313\\
0.2763276	-0.002499847\\
0.2764276	-0.00300444\\
0.2765277	-0.001201173\\
0.2766277	0.00108621\\
0.2767277	0.002680895\\
0.2768277	0.003624591\\
0.2769277	0.003241879\\
0.2770277	0.001464492\\
0.2771277	-0.001192683\\
0.2772277	-0.001988198\\
0.2773277	-0.003742951\\
0.2774277	-0.002034785\\
0.2775278	-7.098086e-05\\
0.2776278	0.0017583\\
0.2777278	0.0004109748\\
0.2778278	0.0007919324\\
0.2779278	0.00187614\\
0.2780278	0.001098013\\
0.2781278	0.0004621814\\
0.2782278	0.001281479\\
0.2783278	0.001687303\\
0.2784278	0.0003936172\\
0.2785279	-0.001100899\\
0.2786279	-0.002146453\\
0.2787279	-0.001218693\\
0.2788279	0.0002093002\\
0.2789279	-0.00105831\\
0.2790279	-0.001772643\\
0.2791279	-0.001655746\\
0.2792279	-0.0009438105\\
0.2793279	-0.0003524031\\
0.2794279	0.0006046608\\
0.279528	0.001891628\\
0.279628	0.003312051\\
0.279728	0.003405804\\
0.279828	0.002416554\\
0.279928	0.002342079\\
0.280028	0.0007368697\\
0.280128	-0.001764903\\
0.280228	-0.0003847578\\
0.280328	-0.0009999322\\
0.280428	-0.002529719\\
0.2805281	-0.003967191\\
0.2806281	-0.004282545\\
0.2807281	-0.004914261\\
0.2808281	-0.001167164\\
0.2809281	0.000610514\\
0.2810281	0.0006005916\\
0.2811281	0.0009657263\\
0.2812281	0.0005873575\\
0.2813281	0.001575342\\
0.2814281	0.002590981\\
0.2815282	0.0005951878\\
0.2816282	0.0007142736\\
0.2817282	0.001077976\\
0.2818282	0.001084648\\
0.2819282	0.001999793\\
0.2820282	0.00212808\\
0.2821282	0.001284737\\
0.2822282	-0.001072352\\
0.2823282	-0.0002300336\\
0.2824282	0.001184126\\
0.2825283	0.00204476\\
0.2826283	-9.958045e-05\\
0.2827283	-0.001450837\\
0.2828283	-0.001846624\\
0.2829283	-0.0002464621\\
0.2830283	-0.002750049\\
0.2831283	-0.002918591\\
0.2832283	-0.0007816122\\
0.2833283	-1.006489e-05\\
0.2834283	-0.001969555\\
0.2835284	-0.0004082749\\
0.2836284	0.001766841\\
0.2837284	0.00151062\\
0.2838284	0.0002970445\\
0.2839284	-1.290363e-05\\
0.2840284	0.0005411049\\
0.2841284	-0.001226806\\
0.2842284	-0.0005040721\\
0.2843284	-0.0009724946\\
0.2844284	-8.37001e-05\\
0.2845285	-0.001013171\\
0.2846285	1.444273e-05\\
0.2847285	0.00118156\\
0.2848285	0.00173037\\
0.2849285	0.000630811\\
0.2850285	0.0009844692\\
0.2851285	0.001497031\\
0.2852285	0.001178288\\
0.2853285	0.0007356336\\
0.2854285	0.0006136192\\
0.2855286	0.0004884469\\
0.2856286	-0.0001423799\\
0.2857286	-0.002163912\\
0.2858286	0.0001236951\\
0.2859286	-0.00183014\\
0.2860286	-0.002019254\\
0.2861286	0.0003310211\\
0.2862286	-0.0006136398\\
0.2863286	0.002159365\\
0.2864286	0.0009953862\\
0.2865287	0.0004671463\\
0.2866287	0.0002238777\\
0.2867287	0.000458653\\
0.2868287	-0.001500958\\
0.2869287	-0.001416546\\
0.2870287	-0.0003535531\\
0.2871287	-0.001010364\\
0.2872287	-0.0007607258\\
0.2873287	0.001051946\\
0.2874287	0.0008262089\\
0.2875288	-0.0007144562\\
0.2876288	-2.705259e-05\\
0.2877288	0.002017759\\
0.2878288	0.0007815661\\
0.2879288	0.0002465384\\
0.2880288	-0.001521664\\
0.2881288	-0.0005799902\\
0.2882288	0.001519505\\
0.2883288	2.131325e-05\\
0.2884288	-0.001624461\\
0.2885289	-0.001636839\\
0.2886289	-0.0009261069\\
0.2887289	-0.0007685583\\
0.2888289	-0.000225832\\
0.2889289	-0.0007300148\\
0.2890289	-0.0001727582\\
0.2891289	0.00130635\\
0.2892289	0.001000077\\
0.2893289	0.002069133\\
0.2894289	0.002411096\\
0.289529	0.00329537\\
0.289629	0.003577188\\
0.289729	0.001333281\\
0.289829	-0.00188629\\
0.289929	-0.002531428\\
0.290029	-0.0001759847\\
0.290129	-0.0001562019\\
0.290229	-0.0004895931\\
0.290329	-0.002398234\\
0.290429	3.569093e-05\\
0.2905291	-0.001003819\\
0.2906291	-0.001331814\\
0.2907291	-0.001654859\\
0.2908291	-0.001483888\\
0.2909291	-0.0003876397\\
0.2910291	0.0002287947\\
0.2911291	-0.0006075796\\
0.2912291	0.0001177253\\
0.2913291	-0.0006147233\\
0.2914291	-0.0007232343\\
0.2915292	-0.0001950264\\
0.2916292	-0.0003487777\\
0.2917292	0.0001067645\\
0.2918292	0.0005268341\\
0.2919292	0.0008665653\\
0.2920292	0.0006058868\\
0.2921292	0.001174808\\
0.2922292	0.002321467\\
0.2923292	0.002586101\\
0.2924292	0.002853472\\
0.2925293	0.00270466\\
0.2926293	0.002828445\\
0.2927293	0.001487614\\
0.2928293	0.000474865\\
0.2929293	-0.003851179\\
0.2930293	-0.0050448\\
0.2931293	-0.003721976\\
0.2932293	-0.002916482\\
0.2933293	-0.001252767\\
0.2934293	-0.0004168679\\
0.2935294	-0.001984458\\
0.2936294	0.0003876833\\
0.2937294	0.001262724\\
0.2938294	0.0009583531\\
0.2939294	0.0008878951\\
0.2940294	0.0002388535\\
0.2941294	0.002189045\\
0.2942294	0.001229209\\
0.2943294	0.001941731\\
0.2944294	0.0003560331\\
0.2945295	0.000618583\\
0.2946295	-0.0001114452\\
0.2947295	-0.001530632\\
0.2948295	-0.003310825\\
0.2949295	-0.002330028\\
0.2950295	0.0003072054\\
0.2951295	-0.0003679807\\
0.2952295	-0.00172022\\
0.2953295	-0.000237609\\
0.2954295	0.000388317\\
0.2955296	-7.307831e-05\\
0.2956296	0.001530586\\
0.2957296	0.001865918\\
0.2958296	0.001567362\\
0.2959296	0.001858383\\
0.2960296	0.0009071709\\
0.2961296	0.001473542\\
0.2962296	0.0006239717\\
0.2963296	-0.001716845\\
0.2964296	-0.002224143\\
0.2965297	-0.002453581\\
0.2966297	-0.002299428\\
0.2967297	-0.0003941501\\
0.2968297	3.028855e-05\\
0.2969297	0.00161745\\
0.2970297	0.002124747\\
0.2971297	0.002550369\\
0.2972297	0.001897847\\
0.2973297	0.0008833332\\
0.2974297	-0.001157361\\
0.2975298	-0.001345928\\
0.2976298	-0.002515925\\
0.2977298	-0.001312303\\
0.2978298	1.418747e-05\\
0.2979298	4.511107e-05\\
0.2980298	-0.001657446\\
0.2981298	-0.00134175\\
0.2982298	-0.001086476\\
0.2983298	0.001045405\\
0.2984298	6.610153e-05\\
0.2985299	-0.0006898964\\
0.2986299	-8.632136e-05\\
0.2987299	0.0004577213\\
0.2988299	-0.001447192\\
0.2989299	-0.002189788\\
0.2990299	0.001088719\\
0.2991299	0.001302243\\
0.2992299	-0.0002198929\\
0.2993299	0.0006960775\\
0.2994299	0.002300402\\
0.29953	0.00277373\\
0.29963	-6.468351e-05\\
0.29973	0.0007203882\\
0.29983	0.001759137\\
0.29993	0.001134169\\
0.30003	0.001542434\\
0.30013	-0.0002037673\\
0.30023	-0.0007085385\\
0.30033	-0.0008457642\\
0.30043	-0.0005348616\\
0.3005301	0.0006238546\\
0.3006301	-0.001291144\\
0.3007301	-0.003206995\\
0.3008301	-0.003538659\\
0.3009301	-0.002057956\\
0.3010301	-0.001945079\\
0.3011301	0.0003044457\\
0.3012301	0.0005650013\\
0.3013301	0.0007416474\\
0.3014301	0.0001695375\\
0.3015302	-0.0001358614\\
0.3016302	0.001058271\\
0.3017302	-0.0001432019\\
0.3018302	-0.002312334\\
0.3019302	-0.0004532125\\
0.3020302	0.0006126111\\
0.3021302	0.001308023\\
0.3022302	0.001162916\\
0.3023302	0.002587727\\
0.3024302	0.00191651\\
0.3025303	0.00260592\\
0.3026303	0.001286706\\
0.3027303	0.0007697991\\
0.3028303	-0.0002154663\\
0.3029303	-0.0009461096\\
0.3030303	-0.003901308\\
0.3031303	-0.00299655\\
0.3032303	-0.003188715\\
0.3033303	-0.001247698\\
0.3034303	0.0005529436\\
0.3035304	0.001287317\\
0.3036304	0.002031469\\
0.3037304	0.0006984368\\
0.3038304	-0.0006580389\\
0.3039304	-0.0004528813\\
0.3040304	0.00177111\\
0.3041304	0.0009548545\\
0.3042304	-0.0003481398\\
0.3043304	3.700877e-06\\
0.3044304	-0.0002377998\\
0.3045305	0.0009680149\\
0.3046305	-0.0002490362\\
0.3047305	-0.001368356\\
0.3048305	-0.0009942994\\
0.3049305	-0.001884938\\
0.3050305	-2.570166e-05\\
0.3051305	0.001072113\\
0.3052305	0.0005173727\\
0.3053305	0.0006730857\\
0.3054305	0.0004306103\\
0.3055306	8.654249e-05\\
0.3056306	-0.002267251\\
0.3057306	-0.003713009\\
0.3058306	-0.002157506\\
0.3059306	-0.001057144\\
0.3060306	-4.187443e-06\\
0.3061306	-0.0002735648\\
0.3062306	0.001471907\\
0.3063306	0.001241003\\
0.3064306	0.002261241\\
0.3065307	7.961422e-05\\
0.3066307	0.002337176\\
0.3067307	0.002179761\\
0.3068307	0.002816721\\
0.3069307	0.001638335\\
0.3070307	-0.0001387829\\
0.3071307	-0.0001289019\\
0.3072307	0.001170117\\
0.3073307	-0.002159548\\
0.3074307	-0.003958594\\
0.3075308	-0.002342759\\
0.3076308	-0.001736778\\
0.3077308	-0.0009667692\\
0.3078308	-0.0001655783\\
0.3079308	-0.001022246\\
0.3080308	0.0004118113\\
0.3081308	0.000779049\\
0.3082308	-0.0006798274\\
0.3083308	-0.0004540233\\
0.3084308	0.0007520657\\
0.3085309	0.001470283\\
0.3086309	0.001902148\\
0.3087309	-0.001102154\\
0.3088309	-0.001878509\\
0.3089309	-0.001390619\\
0.3090309	-0.003238204\\
0.3091309	-0.00191728\\
0.3092309	-0.001670065\\
0.3093309	4.490895e-05\\
0.3094309	0.00220978\\
0.309531	0.0005283361\\
0.309631	0.0005762595\\
0.309731	0.002153436\\
0.309831	0.001868267\\
0.309931	0.00268817\\
0.310031	0.003039143\\
0.310131	0.002715524\\
0.310231	0.0008066011\\
0.310331	0.0005300958\\
0.310431	-0.002233818\\
0.3105311	-0.002350471\\
0.3106311	-0.002717121\\
0.3107311	-0.002914329\\
0.3108311	-0.001849366\\
0.3109311	-0.002097902\\
0.3110311	-0.001088704\\
0.3111311	5.285618e-06\\
0.3112311	-0.001437726\\
0.3113311	-0.0001019175\\
0.3114311	0.0007018093\\
0.3115312	0.001182242\\
0.3116312	0.0007814998\\
0.3117312	0.0004981519\\
0.3118312	-0.0003811491\\
0.3119312	0.0003102432\\
0.3120312	-0.0001101143\\
0.3121312	0.001445553\\
0.3122312	0.0003684172\\
0.3123312	1.162477e-05\\
0.3124312	-0.002132017\\
0.3125313	-0.001099049\\
0.3126313	-0.001457773\\
0.3127313	0.0005414135\\
0.3128313	0.001804714\\
0.3129313	0.0002264196\\
0.3130313	-0.000595771\\
0.3131313	0.001197179\\
0.3132313	3.404775e-05\\
0.3133313	-0.001589327\\
0.3134313	-0.00233221\\
0.3135314	-0.00248798\\
0.3136314	-0.0003146591\\
0.3137314	0.001122876\\
0.3138314	0.0006049748\\
0.3139314	-0.0002561598\\
0.3140314	0.0004626032\\
0.3141314	0.001071265\\
0.3142314	0.0004664466\\
0.3143314	6.71352e-06\\
0.3144314	0.001394669\\
0.3145315	0.002701357\\
0.3146315	9.6731e-05\\
0.3147315	-0.001031093\\
0.3148315	0.0006102326\\
0.3149315	-0.0001639691\\
0.3150315	-0.0002878736\\
0.3151315	-0.0008698606\\
0.3152315	-0.0005199133\\
0.3153315	-0.002643577\\
0.3154315	-0.004398449\\
0.3155316	-0.004753797\\
0.3156316	-0.001954158\\
0.3157316	0.0008828037\\
0.3158316	0.00107503\\
0.3159316	0.0001797854\\
0.3160316	-1.452045e-06\\
0.3161316	0.001182045\\
0.3162316	0.0005624831\\
0.3163316	-6.750945e-05\\
0.3164316	0.0003690849\\
0.3165317	-5.471032e-05\\
0.3166317	-0.0005193869\\
0.3167317	0.0002233079\\
0.3168317	0.000143027\\
0.3169317	0.00173417\\
0.3170317	0.001308282\\
0.3171317	0.002060326\\
0.3172317	0.002171796\\
0.3173317	0.001788709\\
0.3174317	1.002712e-05\\
0.3175318	-0.00120193\\
0.3176318	-0.003479529\\
0.3177318	-0.001990803\\
0.3178318	-0.001510567\\
0.3179318	-0.002056124\\
0.3180318	-0.001100246\\
0.3181318	-0.001163633\\
0.3182318	0.0003696569\\
0.3183318	-0.0006242913\\
0.3184318	-0.0002532779\\
0.3185319	0.0008857615\\
0.3186319	-0.0002711326\\
0.3187319	-0.0002867121\\
0.3188319	-0.00117871\\
0.3189319	-0.0001745237\\
0.3190319	-0.001007072\\
0.3191319	-0.001773197\\
0.3192319	0.0002610674\\
0.3193319	-7.487189e-05\\
0.3194319	-0.001449348\\
0.319532	0.0009367013\\
0.319632	0.001284161\\
0.319732	0.0005991779\\
0.319832	0.001838588\\
0.319932	-0.0007110719\\
0.320032	-0.0007232443\\
0.320132	0.0009813908\\
0.320232	-0.0008371141\\
0.320332	-0.0001158375\\
0.320432	-0.0006373778\\
0.3205321	-0.001228469\\
0.3206321	0.0007058475\\
0.3207321	-0.001238241\\
0.3208321	2.626148e-05\\
0.3209321	-9.619486e-05\\
0.3210321	0.0002633135\\
0.3211321	0.00033451\\
0.3212321	-0.000186467\\
0.3213321	-0.0005030736\\
0.3214321	-0.0003639396\\
0.3215322	-0.0007097202\\
0.3216322	-0.001422428\\
0.3217322	-0.0008346815\\
0.3218322	-0.0006283042\\
0.3219322	-0.001189576\\
0.3220322	-0.001194728\\
0.3221322	-0.0004013785\\
0.3222322	-0.0009148231\\
0.3223322	-0.0009538389\\
0.3224322	-0.0002712206\\
0.3225323	0.002226914\\
0.3226323	0.00339783\\
0.3227323	0.003111674\\
0.3228323	0.000810259\\
0.3229323	0.0001164954\\
0.3230323	-0.0007261058\\
0.3231323	-0.004179662\\
0.3232323	-0.006628982\\
0.3233323	-0.004334931\\
0.3234323	-0.003494967\\
0.3235324	-0.0006830396\\
0.3236324	-0.0003140338\\
0.3237324	0.0004066749\\
0.3238324	0.001389147\\
0.3239324	0.002560008\\
0.3240324	0.002684113\\
0.3241324	0.001810461\\
0.3242324	0.001306407\\
0.3243324	0.002689929\\
0.3244324	0.002379972\\
0.3245325	0.0005812463\\
0.3246325	4.728244e-05\\
0.3247325	-0.0009605069\\
0.3248325	-0.002054741\\
0.3249325	-0.003159755\\
0.3250325	-0.000853876\\
0.3251325	-0.0009820417\\
0.3252325	-0.001337458\\
0.3253325	-0.00102057\\
0.3254325	-0.003160919\\
0.3255326	-0.001504772\\
0.3256326	-0.001533446\\
0.3257326	-0.001736226\\
0.3258326	0.0009125384\\
0.3259326	0.0001978979\\
0.3260326	0.0006082437\\
0.3261326	-0.001200717\\
0.3262326	-0.001017763\\
0.3263326	0.0003656086\\
0.3264326	-0.0008792431\\
0.3265327	-0.001456963\\
0.3266327	-0.001937971\\
0.3267327	-0.001565613\\
0.3268327	0.001421598\\
0.3269327	0.002426809\\
0.3270327	0.001658314\\
0.3271327	0.001108189\\
0.3272327	0.001950307\\
0.3273327	0.002275875\\
0.3274327	-0.0002704806\\
0.3275328	-0.0008586258\\
0.3276328	-0.001144086\\
0.3277328	-0.002086015\\
0.3278328	-0.001525757\\
0.3279328	-0.0001772769\\
0.3280328	-0.0009790987\\
0.3281328	-0.001987672\\
0.3282328	-0.003042141\\
0.3283328	-0.001481522\\
0.3284328	-0.001596994\\
0.3285329	-0.0006271338\\
0.3286329	0.0008388976\\
0.3287329	-0.0003483538\\
0.3288329	0.0004430521\\
0.3289329	0.001120073\\
0.3290329	0.002380885\\
0.3291329	0.0007129257\\
0.3292329	-0.0009106375\\
0.3293329	-0.001934362\\
0.3294329	-0.002666552\\
0.329533	-0.001326958\\
0.329633	0.0004221451\\
0.329733	-0.002441974\\
0.329833	-0.001211885\\
0.329933	1.304047e-05\\
0.330033	-0.0006210105\\
0.330133	6.825042e-05\\
0.330233	0.0006438059\\
0.330333	0.0006671907\\
0.330433	-0.001695501\\
0.3305331	-0.0003348051\\
0.3306331	-0.0002423446\\
0.3307331	-0.0007082531\\
0.3308331	-0.001404322\\
0.3309331	0.0003797229\\
0.3310331	0.0003979188\\
0.3311331	-8.80966e-05\\
0.3312331	-0.0004682668\\
0.3313331	-0.001668439\\
0.3314331	-0.001209752\\
0.3315332	0.000953796\\
0.3316332	0.0004472172\\
0.3317332	-0.00149772\\
0.3318332	0.0006392673\\
0.3319332	0.001165602\\
0.3320332	-6.63478e-05\\
0.3321332	6.575669e-05\\
0.3322332	-0.0007056354\\
0.3323332	-0.001635743\\
0.3324332	-0.002024388\\
0.3325333	-0.001100421\\
0.3326333	-0.0002221204\\
0.3327333	-0.0006643074\\
0.3328333	-0.0003461006\\
0.3329333	-0.001155817\\
0.3330333	-0.00181648\\
0.3331333	-0.001141291\\
0.3332333	-0.001186706\\
0.3333333	-0.001425781\\
0.3334333	-0.001332076\\
0.3335334	-0.0002465065\\
0.3336334	-0.001303513\\
0.3337334	-0.001558624\\
0.3338334	0.0002234615\\
0.3339334	-8.406239e-05\\
0.3340334	-0.001291735\\
0.3341334	-0.001200895\\
0.3342334	-0.0003087997\\
0.3343334	0.0002359153\\
0.3344334	0.001310231\\
0.3345335	0.002722055\\
0.3346335	0.002410099\\
0.3347335	0.002525981\\
0.3348335	0.0004497361\\
0.3349335	-0.0008532369\\
0.3350335	-0.001717686\\
0.3351335	-0.0005860023\\
0.3352335	-0.001658796\\
0.3353335	-0.001219225\\
0.3354335	4.599532e-05\\
0.3355336	-0.002145859\\
0.3356336	-0.003673063\\
0.3357336	-0.004267627\\
0.3358336	-0.002204521\\
0.3359336	-0.001697548\\
0.3360336	-0.0006630645\\
0.3361336	-0.0001231815\\
0.3362336	-0.000538762\\
0.3363336	-0.001383378\\
0.3364336	0.0004945768\\
0.3365337	-0.0001146591\\
0.3366337	-0.0009087771\\
0.3367337	-0.0005344932\\
0.3368337	0.0006039033\\
0.3369337	0.002227739\\
0.3370337	0.001220614\\
0.3371337	0.0003366572\\
0.3372337	0.00016121\\
0.3373337	0.0005752386\\
0.3374337	-7.175762e-05\\
0.3375338	-0.001840667\\
0.3376338	-0.001177846\\
0.3377338	-0.001875291\\
0.3378338	-0.0033434\\
0.3379338	-0.002976645\\
0.3380338	-0.003316561\\
0.3381338	-0.001417074\\
0.3382338	-0.0009618097\\
0.3383338	0.0006178337\\
0.3384338	0.0002968012\\
0.3385339	0.002407164\\
0.3386339	0.003689435\\
0.3387339	0.001971141\\
0.3388339	0.001726024\\
0.3389339	-0.0007548207\\
0.3390339	-0.003846218\\
0.3391339	-0.004303648\\
0.3392339	-0.003470544\\
0.3393339	-0.001707529\\
0.3394339	-0.0003773942\\
0.339534	-0.0007440238\\
0.339634	-0.0002753371\\
0.339734	0.0006034007\\
0.339834	0.00129439\\
0.339934	-0.000332102\\
0.340034	-0.002025657\\
0.340134	-7.660542e-05\\
0.340234	-0.001122589\\
0.340334	-0.0005669644\\
0.340434	3.057264e-05\\
0.3405341	-0.001867739\\
0.3406341	-0.001933553\\
0.3407341	-0.004201885\\
0.3408341	-0.001975184\\
0.3409341	-0.0009468533\\
0.3410341	-0.002745793\\
0.3411341	0.0002773493\\
0.3412341	-0.000436587\\
0.3413341	0.001106927\\
0.3414341	0.002618304\\
0.3415342	0.0007899121\\
0.3416342	0.00198092\\
0.3417342	0.001777566\\
0.3418342	0.002235857\\
0.3419342	-0.000382145\\
0.3420342	-0.002416733\\
0.3421342	-0.002090829\\
0.3422342	-0.002246263\\
0.3423342	-0.002494115\\
0.3424342	-0.002200563\\
0.3425343	-0.003577086\\
0.3426343	0.0002794485\\
0.3427343	0.0005696078\\
0.3428343	-0.000623666\\
0.3429343	-0.0008984286\\
0.3430343	0.0002097239\\
0.3431343	0.0005239777\\
0.3432343	-0.001299068\\
0.3433343	-0.001348731\\
0.3434343	-0.002335718\\
0.3435344	-0.002013894\\
0.3436344	-0.002427935\\
0.3437344	-0.001562056\\
0.3438344	-0.002173438\\
0.3439344	-0.00151349\\
0.3440344	-0.002006083\\
0.3441344	0.001553677\\
0.3442344	0.0004537065\\
0.3443344	0.0003875987\\
0.3444344	0.0007436225\\
0.3445345	-0.001039503\\
0.3446345	0.0002487812\\
0.3447345	0.0007821888\\
0.3448345	0.001411222\\
0.3449345	0.002207365\\
0.3450345	0.0001428108\\
0.3451345	-0.001692177\\
0.3452345	-0.001890018\\
0.3453345	-0.001351018\\
0.3454345	0.0005388532\\
0.3455346	-0.0007896261\\
0.3456346	-0.003044921\\
0.3457346	-0.004717968\\
0.3458346	-0.003368159\\
0.3459346	-0.001102932\\
0.3460346	-0.00141762\\
0.3461346	-0.0009501017\\
0.3462346	0.0001778289\\
0.3463346	0.001748042\\
0.3464346	0.0007759074\\
0.3465347	-0.0009141882\\
0.3466347	-0.002387727\\
0.3467347	6.134463e-05\\
0.3468347	-0.001073597\\
0.3469347	-0.001908121\\
0.3470347	-0.0006093223\\
0.3471347	0.0001645893\\
0.3472347	0.0005356228\\
0.3473347	-0.0005302525\\
0.3474347	0.0003362831\\
0.3475348	-0.0002783703\\
0.3476348	0.0006018782\\
0.3477348	-0.0004861998\\
0.3478348	-0.002716002\\
0.3479348	-0.003350741\\
0.3480348	-0.003562638\\
0.3481348	-0.002766835\\
0.3482348	-0.0004840892\\
0.3483348	-0.001128133\\
0.3484348	-0.00028357\\
0.3485349	-0.0006160842\\
0.3486349	0.0008624937\\
0.3487349	0.0001065778\\
0.3488349	-0.002124075\\
0.3489349	0.0004391646\\
0.3490349	0.001964882\\
0.3491349	0.001257499\\
0.3492349	0.001468635\\
0.3493349	0.0008525721\\
0.3494349	0.0002864118\\
0.349535	-0.000815903\\
0.349635	-0.00175311\\
0.349735	-0.001257567\\
0.349835	-0.002455035\\
0.349935	-0.002608143\\
0.350035	-0.002808103\\
0.350135	-0.002676507\\
0.350235	-0.002293672\\
0.350335	-0.002635742\\
0.350435	-0.0009451124\\
0.3505351	0.0002708423\\
0.3506351	-3.710488e-05\\
0.3507351	-0.001602039\\
0.3508351	-0.00161833\\
0.3509351	-5.108705e-05\\
0.3510351	-0.001914269\\
0.3511351	3.305482e-05\\
0.3512351	0.0005521506\\
0.3513351	0.001843192\\
0.3514351	-0.001244345\\
0.3515352	-0.001125484\\
0.3516352	0.0001878435\\
0.3517352	-0.0005742415\\
0.3518352	-0.0009354352\\
0.3519352	-0.0005492271\\
0.3520352	0.00111462\\
0.3521352	-0.0002327899\\
0.3522352	-0.0001348385\\
0.3523352	-0.001012619\\
0.3524352	-0.001556223\\
0.3525353	-0.001939038\\
0.3526353	0.0005516857\\
0.3527353	0.001687816\\
0.3528353	-0.000105848\\
0.3529353	-0.001632309\\
0.3530353	-0.002182378\\
0.3531353	-0.001968053\\
0.3532353	-0.001761068\\
0.3533353	-0.003005202\\
0.3534353	-0.002250974\\
0.3535354	-0.0004430906\\
0.3536354	5.90338e-05\\
0.3537354	-0.001533099\\
0.3538354	-0.001501759\\
0.3539354	-0.0006262573\\
0.3540354	-0.001688832\\
0.3541354	-0.002273687\\
0.3542354	0.000189196\\
0.3543354	-4.540047e-05\\
0.3544354	0.0002308552\\
0.3545355	0.001915025\\
0.3546355	0.002595534\\
0.3547355	0.002212021\\
0.3548355	-0.001017352\\
0.3549355	-0.002262546\\
0.3550355	-0.003544187\\
0.3551355	-0.002495841\\
0.3552355	-0.001381642\\
0.3553355	-0.0008290144\\
0.3554355	-0.0004499328\\
0.3555356	-0.001119731\\
0.3556356	-0.001165378\\
0.3557356	-0.0006967802\\
0.3558356	-4.366458e-05\\
0.3559356	0.001322274\\
0.3560356	0.0004688842\\
0.3561356	0.0003655219\\
0.3562356	-0.0002257888\\
0.3563356	-0.0001814944\\
0.3564356	-0.001214211\\
0.3565357	-0.002398152\\
0.3566357	-0.002144337\\
0.3567357	-0.002685643\\
0.3568357	-0.002643646\\
0.3569357	-0.001537794\\
0.3570357	0.000248964\\
0.3571357	0.0003903171\\
0.3572357	0.002036326\\
0.3573357	0.001813622\\
0.3574357	0.0003741777\\
0.3575358	0.001994952\\
0.3576358	0.001442742\\
0.3577358	-0.0008598559\\
0.3578358	-0.003531891\\
0.3579358	-0.004250766\\
0.3580358	-0.003821504\\
0.3581358	-0.003368302\\
0.3582358	-0.004193292\\
0.3583358	-0.002924154\\
0.3584358	-0.0003108171\\
0.3585359	-0.0001516929\\
0.3586359	0.0002854429\\
0.3587359	0.001411552\\
0.3588359	0.002660972\\
0.3589359	0.002389261\\
0.3590359	0.001626336\\
0.3591359	0.00174188\\
0.3592359	0.001141597\\
0.3593359	0.00138824\\
0.3594359	-0.0007814123\\
0.359536	-0.002485046\\
0.359636	-0.0002696985\\
0.359736	-0.001402252\\
0.359836	-0.001930593\\
0.359936	-0.0003490029\\
0.360036	-0.001602058\\
0.360136	-0.0008472338\\
0.360236	-0.001387405\\
0.360336	-0.001684238\\
0.360436	-0.0007102378\\
0.3605361	-0.001552865\\
0.3606361	-0.00214382\\
0.3607361	-0.001686003\\
0.3608361	-0.001193935\\
0.3609361	-0.001222636\\
0.3610361	-0.0005182266\\
0.3611361	-0.0005536281\\
0.3612361	0.00114055\\
0.3613361	0.000620225\\
0.3614361	-0.0002764039\\
0.3615362	-0.001760687\\
0.3616362	-0.0005604358\\
0.3617362	0.001622908\\
0.3618362	0.001192823\\
0.3619362	0.0007211055\\
0.3620362	0.001098435\\
0.3621362	0.001691619\\
0.3622362	0.001136171\\
0.3623362	-0.001042121\\
0.3624362	-0.003220644\\
0.3625363	-0.002472554\\
0.3626363	-0.002240217\\
0.3627363	-0.001014045\\
0.3628363	-0.0006045493\\
0.3629363	0.0004062804\\
0.3630363	0.000941905\\
0.3631363	-0.0001175872\\
0.3632363	7.875066e-05\\
0.3633363	-5.494749e-05\\
0.3634363	0.0009581768\\
0.3635364	-0.0002272532\\
0.3636364	-0.001499439\\
0.3637364	-0.001640289\\
0.3638364	5.536865e-05\\
0.3639364	0.0001110795\\
0.3640364	-0.0002145461\\
0.3641364	-0.000503794\\
0.3642364	-0.001686646\\
0.3643364	-0.0004212801\\
0.3644364	0.000438484\\
0.3645365	-0.0001816311\\
0.3646365	-0.00247939\\
0.3647365	-0.0003608944\\
0.3648365	0.00112765\\
0.3649365	-0.0006871689\\
0.3650365	-1.408399e-05\\
0.3651365	0.0001058671\\
0.3652365	0.002184841\\
0.3653365	0.00197916\\
0.3654365	0.001010568\\
0.3655366	-0.001535968\\
0.3656366	-0.001696535\\
0.3657366	-0.00147548\\
0.3658366	-0.004202898\\
0.3659366	-0.0001758225\\
0.3660366	-0.00130295\\
0.3661366	-0.0001798798\\
0.3662366	0.001653968\\
0.3663366	0.002406803\\
0.3664366	0.001590618\\
0.3665367	0.001092263\\
0.3666367	0.00149145\\
0.3667367	0.0002169095\\
0.3668367	-0.0006049918\\
0.3669367	-0.000168968\\
0.3670367	0.001277898\\
0.3671367	0.00144948\\
0.3672367	-0.0006602434\\
0.3673367	3.501206e-05\\
0.3674367	-3.62188e-06\\
0.3675368	-0.002031951\\
0.3676368	-0.002290794\\
0.3677368	-0.0009309722\\
0.3678368	0.001430905\\
0.3679368	0.0004206267\\
0.3680368	-0.002063216\\
0.3681368	-0.001327338\\
0.3682368	-0.0007645718\\
0.3683368	-0.002215073\\
0.3684368	-0.001274461\\
0.3685369	-0.0002701646\\
0.3686369	0.002314657\\
0.3687369	0.001420407\\
0.3688369	0.001938579\\
0.3689369	0.001866368\\
0.3690369	0.0005407463\\
0.3691369	0.0008260983\\
0.3692369	-0.001105451\\
0.3693369	0.0005440145\\
0.3694369	0.001977371\\
0.369537	0.001561916\\
0.369637	0.001221611\\
0.369737	0.001608514\\
0.369837	0.000866924\\
0.369937	0.0008775565\\
0.370037	0.002068302\\
0.370137	-0.0008620778\\
0.370237	-0.0005855798\\
0.370337	-0.0003127474\\
0.370437	-0.0006254298\\
0.3705371	-0.001368813\\
0.3706371	-0.001839477\\
0.3707371	-0.0004798739\\
0.3708371	-0.0007190275\\
0.3709371	-0.0007360674\\
0.3710371	-0.0004943441\\
0.3711371	0.0003416616\\
0.3712371	0.0003404834\\
0.3713371	0.0003231196\\
0.3714371	0.0002066512\\
0.3715372	0.001542916\\
0.3716372	0.002491384\\
0.3717372	0.003712811\\
0.3718372	0.003187687\\
0.3719372	0.0009807786\\
0.3720372	0.001633042\\
0.3721372	0.003188187\\
0.3722372	0.0007589055\\
0.3723372	-0.001044971\\
0.3724372	-0.001711859\\
0.3725373	-0.003722701\\
0.3726373	-0.003337121\\
0.3727373	-0.0004561586\\
0.3728373	0.001350084\\
0.3729373	0.001836132\\
0.3730373	0.001839235\\
0.3731373	0.000990058\\
0.3732373	0.001298275\\
0.3733373	0.003180131\\
0.3734373	0.002750865\\
0.3735374	0.002578112\\
0.3736374	0.002165425\\
0.3737374	0.001012706\\
0.3738374	0.001112112\\
0.3739374	0.0005033328\\
0.3740374	-0.000709637\\
0.3741374	-0.0003895638\\
0.3742374	0.0002068659\\
0.3743374	-0.0005372053\\
0.3744374	-0.000914195\\
0.3745375	0.0002748997\\
0.3746375	0.002293317\\
0.3747375	0.002158033\\
0.3748375	0.001708216\\
0.3749375	0.002758487\\
0.3750375	0.003527158\\
0.3751375	0.001946315\\
0.3752375	0.000446559\\
0.3753375	-0.003012771\\
0.3754375	-0.001128158\\
0.3755376	-0.00112165\\
0.3756376	-0.0003848139\\
0.3757376	-0.00060322\\
0.3758376	0.0008535175\\
0.3759376	0.002585215\\
0.3760376	0.003095625\\
0.3761376	0.003934284\\
0.3762376	0.002534677\\
0.3763376	0.002878041\\
0.3764376	0.002464663\\
0.3765377	0.002308424\\
0.3766377	0.00179855\\
0.3767377	0.001753589\\
0.3768377	-0.0005116968\\
0.3769377	-0.0008587879\\
0.3770377	0.001704274\\
0.3771377	0.002674419\\
0.3772377	0.001489086\\
0.3773377	0.001323946\\
0.3774377	0.002452322\\
0.3775378	0.003536511\\
0.3776378	0.002425324\\
0.3777378	0.00194707\\
0.3778378	0.0001690307\\
0.3779378	0.00127659\\
0.3780378	0.0004957466\\
0.3781378	0.0003130415\\
0.3782378	-0.001330984\\
0.3783378	-4.740534e-05\\
0.3784378	-0.0007443696\\
0.3785379	2.124886e-05\\
0.3786379	0.001051345\\
0.3787379	0.00185831\\
0.3788379	0.003050161\\
0.3789379	0.00174653\\
0.3790379	0.003753836\\
0.3791379	0.003474906\\
0.3792379	0.003865123\\
0.3793379	0.002983316\\
0.3794379	0.003629002\\
0.379538	0.003505035\\
0.379638	0.003125584\\
0.379738	0.001595312\\
0.379838	0.0006318343\\
0.379938	0.002633841\\
0.380038	-0.0006003287\\
0.380138	0.002148278\\
0.380238	0.002871427\\
0.380338	0.002064824\\
0.380438	0.001148047\\
0.3805381	-0.0009475068\\
0.3806381	0.001335662\\
0.3807381	0.001145956\\
0.3808381	0.002409144\\
0.3809381	0.001659955\\
0.3810381	0.003756234\\
0.3811381	0.003319719\\
0.3812381	0.004180383\\
0.3813381	0.002683518\\
0.3814381	0.001065191\\
0.3815382	0.002127375\\
0.3816382	0.0006327266\\
0.3817382	0.002523658\\
0.3818382	0.002333716\\
0.3819382	0.00346492\\
0.3820382	0.002842769\\
0.3821382	0.001904601\\
0.3822382	0.003356975\\
0.3823382	0.00352224\\
0.3824382	0.003522143\\
0.3825383	0.001318163\\
0.3826383	0.001002239\\
0.3827383	0.0001612238\\
0.3828383	0.001413087\\
0.3829383	0.001569037\\
0.3830383	0.003049516\\
0.3831383	0.004375142\\
0.3832383	0.00250394\\
0.3833383	0.001221865\\
0.3834383	-0.0003164918\\
0.3835384	0.002005538\\
0.3836384	0.005091421\\
0.3837384	0.00523685\\
0.3838384	0.004556457\\
0.3839384	0.004849334\\
0.3840384	0.004802572\\
0.3841384	0.003997717\\
0.3842384	0.002427466\\
0.3843384	0.002226008\\
0.3844384	0.004013337\\
0.3845385	0.004441224\\
0.3846385	0.002988846\\
0.3847385	0.002731513\\
0.3848385	0.001104354\\
0.3849385	0.0002104751\\
0.3850385	8.932895e-05\\
0.3851385	0.0007763673\\
0.3852385	0.001485364\\
0.3853385	0.002885198\\
0.3854385	0.003136971\\
0.3855386	0.003557259\\
0.3856386	0.004347563\\
0.3857386	0.003001755\\
0.3858386	0.004461324\\
0.3859386	0.002483716\\
0.3860386	0.002324372\\
0.3861386	0.00503631\\
0.3862386	0.003757889\\
0.3863386	0.003200423\\
0.3864386	0.003878105\\
0.3865387	0.004231408\\
0.3866387	0.003823408\\
0.3867387	0.003684326\\
0.3868387	0.004317836\\
0.3869387	0.003963571\\
0.3870387	0.002479183\\
0.3871387	0.001685756\\
0.3872387	0.002060481\\
0.3873387	0.00304936\\
0.3874387	0.005546251\\
0.3875388	0.005832698\\
0.3876388	0.005310795\\
0.3877388	0.005139214\\
0.3878388	0.004827101\\
0.3879388	0.004508253\\
0.3880388	0.003264874\\
0.3881388	0.001797876\\
0.3882388	-0.0001774439\\
0.3883388	-0.0002099513\\
0.3884388	0.0007628947\\
0.3885389	0.002755289\\
0.3886389	0.003716643\\
0.3887389	0.00480858\\
0.3888389	0.004295938\\
0.3889389	0.00479747\\
0.3890389	0.004730058\\
0.3891389	0.005896874\\
0.3892389	0.004397617\\
0.3893389	0.004233809\\
0.3894389	0.005346073\\
0.389539	0.005462785\\
0.389639	0.004808738\\
0.389739	0.003659837\\
0.389839	0.00373145\\
0.389939	0.002672636\\
0.390039	0.003444197\\
0.390139	0.002256341\\
0.390239	0.004303495\\
0.390339	0.003444646\\
0.390439	0.003046587\\
0.3905391	0.005091161\\
0.3906391	0.003329751\\
0.3907391	0.003762922\\
0.3908391	0.006680265\\
0.3909391	0.00638969\\
0.3910391	0.00420888\\
0.3911391	0.003759794\\
0.3912391	0.002892102\\
0.3913391	0.004109329\\
0.3914391	0.00459697\\
0.3915392	0.004122899\\
0.3916392	0.003103727\\
0.3917392	0.004236032\\
0.3918392	0.004957518\\
0.3919392	0.006192865\\
0.3920392	0.006386537\\
0.3921392	0.006028829\\
0.3922392	0.006419235\\
0.3923392	0.005060193\\
0.3924392	0.005106174\\
0.3925393	0.002931696\\
0.3926393	0.0014109\\
0.3927393	0.000373408\\
0.3928393	0.00191937\\
0.3929393	0.003046959\\
0.3930393	0.002509927\\
0.3931393	0.003582544\\
0.3932393	0.003333109\\
0.3933393	0.004962089\\
0.3934393	0.006786937\\
0.3935394	0.007296598\\
0.3936394	0.008338844\\
0.3937394	0.007492192\\
0.3938394	0.006682732\\
0.3939394	0.006033241\\
0.3940394	0.00479106\\
0.3941394	0.004895283\\
0.3942394	0.003467229\\
0.3943394	0.003444985\\
0.3944394	0.004150169\\
0.3945395	0.005592372\\
0.3946395	0.005708218\\
0.3947395	0.004724381\\
0.3948395	0.005259111\\
0.3949395	0.00574961\\
0.3950395	0.00396732\\
0.3951395	0.002465412\\
0.3952395	0.005712644\\
0.3953395	0.005638503\\
0.3954395	0.005938791\\
0.3955396	0.00325482\\
0.3956396	0.003549869\\
0.3957396	0.003011293\\
0.3958396	0.001921776\\
0.3959396	0.002781013\\
0.3960396	0.004151461\\
0.3961396	0.006122751\\
0.3962396	0.006644708\\
0.3963396	0.007508937\\
0.3964396	0.007732074\\
0.3965397	0.007010094\\
0.3966397	0.007556475\\
0.3967397	0.005509834\\
0.3968397	0.004438339\\
0.3969397	0.003498894\\
0.3970397	0.003803992\\
0.3971397	0.005104605\\
0.3972397	0.005467483\\
0.3973397	0.005630093\\
0.3974397	0.004366891\\
0.3975398	0.005972675\\
0.3976398	0.004373604\\
0.3977398	0.005345387\\
0.3978398	0.004897479\\
0.3979398	0.004834778\\
0.3980398	0.00597825\\
0.3981398	0.005581994\\
0.3982398	0.004946184\\
0.3983398	0.00538057\\
0.3984398	0.00502265\\
0.3985399	0.006757185\\
0.3986399	0.005983377\\
0.3987399	0.004942052\\
0.3988399	0.006377619\\
0.3989399	0.005146364\\
0.3990399	0.00409676\\
0.3991399	0.004465482\\
0.3992399	0.002506843\\
0.3993399	0.003425105\\
0.3994399	0.006003192\\
0.39954	0.005968407\\
0.39964	0.006944333\\
0.39974	0.007123873\\
0.39984	0.005833785\\
0.39994	0.00617481\\
0.40004	0.005028364\\
};
\addplot [color=mycolor2,solid,forget plot]
  table[row sep=crcr]{%
0.40004	0.005028364\\
0.40014	0.003675153\\
0.40024	0.005747546\\
0.40034	0.006970603\\
0.40044	0.005487722\\
0.4005401	0.004372886\\
0.4006401	0.003047542\\
0.4007401	0.005759343\\
0.4008401	0.005313498\\
0.4009401	0.002799473\\
0.4010401	0.003008902\\
0.4011401	0.005454961\\
0.4012401	0.006847822\\
0.4013401	0.008491072\\
0.4014401	0.006408462\\
0.4015402	0.005644323\\
0.4016402	0.007928362\\
0.4017402	0.00653884\\
0.4018402	0.00639158\\
0.4019402	0.006897292\\
0.4020402	0.006384526\\
0.4021402	0.004644701\\
0.4022402	0.005498725\\
0.4023402	0.00555454\\
0.4024402	0.004071921\\
0.4025403	0.003028892\\
0.4026403	0.003839615\\
0.4027403	0.003613395\\
0.4028403	0.003673133\\
0.4029403	0.002829092\\
0.4030403	0.004674411\\
0.4031403	0.003545547\\
0.4032403	0.004950086\\
0.4033403	0.004905286\\
0.4034403	0.0064135\\
0.4035404	0.007342934\\
0.4036404	0.007376649\\
0.4037404	0.006110891\\
0.4038404	0.005272114\\
0.4039404	0.006655604\\
0.4040404	0.006393593\\
0.4041404	0.005399432\\
0.4042404	0.005257821\\
0.4043404	0.006359141\\
0.4044404	0.005105138\\
0.4045405	0.004889948\\
0.4046405	0.005028571\\
0.4047405	0.005813776\\
0.4048405	0.006348283\\
0.4049405	0.007074036\\
0.4050405	0.003483634\\
0.4051405	0.003385258\\
0.4052405	0.004584566\\
0.4053405	0.005735114\\
0.4054405	0.003674736\\
0.4055406	0.003386361\\
0.4056406	0.004857618\\
0.4057406	0.003730175\\
0.4058406	0.0041251\\
0.4059406	0.003948941\\
0.4060406	0.00470329\\
0.4061406	0.005295776\\
0.4062406	0.00514656\\
0.4063406	0.006016962\\
0.4064406	0.005429555\\
0.4065407	0.00555164\\
0.4066407	0.00799249\\
0.4067407	0.006682286\\
0.4068407	0.006009509\\
0.4069407	0.005814772\\
0.4070407	0.003698884\\
0.4071407	0.002225665\\
0.4072407	0.002600903\\
0.4073407	0.002366003\\
0.4074407	0.001204854\\
0.4075408	0.003819499\\
0.4076408	0.005151364\\
0.4077408	0.006278663\\
0.4078408	0.00711526\\
0.4079408	0.005264016\\
0.4080408	0.006064415\\
0.4081408	0.005090911\\
0.4082408	0.005737301\\
0.4083408	0.004539523\\
0.4084408	0.004252178\\
0.4085409	0.003973441\\
0.4086409	0.003806928\\
0.4087409	0.002127807\\
0.4088409	0.003474224\\
0.4089409	0.005499009\\
0.4090409	0.003585272\\
0.4091409	0.004435665\\
0.4092409	0.006054739\\
0.4093409	0.005287173\\
0.4094409	0.004027783\\
0.409541	0.005421515\\
0.409641	0.004024439\\
0.409741	0.003917396\\
0.409841	0.004365129\\
0.409941	0.00355805\\
0.410041	0.000629279\\
0.410141	0.001627378\\
0.410241	0.002324498\\
0.410341	0.003334103\\
0.410441	0.00390968\\
0.4105411	0.003530833\\
0.4106411	0.003165944\\
0.4107411	0.002427983\\
0.4108411	0.00300808\\
0.4109411	0.004599485\\
0.4110411	0.004732958\\
0.4111411	0.00652727\\
0.4112411	0.006313835\\
0.4113411	0.004866823\\
0.4114411	0.004772489\\
0.4115412	0.002650109\\
0.4116412	0.0008491046\\
0.4117412	0.0004889485\\
0.4118412	0.001456588\\
0.4119412	0.002062045\\
0.4120412	0.003754681\\
0.4121412	0.006107043\\
0.4122412	0.004715578\\
0.4123412	0.005997376\\
0.4124412	0.005883603\\
0.4125413	0.005168321\\
0.4126413	0.001721073\\
0.4127413	-0.0005410386\\
0.4128413	0.0009461004\\
0.4129413	0.00012291\\
0.4130413	0.0001233173\\
0.4131413	0.0003040666\\
0.4132413	0.0002556099\\
0.4133413	0.0009286269\\
0.4134413	0.001329419\\
0.4135414	0.002311894\\
0.4136414	0.00395197\\
0.4137414	0.002452347\\
0.4138414	0.003487056\\
0.4139414	0.003914131\\
0.4140414	0.00376384\\
0.4141414	0.004027575\\
0.4142414	0.00325692\\
0.4143414	0.003477704\\
0.4144414	0.003876084\\
0.4145415	-3.579043e-05\\
0.4146415	0.0009868114\\
0.4147415	0.001883384\\
0.4148415	0.001848797\\
0.4149415	0.001250673\\
0.4150415	0.0007390403\\
0.4151415	-0.0006280245\\
0.4152415	3.734683e-05\\
0.4153415	-0.000218486\\
0.4154415	0.0007567072\\
0.4155416	0.002287778\\
0.4156416	0.0004205697\\
0.4157416	0.001238161\\
0.4158416	0.0007872089\\
0.4159416	0.001011226\\
0.4160416	0.002170354\\
0.4161416	0.001333592\\
0.4162416	0.002314725\\
0.4163416	0.001768393\\
0.4164416	0.001233247\\
0.4165417	0.002837581\\
0.4166417	0.0009845742\\
0.4167417	-0.001845189\\
0.4168417	-0.002840391\\
0.4169417	-0.0008887929\\
0.4170417	0.001301057\\
0.4171417	0.0003323379\\
0.4172417	-0.0002707049\\
0.4173417	-0.0009877626\\
0.4174417	-0.001132973\\
0.4175418	-0.001200077\\
0.4176418	-0.001509592\\
0.4177418	0.0002288041\\
0.4178418	0.002149578\\
0.4179418	-0.0001239009\\
0.4180418	-0.0009940655\\
0.4181418	-0.0001758864\\
0.4182418	-0.002156103\\
0.4183418	-0.003195263\\
0.4184418	-0.001877955\\
0.4185419	0.001172773\\
0.4186419	-0.0001549103\\
0.4187419	3.818304e-05\\
0.4188419	0.001178626\\
0.4189419	0.0006902471\\
0.4190419	-0.0004861273\\
0.4191419	-0.001732684\\
0.4192419	-0.0009192713\\
0.4193419	0.0002161603\\
0.4194419	-0.0009491143\\
0.419542	-0.003386863\\
0.419642	-0.004326946\\
0.419742	-0.002743675\\
0.419842	-0.002243872\\
0.419942	-0.002318663\\
0.420042	-0.003137123\\
0.420142	-0.002056323\\
0.420242	-0.003903613\\
0.420342	-0.002616888\\
0.420442	-0.003633417\\
0.4205421	-0.004570063\\
0.4206421	-0.003409349\\
0.4207421	-0.001104719\\
0.4208421	-0.003201493\\
0.4209421	-0.002654238\\
0.4210421	-0.001541007\\
0.4211421	-0.001041763\\
0.4212421	-0.002544557\\
0.4213421	-0.003623798\\
0.4214421	-0.003142129\\
0.4215422	-0.002856355\\
0.4216422	-0.005143274\\
0.4217422	-0.006692993\\
0.4218422	-0.004083266\\
0.4219422	-0.001894954\\
0.4220422	-0.001082935\\
0.4221422	-0.002422409\\
0.4222422	-0.002804133\\
0.4223422	-0.0027842\\
0.4224422	-0.004148217\\
0.4225423	-0.004579275\\
0.4226423	-0.004927639\\
0.4227423	-0.0049779\\
0.4228423	-0.005751983\\
0.4229423	-0.005927367\\
0.4230423	-0.005777768\\
0.4231423	-0.005647142\\
0.4232423	-0.006711337\\
0.4233423	-0.006140928\\
0.4234423	-0.007283833\\
0.4235424	-0.006489554\\
0.4236424	-0.005749935\\
0.4237424	-0.004439485\\
0.4238424	-0.003485063\\
0.4239424	-0.003609734\\
0.4240424	-0.003921848\\
0.4241424	-0.004495329\\
0.4242424	-0.005616024\\
0.4243424	-0.005985463\\
0.4244424	-0.007676962\\
0.4245425	-0.007022527\\
0.4246425	-0.007030001\\
0.4247425	-0.007556402\\
0.4248425	-0.006685724\\
0.4249425	-0.007390431\\
0.4250425	-0.006034526\\
0.4251425	-0.006137092\\
0.4252425	-0.00854008\\
0.4253425	-0.005830092\\
0.4254425	-0.007959112\\
0.4255426	-0.00661762\\
0.4256426	-0.006814738\\
0.4257426	-0.006394419\\
0.4258426	-0.008265675\\
0.4259426	-0.01041125\\
0.4260426	-0.0101398\\
0.4261426	-0.009039204\\
0.4262426	-0.006743349\\
0.4263426	-0.00497391\\
0.4264426	-0.007834932\\
0.4265427	-0.007983678\\
0.4266427	-0.006973478\\
0.4267427	-0.00605739\\
0.4268427	-0.006752735\\
0.4269427	-0.00732171\\
0.4270427	-0.008455926\\
0.4271427	-0.007471889\\
0.4272427	-0.00924641\\
0.4273427	-0.01075495\\
0.4274427	-0.01408007\\
0.4275428	-0.01412023\\
0.4276428	-0.01271524\\
0.4277428	-0.01274296\\
0.4278428	-0.01135675\\
0.4279428	-0.009698353\\
0.4280428	-0.008351061\\
0.4281428	-0.008559176\\
0.4282428	-0.007810199\\
0.4283428	-0.006144337\\
0.4284428	-0.008903213\\
0.4285429	-0.01016582\\
0.4286429	-0.009205495\\
0.4287429	-0.01083431\\
0.4288429	-0.009195322\\
0.4289429	-0.01050635\\
0.4290429	-0.01104012\\
0.4291429	-0.01204904\\
0.4292429	-0.01093175\\
0.4293429	-0.01181888\\
0.4294429	-0.01242706\\
0.429543	-0.01193329\\
0.429643	-0.0105428\\
0.429743	-0.01049468\\
0.429843	-0.01119839\\
0.429943	-0.01216579\\
0.430043	-0.01107276\\
0.430143	-0.0117992\\
0.430243	-0.01324258\\
0.430343	-0.01371901\\
0.430443	-0.01303415\\
0.4305431	-0.01246378\\
0.4306431	-0.01224282\\
0.4307431	-0.01511513\\
0.4308431	-0.01528503\\
0.4309431	-0.01379924\\
0.4310431	-0.0124801\\
0.4311431	-0.01330041\\
0.4312431	-0.01154124\\
0.4313431	-0.01042184\\
0.4314431	-0.01168816\\
0.4315432	-0.01273327\\
0.4316432	-0.01223182\\
0.4317432	-0.01362348\\
0.4318432	-0.01445674\\
0.4319432	-0.01564705\\
0.4320432	-0.01548945\\
0.4321432	-0.01400074\\
0.4322432	-0.0133584\\
0.4323432	-0.01228833\\
0.4324432	-0.0128122\\
0.4325433	-0.01465145\\
0.4326433	-0.01498866\\
0.4327433	-0.0150082\\
0.4328433	-0.01381177\\
0.4329433	-0.0148422\\
0.4330433	-0.01635121\\
0.4331433	-0.01646781\\
0.4332433	-0.01621373\\
0.4333433	-0.0153325\\
0.4334433	-0.01542291\\
0.4335434	-0.01348737\\
0.4336434	-0.0133056\\
0.4337434	-0.01498647\\
0.4338434	-0.01471286\\
0.4339434	-0.01461639\\
0.4340434	-0.01602322\\
0.4341434	-0.01737187\\
0.4342434	-0.01809176\\
0.4343434	-0.01549491\\
0.4344434	-0.01616484\\
0.4345435	-0.01721004\\
0.4346435	-0.0173957\\
0.4347435	-0.01414113\\
0.4348435	-0.01638425\\
0.4349435	-0.01559558\\
0.4350435	-0.01604328\\
0.4351435	-0.01505414\\
0.4352435	-0.01704821\\
0.4353435	-0.01833173\\
0.4354435	-0.01880388\\
0.4355436	-0.01877764\\
0.4356436	-0.0185604\\
0.4357436	-0.01741409\\
0.4358436	-0.01780872\\
0.4359436	-0.01774631\\
0.4360436	-0.01633606\\
0.4361436	-0.01518939\\
0.4362436	-0.01555143\\
0.4363436	-0.01670771\\
0.4364436	-0.01792296\\
0.4365437	-0.01644542\\
0.4366437	-0.01616833\\
0.4367437	-0.01606774\\
0.4368437	-0.01514694\\
0.4369437	-0.0164322\\
0.4370437	-0.01828275\\
0.4371437	-0.01909291\\
0.4372437	-0.0193545\\
0.4373437	-0.01953133\\
0.4374437	-0.02141015\\
0.4375438	-0.0216314\\
0.4376438	-0.01985657\\
0.4377438	-0.01940999\\
0.4378438	-0.0183702\\
0.4379438	-0.01856362\\
0.4380438	-0.01822016\\
0.4381438	-0.01774094\\
0.4382438	-0.019954\\
0.4383438	-0.02014464\\
0.4384438	-0.0186483\\
0.4385439	-0.01857835\\
0.4386439	-0.0173904\\
0.4387439	-0.01766702\\
0.4388439	-0.01799007\\
0.4389439	-0.01686634\\
0.4390439	-0.0176475\\
0.4391439	-0.01842933\\
0.4392439	-0.01989078\\
0.4393439	-0.02014284\\
0.4394439	-0.01926518\\
0.439544	-0.01947833\\
0.439644	-0.01783592\\
0.439744	-0.01868533\\
0.439844	-0.01925876\\
0.439944	-0.01959491\\
0.440044	-0.02089983\\
0.440144	-0.01924754\\
0.440244	-0.02151991\\
0.440344	-0.02128018\\
0.440444	-0.02047227\\
0.4405441	-0.02003256\\
0.4406441	-0.0199373\\
0.4407441	-0.02127206\\
0.4408441	-0.01919174\\
0.4409441	-0.01785713\\
0.4410441	-0.01943039\\
0.4411441	-0.01716884\\
0.4412441	-0.01901167\\
0.4413441	-0.01841715\\
0.4414441	-0.01866436\\
0.4415442	-0.01965481\\
0.4416442	-0.0185011\\
0.4417442	-0.0202843\\
0.4418442	-0.02347769\\
0.4419442	-0.02366639\\
0.4420442	-0.02246707\\
0.4421442	-0.01996758\\
0.4422442	-0.02016643\\
0.4423442	-0.01993672\\
0.4424442	-0.01746337\\
0.4425443	-0.0181177\\
0.4426443	-0.01691423\\
0.4427443	-0.01962121\\
0.4428443	-0.02028243\\
0.4429443	-0.01857811\\
0.4430443	-0.01892012\\
0.4431443	-0.01988181\\
0.4432443	-0.0215235\\
0.4433443	-0.02096858\\
0.4434443	-0.02070126\\
0.4435444	-0.02209307\\
0.4436444	-0.02024169\\
0.4437444	-0.01888404\\
0.4438444	-0.01810318\\
0.4439444	-0.01929695\\
0.4440444	-0.01880048\\
0.4441444	-0.01738223\\
0.4442444	-0.0190548\\
0.4443444	-0.01898391\\
0.4444444	-0.01688042\\
0.4445445	-0.01843507\\
0.4446445	-0.01831632\\
0.4447445	-0.02059565\\
0.4448445	-0.0225788\\
0.4449445	-0.02150331\\
0.4450445	-0.02167189\\
0.4451445	-0.0216881\\
0.4452445	-0.02214958\\
0.4453445	-0.0202928\\
0.4454445	-0.01867941\\
0.4455446	-0.01708122\\
0.4456446	-0.01711006\\
0.4457446	-0.01757571\\
0.4458446	-0.01754915\\
0.4459446	-0.01694711\\
0.4460446	-0.02012993\\
0.4461446	-0.01828089\\
0.4462446	-0.01876365\\
0.4463446	-0.01938887\\
0.4464446	-0.01882597\\
0.4465447	-0.01799763\\
0.4466447	-0.01782866\\
0.4467447	-0.01611551\\
0.4468447	-0.01612475\\
0.4469447	-0.01858329\\
0.4470447	-0.01796532\\
0.4471447	-0.01758919\\
0.4472447	-0.01577056\\
0.4473447	-0.01733214\\
0.4474447	-0.01695709\\
0.4475448	-0.01819045\\
0.4476448	-0.02008132\\
0.4477448	-0.01915106\\
0.4478448	-0.01938344\\
0.4479448	-0.01813222\\
0.4480448	-0.01826448\\
0.4481448	-0.01843848\\
0.4482448	-0.01846146\\
0.4483448	-0.0173652\\
0.4484448	-0.01605006\\
0.4485449	-0.01643692\\
0.4486449	-0.01515566\\
0.4487449	-0.01419427\\
0.4488449	-0.01469115\\
0.4489449	-0.01452004\\
0.4490449	-0.0151031\\
0.4491449	-0.01642565\\
0.4492449	-0.01658036\\
0.4493449	-0.01495844\\
0.4494449	-0.01494657\\
0.449545	-0.01462141\\
0.449645	-0.01453882\\
0.449745	-0.01434281\\
0.449845	-0.01501252\\
0.449945	-0.01619264\\
0.450045	-0.01736606\\
0.450145	-0.01725181\\
0.450245	-0.01592943\\
0.450345	-0.01496059\\
0.450445	-0.01228178\\
0.4505451	-0.01265895\\
0.4506451	-0.01364433\\
0.4507451	-0.01306773\\
0.4508451	-0.01394637\\
0.4509451	-0.01376361\\
0.4510451	-0.01201035\\
0.4511451	-0.01139603\\
0.4512451	-0.01056592\\
0.4513451	-0.01163013\\
0.4514451	-0.01156903\\
0.4515452	-0.01253303\\
0.4516452	-0.0127433\\
0.4517452	-0.01338773\\
0.4518452	-0.01324832\\
0.4519452	-0.01222725\\
0.4520452	-0.0127856\\
0.4521452	-0.01275423\\
0.4522452	-0.01347183\\
0.4523452	-0.01255631\\
0.4524452	-0.00942628\\
0.4525453	-0.00873249\\
0.4526453	-0.008875913\\
0.4527453	-0.009680495\\
0.4528453	-0.01067343\\
0.4529453	-0.009164212\\
0.4530453	-0.01024737\\
0.4531453	-0.01012205\\
0.4532453	-0.01076852\\
0.4533453	-0.009468638\\
0.4534453	-0.008174666\\
0.4535454	-0.008380678\\
0.4536454	-0.007684285\\
0.4537454	-0.007280768\\
0.4538454	-0.004876225\\
0.4539454	-0.003827252\\
0.4540454	-0.006005657\\
0.4541454	-0.007772253\\
0.4542454	-0.006682877\\
0.4543454	-0.008546536\\
0.4544454	-0.008878737\\
0.4545455	-0.00800683\\
0.4546455	-0.008409404\\
0.4547455	-0.005301832\\
0.4548455	-0.003538409\\
0.4549455	-0.005422485\\
0.4550455	-0.005235521\\
0.4551455	-0.005547438\\
0.4552455	-0.00570259\\
0.4553455	-0.004353419\\
0.4554455	-0.004441487\\
0.4555456	-0.004560797\\
0.4556456	-0.003740056\\
0.4557456	-0.003508516\\
0.4558456	-0.004956115\\
0.4559456	-0.003976495\\
0.4560456	-0.002731592\\
0.4561456	-0.002669151\\
0.4562456	-0.001981097\\
0.4563456	0.0002015628\\
0.4564456	-0.001779596\\
0.4565457	-0.001725016\\
0.4566457	-0.0005037599\\
0.4567457	0.000779701\\
0.4568457	0.0003024823\\
0.4569457	0.0008024204\\
0.4570457	0.00150806\\
0.4571457	0.001740716\\
0.4572457	0.00141233\\
0.4573457	0.001224465\\
0.4574457	0.002476335\\
0.4575458	0.001209497\\
0.4576458	-0.001027635\\
0.4577458	-0.003454109\\
0.4578458	-0.001278841\\
0.4579458	0.002055786\\
0.4580458	0.002513897\\
0.4581458	0.003074934\\
0.4582458	0.003658165\\
0.4583458	0.004251949\\
0.4584458	0.006375466\\
0.4585459	0.004748669\\
0.4586459	0.004992969\\
0.4587459	0.005547843\\
0.4588459	0.00655625\\
0.4589459	0.008120241\\
0.4590459	0.005592106\\
0.4591459	0.00616981\\
0.4592459	0.005043186\\
0.4593459	0.004607526\\
0.4594459	0.005040277\\
0.459546	0.00610778\\
0.459646	0.007303597\\
0.459746	0.008289084\\
0.459846	0.009151793\\
0.459946	0.008133692\\
0.460046	0.008951109\\
0.460146	0.009314783\\
0.460246	0.01108766\\
0.460346	0.01030277\\
0.460446	0.01036053\\
0.4605461	0.009723304\\
0.4606461	0.00991695\\
0.4607461	0.01039644\\
0.4608461	0.009923372\\
0.4609461	0.009934901\\
0.4610461	0.01114764\\
0.4611461	0.01275735\\
0.4612461	0.01355915\\
0.4613461	0.01550519\\
0.4614461	0.01692173\\
0.4615462	0.01808322\\
0.4616462	0.01632683\\
0.4617462	0.01540152\\
0.4618462	0.01434332\\
0.4619462	0.01400513\\
0.4620462	0.01223302\\
0.4621462	0.01394086\\
0.4622462	0.01284045\\
0.4623462	0.01272469\\
0.4624462	0.01481756\\
0.4625463	0.01745157\\
0.4626463	0.01721609\\
0.4627463	0.01634174\\
0.4628463	0.01659169\\
0.4629463	0.01887327\\
0.4630463	0.02014955\\
0.4631463	0.01934938\\
0.4632463	0.02165287\\
0.4633463	0.02062681\\
0.4634463	0.02191302\\
0.4635464	0.02127417\\
0.4636464	0.02114741\\
0.4637464	0.02128711\\
0.4638464	0.02214931\\
0.4639464	0.02062174\\
0.4640464	0.02086603\\
0.4641464	0.02283146\\
0.4642464	0.02213815\\
0.4643464	0.02273726\\
0.4644464	0.02296644\\
0.4645465	0.02612196\\
0.4646465	0.02649616\\
0.4647465	0.02482941\\
0.4648465	0.02391892\\
0.4649465	0.02251818\\
0.4650465	0.02429012\\
0.4651465	0.02519081\\
0.4652465	0.02640163\\
0.4653465	0.02608124\\
0.4654465	0.02689656\\
0.4655466	0.02652497\\
0.4656466	0.02618439\\
0.4657466	0.02757963\\
0.4658466	0.02857387\\
0.4659466	0.02857902\\
0.4660466	0.02775965\\
0.4661466	0.02989101\\
0.4662466	0.03087874\\
0.4663466	0.02961877\\
0.4664466	0.02971811\\
0.4665467	0.0301071\\
0.4666467	0.03009142\\
0.4667467	0.02979553\\
0.4668467	0.03076317\\
0.4669467	0.03414691\\
0.4670467	0.0346225\\
0.4671467	0.03410994\\
0.4672467	0.03351754\\
0.4673467	0.0329793\\
0.4674467	0.03285931\\
0.4675468	0.03229013\\
0.4676468	0.03258747\\
0.4677468	0.03246244\\
0.4678468	0.03306896\\
0.4679468	0.03545932\\
0.4680468	0.03503647\\
0.4681468	0.03574091\\
0.4682468	0.03695018\\
0.4683468	0.03686416\\
0.4684468	0.03661286\\
0.4685469	0.03693709\\
0.4686469	0.0366889\\
0.4687469	0.03812335\\
0.4688469	0.03740052\\
0.4689469	0.03861034\\
0.4690469	0.03862094\\
0.4691469	0.03974497\\
0.4692469	0.03970021\\
0.4693469	0.03843782\\
0.4694469	0.03833397\\
0.469547	0.0393313\\
0.469647	0.0399028\\
0.469747	0.03954041\\
0.469847	0.04035407\\
0.469947	0.04109566\\
0.470047	0.04128504\\
0.470147	0.04133477\\
0.470247	0.04004175\\
0.470347	0.03950827\\
0.470447	0.0412532\\
0.4705471	0.04103797\\
0.4706471	0.0428182\\
0.4707471	0.04245767\\
0.4708471	0.04254374\\
0.4709471	0.04342389\\
0.4710471	0.04613651\\
0.4711471	0.04749115\\
0.4712471	0.04667167\\
0.4713471	0.04691609\\
0.4714471	0.04684007\\
0.4715472	0.04662399\\
0.4716472	0.04622712\\
0.4717472	0.04399042\\
0.4718472	0.04511341\\
0.4719472	0.04389608\\
0.4720472	0.04451099\\
0.4721472	0.04477333\\
0.4722472	0.04451413\\
0.4723472	0.0463311\\
0.4724472	0.04599894\\
0.4725473	0.04678\\
0.4726473	0.04706495\\
0.4727473	0.04739345\\
0.4728473	0.04775844\\
0.4729473	0.04869573\\
0.4730473	0.04845642\\
0.4731473	0.04875183\\
0.4732473	0.04929921\\
0.4733473	0.04889464\\
0.4734473	0.04701223\\
0.4735474	0.04728147\\
0.4736474	0.0480906\\
0.4737474	0.05032656\\
0.4738474	0.05075693\\
0.4739474	0.05196242\\
0.4740474	0.05050261\\
0.4741474	0.05166927\\
0.4742474	0.05358159\\
0.4743474	0.05159651\\
0.4744474	0.05094644\\
0.4745475	0.04846442\\
0.4746475	0.04860469\\
0.4747475	0.05006413\\
0.4748475	0.05040152\\
0.4749475	0.05066435\\
0.4750475	0.05168821\\
0.4751475	0.0502846\\
0.4752475	0.0499687\\
0.4753475	0.05138284\\
0.4754475	0.05102113\\
0.4755476	0.05241543\\
0.4756476	0.05129942\\
0.4757476	0.05161951\\
0.4758476	0.05327784\\
0.4759476	0.05344076\\
0.4760476	0.05400866\\
0.4761476	0.05364074\\
0.4762476	0.0537027\\
0.4763476	0.0530125\\
0.4764476	0.05357659\\
0.4765477	0.05114976\\
0.4766477	0.050454\\
0.4767477	0.04803617\\
0.4768477	0.04901996\\
0.4769477	0.05131777\\
0.4770477	0.05200165\\
0.4771477	0.05452249\\
0.4772477	0.05347615\\
0.4773477	0.05528662\\
0.4774477	0.0554633\\
0.4775478	0.05410744\\
0.4776478	0.05271015\\
0.4777478	0.05291201\\
0.4778478	0.05248631\\
0.4779478	0.05247676\\
0.4780478	0.05147325\\
0.4781478	0.04892182\\
0.4782478	0.05158405\\
0.4783478	0.05255283\\
0.4784478	0.05162152\\
0.4785479	0.05124583\\
0.4786479	0.05315072\\
0.4787479	0.0554052\\
0.4788479	0.05418806\\
0.4789479	0.05353326\\
0.4790479	0.0531241\\
0.4791479	0.05347968\\
0.4792479	0.05234358\\
0.4793479	0.05290125\\
0.4794479	0.05193916\\
0.479548	0.04858505\\
0.479648	0.05037416\\
0.479748	0.05031632\\
0.479848	0.04959896\\
0.479948	0.05117981\\
0.480048	0.05031941\\
0.480148	0.04772812\\
0.480248	0.04772273\\
0.480348	0.05005009\\
0.480448	0.05155974\\
0.4805481	0.05260354\\
0.4806481	0.05059499\\
0.4807481	0.04999999\\
0.4808481	0.05017295\\
0.4809481	0.04916861\\
0.4810481	0.04727359\\
0.4811481	0.04785629\\
0.4812481	0.05081445\\
0.4813481	0.05105598\\
0.4814481	0.04873355\\
0.4815482	0.04975066\\
0.4816482	0.05136019\\
0.4817482	0.04918282\\
0.4818482	0.04649334\\
0.4819482	0.04728836\\
0.4820482	0.04797522\\
0.4821482	0.0466216\\
0.4822482	0.04576591\\
0.4823482	0.04490986\\
0.4824482	0.04344477\\
0.4825483	0.04307453\\
0.4826483	0.04359845\\
0.4827483	0.04449097\\
0.4828483	0.04477625\\
0.4829483	0.04478317\\
0.4830483	0.04499499\\
0.4831483	0.04385309\\
0.4832483	0.04356289\\
0.4833483	0.04262091\\
0.4834483	0.04250233\\
0.4835484	0.04289973\\
0.4836484	0.04329353\\
0.4837484	0.04205247\\
0.4838484	0.0428176\\
0.4839484	0.04291891\\
0.4840484	0.04040789\\
0.4841484	0.04027273\\
0.4842484	0.03955861\\
0.4843484	0.04014351\\
0.4844484	0.03744731\\
0.4845485	0.03778616\\
0.4846485	0.03862821\\
0.4847485	0.03658875\\
0.4848485	0.03677427\\
0.4849485	0.03760371\\
0.4850485	0.03678015\\
0.4851485	0.03547935\\
0.4852485	0.0349922\\
0.4853485	0.03414046\\
0.4854485	0.03441997\\
0.4855486	0.03404834\\
0.4856486	0.03482579\\
0.4857486	0.03443611\\
0.4858486	0.03500547\\
0.4859486	0.03492862\\
0.4860486	0.03293101\\
0.4861486	0.03135842\\
0.4862486	0.03066038\\
0.4863486	0.02876183\\
0.4864486	0.02946718\\
0.4865487	0.02974648\\
0.4866487	0.02745972\\
0.4867487	0.02544959\\
0.4868487	0.02768575\\
0.4869487	0.02801914\\
0.4870487	0.02512697\\
0.4871487	0.02535302\\
0.4872487	0.02728626\\
0.4873487	0.02742138\\
0.4874487	0.02497148\\
0.4875488	0.02251906\\
0.4876488	0.02148941\\
0.4877488	0.02027193\\
0.4878488	0.01830221\\
0.4879488	0.01972641\\
0.4880488	0.02093143\\
0.4881488	0.02261146\\
0.4882488	0.02228529\\
0.4883488	0.01918218\\
0.4884488	0.01820175\\
0.4885489	0.01868415\\
0.4886489	0.01763332\\
0.4887489	0.01777998\\
0.4888489	0.0188695\\
0.4889489	0.01500198\\
0.4890489	0.01390324\\
0.4891489	0.01460688\\
0.4892489	0.01135658\\
0.4893489	0.01210945\\
0.4894489	0.01014746\\
0.489549	0.009553548\\
0.489649	0.007771547\\
0.489749	0.0092219\\
0.489849	0.00989197\\
0.489949	0.009454964\\
0.490049	0.008218639\\
0.490149	0.006936891\\
0.490249	0.0079943\\
0.490349	0.004006537\\
0.490449	0.003264365\\
0.4905491	0.002033683\\
0.4906491	0.002364667\\
0.4907491	0.002377611\\
0.4908491	0.0006106816\\
0.4909491	-0.001396205\\
0.4910491	-0.001187345\\
0.4911491	-0.0001803438\\
0.4912491	-0.002930931\\
0.4913491	-0.000942298\\
0.4914491	-0.002009628\\
0.4915492	-0.0005054604\\
0.4916492	-8.931829e-05\\
0.4917492	-0.002942618\\
0.4918492	-0.00375908\\
0.4919492	-0.006876634\\
0.4920492	-0.007530807\\
0.4921492	-0.00935396\\
0.4922492	-0.01016122\\
0.4923492	-0.0122979\\
0.4924492	-0.01304649\\
0.4925493	-0.01480098\\
0.4926493	-0.01395927\\
0.4927493	-0.01303254\\
0.4928493	-0.01497906\\
0.4929493	-0.01377412\\
0.4930493	-0.01654261\\
0.4931493	-0.01558267\\
0.4932493	-0.01592588\\
0.4933493	-0.01855843\\
0.4934493	-0.01861287\\
0.4935494	-0.01906009\\
0.4936494	-0.0182735\\
0.4937494	-0.01889372\\
0.4938494	-0.02003531\\
0.4939494	-0.02425724\\
0.4940494	-0.02506214\\
0.4941494	-0.02717712\\
0.4942494	-0.02759217\\
0.4943494	-0.02787951\\
0.4944494	-0.02889082\\
0.4945495	-0.02807262\\
0.4946495	-0.02940666\\
0.4947495	-0.02938564\\
0.4948495	-0.03085159\\
0.4949495	-0.03020796\\
0.4950495	-0.03320105\\
0.4951495	-0.03344873\\
0.4952495	-0.03435756\\
0.4953495	-0.03652345\\
0.4954495	-0.03651828\\
0.4955496	-0.03643635\\
0.4956496	-0.03779804\\
0.4957496	-0.03789266\\
0.4958496	-0.0388497\\
0.4959496	-0.04065917\\
0.4960496	-0.04213365\\
0.4961496	-0.04077106\\
0.4962496	-0.04024414\\
0.4963496	-0.04074562\\
0.4964496	-0.04170814\\
0.4965497	-0.04408483\\
0.4966497	-0.04491907\\
0.4967497	-0.04868917\\
0.4968497	-0.05089719\\
0.4969497	-0.05179832\\
0.4970497	-0.05163646\\
0.4971497	-0.05209655\\
0.4972497	-0.05441808\\
0.4973497	-0.05447468\\
0.4974497	-0.05191743\\
0.4975498	-0.05280194\\
0.4976498	-0.05481564\\
0.4977498	-0.05496178\\
0.4978498	-0.05458012\\
0.4979498	-0.05745767\\
0.4980498	-0.05733738\\
0.4981498	-0.05962894\\
0.4982498	-0.06183388\\
0.4983498	-0.06188306\\
0.4984498	-0.06240216\\
0.4985499	-0.06215159\\
0.4986499	-0.06104389\\
0.4987499	-0.06274664\\
0.4988499	-0.06371721\\
0.4989499	-0.06414572\\
0.4990499	-0.06549965\\
0.4991499	-0.06780021\\
0.4992499	-0.06923111\\
0.4993499	-0.06882326\\
0.4994499	-0.06968448\\
0.49955	-0.07080132\\
0.49965	-0.07148996\\
0.49975	-0.07150257\\
0.49985	-0.07357667\\
0.49995	-0.07354501\\
0.50005	-0.07439071\\
0.50015	-0.07569246\\
0.50025	-0.07771707\\
0.50035	-0.07809397\\
0.50045	-0.07879205\\
0.5005501	-0.07829745\\
0.5006501	-0.07833715\\
0.5007501	-0.07955804\\
0.5008501	-0.07972021\\
0.5009501	-0.08086914\\
0.5010501	-0.0821745\\
0.5011501	-0.08323768\\
0.5012501	-0.08446765\\
0.5013501	-0.08513865\\
0.5014501	-0.0858169\\
0.5015502	-0.08384373\\
0.5016502	-0.08456517\\
0.5017502	-0.08577067\\
0.5018502	-0.08685336\\
0.5019502	-0.08786845\\
0.5020502	-0.08875974\\
0.5021502	-0.09058365\\
0.5022502	-0.09099294\\
0.5023502	-0.09024145\\
0.5024502	-0.09348451\\
0.5025503	-0.0947248\\
0.5026503	-0.09358229\\
0.5027503	-0.09225948\\
0.5028503	-0.09437116\\
0.5029503	-0.09554766\\
0.5030503	-0.09384639\\
0.5031503	-0.09504409\\
0.5032503	-0.0962104\\
0.5033503	-0.09716924\\
0.5034503	-0.09616388\\
0.5035504	-0.09775429\\
0.5036504	-0.09950342\\
0.5037504	-0.1011858\\
0.5038504	-0.1006278\\
0.5039504	-0.09823044\\
0.5040504	-0.09931444\\
0.5041504	-0.1006148\\
0.5042504	-0.1025118\\
0.5043504	-0.1014513\\
0.5044504	-0.1024873\\
0.5045505	-0.1039082\\
0.5046505	-0.1027224\\
0.5047505	-0.1048977\\
0.5048505	-0.1062745\\
0.5049505	-0.1075863\\
0.5050505	-0.1070569\\
0.5051505	-0.1082052\\
0.5052505	-0.1072528\\
0.5053505	-0.1085983\\
0.5054505	-0.1093814\\
0.5055506	-0.1070836\\
0.5056506	-0.1063587\\
0.5057506	-0.1073153\\
0.5058506	-0.1080631\\
0.5059506	-0.1060956\\
0.5060506	-0.1041395\\
0.5061506	-0.1064943\\
0.5062506	-0.1087199\\
0.5063506	-0.1077876\\
0.5064506	-0.1094401\\
0.5065507	-0.1106226\\
0.5066507	-0.1116279\\
0.5067507	-0.113877\\
0.5068507	-0.1153805\\
0.5069507	-0.1140974\\
0.5070507	-0.1142185\\
0.5071507	-0.112293\\
0.5072507	-0.1122283\\
0.5073507	-0.110858\\
0.5074507	-0.1098676\\
0.5075508	-0.1115539\\
0.5076508	-0.112289\\
0.5077508	-0.1137803\\
0.5078508	-0.1131135\\
0.5079508	-0.1135625\\
0.5080508	-0.113801\\
0.5081508	-0.1143315\\
0.5082508	-0.1136837\\
0.5083508	-0.1138781\\
0.5084508	-0.1125325\\
0.5085509	-0.1138041\\
0.5086509	-0.1138782\\
0.5087509	-0.1123731\\
0.5088509	-0.1122135\\
0.5089509	-0.1112201\\
0.5090509	-0.111732\\
0.5091509	-0.1122618\\
0.5092509	-0.1129284\\
0.5093509	-0.1125664\\
0.5094509	-0.1141158\\
0.509551	-0.1122045\\
0.509651	-0.1116662\\
0.509751	-0.1130874\\
0.509851	-0.112903\\
0.509951	-0.1146432\\
0.510051	-0.1129525\\
0.510151	-0.1117181\\
0.510251	-0.112593\\
0.510351	-0.1115043\\
0.510451	-0.1114934\\
0.5105511	-0.1085942\\
0.5106511	-0.1065647\\
0.5107511	-0.1080362\\
0.5108511	-0.1096568\\
0.5109511	-0.1107745\\
0.5110511	-0.1096664\\
0.5111511	-0.1092761\\
0.5112511	-0.1095291\\
0.5113511	-0.1089453\\
0.5114511	-0.1078084\\
0.5115512	-0.1076006\\
0.5116512	-0.1073363\\
0.5117512	-0.1063334\\
0.5118512	-0.1059653\\
0.5119512	-0.1045706\\
0.5120512	-0.1057116\\
0.5121512	-0.1042402\\
0.5122512	-0.1038075\\
0.5123512	-0.1034962\\
0.5124512	-0.1014645\\
0.5125513	-0.1020577\\
0.5126513	-0.1026529\\
0.5127513	-0.102259\\
0.5128513	-0.1012696\\
0.5129513	-0.100655\\
0.5130513	-0.1011569\\
0.5131513	-0.09848206\\
0.5132513	-0.09722533\\
0.5133513	-0.09655723\\
0.5134513	-0.09584211\\
0.5135514	-0.09730855\\
0.5136514	-0.0944679\\
0.5137514	-0.09200661\\
0.5138514	-0.0910001\\
0.5139514	-0.08973526\\
0.5140514	-0.09065665\\
0.5141514	-0.0916986\\
0.5142514	-0.09079763\\
0.5143514	-0.09118975\\
0.5144514	-0.09159716\\
0.5145515	-0.09026405\\
0.5146515	-0.08778363\\
0.5147515	-0.08662698\\
0.5148515	-0.08559001\\
0.5149515	-0.08561129\\
0.5150515	-0.08436591\\
0.5151515	-0.08249802\\
0.5152515	-0.0817708\\
0.5153515	-0.08021887\\
0.5154515	-0.08056895\\
0.5155516	-0.0780082\\
0.5156516	-0.07777362\\
0.5157516	-0.07702079\\
0.5158516	-0.07696294\\
0.5159516	-0.07573308\\
0.5160516	-0.07208923\\
0.5161516	-0.07039719\\
0.5162516	-0.06987816\\
0.5163516	-0.06754726\\
0.5164516	-0.0668765\\
0.5165517	-0.06531547\\
0.5166517	-0.0650128\\
0.5167517	-0.06500785\\
0.5168517	-0.06393693\\
0.5169517	-0.0645431\\
0.5170517	-0.06383032\\
0.5171517	-0.06385525\\
0.5172517	-0.06202189\\
0.5173517	-0.05899304\\
0.5174517	-0.05576692\\
0.5175518	-0.054763\\
0.5176518	-0.05413847\\
0.5177518	-0.05279548\\
0.5178518	-0.05208435\\
0.5179518	-0.05331337\\
0.5180518	-0.04920273\\
0.5181518	-0.0462865\\
0.5182518	-0.04448338\\
0.5183518	-0.043721\\
0.5184518	-0.04272093\\
0.5185519	-0.04136669\\
0.5186519	-0.04046694\\
0.5187519	-0.03923853\\
0.5188519	-0.03806056\\
0.5189519	-0.03645617\\
0.5190519	-0.03433246\\
0.5191519	-0.03176805\\
0.5192519	-0.03023918\\
0.5193519	-0.03093047\\
0.5194519	-0.03115931\\
0.519552	-0.03031301\\
0.519652	-0.02863452\\
0.519752	-0.02521137\\
0.519852	-0.02278533\\
0.519952	-0.02111478\\
0.520052	-0.01909375\\
0.520152	-0.01548556\\
0.520252	-0.01401571\\
0.520352	-0.01360017\\
0.520452	-0.01264879\\
0.5205521	-0.0109349\\
0.5206521	-0.00960085\\
0.5207521	-0.007708238\\
0.5208521	-0.006706427\\
0.5209521	-0.006046011\\
0.5210521	-0.003629365\\
0.5211521	-0.003413648\\
0.5212521	0.0002009227\\
0.5213521	0.002819771\\
0.5214521	0.004480422\\
0.5215522	0.003967127\\
0.5216522	0.004791923\\
0.5217522	0.008113462\\
0.5218522	0.01076787\\
0.5219522	0.01363856\\
0.5220522	0.01498503\\
0.5221522	0.01706255\\
0.5222522	0.01723921\\
0.5223522	0.01977493\\
0.5224522	0.01898403\\
0.5225523	0.02056635\\
0.5226523	0.02073061\\
0.5227523	0.02267442\\
0.5228523	0.02541007\\
0.5229523	0.02931802\\
0.5230523	0.03227782\\
0.5231523	0.03551075\\
0.5232523	0.03749835\\
0.5233523	0.0394039\\
0.5234523	0.04025698\\
0.5235524	0.04336473\\
0.5236524	0.04395545\\
0.5237524	0.04415232\\
0.5238524	0.04548147\\
0.5239524	0.04568091\\
0.5240524	0.04964227\\
0.5241524	0.05158791\\
0.5242524	0.05258091\\
0.5243524	0.05435589\\
0.5244524	0.05654924\\
0.5245525	0.05786414\\
0.5246525	0.06032951\\
0.5247525	0.06285326\\
0.5248525	0.06594183\\
0.5249525	0.06629657\\
0.5250525	0.06757538\\
0.5251525	0.07001628\\
0.5252525	0.06993157\\
0.5253525	0.0718475\\
0.5254525	0.07438609\\
0.5255526	0.07734395\\
0.5256526	0.07858099\\
0.5257526	0.0777733\\
0.5258526	0.08180694\\
0.5259526	0.08298307\\
0.5260526	0.08660341\\
0.5261526	0.08858994\\
0.5262526	0.09107071\\
0.5263526	0.09555084\\
0.5264526	0.09842584\\
0.5265527	0.0983597\\
0.5266527	0.09673608\\
0.5267527	0.09750909\\
0.5268527	0.1007522\\
0.5269527	0.1012382\\
0.5270527	0.1027605\\
0.5271527	0.1047701\\
0.5272527	0.105659\\
0.5273527	0.1087675\\
0.5274527	0.1094748\\
0.5275528	0.1127769\\
0.5276528	0.1161907\\
0.5277528	0.1168432\\
0.5278528	0.116855\\
0.5279528	0.1202528\\
0.5280528	0.1230256\\
0.5281528	0.1243364\\
0.5282528	0.1254713\\
0.5283528	0.1289152\\
0.5284528	0.1290276\\
0.5285529	0.130483\\
0.5286529	0.1321431\\
0.5287529	0.1329954\\
0.5288529	0.1355181\\
0.5289529	0.1366411\\
0.5290529	0.1385823\\
0.5291529	0.1396928\\
0.5292529	0.1406932\\
0.5293529	0.1410369\\
0.5294529	0.1430026\\
0.529553	0.1468748\\
0.529653	0.1477597\\
0.529753	0.1489818\\
0.529853	0.1511912\\
0.529953	0.151391\\
0.530053	0.1546695\\
0.530153	0.1551757\\
0.530253	0.1546723\\
0.530353	0.1571168\\
0.530453	0.1598691\\
0.5305531	0.1618771\\
0.5306531	0.1640467\\
0.5307531	0.1649383\\
0.5308531	0.1672472\\
0.5309531	0.1693063\\
0.5310531	0.1699405\\
0.5311531	0.1708855\\
0.5312531	0.1711611\\
0.5313531	0.1731169\\
0.5314531	0.1727953\\
0.5315532	0.1720892\\
0.5316532	0.173645\\
0.5317532	0.1750681\\
0.5318532	0.1783389\\
0.5319532	0.180326\\
0.5320532	0.1800344\\
0.5321532	0.1803929\\
0.5322532	0.1813477\\
0.5323532	0.1833745\\
0.5324532	0.1860052\\
0.5325533	0.186422\\
0.5326533	0.1874922\\
0.5327533	0.1887574\\
0.5328533	0.1896989\\
0.5329533	0.189803\\
0.5330533	0.1911569\\
0.5331533	0.1921148\\
0.5332533	0.1925905\\
0.5333533	0.1939788\\
0.5334533	0.1938558\\
0.5335534	0.1960101\\
0.5336534	0.1950495\\
0.5337534	0.1979072\\
0.5338534	0.1993877\\
0.5339534	0.2001688\\
0.5340534	0.2016037\\
0.5341534	0.2032392\\
0.5342534	0.2023931\\
0.5343534	0.2016156\\
0.5344534	0.2015101\\
0.5345535	0.2019098\\
0.5346535	0.2031248\\
0.5347535	0.2033677\\
0.5348535	0.203294\\
0.5349535	0.2052726\\
0.5350535	0.2055086\\
0.5351535	0.205954\\
0.5352535	0.2067285\\
0.5353535	0.2073731\\
0.5354535	0.2062699\\
0.5355536	0.2067257\\
0.5356536	0.2080509\\
0.5357536	0.2076181\\
0.5358536	0.208648\\
0.5359536	0.2085554\\
0.5360536	0.2089435\\
0.5361536	0.2098852\\
0.5362536	0.210668\\
0.5363536	0.2124166\\
0.5364536	0.2119451\\
0.5365537	0.2120658\\
0.5366537	0.2104049\\
0.5367537	0.208553\\
0.5368537	0.2093208\\
0.5369537	0.207457\\
0.5370537	0.2075209\\
0.5371537	0.2080932\\
0.5372537	0.2093502\\
0.5373537	0.2075985\\
0.5374537	0.208489\\
0.5375538	0.2108262\\
0.5376538	0.2108083\\
0.5377538	0.2089782\\
0.5378538	0.2093694\\
0.5379538	0.2083808\\
0.5380538	0.2081082\\
0.5381538	0.2094963\\
0.5382538	0.2058612\\
0.5383538	0.2053358\\
0.5384538	0.2059191\\
0.5385539	0.2045571\\
0.5386539	0.2033547\\
0.5387539	0.2033674\\
0.5388539	0.2046278\\
0.5389539	0.2057504\\
0.5390539	0.2038689\\
0.5391539	0.2023026\\
0.5392539	0.2020528\\
0.5393539	0.2009271\\
0.5394539	0.1986671\\
0.539554	0.1957045\\
0.539654	0.1965071\\
0.539754	0.1940693\\
0.539854	0.1944587\\
0.539954	0.1934204\\
0.540054	0.1905697\\
0.540154	0.1915498\\
0.540254	0.1910961\\
0.540354	0.1890769\\
0.540454	0.1898112\\
0.5405541	0.1899032\\
0.5406541	0.1899974\\
0.5407541	0.1883815\\
0.5408541	0.1881603\\
0.5409541	0.1873256\\
0.5410541	0.1832012\\
0.5411541	0.1810492\\
0.5412541	0.1785847\\
0.5413541	0.1789238\\
0.5414541	0.1766836\\
0.5415542	0.1731302\\
0.5416542	0.1717795\\
0.5417542	0.1699084\\
0.5418542	0.1697161\\
0.5419542	0.1687178\\
0.5420542	0.1649246\\
0.5421542	0.1657607\\
0.5422542	0.1661985\\
0.5423542	0.1652269\\
0.5424542	0.1621621\\
0.5425543	0.1577417\\
0.5426543	0.1559933\\
0.5427543	0.1544138\\
0.5428543	0.1520122\\
0.5429543	0.1513334\\
0.5430543	0.1476288\\
0.5431543	0.1480977\\
0.5432543	0.1442872\\
0.5433543	0.1431364\\
0.5434543	0.1430096\\
0.5435544	0.1396606\\
0.5436544	0.1388804\\
0.5437544	0.1361469\\
0.5438544	0.1338893\\
0.5439544	0.1328048\\
0.5440544	0.1286145\\
0.5441544	0.12692\\
0.5442544	0.1234785\\
0.5443544	0.120563\\
0.5444544	0.1175958\\
0.5445545	0.1151082\\
0.5446545	0.1140905\\
0.5447545	0.1133109\\
0.5448545	0.1119232\\
0.5449545	0.1097516\\
0.5450545	0.1054935\\
0.5451545	0.1021904\\
0.5452545	0.09873992\\
0.5453545	0.09699962\\
0.5454545	0.09447395\\
0.5455546	0.09222824\\
0.5456546	0.09004662\\
0.5457546	0.08775903\\
0.5458546	0.0855558\\
0.5459546	0.08279401\\
0.5460546	0.07756245\\
0.5461546	0.07485582\\
0.5462546	0.07207671\\
0.5463546	0.06848295\\
0.5464546	0.06663937\\
0.5465547	0.06570338\\
0.5466547	0.06568328\\
0.5467547	0.06160454\\
0.5468547	0.05612479\\
0.5469547	0.05354697\\
0.5470547	0.05058864\\
0.5471547	0.04631437\\
0.5472547	0.04220746\\
0.5473547	0.03895495\\
0.5474547	0.0368479\\
0.5475548	0.03436409\\
0.5476548	0.0307826\\
0.5477548	0.03012597\\
0.5478548	0.02705473\\
0.5479548	0.02296344\\
0.5480548	0.01884551\\
0.5481548	0.01754918\\
0.5482548	0.01398767\\
0.5483548	0.01146776\\
0.5484548	0.007781877\\
0.5485549	0.0059727\\
0.5486549	0.003127629\\
0.5487549	0.0005495571\\
0.5488549	-0.003440797\\
0.5489549	-0.00887998\\
0.5490549	-0.01401496\\
0.5491549	-0.01635667\\
0.5492549	-0.01950438\\
0.5493549	-0.02328479\\
0.5494549	-0.02841144\\
0.549555	-0.02976587\\
0.549655	-0.03302161\\
0.549755	-0.0364466\\
0.549855	-0.04130736\\
0.549955	-0.04121823\\
0.550055	-0.04381741\\
0.550155	-0.04810334\\
0.550255	-0.05306583\\
0.550355	-0.05575305\\
0.550455	-0.05983674\\
0.5505551	-0.06507213\\
0.5506551	-0.06678591\\
0.5507551	-0.06771828\\
0.5508551	-0.071171\\
0.5509551	-0.07394781\\
0.5510551	-0.07794993\\
0.5511551	-0.0821168\\
0.5512551	-0.08572875\\
0.5513551	-0.0906912\\
0.5514551	-0.0923753\\
0.5515552	-0.09475954\\
0.5516552	-0.09883961\\
0.5517552	-0.104158\\
0.5518552	-0.1089348\\
0.5519552	-0.1116506\\
0.5520552	-0.1145519\\
0.5521552	-0.1171872\\
0.5522552	-0.1194545\\
0.5523552	-0.1245704\\
0.5524552	-0.1277845\\
0.5525553	-0.1313072\\
0.5526553	-0.1346134\\
0.5527553	-0.1380525\\
0.5528553	-0.1428444\\
0.5529553	-0.1444956\\
0.5530553	-0.1464633\\
0.5531553	-0.1500881\\
0.5532553	-0.1526898\\
0.5533553	-0.156418\\
0.5534553	-0.1598076\\
0.5535554	-0.1650222\\
0.5536554	-0.1682315\\
0.5537554	-0.1697827\\
0.5538554	-0.1718611\\
0.5539554	-0.178032\\
0.5540554	-0.180858\\
0.5541554	-0.1849017\\
0.5542554	-0.1883726\\
0.5543554	-0.1920059\\
0.5544554	-0.195533\\
0.5545555	-0.1983215\\
0.5546555	-0.1998951\\
0.5547555	-0.2037625\\
0.5548555	-0.2059419\\
0.5549555	-0.2087414\\
0.5550555	-0.2103509\\
0.5551555	-0.2136657\\
0.5552555	-0.2163683\\
0.5553555	-0.2197537\\
0.5554555	-0.2224435\\
0.5555556	-0.2269551\\
0.5556556	-0.2283386\\
0.5557556	-0.2331546\\
0.5558556	-0.2393114\\
0.5559556	-0.2414255\\
0.5560556	-0.2440309\\
0.5561556	-0.2458872\\
0.5562556	-0.248695\\
0.5563556	-0.249203\\
0.5564556	-0.2504216\\
0.5565557	-0.2534686\\
0.5566557	-0.2550765\\
0.5567557	-0.2578464\\
0.5568557	-0.2630747\\
0.5569557	-0.2675098\\
0.5570557	-0.2710625\\
0.5571557	-0.2732944\\
0.5572557	-0.2766077\\
0.5573557	-0.2781959\\
0.5574557	-0.2819332\\
0.5575558	-0.2818275\\
0.5576558	-0.2825333\\
0.5577558	-0.2854245\\
0.5578558	-0.2889261\\
0.5579558	-0.2906908\\
0.5580558	-0.2914225\\
0.5581558	-0.2927613\\
0.5582558	-0.2946583\\
0.5583558	-0.2969324\\
0.5584558	-0.3014727\\
0.5585559	-0.3035425\\
0.5586559	-0.3048858\\
0.5587559	-0.306062\\
0.5588559	-0.3097553\\
0.5589559	-0.3087587\\
0.5590559	-0.3130905\\
0.5591559	-0.3155184\\
0.5592559	-0.3160703\\
0.5593559	-0.3178039\\
0.5594559	-0.3204999\\
0.559556	-0.321207\\
0.559656	-0.321743\\
0.559756	-0.324238\\
0.559856	-0.3267242\\
0.559956	-0.3281002\\
0.560056	-0.3300883\\
0.560156	-0.3319503\\
0.560256	-0.3326949\\
0.560356	-0.3344379\\
0.560456	-0.3343871\\
0.5605561	-0.3361361\\
0.5606561	-0.3365905\\
0.5607561	-0.336727\\
0.5608561	-0.3375666\\
0.5609561	-0.3363012\\
0.5610561	-0.3381205\\
0.5611561	-0.3389894\\
0.5612561	-0.3391307\\
0.5613561	-0.3411697\\
0.5614561	-0.3435122\\
0.5615562	-0.3443232\\
0.5616562	-0.3451154\\
0.5617562	-0.3471586\\
0.5618562	-0.3472014\\
0.5619562	-0.3474413\\
0.5620562	-0.3477469\\
0.5621562	-0.3489832\\
0.5622562	-0.3483399\\
0.5623562	-0.3464449\\
0.5624562	-0.3454884\\
0.5625563	-0.3461399\\
0.5626563	-0.3474379\\
0.5627563	-0.3473268\\
0.5628563	-0.3460682\\
0.5629563	-0.346337\\
0.5630563	-0.3484113\\
0.5631563	-0.347188\\
0.5632563	-0.345728\\
0.5633563	-0.3453413\\
0.5634563	-0.3482701\\
0.5635564	-0.3460454\\
0.5636564	-0.3442552\\
0.5637564	-0.3441182\\
0.5638564	-0.3437115\\
0.5639564	-0.3427447\\
0.5640564	-0.3437459\\
0.5641564	-0.3420798\\
0.5642564	-0.3419425\\
0.5643564	-0.3418755\\
0.5644564	-0.3401294\\
0.5645565	-0.3368789\\
0.5646565	-0.3369173\\
0.5647565	-0.3376322\\
0.5648565	-0.336502\\
0.5649565	-0.3344278\\
0.5650565	-0.3312083\\
0.5651565	-0.3303299\\
0.5652565	-0.3291827\\
0.5653565	-0.3288954\\
0.5654565	-0.3256731\\
0.5655566	-0.3244606\\
0.5656566	-0.3224384\\
0.5657566	-0.319165\\
0.5658566	-0.3162626\\
0.5659566	-0.3158067\\
0.5660566	-0.3158731\\
0.5661566	-0.3129407\\
0.5662566	-0.3109218\\
0.5663566	-0.3111879\\
0.5664566	-0.3081425\\
0.5665567	-0.3056955\\
0.5666567	-0.3035208\\
0.5667567	-0.3020683\\
0.5668567	-0.2985287\\
0.5669567	-0.2963625\\
0.5670567	-0.2959508\\
0.5671567	-0.2927651\\
0.5672567	-0.2881254\\
0.5673567	-0.2828539\\
0.5674567	-0.2806006\\
0.5675568	-0.2813522\\
0.5676568	-0.279836\\
0.5677568	-0.2735256\\
0.5678568	-0.2714368\\
0.5679568	-0.2693278\\
0.5680568	-0.2663201\\
0.5681568	-0.2628101\\
0.5682568	-0.2578187\\
0.5683568	-0.2552358\\
0.5684568	-0.2527032\\
0.5685569	-0.2487087\\
0.5686569	-0.2472771\\
0.5687569	-0.2428038\\
0.5688569	-0.2378909\\
0.5689569	-0.2347466\\
0.5690569	-0.2299038\\
0.5691569	-0.2266815\\
0.5692569	-0.2238902\\
0.5693569	-0.2198996\\
0.5694569	-0.216302\\
0.569557	-0.2129804\\
0.569657	-0.208094\\
0.569757	-0.2068195\\
0.569857	-0.2037185\\
0.569957	-0.197854\\
0.570057	-0.1941091\\
0.570157	-0.1891108\\
0.570257	-0.185395\\
0.570357	-0.1813126\\
0.570457	-0.1764971\\
0.5705571	-0.1725102\\
0.5706571	-0.1673826\\
0.5707571	-0.1620042\\
0.5708571	-0.1582419\\
0.5709571	-0.152183\\
0.5710571	-0.1487475\\
0.5711571	-0.1441654\\
0.5712571	-0.1387153\\
0.5713571	-0.1333588\\
0.5714571	-0.1279319\\
0.5715572	-0.1248663\\
0.5716572	-0.1226155\\
0.5717572	-0.119455\\
0.5718572	-0.1145244\\
0.5719572	-0.1099285\\
0.5720572	-0.1058613\\
0.5721572	-0.0979493\\
0.5722572	-0.09160554\\
0.5723572	-0.08494424\\
0.5724572	-0.07845296\\
0.5725573	-0.07350936\\
0.5726573	-0.06929302\\
0.5727573	-0.06622973\\
0.5728573	-0.06069072\\
0.5729573	-0.05738471\\
0.5730573	-0.05147755\\
0.5731573	-0.04526979\\
0.5732573	-0.03758754\\
0.5733573	-0.03260451\\
0.5734573	-0.02822394\\
0.5735574	-0.02306297\\
0.5736574	-0.01798885\\
0.5737574	-0.01294586\\
0.5738574	-0.007760103\\
0.5739574	-0.0008801467\\
0.5740574	0.005356826\\
0.5741574	0.01151046\\
0.5742574	0.01468622\\
0.5743574	0.0198337\\
0.5744574	0.02484528\\
0.5745575	0.02976201\\
0.5746575	0.03443351\\
0.5747575	0.03954302\\
0.5748575	0.04606938\\
0.5749575	0.05191802\\
0.5750575	0.06022607\\
0.5751575	0.06727216\\
0.5752575	0.07174239\\
0.5753575	0.0791827\\
0.5754575	0.08545047\\
0.5755576	0.09068659\\
0.5756576	0.09682487\\
0.5757576	0.1018817\\
0.5758576	0.1068008\\
0.5759576	0.1105933\\
0.5760576	0.1168889\\
0.5761576	0.1211375\\
0.5762576	0.1278509\\
0.5763576	0.132289\\
0.5764576	0.1388969\\
0.5765577	0.1460127\\
0.5766577	0.1527763\\
0.5767577	0.1602765\\
0.5768577	0.1640932\\
0.5769577	0.1692045\\
0.5770577	0.1751224\\
0.5771577	0.1807963\\
0.5772577	0.1846134\\
0.5773577	0.1907603\\
0.5774577	0.194727\\
0.5775578	0.2002398\\
0.5776578	0.2088071\\
0.5777578	0.2145541\\
0.5778578	0.2185379\\
0.5779578	0.2262695\\
0.5780578	0.2308221\\
0.5781578	0.235476\\
0.5782578	0.2433736\\
0.5783578	0.2479664\\
0.5784578	0.2542896\\
0.5785579	0.2583719\\
0.5786579	0.2648796\\
0.5787579	0.2695314\\
0.5788579	0.2736215\\
0.5789579	0.2799565\\
0.5790579	0.2865304\\
0.5791579	0.2913277\\
0.5792579	0.2937816\\
0.5793579	0.2996866\\
0.5794579	0.3047057\\
0.579558	0.3111417\\
0.579658	0.3161048\\
0.579758	0.3216336\\
0.579858	0.326406\\
0.579958	0.3327001\\
0.580058	0.3381705\\
0.580158	0.3432159\\
0.580258	0.3468517\\
0.580358	0.3517432\\
0.580458	0.356201\\
0.5805581	0.3595564\\
0.5806581	0.3635008\\
0.5807581	0.3692118\\
0.5808581	0.3736709\\
0.5809581	0.3780075\\
0.5810581	0.384861\\
0.5811581	0.38916\\
0.5812581	0.3948274\\
0.5813581	0.3982319\\
0.5814581	0.4014329\\
0.5815582	0.4053515\\
0.5816582	0.410188\\
0.5817582	0.4138888\\
0.5818582	0.4170262\\
0.5819582	0.4203574\\
0.5820582	0.4251727\\
0.5821582	0.428126\\
0.5822582	0.4309976\\
0.5823582	0.4372609\\
0.5824582	0.440341\\
0.5825583	0.4428354\\
0.5826583	0.4488532\\
0.5827583	0.4505931\\
0.5828583	0.4534799\\
0.5829583	0.4578332\\
0.5830583	0.4619154\\
0.5831583	0.4670111\\
0.5832583	0.470566\\
0.5833583	0.4729014\\
0.5834583	0.4763219\\
0.5835584	0.479705\\
0.5836584	0.4812195\\
0.5837584	0.4831175\\
0.5838584	0.4838\\
0.5839584	0.4875207\\
0.5840584	0.491329\\
0.5841584	0.4929024\\
0.5842584	0.4935325\\
0.5843584	0.4958136\\
0.5844584	0.4968774\\
0.5845585	0.4989869\\
0.5846585	0.5013885\\
0.5847585	0.5036773\\
0.5848585	0.5089507\\
0.5849585	0.5123144\\
0.5850585	0.5120175\\
0.5851585	0.5134278\\
0.5852585	0.5131035\\
0.5853585	0.5146834\\
0.5854585	0.5181239\\
0.5855586	0.5192873\\
0.5856586	0.5208281\\
0.5857586	0.5220561\\
0.5858586	0.5225552\\
0.5859586	0.5224199\\
0.5860586	0.5229392\\
0.5861586	0.5236925\\
0.5862586	0.5260468\\
0.5863586	0.5258866\\
0.5864586	0.52546\\
0.5865587	0.5261342\\
0.5866587	0.5258221\\
0.5867587	0.5268462\\
0.5868587	0.5262829\\
0.5869587	0.5271998\\
0.5870587	0.5260003\\
0.5871587	0.5261775\\
0.5872587	0.5268661\\
0.5873587	0.525564\\
0.5874587	0.5253947\\
0.5875588	0.5229705\\
0.5876588	0.5230543\\
0.5877588	0.5225027\\
0.5878588	0.5229404\\
0.5879588	0.52047\\
0.5880588	0.5195721\\
0.5881588	0.51876\\
0.5882588	0.5182957\\
0.5883588	0.5188494\\
0.5884588	0.5147766\\
0.5885589	0.5144778\\
0.5886589	0.5127228\\
0.5887589	0.5084787\\
0.5888589	0.5058554\\
0.5889589	0.5038559\\
0.5890589	0.5038477\\
0.5891589	0.500401\\
0.5892589	0.496995\\
0.5893589	0.4948695\\
0.5894589	0.4944248\\
0.589559	0.4903847\\
0.589659	0.4886913\\
0.589759	0.4869552\\
0.589859	0.4851282\\
0.589959	0.4818375\\
0.590059	0.4774138\\
0.590159	0.4751792\\
0.590259	0.4707384\\
0.590359	0.4674598\\
0.590459	0.4629312\\
0.5905591	0.4598873\\
0.5906591	0.4553003\\
0.5907591	0.4518696\\
0.5908591	0.4479314\\
0.5909591	0.4450696\\
0.5910591	0.4414931\\
0.5911591	0.4379103\\
0.5912591	0.4324498\\
0.5913591	0.4293324\\
0.5914591	0.4268125\\
0.5915592	0.4215032\\
0.5916592	0.4147217\\
0.5917592	0.4085948\\
0.5918592	0.4028001\\
0.5919592	0.3974044\\
0.5920592	0.3923149\\
0.5921592	0.38778\\
0.5922592	0.3813821\\
0.5923592	0.3771103\\
0.5924592	0.372933\\
0.5925593	0.3689417\\
0.5926593	0.3642233\\
0.5927593	0.3579026\\
0.5928593	0.3531914\\
0.5929593	0.3484109\\
0.5930593	0.3426156\\
0.5931593	0.334284\\
0.5932593	0.3268897\\
0.5933593	0.3194905\\
0.5934593	0.3151861\\
0.5935594	0.3096016\\
0.5936594	0.3021226\\
0.5937594	0.2965942\\
0.5938594	0.2911584\\
0.5939594	0.2849074\\
0.5940594	0.2785561\\
0.5941594	0.2707408\\
0.5942594	0.264044\\
0.5943594	0.2560288\\
0.5944594	0.2489442\\
0.5945595	0.2409209\\
0.5946595	0.2335388\\
0.5947595	0.2257889\\
0.5948595	0.2189322\\
0.5949595	0.2119373\\
0.5950595	0.2035343\\
0.5951595	0.197209\\
0.5952595	0.1906133\\
0.5953595	0.1825784\\
0.5954595	0.1744499\\
0.5955596	0.1676699\\
0.5956596	0.1609702\\
0.5957596	0.1562704\\
0.5958596	0.1487128\\
0.5959596	0.1396869\\
0.5960596	0.1304898\\
0.5961596	0.12069\\
0.5962596	0.1103064\\
0.5963596	0.1030714\\
0.5964596	0.09560756\\
0.5965597	0.08569961\\
0.5966597	0.07734154\\
0.5967597	0.06983376\\
0.5968597	0.06278966\\
0.5969597	0.05536413\\
0.5970597	0.04708958\\
0.5971597	0.03836371\\
0.5972597	0.0289317\\
0.5973597	0.02073569\\
0.5974597	0.01243219\\
0.5975598	0.003770519\\
0.5976598	-0.004643183\\
0.5977598	-0.01166351\\
0.5978598	-0.01955461\\
0.5979598	-0.02908572\\
0.5980598	-0.0389247\\
0.5981598	-0.04710108\\
0.5982598	-0.05543948\\
0.5983598	-0.06734714\\
0.5984598	-0.07557828\\
0.5985599	-0.08475637\\
0.5986599	-0.09329067\\
0.5987599	-0.1017525\\
0.5988599	-0.1107267\\
0.5989599	-0.118564\\
0.5990599	-0.1263562\\
0.5991599	-0.1342137\\
0.5992599	-0.1448654\\
0.5993599	-0.1538065\\
0.5994599	-0.1645681\\
0.59956	-0.1736361\\
0.59966	-0.1826383\\
0.59976	-0.1915629\\
0.59986	-0.1968014\\
0.59996	-0.2053583\\
0.60006	-0.2148664\\
0.60016	-0.2249608\\
0.60026	-0.2332684\\
0.60036	-0.2413126\\
0.60046	-0.2519885\\
0.6005601	-0.2608994\\
0.6006601	-0.2671697\\
0.6007601	-0.2737899\\
0.6008601	-0.283173\\
0.6009601	-0.293874\\
0.6010601	-0.303665\\
0.6011601	-0.3107389\\
0.6012601	-0.3193661\\
0.6013601	-0.3283254\\
0.6014601	-0.3349791\\
0.6015602	-0.3444088\\
0.6016602	-0.3528409\\
0.6017602	-0.3634265\\
0.6018602	-0.3715682\\
0.6019602	-0.3798231\\
0.6020602	-0.388198\\
0.6021602	-0.3967601\\
0.6022602	-0.4044578\\
0.6023602	-0.4105187\\
0.6024602	-0.4198623\\
0.6025603	-0.4235415\\
0.6026603	-0.4339236\\
0.6027603	-0.4409716\\
0.6028603	-0.4493378\\
0.6029603	-0.4600462\\
0.6030603	-0.4674284\\
0.6031603	-0.4749686\\
0.6032603	-0.4802135\\
0.6033603	-0.4874481\\
0.6034603	-0.4934478\\
0.6035604	-0.5021716\\
0.6036604	-0.5078019\\
0.6037604	-0.515133\\
0.6038604	-0.5235213\\
0.6039604	-0.5311052\\
0.6040604	-0.5373831\\
0.6041604	-0.5419131\\
0.6042604	-0.5510133\\
0.6043604	-0.5565117\\
0.6044604	-0.5647353\\
0.6045605	-0.5722399\\
0.6046605	-0.5775432\\
0.6047605	-0.5835827\\
0.6048605	-0.5896146\\
0.6049605	-0.596506\\
0.6050605	-0.60139\\
0.6051605	-0.6074109\\
0.6052605	-0.6127523\\
0.6053605	-0.6190786\\
0.6054605	-0.6239371\\
0.6055606	-0.6289568\\
0.6056606	-0.6338606\\
0.6057606	-0.6373361\\
0.6058606	-0.642279\\
0.6059606	-0.646774\\
0.6060606	-0.6535928\\
0.6061606	-0.6594579\\
0.6062606	-0.6645544\\
0.6063606	-0.6684116\\
0.6064606	-0.6737201\\
0.6065607	-0.6789819\\
0.6066607	-0.6838391\\
0.6067607	-0.6864051\\
0.6068607	-0.6893765\\
0.6069607	-0.6932641\\
0.6070607	-0.6972586\\
0.6071607	-0.7005731\\
0.6072607	-0.7018074\\
0.6073607	-0.7055758\\
0.6074607	-0.7093173\\
0.6075608	-0.7121575\\
0.6076608	-0.7143177\\
0.6077608	-0.7170247\\
0.6078608	-0.7202933\\
0.6079608	-0.7233485\\
0.6080608	-0.7257596\\
0.6081608	-0.7273702\\
0.6082608	-0.7294282\\
0.6083608	-0.7296946\\
0.6084608	-0.731004\\
0.6085609	-0.7326752\\
0.6086609	-0.7344512\\
0.6087609	-0.7367369\\
0.6088609	-0.7350737\\
0.6089609	-0.739177\\
0.6090609	-0.7422705\\
0.6091609	-0.7435446\\
0.6092609	-0.7442296\\
0.6093609	-0.7440689\\
0.6094609	-0.7448067\\
0.609561	-0.7453612\\
0.609661	-0.74355\\
0.609761	-0.7420071\\
0.609861	-0.7409467\\
0.609961	-0.7407389\\
0.610061	-0.7386777\\
0.610161	-0.7364393\\
0.610261	-0.7371668\\
0.610361	-0.7356286\\
0.610461	-0.734206\\
0.6105611	-0.7337986\\
0.6106611	-0.7332205\\
0.6107611	-0.7327416\\
0.6108611	-0.7317293\\
0.6109611	-0.7274101\\
0.6110611	-0.7263774\\
0.6111611	-0.7241058\\
0.6112611	-0.7209852\\
0.6113611	-0.7185605\\
0.6114611	-0.7134305\\
0.6115612	-0.7089972\\
0.6116612	-0.7048988\\
0.6117612	-0.7036366\\
0.6118612	-0.7005334\\
0.6119612	-0.6981131\\
0.6120612	-0.6962795\\
0.6121612	-0.6907894\\
0.6122612	-0.6871938\\
0.6123612	-0.6842357\\
0.6124612	-0.6808451\\
0.6125613	-0.6750547\\
0.6126613	-0.6689208\\
0.6127613	-0.6628555\\
0.6128613	-0.6580939\\
0.6129613	-0.6537562\\
0.6130613	-0.6483115\\
0.6131613	-0.6418082\\
0.6132613	-0.6356345\\
0.6133613	-0.6296334\\
0.6134613	-0.6248402\\
0.6135614	-0.6184391\\
0.6136614	-0.6141783\\
0.6137614	-0.6080678\\
0.6138614	-0.6012091\\
0.6139614	-0.5926393\\
0.6140614	-0.5864748\\
0.6141614	-0.5817938\\
0.6142614	-0.5720257\\
0.6143614	-0.5651659\\
0.6144614	-0.559041\\
0.6145615	-0.5513878\\
0.6146615	-0.5440202\\
0.6147615	-0.5377377\\
0.6148615	-0.5282812\\
0.6149615	-0.5181744\\
0.6150615	-0.5104009\\
0.6151615	-0.502378\\
0.6152615	-0.4940253\\
0.6153615	-0.4854009\\
0.6154615	-0.476899\\
0.6155616	-0.469271\\
0.6156616	-0.4605978\\
0.6157616	-0.4505429\\
0.6158616	-0.4393939\\
0.6159616	-0.4313603\\
0.6160616	-0.4217738\\
0.6161616	-0.4127358\\
0.6162616	-0.4044989\\
0.6163616	-0.3960331\\
0.6164616	-0.3835985\\
0.6165617	-0.3727208\\
0.6166617	-0.3636873\\
0.6167617	-0.3549452\\
0.6168617	-0.3440993\\
0.6169617	-0.3335208\\
0.6170617	-0.3256752\\
0.6171617	-0.3133312\\
0.6172617	-0.3023026\\
0.6173617	-0.2892323\\
0.6174617	-0.2800123\\
0.6175618	-0.2689573\\
0.6176618	-0.2574243\\
0.6177618	-0.2475917\\
0.6178618	-0.2361489\\
0.6179618	-0.2236081\\
0.6180618	-0.2098054\\
0.6181618	-0.1984132\\
0.6182618	-0.1875219\\
0.6183618	-0.1741015\\
0.6184618	-0.1644376\\
0.6185619	-0.1528557\\
0.6186619	-0.1412643\\
0.6187619	-0.1301998\\
0.6188619	-0.1196064\\
0.6189619	-0.1082259\\
0.6190619	-0.09483391\\
0.6191619	-0.08346869\\
0.6192619	-0.07126268\\
0.6193619	-0.06039592\\
0.6194619	-0.04660791\\
0.619562	-0.03164452\\
0.619662	-0.02185915\\
0.619762	-0.010018\\
0.619862	0.003780405\\
0.619962	0.01671253\\
0.620062	0.02888413\\
0.620162	0.0406673\\
0.620262	0.05504047\\
0.620362	0.06878999\\
0.620462	0.08259588\\
0.6205621	0.09443003\\
0.6206621	0.1049334\\
0.6207621	0.1159583\\
0.6208621	0.129236\\
0.6209621	0.1417075\\
0.6210621	0.1555744\\
0.6211621	0.1666835\\
0.6212621	0.1802523\\
0.6213621	0.1943492\\
0.6214621	0.2080699\\
0.6215622	0.2205038\\
0.6216622	0.2325015\\
0.6217622	0.2466832\\
0.6218622	0.2573494\\
0.6219622	0.267664\\
0.6220622	0.2793688\\
0.6221622	0.2929419\\
0.6222622	0.3070093\\
0.6223622	0.3199015\\
0.6224622	0.3319564\\
0.6225623	0.3469659\\
0.6226623	0.3610631\\
0.6227623	0.3710987\\
0.6228623	0.3811631\\
0.6229623	0.3929723\\
0.6230623	0.4066831\\
0.6231623	0.4188573\\
0.6232623	0.4315508\\
0.6233623	0.4427835\\
0.6234623	0.4535105\\
0.6235624	0.466333\\
0.6236624	0.4779552\\
0.6237624	0.4898744\\
0.6238624	0.5027951\\
0.6239624	0.513034\\
0.6240624	0.5239953\\
0.6241624	0.5359945\\
0.6242624	0.5460243\\
0.6243624	0.5604814\\
0.6244624	0.571035\\
0.6245625	0.5822494\\
0.6246625	0.5942521\\
0.6247625	0.6046176\\
0.6248625	0.613831\\
0.6249625	0.6253184\\
0.6250625	0.6351377\\
0.6251625	0.6457729\\
0.6252625	0.6548539\\
0.6253625	0.6652602\\
0.6254625	0.6765801\\
0.6255626	0.6873595\\
0.6256626	0.6963746\\
0.6257626	0.706224\\
0.6258626	0.719426\\
0.6259626	0.7283306\\
0.6260626	0.7375059\\
0.6261626	0.7467948\\
0.6262626	0.7567607\\
0.6263626	0.7652886\\
0.6264626	0.7710417\\
0.6265627	0.7771716\\
0.6266627	0.7856239\\
0.6267627	0.79437\\
0.6268627	0.801866\\
0.6269627	0.8085863\\
0.6270627	0.8184805\\
0.6271627	0.8273445\\
0.6272627	0.8357644\\
0.6273627	0.845623\\
0.6274627	0.8516984\\
0.6275628	0.8597141\\
0.6276628	0.867534\\
0.6277628	0.8723359\\
0.6278628	0.8787504\\
0.6279628	0.8836968\\
0.6280628	0.8904369\\
0.6281628	0.8972112\\
0.6282628	0.9019159\\
0.6283628	0.9092639\\
0.6284628	0.9140884\\
0.6285629	0.9188302\\
0.6286629	0.9241178\\
0.6287629	0.9289611\\
0.6288629	0.9341239\\
0.6289629	0.9379766\\
0.6290629	0.9437798\\
0.6291629	0.9486591\\
0.6292629	0.9524639\\
0.6293629	0.9550681\\
0.6294629	0.9554019\\
0.629563	0.9573315\\
0.629663	0.9610261\\
0.629763	0.9636591\\
0.629863	0.967962\\
0.629963	0.9717163\\
0.630063	0.9744408\\
0.630163	0.976532\\
0.630263	0.9779149\\
0.630363	0.9810914\\
0.630463	0.9822174\\
0.6305631	0.9826691\\
0.6306631	0.9830308\\
0.6307631	0.9844442\\
0.6308631	0.9860296\\
0.6309631	0.984263\\
0.6310631	0.9841495\\
0.6311631	0.9844666\\
0.6312631	0.983149\\
0.6313631	0.9812427\\
0.6314631	0.9790949\\
0.6315632	0.9799965\\
0.6316632	0.9804686\\
0.6317632	0.979102\\
0.6318632	0.976317\\
0.6319632	0.973938\\
0.6320632	0.9723179\\
0.6321632	0.9687649\\
0.6322632	0.9667738\\
0.6323632	0.9644818\\
0.6324632	0.9623354\\
0.6325633	0.9583243\\
0.6326633	0.9569914\\
0.6327633	0.9536552\\
0.6328633	0.9469641\\
0.6329633	0.9417764\\
0.6330633	0.9385153\\
0.6331633	0.9339815\\
0.6332633	0.9277219\\
0.6333633	0.9243626\\
0.6334633	0.9181244\\
0.6335634	0.9114814\\
0.6336634	0.9040204\\
0.6337634	0.8981336\\
0.6338634	0.8923042\\
0.6339634	0.8867154\\
0.6340634	0.8809978\\
0.6341634	0.8737847\\
0.6342634	0.8674924\\
0.6343634	0.8606286\\
0.6344634	0.8507553\\
0.6345635	0.8433949\\
0.6346635	0.8359205\\
0.6347635	0.8271465\\
0.6348635	0.8175585\\
0.6349635	0.8092055\\
0.6350635	0.8008958\\
0.6351635	0.7916557\\
0.6352635	0.7838952\\
0.6353635	0.776256\\
0.6354635	0.7649181\\
0.6355636	0.7539493\\
0.6356636	0.7448599\\
0.6357636	0.7334679\\
0.6358636	0.7241723\\
0.6359636	0.7137243\\
0.6360636	0.7038756\\
0.6361636	0.692423\\
0.6362636	0.6799698\\
0.6363636	0.668648\\
0.6364636	0.6553832\\
0.6365637	0.6433157\\
0.6366637	0.6333876\\
0.6367637	0.619843\\
0.6368637	0.6074383\\
0.6369637	0.5949593\\
0.6370637	0.5818118\\
0.6371637	0.5689457\\
0.6372637	0.5563395\\
0.6373637	0.5432207\\
0.6374637	0.5292386\\
0.6375638	0.5183807\\
0.6376638	0.5036056\\
0.6377638	0.4885553\\
0.6378638	0.4738326\\
0.6379638	0.4619828\\
0.6380638	0.4472431\\
0.6381638	0.4328379\\
0.6382638	0.4181718\\
0.6383638	0.4053602\\
0.6384638	0.3905116\\
0.6385639	0.3754804\\
0.6386639	0.3602735\\
0.6387639	0.3426249\\
0.6388639	0.3296156\\
0.6389639	0.3153863\\
0.6390639	0.2986648\\
0.6391639	0.2815677\\
0.6392639	0.2638724\\
0.6393639	0.2476275\\
0.6394639	0.2331898\\
0.639564	0.2169115\\
0.639664	0.2021424\\
0.639764	0.1845734\\
0.639864	0.1694301\\
0.639964	0.1531685\\
0.640064	0.136272\\
0.640164	0.1226766\\
0.640264	0.1056902\\
0.640364	0.08884729\\
0.640464	0.07135484\\
0.6405641	0.05528891\\
0.6406641	0.03882734\\
0.6407641	0.02095163\\
0.6408641	0.003509135\\
0.6409641	-0.01370545\\
0.6410641	-0.0334195\\
0.6411641	-0.05030319\\
0.6412641	-0.06630842\\
0.6413641	-0.08460978\\
0.6414641	-0.1021427\\
0.6415642	-0.1184643\\
0.6416642	-0.1322058\\
0.6417642	-0.1495002\\
0.6418642	-0.1669041\\
0.6419642	-0.184523\\
0.6420642	-0.2028111\\
0.6421642	-0.2191042\\
0.6422642	-0.2365317\\
0.6423642	-0.2561334\\
0.6424642	-0.2722143\\
0.6425643	-0.289507\\
0.6426643	-0.3042803\\
0.6427643	-0.3222821\\
0.6428643	-0.3387062\\
0.6429643	-0.3542575\\
0.6430643	-0.3732364\\
0.6431643	-0.387892\\
0.6432643	-0.4060705\\
0.6433643	-0.4234275\\
0.6434643	-0.4376072\\
0.6435644	-0.4562312\\
0.6436644	-0.4730438\\
0.6437644	-0.4906648\\
0.6438644	-0.5070081\\
0.6439644	-0.5241321\\
0.6440644	-0.5411175\\
0.6441644	-0.5558884\\
0.6442644	-0.5739071\\
0.6443644	-0.5914144\\
0.6444644	-0.6065705\\
0.6445645	-0.6210798\\
0.6446645	-0.6360386\\
0.6447645	-0.6507856\\
0.6448645	-0.6657332\\
0.6449645	-0.6782512\\
0.6450645	-0.6928359\\
0.6451645	-0.7069228\\
0.6452645	-0.7242563\\
0.6453645	-0.7415748\\
0.6454645	-0.7573866\\
0.6455646	-0.7732614\\
0.6456646	-0.7879714\\
0.6457646	-0.8010152\\
0.6458646	-0.8146008\\
0.6459646	-0.8257998\\
0.6460646	-0.8392747\\
0.6461646	-0.851361\\
0.6462646	-0.8656624\\
0.6463646	-0.8782148\\
0.6464646	-0.8924857\\
0.6465647	-0.9058562\\
0.6466647	-0.9181726\\
0.6467647	-0.9314663\\
0.6468647	-0.9447922\\
0.6469647	-0.9563881\\
0.6470647	-0.9674343\\
0.6471647	-0.979981\\
0.6472647	-0.9918055\\
0.6473647	-1.002325\\
0.6474647	-1.012241\\
0.6475648	-1.023163\\
0.6476648	-1.03374\\
0.6477648	-1.045527\\
0.6478648	-1.05317\\
0.6479648	-1.063648\\
0.6480648	-1.073648\\
0.6481648	-1.082813\\
0.6482648	-1.092526\\
0.6483648	-1.099723\\
0.6484648	-1.107148\\
0.6485649	-1.117677\\
0.6486649	-1.126163\\
0.6487649	-1.135192\\
0.6488649	-1.141698\\
0.6489649	-1.146861\\
0.6490649	-1.155468\\
0.6491649	-1.161966\\
0.6492649	-1.16787\\
0.6493649	-1.176313\\
0.6494649	-1.183963\\
0.649565	-1.188595\\
0.649665	-1.194025\\
0.649765	-1.198123\\
0.649865	-1.202645\\
0.649965	-1.205944\\
0.650065	-1.209536\\
0.650165	-1.214698\\
0.650265	-1.217767\\
0.650365	-1.222037\\
0.650465	-1.223561\\
0.6505651	-1.2267\\
0.6506651	-1.229684\\
0.6507651	-1.230511\\
0.6508651	-1.23331\\
0.6509651	-1.235793\\
0.6510651	-1.238442\\
0.6511651	-1.239801\\
0.6512651	-1.240634\\
0.6513651	-1.240482\\
0.6514651	-1.24105\\
0.6515652	-1.240628\\
0.6516652	-1.240148\\
0.6517652	-1.238571\\
0.6518652	-1.238029\\
0.6519652	-1.236803\\
0.6520652	-1.234231\\
0.6521652	-1.232643\\
0.6522652	-1.231315\\
0.6523652	-1.227588\\
0.6524652	-1.223416\\
0.6525653	-1.221535\\
0.6526653	-1.218043\\
0.6527653	-1.210852\\
0.6528653	-1.20455\\
0.6529653	-1.200831\\
0.6530653	-1.196353\\
0.6531653	-1.193295\\
0.6532653	-1.187845\\
0.6533653	-1.181696\\
0.6534653	-1.175931\\
0.6535654	-1.170114\\
0.6536654	-1.1628\\
0.6537654	-1.155849\\
0.6538654	-1.151135\\
0.6539654	-1.1424\\
0.6540654	-1.133351\\
0.6541654	-1.125405\\
0.6542654	-1.116942\\
0.6543654	-1.107504\\
0.6544654	-1.097243\\
0.6545655	-1.089775\\
0.6546655	-1.079414\\
0.6547655	-1.067846\\
0.6548655	-1.056556\\
0.6549655	-1.047026\\
0.6550655	-1.036821\\
0.6551655	-1.02742\\
0.6552655	-1.015626\\
0.6553655	-1.00206\\
0.6554655	-0.9900925\\
0.6555656	-0.9792766\\
0.6556656	-0.9652888\\
0.6557656	-0.9511677\\
0.6558656	-0.9380983\\
0.6559656	-0.9233911\\
0.6560656	-0.9115974\\
0.6561656	-0.8986069\\
0.6562656	-0.8854656\\
0.6563656	-0.8683778\\
0.6564656	-0.8535267\\
0.6565657	-0.838646\\
0.6566657	-0.8235958\\
0.6567657	-0.8092166\\
0.6568657	-0.7951017\\
0.6569657	-0.7787834\\
0.6570657	-0.7620683\\
0.6571657	-0.744632\\
0.6572657	-0.7296495\\
0.6573657	-0.7118641\\
0.6574657	-0.6920486\\
0.6575658	-0.674339\\
0.6576658	-0.657213\\
0.6577658	-0.6402402\\
0.6578658	-0.622713\\
0.6579658	-0.6035454\\
0.6580658	-0.5864869\\
0.6581658	-0.5691233\\
0.6582658	-0.5513346\\
0.6583658	-0.5335029\\
0.6584658	-0.5129371\\
0.6585659	-0.4939207\\
0.6586659	-0.4746149\\
0.6587659	-0.4533226\\
0.6588659	-0.4344942\\
0.6589659	-0.4158371\\
0.6590659	-0.3950609\\
0.6591659	-0.3730207\\
0.6592659	-0.3514311\\
0.6593659	-0.3308382\\
0.6594659	-0.3122758\\
0.659566	-0.2924922\\
0.659666	-0.2721751\\
0.659766	-0.2529796\\
0.659866	-0.2318324\\
0.659966	-0.209038\\
0.660066	-0.1869103\\
0.660166	-0.1658425\\
0.660266	-0.1450473\\
0.660366	-0.1228431\\
0.660466	-0.1001813\\
0.6605661	-0.07911126\\
0.6606661	-0.05662403\\
0.6607661	-0.03298199\\
0.6608661	-0.01084429\\
0.6609661	0.0133016\\
0.6610661	0.03294084\\
0.6611661	0.05076606\\
0.6612661	0.07351752\\
0.6613661	0.0955517\\
0.6614661	0.114738\\
0.6615662	0.1361028\\
0.6616662	0.1589549\\
0.6617662	0.1819698\\
0.6618662	0.2062034\\
0.6619662	0.226573\\
0.6620662	0.2493311\\
0.6621662	0.2711999\\
0.6622662	0.2938666\\
0.6623662	0.3158179\\
0.6624662	0.3386016\\
0.6625663	0.3620413\\
0.6626663	0.3840602\\
0.6627663	0.4049207\\
0.6628663	0.4262206\\
0.6629663	0.4481243\\
0.6630663	0.4688207\\
0.6631663	0.4904052\\
0.6632663	0.5114209\\
0.6633663	0.5335721\\
0.6634663	0.5566396\\
0.6635664	0.5760226\\
0.6636664	0.5980527\\
0.6637664	0.618308\\
0.6638664	0.6379077\\
0.6639664	0.6597427\\
0.6640664	0.6792909\\
0.6641664	0.7008655\\
0.6642664	0.7200803\\
0.6643664	0.7386814\\
0.6644664	0.7605364\\
0.6645665	0.7798426\\
0.6646665	0.799194\\
0.6647665	0.8190941\\
0.6648665	0.838547\\
0.6649665	0.8574582\\
0.6650665	0.8786274\\
0.6651665	0.8991152\\
0.6652665	0.9170494\\
0.6653665	0.9358478\\
0.6654665	0.9515264\\
0.6655666	0.9689608\\
0.6656666	0.9854462\\
0.6657666	1.002869\\
0.6658666	1.019605\\
0.6659666	1.036826\\
0.6660666	1.054051\\
0.6661666	1.071807\\
0.6662666	1.088732\\
0.6663666	1.104144\\
0.6664666	1.121214\\
0.6665667	1.138348\\
0.6666667	1.150756\\
0.6667667	1.164691\\
0.6668667	1.17882\\
0.6669667	1.193583\\
0.6670667	1.207691\\
0.6671667	1.221531\\
0.6672667	1.235388\\
0.6673667	1.251123\\
0.6674667	1.26424\\
0.6675668	1.278123\\
0.6676668	1.290152\\
0.6677668	1.300642\\
0.6678668	1.310751\\
0.6679668	1.322029\\
0.6680668	1.332685\\
0.6681668	1.344315\\
0.6682668	1.35371\\
0.6683668	1.362686\\
0.6684668	1.373045\\
0.6685669	1.383511\\
0.6686669	1.394026\\
0.6687669	1.402748\\
0.6688669	1.408109\\
0.6689669	1.415246\\
0.6690669	1.423447\\
0.6691669	1.430389\\
0.6692669	1.436917\\
0.6693669	1.442789\\
0.6694669	1.451119\\
0.669567	1.455744\\
0.669667	1.459134\\
0.669767	1.464572\\
0.669867	1.470365\\
0.669967	1.474266\\
0.670067	1.477446\\
0.670167	1.482411\\
0.670267	1.485491\\
0.670367	1.488075\\
0.670467	1.489138\\
0.6705671	1.491571\\
0.6706671	1.494646\\
0.6707671	1.496294\\
0.6708671	1.495407\\
0.6709671	1.494522\\
0.6710671	1.492238\\
0.6711671	1.492327\\
0.6712671	1.490227\\
0.6713671	1.48778\\
0.6714671	1.486943\\
0.6715672	1.486632\\
0.6716672	1.483557\\
0.6717672	1.481701\\
0.6718672	1.476341\\
0.6719672	1.472565\\
0.6720672	1.467415\\
0.6721672	1.462196\\
0.6722672	1.456465\\
0.6723672	1.452148\\
0.6724672	1.444564\\
0.6725673	1.438746\\
0.6726673	1.431489\\
0.6727673	1.424473\\
0.6728673	1.418754\\
0.6729673	1.411033\\
0.6730673	1.401981\\
0.6731673	1.394463\\
0.6732673	1.384364\\
0.6733673	1.373808\\
0.6734673	1.366856\\
0.6735674	1.356862\\
0.6736674	1.344983\\
0.6737674	1.333369\\
0.6738674	1.318913\\
0.6739674	1.306904\\
0.6740674	1.2927\\
0.6741674	1.280891\\
0.6742674	1.267615\\
0.6743674	1.25575\\
0.6744674	1.24121\\
0.6745675	1.227801\\
0.6746675	1.213119\\
0.6747675	1.19743\\
0.6748675	1.185005\\
0.6749675	1.169569\\
0.6750675	1.153677\\
0.6751675	1.1381\\
0.6752675	1.120777\\
0.6753675	1.105873\\
0.6754675	1.087567\\
0.6755676	1.066014\\
0.6756676	1.048046\\
0.6757676	1.029125\\
0.6758676	1.010684\\
0.6759676	0.9909688\\
0.6760676	0.9704198\\
0.6761676	0.9507864\\
0.6762676	0.9326198\\
0.6763676	0.9126145\\
0.6764676	0.8925633\\
0.6765677	0.8722919\\
0.6766677	0.8524105\\
0.6767677	0.8314129\\
0.6768677	0.8089686\\
0.6769677	0.7857064\\
0.6770677	0.7641112\\
0.6771677	0.7413705\\
0.6772677	0.7191042\\
0.6773677	0.6949521\\
0.6774677	0.672985\\
0.6775678	0.6500975\\
0.6776678	0.6276704\\
0.6777678	0.6030083\\
0.6778678	0.5792675\\
0.6779678	0.5546048\\
0.6780678	0.5322738\\
0.6781678	0.5059641\\
0.6782678	0.4817021\\
0.6783678	0.4567177\\
0.6784678	0.4332941\\
0.6785679	0.4073491\\
0.6786679	0.3785134\\
0.6787679	0.3514301\\
0.6788679	0.3262424\\
0.6789679	0.3004955\\
0.6790679	0.2746969\\
0.6791679	0.2493886\\
0.6792679	0.2241776\\
0.6793679	0.1980281\\
0.6794679	0.1719182\\
0.679568	0.1439211\\
0.679668	0.1167372\\
0.679768	0.09102521\\
0.679868	0.06443987\\
0.679968	0.03772868\\
0.680068	0.009966518\\
0.680168	-0.01787349\\
0.680268	-0.0432134\\
0.680368	-0.06953062\\
0.680468	-0.098116\\
0.6805681	-0.1259532\\
0.6806681	-0.1535479\\
0.6807681	-0.1785849\\
0.6808681	-0.2049253\\
0.6809681	-0.2322054\\
0.6810681	-0.2597793\\
0.6811681	-0.2853156\\
0.6812681	-0.3137779\\
0.6813681	-0.3412622\\
0.6814681	-0.3682446\\
0.6815682	-0.3947774\\
0.6816682	-0.423746\\
0.6817682	-0.4502435\\
0.6818682	-0.4776727\\
0.6819682	-0.5042776\\
0.6820682	-0.5314602\\
0.6821682	-0.5554055\\
0.6822682	-0.5819725\\
0.6823682	-0.6059123\\
0.6824682	-0.6327752\\
0.6825683	-0.6607583\\
0.6826683	-0.6843445\\
0.6827683	-0.711527\\
0.6828683	-0.7374343\\
0.6829683	-0.762193\\
0.6830683	-0.7874027\\
0.6831683	-0.8114499\\
0.6832683	-0.836851\\
0.6833683	-0.8604731\\
0.6834683	-0.8849752\\
0.6835684	-0.9108389\\
0.6836684	-0.9354867\\
0.6837684	-0.9578882\\
0.6838684	-0.9801589\\
0.6839684	-1.004228\\
0.6840684	-1.026269\\
0.6841684	-1.050702\\
0.6842684	-1.073311\\
0.6843684	-1.096776\\
0.6844684	-1.117114\\
0.6845685	-1.138514\\
0.6846685	-1.161502\\
0.6847685	-1.181748\\
0.6848685	-1.200345\\
0.6849685	-1.219196\\
0.6850685	-1.242376\\
0.6851685	-1.260994\\
0.6852685	-1.277555\\
0.6853685	-1.297106\\
0.6854685	-1.318682\\
0.6855686	-1.33582\\
0.6856686	-1.353781\\
0.6857686	-1.373364\\
0.6858686	-1.392444\\
0.6859686	-1.407791\\
0.6860686	-1.423425\\
0.6861686	-1.439387\\
0.6862686	-1.45744\\
0.6863686	-1.473434\\
0.6864686	-1.487383\\
0.6865687	-1.50087\\
0.6866687	-1.515829\\
0.6867687	-1.529463\\
0.6868687	-1.543875\\
0.6869687	-1.558669\\
0.6870687	-1.571565\\
0.6871687	-1.582658\\
0.6872687	-1.593872\\
0.6873687	-1.603661\\
0.6874687	-1.613392\\
0.6875688	-1.621733\\
0.6876688	-1.628766\\
0.6877688	-1.639304\\
0.6878688	-1.64867\\
0.6879688	-1.659347\\
0.6880688	-1.668543\\
0.6881688	-1.676734\\
0.6882688	-1.683357\\
0.6883688	-1.690899\\
0.6884688	-1.697719\\
0.6885689	-1.702762\\
0.6886689	-1.708146\\
0.6887689	-1.712893\\
0.6888689	-1.71598\\
0.6889689	-1.718616\\
0.6890689	-1.721942\\
0.6891689	-1.724379\\
0.6892689	-1.725577\\
0.6893689	-1.726495\\
0.6894689	-1.729386\\
0.689569	-1.731669\\
0.689669	-1.730054\\
0.689769	-1.729359\\
0.689869	-1.729507\\
0.689969	-1.728435\\
0.690069	-1.727493\\
0.690169	-1.724007\\
0.690269	-1.721112\\
0.690369	-1.71802\\
0.690469	-1.71498\\
0.6905691	-1.709376\\
0.6906691	-1.703152\\
0.6907691	-1.699746\\
0.6908691	-1.691731\\
0.6909691	-1.68291\\
0.6910691	-1.674625\\
0.6911691	-1.667582\\
0.6912691	-1.660462\\
0.6913691	-1.655718\\
0.6914691	-1.647861\\
0.6915692	-1.639383\\
0.6916692	-1.628334\\
0.6917692	-1.616837\\
0.6918692	-1.604845\\
0.6919692	-1.591749\\
0.6920692	-1.57982\\
0.6921692	-1.567208\\
0.6922692	-1.555058\\
0.6923692	-1.54132\\
0.6924692	-1.528275\\
0.6925693	-1.512329\\
0.6926693	-1.498187\\
0.6927693	-1.48152\\
0.6928693	-1.465792\\
0.6929693	-1.450294\\
0.6930693	-1.434172\\
0.6931693	-1.417662\\
0.6932693	-1.400548\\
0.6933693	-1.381109\\
0.6934693	-1.362802\\
0.6935694	-1.344638\\
0.6936694	-1.32531\\
0.6937694	-1.306621\\
0.6938694	-1.283989\\
0.6939694	-1.264004\\
0.6940694	-1.243846\\
0.6941694	-1.221559\\
0.6942694	-1.196707\\
0.6943694	-1.176229\\
0.6944694	-1.155469\\
0.6945695	-1.133655\\
0.6946695	-1.110891\\
0.6947695	-1.085773\\
0.6948695	-1.061665\\
0.6949695	-1.035097\\
0.6950695	-1.011391\\
0.6951695	-0.9844254\\
0.6952695	-0.9582626\\
0.6953695	-0.9318239\\
0.6954695	-0.9048456\\
0.6955696	-0.8791715\\
0.6956696	-0.8524912\\
0.6957696	-0.8251592\\
0.6958696	-0.8001306\\
0.6959696	-0.7749749\\
0.6960696	-0.7440112\\
0.6961696	-0.7158347\\
0.6962696	-0.6888587\\
0.6963696	-0.6611983\\
0.6964696	-0.6338425\\
0.6965697	-0.6032485\\
0.6966697	-0.5723057\\
0.6967697	-0.5435137\\
0.6968697	-0.5133999\\
0.6969697	-0.4827621\\
0.6970697	-0.4523009\\
0.6971697	-0.422928\\
0.6972697	-0.3908439\\
0.6973697	-0.360547\\
0.6974697	-0.3304588\\
0.6975698	-0.2991363\\
0.6976698	-0.2672445\\
0.6977698	-0.2356696\\
0.6978698	-0.2053395\\
0.6979698	-0.1724723\\
0.6980698	-0.1414313\\
0.6981698	-0.1077704\\
0.6982698	-0.07666737\\
0.6983698	-0.04596325\\
0.6984698	-0.01264143\\
0.6985699	0.01826836\\
0.6986699	0.04932491\\
0.6987699	0.07993942\\
0.6988699	0.1128473\\
0.6989699	0.1445635\\
0.6990699	0.1769399\\
0.6991699	0.2090573\\
0.6992699	0.2414625\\
0.6993699	0.273015\\
0.6994699	0.3055225\\
0.69957	0.338338\\
0.69967	0.3704445\\
0.69977	0.4020552\\
0.69987	0.4330817\\
0.69997	0.4660505\\
0.70007	0.4994144\\
0.70017	0.530174\\
0.70027	0.5605002\\
0.70037	0.5908332\\
0.70047	0.6206875\\
0.7005701	0.6499996\\
0.7006701	0.6823187\\
0.7007701	0.712659\\
0.7008701	0.7434608\\
0.7009701	0.7732705\\
0.7010701	0.8050791\\
0.7011701	0.8344985\\
0.7012701	0.862538\\
0.7013701	0.8938721\\
0.7014701	0.9222585\\
0.7015702	0.9516828\\
0.7016702	0.9783992\\
0.7017702	1.007561\\
0.7018702	1.037293\\
0.7019702	1.066951\\
0.7020702	1.093068\\
0.7021702	1.1197\\
0.7022702	1.146083\\
0.7023702	1.173165\\
0.7024702	1.199805\\
0.7025703	1.225828\\
0.7026703	1.251916\\
0.7027703	1.275317\\
0.7028703	1.302246\\
0.7029703	1.326972\\
0.7030703	1.349648\\
0.7031703	1.373637\\
0.7032703	1.395199\\
0.7033703	1.419526\\
0.7034703	1.443939\\
0.7035704	1.466751\\
0.7036704	1.488132\\
0.7037704	1.508653\\
0.7038704	1.52857\\
0.7039704	1.549382\\
0.7040704	1.570871\\
0.7041704	1.588029\\
0.7042704	1.605289\\
0.7043704	1.622789\\
0.7044704	1.640834\\
0.7045705	1.658112\\
0.7046705	1.676384\\
0.7047705	1.694601\\
0.7048705	1.711683\\
0.7049705	1.72752\\
0.7050705	1.744953\\
0.7051705	1.759773\\
0.7052705	1.773667\\
0.7053705	1.786403\\
0.7054705	1.798846\\
0.7055706	1.812948\\
0.7056706	1.823376\\
0.7057706	1.833009\\
0.7058706	1.842055\\
0.7059706	1.852417\\
0.7060706	1.863167\\
0.7061706	1.871117\\
0.7062706	1.879431\\
0.7063706	1.888371\\
0.7064706	1.895636\\
0.7065707	1.902125\\
0.7066707	1.908723\\
0.7067707	1.915398\\
0.7068707	1.920939\\
0.7069707	1.924658\\
0.7070707	1.927677\\
0.7071707	1.929068\\
0.7072707	1.929664\\
0.7073707	1.931794\\
0.7074707	1.933036\\
0.7075708	1.934657\\
0.7076708	1.937699\\
0.7077708	1.938381\\
0.7078708	1.937534\\
0.7079708	1.934289\\
0.7080708	1.931474\\
0.7081708	1.928222\\
0.7082708	1.923901\\
0.7083708	1.921118\\
0.7084708	1.916009\\
0.7085709	1.910484\\
0.7086709	1.903116\\
0.7087709	1.895252\\
0.7088709	1.888243\\
0.7089709	1.881939\\
0.7090709	1.871834\\
0.7091709	1.863355\\
0.7092709	1.852464\\
0.7093709	1.842496\\
0.7094709	1.833383\\
0.709571	1.820851\\
0.709671	1.810468\\
0.709771	1.797882\\
0.709871	1.784792\\
0.709971	1.77157\\
0.710071	1.759229\\
0.710171	1.743442\\
0.710271	1.725242\\
0.710371	1.709391\\
0.710471	1.691114\\
0.7105711	1.672989\\
0.7106711	1.655943\\
0.7107711	1.636986\\
0.7108711	1.616419\\
0.7109711	1.597844\\
0.7110711	1.579676\\
0.7111711	1.558445\\
0.7112711	1.538704\\
0.7113711	1.515458\\
0.7114711	1.49476\\
0.7115712	1.469979\\
0.7116712	1.447652\\
0.7117712	1.424278\\
0.7118712	1.399379\\
0.7119712	1.375101\\
0.7120712	1.349684\\
0.7121712	1.324861\\
0.7122712	1.298473\\
0.7123712	1.274667\\
0.7124712	1.248028\\
0.7125713	1.221044\\
0.7126713	1.19434\\
0.7127713	1.162544\\
0.7128713	1.133309\\
0.7129713	1.105609\\
0.7130713	1.077622\\
0.7131713	1.046073\\
0.7132713	1.017082\\
0.7133713	0.9873728\\
0.7134713	0.9543833\\
0.7135714	0.9249503\\
0.7136714	0.8903955\\
0.7137714	0.8586089\\
0.7138714	0.8263054\\
0.7139714	0.795427\\
0.7140714	0.7618266\\
0.7141714	0.7295451\\
0.7142714	0.6979791\\
0.7143714	0.6645662\\
0.7144714	0.6289534\\
0.7145715	0.5937764\\
0.7146715	0.5618192\\
0.7147715	0.525251\\
0.7148715	0.4888999\\
0.7149715	0.4546392\\
0.7150715	0.4196565\\
0.7151715	0.3845942\\
0.7152715	0.348292\\
0.7153715	0.3139927\\
0.7154715	0.2792467\\
0.7155716	0.2456518\\
0.7156716	0.2094583\\
0.7157716	0.1732308\\
0.7158716	0.1376783\\
0.7159716	0.1002162\\
0.7160716	0.06217393\\
0.7161716	0.02477132\\
0.7162716	-0.01030849\\
0.7163716	-0.04899473\\
0.7164716	-0.08620704\\
0.7165717	-0.1227865\\
0.7166717	-0.1569022\\
0.7167717	-0.1938565\\
0.7168717	-0.2297956\\
0.7169717	-0.2684034\\
0.7170717	-0.3062014\\
0.7171717	-0.3412557\\
0.7172717	-0.3764469\\
0.7173717	-0.4118683\\
0.7174717	-0.4504596\\
0.7175718	-0.4839909\\
0.7176718	-0.5192101\\
0.7177718	-0.554055\\
0.7178718	-0.589946\\
0.7179718	-0.6263283\\
0.7180718	-0.6634458\\
0.7181718	-0.6993897\\
0.7182718	-0.7340372\\
0.7183718	-0.7709273\\
0.7184718	-0.8031554\\
0.7185719	-0.8368948\\
0.7186719	-0.8703826\\
0.7187719	-0.9057777\\
0.7188719	-0.9411451\\
0.7189719	-0.9739626\\
0.7190719	-1.007838\\
0.7191719	-1.040473\\
0.7192719	-1.071467\\
0.7193719	-1.101222\\
0.7194719	-1.132546\\
0.719572	-1.164508\\
0.719672	-1.196356\\
0.719772	-1.225897\\
0.719872	-1.255125\\
0.719972	-1.287416\\
0.720072	-1.317223\\
0.720172	-1.344936\\
0.720272	-1.373301\\
0.720372	-1.402633\\
0.720472	-1.432914\\
0.7205721	-1.45829\\
0.7206721	-1.485779\\
0.7207721	-1.513412\\
0.7208721	-1.537573\\
0.7209721	-1.564023\\
0.7210721	-1.587712\\
0.7211721	-1.612507\\
0.7212721	-1.637829\\
0.7213721	-1.661385\\
0.7214721	-1.685724\\
0.7215722	-1.707149\\
0.7216722	-1.73119\\
0.7217722	-1.752715\\
0.7218722	-1.772249\\
0.7219722	-1.793496\\
0.7220722	-1.812404\\
0.7221722	-1.831235\\
0.7222722	-1.847749\\
0.7223722	-1.86475\\
0.7224722	-1.88134\\
0.7225723	-1.898293\\
0.7226723	-1.914008\\
0.7227723	-1.927816\\
0.7228723	-1.942299\\
0.7229723	-1.957981\\
0.7230723	-1.972431\\
0.7231723	-1.987877\\
0.7232723	-2.001581\\
0.7233723	-2.01328\\
0.7234723	-2.023593\\
0.7235724	-2.031637\\
0.7236724	-2.040668\\
0.7237724	-2.048322\\
0.7238724	-2.057408\\
0.7239724	-2.064462\\
0.7240724	-2.070191\\
0.7241724	-2.077796\\
0.7242724	-2.082833\\
0.7243724	-2.086828\\
0.7244724	-2.091237\\
0.7245725	-2.095709\\
0.7246725	-2.098294\\
0.7247725	-2.101481\\
0.7248725	-2.10396\\
0.7249725	-2.103709\\
0.7250725	-2.103045\\
0.7251725	-2.100384\\
0.7252725	-2.098368\\
0.7253725	-2.095122\\
0.7254725	-2.092598\\
0.7255726	-2.088713\\
0.7256726	-2.083987\\
0.7257726	-2.079801\\
0.7258726	-2.07249\\
0.7259726	-2.066517\\
0.7260726	-2.059289\\
0.7261726	-2.051823\\
0.7262726	-2.044696\\
0.7263726	-2.034198\\
0.7264726	-2.024651\\
0.7265727	-2.012123\\
0.7266727	-1.999132\\
0.7267727	-1.987677\\
0.7268727	-1.974619\\
0.7269727	-1.962623\\
0.7270727	-1.947594\\
0.7271727	-1.932785\\
0.7272727	-1.917555\\
0.7273727	-1.900971\\
0.7274727	-1.881854\\
0.7275728	-1.863499\\
0.7276728	-1.846177\\
0.7277728	-1.829002\\
0.7278728	-1.808613\\
0.7279728	-1.787182\\
0.7280728	-1.766766\\
0.7281728	-1.745656\\
0.7282728	-1.725737\\
0.7283728	-1.69954\\
0.7284728	-1.677333\\
0.7285729	-1.653842\\
0.7286729	-1.629372\\
0.7287729	-1.603939\\
0.7288729	-1.578109\\
0.7289729	-1.555307\\
0.7290729	-1.528116\\
0.7291729	-1.499411\\
0.7292729	-1.472717\\
0.7293729	-1.443015\\
0.7294729	-1.413397\\
0.729573	-1.385177\\
0.729673	-1.354581\\
0.729773	-1.323003\\
0.729873	-1.291571\\
0.729973	-1.260016\\
0.730073	-1.226285\\
0.730173	-1.192694\\
0.730273	-1.160526\\
0.730373	-1.125837\\
0.730473	-1.091611\\
0.7305731	-1.060442\\
0.7306731	-1.028847\\
0.7307731	-0.994521\\
0.7308731	-0.9594558\\
0.7309731	-0.9251182\\
0.7310731	-0.8879355\\
0.7311731	-0.8514596\\
0.7312731	-0.8149601\\
0.7313731	-0.777116\\
0.7314731	-0.7397303\\
0.7315732	-0.7018243\\
0.7316732	-0.6638634\\
0.7317732	-0.6261426\\
0.7318732	-0.5889019\\
0.7319732	-0.5510667\\
0.7320732	-0.509549\\
0.7321732	-0.472026\\
0.7322732	-0.4344846\\
0.7323732	-0.3937397\\
0.7324732	-0.3529552\\
0.7325733	-0.3115046\\
0.7326733	-0.2731608\\
0.7327733	-0.2363025\\
0.7328733	-0.1959233\\
0.7329733	-0.1537957\\
0.7330733	-0.112731\\
0.7331733	-0.0735925\\
0.7332733	-0.03375756\\
0.7333733	0.007748723\\
0.7334733	0.04906312\\
0.7335734	0.08918361\\
0.7336734	0.1296297\\
0.7337734	0.1697146\\
0.7338734	0.2099757\\
0.7339734	0.2481415\\
0.7340734	0.2885647\\
0.7341734	0.3281915\\
0.7342734	0.3705921\\
0.7343734	0.4124438\\
0.7344734	0.4512187\\
0.7345735	0.4910608\\
0.7346735	0.5306837\\
0.7347735	0.5709641\\
0.7348735	0.610122\\
0.7349735	0.6491772\\
0.7350735	0.6871385\\
0.7351735	0.7265731\\
0.7352735	0.765535\\
0.7353735	0.8044071\\
0.7354735	0.8397897\\
0.7355736	0.8786557\\
0.7356736	0.9172892\\
0.7357736	0.9544933\\
0.7358736	0.9906286\\
0.7359736	1.02623\\
0.7360736	1.06354\\
0.7361736	1.099298\\
0.7362736	1.132703\\
0.7363736	1.169139\\
0.7364736	1.20402\\
0.7365737	1.237346\\
0.7366737	1.27259\\
0.7367737	1.307523\\
0.7368737	1.342272\\
0.7369737	1.37603\\
0.7370737	1.408472\\
0.7371737	1.439105\\
0.7372737	1.468332\\
0.7373737	1.499026\\
0.7374737	1.529249\\
0.7375738	1.558198\\
0.7376738	1.586916\\
0.7377738	1.615953\\
0.7378738	1.646185\\
0.7379738	1.672922\\
0.7380738	1.700855\\
0.7381738	1.728368\\
0.7382738	1.753292\\
0.7383738	1.778816\\
0.7384738	1.802774\\
0.7385739	1.825197\\
0.7386739	1.850427\\
0.7387739	1.871529\\
0.7388739	1.894362\\
0.7389739	1.914598\\
0.7390739	1.933431\\
0.7391739	1.952708\\
0.7392739	1.972588\\
0.7393739	1.991554\\
0.7394739	2.007842\\
0.739574	2.026199\\
0.739674	2.044163\\
0.739774	2.060975\\
0.739874	2.076595\\
0.739974	2.092818\\
0.740074	2.105487\\
0.740174	2.120517\\
0.740274	2.133165\\
0.740374	2.144782\\
0.740474	2.153573\\
0.7405741	2.160245\\
0.7406741	2.168276\\
0.7407741	2.175856\\
0.7408741	2.185389\\
0.7409741	2.192898\\
0.7410741	2.198527\\
0.7411741	2.20536\\
0.7412741	2.20743\\
0.7413741	2.210592\\
0.7414741	2.214853\\
0.7415742	2.216351\\
0.7416742	2.218317\\
0.7417742	2.217245\\
0.7418742	2.216682\\
0.7419742	2.215345\\
0.7420742	2.211231\\
0.7421742	2.20999\\
0.7422742	2.206662\\
0.7423742	2.202428\\
0.7424742	2.2003\\
0.7425743	2.193711\\
0.7426743	2.186405\\
0.7427743	2.179541\\
0.7428743	2.169262\\
0.7429743	2.160965\\
0.7430743	2.149699\\
0.7431743	2.13938\\
0.7432743	2.128085\\
0.7433743	2.115086\\
0.7434743	2.101415\\
0.7435744	2.084071\\
0.7436744	2.070474\\
0.7437744	2.055747\\
0.7438744	2.04059\\
0.7439744	2.023365\\
0.7440744	2.006637\\
0.7441744	1.990251\\
0.7442744	1.97221\\
0.7443744	1.951777\\
0.7444744	1.931187\\
0.7445745	1.90966\\
0.7446745	1.886663\\
0.7447745	1.861444\\
0.7448745	1.838626\\
0.7449745	1.815521\\
0.7450745	1.79169\\
0.7451745	1.766355\\
0.7452745	1.739299\\
0.7453745	1.711536\\
0.7454745	1.684718\\
0.7455746	1.655259\\
0.7456746	1.626352\\
0.7457746	1.597255\\
0.7458746	1.568309\\
0.7459746	1.539155\\
0.7460746	1.506193\\
0.7461746	1.476628\\
0.7462746	1.442267\\
0.7463746	1.40853\\
0.7464746	1.376043\\
0.7465747	1.341299\\
0.7466747	1.307594\\
0.7467747	1.272203\\
0.7468747	1.237056\\
0.7469747	1.202108\\
0.7470747	1.163949\\
0.7471747	1.128756\\
0.7472747	1.092486\\
0.7473747	1.055948\\
0.7474747	1.018102\\
0.7475748	0.978818\\
0.7476748	0.9405092\\
0.7477748	0.9016941\\
0.7478748	0.8646955\\
0.7479748	0.8243613\\
0.7480748	0.7808015\\
0.7481748	0.7393218\\
0.7482748	0.6988022\\
0.7483748	0.658684\\
0.7484748	0.6159852\\
0.7485749	0.5728115\\
0.7486749	0.5321459\\
0.7487749	0.4929758\\
0.7488749	0.4490914\\
0.7489749	0.4057668\\
0.7490749	0.3639601\\
0.7491749	0.321481\\
0.7492749	0.2784156\\
0.7493749	0.2343989\\
0.7494749	0.1915899\\
0.749575	0.1496781\\
0.749675	0.1046054\\
0.749775	0.05917672\\
0.749875	0.01675553\\
0.749975	-0.02678427\\
0.750075	-0.07037597\\
0.750175	-0.1117598\\
0.750275	-0.1535243\\
0.750375	-0.1972934\\
0.750475	-0.2408675\\
0.7505751	-0.2853478\\
0.7506751	-0.3268708\\
0.7507751	-0.3708057\\
0.7508751	-0.4168954\\
0.7509751	-0.4586895\\
0.7510751	-0.4989738\\
0.7511751	-0.5411339\\
0.7512751	-0.5850176\\
0.7513751	-0.6283714\\
0.7514751	-0.670164\\
0.7515752	-0.7114062\\
0.7516752	-0.7549476\\
0.7517752	-0.7963023\\
0.7518752	-0.836741\\
0.7519752	-0.8779207\\
0.7520752	-0.9159279\\
0.7521752	-0.9553275\\
0.7522752	-0.9958393\\
0.7523752	-1.036715\\
0.7524752	-1.075484\\
0.7525753	-1.110947\\
0.7526753	-1.149328\\
0.7527753	-1.186033\\
0.7528753	-1.223775\\
0.7529753	-1.262533\\
0.7530753	-1.29976\\
0.7531753	-1.335273\\
0.7532753	-1.37088\\
0.7533753	-1.407587\\
0.7534753	-1.444498\\
0.7535754	-1.479032\\
0.7536754	-1.509729\\
0.7537754	-1.543011\\
0.7538754	-1.576161\\
0.7539754	-1.606468\\
0.7540754	-1.636824\\
0.7541754	-1.667482\\
0.7542754	-1.696306\\
0.7543754	-1.725777\\
0.7544754	-1.752485\\
0.7545755	-1.781803\\
0.7546755	-1.808427\\
0.7547755	-1.834023\\
0.7548755	-1.860215\\
0.7549755	-1.886493\\
0.7550755	-1.910202\\
0.7551755	-1.935232\\
0.7552755	-1.960408\\
0.7553755	-1.982792\\
0.7554755	-2.004441\\
0.7555756	-2.023612\\
0.7556756	-2.044773\\
0.7557756	-2.064381\\
0.7558756	-2.082327\\
0.7559756	-2.098337\\
0.7560756	-2.11676\\
0.7561756	-2.130346\\
0.7562756	-2.148102\\
0.7563756	-2.165461\\
0.7564756	-2.180236\\
0.7565757	-2.194907\\
0.7566757	-2.206085\\
0.7567757	-2.216093\\
0.7568757	-2.224353\\
0.7569757	-2.232272\\
0.7570757	-2.240448\\
0.7571757	-2.248492\\
0.7572757	-2.255384\\
0.7573757	-2.263975\\
0.7574757	-2.268375\\
0.7575758	-2.272923\\
0.7576758	-2.276543\\
0.7577758	-2.278545\\
0.7578758	-2.281652\\
0.7579758	-2.282633\\
0.7580758	-2.282753\\
0.7581758	-2.281865\\
0.7582758	-2.279512\\
0.7583758	-2.277393\\
0.7584758	-2.272994\\
0.7585759	-2.26813\\
0.7586759	-2.262646\\
0.7587759	-2.256562\\
0.7588759	-2.250754\\
0.7589759	-2.243502\\
0.7590759	-2.23559\\
0.7591759	-2.224237\\
0.7592759	-2.214508\\
0.7593759	-2.204771\\
0.7594759	-2.193755\\
0.759576	-2.180756\\
0.759676	-2.167129\\
0.759776	-2.152764\\
0.759876	-2.135895\\
0.759976	-2.118253\\
0.760076	-2.099483\\
0.760176	-2.082669\\
0.760276	-2.064098\\
0.760376	-2.042503\\
0.760476	-2.022985\\
0.7605761	-2.001748\\
0.7606761	-1.980387\\
0.7607761	-1.959722\\
0.7608761	-1.937016\\
0.7609761	-1.91444\\
0.7610761	-1.8897\\
0.7611761	-1.861423\\
0.7612761	-1.835099\\
0.7613761	-1.808228\\
0.7614761	-1.778452\\
0.7615762	-1.749477\\
0.7616762	-1.719225\\
0.7617762	-1.689404\\
0.7618762	-1.660479\\
0.7619762	-1.627595\\
0.7620762	-1.59745\\
0.7621762	-1.565742\\
0.7622762	-1.530654\\
0.7623762	-1.497443\\
0.7624762	-1.464941\\
0.7625763	-1.428287\\
0.7626763	-1.390791\\
0.7627763	-1.354604\\
0.7628763	-1.318299\\
0.7629763	-1.281809\\
0.7630763	-1.244916\\
0.7631763	-1.207384\\
0.7632763	-1.169075\\
0.7633763	-1.129682\\
0.7634763	-1.090108\\
0.7635764	-1.049963\\
0.7636764	-1.008102\\
0.7637764	-0.9678612\\
0.7638764	-0.9289741\\
0.7639764	-0.8881276\\
0.7640764	-0.8478122\\
0.7641764	-0.8043069\\
0.7642764	-0.7598941\\
0.7643764	-0.7178374\\
0.7644764	-0.6738158\\
0.7645765	-0.6284432\\
0.7646765	-0.5854407\\
0.7647765	-0.5402878\\
0.7648765	-0.4942082\\
0.7649765	-0.44999\\
0.7650765	-0.404661\\
0.7651765	-0.3613628\\
0.7652765	-0.3160667\\
0.7653765	-0.2697508\\
0.7654765	-0.2244567\\
0.7655766	-0.1819978\\
0.7656766	-0.1355676\\
0.7657766	-0.09188825\\
0.7658766	-0.04711429\\
0.7659766	-0.002459149\\
0.7660766	0.04316331\\
0.7661766	0.08884467\\
0.7662766	0.1353978\\
0.7663766	0.1801625\\
0.7664766	0.2267082\\
0.7665767	0.2729415\\
0.7666767	0.3183943\\
0.7667767	0.3636718\\
0.7668767	0.4086104\\
0.7669767	0.4551342\\
0.7670767	0.4979944\\
0.7671767	0.5419033\\
0.7672767	0.5855742\\
0.7673767	0.6307611\\
0.7674767	0.6737918\\
0.7675768	0.7170755\\
0.7676768	0.7606744\\
0.7677768	0.8038847\\
0.7678768	0.8476394\\
0.7679768	0.89093\\
0.7680768	0.931372\\
0.7681768	0.9713099\\
0.7682768	1.013341\\
0.7683768	1.055711\\
0.7684768	1.094709\\
0.7685769	1.135459\\
0.7686769	1.176242\\
0.7687769	1.215722\\
0.7688769	1.254843\\
0.7689769	1.293482\\
0.7690769	1.329288\\
0.7691769	1.366869\\
0.7692769	1.402312\\
0.7693769	1.438712\\
0.7694769	1.474719\\
0.769577	1.508904\\
0.769677	1.541228\\
0.769777	1.576423\\
0.769877	1.610822\\
0.769977	1.643265\\
0.770077	1.675427\\
0.770177	1.705524\\
0.770277	1.737513\\
0.770377	1.766116\\
0.770477	1.795065\\
0.7705771	1.824574\\
0.7706771	1.851506\\
0.7707771	1.877516\\
0.7708771	1.902249\\
0.7709771	1.927701\\
0.7710771	1.951642\\
0.7711771	1.974871\\
0.7712771	1.99937\\
0.7713771	2.020843\\
0.7714771	2.042592\\
0.7715772	2.063476\\
0.7716772	2.084039\\
0.7717772	2.104102\\
0.7718772	2.122745\\
0.7719772	2.139857\\
0.7720772	2.155001\\
0.7721772	2.169935\\
0.7722772	2.182436\\
0.7723772	2.197617\\
0.7724772	2.211642\\
0.7725773	2.223267\\
0.7726773	2.236947\\
0.7727773	2.246832\\
0.7728773	2.255943\\
0.7729773	2.262829\\
0.7730773	2.270263\\
0.7731773	2.276817\\
0.7732773	2.28181\\
0.7733773	2.285768\\
0.7734773	2.289864\\
0.7735774	2.293292\\
0.7736774	2.294207\\
0.7737774	2.29308\\
0.7738774	2.291274\\
0.7739774	2.291227\\
0.7740774	2.288278\\
0.7741774	2.286148\\
0.7742774	2.28307\\
0.7743774	2.280489\\
0.7744774	2.276483\\
0.7745775	2.26955\\
0.7746775	2.261268\\
0.7747775	2.253535\\
0.7748775	2.244632\\
0.7749775	2.234373\\
0.7750775	2.223662\\
0.7751775	2.212627\\
0.7752775	2.198377\\
0.7753775	2.183495\\
0.7754775	2.166192\\
0.7755776	2.149337\\
0.7756776	2.136717\\
0.7757776	2.120327\\
0.7758776	2.101465\\
0.7759776	2.082708\\
0.7760776	2.062935\\
0.7761776	2.042369\\
0.7762776	2.020578\\
0.7763776	1.996426\\
0.7764776	1.973529\\
0.7765777	1.948667\\
0.7766777	1.921965\\
0.7767777	1.894797\\
0.7768777	1.869519\\
0.7769777	1.844051\\
0.7770777	1.815989\\
0.7771777	1.787078\\
0.7772777	1.759897\\
0.7773777	1.729778\\
0.7774777	1.699533\\
0.7775778	1.667202\\
0.7776778	1.634775\\
0.7777778	1.602108\\
0.7778778	1.568439\\
0.7779778	1.532286\\
0.7780778	1.496176\\
0.7781778	1.459742\\
0.7782778	1.421132\\
0.7783778	1.385615\\
0.7784778	1.348847\\
0.7785779	1.312497\\
0.7786779	1.274376\\
0.7787779	1.236263\\
0.7788779	1.19552\\
0.7789779	1.154611\\
0.7790779	1.117192\\
0.7791779	1.076854\\
0.7792779	1.034626\\
0.7793779	0.9941128\\
0.7794779	0.9513583\\
0.779578	0.9102234\\
0.779678	0.8678409\\
0.779778	0.8217751\\
0.779878	0.776069\\
0.779978	0.7320735\\
0.780078	0.6881732\\
0.780178	0.6421724\\
0.780278	0.5972339\\
0.780378	0.5531791\\
0.780478	0.5091589\\
0.7805781	0.4658719\\
0.7806781	0.4194588\\
0.7807781	0.3729492\\
0.7808781	0.3256679\\
0.7809781	0.2773003\\
0.7810781	0.2296007\\
0.7811781	0.1858153\\
0.7812781	0.1404154\\
0.7813781	0.0926553\\
0.7814781	0.04589417\\
0.7815782	-0.0005805175\\
0.7816782	-0.04669155\\
0.7817782	-0.0934121\\
0.7818782	-0.1421854\\
0.7819782	-0.1873016\\
0.7820782	-0.2332722\\
0.7821782	-0.2800365\\
0.7822782	-0.3269657\\
0.7823782	-0.3718569\\
0.7824782	-0.4155945\\
0.7825783	-0.4622371\\
0.7826783	-0.5067802\\
0.7827783	-0.5497631\\
0.7828783	-0.5945674\\
0.7829783	-0.6401298\\
0.7830783	-0.6863431\\
0.7831783	-0.7318888\\
0.7832783	-0.7775745\\
0.7833783	-0.8204314\\
0.7834783	-0.8637431\\
0.7835784	-0.9071853\\
0.7836784	-0.9494032\\
0.7837784	-0.9909246\\
0.7838784	-1.033891\\
0.7839784	-1.075621\\
0.7840784	-1.115402\\
0.7841784	-1.155475\\
0.7842784	-1.195456\\
0.7843784	-1.235692\\
0.7844784	-1.273818\\
0.7845785	-1.312599\\
0.7846785	-1.348502\\
0.7847785	-1.386161\\
0.7848785	-1.423801\\
0.7849785	-1.459302\\
0.7850785	-1.494331\\
0.7851785	-1.527253\\
0.7852785	-1.56076\\
0.7853785	-1.595064\\
0.7854785	-1.628866\\
0.7855786	-1.660359\\
0.7856786	-1.692521\\
0.7857786	-1.724467\\
0.7858786	-1.753846\\
0.7859786	-1.783001\\
0.7860786	-1.809234\\
0.7861786	-1.837879\\
0.7862786	-1.865111\\
0.7863786	-1.889667\\
0.7864786	-1.915162\\
0.7865787	-1.940889\\
0.7866787	-1.965103\\
0.7867787	-1.988365\\
0.7868787	-2.008649\\
0.7869787	-2.028708\\
0.7870787	-2.047704\\
0.7871787	-2.067828\\
0.7872787	-2.087632\\
0.7873787	-2.102652\\
0.7874787	-2.118827\\
0.7875788	-2.136632\\
0.7876788	-2.151125\\
0.7877788	-2.165132\\
0.7878788	-2.17798\\
0.7879788	-2.189393\\
0.7880788	-2.200544\\
0.7881788	-2.212073\\
0.7882788	-2.222268\\
0.7883788	-2.231781\\
0.7884788	-2.238851\\
0.7885789	-2.245132\\
0.7886789	-2.250956\\
0.7887789	-2.254002\\
0.7888789	-2.257151\\
0.7889789	-2.257684\\
0.7890789	-2.25971\\
0.7891789	-2.260267\\
0.7892789	-2.258362\\
0.7893789	-2.254491\\
0.7894789	-2.254253\\
0.789579	-2.252379\\
0.789679	-2.247694\\
0.789779	-2.241622\\
0.789879	-2.238024\\
0.789979	-2.229809\\
0.790079	-2.220836\\
0.790179	-2.20931\\
0.790279	-2.198721\\
0.790379	-2.18685\\
0.790479	-2.176982\\
0.7905791	-2.164588\\
0.7906791	-2.149804\\
0.7907791	-2.139862\\
0.7908791	-2.123713\\
0.7909791	-2.104442\\
0.7910791	-2.086542\\
0.7911791	-2.070637\\
0.7912791	-2.053177\\
0.7913791	-2.031745\\
0.7914791	-2.009632\\
0.7915792	-1.98645\\
0.7916792	-1.963762\\
0.7917792	-1.940127\\
0.7918792	-1.914674\\
0.7919792	-1.887894\\
0.7920792	-1.863024\\
0.7921792	-1.836835\\
0.7922792	-1.807706\\
0.7923792	-1.78004\\
0.7924792	-1.752085\\
0.7925793	-1.722495\\
0.7926793	-1.692463\\
0.7927793	-1.660668\\
0.7928793	-1.628387\\
0.7929793	-1.594679\\
0.7930793	-1.561103\\
0.7931793	-1.525125\\
0.7932793	-1.488508\\
0.7933793	-1.453088\\
0.7934793	-1.417617\\
0.7935794	-1.380904\\
0.7936794	-1.34302\\
0.7937794	-1.305795\\
0.7938794	-1.267277\\
0.7939794	-1.22908\\
0.7940794	-1.192088\\
0.7941794	-1.152798\\
0.7942794	-1.112962\\
0.7943794	-1.071363\\
0.7944794	-1.0302\\
0.7945795	-0.9850023\\
0.7946795	-0.9429935\\
0.7947795	-0.9000063\\
0.7948795	-0.8558181\\
0.7949795	-0.8117599\\
0.7950795	-0.7679442\\
0.7951795	-0.7223984\\
0.7952795	-0.6778728\\
0.7953795	-0.6331706\\
0.7954795	-0.5890125\\
0.7955796	-0.5450418\\
0.7956796	-0.5012033\\
0.7957796	-0.4572131\\
0.7958796	-0.4111753\\
0.7959796	-0.3634484\\
0.7960796	-0.3178387\\
0.7961796	-0.270953\\
0.7962796	-0.2232978\\
0.7963796	-0.1750663\\
0.7964796	-0.1296001\\
0.7965797	-0.08247165\\
0.7966797	-0.03738833\\
0.7967797	0.009852623\\
0.7968797	0.05667156\\
0.7969797	0.1055151\\
0.7970797	0.1517483\\
0.7971797	0.1986843\\
0.7972797	0.2446412\\
0.7973797	0.29009\\
0.7974797	0.3387757\\
0.7975798	0.3854441\\
0.7976798	0.4287559\\
0.7977798	0.4739476\\
0.7978798	0.5171808\\
0.7979798	0.5623335\\
0.7980798	0.6082143\\
0.7981798	0.6531962\\
0.7982798	0.697174\\
0.7983798	0.7407747\\
0.7984798	0.7824326\\
0.7985799	0.8265744\\
0.7986799	0.8700894\\
0.7987799	0.911912\\
0.7988799	0.9540762\\
0.7989799	0.9948053\\
0.7990799	1.03658\\
0.7991799	1.078041\\
0.7992799	1.118866\\
0.7993799	1.15762\\
0.7994799	1.19766\\
0.79958	1.236776\\
0.79968	1.275\\
0.79978	1.312645\\
0.79988	1.350342\\
0.79998	1.386437\\
0.80008	1.422565\\
};
\addplot [color=mycolor2,solid]
  table[row sep=crcr]{%
0.80008	1.422565\\
0.80018	1.457125\\
0.80028	1.490887\\
0.80038	1.525583\\
0.80048	1.558491\\
0.8005801	1.588135\\
0.8006801	1.62061\\
0.8007801	1.650073\\
0.8008801	1.681307\\
0.8009801	1.7099\\
0.8010801	1.738541\\
0.8011801	1.764826\\
0.8012801	1.792308\\
0.8013801	1.818964\\
0.8014801	1.844674\\
0.8015802	1.865536\\
0.8016802	1.890617\\
0.8017802	1.916584\\
0.8018802	1.940284\\
0.8019802	1.95943\\
0.8020802	1.978846\\
0.8021802	1.996537\\
0.8022802	2.015018\\
0.8023802	2.032078\\
0.8024802	2.048642\\
0.8025803	2.061919\\
0.8026803	2.079059\\
0.8027803	2.093817\\
0.8028803	2.10616\\
0.8029803	2.11825\\
0.8030803	2.128349\\
0.8031803	2.139084\\
0.8032803	2.147328\\
0.8033803	2.154749\\
0.8034803	2.160072\\
0.8035804	2.167248\\
0.8036804	2.172585\\
0.8037804	2.177718\\
0.8038804	2.180422\\
0.8039804	2.180793\\
0.8040804	2.182085\\
0.8041804	2.184102\\
0.8042804	2.183113\\
0.8043804	2.182051\\
0.8044804	2.176879\\
0.8045805	2.173274\\
0.8046805	2.166421\\
0.8047805	2.161187\\
0.8048805	2.154388\\
0.8049805	2.145183\\
0.8050805	2.138796\\
0.8051805	2.131552\\
0.8052805	2.120669\\
0.8053805	2.107725\\
0.8054805	2.094827\\
0.8055806	2.080204\\
0.8056806	2.062729\\
0.8057806	2.046656\\
0.8058806	2.03155\\
0.8059806	2.014157\\
0.8060806	1.997299\\
0.8061806	1.980052\\
0.8062806	1.959738\\
0.8063806	1.938542\\
0.8064806	1.916378\\
0.8065807	1.891654\\
0.8066807	1.868253\\
0.8067807	1.84392\\
0.8068807	1.821168\\
0.8069807	1.793294\\
0.8070807	1.766819\\
0.8071807	1.739894\\
0.8072807	1.711626\\
0.8073807	1.684836\\
0.8074807	1.653671\\
0.8075808	1.625149\\
0.8076808	1.594806\\
0.8077808	1.563363\\
0.8078808	1.528017\\
0.8079808	1.495668\\
0.8080808	1.46188\\
0.8081808	1.426863\\
0.8082808	1.38798\\
0.8083808	1.351368\\
0.8084808	1.313675\\
0.8085809	1.273074\\
0.8086809	1.236145\\
0.8087809	1.19947\\
0.8088809	1.160705\\
0.8089809	1.121079\\
0.8090809	1.083226\\
0.8091809	1.044116\\
0.8092809	1.004462\\
0.8093809	0.9631974\\
0.8094809	0.9223603\\
0.809581	0.8797903\\
0.809681	0.8381115\\
0.809781	0.7960772\\
0.809881	0.7503863\\
0.809981	0.7059355\\
0.810081	0.6616712\\
0.810181	0.6167766\\
0.810281	0.5719563\\
0.810381	0.5288271\\
0.810481	0.4846032\\
0.8105811	0.4403304\\
0.8106811	0.3958084\\
0.8107811	0.3514905\\
0.8108811	0.3057063\\
0.8109811	0.2605968\\
0.8110811	0.2121018\\
0.8111811	0.1667129\\
0.8112811	0.1216422\\
0.8113811	0.07694014\\
0.8114811	0.02993849\\
0.8115812	-0.01882612\\
0.8116812	-0.0631793\\
0.8117812	-0.1087812\\
0.8118812	-0.1526573\\
0.8119812	-0.1992596\\
0.8120812	-0.2432543\\
0.8121812	-0.2889729\\
0.8122812	-0.3341881\\
0.8123812	-0.3806677\\
0.8124812	-0.4247892\\
0.8125813	-0.4675192\\
0.8126813	-0.511997\\
0.8127813	-0.5561387\\
0.8128813	-0.6010222\\
0.8129813	-0.6443158\\
0.8130813	-0.6892862\\
0.8131813	-0.7330663\\
0.8132813	-0.7752796\\
0.8133813	-0.8161903\\
0.8134813	-0.8581515\\
0.8135814	-0.9000977\\
0.8136814	-0.9393388\\
0.8137814	-0.9777948\\
0.8138814	-1.015525\\
0.8139814	-1.056545\\
0.8140814	-1.094683\\
0.8141814	-1.134061\\
0.8142814	-1.17394\\
0.8143814	-1.210667\\
0.8144814	-1.24579\\
0.8145815	-1.280176\\
0.8146815	-1.315182\\
0.8147815	-1.35019\\
0.8148815	-1.383525\\
0.8149815	-1.417098\\
0.8150815	-1.452201\\
0.8151815	-1.484145\\
0.8152815	-1.514582\\
0.8153815	-1.544107\\
0.8154815	-1.573874\\
0.8155816	-1.604824\\
0.8156816	-1.632677\\
0.8157816	-1.661462\\
0.8158816	-1.688366\\
0.8159816	-1.712766\\
0.8160816	-1.736865\\
0.8161816	-1.761876\\
0.8162816	-1.785787\\
0.8163816	-1.807842\\
0.8164816	-1.829786\\
0.8165817	-1.850652\\
0.8166817	-1.871453\\
0.8167817	-1.888917\\
0.8168817	-1.907511\\
0.8169817	-1.924892\\
0.8170817	-1.940777\\
0.8171817	-1.955003\\
0.8172817	-1.969525\\
0.8173817	-1.98342\\
0.8174817	-1.994115\\
0.8175818	-2.003834\\
0.8176818	-2.014054\\
0.8177818	-2.024923\\
0.8178818	-2.032737\\
0.8179818	-2.04091\\
0.8180818	-2.047983\\
0.8181818	-2.055138\\
0.8182818	-2.05938\\
0.8183818	-2.064749\\
0.8184818	-2.068807\\
0.8185819	-2.068171\\
0.8186819	-2.069806\\
0.8187819	-2.070079\\
0.8188819	-2.068352\\
0.8189819	-2.066258\\
0.8190819	-2.063611\\
0.8191819	-2.060616\\
0.8192819	-2.056579\\
0.8193819	-2.049959\\
0.8194819	-2.040268\\
0.819582	-2.031867\\
0.819682	-2.023383\\
0.819782	-2.013115\\
0.819882	-2.00192\\
0.819982	-1.99038\\
0.820082	-1.977837\\
0.820182	-1.962245\\
0.820282	-1.949635\\
0.820382	-1.934762\\
0.820482	-1.917941\\
0.8205821	-1.899243\\
0.8206821	-1.882397\\
0.8207821	-1.86573\\
0.8208821	-1.843851\\
0.8209821	-1.823934\\
0.8210821	-1.801165\\
0.8211821	-1.779744\\
0.8212821	-1.755476\\
0.8213821	-1.731108\\
0.8214821	-1.707545\\
0.8215822	-1.68032\\
0.8216822	-1.65318\\
0.8217822	-1.623606\\
0.8218822	-1.595477\\
0.8219822	-1.565837\\
0.8220822	-1.534425\\
0.8221822	-1.506125\\
0.8222822	-1.475021\\
0.8223822	-1.444509\\
0.8224822	-1.413134\\
0.8225823	-1.382502\\
0.8226823	-1.346853\\
0.8227823	-1.314042\\
0.8228823	-1.277579\\
0.8229823	-1.242721\\
0.8230823	-1.2078\\
0.8231823	-1.17101\\
0.8232823	-1.134998\\
0.8233823	-1.095462\\
0.8234823	-1.057524\\
0.8235824	-1.017625\\
0.8236824	-0.980109\\
0.8237824	-0.9384968\\
0.8238824	-0.8964334\\
0.8239824	-0.8585996\\
0.8240824	-0.8175758\\
0.8241824	-0.7763945\\
0.8242824	-0.7339398\\
0.8243824	-0.693009\\
0.8244824	-0.6525707\\
0.8245825	-0.6087193\\
0.8246825	-0.5674026\\
0.8247825	-0.5243506\\
0.8248825	-0.479778\\
0.8249825	-0.435184\\
0.8250825	-0.3905363\\
0.8251825	-0.3476039\\
0.8252825	-0.3038877\\
0.8253825	-0.2586923\\
0.8254825	-0.2150584\\
0.8255826	-0.1696096\\
0.8256826	-0.127195\\
0.8257826	-0.08541899\\
0.8258826	-0.04285619\\
0.8259826	-0.002018526\\
0.8260826	0.04307589\\
0.8261826	0.0865712\\
0.8262826	0.1307359\\
0.8263826	0.1763963\\
0.8264826	0.2225662\\
0.8265827	0.2670852\\
0.8266827	0.3113816\\
0.8267827	0.3546565\\
0.8268827	0.3954077\\
0.8269827	0.4386819\\
0.8270827	0.4792884\\
0.8271827	0.5235243\\
0.8272827	0.5653237\\
0.8273827	0.6049344\\
0.8274827	0.6458032\\
0.8275828	0.686608\\
0.8276828	0.7307233\\
0.8277828	0.7705103\\
0.8278828	0.8119366\\
0.8279828	0.8502919\\
0.8280828	0.8892246\\
0.8281828	0.9276222\\
0.8282828	0.9640878\\
0.8283828	1.003838\\
0.8284828	1.040829\\
0.8285829	1.076695\\
0.8286829	1.111656\\
0.8287829	1.147355\\
0.8288829	1.182302\\
0.8289829	1.213156\\
0.8290829	1.245417\\
0.8291829	1.279756\\
0.8292829	1.312473\\
0.8293829	1.344239\\
0.8294829	1.376158\\
0.829583	1.406234\\
0.829683	1.437008\\
0.829783	1.464023\\
0.829883	1.493663\\
0.829983	1.520538\\
0.830083	1.546443\\
0.830183	1.57198\\
0.830283	1.595753\\
0.830383	1.619785\\
0.830483	1.640704\\
0.8305831	1.66424\\
0.8306831	1.685524\\
0.8307831	1.707505\\
0.8308831	1.726932\\
0.8309831	1.744147\\
0.8310831	1.762704\\
0.8311831	1.780249\\
0.8312831	1.795788\\
0.8313831	1.810938\\
0.8314831	1.824333\\
0.8315832	1.838829\\
0.8316832	1.852834\\
0.8317832	1.865649\\
0.8318832	1.877706\\
0.8319832	1.886963\\
0.8320832	1.89629\\
0.8321832	1.904661\\
0.8322832	1.911196\\
0.8323832	1.917289\\
0.8324832	1.921367\\
0.8325833	1.925621\\
0.8326833	1.927286\\
0.8327833	1.930236\\
0.8328833	1.932534\\
0.8329833	1.932114\\
0.8330833	1.928739\\
0.8331833	1.924819\\
0.8332833	1.922315\\
0.8333833	1.918533\\
0.8334833	1.913329\\
0.8335834	1.909286\\
0.8336834	1.905071\\
0.8337834	1.897042\\
0.8338834	1.89032\\
0.8339834	1.877641\\
0.8340834	1.869707\\
0.8341834	1.858495\\
0.8342834	1.846199\\
0.8343834	1.83317\\
0.8344834	1.819682\\
0.8345835	1.807573\\
0.8346835	1.791511\\
0.8347835	1.772616\\
0.8348835	1.754869\\
0.8349835	1.737099\\
0.8350835	1.718305\\
0.8351835	1.698533\\
0.8352835	1.67894\\
0.8353835	1.657507\\
0.8354835	1.634695\\
0.8355836	1.610416\\
0.8356836	1.584527\\
0.8357836	1.559324\\
0.8358836	1.535277\\
0.8359836	1.509213\\
0.8360836	1.481544\\
0.8361836	1.454993\\
0.8362836	1.427009\\
0.8363836	1.400313\\
0.8364836	1.368435\\
0.8365837	1.336498\\
0.8366837	1.305076\\
0.8367837	1.272412\\
0.8368837	1.238122\\
0.8369837	1.204755\\
0.8370837	1.174265\\
0.8371837	1.140061\\
0.8372837	1.104679\\
0.8373837	1.070013\\
0.8374837	1.034243\\
0.8375838	1.000385\\
0.8376838	0.9639424\\
0.8377838	0.9258256\\
0.8378838	0.8895001\\
0.8379838	0.850658\\
0.8380838	0.8116542\\
0.8381838	0.7742203\\
0.8382838	0.7338679\\
0.8383838	0.6950231\\
0.8384838	0.6546038\\
0.8385839	0.6135145\\
0.8386839	0.5753025\\
0.8387839	0.5324517\\
0.8388839	0.491537\\
0.8389839	0.4519891\\
0.8390839	0.4104314\\
0.8391839	0.3710338\\
0.8392839	0.3300314\\
0.8393839	0.2894577\\
0.8394839	0.2473644\\
0.839584	0.2062879\\
0.839684	0.1658891\\
0.839784	0.1211896\\
0.839884	0.07820487\\
0.839984	0.03715401\\
0.840084	-0.0035415\\
0.840184	-0.0452865\\
0.840284	-0.08904238\\
0.840384	-0.132411\\
0.840484	-0.1730296\\
0.8405841	-0.2153391\\
0.8406841	-0.2546125\\
0.8407841	-0.2966781\\
0.8408841	-0.3384096\\
0.8409841	-0.3771019\\
0.8410841	-0.4184522\\
0.8411841	-0.4582601\\
0.8412841	-0.4976604\\
0.8413841	-0.535901\\
0.8414841	-0.5749119\\
0.8415842	-0.6157031\\
0.8416842	-0.6565362\\
0.8417842	-0.6951079\\
0.8418842	-0.7308785\\
0.8419842	-0.7674691\\
0.8420842	-0.8052514\\
0.8421842	-0.8404988\\
0.8422842	-0.8732468\\
0.8423842	-0.9080332\\
0.8424842	-0.9433501\\
0.8425843	-0.9782609\\
0.8426843	-1.012142\\
0.8427843	-1.046566\\
0.8428843	-1.080063\\
0.8429843	-1.111707\\
0.8430843	-1.143179\\
0.8431843	-1.174859\\
0.8432843	-1.208779\\
0.8433843	-1.239287\\
0.8434843	-1.269137\\
0.8435844	-1.298561\\
0.8436844	-1.324884\\
0.8437844	-1.351747\\
0.8438844	-1.377132\\
0.8439844	-1.400662\\
0.8440844	-1.426021\\
0.8441844	-1.447648\\
0.8442844	-1.470517\\
0.8443844	-1.493905\\
0.8444844	-1.513568\\
0.8445845	-1.535126\\
0.8446845	-1.554154\\
0.8447845	-1.576188\\
0.8448845	-1.597852\\
0.8449845	-1.61415\\
0.8450845	-1.628855\\
0.8451845	-1.645127\\
0.8452845	-1.659689\\
0.8453845	-1.673266\\
0.8454845	-1.68524\\
0.8455846	-1.696711\\
0.8456846	-1.707053\\
0.8457846	-1.719485\\
0.8458846	-1.730204\\
0.8459846	-1.73895\\
0.8460846	-1.748695\\
0.8461846	-1.753743\\
0.8462846	-1.75941\\
0.8463846	-1.763045\\
0.8464846	-1.767163\\
0.8465847	-1.77042\\
0.8466847	-1.771055\\
0.8467847	-1.773332\\
0.8468847	-1.772897\\
0.8469847	-1.771475\\
0.8470847	-1.772622\\
0.8471847	-1.770175\\
0.8472847	-1.766893\\
0.8473847	-1.763632\\
0.8474847	-1.757628\\
0.8475848	-1.75414\\
0.8476848	-1.747157\\
0.8477848	-1.736432\\
0.8478848	-1.727578\\
0.8479848	-1.717951\\
0.8480848	-1.708078\\
0.8481848	-1.695325\\
0.8482848	-1.682285\\
0.8483848	-1.671047\\
0.8484848	-1.657918\\
0.8485849	-1.642401\\
0.8486849	-1.624841\\
0.8487849	-1.608721\\
0.8488849	-1.590753\\
0.8489849	-1.573449\\
0.8490849	-1.554355\\
0.8491849	-1.534475\\
0.8492849	-1.515548\\
0.8493849	-1.493386\\
0.8494849	-1.46859\\
0.849585	-1.445238\\
0.849685	-1.421705\\
0.849785	-1.399213\\
0.849885	-1.37435\\
0.849985	-1.346072\\
0.850085	-1.319324\\
0.850185	-1.292994\\
0.850285	-1.264837\\
0.850385	-1.236179\\
0.850485	-1.208856\\
0.8505851	-1.181253\\
0.8506851	-1.152959\\
0.8507851	-1.120307\\
0.8508851	-1.089205\\
0.8509851	-1.058681\\
0.8510851	-1.026299\\
0.8511851	-0.9927625\\
0.8512851	-0.960092\\
0.8513851	-0.9251878\\
0.8514851	-0.8911143\\
0.8515852	-0.8571623\\
0.8516852	-0.8212711\\
0.8517852	-0.7880079\\
0.8518852	-0.751422\\
0.8519852	-0.713472\\
0.8520852	-0.674091\\
0.8521852	-0.636463\\
0.8522852	-0.5984984\\
0.8523852	-0.5611611\\
0.8524852	-0.5248693\\
0.8525853	-0.4870576\\
0.8526853	-0.4530736\\
0.8527853	-0.4130152\\
0.8528853	-0.373036\\
0.8529853	-0.3353018\\
0.8530853	-0.2960487\\
0.8531853	-0.2586633\\
0.8532853	-0.2206225\\
0.8533853	-0.1801112\\
0.8534853	-0.1413515\\
0.8535854	-0.1023859\\
0.8536854	-0.06249691\\
0.8537854	-0.02591958\\
0.8538854	0.01331128\\
0.8539854	0.05546739\\
0.8540854	0.09283018\\
0.8541854	0.130948\\
0.8542854	0.1692879\\
0.8543854	0.2078568\\
0.8544854	0.2478174\\
0.8545855	0.2835013\\
0.8546855	0.3209576\\
0.8547855	0.3593545\\
0.8548855	0.3976102\\
0.8549855	0.435612\\
0.8550855	0.4733933\\
0.8551855	0.5103354\\
0.8552855	0.5469163\\
0.8553855	0.583618\\
0.8554855	0.6191596\\
0.8555856	0.6519664\\
0.8556856	0.6852686\\
0.8557856	0.7204672\\
0.8558856	0.754156\\
0.8559856	0.78618\\
0.8560856	0.8189079\\
0.8561856	0.8509804\\
0.8562856	0.8862781\\
0.8563856	0.918559\\
0.8564856	0.9510863\\
0.8565857	0.9809477\\
0.8566857	1.011511\\
0.8567857	1.042742\\
0.8568857	1.072105\\
0.8569857	1.098787\\
0.8570857	1.124522\\
0.8571857	1.151925\\
0.8572857	1.179111\\
0.8573857	1.203565\\
0.8574857	1.227671\\
0.8575858	1.250774\\
0.8576858	1.272958\\
0.8577858	1.296973\\
0.8578858	1.316087\\
0.8579858	1.337055\\
0.8580858	1.358081\\
0.8581858	1.378824\\
0.8582858	1.397262\\
0.8583858	1.417855\\
0.8584858	1.435203\\
0.8585859	1.451608\\
0.8586859	1.468777\\
0.8587859	1.482561\\
0.8588859	1.496829\\
0.8589859	1.51133\\
0.8590859	1.522449\\
0.8591859	1.532825\\
0.8592859	1.542464\\
0.8593859	1.552813\\
0.8594859	1.562932\\
0.859586	1.571182\\
0.859686	1.57939\\
0.859786	1.586508\\
0.859886	1.593751\\
0.859986	1.598198\\
0.860086	1.601594\\
0.860186	1.605466\\
0.860286	1.60834\\
0.860386	1.608791\\
0.860486	1.609972\\
0.8605861	1.607133\\
0.8606861	1.602618\\
0.8607861	1.600588\\
0.8608861	1.59751\\
0.8609861	1.593655\\
0.8610861	1.589748\\
0.8611861	1.584259\\
0.8612861	1.577108\\
0.8613861	1.570243\\
0.8614861	1.561641\\
0.8615862	1.549857\\
0.8616862	1.540502\\
0.8617862	1.531017\\
0.8618862	1.519954\\
0.8619862	1.509163\\
0.8620862	1.496396\\
0.8621862	1.480786\\
0.8622862	1.464322\\
0.8623862	1.449437\\
0.8624862	1.433842\\
0.8625863	1.416973\\
0.8626863	1.401385\\
0.8627863	1.382592\\
0.8628863	1.363467\\
0.8629863	1.344121\\
0.8630863	1.32313\\
0.8631863	1.300064\\
0.8632863	1.277097\\
0.8633863	1.252682\\
0.8634863	1.227661\\
0.8635864	1.204058\\
0.8636864	1.179646\\
0.8637864	1.154874\\
0.8638864	1.128929\\
0.8639864	1.104616\\
0.8640864	1.078124\\
0.8641864	1.050289\\
0.8642864	1.020608\\
0.8643864	0.9932636\\
0.8644864	0.9650403\\
0.8645865	0.935746\\
0.8646865	0.9047483\\
0.8647865	0.8750095\\
0.8648865	0.8429978\\
0.8649865	0.8093584\\
0.8650865	0.776904\\
0.8651865	0.7427233\\
0.8652865	0.7098336\\
0.8653865	0.6790066\\
0.8654865	0.6463046\\
0.8655866	0.6125185\\
0.8656866	0.5801051\\
0.8657866	0.5452069\\
0.8658866	0.5092453\\
0.8659866	0.4748579\\
0.8660866	0.4404398\\
0.8661866	0.4057592\\
0.8662866	0.370233\\
0.8663866	0.3338977\\
0.8664866	0.2998178\\
0.8665867	0.2638901\\
0.8666867	0.2303513\\
0.8667867	0.1953533\\
0.8668867	0.1592344\\
0.8669867	0.1257124\\
0.8670867	0.0900986\\
0.8671867	0.05279458\\
0.8672867	0.01617817\\
0.8673867	-0.02067052\\
0.8674867	-0.05778823\\
0.8675868	-0.09318442\\
0.8676868	-0.1261708\\
0.8677868	-0.1622909\\
0.8678868	-0.1976348\\
0.8679868	-0.2324612\\
0.8680868	-0.2673651\\
0.8681868	-0.3040513\\
0.8682868	-0.3382709\\
0.8683868	-0.3741679\\
0.8684868	-0.4102932\\
0.8685869	-0.4420184\\
0.8686869	-0.4741771\\
0.8687869	-0.5075088\\
0.8688869	-0.5408842\\
0.8689869	-0.5717116\\
0.8690869	-0.6034246\\
0.8691869	-0.6347678\\
0.8692869	-0.6645443\\
0.8693869	-0.6947816\\
0.8694869	-0.7254369\\
0.869587	-0.7565952\\
0.869687	-0.7847268\\
0.869787	-0.8129296\\
0.869887	-0.8408329\\
0.869987	-0.8694275\\
0.870087	-0.8988793\\
0.870187	-0.9254281\\
0.870287	-0.9500726\\
0.870387	-0.9763103\\
0.870487	-1.003674\\
0.8705871	-1.026525\\
0.8706871	-1.050389\\
0.8707871	-1.074026\\
0.8708871	-1.096064\\
0.8709871	-1.116401\\
0.8710871	-1.137918\\
0.8711871	-1.157559\\
0.8712871	-1.177935\\
0.8713871	-1.197316\\
0.8714871	-1.21624\\
0.8715872	-1.233343\\
0.8716872	-1.252979\\
0.8717872	-1.272004\\
0.8718872	-1.286741\\
0.8719872	-1.301938\\
0.8720872	-1.315589\\
0.8721872	-1.328942\\
0.8722872	-1.340332\\
0.8723872	-1.351607\\
0.8724872	-1.3611\\
0.8725873	-1.370224\\
0.8726873	-1.37721\\
0.8727873	-1.387473\\
0.8728873	-1.397222\\
0.8729873	-1.404418\\
0.8730873	-1.409411\\
0.8731873	-1.414742\\
0.8732873	-1.421131\\
0.8733873	-1.426217\\
0.8734873	-1.429727\\
0.8735874	-1.431454\\
0.8736874	-1.43339\\
0.8737874	-1.433176\\
0.8738874	-1.432672\\
0.8739874	-1.429021\\
0.8740874	-1.428188\\
0.8741874	-1.424275\\
0.8742874	-1.420195\\
0.8743874	-1.416127\\
0.8744874	-1.41024\\
0.8745875	-1.40477\\
0.8746875	-1.396963\\
0.8747875	-1.388124\\
0.8748875	-1.380722\\
0.8749875	-1.372687\\
0.8750875	-1.365665\\
0.8751875	-1.356329\\
0.8752875	-1.3452\\
0.8753875	-1.33353\\
0.8754875	-1.318527\\
0.8755876	-1.306151\\
0.8756876	-1.291373\\
0.8757876	-1.277105\\
0.8758876	-1.262447\\
0.8759876	-1.246189\\
0.8760876	-1.228123\\
0.8761876	-1.211593\\
0.8762876	-1.193586\\
0.8763876	-1.173028\\
0.8764876	-1.152435\\
0.8765877	-1.132773\\
0.8766877	-1.111642\\
0.8767877	-1.088383\\
0.8768877	-1.065186\\
0.8769877	-1.044655\\
0.8770877	-1.020058\\
0.8771877	-0.9938901\\
0.8772877	-0.9716415\\
0.8773877	-0.9472377\\
0.8774877	-0.9220604\\
0.8775878	-0.8963103\\
0.8776878	-0.870762\\
0.8777878	-0.8428134\\
0.8778878	-0.814444\\
0.8779878	-0.7890323\\
0.8780878	-0.763746\\
0.8781878	-0.7366711\\
0.8782878	-0.7079646\\
0.8783878	-0.678951\\
0.8784878	-0.6492066\\
0.8785879	-0.6193874\\
0.8786879	-0.5884757\\
0.8787879	-0.5560832\\
0.8788879	-0.5245168\\
0.8789879	-0.4938137\\
0.8790879	-0.4629226\\
0.8791879	-0.4334427\\
0.8792879	-0.4041339\\
0.8793879	-0.3731604\\
0.8794879	-0.3409931\\
0.879588	-0.3084175\\
0.879688	-0.2757876\\
0.879788	-0.2434428\\
0.879888	-0.2097449\\
0.879988	-0.1770686\\
0.880088	-0.1459431\\
0.880188	-0.1122731\\
0.880288	-0.08053405\\
0.880388	-0.0472989\\
0.880488	-0.01582008\\
0.8805881	0.01578841\\
0.8806881	0.0487368\\
0.8807881	0.07922211\\
0.8808881	0.1111614\\
0.8809881	0.1421228\\
0.8810881	0.1744611\\
0.8811881	0.2052139\\
0.8812881	0.2375877\\
0.8813881	0.2709957\\
0.8814881	0.3035021\\
0.8815882	0.3326097\\
0.8816882	0.3618331\\
0.8817882	0.3932631\\
0.8818882	0.4226355\\
0.8819882	0.4537252\\
0.8820882	0.482008\\
0.8821882	0.5106805\\
0.8822882	0.5387777\\
0.8823882	0.5649793\\
0.8824882	0.59276\\
0.8825883	0.6212184\\
0.8826883	0.6463133\\
0.8827883	0.6740684\\
0.8828883	0.7020705\\
0.8829883	0.7258391\\
0.8830883	0.7517458\\
0.8831883	0.7777223\\
0.8832883	0.8027038\\
0.8833883	0.8259845\\
0.8834883	0.8490816\\
0.8835884	0.8708884\\
0.8836884	0.8926044\\
0.8837884	0.914524\\
0.8838884	0.9356909\\
0.8839884	0.9563583\\
0.8840884	0.9778434\\
0.8841884	0.9958839\\
0.8842884	1.013958\\
0.8843884	1.033753\\
0.8844884	1.052753\\
0.8845885	1.067874\\
0.8846885	1.082911\\
0.8847885	1.101083\\
0.8848885	1.114767\\
0.8849885	1.12823\\
0.8850885	1.141276\\
0.8851885	1.15663\\
0.8852885	1.166388\\
0.8853885	1.176436\\
0.8854885	1.188544\\
0.8855886	1.198037\\
0.8856886	1.207665\\
0.8857886	1.215267\\
0.8858886	1.219944\\
0.8859886	1.22569\\
0.8860886	1.231013\\
0.8861886	1.235678\\
0.8862886	1.240294\\
0.8863886	1.245929\\
0.8864886	1.248364\\
0.8865887	1.251611\\
0.8866887	1.255912\\
0.8867887	1.257783\\
0.8868887	1.259648\\
0.8869887	1.260639\\
0.8870887	1.260368\\
0.8871887	1.258511\\
0.8872887	1.254263\\
0.8873887	1.250497\\
0.8874887	1.245271\\
0.8875888	1.23895\\
0.8876888	1.234597\\
0.8877888	1.228878\\
0.8878888	1.220823\\
0.8879888	1.209503\\
0.8880888	1.201658\\
0.8881888	1.192757\\
0.8882888	1.184978\\
0.8883888	1.176173\\
0.8884888	1.165565\\
0.8885889	1.152804\\
0.8886889	1.140845\\
0.8887889	1.125803\\
0.8888889	1.113212\\
0.8889889	1.098932\\
0.8890889	1.08321\\
0.8891889	1.067902\\
0.8892889	1.05132\\
0.8893889	1.034248\\
0.8894889	1.016969\\
0.889589	0.9991939\\
0.889689	0.980553\\
0.889789	0.9608818\\
0.889889	0.9401671\\
0.889989	0.9207326\\
0.890089	0.8998427\\
0.890189	0.8787238\\
0.890289	0.8585976\\
0.890389	0.8381109\\
0.890489	0.8160441\\
0.8905891	0.7920066\\
0.8906891	0.7671941\\
0.8907891	0.7418851\\
0.8908891	0.7160459\\
0.8909891	0.6910799\\
0.8910891	0.6678802\\
0.8911891	0.6448501\\
0.8912891	0.6201493\\
0.8913891	0.5949979\\
0.8914891	0.570075\\
0.8915892	0.5445928\\
0.8916892	0.515733\\
0.8917892	0.4895427\\
0.8918892	0.4611676\\
0.8919892	0.4310292\\
0.8920892	0.4030365\\
0.8921892	0.3787999\\
0.8922892	0.3516442\\
0.8923892	0.3207574\\
0.8924892	0.2901282\\
0.8925893	0.262846\\
0.8926893	0.2361658\\
0.8927893	0.2056905\\
0.8928893	0.1758543\\
0.8929893	0.1487997\\
0.8930893	0.1198738\\
0.8931893	0.09100693\\
0.8932893	0.06000178\\
0.8933893	0.03142792\\
0.8934893	0.002779611\\
0.8935894	-0.02591966\\
0.8936894	-0.0530826\\
0.8937894	-0.08194059\\
0.8938894	-0.1096781\\
0.8939894	-0.1383625\\
0.8940894	-0.1679015\\
0.8941894	-0.1957782\\
0.8942894	-0.2220104\\
0.8943894	-0.2484816\\
0.8944894	-0.2753667\\
0.8945895	-0.3021183\\
0.8946895	-0.327995\\
0.8947895	-0.3548711\\
0.8948895	-0.381093\\
0.8949895	-0.4079458\\
0.8950895	-0.4341612\\
0.8951895	-0.4602381\\
0.8952895	-0.4847139\\
0.8953895	-0.510985\\
0.8954895	-0.5358179\\
0.8955896	-0.5600983\\
0.8956896	-0.5845101\\
0.8957896	-0.608333\\
0.8958896	-0.6305973\\
0.8959896	-0.6518726\\
0.8960896	-0.6743112\\
0.8961896	-0.6971819\\
0.8962896	-0.7186255\\
0.8963896	-0.7383529\\
0.8964896	-0.7586307\\
0.8965897	-0.7780458\\
0.8966897	-0.8004301\\
0.8967897	-0.8176645\\
0.8968897	-0.835577\\
0.8969897	-0.8542767\\
0.8970897	-0.8713867\\
0.8971897	-0.8850158\\
0.8972897	-0.8979029\\
0.8973897	-0.9143206\\
0.8974897	-0.930034\\
0.8975898	-0.9438373\\
0.8976898	-0.9564985\\
0.8977898	-0.9696737\\
0.8978898	-0.9813718\\
0.8979898	-0.993727\\
0.8980898	-1.004901\\
0.8981898	-1.015215\\
0.8982898	-1.027974\\
0.8983898	-1.038534\\
0.8984898	-1.048425\\
0.8985899	-1.055272\\
0.8986899	-1.060802\\
0.8987899	-1.067384\\
0.8988899	-1.071373\\
0.8989899	-1.075337\\
0.8990899	-1.078122\\
0.8991899	-1.082879\\
0.8992899	-1.086604\\
0.8993899	-1.088017\\
0.8994899	-1.088134\\
0.89959	-1.090732\\
0.89969	-1.093504\\
0.89979	-1.094315\\
0.89989	-1.094542\\
0.89999	-1.093202\\
0.90009	-1.090637\\
0.90019	-1.088429\\
0.90029	-1.084512\\
0.90039	-1.080554\\
0.90049	-1.075699\\
0.9005901	-1.071468\\
0.9006901	-1.066335\\
0.9007901	-1.060127\\
0.9008901	-1.052687\\
0.9009901	-1.042634\\
0.9010901	-1.034347\\
0.9011901	-1.023444\\
0.9012901	-1.012179\\
0.9013901	-1.001449\\
0.9014901	-0.9894406\\
0.9015902	-0.9777434\\
0.9016902	-0.9647447\\
0.9017902	-0.9535905\\
0.9018902	-0.9415285\\
0.9019902	-0.9286746\\
0.9020902	-0.9145417\\
0.9021902	-0.8985904\\
0.9022902	-0.8827246\\
0.9023902	-0.8673122\\
0.9024902	-0.8504306\\
0.9025903	-0.8326797\\
0.9026903	-0.8166581\\
0.9027903	-0.7980608\\
0.9028903	-0.7801493\\
0.9029903	-0.7627838\\
0.9030903	-0.7431211\\
0.9031903	-0.7220874\\
0.9032903	-0.7020374\\
0.9033903	-0.6832111\\
0.9034903	-0.6636544\\
0.9035904	-0.6421948\\
0.9036904	-0.6216696\\
0.9037904	-0.6020504\\
0.9038904	-0.58021\\
0.9039904	-0.5568354\\
0.9040904	-0.5315256\\
0.9041904	-0.5078045\\
0.9042904	-0.4858586\\
0.9043904	-0.4594731\\
0.9044904	-0.4342897\\
0.9045905	-0.4091243\\
0.9046905	-0.3864143\\
0.9047905	-0.3630887\\
0.9048905	-0.3379578\\
0.9049905	-0.3137951\\
0.9050905	-0.2899586\\
0.9051905	-0.2651687\\
0.9052905	-0.2416732\\
0.9053905	-0.2151894\\
0.9054905	-0.1890896\\
0.9055906	-0.1633622\\
0.9056906	-0.1369932\\
0.9057906	-0.1119403\\
0.9058906	-0.08610143\\
0.9059906	-0.06119174\\
0.9060906	-0.03700839\\
0.9061906	-0.01149166\\
0.9062906	0.01370871\\
0.9063906	0.03802645\\
0.9064906	0.06336298\\
0.9065907	0.0894823\\
0.9066907	0.1129449\\
0.9067907	0.1358884\\
0.9068907	0.1626393\\
0.9069907	0.1873285\\
0.9070907	0.2120884\\
0.9071907	0.2376775\\
0.9072907	0.2610177\\
0.9073907	0.2862497\\
0.9074907	0.3088393\\
0.9075908	0.3328451\\
0.9076908	0.3542495\\
0.9077908	0.3744653\\
0.9078908	0.3970463\\
0.9079908	0.4229492\\
0.9080908	0.4460535\\
0.9081908	0.4649621\\
0.9082908	0.4838242\\
0.9083908	0.5052118\\
0.9084908	0.5275245\\
0.9085909	0.5457104\\
0.9086909	0.5653038\\
0.9087909	0.5856582\\
0.9088909	0.6085927\\
0.9089909	0.626992\\
0.9090909	0.6442607\\
0.9091909	0.661264\\
0.9092909	0.6796098\\
0.9093909	0.6958338\\
0.9094909	0.7123328\\
0.909591	0.7284706\\
0.909691	0.7436565\\
0.909791	0.7584011\\
0.909891	0.770981\\
0.909991	0.7849587\\
0.910091	0.7984511\\
0.910191	0.8098303\\
0.910291	0.8228157\\
0.910391	0.8335185\\
0.910491	0.8448614\\
0.9105911	0.8548903\\
0.9106911	0.8643194\\
0.9107911	0.8718712\\
0.9108911	0.8805112\\
0.9109911	0.8884397\\
0.9110911	0.894665\\
0.9111911	0.9025131\\
0.9112911	0.9098598\\
0.9113911	0.917342\\
0.9114911	0.9227096\\
0.9115912	0.9297463\\
0.9116912	0.9353914\\
0.9117912	0.9399443\\
0.9118912	0.9423623\\
0.9119912	0.9437673\\
0.9120912	0.9467853\\
0.9121912	0.948238\\
0.9122912	0.9474451\\
0.9123912	0.9469032\\
0.9124912	0.9444986\\
0.9125913	0.9437877\\
0.9126913	0.9414118\\
0.9127913	0.9370481\\
0.9128913	0.9328054\\
0.9129913	0.9317619\\
0.9130913	0.9276791\\
0.9131913	0.9232556\\
0.9132913	0.9165582\\
0.9133913	0.9099605\\
0.9134913	0.9054601\\
0.9135914	0.8981872\\
0.9136914	0.8909647\\
0.9137914	0.8828297\\
0.9138914	0.8743782\\
0.9139914	0.8658698\\
0.9140914	0.8551222\\
0.9141914	0.8444898\\
0.9142914	0.8314619\\
0.9143914	0.8188986\\
0.9144914	0.8059581\\
0.9145915	0.7934654\\
0.9146915	0.7828614\\
0.9147915	0.7703218\\
0.9148915	0.7569447\\
0.9149915	0.7440011\\
0.9150915	0.7312053\\
0.9151915	0.7167854\\
0.9152915	0.7015528\\
0.9153915	0.6858445\\
0.9154915	0.6664525\\
0.9155916	0.6483022\\
0.9156916	0.6313056\\
0.9157916	0.6152921\\
0.9158916	0.597717\\
0.9159916	0.5787638\\
0.9160916	0.5607109\\
0.9161916	0.5429814\\
0.9162916	0.5248371\\
0.9163916	0.5044361\\
0.9164916	0.4828874\\
0.9165917	0.4627191\\
0.9166917	0.4442403\\
0.9167917	0.4243986\\
0.9168917	0.4043083\\
0.9169917	0.3850307\\
0.9170917	0.3632136\\
0.9171917	0.3404716\\
0.9172917	0.3190807\\
0.9173917	0.2983779\\
0.9174917	0.2758359\\
0.9175918	0.2533214\\
0.9176918	0.230047\\
0.9177918	0.2081071\\
0.9178918	0.1865851\\
0.9179918	0.165369\\
0.9180918	0.1428047\\
0.9181918	0.1183039\\
0.9182918	0.09551076\\
0.9183918	0.07311625\\
0.9184918	0.05004763\\
0.9185919	0.03007829\\
0.9186919	0.008269126\\
0.9187919	-0.01275915\\
0.9188919	-0.03448055\\
0.9189919	-0.05519299\\
0.9190919	-0.07685766\\
0.9191919	-0.09788912\\
0.9192919	-0.1217366\\
0.9193919	-0.1423783\\
0.9194919	-0.1645607\\
0.919592	-0.1858012\\
0.919692	-0.2057844\\
0.919792	-0.2268543\\
0.919892	-0.2479616\\
0.919992	-0.2702627\\
0.920092	-0.2908486\\
0.920192	-0.3099778\\
0.920292	-0.3283096\\
0.920392	-0.3470439\\
0.920492	-0.3679502\\
0.9205921	-0.387369\\
0.9206921	-0.4087707\\
0.9207921	-0.4267337\\
0.9208921	-0.4468012\\
0.9209921	-0.4649486\\
0.9210921	-0.4843567\\
0.9211921	-0.5014645\\
0.9212921	-0.5172461\\
0.9213921	-0.5328065\\
0.9214921	-0.547928\\
0.9215922	-0.5602587\\
0.9216922	-0.5743869\\
0.9217922	-0.5891183\\
0.9218922	-0.603505\\
0.9219922	-0.6179732\\
0.9220922	-0.6301918\\
0.9221922	-0.643749\\
0.9222922	-0.6575112\\
0.9223922	-0.6684104\\
0.9224922	-0.6821993\\
0.9225923	-0.6939728\\
0.9226923	-0.704257\\
0.9227923	-0.7144639\\
0.9228923	-0.724275\\
0.9229923	-0.7339172\\
0.9230923	-0.7416328\\
0.9231923	-0.7505693\\
0.9232923	-0.7568413\\
0.9233923	-0.7645535\\
0.9234923	-0.7722149\\
0.9235924	-0.779517\\
0.9236924	-0.7842728\\
0.9237924	-0.7885006\\
0.9238924	-0.789807\\
0.9239924	-0.7965579\\
0.9240924	-0.799746\\
0.9241924	-0.8051725\\
0.9242924	-0.8078231\\
0.9243924	-0.8101866\\
0.9244924	-0.8107603\\
0.9245925	-0.8103186\\
0.9246925	-0.8073115\\
0.9247925	-0.8102197\\
0.9248925	-0.808565\\
0.9249925	-0.8108257\\
0.9250925	-0.8091342\\
0.9251925	-0.8074705\\
0.9252925	-0.8040123\\
0.9253925	-0.8014717\\
0.9254925	-0.7986967\\
0.9255926	-0.7921106\\
0.9256926	-0.7880878\\
0.9257926	-0.7806578\\
0.9258926	-0.7730362\\
0.9259926	-0.7653752\\
0.9260926	-0.758987\\
0.9261926	-0.75104\\
0.9262926	-0.7447699\\
0.9263926	-0.7351883\\
0.9264926	-0.7269011\\
0.9265927	-0.7180506\\
0.9266927	-0.7082555\\
0.9267927	-0.6993566\\
0.9268927	-0.6890224\\
0.9269927	-0.6782856\\
0.9270927	-0.6636476\\
0.9271927	-0.6530425\\
0.9272927	-0.6416724\\
0.9273927	-0.6298594\\
0.9274927	-0.6157342\\
0.9275928	-0.6029668\\
0.9276928	-0.5898353\\
0.9277928	-0.5747377\\
0.9278928	-0.5585892\\
0.9279928	-0.5435868\\
0.9280928	-0.5295024\\
0.9281928	-0.5145962\\
0.9282928	-0.5022153\\
0.9283928	-0.4850544\\
0.9284928	-0.4704957\\
0.9285929	-0.4551954\\
0.9286929	-0.4381585\\
0.9287929	-0.421259\\
0.9288929	-0.4031791\\
0.9289929	-0.3837795\\
0.9290929	-0.3633956\\
0.9291929	-0.3448757\\
0.9292929	-0.3250129\\
0.9293929	-0.3075004\\
0.9294929	-0.2882685\\
0.929593	-0.2685908\\
0.929693	-0.2506756\\
0.929793	-0.2302451\\
0.929893	-0.2118704\\
0.929993	-0.1921792\\
0.930093	-0.1752592\\
0.930193	-0.1580911\\
0.930293	-0.1400063\\
0.930393	-0.1203132\\
0.930493	-0.09872919\\
0.9305931	-0.08068705\\
0.9306931	-0.06027234\\
0.9307931	-0.04224378\\
0.9308931	-0.02219277\\
0.9309931	-0.003805226\\
0.9310931	0.01532682\\
0.9311931	0.03648722\\
0.9312931	0.05366342\\
0.9313931	0.07515263\\
0.9314931	0.09210438\\
0.9315932	0.1121144\\
0.9316932	0.1313661\\
0.9317932	0.148018\\
0.9318932	0.1675485\\
0.9319932	0.1858983\\
0.9320932	0.205855\\
0.9321932	0.2235816\\
0.9322932	0.2390681\\
0.9323932	0.2570909\\
0.9324932	0.2748862\\
0.9325933	0.2935898\\
0.9326933	0.3098846\\
0.9327933	0.3277435\\
0.9328933	0.344738\\
0.9329933	0.3590149\\
0.9330933	0.3743536\\
0.9331933	0.3892753\\
0.9332933	0.4041098\\
0.9333933	0.419444\\
0.9334933	0.4331205\\
0.9335934	0.4488325\\
0.9336934	0.4640048\\
0.9337934	0.4773924\\
0.9338934	0.4912788\\
0.9339934	0.5027112\\
0.9340934	0.5181542\\
0.9341934	0.5308504\\
0.9342934	0.5422838\\
0.9343934	0.5535456\\
0.9344934	0.563058\\
0.9345935	0.5739467\\
0.9346935	0.5812044\\
0.9347935	0.5925302\\
0.9348935	0.6026585\\
0.9349935	0.6113032\\
0.9350935	0.6187109\\
0.9351935	0.6243006\\
0.9352935	0.6319596\\
0.9353935	0.639494\\
0.9354935	0.6454222\\
0.9355936	0.6505619\\
0.9356936	0.6568359\\
0.9357936	0.6611002\\
0.9358936	0.6683714\\
0.9359936	0.6709431\\
0.9360936	0.6771258\\
0.9361936	0.6812596\\
0.9362936	0.683804\\
0.9363936	0.6910326\\
0.9364936	0.6927268\\
0.9365937	0.6946148\\
0.9366937	0.6951289\\
0.9367937	0.692932\\
0.9368937	0.6929812\\
0.9369937	0.6913584\\
0.9370937	0.6902787\\
0.9371937	0.6890349\\
0.9372937	0.6874681\\
0.9373937	0.6868005\\
0.9374937	0.6858923\\
0.9375938	0.6804772\\
0.9376938	0.6782615\\
0.9377938	0.6736859\\
0.9378938	0.6678656\\
0.9379938	0.6630653\\
0.9380938	0.6567907\\
0.9381938	0.652189\\
0.9382938	0.6442366\\
0.9383938	0.6363832\\
0.9384938	0.6287995\\
0.9385939	0.6214661\\
0.9386939	0.6123621\\
0.9387939	0.6044593\\
0.9388939	0.5971073\\
0.9389939	0.5868071\\
0.9390939	0.579648\\
0.9391939	0.5699668\\
0.9392939	0.5590801\\
0.9393939	0.5494423\\
0.9394939	0.5377022\\
0.939594	0.5258037\\
0.939694	0.5170824\\
0.939794	0.5054062\\
0.939894	0.4926254\\
0.939994	0.4801084\\
0.940094	0.4643176\\
0.940194	0.4526101\\
0.940294	0.4397209\\
0.940394	0.4263951\\
0.940494	0.4154968\\
0.9405941	0.3992301\\
0.9406941	0.386351\\
0.9407941	0.3717596\\
0.9408941	0.354719\\
0.9409941	0.3365951\\
0.9410941	0.3209485\\
0.9411941	0.3028771\\
0.9412941	0.2868951\\
0.9413941	0.2757163\\
0.9414941	0.259086\\
0.9415942	0.2430118\\
0.9416942	0.229437\\
0.9417942	0.2135866\\
0.9418942	0.1966692\\
0.9419942	0.1818452\\
0.9420942	0.1632977\\
0.9421942	0.1472352\\
0.9422942	0.1301245\\
0.9423942	0.1122909\\
0.9424942	0.09532302\\
0.9425943	0.07781573\\
0.9426943	0.05950104\\
0.9427943	0.04124896\\
0.9428943	0.0254539\\
0.9429943	0.008057716\\
0.9430943	-0.01041668\\
0.9431943	-0.02694633\\
0.9432943	-0.04176401\\
0.9433943	-0.05639745\\
0.9434943	-0.07193964\\
0.9435944	-0.08880194\\
0.9436944	-0.1020565\\
0.9437944	-0.1179563\\
0.9438944	-0.1355914\\
0.9439944	-0.1513841\\
0.9440944	-0.1667127\\
0.9441944	-0.1816671\\
0.9442944	-0.1986606\\
0.9443944	-0.2127495\\
0.9444944	-0.2274523\\
0.9445945	-0.2461043\\
0.9446945	-0.2587913\\
0.9447945	-0.274354\\
0.9448945	-0.2885013\\
0.9449945	-0.3014868\\
0.9450945	-0.3155852\\
0.9451945	-0.3308993\\
0.9452945	-0.3418049\\
0.9453945	-0.355462\\
0.9454945	-0.3696215\\
0.9455946	-0.3827467\\
0.9456946	-0.3953629\\
0.9457946	-0.4081283\\
0.9458946	-0.4208305\\
0.9459946	-0.4299903\\
0.9460946	-0.4426987\\
0.9461946	-0.452162\\
0.9462946	-0.4595803\\
0.9463946	-0.4686461\\
0.9464946	-0.4786891\\
0.9465947	-0.4881278\\
0.9466947	-0.4937415\\
0.9467947	-0.5051951\\
0.9468947	-0.5125183\\
0.9469947	-0.5170892\\
0.9470947	-0.5232365\\
0.9471947	-0.5290141\\
0.9472947	-0.5368194\\
0.9473947	-0.5449753\\
0.9474947	-0.550772\\
0.9475948	-0.5571507\\
0.9476948	-0.5625066\\
0.9477948	-0.567907\\
0.9478948	-0.5733187\\
0.9479948	-0.5742678\\
0.9480948	-0.5777067\\
0.9481948	-0.5800934\\
0.9482948	-0.5811421\\
0.9483948	-0.5835472\\
0.9484948	-0.5847174\\
0.9485949	-0.5862346\\
0.9486949	-0.5869702\\
0.9487949	-0.588034\\
0.9488949	-0.5890128\\
0.9489949	-0.5881389\\
0.9490949	-0.587564\\
0.9491949	-0.5866152\\
0.9492949	-0.5852898\\
0.9493949	-0.5815114\\
0.9494949	-0.5778659\\
0.949595	-0.5764794\\
0.949695	-0.5738932\\
0.949795	-0.5687367\\
0.949895	-0.5636865\\
0.949995	-0.5618224\\
0.950095	-0.5567372\\
0.950195	-0.5489893\\
0.950295	-0.5432611\\
0.950395	-0.5373224\\
0.950495	-0.5298693\\
0.9505951	-0.5213522\\
0.9506951	-0.5150035\\
0.9507951	-0.5079639\\
0.9508951	-0.5005443\\
0.9509951	-0.494829\\
0.9510951	-0.4870767\\
0.9511951	-0.4773586\\
0.9512951	-0.4673453\\
0.9513951	-0.4578918\\
0.9514951	-0.4479042\\
0.9515952	-0.4371615\\
0.9516952	-0.4251189\\
0.9517952	-0.4167456\\
0.9518952	-0.405164\\
0.9519952	-0.394967\\
0.9520952	-0.3828356\\
0.9521952	-0.3699015\\
0.9522952	-0.3589905\\
0.9523952	-0.3464107\\
0.9524952	-0.3337287\\
0.9525953	-0.3238851\\
0.9526953	-0.31158\\
0.9527953	-0.2985392\\
0.9528953	-0.2862772\\
0.9529953	-0.2747818\\
0.9530953	-0.2620358\\
0.9531953	-0.2476537\\
0.9532953	-0.2345116\\
0.9533953	-0.2223549\\
0.9534953	-0.2078036\\
0.9535954	-0.1912056\\
0.9536954	-0.176451\\
0.9537954	-0.162151\\
0.9538954	-0.1471081\\
0.9539954	-0.1303002\\
0.9540954	-0.1155004\\
0.9541954	-0.1002005\\
0.9542954	-0.0869588\\
0.9543954	-0.07213975\\
0.9544954	-0.05693399\\
0.9545955	-0.04201564\\
0.9546955	-0.03044921\\
0.9547955	-0.01468818\\
0.9548955	0.0009322527\\
0.9549955	0.01453861\\
0.9550955	0.0284872\\
0.9551955	0.04070618\\
0.9552955	0.05534169\\
0.9553955	0.06988208\\
0.9554955	0.08279017\\
0.9555956	0.09500604\\
0.9556956	0.1103717\\
0.9557956	0.1244758\\
0.9558956	0.1350068\\
0.9559956	0.1458697\\
0.9560956	0.1615612\\
0.9561956	0.1778002\\
0.9562956	0.1920798\\
0.9563956	0.2046407\\
0.9564956	0.2200337\\
0.9565957	0.2340579\\
0.9566957	0.2457756\\
0.9567957	0.2556848\\
0.9568957	0.2650644\\
0.9569957	0.2756193\\
0.9570957	0.2877628\\
0.9571957	0.3003273\\
0.9572957	0.311744\\
0.9573957	0.3201091\\
0.9574957	0.3313737\\
0.9575958	0.3430425\\
0.9576958	0.3541213\\
0.9577958	0.3639757\\
0.9578958	0.3735513\\
0.9579958	0.3833538\\
0.9580958	0.3935299\\
0.9581958	0.4004696\\
0.9582958	0.4071937\\
0.9583958	0.4137999\\
0.9584958	0.421178\\
0.9585959	0.4273501\\
0.9586959	0.4350438\\
0.9587959	0.4414466\\
0.9588959	0.4462327\\
0.9589959	0.4528118\\
0.9590959	0.459069\\
0.9591959	0.4670567\\
0.9592959	0.4700565\\
0.9593959	0.4743002\\
0.9594959	0.4803741\\
0.959596	0.4838832\\
0.959696	0.4855088\\
0.959796	0.4868653\\
0.959896	0.4903289\\
0.959996	0.492274\\
0.960096	0.4931409\\
0.960196	0.4942626\\
0.960296	0.4946533\\
0.960396	0.492933\\
0.960496	0.4944746\\
0.9605961	0.4970297\\
0.9606961	0.4976208\\
0.9607961	0.498759\\
0.9608961	0.4966892\\
0.9609961	0.4961983\\
0.9610961	0.4960204\\
0.9611961	0.4949377\\
0.9612961	0.4918593\\
0.9613961	0.4876585\\
0.9614961	0.4845356\\
0.9615962	0.4807442\\
0.9616962	0.4777984\\
0.9617962	0.4744884\\
0.9618962	0.4682385\\
0.9619962	0.462456\\
0.9620962	0.4585433\\
0.9621962	0.4556436\\
0.9622962	0.4486361\\
0.9623962	0.4424553\\
0.9624962	0.4362261\\
0.9625963	0.4264237\\
0.9626963	0.4196242\\
0.9627963	0.4131149\\
0.9628963	0.4083897\\
0.9629963	0.3991553\\
0.9630963	0.3901263\\
0.9631963	0.3796171\\
0.9632963	0.369596\\
0.9633963	0.362852\\
0.9634963	0.3533937\\
0.9635964	0.3403924\\
0.9636964	0.3291811\\
0.9637964	0.3223477\\
0.9638964	0.3138836\\
0.9639964	0.3052061\\
0.9640964	0.2965102\\
0.9641964	0.2866703\\
0.9642964	0.2770653\\
0.9643964	0.2662347\\
0.9644964	0.2546668\\
0.9645965	0.2434261\\
0.9646965	0.2342416\\
0.9647965	0.2231614\\
0.9648965	0.2096901\\
0.9649965	0.1964625\\
0.9650965	0.1834957\\
0.9651965	0.1727879\\
0.9652965	0.1611693\\
0.9653965	0.1476704\\
0.9654965	0.1340687\\
0.9655966	0.121338\\
0.9656966	0.1075405\\
0.9657966	0.09329792\\
0.9658966	0.0818907\\
0.9659966	0.07256819\\
0.9660966	0.06008257\\
0.9661966	0.04767925\\
0.9662966	0.03529349\\
0.9663966	0.0217918\\
0.9664966	0.009768708\\
0.9665967	-0.00200381\\
0.9666967	-0.01298044\\
0.9667967	-0.02632451\\
0.9668967	-0.03944934\\
0.9669967	-0.05260353\\
0.9670967	-0.06421152\\
0.9671967	-0.07775639\\
0.9672967	-0.08931448\\
0.9673967	-0.1001161\\
0.9674967	-0.1114681\\
0.9675968	-0.1244901\\
0.9676968	-0.1378803\\
0.9677968	-0.1502735\\
0.9678968	-0.1614506\\
0.9679968	-0.1717876\\
0.9680968	-0.1821858\\
0.9681968	-0.1906207\\
0.9682968	-0.1992302\\
0.9683968	-0.2106283\\
0.9684968	-0.2213311\\
0.9685969	-0.2308891\\
0.9686969	-0.2428489\\
0.9687969	-0.2519443\\
0.9688969	-0.2612712\\
0.9689969	-0.268912\\
0.9690969	-0.2774621\\
0.9691969	-0.2879768\\
0.9692969	-0.2969576\\
0.9693969	-0.3054783\\
0.9694969	-0.3142846\\
0.969597	-0.323468\\
0.969697	-0.3305872\\
0.969797	-0.3374242\\
0.969897	-0.3423471\\
0.969997	-0.3507728\\
0.970097	-0.3596987\\
0.970197	-0.3661942\\
0.970297	-0.3723006\\
0.970397	-0.3788382\\
0.970497	-0.3857245\\
0.9705971	-0.3893384\\
0.9706971	-0.3917698\\
0.9707971	-0.3960652\\
0.9708971	-0.3999947\\
0.9709971	-0.4040909\\
0.9710971	-0.4072901\\
0.9711971	-0.4103468\\
0.9712971	-0.4157353\\
0.9713971	-0.4182068\\
0.9714971	-0.4185705\\
0.9715972	-0.420485\\
0.9716972	-0.4204737\\
0.9717972	-0.4217343\\
0.9718972	-0.4211334\\
0.9719972	-0.4198539\\
0.9720972	-0.4221888\\
0.9721972	-0.4242988\\
0.9722972	-0.4223396\\
0.9723972	-0.4215439\\
0.9724972	-0.4216923\\
0.9725973	-0.4209795\\
0.9726973	-0.4201031\\
0.9727973	-0.4184904\\
0.9728973	-0.4163349\\
0.9729973	-0.4141077\\
0.9730973	-0.412146\\
0.9731973	-0.4102875\\
0.9732973	-0.4090167\\
0.9733973	-0.4060732\\
0.9734973	-0.4014042\\
0.9735974	-0.3970277\\
0.9736974	-0.3920365\\
0.9737974	-0.3852299\\
0.9738974	-0.3779865\\
0.9739974	-0.3719043\\
0.9740974	-0.3678421\\
0.9741974	-0.3623654\\
0.9742974	-0.3557642\\
0.9743974	-0.3519238\\
0.9744974	-0.3451432\\
0.9745975	-0.338387\\
0.9746975	-0.331317\\
0.9747975	-0.3242943\\
0.9748975	-0.3156464\\
0.9749975	-0.3067839\\
0.9750975	-0.2986403\\
0.9751975	-0.2882986\\
0.9752975	-0.2790007\\
0.9753975	-0.2702839\\
0.9754975	-0.2618637\\
0.9755976	-0.2534772\\
0.9756976	-0.246663\\
0.9757976	-0.2378979\\
0.9758976	-0.2311779\\
0.9759976	-0.2224332\\
0.9760976	-0.2116426\\
0.9761976	-0.2028448\\
0.9762976	-0.192631\\
0.9763976	-0.1806018\\
0.9764976	-0.1698679\\
0.9765977	-0.1582049\\
0.9766977	-0.1480279\\
0.9767977	-0.1355806\\
0.9768977	-0.1239904\\
0.9769977	-0.1142967\\
0.9770977	-0.1047859\\
0.9771977	-0.09370314\\
0.9772977	-0.08470296\\
0.9773977	-0.07600253\\
0.9774977	-0.06495239\\
0.9775978	-0.05657945\\
0.9776978	-0.04536228\\
0.9777978	-0.03655159\\
0.9778978	-0.02484307\\
0.9779978	-0.01298707\\
0.9780978	-0.001061766\\
0.9781978	0.01064782\\
0.9782978	0.02130736\\
0.9783978	0.0316482\\
0.9784978	0.04334533\\
0.9785979	0.05433399\\
0.9786979	0.06720639\\
0.9787979	0.07759426\\
0.9788979	0.08975966\\
0.9789979	0.1006138\\
0.9790979	0.1105342\\
0.9791979	0.1218409\\
0.9792979	0.1311086\\
0.9793979	0.1379543\\
0.9794979	0.1455527\\
0.979598	0.1565642\\
0.979698	0.167541\\
0.979798	0.1759281\\
0.979898	0.1834199\\
0.979998	0.1921495\\
0.980098	0.2008183\\
0.980198	0.2097816\\
0.980298	0.2173901\\
0.980398	0.2250704\\
0.980498	0.233218\\
0.9805981	0.2419339\\
0.9806981	0.2490882\\
0.9807981	0.255778\\
0.9808981	0.264248\\
0.9809981	0.2702995\\
0.9810981	0.2778741\\
0.9811981	0.2850838\\
0.9812981	0.2916867\\
0.9813981	0.2973934\\
0.9814981	0.3027791\\
0.9815982	0.3093554\\
0.9816982	0.3140344\\
0.9817982	0.32056\\
0.9818982	0.3262678\\
0.9819982	0.3316494\\
0.9820982	0.3354403\\
0.9821982	0.3400692\\
0.9822982	0.3444675\\
0.9823982	0.3488369\\
0.9824982	0.3522171\\
0.9825983	0.354164\\
0.9826983	0.3562352\\
0.9827983	0.3582822\\
0.9828983	0.3602299\\
0.9829983	0.3628679\\
0.9830983	0.3648848\\
0.9831983	0.3658908\\
0.9832983	0.3669327\\
0.9833983	0.3670756\\
0.9834983	0.367451\\
0.9835984	0.3665226\\
0.9836984	0.3662381\\
0.9837984	0.3652911\\
0.9838984	0.3655845\\
0.9839984	0.3665581\\
0.9840984	0.3663347\\
0.9841984	0.3643606\\
0.9842984	0.3632837\\
0.9843984	0.3612446\\
0.9844984	0.3589601\\
0.9845985	0.3587334\\
0.9846985	0.3570748\\
0.9847985	0.3536955\\
0.9848985	0.3511992\\
0.9849985	0.3469403\\
0.9850985	0.3446019\\
0.9851985	0.340773\\
0.9852985	0.33773\\
0.9853985	0.3340128\\
0.9854985	0.3310566\\
0.9855986	0.3295748\\
0.9856986	0.3270249\\
0.9857986	0.3233911\\
0.9858986	0.3193269\\
0.9859986	0.3154486\\
0.9860986	0.3117336\\
0.9861986	0.3089729\\
0.9862986	0.3058004\\
0.9863986	0.3040407\\
0.9864986	0.3017524\\
0.9865987	0.299382\\
0.9866987	0.2971416\\
0.9867987	0.2945383\\
0.9868987	0.2898725\\
0.9869987	0.2871816\\
0.9870987	0.2827325\\
0.9871987	0.2800072\\
0.9872987	0.2776064\\
0.9873987	0.2747684\\
0.9874987	0.2727094\\
0.9875988	0.2678713\\
0.9876988	0.2653315\\
0.9877988	0.2609049\\
0.9878988	0.2561691\\
0.9879988	0.2521826\\
0.9880988	0.247484\\
0.9881988	0.2427369\\
0.9882988	0.237257\\
0.9883988	0.2330031\\
0.9884988	0.2281965\\
0.9885989	0.2225634\\
0.9886989	0.2156332\\
0.9887989	0.2092343\\
0.9888989	0.2045617\\
0.9889989	0.2002445\\
0.9890989	0.1927772\\
0.9891989	0.186301\\
0.9892989	0.1810907\\
0.9893989	0.1769253\\
0.9894989	0.1726081\\
0.989599	0.1698845\\
0.989699	0.1681956\\
0.989799	0.1646312\\
0.989899	0.1611549\\
0.989999	0.1567643\\
0.990099	0.1532928\\
0.990199	0.148284\\
0.990299	0.144163\\
0.990399	0.1410361\\
0.990499	0.1384102\\
0.9905991	0.1335693\\
0.9906991	0.1302041\\
0.9907991	0.1240498\\
0.9908991	0.1209733\\
0.9909991	0.1148119\\
0.9910991	0.1099225\\
0.9911991	0.1030058\\
0.9912991	0.09634938\\
0.9913991	0.09293408\\
0.9914991	0.08597533\\
0.9915992	0.07927353\\
0.9916992	0.07447438\\
0.9917992	0.07117196\\
0.9918992	0.06739463\\
0.9919992	0.0635175\\
0.9920992	0.05893427\\
0.9921992	0.05475966\\
0.9922992	0.0498781\\
0.9923992	0.04478784\\
0.9924992	0.04138685\\
0.9925993	0.03686293\\
0.9926993	0.03018379\\
0.9927993	0.02413468\\
0.9928993	0.01656235\\
0.9929993	0.01307869\\
0.9930993	0.007982238\\
0.9931993	0.003065718\\
0.9932993	-0.000908818\\
0.9933993	-0.00388665\\
0.9934993	-0.006290269\\
0.9935994	-0.01053804\\
0.9936994	-0.01308806\\
0.9937994	-0.01759906\\
0.9938994	-0.02043064\\
0.9939994	-0.02333982\\
0.9940994	-0.02716152\\
0.9941994	-0.03313453\\
0.9942994	-0.03618255\\
0.9943994	-0.04015866\\
0.9944994	-0.04399792\\
0.9945995	-0.04624639\\
0.9946995	-0.04738769\\
0.9947995	-0.04895307\\
0.9948995	-0.0527765\\
0.9949995	-0.05609447\\
0.9950995	-0.06000801\\
0.9951995	-0.06559563\\
0.9952995	-0.07240253\\
0.9953995	-0.07624982\\
0.9954995	-0.08043311\\
0.9955996	-0.08462096\\
0.9956996	-0.08915276\\
0.9957996	-0.09300281\\
0.9958996	-0.09629444\\
0.9959996	-0.09916009\\
0.9960996	-0.1036796\\
0.9961996	-0.1075946\\
0.9962996	-0.110159\\
0.9963996	-0.1146994\\
0.9964996	-0.1194568\\
0.9965997	-0.1219368\\
0.9966997	-0.1238106\\
0.9967997	-0.1254491\\
0.9968997	-0.1253265\\
0.9969997	-0.1275029\\
0.9970997	-0.1285938\\
0.9971997	-0.1297638\\
0.9972997	-0.1336577\\
0.9973997	-0.1353209\\
0.9974997	-0.1379713\\
0.9975998	-0.1391612\\
0.9976998	-0.1399976\\
0.9977998	-0.1408983\\
0.9978998	-0.1404964\\
0.9979998	-0.1412137\\
0.9980998	-0.1432231\\
0.9981998	-0.1454826\\
0.9982998	-0.1477692\\
0.9983998	-0.1502429\\
0.9984998	-0.153985\\
0.9985999	-0.157245\\
0.9986999	-0.1569888\\
0.9987999	-0.1584671\\
0.9988999	-0.1601096\\
0.9989999	-0.1630507\\
0.9990999	-0.1657033\\
0.9991999	-0.1691009\\
0.9992999	-0.1710002\\
0.9993999	-0.1741735\\
0.9994999	-0.1745732\\
0.9996	-0.1752182\\
0.9997	-0.1759186\\
0.9998	-0.1749133\\
0.9999	-0.1771138\\
1	-0.1777804\\
};
\addlegendentry{$\text{imag(}\psi{}_\text{n}\text{(x,t))}$};

\addplot [color=red,mark size=0.7pt,only marks,mark=*,mark options={solid}]
  table[row sep=crcr]{%
0	1.64442355057845e-22\\
0.0167016701670167	1.5197240973474e-21\\
0.0334033403340334	1.33763780455174e-20\\
0.0501050105010501	1.12133413116993e-19\\
0.0668066806680668	8.95270493104685e-19\\
0.0835083508350835	6.80763438269121e-18\\
0.1002100210021	4.93015868195705e-17\\
0.116911691169117	3.40054318759744e-16\\
0.133613361336134	2.23387266838271e-15\\
0.15031503150315	1.39762691885738e-14\\
0.167016701670167	8.32811706951729e-14\\
0.183718371837184	4.72634161474762e-13\\
0.2004200420042	2.55461888610083e-12\\
0.217121712171217	1.31507296853249e-11\\
0.233823382338234	6.44757356102553e-11\\
0.250525052505251	3.01068613919693e-10\\
0.267226722672267	1.33892849120405e-09\\
0.283928392839284	5.67116137544888e-09\\
0.300630063006301	2.28775414288284e-08\\
0.317331733173317	8.78960620290514e-08\\
0.334033403340334	3.2162678389677e-07\\
0.350735073507351	1.12087638809504e-06\\
0.367436743674367	3.72036847520886e-06\\
0.384138413841384	1.17608013199936e-05\\
0.400840084008401	3.54087498300945e-05\\
0.417541754175418	0.000101532948022041\\
0.434243424342434	0.000277284837026129\\
0.450945094509451	0.000721220344715269\\
0.467646764676468	0.0017866214319633\\
0.484348434843484	0.00421521598203439\\
0.501050105010501	0.00947174173184439\\
0.517751775177518	0.020270412443154\\
0.534453445344534	0.0413159787886048\\
0.551155115511551	0.0802040370079134\\
0.567856785678568	0.148284970173713\\
0.584558455845585	0.261108351062417\\
0.601260126012601	0.437892102505805\\
0.617961796179618	0.699417002653674\\
0.634663466346635	1.06396656679291\\
0.651365136513651	1.54149634514675\\
0.668066806680668	2.12705999685486\\
0.684768476847685	2.79537259493006\\
0.701470147014701	3.4988263520665\\
0.718171817181718	4.170880980297\\
0.734873487348735	4.73539190934414\\
0.751575157515752	5.12043409924688\\
0.768276827682768	5.27327423030381\\
0.784978497849785	5.17221602143576\\
0.801680168016802	4.83165232738973\\
0.818381838183818	4.2987027920769\\
0.835083508350835	3.64251949987771\\
0.851785178517852	2.93960565285496\\
0.868486848684868	2.25943048880823\\
0.885188518851885	1.6539851970363\\
0.901890189018902	1.15315310942942\\
0.918591859185919	0.76571129420871\\
0.935293529352935	0.484245790431272\\
0.951995199519952	0.291668398961461\\
0.968696869686969	0.1673152874426\\
0.985398539853985	0.0914122908027332\\
};
\addlegendentry{$\text{|}\psi{}_\text{a}\text{(x,t)|}^\text{2}$};

\addplot [color=black,mark size=0.7pt,only marks,mark=*,mark options={solid}]
  table[row sep=crcr]{%
0	-1.27675569491282e-11\\
0.0167016701670167	-1.44320639101746e-11\\
0.0334033403340334	8.33166972643569e-11\\
0.0501050105010501	3.18312386338341e-10\\
0.0668066806680668	2.56338629711749e-10\\
0.0835083508350835	-1.48390095555069e-09\\
0.1002100210021	-6.89785251261254e-09\\
0.116911691169117	-1.62836795869618e-08\\
0.133613361336134	-2.29601284074641e-08\\
0.15031503150315	-3.97835901640648e-09\\
0.167016701670167	9.58365455090127e-08\\
0.183718371837184	3.9342699718733e-07\\
0.2004200420042	1.12247026913904e-06\\
0.217121712171217	2.71604212208874e-06\\
0.233823382338234	5.83222812402185e-06\\
0.250525052505251	1.08836798290307e-05\\
0.267226722672267	1.56401123441653e-05\\
0.283928392839284	7.80587231741371e-06\\
0.300630063006301	-4.97829872645521e-05\\
0.317331733173317	-0.000227754095430591\\
0.334033403340334	-0.000566335031112856\\
0.350735073507351	-0.00081596484089443\\
0.367436743674367	-6.75446904657783e-05\\
0.384138413841384	0.00271033068745036\\
0.400840084008401	0.00550748163683275\\
0.417541754175418	0.000335102021987193\\
0.434243424342434	-0.0156226862771179\\
0.450945094509451	-0.0145407941265323\\
0.467646764676468	0.032613847384848\\
0.484348434843484	0.0420903518300913\\
0.501050105010501	-0.0792204046795816\\
0.517751775177518	-0.059717306962364\\
0.534453445344534	0.201146948732901\\
0.551155115511551	-0.0688004273915454\\
0.567856785678568	-0.285238552995914\\
0.584558455845585	0.496161249739722\\
0.601260126012601	-0.289391448348718\\
0.617961796179618	-0.265178612190064\\
0.634663466346635	0.866007584935154\\
0.651365136513651	-1.24156790836517\\
0.668066806680668	1.30089178426692\\
0.684768476847685	-1.11760633438503\\
0.701470147014701	0.837080846817071\\
0.718171817181718	-0.598202563403678\\
0.734873487348735	0.500297253588819\\
0.751575157515752	-0.599466577921309\\
0.768276827682768	0.906250388692695\\
0.784978497849785	-1.36668112283849\\
0.801680168016802	1.83172467689519\\
0.818381838183818	-2.05191201673006\\
0.835083508350835	1.75832469348223\\
0.851785178517852	-0.861865676153611\\
0.868486848684868	-0.3271773127962\\
0.885188518851885	1.12626490495303\\
0.901890189018902	-0.969203475363891\\
0.918591859185919	0.0657930174644708\\
0.935293529352935	0.612604447859163\\
0.951995199519952	-0.391822537920453\\
0.968696869686969	-0.211318712235213\\
0.985398539853985	0.275047076370718\\
};
\addlegendentry{$\text{imag(}\psi{}_\text{a}\text{(x,t))}$};

\end{axis}
\end{tikzpicture}%
		\caption{The numerical solution ($\psi_n$) plotted against the analytical solution ($\psi_a$).}
		\label{fig:periodicInfPlot}
	\end{subfigure}
	\begin{subfigure}{.9\linewidth}
		\setlength\figureheight{.5\linewidth}
		\setlength\figurewidth{.9\linewidth}
		% This file was created by matlab2tikz.
% Minimal pgfplots version: 1.3
%
%The latest updates can be retrieved from
%  http://www.mathworks.com/matlabcentral/fileexchange/22022-matlab2tikz
%where you can also make suggestions and rate matlab2tikz.
%
\definecolor{mycolor1}{rgb}{0.00000,0.44700,0.74100}%
%
\begin{tikzpicture}

\begin{axis}[%
width=0.95092\figurewidth,
height=\figureheight,
at={(0\figurewidth,0\figureheight)},
scale only axis,
xmin=0,
xmax=1,
xlabel={Position},
ymin=-0.06,
ymax=0.01,
ylabel={Magnitude},
title style={font=\bfseries},
title={$\text{Error of |}\psi{}_\text{n}\text{(x,t)|}^\text{2}\text{, dt = 2e-06, dx = 0.0001}$},
title style={font=\small},ticklabel style={font=\tiny}
]
\addplot [color=mycolor1,solid,forget plot]
  table[row sep=crcr]{%
0	-0.0515691\\
0.00010001	-0.05136202\\
0.00020002	-0.05115568\\
0.00030003	-0.05095009\\
0.00040004	-0.05074523\\
0.00050005	-0.0505411\\
0.00060006	-0.05033771\\
0.00070007	-0.05013505\\
0.00080008	-0.04993312\\
0.00090009	-0.04973191\\
0.0010001	-0.04953143\\
0.00110011	-0.04933167\\
0.00120012	-0.04913263\\
0.00130013	-0.04893431\\
0.00140014	-0.0487367\\
0.00150015	-0.04853981\\
0.00160016	-0.04834363\\
0.00170017	-0.04814815\\
0.00180018	-0.04795339\\
0.00190019	-0.04775932\\
0.0020002	-0.04756596\\
0.00210021	-0.0473733\\
0.00220022	-0.04718134\\
0.00230023	-0.04699007\\
0.00240024	-0.0467995\\
0.00250025	-0.04660961\\
0.00260026	-0.04642042\\
0.00270027	-0.04623191\\
0.00280028	-0.04604409\\
0.00290029	-0.04585696\\
0.0030003	-0.0456705\\
0.00310031	-0.04548472\\
0.00320032	-0.04529962\\
0.00330033	-0.04511519\\
0.00340034	-0.04493143\\
0.00350035	-0.04474835\\
0.00360036	-0.04456593\\
0.00370037	-0.04438418\\
0.00380038	-0.04420309\\
0.00390039	-0.04402267\\
0.0040004	-0.0438429\\
0.00410041	-0.04366379\\
0.00420042	-0.04348534\\
0.00430043	-0.04330754\\
0.00440044	-0.04313039\\
0.00450045	-0.0429539\\
0.00460046	-0.04277805\\
0.00470047	-0.04260284\\
0.00480048	-0.04242828\\
0.00490049	-0.04225436\\
0.0050005	-0.04208108\\
0.00510051	-0.04190844\\
0.00520052	-0.04173643\\
0.00530053	-0.04156505\\
0.00540054	-0.04139431\\
0.00550055	-0.0412242\\
0.00560056	-0.04105471\\
0.00570057	-0.04088585\\
0.00580058	-0.04071761\\
0.00590059	-0.04054999\\
0.0060006	-0.04038299\\
0.00610061	-0.04021661\\
0.00620062	-0.04005085\\
0.00630063	-0.0398857\\
0.00640064	-0.03972116\\
0.00650065	-0.03955723\\
0.00660066	-0.03939391\\
0.00670067	-0.03923119\\
0.00680068	-0.03906908\\
0.00690069	-0.03890757\\
0.0070007	-0.03874666\\
0.00710071	-0.03858635\\
0.00720072	-0.03842663\\
0.00730073	-0.03826751\\
0.00740074	-0.03810898\\
0.00750075	-0.03795104\\
0.00760076	-0.03779368\\
0.00770077	-0.03763692\\
0.00780078	-0.03748074\\
0.00790079	-0.03732514\\
0.0080008	-0.03717013\\
0.00810081	-0.03701569\\
0.00820082	-0.03686183\\
0.00830083	-0.03670854\\
0.00840084	-0.03655583\\
0.00850085	-0.03640369\\
0.00860086	-0.03625212\\
0.00870087	-0.03610112\\
0.00880088	-0.03595069\\
0.00890089	-0.03580082\\
0.0090009	-0.03565151\\
0.00910091	-0.03550276\\
0.00920092	-0.03535457\\
0.00930093	-0.03520694\\
0.00940094	-0.03505986\\
0.00950095	-0.03491334\\
0.00960096	-0.03476736\\
0.00970097	-0.03462194\\
0.00980098	-0.03447707\\
0.00990099	-0.03433274\\
0.010001	-0.03418895\\
0.01010101	-0.03404571\\
0.01020102	-0.03390301\\
0.01030103	-0.03376085\\
0.01040104	-0.03361923\\
0.01050105	-0.03347814\\
0.01060106	-0.03333758\\
0.01070107	-0.03319756\\
0.01080108	-0.03305807\\
0.01090109	-0.03291911\\
0.0110011	-0.03278067\\
0.01110111	-0.03264276\\
0.01120112	-0.03250537\\
0.01130113	-0.0323685\\
0.01140114	-0.03223215\\
0.01150115	-0.03209633\\
0.01160116	-0.03196101\\
0.01170117	-0.03182622\\
0.01180118	-0.03169193\\
0.01190119	-0.03155816\\
0.0120012	-0.0314249\\
0.01210121	-0.03129214\\
0.01220122	-0.03115989\\
0.01230123	-0.03102815\\
0.01240124	-0.03089691\\
0.01250125	-0.03076617\\
0.01260126	-0.03063593\\
0.01270127	-0.03050619\\
0.01280128	-0.03037694\\
0.01290129	-0.03024819\\
0.0130013	-0.03011993\\
0.01310131	-0.02999216\\
0.01320132	-0.02986489\\
0.01330133	-0.0297381\\
0.01340134	-0.0296118\\
0.01350135	-0.02948598\\
0.01360136	-0.02936065\\
0.01370137	-0.02923579\\
0.01380138	-0.02911142\\
0.01390139	-0.02898753\\
0.0140014	-0.02886411\\
0.01410141	-0.02874117\\
0.01420142	-0.0286187\\
0.01430143	-0.0284967\\
0.01440144	-0.02837518\\
0.01450145	-0.02825412\\
0.01460146	-0.02813353\\
0.01470147	-0.02801341\\
0.01480148	-0.02789375\\
0.01490149	-0.02777455\\
0.0150015	-0.02765581\\
0.01510151	-0.02753754\\
0.01520152	-0.02741972\\
0.01530153	-0.02730235\\
0.01540154	-0.02718545\\
0.01550155	-0.02706899\\
0.01560156	-0.02695299\\
0.01570157	-0.02683744\\
0.01580158	-0.02672233\\
0.01590159	-0.02660767\\
0.0160016	-0.02649346\\
0.01610161	-0.0263797\\
0.01620162	-0.02626637\\
0.01630163	-0.02615349\\
0.01640164	-0.02604104\\
0.01650165	-0.02592904\\
0.01660166	-0.02581747\\
0.01670167	-0.02570634\\
0.01680168	-0.02559564\\
0.01690169	-0.02548537\\
0.0170017	-0.02537553\\
0.01710171	-0.02526612\\
0.01720172	-0.02515714\\
0.01730173	-0.02504859\\
0.01740174	-0.02494046\\
0.01750175	-0.02483275\\
0.01760176	-0.02472547\\
0.01770177	-0.0246186\\
0.01780178	-0.02451216\\
0.01790179	-0.02440613\\
0.0180018	-0.02430052\\
0.01810181	-0.02419532\\
0.01820182	-0.02409054\\
0.01830183	-0.02398617\\
0.01840184	-0.02388221\\
0.01850185	-0.02377866\\
0.01860186	-0.02367551\\
0.01870187	-0.02357277\\
0.01880188	-0.02347044\\
0.01890189	-0.02336851\\
0.0190019	-0.02326698\\
0.01910191	-0.02316586\\
0.01920192	-0.02306513\\
0.01930193	-0.0229648\\
0.01940194	-0.02286486\\
0.01950195	-0.02276533\\
0.01960196	-0.02266618\\
0.01970197	-0.02256743\\
0.01980198	-0.02246907\\
0.01990199	-0.02237109\\
0.020002	-0.02227351\\
0.02010201	-0.02217631\\
0.02020202	-0.0220795\\
0.02030203	-0.02198307\\
0.02040204	-0.02188703\\
0.02050205	-0.02179137\\
0.02060206	-0.02169608\\
0.02070207	-0.02160118\\
0.02080208	-0.02150665\\
0.02090209	-0.0214125\\
0.0210021	-0.02131873\\
0.02110211	-0.02122533\\
0.02120212	-0.0211323\\
0.02130213	-0.02103964\\
0.02140214	-0.02094735\\
0.02150215	-0.02085543\\
0.02160216	-0.02076388\\
0.02170217	-0.02067269\\
0.02180218	-0.02058187\\
0.02190219	-0.02049141\\
0.0220022	-0.02040131\\
0.02210221	-0.02031157\\
0.02220222	-0.02022219\\
0.02230223	-0.02013317\\
0.02240224	-0.02004451\\
0.02250225	-0.0199562\\
0.02260226	-0.01986825\\
0.02270227	-0.01978065\\
0.02280228	-0.0196934\\
0.02290229	-0.0196065\\
0.0230023	-0.01951995\\
0.02310231	-0.01943375\\
0.02320232	-0.0193479\\
0.02330233	-0.01926239\\
0.02340234	-0.01917723\\
0.02350235	-0.01909241\\
0.02360236	-0.01900793\\
0.02370237	-0.01892379\\
0.02380238	-0.01883999\\
0.02390239	-0.01875653\\
0.0240024	-0.01867341\\
0.02410241	-0.01859062\\
0.02420242	-0.01850817\\
0.02430243	-0.01842605\\
0.02440244	-0.01834426\\
0.02450245	-0.01826281\\
0.02460246	-0.01818168\\
0.02470247	-0.01810089\\
0.02480248	-0.01802042\\
0.02490249	-0.01794027\\
0.0250025	-0.01786046\\
0.02510251	-0.01778096\\
0.02520252	-0.01770179\\
0.02530253	-0.01762294\\
0.02540254	-0.01754441\\
0.02550255	-0.0174662\\
0.02560256	-0.01738831\\
0.02570257	-0.01731074\\
0.02580258	-0.01723348\\
0.02590259	-0.01715654\\
0.0260026	-0.01707991\\
0.02610261	-0.01700359\\
0.02620262	-0.01692759\\
0.02630263	-0.01685189\\
0.02640264	-0.0167765\\
0.02650265	-0.01670143\\
0.02660266	-0.01662665\\
0.02670267	-0.01655219\\
0.02680268	-0.01647803\\
0.02690269	-0.01640417\\
0.0270027	-0.01633062\\
0.02710271	-0.01625736\\
0.02720272	-0.01618441\\
0.02730273	-0.01611176\\
0.02740274	-0.0160394\\
0.02750275	-0.01596734\\
0.02760276	-0.01589558\\
0.02770277	-0.01582411\\
0.02780278	-0.01575294\\
0.02790279	-0.01568206\\
0.0280028	-0.01561147\\
0.02810281	-0.01554117\\
0.02820282	-0.01547116\\
0.02830283	-0.01540144\\
0.02840284	-0.015332\\
0.02850285	-0.01526285\\
0.02860286	-0.01519399\\
0.02870287	-0.01512541\\
0.02880288	-0.01505712\\
0.02890289	-0.01498911\\
0.0290029	-0.01492138\\
0.02910291	-0.01485392\\
0.02920292	-0.01478675\\
0.02930293	-0.01471986\\
0.02940294	-0.01465324\\
0.02950295	-0.0145869\\
0.02960296	-0.01452083\\
0.02970297	-0.01445504\\
0.02980298	-0.01438952\\
0.02990299	-0.01432427\\
0.030003	-0.0142593\\
0.03010301	-0.01419459\\
0.03020302	-0.01413015\\
0.03030303	-0.01406598\\
0.03040304	-0.01400208\\
0.03050305	-0.01393844\\
0.03060306	-0.01387507\\
0.03070307	-0.01381196\\
0.03080308	-0.01374912\\
0.03090309	-0.01368653\\
0.0310031	-0.01362421\\
0.03110311	-0.01356215\\
0.03120312	-0.01350035\\
0.03130313	-0.0134388\\
0.03140314	-0.01337752\\
0.03150315	-0.01331648\\
0.03160316	-0.01325571\\
0.03170317	-0.01319519\\
0.03180318	-0.01313492\\
0.03190319	-0.0130749\\
0.0320032	-0.01301514\\
0.03210321	-0.01295562\\
0.03220322	-0.01289636\\
0.03230323	-0.01283734\\
0.03240324	-0.01277858\\
0.03250325	-0.01272005\\
0.03260326	-0.01266178\\
0.03270327	-0.01260375\\
0.03280328	-0.01254596\\
0.03290329	-0.01248842\\
0.0330033	-0.01243112\\
0.03310331	-0.01237406\\
0.03320332	-0.01231724\\
0.03330333	-0.01226066\\
0.03340334	-0.01220432\\
0.03350335	-0.01214821\\
0.03360336	-0.01209235\\
0.03370337	-0.01203672\\
0.03380338	-0.01198132\\
0.03390339	-0.01192616\\
0.0340034	-0.01187123\\
0.03410341	-0.01181653\\
0.03420342	-0.01176207\\
0.03430343	-0.01170783\\
0.03440344	-0.01165383\\
0.03450345	-0.01160005\\
0.03460346	-0.01154651\\
0.03470347	-0.01149318\\
0.03480348	-0.01144009\\
0.03490349	-0.01138722\\
0.0350035	-0.01133458\\
0.03510351	-0.01128215\\
0.03520352	-0.01122996\\
0.03530353	-0.01117798\\
0.03540354	-0.01112622\\
0.03550355	-0.01107469\\
0.03560356	-0.01102337\\
0.03570357	-0.01097228\\
0.03580358	-0.0109214\\
0.03590359	-0.01087074\\
0.0360036	-0.01082029\\
0.03610361	-0.01077006\\
0.03620362	-0.01072004\\
0.03630363	-0.01067024\\
0.03640364	-0.01062065\\
0.03650365	-0.01057127\\
0.03660366	-0.01052211\\
0.03670367	-0.01047315\\
0.03680368	-0.0104244\\
0.03690369	-0.01037586\\
0.0370037	-0.01032753\\
0.03710371	-0.01027941\\
0.03720372	-0.01023149\\
0.03730373	-0.01018378\\
0.03740374	-0.01013628\\
0.03750375	-0.01008897\\
0.03760376	-0.01004187\\
0.03770377	-0.009994977\\
0.03780378	-0.009948282\\
0.03790379	-0.009901788\\
0.0380038	-0.009855493\\
0.03810381	-0.009809398\\
0.03820382	-0.009763502\\
0.03830383	-0.009717803\\
0.03840384	-0.009672301\\
0.03850385	-0.009626996\\
0.03860386	-0.009581886\\
0.03870387	-0.00953697\\
0.03880388	-0.009492249\\
0.03890389	-0.00944772\\
0.0390039	-0.009403385\\
0.03910391	-0.00935924\\
0.03920392	-0.009315287\\
0.03930393	-0.009271524\\
0.03940394	-0.009227951\\
0.03950395	-0.009184566\\
0.03960396	-0.009141369\\
0.03970397	-0.00909836\\
0.03980398	-0.009055537\\
0.03990399	-0.009012899\\
0.040004	-0.008970447\\
0.04010401	-0.008928179\\
0.04020402	-0.008886095\\
0.04030403	-0.008844194\\
0.04040404	-0.008802475\\
0.04050405	-0.008760937\\
0.04060406	-0.00871958\\
0.04070407	-0.008678404\\
0.04080408	-0.008637406\\
0.04090409	-0.008596587\\
0.0410041	-0.008555946\\
0.04110411	-0.008515483\\
0.04120412	-0.008475196\\
0.04130413	-0.008435085\\
0.04140414	-0.008395148\\
0.04150415	-0.008355387\\
0.04160416	-0.008315799\\
0.04170417	-0.008276384\\
0.04180418	-0.008237142\\
0.04190419	-0.008198071\\
0.0420042	-0.008159172\\
0.04210421	-0.008120443\\
0.04220422	-0.008081883\\
0.04230423	-0.008043493\\
0.04240424	-0.008005271\\
0.04250425	-0.007967217\\
0.04260426	-0.007929329\\
0.04270427	-0.007891608\\
0.04280428	-0.007854053\\
0.04290429	-0.007816663\\
0.0430043	-0.007779437\\
0.04310431	-0.007742375\\
0.04320432	-0.007705476\\
0.04330433	-0.00766874\\
0.04340434	-0.007632165\\
0.04350435	-0.007595752\\
0.04360436	-0.007559499\\
0.04370437	-0.007523405\\
0.04380438	-0.007487472\\
0.04390439	-0.007451696\\
0.0440044	-0.007416079\\
0.04410441	-0.007380619\\
0.04420442	-0.007345316\\
0.04430443	-0.007310168\\
0.04440444	-0.007275177\\
0.04450445	-0.00724034\\
0.04460446	-0.007205657\\
0.04470447	-0.007171128\\
0.04480448	-0.007136752\\
0.04490449	-0.007102528\\
0.0450045	-0.007068456\\
0.04510451	-0.007034535\\
0.04520452	-0.007000765\\
0.04530453	-0.006967145\\
0.04540454	-0.006933674\\
0.04550455	-0.006900351\\
0.04560456	-0.006867177\\
0.04570457	-0.006834151\\
0.04580458	-0.006801271\\
0.04590459	-0.006768538\\
0.0460046	-0.00673595\\
0.04610461	-0.006703508\\
0.04620462	-0.00667121\\
0.04630463	-0.006639056\\
0.04640464	-0.006607046\\
0.04650465	-0.006575178\\
0.04660466	-0.006543453\\
0.04670467	-0.00651187\\
0.04680468	-0.006480427\\
0.04690469	-0.006449125\\
0.0470047	-0.006417963\\
0.04710471	-0.006386941\\
0.04720472	-0.006356057\\
0.04730473	-0.006325312\\
0.04740474	-0.006294704\\
0.04750475	-0.006264233\\
0.04760476	-0.006233899\\
0.04770477	-0.006203702\\
0.04780478	-0.006173639\\
0.04790479	-0.006143712\\
0.0480048	-0.006113919\\
0.04810481	-0.006084259\\
0.04820482	-0.006054733\\
0.04830483	-0.00602534\\
0.04840484	-0.005996079\\
0.04850485	-0.00596695\\
0.04860486	-0.005937952\\
0.04870487	-0.005909084\\
0.04880488	-0.005880346\\
0.04890489	-0.005851738\\
0.0490049	-0.005823259\\
0.04910491	-0.005794909\\
0.04920492	-0.005766686\\
0.04930493	-0.005738591\\
0.04940494	-0.005710623\\
0.04950495	-0.005682781\\
0.04960496	-0.005655065\\
0.04970497	-0.005627474\\
0.04980498	-0.005600008\\
0.04990499	-0.005572667\\
0.050005	-0.005545449\\
0.05010501	-0.005518354\\
0.05020502	-0.005491382\\
0.05030503	-0.005464533\\
0.05040504	-0.005437805\\
0.05050505	-0.005411199\\
0.05060506	-0.005384713\\
0.05070507	-0.005358348\\
0.05080508	-0.005332102\\
0.05090509	-0.005305976\\
0.0510051	-0.005279968\\
0.05110511	-0.005254079\\
0.05120512	-0.005228308\\
0.05130513	-0.005202653\\
0.05140514	-0.005177116\\
0.05150515	-0.005151695\\
0.05160516	-0.00512639\\
0.05170517	-0.0051012\\
0.05180518	-0.005076125\\
0.05190519	-0.005051165\\
0.0520052	-0.005026318\\
0.05210521	-0.005001585\\
0.05220522	-0.004976965\\
0.05230523	-0.004952458\\
0.05240524	-0.004928063\\
0.05250525	-0.004903779\\
0.05260526	-0.004879606\\
0.05270527	-0.004855544\\
0.05280528	-0.004831592\\
0.05290529	-0.00480775\\
0.0530053	-0.004784017\\
0.05310531	-0.004760393\\
0.05320532	-0.004736878\\
0.05330533	-0.00471347\\
0.05340534	-0.00469017\\
0.05350535	-0.004666977\\
0.05360536	-0.00464389\\
0.05370537	-0.00462091\\
0.05380538	-0.004598035\\
0.05390539	-0.004575266\\
0.0540054	-0.004552601\\
0.05410541	-0.00453004\\
0.05420542	-0.004507584\\
0.05430543	-0.004485231\\
0.05440544	-0.004462981\\
0.05450545	-0.004440834\\
0.05460546	-0.004418788\\
0.05470547	-0.004396845\\
0.05480548	-0.004375003\\
0.05490549	-0.004353262\\
0.0550055	-0.004331621\\
0.05510551	-0.00431008\\
0.05520552	-0.004288639\\
0.05530553	-0.004267297\\
0.05540554	-0.004246054\\
0.05550555	-0.00422491\\
0.05560556	-0.004203863\\
0.05570557	-0.004182914\\
0.05580558	-0.004162061\\
0.05590559	-0.004141306\\
0.0560056	-0.004120647\\
0.05610561	-0.004100084\\
0.05620562	-0.004079616\\
0.05630563	-0.004059243\\
0.05640564	-0.004038965\\
0.05650565	-0.004018782\\
0.05660566	-0.003998692\\
0.05670567	-0.003978695\\
0.05680568	-0.003958792\\
0.05690569	-0.003938981\\
0.0570057	-0.003919263\\
0.05710571	-0.003899637\\
0.05720572	-0.003880102\\
0.05730573	-0.003860658\\
0.05740574	-0.003841305\\
0.05750575	-0.003822042\\
0.05760576	-0.003802869\\
0.05770577	-0.003783786\\
0.05780578	-0.003764792\\
0.05790579	-0.003745886\\
0.0580058	-0.00372707\\
0.05810581	-0.003708341\\
0.05820582	-0.0036897\\
0.05830583	-0.003671146\\
0.05840584	-0.003652679\\
0.05850585	-0.003634298\\
0.05860586	-0.003616004\\
0.05870587	-0.003597796\\
0.05880588	-0.003579672\\
0.05890589	-0.003561635\\
0.0590059	-0.003543681\\
0.05910591	-0.003525812\\
0.05920592	-0.003508027\\
0.05930593	-0.003490326\\
0.05940594	-0.003472708\\
0.05950595	-0.003455173\\
0.05960596	-0.00343772\\
0.05970597	-0.003420349\\
0.05980598	-0.003403061\\
0.05990599	-0.003385853\\
0.060006	-0.003368727\\
0.06010601	-0.003351682\\
0.06020602	-0.003334717\\
0.06030603	-0.003317832\\
0.06040604	-0.003301027\\
0.06050605	-0.003284301\\
0.06060606	-0.003267655\\
0.06070607	-0.003251087\\
0.06080608	-0.003234597\\
0.06090609	-0.003218185\\
0.0610061	-0.003201851\\
0.06110611	-0.003185595\\
0.06120612	-0.003169415\\
0.06130613	-0.003153312\\
0.06140614	-0.003137285\\
0.06150615	-0.003121335\\
0.06160616	-0.00310546\\
0.06170617	-0.00308966\\
0.06180618	-0.003073935\\
0.06190619	-0.003058285\\
0.0620062	-0.00304271\\
0.06210621	-0.003027208\\
0.06220622	-0.00301178\\
0.06230623	-0.002996426\\
0.06240624	-0.002981144\\
0.06250625	-0.002965935\\
0.06260626	-0.002950799\\
0.06270627	-0.002935735\\
0.06280628	-0.002920743\\
0.06290629	-0.002905822\\
0.0630063	-0.002890972\\
0.06310631	-0.002876193\\
0.06320632	-0.002861485\\
0.06330633	-0.002846847\\
0.06340634	-0.002832279\\
0.06350635	-0.00281778\\
0.06360636	-0.002803351\\
0.06370637	-0.002788991\\
0.06380638	-0.0027747\\
0.06390639	-0.002760477\\
0.0640064	-0.002746322\\
0.06410641	-0.002732234\\
0.06420642	-0.002718215\\
0.06430643	-0.002704262\\
0.06440644	-0.002690377\\
0.06450645	-0.002676558\\
0.06460646	-0.002662805\\
0.06470647	-0.002649119\\
0.06480648	-0.002635498\\
0.06490649	-0.002621943\\
0.0650065	-0.002608453\\
0.06510651	-0.002595027\\
0.06520652	-0.002581667\\
0.06530653	-0.00256837\\
0.06540654	-0.002555138\\
0.06550655	-0.002541969\\
0.06560656	-0.002528864\\
0.06570657	-0.002515822\\
0.06580658	-0.002502842\\
0.06590659	-0.002489926\\
0.0660066	-0.002477072\\
0.06610661	-0.002464279\\
0.06620662	-0.002451549\\
0.06630663	-0.00243888\\
0.06640664	-0.002426272\\
0.06650665	-0.002413726\\
0.06660666	-0.002401239\\
0.06670667	-0.002388814\\
0.06680668	-0.002376448\\
0.06690669	-0.002364143\\
0.0670067	-0.002351897\\
0.06710671	-0.00233971\\
0.06720672	-0.002327582\\
0.06730673	-0.002315514\\
0.06740674	-0.002303503\\
0.06750675	-0.002291551\\
0.06760676	-0.002279657\\
0.06770677	-0.002267821\\
0.06780678	-0.002256043\\
0.06790679	-0.002244321\\
0.0680068	-0.002232657\\
0.06810681	-0.002221049\\
0.06820682	-0.002209498\\
0.06830683	-0.002198003\\
0.06840684	-0.002186564\\
0.06850685	-0.002175181\\
0.06860686	-0.002163853\\
0.06870687	-0.00215258\\
0.06880688	-0.002141363\\
0.06890689	-0.0021302\\
0.0690069	-0.002119092\\
0.06910691	-0.002108038\\
0.06920692	-0.002097038\\
0.06930693	-0.002086091\\
0.06940694	-0.002075198\\
0.06950695	-0.002064359\\
0.06960696	-0.002053572\\
0.06970697	-0.002042839\\
0.06980698	-0.002032157\\
0.06990699	-0.002021528\\
0.070007	-0.002010952\\
0.07010701	-0.002000427\\
0.07020702	-0.001989953\\
0.07030703	-0.001979531\\
0.07040704	-0.00196916\\
0.07050705	-0.001958841\\
0.07060706	-0.001948571\\
0.07070707	-0.001938352\\
0.07080708	-0.001928184\\
0.07090709	-0.001918065\\
0.0710071	-0.001907996\\
0.07110711	-0.001897977\\
0.07120712	-0.001888007\\
0.07130713	-0.001878086\\
0.07140714	-0.001868214\\
0.07150715	-0.001858391\\
0.07160716	-0.001848616\\
0.07170717	-0.001838889\\
0.07180718	-0.00182921\\
0.07190719	-0.001819579\\
0.0720072	-0.001809996\\
0.07210721	-0.00180046\\
0.07220722	-0.001790971\\
0.07230723	-0.001781528\\
0.07240724	-0.001772133\\
0.07250725	-0.001762784\\
0.07260726	-0.001753481\\
0.07270727	-0.001744224\\
0.07280728	-0.001735014\\
0.07290729	-0.001725848\\
0.0730073	-0.001716729\\
0.07310731	-0.001707654\\
0.07320732	-0.001698624\\
0.07330733	-0.001689639\\
0.07340734	-0.001680699\\
0.07350735	-0.001671803\\
0.07360736	-0.001662952\\
0.07370737	-0.001654144\\
0.07380738	-0.00164538\\
0.07390739	-0.00163666\\
0.0740074	-0.001627983\\
0.07410741	-0.001619349\\
0.07420742	-0.001610758\\
0.07430743	-0.00160221\\
0.07440744	-0.001593704\\
0.07450745	-0.001585241\\
0.07460746	-0.00157682\\
0.07470747	-0.001568441\\
0.07480748	-0.001560104\\
0.07490749	-0.001551809\\
0.0750075	-0.001543554\\
0.07510751	-0.001535342\\
0.07520752	-0.00152717\\
0.07530753	-0.001519039\\
0.07540754	-0.001510948\\
0.07550755	-0.001502898\\
0.07560756	-0.001494889\\
0.07570757	-0.001486919\\
0.07580758	-0.001478989\\
0.07590759	-0.001471099\\
0.0760076	-0.001463249\\
0.07610761	-0.001455438\\
0.07620762	-0.001447666\\
0.07630763	-0.001439933\\
0.07640764	-0.001432239\\
0.07650765	-0.001424583\\
0.07660766	-0.001416966\\
0.07670767	-0.001409388\\
0.07680768	-0.001401847\\
0.07690769	-0.001394344\\
0.0770077	-0.001386879\\
0.07710771	-0.001379451\\
0.07720772	-0.001372061\\
0.07730773	-0.001364708\\
0.07740774	-0.001357392\\
0.07750775	-0.001350113\\
0.07760776	-0.001342871\\
0.07770777	-0.001335665\\
0.07780778	-0.001328496\\
0.07790779	-0.001321362\\
0.0780078	-0.001314265\\
0.07810781	-0.001307204\\
0.07820782	-0.001300178\\
0.07830783	-0.001293187\\
0.07840784	-0.001286233\\
0.07850785	-0.001279313\\
0.07860786	-0.001272428\\
0.07870787	-0.001265578\\
0.07880788	-0.001258763\\
0.07890789	-0.001251982\\
0.0790079	-0.001245235\\
0.07910791	-0.001238523\\
0.07920792	-0.001231845\\
0.07930793	-0.001225201\\
0.07940794	-0.00121859\\
0.07950795	-0.001212013\\
0.07960796	-0.001205469\\
0.07970797	-0.001198959\\
0.07980798	-0.001192481\\
0.07990799	-0.001186037\\
0.080008	-0.001179625\\
0.08010801	-0.001173246\\
0.08020802	-0.001166899\\
0.08030803	-0.001160585\\
0.08040804	-0.001154303\\
0.08050805	-0.001148053\\
0.08060806	-0.001141834\\
0.08070807	-0.001135648\\
0.08080808	-0.001129493\\
0.08090809	-0.00112336899999999\\
0.0810081	-0.00111727599999999\\
0.08110811	-0.00111121499999999\\
0.08120812	-0.00110518499999999\\
0.08130813	-0.00109918499999999\\
0.08140814	-0.00109321599999999\\
0.08150815	-0.00108727799999999\\
0.08160816	-0.00108136999999999\\
0.08170817	-0.00107549199999999\\
0.08180818	-0.00106964399999999\\
0.08190819	-0.00106382599999999\\
0.0820082	-0.00105803799999999\\
0.08210821	-0.00105227999999999\\
0.08220822	-0.00104655099999999\\
0.08230823	-0.00104085099999999\\
0.08240824	-0.00103518099999999\\
0.08250825	-0.00102953999999999\\
0.08260826	-0.00102392799999999\\
0.08270827	-0.00101834399999999\\
0.08280828	-0.00101278899999999\\
0.08290829	-0.00100726299999999\\
0.0830083	-0.00100176499999999\\
0.08310831	-0.000996295899999993\\
0.08320832	-0.000990854499999993\\
0.08330833	-0.000985441099999993\\
0.08340834	-0.000980055499999993\\
0.08350835	-0.000974697699999993\\
0.08360836	-0.000969367399999993\\
0.08370837	-0.000964064699999993\\
0.08380838	-0.000958789199999993\\
0.08390839	-0.000953540999999993\\
0.0840084	-0.000948319799999993\\
0.08410841	-0.000943125599999993\\
0.08420842	-0.000937958199999993\\
0.08430843	-0.000932817399999993\\
0.08440844	-0.000927703299999992\\
0.08450845	-0.000922615499999992\\
0.08460846	-0.000917553999999992\\
0.08470847	-0.000912518799999992\\
0.08480848	-0.000907509499999992\\
0.08490849	-0.000902526199999992\\
0.0850085	-0.000897568699999992\\
0.08510851	-0.000892636799999992\\
0.08520852	-0.000887730499999992\\
0.08530853	-0.000882849599999992\\
0.08540854	-0.000877993999999991\\
0.08550855	-0.000873163599999991\\
0.08560856	-0.000868358299999991\\
0.08570857	-0.000863577799999991\\
0.08580858	-0.000858822199999991\\
0.08590859	-0.000854091299999991\\
0.0860086	-0.000849384999999991\\
0.08610861	-0.000844703099999991\\
0.08620862	-0.000840045599999991\\
0.08630863	-0.00083541229999999\\
0.08640864	-0.00083080299999999\\
0.08650865	-0.00082621779999999\\
0.08660866	-0.00082165649999999\\
0.08670867	-0.00081711879999999\\
0.08680868	-0.00081260489999999\\
0.08690869	-0.00080811439999999\\
0.0870087	-0.00080364739999999\\
0.08710871	-0.000799203699999989\\
0.08720872	-0.000794783099999989\\
0.08730873	-0.000790385599999989\\
0.08740874	-0.000786011099999989\\
0.08750875	-0.000781659399999989\\
0.08760876	-0.000777330399999989\\
0.08770877	-0.000773024099999989\\
0.08780878	-0.000768740299999989\\
0.08790879	-0.000764478899999988\\
0.0880088	-0.000760239699999988\\
0.08810881	-0.000756022799999988\\
0.08820882	-0.000751827899999988\\
0.08830883	-0.000747654999999988\\
0.08840884	-0.000743503999999988\\
0.08850885	-0.000739374699999988\\
0.08860886	-0.000735267099999987\\
0.08870887	-0.000731180999999987\\
0.08880888	-0.000727116299999987\\
0.08890889	-0.000723072999999987\\
0.0890089	-0.000719050899999987\\
0.08910891	-0.000715049899999987\\
0.08920892	-0.000711069999999987\\
0.08930893	-0.000707110999999986\\
0.08940894	-0.000703172699999986\\
0.08950895	-0.000699255199999986\\
0.08960896	-0.000695358299999986\\
0.08970897	-0.000691481899999986\\
0.08980898	-0.000687625899999986\\
0.08990899	-0.000683790199999985\\
0.090009	-0.000679974799999985\\
0.09010901	-0.000676179399999985\\
0.09020902	-0.000672403999999985\\
0.09030903	-0.000668648599999985\\
0.09040904	-0.000664912999999985\\
0.09050905	-0.000661196999999984\\
0.09060906	-0.000657500699999984\\
0.09070907	-0.000653823999999984\\
0.09080908	-0.000650166599999984\\
0.09090909	-0.000646528599999984\\
0.0910091	-0.000642909799999983\\
0.09110911	-0.000639310099999983\\
0.09120912	-0.000635729499999983\\
0.09130913	-0.000632167799999983\\
0.09140914	-0.000628624999999982\\
0.09150915	-0.000625100999999982\\
0.09160916	-0.000621595599999982\\
0.09170917	-0.000618108799999982\\
0.09180918	-0.000614640499999982\\
0.09190919	-0.000611190499999981\\
0.0920092	-0.000607758899999981\\
0.09210921	-0.000604345499999981\\
0.09220922	-0.000600950199999981\\
0.09230923	-0.000597572899999981\\
0.09240924	-0.00059421359999998\\
0.09250925	-0.00059087219999998\\
0.09260926	-0.00058754849999998\\
0.09270927	-0.00058424239999998\\
0.09280928	-0.000580953999999979\\
0.09290929	-0.000577682999999979\\
0.0930093	-0.000574429499999979\\
0.09310931	-0.000571193299999979\\
0.09320932	-0.000567974299999978\\
0.09330933	-0.000564772499999978\\
0.09340934	-0.000561587799999978\\
0.09350935	-0.000558419999999978\\
0.09360936	-0.000555269099999977\\
0.09370937	-0.000552135099999977\\
0.09380938	-0.000549017799999977\\
0.09390939	-0.000545917099999976\\
0.0940094	-0.000542832999999976\\
0.09410941	-0.000539765299999976\\
0.09420942	-0.000536714099999976\\
0.09430943	-0.000533679199999975\\
0.09440944	-0.000530660499999975\\
0.09450945	-0.000527657999999975\\
0.09460946	-0.000524671499999975\\
0.09470947	-0.000521700999999974\\
0.09480948	-0.000518746499999974\\
0.09490949	-0.000515807699999974\\
0.0950095	-0.000512884799999973\\
0.09510951	-0.000509977499999973\\
0.09520952	-0.000507085699999973\\
0.09530953	-0.000504209499999972\\
0.09540954	-0.000501348799999972\\
0.09550955	-0.000498503399999972\\
0.09560956	-0.000495673299999971\\
0.09570957	-0.000492858399999971\\
0.09580958	-0.000490058599999971\\
0.09590959	-0.00048727379999997\\
0.0960096	-0.00048450409999997\\
0.09610961	-0.00048174919999997\\
0.09620962	-0.000479009199999969\\
0.09630963	-0.000476283999999969\\
0.09640964	-0.000473573399999968\\
0.09650965	-0.000470877399999968\\
0.09660966	-0.000468195899999968\\
0.09670967	-0.000465528899999967\\
0.09680968	-0.000462876299999967\\
0.09690969	-0.000460237999999967\\
0.0970097	-0.000457613899999966\\
0.09710971	-0.000455003999999966\\
0.09720972	-0.000452408199999965\\
0.09730973	-0.000449826399999965\\
0.09740974	-0.000447258499999964\\
0.09750975	-0.000444704599999964\\
0.09760976	-0.000442164399999964\\
0.09770977	-0.000439637999999963\\
0.09780978	-0.000437125299999963\\
0.09790979	-0.000434626099999962\\
0.0980098	-0.000432140499999962\\
0.09810981	-0.000429668399999961\\
0.09820982	-0.000427209599999961\\
0.09830983	-0.000424764199999961\\
0.09840984	-0.00042233209999996\\
0.09850985	-0.00041991309999996\\
0.09860986	-0.000417507199999959\\
0.09870987	-0.000415114499999959\\
0.09880988	-0.000412734699999958\\
0.09890989	-0.000410367799999958\\
0.0990099	-0.000408013799999957\\
0.09910991	-0.000405672599999957\\
0.09920992	-0.000403344099999956\\
0.09930993	-0.000401028299999956\\
0.09940994	-0.000398724999999955\\
0.09950995	-0.000396434399999955\\
0.09960996	-0.000394156099999954\\
0.09970997	-0.000391890299999953\\
0.09980998	-0.000389636899999953\\
0.09990999	-0.000387395699999952\\
0.10001	-0.000385166699999952\\
0.10011	-0.000382949899999951\\
0.10021	-0.000380745199999951\\
0.10031	-0.00037855249999995\\
0.10041	-0.00037637179999995\\
0.1005101	-0.000374202899999949\\
0.1006101	-0.000372045999999948\\
0.1007101	-0.000369900799999948\\
0.1008101	-0.000367767299999947\\
0.1009101	-0.000365645499999947\\
0.1010101	-0.000363535399999946\\
0.1011101	-0.000361436699999945\\
0.1012101	-0.000359349599999945\\
0.1013101	-0.000357273799999944\\
0.1014101	-0.000355209499999943\\
0.1015102	-0.000353156399999943\\
0.1016102	-0.000351114599999942\\
0.1017102	-0.000349083999999941\\
0.1018102	-0.000347064599999941\\
0.1019102	-0.00034505619999994\\
0.1020102	-0.000343058799999939\\
0.1021102	-0.000341072399999938\\
0.1022102	-0.000339096899999938\\
0.1023102	-0.000337132299999937\\
0.1024102	-0.000335178499999936\\
0.1025103	-0.000333235399999935\\
0.1026103	-0.000331302899999935\\
0.1027103	-0.000329381199999934\\
0.1028103	-0.000327469899999933\\
0.1029103	-0.000325569199999932\\
0.1030103	-0.000323678999999932\\
0.1031103	-0.000321799199999931\\
0.1032103	-0.00031992969999993\\
0.1033103	-0.000318070599999929\\
0.1034103	-0.000316221699999928\\
0.1035104	-0.000314382999999928\\
0.1036104	-0.000312554399999927\\
0.1037104	-0.000310735999999926\\
0.1038104	-0.000308927499999925\\
0.1039104	-0.000307129099999924\\
0.1040104	-0.000305340599999923\\
0.1041104	-0.000303561999999922\\
0.1042104	-0.000301793199999921\\
0.1043104	-0.00030003419999992\\
0.1044104	-0.000298284899999919\\
0.1045105	-0.000296545299999919\\
0.1046105	-0.000294815299999918\\
0.1047105	-0.000293094899999917\\
0.1048105	-0.000291384099999916\\
0.1049105	-0.000289682699999915\\
0.1050105	-0.000287990799999914\\
0.1051105	-0.000286308199999913\\
0.1052105	-0.000284634999999912\\
0.1053105	-0.000282970999999911\\
0.1054105	-0.00028131629999991\\
0.1055106	-0.000279670799999908\\
0.1056106	-0.000278034399999907\\
0.1057106	-0.000276407099999906\\
0.1058106	-0.000274788799999905\\
0.1059106	-0.000273179599999904\\
0.1060106	-0.000271579299999903\\
0.1061106	-0.000269987899999902\\
0.1062106	-0.000268405299999901\\
0.1063106	-0.0002668315999999\\
0.1064106	-0.000265266599999898\\
0.1065107	-0.000263710299999897\\
0.1066107	-0.000262162799999896\\
0.1067107	-0.000260623799999895\\
0.1068107	-0.000259093399999894\\
0.1069107	-0.000257571599999892\\
0.1070107	-0.000256058199999891\\
0.1071107	-0.00025455329999989\\
0.1072107	-0.000253056799999889\\
0.1073107	-0.000251568699999887\\
0.1074107	-0.000250088899999886\\
0.1075108	-0.000248617299999885\\
0.1076108	-0.000247153999999883\\
0.1077108	-0.000245698799999882\\
0.1078108	-0.000244251799999881\\
0.1079108	-0.000242812899999879\\
0.1080108	-0.000241382099999878\\
0.1081108	-0.000239959199999876\\
0.1082108	-0.000238544299999875\\
0.1083108	-0.000237137399999873\\
0.1084108	-0.000235738299999872\\
0.1085109	-0.00023434709999987\\
0.1086109	-0.000232963699999869\\
0.1087109	-0.000231588099999867\\
0.1088109	-0.000230220199999866\\
0.1089109	-0.000228859899999864\\
0.1090109	-0.000227507299999863\\
0.1091109	-0.000226162299999861\\
0.1092109	-0.00022482489999986\\
0.1093109	-0.000223494899999858\\
0.1094109	-0.000222172499999856\\
0.109511	-0.000220857499999855\\
0.109611	-0.000219549899999853\\
0.109711	-0.000218249599999851\\
0.109811	-0.000216956699999849\\
0.109911	-0.000215670999999848\\
0.110011	-0.000214392599999846\\
0.110111	-0.000213121399999844\\
0.110211	-0.000211857399999842\\
0.110311	-0.000210600499999841\\
0.110411	-0.000209350699999839\\
0.1105111	-0.000208107999999837\\
0.1106111	-0.000206872199999835\\
0.1107111	-0.000205643499999833\\
0.1108111	-0.000204421699999831\\
0.1109111	-0.000203206699999829\\
0.1110111	-0.000201998699999827\\
0.1111111	-0.000200797499999825\\
0.1112111	-0.000199603099999823\\
0.1113111	-0.000198415399999821\\
0.1114111	-0.000197234499999819\\
0.1115112	-0.000196060199999817\\
0.1116112	-0.000194892599999815\\
0.1117112	-0.000193731599999813\\
0.1118112	-0.00019257719999981\\
0.1119112	-0.000191429399999808\\
0.1120112	-0.000190287999999806\\
0.1121112	-0.000189153099999804\\
0.1122112	-0.000188024699999802\\
0.1123112	-0.000186902699999799\\
0.1124112	-0.000185786999999797\\
0.1125113	-0.000184677699999795\\
0.1126113	-0.000183574699999792\\
0.1127113	-0.00018247789999979\\
0.1128113	-0.000181387399999787\\
0.1129113	-0.000180303099999785\\
0.1130113	-0.000179224899999782\\
0.1131113	-0.00017815289999978\\
0.1132113	-0.000177086999999777\\
0.1133113	-0.000176027199999775\\
0.1134113	-0.000174973399999772\\
0.1135114	-0.00017392559999977\\
0.1136114	-0.000172883799999767\\
0.1137114	-0.000171847899999764\\
0.1138114	-0.000170817899999762\\
0.1139114	-0.000169793799999759\\
0.1140114	-0.000168775599999756\\
0.1141114	-0.000167763099999753\\
0.1142114	-0.00016675649999975\\
0.1143114	-0.000165755599999747\\
0.1144114	-0.000164760399999745\\
0.1145115	-0.000163770899999742\\
0.1146115	-0.000162786999999739\\
0.1147115	-0.000161808799999736\\
0.1148115	-0.000160836199999733\\
0.1149115	-0.000159869199999729\\
0.1150115	-0.000158907699999726\\
0.1151115	-0.000157951599999723\\
0.1152115	-0.00015700109999972\\
0.1153115	-0.000156055999999717\\
0.1154115	-0.000155116399999714\\
0.1155116	-0.00015418209999971\\
0.1156116	-0.000153253199999707\\
0.1157116	-0.000152329599999704\\
0.1158116	-0.0001514112999997\\
0.1159116	-0.000150498299999697\\
0.1160116	-0.000149590499999693\\
0.1161116	-0.00014868799999969\\
0.1162116	-0.000147790599999686\\
0.1163116	-0.000146898399999682\\
0.1164116	-0.000146011299999679\\
0.1165117	-0.000145129299999675\\
0.1166117	-0.000144252399999671\\
0.1167117	-0.000143380599999668\\
0.1168117	-0.000142513699999664\\
0.1169117	-0.00014165189999966\\
0.1170117	-0.000140794999999656\\
0.1171117	-0.000139943099999652\\
0.1172117	-0.000139095999999648\\
0.1173117	-0.000138253899999644\\
0.1174117	-0.00013741659999964\\
0.1175118	-0.000136584199999636\\
0.1176118	-0.000135756499999632\\
0.1177118	-0.000134933599999627\\
0.1178118	-0.000134115499999623\\
0.1179118	-0.000133302099999619\\
0.1180118	-0.000132493399999614\\
0.1181118	-0.00013168939999961\\
0.1182118	-0.000130890099999606\\
0.1183118	-0.000130095299999601\\
0.1184118	-0.000129305199999597\\
0.1185119	-0.000128519599999592\\
0.1186119	-0.000127738599999587\\
0.1187119	-0.000126962099999582\\
0.1188119	-0.000126190099999578\\
0.1189119	-0.000125422599999573\\
0.1190119	-0.000124659499999568\\
0.1191119	-0.000123900799999563\\
0.1192119	-0.000123146599999558\\
0.1193119	-0.000122396699999553\\
0.1194119	-0.000121651199999548\\
0.119512	-0.000120909999999543\\
0.119612	-0.000120173099999537\\
0.119712	-0.000119440499999532\\
0.119812	-0.000118712199999527\\
0.119912	-0.000117988099999521\\
0.120012	-0.000117268199999516\\
0.120112	-0.00011655249999951\\
0.120212	-0.000115840899999505\\
0.120312	-0.000115133499999499\\
0.120412	-0.000114430199999493\\
0.1205121	-0.000113730999999488\\
0.1206121	-0.000113035899999482\\
0.1207121	-0.000112344899999476\\
0.1208121	-0.00011165779999947\\
0.1209121	-0.000110974799999464\\
0.1210121	-0.000110295799999458\\
0.1211121	-0.000109620699999452\\
0.1212121	-0.000108949599999445\\
0.1213121	-0.000108282299999439\\
0.1214121	-0.000107618999999433\\
0.1215122	-0.000106959599999426\\
0.1216122	-0.00010630399999942\\
0.1217122	-0.000105652199999413\\
0.1218122	-0.000105004299999406\\
0.1219122	-0.000104360099999399\\
0.1220122	-0.000103719699999393\\
0.1221122	-0.000103083099999386\\
0.1222122	-0.000102450199999379\\
0.1223122	-0.000101820999999372\\
0.1224122	-0.000101195499999364\\
0.1225123	-0.000100573699999357\\
0.1226123	-9.99554499993499e-05\\
0.1227123	-9.93408799993425e-05\\
0.1228123	-9.87299099993351e-05\\
0.1229123	-9.81225299993275e-05\\
0.1230123	-9.75187199993198e-05\\
0.1231123	-9.69184499993121e-05\\
0.1232123	-9.63217099993043e-05\\
0.1233123	-9.57284799992964e-05\\
0.1234123	-9.51387299992884e-05\\
0.1235124	-9.45524499992803e-05\\
0.1236124	-9.39696199992721e-05\\
0.1237124	-9.33902199992638e-05\\
0.1238124	-9.28142299992554e-05\\
0.1239124	-9.2241629999247e-05\\
0.1240124	-9.16723999992384e-05\\
0.1241124	-9.11065299992298e-05\\
0.1242124	-9.05439899992211e-05\\
0.1243124	-8.99847699992122e-05\\
0.1244124	-8.94288499992033e-05\\
0.1245125	-8.88761999991942e-05\\
0.1246125	-8.83268199991851e-05\\
0.1247125	-8.77806799991758e-05\\
0.1248125	-8.72377699991665e-05\\
0.1249125	-8.6698059999157e-05\\
0.1250125	-8.61615399991475e-05\\
0.1251125	-8.56281899991378e-05\\
0.1252125	-8.5097989999128e-05\\
0.1253125	-8.45709399991181e-05\\
0.1254125	-8.40469899991082e-05\\
0.1255126	-8.3526149999098e-05\\
0.1256126	-8.30083899990878e-05\\
0.1257126	-8.24936999990775e-05\\
0.1258126	-8.19820599990671e-05\\
0.1259126	-8.14734499990565e-05\\
0.1260126	-8.09678499990458e-05\\
0.1261126	-8.0465249999035e-05\\
0.1262126	-7.99656299990241e-05\\
0.1263126	-7.9468969999013e-05\\
0.1264126	-7.89752699990019e-05\\
0.1265127	-7.84844899989905e-05\\
0.1266127	-7.79966299989791e-05\\
0.1267127	-7.75116599989676e-05\\
0.1268127	-7.70295799989559e-05\\
0.1269127	-7.65503599989441e-05\\
0.1270127	-7.60739899989322e-05\\
0.1271127	-7.56004499989201e-05\\
0.1272127	-7.51297299989079e-05\\
0.1273127	-7.46618199988956e-05\\
0.1274127	-7.41966799988831e-05\\
0.1275128	-7.37343199988705e-05\\
0.1276128	-7.32747099988577e-05\\
0.1277128	-7.28178399988448e-05\\
0.1278128	-7.23636899988318e-05\\
0.1279128	-7.19122499988186e-05\\
0.1280128	-7.14634999988052e-05\\
0.1281128	-7.10174299987918e-05\\
0.1282128	-7.05740199987781e-05\\
0.1283128	-7.01332599987643e-05\\
0.1284128	-6.96951299987504e-05\\
0.1285129	-6.92596099987363e-05\\
0.1286129	-6.8826699998722e-05\\
0.1287129	-6.83963699987076e-05\\
0.1288129	-6.79686199986931e-05\\
0.1289129	-6.75434299986783e-05\\
0.1290129	-6.71207699986634e-05\\
0.1291129	-6.67006499986484e-05\\
0.1292129	-6.62830399986332e-05\\
0.1293129	-6.58679299986177e-05\\
0.1294129	-6.54553099986022e-05\\
0.129513	-6.50451599985864e-05\\
0.129613	-6.46374599985705e-05\\
0.129713	-6.42322099985544e-05\\
0.129813	-6.38293899985382e-05\\
0.129913	-6.34289899985217e-05\\
0.130013	-6.30309799985051e-05\\
0.130113	-6.26353699984883e-05\\
0.130213	-6.22421299984713e-05\\
0.130313	-6.18512499984541e-05\\
0.130413	-6.14627199984367e-05\\
0.1305131	-6.10765199984191e-05\\
0.1306131	-6.06926499984013e-05\\
0.1307131	-6.03110799983833e-05\\
0.1308131	-5.99318099983652e-05\\
0.1309131	-5.95548099983468e-05\\
0.1310131	-5.91800899983282e-05\\
0.1311131	-5.88076199983095e-05\\
0.1312131	-5.84373899982905e-05\\
0.1313131	-5.80693899982713e-05\\
0.1314131	-5.77036099982519e-05\\
0.1315132	-5.73400299982322e-05\\
0.1316132	-5.69786499982124e-05\\
0.1317132	-5.66194399981923e-05\\
0.1318132	-5.62623999981721e-05\\
0.1319132	-5.59075099981516e-05\\
0.1320132	-5.55547599981308e-05\\
0.1321132	-5.52041499981099e-05\\
0.1322132	-5.48556499980887e-05\\
0.1323132	-5.45092499980672e-05\\
0.1324132	-5.41649499980456e-05\\
0.1325133	-5.38227199980236e-05\\
0.1326133	-5.34825699980015e-05\\
0.1327133	-5.31444699979791e-05\\
0.1328133	-5.28084099979565e-05\\
0.1329133	-5.24743899979336e-05\\
0.1330133	-5.21423899979104e-05\\
0.1331133	-5.18123999978871e-05\\
0.1332133	-5.14844099978634e-05\\
0.1333133	-5.11583999978395e-05\\
0.1334133	-5.08343699978153e-05\\
0.1335134	-5.05123099977908e-05\\
0.1336134	-5.01921899977661e-05\\
0.1337134	-4.98740099977411e-05\\
0.1338134	-4.95577699977159e-05\\
0.1339134	-4.92434399976903e-05\\
0.1340134	-4.89310199976645e-05\\
0.1341134	-4.86204999976384e-05\\
0.1342134	-4.8311859997612e-05\\
0.1343134	-4.80050999975853e-05\\
0.1344134	-4.77001999975583e-05\\
0.1345135	-4.7397149997531e-05\\
0.1346135	-4.70959499975034e-05\\
0.1347135	-4.67965799974756e-05\\
0.1348135	-4.64990199974474e-05\\
0.1349135	-4.62032799974189e-05\\
0.1350135	-4.59093399973901e-05\\
0.1351135	-4.56171899973609e-05\\
0.1352135	-4.53268099973315e-05\\
0.1353135	-4.50382099973017e-05\\
0.1354135	-4.47513599972716e-05\\
0.1355136	-4.44662599972411e-05\\
0.1356136	-4.41828999972104e-05\\
0.1357136	-4.39012699971793e-05\\
0.1358136	-4.36213599971478e-05\\
0.1359136	-4.3343149997116e-05\\
0.1360136	-4.30666399970839e-05\\
0.1361136	-4.27918199970514e-05\\
0.1362136	-4.25186799970185e-05\\
0.1363136	-4.22472099969853e-05\\
0.1364136	-4.19773899969517e-05\\
0.1365137	-4.17092299969178e-05\\
0.1366137	-4.14426999968834e-05\\
0.1367137	-4.11778099968488e-05\\
0.1368137	-4.09145299968137e-05\\
0.1369137	-4.06528699967782e-05\\
0.1370137	-4.03928099967424e-05\\
0.1371137	-4.01343399967061e-05\\
0.1372137	-3.98774499966695e-05\\
0.1373137	-3.96221399966324e-05\\
0.1374137	-3.9368389996595e-05\\
0.1375138	-3.91161999965571e-05\\
0.1376138	-3.88655599965188e-05\\
0.1377138	-3.86164499964801e-05\\
0.1378138	-3.8368879996441e-05\\
0.1379138	-3.81228199964015e-05\\
0.1380138	-3.78782699963615e-05\\
0.1381138	-3.76352299963211e-05\\
0.1382138	-3.73936799962802e-05\\
0.1383138	-3.71536099962389e-05\\
0.1384138	-3.69150199961971e-05\\
0.1385139	-3.66778899961549e-05\\
0.1386139	-3.64422299961122e-05\\
0.1387139	-3.6208019996069e-05\\
0.1388139	-3.59752399960254e-05\\
0.1389139	-3.57439099959813e-05\\
0.1390139	-3.55139899959368e-05\\
0.1391139	-3.52854899958917e-05\\
0.1392139	-3.50584099958461e-05\\
0.1393139	-3.48327199958001e-05\\
0.1394139	-3.46084199957535e-05\\
0.139514	-3.43855099957064e-05\\
0.139614	-3.41639699956588e-05\\
0.139714	-3.39437999956107e-05\\
0.139814	-3.37249899955621e-05\\
0.139914	-3.3507539995513e-05\\
0.140014	-3.32914199954633e-05\\
0.140114	-3.3076649995413e-05\\
0.140214	-3.28631999953623e-05\\
0.140314	-3.26510699953109e-05\\
0.140414	-3.2440249995259e-05\\
0.1405141	-3.22307399952065e-05\\
0.1406141	-3.20225199951535e-05\\
0.1407141	-3.18155999950999e-05\\
0.1408141	-3.16099599950457e-05\\
0.1409141	-3.14055899949909e-05\\
0.1410141	-3.12024899949355e-05\\
0.1411141	-3.10006399948795e-05\\
0.1412141	-3.08000599948229e-05\\
0.1413141	-3.06007099947657e-05\\
0.1414141	-3.04025999947079e-05\\
0.1415142	-3.02057299946493e-05\\
0.1416142	-3.00100699945902e-05\\
0.1417142	-2.98156299945305e-05\\
0.1418142	-2.96223999944701e-05\\
0.1419142	-2.9430379994409e-05\\
0.1420142	-2.92395399943473e-05\\
0.1421142	-2.90498999942849e-05\\
0.1422142	-2.88614299942218e-05\\
0.1423142	-2.86741399941581e-05\\
0.1424142	-2.84880099940936e-05\\
0.1425143	-2.83030399940284e-05\\
0.1426143	-2.81192299939626e-05\\
0.1427143	-2.7936559993896e-05\\
0.1428143	-2.77550299938287e-05\\
0.1429143	-2.75746299937607e-05\\
0.1430143	-2.73953599936919e-05\\
0.1431143	-2.72172099936224e-05\\
0.1432143	-2.70401699935521e-05\\
0.1433143	-2.68642299934811e-05\\
0.1434143	-2.66893999934093e-05\\
0.1435144	-2.65156499933366e-05\\
0.1436144	-2.63429999932633e-05\\
0.1437144	-2.61714199931891e-05\\
0.1438144	-2.60009099931141e-05\\
0.1439144	-2.58314799930383e-05\\
0.1440144	-2.56630999929617e-05\\
0.1441144	-2.54957799928843e-05\\
0.1442144	-2.5329499992806e-05\\
0.1443144	-2.51642699927269e-05\\
0.1444144	-2.50000699926469e-05\\
0.1445145	-2.4836899992566e-05\\
0.1446145	-2.46747499924843e-05\\
0.1447145	-2.45136299924017e-05\\
0.1448145	-2.43535099923182e-05\\
0.1449145	-2.41943899922337e-05\\
0.1450145	-2.40362799921484e-05\\
0.1451145	-2.38791599920622e-05\\
0.1452145	-2.3723019991975e-05\\
0.1453145	-2.35678699918869e-05\\
0.1454145	-2.34136899917978e-05\\
0.1455146	-2.32604799917077e-05\\
0.1456146	-2.31082299916167e-05\\
0.1457146	-2.29569399915247e-05\\
0.1458146	-2.28065999914317e-05\\
0.1459146	-2.26572099913377e-05\\
0.1460146	-2.25087599912427e-05\\
0.1461146	-2.23612399911466e-05\\
0.1462146	-2.22146499910496e-05\\
0.1463146	-2.20689899909514e-05\\
0.1464146	-2.19242399908522e-05\\
0.1465147	-2.17804099907519e-05\\
0.1466147	-2.16374799906505e-05\\
0.1467147	-2.14954499905481e-05\\
0.1468147	-2.13543199904446e-05\\
0.1469147	-2.12140799903399e-05\\
0.1470147	-2.10747299902341e-05\\
0.1471147	-2.09362499901272e-05\\
0.1472147	-2.07986499900191e-05\\
0.1473147	-2.06619199899099e-05\\
0.1474147	-2.05260499897995e-05\\
0.1475148	-2.03910399896877e-05\\
0.1476148	-2.02568799895749e-05\\
0.1477148	-2.01235699894609e-05\\
0.1478148	-1.99910999893456e-05\\
0.1479148	-1.98594799892291e-05\\
0.1480148	-1.97286799891114e-05\\
0.1481148	-1.95987099889923e-05\\
0.1482148	-1.9469559988872e-05\\
0.1483148	-1.93412399887504e-05\\
0.1484148	-1.92137199886275e-05\\
0.1485149	-1.90870099885031e-05\\
0.1486149	-1.89611099883776e-05\\
0.1487149	-1.88359999882506e-05\\
0.1488149	-1.87116799881223e-05\\
0.1489149	-1.85881499879927e-05\\
0.1490149	-1.84654099878616e-05\\
0.1491149	-1.83434399877291e-05\\
0.1492149	-1.82222499875952e-05\\
0.1493149	-1.81018299874599e-05\\
0.1494149	-1.79821699873231e-05\\
0.149515	-1.78632699871847e-05\\
0.149615	-1.77451199870449e-05\\
0.149715	-1.76277299869037e-05\\
0.149815	-1.75110799867609e-05\\
0.149915	-1.73951699866166e-05\\
0.150015	-1.72799999864708e-05\\
0.150115	-1.71655599863234e-05\\
0.150215	-1.70518499861744e-05\\
0.150315	-1.69388599860238e-05\\
0.150415	-1.68265899858716e-05\\
0.1505151	-1.67150299857176e-05\\
0.1506151	-1.66041899855621e-05\\
0.1507151	-1.64940499854049e-05\\
0.1508151	-1.63846099852461e-05\\
0.1509151	-1.62758599850855e-05\\
0.1510151	-1.61678099849232e-05\\
0.1511151	-1.60604499847592e-05\\
0.1512151	-1.59537699845935e-05\\
0.1513151	-1.58477699844259e-05\\
0.1514151	-1.57424499842566e-05\\
0.1515152	-1.56377999840853e-05\\
0.1516152	-1.55338099839123e-05\\
0.1517152	-1.54304899837374e-05\\
0.1518152	-1.53278299835607e-05\\
0.1519152	-1.52258199833821e-05\\
0.1520152	-1.51244699832016e-05\\
0.1521152	-1.50237599830192e-05\\
0.1522152	-1.49236999828348e-05\\
0.1523152	-1.48242699826484e-05\\
0.1524152	-1.47254799824601e-05\\
0.1525153	-1.46273199822695e-05\\
0.1526153	-1.45297899820771e-05\\
0.1527153	-1.44328799818826e-05\\
0.1528153	-1.43365899816861e-05\\
0.1529153	-1.42409199814874e-05\\
0.1530153	-1.41458599812867e-05\\
0.1531153	-1.40514099810838e-05\\
0.1532153	-1.39575699808787e-05\\
0.1533153	-1.38643199806714e-05\\
0.1534153	-1.3771669980462e-05\\
0.1535154	-1.367961998025e-05\\
0.1536154	-1.35881599800361e-05\\
0.1537154	-1.34972799798198e-05\\
0.1538154	-1.34069799796012e-05\\
0.1539154	-1.33172699793803e-05\\
0.1540154	-1.32281299791571e-05\\
0.1541154	-1.31395599789315e-05\\
0.1542154	-1.30515699787034e-05\\
0.1543154	-1.2964129978473e-05\\
0.1544154	-1.287725997824e-05\\
0.1545155	-1.27909499780044e-05\\
0.1546155	-1.27051899777665e-05\\
0.1547155	-1.2619989977526e-05\\
0.1548155	-1.2535329977283e-05\\
0.1549155	-1.24512199770374e-05\\
0.1550155	-1.23676399767892e-05\\
0.1551155	-1.22846099765383e-05\\
0.1552155	-1.22021099762848e-05\\
0.1553155	-1.21201499760286e-05\\
0.1554155	-1.20387099757697e-05\\
0.1555156	-1.19577999755077e-05\\
0.1556156	-1.18774099752432e-05\\
0.1557156	-1.17975399749759e-05\\
0.1558156	-1.17181799747058e-05\\
0.1559156	-1.16393399744327e-05\\
0.1560156	-1.15609999741568e-05\\
0.1561156	-1.1483179973878e-05\\
0.1562156	-1.14058499735962e-05\\
0.1563156	-1.13290299733114e-05\\
0.1564156	-1.12526999730235e-05\\
0.1565157	-1.11768699727324e-05\\
0.1566157	-1.11015299724384e-05\\
0.1567157	-1.10266699721413e-05\\
0.1568157	-1.0952309971841e-05\\
0.1569157	-1.08784199715376e-05\\
0.1570157	-1.08050099712309e-05\\
0.1571157	-1.0732079970921e-05\\
0.1572157	-1.06596199706078e-05\\
0.1573157	-1.05876399702913e-05\\
0.1574157	-1.05161199699715e-05\\
0.1575158	-1.04450599696479e-05\\
0.1576158	-1.03744699693212e-05\\
0.1577158	-1.0304339968991e-05\\
0.1578158	-1.02346599686574e-05\\
0.1579158	-1.01654399683202e-05\\
0.1580158	-1.00966699679794e-05\\
0.1581158	-1.0028339967635e-05\\
0.1582158	-9.96046396728702e-06\\
0.1583158	-9.89302796693532e-06\\
0.1584158	-9.8260309665799e-06\\
0.1585159	-9.75947196622035e-06\\
0.1586159	-9.69334696585736e-06\\
0.1587159	-9.62765396549053e-06\\
0.1588159	-9.56238996511982e-06\\
0.1589159	-9.49755296474519e-06\\
0.1590159	-9.4331389643666e-06\\
0.1591159	-9.36914696398401e-06\\
0.1592159	-9.30557396359737e-06\\
0.1593159	-9.24241696320664e-06\\
0.1594159	-9.17967296281179e-06\\
0.159516	-9.11734096241236e-06\\
0.159616	-9.05541696200912e-06\\
0.159716	-8.99389896160161e-06\\
0.159816	-8.93278396118981e-06\\
0.159916	-8.87207096077365e-06\\
0.160016	-8.8117559603531e-06\\
0.160116	-8.75183695992812e-06\\
0.160216	-8.69231095949864e-06\\
0.160316	-8.63317695906464e-06\\
0.160416	-8.57443195862606e-06\\
0.1605161	-8.51607195818241e-06\\
0.1606161	-8.45809695773452e-06\\
0.1607161	-8.40050295728191e-06\\
0.1608161	-8.34328895682454e-06\\
0.1609161	-8.28645095636234e-06\\
0.1610161	-8.22998695589526e-06\\
0.1611161	-8.17389595542327e-06\\
0.1612161	-8.11817395494631e-06\\
0.1613161	-8.06281995446432e-06\\
0.1614161	-8.00783095397725e-06\\
0.1615162	-7.95320395348457e-06\\
0.1616162	-7.89893795298719e-06\\
0.1617162	-7.84502995248457e-06\\
0.1618162	-7.79147895197667e-06\\
0.1619162	-7.73828095146342e-06\\
0.1620162	-7.68543395094477e-06\\
0.1621162	-7.63293695042067e-06\\
0.1622162	-7.58078694989105e-06\\
0.1623162	-7.52898094935587e-06\\
0.1624162	-7.47751894881505e-06\\
0.1625163	-7.42639594826801e-06\\
0.1626163	-7.37561194771576e-06\\
0.1627163	-7.32516394715772e-06\\
0.1628163	-7.27505094659381e-06\\
0.1629163	-7.22526794602397e-06\\
0.1630163	-7.17581594544815e-06\\
0.1631163	-7.12669094486629e-06\\
0.1632163	-7.07789194427831e-06\\
0.1633163	-7.02941594368416e-06\\
0.1634163	-6.98126094308378e-06\\
0.1635164	-6.93342594247649e-06\\
0.1636164	-6.88590794186344e-06\\
0.1637164	-6.83870394124395e-06\\
0.1638164	-6.79181394061797e-06\\
0.1639164	-6.74523493998542e-06\\
0.1640164	-6.69896493934624e-06\\
0.1641164	-6.65300193870037e-06\\
0.1642164	-6.60734393804772e-06\\
0.1643164	-6.56198893738823e-06\\
0.1644164	-6.51693493672183e-06\\
0.1645165	-6.47217893604778e-06\\
0.1646165	-6.42772193536734e-06\\
0.1647165	-6.38355893467978e-06\\
0.1648165	-6.33968893398502e-06\\
0.1649165	-6.29611093328298e-06\\
0.1650165	-6.2528219325736e-06\\
0.1651165	-6.20982093185679e-06\\
0.1652165	-6.16710493113249e-06\\
0.1653165	-6.1246729304006e-06\\
0.1654165	-6.08252192966106e-06\\
0.1655166	-6.04065192891304e-06\\
0.1656166	-5.99905992815795e-06\\
0.1657166	-5.95774392739496e-06\\
0.1658166	-5.916702926624e-06\\
0.1659166	-5.87593392584498e-06\\
0.1660166	-5.83543592505782e-06\\
0.1661166	-5.79520692426244e-06\\
0.1662166	-5.75524592345875e-06\\
0.1663166	-5.71554992264666e-06\\
0.1664166	-5.6761169218261e-06\\
0.1665167	-5.63694692099614e-06\\
0.1666167	-5.59803692015835e-06\\
0.1667167	-5.55938491931181e-06\\
0.1668167	-5.52099091845644e-06\\
0.1669167	-5.48285091759215e-06\\
0.1670167	-5.44496491671884e-06\\
0.1671167	-5.40732991583643e-06\\
0.1672167	-5.36994591494481e-06\\
0.1673167	-5.3328099140439e-06\\
0.1674167	-5.2959209131336e-06\\
0.1675168	-5.25927691221289e-06\\
0.1676168	-5.2228759112835e-06\\
0.1677168	-5.18671691034444e-06\\
0.1678168	-5.15079890939559e-06\\
0.1679168	-5.11511890843686e-06\\
0.1680168	-5.07967690746815e-06\\
0.1681168	-5.04446990648936e-06\\
0.1682168	-5.00949690550037e-06\\
0.1683168	-4.97475590450109e-06\\
0.1684168	-4.94024590349142e-06\\
0.1685169	-4.90596590247021e-06\\
0.1686169	-4.87191390143941e-06\\
0.1687169	-4.83808790039789e-06\\
0.1688169	-4.80448589934553e-06\\
0.1689169	-4.77110889828224e-06\\
0.1690169	-4.73795289720789e-06\\
0.1691169	-4.70501789612238e-06\\
0.1692169	-4.67230089502559e-06\\
0.1693169	-4.6398028939174e-06\\
0.1694169	-4.6075208927977e-06\\
0.169517	-4.57545289166523e-06\\
0.169617	-4.54359889052214e-06\\
0.169717	-4.51195688936719e-06\\
0.169817	-4.48052488820024e-06\\
0.169917	-4.44930288702119e-06\\
0.170017	-4.4182878858299e-06\\
0.170117	-4.38747988462625e-06\\
0.170217	-4.35687788341011e-06\\
0.170317	-4.32647888218136e-06\\
0.170417	-4.29628288093987e-06\\
0.1705171	-4.26628787968424e-06\\
0.1706171	-4.23649387841687e-06\\
0.1707171	-4.20689787713635e-06\\
0.1708171	-4.17749887584257e-06\\
0.1709171	-4.14829687453538e-06\\
0.1710171	-4.11928887321465e-06\\
0.1711171	-4.09047587188024e-06\\
0.1712171	-4.06185487053202e-06\\
0.1713171	-4.03342486916983e-06\\
0.1714171	-4.00518486779354e-06\\
0.1715172	-3.97713386640161e-06\\
0.1716172	-3.94927086499668e-06\\
0.1717172	-3.92159386357721e-06\\
0.1718172	-3.89410286214305e-06\\
0.1719172	-3.86679486069407e-06\\
0.1720172	-3.83967085923009e-06\\
0.1721172	-3.81272885775099e-06\\
0.1722172	-3.78596685625659e-06\\
0.1723172	-3.75938485474674e-06\\
0.1724172	-3.7329818532213e-06\\
0.1725173	-3.70675585167854e-06\\
0.1726173	-3.6807058501214e-06\\
0.1727173	-3.65483084854817e-06\\
0.1728173	-3.6291308469587e-06\\
0.1729173	-3.60360384535281e-06\\
0.1730173	-3.57824884373035e-06\\
0.1731173	-3.55306384209115e-06\\
0.1732173	-3.52804984043502e-06\\
0.1733173	-3.50320383876181e-06\\
0.1734173	-3.47852683707134e-06\\
0.1735174	-3.45401583536172e-06\\
0.1736174	-3.42967083363618e-06\\
0.1737174	-3.40549083189285e-06\\
0.1738174	-3.38147383013155e-06\\
0.1739174	-3.3576198283521e-06\\
0.1740174	-3.33392782655431e-06\\
0.1741174	-3.310396824738e-06\\
0.1742174	-3.28702582290297e-06\\
0.1743174	-3.26381382104905e-06\\
0.1744174	-3.24075881917603e-06\\
0.1745175	-3.21786181728183e-06\\
0.1746175	-3.19511981537003e-06\\
0.1747175	-3.17253381343855e-06\\
0.1748175	-3.1501018114872e-06\\
0.1749175	-3.12782280951577e-06\\
0.1750175	-3.10569580752406e-06\\
0.1751175	-3.08371980551186e-06\\
0.1752175	-3.06189480347897e-06\\
0.1753175	-3.04021980142518e-06\\
0.1754175	-3.01869279935028e-06\\
0.1755176	-2.99731279725194e-06\\
0.1756176	-2.97608079513415e-06\\
0.1757176	-2.95499379299459e-06\\
0.1758176	-2.93405279083306e-06\\
0.1759176	-2.91325478864933e-06\\
0.1760176	-2.89260078644317e-06\\
0.1761176	-2.87208878421436e-06\\
0.1762176	-2.85171878196267e-06\\
0.1763176	-2.83148877968787e-06\\
0.1764176	-2.81139877738973e-06\\
0.1765177	-2.79144777506567e-06\\
0.1766177	-2.7716347727201e-06\\
0.1767177	-2.75195977035048e-06\\
0.1768177	-2.73241976795656e-06\\
0.1769177	-2.71301576553809e-06\\
0.1770177	-2.69374676309483e-06\\
0.1771177	-2.67461176062653e-06\\
0.1772177	-2.65560875813293e-06\\
0.1773177	-2.63673875561378e-06\\
0.1774177	-2.61799975306883e-06\\
0.1775178	-2.59939075049523e-06\\
0.1776178	-2.58091174789786e-06\\
0.1777178	-2.56256174527389e-06\\
0.1778178	-2.54433974262306e-06\\
0.1779178	-2.5262447399451e-06\\
0.1780178	-2.50827573723974e-06\\
0.1781178	-2.4904327345067e-06\\
0.1782178	-2.4727147317457e-06\\
0.1783178	-2.45511972895645e-06\\
0.1784178	-2.43764872613869e-06\\
0.1785179	-2.42029972328926e-06\\
0.1786179	-2.40307272041355e-06\\
0.1787179	-2.38596571750846e-06\\
0.1788179	-2.36897871457368e-06\\
0.1789179	-2.35211071160891e-06\\
0.1790179	-2.33536070861386e-06\\
0.1791179	-2.31872870558822e-06\\
0.1792179	-2.30221370253168e-06\\
0.1793179	-2.28581469944393e-06\\
0.1794179	-2.26953069632467e-06\\
0.179518	-2.2533606931704e-06\\
0.179618	-2.23730468998711e-06\\
0.179718	-2.22136068677134e-06\\
0.179818	-2.20552968352276e-06\\
0.179918	-2.18980868024106e-06\\
0.180018	-2.17419967692588e-06\\
0.180118	-2.15869967357691e-06\\
0.180218	-2.1433076701938e-06\\
0.180318	-2.12802466677621e-06\\
0.180418	-2.1128496633238e-06\\
0.1805181	-2.0977806598327e-06\\
0.1806181	-2.08281765630955e-06\\
0.1807181	-2.06795965275051e-06\\
0.1808181	-2.05320564915523e-06\\
0.1809181	-2.03855564552335e-06\\
0.1810181	-2.0240086418545e-06\\
0.1811181	-2.0095636381483e-06\\
0.1812181	-1.9952196344044e-06\\
0.1813181	-1.9809766306224e-06\\
0.1814181	-1.96683362680193e-06\\
0.1815182	-1.95278862293873e-06\\
0.1816182	-1.93884361904012e-06\\
0.1817182	-1.92499561510187e-06\\
0.1818182	-1.91124361112359e-06\\
0.1819182	-1.89758860710488e-06\\
0.1820182	-1.88402960304534e-06\\
0.1821182	-1.87056459894455e-06\\
0.1822182	-1.8571945948021e-06\\
0.1823182	-1.84391659061759e-06\\
0.1824182	-1.83073258639058e-06\\
0.1825183	-1.81763958211636e-06\\
0.1826183	-1.80463857780306e-06\\
0.1827183	-1.79172757344597e-06\\
0.1828183	-1.77890656904468e-06\\
0.1829183	-1.76617456459873e-06\\
0.1830183	-1.75353156010768e-06\\
0.1831183	-1.74097555557109e-06\\
0.1832183	-1.72850655098849e-06\\
0.1833183	-1.71612454635944e-06\\
0.1834183	-1.70382854168347e-06\\
0.1835184	-1.69161753695536e-06\\
0.1836184	-1.67949153218408e-06\\
0.1837184	-1.66744852736447e-06\\
0.1838184	-1.65548852249605e-06\\
0.1839184	-1.64361151757831e-06\\
0.1840184	-1.63181651261079e-06\\
0.1841184	-1.62010250759297e-06\\
0.1842184	-1.60846850252436e-06\\
0.1843184	-1.59691549740445e-06\\
0.1844184	-1.58544149223275e-06\\
0.1845185	-1.57404648700347e-06\\
0.1846185	-1.56272948172654e-06\\
0.1847185	-1.55149047639626e-06\\
0.1848185	-1.54032747101207e-06\\
0.1849185	-1.52924146557346e-06\\
0.1850185	-1.51823146007987e-06\\
0.1851185	-1.50729645453077e-06\\
0.1852185	-1.4964364489256e-06\\
0.1853185	-1.4856504432638e-06\\
0.1854185	-1.47493743754482e-06\\
0.1855186	-1.46429843176228e-06\\
0.1856186	-1.45373042592715e-06\\
0.1857186	-1.44323542003313e-06\\
0.1858186	-1.43281141407961e-06\\
0.1859186	-1.42245840806601e-06\\
0.1860186	-1.41217540199174e-06\\
0.1861186	-1.40196339585619e-06\\
0.1862186	-1.39181938965875e-06\\
0.1863186	-1.38174438339882e-06\\
0.1864186	-1.37173737707578e-06\\
0.1865187	-1.36179937068257e-06\\
0.1866187	-1.35192736423135e-06\\
0.1867187	-1.34212335771512e-06\\
0.1868187	-1.33238435113323e-06\\
0.1869187	-1.32271234448504e-06\\
0.1870187	-1.31310533776989e-06\\
0.1871187	-1.30356333098713e-06\\
0.1872187	-1.29408632413607e-06\\
0.1873187	-1.28467331721605e-06\\
0.1874187	-1.27532331022638e-06\\
0.1875188	-1.26603730315928e-06\\
0.1876188	-1.25681329602817e-06\\
0.1877188	-1.24765228882533e-06\\
0.1878188	-1.23855328155005e-06\\
0.1879188	-1.22951627420161e-06\\
0.1880188	-1.2205392667793e-06\\
0.1881188	-1.21162425928237e-06\\
0.1882188	-1.2027692517101e-06\\
0.1883188	-1.19397424406174e-06\\
0.1884188	-1.18523823633654e-06\\
0.1885189	-1.1765622285259e-06\\
0.1886189	-1.16794522064466e-06\\
0.1887189	-1.15938621268428e-06\\
0.1888189	-1.15088620464398e-06\\
0.1889189	-1.14244319652298e-06\\
0.1890189	-1.13405818832048e-06\\
0.1891189	-1.12573118003568e-06\\
0.1892189	-1.11746017166776e-06\\
0.1893189	-1.10924516321591e-06\\
0.1894189	-1.10108715467929e-06\\
0.189519	-1.09298514604842e-06\\
0.189619	-1.08493913733968e-06\\
0.189719	-1.07694712854366e-06\\
0.189819	-1.06901111965949e-06\\
0.189919	-1.0611291106863e-06\\
0.190019	-1.05330210162322e-06\\
0.190119	-1.04552909246936e-06\\
0.190219	-1.03780908322384e-06\\
0.190319	-1.03014307388574e-06\\
0.190419	-1.02253106445417e-06\\
0.1905191	-1.01497105491862e-06\\
0.1906191	-1.00746404529722e-06\\
0.1907191	-1.00001003557955e-06\\
0.1908191	-9.9260722576468e-07\\
0.1909191	-9.85256715851641e-07\\
0.1910191	-9.77957905839473e-07\\
0.1911191	-9.70710395727201e-07\\
0.1912191	-9.63514085513845e-07\\
0.1913191	-9.56368575198412e-07\\
0.1914191	-9.49273664779899e-07\\
0.1915192	-9.42229154246721e-07\\
0.1916192	-9.35234743618901e-07\\
0.1917192	-9.28290232884937e-07\\
0.1918192	-9.21395322043789e-07\\
0.1919192	-9.14549811094406e-07\\
0.1920192	-9.07753400035728e-07\\
0.1921192	-9.01005888866682e-07\\
0.1922192	-8.94307077586188e-07\\
0.1923192	-8.87656766193153e-07\\
0.1924192	-8.81054554686476e-07\\
0.1925193	-8.74500343053364e-07\\
0.1926193	-8.67993831315936e-07\\
0.1927193	-8.61534819461493e-07\\
0.1928193	-8.5512300748889e-07\\
0.1929193	-8.48758295396971e-07\\
0.1930193	-8.42440283184569e-07\\
0.1931193	-8.36168870850505e-07\\
0.1932193	-8.29943658393589e-07\\
0.1933193	-8.23764645812619e-07\\
0.1934193	-8.17631433106384e-07\\
0.1935194	-8.11543820260762e-07\\
0.1936194	-8.05501607300181e-07\\
0.1937194	-7.99504494210624e-07\\
0.1938194	-7.93552180990832e-07\\
0.1939194	-7.87644667639532e-07\\
0.1940194	-7.81781454155439e-07\\
0.1941194	-7.75962440537258e-07\\
0.1942194	-7.70187326783679e-07\\
0.1943194	-7.64455912893382e-07\\
0.1944194	-7.58767998865031e-07\\
0.1945195	-7.53123184683042e-07\\
0.1946195	-7.47521370374389e-07\\
0.1947195	-7.41962155923603e-07\\
0.1948195	-7.36445441329296e-07\\
0.1949195	-7.30970926590071e-07\\
0.1950195	-7.25538311704514e-07\\
0.1951195	-7.20147296671199e-07\\
0.1952195	-7.14797781488686e-07\\
0.1953195	-7.09489366155523e-07\\
0.1954195	-7.04221950670241e-07\\
0.1955196	-6.98995035015644e-07\\
0.1956196	-6.93808519221512e-07\\
0.1957196	-6.88662103270774e-07\\
0.1958196	-6.83555487161906e-07\\
0.1959196	-6.78488370893368e-07\\
0.1960196	-6.73460554463608e-07\\
0.1961196	-6.68471737871056e-07\\
0.1962196	-6.63521621114132e-07\\
0.1963196	-6.58610004191234e-07\\
0.1964196	-6.53736487100751e-07\\
0.1965197	-6.48900869823709e-07\\
0.1966197	-6.4410285239298e-07\\
0.1967197	-6.39342034789731e-07\\
0.1968197	-6.34618317012286e-07\\
0.1969197	-6.29931299058951e-07\\
0.1970197	-6.2528068092802e-07\\
0.1971197	-6.20666162617769e-07\\
0.1972197	-6.16087544126455e-07\\
0.1973197	-6.11544425452324e-07\\
0.1974197	-6.07036506593599e-07\\
0.1975198	-6.02563587529352e-07\\
0.1976198	-5.98125268295864e-07\\
0.1977198	-5.93721248872358e-07\\
0.1978198	-5.89351329256991e-07\\
0.1979198	-5.85015009447904e-07\\
0.1980198	-5.80712189443219e-07\\
0.1981198	-5.7644246924104e-07\\
0.1982198	-5.72205548839454e-07\\
0.1983198	-5.6800112823653e-07\\
0.1984198	-5.63828807430318e-07\\
0.1985199	-5.59688486397733e-07\\
0.1986199	-5.55579665178811e-07\\
0.1987199	-5.51502043750637e-07\\
0.1988199	-5.47455422111186e-07\\
0.1989199	-5.43439400258413e-07\\
0.1990199	-5.39453678190256e-07\\
0.1991199	-5.3549805590463e-07\\
0.1992199	-5.31572033399432e-07\\
0.1993199	-5.2767551067254e-07\\
0.1994199	-5.23807987721811e-07\\
0.19952	-5.19969264521791e-07\\
0.19962	-5.16158941116647e-07\\
0.19972	-5.12376917481112e-07\\
0.19982	-5.08622693612961e-07\\
0.19992	-5.04895969509946e-07\\
0.20002	-5.01196545169801e-07\\
0.20012	-4.97524120590235e-07\\
0.20022	-4.93878295768938e-07\\
0.20032	-4.90258770703577e-07\\
0.20042	-4.86665345391798e-07\\
0.2005201	-4.83097719805537e-07\\
0.2006201	-4.79555593993515e-07\\
0.2007201	-4.76038667927872e-07\\
0.2008201	-4.72546641606165e-07\\
0.2009201	-4.69079215025924e-07\\
0.2010201	-4.65636088184658e-07\\
0.2011201	-4.62217061079852e-07\\
0.2012201	-4.58821833708967e-07\\
0.2013201	-4.55450106069441e-07\\
0.2014201	-4.52101678158686e-07\\
0.2015202	-4.48776249945768e-07\\
0.2016202	-4.45473421484418e-07\\
0.2017202	-4.42193192743929e-07\\
0.2018202	-4.38935163721613e-07\\
0.2019202	-4.3569903441476e-07\\
0.2020202	-4.32484704820633e-07\\
0.2021202	-4.29291774936468e-07\\
0.2022202	-4.26120044759478e-07\\
0.2023202	-4.22969414286846e-07\\
0.2024202	-4.19839583515732e-07\\
0.2025203	-4.16730252412042e-07\\
0.2026203	-4.13641221035024e-07\\
0.2027203	-4.10572289350834e-07\\
0.2028203	-4.07523357356522e-07\\
0.2029203	-4.0449402504911e-07\\
0.2030203	-4.01484192425591e-07\\
0.2031203	-3.98493659482932e-07\\
0.2032203	-3.95522326218069e-07\\
0.2033203	-3.92569792627911e-07\\
0.2034203	-3.89636058709338e-07\\
0.2035204	-3.86720824424781e-07\\
0.2036204	-3.8382398983956e-07\\
0.2037204	-3.80945354916381e-07\\
0.2038204	-3.78084819652004e-07\\
0.2039204	-3.7524208404316e-07\\
0.2040204	-3.72417148086546e-07\\
0.2041204	-3.69609811778832e-07\\
0.2042204	-3.66819975116654e-07\\
0.2043204	-3.64047438096618e-07\\
0.2044204	-3.61292000715297e-07\\
0.2045205	-3.58553662931302e-07\\
0.2046205	-3.55832324816633e-07\\
0.2047205	-3.53127686330202e-07\\
0.2048205	-3.50439847468451e-07\\
0.2049205	-3.47768608227791e-07\\
0.2050205	-3.45113768604598e-07\\
0.2051205	-3.42475428595214e-07\\
0.2052205	-3.39853388195946e-07\\
0.2053205	-3.37247547403068e-07\\
0.2054205	-3.34657806212819e-07\\
0.2055206	-3.32084164579608e-07\\
0.2056206	-3.29526522582782e-07\\
0.2057206	-3.26984780177083e-07\\
0.2058206	-3.24458937358606e-07\\
0.2059206	-3.21948794123411e-07\\
0.2060206	-3.19454450467519e-07\\
0.2061206	-3.16975806386915e-07\\
0.2062206	-3.14512761877547e-07\\
0.2063206	-3.12065416935324e-07\\
0.2064206	-3.09633471556117e-07\\
0.2065207	-3.07217125689717e-07\\
0.2066207	-3.04816279423554e-07\\
0.2067207	-3.02430932707785e-07\\
0.2068207	-3.00060885538122e-07\\
0.2069207	-2.97706337910243e-07\\
0.2070207	-2.95367189819779e-07\\
0.2071207	-2.93043341262326e-07\\
0.2072207	-2.90734892233436e-07\\
0.2073207	-2.88441842728619e-07\\
0.2074207	-2.86164092743345e-07\\
0.2075208	-2.83901642222326e-07\\
0.2076208	-2.81654591261884e-07\\
0.2077208	-2.79422839807132e-07\\
0.2078208	-2.77206387853368e-07\\
0.2079208	-2.75005335395846e-07\\
0.2080208	-2.72819482429775e-07\\
0.2081208	-2.70649128950322e-07\\
0.2082208	-2.68493974952605e-07\\
0.2083208	-2.663542204317e-07\\
0.2084208	-2.64229865382635e-07\\
0.2085209	-2.62120909744542e-07\\
0.2086209	-2.60027253623517e-07\\
0.2087209	-2.57948996959134e-07\\
0.2088209	-2.55886139746235e-07\\
0.2089209	-2.53838681979611e-07\\
0.2090209	-2.51806723654009e-07\\
0.2091209	-2.49790064764123e-07\\
0.2092209	-2.47788905304601e-07\\
0.2093209	-2.45803245270039e-07\\
0.2094209	-2.43832984654983e-07\\
0.209521	-2.41878223392434e-07\\
0.209621	-2.39938861599235e-07\\
0.209721	-2.3801509920887e-07\\
0.209821	-2.36106736215682e-07\\
0.209921	-2.34213872613959e-07\\
0.210021	-2.32336508397937e-07\\
0.210121	-2.30474643561798e-07\\
0.210221	-2.28628178099669e-07\\
0.210321	-2.26797312005626e-07\\
0.210421	-2.24981845273686e-07\\
0.2105211	-2.23181877830113e-07\\
0.2106211	-2.21397409803561e-07\\
0.2107211	-2.19628341120829e-07\\
0.2108211	-2.1787467177571e-07\\
0.2109211	-2.1613650176194e-07\\
0.2110211	-2.14413631073199e-07\\
0.2111211	-2.12706159703106e-07\\
0.2112211	-2.11013987645221e-07\\
0.2113211	-2.09337214893047e-07\\
0.2114211	-2.07675641440025e-07\\
0.2115212	-2.06029267205017e-07\\
0.2116212	-2.0439819232966e-07\\
0.2117212	-2.02782216733405e-07\\
0.2118212	-2.01181340409446e-07\\
0.2119212	-1.99595563350917e-07\\
0.2120212	-1.98024785550885e-07\\
0.2121212	-1.96468907002356e-07\\
0.2122212	-1.9492792769827e-07\\
0.2123212	-1.93401747631501e-07\\
0.2124212	-1.91890266794858e-07\\
0.2125213	-1.90393585099079e-07\\
0.2126213	-1.88911402700062e-07\\
0.2127213	-1.8744371950919e-07\\
0.2128213	-1.85990535519004e-07\\
0.2129213	-1.84551650721974e-07\\
0.2130213	-1.83126965110501e-07\\
0.2131213	-1.81716478676917e-07\\
0.2132213	-1.80319991413484e-07\\
0.2133213	-1.7893750331239e-07\\
0.2134213	-1.77568814365753e-07\\
0.2135214	-1.76213824475387e-07\\
0.2136214	-1.7487243381286e-07\\
0.2137214	-1.73544542280695e-07\\
0.2138214	-1.72230049870716e-07\\
0.2139214	-1.7092875657467e-07\\
0.2140214	-1.6964056238423e-07\\
0.2141214	-1.68365267290989e-07\\
0.2142214	-1.67102871286467e-07\\
0.2143214	-1.65853174362102e-07\\
0.2144214	-1.64615976509256e-07\\
0.2145215	-1.63391177619948e-07\\
0.2146215	-1.62178577882953e-07\\
0.2147215	-1.60978077191072e-07\\
0.2148215	-1.59789575535343e-07\\
0.2149215	-1.58612772906722e-07\\
0.2150215	-1.57447569296083e-07\\
0.2151215	-1.56293864694217e-07\\
0.2152215	-1.55151359091827e-07\\
0.2153215	-1.54019952479534e-07\\
0.2154215	-1.52899444847871e-07\\
0.2155216	-1.51789736078107e-07\\
0.2156216	-1.50690626377914e-07\\
0.2157216	-1.49601915629419e-07\\
0.2158216	-1.48523303822802e-07\\
0.2159216	-1.47454790948156e-07\\
0.2160216	-1.4639617699548e-07\\
0.2161216	-1.45347161954684e-07\\
0.2162216	-1.44307645815584e-07\\
0.2163216	-1.43277428567903e-07\\
0.2164216	-1.42256310201271e-07\\
0.2165217	-1.41244090585157e-07\\
0.2166217	-1.40240669947983e-07\\
0.2167217	-1.3924574816016e-07\\
0.2168217	-1.38259125210933e-07\\
0.2169217	-1.37280701089448e-07\\
0.2170217	-1.36310275784749e-07\\
0.2171217	-1.35347649285784e-07\\
0.2172217	-1.34392621581398e-07\\
0.2173217	-1.33444892660333e-07\\
0.2174217	-1.32504462511229e-07\\
0.2175218	-1.31571130990611e-07\\
0.2176218	-1.30644598349678e-07\\
0.2177218	-1.29724764445988e-07\\
0.2178218	-1.28811429267761e-07\\
0.2179218	-1.27904392803109e-07\\
0.2180218	-1.27003455040033e-07\\
0.2181218	-1.26108515966429e-07\\
0.2182218	-1.25219375570078e-07\\
0.2183218	-1.24335833838654e-07\\
0.2184218	-1.23457790759713e-07\\
0.2185219	-1.2258494617558e-07\\
0.2186219	-1.21717300362453e-07\\
0.2187219	-1.2085465316379e-07\\
0.2188219	-1.19996704566691e-07\\
0.2189219	-1.19143554558138e-07\\
0.2190219	-1.18294803124992e-07\\
0.2191219	-1.17450450253998e-07\\
0.2192219	-1.16610295931779e-07\\
0.2193219	-1.15774240144834e-07\\
0.2194219	-1.14942182879541e-07\\
0.219522	-1.14113923962647e-07\\
0.219622	-1.13289263697784e-07\\
0.219722	-1.12468301912945e-07\\
0.219822	-1.11650738594006e-07\\
0.219922	-1.10836473726712e-07\\
0.220022	-1.10025407296682e-07\\
0.220122	-1.09217539289401e-07\\
0.220222	-1.08412669690225e-07\\
0.220322	-1.07610798484374e-07\\
0.220422	-1.06811725656936e-07\\
0.2205221	-1.06015351017575e-07\\
0.2206221	-1.05221674900023e-07\\
0.2207221	-1.04430597115312e-07\\
0.2208221	-1.03642117647981e-07\\
0.2209221	-1.02856036482428e-07\\
0.2210221	-1.02072353602908e-07\\
0.2211221	-1.01291068993536e-07\\
0.2212221	-1.00512082638281e-07\\
0.2213221	-9.97353445209658e-08\\
0.2214221	-9.89608146252686e-08\\
0.2215222	-9.81884827421242e-08\\
0.2216222	-9.74182992382818e-08\\
0.2217222	-9.66502439061795e-08\\
0.2218222	-9.58843167288948e-08\\
0.2219222	-9.5120487689352e-08\\
0.2220222	-9.43587467703206e-08\\
0.2221222	-9.35991039544145e-08\\
0.2222222	-9.28415492240898e-08\\
0.2223222	-9.20861025616443e-08\\
0.2224222	-9.13327539492153e-08\\
0.2225223	-9.05815431572073e-08\\
0.2226223	-8.98324805885807e-08\\
0.2227223	-8.90855960153894e-08\\
0.2228223	-8.8340919419116e-08\\
0.2229223	-8.75984807810753e-08\\
0.2230223	-8.68583300824138e-08\\
0.2231223	-8.61204973041073e-08\\
0.2232223	-8.53850424269599e-08\\
0.2233223	-8.46520054316024e-08\\
0.2234223	-8.39214662984908e-08\\
0.2235224	-8.31934547755275e-08\\
0.2236224	-8.24680613053807e-08\\
0.2237224	-8.17453356377628e-08\\
0.2238224	-8.10253577524144e-08\\
0.2239224	-8.03081876288936e-08\\
0.2240224	-7.95939252465741e-08\\
0.2241224	-7.88826405846437e-08\\
0.2242224	-7.81744136221026e-08\\
0.2243224	-7.74693343377618e-08\\
0.2244224	-7.67674827102413e-08\\
0.2245225	-7.60689584627858e-08\\
0.2246225	-7.53738620815969e-08\\
0.2247225	-7.46822732919045e-08\\
0.2248225	-7.39942920715478e-08\\
0.2249225	-7.33100283981669e-08\\
0.2250225	-7.26295722492005e-08\\
0.2251225	-7.19530336018844e-08\\
0.2252225	-7.12805124332496e-08\\
0.2253225	-7.06120987201208e-08\\
0.2254225	-6.9947922439114e-08\\
0.2255226	-6.92880632864578e-08\\
0.2256226	-6.86326517960738e-08\\
0.2257226	-6.79817676663678e-08\\
0.2258226	-6.73355408731039e-08\\
0.2259226	-6.66940713918285e-08\\
0.2260226	-6.60574591978686e-08\\
0.2261226	-6.54258042663294e-08\\
0.2262226	-6.47992265720929e-08\\
0.2263226	-6.41778160898157e-08\\
0.2264226	-6.35616927939271e-08\\
0.2265227	-6.29509363510616e-08\\
0.2266227	-6.23456573474395e-08\\
0.2267227	-6.17459554520828e-08\\
0.2268227	-6.11519306384911e-08\\
0.2269227	-6.05636628799261e-08\\
0.2270227	-5.99812621494103e-08\\
0.2271227	-5.94048084197242e-08\\
0.2272227	-5.88344016634047e-08\\
0.2273227	-5.82701118527429e-08\\
0.2274227	-5.77120189597815e-08\\
0.2275228	-5.71602226187434e-08\\
0.2276228	-5.66147934731552e-08\\
0.2277228	-5.6075791159857e-08\\
0.2278228	-5.55432956498774e-08\\
0.2279228	-5.50173769139858e-08\\
0.2280228	-5.44980849226901e-08\\
0.2281228	-5.39854996462348e-08\\
0.2282228	-5.34796610545982e-08\\
0.2283228	-5.29806191174906e-08\\
0.2284228	-5.24884238043516e-08\\
0.2285229	-5.20031347139121e-08\\
0.2286229	-5.15247625524809e-08\\
0.2287229	-5.10533669216582e-08\\
0.2288229	-5.05889577897768e-08\\
0.2289229	-5.01315751248874e-08\\
0.2290229	-4.96812288947554e-08\\
0.2291229	-4.92379390668584e-08\\
0.2292229	-4.88017256083844e-08\\
0.2293229	-4.83725784862286e-08\\
0.2294229	-4.79505076669909e-08\\
0.229523	-4.75355227105453e-08\\
0.229623	-4.7127594391969e-08\\
0.229723	-4.67267122742801e-08\\
0.229823	-4.63328663228719e-08\\
0.229923	-4.59460365028297e-08\\
0.230023	-4.55661827789276e-08\\
0.230123	-4.51932751156264e-08\\
0.230223	-4.48272834770704e-08\\
0.230323	-4.44681578270851e-08\\
0.230423	-4.4115838129174e-08\\
0.2305231	-4.37702939006765e-08\\
0.2306231	-4.34314559919825e-08\\
0.2307231	-4.30992539238768e-08\\
0.2308231	-4.2773627658547e-08\\
0.2309231	-4.24545071578449e-08\\
0.2310231	-4.21418023832833e-08\\
0.2311231	-4.18354432960332e-08\\
0.2312231	-4.15353298569205e-08\\
0.2313231	-4.12413920264235e-08\\
0.2314231	-4.09535097646695e-08\\
0.2315232	-4.06715925424474e-08\\
0.2316232	-4.03955412926106e-08\\
0.2317232	-4.01252454897222e-08\\
0.2318232	-3.98605850924753e-08\\
0.2319232	-3.96014500591964e-08\\
0.2320232	-3.93477203478427e-08\\
0.2321232	-3.90992659159987e-08\\
0.2322232	-3.8855966720873e-08\\
0.2323232	-3.86176827192951e-08\\
0.2324232	-3.83842938677119e-08\\
0.2325233	-3.81556395859751e-08\\
0.2326233	-3.79316008972156e-08\\
0.2327233	-3.77120272254193e-08\\
0.2328233	-3.74967685254696e-08\\
0.2329233	-3.72856847518509e-08\\
0.2330233	-3.70786258586448e-08\\
0.2331233	-3.68754317995267e-08\\
0.2332233	-3.66759425277626e-08\\
0.2333233	-3.64800179962049e-08\\
0.2334233	-3.62874981572895e-08\\
0.2335234	-3.60982023751411e-08\\
0.2336234	-3.59120017717045e-08\\
0.2337234	-3.57287057156319e-08\\
0.2338234	-3.55481741576553e-08\\
0.2339234	-3.53702270480717e-08\\
0.2340234	-3.51947143367393e-08\\
0.2341234	-3.50214659730737e-08\\
0.2342234	-3.48503119060442e-08\\
0.2343234	-3.46810920841697e-08\\
0.2344234	-3.45136564555152e-08\\
0.2345235	-3.43478243232495e-08\\
0.2346235	-3.41834369174549e-08\\
0.2347235	-3.40203435462575e-08\\
0.2348235	-3.38583741558657e-08\\
0.2349235	-3.36973686920143e-08\\
0.2350235	-3.35371670999597e-08\\
0.2351235	-3.33776293244764e-08\\
0.2352235	-3.32185853098523e-08\\
0.2353235	-3.3059894999885e-08\\
0.2354235	-3.29013983378775e-08\\
0.2355236	-3.27429445603353e-08\\
0.2356236	-3.25844050156597e-08\\
0.2357236	-3.24256289457833e-08\\
0.2358236	-3.22664762919865e-08\\
0.2359236	-3.2106816995034e-08\\
0.2360236	-3.19465109951694e-08\\
0.2361236	-3.17854282321113e-08\\
0.2362236	-3.16234486450484e-08\\
0.2363236	-3.14604421726354e-08\\
0.2364236	-3.12962987529881e-08\\
0.2365237	-3.11309075497201e-08\\
0.2366237	-3.09641500406687e-08\\
0.2367237	-3.07959253953873e-08\\
0.2368237	-3.06261235497893e-08\\
0.2369237	-3.04546644392258e-08\\
0.2370237	-3.0281437998481e-08\\
0.2371237	-3.01063641617669e-08\\
0.2372237	-2.99293628627187e-08\\
0.2373237	-2.97503640343901e-08\\
0.2374237	-2.95692776092476e-08\\
0.2375238	-2.93860426712172e-08\\
0.2376238	-2.92006008397071e-08\\
0.2377238	-2.90129012051431e-08\\
0.2378238	-2.88228836975937e-08\\
0.2379238	-2.86304982465152e-08\\
0.2380238	-2.84357047807466e-08\\
0.2381238	-2.82384832285039e-08\\
0.2382238	-2.80387835173753e-08\\
0.2383238	-2.78365955743153e-08\\
0.2384238	-2.76318893256398e-08\\
0.2385239	-2.74246737681725e-08\\
0.2386239	-2.72149106761382e-08\\
0.2387239	-2.70026290534722e-08\\
0.2388239	-2.67878088238795e-08\\
0.2389239	-2.65704699103985e-08\\
0.2390239	-2.63506222353954e-08\\
0.2391239	-2.61282957205586e-08\\
0.2392239	-2.59035002868925e-08\\
0.2393239	-2.56762858547124e-08\\
0.2394239	-2.54466823436378e-08\\
0.239524	-2.52147186553043e-08\\
0.239624	-2.49804667332058e-08\\
0.239724	-2.47439554867574e-08\\
0.239824	-2.45052548327368e-08\\
0.239924	-2.42644246871963e-08\\
0.240024	-2.40215349654565e-08\\
0.240124	-2.37766555821004e-08\\
0.240224	-2.35298564509665e-08\\
0.240324	-2.32812374851432e-08\\
0.240424	-2.30308685969617e-08\\
0.2405241	-2.27788385840514e-08\\
0.2406241	-2.25252495749438e-08\\
0.2407241	-2.22702003757766e-08\\
0.2408241	-2.20138008957886e-08\\
0.2409241	-2.17561410434288e-08\\
0.2410241	-2.14973407263501e-08\\
0.2411241	-2.12375098514026e-08\\
0.2412241	-2.09767583246267e-08\\
0.2413241	-2.07152260512459e-08\\
0.2414241	-2.04530229356606e-08\\
0.2415242	-2.0190267661879e-08\\
0.2416242	-1.99271025606713e-08\\
0.2417242	-1.966365632535e-08\\
0.2418242	-1.94000588569501e-08\\
0.2419242	-1.91364400556477e-08\\
0.2420242	-1.88729398207524e-08\\
0.2421242	-1.86097080507006e-08\\
0.2422242	-1.83468646430474e-08\\
0.2423242	-1.80845694944598e-08\\
0.2424242	-1.78229625007086e-08\\
0.2425243	-1.75621722217009e-08\\
0.2426243	-1.73023612092044e-08\\
0.2427243	-1.70436580332983e-08\\
0.2428243	-1.67862225860926e-08\\
0.2429243	-1.65301947587632e-08\\
0.2430243	-1.62757144415433e-08\\
0.2431243	-1.6022921523716e-08\\
0.2432243	-1.57719758936063e-08\\
0.2433243	-1.55230074385726e-08\\
0.2434243	-1.5276166044999e-08\\
0.2435244	-1.50315801372686e-08\\
0.2436244	-1.47894125086014e-08\\
0.2437244	-1.45497715944999e-08\\
0.2438244	-1.43128172773664e-08\\
0.2439244	-1.40786794385869e-08\\
0.2440244	-1.38474879585225e-08\\
0.2441244	-1.3619362716501e-08\\
0.2442244	-1.3394443590808e-08\\
0.2443244	-1.31728504586782e-08\\
0.2444244	-1.29547131962865e-08\\
0.2445245	-1.27401300800449e-08\\
0.2446245	-1.2529234166926e-08\\
0.2447245	-1.23221337454943e-08\\
0.2448245	-1.21189386875948e-08\\
0.2449245	-1.19197488639673e-08\\
0.2450245	-1.17246641442374e-08\\
0.2451245	-1.15337843969069e-08\\
0.2452245	-1.13472094893443e-08\\
0.2453245	-1.11650092877757e-08\\
0.2454245	-1.09872936572749e-08\\
0.2455246	-1.08141107127219e-08\\
0.2456246	-1.06455537991487e-08\\
0.2457246	-1.04816910447147e-08\\
0.2458246	-1.03225923097858e-08\\
0.2459246	-1.01683074535266e-08\\
0.2460246	-1.00188863338902e-08\\
0.2461246	-9.87439880760805e-09\\
0.2462246	-9.73486973017974e-09\\
0.2463246	-9.60034695586264e-09\\
0.2464246	-9.47086133766154e-09\\
0.2465247	-9.34643981415339e-09\\
0.2466247	-9.22710904491874e-09\\
0.2467247	-9.11288298304208e-09\\
0.2468247	-9.00377347641075e-09\\
0.2469247	-8.8997853716062e-09\\
0.2470247	-8.80091951389305e-09\\
0.2471247	-8.70716974720808e-09\\
0.2472247	-8.61852391414912e-09\\
0.2473247	-8.53496685596388e-09\\
0.2474247	-8.45647441253866e-09\\
0.2475248	-8.38301733005991e-09\\
0.2476248	-8.31456461151942e-09\\
0.2477248	-8.25107401895021e-09\\
0.2478248	-8.19250238667685e-09\\
0.2479248	-8.13879754760472e-09\\
0.2480248	-8.08990333320814e-09\\
0.2481248	-8.04575957351841e-09\\
0.2482248	-8.00629909711182e-09\\
0.2483248	-7.9714487310975e-09\\
0.2484248	-7.94113230110518e-09\\
0.2485249	-7.9152653434139e-09\\
0.2486249	-7.89376423586897e-09\\
0.2487249	-7.87653553205543e-09\\
0.2488249	-7.86348205155963e-09\\
0.2489249	-7.85450161242584e-09\\
0.2490249	-7.84949003114343e-09\\
0.2491249	-7.84833512263395e-09\\
0.2492249	-7.85092370023806e-09\\
0.2493249	-7.85713657570241e-09\\
0.2494249	-7.86685055916633e-09\\
0.249525	-7.8799369579314e-09\\
0.249625	-7.89627055894264e-09\\
0.249725	-7.915715688398e-09\\
0.249825	-7.93813614987129e-09\\
0.249925	-7.9633897452611e-09\\
0.250025	-7.99133627477692e-09\\
0.250125	-8.02182753692507e-09\\
0.250225	-8.05471632849459e-09\\
0.250325	-8.089851444543e-09\\
0.250425	-8.12708067838191e-09\\
0.2505251	-8.16624608757832e-09\\
0.2506251	-8.20719590547021e-09\\
0.2507251	-8.24976921025369e-09\\
0.2508251	-8.29380578810895e-09\\
0.2509251	-8.33914642339666e-09\\
0.2510251	-8.38562689864294e-09\\
0.2511251	-8.43308499452411e-09\\
0.2512251	-8.4813584898514e-09\\
0.2513251	-8.53028216155549e-09\\
0.2514251	-8.57969278467098e-09\\
0.2515252	-8.62942314444121e-09\\
0.2516252	-8.67931696120079e-09\\
0.2517252	-8.72920604270962e-09\\
0.2518252	-8.77893015625769e-09\\
0.2519252	-8.82832606715919e-09\\
0.2520252	-8.87723453873608e-09\\
0.2521252	-8.9254963323017e-09\\
0.2522252	-8.97295320714413e-09\\
0.2523252	-9.01944992050952e-09\\
0.2524252	-9.06483322758513e-09\\
0.2525253	-9.10894661671246e-09\\
0.2526253	-9.15164833942246e-09\\
0.2527253	-9.19278790862874e-09\\
0.2528253	-9.23222207110977e-09\\
0.2529253	-9.26980757149881e-09\\
0.2530253	-9.3054091522662e-09\\
0.2531253	-9.33888955370155e-09\\
0.2532253	-9.37011851389577e-09\\
0.2533253	-9.39896776872292e-09\\
0.2534253	-9.42531405182199e-09\\
0.2535254	-9.44903252789515e-09\\
0.2536254	-9.47001602777481e-09\\
0.2537254	-9.48814774297357e-09\\
0.2538254	-9.50332239800213e-09\\
0.2539254	-9.51543471504254e-09\\
0.2540254	-9.52438941392906e-09\\
0.2541254	-9.53009121212881e-09\\
0.2542254	-9.53245382472235e-09\\
0.2543254	-9.53139196438404e-09\\
0.2544254	-9.52682834136225e-09\\
0.2545255	-9.51868676763892e-09\\
0.2546255	-9.50690670569372e-09\\
0.2547255	-9.49142299675576e-09\\
0.2548255	-9.47217934116607e-09\\
0.2549255	-9.44912443673835e-09\\
0.2550255	-9.42221197873834e-09\\
0.2551255	-9.39140465986283e-09\\
0.2552255	-9.3566671702186e-09\\
0.2553255	-9.31797319730122e-09\\
0.2554255	-9.27529842597357e-09\\
0.2555256	-9.22862328387339e-09\\
0.2556256	-9.17794692207728e-09\\
0.2557256	-9.12325980012249e-09\\
0.2558256	-9.06456359212104e-09\\
0.2559256	-9.00186596944265e-09\\
0.2560256	-8.93518060069223e-09\\
0.2561256	-8.86452615168734e-09\\
0.2562256	-8.78992728543529e-09\\
0.2563256	-8.71141466211021e-09\\
0.2564256	-8.62902693902983e-09\\
0.2565257	-8.54279912510477e-09\\
0.2566257	-8.45279112195531e-09\\
0.2567257	-8.35904997327808e-09\\
0.2568257	-8.26163632472687e-09\\
0.2569257	-8.16061381898036e-09\\
0.2570257	-8.05605209571791e-09\\
0.2571257	-7.94802779159498e-09\\
0.2572257	-7.83661954021846e-09\\
0.2573257	-7.72191497212182e-09\\
0.2574257	-7.60400171473999e-09\\
0.2575258	-7.48297232088103e-09\\
0.2576258	-7.35893551007846e-09\\
0.2577258	-7.23199087306695e-09\\
0.2578258	-7.10224802462932e-09\\
0.2579258	-6.96981857632142e-09\\
0.2580258	-6.83482013644582e-09\\
0.2581258	-6.69737431002536e-09\\
0.2582258	-6.55760669877637e-09\\
0.2583258	-6.41564390108178e-09\\
0.2584258	-6.27161951196398e-09\\
0.2585259	-6.12566258750749e-09\\
0.2586259	-5.97792173841247e-09\\
0.2587259	-5.8285340621066e-09\\
0.2588259	-5.67764313989091e-09\\
0.2589259	-5.52539454956693e-09\\
0.2590259	-5.37193686540824e-09\\
0.2591259	-5.2174226581318e-09\\
0.2592259	-5.06200349486902e-09\\
0.2593259	-4.90583293913666e-09\\
0.2594259	-4.74906855080747e-09\\
0.259526	-4.59185984510247e-09\\
0.259626	-4.43437840352452e-09\\
0.259726	-4.27677678635377e-09\\
0.259826	-4.11921453858489e-09\\
0.259926	-3.96185220141826e-09\\
0.260026	-3.80485031222915e-09\\
0.260126	-3.64836740453669e-09\\
0.260226	-3.49256600797267e-09\\
0.260326	-3.33760464824999e-09\\
0.260426	-3.18364184713092e-09\\
0.2605261	-3.0308305310181e-09\\
0.2606261	-2.87933933877572e-09\\
0.2607261	-2.7293172459064e-09\\
0.2608261	-2.58091975804329e-09\\
0.2609261	-2.43429837670631e-09\\
0.2610261	-2.28960359926895e-09\\
0.2611261	-2.14698391892465e-09\\
0.2612261	-2.00658482465305e-09\\
0.2613261	-1.86855080118593e-09\\
0.2614261	-1.73302132897283e-09\\
0.2615262	-1.60012869350841e-09\\
0.2616262	-1.47001968508324e-09\\
0.2617262	-1.3428206426853e-09\\
0.2618262	-1.21865802928211e-09\\
0.2619262	-1.0976563033832e-09\\
0.2620262	-9.79935919004143e-10\\
0.2621262	-8.6561332563026e-10\\
0.2622262	-7.54800968180105e-10\\
0.2623262	-6.47606286968627e-10\\
0.2624262	-5.44132717670052e-10\\
0.2625263	-4.44471848300673e-10\\
0.2626263	-3.48733722780796e-10\\
0.2627263	-2.56998987395725e-10\\
0.2628263	-1.69352058874958e-10\\
0.2629263	-8.58733491172364e-11\\
0.2630263	-6.63486515165319e-12\\
0.2631263	6.82931909015424e-11\\
0.2632263	1.38847821870445e-10\\
0.2633263	2.04971435571289e-10\\
0.2634263	2.666117448486e-10\\
0.2635264	3.23730720589391e-10\\
0.2636264	3.76270454218857e-10\\
0.2637264	4.24204353164724e-10\\
0.2638264	4.67503150797314e-10\\
0.2639264	5.0614358572066e-10\\
0.2640264	5.4010790181447e-10\\
0.2641264	5.69384848276526e-10\\
0.2642264	5.93968579665356e-10\\
0.2643264	6.13859655943285e-10\\
0.2644264	6.29064142519731e-10\\
0.2645265	6.39603835868633e-10\\
0.2646265	6.45478242175876e-10\\
0.2647265	6.4672088881627e-10\\
0.2648265	6.43361763433823e-10\\
0.2649265	6.35436659341611e-10\\
0.2650265	6.22987375567224e-10\\
0.2651265	6.06061416898524e-10\\
0.2652265	5.8471189392978e-10\\
0.2653265	5.58997523108158e-10\\
0.2654265	5.28982826780555e-10\\
0.2655266	4.9474769938676e-10\\
0.2656266	4.56347130929882e-10\\
0.2657266	4.13871740623464e-10\\
0.2658266	3.67407174895274e-10\\
0.2659266	3.170441863126e-10\\
0.2660266	2.62878533631318e-10\\
0.2661266	2.05010381845384e-10\\
0.2662266	1.43546102236605e-10\\
0.2663266	7.85947724249064e-11\\
0.2664266	1.02701764189163e-11\\
0.2665267	-6.12978148298817e-11\\
0.2666267	-1.36009469666118e-10\\
0.2667267	-2.13724999019764e-10\\
0.2668267	-2.94316592806822e-10\\
0.2669267	-3.7764843429553e-10\\
0.2670267	-4.63583700053413e-10\\
0.2671267	-5.51982559893854e-10\\
0.2672267	-6.42701176822281e-10\\
0.2673267	-7.35592706981912e-10\\
0.2674267	-8.30507299599104e-10\\
0.2675268	-9.2728202198852e-10\\
0.2676268	-1.02578605516151e-09\\
0.2677268	-1.12585255554172e-09\\
0.2678268	-1.22732264412587e-09\\
0.2679268	-1.3300364347141e-09\\
0.2680268	-1.43383403385274e-09\\
0.2681268	-1.53855154077661e-09\\
0.2682268	-1.64402704735088e-09\\
0.2683268	-1.75009563801259e-09\\
0.2684268	-1.85659238971151e-09\\
0.2685269	-1.9633392155797e-09\\
0.2686269	-2.07019737677594e-09\\
0.2687269	-2.17698888338881e-09\\
0.2688269	-2.28354678181219e-09\\
0.2689269	-2.38970911065071e-09\\
0.2690269	-2.49530990065798e-09\\
0.2691269	-2.60018817467427e-09\\
0.2692269	-2.70418094756379e-09\\
0.2693269	-2.80712922615147e-09\\
0.2694269	-2.90887300915919e-09\\
0.269527	-3.00924195526734e-09\\
0.269627	-3.10810858751252e-09\\
0.269727	-3.20530667005073e-09\\
0.269827	-3.30068616853642e-09\\
0.269927	-3.39409904019499e-09\\
0.270027	-3.48540023375602e-09\\
0.270127	-3.57444768938613e-09\\
0.270227	-3.6611023386212e-09\\
0.270327	-3.74523010429812e-09\\
0.270427	-3.82669690048604e-09\\
0.2705271	-3.90536102268308e-09\\
0.2706271	-3.98112845295901e-09\\
0.2707271	-4.05386360153396e-09\\
0.2708271	-4.12345034669864e-09\\
0.2709271	-4.18977655762405e-09\\
0.2710271	-4.25273609428953e-09\\
0.2711271	-4.31222580741023e-09\\
0.2712271	-4.36814753836397e-09\\
0.2713271	-4.4204101191176e-09\\
0.2714271	-4.46892337215278e-09\\
0.2715272	-4.5135901119016e-09\\
0.2716272	-4.55436499331597e-09\\
0.2717272	-4.59115995558377e-09\\
0.2718272	-4.62390878248251e-09\\
0.2719272	-4.65254924792477e-09\\
0.2720272	-4.6770271158805e-09\\
0.2721272	-4.69729214029878e-09\\
0.2722272	-4.71330106502888e-09\\
0.2723272	-4.72501562374086e-09\\
0.2724272	-4.73240253984543e-09\\
0.2725273	-4.73541901892495e-09\\
0.2726273	-4.73407962072589e-09\\
0.2727273	-4.72835168647265e-09\\
0.2728273	-4.71822789771501e-09\\
0.2729273	-4.70370492533375e-09\\
0.2730273	-4.6847864294568e-09\\
0.2731273	-4.66148305937484e-09\\
0.2732273	-4.63380845345618e-09\\
0.2733273	-4.60178623906106e-09\\
0.2734273	-4.56544203245525e-09\\
0.2735274	-4.52478929188251e-09\\
0.2736274	-4.479907733338e-09\\
0.2737274	-4.43082296251636e-09\\
0.2738274	-4.37758355041701e-09\\
0.2739274	-4.32024705650296e-09\\
0.2740274	-4.25887602861042e-09\\
0.2741274	-4.19353600285775e-09\\
0.2742274	-4.12430150355368e-09\\
0.2743274	-4.05124904310485e-09\\
0.2744274	-3.97446312192263e-09\\
0.2745275	-3.8940093008498e-09\\
0.2746275	-3.8100247247208e-09\\
0.2747275	-3.72258611463647e-09\\
0.2748275	-3.63179492206891e-09\\
0.2749275	-3.5377605860184e-09\\
0.2750275	-3.44059353291587e-09\\
0.2751275	-3.34041017652473e-09\\
0.2752275	-3.23732991784191e-09\\
0.2753275	-3.13147914499809e-09\\
0.2754275	-3.02298323315738e-09\\
0.2755276	-2.91195068318878e-09\\
0.2756276	-2.79856336421425e-09\\
0.2757276	-2.6829379512481e-09\\
0.2758276	-2.56521376655458e-09\\
0.2759276	-2.4455351189174e-09\\
0.2760276	-2.32404830353463e-09\\
0.2761276	-2.20090460191285e-09\\
0.2762276	-2.07625428176047e-09\\
0.2763276	-1.95025059688021e-09\\
0.2764276	-1.82304978706083e-09\\
0.2765277	-1.69478211710493e-09\\
0.2766277	-1.56565950058351e-09\\
0.2767277	-1.43581439150002e-09\\
0.2768277	-1.30540897247348e-09\\
0.2769277	-1.17460441155494e-09\\
0.2770277	-1.04356286211423e-09\\
0.2771277	-9.12447462725862e-10\\
0.2772277	-7.81420337053996e-10\\
0.2773277	-6.50645593736629e-10\\
0.2774277	-5.20284326268802e-10\\
0.2775278	-3.90470372685586e-10\\
0.2776278	-2.61422037883648e-10\\
0.2777278	-1.33271365417956e-10\\
0.2778278	-6.17738701128791e-12\\
0.2779278	1.19702881353503e-10\\
0.2780278	2.442114395557e-10\\
0.2781278	3.67194303459586e-10\\
0.2782278	4.88499505038443e-10\\
0.2783278	6.07977092499343e-10\\
0.2784278	7.25477130409005e-10\\
0.2785279	8.4088741398234e-10\\
0.2786279	9.54002871247412e-10\\
0.2787279	1.06471507421406e-09\\
0.2788279	1.17288615418122e-09\\
0.2789279	1.27838525945083e-09\\
0.2790279	1.38108055545938e-09\\
0.2791279	1.48084722491034e-09\\
0.2792279	1.57756246790781e-09\\
0.2793279	1.67110950209097e-09\\
0.2794279	1.76137456276956e-09\\
0.279528	1.84827930193592e-09\\
0.279628	1.93165447359429e-09\\
0.279728	2.01143148736989e-09\\
0.279828	2.08751565064891e-09\\
0.279928	2.15981528918117e-09\\
0.280028	2.22824374722202e-09\\
0.280128	2.29272238767499e-09\\
0.280228	2.35317459223565e-09\\
0.280328	2.4095297615365e-09\\
0.280428	2.46172331529298e-09\\
0.2805281	2.50973300420885e-09\\
0.2806281	2.55343096760975e-09\\
0.2807281	2.59280769307412e-09\\
0.2808281	2.62781767815066e-09\\
0.2809281	2.6584274402175e-09\\
0.2810281	2.68460351663493e-09\\
0.2811281	2.70632246489917e-09\\
0.2812281	2.72356386279719e-09\\
0.2813281	2.73631530856285e-09\\
0.2814281	2.74456942103409e-09\\
0.2815282	2.74836431143288e-09\\
0.2816282	2.74762602734409e-09\\
0.2817282	2.74240239438133e-09\\
0.2818282	2.73271211541593e-09\\
0.2819282	2.71857591472736e-09\\
0.2820282	2.70002253816743e-09\\
0.2821282	2.67708475332599e-09\\
0.2822282	2.64980334969775e-09\\
0.2823282	2.61822313885016e-09\\
0.2824282	2.58239595459298e-09\\
0.2825283	2.54241855192453e-09\\
0.2826283	2.49827037030963e-09\\
0.2827283	2.45005985479171e-09\\
0.2828283	2.39785992989015e-09\\
0.2829283	2.34174854323213e-09\\
0.2830283	2.28180866572948e-09\\
0.2831283	2.21812829175703e-09\\
0.2832283	2.1508004393322e-09\\
0.2833283	2.07992315029587e-09\\
0.2834283	2.00559849049466e-09\\
0.2835284	1.92797916511301e-09\\
0.2836284	1.8470844466645e-09\\
0.2837284	1.76307470401914e-09\\
0.2838284	1.67607110092078e-09\\
0.2839284	1.58619482605073e-09\\
0.2840284	1.49357409321834e-09\\
0.2841284	1.39833914155278e-09\\
0.2842284	1.30062223569644e-09\\
0.2843284	1.20056266599955e-09\\
0.2844284	1.09829974871644e-09\\
0.2845285	9.94026470608144e-10\\
0.2846285	8.87787332579117e-10\\
0.2847285	7.79780956542559e-10\\
0.2848285	6.70156764386304e-10\\
0.2849285	5.59067204904968e-10\\
0.2850285	4.46665754005125e-10\\
0.2851285	3.33108914911618e-10\\
0.2852285	2.18552218375735e-10\\
0.2853285	1.03155222884567e-10\\
0.2854285	-1.29234851277743e-11\\
0.2855286	-1.29469278983183e-10\\
0.2856286	-2.46431083505034e-10\\
0.2857286	-3.635936677438e-10\\
0.2858286	-4.80795331306659e-10\\
0.2859286	-5.97876344774646e-10\\
0.2860286	-7.14675949482194e-10\\
0.2861286	-8.31033357294998e-10\\
0.2862286	-9.46789750386188e-10\\
0.2863286	-1.06178728101077e-09\\
0.2864286	-1.17586807127872e-09\\
0.2865287	-1.28881646718677e-09\\
0.2866287	-1.40059652677634e-09\\
0.2867287	-1.51099802518737e-09\\
0.2868287	-1.61986896161385e-09\\
0.2869287	-1.72706230394374e-09\\
0.2870287	-1.83242998852176e-09\\
0.2871287	-1.93582991991046e-09\\
0.2872287	-2.03712197064932e-09\\
0.2873287	-2.13616698101239e-09\\
0.2874287	-2.23283075876402e-09\\
0.2875288	-2.32691520382056e-09\\
0.2876288	-2.41842227251326e-09\\
0.2877288	-2.50717133005141e-09\\
0.2878288	-2.59302805164818e-09\\
0.2879288	-2.67588807875825e-09\\
0.2880288	-2.75562701882288e-09\\
0.2881288	-2.83215044501265e-09\\
0.2882288	-2.9053438959688e-09\\
0.2883288	-2.97511287554221e-09\\
0.2884288	-3.04136285253075e-09\\
0.2885289	-3.10392882821803e-09\\
0.2886289	-3.1628664844073e-09\\
0.2887289	-3.21803132677606e-09\\
0.2888289	-3.26934868120863e-09\\
0.2889289	-3.31674383719374e-09\\
0.2890289	-3.36015204754996e-09\\
0.2891289	-3.39952852814944e-09\\
0.2892289	-3.43479845763959e-09\\
0.2893289	-3.46592697716247e-09\\
0.2894289	-3.49286919007246e-09\\
0.289529	-3.51551371003502e-09\\
0.289629	-3.53397783849054e-09\\
0.289729	-3.5481607357836e-09\\
0.289829	-3.55804735121284e-09\\
0.289929	-3.56362259484567e-09\\
0.290029	-3.56487133722323e-09\\
0.290129	-3.56179840906308e-09\\
0.290229	-3.55439860096035e-09\\
0.290329	-3.54269666308576e-09\\
0.290429	-3.52669730488242e-09\\
0.2905291	-3.50634222414715e-09\\
0.2906291	-3.48182130834518e-09\\
0.2907291	-3.45308684767204e-09\\
0.2908291	-3.42016338553712e-09\\
0.2909291	-3.3831154230702e-09\\
0.2910291	-3.34197741880442e-09\\
0.2911291	-3.29681378835694e-09\\
0.2912291	-3.24766890410739e-09\\
0.2913291	-3.19461709487374e-09\\
0.2914291	-3.13773264558602e-09\\
0.2915292	-3.07699976761142e-09\\
0.2916292	-3.01267197868506e-09\\
0.2917292	-2.94474413199701e-09\\
0.2918292	-2.87331033356936e-09\\
0.2919292	-2.79846464386866e-09\\
0.2920292	-2.7203110774651e-09\\
0.2921292	-2.63894360268926e-09\\
0.2922292	-2.5544661412866e-09\\
0.2923292	-2.46700256806925e-09\\
0.2924292	-2.3766667105653e-09\\
0.2925293	-2.28347467756884e-09\\
0.2926293	-2.18774474524572e-09\\
0.2927293	-2.0895137176309e-09\\
0.2928293	-1.98889522951046e-09\\
0.2929293	-1.88604286659426e-09\\
0.2930293	-1.78107016515029e-09\\
0.2931293	-1.67413061163601e-09\\
0.2932293	-1.56535764232658e-09\\
0.2933293	-1.45490464294149e-09\\
0.2934293	-1.34290494826699e-09\\
0.2935294	-1.22940589928331e-09\\
0.2936294	-1.11476174917372e-09\\
0.2937294	-9.99030591883237e-10\\
0.2938294	-8.82345554592782e-10\\
0.2939294	-7.64879711625928e-10\\
0.2940294	-6.46776084054978e-10\\
0.2941294	-5.28187639305328e-10\\
0.2942294	-4.09277290756445e-10\\
0.2943294	-2.90197897340086e-10\\
0.2944294	-1.71102263135791e-10\\
0.2945295	-5.20282431948291e-11\\
0.2946295	6.66326162711905e-11\\
0.2947295	1.84841644881487e-10\\
0.2948295	3.0244626212723e-10\\
0.2949295	4.19293944419005e-10\\
0.2950295	5.35222225509784e-10\\
0.2951295	6.50098696919823e-10\\
0.2952295	7.63761008365252e-10\\
0.2953295	8.76066868189453e-10\\
0.2954295	9.86864043797407e-10\\
0.2955296	1.09614494111095e-09\\
0.2956296	1.20350929993744e-09\\
0.2957296	1.3089386435034e-09\\
0.2958296	1.41231097992987e-09\\
0.2959296	1.51349437861959e-09\\
0.2960296	1.61233697071057e-09\\
0.2961296	1.70873694953303e-09\\
0.2962296	1.80255257106893e-09\\
0.2963296	1.89367215441577e-09\\
0.2964296	1.98198408225235e-09\\
0.2965297	2.06749185780028e-09\\
0.2966297	2.14984497291705e-09\\
0.2967297	2.22904597537521e-09\\
0.2968297	2.30501350697855e-09\\
0.2969297	2.37764627549605e-09\\
0.2970297	2.44684305514853e-09\\
0.2971297	2.51253268709898e-09\\
0.2972297	2.57462407994671e-09\\
0.2973297	2.63304621022404e-09\\
0.2974297	2.68771812289741e-09\\
0.2975298	2.73872532077296e-09\\
0.2976298	2.78571539209821e-09\\
0.2977298	2.82877280567288e-09\\
0.2978298	2.8678568853497e-09\\
0.2979298	2.90289702597384e-09\\
0.2980298	2.93387269390505e-09\\
0.2981298	2.96074342754437e-09\\
0.2982298	2.9834888378637e-09\\
0.2983298	3.00205860893952e-09\\
0.2984298	3.01646249849063e-09\\
0.2985299	3.02682898216692e-09\\
0.2986299	3.03283195794327e-09\\
0.2987299	3.03464878265476e-09\\
0.2988299	3.03225951411588e-09\\
0.2989299	3.02567428652808e-09\\
0.2990299	3.01492331104135e-09\\
0.2991299	3.00001687631845e-09\\
0.2992299	2.98097534910349e-09\\
0.2993299	2.95784917479492e-09\\
0.2994299	2.93064887802194e-09\\
0.29953	2.89960695689466e-09\\
0.29963	2.86442169145249e-09\\
0.29973	2.82531436945258e-09\\
0.29983	2.7823558383777e-09\\
0.29993	2.73558702788732e-09\\
0.30003	2.68507895041867e-09\\
0.30013	2.63092270179261e-09\\
0.30023	2.57317946182375e-09\\
0.30033	2.51194049493523e-09\\
0.30043	2.44727715077638e-09\\
0.3005301	2.37947708166755e-09\\
0.3006301	2.30827087031909e-09\\
0.3007301	2.23393485991428e-09\\
0.3008301	2.15657074737164e-09\\
0.3009301	2.07628031799616e-09\\
0.3010301	1.99318544612482e-09\\
0.3011301	1.90740809577584e-09\\
0.3012301	1.81905032130317e-09\\
0.3013301	1.72823426805575e-09\\
0.3014301	1.63509217304059e-09\\
0.3015302	1.53994806286506e-09\\
0.3016302	1.44253258027253e-09\\
0.3017302	1.34318833615131e-09\\
0.3018302	1.24202794096484e-09\\
0.3019302	1.1392041002239e-09\\
0.3020302	1.03485961517799e-09\\
0.3021302	9.29127383512413e-10\\
0.3022302	8.22150400050141e-10\\
0.3023302	7.14081757457795e-10\\
0.3024302	6.05064646957689e-10\\
0.3025303	4.95450784492084e-10\\
0.3026303	3.84968454601977e-10\\
0.3027303	2.7397784253628e-10\\
0.3028303	1.62632541056524e-10\\
0.3029303	5.10762451132885e-11\\
0.3030303	-6.05472474114823e-11\\
0.3031303	-1.72074034957907e-10\\
0.3032303	-2.83370111535775e-10\\
0.3033303	-3.94271365966614e-10\\
0.3034303	-5.04643581121398e-10\\
0.3035304	-6.14095934581894e-10\\
0.3036304	-7.22965107094222e-10\\
0.3037304	-8.30851930936088e-10\\
0.3038304	-9.37621757330345e-10\\
0.3039304	-1.04314982765413e-09\\
0.3040304	-1.14727127264279e-09\\
0.3041304	-1.24986111158931e-09\\
0.3042304	-1.35078425153825e-09\\
0.3043304	-1.44990548647353e-09\\
0.3044304	-1.54708949650025e-09\\
0.3045305	-1.64196482585525e-09\\
0.3046305	-1.73490593071409e-09\\
0.3047305	-1.82554314372379e-09\\
0.3048305	-1.91376068106762e-09\\
0.3049305	-1.99943264087596e-09\\
0.3050305	-2.08245300237309e-09\\
0.3051305	-2.16271562501908e-09\\
0.3052305	-2.24012424764616e-09\\
0.3053305	-2.31456248758771e-09\\
0.3054305	-2.38594383980258e-09\\
0.3055306	-2.45389457040179e-09\\
0.3056306	-2.51887993942847e-09\\
0.3057306	-2.58054814556202e-09\\
0.3058306	-2.63882218525195e-09\\
0.3059306	-2.69361492809877e-09\\
0.3060306	-2.74486911594021e-09\\
0.3061306	-2.79251736193346e-09\\
0.3062306	-2.83649214962705e-09\\
0.3063306	-2.87675583203033e-09\\
0.3064306	-2.91324063067338e-09\\
0.3065307	-2.94560876208297e-09\\
0.3066307	-2.97442955323296e-09\\
0.3067307	-2.99934730860806e-09\\
0.3068307	-3.020363714169e-09\\
0.3069307	-3.03743031960386e-09\\
0.3070307	-3.0505385373504e-09\\
0.3071307	-3.05967964161071e-09\\
0.3072307	-3.06483476735874e-09\\
0.3073307	-3.06600490934287e-09\\
0.3074307	-3.06320092107865e-09\\
0.3075308	-3.05609906197888e-09\\
0.3076308	-3.04536624114224e-09\\
0.3077308	-3.03069097340173e-09\\
0.3078308	-3.01209353700344e-09\\
0.3079308	-2.98962406382581e-09\\
0.3080308	-2.96329253833351e-09\\
0.3081308	-2.93315879652179e-09\\
0.3082308	-2.89925252485324e-09\\
0.3083308	-2.86162325919021e-09\\
0.3084308	-2.82033038371508e-09\\
0.3085309	-2.77508214706537e-09\\
0.3086309	-2.7266468265458e-09\\
0.3087309	-2.67473510665271e-09\\
0.3088309	-2.61942575377866e-09\\
0.3089309	-2.56079737713829e-09\\
0.3090309	-2.4989184276473e-09\\
0.3091309	-2.43388719679229e-09\\
0.3092309	-2.36580181549258e-09\\
0.3093309	-2.29474025295554e-09\\
0.3094309	-2.2208003155236e-09\\
0.309531	-2.14371003049488e-09\\
0.309631	-2.06433312043851e-09\\
0.309731	-1.98240024289522e-09\\
0.309831	-1.89801854088458e-09\\
0.309931	-1.81131498867472e-09\\
0.310031	-1.72238639057899e-09\\
0.310131	-1.63136937974766e-09\\
0.310231	-1.5383804169482e-09\\
0.310331	-1.44355578934114e-09\\
0.310431	-1.34700160924486e-09\\
0.3105311	-1.24846330361756e-09\\
0.3106311	-1.1488744298524e-09\\
0.3107311	-1.04796325451375e-09\\
0.3108311	-9.45875177853193e-10\\
0.3109311	-8.42725418978393e-10\\
0.3110311	-7.38679014566231e-10\\
0.3111311	-6.33850817568188e-10\\
0.3112311	-5.28395495907112e-10\\
0.3113311	-4.22447531164087e-10\\
0.3114311	-3.16151217258372e-10\\
0.3115312	-2.09206821025401e-10\\
0.3116312	-1.02632459788065e-10\\
0.3117312	3.86853455596376e-12\\
0.3118312	1.10162632327675e-10\\
0.3119312	2.16096498256203e-10\\
0.3120312	3.2153699285299e-10\\
0.3121312	4.26341173799798e-10\\
0.3122312	5.30386297342442e-10\\
0.3123312	6.33509819695525e-10\\
0.3124312	7.35599398456649e-10\\
0.3125313	8.36992680687198e-10\\
0.3126313	9.36601903811762e-10\\
0.3127313	1.03477140709163e-09\\
0.3128313	1.13136966819782e-09\\
0.3129313	1.22628537339907e-09\\
0.3130313	1.31938741903729e-09\\
0.3131313	1.41055491300923e-09\\
0.3132313	1.49966717625889e-09\\
0.3133313	1.58661374428009e-09\\
0.3134313	1.67127436862983e-09\\
0.3135314	1.75407757216913e-09\\
0.3136314	1.83387047511867e-09\\
0.3137314	1.91106803124524e-09\\
0.3138314	1.98557087191394e-09\\
0.3139314	2.05728985227325e-09\\
0.3140314	2.12613605282878e-09\\
0.3141314	2.1920107810328e-09\\
0.3142314	2.25484557287654e-09\\
0.3143314	2.31455219450044e-09\\
0.3144314	2.37105264381228e-09\\
0.3145315	2.42484950462227e-09\\
0.3146315	2.47475889310432e-09\\
0.3147315	2.52127954276051e-09\\
0.3148315	2.5643343952039e-09\\
0.3149315	2.60389663207154e-09\\
0.3150315	2.63990967670941e-09\\
0.3151315	2.67232719586765e-09\\
0.3152315	2.70112310140942e-09\\
0.3153315	2.72625155202859e-09\\
0.3154315	2.74769695498155e-09\\
0.3155316	2.76602937959967e-09\\
0.3156316	2.78004760603231e-09\\
0.3157316	2.79031755070754e-09\\
0.3158316	2.79683463292332e-09\\
0.3159316	2.79961452936887e-09\\
0.3160316	2.79864317592774e-09\\
0.3161316	2.79392676948999e-09\\
0.3162316	2.78549176977824e-09\\
0.3163316	2.77333490118512e-09\\
0.3164316	2.75750315462498e-09\\
0.3165317	2.73865776622716e-09\\
0.3166317	2.71554337066623e-09\\
0.3167317	2.6888267256298e-09\\
0.3168317	2.65857590714057e-09\\
0.3169317	2.6248092671821e-09\\
0.3170317	2.58759543562555e-09\\
0.3171317	2.54698332216618e-09\\
0.3172317	2.50303211827843e-09\\
0.3173317	2.45580129917824e-09\\
0.3174317	2.4053706258006e-09\\
0.3175318	2.35251645754762e-09\\
0.3176318	2.29590196208921e-09\\
0.3177318	2.23631867023215e-09\\
0.3178318	2.17382750612987e-09\\
0.3179318	2.10854968974575e-09\\
0.3180318	2.04054673891338e-09\\
0.3181318	1.96993047140681e-09\\
0.3182318	1.89680300702607e-09\\
0.3183318	1.82124676969968e-09\\
0.3184318	1.74338448959639e-09\\
0.3185319	1.66407190081472e-09\\
0.3186319	1.58191283342292e-09\\
0.3187319	1.49776781637007e-09\\
0.3188319	1.41176082652337e-09\\
0.3189319	1.32399615777373e-09\\
0.3190319	1.23460842323701e-09\\
0.3191319	1.14372255746598e-09\\
0.3192319	1.05138381868053e-09\\
0.3193319	9.57907791012234e-10\\
0.3194319	8.63240386760296e-10\\
0.319532	7.68351282210626e-10\\
0.319632	6.71746510616553e-10\\
0.319732	5.7449013824432e-10\\
0.319832	4.76529413295814e-10\\
0.319932	3.78111923663976e-10\\
0.320032	2.79385599283479e-10\\
0.320132	1.80398714495406e-10\\
0.320232	8.12998904276481e-11\\
0.320332	-1.77619026071632e-11\\
0.320432	-1.16537342698375e-10\\
0.3205321	-2.14287905266039e-10\\
0.3206321	-3.12334443967783e-10\\
0.3207321	-4.09944479470726e-10\\
0.3208321	-5.06767258665023e-10\\
0.3209321	-6.02851664533357e-10\\
0.3210321	-6.97946213645042e-10\\
0.3211321	-7.91899053625857e-10\\
0.3212321	-8.84757960617488e-10\\
0.3213321	-9.76270336716397e-10\\
0.3214321	-1.06638320739615e-09\\
0.3215322	-1.15398393096179e-09\\
0.3216322	-1.24083001412911e-09\\
0.3217322	-1.32591532822522e-09\\
0.3218322	-1.40908536573437e-09\\
0.3219322	-1.4902852293748e-09\\
0.3220322	-1.56935962941572e-09\\
0.3221322	-1.64635288098292e-09\\
0.3222322	-1.72090890133666e-09\\
0.3223322	-1.79307120714706e-09\\
0.3224322	-1.86288291173262e-09\\
0.3225323	-1.92895159902667e-09\\
0.3226323	-1.9934819192531e-09\\
0.3227323	-2.05518847202046e-09\\
0.3228323	-2.11411272896407e-09\\
0.3229323	-2.17019574433432e-09\\
0.3230323	-2.22327815213769e-09\\
0.3231323	-2.2734001632561e-09\\
0.3232323	-2.3203015625443e-09\\
0.3233323	-2.36422170591632e-09\\
0.3234323	-2.40489951740272e-09\\
0.3235324	-2.4411567333976e-09\\
0.3236324	-2.47525641393201e-09\\
0.3237324	-2.50602785110543e-09\\
0.3238324	-2.53350821121278e-09\\
0.3239324	-2.55753421368549e-09\\
0.3240324	-2.57824212803347e-09\\
0.3241324	-2.59546777077111e-09\\
0.3242324	-2.60914650232399e-09\\
0.3243324	-2.61951322391164e-09\\
0.3244324	-2.62630237441416e-09\\
0.3245325	-2.6284433242844e-09\\
0.3246325	-2.62836964055295e-09\\
0.3247325	-2.62481882945454e-09\\
0.3248325	-2.61772344825648e-09\\
0.3249325	-2.60731557589563e-09\\
0.3250325	-2.59342680971367e-09\\
0.3251325	-2.57608826217701e-09\\
0.3252325	-2.55543055757475e-09\\
0.3253325	-2.53148382869253e-09\\
0.3254325	-2.50427771346671e-09\\
0.3255326	-2.47244222148544e-09\\
0.3256326	-2.43879441362867e-09\\
0.3257326	-2.40197305823116e-09\\
0.3258326	-2.36200577942156e-09\\
0.3259326	-2.31901968941248e-09\\
0.3260326	-2.27304138502973e-09\\
0.3261326	-2.22419694419906e-09\\
0.3262326	-2.17251192243124e-09\\
0.3263326	-2.11811134926784e-09\\
0.3264326	-2.06101972471903e-09\\
0.3265327	-1.99986019265943e-09\\
0.3266327	-1.93764724706759e-09\\
0.3267327	-1.87291345946284e-09\\
0.3268327	-1.80588117499481e-09\\
0.3269327	-1.73667219107175e-09\\
0.3270327	-1.66530775364961e-09\\
0.3271327	-1.59180855349646e-09\\
0.3272327	-1.51639472243308e-09\\
0.3273327	-1.43918582954714e-09\\
0.3274327	-1.36020087738861e-09\\
0.3275328	-1.2781480937391e-09\\
0.3276328	-1.19595436426075e-09\\
0.3277328	-1.11243806245496e-09\\
0.3278328	-1.02761588554794e-09\\
0.3279328	-9.41603944809453e-10\\
0.3280328	-8.54517761597378e-10\\
0.3281328	-7.66572263361316e-10\\
0.3282328	-6.77781779646977e-10\\
0.3283328	-5.88260038053877e-10\\
0.3284328	-4.98120160172654e-10\\
0.3285329	-4.05946823775196e-10\\
0.3286329	-3.14995355575876e-10\\
0.3287329	-2.23861349787299e-10\\
0.3288329	-1.32655462207732e-10\\
0.3289329	-4.14877219248971e-11\\
0.3290329	4.96324729118805e-11\\
0.3291329	1.40396359331712e-10\\
0.3292329	2.30795813824618e-10\\
0.3293329	3.20723356642449e-10\\
0.3294329	4.09972156128123e-10\\
0.329533	5.002903424829e-10\\
0.329633	5.87976931580862e-10\\
0.329733	6.74668310119922e-10\\
0.329833	7.60360283341756e-10\\
0.329933	8.44749326655789e-10\\
0.330033	9.27932590143959e-10\\
0.330133	1.00960790308272e-09\\
0.330233	1.08977377850511e-09\\
0.330333	1.16832941778712e-09\\
0.330433	1.24517471526554e-09\\
0.3305331	1.32210053430575e-09\\
0.3306331	1.39524176875381e-09\\
0.3307331	1.46627665112266e-09\\
0.3308331	1.53530789505328e-09\\
0.3309331	1.6021389306822e-09\\
0.3310331	1.66667390943253e-09\\
0.3311331	1.72881770884217e-09\\
0.3312331	1.78857593742516e-09\\
0.3313331	1.84575493955603e-09\\
0.3314331	1.90046180039162e-09\\
0.3315332	1.95454075492414e-09\\
0.3316332	2.00384277543056e-09\\
0.3317332	2.05039851419586e-09\\
0.3318332	2.09411807010283e-09\\
0.3319332	2.13491230789246e-09\\
0.3320332	2.17279286328078e-09\\
0.3321332	2.20777214810041e-09\\
0.3322332	2.23966335547738e-09\\
0.3323332	2.26858046504e-09\\
0.3324332	2.29443824816234e-09\\
0.3325333	2.31934571224337e-09\\
0.3326333	2.33884868112689e-09\\
0.3327333	2.35534155867272e-09\\
0.3328333	2.36864233714636e-09\\
0.3329333	2.37866982728172e-09\\
0.3330333	2.38564366372282e-09\\
0.3331333	2.38938431051248e-09\\
0.3332333	2.38991306658988e-09\\
0.3333333	2.38725207136263e-09\\
0.3334333	2.38152431027576e-09\\
0.3335334	2.37491577851775e-09\\
0.3336334	2.36284439881903e-09\\
0.3337334	2.34768046822469e-09\\
0.3338334	2.32945041378296e-09\\
0.3339334	2.30808153705028e-09\\
0.3340334	2.28380201988952e-09\\
0.3341334	2.25654093031446e-09\\
0.3342334	2.22622822835237e-09\\
0.3343334	2.19309477196179e-09\\
0.3344334	2.15717232298671e-09\\
0.3345335	2.12093695004071e-09\\
0.3346335	2.07945429133588e-09\\
0.3347335	2.03528354386585e-09\\
0.3348335	1.98856014615855e-09\\
0.3349335	1.93932047094163e-09\\
0.3350335	1.88750183130891e-09\\
0.3351335	1.83344248695268e-09\\
0.3352335	1.77698165040379e-09\\
0.3353335	1.71825949333452e-09\\
0.3354335	1.65741715288979e-09\\
0.3355336	1.59723478631131e-09\\
0.3356336	1.53239962131816e-09\\
0.3357336	1.46567368574559e-09\\
0.3358336	1.39710204376706e-09\\
0.3359336	1.32683075732222e-09\\
0.3360336	1.25500689267509e-09\\
0.3361336	1.18167852704609e-09\\
0.3362336	1.1068947552618e-09\\
0.3363336	1.03080569645829e-09\\
0.3364336	9.53562500834087e-10\\
0.3365337	8.78164422119507e-10\\
0.3366337	7.98792289957185e-10\\
0.3367337	7.18525880779288e-10\\
0.3368337	6.37620538347652e-10\\
0.3369337	5.55932671855059e-10\\
0.3370337	4.73719762905047e-10\\
0.3371337	3.91140372577684e-10\\
0.3372337	3.08154148504312e-10\\
0.3373337	2.24921832005413e-10\\
0.3374337	1.41605265271977e-10\\
0.3375338	6.1438866283879e-11\\
0.3376338	-2.17329094619811e-11\\
0.3377338	-1.04696565460027e-10\\
0.3378338	-1.8738578223275e-10\\
0.3379338	-2.69633102860115e-10\\
0.3380338	-3.51269925529103e-10\\
0.3381338	-4.32226496031003e-10\\
0.3382338	-5.12531900220229e-10\\
0.3383338	-5.91914056425988e-10\\
0.3384338	-6.70399707810202e-10\\
0.3385339	-7.44402072996372e-10\\
0.3386339	-8.20545170190037e-10\\
0.3387339	-8.95464686510434e-10\\
0.3388339	-9.69082585892418e-10\\
0.3389339	-1.04121961818978e-09\\
0.3390339	-1.11179531124124e-09\\
0.3391339	-1.18072796288784e-09\\
0.3392339	-1.24793463295237e-09\\
0.3393339	-1.31333113515561e-09\\
0.3394339	-1.37693202900534e-09\\
0.339534	-1.434779763086e-09\\
0.339634	-1.49430119609414e-09\\
0.339734	-1.55166290197045e-09\\
0.339834	-1.60687433974032e-09\\
0.339934	-1.65984367280382e-09\\
0.340034	-1.71047776048642e-09\\
0.340134	-1.75868214956312e-09\\
0.340234	-1.80456106572473e-09\\
0.340334	-1.84781740497117e-09\\
0.340434	-1.8886527249853e-09\\
0.3405341	-1.92291900922372e-09\\
0.3406341	-1.95838274385666e-09\\
0.3407341	-1.99132181187516e-09\\
0.3408341	-2.02143232873705e-09\\
0.3409341	-2.04880902754495e-09\\
0.3410341	-2.07344525005669e-09\\
0.3411341	-2.09533293766701e-09\\
0.3412341	-2.11436262232279e-09\\
0.3413341	-2.13052341738218e-09\\
0.3414341	-2.14390300843556e-09\\
0.3415342	-2.15004184418193e-09\\
0.3416342	-2.15768540821427e-09\\
0.3417342	-2.16240194815845e-09\\
0.3418342	-2.1642733439452e-09\\
0.3419342	-2.16318000088939e-09\\
0.3420342	-2.15940084016855e-09\\
0.3421342	-2.15261328919992e-09\\
0.3422342	-2.1430932720026e-09\\
0.3423342	-2.13081519947681e-09\\
0.3424342	-2.11565195962265e-09\\
0.3425343	-2.09330993064434e-09\\
0.3426343	-2.07265571961607e-09\\
0.3427343	-2.04932553674057e-09\\
0.3428343	-2.02338607791725e-09\\
0.3429343	-1.99490246635969e-09\\
0.3430343	-1.96373824245938e-09\\
0.3431343	-1.93015535357066e-09\\
0.3432343	-1.89411414375882e-09\\
0.3433343	-1.85567334345753e-09\\
0.3434343	-1.81479005908796e-09\\
0.3435344	-1.76681249916893e-09\\
0.3436344	-1.7214734605948e-09\\
0.3437344	-1.6739531596065e-09\\
0.3438344	-1.62430209971004e-09\\
0.3439344	-1.57266910746453e-09\\
0.3440344	-1.51920132170681e-09\\
0.3441344	-1.46374418271897e-09\\
0.3442344	-1.40654142130605e-09\\
0.3443344	-1.34763504781384e-09\\
0.3444344	-1.28716534110156e-09\\
0.3445345	-1.21979657392857e-09\\
0.3446345	-1.15627593540451e-09\\
0.3447345	-1.09140203394243e-09\\
0.3448345	-1.02520810886008e-09\\
0.3449345	-9.57925611716627e-10\\
0.3450345	-8.89484194889401e-10\\
0.3451345	-8.20111700036159e-10\\
0.3452345	-7.49834146500367e-10\\
0.3453345	-6.78675719651568e-10\\
0.3454345	-6.06958759155694e-10\\
0.3455346	-5.28836059120496e-10\\
0.3456346	-4.55920747484674e-10\\
0.3457346	-3.82602444307979e-10\\
0.3458346	-3.08995985746598e-10\\
0.3459346	-2.35114302488613e-10\\
0.3460346	-1.61268407575436e-10\\
0.3461346	-8.7267384180669e-11\\
0.3462346	-1.34183732790247e-11\\
0.3463346	6.02734387487725e-11\\
0.3464346	1.33604832561111e-10\\
0.3465347	2.12663929229976e-10\\
0.3466347	2.85216549895981e-10\\
0.3467347	3.57111326204657e-10\\
0.3468347	4.28453035229805e-10\\
0.3469347	4.98948484571065e-10\\
0.3470347	5.68606525288419e-10\\
0.3471347	6.3723806489863e-10\\
0.3472347	7.04956080458208e-10\\
0.3473347	7.71475631733931e-10\\
0.3474347	8.3671387442508e-10\\
0.3475348	9.0733129900568e-10\\
0.3476348	9.6991376300081e-10\\
0.3477348	1.03117941935796e-09\\
0.3478348	1.09085394371058e-09\\
0.3479348	1.14886517503692e-09\\
0.3480348	1.20524312941154e-09\\
0.3481348	1.2598200137927e-09\\
0.3482348	1.31253023991935e-09\\
0.3483348	1.36341043830412e-09\\
0.3484348	1.41229947226231e-09\\
0.3485349	1.46616380533614e-09\\
0.3486349	1.51104636840083e-09\\
0.3487349	1.55376824253996e-09\\
0.3488349	1.59427736691604e-09\\
0.3489349	1.63252398510063e-09\\
0.3490349	1.66856065962035e-09\\
0.3491349	1.70224228665607e-09\\
0.3492349	1.73382611074131e-09\\
0.3493349	1.76277173965106e-09\\
0.3494349	1.78864115927873e-09\\
0.349535	1.82034269924711e-09\\
0.349635	1.84220908509064e-09\\
0.349735	1.86110000923018e-09\\
0.349835	1.87738712360832e-09\\
0.349935	1.89144453431412e-09\\
0.350035	1.90364881701731e-09\\
0.350135	1.91237903256444e-09\\
0.350235	1.91801674259935e-09\\
0.350335	1.92194602532875e-09\\
0.350435	1.9235534913559e-09\\
0.3505351	1.93032766719529e-09\\
0.3506351	1.92651919699921e-09\\
0.3507351	1.91956449175786e-09\\
0.3508351	1.91086045457288e-09\\
0.3509351	1.898806601616e-09\\
0.3510351	1.88480507851463e-09\\
0.3511351	1.86826067687249e-09\\
0.3512351	1.84958085088588e-09\\
0.3513351	1.8281757340349e-09\\
0.3514351	1.80445815592818e-09\\
0.3515352	1.78753776617828e-09\\
0.3516352	1.75850635504335e-09\\
0.3517352	1.72841774010243e-09\\
0.3518352	1.6956957082435e-09\\
0.3519352	1.66076682806269e-09\\
0.3520352	1.62406046725296e-09\\
0.3521352	1.58400881011917e-09\\
0.3522352	1.5430468752188e-09\\
0.3523352	1.50061253306558e-09\\
0.3524352	1.45514652400003e-09\\
0.3525353	1.41742330955633e-09\\
0.3526353	1.36929383964385e-09\\
0.3527353	1.319472773498e-09\\
0.3528353	1.26741251409187e-09\\
0.3529353	1.21456842495939e-09\\
0.3530353	1.1593988487361e-09\\
0.3531353	1.10336512566402e-09\\
0.3532353	1.04593161236452e-09\\
0.3533353	9.8656570058703e-10\\
0.3534353	9.26737836137951e-10\\
0.3535354	8.75933921275288e-10\\
0.3536354	8.14676528205445e-10\\
0.3537354	7.51387516413124e-10\\
0.3538354	6.87549742514473e-10\\
0.3539354	6.23649213448878e-10\\
0.3540354	5.58175106112526e-10\\
0.3541354	4.92619787050571e-10\\
0.3542354	4.26478832279056e-10\\
0.3543354	3.6025104724617e-10\\
0.3544354	2.93438486849486e-10\\
0.3545355	2.3728825089684e-10\\
0.3546355	1.69901088253414e-10\\
0.3547355	1.03455527803534e-10\\
0.3548355	3.64668286672984e-11\\
0.3549355	-3.05463984765415e-11\\
0.3550355	-9.70621705779739e-11\\
0.3551355	-1.63555111356511e-10\\
0.3552355	-2.29496430257447e-10\\
0.3553355	-2.95353901302984e-10\\
0.3554355	-3.6059184187061e-10\\
0.3555356	-4.13148870206153e-10\\
0.3556356	-4.76445793972722e-10\\
0.3557356	-5.39494636783853e-10\\
0.3558356	-6.00745670566439e-10\\
0.3559356	-6.61645602334386e-10\\
0.3560356	-7.21637552184039e-10\\
0.3561356	-7.80161031097319e-10\\
0.3562356	-8.37651918692942e-10\\
0.3563356	-8.9454244080099e-10\\
0.3564356	-9.49261146933255e-10\\
0.3565357	-9.89875750062585e-10\\
0.3566357	-1.04243503717602e-09\\
0.3567357	-1.09308528700492e-09\\
0.3568357	-1.1412401119053e-09\\
0.3569357	-1.18830933319834e-09\\
0.3570357	-1.23469895782482e-09\\
0.3571357	-1.27781115486321e-09\\
0.3572357	-1.32004423188709e-09\\
0.3573357	-1.35979261126084e-09\\
0.3574357	-1.39844680621631e-09\\
0.3575358	-1.42114323803303e-09\\
0.3576358	-1.4556722118487e-09\\
0.3577358	-1.48825422193286e-09\\
0.3578358	-1.51826389935584e-09\\
0.3579358	-1.54607184449804e-09\\
0.3580358	-1.57204460234856e-09\\
0.3581358	-1.5955446375825e-09\\
0.3582358	-1.61793030955173e-09\\
0.3583358	-1.63655584715392e-09\\
0.3584358	-1.65477132347666e-09\\
0.3585359	-1.65471748116498e-09\\
0.3586359	-1.66804725027849e-09\\
0.3587359	-1.67799132271089e-09\\
0.3588359	-1.6868828880848e-09\\
0.3589359	-1.69305085145436e-09\\
0.3590359	-1.69681980707832e-09\\
0.3591359	-1.69851001211365e-09\\
0.3592359	-1.69743736006422e-09\\
0.3593359	-1.69491335413346e-09\\
0.3594359	-1.68924508042681e-09\\
0.359536	-1.66750896394022e-09\\
0.359636	-1.65734971341148e-09\\
0.359736	-1.64593997181319e-09\\
0.359836	-1.63256888108885e-09\\
0.359936	-1.61652102967137e-09\\
0.360036	-1.59807642472214e-09\\
0.360136	-1.57851046424805e-09\\
0.360236	-1.55609390898704e-09\\
0.360336	-1.53209285420589e-09\\
0.360436	-1.50676870130463e-09\\
0.3605361	-1.46206040010016e-09\\
0.3606361	-1.4327421462382e-09\\
0.3607361	-1.40085579359526e-09\\
0.3608361	-1.3676436760739e-09\\
0.3609361	-1.33334328923235e-09\\
0.3610361	-1.29618726090534e-09\\
0.3611361	-1.25840332172162e-09\\
0.3612361	-1.21921427528244e-09\\
0.3613361	-1.17783796831923e-09\\
0.3614361	-1.13548726061167e-09\\
0.3615362	-1.07388566867227e-09\\
0.3616362	-1.02908367534256e-09\\
0.3617362	-9.82914831012307e-10\\
0.3618362	-9.35571738535049e-10\\
0.3619362	-8.86241860865657e-10\\
0.3620362	-8.37107489993407e-10\\
0.3621362	-7.86345715646251e-10\\
0.3622362	-7.35128393819287e-10\\
0.3623362	-6.83622115122299e-10\\
0.3624362	-6.3098817297282e-10\\
0.3625363	-5.5865159301759e-10\\
0.3626363	-5.04095607210355e-10\\
0.3627363	-4.49862876604836e-10\\
0.3628363	-3.95093178510835e-10\\
0.3629363	-3.39920831181774e-10\\
0.3630363	-2.84474660871652e-10\\
0.3631363	-2.28877968694882e-10\\
0.3632363	-1.73248497394221e-10\\
0.3633363	-1.16698397789982e-10\\
0.3634363	-6.13341951084645e-11\\
0.3635364	1.38060488550608e-11\\
0.3636364	6.9639626662808e-11\\
0.3637364	1.25002738391724e-10\\
0.3638364	1.79811707514916e-10\\
0.3639364	2.33988654395695e-10\\
0.3640364	2.87461531146369e-10\\
0.3641364	3.4116415662407e-10\\
0.3642364	3.93036251692208e-10\\
0.3643364	4.45023474648867e-10\\
0.3644364	4.96077456868285e-10\\
0.3645365	5.67641323435167e-10\\
0.3646365	6.16855262398831e-10\\
0.3647365	6.65028025364437e-10\\
0.3648365	7.12135502455206e-10\\
0.3649365	7.58159738196844e-10\\
0.3650365	8.03088968253353e-10\\
0.3651365	8.4691765648346e-10\\
0.3652365	8.89646532362526e-10\\
0.3653365	9.30282628256844e-10\\
0.3654365	9.69839317283758e-10\\
0.3655366	1.03134123837047e-09\\
0.3656366	1.06796226778218e-09\\
0.3657366	1.10358346984606e-09\\
0.3658366	1.13624396813214e-09\\
0.3659366	1.16898941866312e-09\\
0.3660366	1.19887204887234e-09\\
0.3661366	1.22695069674666e-09\\
0.3662366	1.25429085034007e-09\\
0.3663366	1.27896468732258e-09\\
0.3664366	1.30205111491889e-09\\
0.3665367	1.34826309121374e-09\\
0.3666367	1.36760668037214e-09\\
0.3667367	1.3846414638295e-09\\
0.3668367	1.40047367699237e-09\\
0.3669367	1.41421648753819e-09\\
0.3670367	1.42599003665804e-09\\
0.3671367	1.43592148049793e-09\\
0.3672367	1.44414503186161e-09\\
0.3673367	1.44980200205715e-09\\
0.3674367	1.45304084315716e-09\\
0.3675368	1.48137651189218e-09\\
0.3676368	1.48143271079817e-09\\
0.3677368	1.47956057988752e-09\\
0.3678368	1.47593756769606e-09\\
0.3679368	1.4697484777366e-09\\
0.3680368	1.46218551224513e-09\\
0.3681368	1.45144831591191e-09\\
0.3682368	1.4397440199856e-09\\
0.3683368	1.42628728661039e-09\\
0.3684368	1.41030035326545e-09\\
0.3685369	1.42122114762461e-09\\
0.3686369	1.40206261210387e-09\\
0.3687369	1.38108774021807e-09\\
0.3688369	1.35854952954635e-09\\
0.3689369	1.33370877965675e-09\\
0.3690369	1.30683413824457e-09\\
0.3691369	1.27920214735243e-09\\
0.3692369	1.25009729003766e-09\\
0.3693369	1.21881203717811e-09\\
0.3694369	1.18564689446754e-09\\
0.369537	1.18109146795204e-09\\
0.369637	1.14530484669656e-09\\
0.369737	1.1085898661814e-09\\
0.369837	1.07027965622962e-09\\
0.369937	1.03071562098633e-09\\
0.370037	9.89247487671734e-10\\
0.370137	9.47233355498021e-10\\
0.370237	9.04039744867435e-10\\
0.370337	8.59041646886934e-10\\
0.370437	8.13622573128461e-10\\
0.3705371	7.99460718098658e-10\\
0.3706371	7.5160263548016e-10\\
0.3707371	7.0352715001677e-10\\
0.3708371	6.54652366951703e-10\\
0.3709371	6.04405165044618e-10\\
0.3710371	5.54221247885357e-10\\
0.3711371	5.0354519570853e-10\\
0.3712371	4.51830517356079e-10\\
0.3713371	3.99539702494556e-10\\
0.3714371	3.47144274339297e-10\\
0.3715372	3.2865662257218e-10\\
0.3716372	2.75735549920458e-10\\
0.3717372	2.23180667625684e-10\\
0.3718372	1.695001733685e-10\\
0.3719372	1.16211565529856e-10\\
0.3720372	6.28416974939367e-11\\
0.3721372	9.92683229267575e-12\\
0.3722372	-4.39873024674214e-11\\
0.3723372	-9.63454594391248e-11\\
0.3724372	-1.49582815906899e-10\\
0.3725373	-1.65197961774864e-10\\
0.3726373	-2.17212593248942e-10\\
0.3727373	-2.68352345998764e-10\\
0.3728373	-3.1901352506986e-10\\
0.3729373	-3.69582577912807e-10\\
0.3730373	-4.19436037203188e-10\\
0.3731373	-4.68940462976162e-10\\
0.3732373	-5.17452384678771e-10\\
0.3733373	-5.64318242956147e-10\\
0.3734373	-6.11874330874917e-10\\
0.3735374	-6.17965623361501e-10\\
0.3736374	-6.62605650745377e-10\\
0.3737374	-7.05881611437825e-10\\
0.3738374	-7.49088605551508e-10\\
0.3739374	-7.90511287616863e-10\\
0.3740374	-8.3142380620788e-10\\
0.3741374	-8.70089742918714e-10\\
0.3742374	-9.08762051132502e-10\\
0.3743374	-9.45682994583306e-10\\
0.3744374	-9.8108408534626e-10\\
0.3745375	-9.72981737639279e-10\\
0.3746375	-1.00471235243581e-09\\
0.3747375	-1.0365507106491e-09\\
0.3748375	-1.06568468126808e-09\\
0.3749375	-1.09429106715237e-09\\
0.3750375	-1.12053554100792e-09\\
0.3751375	-1.1465725812872e-09\\
0.3752375	-1.16954540736106e-09\\
0.3753375	-1.1925859146539e-09\\
0.3754375	-1.21281460935994e-09\\
0.3755376	-1.18623343424882e-09\\
0.3756376	-1.20385357520439e-09\\
0.3757376	-1.21895250815816e-09\\
0.3758376	-1.23360451021666e-09\\
0.3759376	-1.24587213732315e-09\\
0.3760376	-1.25580615668191e-09\\
0.3761376	-1.26544547900412e-09\\
0.3762376	-1.27281709027535e-09\\
0.3763376	-1.27893598319161e-09\\
0.3764376	-1.28280508832063e-09\\
0.3765377	-1.23721435290165e-09\\
0.3766377	-1.23822378223105e-09\\
0.3767377	-1.2369171076444e-09\\
0.3768377	-1.23424830193148e-09\\
0.3769377	-1.23015892538308e-09\\
0.3770377	-1.22457805467946e-09\\
0.3771377	-1.21742221137874e-09\\
0.3772377	-1.20859528986148e-09\\
0.3773377	-1.19898848518155e-09\\
0.3774377	-1.18748022007567e-09\\
0.3775378	-1.12343862536652e-09\\
0.3776378	-1.10836999125052e-09\\
0.3777378	-1.09295563274446e-09\\
0.3778378	-1.07502213604292e-09\\
0.3779378	-1.05638294583705e-09\\
0.3780378	-1.03683829013809e-09\\
0.3781378	-1.01617510469284e-09\\
0.3782378	-9.93166957332368e-10\\
0.3783378	-9.7057397153954e-10\\
0.3784378	-9.46142749762383e-10\\
0.3785379	-8.65596767004844e-10\\
0.3786379	-8.38310878766517e-10\\
0.3787379	-8.1134236165418e-10\\
0.3788379	-7.82382945231061e-10\\
0.3789379	-7.53110448371394e-10\\
0.3790379	-7.2218870015687e-10\\
0.3791379	-6.91267460202773e-10\\
0.3792379	-6.59982338894224e-10\\
0.3793379	-6.27954716495022e-10\\
0.3794379	-5.94791662747267e-10\\
0.379538	-5.02335379083598e-10\\
0.379638	-4.68277840425508e-10\\
0.379738	-4.33816956465246e-10\\
0.379838	-3.98501722187483e-10\\
0.379938	-3.62866410999661e-10\\
0.380038	-3.27430491271061e-10\\
0.380138	-2.9169854248731e-10\\
0.380238	-2.55160170715887e-10\\
0.380338	-2.19289924220095e-10\\
0.380438	-1.8354720769602e-10\\
0.3805381	-8.46417505447625e-11\\
0.3806381	-4.80590147942283e-11\\
0.3807381	-1.18877100664764e-11\\
0.3808381	2.34742089827466e-11\\
0.3809381	5.96443642190264e-11\\
0.3810381	9.52560418628605e-11\\
0.3811381	1.29958280915772e-10\\
0.3812381	1.64415962098771e-10\\
0.3813381	1.98309897188437e-10\\
0.3814381	2.3233691892148e-10\\
0.3815382	3.32186394482188e-10\\
0.3816382	3.65073569980187e-10\\
0.3817382	3.97284114983794e-10\\
0.3818382	4.31579870164062e-10\\
0.3819382	4.56739153649363e-10\\
0.3820382	4.90556853561326e-10\\
0.3821382	5.15844521257848e-10\\
0.3822382	5.45430465082161e-10\\
0.3823382	5.72159844413876e-10\\
0.3824382	5.98894764377835e-10\\
0.3825383	6.90006629925919e-10\\
0.3826383	7.15874442427792e-10\\
0.3827383	7.40439661896635e-10\\
0.3828383	7.66633150633416e-10\\
0.3829383	7.87403197440202e-10\\
0.3830383	8.05715615509765e-10\\
0.3831383	8.24553840621413e-10\\
0.3832383	8.36919029616342e-10\\
0.3833383	8.5583015999556e-10\\
0.3834383	8.6432412914461e-10\\
0.3835384	9.61754576355562e-10\\
0.3836384	9.69094352914991e-10\\
0.3837384	9.75239287255277e-10\\
0.3838384	9.93298983514211e-10\\
0.3839384	9.96401472555913e-10\\
0.3840384	1.00769331484249e-09\\
0.3841384	1.01033970393858e-09\\
0.3842384	1.00752457032618e-09\\
0.3843384	1.01245068599729e-09\\
0.3844384	1.00833976920877e-09\\
0.3845385	1.08984613403417e-09\\
0.3846385	1.08793172390594e-09\\
0.3847385	1.08676346299861e-09\\
0.3848385	1.07963993639516e-09\\
0.3849385	1.06987920918705e-09\\
0.3850385	1.06081893462615e-09\\
0.3851385	1.05581646339193e-09\\
0.3852385	1.03824895267607e-09\\
0.3853385	1.03151347656354e-09\\
0.3854385	1.01902713622085e-09\\
0.3855386	1.0810826459944e-09\\
0.3856386	1.06798942779029e-09\\
0.3857386	1.04952141561284e-09\\
0.3858386	1.0291769578863e-09\\
0.3859386	1.01047499206463e-09\\
0.3860386	9.86955158597473e-10\\
0.3861386	9.62177915706129e-10\\
0.3862386	9.39724654540434e-10\\
0.3863386	9.13197815055524e-10\\
0.3864386	8.86221002405739e-10\\
0.3865387	9.45083438283245e-10\\
0.3866387	9.18761444633048e-10\\
0.3867387	8.8299215636524e-10\\
0.3868387	8.51485037392638e-10\\
0.3869387	8.17971308210013e-10\\
0.3870387	7.86204065634456e-10\\
0.3871387	7.49958403907671e-10\\
0.3872387	7.13031535635043e-10\\
0.3873387	6.6924291387218e-10\\
0.3874387	6.32434354694373e-10\\
0.3875388	6.95271229224213e-10\\
0.3876388	6.54674995497875e-10\\
0.3877388	6.12723605717636e-10\\
0.3878388	5.73349588524669e-10\\
0.3879388	5.3050845764381e-10\\
0.3880388	4.88178838407963e-10\\
0.3881388	4.40362594608353e-10\\
0.3882388	4.01084956161435e-10\\
0.3883388	3.54394647437658e-10\\
0.3884388	3.14364016077356e-10\\
0.3885389	3.70436974673973e-10\\
0.3886389	3.26714829083414e-10\\
0.3887389	2.82016599251889e-10\\
0.3888389	2.30510453041018e-10\\
0.3889389	1.86388836239819e-10\\
0.3890389	1.43868605542691e-10\\
0.3891389	9.71911623278115e-11\\
0.3892389	5.06225865669644e-11\\
0.3893389	-1.54622774545081e-12\\
0.3894389	-4.49994373428781e-11\\
0.389539	1.69118302295482e-11\\
0.389639	-2.53425441710479e-11\\
0.389739	-7.57783169641978e-11\\
0.389839	-1.19978451666399e-10\\
0.389939	-1.63500273870064e-10\\
0.390039	-2.11875330987017e-10\\
0.390139	-2.60609252555118e-10\\
0.390239	-3.05181608123781e-10\\
0.390339	-3.41045766012916e-10\\
0.390439	-3.9362875010887e-10\\
0.3905391	-3.18624938031844e-10\\
0.3906391	-3.60055721606504e-10\\
0.3907391	-3.94322235196883e-10\\
0.3908391	-4.36744658332129e-10\\
0.3909391	-4.82616102063951e-10\\
0.3910391	-5.17202462278153e-10\\
0.3911391	-5.65742271548675e-10\\
0.3912391	-6.03446551093663e-10\\
0.3913391	-6.35498661008695e-10\\
0.3914391	-6.77054150604824e-10\\
0.3915392	-5.85672073196515e-10\\
0.3916392	-6.20776240463784e-10\\
0.3917392	-6.5067711623533e-10\\
0.3918392	-6.90417462787587e-10\\
0.3919392	-7.25011470469245e-10\\
0.3920392	-7.4944460251068e-10\\
0.3921392	-7.88673439953942e-10\\
0.3922392	-8.17625525996387e-10\\
0.3923392	-8.41199208170319e-10\\
0.3924392	-8.74263481464313e-10\\
0.3925393	-7.65735230875731e-10\\
0.3926393	-7.91405998860331e-10\\
0.3927393	-8.10991402464965e-10\\
0.3928393	-8.39241402446952e-10\\
0.3929393	-8.60875806896205e-10\\
0.3930393	-8.80584108012099e-10\\
0.3931393	-9.03025319017731e-10\\
0.3932393	-9.22827810515844e-10\\
0.3933393	-9.34589144901363e-10\\
0.3934393	-9.52875911272686e-10\\
0.3935394	-8.27426150023321e-10\\
0.3936394	-8.4142162354867e-10\\
0.3937394	-8.55449242177656e-10\\
0.3938394	-8.63949995572722e-10\\
0.3939394	-8.71333058744645e-10\\
0.3940394	-8.81975621120588e-10\\
0.3941394	-9.00222715484098e-10\\
0.3942394	-9.00387045137634e-10\\
0.3943394	-9.16748811222558e-10\\
0.3944394	-9.13555537835009e-10\\
0.3945395	-7.70798271730084e-10\\
0.3946395	-7.70131828362037e-10\\
0.3947395	-7.72449236434387e-10\\
0.3948395	-7.81865522070266e-10\\
0.3949395	-7.82462150875383e-10\\
0.3950395	-7.78286849030529e-10\\
0.3951395	-7.83353422777991e-10\\
0.3952395	-7.8164157711643e-10\\
0.3953395	-7.77096734769727e-10\\
0.3954395	-7.73629852104989e-10\\
0.3955396	-6.10883635106393e-10\\
0.3956396	-6.0013246011475e-10\\
0.3957396	-6.01977265964445e-10\\
0.3958396	-5.90189263807066e-10\\
0.3959396	-5.88504270645086e-10\\
0.3960396	-5.80622521128394e-10\\
0.3961396	-5.70208478327947e-10\\
0.3962396	-5.60890643301035e-10\\
0.3963396	-5.56261364610509e-10\\
0.3964396	-5.49876645390311e-10\\
0.3965397	-3.70394495270325e-10\\
0.3966397	-3.59922531072843e-10\\
0.3967397	-3.48136558264104e-10\\
0.3968397	-3.38445223195126e-10\\
0.3969397	-3.24219852757722e-10\\
0.3970397	-3.18794255889512e-10\\
0.3971397	-3.05464525574776e-10\\
0.3972397	-2.97488838772698e-10\\
0.3973397	-2.78087255578264e-10\\
0.3974397	-2.70441518327889e-10\\
0.3975398	-7.15516745482812e-11\\
0.3976398	-6.56431069508243e-11\\
0.3977398	-5.07967160833802e-11\\
0.3978398	-4.00388951492605e-11\\
0.3979398	-2.63566961868588e-11\\
0.3980398	-2.26976214855027e-11\\
0.3981398	-1.19694157878344e-11\\
0.3982398	2.96014392671732e-12\\
0.3983398	9.26345864910873e-12\\
0.3984398	1.41533191044048e-11\\
0.3985399	2.22999812873496e-10\\
0.3986399	2.28100559687488e-10\\
0.3987399	2.43678051661236e-10\\
0.3988399	2.47108817830438e-10\\
0.3989399	2.55810850563617e-10\\
0.3990399	2.57243824765968e-10\\
0.3991399	2.68909315374508e-10\\
0.3992399	2.68351018774368e-10\\
0.3993399	2.73154972877103e-10\\
0.3994399	2.80949779330505e-10\\
0.39954	4.90228837585942e-10\\
0.39964	4.98374779799352e-10\\
0.39974	4.92662780864528e-10\\
0.39984	5.00893985184016e-10\\
0.39994	5.00913228214037e-10\\
0.40004	5.0060926468109e-10\\
};
\addplot [color=mycolor1,solid,forget plot]
  table[row sep=crcr]{%
0.40004	5.0060926468109e-10\\
0.40014	4.97914999331259e-10\\
0.40024	4.9080771697828e-10\\
0.40034	4.87309314543838e-10\\
0.40044	4.85486534566209e-10\\
0.4005401	6.97752466123234e-10\\
0.4006401	6.95053753206685e-10\\
0.4007401	6.88510534304872e-10\\
0.4008401	6.76371331975712e-10\\
0.4009401	6.66930696643041e-10\\
0.4010401	6.58529445505943e-10\\
0.4011401	6.39554905305169e-10\\
0.4012401	6.28441152849407e-10\\
0.4013401	6.13669259105656e-10\\
0.4014401	6.03767533852943e-10\\
0.4015402	8.25910644007013e-10\\
0.4016402	8.03000768018257e-10\\
0.4017402	7.90840577046144e-10\\
0.4018402	7.68149851039024e-10\\
0.4019402	7.43696847994282e-10\\
0.4020402	7.26298556311068e-10\\
0.4021402	7.04820947299587e-10\\
0.4022402	6.78179229293078e-10\\
0.4023402	6.45338102197462e-10\\
0.4024402	6.25312014712923e-10\\
0.4025403	8.50927028526202e-10\\
0.4026403	8.15340220489678e-10\\
0.4027403	7.89922059830732e-10\\
0.4028403	7.53888626913303e-10\\
0.4029403	7.26507051295854e-10\\
0.4030403	6.97095774991098e-10\\
0.4031403	6.55024816769475e-10\\
0.4032403	6.19716040115047e-10\\
0.4033403	5.90643418946397e-10\\
0.4034403	5.47333307616851e-10\\
0.4035404	7.79203446245867e-10\\
0.4036404	7.37868100062107e-10\\
0.4037404	7.01201024287256e-10\\
0.4038404	6.58940859613478e-10\\
0.4039404	6.20879989667023e-10\\
0.4040404	5.76864818547628e-10\\
0.4041404	5.26796046400931e-10\\
0.4042404	4.80628950999323e-10\\
0.4043404	4.38373666333076e-10\\
0.4044404	3.9009546542448e-10\\
0.4045405	6.32797175729592e-10\\
0.4046405	5.84650317294786e-10\\
0.4047405	5.31020289406289e-10\\
0.4048405	4.82195819236856e-10\\
0.4049405	4.38522200953281e-10\\
0.4050405	3.80401586092146e-10\\
0.4051405	3.28293273574548e-10\\
0.4052405	2.82714003581677e-10\\
0.4053405	2.24238250431702e-10\\
0.4054405	1.73498517886761e-10\\
0.4055406	4.26132044256157e-10\\
0.4056406	3.74860042935991e-10\\
0.4057406	3.13583561468132e-10\\
0.4058406	2.63170260716571e-10\\
0.4059406	2.04547328443053e-10\\
0.4060406	1.48701781662079e-10\\
0.4061406	9.66807738535702e-11\\
0.4062406	3.95918985259241e-11\\
0.4063406	-1.13965010117758e-11\\
0.4064406	-7.50550223032709e-11\\
0.4065407	1.93996141687695e-10\\
0.4066407	1.40907046637271e-10\\
0.4067407	7.92169457743304e-11\\
0.4068407	2.04026586952759e-11\\
0.4069407	-3.39963679020614e-11\\
0.4070407	-9.23777431502751e-11\\
0.4071407	-1.43075811845383e-10\\
0.4072407	-2.04361334373514e-10\\
0.4073407	-2.64441162977497e-10\\
0.4074407	-3.21457919012997e-10\\
0.4075408	-2.91191727441785e-11\\
0.4076408	-9.20861009719317e-11\\
0.4077408	-1.46017087249227e-10\\
0.4078408	-1.98794493006297e-10\\
0.4079408	-2.58234806320868e-10\\
0.4080408	-3.12088309168334e-10\\
0.4081408	-3.68038741735237e-10\\
0.4082408	-4.23702968015333e-10\\
0.4083408	-4.76630636665081e-10\\
0.4084408	-5.34303843160164e-10\\
0.4085409	-2.18282383067738e-10\\
0.4086409	-2.7540417983521e-10\\
0.4087409	-3.29296305020155e-10\\
0.4088409	-3.77167147855425e-10\\
0.4089409	-4.36155822965351e-10\\
0.4090409	-4.83331823667571e-10\\
0.4091409	-5.35694671659355e-10\\
0.4092409	-5.90173563699023e-10\\
0.4093409	-6.33627020501322e-10\\
0.4094409	-6.92842529997754e-10\\
0.409541	-3.45928507505268e-10\\
0.409641	-3.94397086265152e-10\\
0.409741	-4.48546949352569e-10\\
0.409841	-4.94877801417029e-10\\
0.409941	-5.39816513334839e-10\\
0.410041	-5.89716760016415e-10\\
0.410141	-6.30858652562192e-10\\
0.410241	-6.79448370247246e-10\\
0.410341	-7.21617791481785e-10\\
0.410441	-7.73424121066637e-10\\
0.4105411	-4.08148821936572e-10\\
0.4106411	-4.44614469712252e-10\\
0.4107411	-4.88422654114384e-10\\
0.4108411	-5.35328062667385e-10\\
0.4109411	-5.71008825834586e-10\\
0.4110411	-6.11066135327654e-10\\
0.4111411	-6.5102386139328e-10\\
0.4112411	-6.96328166839306e-10\\
0.4113411	-7.3234712154454e-10\\
0.4114411	-7.743703106587e-10\\
0.4115412	-3.79401031641927e-10\\
0.4116412	-4.16354441104116e-10\\
0.4117412	-4.54691044107021e-10\\
0.4118412	-4.89382492673386e-10\\
0.4119412	-5.25319985374053e-10\\
0.4120412	-5.67313867541261e-10\\
0.4121412	-6.0009323039705e-10\\
0.4122412	-6.3830550843828e-10\\
0.4123412	-6.76516074388754e-10\\
0.4124412	-7.09207832326648e-10\\
0.4125413	-2.75575213920542e-10\\
0.4126413	-3.07561470906724e-10\\
0.4127413	-3.46962919265076e-10\\
0.4128413	-3.77928400813882e-10\\
0.4129413	-4.14522230153232e-10\\
0.4130413	-4.40723772987496e-10\\
0.4131413	-4.80427028658171e-10\\
0.4132413	-5.07440207922157e-10\\
0.4133413	-5.45485310309642e-10\\
0.4134413	-5.78197697741743e-10\\
0.4135414	-1.15348729862446e-10\\
0.4136414	-1.45006459259783e-10\\
0.4137414	-1.79696787330308e-10\\
0.4138414	-2.12704149568166e-10\\
0.4139414	-2.47224189714847e-10\\
0.4140414	-2.76363322322865e-10\\
0.4141414	-3.13138294116768e-10\\
0.4142414	-3.40475743708683e-10\\
0.4143414	-3.81211758530451e-10\\
0.4144414	-4.08091431022719e-10\\
0.4145415	8.02388722413901e-11\\
0.4146415	4.6320223847404e-11\\
0.4147415	1.8590666441172e-11\\
0.4148415	-1.53263314853994e-11\\
0.4149415	-4.77140531539343e-11\\
0.4150415	-8.07620768086261e-11\\
0.4151415	-1.16565816948942e-10\\
0.4152415	-1.57126062676335e-10\\
0.4153415	-1.94348517612358e-10\\
0.4154415	-2.30043332188792e-10\\
0.4155416	2.90062148603008e-10\\
0.4156416	2.55674130806194e-10\\
0.4157416	2.17979815065618e-10\\
0.4158416	1.85556042631304e-10\\
0.4159416	1.4707756121671e-10\\
0.4160416	1.01317502009131e-10\\
0.4161416	6.71478594976198e-11\\
0.4162416	2.35399722175837e-11\\
0.4163416	-2.04349939205394e-11\\
0.4164416	-5.56055591925241e-11\\
0.4165417	4.87110689132743e-10\\
0.4166417	4.40954237370967e-10\\
0.4167417	4.01742918527395e-10\\
0.4168417	3.59054464898209e-10\\
0.4169417	3.12569384654585e-10\\
0.4170417	2.72071456374401e-10\\
0.4171417	2.17448232583592e-10\\
0.4172417	1.68691541055474e-10\\
0.4173417	8.58979908551069e-11\\
0.4174417	6.92694793262257e-11\\
0.4175418	6.34690415024855e-10\\
0.4176418	5.55108512284509e-10\\
0.4177418	5.72954747651822e-10\\
0.4178418	5.08857395104109e-10\\
0.4179418	4.83552587592561e-10\\
0.4180418	4.17884836638265e-10\\
0.4181418	3.32807554735345e-10\\
0.4182418	2.49383583621481e-10\\
0.4183418	1.88785718096629e-10\\
0.4184418	1.72297238379849e-10\\
0.4185419	7.84703311037431e-10\\
0.4186419	7.24625853678468e-10\\
0.4187419	6.7319753572115e-10\\
0.4188419	5.52149031791266e-10\\
0.4189419	4.83324185551358e-10\\
0.4190419	4.88680553447783e-10\\
0.4191419	3.9028994928492e-10\\
0.4192419	3.10338994858662e-10\\
0.4193419	2.71129669239469e-10\\
0.4194419	1.95079862325338e-10\\
0.419542	8.08086017195573e-10\\
0.419642	7.30195063498884e-10\\
0.419742	6.83441159731162e-10\\
0.419842	5.90711537410554e-10\\
0.419942	4.75012141762637e-10\\
0.420042	4.59468199083416e-10\\
0.420142	3.67324785768407e-10\\
0.420242	3.2194740189756e-10\\
0.420342	2.4682254734531e-10\\
0.420442	1.65558296455828e-10\\
0.4205421	7.47490113954682e-10\\
0.4206421	7.29613925618553e-10\\
0.4207421	5.77181871307814e-10\\
0.4208421	5.1429427781898e-10\\
0.4209421	4.65175974210521e-10\\
0.4210421	3.54176881426542e-10\\
0.4211421	3.05772609977221e-10\\
0.4212421	2.44565055408869e-10\\
0.4213421	9.52829996435541e-11\\
0.4214421	-1.7217286298922e-11\\
0.4215422	7.22289364413599e-10\\
0.4216422	6.63033904391369e-10\\
0.4217422	5.15925329503753e-10\\
0.4218422	4.06336072617973e-10\\
0.4219422	3.59769107896851e-10\\
0.4220422	3.01858568577205e-10\\
0.4221422	1.5837036806206e-10\\
0.4222422	5.52028219956337e-11\\
0.4223422	1.83872761309357e-11\\
0.4224422	-1.25911265002601e-10\\
0.4225423	5.8600163984528e-10\\
0.4226423	5.10628437536825e-10\\
0.4227423	4.07102163991254e-10\\
0.4228423	3.02128933552187e-10\\
0.4229423	2.22551701866509e-10\\
0.4230423	9.53509117637089e-11\\
0.4231423	4.76451417923997e-11\\
0.4232423	-9.33082465188594e-11\\
0.4233423	-2.00112451538862e-10\\
0.4234423	-3.45230578797711e-10\\
0.4235424	4.86215766948425e-10\\
0.4236424	3.52775304872073e-10\\
0.4237424	2.64506800549914e-10\\
0.4238424	1.49514279456546e-10\\
0.4239424	3.60451828974873e-11\\
0.4240424	-4.75089599008983e-11\\
0.4241424	-1.72611858075163e-10\\
0.4242424	-3.10581778773921e-10\\
0.4243424	-3.32590869423074e-10\\
0.4244424	-5.09664469482141e-10\\
0.4245425	3.2711603133802e-10\\
0.4246425	2.32846076133383e-10\\
0.4247425	1.71352798486804e-10\\
0.4248425	7.22051244207537e-11\\
0.4249425	-3.48777499394569e-11\\
0.4250425	-2.20025659746891e-10\\
0.4251425	-3.53216767195585e-10\\
0.4252425	-4.04276851958315e-10\\
0.4253425	-5.4287860665192e-10\\
0.4254425	-6.38540918111805e-10\\
0.4255426	2.34699648604675e-10\\
0.4256426	1.22698184945591e-10\\
0.4257426	4.6040519541355e-11\\
0.4258426	-6.41696783058344e-11\\
0.4259426	-1.76671325084667e-10\\
0.4260426	-3.60045194206697e-10\\
0.4261426	-3.8271318181086e-10\\
0.4262426	-5.1293756877591e-10\\
0.4263426	-6.18820284944976e-10\\
0.4264426	-7.68302159712781e-10\\
0.4265427	1.2478506173242e-10\\
0.4266427	9.09677662244788e-11\\
0.4267427	-8.92644942609034e-11\\
0.4268427	-1.83200506820953e-10\\
0.4269427	-2.57964200754647e-10\\
0.4270427	-3.80513885157863e-10\\
0.4271427	-5.1764148314439e-10\\
0.4272427	-6.35971767273553e-10\\
0.4273427	-7.01961584589602e-10\\
0.4274427	-8.81899087312949e-10\\
0.4275428	1.73911525317479e-10\\
0.4276428	7.42639642272933e-11\\
0.4277428	-3.71414520504171e-11\\
0.4278428	-2.25910108382719e-10\\
0.4279428	-3.57474773849016e-10\\
0.4280428	-3.97094805822118e-10\\
0.4281428	-5.09855357152394e-10\\
0.4282428	-6.60666573857292e-10\\
0.4283428	-8.14262792798613e-10\\
0.4284428	-9.35201737638182e-10\\
0.4285429	1.93233042197413e-10\\
0.4286429	5.13680447461994e-11\\
0.4287429	-5.04089729214206e-11\\
0.4288429	-1.75940962591093e-10\\
0.4289429	-2.88890180061372e-10\\
0.4290429	-4.52737357504876e-10\\
0.4291429	-5.30780872650283e-10\\
0.4292429	-6.8613591937554e-10\\
0.4293429	-7.81733668737353e-10\\
0.4294429	-8.80320432462877e-10\\
0.429543	2.05512335990123e-10\\
0.429643	1.20543250836831e-10\\
0.429743	4.55017449500078e-11\\
0.429843	-8.16103915945565e-11\\
0.429943	-2.22602258376442e-10\\
0.430043	-3.39092983157657e-10\\
0.430143	-4.92510858729174e-10\\
0.430243	-5.44092473007442e-10\\
0.430343	-7.54881842988551e-10\\
0.430443	-8.85729536345995e-10\\
0.4305431	3.25323175055748e-10\\
0.4306431	2.80063987083932e-10\\
0.4307431	1.33404079747995e-10\\
0.4308431	2.52756844046267e-11\\
0.4309431	-1.04191097888067e-10\\
0.4310431	-2.14667399660176e-10\\
0.4311431	-3.65624685800918e-10\\
0.4312431	-5.16333852979218e-10\\
0.4313431	-6.25864322959533e-10\\
0.4314431	-7.53083132746385e-10\\
0.4315432	5.42571991648351e-10\\
0.4316432	4.12075543663754e-10\\
0.4317432	2.88554949520129e-10\\
0.4318432	1.13962086946071e-10\\
0.4319432	3.04558227595675e-11\\
0.4320432	-1.19597058493605e-10\\
0.4321432	-2.93620912794005e-10\\
0.4322432	-4.48830302242874e-10\\
0.4323432	-5.42229051101514e-10\\
0.4324432	-6.30609298070059e-10\\
0.4325433	7.09452401231707e-10\\
0.4326433	5.69898213906642e-10\\
0.4327433	3.66309694762364e-10\\
0.4328433	2.42751200488656e-10\\
0.4329433	1.4350356168126e-10\\
0.4330433	1.3065054246808e-11\\
0.4331433	-2.03847630398865e-10\\
0.4332433	-2.62298404621764e-10\\
0.4333433	-4.17130806350079e-10\\
0.4334433	-6.22967012556871e-10\\
0.4335434	8.30948831661085e-10\\
0.4336434	6.6889229604402e-10\\
0.4337434	4.9334964542444e-10\\
0.4338434	3.50594289819268e-10\\
0.4339434	2.87125985043001e-10\\
0.4340434	4.96718347334744e-11\\
0.4341434	-1.48126965337711e-11\\
0.4342434	-2.5914276182825e-10\\
0.4343434	-3.35903118091774e-10\\
0.4344434	-4.97447100438614e-10\\
0.4345435	9.59008133755851e-10\\
0.4346435	7.81003099649055e-10\\
0.4347435	6.62601737908865e-10\\
0.4348435	4.52386958460244e-10\\
0.4349435	2.99178276716352e-10\\
0.4350435	1.52032860422074e-10\\
0.4351435	-3.97534230468961e-11\\
0.4352435	-1.26644942742473e-10\\
0.4353435	-3.58865247414297e-10\\
0.4354435	-4.86396001817338e-10\\
0.4355436	1.09050064686861e-09\\
0.4356436	9.33107704385178e-10\\
0.4357436	7.31972209190232e-10\\
0.4358436	5.38091012231619e-10\\
0.4359436	4.02708234651956e-10\\
0.4360436	1.77316357723081e-10\\
0.4361436	1.36573186570663e-11\\
0.4362436	-1.36276398152169e-10\\
0.4363436	-3.20240649861851e-10\\
0.4364436	-4.85738534681049e-10\\
0.4365437	1.16909480767048e-09\\
0.4366437	1.00928780068753e-09\\
0.4367437	8.27018944429478e-10\\
0.4368437	5.75807592524593e-10\\
0.4369437	4.09431453367651e-10\\
0.4370437	1.81927723049129e-10\\
0.4371437	4.75942277668e-11\\
0.4372437	-1.39009444309179e-10\\
0.4373437	-4.23060786815081e-10\\
0.4374437	-5.49473237309445e-10\\
0.4375438	1.19117196239072e-09\\
0.4376438	9.57155551920084e-10\\
0.4377438	7.4768968116477e-10\\
0.4378438	6.18929016219828e-10\\
0.4379438	4.27298100006378e-10\\
0.4380438	1.29492528028533e-10\\
0.4381438	-1.75198709164402e-11\\
0.4382438	-2.56497851333203e-10\\
0.4383438	-4.29925555388549e-10\\
0.4384438	-6.80011324647681e-10\\
0.4385439	1.21591062484392e-09\\
0.4386439	9.98361959672189e-10\\
0.4387439	7.79255795307909e-10\\
0.4388439	5.17499309626423e-10\\
0.4389439	3.72281528250151e-10\\
0.4390439	1.03074536373596e-10\\
0.4391439	-1.30365287141776e-10\\
0.4392439	-3.67996044769482e-10\\
0.4393439	-5.4948908503839e-10\\
0.4394439	-8.14227756886724e-10\\
0.439544	1.17967192416434e-09\\
0.439644	9.43419728460532e-10\\
0.439744	7.07573931141726e-10\\
0.439844	4.33915947443964e-10\\
0.439944	2.84521467421462e-10\\
0.440044	2.1761713741783e-11\\
0.440144	-1.91695281826984e-10\\
0.440244	-4.92883401247927e-10\\
0.440344	-7.18537157833217e-10\\
0.440444	-9.05090408924534e-10\\
0.4405441	1.11482048511255e-09\\
0.4406441	9.10975122485162e-10\\
0.4407441	6.38810206826396e-10\\
0.4408441	3.63107777840658e-10\\
0.4409441	1.48957046350318e-10\\
0.4410441	-3.82442961847917e-11\\
0.4411441	-3.32788753422842e-10\\
0.4412441	-5.68657699704864e-10\\
0.4413441	-8.79520047774879e-10\\
0.4414441	-1.09873091522347e-09\\
0.4415442	1.07311743941917e-09\\
0.4416442	8.51665750238855e-10\\
0.4417442	6.23764164831616e-10\\
0.4418442	3.57326535168571e-10\\
0.4419442	1.20587263590342e-10\\
0.4420442	-2.17897326484223e-10\\
0.4421442	-4.89247621799152e-10\\
0.4422442	-7.24259342495086e-10\\
0.4423442	-9.5340216407433e-10\\
0.4424442	-1.20681832907997e-09\\
0.4425443	1.05382463741694e-09\\
0.4426443	7.76703592338546e-10\\
0.4427443	5.86932031622356e-10\\
0.4428443	2.55691746259019e-10\\
0.4429443	5.44989640438934e-11\\
0.4430443	-2.44794225634417e-10\\
0.4431443	-5.69998451921145e-10\\
0.4432443	-8.48585636058653e-10\\
0.4433443	-1.10768755859315e-09\\
0.4434443	-1.37409442114642e-09\\
0.4435444	1.03666148376039e-09\\
0.4436444	7.91324915913171e-10\\
0.4437444	5.60451146537806e-10\\
0.4438444	2.18631121128195e-10\\
0.4439444	-5.91953678999793e-11\\
0.4440444	-2.97738215267358e-10\\
0.4441444	-6.21355517277204e-10\\
0.4442444	-8.54052101826111e-10\\
0.4443444	-1.21947803743104e-09\\
0.4444444	-1.44092714171075e-09\\
0.4445445	1.11975809310647e-09\\
0.4446445	8.33250997931204e-10\\
0.4447445	5.23069715796173e-10\\
0.4448445	2.67360801141543e-10\\
0.4449445	-5.5365405619498e-11\\
0.4450445	-3.66233259972877e-10\\
0.4451445	-6.86000274814635e-10\\
0.4452445	-9.35055590248202e-10\\
0.4453445	-1.23341843574591e-09\\
0.4454445	-1.6007365812639e-09\\
0.4455446	1.16275064197206e-09\\
0.4456446	8.16305753244609e-10\\
0.4457446	5.44288196869906e-10\\
0.4458446	2.28551977365411e-10\\
0.4459446	-4.86696301749828e-11\\
0.4460446	-4.04762508774727e-10\\
0.4461446	-6.56730097170141e-10\\
0.4462446	-1.02119181535389e-09\\
0.4463446	-1.31438146575447e-09\\
0.4464446	-1.65214562462736e-09\\
0.4465447	1.23516693346839e-09\\
0.4466447	9.79338483138997e-10\\
0.4467447	6.33820737864707e-10\\
0.4468447	2.8433252681568e-10\\
0.4469447	1.69879992234848e-11\\
0.4470447	-3.81701745081134e-10\\
0.4471447	-6.24827014889147e-10\\
0.4472447	-1.02507787078533e-09\\
0.4473447	-1.29474248466641e-09\\
0.4474447	-1.64570546697541e-09\\
0.4475448	1.37025221432028e-09\\
0.4476448	1.04060199422173e-09\\
0.4477448	6.96527110077491e-10\\
0.4478448	4.2777428973205e-10\\
0.4479448	2.45021954115021e-11\\
0.4480448	-3.22716866142485e-10\\
0.4481448	-6.22895265532317e-10\\
0.4482448	-9.84628322235268e-10\\
0.4483448	-1.31609260165325e-09\\
0.4484448	-1.72504419993294e-09\\
0.4485449	1.52438720619954e-09\\
0.4486449	1.15773821595094e-09\\
0.4487449	7.92967468164069e-10\\
0.4488449	5.24018496021565e-10\\
0.4489449	1.45263987037098e-10\\
0.4490449	-2.48492455345167e-10\\
0.4491449	-5.62014560107418e-10\\
0.4492449	-9.9963161066894e-10\\
0.4493449	-1.36523668173467e-09\\
0.4494449	-1.66228485078308e-09\\
0.449545	1.6422521637938e-09\\
0.449645	1.29352533892457e-09\\
0.449745	9.05734105771661e-10\\
0.449845	5.77193713939723e-10\\
0.449945	2.06666401982199e-10\\
0.450045	-2.06636796757677e-10\\
0.450145	-5.63054225097802e-10\\
0.450245	-9.62471775022011e-10\\
0.450345	-1.30432105242308e-09\\
0.450445	-1.68757753409542e-09\\
0.4505451	1.82789246445155e-09\\
0.4506451	1.48754210316758e-09\\
0.4507451	1.11171602522678e-09\\
0.4508451	7.03282278008595e-10\\
0.4509451	2.65574339104215e-10\\
0.4510451	-9.7606999631232e-11\\
0.4511451	-4.81991747620751e-10\\
0.4512451	-8.82838819735686e-10\\
0.4513451	-1.29493414515597e-09\\
0.4514451	-1.71258874670597e-09\\
0.4515452	2.02184282131149e-09\\
0.4516452	1.63390652288369e-09\\
0.4517452	1.16045376444013e-09\\
0.4518452	8.0909277422412e-10\\
0.4519452	3.87916309100775e-10\\
0.4520452	-9.44964074437588e-11\\
0.4521452	-5.29077727809413e-10\\
0.4522452	-9.06269634207493e-10\\
0.4523452	-1.31602175955804e-09\\
0.4524452	-1.74778942002834e-09\\
0.4525453	2.18446871171323e-09\\
0.4526453	1.66524773148342e-09\\
0.4527453	1.25874324342651e-09\\
0.4528453	8.77497536764214e-10\\
0.4529453	4.34557170195943e-10\\
0.4530453	-5.65250168720055e-11\\
0.4531453	-4.81687875675783e-10\\
0.4532453	-9.26359935049699e-10\\
0.4533453	-1.37545736855326e-09\\
0.4534453	-1.91338195060144e-09\\
0.4535454	2.28568485134955e-09\\
0.4536454	1.84307093924804e-09\\
0.4537454	1.36164884174863e-09\\
0.4538454	8.59095407551407e-10\\
0.4539454	4.53612189802409e-10\\
0.4540454	-3.60724902639248e-11\\
0.4541454	-4.90701452453106e-10\\
0.4542454	-9.90486559741474e-10\\
0.4543454	-1.51510663195208e-09\\
0.4544454	-1.94370531995947e-09\\
0.4545455	2.40121173758403e-09\\
0.4546455	1.85467662701302e-09\\
0.4547455	1.37008682318199e-09\\
0.4548455	9.70461064265203e-10\\
0.4549455	4.79363893740072e-10\\
0.4550455	-7.90921862508717e-11\\
0.4551455	-5.80244571900083e-10\\
0.4552455	-1.09887838184355e-09\\
0.4553455	-1.60922429515934e-09\\
0.4554455	-2.08495637117959e-09\\
0.4555456	2.51553233310883e-09\\
0.4556456	2.01679609330204e-09\\
0.4557456	1.43514141944055e-09\\
0.4558456	8.9914322515261e-10\\
0.4559456	4.37944036612604e-10\\
0.4560456	-1.18743775771347e-10\\
0.4561456	-6.4063577294611e-10\\
0.4562456	-1.19687321468651e-09\\
0.4563456	-1.75602081324295e-09\\
0.4564456	-2.28606449135388e-09\\
0.4565457	2.53171163156971e-09\\
0.4566457	1.98610624235936e-09\\
0.4567457	1.46927645224004e-09\\
0.4568457	9.15574760538412e-10\\
0.4569457	8.59943818194361e-10\\
0.4570457	-5.62081276488102e-10\\
0.4571457	-5.14370666069089e-10\\
0.4572457	-8.60197439320198e-10\\
0.4573457	-1.46223530180824e-09\\
0.4574457	-2.18255627026237e-09\\
0.4575458	2.7882426424132e-09\\
0.4576458	2.27678713295848e-09\\
0.4577458	1.06460609078374e-09\\
0.4578458	1.29205830713339e-09\\
0.4579458	1.00116000342418e-10\\
0.4580458	-3.69632810904436e-10\\
0.4581458	-9.74981925055962e-10\\
0.4582458	-1.57310458474662e-09\\
0.4583458	-2.02055107237711e-09\\
0.4584458	-2.17324633202537e-09\\
0.4585459	3.08308263099154e-09\\
0.4586459	1.9852862092988e-09\\
0.4587459	1.61838870343239e-09\\
0.4588459	1.12899136415524e-09\\
0.4589459	6.64332903141732e-10\\
0.4590459	-6.27708063478949e-10\\
0.4591459	-5.98610552186157e-10\\
0.4592459	-1.09920876524214e-09\\
0.4593459	-1.97968960235356e-09\\
0.4594459	-3.08959020997243e-09\\
0.459546	3.00504373403239e-09\\
0.459646	1.92245329442946e-09\\
0.459746	1.06590788365862e-09\\
0.459846	5.8849500706587e-10\\
0.459946	6.43964428439711e-10\\
0.460046	-6.13269305780803e-10\\
0.460146	-1.02812432200595e-09\\
0.460246	-1.4448488874514e-09\\
0.460346	-2.70701886817895e-09\\
0.460446	-2.65753516675432e-09\\
0.4605461	2.47270203574887e-09\\
0.4606461	2.65321045949535e-09\\
0.4607461	1.62090475048192e-09\\
0.4608461	5.35610241055029e-10\\
0.4609461	5.57840109391763e-10\\
0.4610461	-1.51202010113816e-10\\
0.4611461	-1.42961940658055e-09\\
0.4612461	-2.11481968504131e-09\\
0.4613461	-2.04351213750099e-09\\
0.4614461	-3.05170510346217e-09\\
0.4615462	2.98098896023453e-09\\
0.4616462	2.34392056762053e-09\\
0.4617462	2.12372526487391e-09\\
0.4618462	4.87225733716845e-10\\
0.4619462	6.01958894839646e-10\\
0.4620462	-3.63821385344909e-10\\
0.4621462	-1.24114162198576e-09\\
0.4622462	-1.86030601101624e-09\\
0.4623462	-3.05089370390352e-09\\
0.4624462	-3.64175610956426e-09\\
0.4625463	2.85562834099681e-09\\
0.4626463	2.01761778226457e-09\\
0.4627463	1.29734580868977e-09\\
0.4628463	8.68900086646932e-10\\
0.4629463	-9.2890252998204e-11\\
0.4630463	-4.12451846331166e-10\\
0.4631463	-9.13464330868849e-10\\
0.4632463	-2.41885756411576e-09\\
0.4633463	-2.75080880513326e-09\\
0.4634463	-3.73073994527212e-09\\
0.4635464	3.5155817462891e-09\\
0.4636464	2.8172625519355e-09\\
0.4637464	2.01143944317424e-09\\
0.4638464	1.27974098007057e-09\\
0.4639464	-1.95434747161591e-10\\
0.4640464	-2.30917295267891e-10\\
0.4641464	-1.64276101424256e-09\\
0.4642464	-2.24624213608811e-09\\
0.4643464	-2.85585594343891e-09\\
0.4644464	-3.28531386101748e-09\\
0.4645465	3.74366365186486e-09\\
0.4646465	2.27718252240883e-09\\
0.4647465	1.55464415169462e-09\\
0.4648465	7.65502360718789e-10\\
0.4649465	1.00009368408233e-10\\
0.4650465	-2.50781259257191e-10\\
0.4651465	-1.09501164797902e-09\\
0.4652465	-2.24001669426807e-09\\
0.4653465	-3.49232106384642e-09\\
0.4654465	-3.65763624921543e-09\\
0.4655466	3.96548674127563e-09\\
0.4656466	2.60285948653835e-09\\
0.4657466	1.91519307444349e-09\\
0.4658466	1.10005949163251e-09\\
0.4659466	3.55858889664745e-10\\
0.4660466	-1.18177364595573e-10\\
0.4661466	-1.12198366656463e-09\\
0.4662466	-2.45465714551404e-09\\
0.4663466	-2.91445458348513e-09\\
0.4664466	-4.29878937177955e-09\\
0.4665467	3.53689583675564e-09\\
0.4666467	2.95922081023497e-09\\
0.4667467	2.07006491006849e-09\\
0.4668467	1.07541995252557e-09\\
0.4669467	1.82136590965581e-10\\
0.4670467	-4.02072580524965e-10\\
0.4671467	-1.46862989889966e-09\\
0.4672467	-1.80808950884376e-09\\
0.4673467	-3.21013421183611e-09\\
0.4674467	-4.46357229975068e-09\\
0.4675468	4.04004732988163e-09\\
0.4676468	3.76759180624464e-09\\
0.4677468	2.28281082643826e-09\\
0.4678468	1.80042661248464e-09\\
0.4679468	5.36051780221369e-10\\
0.4680468	-2.93807484851721e-10\\
0.4681468	-1.47174819250043e-09\\
0.4682468	-2.77946732578534e-09\\
0.4683468	-2.99775863269076e-09\\
0.4684468	-3.90650935428241e-09\\
0.4685469	4.58828838920296e-09\\
0.4686469	4.01146323905817e-09\\
0.4687469	2.41021177755513e-09\\
0.4688469	2.00830625220194e-09\\
0.4689469	1.0304417518911e-09\\
0.4690469	-2.97760488357848e-10\\
0.4691469	-1.74974979102148e-09\\
0.4692469	-2.09804272110863e-09\\
0.4693469	-3.11421977094994e-09\\
0.4694469	-4.56892200548925e-09\\
0.469547	5.1399877535642e-09\\
0.469647	3.55122543654537e-09\\
0.469747	3.21791740302835e-09\\
0.469847	1.37321472899396e-09\\
0.469947	2.51224726351834e-10\\
0.470047	8.70143393102984e-11\\
0.470147	-1.88338649190212e-09\\
0.470247	-2.4229813693627e-09\\
0.470347	-3.29380407931151e-09\\
0.470447	-4.25691510860804e-09\\
0.4705471	4.82146699691891e-09\\
0.4706471	4.44801755063867e-09\\
0.4707471	2.70521474188615e-09\\
0.4708471	1.83592662860582e-09\\
0.4709471	1.08401182865106e-09\\
0.4710471	-3.056769945671e-10\\
0.4711471	-1.08728963462909e-09\\
0.4712471	-3.01397483620097e-09\\
0.4713471	-3.8378767945775e-09\\
0.4714471	-5.31013163345209e-09\\
0.4715472	5.2591776558697e-09\\
0.4716472	4.29683704892755e-09\\
0.4717472	3.43907150196845e-09\\
0.4718472	1.9388136833845e-09\\
0.4719472	1.05002208750787e-09\\
0.4720472	2.76846358338023e-11\\
0.4721472	-1.87217771797574e-09\\
0.4722472	-2.39250741397232e-09\\
0.4723472	-4.27520665310238e-09\\
0.4724472	-5.2611338163637e-09\\
0.4725473	5.92126593763898e-09\\
0.4726473	4.56905071243019e-09\\
0.4727473	2.89759474183734e-09\\
0.4728473	2.17025251155606e-09\\
0.4729473	6.51440593742864e-10\\
0.4730473	-3.93358652941989e-10\\
0.4731473	-1.69759339831849e-09\\
0.4732473	-2.99363866161639e-09\\
0.4733473	-4.01279257912501e-09\\
0.4734473	-5.48527271998153e-09\\
0.4735474	5.4686626260389e-09\\
0.4736474	4.96444437598706e-09\\
0.4737474	2.82304170665104e-09\\
0.4738474	2.31859789440875e-09\\
0.4739474	7.26355563369918e-10\\
0.4740474	-6.77339585895137e-10\\
0.4741474	-1.61503493304599e-09\\
0.4742474	-2.80816717058638e-09\\
0.4743474	-4.97705847845795e-09\\
0.4744474	-5.84091274045268e-09\\
0.4745475	6.11583108201669e-09\\
0.4746475	4.77294375290233e-09\\
0.4747475	3.58451482298378e-09\\
0.4748475	1.83585320209789e-09\\
0.4749475	8.1340535782598e-10\\
0.4750475	-1.95240817870562e-10\\
0.4751475	-1.90135209660047e-09\\
0.4752475	-3.01504603922309e-09\\
0.4753475	-5.24528706927208e-09\\
0.4754475	-6.29988256830791e-09\\
0.4755476	6.00130141067967e-09\\
0.4756476	5.24613395346679e-09\\
0.4757476	3.55046775183179e-09\\
0.4758476	2.21116415455302e-09\\
0.4759476	5.26261304499565e-10\\
0.4760476	-2.05021899631064e-10\\
0.4761476	-1.68228178436258e-09\\
0.4762476	-3.60392593285602e-09\\
0.4763476	-4.66716916416171e-09\\
0.4764476	-6.56802950265226e-09\\
0.4765477	6.56811322300929e-09\\
0.4766477	4.97869953987665e-09\\
0.4767477	3.47114388447867e-09\\
0.4768477	2.35425696342231e-09\\
0.4769477	9.3806652246356e-10\\
0.4770477	-4.66178571006254e-10\\
0.4771477	-2.54600426678742e-09\\
0.4772477	-3.98770718052291e-09\\
0.4773477	-5.47635052193171e-09\\
0.4774477	-6.69575995213348e-09\\
0.4775478	6.95428367929879e-09\\
0.4776478	5.29990112430392e-09\\
0.4777478	3.87106263450426e-09\\
0.4778478	1.98893539268483e-09\\
0.4779478	9.75944885669988e-10\\
0.4780478	-8.44220918432698e-10\\
0.4781478	-2.14660735454314e-09\\
0.4782478	-3.60498886905342e-09\\
0.4783478	-5.89186482198231e-09\\
0.4784478	-7.67845520933258e-09\\
0.4785479	7.39341087527615e-09\\
0.4786479	5.67519802419786e-09\\
0.4787479	4.45166236685834e-09\\
0.4788479	2.05674495100047e-09\\
0.4789479	8.25687443764545e-10\\
0.4790479	-9.04963560251909e-10\\
0.4791479	-2.79735230205441e-09\\
0.4792479	-4.51230949740505e-09\\
0.4793479	-5.70934802733569e-09\\
0.4794479	-8.04665863021053e-09\\
0.479548	7.62551594915215e-09\\
0.479648	6.11811955741545e-09\\
0.479748	4.50420175983346e-09\\
0.479848	2.13090531114499e-09\\
0.479948	3.46716925569285e-10\\
0.480048	-4.98528343172849e-10\\
0.480148	-3.05364264213334e-09\\
0.480248	-3.96608100354523e-09\\
0.480348	-5.88193680870971e-09\\
0.480448	-7.44593743520067e-09\\
0.4805481	8.31821973064015e-09\\
0.4806481	6.61247990868183e-09\\
0.4807481	4.33301393593447e-09\\
0.4808481	2.84060453414545e-09\\
0.4809481	4.97422769640821e-10\\
0.4810481	-1.33296741786776e-09\\
0.4811481	-2.28560471113651e-09\\
0.4812481	-4.99412592790821e-09\\
0.4813481	-6.0907614242954e-09\\
0.4814481	-8.20633061027545e-09\\
0.4815482	8.49834177118688e-09\\
0.4816482	6.54502392083769e-09\\
0.4817482	4.68919963015485e-09\\
0.4818482	2.30574038985118e-09\\
0.4819482	7.70951517035934e-10\\
0.4820482	-5.37423248264668e-10\\
0.4821482	-3.24019710108009e-09\\
0.4822482	-4.95673558578366e-09\\
0.4823482	-6.30495193294084e-09\\
0.4824482	-8.90130240523146e-09\\
0.4825483	8.99400124732344e-09\\
0.4826483	7.14858206787503e-09\\
0.4827483	5.21483699148298e-09\\
0.4828483	2.58218569096358e-09\\
0.4829483	6.41528237650679e-10\\
0.4830483	-1.21475025871565e-09\\
0.4831483	-2.59277487327236e-09\\
0.4832483	-5.09717620108402e-09\\
0.4833483	-7.33108556846829e-09\\
0.4834483	-8.89613027514613e-09\\
0.4835484	8.88729012951672e-09\\
0.4836484	6.95580424595749e-09\\
0.4837484	4.89777655671847e-09\\
0.4838484	3.11764505878109e-09\\
0.4839484	1.02137573225469e-09\\
0.4840484	-9.83532701462342e-10\\
0.4841484	-3.48804392227148e-09\\
0.4842484	-5.08157916093155e-09\\
0.4843484	-7.35201240685546e-09\\
0.4844484	-9.88566547545366e-09\\
0.4845485	9.97757928014598e-09\\
0.4846485	7.26353197727514e-09\\
0.4847485	5.53708015983007e-09\\
0.4848485	3.2181609530002e-09\\
0.4849485	7.28288120782628e-10\\
0.4850485	-1.50944305405548e-09\\
0.4851485	-3.07035079392254e-09\\
0.4852485	-5.52816190827615e-09\\
0.4853485	-7.45500686270134e-09\\
0.4854485	-9.42141472736607e-09\\
0.4855486	1.0255508084249e-08\\
0.4856486	7.6078587015263e-09\\
0.4857486	5.21915216426805e-09\\
0.4858486	3.52531628455138e-09\\
0.4859486	9.63905220398298e-10\\
0.4860486	-1.02589546580045e-09\\
0.4861486	-3.00326379043536e-09\\
0.4862486	-5.52573628260361e-09\\
0.4863486	-8.1492029322533e-09\\
0.4864486	-1.04279020861006e-08\\
0.4865487	1.03876973559305e-08\\
0.4866487	8.24991664219454e-09\\
0.4867487	5.80460060833204e-09\\
0.4868487	3.50417007043136e-09\\
0.4869487	8.02722951477597e-10\\
0.4870487	-8.43960438860758e-10\\
0.4871487	-3.97841236170204e-09\\
0.4872487	-6.14147244518137e-09\\
0.4873487	-8.87228250937011e-09\\
0.4874487	-1.07082814369136e-08\\
0.4875488	1.12122138596141e-08\\
0.4876488	8.67241747772723e-09\\
0.4877488	6.42585417775371e-09\\
0.4878488	3.9419508117769e-09\\
0.4879488	1.69186328922294e-09\\
0.4880488	-8.51518144190377e-10\\
0.4881488	-3.21356366690673e-09\\
0.4882488	-5.9178986173275e-09\\
0.4883488	-8.48639819788638e-09\\
0.4884488	-1.14391822412033e-08\\
0.4885489	1.12448027157069e-08\\
0.4886489	9.08696187566782e-09\\
0.4887489	6.99554880776582e-09\\
0.4888489	4.45752029384905e-09\\
0.4889489	9.61615168569574e-10\\
0.4890489	-1.00164041384049e-09\\
0.4891489	-3.93992748148886e-09\\
0.4892489	-6.35912893176549e-09\\
0.4893489	-8.76332413666386e-09\\
0.4894489	-1.16547835419048e-08\\
0.489549	1.21958913870679e-08\\
0.489649	9.95212010024282e-09\\
0.489749	6.72568502019938e-09\\
0.489849	4.02160741427154e-09\\
0.489949	1.34674466023554e-09\\
0.490049	-1.79020438490635e-09\\
0.490149	-3.87869378976058e-09\\
0.490249	-6.40632514658374e-09\\
0.490349	-9.85884201533394e-09\\
0.490449	-1.27201245005842e-08\\
0.4905491	1.24983553098992e-08\\
0.4906491	9.50227283923882e-09\\
0.4907491	6.65754339070762e-09\\
0.4908491	4.48779807413202e-09\\
0.4909491	1.51855915862809e-09\\
0.4910491	-1.7227543480311e-09\\
0.4911491	-4.70682104056208e-09\\
0.4912491	-6.90241162924071e-09\\
0.4913491	-9.77638323594598e-09\\
0.4914491	-1.27936738581608e-08\\
0.4915492	1.28460218851156e-08\\
0.4916492	1.02871868289459e-08\\
0.4917492	7.20233171713769e-09\\
0.4918492	4.13425339275408e-09\\
0.4919492	1.62769609856478e-09\\
0.4920492	-1.77064289098861e-09\\
0.4921492	-4.51210759467441e-09\\
0.4922492	-7.04607756168985e-09\\
0.4923492	-1.08199622386729e-08\\
0.4924492	-1.32791951870312e-08\\
0.4925493	1.37428738883685e-08\\
0.4926493	1.07222856175412e-08\\
0.4927493	7.69252229797235e-09\\
0.4928493	4.21611313049242e-09\\
0.4929493	1.85759191882739e-09\\
0.4930493	-1.81649711678694e-09\\
0.4931493	-4.23759364230614e-09\\
0.4932493	-7.83511530481895e-09\\
0.4933493	-1.10364519667269e-08\\
0.4934493	-1.42669599016709e-08\\
0.4935494	1.40628991394595e-08\\
0.4936494	1.16495600640046e-08\\
0.4937494	7.9438227315598e-09\\
0.4938494	4.52852583383845e-09\\
0.4939494	1.98857091990995e-09\\
0.4940494	-2.08907173070844e-09\\
0.4941494	-5.11535715869149e-09\\
0.4942494	-8.49915984552779e-09\\
0.4943494	-1.16472677691784e-08\\
0.4944494	-1.49643765079074e-08\\
0.4945495	1.4620516344474e-08\\
0.4946495	1.19090471858985e-08\\
0.4947495	8.82770695952162e-09\\
0.4948495	4.98023128032293e-09\\
0.4949495	1.97247797609212e-09\\
0.4950495	-1.58756695550072e-09\\
0.4951495	-5.08978331269921e-09\\
0.4952495	-8.92191078134819e-09\\
0.4953495	-1.24695427429836e-08\\
0.4954495	-1.61161204086796e-08\\
0.4955496	1.57514907150533e-08\\
0.4956496	1.19207893547676e-08\\
0.4957496	8.8544166608126e-09\\
0.4958496	5.17760307631315e-09\\
0.4959496	1.51776155185579e-09\\
0.4960496	-1.49550636371204e-09\\
0.4961496	-5.23040435615463e-09\\
0.4962496	-9.05293534234075e-09\\
0.4963496	-1.2326895239799e-08\\
0.4964496	-1.64138669080105e-08\\
0.4965497	1.69042353387674e-08\\
0.4966497	1.32771676032012e-08\\
0.4967497	8.76632566450958e-09\\
0.4968497	6.01904272257148e-09\\
0.4969497	1.6848958625873e-09\\
0.4970497	-1.58428777388697e-09\\
0.4971497	-6.13442474074744e-09\\
0.4972497	-9.30916920285907e-09\\
0.4973497	-1.34499065322646e-08\\
0.4974497	-1.68957472075626e-08\\
0.4975498	1.72413763405155e-08\\
0.4976498	1.3345485732133e-08\\
0.4977498	9.14154125661315e-09\\
0.4978498	5.29959683255588e-09\\
0.4979498	1.49201270956134e-09\\
0.4980498	-1.60653838149338e-09\\
0.4981498	-6.31906489287348e-09\\
0.4982498	-9.96625016021213e-09\\
0.4983498	-1.38664461354215e-08\\
0.4984498	-1.83356669829909e-08\\
0.4985499	1.82514407354867e-08\\
0.4986499	1.38806618369541e-08\\
0.4987499	1.00075977470249e-08\\
0.4988499	6.32565110882111e-09\\
0.4989499	1.53059413754431e-09\\
0.4990499	-1.67942503480345e-09\\
0.4991499	-6.60387597731238e-09\\
0.4992499	-1.05398394803502e-08\\
0.4993499	-1.47820013191718e-08\\
0.4994499	-1.86226456616589e-08\\
0.49955	1.83705063661055e-08\\
0.49965	1.46478645376286e-08\\
0.49975	1.04650775406556e-08\\
0.49985	6.53953421969611e-09\\
0.49995	2.59105726631292e-09\\
0.50005	-1.65809039241738e-09\\
0.50015	-6.48319905746653e-09\\
0.50025	-1.11571056837789e-08\\
0.50035	-1.49501876000463e-08\\
0.50045	-1.91303557162126e-08\\
0.5005501	1.96136328671492e-08\\
0.5006501	1.50549114095316e-08\\
0.5007501	1.13209746317039e-08\\
0.5008501	6.15384449127732e-09\\
0.5009501	2.29804201015704e-09\\
0.5010501	-2.49940618028688e-09\\
0.5011501	-6.48896145670486e-09\\
0.5012501	-1.09185662941491e-08\\
0.5013501	-1.60336379034787e-08\\
0.5014501	-2.00770615410573e-08\\
0.5015502	2.02158717283601e-08\\
0.5016502	1.57942249643522e-08\\
0.5017502	1.17315941856877e-08\\
0.5018502	6.7952907554375e-09\\
0.5019502	2.75519106140987e-09\\
0.5020502	-2.61625673035615e-09\\
0.5021502	-6.54402601725501e-09\\
0.5022502	-1.52505050775298e-08\\
0.5023502	-1.39554904420719e-08\\
0.5024502	-1.98761800889596e-08\\
0.5025503	2.52826316601146e-08\\
0.5026503	1.74941198576029e-08\\
0.5027503	1.48547596359044e-08\\
0.5028503	8.15781670340798e-09\\
0.5029503	-1.80081119287301e-09\\
0.5030503	-4.22258744810222e-09\\
0.5031503	-8.30632982187773e-09\\
0.5032503	-1.32482039019721e-08\\
0.5033503	-1.8241715946804e-08\\
0.5034503	-2.24777064721654e-08\\
0.5035504	2.44491523120755e-08\\
0.5036504	1.43791735914001e-08\\
0.5037504	7.51107686811081e-09\\
0.5038504	4.66475597385441e-09\\
0.5039504	6.6628046656303e-09\\
0.5040504	-5.66947663468986e-09\\
0.5041504	-1.15040738862121e-08\\
0.5042504	-1.001025261467e-08\\
0.5043504	-2.03545510515935e-08\\
0.5044504	-2.17007733189362e-08\\
0.5045505	1.85488175132031e-08\\
0.5046505	1.79394071039146e-08\\
0.5047505	8.8539799311671e-09\\
0.5048505	1.2139740010117e-08\\
0.5049505	-1.35334010491206e-09\\
0.5050505	-7.72512423821703e-10\\
0.5051505	-5.26224645455708e-09\\
0.5052505	-1.39642224675912e-08\\
0.5053505	-1.60173245535022e-08\\
0.5054505	-2.05576335329805e-08\\
0.5055506	2.72900131138842e-08\\
0.5056506	2.06079805824266e-08\\
0.5057506	1.40482611255682e-08\\
0.5058506	8.48605889378073e-09\\
0.5059506	4.79941622728852e-09\\
0.5060506	-6.1307796721255e-09\\
0.5061506	-3.42078954745917e-09\\
0.5062506	-1.61840152286541e-08\\
0.5063506	-2.35309925208232e-08\\
0.5064506	-2.45693842979011e-08\\
0.5065507	2.79412120975098e-08\\
0.5066507	2.24470822624001e-08\\
0.5067507	1.59567600001298e-08\\
0.5068507	9.37414763868549e-09\\
0.5069507	3.6060556138362e-09\\
0.5070507	-4.37790389010151e-10\\
0.5071507	-1.18447423900353e-08\\
0.5072507	-9.69922302716808e-09\\
0.5073507	-2.30827186514099e-08\\
0.5074507	-3.1073771921894e-08\\
0.5075508	2.60239417220615e-08\\
0.5076508	2.18416638751023e-08\\
0.5077508	1.58348640692807e-08\\
0.5078508	8.93684765773672e-09\\
0.5079508	2.08389885382465e-09\\
0.5080508	-3.78471177661655e-09\\
0.5081508	-7.72672037810396e-09\\
0.5082508	-1.87968627659435e-08\\
0.5083508	-2.60468671024056e-08\\
0.5084508	-2.85254468518348e-08\\
0.5085509	2.60132637167065e-08\\
0.5086509	2.62006507826401e-08\\
0.5087509	1.40324439481393e-08\\
0.5088509	1.0472063362299e-08\\
0.5089509	6.4859797584832e-09\\
0.5090509	-6.95627837797586e-09\\
0.5091509	-8.88211761432278e-09\\
0.5092509	-1.83158723465748e-08\\
0.5093509	-2.42787975210379e-08\\
0.5094509	-2.57890617194612e-08\\
0.509551	3.20453688480715e-08\\
0.509651	2.26652249555453e-08\\
0.509751	2.0702821122362e-08\\
0.509851	7.15241063251049e-09\\
0.509951	3.01136931016299e-09\\
0.510051	-7.19796845391474e-10\\
0.510151	-1.30374447314396e-08\\
0.510251	-1.29347866341389e-08\\
0.510351	-1.94018832371662e-08\\
0.510451	-3.14256360847004e-08\\
0.5105511	2.86318630107057e-08\\
0.5106511	1.88237556618398e-08\\
0.5107511	1.65183383377815e-08\\
0.5108511	1.27414215411042e-08\\
0.5109511	-1.4779888545674e-09\\
0.5110511	-5.10768417318064e-09\\
0.5111511	-7.11224569169278e-09\\
0.5112511	-1.64530373804317e-08\\
0.5113511	-2.20881984838023e-08\\
0.5114511	-3.29726361561478e-08\\
0.5115512	3.13802892750598e-08\\
0.5116512	2.34329190929722e-08\\
0.5117512	1.33920721925168e-08\\
0.5118512	1.23158459056083e-08\\
0.5119512	1.26560625868244e-09\\
0.5120512	1.30599510372864e-09\\
0.5121512	-6.49506208885708e-09\\
0.5122512	-2.10663492344887e-08\\
0.5123512	-2.13333521126291e-08\\
0.5124512	-3.62182510128817e-08\\
0.5125513	2.77204065569914e-08\\
0.5126513	2.71445248743729e-08\\
0.5127513	1.52052021910876e-08\\
0.5128513	1.29935570879203e-08\\
0.5129513	1.6040505347592e-09\\
0.5130513	-7.86550666612151e-09\\
0.5131513	-1.43139466599052e-08\\
0.5132513	-1.66367369602849e-08\\
0.5133513	-2.372597316877e-08\\
0.5134513	-3.44703714252881e-08\\
0.5135514	3.76357101872227e-08\\
0.5136514	2.32375317078815e-08\\
0.5137514	1.85395158704138e-08\\
0.5138514	1.46665414917091e-08\\
0.5139514	2.74690377735642e-09\\
0.5140514	-6.0876780769159e-09\\
0.5141514	-1.07020538125946e-08\\
0.5142514	-1.99576340835506e-08\\
0.5143514	-2.27123835988552e-08\\
0.5144514	-3.78208133006153e-08\\
0.5145515	3.43996552146231e-08\\
0.5146515	2.83547095490666e-08\\
0.5147515	2.34145045852063e-08\\
0.5148515	1.07384218114848e-08\\
0.5149515	1.48933364252368e-09\\
0.5150515	-3.16638919251044e-09\\
0.5151515	-1.20588701975344e-08\\
0.5152515	-2.40147194768836e-08\\
0.5153515	-2.78570265639655e-08\\
0.5154515	-3.24053526443158e-08\\
0.5155516	3.53160067405323e-08\\
0.5156516	3.32433701087653e-08\\
0.5157516	2.40284675767644e-08\\
0.5158516	8.8659290811699e-09\\
0.5159516	8.95395002240007e-09\\
0.5160516	-4.50570083673774e-09\\
0.5161516	-1.03076746305053e-08\\
0.5162516	-1.72430340214524e-08\\
0.5163516	-3.40992463536416e-08\\
0.5164516	-3.96601758066717e-08\\
0.5165517	4.24627147818657e-08\\
0.5166517	3.3499559834449e-08\\
0.5167517	1.95030168161081e-08\\
0.5168517	1.17037122389363e-08\\
0.5169517	1.33591308501546e-09\\
0.5170517	-3.62465731040817e-10\\
0.5171517	-1.21498536995168e-08\\
0.5172517	-2.27810173503151e-08\\
0.5173517	-3.10070526789163e-08\\
0.5174517	-3.55753775882484e-08\\
0.5175518	4.34386758473015e-08\\
0.5176518	3.03150731045809e-08\\
0.5177518	2.46303355766475e-08\\
0.5178518	1.76518360692213e-08\\
0.5179518	6.50663072032653e-10\\
0.5180518	-5.09837189480167e-09\\
0.5181518	-1.83167265695172e-08\\
0.5182518	-2.77221209876721e-08\\
0.5183518	-3.20285292211564e-08\\
0.5184518	-3.9946172307459e-08\\
0.5185519	4.21127116304043e-08\\
0.5186519	3.12266158798702e-08\\
0.5187519	2.06225187725229e-08\\
0.5188519	1.16052906382058e-08\\
0.5189519	5.48359248633834e-09\\
0.5190519	-6.43011645665159e-09\\
0.5191519	-1.2819571341971e-08\\
0.5192519	-2.23646940619715e-08\\
0.5193519	-3.37415858482626e-08\\
0.5194519	-4.56225195920312e-08\\
0.519552	3.93740638322815e-08\\
0.519652	3.08664104102518e-08\\
0.519752	2.58622818369303e-08\\
0.519852	1.57047971675872e-08\\
0.519952	1.74094104757327e-09\\
0.520052	-4.67842845713884e-09\\
0.520152	-1.21985725166918e-08\\
0.520252	-2.94608639567207e-08\\
0.520352	-3.51027800403469e-08\\
0.520452	-4.77578944932389e-08\\
0.5205521	4.38836659441555e-08\\
0.5206521	3.17135047216144e-08\\
0.5207521	2.66542988479335e-08\\
0.5208521	1.00881664376329e-08\\
0.5209521	3.40116672672974e-09\\
0.5210521	-2.01669285437855e-09\\
0.5211521	-1.47714491367801e-08\\
0.5212521	-2.34651757399784e-08\\
0.5213521	-3.66959754180551e-08\\
0.5214521	-4.30579727775382e-08\\
0.5215522	4.28254251401416e-08\\
0.5216522	3.4845010418727e-08\\
0.5217522	2.79763882234108e-08\\
0.5218522	1.36414257043138e-08\\
0.5219522	3.26600559791279e-09\\
0.5220522	-1.71996630479732e-09\\
0.5221522	-1.98825536382174e-08\\
0.5222522	-2.97837818470648e-08\\
0.5223522	-3.99816313806367e-08\\
0.5224522	-4.90300293143342e-08\\
0.5225523	5.26567048957094e-08\\
0.5226523	4.06864604993173e-08\\
0.5227523	2.42297384968337e-08\\
0.5228523	1.47489020718705e-08\\
0.5229523	3.7104042997782e-09\\
0.5230523	-7.41520381902783e-09\\
0.5231523	-1.71532663698071e-08\\
0.5232523	-2.40250149288745e-08\\
0.5233523	-3.65475612308652e-08\\
0.5234523	-5.32338896679085e-08\\
0.5235524	4.98571724391605e-08\\
0.5236524	3.97604454534239e-08\\
0.5237524	2.99873228019887e-08\\
0.5238524	1.20414077285635e-08\\
0.5239524	7.43046769599576e-09\\
0.5240524	-2.33355789552858e-09\\
0.5241524	-1.57345506034134e-08\\
0.5242524	-3.12522053479147e-08\\
0.5243524	-3.73620230968708e-08\\
0.5244524	-5.25353038259568e-08\\
0.5245525	5.16751313510422e-08\\
0.5246525	4.34326450186873e-08\\
0.5247525	3.07396215946942e-08\\
0.5248525	1.51416444903185e-08\\
0.5249525	8.18853495079819e-09\\
0.5250525	-8.56563994386383e-09\\
0.5251525	-1.35625605017509e-08\\
0.5252525	-3.52396468811245e-08\\
0.5253525	-4.20300517757488e-08\\
0.5254525	-5.23626530642141e-08\\
0.5255526	5.68704317284452e-08\\
0.5256526	4.46543445067893e-08\\
0.5257526	3.36367350778355e-08\\
0.5258526	1.54059020297803e-08\\
0.5259526	1.55445525201747e-09\\
0.5260526	-6.32067675282455e-09\\
0.5261526	-1.661823966978e-08\\
0.5262526	-2.77326460303151e-08\\
0.5263526	-4.80539679888203e-08\\
0.5264526	-5.59679298670068e-08\\
0.5265527	5.6453006911511e-08\\
0.5266527	4.87004808806535e-08\\
0.5267527	2.82259292674891e-08\\
0.5268527	1.66610948683366e-08\\
0.5269527	5.64210465153758e-09\\
0.5270527	-3.19052392491748e-09\\
0.5271527	-1.8191874669865e-08\\
0.5272527	-3.77126261746907e-08\\
0.5273527	-5.0099044312385e-08\\
0.5274527	-6.36929752134696e-08\\
0.5275528	5.4416057167217e-08\\
0.5276528	4.39022355067098e-08\\
0.5277528	3.7183627035059e-08\\
0.5278528	1.59361409901582e-08\\
0.5279528	1.84014217752271e-09\\
0.5280528	-3.41954186494631e-09\\
0.5281528	-1.81536136956151e-08\\
0.5282528	-3.06682990044993e-08\\
0.5283528	-4.92653393427411e-08\\
0.5284528	-6.22419852255862e-08\\
0.5285529	5.84628558342182e-08\\
0.5286529	5.23731453414e-08\\
0.5287529	3.70408142560952e-08\\
0.5288529	2.41866520492295e-08\\
0.5289529	5.53597461622202e-09\\
0.5290529	-7.18136844435335e-09\\
0.5291529	-2.22309873912363e-08\\
0.5292529	-3.7873944508382e-08\\
0.5293529	-5.23667477367318e-08\\
0.5294529	-6.39613433189856e-08\\
0.529553	6.07261262663972e-08\\
0.529653	5.07276772115928e-08\\
0.529753	3.89008001366875e-08\\
0.529853	1.70118692952026e-08\\
0.529953	6.83185542166287e-09\\
0.530053	-9.86366749389234e-09\\
0.530153	-2.12945152622268e-08\\
0.530253	-3.56758864811235e-08\\
0.530353	-5.12183556608536e-08\\
0.530453	-6.61278664673692e-08\\
0.5305531	6.84788911756162e-08\\
0.5306531	5.07911633793046e-08\\
0.5307531	3.914822767348e-08\\
0.5308531	2.53627373686394e-08\\
0.5309531	1.1252011072882e-08\\
0.5310531	-1.13619606639803e-08\\
0.5311531	-2.06525082938058e-08\\
0.5312531	-3.47882772450125e-08\\
0.5313531	-5.19332199327982e-08\\
0.5314531	-7.02465899130034e-08\\
0.5315532	6.48356697913188e-08\\
0.5316532	5.02999697726825e-08\\
0.5317532	4.0148201166379e-08\\
0.5318532	2.62399792588797e-08\\
0.5319532	1.0439651844385e-08\\
0.5320532	-5.38369368530134e-09\\
0.5321532	-1.93562239292344e-08\\
0.5322532	-3.95993526927652e-08\\
0.5323532	-5.42297343697484e-08\\
0.5324532	-7.1359257239334e-08\\
0.5325533	6.94428519756873e-08\\
0.5326533	5.35907216867026e-08\\
0.5327533	4.09343524215844e-08\\
0.5328533	2.33809907962401e-08\\
0.5329533	2.842682478299e-09\\
0.5330533	-8.76372142910586e-09\\
0.5331533	-2.95165574273892e-08\\
0.5332533	-3.74893441359991e-08\\
0.5333533	-6.0750774452667e-08\\
0.5334533	-7.73647099994168e-08\\
0.5335534	6.91570477551195e-08\\
0.5336534	5.22773853858793e-08\\
0.5337534	3.7884639821073e-08\\
0.5338534	2.79343451217717e-08\\
0.5339534	4.38689905929435e-09\\
0.5340534	-1.07924305950413e-08\\
0.5341534	-2.56334995660246e-08\\
0.5342534	-3.81612806424347e-08\\
0.5343534	-5.63958575949819e-08\\
0.5344534	-7.83524185601903e-08\\
0.5345535	7.87101748778163e-08\\
0.5346535	5.59149571574546e-08\\
0.5347535	4.53836289521226e-08\\
0.5348535	2.91206517927245e-08\\
0.5349535	9.13541417513208e-09\\
0.5350535	-1.25577622259176e-08\\
0.5351535	-2.39396166917816e-08\\
0.5352535	-4.29859425554735e-08\\
0.5353535	-5.76675822788753e-08\\
0.5354535	-7.59504200087413e-08\\
0.5355536	8.13600040117102e-08\\
0.5356536	6.26484859961307e-08\\
0.5357536	4.64699769359145e-08\\
0.5358536	2.48784892478704e-08\\
0.5359536	9.93302448626388e-09\\
0.5360536	-6.3024219662533e-09\\
0.5361536	-3.17588542547709e-08\\
0.5362536	-4.43622702878121e-08\\
0.5363536	-6.20336556358958e-08\\
0.5364536	-8.26889776095463e-08\\
0.5365537	7.95248561238915e-08\\
0.5366537	5.98461102271908e-08\\
0.5367537	4.34677960015062e-08\\
0.5368537	2.24940831863862e-08\\
0.5369537	9.03418947290069e-09\\
0.5370537	-1.47976138703543e-08\\
0.5371537	-2.68819963095135e-08\\
0.5372537	-4.509456194568e-08\\
0.5373537	-6.7305843870491e-08\\
0.5374537	-8.1381298776273e-08\\
0.5375538	8.54010979295383e-08\\
0.5376538	6.47147590307484e-08\\
0.5377538	4.86005482594476e-08\\
0.5378538	2.92133785564919e-08\\
0.5379538	8.71326757967417e-09\\
0.5380538	-1.07346555117127e-08\\
0.5381538	-2.69601452551615e-08\\
0.5382538	-4.77878342106108e-08\\
0.5383538	-7.10372273990401e-08\\
0.5384538	-8.45226965223711e-08\\
0.5385539	9.1562060565431e-08\\
0.5386539	6.48971358249706e-08\\
0.5387539	4.45860775036278e-08\\
0.5388539	3.28351065503885e-08\\
0.5389539	1.18556045050622e-08\\
0.5390539	-6.13588111569863e-09\\
0.5391539	-2.89176316919937e-08\\
0.5392539	-5.42627516184058e-08\\
0.5393539	-6.99391632594248e-08\\
0.5394539	-9.37096022307626e-08\\
0.539554	9.15369579856939e-08\\
0.539654	6.90485574481037e-08\\
0.539754	5.5211356160545e-08\\
0.539854	3.22834260713112e-08\\
0.539954	1.25280528173732e-08\\
0.540054	-1.17862600929053e-08\\
0.540154	-2.83857843619284e-08\\
0.540254	-5.49915632822029e-08\\
0.540354	-7.93194059531266e-08\\
0.540454	-9.90798824374028e-08\\
0.5405541	9.03683597119898e-08\\
0.5406541	7.73922016195083e-08\\
0.5407541	5.58854920673646e-08\\
0.5408541	2.81586727199357e-08\\
0.5409541	6.52744916451731e-09\\
0.5410541	-1.66872040213706e-08\\
0.5411541	-2.91590383089124e-08\\
0.5412541	-5.85565267177679e-08\\
0.5413541	-7.25428586922927e-08\\
0.5414541	-9.87759350493111e-08\\
0.5415542	9.51467284196483e-08\\
0.5416542	7.22512686762711e-08\\
0.5417542	5.41694988795416e-08\\
0.5418542	3.32647238990291e-08\\
0.5419542	1.19055606068175e-08\\
0.5420542	-7.53405823145092e-09\\
0.5421542	-3.26748778589625e-08\\
0.5422542	-6.11323182839185e-08\\
0.5423542	-8.05164689990367e-08\\
0.5424542	-9.84320849412024e-08\\
0.5425543	1.05520511363422e-07\\
0.5426543	7.85565433397695e-08\\
0.5427543	6.02813132952185e-08\\
0.5428543	3.31114556814671e-08\\
0.5429543	9.46896180908929e-09\\
0.5430543	-8.21881556561932e-09\\
0.5431543	-3.75191580789802e-08\\
0.5432543	-5.59939773991935e-08\\
0.5433543	-8.11998109584433e-08\\
0.5434543	-1.10687817841959e-07\\
0.5435544	1.04180230553474e-07\\
0.5436544	8.43277991463554e-08\\
0.5437544	5.75744975306325e-08\\
0.5438544	3.63907273343012e-08\\
0.5439544	1.32522895920917e-08\\
0.5440544	-9.35961129433505e-09\\
0.5441544	-3.89583647286695e-08\\
0.5442544	-6.30519480307457e-08\\
0.5443544	-8.91429235427843e-08\\
0.5444544	-1.14728434703892e-07\\
0.5445545	1.07314998408492e-07\\
0.5446545	8.11274686735031e-08\\
0.5447545	6.29890477554795e-08\\
0.5448545	3.54243124556231e-08\\
0.5449545	1.09632777162139e-08\\
0.5450545	-1.78585999044412e-08\\
0.5451545	-3.85004179093151e-08\\
0.5452545	-6.84158247554745e-08\\
0.5453545	-8.50530161855545e-08\\
0.5454545	-1.15854731155829e-07\\
0.5455546	1.1503984279071e-07\\
0.5456546	8.44849600417641e-08\\
0.5457546	6.74726866739528e-08\\
0.5458546	3.65821499204166e-08\\
0.5459546	1.43979510952574e-08\\
0.5460546	-1.64898313198547e-08\\
0.5461546	-4.34856379794013e-08\\
0.5462546	-6.39884261133261e-08\\
0.5463546	-9.53916654278242e-08\\
0.5464546	-1.15083334903399e-07\\
0.5465547	1.11792209211981e-07\\
0.5466547	9.42899868738856e-08\\
0.5467547	6.63710025411435e-08\\
0.5468547	4.06692767795924e-08\\
0.5469547	9.82433558116558e-09\\
0.5470547	-1.35187855077046e-08\\
0.5471547	-3.67095387698857e-08\\
0.5472547	-6.70918615525151e-08\\
0.5473547	-9.20041726554244e-08\\
0.5474547	-1.1877937039495e-07\\
0.5475548	1.16695969776259e-07\\
0.5476548	9.51532331816551e-08\\
0.5477548	6.97847566122345e-08\\
0.5478548	4.32797594845935e-08\\
0.5479548	8.33299637625906e-09\\
0.5480548	-1.23552409830863e-08\\
0.5481548	-4.60791200751487e-08\\
0.5482548	-7.01272667297248e-08\\
0.5483548	-1.01782760650315e-07\\
0.5484548	-1.28323133910535e-07\\
0.5485549	1.23891246789354e-07\\
0.5486549	9.67326686113212e-08\\
0.5487549	7.28926679571806e-08\\
0.5488549	4.51159332531814e-08\\
0.5489549	1.61527132225281e-08\\
0.5490549	-1.12411815134106e-08\\
0.5491549	-4.43043760706141e-08\\
0.5492549	-7.02699293197995e-08\\
0.5493549	-1.06365331278835e-07\\
0.5494549	-1.29812501498128e-07\\
0.549555	1.32828337412461e-07\\
0.549655	1.04023784280871e-07\\
0.549755	7.62379466917595e-08\\
0.549855	4.22712137654324e-08\\
0.549955	1.49295553419915e-08\\
0.550055	-1.29754762767664e-08\\
0.550155	-4.8626744533764e-08\\
0.550255	-7.9201526323458e-08\\
0.550355	-1.0187151171237e-07\\
0.550455	-1.33802800506011e-07\\
0.5505551	1.38523990861339e-07\\
0.5506551	1.07612011385205e-07\\
0.5507551	7.59768363362001e-08\\
0.5508551	4.64747450107383e-08\\
0.5509551	1.19676149962222e-08\\
0.5510551	-1.46770774839267e-08\\
0.5511551	-5.05862561978043e-08\\
0.5512551	-8.28812431880044e-08\\
0.5513551	-1.08677757543196e-07\\
0.5514551	-1.45085914449172e-07\\
0.5515552	1.4177828343287e-07\\
0.5516552	1.03885557892203e-07\\
0.5517552	7.40873909860218e-08\\
0.5518552	4.52961015501296e-08\\
0.5519552	1.04296183633945e-08\\
0.5520552	-1.75885192521541e-08\\
0.5521552	-4.58291600918814e-08\\
0.5522552	-8.1357540329674e-08\\
0.5523552	-1.11233282223955e-07\\
0.5524552	-1.42510394859907e-07\\
0.5525553	1.39350299341312e-07\\
0.5526553	1.15206871223461e-07\\
0.5527553	7.85367065547771e-08\\
0.5528553	4.23082675699016e-08\\
0.5529553	1.94956328858353e-08\\
0.5530553	-1.69214994238942e-08\\
0.5531553	-5.39578130925866e-08\\
0.5532553	-8.86223725932167e-08\\
0.5533553	-1.17918622838387e-07\\
0.5534553	-1.48844388925262e-07\\
0.5535554	1.44090822426501e-07\\
0.5536554	1.1004093222744e-07\\
0.5537554	8.34047951675521e-08\\
0.5538554	4.72070735280106e-08\\
0.5539554	1.44780507999043e-08\\
0.5540554	-2.17463684559771e-08\\
0.5541554	-4.84246589216308e-08\\
0.5542554	-8.25096743428677e-08\\
0.5543554	-1.20948647319707e-07\\
0.5544554	-1.6068319080853e-07\\
0.5545555	1.55030211959617e-07\\
0.5546555	1.23038549543208e-07\\
0.5547555	8.89632799372375e-08\\
0.5548555	4.58852770918172e-08\\
0.5549555	1.68910338771422e-08\\
0.5550555	-1.49273386429449e-08\\
0.5551555	-5.64721105433819e-08\\
0.5552555	-9.46399339868398e-08\\
0.5553555	-1.26321844595711e-07\\
0.5554555	-1.58403261174556e-07\\
0.5555556	1.57419637408784e-07\\
0.5556556	1.25072827608652e-07\\
0.5557556	9.17070808459375e-08\\
0.5558556	5.04594422773286e-08\\
0.5559556	1.44725696427095e-08\\
0.5560556	-2.31052682542199e-08\\
0.5561556	-5.91201920197015e-08\\
0.5562556	-9.04127126638565e-08\\
0.5563556	-1.33817733102837e-07\\
0.5564556	-1.66164548634495e-07\\
0.5565557	1.62723813756926e-07\\
0.5566557	1.23226952628008e-07\\
0.5567557	9.43372290701516e-08\\
0.5568557	4.92477686686588e-08\\
0.5569557	2.11572977470986e-08\\
0.5570557	-6.72985989158548e-09\\
0.5571557	-7.1203781440432e-08\\
0.5572557	-1.19048948990508e-07\\
0.5573557	-1.47044250428663e-07\\
0.5574557	-1.51962981448373e-07\\
0.5575558	1.48563391769163e-07\\
0.5576558	1.00731445226709e-07\\
0.5577558	8.56929653486604e-08\\
0.5578558	6.69701348543494e-09\\
0.5579558	-3.30017661392956e-08\\
0.5580558	-3.01431483662462e-08\\
0.5581558	-8.14613287936083e-08\\
0.5582558	-8.36849270852147e-08\\
0.5583558	-1.33536987814709e-07\\
0.5584558	-2.27734983981875e-07\\
0.5585559	1.2860512560442e-07\\
0.5586559	1.56849003729587e-07\\
0.5587559	5.0631239367771e-08\\
0.5588559	1.3256630168601e-08\\
0.5589559	4.80355335508165e-08\\
0.5590559	-4.17161356847107e-08\\
0.5591559	-5.26769091818702e-08\\
0.5592559	-8.15197654380206e-08\\
0.5593559	-1.2491213441379e-07\\
0.5594559	-1.79515899936811e-07\\
0.559556	1.6239795588846e-07\\
0.559656	9.67050752115339e-08\\
0.559756	1.29851751476395e-07\\
0.559856	6.5198261400834e-08\\
0.559956	6.11041245379695e-09\\
0.560056	-4.40404601537159e-08\\
0.560156	-8.18774965327673e-08\\
0.560256	-1.04018315655319e-07\\
0.560356	-1.0707501886531e-07\\
0.560456	-1.87654193448128e-07\\
0.5605561	1.75153119175153e-07\\
0.5606561	1.51062267167434e-07\\
0.5607561	5.96656454937561e-08\\
0.5608561	1.04378694035234e-07\\
0.5609561	-1.13776516125208e-08\\
0.5610561	1.58230368668688e-08\\
0.5611561	-1.10587324103628e-07\\
0.5612561	-8.71713344285929e-08\\
0.5613561	-1.10486112661623e-07\\
0.5614561	-1.77083301453052e-07\\
0.5615562	1.47466422836806e-07\\
0.5616562	1.06036733824033e-07\\
0.5617562	1.3170598947565e-07\\
0.5618562	2.79444158646225e-08\\
0.5619562	-1.77230553843888e-09\\
0.5620562	-5.39630440749006e-08\\
0.5621562	-2.51412218266189e-08\\
0.5622562	-1.11814820444311e-07\\
0.5623562	-1.10486384388708e-07\\
0.5624562	-2.17653026254871e-07\\
0.5625563	2.14980792098696e-07\\
0.5626563	2.02754648293002e-07\\
0.5627563	9.25781544619531e-08\\
0.5628563	8.79758832728061e-08\\
0.5629563	-7.52218554112005e-09\\
0.5630563	9.61933047105568e-09\\
0.5631563	-5.70587891379759e-08\\
0.5632563	-1.04010373702668e-07\\
0.5633563	-1.27683864981076e-07\\
0.5634563	-2.24522323677423e-07\\
0.5635564	1.67987228183453e-07\\
0.5636564	1.36946902401514e-07\\
0.5637564	1.43448235900845e-07\\
0.5638564	9.10696422717905e-08\\
0.5639564	-1.66051103145382e-08\\
0.5640564	-7.59869059341334e-08\\
0.5641564	-8.34812874661583e-08\\
0.5642564	-1.35488461450173e-07\\
0.5643564	-1.28403305099667e-07\\
0.5644564	-2.58615371145909e-07\\
0.5645565	2.50962312320357e-07\\
0.5646565	1.58480160578067e-07\\
0.5647565	1.39567637669202e-07\\
0.5648565	9.78564296261197e-08\\
0.5649565	3.69835149727216e-08\\
0.5650565	-3.94088421340477e-08\\
0.5651565	-2.76730970916184e-08\\
0.5652565	-1.24156432607103e-07\\
0.5653565	-1.25200766060152e-07\\
0.5654565	-2.27142753689291e-07\\
0.5655566	2.6204039166311e-07\\
0.5656566	1.70822414102823e-07\\
0.5657566	8.97319830384102e-08\\
0.5658566	2.24534149328992e-08\\
0.5659566	-2.73237466541509e-08\\
0.5660566	-5.59047411974323e-08\\
0.5661566	-5.95895977284489e-08\\
0.5662566	-1.34673142471176e-07\\
0.5663566	-1.77445005328236e-07\\
0.5664566	-1.84189628243958e-07\\
0.5665567	2.52418618645889e-07\\
0.5666567	1.30441206391829e-07\\
0.5667567	1.55672496671144e-07\\
0.5668567	3.18487300698944e-08\\
0.5669567	6.27112975160049e-08\\
0.5670567	-4.79932682206918e-08\\
0.5671567	-9.65133001851193e-08\\
0.5672567	-1.79092005281323e-07\\
0.5673567	-1.91967471530052e-07\\
0.5674567	-2.31372679626585e-07\\
0.5675568	2.25693001454363e-07\\
0.5676568	1.46132684564204e-07\\
0.5677568	1.51378050677886e-07\\
0.5678568	4.52164829745083e-08\\
0.5679568	3.14404299261994e-08\\
0.5680568	1.38474019195645e-08\\
0.5681568	-1.03760041308965e-07\\
0.5682568	-1.17574298463818e-07\\
0.5683568	-2.23782734481981e-07\\
0.5684568	-2.18567691029348e-07\\
0.5685569	2.37123675139728e-07\\
0.5686569	1.78279695450811e-07\\
0.5687569	1.42346059855969e-07\\
0.5688569	3.31604369796512e-08\\
0.5689569	5.45654742778456e-08\\
0.5690569	1.04087873875613e-08\\
0.5691569	-9.545704818259e-08\\
0.5692569	-1.59174507352589e-07\\
0.5693569	-1.76881123581385e-07\\
0.5694569	-2.447095023006e-07\\
0.569557	2.92827592229816e-07\\
0.569657	2.38037121708112e-07\\
0.569757	1.44760448100856e-07\\
0.569857	1.16884596335076e-07\\
0.569957	5.83014731214693e-08\\
0.570057	-2.70921458600526e-08\\
0.570157	-1.35394622846619e-07\\
0.570257	-1.62699469452576e-07\\
0.570357	-2.05095358363216e-07\\
0.570457	-2.58666132180174e-07\\
0.5705571	2.48896890597816e-07\\
0.5706571	1.86442722177604e-07\\
0.5707571	1.24594944583833e-07\\
0.5708571	6.72889314545788e-08\\
0.5709571	1.84648323575676e-08\\
0.5710571	-1.79324377014289e-08\\
0.5711571	-1.3795321052168e-07\\
0.5712571	-1.37643074188398e-07\\
0.5713571	-2.13042885804926e-07\\
0.5714571	-2.60188781020521e-07\\
0.5715572	2.10441191750288e-07\\
0.5716572	2.33451730086154e-07\\
0.5717572	9.66399645585536e-08\\
0.5718572	1.03988531424681e-07\\
0.5719572	5.9484729836301e-08\\
0.5720572	-3.28794909632002e-08\\
0.5721572	-6.91075409675612e-08\\
0.5722572	-1.45198203543861e-07\\
0.5723572	-2.57145646082035e-07\\
0.5724572	-3.00939433067748e-07\\
0.5725573	2.30552149815821e-07\\
0.5726573	2.3689353589873e-07\\
0.5727573	1.2345140959491e-07\\
0.5728573	9.4254502674973e-08\\
0.5729573	5.33360877408295e-08\\
0.5730573	4.73396788436276e-09\\
0.5731573	-4.75095381924362e-08\\
0.5732573	-1.99347607421352e-07\\
0.5733573	-2.46728926428519e-07\\
0.5734573	-2.85597703181262e-07\\
0.5735574	3.09188573299757e-07\\
0.5736574	2.01348777872923e-07\\
0.5737574	1.14219710384278e-07\\
0.5738574	5.18749486977388e-08\\
0.5739574	1.83924832031224e-08\\
0.5740574	-8.21452958432722e-08\\
0.5741574	-1.45651615185427e-07\\
0.5742574	-1.68035327791083e-07\\
0.5743574	-2.45200928700173e-07\\
0.5744574	-2.73048568338874e-07\\
0.5745575	2.91980516597956e-07\\
0.5746575	1.76945243318682e-07\\
0.5747575	1.23557521991557e-07\\
0.5748575	1.35934444422947e-07\\
0.5749575	1.81973774937561e-08\\
0.5750575	-2.55280520888856e-08\\
0.5751575	-9.11119717728059e-08\\
0.5752575	-1.74420277304632e-07\\
0.5753575	-2.713146478317e-07\\
0.5754575	-2.77652560193298e-07\\
0.5755576	2.68950746645791e-07\\
0.5756576	2.58070989300752e-07\\
0.5757576	1.50204483545524e-07\\
0.5758576	4.95104180708861e-08\\
0.5759576	6.01521097098878e-08\\
0.5760576	-1.37030118640968e-08\\
0.5761576	-6.78834189971056e-08\\
0.5762576	-1.98213503443956e-07\\
0.5763576	-2.00513588011209e-07\\
0.5764576	-2.70599944673933e-07\\
0.5765577	2.73152072877147e-07\\
0.5766577	2.82003176838019e-07\\
0.5767577	1.35647542143236e-07\\
0.5768577	1.38284950740308e-07\\
0.5769577	-5.88084306252235e-09\\
0.5770577	7.3578713122302e-09\\
0.5771577	-1.17787254866331e-07\\
0.5772577	-1.77100645343264e-07\\
0.5773577	-2.66362816470078e-07\\
0.5774577	-2.81350391900803e-07\\
0.5775578	2.79219046495083e-07\\
0.5776578	2.27451126766365e-07\\
0.5777578	1.62655353436181e-07\\
0.5778578	8.9070508274558e-08\\
0.5779578	1.09391815050763e-08\\
0.5780578	-6.74922446264326e-08\\
0.5781578	-1.41973612227675e-07\\
0.5782578	-2.08251008160243e-07\\
0.5783578	-2.62066776390846e-07\\
0.5784578	-2.99159538696969e-07\\
0.5785579	3.01832613874842e-07\\
0.5786579	2.13012200683815e-07\\
0.5787579	1.53725370100055e-07\\
0.5788579	1.28248221598071e-07\\
0.5789579	4.08604924340139e-08\\
0.5790579	-4.15446321788693e-09\\
0.5791579	-1.0250969070813e-07\\
0.5792579	-1.49914653968519e-07\\
0.5793579	-2.42075253009677e-07\\
0.5794579	-3.74693842092322e-07\\
0.579558	2.94096367314811e-07\\
0.579658	2.95539130223954e-07\\
0.579758	1.69442241265694e-07\\
0.579858	1.20117349011739e-07\\
0.579958	5.18795547355122e-08\\
0.580058	-3.09526045072506e-08\\
0.580158	-1.24057176559278e-07\\
0.580258	-2.2310881112042e-07\\
0.580358	-2.23778782615369e-07\\
0.580458	-3.21735003777635e-07\\
0.5805581	3.45823070208029e-07\\
0.5806581	2.68417508803021e-07\\
0.5807581	2.06746262165769e-07\\
0.5808581	6.51546609720111e-08\\
0.5809581	4.79913006279364e-08\\
0.5810581	-4.0391982314425e-08\\
0.5811581	-9.56401286222874e-08\\
0.5812581	-2.13394875936412e-07\\
0.5813581	-2.89294779132199e-07\\
0.5814581	-3.18975230562124e-07\\
0.5815582	3.81730255599733e-07\\
0.5816582	2.59752228809473e-07\\
0.5817582	1.9711062798633e-07\\
0.5818582	9.81825137302827e-08\\
0.5819582	6.73480070878529e-08\\
0.5820582	-9.10097300577117e-08\\
0.5821582	-1.72504515172367e-07\\
0.5822582	-1.72747167598297e-07\\
0.5823582	-2.8734552659504e-07\\
0.5824582	-4.11904476804725e-07\\
0.5825583	3.59543698702502e-07\\
0.5826583	2.30461684402883e-07\\
0.5827583	2.0462636882157e-07\\
0.5828583	8.64444909820516e-08\\
0.5829583	8.03256373638206e-08\\
0.5830583	-9.31777258128541e-09\\
0.5831583	-1.78070512557582e-07\\
0.5832583	-2.21514570464265e-07\\
0.5833583	-3.35229168813012e-07\\
0.5834583	-4.14790789105002e-07\\
0.5835584	3.68007713968943e-07\\
0.5836584	2.7227869320301e-07\\
0.5837584	2.23993875503314e-07\\
0.5838584	1.27587531317808e-07\\
0.5839584	-1.25034382847389e-08\\
0.5840584	-9.18395214877243e-08\\
0.5841584	-1.05978622322311e-07\\
0.5842584	-2.5047607987716e-07\\
0.5843584	-3.20884692539547e-07\\
0.5844584	-4.12754740164445e-07\\
0.5845585	3.24801233519967e-07\\
0.5846585	3.0565710312791e-07\\
0.5847585	1.78420030305659e-07\\
0.5848585	1.47549577500516e-07\\
0.5849585	1.75077105035726e-08\\
0.5850585	-7.2412296714397e-09\\
0.5851585	-1.22230543020407e-07\\
0.5852585	-2.22991201337397e-07\\
0.5853585	-3.05051865934214e-07\\
0.5854585	-3.6393891622577e-07\\
0.5855586	3.74358774479777e-07\\
0.5856586	2.77583200425369e-07\\
0.5857586	2.17420047521344e-07\\
0.5858586	9.8351833732746e-08\\
0.5859586	2.48632399113546e-08\\
0.5860586	-9.8558913974589e-08\\
0.5861586	-1.67425695585255e-07\\
0.5862586	-2.77246081092919e-07\\
0.5863586	-3.23526982815014e-07\\
0.5864586	-4.01773272484807e-07\\
0.5865587	3.85595429253094e-07\\
0.5866587	2.59291061899258e-07\\
0.5867587	2.14523179831172e-07\\
0.5868587	1.55794822653821e-07\\
0.5869587	-1.23890558922923e-08\\
0.5870587	-8.55216162332084e-08\\
0.5871587	-1.59094151241934e-07\\
0.5872587	-2.28596118090341e-07\\
0.5873587	-2.89515161289167e-07\\
0.5874587	-4.37337140002381e-07\\
0.5875588	4.49535162649894e-07\\
0.5876588	3.43881145781211e-07\\
0.5877588	2.64881596379052e-07\\
0.5878588	1.17057541992871e-07\\
0.5879588	4.93166713022575e-09\\
0.5880588	-6.69717152845806e-08\\
0.5881588	-1.9412668905705e-07\\
0.5882588	-2.72005762191885e-07\\
0.5883588	-2.96079892825318e-07\\
0.5884588	-4.61818516372947e-07\\
0.5885589	3.76841736537958e-07\\
0.5886589	3.43841448313675e-07\\
0.5887589	1.82781764834772e-07\\
0.5888589	9.81990728177173e-08\\
0.5889589	-5.36885325086445e-09\\
0.5890589	-2.33828770235256e-08\\
0.5891589	-1.5130253444573e-07\\
0.5892589	-2.84586052168301e-07\\
0.5893589	-3.18690380418651e-07\\
0.5894589	-4.49071221098052e-07\\
0.589559	3.95251741669433e-07\\
0.589659	2.88470722531464e-07\\
0.589759	1.99057853422779e-07\\
0.589859	1.31562152005937e-07\\
0.589959	-9.46625433595472e-09\\
0.590059	-1.94761584637604e-08\\
0.590159	-1.93915299762093e-07\\
0.590259	-2.28230395526996e-07\\
0.590359	-3.17867166199637e-07\\
0.590459	-4.58270367376112e-07\\
0.5905591	4.46909263762674e-07\\
0.5906591	3.21203285191807e-07\\
0.5907591	2.58406677611234e-07\\
0.5908591	1.63078262505056e-07\\
0.5909591	3.97776815375117e-08\\
0.5910591	-1.06934631594058e-07\\
0.5911591	-1.72497479855593e-07\\
0.5912591	-2.52348937690439e-07\\
0.5913591	-3.41926372049439e-07\\
0.5914591	-4.36666479641801e-07\\
0.5915592	4.85601872712849e-07\\
0.5916592	3.96835127869366e-07\\
0.5917592	2.16603927127856e-07\\
0.5918592	1.49473970090241e-07\\
0.5919592	1.14772635839699e-11\\
0.5920592	-2.72168384496396e-08\\
0.5921592	-1.27643801417499e-07\\
0.5922592	-2.96701804958222e-07\\
0.5923592	-3.29822840816618e-07\\
0.5924592	-5.2243853160272e-07\\
0.5925593	4.73897683761493e-07\\
0.5926593	3.78651073384972e-07\\
0.5927593	2.37621422705114e-07\\
0.5928593	1.55378281907481e-07\\
0.5929593	3.64914147787943e-08\\
0.5930593	-1.4469231424119e-08\\
0.5931593	-1.92933556242281e-07\\
0.5932593	-2.94331338579479e-07\\
0.5933593	-4.14092270695221e-07\\
0.5934593	-4.47645985890421e-07\\
0.5935594	4.80183392437272e-07\\
0.5936594	3.3545291810988e-07\\
0.5937594	2.90645357448049e-07\\
0.5938594	1.50330984161062e-07\\
0.5939594	1.90799729904434e-08\\
0.5940594	-9.85376381601455e-08\\
0.5941594	-1.9795197858663e-07\\
0.5942594	-2.74593379201438e-07\\
0.5943594	-4.23892401457682e-07\\
0.5944594	-5.41279874310963e-07\\
0.5945595	4.75603303407013e-07\\
0.5946595	3.38488554130212e-07\\
0.5947595	2.46995516595128e-07\\
0.5948595	1.05691969953803e-07\\
0.5949595	1.9145264262832e-08\\
0.5950595	-1.08077715332833e-07\\
0.5951595	-1.71410579785647e-07\\
0.5952595	-2.66287471684112e-07\\
0.5953595	-3.88143099772087e-07\\
0.5954595	-5.32412768428081e-07\\
0.5955596	5.30899454731504e-07\\
0.5956596	3.58282225498829e-07\\
0.5957596	2.76945674893003e-07\\
0.5958596	1.91451767606043e-07\\
0.5959596	6.36169494860894e-09\\
0.5960596	-7.37641500103692e-08\\
0.5961596	-1.44366211818348e-07\\
0.5962596	-3.00885805992746e-07\\
0.5963596	-4.38765155374998e-07\\
0.5964596	-5.53447421891562e-07\\
0.5965597	5.13153138592681e-07\\
0.5966597	3.61366200962721e-07\\
0.5967597	2.46445420792885e-07\\
0.5968597	1.72943523335523e-07\\
0.5969597	4.54121188431245e-08\\
0.5970597	-3.15983332277447e-08\\
0.5971597	-1.53538558644861e-07\\
0.5972597	-3.15860506194809e-07\\
0.5973597	-4.1401737893576e-07\\
0.5974597	-5.43463676894174e-07\\
0.5975598	4.82428163584014e-07\\
0.5976598	4.0691424585404e-07\\
0.5977598	3.13743838964431e-07\\
0.5978598	2.07456915313209e-07\\
0.5979598	9.25919719230173e-08\\
0.5980598	-2.63139977030313e-08\\
0.5981598	-1.44725547934765e-07\\
0.5982598	-3.58108814046965e-07\\
0.5983598	-4.61931548068062e-07\\
0.5984598	-5.51663158321336e-07\\
0.5985599	4.88316427571522e-07\\
0.5986599	4.43277455464131e-07\\
0.5987599	3.25915616883332e-07\\
0.5988599	1.40754516486652e-07\\
0.5989599	9.23159212407043e-08\\
0.5990599	-1.14880278678786e-07\\
0.5991599	-1.76316106659247e-07\\
0.5992599	-2.87475535820647e-07\\
0.5993599	-4.43844528463799e-07\\
0.5994599	-5.40911070545658e-07\\
0.59956	5.66386422484744e-07\\
0.59966	4.04423098787632e-07\\
0.59976	3.15290234087495e-07\\
0.59986	2.03491359318519e-07\\
0.59996	7.35277895236308e-08\\
0.60006	-7.01014158388524e-08\\
0.60016	-2.22899485313732e-07\\
0.60026	-2.80371976912885e-07\\
0.60036	-4.38026819205017e-07\\
0.60046	-5.913743451047e-07\\
0.6005601	5.3453551318583e-07\\
0.6006601	4.06277322340109e-07\\
0.6007601	2.95785102522039e-07\\
0.6008601	2.07538503826932e-07\\
0.6009601	4.60145771485898e-08\\
0.6010601	-8.43122665572515e-08\\
0.6011601	-1.78970295416914e-07\\
0.6012601	-3.33490494275335e-07\\
0.6013601	-4.43406606709384e-07\\
0.6014601	-6.04255169567303e-07\\
0.6015602	5.89246896220974e-07\\
0.6016602	4.42972641012673e-07\\
0.6017602	3.59143592698619e-07\\
0.6018602	1.42211627307542e-07\\
0.6019602	9.66256245571451e-08\\
0.6020602	-7.31685691479633e-08\\
0.6021602	-1.62728187425021e-07\\
0.6022602	-3.6761357413706e-07\\
0.6023602	-4.83388231109494e-07\\
0.6024602	-6.05618854787604e-07\\
0.6025603	6.01752379369991e-07\\
0.6026603	4.83001358686508e-07\\
0.6027603	2.71079121150208e-07\\
0.6028603	1.70405776600546e-07\\
0.6029603	8.53980372528262e-08\\
0.6030603	-7.95308264800454e-08\\
0.6031603	-2.19971022508147e-07\\
0.6032603	-3.31516282536803e-07\\
0.6033603	-5.09763900369276e-07\\
0.6034603	-6.50314775973992e-07\\
0.6035604	6.14102174056885e-07\\
0.6036604	4.65276147665339e-07\\
0.6037604	3.6732651570448e-07\\
0.6038604	2.24637544565809e-07\\
0.6039604	4.15896895256118e-08\\
0.6040604	-7.74404469994217e-08\\
0.6041604	-2.2808016236775e-07\\
0.6042604	-4.0596068739962e-07\\
0.6043604	-5.06717232617859e-07\\
0.6044604	-6.25989031000085e-07\\
0.6045605	6.35143156846318e-07\\
0.6046605	4.95099229946838e-07\\
0.6047605	3.49602144567029e-07\\
0.6048605	2.02996157494528e-07\\
0.6049605	5.96212880732416e-08\\
0.6050605	-7.61867177678788e-08\\
0.6051605	-2.00096435143404e-07\\
0.6052605	-4.07780799627222e-07\\
0.6053605	-4.94917151205865e-07\\
0.6054605	-6.57187277486315e-07\\
0.6055606	6.36407096799019e-07\\
0.6056606	4.40041696192495e-07\\
0.6057606	3.8147434383573e-07\\
0.6058606	1.65005026198095e-07\\
0.6059606	9.49290656082624e-08\\
0.6060606	-1.24462921857749e-07\\
0.6061606	-1.88885069918321e-07\\
0.6062606	-3.9405631202527e-07\\
0.6063606	-5.35700413140106e-07\\
0.6064606	-7.09546023602847e-07\\
0.6065607	6.47910591844436e-07\\
0.6066607	5.25734727196969e-07\\
0.6067607	3.84145926290991e-07\\
0.6068607	2.273955691523e-07\\
0.6069607	5.97299312232735e-08\\
0.6070607	-1.1460985838152e-07\\
0.6071607	-1.91387861669945e-07\\
0.6072607	-3.66373375304541e-07\\
0.6073607	-5.35340976548504e-07\\
0.6074607	-6.94070568130201e-07\\
0.6075608	6.53868592603857e-07\\
0.6076608	5.31574560813652e-07\\
0.6077608	3.32150639215101e-07\\
0.6078608	1.59795173870769e-07\\
0.6079608	1.87009621122058e-08\\
0.6080608	-8.69447930318401e-08\\
0.6081608	-2.52960528523971e-07\\
0.6082608	-3.75170363220434e-07\\
0.6083608	-5.49404151839461e-07\\
0.6084608	-6.71497519366682e-07\\
0.6085609	6.88323506081723e-07\\
0.6086609	4.86343300631908e-07\\
0.6087609	3.48965674468182e-07\\
0.6088609	1.80331434673775e-07\\
0.6089609	8.45753886924427e-08\\
0.6090609	-1.34173707033725e-07\\
0.6091609	-2.71793190398384e-07\\
0.6092609	-4.24166543111504e-07\\
0.6093609	-5.87183433209404e-07\\
0.6094609	-7.56739763518866e-07\\
0.609561	7.3069213091248e-07\\
0.609661	5.63747741044729e-07\\
0.609761	4.0254241828741e-07\\
0.609861	2.51154855135383e-07\\
0.609961	1.36572759767972e-08\\
0.610061	-1.05884607193296e-07\\
0.610161	-3.03411639812623e-07\\
0.610261	-3.7487127446667e-07\\
0.610361	-6.16217619930914e-07\\
0.610461	-7.23411482272951e-07\\
0.6105611	7.01232731326051e-07\\
0.6106611	5.77878603968962e-07\\
0.6107611	4.00757884211345e-07\\
0.6108611	1.73882489629129e-07\\
0.6109611	1.01257395246357e-07\\
0.6110611	-1.13119401068396e-07\\
0.6111611	-2.65256934861569e-07\\
0.6112611	-4.51171318749388e-07\\
0.6113611	-5.66885791530858e-07\\
0.6114611	-7.08430766493962e-07\\
0.6115612	7.56434836102216e-07\\
0.6116612	5.78592750266615e-07\\
0.6117612	3.86789349882122e-07\\
0.6118612	1.84965042726581e-07\\
0.6119612	7.70528243432267e-08\\
0.6120612	-1.3302177270802e-07\\
0.6121612	-2.41340721074401e-07\\
0.6122612	-4.43993555032662e-07\\
0.6123612	-6.37077409382236e-07\\
0.6124612	-7.16697079905515e-07\\
0.6125613	6.84334402589748e-07\\
0.6126613	5.4682104100312e-07\\
0.6127613	4.34414950789552e-07\\
0.6128613	2.50980234040021e-07\\
0.6129613	1.003730899507e-07\\
0.6130613	-1.13558221159771e-07\\
0.6131613	-2.86973434682203e-07\\
0.6132613	-4.16040329853651e-07\\
0.6133613	-5.9693477727496e-07\\
0.6134613	-8.25840788420962e-07\\
0.6135614	6.9975727123861e-07\\
0.6136614	5.89804783235337e-07\\
0.6137614	4.43243147518757e-07\\
0.6138614	2.63855289728276e-07\\
0.6139614	5.54157481014173e-08\\
0.6140614	-7.83093748735197e-08\\
0.6141614	-2.33562462481629e-07\\
0.6142614	-4.06594433233209e-07\\
0.6143614	-5.93664789216852e-07\\
0.6144614	-7.91041665948455e-07\\
0.6145615	7.39493895118848e-07\\
0.6146615	6.36263761677469e-07\\
0.6147615	4.33874093319631e-07\\
0.6148615	2.36021709842937e-07\\
0.6149615	4.63945497575224e-08\\
0.6150615	-1.31328386387963e-07\\
0.6151615	-2.93477075752335e-07\\
0.6152615	-4.36390531155517e-07\\
0.6153615	-6.56416848765495e-07\\
0.6154615	-8.49913257550128e-07\\
0.6155616	7.57408605056575e-07\\
0.6156616	6.3149913875904e-07\\
0.6157616	4.42996201055479e-07\\
0.6158616	2.95505514502636e-07\\
0.6159616	9.26234194897901e-08\\
0.6160616	-6.20631818160078e-08\\
0.6161616	-2.64976871333644e-07\\
0.6162616	-5.12549770181892e-07\\
0.6163616	-6.01223584806299e-07\\
0.6164616	-8.27449660012469e-07\\
0.6165617	8.19486730918584e-07\\
0.6166617	6.32434562053419e-07\\
0.6167617	4.18421270298452e-07\\
0.6168617	2.80956435338986e-07\\
0.6169617	2.35397484837918e-08\\
0.6170617	-1.50339043436531e-07\\
0.6171617	-3.37200192990217e-07\\
0.6172617	-4.33573993818825e-07\\
0.6173617	-6.36000839104867e-07\\
0.6174617	-8.41031268961068e-07\\
0.6175618	7.98823195147236e-07\\
0.6176618	6.02599098176171e-07\\
0.6177618	4.14061803288845e-07\\
0.6178618	2.36619651206027e-07\\
0.6179618	7.367057830443e-08\\
0.6180618	-7.13979372291007e-08\\
0.6181618	-2.95208923528634e-07\\
0.6182618	-4.94395971162298e-07\\
0.6183618	-6.65603279603744e-07\\
0.6184618	-9.05485713631471e-07\\
0.6185619	8.70556232102082e-07\\
0.6186619	6.07055682166902e-07\\
0.6187619	4.84854223881115e-07\\
0.6188619	3.07253808506935e-07\\
0.6189619	7.75454691570587e-08\\
0.6190619	-1.00990738105544e-07\\
0.6191619	-3.25085785957135e-07\\
0.6192619	-4.91481722431075e-07\\
0.6193619	-6.96931725441274e-07\\
0.6194619	-8.38200156727531e-07\\
0.619562	8.06750183324212e-07\\
0.619662	6.072790428302e-07\\
0.619762	4.81631963755724e-07\\
0.619862	2.32999322502536e-07\\
0.619962	6.45600498616616e-08\\
0.620062	-1.20518420798632e-07\\
0.620162	-3.19080204258526e-07\\
0.620262	-5.27981016240631e-07\\
0.620362	-7.44088224924333e-07\\
0.620462	-8.6428090628754e-07\\
0.6205621	8.71231383925419e-07\\
0.6206621	6.55987113074019e-07\\
0.6207621	4.45952463401511e-07\\
0.6208621	2.44200996646349e-07\\
0.6209621	5.37943090073156e-08\\
0.6210621	-1.22218027032162e-07\\
0.6211621	-2.80798509666447e-07\\
0.6212621	-5.18921773551284e-07\\
0.6213621	-7.33574624267597e-07\\
0.6214621	-9.21756102889582e-07\\
0.6215622	9.14381380723306e-07\\
0.6216622	6.91930251428552e-07\\
0.6217622	5.04882401108908e-07\\
0.6218622	2.56189318093014e-07\\
0.6219622	4.87899917134271e-08\\
0.6220622	-1.14389138050619e-07\\
0.6221622	-3.30434236861521e-07\\
0.6222622	-4.96444124586937e-07\\
0.6223622	-7.09530332931863e-07\\
0.6224622	-9.66817152669996e-07\\
0.6225623	8.67891895794948e-07\\
0.6226623	7.34642591093682e-07\\
0.6227623	4.6574551859635e-07\\
0.6228623	2.64024792917539e-07\\
0.6229623	3.22914954775655e-08\\
0.6230623	-1.26656369614508e-07\\
0.6231623	-3.10033935635445e-07\\
0.6232623	-5.15069521922484e-07\\
0.6233623	-7.39004695993195e-07\\
0.6234623	-9.79094312292261e-07\\
0.6235624	9.39486050288707e-07\\
0.6236624	6.79160181316085e-07\\
0.6237624	5.10837694034372e-07\\
0.6238624	2.3720999919874e-07\\
0.6239624	6.0954952085801e-08\\
0.6240624	-1.15263209310434e-07\\
0.6241624	-3.88793916505925e-07\\
0.6242624	-5.57000320866585e-07\\
0.6243624	-7.17259354576782e-07\\
0.6244624	-9.66961773207764e-07\\
0.6245625	9.07610625211497e-07\\
0.6246625	6.90710592499144e-07\\
0.6247625	4.92113945815831e-07\\
0.6248625	3.14374060228495e-07\\
0.6249625	6.00302114683871e-08\\
0.6250625	-1.68392473076651e-07\\
0.6251625	-3.68383073157119e-07\\
0.6252625	-5.37444921455688e-07\\
0.6253625	-7.73095671591228e-07\\
0.6254625	-9.72867334914085e-07\\
0.6255626	9.16104167414389e-07\\
0.6256626	6.99378895241409e-07\\
0.6257626	5.25852357768919e-07\\
0.6258626	2.97934530046895e-07\\
0.6259626	1.18020748840486e-07\\
0.6260626	-1.11508332678589e-07\\
0.6261626	-3.88286803265636e-07\\
0.6262626	-6.09963546227554e-07\\
0.6263626	-7.74202294717696e-07\\
0.6264626	-9.78681684249416e-07\\
0.6265627	9.68846062265882e-07\\
0.6266627	6.94755107222278e-07\\
0.6267627	4.87299922857787e-07\\
0.6268627	3.48741712730138e-07\\
0.6269627	8.13265085275461e-08\\
0.6270627	-1.12714883204745e-07\\
0.6271627	-3.31166934430804e-07\\
0.6272627	-5.71829449347305e-07\\
0.6273627	-8.32517614957062e-07\\
0.6274627	-1.01106206773505e-06\\
0.6275628	1.02439167959378e-06\\
0.6276628	7.20568239853669e-07\\
0.6277628	5.05305562836789e-07\\
0.6278628	2.80710709454901e-07\\
0.6279628	4.88750222515932e-08\\
0.6280628	-1.88125911404491e-07\\
0.6281628	-3.28232319990462e-07\\
0.6282628	-5.6940030568331e-07\\
0.6283628	-8.09601889639744e-07\\
0.6284628	-1.0468250737361e-06\\
0.6285629	9.90599014039084e-07\\
0.6286629	7.69312388682053e-07\\
0.6287629	5.56945715324275e-07\\
0.6288629	3.55446530653225e-07\\
0.6289629	6.6746125404471e-08\\
0.6290629	-1.07240508806505e-07\\
0.6291629	-3.6461473362781e-07\\
0.6292629	-6.03494317297226e-07\\
0.6293629	-8.22013489276685e-07\\
0.6294629	-1.01832298970983e-06\\
0.629563	1.01925223838784e-06\\
0.629663	7.76870282326314e-07\\
0.629763	5.62147938309998e-07\\
0.629863	2.7686785275538e-07\\
0.629963	1.22795895340921e-07\\
0.630063	-9.83189082370828e-08\\
0.630163	-3.84744414327898e-07\\
0.630263	-6.34765423335182e-07\\
0.630363	-8.46683732680376e-07\\
0.630463	-1.11881818865012e-06\\
0.6305631	1.00068786934582e-06\\
0.6306631	8.17140027531416e-07\\
0.6307631	5.78318288702029e-07\\
0.6308631	2.85835057289852e-07\\
0.6309631	4.12854186393474e-08\\
0.6310631	-1.53752909182181e-07\\
0.6311631	-3.97719632783122e-07\\
0.6312631	-5.89071935119811e-07\\
0.6313631	-8.26284529353849e-07\\
0.6314631	-1.107849708748e-06\\
0.6315632	1.05842925035571e-06\\
0.6316632	7.96670746039396e-07\\
0.6317632	5.94978180168404e-07\\
0.6318632	3.5478837367986e-07\\
0.6319632	7.75202996239344e-08\\
0.6320632	-1.35424971570153e-07\\
0.6321632	-4.82664329837235e-07\\
0.6322632	-3.62832665912194e-07\\
0.6323632	-1.17458293580341e-06\\
0.6324632	-9.16586208710157e-07\\
0.6325633	8.43835330721632e-07\\
0.6326633	1.24931311029641e-06\\
0.6327633	7.28416727380221e-07\\
0.6328633	2.82402105256097e-07\\
0.6329633	-8.74932077898904e-08\\
0.6330633	-3.80050098502949e-07\\
0.6331633	-5.94067936399156e-07\\
0.6332633	-7.28364629054568e-07\\
0.6333633	-7.81776672953427e-07\\
0.6334633	-7.53159207889098e-07\\
0.6335634	8.30769697479639e-07\\
0.6336634	1.03089074099394e-06\\
0.6337634	3.16364573205163e-07\\
0.6338634	6.88260941350549e-07\\
0.6339634	1.47630682567623e-07\\
0.6340634	-3.04494313763115e-07\\
0.6341634	-6.67101180962248e-07\\
0.6342634	-9.39196106219953e-07\\
0.6343634	-1.11980439476689e-06\\
0.6344634	-1.2079705151713e-06\\
0.6345635	1.31029609518052e-06\\
0.6346635	4.13898717299688e-07\\
0.6347635	6.12695455259882e-07\\
0.6348635	-9.24353686926338e-08\\
0.6349635	2.99365139788677e-07\\
0.6350635	-2.11063607036266e-07\\
0.6351635	-6.22901727087566e-07\\
0.6352635	-9.35348922315171e-07\\
0.6353635	-1.14762452918526e-06\\
0.6354635	-1.2589675706387e-06\\
0.6355636	1.28540665267352e-06\\
0.6356636	3.82235412876852e-07\\
0.6357636	5.82161719275831e-07\\
0.6358636	-1.14132735040684e-07\\
0.6359636	2.94013795310022e-07\\
0.6360636	-1.92756938632854e-07\\
0.6361636	-5.73823232707937e-07\\
0.6362636	-8.48583490586918e-07\\
0.6363636	-1.01645625671942e-06\\
0.6364636	-1.07688028161412e-06\\
0.6365637	1.5657892198373e-06\\
0.6366637	7.25974130144635e-07\\
0.6367637	9.95170294393333e-07\\
0.6368637	3.73857626412288e-07\\
0.6369637	-1.37504421404699e-07\\
0.6370637	-5.38476904621277e-07\\
0.6371637	-8.28641439243327e-07\\
0.6372637	-1.00760024812629e-06\\
0.6373637	-1.07497621737451e-06\\
0.6374637	-1.03041294430284e-06\\
0.6375638	7.62640351981148e-07\\
0.6376638	1.0361809632542e-06\\
0.6377638	4.22605536121168e-07\\
0.6378638	-7.78128970146952e-08\\
0.6379638	5.3517773257461e-07\\
0.6380638	2.61808478674297e-07\\
0.6381638	-8.97710667802087e-07\\
0.6382638	-9.43190829438834e-07\\
0.6383638	-8.74464299416289e-07\\
0.6384638	-1.69138457839502e-06\\
0.6385639	1.28353045614205e-06\\
0.6386639	6.99785571933731e-07\\
0.6387639	2.30705660575481e-07\\
0.6388639	-1.23648163041779e-07\\
0.6389639	-3.63236239753206e-07\\
0.6390639	-4.88040425850045e-07\\
0.6391639	-4.98064138154675e-07\\
0.6392639	-3.93332406201097e-07\\
0.6393639	-1.17389192233475e-06\\
0.6394639	-8.39811087782749e-07\\
0.639564	1.32732786961576e-06\\
0.639664	8.9451149265507e-07\\
0.639764	5.75999367757873e-07\\
0.639864	3.71635739737641e-07\\
0.639964	2.81242897859357e-07\\
0.640064	3.04621127211746e-07\\
0.640164	-5.58451338505961e-07\\
0.640264	-3.08218365718815e-07\\
0.640364	-9.44945978575973e-07\\
0.640464	-1.46892239127894e-06\\
0.6405641	8.7918856173097e-07\\
0.6406641	5.83872795667517e-07\\
0.6407641	4.00309537740284e-07\\
0.6408641	3.28121275483539e-07\\
0.6409641	3.66908047277903e-07\\
0.6410641	-4.83752588209541e-07\\
0.6411641	-2.24305607421371e-07\\
0.6412641	-8.55218573292404e-07\\
0.6413641	-1.37698167601563e-06\\
0.6414641	-1.79010778689204e-06\\
0.6415642	1.70561839674122e-06\\
0.6416642	5.12243966133141e-07\\
0.6417642	4.2583078108116e-07\\
0.6418642	4.45774762214413e-07\\
0.6419642	5.71448913744277e-07\\
0.6420642	-1.97796719003662e-07\\
0.6421642	-8.62635103349163e-07\\
0.6422642	-4.23762256174243e-07\\
0.6423642	-8.8189730118593e-07\\
0.6424642	-1.23778251492901e-06\\
0.6425643	1.34961476971895e-06\\
0.6426643	1.20000987458013e-06\\
0.6427643	1.15028781899618e-06\\
0.6428643	1.99613248152986e-07\\
0.6429643	3.47127421029825e-07\\
0.6430643	-4.08051821665367e-07\\
0.6431643	-6.68301132211724e-08\\
0.6432643	-6.30136607338017e-07\\
0.6433643	-1.09892402266887e-06\\
0.6434643	-1.47416868334638e-06\\
0.6435644	1.12589470213642e-06\\
0.6436644	9.38802898087232e-07\\
0.6437644	8.42181694604704e-07\\
0.6438644	8.34959847484384e-07\\
0.6439644	-8.39577232181199e-08\\
0.6440644	8.4310022518963e-08\\
0.6441644	-6.613798027022e-07\\
0.6442644	-3.22194056989744e-07\\
0.6443644	-8.99323614689251e-07\\
0.6444644	-1.3939834047072e-06\\
0.6445645	1.11621633491765e-06\\
0.6446645	7.86834136956571e-07\\
0.6447645	5.36130799311252e-07\\
0.6448645	3.62794682517986e-07\\
0.6449645	2.65489868311874e-07\\
0.6450645	2.42856105225897e-07\\
0.6451645	-7.06491220192618e-07\\
0.6452645	-5.83961134115185e-07\\
0.6453645	-1.39098711970931e-06\\
0.6454645	-1.12902715820873e-06\\
0.6455646	1.16480097589466e-06\\
0.6456646	5.64325908225527e-07\\
0.6457646	1.02831387382984e-06\\
0.6458646	5.55208437891253e-07\\
0.6459646	1.43428452226146e-07\\
0.6460646	-2.08631985687546e-07\\
0.6461646	-5.02603580976313e-07\\
0.6462646	-7.4014187867455e-07\\
0.6463646	-9.22927307911436e-07\\
0.6464646	-1.05266522432146e-06\\
0.6465647	8.73862730621511e-07\\
0.6466647	8.49052536278094e-07\\
0.6467647	8.72022019926533e-07\\
0.6468647	-5.90343218753731e-08\\
0.6469647	5.40528743986357e-08\\
0.6470647	2.09427795327954e-07\\
0.6471647	-5.94790596109007e-07\\
0.6472647	-3.60508588315156e-07\\
0.6473647	-1.08965777090297e-06\\
0.6474647	-1.78419507546046e-06\\
0.6475648	1.59925294407337e-06\\
0.6476648	9.71996649168361e-07\\
0.6477648	3.73326945801722e-07\\
0.6478648	8.0118511092131e-07\\
0.6479648	2.53486877932474e-07\\
0.6480648	-2.71877602209258e-07\\
0.6481648	-7.7704379797261e-07\\
0.6482648	-1.26417284618263e-06\\
0.6483648	-7.35451582967173e-07\\
0.6484648	-1.19309258206357e-06\\
0.6485649	1.44622650899429e-06\\
0.6486649	1.01312808298282e-06\\
0.6487649	5.86872517693848e-07\\
0.6488649	1.65143854147232e-07\\
0.6489649	-2.54399803889527e-07\\
0.6490649	-6.7412633009134e-07\\
0.6491649	-9.64296054117852e-08\\
0.6492649	-5.23729574153364e-07\\
0.6493649	-9.58472260892407e-07\\
0.6494649	-1.4031298238848e-06\\
0.649565	1.26533729249978e-06\\
0.649665	7.97312605493872e-07\\
0.649765	3.1179686033056e-07\\
0.649865	8.06212990367428e-07\\
0.649965	2.77957604932766e-07\\
0.650065	-2.75599038035423e-07\\
0.650165	-8.57113067853632e-07\\
0.650265	-4.69267045888344e-07\\
0.650365	-1.11476999453686e-06\\
0.650465	-1.79635743657869e-06\\
0.6505651	1.64846976469235e-06\\
0.6506651	8.903575452468e-07\\
0.6507651	1.08779207041998e-06\\
0.6508651	2.37931428115346e-07\\
0.6509651	3.37907024139383e-07\\
0.6510651	-6.15176448448551e-07\\
0.6511651	-6.24241046187635e-07\\
0.6512651	-6.92235625487925e-07\\
0.6513651	-8.22135850730987e-07\\
0.6514651	-1.0169442468122e-06\\
0.6515652	9.25013999841795e-07\\
0.6516652	5.95201552267e-07\\
0.6517652	1.19130877718199e-06\\
0.6518652	7.1022533987275e-07\\
0.6519652	1.48813893652289e-07\\
0.6520652	-4.96089981316672e-07\\
0.6521652	-2.27677797282055e-07\\
0.6522652	-1.04916819809731e-06\\
0.6523652	-9.63806991549632e-07\\
0.6524652	-1.97486718156448e-06\\
0.6525653	1.15819119828053e-06\\
0.6526653	9.48255524990671e-07\\
0.6527653	6.31912562498727e-07\\
0.6528653	2.05780161577351e-07\\
0.6529653	-3.3355118578271e-07\\
0.6530653	1.04816162416199e-08\\
0.6531653	-7.65585759010534e-07\\
0.6532653	-6.65245091679978e-07\\
0.6533653	-1.6920156604705e-06\\
0.6534653	-1.84944425796196e-06\\
0.6535654	1.14153672847728e-06\\
0.6536654	7.15901815162567e-07\\
0.6537654	1.14879942203494e-06\\
0.6538654	4.36572329176599e-07\\
0.6539654	5.75535650915171e-07\\
0.6540654	-4.38023206816851e-07\\
0.6541654	-6.07844575339911e-07\\
0.6542654	-9.37696549740963e-07\\
0.6543654	-1.4313750218431e-06\\
0.6544654	-1.09270371262404e-06\\
0.6545655	1.39554784661655e-06\\
0.6546655	1.3911587835036e-06\\
0.6547655	1.20747717580194e-06\\
0.6548655	8.40567680349125e-07\\
0.6549655	2.86466983689948e-07\\
0.6550655	-4.58816231896364e-07\\
0.6551655	-3.99301311126976e-07\\
0.6552655	-5.39035653890352e-07\\
0.6553655	-8.82094744802941e-07\\
0.6554655	-1.43258219420517e-06\\
0.6555656	1.16450294318504e-06\\
0.6556656	1.19051758695754e-06\\
0.6557656	9.96619919479969e-07\\
0.6558656	5.78593590816467e-07\\
0.6559656	-6.78060156911187e-08\\
0.6560656	5.31482253762761e-08\\
0.6561656	-6.28448759698585e-08\\
0.6562656	-4.20114832033747e-07\\
0.6563656	-1.0230195128802e-06\\
0.6564656	-1.87594516987133e-06\\
0.6565657	1.41345935067427e-06\\
0.6566657	1.05095821800738e-06\\
0.6567657	4.25102322232362e-07\\
0.6568657	5.31391602054399e-07\\
0.6569657	3.65297486037974e-07\\
0.6570657	-7.7737126291666e-08\\
0.6571657	-8.02297896784765e-07\\
0.6572657	-8.12999069621512e-07\\
0.6573657	-1.11448348660836e-06\\
0.6574657	-1.7114226369408e-06\\
0.6575658	1.82543645288469e-06\\
0.6576658	1.62715151130755e-06\\
0.6577658	1.11922061352132e-06\\
0.6578658	2.96857500137193e-07\\
0.6579658	1.5524716201476e-07\\
0.6580658	-3.10454187024689e-07\\
0.6581658	-1.05119116522445e-07\\
0.6582658	-1.23364901227774e-06\\
0.6583658	-1.70097409690761e-06\\
0.6584658	-1.51205345422234e-06\\
0.6585659	1.79879091999346e-06\\
0.6586659	1.28885395511702e-06\\
0.6587659	4.20107050258522e-07\\
0.6588659	1.8747544316966e-07\\
0.6589659	5.85855410406566e-07\\
0.6590659	-3.89885754348285e-07\\
0.6591659	-7.44909768535251e-07\\
0.6592659	-4.84407370437978e-07\\
0.6593659	-1.61359835026964e-06\\
0.6594659	-1.13773155474561e-06\\
0.659566	1.44479063513003e-06\\
0.659666	1.11850185735562e-06\\
0.659766	3.81344351874091e-07\\
0.659866	2.27952781539642e-07\\
0.659966	-3.47067349082408e-07\\
0.660066	-3.49139729571846e-07\\
0.660166	-7.83717231112036e-07\\
0.660266	-6.56281949007465e-07\\
0.660366	-9.72345209904191e-07\\
0.660466	-1.73744759579542e-06\\
0.6605661	1.585393940573e-06\\
0.6606661	9.09011457217446e-07\\
0.6607661	7.66786789396789e-07\\
0.6608661	1.5306214917743e-07\\
0.6609661	6.21504145925655e-08\\
0.6610661	-5.11664889346619e-07\\
0.6611661	-5.74129605412921e-07\\
0.6612661	-1.13101895848899e-06\\
0.6613661	-1.18813758032132e-06\\
0.6614661	-1.75131950852503e-06\\
0.6615662	1.75124049328623e-06\\
0.6616662	1.16179131004479e-06\\
0.6617662	1.04859411531599e-06\\
0.6618662	4.0569702863813e-07\\
0.6619662	2.27118681861427e-07\\
0.6620662	-4.93151790292856e-07\\
0.6621662	-7.61154784534313e-07\\
0.6622662	-5.82960213879602e-07\\
0.6623662	-9.64667536296204e-07\\
0.6624662	-1.91240576685736e-06\\
0.6625663	1.17986014047666e-06\\
0.6626663	1.08497327722112e-06\\
0.6627663	4.05484765320097e-07\\
0.6628663	1.35147215196696e-07\\
0.6629663	2.67683613497383e-07\\
0.6630663	-2.03212679572573e-07\\
0.6631663	-2.83877943374478e-07\\
0.6632663	-9.80678107165289e-07\\
0.6633663	-1.30000876663594e-06\\
0.6634663	-1.2482951894377e-06\\
0.6635664	1.81410579358499e-06\\
0.6636664	1.59186815018586e-06\\
0.6637664	7.21214272747517e-07\\
0.6638664	1.95600069252677e-07\\
0.6639664	8.4517302045839e-09\\
0.6640664	1.5316570611823e-07\\
0.6641664	-3.76891285069902e-07\\
0.6642664	-5.8838228556013e-07\\
0.6643664	-1.48800008736849e-06\\
0.6644664	-2.08246724886507e-06\\
0.6645665	1.30081640814517e-06\\
0.6646665	1.29965195161397e-06\\
0.6647665	5.83284944788431e-07\\
0.6648665	1.44873655738564e-07\\
0.6649665	-2.2453450654325e-08\\
0.6650665	7.44022772281028e-08\\
0.6651665	-5.71490327061497e-07\\
0.6652665	-9.67092251613622e-07\\
0.6653665	-1.11939431546304e-06\\
0.6654665	-2.03541717080924e-06\\
0.6655666	1.98971576415019e-06\\
0.6656666	1.52828859167897e-06\\
0.6657666	1.2818924748359e-06\\
0.6658666	2.43387339615708e-07\\
0.6659666	4.05603247433106e-07\\
0.6660666	-2.3865960230296e-07\\
0.6661666	-6.96630874763571e-07\\
0.6662666	-9.7557012201932e-07\\
0.6663666	-1.08276675714336e-06\\
0.6664666	-2.025540078332e-06\\
0.6665667	1.93255269076431e-06\\
0.6666667	1.29969457063339e-06\\
0.6667667	8.09115152655693e-07\\
0.6668667	4.53375556652702e-07\\
0.6669667	2.25007027676583e-07\\
0.6670667	1.16510912029355e-07\\
0.6671667	-8.7964134287688e-07\\
0.6672667	-7.71008176858601e-07\\
0.6673667	-1.56517793525879e-06\\
0.6674667	-1.26976885272967e-06\\
0.6675668	1.88248822352577e-06\\
0.6676668	1.33715108274757e-06\\
0.6677668	8.58350956089993e-07\\
0.6678668	4.38349989639875e-07\\
0.6679668	6.93804276252763e-08\\
0.6680668	-2.56355378702722e-07\\
0.6681668	-5.46684978175449e-07\\
0.6682668	-8.09465800610809e-07\\
0.6683668	-1.05258518434681e-06\\
0.6684668	-1.28396034337896e-06\\
0.6685669	1.29373477530947e-06\\
0.6686669	1.0649686812414e-06\\
0.6687669	8.24007824640205e-07\\
0.6688669	5.62815419602458e-07\\
0.6689669	2.73324810784459e-07\\
0.6690669	-5.25605261536555e-08\\
0.6691669	-4.22966992097429e-07\\
0.6692669	-8.4605084582634e-07\\
0.6693669	-1.32999820268154e-06\\
0.6694669	-1.8830250443358e-06\\
0.669567	1.32145249676086e-06\\
0.669667	1.60840961926922e-06\\
0.669767	8.01451934151487e-07\\
0.669867	8.9224406929489e-07\\
0.669967	-1.2757917700057e-07\\
0.670067	-2.66412826466222e-07\\
0.670167	-5.32681716070016e-07\\
0.670267	-9.34840491684241e-07\\
0.670367	-1.48137359534672e-06\\
0.670467	-2.1807952785835e-06\\
0.6705671	1.8219074267023e-06\\
0.6706671	7.93872433124676e-07\\
0.6707671	5.87218841552328e-07\\
0.6708671	1.93313289376107e-07\\
0.6709671	6.03492651851667e-07\\
0.6710671	-1.90935931954073e-07\\
0.6711671	-1.98695058184484e-07\\
0.6712671	-1.42853705042967e-06\\
0.6713671	-1.88924394706191e-06\\
0.6714671	-1.58962748875169e-06\\
0.6715672	1.35289614489054e-06\\
0.6716672	1.14934347328344e-06\\
0.6717672	6.79491888533335e-07\\
0.6718672	9.34410900654825e-07\\
0.6719672	-9.48596272642988e-08\\
0.6720672	-4.17309468758731e-07\\
0.6721672	-1.04195802697049e-06\\
0.6722672	-9.77854317518734e-07\\
0.6723672	-1.23407695573619e-06\\
0.6724672	-1.81973415624981e-06\\
0.6725673	2.17444105921061e-06\\
0.6726673	9.05119495797635e-07\\
0.6727673	1.278852144182e-06\\
0.6728673	2.86412548344828e-07\\
0.6729673	-8.14552687522507e-08\\
0.6730673	1.65963206733011e-07\\
0.6731673	-9.80646995607515e-07\\
0.6732673	-1.53063032648504e-06\\
0.6733673	-1.4933606897749e-06\\
0.6734673	-1.87824144282089e-06\\
0.6735674	1.24976051241177e-06\\
0.6736674	9.9480543891417e-07\\
0.6737674	1.28930390985715e-06\\
0.6738674	1.23734926305019e-07\\
0.6739674	4.8854813394783e-07\\
0.6740674	-6.25836173551164e-07\\
0.6741674	-2.29027021347861e-07\\
0.6742674	-1.33066274710103e-06\\
0.6743674	-1.94041097500985e-06\\
0.6744674	-2.06796860346969e-06\\
0.6745675	1.24651733068859e-06\\
0.6746675	1.05659102578315e-06\\
0.6747675	1.31957938753757e-06\\
0.6748675	1.02566859316866e-06\\
0.6749675	1.65015650921418e-07\\
0.6750675	-2.72251595490047e-07\\
0.6751675	-2.96034426749969e-07\\
0.6752675	-9.16263237282777e-07\\
0.6753675	-1.14289751484975e-06\\
0.6754675	-1.98592581446277e-06\\
0.6755676	1.5383494500476e-06\\
0.6756676	1.43480976655397e-06\\
0.6757676	6.84726101152222e-07\\
0.6758676	2.77993807085153e-07\\
0.6759676	2.04479282395198e-07\\
0.6760676	-5.45980011867897e-07\\
0.6761676	-9.83575523250124e-07\\
0.6762676	-1.11852759010489e-06\\
0.6763676	-1.9610854131713e-06\\
0.6764676	-1.52152703059016e-06\\
0.6765677	2.20668566397464e-06\\
0.6766677	1.18178339025832e-06\\
0.6767677	1.40798003567255e-06\\
0.6768677	8.74882424284351e-07\\
0.6769677	5.72068663551306e-07\\
0.6770677	-5.10911817297455e-07\\
0.6771677	-3.84538259901035e-07\\
0.6772677	-1.05931853600083e-06\\
0.6773677	-1.54578911937264e-06\\
0.6774677	-1.85451506773404e-06\\
0.6775678	2.04284955396261e-06\\
0.6776678	1.05995472887699e-06\\
0.6777678	1.22292750548425e-06\\
0.6778678	5.21088759253274e-07\\
0.6779678	-5.62690738448168e-08\\
0.6780678	-5.19881963256807e-07\\
0.6781678	-8.80514263723597e-07\\
0.6782678	-1.14895867575626e-06\\
0.6783678	-1.33603622431977e-06\\
0.6784678	-1.45259622241767e-06\\
0.6785679	1.55045405225707e-06\\
0.6786679	1.54431147558753e-06\\
0.6787679	5.75958127413401e-07\\
0.6788679	6.34431815527137e-07\\
0.6789679	-2.9125779299477e-07\\
0.6790679	-2.12129123600135e-07\\
0.6791679	-1.13922867850391e-06\\
0.6792679	-1.08363100004283e-06\\
0.6793679	-1.05643862946536e-06\\
0.6794679	-2.06878208164341e-06\\
0.679568	1.94808904785404e-06\\
0.679668	1.82510320767548e-06\\
0.679768	6.29011275066205e-07\\
0.679868	3.48571167485545e-07\\
0.679968	-2.74870042282771e-08\\
0.680068	-5.1046090199236e-07\\
0.680168	-1.11167590644001e-06\\
0.680268	-8.42485102126034e-07\\
0.680368	-1.71426922790374e-06\\
0.680468	-1.73843665018936e-06\\
0.6805681	2.17230387900003e-06\\
0.6806681	1.81085343875509e-06\\
0.6807681	1.2626177356978e-06\\
0.6808681	5.16078256573849e-07\\
0.6809681	5.59689051549128e-07\\
0.6810681	-6.18123233486756e-07\\
0.6811681	-1.02895930353952e-06\\
0.6812681	-6.84447200249849e-07\\
0.6813681	-1.59624223972088e-06\\
0.6814681	-1.77602698592239e-06\\
0.6815682	1.88088640840078e-06\\
0.6816682	1.13166846915647e-06\\
0.6817682	1.07923829517631e-06\\
0.6818682	7.11804711173158e-07\\
0.6819682	1.75495062926245e-08\\
0.6820682	-1.53725330243049e-08\\
0.6821682	-3.98833587933467e-07\\
0.6822682	-1.14473275480975e-06\\
0.6823682	-1.26499600572316e-06\\
0.6824682	-1.77157615599555e-06\\
0.6825683	1.45643996329881e-06\\
0.6826683	1.1428441282213e-06\\
0.6827683	1.40690032512936e-06\\
0.6828683	2.36548798326908e-07\\
0.6829683	6.19703170556818e-07\\
0.6830683	-4.55749508354586e-07\\
0.6831683	-1.00194872221238e-06\\
0.6832683	-1.03106043747303e-06\\
0.6833683	-1.55527708001202e-06\\
0.6834683	-1.58681745654476e-06\\
0.6835684	2.01025901880314e-06\\
0.6836684	9.28771382024962e-07\\
0.6837684	1.30313333635002e-06\\
0.6838684	1.21020903964819e-07\\
0.6839684	3.70083944822852e-07\\
0.6840684	3.79462066213421e-08\\
0.6841684	-8.87794647574935e-07\\
0.6842684	-1.41956697641277e-06\\
0.6843684	-1.56982511390069e-06\\
0.6844684	-2.35104932455599e-06\\
0.6845685	1.38650431136256e-06\\
0.6846685	1.30714139690014e-06\\
0.6847685	5.59203952477105e-07\\
0.6848685	1.30108457341294e-07\\
0.6849685	7.24570714538686e-09\\
0.6850685	1.77980860271987e-07\\
0.6851685	-3.70346505906838e-07\\
0.6852685	-6.50422346648583e-07\\
0.6853685	-1.674958093556e-06\\
0.6854685	-1.45669063122966e-06\\
0.6855686	2.16667747432453e-06\\
0.6856686	1.83345013615721e-06\\
0.6857686	7.0465050905355e-07\\
0.6858686	7.6744049071209e-07\\
0.6859686	8.95679796997229e-09\\
0.6860686	-5.83688951838468e-07\\
0.6861686	-1.02341021834818e-06\\
0.6862686	-1.32314547096613e-06\\
0.6863686	-1.49585813380426e-06\\
0.6864686	-1.55453653727378e-06\\
0.6865687	1.67439519582757e-06\\
0.6866687	1.8058024360279e-06\\
0.6867687	1.01211699776727e-06\\
0.6868687	2.80251467366099e-07\\
0.6869687	5.97093793963666e-07\\
0.6870687	-5.04926438438247e-08\\
0.6871687	-6.75668978988853e-07\\
0.6872687	-1.2916208111946e-06\\
0.6873687	-1.91155812734323e-06\\
0.6874687	-1.54871528579292e-06\\
0.6875688	1.98046226174142e-06\\
0.6876688	1.27001488792189e-06\\
0.6877688	5.02485827169608e-07\\
0.6878688	6.64543910566806e-07\\
0.6879688	-2.57166089756566e-07\\
0.6880688	-2.76023407685955e-07\\
0.6881688	-4.05431204164586e-07\\
0.6882688	-6.58816516008187e-07\\
0.6883688	-2.04963020600246e-06\\
0.6884688	-1.59134689381446e-06\\
0.6885689	1.90824269541423e-06\\
0.6886689	1.02501673104882e-06\\
0.6887689	9.50308410185841e-07\\
0.6888689	6.70548668590953e-07\\
0.6889689	1.72144981824829e-07\\
0.6890689	-5.58518565618726e-07\\
0.6891689	-5.35081214447786e-07\\
0.6892689	-7.71205480809556e-07\\
0.6893689	-1.28057707859952e-06\\
0.6894689	-2.07690486009326e-06\\
0.689569	2.03932762277148e-06\\
0.689669	1.6285473001254e-06\\
0.689769	8.89532205672339e-07\\
0.689869	8.08481530167882e-07\\
0.689969	3.71571645452207e-07\\
0.690069	-4.35043849389416e-07\\
0.690169	-6.25234040985134e-07\\
0.690269	-1.21289063503838e-06\\
0.690369	-1.21192791002755e-06\\
0.690469	-1.63628263294413e-06\\
0.6905691	1.71949834149032e-06\\
0.6906691	1.40314849339518e-06\\
0.6907691	6.19522558586993e-07\\
0.6908691	3.54594424045729e-07\\
0.6909691	5.94315818336355e-07\\
0.6910691	-6.75383628134796e-07\\
0.6911691	-4.68596313307756e-07\\
0.6912691	-7.99436584042468e-07\\
0.6913691	-1.68204069161959e-06\\
0.6914691	-2.13056669551293e-06\\
0.6915692	2.06498237931285e-06\\
0.6916692	1.44245002697829e-06\\
0.6917692	1.21137711017738e-06\\
0.6918692	3.57518952665004e-07\\
0.6919692	-1.33390598566763e-07\\
0.6920692	-2.75639095104196e-07\\
0.6921692	-1.08353542938033e-06\\
0.6922692	-1.57140974410552e-06\\
0.6923692	-1.75361338206059e-06\\
0.6924692	-1.64451879225425e-06\\
0.6925693	1.96900030369207e-06\\
0.6926693	1.61774513429691e-06\\
0.6927693	5.14530387629719e-07\\
0.6928693	6.44899839308266e-07\\
0.6929693	-5.62349544708241e-09\\
0.6930693	-4.51537282231129e-07\\
0.6931693	-7.07359792961881e-07\\
0.6932693	-7.87629842680104e-07\\
0.6933693	-1.70690670042717e-06\\
0.6934693	-2.47977002310051e-06\\
0.6935694	2.10860026816917e-06\\
0.6936694	1.58485373935591e-06\\
0.6937694	1.16364554836323e-06\\
0.6938694	8.30315219246813e-07\\
0.6939694	5.70182269399311e-07\\
0.6940694	-6.31453724864173e-07\\
0.6941694	-7.89313031823724e-07\\
0.6942694	-9.18135729222058e-07\\
0.6943694	-2.03268156973024e-06\\
0.6944694	-2.14772994722168e-06\\
0.6945695	1.95177723671947e-06\\
0.6946695	1.79126994304113e-06\\
0.6947695	5.85790097762384e-07\\
0.6948695	3.20480552762348e-07\\
0.6949695	-1.95350695619823e-08\\
0.6950695	-4.49152308235057e-07\\
0.6951695	-9.83285788791477e-07\\
0.6952695	-1.63686912690864e-06\\
0.6953695	-1.42485486787081e-06\\
0.6954695	-2.36221438942863e-06\\
0.6955696	1.76487316494445e-06\\
0.6956696	1.48359007168253e-06\\
0.6957696	1.00789176959992e-06\\
0.6958696	3.22732294222305e-07\\
0.6959696	4.13047251957011e-07\\
0.6960696	-7.36246112964523e-07\\
0.6961696	-1.14024882424957e-06\\
0.6962696	-8.14080113986648e-07\\
0.6963696	-1.77287730096509e-06\\
0.6964696	-2.03179574853607e-06\\
0.6965697	1.62025415262335e-06\\
0.6966697	1.71521732372781e-06\\
0.6967697	1.46447061766253e-06\\
0.6968697	8.5278736783323e-07\\
0.6969697	-1.35076684948388e-07\\
0.6970697	-5.14383328820145e-07\\
0.6971697	-3.00411802989231e-07\\
0.6972697	-1.50845869173466e-06\\
0.6973697	-1.15383786392087e-06\\
0.6974697	-2.25188036617752e-06\\
0.6975698	1.40426022898765e-06\\
0.6976698	1.35433851733424e-06\\
0.6977698	8.05642661472206e-07\\
0.6978698	7.42773700324051e-07\\
0.6979698	1.50315920599553e-07\\
0.6980698	1.28369306295895e-08\\
0.6981698	-6.85112245601971e-07\\
0.6982698	-9.58997069755441e-07\\
0.6983698	-1.82429943373918e-06\\
0.6984698	-2.29651753036464e-06\\
0.6985699	1.82542284354881e-06\\
0.6986699	1.09216809063284e-06\\
0.6987699	7.05390403421546e-07\\
0.6988699	6.49527149398921e-07\\
0.6989699	-9.10001674014893e-08\\
0.6990699	-5.31785831814346e-07\\
0.6991699	-6.88439841578514e-07\\
0.6992699	-1.57658779231795e-06\\
0.6993699	-1.21187080770468e-06\\
0.6994699	-1.60994544096127e-06\\
0.69957	1.42294502403217e-06\\
0.69967	1.45145422081328e-06\\
0.69977	6.70094756571871e-07\\
0.69987	6.31492471647732e-08\\
0.69997	6.14885335536997e-07\\
0.70007	-6.9044421735498e-07\\
0.70017	-8.68601450765283e-07\\
0.70027	-9.35363099419106e-07\\
0.70037	-1.90652049347761e-06\\
0.70047	-1.79787950616017e-06\\
0.7005701	1.57543844858665e-06\\
0.7006701	1.79524127075226e-06\\
0.7007701	1.04732289729625e-06\\
0.7008701	3.15820341434403e-07\\
0.7009701	5.84856541863132e-07\\
0.7010701	-1.61459491998528e-07\\
0.7011701	-9.39032620550506e-07\\
0.7012701	-7.63781470070057e-07\\
0.7013701	-1.65163832654969e-06\\
0.7014701	-1.61854906588488e-06\\
0.7015702	1.509911660591e-06\\
0.7016702	1.33588274531249e-06\\
0.7017702	1.03486585389945e-06\\
0.7018702	5.90861762184147e-07\\
0.7019702	-1.21418555210084e-08\\
0.7020702	-7.90170321174344e-07\\
0.7021702	-7.59261866711824e-07\\
0.7022702	-9.35467516338662e-07\\
0.7023702	-1.3348510194966e-06\\
0.7024702	-1.97348872044145e-06\\
0.7025703	1.31100284317398e-06\\
0.7026703	1.14429819841888e-06\\
0.7027703	6.90019470006575e-07\\
0.7028703	9.32040826295832e-07\\
0.7029703	-1.45775692406147e-07\\
0.7030703	-5.59580073833388e-07\\
0.7031703	-3.25534219136614e-07\\
0.7032703	-1.45981185895039e-06\\
0.7033703	-1.97859845041393e-06\\
0.7034703	-1.89809107720151e-06\\
0.7035704	1.93045066421149e-06\\
0.7036704	1.15946745360418e-06\\
0.7037704	9.39101959307465e-07\\
0.7038704	2.53111583248256e-07\\
0.7039704	8.5242600977864e-08\\
0.7040704	-5.80769740299303e-07\\
0.7041704	-7.6120111414113e-07\\
0.7042704	-1.47233801239821e-06\\
0.7043704	-1.73047765272827e-06\\
0.7044704	-1.55192786666092e-06\\
0.7045705	2.19679585145016e-06\\
0.7046705	1.19815487531483e-06\\
0.7047705	5.87200773782826e-07\\
0.7048705	3.47584256044087e-07\\
0.7049705	4.62945917600877e-07\\
0.7050705	-8.30836666132484e-08\\
0.7051705	-3.06883805478719e-07\\
0.7052705	-1.22484360565167e-06\\
0.7053705	-1.85336187064067e-06\\
0.7054705	-2.20884700086188e-06\\
0.7055706	1.82530620707055e-06\\
0.7056706	9.64856285978755e-07\\
0.7057706	1.3281413648869e-06\\
0.7058706	8.98715728503419e-07\\
0.7059706	-3.3987542380487e-07\\
0.7060706	-4.04095846118224e-07\\
0.7061706	-3.10418159621406e-07\\
0.7062706	-1.07532373183616e-06\\
0.7063706	-1.71530258041841e-06\\
0.7064706	-2.24685326699614e-06\\
0.7065707	1.4281170797048e-06\\
0.7066707	1.06196053950569e-06\\
0.7067707	7.54669760016924e-07\\
0.7068707	4.89713070095377e-07\\
0.7069707	2.50550782787684e-07\\
0.7070707	2.06353121257052e-08\\
0.7071707	-2.16588738055634e-07\\
0.7072707	-1.4776844352582e-06\\
0.7073707	-1.77922243738138e-06\\
0.7074707	-2.13778087454486e-06\\
0.7075708	1.52457920954419e-06\\
0.7076708	1.00011763848329e-06\\
0.7077708	1.36884086510136e-06\\
0.7078708	6.1414193419651e-07\\
0.7079708	-2.80593044355015e-07\\
0.7080708	-3.31984785439943e-07\\
0.7081708	-5.56660718853408e-07\\
0.7082708	-9.7125487741323e-07\\
0.7083708	-1.59240779717962e-06\\
0.7084708	-1.436766405849e-06\\
0.7085709	1.55180527539045e-06\\
0.7086709	1.20880466836226e-06\\
0.7087709	5.9260370433023e-07\\
0.7088709	6.8653100493421e-07\\
0.7089709	4.73909343323697e-07\\
0.7090709	-6.19442266369674e-08\\
0.7091709	-9.37718271032395e-07\\
0.7092709	-1.17010687095132e-06\\
0.7093709	-1.7758094958964e-06\\
0.7094709	-1.77153090064053e-06\\
0.709571	1.87540639062789e-06\\
0.709671	1.0470807776386e-06\\
0.709771	7.78574799298326e-07\\
0.709871	5.3163675683976e-08\\
0.709971	-1.45882100621719e-07\\
0.710071	1.6470334029961e-07\\
0.710171	-1.03181862831647e-06\\
0.710271	-7.52191023689619e-07\\
0.710371	-1.01316114164973e-06\\
0.710471	-1.8314804410835e-06\\
0.7105711	1.800409210162e-06\\
0.7106711	1.81452181768904e-06\\
0.7107711	1.22099023691646e-06\\
0.7108711	1.00304747796542e-06\\
0.7109711	1.43922946005404e-07\\
0.7110711	-3.73157432065341e-07\\
0.7111711	-5.6497110767495e-07\\
0.7112711	-1.44829878445663e-06\\
0.7113711	-1.039924315549e-06\\
0.7114711	-2.35663457814894e-06\\
0.7115712	1.58234413172309e-06\\
0.7116712	1.76232515070041e-06\\
0.7117712	1.16682697592907e-06\\
0.7118712	7.79051743116099e-07\\
0.7119712	5.82199147558526e-07\\
0.7120712	-4.40533463041959e-07\\
0.7121712	-3.05950963230828e-07\\
0.7122712	-1.03086034819455e-06\\
0.7123712	-1.63207059555859e-06\\
0.7124712	-2.12639257268776e-06\\
0.7125713	1.4384948188173e-06\\
0.7126713	1.10457469704173e-06\\
0.7127713	8.27083323162725e-07\\
0.7128713	5.89203467793453e-07\\
0.7129713	3.74116604806574e-07\\
0.7130713	1.65003028129718e-07\\
0.7131713	-1.05495803914835e-06\\
0.7132713	-1.30258831276819e-06\\
0.7133713	-1.59471036420555e-06\\
0.7134713	-1.94814746734551e-06\\
0.7135714	1.55929870659932e-06\\
0.7136714	1.02965601556093e-06\\
0.7137714	1.38820849659993e-06\\
0.7138714	6.18131184459969e-07\\
0.7139714	-2.9740103446585e-07\\
0.7140714	-3.7521327111989e-07\\
0.7141714	-6.32130530142661e-07\\
0.7142714	-1.08497760287207e-06\\
0.7143714	-1.75057893336827e-06\\
0.7144714	-1.64575850725157e-06\\
0.7145715	2.11988824272424e-06\\
0.7146715	1.71181099162254e-06\\
0.7147715	1.02367064869213e-06\\
0.7148715	3.86462390977726e-08\\
0.7149715	-2.60082154035501e-07\\
0.7150715	1.10666717567653e-07\\
0.7151715	-8.6592460846191e-07\\
0.7152715	-1.2066721835069e-06\\
0.7153715	-1.92839053614335e-06\\
0.7154715	-2.04789255420224e-06\\
0.7155716	1.29176192675828e-06\\
0.7156716	1.32282128806338e-06\\
0.7157716	9.05652589366923e-07\\
0.7158716	2.34507266938522e-08\\
0.7159716	-3.40587171798745e-07\\
0.7160716	-2.03261613762606e-07\\
0.7161716	-5.81370621866029e-07\\
0.7162716	-1.49170961361023e-06\\
0.7163716	-9.51071293719963e-07\\
0.7164716	-1.97624552100706e-06\\
0.7165717	1.25457434041465e-06\\
0.7166717	1.04380962451245e-06\\
0.7167717	1.21686424670031e-06\\
0.7168717	7.56960930381467e-07\\
0.7169717	-3.52674181414159e-07\\
0.7170717	-1.28811361577164e-07\\
0.7171717	-5.88217214492204e-07\\
0.7172717	-7.47654535615538e-07\\
0.7173717	-1.62388219315801e-06\\
0.7174717	-2.23365499607553e-06\\
0.7175718	1.20803369885181e-06\\
0.7176718	1.07714747876742e-06\\
0.7177718	1.16246038750489e-06\\
0.7178718	4.47235001921342e-07\\
0.7179718	-8.52614432389487e-08\\
0.7180718	-4.51756950958782e-07\\
0.7181718	-6.6897461703519e-07\\
0.7182718	-7.53632532379811e-07\\
0.7183718	-1.72244363927376e-06\\
0.7184718	-1.59211563044437e-06\\
0.7185719	1.38389564341423e-06\\
0.7186719	1.65845785549124e-06\\
0.7187719	9.82052205777961e-07\\
0.7188719	3.37993257382152e-07\\
0.7189719	-2.903985425462e-07\\
0.7190719	8.02032289470844e-08\\
0.7191719	-5.66868877172055e-07\\
0.7192719	-1.24827609582212e-06\\
0.7193719	-9.80673295458701e-07\\
0.7194719	-1.7807088594779e-06\\
0.719572	2.05804136221843e-06\\
0.719672	1.06870117289048e-06\\
0.719772	9.61800907361976e-07\\
0.719872	7.20719321556373e-07\\
0.719972	3.28842256536177e-07\\
0.720072	-2.30437230186453e-07\\
0.720172	-9.73718753627395e-07\\
0.720272	-9.17594468852201e-07\\
0.720372	-1.07864894616228e-06\\
0.720472	-1.47345905165963e-06\\
0.7205721	1.56262820993902e-06\\
0.7206721	1.64633247390356e-06\\
0.7207721	4.46581389645928e-07\\
0.7208721	9.468301263027e-07\\
0.7209721	1.30542176179915e-07\\
0.7210721	-1.88105291343277e-08\\
0.7211721	-5.17747495543119e-07\\
0.7212721	-1.38277954953026e-06\\
0.7213721	-1.63040870493347e-06\\
0.7214721	-1.27712804331992e-06\\
0.7215722	1.29830061812441e-06\\
0.7216722	7.99517707150699e-07\\
0.7217722	8.52203706713794e-07\\
0.7218722	4.39902508198031e-07\\
0.7219722	5.46167529869024e-07\\
0.7220722	1.54561867304892e-07\\
0.7221722	-7.51341593385746e-07\\
0.7222722	-1.18796005477151e-06\\
0.7223722	-1.17170067248651e-06\\
0.7224722	-1.71896044687259e-06\\
0.7225723	1.74644903339072e-06\\
0.7226723	1.01839674826465e-06\\
0.7227723	6.77680151994764e-07\\
0.7228723	7.07944136202343e-07\\
0.7229723	9.28443748549057e-08\\
0.7230723	-1.83952564292156e-07\\
0.7231723	-1.38769086888146e-07\\
0.7232723	-7.87916446753911e-07\\
0.7233723	-1.14769463799291e-06\\
0.7234723	-1.23439225152566e-06\\
0.7235724	1.48150434231553e-06\\
0.7236724	8.8738070314065e-07\\
0.7237724	5.17525209353664e-07\\
0.7238724	3.55696085918566e-07\\
0.7239724	3.85663576629725e-07\\
0.7240724	-4.08789950334665e-07\\
0.7241724	-1.04386986521376e-06\\
0.7242724	-5.35769162368638e-07\\
0.7243724	-9.00668327474818e-07\\
0.7244724	-1.15473523276677e-06\\
0.7245725	1.18325512055151e-06\\
0.7246725	1.09747074450439e-06\\
0.7247725	1.07407591531938e-06\\
0.7248725	9.69545110862668e-08\\
0.7249725	1.50003675258858e-07\\
0.7250725	2.17133895752397e-07\\
0.7251725	-7.17730842758613e-07\\
0.7252725	-6.70652941181515e-07\\
0.7253725	-1.65768107418529e-06\\
0.7254725	-1.69485005319103e-06\\
0.7255726	1.64917500722339e-06\\
0.7256726	1.45858577038638e-06\\
0.7257726	1.16981964382035e-06\\
0.7258726	7.66898485160539e-07\\
0.7259726	2.33858630238615e-07\\
0.7260726	-4.45248993230507e-07\\
0.7261726	-2.86358733170289e-07\\
0.7262726	-1.30539010267938e-06\\
0.7263726	-1.51824765026731e-06\\
0.7264726	-1.94082084092884e-06\\
0.7265727	1.8067472096206e-06\\
0.7266727	9.11885931209611e-07\\
0.7267727	7.59716543896616e-07\\
0.7268727	3.34411160096693e-07\\
0.7269727	-3.79842392206342e-07\\
0.7270727	-3.98840459325811e-07\\
0.7271727	-7.38363454821922e-07\\
0.7272727	-1.41417572230296e-06\\
0.7273727	-1.44202541196847e-06\\
0.7274727	-1.83764436822997e-06\\
0.7275728	1.72577322121725e-06\\
0.7276728	1.54207902802739e-06\\
0.7277728	9.4350345580807e-07\\
0.7278728	9.1438114946385e-07\\
0.7279728	4.39063667201367e-07\\
0.7280728	-4.98080382271837e-07\\
0.7281728	-9.12665226060483e-07\\
0.7282728	-8.20287798575237e-07\\
0.7283728	-1.23652764028037e-06\\
0.7284728	-1.17694674717228e-06\\
0.7285729	1.63065257030581e-06\\
0.7286729	5.89696651509541e-07\\
0.7287729	9.77966370108163e-07\\
0.7288729	7.79971092157439e-07\\
0.7289729	-1.97617051611587e-08\\
0.7290729	-4.36686248050933e-07\\
0.7291729	-4.86238407759743e-07\\
0.7292729	-1.18383554070789e-06\\
0.7293729	-1.54487638770462e-06\\
0.7294729	-1.58474093847616e-06\\
0.729573	9.12620569337719e-07\\
0.729673	1.46332701422835e-06\\
0.729773	2.8916813743507e-07\\
0.729873	3.74840150918487e-07\\
0.729973	-2.94941384204606e-07\\
0.730073	2.64558558171757e-07\\
0.730173	-9.61905402441232e-07\\
0.730273	-9.89558935238222e-07\\
0.730373	-8.33607884942467e-07\\
0.730473	-1.50923814867809e-06\\
0.7305731	1.14193071887314e-06\\
0.7306731	7.5179108716128e-07\\
0.7307731	4.84618248641766e-07\\
0.7308731	3.25307335735658e-07\\
0.7309731	2.58774015549079e-07\\
0.7310731	2.69954634646297e-07\\
0.7311731	-6.56193667403215e-07\\
0.7312731	-5.34692856213326e-07\\
0.7313731	-1.38054386233932e-06\\
0.7314731	-1.20872646736103e-06\\
0.7315732	1.07996870202243e-06\\
0.7316732	1.23624901959829e-06\\
0.7317732	3.65371539245984e-07\\
0.7318732	4.5244228186192e-07\\
0.7319732	4.82588990990962e-07\\
0.7320732	-5.5903872375751e-07\\
0.7321732	-6.87269269050717e-07\\
0.7322732	-9.169089594252e-07\\
0.7323732	-1.26274189682363e-06\\
0.7324732	-1.73952984905412e-06\\
0.7325733	1.69128454263756e-06\\
0.7326733	9.02223399990021e-07\\
0.7327733	9.38042425069341e-07\\
0.7328733	7.8407038106576e-07\\
0.7329733	4.25658958747022e-07\\
0.7330733	-1.5181713663992e-07\\
0.7331733	-9.62960016615e-07\\
0.7332733	-1.0223485338301e-06\\
0.7333733	-1.34453814748525e-06\\
0.7334733	-9.44060807306357e-07\\
0.7335734	1.15552988333434e-06\\
0.7336734	9.51526300418948e-07\\
0.7337734	4.26721565638388e-07\\
0.7338734	5.66678940217002e-07\\
0.7339734	3.5698577782739e-07\\
0.7340734	-2.16746366632492e-07\\
0.7341734	-1.68881614470706e-07\\
0.7342734	-5.13759649045653e-07\\
0.7343734	-1.26569560610079e-06\\
0.7344734	-1.43897994586695e-06\\
0.7345735	8.79286964128312e-07\\
0.7346735	8.14076728694602e-07\\
0.7347735	2.84781664028344e-07\\
0.7348735	2.77211166910263e-07\\
0.7349735	-2.22800150773139e-07\\
0.7350735	-2.29392294492925e-07\\
0.7351735	-7.56679807700777e-07\\
0.7352735	-8.18751654918515e-07\\
0.7353735	-1.4296710864059e-06\\
0.7354735	-1.60347554167828e-06\\
0.7355736	1.5077763579896e-06\\
0.7356736	1.15959548629974e-06\\
0.7357736	2.06559344739787e-07\\
0.7358736	6.34734916182822e-07\\
0.7359736	4.30215567170933e-07\\
0.7360736	-4.20878845552863e-07\\
0.7361736	-9.32401855280318e-07\\
0.7362736	-1.11818027104249e-06\\
0.7363736	-9.92014070888558e-07\\
0.7364736	-1.56767629011512e-06\\
0.7365737	9.36430140541233e-07\\
0.7366737	9.09164188378497e-07\\
0.7367737	1.38899846291451e-07\\
0.7368737	6.11973008268762e-07\\
0.7369737	3.14747088836498e-07\\
0.7370737	2.33613105748987e-07\\
0.7371737	-6.45010196365092e-07\\
0.7372737	-3.34676222735197e-07\\
0.7373737	-8.48910435280459e-07\\
0.7374737	-1.2012102335035e-06\\
0.7375738	1.32231762872692e-06\\
0.7376738	1.24663555123306e-06\\
0.7377738	2.92551746383651e-07\\
0.7378738	4.46682193455672e-07\\
0.7379738	-3.04328536593346e-07\\
0.7380738	2.61928700950875e-08\\
0.7381738	-5.75051463336251e-07\\
0.7382738	-1.12133045870877e-06\\
0.7383738	-6.25884005955868e-07\\
0.7384738	-1.10192282587462e-06\\
0.7385739	1.09541069992503e-06\\
0.7386739	6.29881935054755e-07\\
0.7387739	1.53398637792179e-07\\
0.7388739	6.52867816697267e-07\\
0.7389739	1.1522618947879e-07\\
0.7390739	-4.72559738007305e-07\\
0.7391739	-1.23493526515972e-07\\
0.7392739	-8.50548737219015e-07\\
0.7393739	-6.66668794835346e-07\\
0.7394739	-1.58476689193776e-06\\
0.739574	9.69675872930509e-07\\
0.739674	8.01869448174841e-07\\
0.739774	4.93515051047666e-07\\
0.739874	3.18215231942531e-08\\
0.739974	4.04028460465611e-07\\
0.740074	-4.02593673953788e-07\\
0.740174	-4.00743437545259e-07\\
0.740274	-6.03088315287437e-07\\
0.740374	-1.02226460274579e-06\\
0.740474	-1.67087729874282e-06\\
0.7405741	9.53980535456367e-07\\
0.7406741	8.01544844719615e-07\\
0.7407741	3.82033996437769e-07\\
0.7408741	6.82969218956941e-07\\
0.7409741	-3.08096463008667e-07\\
0.7410741	-6.03578102165159e-07\\
0.7411741	-2.15858741547947e-07\\
0.7412741	-1.15728929817038e-06\\
0.7413741	-4.40188485661963e-07\\
0.7414741	-1.07684268169095e-06\\
0.7415742	1.36280072737804e-06\\
0.7416742	9.74522847307924e-07\\
0.7417742	1.95813609771278e-07\\
0.7418742	1.45169698484438e-08\\
0.7419742	4.18509730870653e-07\\
0.7420742	-6.04298382889112e-07\\
0.7421742	-6.59646079981258e-08\\
0.7422742	-9.7851305458363e-07\\
0.7423742	-1.35393458755573e-06\\
0.7424742	-1.20418675031431e-06\\
0.7425743	8.267182582955e-07\\
0.7426743	9.8356147493206e-07\\
0.7427743	6.29890616110629e-07\\
0.7428743	7.53882497761538e-07\\
0.7429743	3.43747782238779e-07\\
0.7430743	-6.12268942212779e-07\\
0.7431743	-1.25889066282525e-07\\
0.7432743	-2.08799845857754e-07\\
0.7433743	-8.72654330130729e-07\\
0.7434743	-1.12907122318262e-06\\
0.7435744	1.30269486486867e-06\\
0.7436744	8.18813282066344e-07\\
0.7437744	7.07708529112949e-07\\
0.7438744	-4.20998285122209e-08\\
0.7439744	-4.42057417338049e-07\\
0.7440744	-5.03574969279441e-07\\
0.7441744	-2.38028206922536e-07\\
0.7442744	-6.56757775274741e-07\\
0.7443744	-7.71069122329493e-07\\
0.7444744	-5.92232419549532e-07\\
0.7445745	1.0841116440119e-06\\
0.7446745	8.0784065215056e-07\\
0.7447745	7.91108688780184e-07\\
0.7448745	2.27876846281561e-08\\
0.7449745	4.91785334233441e-07\\
0.7450745	1.87045167798772e-07\\
0.7451745	9.75466578623241e-08\\
0.7452745	-7.87694691695151e-07\\
0.7453745	-4.79627252580883e-07\\
0.7454745	-9.89163178033436e-07\\
0.7455746	8.10561884279082e-07\\
0.7456746	6.25383347774289e-07\\
0.7457746	5.90072406936315e-07\\
0.7458746	6.93862725498207e-07\\
0.7459746	-7.39753645007113e-08\\
0.7460746	2.75865240872974e-07\\
0.7461746	-2.67271509812872e-07\\
0.7462746	-7.14004717572436e-07\\
0.7463746	-1.07491644651958e-06\\
0.7464746	-1.36055164645654e-06\\
0.7465747	4.77386003616687e-07\\
0.7466747	3.02867030121945e-07\\
0.7467747	1.72202963888424e-07\\
0.7468747	7.49982804748583e-08\\
0.7469747	8.95011176282878e-10\\
0.7470747	-6.04271495063813e-08\\
0.7471747	-1.19250765706624e-07\\
0.7472747	-1.85820580256291e-07\\
0.7473747	-2.70343413433238e-07\\
0.7474747	-1.38298809115867e-06\\
0.7475748	4.44937231058873e-07\\
0.7476748	2.37641445188785e-07\\
0.7477748	-2.80638214888995e-08\\
0.7478748	6.37805508851841e-07\\
0.7479748	2.25271949538808e-07\\
0.7480748	-2.75603484034548e-07\\
0.7481748	1.25278833529308e-07\\
0.7482748	-5.81942796173962e-07\\
0.7483748	-4.07091303777918e-07\\
0.7484748	-1.3599507697748e-06\\
0.7485749	4.47567301442575e-07\\
0.7486749	2.01937484511916e-07\\
0.7487749	7.99468740986242e-07\\
0.7488749	2.30533263589905e-07\\
0.7489749	4.85542499539804e-07\\
0.7490749	-4.45052768860421e-07\\
0.7491749	-5.70762317408935e-07\\
0.7492749	-9.0105642058802e-07\\
0.7493749	-4.45365773593664e-07\\
0.7494749	-1.2130814006639e-06\\
0.749575	6.02321346576673e-07\\
0.749675	3.51532460030057e-07\\
0.749775	8.49393780200103e-07\\
0.749875	8.66737908111759e-08\\
0.749975	5.41810427634459e-08\\
0.750075	-2.57235780232179e-07\\
0.750175	1.43312235856285e-07\\
0.750275	-7.53245702256322e-07\\
0.750375	-9.55940016922341e-07\\
0.750475	-4.73760663588507e-07\\
0.7505751	4.17331851743086e-07\\
0.7506751	2.34113416652804e-07\\
0.7507751	7.09033396439906e-07\\
0.7508751	-1.66735558693176e-07\\
0.7509751	-4.01979953323917e-07\\
0.7510751	-5.44539702218572e-09\\
0.7511751	1.4163497930042e-08\\
0.7512751	-3.51816840549191e-07\\
0.7513751	-1.11200882901841e-06\\
0.7514751	-1.27499369906303e-06\\
0.7515752	7.99901578751872e-07\\
0.7516752	7.97328405965914e-07\\
0.7517752	3.66458215772525e-07\\
0.7518752	4.98875381893527e-07\\
0.7519752	1.8620585873208e-07\\
0.7520752	-5.7988277735177e-07\\
0.7521752	1.92318775127376e-07\\
0.7522752	-5.05438400466574e-07\\
0.7523752	-6.81361363774613e-07\\
0.7524752	-3.43615251807705e-07\\
0.7525753	1.06426591006681e-06\\
0.7526753	3.96516259826285e-07\\
0.7527753	2.18184508682384e-07\\
0.7528753	5.2127396799051e-07\\
0.7529753	2.97830213824568e-07\\
0.7530753	-4.60058833695598e-07\\
0.7531753	-7.60262843790827e-07\\
0.7532753	-6.10609016682417e-07\\
0.7533753	-1.01888199921518e-06\\
0.7534753	-9.92823841805546e-07\\
0.7535754	9.39025260215942e-07\\
0.7536754	8.02105520669727e-07\\
0.7537754	7.65395320456719e-08\\
0.7538754	-2.45243593433031e-07\\
0.7539754	-1.70771810914516e-07\\
0.7540754	2.92469923834915e-07\\
0.7541754	1.37039739556144e-07\\
0.7542754	-6.44461120025142e-07\\
0.7543754	-5.93882054644723e-08\\
0.7544754	-1.11505379951637e-06\\
0.7545755	5.7423870458706e-07\\
0.7546755	2.06673731462104e-07\\
0.7547755	1.76686401509585e-07\\
0.7548755	4.77138137000566e-07\\
0.7549755	1.00933926994173e-07\\
0.7550755	4.10223996993864e-08\\
0.7551755	-7.09604127457908e-07\\
0.7552755	-1.5790958141082e-07\\
0.7553755	-3.10814077586485e-07\\
0.7554755	-1.75193855866951e-07\\
0.7555756	5.48170312697494e-07\\
0.7556756	2.31658570548632e-07\\
0.7557756	1.8329994233568e-07\\
0.7558756	3.96394318258331e-07\\
0.7559756	-1.35714246063401e-07\\
0.7560756	-4.19637470550072e-07\\
0.7561756	-4.61942785356939e-07\\
0.7562756	-2.69153295207047e-07\\
0.7563756	-8.47747702259483e-07\\
0.7564756	-2.0416026647041e-07\\
0.7565757	8.73680037649649e-07\\
0.7566757	-6.62872210455134e-08\\
0.7567757	1.96885590497686e-07\\
0.7568757	-3.43057383389578e-07\\
0.7569757	3.0767272907184e-07\\
0.7570757	1.42909579281536e-07\\
0.7571757	1.56531646311464e-07\\
0.7572757	-6.57537697179578e-07\\
0.7573757	-3.05330149608096e-07\\
0.7574757	-7.92832412166433e-07\\
0.7575758	4.2514756160017e-09\\
0.7576758	8.10695207853485e-07\\
0.7577758	-2.40266359696761e-07\\
0.7578758	-1.54439403488027e-07\\
0.7579758	6.24151219597024e-08\\
0.7580758	4.04581585122799e-07\\
0.7581758	-1.33610322805566e-07\\
0.7582758	-5.57785496901886e-07\\
0.7583758	-8.73523411826227e-07\\
0.7584758	-8.63580300602962e-08\\
0.7585759	8.39649227479811e-07\\
0.7586759	8.07290746251965e-07\\
0.7587759	-1.38498305268797e-07\\
0.7588759	-3.06941316807752e-09\\
0.7589759	2.08271664803306e-07\\
0.7590759	-5.09735060383321e-07\\
0.7591759	-1.62303756212623e-07\\
0.7592759	-7.54602716313002e-07\\
0.7593759	-2.91754340686623e-07\\
0.7594759	-7.78835070214257e-07\\
0.759576	7.31198992909299e-07\\
0.759676	3.20252351748707e-07\\
0.759776	-5.55814692049239e-08\\
0.759876	5.98805386609058e-07\\
0.759976	2.78566952616188e-07\\
0.760076	-2.10965422908771e-08\\
0.760176	-3.04938600592664e-07\\
0.760276	-5.77666432022284e-07\\
0.760376	1.56059084766014e-07\\
0.760476	-1.08376555729706e-07\\
0.7605761	4.86684028189188e-07\\
0.7606761	2.03258791131589e-07\\
0.7607761	-9.1896894538479e-08\\
0.7608761	5.96788424012118e-07\\
0.7609761	2.64932765503545e-07\\
0.7610761	-9.17992313276272e-08\\
0.7611761	-4.77696294964858e-07\\
0.7612761	1.02999556439443e-07\\
0.7613761	-3.53906993488806e-07\\
0.7614761	-8.52564491893304e-07\\
0.7615762	3.74851992113179e-07\\
0.7616762	-2.28617754061133e-07\\
0.7617762	1.13992840056198e-07\\
0.7618762	3.98722662886541e-07\\
0.7619762	-3.78342450702007e-07\\
0.7620762	-2.2106970565261e-07\\
0.7621762	-1.33279296932187e-07\\
0.7622762	-1.18744384103309e-07\\
0.7623762	-1.81191055936836e-07\\
0.7624762	-3.24298295772962e-07\\
0.7625763	1.29526748438025e-07\\
0.7626763	-1.94839683942405e-07\\
0.7627763	3.89375770737388e-07\\
0.7628763	-1.21317100365559e-07\\
0.7629763	2.69638745109546e-07\\
0.7630763	-4.4115235464659e-07\\
0.7631763	-2.57038752060623e-07\\
0.7632763	-1.81321445857918e-07\\
0.7633763	-2.17254064516226e-07\\
0.7634763	-3.68042822884718e-07\\
0.7635764	-4.66799212617275e-08\\
0.7636764	5.54267467478553e-07\\
0.7637764	3.10217167509563e-08\\
0.7638764	3.80566548052741e-07\\
0.7639764	-4.0006677348714e-07\\
0.7640764	-3.1379942377896e-07\\
0.7641764	-3.63504985223528e-07\\
0.7642764	-5.52009420395905e-07\\
0.7643764	1.17908941277278e-07\\
0.7644764	-3.56480544638771e-07\\
0.7645765	5.2093880054116e-07\\
0.7646765	-2.59218523979143e-07\\
0.7647765	-1.9159201514185e-07\\
0.7648765	-2.78721388191627e-07\\
0.7649765	4.76901432122645e-07\\
0.7650765	7.2832303743553e-08\\
0.7651765	-4.93325075012763e-07\\
0.7652765	-2.23919171205011e-07\\
0.7653765	-1.21250580242815e-07\\
0.7654765	-1.87572015697413e-07\\
0.7655766	-1.79167116698409e-08\\
0.7656766	5.62040879614756e-07\\
0.7657766	-3.34642793475837e-08\\
0.7658766	1.9350688518216e-07\\
0.7659766	2.40941416151941e-07\\
0.7660766	1.06874352923114e-07\\
0.7661766	-2.10611239559455e-07\\
0.7662766	2.8661572137878e-07\\
0.7663766	-4.03265628889926e-07\\
0.7664766	-2.82028091547204e-07\\
0.7665767	-3.60662193443773e-08\\
0.7666767	-3.06910086678158e-07\\
0.7667767	2.28333451701701e-07\\
0.7668767	-4.31915956333739e-07\\
0.7669767	-2.89190523972138e-07\\
0.7670767	-3.44974316668356e-07\\
0.7671767	3.99296759212575e-07\\
0.7672767	-5.77650238753336e-08\\
0.7673767	2.8250079608938e-07\\
0.7674767	-5.81197123317168e-07\\
0.7675768	5.73221565325355e-07\\
0.7676768	2.88708001505711e-07\\
0.7677768	-2.03355147831985e-07\\
0.7678768	9.59337196348997e-08\\
0.7679768	1.85524477558374e-07\\
0.7680768	6.44152384765562e-08\\
0.7681768	-2.68347607246255e-07\\
0.7682768	1.86330614049268e-07\\
0.7683768	4.27592847351832e-07\\
0.7684768	-5.45369652549255e-07\\
0.7685769	3.97882361724555e-07\\
0.7686769	-1.49787142689206e-08\\
0.7687769	3.55798399631624e-07\\
0.7688769	5.09598227615982e-07\\
0.7689769	4.45853620689718e-07\\
0.7690769	1.64045775186139e-07\\
0.7691769	-3.36295797431774e-07\\
0.7692769	-5.55932428980555e-08\\
0.7693769	5.77963366055201e-09\\
0.7694769	-1.52502620309747e-07\\
0.769577	5.08290002976253e-07\\
0.769677	-9.93057707177059e-08\\
0.769777	7.27568814085089e-08\\
0.769877	2.43459421511716e-08\\
0.769977	-2.44622233580571e-07\\
0.770077	2.65817051037232e-07\\
0.770177	-4.44323150716741e-07\\
0.770277	-3.74981423689746e-07\\
0.770377	4.73951994983679e-07\\
0.770477	1.02635230980752e-07\\
0.7705771	4.58068837971837e-07\\
0.7706771	-3.62300175638097e-07\\
0.7707771	-4.02154581635727e-07\\
0.7708771	3.38857181425567e-07\\
0.7709771	-1.38864969301267e-07\\
0.7710771	1.65127211460003e-07\\
0.7711771	2.51330302347696e-07\\
0.7712771	1.20289223914938e-07\\
0.7713771	-2.27402789043651e-07\\
0.7714771	2.08895825082323e-07\\
0.7715772	2.84483897772247e-07\\
0.7716772	2.81667515089623e-07\\
0.7717772	6.50571223559382e-08\\
0.7718772	-3.64512447781351e-07\\
0.7719772	-6.15807049797468e-09\\
0.7720772	1.41051657465141e-07\\
0.7721772	7.80964075630664e-08\\
0.7722772	-1.93995874475661e-07\\
0.7723772	3.25851003069033e-07\\
0.7724772	-3.61238514656748e-07\\
0.7725773	-4.91591690021664e-07\\
0.7726773	4.01808340377841e-07\\
0.7727773	9.19359708362322e-08\\
0.7728773	-4.19891437175579e-07\\
0.7729773	-1.32308313638418e-07\\
0.7730773	-4.39009131270041e-08\\
0.7731773	-1.53207321496041e-07\\
0.7732773	5.41282526356213e-07\\
0.7733773	4.11268423761157e-08\\
0.7734773	3.47931965016812e-07\\
0.7735774	1.33868041984897e-07\\
0.7736774	5.04174302307092e-08\\
0.7737774	-2.20962958152882e-07\\
0.7738774	3.21525645219367e-07\\
0.7739774	-3.20269949227736e-07\\
0.7740774	-1.44454891604084e-07\\
0.7741774	-1.49086314849001e-07\\
0.7742774	-3.32173332395769e-07\\
0.7743774	3.08322913866732e-07\\
0.7744774	-2.25510741636015e-07\\
0.7745775	-3.52835498595994e-07\\
0.7746775	-2.38045569922463e-07\\
0.7747775	-2.9103514975759e-07\\
0.7748775	-5.09525665215449e-07\\
0.7749775	1.08809325105597e-07\\
0.7750775	-4.33655865883509e-07\\
0.7751775	-1.34499110515662e-07\\
0.7752775	8.7495388712e-09\\
0.7753775	-1.3921770403158e-09\\
0.7754775	-1.62358751332192e-07\\
0.7755776	1.55760950804051e-08\\
0.7756776	-4.4829692225079e-07\\
0.7757776	-5.50091634465844e-08\\
0.7758776	1.98195774459009e-07\\
0.7759776	3.14121942679435e-07\\
0.7760776	2.95621022772252e-07\\
0.7761776	1.45592296441066e-07\\
0.7762776	-1.33017370451682e-07\\
0.7763776	4.62786434951568e-07\\
0.7764776	-6.39543502600759e-08\\
0.7765777	-3.14359724917779e-07\\
0.7766777	-8.59886286619371e-08\\
0.7767777	2.92517938760284e-08\\
0.7768777	3.4593364617308e-08\\
0.7769777	-6.66847004282545e-08\\
0.7770777	-2.71255832906547e-07\\
0.7771777	4.24253871145197e-07\\
0.7772777	2.32656169885104e-08\\
0.7773777	5.29247885516781e-07\\
0.7774777	-5.42836060546392e-08\\
0.7775778	-4.1898172131738e-07\\
0.7776778	-1.79885270235047e-07\\
0.7777778	-1.94685885190893e-08\\
0.7778778	6.59727490415207e-08\\
0.7779778	8.01902499958373e-08\\
0.7780778	2.69824846910183e-08\\
0.7781778	-8.98049670183809e-08\\
0.7782778	-2.66279541527581e-07\\
0.7783778	5.01498288230096e-07\\
0.7784778	2.17514956268872e-07\\
0.7785779	9.99450389116419e-08\\
0.7786779	-2.84455393462224e-07\\
0.7787779	2.91627627824198e-07\\
0.7788779	-1.676320868782e-07\\
0.7789779	3.41986002183603e-07\\
0.7790779	-1.75250856493392e-07\\
0.7791779	2.84971250152921e-07\\
0.7792779	7.27012859691456e-07\\
0.7793779	1.55281110636452e-07\\
0.7794779	5.74229678917959e-07\\
0.779578	1.12267501073404e-07\\
0.779678	-4.82874203733274e-07\\
0.779778	-7.36909875342917e-08\\
0.779878	3.44456690726247e-07\\
0.779978	-2.23745265870434e-07\\
0.780078	2.26435361128097e-07\\
0.780178	-3.00222939841888e-07\\
0.780278	2.01104555230813e-07\\
0.780378	7.35288753439534e-07\\
0.780478	3.0724671429283e-07\\
0.7805781	-4.39251488515424e-08\\
0.7806781	-3.90433850583349e-07\\
0.7807781	-6.84136090001175e-07\\
0.7808781	8.00693964464472e-08\\
0.7809781	-9.26702021786241e-08\\
0.7810781	-1.97161808657143e-07\\
0.7811781	-2.28166515015005e-07\\
0.7812781	-1.80399612581539e-07\\
0.7813781	-4.85306372866035e-08\\
0.7814781	1.72816566390566e-07\\
0.7815782	-5.66075121000154e-07\\
0.7816782	-1.58357291724087e-07\\
0.7817782	-6.44753475498305e-07\\
0.7818782	-1.97051361894296e-08\\
0.7819782	-2.77608254606321e-07\\
0.7820782	-4.12813415451296e-07\\
0.7821782	5.80374171832432e-07\\
0.7822782	7.07694627344324e-07\\
0.7823782	-2.50666785106546e-08\\
0.7824782	3.87920849220791e-07\\
0.7825783	-1.91329038834454e-07\\
0.7826783	-4.78010908189219e-07\\
0.7827783	-6.01175351278016e-07\\
0.7828783	-5.54811374620101e-07\\
0.7829783	-3.32863001162309e-07\\
0.7830783	7.07706329095004e-08\\
0.7831783	6.62235271065015e-07\\
0.7832783	4.47721446583671e-07\\
0.7833783	4.33464445670495e-07\\
0.7834783	6.25744253746063e-07\\
0.7835784	-2.01105208041952e-07\\
0.7836784	-5.85510898432062e-07\\
0.7837784	-7.44265759244911e-07\\
0.7838784	3.29088455153226e-07\\
0.7839784	-3.58945554168599e-07\\
0.7840784	1.98179253452224e-07\\
0.7841784	7.05423008895423e-09\\
0.7842784	7.43149914939067e-08\\
0.7843784	4.06641335715108e-07\\
0.7844784	1.01075719616261e-06\\
0.7845785	-4.26049547641583e-07\\
0.7846785	-2.66717799846106e-07\\
0.7847785	1.84846387973892e-07\\
0.7848785	-6.44570539165557e-08\\
0.7849785	-7.68433583431261e-09\\
0.7850785	3.62152098531965e-07\\
0.7851785	5.20835348183368e-08\\
0.7852785	6.9184915574283e-08\\
0.7853785	4.2057478299995e-07\\
0.7854785	1.11341522845976e-06\\
0.7855786	-2.51374282278505e-07\\
0.7856786	-8.6261305121127e-07\\
0.7857786	-1.10647075501902e-07\\
0.7858786	1.18593082021334e-08\\
0.7859786	-4.87715018593349e-07\\
0.7860786	3.98051990480042e-07\\
0.7861786	6.76625485418469e-07\\
0.7862786	3.55513642524841e-07\\
0.7863786	4.42267618971925e-07\\
0.7864786	9.44481459264068e-07\\
0.7865787	-6.22572799535703e-07\\
0.7866787	-2.75051124987158e-07\\
0.7867787	-4.89022900396208e-07\\
0.7868787	-2.56723063252196e-07\\
0.7869787	4.29656021871949e-07\\
0.7870787	5.7796449048908e-07\\
0.7871787	1.96094926963042e-07\\
0.7872787	2.91982284572612e-07\\
0.7873787	8.73603822171276e-07\\
0.7874787	9.48979032244779e-07\\
0.7875788	-1.05150387952335e-06\\
0.7876788	-9.72880266125742e-07\\
0.7877788	-3.76183610484304e-07\\
0.7878788	-2.53226135171758e-07\\
0.7879788	4.04221834671148e-07\\
0.7880788	6.04431787287751e-07\\
0.7881788	3.55716953670537e-07\\
0.7882788	6.66432247165005e-07\\
0.7883788	5.44974181337921e-07\\
0.7884788	9.99780797705796e-07\\
0.7885789	-6.22837841746104e-07\\
0.7886789	-9.98424424025757e-07\\
0.7887789	-7.72175143382015e-07\\
0.7888789	6.45134772270239e-08\\
0.7889789	-4.7971390859658e-07\\
0.7890789	6.03828440581822e-07\\
0.7891789	3.23867285878521e-07\\
0.7892789	6.89170323475707e-07\\
0.7893789	7.08546131988896e-07\\
0.7894789	3.90844073550056e-07\\
0.789579	-1.0008569395481e-06\\
0.789679	-9.74318619562098e-07\\
0.789779	-2.58057085034125e-07\\
0.789879	1.56939488960006e-07\\
0.789979	2.79723320950609e-07\\
0.790079	1.19386983854497e-07\\
0.790179	6.85063288763388e-07\\
0.790279	9.85925234786578e-07\\
0.790379	1.03118591709972e-06\\
0.790479	8.30098463744378e-07\\
0.7905791	-4.36601745867904e-07\\
0.7906791	-1.10689473586945e-07\\
0.7907791	-3.11724690504889e-09\\
0.7908791	-1.04472622552976e-07\\
0.7909791	-4.05303518746791e-07\\
0.7910791	1.03881686364105e-07\\
0.7911791	4.32614073453408e-07\\
0.7912791	5.9046411227115e-07\\
0.7913791	5.87041561139756e-07\\
0.7914791	4.31995405669738e-07\\
0.7915792	-7.75355013082901e-07\\
0.7916792	-2.12672297550398e-07\\
0.7917792	-7.72421965500314e-07\\
0.7918792	-4.44798949672531e-07\\
0.7919792	-2.19959371605682e-07\\
0.7920792	-8.80206396658423e-08\\
0.7921792	-3.90615157996876e-08\\
0.7922792	9.36877775359335e-07\\
0.7923792	8.49795501522976e-07\\
0.7924792	7.09728287695555e-07\\
0.7925793	-4.64454001480874e-07\\
0.7926793	-6.88255784808689e-07\\
0.7927793	-9.34692974574602e-07\\
0.7928793	-1.93576214613245e-07\\
0.7929793	-4.54678197314706e-07\\
0.7930793	2.92266259194207e-07\\
0.7931793	5.7560106903054e-08\\
0.7932793	8.51544001534421e-07\\
0.7933793	6.84596199285181e-07\\
0.7934793	5.67132477868881e-07\\
0.7935794	-5.61421989786481e-07\\
0.7936794	-5.56444896382402e-07\\
0.7937794	-4.7050162077511e-07\\
0.7938794	-2.93027154540937e-07\\
0.7939794	-1.34194273471167e-08\\
0.7940794	3.78960627323011e-07\\
0.7941794	-1.05211025669405e-07\\
0.7942794	5.44778373523513e-07\\
0.7943794	3.39678305039115e-07\\
0.7944794	1.29027487005828e-06\\
0.7945795	-7.42409213216888e-07\\
0.7946795	-4.55732202198078e-07\\
0.7947795	1.92295042111823e-08\\
0.7948795	-3.06592372822934e-07\\
0.7949795	-4.2222994345309e-07\\
0.7950795	-3.16679238387962e-07\\
0.7951795	2.10996820015907e-08\\
0.7952795	6.02182657516437e-07\\
0.7953795	4.37681313414373e-07\\
0.7954795	1.53874298280954e-06\\
0.7955796	-1.31093332900178e-06\\
0.7956796	-6.52867701234072e-07\\
0.7957796	-6.95572521358656e-07\\
0.7958796	-4.27758556575952e-07\\
0.7959796	1.61898665140825e-07\\
0.7960796	8.47587386942905e-08\\
0.7961796	3.522163050107e-07\\
0.7962796	9.7570095203281e-07\\
0.7963796	9.66677107250291e-07\\
0.7964796	1.33664395995936e-06\\
0.7965797	-1.20690884308772e-06\\
0.7966797	-1.05191661781134e-06\\
0.7967797	-4.83216969549005e-07\\
0.7968797	-4.89172637507806e-07\\
0.7969797	-5.81120831455451e-08\\
0.7970797	-1.78329610633909e-07\\
0.7971797	1.61914550211861e-07\\
0.7972797	9.74394154695801e-07\\
0.7973797	1.27091682244895e-06\\
0.7974797	1.06332395688469e-06\\
0.7975798	-1.01595509693908e-06\\
0.7976798	-1.2035945848865e-06\\
0.7977798	-8.59611771986124e-07\\
0.7978798	2.79688885385099e-08\\
0.7979798	-5.28843773217602e-07\\
0.7980798	4.8199224966794e-07\\
0.7981798	7.25520505895361e-08\\
0.7982798	2.54943693711596e-07\\
0.7983798	1.04130813483039e-06\\
0.7984798	1.4438190865107e-06\\
0.7985799	-9.78971611331758e-07\\
0.7986799	-3.14869131656792e-07\\
0.7987799	-9.97891031140341e-07\\
0.7988799	-1.57334705264134e-08\\
0.7989799	-3.56060326112129e-07\\
0.7990799	-6.50332143692367e-09\\
0.7991799	4.53378810050253e-08\\
0.7992799	8.11895592534029e-07\\
0.7993799	1.30563396805172e-06\\
0.7994799	1.53904891853074e-06\\
0.79958	-1.0019697072039e-06\\
0.79968	-1.25881631607427e-06\\
0.79978	-7.38296534130711e-07\\
0.79988	-4.27788435786169e-07\\
0.79998	-3.14638875842377e-07\\
0.80008	6.1383641991597e-07\\
};
\addplot [color=mycolor1,solid,forget plot]
  table[row sep=crcr]{%
0.80008	6.1383641991597e-07\\
0.80018	3.70352756462466e-07\\
0.80028	9.67656347050649e-07\\
0.80038	1.41852421897681e-06\\
0.80048	1.73576411377496e-06\\
0.8005801	-6.66148847194847e-07\\
0.8006801	-5.84720881491307e-07\\
0.8007801	-5.98301586940408e-07\\
0.8008801	-6.93961441200486e-07\\
0.8009801	1.4125927272346e-07\\
0.8010801	-7.96496797406121e-08\\
0.8011801	6.56331453185999e-07\\
0.8012801	3.62252253083284e-07\\
0.8013801	1.05119206672555e-06\\
0.8014801	1.73625986565895e-06\\
0.8015802	-1.23820596709123e-06\\
0.8016802	-5.2840819986244e-07\\
0.8017802	-7.82965548395964e-07\\
0.8018802	1.13484830421839e-08\\
0.8019802	-1.32210533188015e-07\\
0.8020802	-2.00358020840952e-07\\
0.8021802	8.2021946479216e-07\\
0.8022802	9.4286414498157e-07\\
0.8023802	1.18094690115811e-06\\
0.8024802	1.54786716599631e-06\\
0.8025803	-6.8086580551352e-07\\
0.8026803	-1.02279610025846e-06\\
0.8027803	-1.95506875577678e-07\\
0.8028803	-1.85485572679056e-07\\
0.8029803	2.08083879016385e-08\\
0.8030803	4.36943460968564e-07\\
0.8031803	7.65158763016416e-08\\
0.8032803	9.53149545068754e-07\\
0.8033803	1.08049592384418e-06\\
0.8034803	1.47223392321649e-06\\
0.8035804	-1.66362010922683e-06\\
0.8036804	-7.08654213532611e-07\\
0.8037804	-4.48082651871573e-07\\
0.8038804	1.31882172738074e-07\\
0.8039804	4.50547430475012e-08\\
0.8040804	3.05276317114078e-07\\
0.8041804	9.26414803537057e-07\\
0.8042804	9.22364664646125e-07\\
0.8043804	1.30704679968119e-06\\
0.8044804	1.09440843409914e-06\\
0.8045805	-1.57366329478492e-06\\
0.8046805	-9.45558640275124e-07\\
0.8047805	-8.72762004888727e-07\\
0.8048805	-3.41221992883334e-07\\
0.8049805	-3.36861464056426e-07\\
0.8050805	1.54422381548613e-07\\
0.8051805	1.46757849428525e-07\\
0.8052805	6.54298661473263e-07\\
0.8053805	6.9122383372644e-07\\
0.8054805	1.27173757391574e-06\\
0.8055806	-1.52701207856154e-06\\
0.8056806	-8.23029243868234e-07\\
0.8057806	-5.32681015741332e-07\\
0.8058806	-6.41663592482189e-07\\
0.8059806	-1.3564857326287e-07\\
0.8060806	-2.83059797823171e-10\\
0.8061806	7.78810222534787e-07\\
0.8062806	1.2160327953481e-06\\
0.8063806	1.32581033529533e-06\\
0.8064806	1.12259253537417e-06\\
0.8065807	-1.37979631364971e-06\\
0.8066807	-1.17183694214873e-06\\
0.8067807	-2.33366288959758e-07\\
0.8068807	-5.4983969999256e-07\\
0.8069807	-1.06689066470267e-07\\
0.8070807	1.10677047082675e-07\\
0.8071807	1.16873278166452e-07\\
0.8072807	9.26537357059942e-07\\
0.8073807	1.55432999893179e-06\\
0.8074807	1.01493476378778e-06\\
0.8075808	-1.73970831429671e-06\\
0.8076808	-5.75468124708323e-07\\
0.8077808	-5.34214377623243e-07\\
0.8078808	-6.01173229419771e-07\\
0.8079808	2.3845142571588e-07\\
0.8080808	-5.2214232937331e-10\\
0.8081808	6.96746358386235e-07\\
0.8082808	1.34511913518764e-06\\
0.8083808	9.59480185969142e-07\\
0.8084808	1.55473519569682e-06\\
0.8085809	-9.77597626850013e-07\\
0.8086809	-1.38173329045088e-06\\
0.8087809	-7.60114110320842e-07\\
0.8088809	-9.77488552322825e-08\\
0.8089809	-3.79625235247261e-07\\
0.8090809	4.0929002054213e-07\\
0.8091809	2.84051015597697e-07\\
0.8092809	1.25973258668921e-06\\
0.8093809	1.35143016244399e-06\\
0.8094809	1.5742596675139e-06\\
0.809581	-1.23919825423258e-06\\
0.809681	-7.1450677818774e-07\\
0.809781	-1.01319871248506e-06\\
0.809881	-1.20077380927341e-07\\
0.809981	-1.99262197853045e-08\\
0.810081	3.02491105053093e-07\\
0.810181	8.62430571402228e-07\\
0.810281	6.75167673769295e-07\\
0.810381	7.5599731985676e-07\\
0.810481	1.12023368892267e-06\\
0.8105811	-1.45697524978061e-06\\
0.8106811	-1.48558455670411e-06\\
0.8107811	-1.18471471033388e-06\\
0.8108811	-5.38975574393419e-07\\
0.8109811	-5.32958343235634e-07\\
0.8110811	-1.51235655998505e-07\\
0.8111811	6.21638300835059e-07\\
0.8112811	8.01127639782351e-07\\
0.8113811	1.40271467685693e-06\\
0.8114811	1.44189979511822e-06\\
0.8115812	-1.36207739132033e-06\\
0.8116812	-1.40664682124481e-06\\
0.8117812	-9.66994222473261e-07\\
0.8118812	-2.75481131239985e-08\\
0.8119812	4.2728045457352e-07\\
0.8120812	4.13097771101434e-07\\
0.8121812	9.45527347084862e-07\\
0.8122812	1.04020980185027e-06\\
0.8123812	1.7128027183233e-06\\
0.8124812	1.97898054477008e-06\\
0.8125813	-1.49638290114495e-06\\
0.8126813	-1.00131225355682e-06\\
0.8127813	-8.65517711723385e-07\\
0.8128813	-7.32586062923701e-08\\
0.8129813	3.91221970375e-07\\
0.8130813	5.43697052712844e-07\\
0.8131813	3.99955673024976e-07\\
0.8132813	9.75802746161492e-07\\
0.8133813	1.28705893942538e-06\\
0.8134813	1.34956055131141e-06\\
0.8135814	-1.22462477936125e-06\\
0.8136814	-6.17270588421093e-07\\
0.8137814	-1.21105360939566e-06\\
0.8138814	9.92377913178188e-09\\
0.8139814	6.15742248299966e-08\\
0.8140814	-4.01747373146577e-08\\
0.8141814	7.20619211413975e-07\\
0.8142814	1.35991304084371e-06\\
0.8143814	8.93678241631335e-07\\
0.8144814	1.33790072354145e-06\\
0.8145815	-1.74658261808958e-06\\
0.8146815	-1.43848037570393e-06\\
0.8147815	-1.17186191506136e-06\\
0.8148815	6.9315084338939e-08\\
0.8149815	3.01106720890232e-07\\
0.8150815	5.39582767800084e-07\\
0.8151815	8.00826539304467e-07\\
0.8152815	1.10093476823891e-06\\
0.8153815	1.45601749679258e-06\\
0.8154815	1.88219793351152e-06\\
0.8155816	-1.109327695481e-06\\
0.8156816	-1.49741864952091e-06\\
0.8157816	-7.65948266590044e-07\\
0.8158816	-8.98741852495277e-07\\
0.8159816	1.20387852575732e-07\\
0.8160816	3.07640537933196e-07\\
0.8161816	6.79228210564986e-07\\
0.8162816	1.25137506756801e-06\\
0.8163816	1.04031737091503e-06\\
0.8164816	2.06230333343882e-06\\
0.8165817	-1.21950802345339e-06\\
0.8166817	-6.87369249519065e-07\\
0.8167817	-8.73355482156057e-07\\
0.8168817	-7.61171953023165e-07\\
0.8169817	-3.34512577637724e-07\\
0.8170817	4.22939936406408e-07\\
0.8171817	5.27513964954096e-07\\
0.8172817	9.95548837146032e-07\\
0.8173817	1.84339472308892e-06\\
0.8174817	2.08741249529965e-06\\
0.8175818	-1.85566037025353e-06\\
0.8176818	-7.74736828468292e-07\\
0.8177818	-1.24847933680172e-06\\
0.8178818	-2.60485385261688e-07\\
0.8179818	2.05657614493759e-07\\
0.8180818	1.66372235099743e-07\\
0.8181818	6.38090893900767e-07\\
0.8182818	6.3725572463369e-07\\
0.8183818	1.18031847851086e-06\\
0.8184818	1.28374038510515e-06\\
0.8185819	-1.68053602234153e-06\\
0.8186819	-1.41137327958063e-06\\
0.8187819	-5.32395575625344e-07\\
0.8188819	-2.7104985811377e-08\\
0.8189819	1.21005278685971e-07\\
0.8190819	-7.15492536329521e-08\\
0.8191819	4.11755555873583e-07\\
0.8192819	5.87452359646079e-07\\
0.8193819	1.47208216283445e-06\\
0.8194819	2.08219423303291e-06\\
0.819582	-1.25342746049029e-06\\
0.819682	-1.14690377284177e-06\\
0.819782	-1.26518516907481e-06\\
0.819882	-5.91690606732698e-07\\
0.819982	-1.09831404415672e-07\\
0.820082	1.96988630740691e-07\\
0.820182	3.45373095456125e-07\\
0.820282	1.35193283234969e-06\\
0.820382	1.23328585299731e-06\\
0.820482	2.0060571719327e-06\\
0.8205821	-2.04248291346687e-06\\
0.8206821	-1.44103940158402e-06\\
0.8207821	-8.98244937985737e-07\\
0.8208821	-3.97447622368929e-07\\
0.8209821	7.80108635467514e-08\\
0.8210821	5.44795129009401e-07\\
0.8211821	1.01957594544899e-06\\
0.8212821	1.51903013456689e-06\\
0.8213821	1.05984043763385e-06\\
0.8214821	1.65869541213226e-06\\
0.8215822	-1.43699615140491e-06\\
0.8216822	-1.6758638130554e-06\\
0.8217822	-8.06570861655587e-07\\
0.8218822	-8.1240675964267e-07\\
0.8219822	3.23344232278089e-07\\
0.8220822	6.17402923452914e-07\\
0.8221822	1.0864950734657e-06\\
0.8222822	7.47351271535024e-07\\
0.8223822	1.61670681286807e-06\\
0.8224822	1.71130157688992e-06\\
0.8225823	-1.75965879378737e-06\\
0.8226823	-1.16808107009092e-06\\
0.8227823	-1.30100140616207e-06\\
0.8228823	-1.41662781061314e-07\\
0.8229823	3.26695801966537e-07\\
0.8230823	1.20839215966839e-07\\
0.8231823	2.57536060033203e-07\\
0.8232823	7.53558553334699e-07\\
0.8233823	1.62568241801608e-06\\
0.8234823	1.89068675116033e-06\\
0.8235824	-1.27876278277483e-06\\
0.8236824	-1.18121245762381e-06\\
0.8237824	-6.40408510221846e-07\\
0.8238824	-6.39559539550305e-07\\
0.8239824	-1.61871358095311e-07\\
0.8240824	8.09452876282535e-07\\
0.8241824	2.91212542968822e-07\\
0.8242824	1.30020943700515e-06\\
0.8243824	1.85324764956363e-06\\
0.8244824	1.96713345168575e-06\\
0.8245825	-1.22034056015963e-06\\
0.8246825	-9.37729777383822e-07\\
0.8247825	-1.04382390819779e-06\\
0.8248825	-5.21809172759902e-07\\
0.8249825	-3.54870199892332e-07\\
0.8250825	4.73809834922179e-07\\
0.8251825	9.8104910950525e-07\\
0.8252825	1.18366702128725e-06\\
0.8253825	1.09848406726343e-06\\
0.8254825	1.74232173577593e-06\\
0.8255826	-1.78023102925806e-06\\
0.8256826	-1.63111321249687e-06\\
0.8257826	-7.02488754278363e-07\\
0.8258826	2.2466596583115e-08\\
0.8259826	-4.39422508158316e-07\\
0.8260826	-7.13311587574594e-08\\
0.8261826	1.14356573277519e-06\\
0.8262826	1.22209328656808e-06\\
0.8263826	1.18107654767385e-06\\
0.8264826	2.03734037151904e-06\\
0.8265827	-2.13605921439708e-06\\
0.8266827	-1.43782172967732e-06\\
0.8267827	-7.91814641853961e-07\\
0.8268827	-1.81215047856398e-07\\
0.8269827	4.10799171568499e-07\\
0.8270827	1.04924824384511e-09\\
0.8271827	6.06355388121926e-07\\
0.8272827	1.24353668562449e-06\\
0.8273827	1.92941096877064e-06\\
0.8274827	1.68079471984939e-06\\
0.8275828	-1.4591180437229e-06\\
0.8276828	-1.52916442131001e-06\\
0.8277828	-4.83244432025742e-07\\
0.8278828	-3.04548217755496e-07\\
0.8279828	2.37321495788478e-08\\
0.8280828	5.18402524463113e-07\\
0.8281828	1.19626660000094e-06\\
0.8282828	1.07412577809995e-06\\
0.8283828	1.16877905975699e-06\\
0.8284828	1.4970229220701e-06\\
0.8285829	-1.92614073846187e-06\\
0.8286829	-1.08306133705938e-06\\
0.8287829	-9.56001388274785e-07\\
0.8288829	-5.28175623237814e-07\\
0.8289829	2.17198112384409e-07\\
0.8290829	2.9689878600081e-07\\
0.8291829	7.27702040403955e-07\\
0.8292829	1.52638007833161e-06\\
0.8293829	1.70970156787575e-06\\
0.8294829	2.29443151544828e-06\\
0.829583	-1.73095257016698e-06\\
0.829683	-1.29568260565094e-06\\
0.829783	-4.08715533417592e-07\\
0.829883	-5.33021107251841e-08\\
0.829983	-2.12697348622726e-07\\
0.830083	1.2983941477529e-07\\
0.830183	9.9104439099662e-07\\
0.830283	1.38764922930079e-06\\
0.830383	1.33638090016319e-06\\
0.830483	1.85396159002593e-06\\
0.8305831	-2.0959911153362e-06\\
0.8306831	-1.39295489720581e-06\\
0.8307831	-1.07091624323985e-06\\
0.8308831	-1.13173230342767e-07\\
0.8309831	-5.03029295106927e-07\\
0.8310831	7.76206676711411e-07\\
0.8311831	7.41220241184948e-07\\
0.8312831	1.40869125653253e-06\\
0.8313831	1.79529381894028e-06\\
0.8314831	1.91769610546899e-06\\
0.8315832	-1.28368416607572e-06\\
0.8316832	-1.64192455853751e-06\\
0.8317832	-1.21438051348122e-06\\
0.8318832	1.55914583466199e-08\\
0.8319832	6.46283968563921e-08\\
0.8320832	-5.06392150434465e-08\\
0.8321832	6.86412436667894e-07\\
0.8322832	1.29240039870027e-06\\
0.8323832	1.78393482475059e-06\\
0.8324832	2.17761888654522e-06\\
0.8325833	-1.60767505708037e-06\\
0.8326833	-1.36196710531777e-06\\
0.8327833	-1.1643264783423e-06\\
0.8328833	1.82093407019579e-09\\
0.8329833	1.53041707307011e-07\\
0.8330833	3.05894777063997e-07\\
0.8331833	4.76931332205055e-07\\
0.8332833	6.82694710985032e-07\\
0.8333833	1.93972027684453e-06\\
0.8334833	2.26453534768467e-06\\
0.8335834	-1.44388511502669e-06\\
0.8336834	-9.35832855208929e-07\\
0.8337834	-1.31044215567755e-06\\
0.8338834	-5.51219060174191e-07\\
0.8339834	3.58321804672102e-07\\
0.8340834	4.34657083658863e-07\\
0.8341834	6.94254627564561e-07\\
0.8342834	1.15357336660793e-06\\
0.8343834	1.82906321066767e-06\\
0.8344834	1.73716494389353e-06\\
0.8345835	-1.24140360346914e-06\\
0.8346835	-8.20519078903459e-07\\
0.8347835	-1.11773969857509e-06\\
0.8348835	-1.16662233207165e-07\\
0.8349835	1.99106900744539e-07\\
0.8350835	8.45951535488609e-07\\
0.8351835	8.40245648170423e-07\\
0.8352835	1.19835326373163e-06\\
0.8353835	1.93662834080399e-06\\
0.8354835	2.07141467845062e-06\\
0.8355836	-1.5331954736908e-06\\
0.8356836	-1.55795903689082e-06\\
0.8357836	-1.13722563588325e-06\\
0.8358836	-2.54693155010699e-07\\
0.8359836	1.05929841787145e-07\\
0.8360836	-3.90759749002711e-08\\
0.8361836	3.26559177032948e-07\\
0.8362836	1.21909411321397e-06\\
0.8363836	1.65477655711399e-06\\
0.8364836	1.64984305994764e-06\\
0.8365837	-1.94661792374262e-06\\
0.8366837	-1.78551893093015e-06\\
0.8367837	-1.01637811145139e-06\\
0.8368837	-6.23004648314662e-07\\
0.8369837	4.10780597981386e-07\\
0.8370837	1.0114498305569e-07\\
0.8371837	4.64243991249447e-07\\
0.8372837	1.51622111665972e-06\\
0.8373837	1.27320778764428e-06\\
0.8374837	1.75132324775618e-06\\
0.8375838	-2.21373680187043e-06\\
0.8376838	-1.24629380771069e-06\\
0.8377838	-5.09422226624423e-07\\
0.8378838	1.29474964083443e-08\\
0.8379838	3.36872254536047e-07\\
0.8380838	4.78396172010065e-07\\
0.8381838	4.53550528245472e-07\\
0.8382838	1.2783536349481e-06\\
0.8383838	1.96881075265054e-06\\
0.8384838	1.54091399862111e-06\\
0.8385839	-2.18143444108421e-06\\
0.8386839	-1.79919409060147e-06\\
0.8387839	-4.87395172665828e-07\\
0.8388839	-2.30099134412853e-07\\
0.8389839	-1.13810543211912e-08\\
0.8390839	1.84670271075049e-07\\
0.8391839	3.73952225629637e-07\\
0.8392839	1.57234829023523e-06\\
0.8393839	1.7957279330183e-06\\
0.8394839	2.05994652979768e-06\\
0.839584	-1.82130069337205e-06\\
0.839684	-1.42881451337118e-06\\
0.839784	-9.47992983224566e-07\\
0.839884	-3.63038076933719e-07\\
0.839984	3.41833675587822e-07\\
0.840084	1.82391076020139e-07\\
0.840184	1.17438818758586e-06\\
0.840284	1.33356422749031e-06\\
0.840384	1.67564347286486e-06\\
0.840484	2.21633518560083e-06\\
0.8405841	-2.23929949871859e-06\\
0.8406841	-1.25507781767453e-06\\
0.8407841	-1.02519129940859e-06\\
0.8408841	-5.33991717510673e-07\\
0.8409841	2.34153680001015e-07\\
0.8410841	2.94862074756708e-07\\
0.8411841	6.63734990968834e-07\\
0.8412841	1.35635820264213e-06\\
0.8413841	1.38830164564396e-06\\
0.8414841	1.7751193315263e-06\\
0.8415842	-1.68520344479006e-06\\
0.8416842	-1.54264570539908e-06\\
0.8417842	-9.9863264901856e-07\\
0.8418842	-3.76748778840863e-08\\
0.8419842	3.55700637388878e-07\\
0.8420842	1.96950468822621e-07\\
0.8421842	5.01514637374356e-07\\
0.8422842	1.28481655536916e-06\\
0.8423842	1.56226289549366e-06\\
0.8424842	2.34924353081922e-06\\
0.8425843	-1.56178931698392e-06\\
0.8426843	-1.71009017124746e-06\\
0.8427843	-1.30277345489205e-06\\
0.8428843	-3.24517392957802e-07\\
0.8429843	2.39982561378582e-07\\
0.8430843	4.06013636045088e-07\\
0.8431843	1.18884565569743e-06\\
0.8432843	1.60373095603461e-06\\
0.8433843	1.66590430739078e-06\\
0.8434843	2.39058281659155e-06\\
0.8435844	-1.43378800832039e-06\\
0.8436844	-1.33881842234729e-06\\
0.8437844	-5.35783967858805e-07\\
0.8438844	-9.53901269085122e-09\\
0.8439844	2.55044005470495e-07\\
0.8440844	2.73074492795899e-07\\
0.8441844	1.05964361774014e-06\\
0.8442844	1.62982422935443e-06\\
0.8443844	1.99867078842786e-06\\
0.8444844	2.18121926165438e-06\\
0.8445845	-2.03658239206206e-06\\
0.8446845	-1.181745291845e-06\\
0.8447845	-4.68195100911117e-07\\
0.8448845	-8.80970585370733e-07\\
0.8449845	-4.05129422897943e-07\\
0.8450845	-2.57482395404907e-08\\
0.8451845	1.27207729549994e-06\\
0.8452845	1.50323237946282e-06\\
0.8453845	1.68258302357671e-06\\
0.8454845	1.82497594725106e-06\\
0.8455846	-1.28464735604794e-06\\
0.8456846	-1.1717071304318e-06\\
0.8457846	-1.05128613769878e-06\\
0.8458846	-9.08615578776306e-07\\
0.8459846	2.71053687050937e-07\\
0.8460846	5.02451042549978e-07\\
0.8461846	8.00286053426191e-07\\
0.8462846	1.1792483807227e-06\\
0.8463846	1.65400772234037e-06\\
0.8464846	2.23921371000912e-06\\
0.8465847	-1.27972637642415e-06\\
0.8466847	-1.42961176896605e-06\\
0.8467847	-4.25208082521777e-07\\
0.8468847	-2.51946668416991e-07\\
0.8469847	1.04720719651397e-07\\
0.8470847	6.59321804707957e-07\\
0.8471847	4.26363742533908e-07\\
0.8472847	1.42033303873745e-06\\
0.8473847	1.65569547672106e-06\\
0.8474847	2.14689604494112e-06\\
0.8475848	-1.31873959663409e-06\\
0.8476848	-1.27231947866235e-06\\
0.8477848	-9.26837389059187e-07\\
0.8478848	-2.67932270592297e-07\\
0.8479848	-2.81264217250055e-07\\
0.8480848	4.74854449095119e-08\\
0.8481848	7.32614093834627e-07\\
0.8482848	7.88397732875978e-07\\
0.8483848	1.22909092192813e-06\\
0.8484848	2.06892672816039e-06\\
0.8485849	-1.90141857014936e-06\\
0.8486849	-1.22025013782334e-06\\
0.8487849	-1.09735554421064e-06\\
0.8488849	-5.18588501030592e-07\\
0.8489849	-4.69824570803468e-07\\
0.8490849	6.3038750219846e-08\\
0.8491849	1.09408196902194e-06\\
0.8492849	1.63736353231414e-06\\
0.8493849	1.70691975220905e-06\\
0.8494849	2.31676474760079e-06\\
0.849585	-1.73766342781434e-06\\
0.849685	-1.00471243014155e-06\\
0.849785	-6.89550239130199e-07\\
0.849885	-7.78252218669451e-07\\
0.849985	-2.5691627048019e-07\\
0.850085	-1.11662895818654e-07\\
0.850185	6.71364744686542e-07\\
0.850285	1.10600076297374e-06\\
0.850385	1.20605647957461e-06\\
0.850485	1.9853203458986e-06\\
0.8505851	-1.75461842166769e-06\\
0.8506851	-1.57495101449356e-06\\
0.8507851	-6.74834140568237e-07\\
0.8508851	-4.05714333240326e-08\\
0.8509851	3.41510293377922e-07\\
0.8510851	4.85061006916254e-07\\
0.8511851	4.03707358209715e-07\\
0.8512851	1.11105263433942e-06\\
0.8513851	1.62067670217425e-06\\
0.8514851	1.94613593906823e-06\\
0.8515852	-2.10346238960213e-06\\
0.8516852	-1.10490830174115e-06\\
0.8517852	-1.24997796824289e-06\\
0.8518852	-5.25209595814147e-07\\
0.8519852	8.2834825665401e-08\\
0.8520852	5.87569445098524e-07\\
0.8521852	1.00238451183898e-06\\
0.8522852	1.34064629930819e-06\\
0.8523852	1.61569705303677e-06\\
0.8524852	1.84085492671571e-06\\
0.8525853	-2.16591087776052e-06\\
0.8526853	-9.9969760336549e-07\\
0.8527853	-8.43554602614205e-07\\
0.8528853	-6.84260667682679e-07\\
0.8529853	-5.08618968275698e-07\\
0.8530853	6.96542880262996e-07\\
0.8531853	9.44372795519399e-07\\
0.8532853	1.24799412848375e-06\\
0.8533853	1.62050563368155e-06\\
0.8534853	2.0749814111376e-06\\
0.8535854	-1.56042726517924e-06\\
0.8536854	-9.01785015106071e-07\\
0.8537854	-1.12209171820865e-06\\
0.8538854	-2.08372484156172e-07\\
0.8539854	-1.47677360118337e-07\\
0.8540854	7.2918624471896e-08\\
0.8541854	4.66315395630801e-07\\
0.8542854	1.04538778966656e-06\\
0.8543854	1.82298549589177e-06\\
0.8544854	1.81193300141658e-06\\
0.8545855	-2.14814040866429e-06\\
0.8546855	-1.69687759044024e-06\\
0.8547855	-9.95930489189334e-07\\
0.8548855	-3.25759632602285e-08\\
0.8549855	2.05883673665141e-07\\
0.8550855	7.32120589930219e-07\\
0.8551855	5.5878140869936e-07\\
0.8552855	6.98487128580894e-07\\
0.8553855	1.16383309434198e-06\\
0.8554855	1.96738893221138e-06\\
0.8555856	-2.03846694013521e-06\\
0.8556856	-1.51951602767042e-06\\
0.8557856	-6.24788625724904e-07\\
0.8558856	-3.41818465088295e-07\\
0.8559856	3.41834763162296e-07\\
0.8560856	4.38585372020128e-07\\
0.8561856	9.60821618267715e-07\\
0.8562856	9.20905665502403e-07\\
0.8563856	1.33117352341117e-06\\
0.8564856	2.20393501937366e-06\\
0.8565857	-1.59443651126523e-06\\
0.8566857	-1.75837012150737e-06\\
0.8567857	-4.23026030205165e-07\\
0.8568857	-5.76199676860512e-07\\
0.8569857	-2.05712919143508e-07\\
0.8570857	7.00585917723373e-07\\
0.8571857	1.15482199092298e-06\\
0.8572857	1.16909391634223e-06\\
0.8573857	1.75547369352103e-06\\
0.8574857	1.92600668524889e-06\\
0.8575858	-1.43771883953647e-06\\
0.8576858	-1.06123597864993e-06\\
0.8577858	-1.0646121633151e-06\\
0.8578858	-4.35909071949681e-07\\
0.8579858	-1.63215236437253e-07\\
0.8580858	7.65353906029276e-07\\
0.8581858	3.61655982494824e-07\\
0.8582858	1.63752162718467e-06\\
0.8583858	1.60475445820651e-06\\
0.8584858	2.27513102446508e-06\\
0.8585859	-1.45335174916994e-06\\
0.8586859	-1.33973406546417e-06\\
0.8587859	-4.87794119763407e-07\\
0.8588859	-8.85864058997043e-07\\
0.8589859	-5.22303292171955e-07\\
0.8590859	6.14501463491735e-07\\
0.8591859	5.36136153694855e-07\\
0.8592859	1.25415932927808e-06\\
0.8593859	1.78010214435531e-06\\
0.8594859	2.12546827249582e-06\\
0.859586	-1.79416964041224e-06\\
0.859686	-1.77370778287056e-06\\
0.859786	-8.99465283143996e-07\\
0.859886	-1.60048675645896e-07\\
0.859986	-5.4409215755058e-07\\
0.860086	-4.02575976998776e-08\\
0.860186	3.62765420813815e-07\\
0.860286	6.76259563903159e-07\\
0.860386	1.91147971451144e-06\\
0.860486	2.07965293785861e-06\\
0.8605861	-1.88493245634191e-06\\
0.8606861	-8.15322449110312e-07\\
0.8607861	-7.79234312631161e-07\\
0.8608861	-7.65552634618416e-07\\
0.8609861	2.36810012044231e-07\\
0.8610861	2.38913023586917e-07\\
0.8611861	1.25178775567392e-06\\
0.8612861	1.28643745789248e-06\\
0.8613861	1.35383727917215e-06\\
0.8614861	2.46493420608118e-06\\
0.8615862	-1.42615527298773e-06\\
0.8616862	-1.19286498989624e-06\\
0.8617862	-8.83195017120642e-07\\
0.8618862	-4.8631131610577e-07\\
0.8619862	8.59185567136933e-09\\
0.8620862	6.12291890522698e-07\\
0.8621862	3.35537818862974e-07\\
0.8622862	1.1890502769063e-06\\
0.8623862	1.1835214719369e-06\\
0.8624862	2.32961515944652e-06\\
0.8625863	-1.39763928386216e-06\\
0.8626863	-9.14245139149727e-07\\
0.8627863	-1.24739768025961e-06\\
0.8628863	-3.86547279518368e-07\\
0.8629863	-3.21172907380429e-07\\
0.8630863	-4.07821483250359e-08\\
0.8631863	4.65088759060706e-07\\
0.8632863	1.20687490356985e-06\\
0.8633863	1.19498266748508e-06\\
0.8634863	2.43978972180869e-06\\
0.8635864	-2.06170506755754e-06\\
0.8636864	-1.27019897355751e-06\\
0.8637864	-1.19102295448315e-06\\
0.8638864	-8.13914541986094e-07\\
0.8639864	-1.28640113672418e-07\\
0.8640864	-1.24994929961986e-07\\
0.8641864	1.2071968473748e-06\\
0.8642864	8.78082133670688e-07\\
0.8643864	1.89777890025056e-06\\
0.8644864	2.27637614047538e-06\\
0.8645865	-1.96612956582243e-06\\
0.8646865	-8.37196372227567e-07\\
0.8647865	-1.31925977520098e-06\\
0.8648865	-4.0234690734664e-07\\
0.8649865	-7.65139906988566e-08\\
0.8650865	6.68153651073311e-07\\
0.8651865	8.41541572427218e-07\\
0.8652865	1.45350618208084e-06\\
0.8653865	1.51387472557829e-06\\
0.8654865	2.03244525476265e-06\\
0.8655866	-1.94678834741779e-06\\
0.8656866	-1.48005391542583e-06\\
0.8657866	-5.25888927516149e-07\\
0.8658866	-7.46122883477085e-08\\
0.8659866	-1.16572191899422e-07\\
0.8660866	3.5785388607934e-07\\
0.8661866	3.58259134358718e-07\\
0.8662866	8.94207407675651e-07\\
0.8663866	1.97523321698867e-06\\
0.8664866	1.61084169203463e-06\\
0.8665867	-1.13000528889629e-06\\
0.8666867	-1.35425541980538e-06\\
0.8667867	-9.95574475304295e-07\\
0.8668867	-4.45749948063678e-08\\
0.8669867	-4.91898978438599e-07\\
0.8670867	6.71782104610941e-07\\
0.8671867	4.55767308338295e-07\\
0.8672867	8.69326183039476e-07\\
0.8673867	1.92169876545023e-06\\
0.8674867	1.62209556098247e-06\\
0.8675868	-1.93461186137611e-06\\
0.8676868	-9.07982348419267e-07\\
0.8677868	-1.20586528895572e-06\\
0.8678868	-8.19168450405527e-07\\
0.8679868	2.61170800097688e-07\\
0.8680868	4.41854850485868e-08\\
0.8681868	5.38879021405592e-07\\
0.8682868	7.54225175203516e-07\\
0.8683868	1.69916807024606e-06\\
0.8684868	2.38262217866492e-06\\
0.8685869	-2.07371881666418e-06\\
0.8686869	-8.83856875688593e-07\\
0.8687869	-9.28909554165358e-07\\
0.8688869	-2.00081143653108e-07\\
0.8689869	3.11394346752536e-07\\
0.8690869	6.14253187869451e-07\\
0.8691869	7.17201924516075e-07\\
0.8692869	6.28917357303749e-07\\
0.8693869	1.3580465383356e-06\\
0.8694869	1.91320676767859e-06\\
0.869587	-1.55620306818705e-06\\
0.869687	-1.32040048539395e-06\\
0.869787	-1.2328850838017e-06\\
0.869887	-2.85158715307432e-07\\
0.869987	-4.68753021287682e-07\\
0.870087	2.24770540402375e-07\\
0.870187	8.03820720207682e-07\\
0.870287	1.27677643169122e-06\\
0.870387	1.6519867687137e-06\\
0.870487	1.93777099566361e-06\\
0.8705871	-1.68791315813266e-06\\
0.8706871	-1.55321146966259e-06\\
0.8707871	-4.83148926200272e-07\\
0.8708871	-4.6952567522851e-07\\
0.8709871	-5.04171722148783e-07\\
0.8710871	4.21053067078958e-07\\
0.8711871	3.14258962408331e-07\\
0.8712871	1.18352636002683e-06\\
0.8713871	1.03690578212579e-06\\
0.8714871	1.88241787935084e-06\\
0.8715872	-1.07259673054472e-06\\
0.8716872	-1.21586481505176e-06\\
0.8717872	-3.43109664235897e-07\\
0.8718872	-4.46430201250791e-07\\
0.8719872	-5.1795523914322e-07\\
0.8720872	4.50156521392131e-07\\
0.8721872	4.65716475162026e-07\\
0.8722872	1.53650613787804e-06\\
0.8723872	1.67027711617251e-06\\
0.8724872	1.87475112234381e-06\\
0.8725873	-1.61255385577874e-06\\
0.8726873	-1.24053836136895e-06\\
0.8727873	-7.7482646254623e-07\\
0.8728873	-2.07816066133404e-07\\
0.8729873	-5.31934978731385e-07\\
0.8730873	2.60359100501262e-07\\
0.8731873	1.17657857767739e-06\\
0.8732873	1.22420596504469e-06\\
0.8733873	1.41069388215342e-06\\
0.8734873	1.74346506032208e-06\\
0.8735874	-1.50902034157241e-06\\
0.8736874	-8.58369293865735e-07\\
0.8737874	-1.0393388008012e-06\\
0.8738874	-4.46256969155456e-08\\
0.8739874	1.3304329815611e-07\\
0.8740874	5.00911597089981e-07\\
0.8741874	1.0661927412059e-06\\
0.8742874	8.36070399135025e-07\\
0.8743874	1.81769839224799e-06\\
0.8744874	2.01820066791925e-06\\
0.8745875	-1.26228523633642e-06\\
0.8746875	-1.59954537748419e-06\\
0.8747875	-6.96731456972799e-07\\
0.8748875	-5.468389234764e-07\\
0.8749875	-1.42893069376981e-07\\
0.8750875	5.22050983775557e-07\\
0.8751875	4.54908301250612e-07\\
0.8752875	6.62564132447585e-07\\
0.8753875	1.15187392180083e-06\\
0.8754875	1.92966330869737e-06\\
0.8755876	-1.67154724661778e-06\\
0.8756876	-1.29313548091758e-06\\
0.8757876	-6.05939195486016e-07\\
0.8758876	-6.03251906206026e-07\\
0.8759876	-2.78396887321009e-07\\
0.8760876	3.75272829700179e-07\\
0.8761876	3.64374471262252e-07\\
0.8762876	6.95495522062828e-07\\
0.8763876	1.37519375198458e-06\\
0.8764876	1.40999720121826e-06\\
0.8765877	-1.83451473989571e-06\\
0.8766877	-1.06666424137813e-06\\
0.8767877	-9.24296175153216e-07\\
0.8768877	-4.01001301719717e-07\\
0.8769877	-4.90400041019967e-07\\
0.8770877	8.13857521819727e-07\\
0.8771877	5.18091660905995e-07\\
0.8772877	6.28593014528889e-07\\
0.8773877	1.1516225983943e-06\\
0.8774877	2.09341181411915e-06\\
0.8775878	-1.14675452289603e-06\\
0.8776878	-1.34543599306092e-06\\
0.8777878	-1.10683503118381e-06\\
0.8778878	-4.24838585644238e-07\\
0.8779878	-2.93363151770976e-07\\
0.8780878	2.93645241233875e-07\\
0.8781878	3.42211037063223e-07\\
0.8782878	8.58329176400829e-07\\
0.8783878	1.84796509627994e-06\\
0.8784878	1.3170547703556e-06\\
0.8785879	-1.30079447258069e-06\\
0.8786879	-8.51613096930492e-07\\
0.8787879	-9.05340993062609e-07\\
0.8788879	-4.56160008655004e-07\\
0.8789879	-4.98281383931598e-07\\
0.8790879	-2.59457686446751e-08\\
0.8791879	9.66576820360388e-07\\
0.8792879	1.48498700580646e-06\\
0.8793879	1.53495606669551e-06\\
0.8794879	2.12212595362971e-06\\
0.879588	-1.28498566609458e-06\\
0.879688	-6.03054316483664e-07\\
0.879788	-3.67166542281083e-07\\
0.879888	-5.71797563742038e-07\\
0.879988	-2.11451840925747e-07\\
0.880088	7.19336940679938e-07\\
0.880188	2.26005891335035e-07\\
0.880288	1.31396292801789e-06\\
0.880388	9.88586805084068e-07\\
0.880488	1.25522711269532e-06\\
0.8805881	-1.38212945488547e-06\\
0.8806881	-9.11918716095528e-07\\
0.8807881	-8.33812213496543e-07\\
0.8808881	-1.42576794148752e-07\\
0.8809881	1.66991638739233e-07\\
0.8810881	1.00068156427824e-07\\
0.8811881	6.61798804868496e-07\\
0.8812881	8.57300637591152e-07\\
0.8813881	1.6916617360252e-06\\
0.8814881	2.16994122648195e-06\\
0.8815882	-1.16787550119568e-06\\
0.8816882	-1.3830411143001e-06\\
0.8817882	-9.39279508616053e-07\\
0.8818882	-8.31647189114548e-07\\
0.8819882	-5.52295071898357e-08\\
0.8820882	3.94859360319799e-07\\
0.8821882	5.23476436864456e-07\\
0.8822882	1.33544996239188e-06\\
0.8823882	8.35579420099819e-07\\
0.8824882	2.02863556242328e-06\\
0.8825883	-1.5088965912291e-06\\
0.8826883	-9.12084924031831e-07\\
0.8827883	-6.0820182845589e-07\\
0.8828883	-5.92591289594324e-07\\
0.8829883	1.39374089380695e-07\\
0.8830883	5.92293112333664e-07\\
0.8831883	7.70736021227236e-07\\
0.8832883	6.79244506418542e-07\\
0.8833883	1.32233173921747e-06\\
0.8834883	1.70448239766863e-06\\
0.8835884	-1.56084653135657e-06\\
0.8836884	-6.83478690310935e-07\\
0.8837884	-1.05375972370148e-06\\
0.8838884	-6.67318709623999e-07\\
0.8839884	-5.19813091015564e-07\\
0.8840884	3.93071344140239e-07\\
0.8841884	1.0756204942286e-06\\
0.8842884	5.32091968930004e-07\\
0.8843884	1.76671511620263e-06\\
0.8844884	1.78369104908427e-06\\
0.8845885	-1.7661072877484e-06\\
0.8846885	-1.16814251716058e-06\\
0.8847885	-7.75386661766575e-07\\
0.8848885	-5.83751304406022e-07\\
0.8849885	4.10823868701726e-07\\
0.8850885	2.1237110958694e-07\\
0.8851885	8.24894618833838e-07\\
0.8852885	1.25237059522831e-06\\
0.8853885	1.49874724120203e-06\\
0.8854885	1.5679448088779e-06\\
0.8855886	-1.85133190400144e-06\\
0.8856886	-1.12101092319961e-06\\
0.8857886	-5.56271890328475e-07\\
0.8858886	-1.53306120465047e-07\\
0.8859886	9.1667265333939e-08\\
0.8860886	1.82401365789175e-07\\
0.8861886	1.22621531595257e-07\\
0.8862886	9.16025390651498e-07\\
0.8863886	1.56628288361382e-06\\
0.8864886	2.07703629206968e-06\\
0.8865887	-1.82478968269173e-06\\
0.8866887	-5.78358596570183e-07\\
0.8867887	-4.60666849111391e-07\\
0.8868887	-4.6818251298042e-07\\
0.8869887	4.02598836934942e-07\\
0.8870887	1.55154155967452e-07\\
0.8871887	7.92932953963543e-07\\
0.8872887	1.31935734160038e-06\\
0.8873887	1.73782204759831e-06\\
0.8874887	2.05169446054754e-06\\
0.8875888	-9.73520292824048e-07\\
0.8876888	-8.5493580948004e-07\\
0.8877888	-8.31001924161967e-07\\
0.8878888	1.01539684882468e-07\\
0.8879888	-5.40798528181341e-08\\
0.8880888	7.05343453422103e-07\\
0.8881888	3.82986480529368e-07\\
0.8882888	9.81999026539171e-07\\
0.8883888	1.50550384692139e-06\\
0.8884888	1.95659668444925e-06\\
0.8885889	-8.60303633221093e-07\\
0.8886889	-5.40920189529004e-07\\
0.8887889	-2.84820213192205e-07\\
0.8888889	-8.9015688597982e-08\\
0.8889889	4.94545648876965e-08\\
0.8890889	1.33524926670958e-07\\
0.8891889	1.66103005128093e-07\\
0.8892889	1.15006967571496e-06\\
0.8893889	1.08827912814746e-06\\
0.8894889	1.98355887648383e-06\\
0.889589	-1.32045218803611e-06\\
0.889689	-4.9869172769057e-07\\
0.889789	-7.11535288244747e-07\\
0.889889	4.37383456119989e-08\\
0.889989	-2.30176102844482e-07\\
0.890089	4.6938965247989e-07\\
0.890189	1.45077488467038e-07\\
0.890289	7.9950290832187e-07\\
0.890389	1.4352550872232e-06\\
0.890489	1.05489690294114e-06\\
0.8905891	-1.45843291399217e-06\\
0.8906891	-8.59438006317603e-07\\
0.8907891	-2.69019761800493e-07\\
0.8908891	-6.84720138588446e-07\\
0.8909891	-1.04107201748249e-07\\
0.8910891	4.75224916085537e-07\\
0.8911891	1.05565604346936e-06\\
0.8912891	6.39540025382246e-07\\
0.8913891	1.22920474110444e-06\\
0.8914891	1.82695217132434e-06\\
0.8915892	-1.64432555083138e-06\\
0.8916892	-1.01959653409089e-06\\
0.8917892	-3.80031788926871e-07\\
0.8918892	-7.23432649119005e-07\\
0.8919892	-4.76261541315637e-08\\
0.8920892	6.49534959595854e-07\\
0.8921892	3.70172325903084e-07\\
0.8922892	1.11638197419772e-06\\
0.8923892	8.90234371109244e-07\\
0.8924892	1.69377446268548e-06\\
0.8925893	-1.51012449922661e-06\\
0.8926893	-6.37141364556726e-07\\
0.8927893	-7.28486458267952e-07\\
0.8928893	-7.82216540873293e-07\\
0.8929893	2.03586314251325e-07\\
0.8930893	2.30814750423747e-07\\
0.8931893	3.01336178898381e-07\\
0.8932893	4.16992812946049e-07\\
0.8933893	1.57960170987792e-06\\
0.8934893	1.79095482311098e-06\\
0.8935894	-9.45891241110886e-07\\
0.8936894	-6.27720981638547e-07\\
0.8937894	-2.55579218455537e-07\\
0.8938894	-8.27774062184261e-07\\
0.8939894	-3.4263852821681e-07\\
0.8940894	2.01469514848895e-07\\
0.8941894	8.06167368860855e-07\\
0.8942894	4.73047566451612e-07\\
0.8943894	1.20367790334974e-06\\
0.8944894	9.99601486695312e-07\\
0.8945895	-1.09576455464833e-06\\
0.8946895	-1.16065466881921e-06\\
0.8947895	-1.5576848011456e-07\\
0.8948895	-7.96612418252352e-08\\
0.8949895	6.90873167563666e-08\\
0.8950895	2.91873034408141e-07\\
0.8951895	5.90067363193114e-07\\
0.8952895	9.65017405762936e-07\\
0.8953895	1.41804596887063e-06\\
0.8954895	9.50451595649326e-07\\
0.8955896	-1.35383553057089e-06\\
0.8956896	-6.54794415799742e-07\\
0.8957896	-8.72624481118578e-07\\
0.8958896	-6.12379236208938e-09\\
0.8959896	-5.41144531407411e-08\\
0.8960896	-1.54425483511744e-08\\
0.8961896	1.11021880133677e-07\\
0.8962896	1.3263848912537e-06\\
0.8963896	6.31728689137745e-07\\
0.8964896	1.02811166313721e-06\\
0.8965897	-1.35989468508235e-06\\
0.8966897	-7.74259625524465e-07\\
0.8967897	-1.09455166064087e-06\\
0.8968897	-3.19807219817392e-07\\
0.8969897	-4.49086315246205e-07\\
0.8970897	5.18527500048549e-07\\
0.8971897	5.83927180830557e-07\\
0.8972897	7.47982232152111e-07\\
0.8973897	1.01153875875504e-06\\
0.8974897	1.37541950895326e-06\\
0.8975898	-9.95058201169741e-07\\
0.8976898	-4.24051668668213e-07\\
0.8977898	-7.50391596415767e-07\\
0.8978898	2.66517798941379e-08\\
0.8979898	-9.22148961635116e-08\\
0.8980898	-1.06308064573568e-07\\
0.8981898	9.85032816425147e-07\\
0.8982898	1.1824452947895e-06\\
0.8983898	1.48654398435255e-06\\
0.8984898	8.97920611420133e-07\\
0.8985899	-1.37728055737973e-06\\
0.8986899	-7.45555541525889e-07\\
0.8987899	-1.00491376442413e-06\\
0.8988899	-1.54854607803401e-07\\
0.8989899	-1.94900106276208e-07\\
0.8990899	-1.24594898842645e-07\\
0.8991899	5.64938171798701e-08\\
0.8992899	3.48776334702094e-07\\
0.8993899	7.52640488599354e-07\\
0.8994899	1.26845169523548e-06\\
0.89959	-8.56760580481009e-07\\
0.89969	-1.1119356604361e-06\\
0.89979	-2.54201141247634e-07\\
0.89989	-2.83280802371166e-07\\
0.89999	-1.98920590444018e-07\\
0.90009	-8.88577211810571e-10\\
0.90019	3.11025085686367e-07\\
0.90029	7.37008231554981e-07\\
0.90039	1.27722670506536e-06\\
0.90049	9.31824425398631e-07\\
0.9005901	-1.01124812412756e-06\\
0.9006901	-1.12343298286355e-06\\
0.9007901	-1.20937850622838e-07\\
0.9008901	-3.70605901522936e-09\\
0.9009901	2.28297379001319e-07\\
0.9010901	5.75085817855481e-07\\
0.9011901	3.6651023993528e-08\\
0.9012901	6.12963232171637e-07\\
0.9013901	1.30397118724979e-06\\
0.9014901	1.10960220212064e-06\\
0.9015902	-6.41258359523889e-07\\
0.9016902	-6.02570445762396e-07\\
0.9017902	-4.49605811914111e-07\\
0.9018902	-1.82522417180309e-07\\
0.9019902	1.98500593073803e-07\\
0.9020902	6.93262934170491e-07\\
0.9021902	3.01543233272028e-07\\
0.9022902	1.02309908811549e-06\\
0.9023902	8.57667099563031e-07\\
0.9024902	8.0496293808352e-07\\
0.9025903	-7.65200734109683e-07\\
0.9026903	-5.89273804862955e-07\\
0.9027903	-3.01597518070906e-07\\
0.9028903	9.74605387504823e-08\\
0.9029903	-3.92487916389683e-07\\
0.9030903	2.28148208325507e-07\\
0.9031903	9.58939415651727e-07\\
0.9032903	7.99435684095684e-07\\
0.9033903	7.49166505942256e-07\\
0.9034903	8.07640944344712e-07\\
0.9035904	-6.14429614076428e-07\\
0.9036904	-3.35914784432845e-07\\
0.9037904	-9.50251776155042e-07\\
0.9038904	-4.58012755277792e-07\\
0.9039904	1.4020993988062e-07\\
0.9040904	-1.56196154010857e-07\\
0.9041904	6.52136428413996e-07\\
0.9042904	5.64555131310485e-07\\
0.9043904	5.80387434556684e-07\\
0.9044904	1.69894089707689e-06\\
0.9045905	-6.28223413068341e-07\\
0.9046905	-3.02283465147823e-07\\
0.9047905	-8.75819434531522e-07\\
0.9048905	-3.49602927496306e-07\\
0.9049905	2.75574777885979e-07\\
0.9050905	-1.09719722196644e-09\\
0.9051905	8.19550711739225e-07\\
0.9052905	7.36668566370113e-07\\
0.9053905	7.49386976472977e-07\\
0.9054905	8.56817152339318e-07\\
0.9055906	-1.44869916884893e-06\\
0.9056906	-1.15049676185919e-06\\
0.9057906	-7.60365789664874e-07\\
0.9058906	-2.7927213830381e-07\\
0.9059906	2.91799167717599e-07\\
0.9060906	-4.81559816467581e-08\\
0.9061906	6.99839268758495e-07\\
0.9062906	5.34742790714304e-07\\
0.9063906	1.4554935274802e-06\\
0.9064906	1.46101155085177e-06\\
0.9065907	-9.15669207346426e-07\\
0.9066907	-7.39849675079896e-07\\
0.9067907	-4.82616538732827e-07\\
0.9068907	-1.45124736672386e-07\\
0.9069907	2.7145218828295e-07\\
0.9070907	-2.34077874639027e-07\\
0.9071907	3.37074455902808e-07\\
0.9072907	9.83680104482687e-07\\
0.9073907	7.04491595726608e-07\\
0.9074907	1.49824310047286e-06\\
0.9075908	-1.06144666944807e-06\\
0.9076908	-1.12161606224959e-06\\
0.9077908	-1.12753826941869e-07\\
0.9078908	-3.6198683606159e-08\\
0.9079908	6.69256416951924e-09\\
0.9080908	2.14545079479933e-07\\
0.9081908	4.85966044960762e-07\\
0.9082908	7.19544717253662e-07\\
0.9083908	9.13852475470378e-07\\
0.9084908	1.16744288902737e-06\\
0.9085909	-1.00560639582081e-06\\
0.9086909	-7.33805419317335e-07\\
0.9087909	-5.07169003571839e-07\\
0.9088909	-2.27214354109329e-07\\
0.9089909	4.52376658355291e-09\\
0.9090909	1.86493102849461e-07\\
0.9091909	4.17123942608733e-07\\
0.9092909	6.94829183411905e-07\\
0.9093909	9.18004380179305e-07\\
0.9094909	1.18502779999918e-06\\
0.909591	-9.49707717801118e-07\\
0.909691	-6.95882214074572e-07\\
0.909791	-5.03178559485562e-07\\
0.909891	-2.73287120378107e-07\\
0.909991	-7.91528442789513e-09\\
0.910091	1.91212593203538e-07\\
0.910191	4.22355245555117e-07\\
0.910291	6.83754542762927e-07\\
0.910391	8.73635550013852e-07\\
0.910491	1.09020657690362e-06\\
0.9105911	-9.71985460562763e-07\\
0.9106911	-7.03454058781361e-07\\
0.9107911	-5.13709532135032e-07\\
0.9108911	-2.04610062781718e-07\\
0.9109911	2.19696805192626e-08\\
0.9110911	1.64138606040254e-07\\
0.9111911	4.19989234501372e-07\\
0.9112911	6.87597766835246e-07\\
0.9113911	8.65024131768699e-07\\
0.9114911	1.15031204284133e-06\\
0.9115912	-9.22015489956962e-07\\
0.9116912	-7.22935046826301e-07\\
0.9117912	-4.21960695828894e-07\\
0.9118912	-2.21113070963952e-07\\
0.9119912	-2.24287530770084e-08\\
0.9120912	1.72039780532707e-07\\
0.9121912	4.60224212961435e-07\\
0.9122912	6.40040439980893e-07\\
0.9123912	9.09388624736884e-07\\
0.9124912	1.0661532480194e-06\\
0.9125913	-9.15360862774328e-07\\
0.9126913	-6.86190303245127e-07\\
0.9127913	-4.76045566744077e-07\\
0.9128913	-2.87104366281099e-07\\
0.9129913	-2.15598284780683e-08\\
0.9130913	2.18379563543536e-07\\
0.9131913	4.30490029068586e-07\\
0.9132913	6.12532530697152e-07\\
0.9133913	8.62252837574573e-07\\
0.9134913	1.07738157706372e-06\\
0.9135914	-9.28204683781075e-07\\
0.9136914	-6.85167829361788e-07\\
0.9137914	-4.83623298408453e-07\\
0.9138914	-2.2590051051985e-07\\
0.9139914	-1.43437571820115e-08\\
0.9140914	1.48687849588214e-07\\
0.9141914	4.60820431924525e-07\\
0.9142914	6.19665398771119e-07\\
0.9143914	8.22819502643846e-07\\
0.9144914	1.06786488895949e-06\\
0.9145915	-8.91975458716665e-07\\
0.9146915	-6.66523435111088e-07\\
0.9147915	-4.06523249396606e-07\\
0.9148915	-2.14450652569198e-07\\
0.9149915	7.20426884814884e-09\\
0.9150915	1.55937145973617e-07\\
0.9151915	4.29229376153195e-07\\
0.9152915	6.24548190719665e-07\\
0.9153915	8.39346691483556e-07\\
0.9154915	1.07106391411538e-06\\
0.9155916	-8.8797079400571e-07\\
0.9156916	-6.26244575530066e-07\\
0.9157916	-4.55369011276296e-07\\
0.9158916	-1.77960818636969e-07\\
0.9159916	3.34948624391984e-09\\
0.9160916	1.85917639838884e-07\\
0.9161916	3.670856847382e-07\\
0.9162916	6.44182023834539e-07\\
0.9163916	8.14521474712393e-07\\
0.9164916	9.75405320624567e-07\\
0.9165917	-8.41984969079945e-07\\
0.9166917	-6.04278610816245e-07\\
0.9167917	-3.84207417547522e-07\\
0.9168917	-1.84523717527973e-07\\
0.9169917	-7.99310539978393e-09\\
0.9170917	1.42605612318825e-07\\
0.9171917	3.64480465697703e-07\\
0.9172917	5.54826386722773e-07\\
0.9173917	8.10825248387381e-07\\
0.9174917	1.0296459229675e-06\\
0.9175918	-8.189458955421e-07\\
0.9176918	-5.79171016035218e-07\\
0.9177918	-3.85148045456418e-07\\
0.9178918	-2.39759584297339e-07\\
0.9179918	-4.59009739106619e-08\\
0.9180918	1.93519765212535e-07\\
0.9181918	3.75581985290196e-07\\
0.9182918	5.97352462716749e-07\\
0.9183918	7.55885454895022e-07\\
0.9184918	9.48222746544047e-07\\
0.9185919	-8.1756678271816e-07\\
0.9186919	-5.62718785679373e-07\\
0.9187919	-3.83018411542757e-07\\
0.9188919	-1.81473231886109e-07\\
0.9189919	-6.11030220865061e-08\\
0.9190919	1.75060292040286e-07\\
0.9191919	4.23972684848728e-07\\
0.9192919	5.82578083219154e-07\\
0.9193919	7.47808426182495e-07\\
0.9194919	9.16583707044616e-07\\
0.919592	-7.65017645232291e-07\\
0.919692	-5.94644209650497e-07\\
0.919792	-4.30040090759753e-07\\
0.919892	-1.74332537183908e-07\\
0.919992	-3.06604819311218e-08\\
0.920092	1.97825520298345e-07\\
0.920192	4.07963341553241e-07\\
0.920292	5.96579332601266e-07\\
0.920392	7.60488378270807e-07\\
0.920492	8.96493945301557e-07\\
0.9205921	-8.11621712948352e-07\\
0.9206921	-6.37294043870362e-07\\
0.9207921	-4.00530656974318e-07\\
0.9208921	-2.04573234463545e-07\\
0.9209921	-5.26746141149559e-08\\
0.9210921	1.51901263678589e-07\\
0.9211921	4.05879407416876e-07\\
0.9212921	5.05973829501016e-07\\
0.9213921	7.48887591606184e-07\\
0.9214921	9.31312868490686e-07\\
0.9215922	-7.25581542404541e-07\\
0.9216922	-5.70368856145365e-07\\
0.9217922	-3.85636342303819e-07\\
0.9218922	-1.74734901547957e-07\\
0.9219922	-4.10260672101259e-08\\
0.9220922	2.12118040354703e-07\\
0.9221922	3.81314768760355e-07\\
0.9222922	5.63170986289663e-07\\
0.9223922	7.54283131620959e-07\\
0.9224922	9.51237265178762e-07\\
0.9225923	-7.87739592111869e-07\\
0.9226923	-5.85687290333148e-07\\
0.9227923	-3.88100018566817e-07\\
0.9228923	-1.98432720477371e-07\\
0.9229923	-2.01504642971528e-08\\
0.9230923	1.43271615238305e-07\\
0.9231923	3.88348363244972e-07\\
0.9232923	5.11584658080544e-07\\
0.9233923	7.0947546060296e-07\\
0.9234923	8.7850587071614e-07\\
0.9235924	-6.8637772221436e-07\\
0.9236924	-5.8198970875889e-07\\
0.9237924	-4.17069030089756e-07\\
0.9238924	-1.9516955285237e-07\\
0.9239924	-1.9854752397741e-08\\
0.9240924	2.05302336464541e-07\\
0.9241924	3.76719166750306e-07\\
0.9242924	4.90803734431999e-07\\
0.9243924	7.43954625348486e-07\\
0.9244924	8.32561067420912e-07\\
0.9245925	-7.12060054031127e-07\\
0.9246925	-5.59785616860431e-07\\
0.9247925	-3.82947229038422e-07\\
0.9248925	-1.85192600143935e-07\\
0.9249925	2.9821453528811e-08\\
0.9250925	1.58429061980492e-07\\
0.9251925	2.96955346645866e-07\\
0.9252925	5.41716473456688e-07\\
0.9253925	6.89019703181692e-07\\
0.9254925	8.35163435963793e-07\\
0.9255926	-7.52523328206856e-07\\
0.9256926	-5.16248997883828e-07\\
0.9257926	-3.92295320450486e-07\\
0.9258926	-1.84398828984023e-07\\
0.9259926	3.69533714739134e-09\\
0.9260926	1.68233484365565e-07\\
0.9261926	3.05453405680112e-07\\
0.9262926	5.11584437590251e-07\\
0.9263926	6.82847504984707e-07\\
0.9264926	8.15455175717794e-07\\
0.9265927	-6.87618856187733e-07\\
0.9266927	-5.40165637441525e-07\\
0.9267927	-3.42783601947616e-07\\
0.9268927	-1.99293141323054e-07\\
0.9269927	-1.3522758912643e-08\\
0.9270927	1.10690973387229e-07\\
0.9271927	3.69503469155319e-07\\
0.9272927	4.5906217338576e-07\\
0.9273927	6.75506612757282e-07\\
0.9274927	8.14968442841568e-07\\
0.9275928	-6.84309866216815e-07\\
0.9276928	-5.06936215005638e-07\\
0.9277928	-3.18200837012839e-07\\
0.9278928	-2.22003079364264e-07\\
0.9279928	-2.2249917686068e-08\\
0.9280928	1.77144091395576e-07\\
0.9281928	2.72256866384879e-07\\
0.9282928	4.59158835730555e-07\\
0.9283928	6.33912995762209e-07\\
0.9284928	7.92574954888714e-07\\
0.9285929	-6.91728001656067e-07\\
0.9286929	-4.73638910536422e-07\\
0.9287929	-3.83523004021669e-07\\
0.9288929	-1.25353750379809e-07\\
0.9289929	-3.11176395761947e-09\\
0.9290929	1.79215240758701e-07\\
0.9291929	3.17632495749898e-07\\
0.9292929	5.08138228383359e-07\\
0.9293929	6.4672370836405e-07\\
0.9294929	8.29373294930491e-07\\
0.929593	-6.36292429967433e-07\\
0.929693	-4.74154965179885e-07\\
0.929793	-2.80045347245128e-07\\
0.929893	-1.5800639108221e-07\\
0.929993	-1.20875912612561e-08\\
0.930093	1.53654933776615e-07\\
0.930193	3.35158480968545e-07\\
0.930293	4.28353817860661e-07\\
0.930393	6.29165220922268e-07\\
0.930493	7.33510524764114e-07\\
0.9305931	-6.16894980209892e-07\\
0.9306931	-5.14360598891095e-07\\
0.9307931	-3.20581091672878e-07\\
0.9308931	-1.39663929421552e-07\\
0.9309931	2.42772091318955e-08\\
0.9310931	1.67122481742332e-07\\
0.9311931	2.84745930501096e-07\\
0.9312931	4.73015526969434e-07\\
0.9313931	6.27793216301242e-07\\
0.9314931	7.44934965712929e-07\\
0.9315932	-6.00154437035982e-07\\
0.9316932	-4.67388377556333e-07\\
0.9317932	-2.84729938737627e-07\\
0.9318932	-1.56346618096492e-07\\
0.9319932	1.3588335900927e-08\\
0.9320932	1.20895968569457e-07\\
0.9321932	2.61391664091448e-07\\
0.9322932	4.30885195856767e-07\\
0.9323932	6.25180763380584e-07\\
0.9324932	7.40077042982179e-07\\
0.9325933	-6.15743028453686e-07\\
0.9326933	-4.68961099064913e-07\\
0.9327933	-3.14219349650813e-07\\
0.9328933	-1.55740756513367e-07\\
0.9329933	2.24639895574086e-09\\
0.9330933	1.55508582566632e-07\\
0.9331933	2.9980704630006e-07\\
0.9332933	4.30897876713132e-07\\
0.9333933	5.44532037571877e-07\\
0.9334933	7.36455414318549e-07\\
0.9335934	-5.51787961211581e-07\\
0.9336934	-4.12800960858384e-07\\
0.9337934	-3.08321389974608e-07\\
0.9338934	-1.4262324044001e-07\\
0.9339934	-1.99853622540047e-08\\
0.9340934	1.55308578997371e-07\\
0.9341934	2.7897014792666e-07\\
0.9342934	4.4670617987963e-07\\
0.9343934	5.5421883227158e-07\\
0.9344934	6.97205616573271e-07\\
0.9345935	-5.50350636507879e-07\\
0.9346935	-4.46116696750654e-07\\
0.9347935	-3.19347813360071e-07\\
0.9348935	-1.7436460242326e-07\\
0.9349935	-1.54921084849491e-08\\
0.9350935	1.52940242692523e-07\\
0.9351935	2.26598689034851e-07\\
0.9352935	4.01145170692008e-07\\
0.9353935	5.72237375595641e-07\\
0.9354935	7.35528781992123e-07\\
0.9355936	-6.02986097053559e-07\\
0.9356936	-4.65171172581158e-07\\
0.9357936	-2.48225814181158e-07\\
0.9358936	-1.56512964866184e-07\\
0.9359936	5.6004321735692e-09\\
0.9360936	1.33743476904424e-07\\
0.9361936	2.23541354871237e-07\\
0.9362936	3.70615379385875e-07\\
0.9363936	5.70583030046201e-07\\
0.9364936	7.19058003029183e-07\\
0.9365937	-5.46385001409799e-07\\
0.9366937	-4.1093173608342e-07\\
0.9367937	-2.40156792064017e-07\\
0.9368937	-1.3846122409733e-07\\
0.9369937	-1.02496631182447e-08\\
0.9370937	1.40069712162294e-07\\
0.9371937	3.08085226607968e-07\\
0.9372937	3.89381746901396e-07\\
0.9373937	4.79540719799942e-07\\
0.9374937	6.74140217093999e-07\\
0.9375938	-5.58100333669653e-07\\
0.9376938	-3.64805510633115e-07\\
0.9377938	-2.80360238691379e-07\\
0.9378938	-1.09199551456829e-07\\
0.9379938	-5.57616571805752e-08\\
0.9380938	7.55121034590189e-08\\
0.9381938	2.80177294309514e-07\\
0.9382938	3.53786426821046e-07\\
0.9383938	4.91888999143431e-07\\
0.9384938	6.90031533523427e-07\\
0.9385939	-5.5236081858423e-07\\
0.9386939	-4.44461586923683e-07\\
0.9387939	-2.8990472711321e-07\\
0.9388939	-9.3155214264673e-08\\
0.9389939	4.13191988402062e-08\\
0.9390939	1.09048028495984e-07\\
0.9391939	2.05558088883784e-07\\
0.9392939	3.2637353952758e-07\\
0.9393939	4.67015920546654e-07\\
0.9394939	6.23004193123222e-07\\
0.939594	-4.7597293278967e-07\\
0.939694	-3.9974198662307e-07\\
0.939794	-2.21627842522665e-07\\
0.939894	-1.46121471589211e-07\\
0.939994	2.22837701069345e-08\\
0.940094	7.90921786397725e-08\\
0.940194	2.19805742052159e-07\\
0.940294	3.39924176862016e-07\\
0.940394	5.34944967434114e-07\\
0.940494	6.00363402158965e-07\\
0.9405941	-5.04313149074509e-07\\
0.9406941	-4.08663015094568e-07\\
0.9407941	-2.56146664434631e-07\\
0.9408941	-1.51277209203826e-07\\
0.9409941	1.43023420884347e-09\\
0.9410941	9.74585834812025e-08\\
0.9411941	2.32288822121784e-07\\
0.9412941	4.01400042426836e-07\\
0.9413941	5.00269480507853e-07\\
0.9414941	6.24372553936769e-07\\
0.9415942	-5.37411991297088e-07\\
0.9416942	-3.7350857584606e-07\\
0.9417942	-1.97960348613524e-07\\
0.9418942	-1.15298800007491e-07\\
0.9419942	-3.0057055511179e-08\\
0.9420942	1.5323016183677e-07\\
0.9421942	2.3002656618365e-07\\
0.9422942	3.95794347907152e-07\\
0.9423942	4.45994207565192e-07\\
0.9424942	5.76085393388936e-07\\
0.9425943	-4.96131228933105e-07\\
0.9426943	-3.17016747741761e-07\\
0.9427943	-2.71646099392342e-07\\
0.9428943	-1.64565488791624e-07\\
0.9429943	-3.2239455549643e-10\\
0.9430943	1.16534466909624e-07\\
0.9431943	1.81455178938084e-07\\
0.9432943	3.89888655139004e-07\\
0.9433943	4.37282678289463e-07\\
0.9434943	6.19083935837761e-07\\
0.9435944	-4.18435473270584e-07\\
0.9436944	-3.78660881150683e-07\\
0.9437944	-2.18149779185328e-07\\
0.9438944	-1.41459518043785e-07\\
0.9439944	-5.31483686505574e-08\\
0.9440944	1.4222451033552e-07\\
0.9441944	2.40099107151615e-07\\
0.9442944	3.35914592530795e-07\\
0.9443944	4.25109352775177e-07\\
0.9444944	6.03121024955744e-07\\
0.9445945	-4.55759208928264e-07\\
0.9446945	-3.11026224719679e-07\\
0.9447945	-1.91172139341056e-07\\
0.9448945	-1.00761975374564e-07\\
0.9449945	-4.43613346612359e-08\\
0.9450945	7.34636351196194e-08\\
0.9451945	2.48146271086824e-07\\
0.9452945	2.75119436010662e-07\\
0.9453945	4.49815541925247e-07\\
0.9454945	5.67666590567395e-07\\
0.9455946	-4.69470220754165e-07\\
0.9456946	-3.76283038339142e-07\\
0.9457946	-2.53651623260964e-07\\
0.9458946	-1.06145295108995e-07\\
0.9459946	-3.83336222209429e-08\\
0.9460946	1.45213614632933e-07\\
0.9461946	2.39926449208117e-07\\
0.9462946	3.41234767053944e-07\\
0.9463946	4.44568335422613e-07\\
0.9464946	5.4535683524648e-07\\
0.9465947	-4.27430168814524e-07\\
0.9466947	-3.42756914684106e-07\\
0.9467947	-1.74344446268737e-07\\
0.9468947	-1.26763102459293e-07\\
0.9469947	-4.58315019269406e-09\\
0.9470947	8.76252528869692e-08\\
0.9471947	2.45292084477544e-07\\
0.9472947	2.63847491355218e-07\\
0.9473947	4.38721822926258e-07\\
0.9474947	5.65345659064853e-07\\
0.9475948	-4.00652965526493e-07\\
0.9476948	-2.81596974349707e-07\\
0.9477948	-2.24502607504284e-07\\
0.9478948	-1.33938044322157e-07\\
0.9479948	-1.44710810001669e-08\\
0.9480948	1.29330904929947e-07\\
0.9481948	1.9290098385305e-07\\
0.9482948	2.71672704388326e-07\\
0.9483948	3.61080124189339e-07\\
0.9484948	4.56557841255201e-07\\
0.9485949	-4.60061466367456e-07\\
0.9486949	-2.63542400291694e-07\\
0.9487949	-1.74650272799504e-07\\
0.9488949	-9.79480290763668e-08\\
0.9489949	-3.79979238074313e-08\\
0.9490949	1.00638506417283e-07\\
0.9491949	2.13400474013969e-07\\
0.9492949	2.95727967891679e-07\\
0.9493949	3.43061787533294e-07\\
0.9494949	4.50843566923709e-07\\
0.949595	-3.73342819326439e-07\\
0.949695	-2.55787671987751e-07\\
0.949795	-1.9145890528538e-07\\
0.949895	-8.49112475331815e-08\\
0.949995	-4.06984409506705e-08\\
0.950095	1.36626778501903e-07\\
0.950195	1.42512715961551e-07\\
0.950295	2.72408743617891e-07\\
0.950395	4.21765329827739e-07\\
0.950495	4.86034067270769e-07\\
0.9505951	-4.01902745861271e-07\\
0.9506951	-3.18946745292426e-07\\
0.9507951	-2.34721332037502e-07\\
0.9508951	-5.37701328462425e-08\\
0.9509951	1.93644885682076e-08\\
0.9510951	8.01414619666652e-08\\
0.9511951	2.240210360549e-07\\
0.9512951	2.46464807451829e-07\\
0.9513951	3.42935745856199e-07\\
0.9514951	5.08898226780907e-07\\
0.9515952	-3.97918830108424e-07\\
0.9516952	-3.04116245630759e-07\\
0.9517952	-1.54424846188217e-07\\
0.9518952	-5.33743790276375e-08\\
0.9519952	-5.49304968178177e-09\\
0.9520952	8.4692495683214e-08\\
0.9521952	2.12657204012867e-07\\
0.9522952	2.73877637846365e-07\\
0.9523952	3.63832005079168e-07\\
0.9524952	4.78000181880489e-07\\
0.9525953	-4.01492845802931e-07\\
0.9526953	-2.5003768872045e-07\\
0.9527953	-1.8792332334705e-07\\
0.9528953	-1.19662930120334e-07\\
0.9529953	-4.9767888643526e-08\\
0.9530953	1.17252246856214e-07\\
0.9531953	1.76889774550215e-07\\
0.9532953	2.2463886872881e-07\\
0.9533953	3.55995607104198e-07\\
0.9534953	4.66457994341418e-07\\
0.9535954	-3.37901952918163e-07\\
0.9536954	-2.80358555093674e-07\\
0.9537954	-1.57208166873257e-07\\
0.9538954	-7.29448170755376e-08\\
0.9539954	-3.20604828796256e-08\\
0.9540954	6.09549402064147e-08\\
0.9541954	2.01613655770316e-07\\
0.9542954	2.85429996582653e-07\\
0.9543954	3.07920446696031e-07\\
0.9544954	4.64603666483487e-07\\
0.9545955	-3.14948559620998e-07\\
0.9546955	-3.00991996993183e-07\\
0.9547955	-1.66277857094332e-07\\
0.9548955	-1.15278531498397e-07\\
0.9549955	-5.24641140531301e-08\\
0.9550955	1.17697621315749e-07\\
0.9551955	1.90741245298387e-07\\
0.9552955	2.62203696421714e-07\\
0.9553955	3.27624303853025e-07\\
0.9554955	3.82544811217134e-07\\
0.9555956	-3.20408455489751e-07\\
0.9556956	-2.97574775709464e-07\\
0.9557956	-1.98605691781317e-07\\
0.9558956	-1.27949565886798e-07\\
0.9559956	9.94776933049835e-09\\
0.9560956	1.10643034345603e-07\\
0.9561956	1.69695527463265e-07\\
0.9562956	2.82667143580717e-07\\
0.9563956	3.45122398603959e-07\\
0.9564956	4.52628452385362e-07\\
0.9565957	-3.19576790641651e-07\\
0.9566957	-2.33022996121157e-07\\
0.9567957	-2.1470526287426e-07\\
0.9568957	-6.90456322394173e-08\\
0.9569957	-4.63390409377595e-10\\
0.9570957	8.66249517950113e-08\\
0.9571957	1.87805681534092e-07\\
0.9572957	1.98667905820793e-07\\
0.9573957	3.14803572820654e-07\\
0.9574957	4.31807493428416e-07\\
0.9575958	-3.52911306261694e-07\\
0.9576958	-2.45185023256544e-07\\
0.9577958	-1.49793060150172e-07\\
0.9578958	-7.11289390997738e-08\\
0.9579958	-1.35832143866033e-08\\
0.9580958	1.18456549319523e-07\\
0.9581958	1.20605799375229e-07\\
0.9582958	1.88483012475427e-07\\
0.9583958	3.17709718339332e-07\\
0.9584958	4.03910517837724e-07\\
0.9585959	-3.33772173344293e-07\\
0.9586959	-2.44590943970779e-07\\
0.9587959	-1.11547541814527e-07\\
0.9588959	-3.90048701393031e-08\\
0.9589959	-3.13226577730585e-08\\
0.9590959	1.07142560656914e-07\\
0.9591959	1.72037462775609e-07\\
0.9592959	2.5901196110123e-07\\
0.9593959	2.63719219489866e-07\\
0.9594959	3.81815675798425e-07\\
0.959596	-3.46257665290173e-07\\
0.959696	-2.12297771501335e-07\\
0.959796	-1.77963870168218e-07\\
0.959896	-4.75862358129131e-08\\
0.959996	-2.54917744690886e-08\\
0.960096	8.39959968523818e-08\\
0.960196	1.76556966963171e-07\\
0.960296	2.47874453296326e-07\\
0.960396	2.9363521600323e-07\\
0.960496	4.09529479300508e-07\\
0.9605961	-3.43134819003055e-07\\
0.9606961	-1.97829564091423e-07\\
0.9607961	-1.95303466427044e-07\\
0.9608961	-3.98522601585682e-08\\
0.9609961	-3.57681229634643e-08\\
0.9610961	1.12660341167192e-07\\
0.9611961	1.01148119130423e-07\\
0.9612961	2.25413808313535e-07\\
0.9613961	2.81179633215833e-07\\
0.9614961	3.64171463318463e-07\\
0.9615962	-2.43864157228879e-07\\
0.9616962	-2.17211484049296e-07\\
0.9617962	-1.76137410629051e-07\\
0.9618962	-2.49013067943338e-08\\
0.9619962	3.22411896624963e-08\\
0.9620962	9.10381914576153e-08\\
0.9621962	1.47241577541646e-07\\
0.9622962	1.96607010422012e-07\\
0.9623962	2.34893955869397e-07\\
0.9624962	3.57865695188586e-07\\
0.9625963	-2.32717427273155e-07\\
0.9626963	-2.51096792136529e-07\\
0.9627963	-9.7482667854587e-08\\
0.9628963	-7.60963289048178e-08\\
0.9629963	8.84484810392827e-09\\
0.9630963	5.3127404531228e-08\\
0.9631963	1.5254181473523e-07\\
0.9632963	2.02882502770008e-07\\
0.9633963	2.99947857690386e-07\\
0.9634963	3.39540250637871e-07\\
0.9635964	-2.56987288560495e-07\\
0.9636964	-2.42985655568573e-07\\
0.9637964	-9.90300099701358e-08\\
0.9638964	-2.93018939667711e-08\\
0.9639964	-3.79787899740247e-08\\
0.9640964	7.07658952547874e-08\\
0.9641964	9.27628455160701e-08\\
0.9642964	2.2384684900123e-07\\
0.9643964	2.59856818091153e-07\\
0.9644964	2.96635798646694e-07\\
0.9645965	-2.25287745514446e-07\\
0.9646965	-1.97534590906656e-07\\
0.9647965	-1.81462461751325e-07\\
0.9648965	-8.12116142945296e-08\\
0.9649965	-9.1809293500944e-10\\
0.9650965	5.52862806368903e-08\\
0.9651965	8.32739249223557e-08\\
0.9652965	1.78921511773122e-07\\
0.9653965	2.38109980732482e-07\\
0.9654965	3.56724554173571e-07\\
0.9655966	-3.05944106132605e-07\\
0.9656966	-1.78955303908879e-07\\
0.9657966	-1.04862998062671e-07\\
0.9658966	-8.77646938735044e-08\\
0.9659966	-3.1753543738855e-08\\
0.9660966	5.90816644219228e-08\\
0.9661966	8.06565223221423e-08\\
0.9662966	2.28891014530941e-07\\
0.9663966	1.99709531023551e-07\\
0.9664966	2.89040883116876e-07\\
0.9665967	-2.25471135290034e-07\\
0.9666967	-2.09501586162508e-07\\
0.9667967	-8.72101779680889e-08\\
0.9668967	-6.2650286097865e-08\\
0.9669967	-3.98707999949011e-08\\
0.9670967	7.70838872177215e-08\\
0.9671967	8.41738944656889e-08\\
0.9672967	1.77363863207436e-07\\
0.9673967	2.52622971208538e-07\\
0.9674967	3.05924944765668e-07\\
0.9675968	-2.67138049708793e-07\\
0.9676968	-1.68042914860944e-07\\
0.9677968	-1.02960330566892e-07\\
0.9678968	-7.58982484638793e-08\\
0.9679968	9.13999032015056e-09\\
0.9680968	4.81556658715743e-08\\
0.9681968	1.37154692714248e-07\\
0.9682968	1.72147630983499e-07\\
0.9683968	2.49149698011708e-07\\
0.9684968	2.64180780140677e-07\\
0.9685969	-2.69618900622381e-07\\
0.9686969	-1.88723293575999e-07\\
0.9687969	-8.17148425591441e-08\\
0.9688969	-5.25548638219409e-08\\
0.9689969	-5.19994725056705e-09\\
0.9690969	5.63980566692379e-08\\
0.9691969	1.28292043988232e-07\\
0.9692969	2.06539672364947e-07\\
0.9693969	1.87203369389488e-07\\
0.9694969	2.6635034522382e-07\\
0.969597	-2.25726881047938e-07\\
0.969697	-1.59703866908778e-07\\
0.969797	-1.06971061758498e-07\\
0.969897	-7.14420160807805e-08\\
0.969997	4.29745544150428e-08\\
0.970097	3.23747796071583e-08\\
0.970197	9.28596459726094e-08\\
0.970297	1.20535005254663e-07\\
0.970397	2.11511586106772e-07\\
0.970497	2.61905002546525e-07\\
0.9705971	-1.81230998957949e-07\\
0.9706971	-1.2198774340666e-07\\
0.9707971	-1.14955255203775e-07\\
0.9708971	-6.39982692951513e-08\\
0.9709971	2.70234151722448e-08\\
0.9710971	5.42549443149909e-08\\
0.9711971	1.1384642090162e-07\\
0.9712971	2.01952910172665e-07\\
0.9713971	2.14734452330667e-07\\
0.9714971	2.48356070486899e-07\\
0.9715972	-2.33753551820426e-07\\
0.9716972	-1.68325459709573e-07\\
0.9717972	-9.3536047213183e-08\\
0.9718972	-1.32002611064053e-08\\
0.9719972	-3.11280191067631e-08\\
0.9720972	4.88757985750432e-08\\
0.9721972	1.23011358071912e-07\\
0.9722972	1.87483881114447e-07\\
0.9723972	2.38503653243738e-07\\
0.9724972	2.72286032304558e-07\\
0.9725973	-2.31746757034523e-07\\
0.9726973	-1.42199436642398e-07\\
0.9727973	-8.12166827957039e-08\\
0.9728973	-5.25627578407128e-08\\
0.9729973	4.00031893998953e-08\\
0.9730973	9.27271325001922e-08\\
0.9731973	1.01860177154078e-07\\
0.9732973	1.63658565560665e-07\\
0.9733973	1.74383687184809e-07\\
0.9734973	2.30302085513223e-07\\
0.9735974	-1.73546890414045e-07\\
0.9736974	-1.36885772827888e-07\\
0.9737974	-6.62044600530098e-08\\
0.9738974	-6.52157101099782e-08\\
0.9739974	-3.76270878343909e-08\\
0.9740974	1.28590410819474e-08\\
0.9741974	8.25455189756852e-08\\
0.9742974	1.67740401679772e-07\\
0.9743974	1.64756967041546e-07\\
0.9744974	2.69913721401283e-07\\
0.9745975	-2.06504214061987e-07\\
0.9746975	-1.945916548729e-07\\
0.9747975	-8.55556081258424e-08\\
0.9748975	-8.30565774767589e-08\\
0.9749975	9.2501964699121e-09\\
0.9750975	8.77147431910252e-08\\
0.9751975	4.86923681930485e-08\\
0.9752975	1.88543661655549e-07\\
0.9753975	2.03634501788619e-07\\
0.9754975	1.9033606363239e-07\\
0.9755976	-2.26186967544617e-07\\
0.9756976	-1.0566666308609e-07\\
0.9757976	-1.24393882722229e-07\\
0.9758976	-8.59762000759989e-08\\
0.9759976	5.98413926744534e-09\\
0.9760976	4.78902228429057e-08\\
0.9761976	1.36150476864882e-07\\
0.9762976	1.671786730717e-07\\
0.9763976	1.37393933424823e-07\\
0.9764976	2.43220736467764e-07\\
0.9765977	-1.75657662609963e-07\\
0.9766977	-1.07886189426587e-07\\
0.9767977	-1.15201098799256e-07\\
0.9768977	-1.15642492803936e-09\\
0.9769977	3.069918248777e-08\\
0.9770977	7.68224632380354e-08\\
0.9771977	1.33675551419743e-07\\
0.9772977	9.77259811407727e-08\\
0.9773977	1.65446692984572e-07\\
0.9774977	2.33316036046816e-07\\
0.9775978	-1.44819880795732e-07\\
0.9776978	-8.5805194185995e-08\\
0.9777978	-1.37177819617418e-07\\
0.9778978	-2.43771078323185e-09\\
0.9779978	1.49206115518385e-08\\
0.9780978	1.14080665053473e-08\\
0.9781978	8.3541017409372e-08\\
0.9782978	1.27841275038287e-07\\
0.9783978	1.40836103917907e-07\\
0.9784978	2.19058225517377e-07\\
0.9785979	-1.69833780164219e-07\\
0.9786979	-1.70180476155579e-07\\
0.9787979	-1.15672375677889e-07\\
0.9788979	-9.75487925158891e-09\\
0.9789979	4.41320907179632e-08\\
0.9790979	4.25540949006153e-08\\
0.9791979	8.20821799696336e-08\\
0.9792979	1.59292882828677e-07\\
0.9793979	1.70768235996377e-07\\
0.9794979	2.13095768766824e-07\\
0.979598	-1.32598464153144e-07\\
0.979698	-1.37459629012682e-07\\
0.979798	-1.21676392725312e-07\\
0.979898	-8.86391915699836e-08\\
0.979998	-4.17329438923142e-08\\
0.980098	1.56629497705119e-08\\
0.980198	8.01746125322911e-08\\
0.980298	1.48433693461514e-07\\
0.980398	1.17077371505903e-07\\
0.980498	1.82748358057427e-07\\
0.9805981	-1.60299386292584e-07\\
0.9806981	-1.09334845746489e-07\\
0.9807981	-7.1384973099331e-08\\
0.9808981	-4.97848901409581e-08\\
0.9809981	-4.78641706408434e-08\\
0.9810981	3.10531620001386e-08\\
0.9811981	8.36486377525647e-08\\
0.9812981	1.06609342756525e-07\\
0.9813981	1.9662792097308e-07\\
0.9814981	1.50402579501829e-07\\
0.9815982	-1.25018916566355e-07\\
0.9816982	-1.52359698940563e-07\\
0.9817982	-1.25819673291172e-07\\
0.9818982	-4.8678358594012e-08\\
0.9819982	-2.42096992314478e-08\\
0.9820982	4.43179347664469e-08\\
0.9821982	5.36417516239229e-08\\
0.9822982	1.00504539621293e-07\\
0.9823982	1.81654668890419e-07\\
0.9824982	1.93846093760053e-07\\
0.9825983	-1.43408219521746e-07\\
0.9826983	-7.76270458330286e-08\\
0.9827983	-9.05124164507498e-08\\
0.9828983	-8.52880163437764e-08\\
0.9829983	3.4828063391612e-08\\
0.9830983	3.66233258997095e-08\\
0.9831983	7.68908715170769e-08\\
0.9832983	1.12429398832492e-07\\
0.9833983	1.50043205596195e-07\\
0.9834983	1.86542191124572e-07\\
0.9835984	-1.46418555502237e-07\\
0.9836984	-1.10506168798929e-07\\
0.9837984	-7.52484542870802e-08\\
0.9838984	-3.38130883847931e-08\\
0.9839984	6.3785954296236e-10\\
0.9840984	3.49479303829447e-08\\
0.9841984	7.59662738825106e-08\\
0.9842984	1.10547651130388e-07\\
0.9843984	1.45552435007468e-07\\
0.9844984	1.77846611854049e-07\\
0.9845985	-1.39090144110621e-07\\
0.9846985	-1.10437290423704e-07\\
0.9847985	-7.38665137944849e-08\\
0.9848985	-3.24893629638501e-08\\
0.9849985	5.88230866793715e-10\\
0.9850985	3.22659535895164e-08\\
0.9851985	6.94491093106153e-08\\
0.9852985	9.90486203750196e-08\\
0.9853985	1.37981028186518e-07\\
0.9854985	1.73168493697307e-07\\
0.9855986	-1.40396697656531e-07\\
0.9856986	-1.00781570602781e-07\\
0.9857986	-6.41142410967577e-08\\
0.9858986	-3.34500618731637e-08\\
0.9859986	-1.83876468218269e-09\\
0.9860986	3.76755391745531e-08\\
0.9861986	6.20543603585544e-08\\
0.9862986	9.82648291902866e-08\\
0.9863986	1.33279697625555e-07\\
0.9864986	1.6407733825774e-07\\
0.9865987	-1.33143761876786e-07\\
0.9866987	-9.87247244371714e-08\\
0.9867987	-6.75570754060661e-08\\
0.9868987	-3.26399548084e-08\\
0.9869987	3.03311697980035e-09\\
0.9870987	3.6474238601647e-08\\
0.9871987	6.47011280385801e-08\\
0.9872987	9.47371216059034e-08\\
0.9873987	1.23611173200033e-07\\
0.9874987	1.5835785480145e-07\\
0.9875988	-1.23918980063364e-07\\
0.9876988	-9.5232375860177e-08\\
0.9877988	-5.95386748375448e-08\\
0.9878988	-2.97808370186559e-08\\
0.9879988	1.10379190509935e-09\\
0.9880988	3.01834784666166e-08\\
0.9881988	6.45321028835966e-08\\
0.9882988	9.12291552579436e-08\\
0.9883988	1.27359736312727e-07\\
0.9884988	1.50014555846956e-07\\
0.9885989	-1.23092427931071e-07\\
0.9886989	-9.50552958983719e-08\\
0.9887989	-5.91910399722062e-08\\
0.9888989	-2.83865190242594e-08\\
0.9889989	4.47701177519466e-09\\
0.9890989	2.65238995489359e-08\\
0.9891989	6.48840922518046e-08\\
0.9892989	8.66931372384139e-08\\
0.9893989	1.1909218004813e-07\\
0.9894989	1.49227962376042e-07\\
0.989599	-1.24865150760711e-07\\
0.989699	-8.67826575601249e-08\\
0.989799	-5.94924678826159e-08\\
0.989899	-2.5825463420559e-08\\
0.989999	1.39306302604503e-09\\
0.990099	2.93434053583175e-08\\
0.990199	5.52114426538575e-08\\
0.990299	8.61886376712917e-08\\
0.990399	1.19472034265655e-07\\
0.990499	1.4226425601449e-07\\
0.9905991	-1.17364065077563e-07\\
0.9906991	-8.29414756597924e-08\\
0.9907991	-5.73715096263694e-08\\
0.9908991	-3.34292394182922e-08\\
0.9909991	-3.88416758478893e-09\\
0.9910991	2.84997707378221e-08\\
0.9911991	6.09642056559334e-08\\
0.9912991	8.0756330525511e-08\\
0.9913991	1.15128899191852e-07\\
0.9914991	1.41340224121822e-07\\
0.9915992	-1.12781393460382e-07\\
0.9916992	-8.01403891442787e-08\\
0.9917992	-5.38550860296194e-08\\
0.9918992	-2.66449557695392e-08\\
0.9919992	-1.22392380952352e-09\\
0.9920992	2.9699628492641e-08\\
0.9921992	5.34228611764886e-08\\
0.9922992	7.72484724381739e-08\\
0.9923992	1.08484697725064e-07\\
0.9924992	1.34445304847408e-07\\
0.9925993	-1.07556808473519e-07\\
0.9926993	-7.92559815693261e-08\\
0.9927993	-5.42589624896728e-08\\
0.9928993	-2.52298708175447e-08\\
0.9929993	-4.82730756756755e-09\\
0.9930993	2.42956429002961e-08\\
0.9931993	4.94914098037036e-08\\
0.9932993	7.81179332332016e-08\\
0.9933993	1.07538660587547e-07\\
0.9934993	1.35122543903909e-07\\
0.9935994	-1.0260049550237e-07\\
0.9936994	-7.56598334400449e-08\\
0.9937994	-4.84187748783782e-08\\
0.9938994	-2.34863753861836e-08\\
0.9939994	-3.46620182556379e-09\\
0.9940994	2.90436644040382e-08\\
0.9941994	5.14506241500179e-08\\
0.9942994	8.1167556814643e-08\\
0.9943994	1.05612817949408e-07\\
0.9944994	1.22210234584919e-07\\
0.9945995	-1.03555344979633e-07\\
0.9946995	-7.94845309431969e-08\\
0.9947995	-5.09598393483301e-08\\
0.9948995	-2.05355843441168e-08\\
0.9949995	-7.60624813134214e-10\\
0.9950995	2.58216304171999e-08\\
0.9951995	4.66732201220843e-08\\
0.9952995	6.92616267117252e-08\\
0.9953995	1.01059772258938e-07\\
0.9954995	1.19546015198146e-07\\
0.9955996	-1.0109652153617e-07\\
0.9956996	-7.59268771535759e-08\\
0.9957996	-5.16039842940463e-08\\
0.9958996	-2.06277727191062e-08\\
0.9959996	4.50724464695629e-09\\
0.9960996	2.13119678571516e-08\\
0.9961996	4.73027060507447e-08\\
0.9962996	7.00011734203199e-08\\
0.9963996	9.69344846543652e-08\\
0.9964996	1.15635152467264e-07\\
0.9965997	-9.12666656902039e-08\\
0.9966997	-6.55865610710293e-08\\
0.9967997	-4.55116722225535e-08\\
0.9968997	-2.34879352911666e-08\\
0.9969997	-1.95590992707606e-09\\
0.9970997	2.66492158201848e-08\\
0.9971997	4.9897621437478e-08\\
0.9972997	6.53648498522008e-08\\
0.9973997	9.06318030000075e-08\\
0.9974997	1.13284737437991e-07\\
0.9975998	-8.58450210222683e-08\\
0.9976998	-6.48385842516586e-08\\
0.9977998	-4.36577078030687e-08\\
0.9978998	-2.46947556986621e-08\\
0.9979998	-3.36759273245057e-10\\
0.9980998	2.70345776876013e-08\\
0.9981998	4.50428748882681e-08\\
0.9982998	7.1317070478305e-08\\
0.9983998	8.34914168287848e-08\\
0.9984998	1.09205475287699e-07\\
0.9985999	-8.27487905211144e-08\\
0.9986999	-6.62377475577069e-08\\
0.9987999	-4.32388397031414e-08\\
0.9988999	-2.6091309564269e-08\\
0.9989999	2.87088567502147e-09\\
0.9990999	2.13190705228294e-08\\
0.9991999	4.69298448366184e-08\\
0.9992999	6.7385079863258e-08\\
0.9993999	8.03719130623715e-08\\
0.9994999	1.03582743306296e-07\\
0.9996	-8.64650692017355e-08\\
0.9997	-5.89535509090688e-08\\
0.9998	-3.8111609590441e-08\\
0.9999	-1.62258455352338e-08\\
1	4.42237758913056e-09\\
};
\end{axis}
\end{tikzpicture}%
		\caption{The error is defined as the difference between the analytical solution and numerical solution.}
		\label{fig:periodicInfError}
	\end{subfigure}
	\label{fig:smallError}
	\caption{The numerical and analytical solution as it passes its solution space of $x=[0,1]$. One can clearly see in figure \ref{fig:periodicInfPlot} that the numerical solution is periodic. The error caused by this is highlighted in figure \ref{fig:periodicInfError}}
\end{figure}
\fi

\newpage
\section{Potential Well}
We can see in figure \ref{fig:potWell} the effect of a potential barrier. Part of the wavefunction is reflected by it, while a large part goes trough it. Here we had to change the time step to $1\cdot10^{-7}$ in order to get a smooth plot.
\begin{figure}[H]
	\centering
	\begin{subfigure}{.9\linewidth}
		\setlength\figureheight{.5\linewidth}
		\setlength\figurewidth{.9\linewidth}
		% This file was created by matlab2tikz.
% Minimal pgfplots version: 1.3
%
%The latest updates can be retrieved from
%  http://www.mathworks.com/matlabcentral/fileexchange/22022-matlab2tikz
%where you can also make suggestions and rate matlab2tikz.
%
\definecolor{mycolor1}{rgb}{0.00000,0.44700,0.74100}%
\definecolor{mycolor2}{rgb}{0.85000,0.32500,0.09800}%
\definecolor{mycolor3}{rgb}{1.00000,0.50000,1.00000}%
%
\begin{tikzpicture}

\begin{axis}[%
width=0.95092\figurewidth,
height=\figureheight,
at={(0\figurewidth,0\figureheight)},
scale only axis,
xmin=0,
xmax=1,
xlabel={Position},
ymin=-2,
ymax=4,
title style={font=\bfseries},
title={Numerical Solution with a Potential Well},
legend style={at={(0.03,0.97)},anchor=north west,legend cell align=left,align=left,draw=white!15!black},
title style={font=\small},ticklabel style={font=\tiny}
]
\addplot [color=mycolor1,solid]
  table[row sep=crcr]{%
0	1.014437\\
0.001001001	1.066478\\
0.002002002	1.049845\\
0.003003003	0.9822696\\
0.004004004	0.8893669\\
0.005005005	0.7973189\\
0.006006006	0.7261089\\
0.007007007	0.6848868\\
0.008008008	0.6705092\\
0.009009009	0.6695216\\
0.01001001	0.6630119\\
0.01101101	0.6330601\\
0.01201201	0.5690947\\
0.01301301	0.4724555\\
0.01401401	0.357853\\
0.01501502	0.2511453\\
0.01601602	0.1837474\\
0.01701702	0.184855\\
0.01801802	0.273285\\
0.01901902	0.4509592\\
0.02002002	0.6998064\\
0.02102102	0.9831633\\
0.02202202	1.25176\\
0.02302302	1.453294\\
0.02402402	1.543681\\
0.02502503	1.497532\\
0.02602603	1.315412\\
0.02702703	1.026009\\
0.02802803	0.68235\\
0.02902903	0.3525035\\
0.03003003	0.106444\\
0.03103103	0.001708588\\
0.03203203	0.07086464\\
0.03303303	0.3135571\\
0.03403403	0.6950302\\
0.03503504	1.151693\\
0.03603604	1.602804\\
0.03703704	1.966028\\
0.03803804	2.173715\\
0.03903904	2.186561\\
0.04004004	2.001789\\
0.04104104	1.654098\\
0.04204204	1.209126\\
0.04304304	0.7507443\\
0.04404404	0.3647441\\
0.04504505	0.1222343\\
0.04604605	0.06609233\\
0.04704705	0.2031405\\
0.04804805	0.5035093\\
0.04904905	0.9071619\\
0.05005005	1.336103\\
0.05105105	1.709693\\
0.05205205	1.959951\\
0.05305305	2.043842\\
0.05405405	1.950275\\
0.05505506	1.700686\\
0.05605606	1.343408\\
0.05705706	0.9432502\\
0.05805806	0.5685808\\
0.05905906	0.2785414\\
0.06006006	0.1128546\\
0.06106106	0.08599415\\
0.06206206	0.1865125\\
0.06306306	0.3812537\\
0.06406406	0.6232425\\
0.06506507	0.861428\\
0.06606607	1.050254\\
0.06706707	1.157254\\
0.06806807	1.167434\\
0.06906907	1.083956\\
0.07007007	0.9254431\\
0.07107107	0.7208296\\
0.07207207	0.5030801\\
0.07307307	0.303148\\
0.07407407	0.1452982\\
0.07507508	0.04446858\\
0.07607608	0.005799156\\
0.07707708	0.02596121\\
0.07807808	0.09558007\\
0.07907908	0.2019286\\
0.08008008	0.3311819\\
0.08108108	0.4698146\\
0.08208208	0.6050902\\
0.08308308	0.7249389\\
0.08408408	0.8177287\\
0.08508509	0.8724645\\
0.08608609	0.8797769\\
0.08708709	0.8337457\\
0.08808809	0.7342311\\
0.08908909	0.5890702\\
0.09009009	0.4153417\\
0.09109109	0.2389649\\
0.09209209	0.0922012\\
0.09309309	0.009095339\\
0.09409409	0.01944146\\
0.0950951	0.1423372\\
0.0960961	0.3806777\\
0.0970971	0.7179431\\
0.0980981	1.118317\\
0.0990991	1.530581\\
0.1001001	1.895488\\
0.1011011	2.155515\\
0.1021021	2.265351\\
0.1031031	2.201111\\
0.1041041	1.966389\\
0.1051051	1.593747\\
0.1061061	1.141015\\
0.1071071	0.682766\\
0.1081081	0.2982719\\
0.1091091	0.057986\\
0.1101101	0.01096454\\
0.1111111	0.1755646\\
0.1121121	0.5352214\\
0.1131131	1.040223\\
0.1141141	1.61532\\
0.1151151	2.171932\\
0.1161161	2.622866\\
0.1171171	2.896962\\
0.1181181	2.951094\\
0.1191191	2.777385\\
0.1201201	2.404389\\
0.1211211	1.892028\\
0.1221221	1.321225\\
0.1231231	0.7800965\\
0.1241241	0.3491604\\
0.1251251	0.08816118\\
0.1261261	0.02679061\\
0.1271271	0.1608326\\
0.1281281	0.4542652\\
0.1291291	0.8467683\\
0.1301301	1.265128\\
0.1311311	1.636364\\
0.1321321	1.900171\\
0.1331331	2.018451\\
0.1341341	1.980359\\
0.1351351	1.80215\\
0.1361361	1.522139\\
0.1371371	1.191983\\
0.1381381	0.8661603\\
0.1391391	0.5917739\\
0.1401401	0.4006423\\
0.1411411	0.3050955\\
0.1421421	0.2980648\\
0.1431431	0.3571416\\
0.1441441	0.4514618\\
0.1451451	0.5497182\\
0.1461461	0.6274385\\
0.1471471	0.6719083\\
0.1481481	0.6837229\\
0.1491491	0.6747732\\
0.1501502	0.6633266\\
0.1511512	0.667559\\
0.1521522	0.6992611\\
0.1531532	0.7594021\\
0.1541542	0.8367734\\
0.1551552	0.9101793\\
0.1561562	0.9537368\\
0.1571572	0.9440367\\
0.1581582	0.8673784\\
0.1591592	0.7251771\\
0.1601602	0.5359946\\
0.1611612	0.3333955\\
0.1621622	0.1598147\\
0.1631632	0.05761835\\
0.1641642	0.05929395\\
0.1651652	0.1790399\\
0.1661662	0.4078303\\
0.1671672	0.7133338\\
0.1681682	1.045013\\
0.1691692	1.343552\\
0.1701702	1.552733\\
0.1711712	1.631242\\
0.1721722	1.56182\\
0.1731732	1.355656\\
0.1741742	1.050942\\
0.1751752	0.7057516\\
0.1761762	0.3866547\\
0.1771772	0.1554302\\
0.1781782	0.05667639\\
0.1791792	0.1089251\\
0.1801802	0.3011192\\
0.1811812	0.5951495\\
0.1821822	0.9338451\\
0.1831832	1.252664\\
0.1841842	1.492575\\
0.1851852	1.611434\\
0.1861862	1.591554\\
0.1871872	1.442049\\
0.1881882	1.195678\\
0.1891892	0.9011046\\
0.1901902	0.6123843\\
0.1911912	0.3780034\\
0.1921922	0.23176\\
0.1931932	0.1872516\\
0.1941942	0.2368713\\
0.1951952	0.3552052\\
0.1961962	0.5057943\\
0.1971972	0.6495869\\
0.1981982	0.7531679\\
0.1991992	0.7950418\\
0.2002002	0.7688045\\
0.2012012	0.6827986\\
0.2022022	0.5566541\\
0.2032032	0.4157601\\
0.2042042	0.2850883\\
0.2052052	0.1838094\\
0.2062062	0.1218412\\
0.2072072	0.09892808\\
0.2082082	0.1062096\\
0.2092092	0.1296564\\
0.2102102	0.1543517\\
0.2112112	0.1684739\\
0.2122122	0.1659959\\
0.2132132	0.1475092\\
0.2142142	0.1191062\\
0.2152152	0.08976709\\
0.2162162	0.0680844\\
0.2172172	0.05930977\\
0.2182182	0.06360164\\
0.2192192	0.07600835\\
0.2202202	0.08823142\\
0.2212212	0.09170342\\
0.2222222	0.08111432\\
0.2232232	0.05733917\\
0.2242242	0.02881216\\
0.2252252	0.0107459\\
0.2262262	0.02213288\\
0.2272272	0.08106051\\
0.2282282	0.1993754\\
0.2292292	0.3780142\\
0.2302302	0.6042943\\
0.2312312	0.8521104\\
0.2322322	1.085381\\
0.2332332	1.264343\\
0.2342342	1.353604\\
0.2352352	1.330323\\
0.2362362	1.190745\\
0.2372372	0.9534784\\
0.2382382	0.6585118\\
0.2392392	0.3617612\\
0.2402402	0.1258857\\
0.2412412	0.008923388\\
0.2422422	0.05285275\\
0.2432432	0.2743312\\
0.2442442	0.6595592\\
0.2452452	1.164512\\
0.2462462	1.720801\\
0.2472472	2.246347\\
0.2482482	2.659105\\
0.2492492	2.891442\\
0.2502503	2.902535\\
0.2512513	2.686484\\
0.2522523	2.274537\\
0.2532533	1.730854\\
0.2542543	1.142433\\
0.2552553	0.6048461\\
0.2562563	0.2062597\\
0.2572573	0.01250613\\
0.2582583	0.05585023\\
0.2592593	0.3294354\\
0.2602603	0.7884057\\
0.2612613	1.357515\\
0.2622623	1.943893\\
0.2632633	2.452715\\
0.2642643	2.803047\\
0.2652653	2.941087\\
0.2662663	2.848543\\
0.2672673	2.544768\\
0.2682683	2.082388\\
0.2692693	1.537337\\
0.2702703	0.9952089\\
0.2712713	0.5364191\\
0.2722723	0.2228764\\
0.2732733	0.08847659\\
0.2742743	0.1349897\\
0.2752753	0.3338602\\
0.2762763	0.6333235\\
0.2772773	0.9692683\\
0.2782783	1.277624\\
0.2792793	1.505855\\
0.2802803	1.621412\\
0.2812813	1.615709\\
0.2822823	1.503124\\
0.2832833	1.315571\\
0.2842843	1.094068\\
0.2852853	0.8792809\\
0.2862863	0.7031196\\
0.2872873	0.583151\\
0.2882883	0.5208597\\
0.2892893	0.5038762\\
0.2902903	0.5113726\\
0.2912913	0.5211132\\
0.2922923	0.5163087\\
0.2932933	0.4905263\\
0.2942943	0.4494292\\
0.2952953	0.4089367\\
0.2962963	0.3903056\\
0.2972973	0.4134266\\
0.2982983	0.4901038\\
0.2992993	0.6191292\\
0.3003003	0.7845511\\
0.3013013	0.9577743\\
0.3023023	1.103179\\
0.3033033	1.186061\\
0.3043043	1.181055\\
0.3053053	1.079026\\
0.3063063	0.8906842\\
0.3073073	0.6458853\\
0.3083083	0.3885797\\
0.3093093	0.1683546\\
0.3103103	0.03034541\\
0.3113113	0.005700512\\
0.3123123	0.1047031\\
0.3133133	0.3140961\\
0.3143143	0.599242\\
0.3153153	0.9106929\\
0.3163163	1.193791\\
0.3173173	1.399274\\
0.3183183	1.49267\\
0.3193193	1.46057\\
0.3203203	1.312573\\
0.3213213	1.078632\\
0.3223223	0.802541\\
0.3233233	0.5330203\\
0.3243243	0.3143474\\
0.3253253	0.1784182\\
0.3263263	0.1397449\\
0.3273273	0.194177\\
0.3283283	0.3213102\\
0.3293293	0.4897914\\
0.3303303	0.6641967\\
0.3313313	0.8119563\\
0.3323323	0.9089377\\
0.3333333	0.9427233\\
0.3343343	0.9132042\\
0.3353353	0.8307236\\
0.3363363	0.7124896\\
0.3373373	0.578261\\
0.3383383	0.4463241\\
0.3393393	0.3305705\\
0.3403403	0.2391116\\
0.3413413	0.174439\\
0.3423423	0.1347621\\
0.3433433	0.1159245\\
0.3443443	0.1132425\\
0.3453453	0.1227459\\
0.3463463	0.1415559\\
0.3473473	0.1674523\\
0.3483483	0.197957\\
0.3493493	0.2294174\\
0.3503504	0.2565838\\
0.3513514	0.2730101\\
0.3523524	0.2723438\\
0.3533534	0.2502497\\
0.3543544	0.2064531\\
0.3553554	0.1462485\\
0.3563564	0.08086152\\
0.3573574	0.02628011\\
0.3583584	0.0005347384\\
0.3593594	0.01983165\\
0.3603604	0.09431603\\
0.3613614	0.2244633\\
0.3623624	0.3990931\\
0.3633634	0.5957478\\
0.3643644	0.783714\\
0.3653654	0.9293763\\
0.3663664	1.003016\\
0.3673674	0.9857263\\
0.3683684	0.874937\\
0.3693694	0.6871865\\
0.3703704	0.4572283\\
0.3713714	0.2332672\\
0.3723724	0.06891149\\
0.3733734	0.01317741\\
0.3743744	0.1003915\\
0.3753754	0.3420095\\
0.3763764	0.7221332\\
0.3773774	1.197898\\
0.3783784	1.705018\\
0.3793794	2.167782\\
0.3803804	2.511909\\
0.3813814	2.677998\\
0.3823824	2.63312\\
0.3833834	2.378303\\
0.3843844	1.950315\\
0.3853854	1.417175\\
0.3863864	0.8678957\\
0.3873874	0.3980902\\
0.3883884	0.09382153\\
0.3893894	0.01651536\\
0.3903904	0.1916212\\
0.3913914	0.6031267\\
0.3923924	1.195044\\
0.3933934	1.879786\\
0.3943944	2.552128\\
0.3953954	3.106458\\
0.3963964	3.454384\\
0.3973974	3.539679\\
0.3983984	3.347934\\
0.3993994	2.909225\\
0.4004004	2.293248\\
0.4014014	1.597749\\
0.4024024	0.9321898\\
0.4034034	0.3994848\\
0.4044044	0.0789638\\
0.4054054	0.013501\\
0.4064064	0.2030181\\
0.4074074	0.605426\\
0.4084084	1.144746\\
0.4094094	1.724868\\
0.4104104	2.246433\\
0.4114114	2.623768\\
0.4124124	2.798891\\
0.4134134	2.750155\\
0.4144144	2.494163\\
0.4154154	2.080841\\
0.4164164	1.58284\\
0.4174174	1.081442\\
0.4184184	0.6517453\\
0.4194194	0.3499059\\
0.4204204	0.2047445\\
0.4214214	0.2150413\\
0.4224224	0.3526982\\
0.4234234	0.5707591\\
0.4244244	0.814372\\
0.4254254	1.032274\\
0.4264264	1.186418\\
0.4274274	1.257876\\
0.4284284	1.248077\\
0.4294294	1.175473\\
0.4304304	1.068755\\
0.4314314	0.9584399\\
0.4324324	0.8688982\\
0.4334334	0.8127308\\
0.4344344	0.7886888\\
0.4354354	0.7834937\\
0.4364364	0.7768936\\
0.4374374	0.7485615\\
0.4384384	0.6849672\\
0.4394394	0.5844253\\
0.4404404	0.4589529\\
0.4414414	0.3323968\\
0.4424424	0.2351944\\
0.4434434	0.1969874\\
0.4444444	0.2388548\\
0.4454454	0.3670557\\
0.4464464	0.5698759\\
0.4474474	0.8184442\\
0.4484484	1.071526\\
0.4494494	1.283331\\
0.4504505	1.412744\\
0.4514515	1.431956\\
0.4524525	1.332684\\
0.4534535	1.128551\\
0.4544545	0.8531487\\
0.4554555	0.5540604\\
0.4564565	0.2841145\\
0.4574575	0.09154888\\
0.4584585	0.01108607\\
0.4594595	0.05759558\\
0.4604605	0.2235451\\
0.4614615	0.4806186\\
0.4624625	0.7850769\\
0.4634635	1.085771\\
0.4644645	1.333185\\
0.4654655	1.487919\\
0.4664665	1.526987\\
0.4674675	1.447111\\
0.4684685	1.264366\\
0.4694695	1.010726\\
0.4704705	0.7279901\\
0.4714715	0.4605918\\
0.4724725	0.2482498\\
0.4734735	0.1199922\\
0.4744745	0.09010416\\
0.4754755	0.1567897\\
0.4764765	0.3033246\\
0.4774775	0.5014378\\
0.4784785	0.7163163\\
0.4794795	0.9120036\\
0.4804805	1.057173\\
0.4814815	1.128973\\
0.4824825	1.117079\\
0.4834835	1.023008\\
0.4844845	0.8631968\\
0.4854855	0.6535388\\
0.4864865	0.3911484\\
0.4874875	0.1593423\\
0.4884885	0.04349088\\
0.4894895	0.08731621\\
0.4904905	0.2776077\\
0.4914915	0.5474213\\
0.4924925	0.7990549\\
0.4934935	0.939549\\
0.4944945	0.9147384\\
0.4954955	0.7306585\\
0.4964965	0.4527538\\
0.4974975	0.1834906\\
0.4984985	0.02526293\\
0.4994995	0.04235872\\
0.5005005	0.2358214\\
0.5015015	0.5410848\\
0.5025025	0.8499569\\
0.5035035	1.049601\\
0.5045045	1.064454\\
0.5055055	0.8857707\\
0.5065065	0.5770247\\
0.5075075	0.2528539\\
0.5085085	0.03788027\\
0.5095095	0.02074789\\
0.5105105	0.2197257\\
0.5115115	0.5733432\\
0.5125125	0.9607497\\
0.5135135	1.243582\\
0.5145145	1.318556\\
0.5155155	1.152665\\
0.5165165	0.8137831\\
0.5175175	0.4645584\\
0.5185185	0.1821551\\
0.5195195	0.0217516\\
0.5205205	0.0238752\\
0.5215215	0.2046393\\
0.5225225	0.5523533\\
0.5235235	1.028271\\
0.5245245	1.571138\\
0.5255255	2.10645\\
0.5265265	2.557158\\
0.5275275	2.856149\\
0.5285285	2.956634\\
0.5295295	2.840513\\
0.5305305	2.521757\\
0.5315315	2.045528\\
0.5325325	1.481899\\
0.5335335	0.9160021\\
0.5345345	0.4352628\\
0.5355355	0.1162407\\
0.5365365	0.01266285\\
0.5375375	0.1468109\\
0.5385385	0.5054936\\
0.5395395	1.041374\\
0.5405405	1.679585\\
0.5415415	2.328604\\
0.5425425	2.893949\\
0.5435435	3.292417\\
0.5445445	3.464896\\
0.5455455	3.385482\\
0.5465465	3.065686\\
0.5475475	2.552739\\
0.5485485	1.922452\\
0.5495495	1.267476\\
0.5505506	0.6829821\\
0.5515516	0.2518822\\
0.5525526	0.03207325\\
0.5535536	0.0476882\\
0.5545546	0.2858248\\
0.5555556	0.6992452\\
0.5565566	1.214574\\
0.5575576	1.744619\\
0.5585586	2.20268\\
0.5595596	2.516487\\
0.5605606	2.639349\\
0.5615616	2.556765\\
0.5625626	2.28738\\
0.5635636	1.878367\\
0.5645646	1.396125\\
0.5655656	0.9141466\\
0.5665666	0.5002623\\
0.5675676	0.2056024\\
0.5685686	0.05715809\\
0.5695696	0.05514129\\
0.5705706	0.1753528\\
0.5715716	0.3758268\\
0.5725726	0.6062218\\
0.5735736	0.817957\\
0.5745746	0.9730686\\
0.5755756	1.050101\\
0.5765766	1.046088\\
0.5775776	0.9745067\\
0.5785786	0.8600025\\
0.5795796	0.731274\\
0.5805806	0.6138873\\
0.5815816	0.5246537\\
0.5825826	0.4687882\\
0.5835836	0.440326\\
0.5845846	0.4254944\\
0.5855856	0.4080174\\
0.5865866	0.3748956\\
0.5875876	0.3211157\\
0.5885886	0.2520326\\
0.5895896	0.1827631\\
0.5905906	0.1346756\\
0.5915916	0.1298019\\
0.5925926	0.1845283\\
0.5935936	0.3041523\\
0.5945946	0.4797287\\
0.5955956	0.6881408\\
0.5965966	0.8955992\\
0.5975976	1.063982\\
0.5985986	1.15875\\
0.5995996	1.156755\\
0.6006006	1.052205\\
0.6016016	0.8593906\\
0.6026026	0.6113915\\
0.6036036	0.3548345\\
0.6046046	0.1415722\\
0.6056056	0.01883608\\
0.6066066	0.01977724\\
0.6076076	0.1562872\\
0.6086086	0.4156009\\
0.6096096	0.761481\\
0.6106106	1.139914\\
0.6116116	1.488375\\
0.6126126	1.747019\\
0.6136136	1.86974\\
0.6146146	1.833039\\
0.6156156	1.640985\\
0.6166166	1.325272\\
0.6176176	0.940241\\
0.6186186	0.5536831\\
0.6196196	0.2350089\\
0.6206206	0.04289629\\
0.6216216	0.01464098\\
0.6226226	0.1591644\\
0.6236236	0.4549909\\
0.6246246	0.8536245\\
0.6256256	1.287782\\
0.6266266	1.683063\\
0.6276276	1.970998\\
0.6286286	2.101175\\
0.6296296	2.050266\\
0.6306306	1.826353\\
0.6316316	1.46776\\
0.6326326	1.036584\\
0.6336336	0.6080439\\
0.6346346	0.2575112\\
0.6356356	0.04748312\\
0.6366366	0.01675075\\
0.6376376	0.1736368\\
0.6386386	0.4944518\\
0.6396396	0.9274169\\
0.6406406	1.401363\\
0.6416416	1.837711\\
0.6426426	2.163695\\
0.6436436	2.324606\\
0.6446446	2.293016\\
0.6456456	2.073467\\
0.6466466	1.70188\\
0.6476476	1.239806\\
0.6486486	0.7644986\\
0.6496496	0.3564335\\
0.6506507	0.08631381\\
0.6516517	0.003645072\\
0.6526527	0.1286946\\
0.6536537	0.4490867\\
0.6546547	0.9215398\\
0.6556557	1.478438\\
0.6566567	2.038184\\
0.6576577	2.5177\\
0.6586587	2.845155\\
0.6596597	2.970973\\
0.6606607	2.875464\\
0.6616617	2.571973\\
0.6626627	2.105089\\
0.6636637	1.544223\\
0.6646647	0.9735074\\
0.6656657	0.4795168\\
0.6666667	0.1385847\\
0.6676677	0.00554807\\
0.6686687	0.1055343\\
0.6696697	0.4299745\\
0.6706707	0.9374354\\
0.6716717	1.559193\\
0.6726727	2.20881\\
0.6736737	2.794404\\
0.6746747	3.231909\\
0.6756757	3.457438\\
0.6766767	3.436941\\
0.6776777	3.171711\\
0.6786787	2.6988\\
0.6796797	2.086164\\
0.6806807	1.423105\\
0.6816817	0.8073016\\
0.6826827	0.3302939\\
0.6836837	0.06359084\\
0.6846847	0.04755004\\
0.6856857	0.2848303\\
0.6866867	0.7395558\\
0.6876877	1.342461\\
0.6886887	2.001333\\
0.6896897	2.615184\\
0.6906907	3.089954\\
0.6916917	3.353211\\
0.6926927	3.365459\\
0.6936937	3.126163\\
0.6946947	2.673478\\
0.6956957	2.077689\\
0.6966967	1.429483\\
0.6976977	0.8250051\\
0.6986987	0.3502527\\
0.6996997	0.0674464\\
0.7007007	0.005667777\\
0.7017017	0.1572917\\
0.7027027	0.4807038\\
0.7037037	0.908678\\
0.7047047	1.360793\\
0.7057057	1.757584\\
0.7067067	2.03386\\
0.7077077	2.148871\\
0.7087087	2.091613\\
0.7097097	1.880572\\
0.7107107	1.558225\\
0.7117117	1.18161\\
0.7127127	0.810959\\
0.7137137	0.4986551\\
0.7147147	0.2806088\\
0.7157157	0.1715472\\
0.7167167	0.1648542\\
0.7177177	0.2366425\\
0.7187187	0.3528868\\
0.7197197	0.477884\\
0.7207207	0.582126\\
0.7217217	0.647913\\
0.7227227	0.671628\\
0.7237237	0.6623994\\
0.7247247	0.6377152\\
0.7257257	0.6172387\\
0.7267267	0.6164508\\
0.7277277	0.6417421\\
0.7287287	0.6881934\\
0.7297297	0.7406161\\
0.7307307	0.7776193\\
0.7317317	0.7777201\\
0.7327327	0.7259876\\
0.7337337	0.6195268\\
0.7347347	0.470329\\
0.7357357	0.304575\\
0.7367367	0.1582839\\
0.7377377	0.0700531\\
0.7387387	0.07236409\\
0.7397397	0.1833516\\
0.7407407	0.4009541\\
0.7417417	0.7009576\\
0.7427427	1.039681\\
0.7437437	1.36109\\
0.7447447	1.607147\\
0.7457457	1.729438\\
0.7467467	1.699737\\
0.7477477	1.517217\\
0.7487487	1.210625\\
0.7497497	0.834644\\
0.7507508	0.46085\\
0.7517518	0.164771\\
0.7527528	0.01144403\\
0.7537538	0.0422718\\
0.7547548	0.2658478\\
0.7557558	0.6547421\\
0.7567568	1.149128\\
0.7577578	1.666808\\
0.7587588	2.117909\\
0.7597598	2.421554\\
0.7607608	2.521334\\
0.7617618	2.396579\\
0.7627628	2.067136\\
0.7637638	1.59063\\
0.7647648	1.052566\\
0.7657658	0.5510491\\
0.7667668	0.1789472\\
0.7677678	0.006819186\\
0.7687688	0.06982197\\
0.7697698	0.3610539\\
0.7707708	0.8325808\\
0.7717718	1.403946\\
0.7727728	1.976566\\
0.7737738	2.451348\\
0.7747748	2.746283\\
0.7757758	2.810843\\
0.7767768	2.634633\\
0.7777778	2.248856\\
0.7787788	1.720457\\
0.7797798	1.140142\\
0.7807808	0.6065166\\
0.7817818	0.2091942\\
0.7827828	0.01382391\\
0.7837838	0.05154189\\
0.7847848	0.3144964\\
0.7857858	0.7579949\\
0.7867868	1.308675\\
0.7877878	1.87712\\
0.7887888	2.372663\\
0.7897898	2.717898\\
0.7907908	2.860551\\
0.7917918	2.780956\\
0.7927928	2.494157\\
0.7937938	2.046569\\
0.7947948	1.507976\\
0.7957958	0.9603366\\
0.7967968	0.48523\\
0.7977978	0.1518765\\
0.7987988	0.007441454\\
0.7997998	0.07088944\\
0.8008008	0.3310619\\
0.8018018	0.7490234\\
0.8028028	1.264143\\
0.8038038	1.802928\\
0.8048048	2.28933\\
0.8058058	2.655154\\
0.8068068	2.849243\\
0.8078078	2.844321\\
0.8088088	2.640684\\
0.8098098	2.266302\\
0.8108108	1.773294\\
0.8118118	1.231174\\
0.8128128	0.717646\\
0.8138138	0.3080323\\
0.8148148	0.06469424\\
0.8158158	0.02784209\\
0.8168168	0.209082\\
0.8178178	0.5887595\\
0.8188188	1.117706\\
0.8198198	1.723399\\
0.8208208	2.319887\\
0.8218218	2.820183\\
0.8228228	3.149378\\
0.8238238	3.256438\\
0.8248248	3.122761\\
0.8258258	2.765953\\
0.8268268	2.238004\\
0.8278278	1.617897\\
0.8288288	0.9996801\\
0.8298298	0.4777855\\
0.8308308	0.1319749\\
0.8318318	0.01440143\\
0.8328328	0.1410154\\
0.8338338	0.4888399\\
0.8348348	0.9996635\\
0.8358358	1.58959\\
0.8368368	2.162857\\
0.8378378	2.627575\\
0.8388388	2.910686\\
0.8398398	2.969586\\
0.8408408	2.79842\\
0.8418418	2.428062\\
0.8428428	1.919869\\
0.8438438	1.354438\\
0.8448448	0.8174548\\
0.8458458	0.3852013\\
0.8468468	0.1122852\\
0.8478478	0.02368447\\
0.8488488	0.1123491\\
0.8498498	0.3425454\\
0.8508509	0.6580614\\
0.8518519	0.9935301\\
0.8528529	1.28662\\
0.8538539	1.488779\\
0.8548549	1.572617\\
0.8558559	1.534738\\
0.8568569	1.393803\\
0.8578579	1.184524\\
0.8588589	0.9491047\\
0.8598599	0.7280299\\
0.8608609	0.5521558\\
0.8618619	0.4376723\\
0.8628629	0.3847962\\
0.8638639	0.3802137\\
0.8648649	0.4024525\\
0.8658659	0.4287467\\
0.8668669	0.4416641\\
0.8678679	0.433877\\
0.8688689	0.4099328\\
0.8698699	0.3846148\\
0.8708709	0.3783189\\
0.8718719	0.410606\\
0.8728729	0.493569\\
0.8738739	0.6267506\\
0.8748749	0.7950368\\
0.8758759	0.9702915\\
0.8768769	1.116624\\
0.8778779	1.198291\\
0.8788789	1.188539\\
0.8798799	1.077321\\
0.8808809	0.8759641\\
0.8818819	0.617396\\
0.8828829	0.3514689\\
0.8838839	0.1360006\\
0.8848849	0.02516504\\
0.8858859	0.05757971\\
0.8868869	0.2466778\\
0.8878879	0.5756322\\
0.8888889	0.9982607\\
0.8898899	1.44614\\
0.8908909	1.840837\\
0.8918919	2.109006\\
0.8928929	2.197365\\
0.8938939	2.084443\\
0.8948949	1.786502\\
0.8958959	1.356147\\
0.8968969	0.8736\\
0.8978979	0.4321084\\
0.8988989	0.1202506\\
0.8998999	0.004591274\\
0.9009009	0.1161778\\
0.9019019	0.4436634\\
0.9029029	0.9345558\\
0.9039039	1.504477\\
0.9049049	2.052715\\
0.9059059	2.481067\\
0.9069069	2.712329\\
0.9079079	2.70483\\
0.9089089	2.460212\\
0.9099099	2.02301\\
0.9109109	1.472171\\
0.9119119	0.9062454\\
0.9129129	0.4251462\\
0.9139139	0.1119904\\
0.9149149	0.01841177\\
0.9159159	0.1559923\\
0.9169169	0.495211\\
0.9179179	0.9718506\\
0.9189189	1.499412\\
0.9199199	1.985037\\
0.9209209	2.345907\\
0.9219219	2.523154\\
0.9229229	2.490934\\
0.9239239	2.259314\\
0.9249249	1.870832\\
0.9259259	1.391723\\
0.9269269	0.899664\\
0.9279279	0.4704007\\
0.9289289	0.165595\\
0.9299299	0.02385201\\
0.9309309	0.05615505\\
0.9319319	0.2460774\\
0.9329329	0.5542964\\
0.9339339	0.9262678\\
0.9349349	1.301527\\
0.9359359	1.623003\\
0.9369369	1.844934\\
0.9379379	1.938392\\
0.9389389	1.893938\\
0.9399399	1.721414\\
0.9409409	1.447313\\
0.9419419	1.110388\\
0.9429429	0.7562599\\
0.9439439	0.4317254\\
0.9449449	0.1793514\\
0.9459459	0.03276747\\
0.9469469	0.01296105\\
0.9479479	0.1257987\\
0.9489489	0.3609682\\
0.9499499	0.6925215\\
0.950951	1.081151\\
0.951952	1.47822\\
0.952953	1.831364\\
0.953954	2.091211\\
0.954955	2.218482\\
0.955956	2.190472\\
0.956957	2.005829\\
0.957958	1.686591\\
0.958959	1.276784\\
0.95996	0.8373555\\
0.960961	0.4378738\\
0.961962	0.1460533\\
0.962963	0.01669652\\
0.963964	0.08192197\\
0.964965	0.3444911\\
0.965966	0.7756428\\
0.966967	1.318132\\
0.967968	1.894265\\
0.968969	2.417812\\
0.96997	2.807886\\
0.970971	3.002457\\
0.971972	2.969105\\
0.972973	2.711037\\
0.973974	2.267191\\
0.974975	1.706253\\
0.975976	1.115557\\
0.976977	0.5866987\\
0.977978	0.2003838\\
0.978979	0.01312459\\
0.97998	0.04812018\\
0.980981	0.2918881\\
0.981982	0.697189\\
0.982983	1.191656\\
0.983984	1.69053\\
0.984985	2.111186\\
0.985986	2.386864\\
0.986987	2.477214\\
0.987988	2.373886\\
0.988989	2.100356\\
0.98999	1.706239\\
0.990991	1.257324\\
0.991992	0.8233214\\
0.992993	0.4656574\\
0.993994	0.227518\\
0.994995	0.1278672\\
0.995996	0.1603088\\
0.996997	0.2967095\\
0.997998	0.4945852\\
0.998999	0.7065696\\
1	0.889962\\
};
\addlegendentry{$\text{|}\psi{}_\text{n}\text{(x,t)|}^\text{2}$};

\addplot [color=mycolor2,solid]
  table[row sep=crcr]{%
0	0.02727616\\
0.001001001	0.1377108\\
0.002002002	0.2543422\\
0.003003003	0.3726817\\
0.004004004	0.487628\\
0.005005005	0.5936624\\
0.006006006	0.6850907\\
0.007007007	0.7563248\\
0.008008008	0.8021899\\
0.009009009	0.8182424\\
0.01001001	0.8010812\\
0.01101101	0.7486316\\
0.01201201	0.6603827\\
0.01301301	0.537559\\
0.01401401	0.3832097\\
0.01501502	0.2022025\\
0.01601602	0.001113658\\
0.01701702	-0.2119875\\
0.01801802	-0.4278561\\
0.01901902	-0.6364795\\
0.02002002	-0.8275582\\
0.02102102	-0.9910372\\
0.02202202	-1.117658\\
0.02302302	-1.199499\\
0.02402402	-1.230479\\
0.02502503	-1.206776\\
0.02602603	-1.127151\\
0.02702703	-0.9931376\\
0.02802803	-0.8090857\\
0.02902903	-0.5820504\\
0.03003003	-0.3215236\\
0.03103103	-0.03901882\\
0.03203203	0.2524728\\
0.03303303	0.5391264\\
0.03403403	0.806995\\
0.03503504	1.042747\\
0.03603604	1.23439\\
0.03703704	1.371934\\
0.03803804	1.447957\\
0.03903904	1.458044\\
0.04004004	1.401055\\
0.04104104	1.279227\\
0.04204204	1.098077\\
0.04304304	0.8661362\\
0.04404404	0.5945066\\
0.04504505	0.2962795\\
0.04604605	-0.01415784\\
0.04704705	-0.3218841\\
0.04804805	-0.6122323\\
0.04904905	-0.8715582\\
0.05005005	-1.087946\\
0.05105105	-1.25181\\
0.05205205	-1.356361\\
0.05305305	-1.397912\\
0.05405405	-1.376005\\
0.05505506	-1.293365\\
0.05605606	-1.155673\\
0.05705706	-0.9711893\\
0.05805806	-0.7502416\\
0.05905906	-0.5046167\\
0.06006006	-0.2468904\\
0.06106106	0.01026265\\
0.06206206	0.2547415\\
0.06306306	0.4756342\\
0.06406406	0.6637441\\
0.06506507	0.8120006\\
0.06606607	0.9157348\\
0.06706707	0.9728109\\
0.06806807	0.9836109\\
0.06906907	0.9508802\\
0.07007007	0.8794498\\
0.07107107	0.7758549\\
0.07207207	0.6478776\\
0.07307307	0.5040434\\
0.07407407	0.3531019\\
0.07507508	0.2035221\\
0.07607608	0.06303168\\
0.07707708	-0.06177773\\
0.07807808	-0.1657573\\
0.07907908	-0.2454088\\
0.08008008	-0.2989848\\
0.08108108	-0.3264817\\
0.08208208	-0.329531\\
0.08308308	-0.3111994\\
0.08408408	-0.275715\\
0.08508509	-0.228137\\
0.08608609	-0.173992\\
0.08708709	-0.1188985\\
0.08808809	-0.06820239\\
0.08908909	-0.02664323\\
0.09009009	0.001929891\\
0.09109109	0.01477928\\
0.09209209	0.0104254\\
0.09309309	-0.01128609\\
0.09409409	-0.04919828\\
0.0950951	-0.1009348\\
0.0960961	-0.1630617\\
0.0970971	-0.2313093\\
0.0980981	-0.3008387\\
0.0990991	-0.3665385\\
0.1001001	-0.4233344\\
0.1011011	-0.4664927\\
0.1021021	-0.4919046\\
0.1031031	-0.496332\\
0.1041041	-0.4776058\\
0.1051051	-0.4347644\\
0.1061061	-0.3681259\\
0.1071071	-0.2792926\\
0.1081081	-0.1710861\\
0.1091091	-0.04741842\\
0.1101101	0.08689324\\
0.1111111	0.2263593\\
0.1121121	0.3650841\\
0.1131131	0.4970526\\
0.1141141	0.6164217\\
0.1151151	0.7178026\\
0.1161161	0.7965184\\
0.1171171	0.8488242\\
0.1181181	0.872081\\
0.1191191	0.8648725\\
0.1201201	0.8270622\\
0.1211211	0.7597878\\
0.1221221	0.6653951\\
0.1231231	0.5473165\\
0.1241241	0.4099008\\
0.1251251	0.2582063\\
0.1261261	0.09776639\\
0.1271271	-0.06565665\\
0.1281281	-0.2263188\\
0.1291291	-0.3787284\\
0.1301301	-0.5178569\\
0.1311311	-0.639316\\
0.1321321	-0.7394949\\
0.1331331	-0.8156538\\
0.1341341	-0.8659709\\
0.1351351	-0.8895468\\
0.1361361	-0.8863685\\
0.1371371	-0.8572409\\
0.1381381	-0.8036922\\
0.1391391	-0.727864\\
0.1401401	-0.6323931\\
0.1411411	-0.5202942\\
0.1421421	-0.3948514\\
0.1431431	-0.2595217\\
0.1441441	-0.1178558\\
0.1451451	0.02656408\\
0.1461461	0.1701711\\
0.1471471	0.3094477\\
0.1481481	0.4409531\\
0.1491491	0.5613486\\
0.1501502	0.6674281\\
0.1511512	0.7561595\\
0.1521522	0.82474\\
0.1531532	0.8706694\\
0.1541542	0.8918404\\
0.1551552	0.8866437\\
0.1561562	0.8540831\\
0.1571572	0.7938918\\
0.1581582	0.706642\\
0.1591592	0.5938357\\
0.1601602	0.4579655\\
0.1611612	0.3025361\\
0.1621622	0.1320364\\
0.1631632	-0.0481449\\
0.1641642	-0.2318607\\
0.1651652	-0.4124061\\
0.1661662	-0.5827706\\
0.1671672	-0.7359335\\
0.1681682	-0.8651898\\
0.1691692	-0.9644875\\
0.1701702	-1.02876\\
0.1711712	-1.05423\\
0.1721722	-1.03867\\
0.1731732	-0.9815911\\
0.1741742	-0.8843585\\
0.1751752	-0.7502059\\
0.1761762	-0.5841559\\
0.1771772	-0.3928379\\
0.1781782	-0.1842121\\
0.1791792	0.0327896\\
0.1801802	0.2486879\\
0.1811812	0.4539238\\
0.1821822	0.6393445\\
0.1831832	0.7966787\\
0.1841842	0.9189746\\
0.1851852	1.000974\\
0.1861862	1.039398\\
0.1871872	1.033128\\
0.1881882	0.9832672\\
0.1891892	0.89308\\
0.1901902	0.7678061\\
0.1911912	0.6143654\\
0.1921922	0.4409654\\
0.1931932	0.256635\\
0.1941942	0.07071253\\
0.1951952	-0.1076816\\
0.1961962	-0.2701567\\
0.1971972	-0.4095156\\
0.1981982	-0.5201433\\
0.1991992	-0.5982966\\
0.2002002	-0.642276\\
0.2012012	-0.6524703\\
0.2022022	-0.6312716\\
0.2032032	-0.5828659\\
0.2042042	-0.5129126\\
0.2052052	-0.4281351\\
0.2062062	-0.3358472\\
0.2072072	-0.2434475\\
0.2082082	-0.1579129\\
0.2092092	-0.08532638\\
0.2102102	-0.03046756\\
0.2112112	0.003505824\\
0.2122122	0.01526477\\
0.2132132	0.00539057\\
0.2142142	-0.0236748\\
0.2152152	-0.06780654\\
0.2162162	-0.1214921\\
0.2172172	-0.1782246\\
0.2182182	-0.2309692\\
0.2192192	-0.2726725\\
0.2202202	-0.2967815\\
0.2212212	-0.2977385\\
0.2222222	-0.2714186\\
0.2232232	-0.215482\\
0.2242242	-0.1296156\\
0.2252252	-0.01564766\\
0.2262262	0.1224735\\
0.2272272	0.2788383\\
0.2282282	0.4458628\\
0.2292292	0.6146901\\
0.2302302	0.7756808\\
0.2312312	0.9189657\\
0.2322322	1.035027\\
0.2332332	1.115276\\
0.2342342	1.152587\\
0.2352352	1.141767\\
0.2362362	1.079916\\
0.2372372	0.9666807\\
0.2382382	0.8043527\\
0.2392392	0.5978406\\
0.2402402	0.3544885\\
0.2412412	0.08376361\\
0.2422422	-0.2031741\\
0.2432432	-0.4939985\\
0.2442442	-0.7758116\\
0.2452452	-1.035774\\
0.2462462	-1.261742\\
0.2472472	-1.442871\\
0.2482482	-1.570163\\
0.2492492	-1.636928\\
0.2502503	-1.639129\\
0.2512513	-1.575606\\
0.2522523	-1.448157\\
0.2532533	-1.26148\\
0.2542543	-1.022974\\
0.2552553	-0.7424112\\
0.2562563	-0.4315003\\
0.2572573	-0.1033518\\
0.2582583	0.2281189\\
0.2592593	0.5488277\\
0.2602603	0.845166\\
0.2612613	1.104623\\
0.2622623	1.316359\\
0.2632633	1.471695\\
0.2642643	1.564511\\
0.2652653	1.591517\\
0.2662663	1.552398\\
0.2672673	1.449814\\
0.2682683	1.289266\\
0.2692693	1.078823\\
0.2702703	0.8287339\\
0.2712713	0.550929\\
0.2722723	0.2584453\\
0.2732733	-0.03520216\\
0.2742743	-0.3166716\\
0.2752753	-0.5734368\\
0.2762763	-0.7943876\\
0.2772773	-0.9703592\\
0.2782783	-1.094563\\
0.2792793	-1.162895\\
0.2802803	-1.174107\\
0.2812813	-1.129828\\
0.2822823	-1.034439\\
0.2832833	-0.8947981\\
0.2842843	-0.7198422\\
0.2852853	-0.520074\\
0.2862863	-0.3069702\\
0.2872873	-0.09234123\\
0.2882883	0.112323\\
0.2892893	0.2964825\\
0.2902903	0.4511163\\
0.2912913	0.5691992\\
0.2922923	0.6460566\\
0.2932933	0.6795786\\
0.2942943	0.6702792\\
0.2952953	0.6211988\\
0.2962963	0.5376557\\
0.2972973	0.4268625\\
0.2982983	0.2974323\\
0.2992993	0.158805\\
0.3003003	0.02063176\\
0.3013013	-0.1078451\\
0.3023023	-0.2183754\\
0.3033033	-0.304211\\
0.3043043	-0.3605109\\
0.3053053	-0.3846135\\
0.3063063	-0.3761623\\
0.3073073	-0.3370756\\
0.3083083	-0.2713649\\
0.3093093	-0.1848155\\
0.3103103	-0.08455112\\
0.3113113	0.02148508\\
0.3123123	0.1250991\\
0.3133133	0.2184237\\
0.3143143	0.2944654\\
0.3153153	0.3475847\\
0.3163163	0.3738754\\
0.3173173	0.3714193\\
0.3183183	0.3403992\\
0.3193193	0.2830634\\
0.3203203	0.2035468\\
0.3213213	0.107561\\
0.3223223	0.00197749\\
0.3233233	-0.1056667\\
0.3243243	-0.2077053\\
0.3253253	-0.296883\\
0.3263263	-0.3668635\\
0.3273273	-0.4126677\\
0.3283283	-0.431015\\
0.3293293	-0.4205441\\
0.3303303	-0.3818984\\
0.3313313	-0.3176736\\
0.3323323	-0.2322313\\
0.3333333	-0.1313928\\
0.3343343	-0.0220367\\
0.3353353	0.08837248\\
0.3363363	0.1922912\\
0.3373373	0.2826052\\
0.3383383	0.3531047\\
0.3393393	0.3988957\\
0.3403403	0.4167216\\
0.3413413	0.405174\\
0.3423423	0.3647829\\
0.3433433	0.29798\\
0.3443443	0.2089418\\
0.3453453	0.103323\\
0.3463463	-0.01210042\\
0.3473473	-0.1298535\\
0.3483483	-0.2422249\\
0.3493493	-0.3417387\\
0.3503504	-0.4216076\\
0.3513514	-0.4761407\\
0.3523524	-0.5010836\\
0.3533534	-0.4938714\\
0.3543544	-0.4537829\\
0.3553554	-0.3819897\\
0.3563564	-0.2815003\\
0.3573574	-0.1570032\\
0.3583584	-0.01462221\\
0.3593594	0.1384022\\
0.3603604	0.2940874\\
0.3613614	0.4441178\\
0.3623624	0.5802675\\
0.3633634	0.6948162\\
0.3643644	0.7809373\\
0.3653654	0.8330419\\
0.3663664	0.8470625\\
0.3673674	0.8206651\\
0.3683684	0.7533817\\
0.3693694	0.6466589\\
0.3703704	0.5038195\\
0.3713714	0.3299433\\
0.3723724	0.1316675\\
0.3733734	-0.08307914\\
0.3743744	-0.3054027\\
0.3753754	-0.5258143\\
0.3763764	-0.7346317\\
0.3773774	-0.9223948\\
0.3783784	-1.08028\\
0.3793794	-1.200495\\
0.3803804	-1.276639\\
0.3813814	-1.304012\\
0.3823824	-1.279857\\
0.3833834	-1.203533\\
0.3843844	-1.076588\\
0.3853854	-0.9027571\\
0.3863864	-0.6878509\\
0.3873874	-0.4395605\\
0.3883884	-0.1671719\\
0.3893894	0.1187977\\
0.3903904	0.4070256\\
0.3913914	0.6858569\\
0.3923924	0.9438028\\
0.3933934	1.170045\\
0.3943944	1.354921\\
0.3953954	1.490362\\
0.3963964	1.570278\\
0.3973974	1.590848\\
0.3983984	1.550719\\
0.3993994	1.45109\\
0.4004004	1.295682\\
0.4014014	1.090592\\
0.4024024	0.8440319\\
0.4034034	0.565969\\
0.4044044	0.2676843\\
0.4054054	-0.03873272\\
0.4064064	-0.3409286\\
0.4074074	-0.626839\\
0.4084084	-0.8852277\\
0.4094094	-1.106181\\
0.4104104	-1.281536\\
0.4114114	-1.405217\\
0.4124124	-1.473472\\
0.4134134	-1.484991\\
0.4144144	-1.440902\\
0.4154154	-1.344654\\
0.4164164	-1.201787\\
0.4174174	-1.019614\\
0.4184184	-0.8068125\\
0.4194194	-0.5729822\\
0.4204204	-0.3281596\\
0.4214214	-0.08233868\\
0.4224224	0.1549842\\
0.4234234	0.3752209\\
0.4244244	0.5710296\\
0.4254254	0.7365742\\
0.4264264	0.867686\\
0.4274274	0.9619316\\
0.4284284	1.018581\\
0.4294294	1.038494\\
0.4304304	1.023922\\
0.4314314	0.9782588\\
0.4324324	0.9057511\\
0.4334334	0.8111882\\
0.4344344	0.6995963\\
0.4354354	0.5759593\\
0.4364364	0.4449739\\
0.4374374	0.3108661\\
0.4384384	0.1772622\\
0.4394394	0.0471325\\
0.4404404	-0.07720748\\
0.4414414	-0.1940319\\
0.4424424	-0.3021021\\
0.4434434	-0.4005232\\
0.4444444	-0.4885975\\
0.4454454	-0.5656699\\
0.4464464	-0.6310049\\
0.4474474	-0.6836877\\
0.4484484	-0.7225835\\
0.4494494	-0.7463381\\
0.4504505	-0.7534481\\
0.4514515	-0.7423684\\
0.4524525	-0.7116805\\
0.4534535	-0.6602737\\
0.4544545	-0.5875604\\
0.4554555	-0.4936653\\
0.4564565	-0.3796125\\
0.4574575	-0.2474446\\
0.4584585	-0.1003068\\
0.4594595	0.05757006\\
0.4604605	0.2209387\\
0.4614615	0.3837192\\
0.4624625	0.5392269\\
0.4634635	0.6805059\\
0.4644645	0.8006714\\
0.4654655	0.8933263\\
0.4664665	0.952932\\
0.4674675	0.9752115\\
0.4684685	0.9574407\\
0.4694695	0.8987364\\
0.4704705	0.8001695\\
0.4714715	0.6648616\\
0.4724725	0.4978586\\
0.4734735	0.3060094\\
0.4744745	0.09759194\\
0.4754755	-0.1179876\\
0.4764765	-0.3307702\\
0.4774775	-0.5306231\\
0.4784785	-0.7079376\\
0.4794795	-0.8539309\\
0.4804805	-0.9614019\\
0.4814815	-1.024735\\
0.4824825	-1.040786\\
0.4834835	-1.008115\\
0.4844845	-0.9289277\\
0.4854855	-0.8016453\\
0.4864865	-0.6021648\\
0.4874875	-0.3439592\\
0.4884885	-0.05287701\\
0.4894895	0.2434183\\
0.4904905	0.5162044\\
0.4914915	0.7394525\\
0.4924925	0.8917522\\
0.4934935	0.9587374\\
0.4944945	0.9341569\\
0.4954955	0.8206219\\
0.4964965	0.6291497\\
0.4974975	0.3781647\\
0.4984985	0.09160857\\
0.4994995	-0.20331\\
0.5005005	-0.4787541\\
0.5015015	-0.708859\\
0.5025025	-0.8721535\\
0.5035035	-0.9535351\\
0.5045045	-0.9456059\\
0.5055055	-0.8493307\\
0.5065065	-0.6738267\\
0.5075075	-0.4355332\\
0.5085085	-0.1565436\\
0.5095095	0.1373225\\
0.5105105	0.4190832\\
0.5115115	0.6628036\\
0.5125125	0.846295\\
0.5135135	0.9525629\\
0.5145145	0.972195\\
0.5155155	0.9024254\\
0.5165165	0.7542511\\
0.5175175	0.5689675\\
0.5185185	0.3594289\\
0.5195195	0.1328726\\
0.5205205	-0.101549\\
0.5215215	-0.3349978\\
0.5225225	-0.5583264\\
0.5235235	-0.7629764\\
0.5245245	-0.9408971\\
0.5255255	-1.085106\\
0.5265265	-1.189731\\
0.5275275	-1.250386\\
0.5285285	-1.26421\\
0.5295295	-1.230093\\
0.5305305	-1.148644\\
0.5315315	-1.02228\\
0.5325325	-0.8550944\\
0.5335335	-0.6528004\\
0.5345345	-0.4224987\\
0.5355355	-0.1724819\\
0.5365365	0.08809106\\
0.5375375	0.3494956\\
0.5385385	0.6018346\\
0.5395395	0.8353965\\
0.5405405	1.041065\\
0.5415415	1.210675\\
0.5425425	1.337376\\
0.5435435	1.415914\\
0.5445445	1.442894\\
0.5455455	1.416924\\
0.5465465	1.338721\\
0.5475475	1.211085\\
0.5485485	1.038818\\
0.5495495	0.8285277\\
0.5505506	0.588362\\
0.5515516	0.3276649\\
0.5525526	0.05658561\\
0.5535536	-0.2143641\\
0.5545546	-0.4747548\\
0.5555556	-0.7146947\\
0.5565566	-0.9252463\\
0.5575576	-1.098808\\
0.5585586	-1.229422\\
0.5595596	-1.313008\\
0.5605606	-1.347505\\
0.5615616	-1.332919\\
0.5625626	-1.271263\\
0.5635636	-1.166421\\
0.5645646	-1.023913\\
0.5655656	-0.8505927\\
0.5665666	-0.6542946\\
0.5675676	-0.4434445\\
0.5685686	-0.2266542\\
0.5695696	-0.01232987\\
0.5705706	0.1916978\\
0.5715716	0.3784899\\
0.5725726	0.5422595\\
0.5735736	0.6785541\\
0.5745746	0.7843621\\
0.5755756	0.8581342\\
0.5765766	0.8997309\\
0.5775776	0.9102956\\
0.5785786	0.892073\\
0.5795796	0.8481816\\
0.5805806	0.782362\\
0.5815816	0.6987157\\
0.5825826	0.6014555\\
0.5835836	0.4946796\\
0.5845846	0.3821885\\
0.5855856	0.2673479\\
0.5865866	0.1530104\\
0.5875876	0.04149091\\
0.5885886	-0.06540257\\
0.5895896	-0.1662951\\
0.5905906	-0.2601347\\
0.5915916	-0.3460653\\
0.5925926	-0.4232932\\
0.5935936	-0.4909671\\
0.5945946	-0.5480835\\
0.5955956	-0.5934329\\
0.5965966	-0.6255908\\
0.5975976	-0.6429646\\
0.5985986	-0.643889\\
0.5995996	-0.6267696\\
0.6006006	-0.5902609\\
0.6016016	-0.5334665\\
0.6026026	-0.4561418\\
0.6036036	-0.3588822\\
0.6046046	-0.2432742\\
0.6056056	-0.1119951\\
0.6066066	0.03115787\\
0.6076076	0.1813165\\
0.6086086	0.3326758\\
0.6096096	0.4787112\\
0.6106106	0.6124717\\
0.6116116	0.7269347\\
0.6126126	0.8154028\\
0.6136136	0.8719183\\
0.6146146	0.8916695\\
0.6156156	0.8713586\\
0.6166166	0.8095065\\
0.6176176	0.7066689\\
0.6186186	0.5655443\\
0.6196196	0.3909616\\
0.6206206	0.1897416\\
0.6216216	-0.02956625\\
0.6226226	-0.2570595\\
0.6236236	-0.4819541\\
0.6246246	-0.6931247\\
0.6256256	-0.8796862\\
0.6266266	-1.031581\\
0.6276276	-1.140138\\
0.6286286	-1.198573\\
0.6296296	-1.202396\\
0.6306306	-1.149702\\
0.6316316	-1.041329\\
0.6326326	-0.8808728\\
0.6336336	-0.674552\\
0.6346346	-0.4309382\\
0.6356356	-0.1605578\\
0.6366366	0.1246075\\
0.6376376	0.4116963\\
0.6386386	0.6875873\\
0.6396396	0.9395308\\
0.6406406	1.155759\\
0.6416416	1.326044\\
0.6426426	1.442173\\
0.6436436	1.498326\\
0.6446446	1.491332\\
0.6456456	1.420798\\
0.6466466	1.289106\\
0.6476476	1.101285\\
0.6486486	0.864768\\
0.6496496	0.5890443\\
0.6506507	0.2852322\\
0.6516517	-0.03440888\\
0.6526527	-0.3570017\\
0.6536537	-0.6695879\\
0.6546547	-0.9596534\\
0.6556557	-1.215623\\
0.6566567	-1.427303\\
0.6576577	-1.586263\\
0.6586587	-1.686138\\
0.6596597	-1.722844\\
0.6606607	-1.694708\\
0.6616617	-1.602505\\
0.6626627	-1.449413\\
0.6636637	-1.24087\\
0.6646647	-0.9843699\\
0.6656657	-0.6891801\\
0.6666667	-0.3660031\\
0.6676677	-0.02659295\\
0.6686687	0.3166648\\
0.6696697	0.6511926\\
0.6706707	0.9646694\\
0.6716717	1.245472\\
0.6726727	1.483097\\
0.6736737	1.668549\\
0.6746747	1.794677\\
0.6756757	1.85646\\
0.6766767	1.851215\\
0.6776777	1.778724\\
0.6786787	1.641282\\
0.6796797	1.44364\\
0.6806807	1.192869\\
0.6816817	0.8981189\\
0.6826827	0.5703039\\
0.6836837	0.2217009\\
0.6846847	-0.1345097\\
0.6856857	-0.4847529\\
0.6866867	-0.815596\\
0.6876877	-1.114292\\
0.6886887	-1.369307\\
0.6896897	-1.570809\\
0.6906907	-1.711096\\
0.6916917	-1.784937\\
0.6926927	-1.789819\\
0.6936937	-1.726073\\
0.6946947	-1.596884\\
0.6956957	-1.408169\\
0.6966967	-1.16833\\
0.6976977	-0.8878964\\
0.6986987	-0.5790609\\
0.6996997	-0.2551369\\
0.7007007	0.07004118\\
0.7017017	0.3827467\\
0.7027027	0.6699876\\
0.7037037	0.920101\\
0.7047047	1.123286\\
0.7057057	1.272045\\
0.7067067	1.361517\\
0.7077077	1.389675\\
0.7087087	1.357391\\
0.7097097	1.268343\\
0.7107107	1.128798\\
0.7117117	0.9472521\\
0.7127127	0.733966\\
0.7137137	0.5004168\\
0.7147147	0.2586889\\
0.7157157	0.02084384\\
0.7167167	-0.2016996\\
0.7177177	-0.3987414\\
0.7187187	-0.5618008\\
0.7197197	-0.6845151\\
0.7207207	-0.7629151\\
0.7217217	-0.7955641\\
0.7227227	-0.7835534\\
0.7237237	-0.7303578\\
0.7247247	-0.6415601\\
0.7257257	-0.5244644\\
0.7267267	-0.3876202\\
0.7277277	-0.2402886\\
0.7287287	-0.0918815\\
0.7297297	0.0485918\\
0.7307307	0.1730381\\
0.7317317	0.2747246\\
0.7327327	0.3486282\\
0.7337337	0.3916719\\
0.7347347	0.4028348\\
0.7357357	0.383135\\
0.7367367	0.3354864\\
0.7377377	0.264444\\
0.7387387	0.1758551\\
0.7397397	0.07644202\\
0.7407407	-0.0266574\\
0.7417417	-0.1263591\\
0.7427427	-0.216083\\
0.7437437	-0.290156\\
0.7447447	-0.3441419\\
0.7457457	-0.3750812\\
0.7467467	-0.381625\\
0.7477477	-0.3640607\\
0.7487487	-0.3242284\\
0.7497497	-0.2653391\\
0.7507508	-0.1917075\\
0.7517518	-0.1084228\\
0.7527528	-0.02097899\\
0.7537538	0.06510749\\
0.7547548	0.1446673\\
0.7557558	0.2132095\\
0.7567568	0.2671938\\
0.7577578	0.3042267\\
0.7587588	0.3231703\\
0.7597598	0.3241597\\
0.7607608	0.3085303\\
0.7617618	0.2786637\\
0.7627628	0.2377653\\
0.7637638	0.1895925\\
0.7647648	0.1381538\\
0.7657658	0.08740288\\
0.7667668	0.04094791\\
0.7677678	0.001799305\\
0.7687688	-0.02782868\\
0.7697698	-0.0466486\\
0.7707708	-0.05435528\\
0.7717718	-0.05160435\\
0.7727728	-0.03991758\\
0.7737738	-0.02152445\\
0.7747748	0.000846461\\
0.7757758	0.02420867\\
0.7767768	0.04557245\\
0.7777778	0.06219284\\
0.7787788	0.07179203\\
0.7797798	0.07273929\\
0.7807808	0.06417547\\
0.7817818	0.04607318\\
0.7827828	0.01922945\\
0.7837838	-0.0148077\\
0.7847848	-0.05387163\\
0.7857858	-0.09535727\\
0.7867868	-0.136434\\
0.7877878	-0.1742739\\
0.7887888	-0.2062776\\
0.7897898	-0.2302788\\
0.7907908	-0.2447128\\
0.7917918	-0.2487339\\
0.7927928	-0.2422756\\
0.7937938	-0.2260464\\
0.7947948	-0.2014637\\
0.7957958	-0.1705303\\
0.7967968	-0.135665\\
0.7977978	-0.09949936\\
0.7987988	-0.06465787\\
0.7997998	-0.03353905\\
0.8008008	-0.008114759\\
0.8018018	0.01023614\\
0.8028028	0.02084628\\
0.8038038	0.02382087\\
0.8048048	0.02003596\\
0.8058058	0.01107359\\
0.8068068	-0.0009017873\\
0.8078078	-0.01331665\\
0.8088088	-0.02339865\\
0.8098098	-0.02840808\\
0.8108108	-0.02587314\\
0.8118118	-0.01381135\\
0.8128128	0.009080013\\
0.8138138	0.04327886\\
0.8148148	0.08835027\\
0.8158158	0.1429227\\
0.8168168	0.2047315\\
0.8178178	0.2707273\\
0.8188188	0.3372432\\
0.8198198	0.4002095\\
0.8208208	0.4554028\\
0.8218218	0.4987114\\
0.8228228	0.5264022\\
0.8238238	0.5353686\\
0.8248248	0.5233451\\
0.8258258	0.489075\\
0.8268268	0.4324183\\
0.8278278	0.3543954\\
0.8288288	0.2571626\\
0.8298298	0.143922\\
0.8308308	0.01877127\\
0.8318318	-0.1134976\\
0.8328328	-0.2476369\\
0.8338338	-0.3782054\\
0.8348348	-0.4998387\\
0.8358358	-0.6075117\\
0.8368368	-0.6967782\\
0.8378378	-0.763976\\
0.8388388	-0.8063853\\
0.8398398	-0.8223351\\
0.8408408	-0.8112525\\
0.8418418	-0.7736555\\
0.8428428	-0.7110922\\
0.8438438	-0.6260317\\
0.8448448	-0.5217176\\
0.8458458	-0.4019922\\
0.8468468	-0.2711046\\
0.8478478	-0.1335138\\
0.8488488	0.006301058\\
0.8498498	0.1440145\\
0.8508509	0.2756027\\
0.8518519	0.3974607\\
0.8528529	0.5064856\\
0.8538539	0.6001243\\
0.8548549	0.6763884\\
0.8558559	0.7338405\\
0.8568569	0.7715579\\
0.8578579	0.7890819\\
0.8588589	0.7863592\\
0.8598599	0.7636856\\
0.8608609	0.7216578\\
0.8618619	0.6611391\\
0.8628629	0.5832434\\
0.8638639	0.4893378\\
0.8648649	0.3810634\\
0.8658659	0.2603696\\
0.8668669	0.1295559\\
0.8678679	-0.008687001\\
0.8688689	-0.1512456\\
0.8698699	-0.2945726\\
0.8708709	-0.4347035\\
0.8718719	-0.5673111\\
0.8728729	-0.6877997\\
0.8738739	-0.791445\\
0.8748749	-0.8735742\\
0.8758759	-0.9297842\\
0.8768769	-0.9561848\\
0.8778779	-0.9496571\\
0.8788789	-0.9081076\\
0.8798799	-0.8307027\\
0.8808809	-0.7180625\\
0.8818819	-0.5723962\\
0.8828829	-0.3975639\\
0.8838839	-0.1990505\\
0.8848849	0.01615562\\
0.8858859	0.2397808\\
0.8868869	0.4625778\\
0.8878879	0.67472\\
0.8888889	0.8662603\\
0.8898899	1.027633\\
0.8908909	1.15017\\
0.8918919	1.226605\\
0.8928929	1.251526\\
0.8938939	1.221757\\
0.8948949	1.136638\\
0.8958959	0.9981741\\
0.8968969	0.8110497\\
0.8978979	0.5824958\\
0.8988989	0.32201\\
0.8998999	0.04094412\\
0.9009009	-0.2480252\\
0.9019019	-0.5315128\\
0.9029029	-0.7960969\\
0.9039039	-1.02901\\
0.9049049	-1.218807\\
0.9059059	-1.355978\\
0.9069069	-1.433458\\
0.9079079	-1.447023\\
0.9089089	-1.395527\\
0.9099099	-1.280979\\
0.9109109	-1.108457\\
0.9119119	-0.8858518\\
0.9129129	-0.6234635\\
0.9139139	-0.3334749\\
0.9149149	-0.02932513\\
0.9159159	0.2749748\\
0.9169169	0.5655501\\
0.9179179	0.8293524\\
0.9189189	1.054796\\
0.9199199	1.232306\\
0.9209209	1.354749\\
0.9219219	1.417731\\
0.9229229	1.419743\\
0.9239239	1.362151\\
0.9249249	1.249049\\
0.9259259	1.086967\\
0.9269269	0.8844717\\
0.9279279	0.651674\\
0.9289289	0.3996839\\
0.9299299	0.1400334\\
0.9309309	-0.1158939\\
0.9319319	-0.3573986\\
0.9329329	-0.5749409\\
0.9339339	-0.7605333\\
0.9349349	-0.9080339\\
0.9359359	-1.013332\\
0.9369369	-1.074424\\
0.9379379	-1.091372\\
0.9389389	-1.066178\\
0.9399399	-1.002555\\
0.9409409	-0.9056405\\
0.9419419	-0.781656\\
0.9429429	-0.6375386\\
0.9439439	-0.4805665\\
0.9449449	-0.3179963\\
0.9459459	-0.156731\\
0.9469469	-0.003031976\\
0.9479479	0.1377128\\
0.9489489	0.261158\\
0.9499499	0.3641046\\
0.950951	0.4445382\\
0.951952	0.5016069\\
0.952953	0.5355453\\
0.953954	0.5475528\\
0.954955	0.5396395\\
0.955956	0.5144475\\
0.956957	0.4750628\\
0.957958	0.424825\\
0.958959	0.3671471\\
0.95996	0.3053518\\
0.960961	0.2425315\\
0.961962	0.1814348\\
0.962963	0.1243837\\
0.963964	0.07321922\\
0.964965	0.02927701\\
0.965966	-0.006612358\\
0.966967	-0.03410033\\
0.967968	-0.05327952\\
0.968969	-0.0646296\\
0.96997	-0.06895428\\
0.970971	-0.06731366\\
0.971972	-0.06095547\\
0.972973	-0.0512482\\
0.973974	-0.03961815\\
0.974975	-0.02749175\\
0.975976	-0.01624386\\
0.976977	-0.007151931\\
0.977978	-0.001355734\\
0.978979	0.0001780726\\
0.97998	-0.003312713\\
0.980981	-0.01235803\\
0.981982	-0.02723092\\
0.982983	-0.04792628\\
0.983984	-0.07414349\\
0.984985	-0.1052742\\
0.985986	-0.1403971\\
0.986987	-0.178281\\
0.987988	-0.2173996\\
0.988989	-0.2559592\\
0.98999	-0.2919422\\
0.990991	-0.3231671\\
0.991992	-0.3473659\\
0.992993	-0.3622796\\
0.993994	-0.3657671\\
0.994995	-0.355928\\
0.995996	-0.3312315\\
0.996997	-0.2906464\\
0.997998	-0.233765\\
0.998999	-0.1609118\\
1	-0.07322808\\
};
\addlegendentry{$\text{imag(}\psi{}_\text{n}\text{(x,t))}$};

\addplot [color=mycolor3,solid]
  table[row sep=crcr]{%
0.4834835	0\\
0.4834835	3.539679\\
};
\addlegendentry{Potential Well};

\addplot [color=mycolor3,solid,forget plot]
  table[row sep=crcr]{%
0.5155155	0\\
0.5155155	3.539679\\
};
\end{axis}
\end{tikzpicture}%
		\caption{The numerical solution of $|\psi_n(x,t)|^2$ after it has hit the potential well.}
		\label{fig:potWellfPlot}
	\end{subfigure}
	\begin{subfigure}{.9\linewidth}
		\setlength\figureheight{.5\linewidth}
		\setlength\figurewidth{.9\linewidth}
		\input{../src/plots/potWellTR.tikz}
		\caption{Here we clearly see the transmission and reflection before and after the wave function hits the potential well.}
		\label{fig:potWellTR}
	\end{subfigure}
	\label{fig:potWell}
	\caption{Here we depict the effect of the wave function hitting the potential well.}
\end{figure}

\newpage
\section{Square Barrier}
In figure \ref{fig:squareBarrTR} we can see the effect of the square barrier on the wave function. After the wave function hits the square barrier part of it is trapped a while before a large part is reflected and a small part tunnels trough. Here we also had the timestep on $1\cdot10^{-7}$

\begin{figure}[H]
	\centering
	\begin{subfigure}{.9\linewidth}
		\includegraphics[width=1\textwidth]{../src/plots/squareBarrPlot.png}
		\caption{The numerical solution of $|\psi_n(x,t)|^2$ after it has hit the square barrier.}
		\label{fig:squareBarrfPlot}
	\end{subfigure}
	\begin{subfigure}{.9\linewidth}
		\setlength\figureheight{.5\linewidth}
		\setlength\figurewidth{.9\linewidth}
		\includegraphics[width=1\textwidth]{../src/plots/squareBarrTR.png}
		\caption{Here we clearly see the transmission and reflection before and after the wave function hits the square barrier.}
		\label{fig:squareBarrTR}
	\end{subfigure}
	\label{fig:squareBarr}
	\caption{Here we depict the effect of the wave function hitting the square barrier. Note in figure \ref{fig:squareBarrTR} how a certain part of the wave function is trapped inside for while before a large part is reflected and a small part tunnels through.}
\end{figure}


\section{The effect of changing $k_0$}
In figure \ref{fig:k0} we can se how the constant $k_0$ affect the transmission and reflection of the wave function. It's clear that with a larger $k_0$ the transmission seem to approach nil while thee the reflection then of course are approaching total reflection.
\begin{figure}[H]
	\centering
	\setlength\figureheight{.5\linewidth}
	\setlength\figurewidth{.9\linewidth}
	\input{../src/plots/trvk0Plot.tikz}
	\caption{The reflektion and transmission through the potential well depending on the value of $k_0$}
	\label{fig:k0}
\end{figure}


\end{document}
