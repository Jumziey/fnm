\documentclass[11pt]{article}
\usepackage{report}
\newif\iftikz
\tikztrue

\begin{document}
\section{Exercise 1}
\subsection{How to run the code}
In the directory \verb+~jeve0010/Documents/jesper/fnm/lab3/ex1+ the written code resides. To execute the code simply write \verb+make+ in the console to be sure to use the updated code, then execute it by writing \verb+./ex1+ in the console.

\subsection{Result}
In table \ref{table:ex1Res} we can see the obtained results for \verb+top1.o+.
\begin{table}[h]\center\Large
  \caption{The obtained obtained values}
  \begin{tabular}{c|l}
  M & 100.995g \\ \hline
  l & 3.00630cm \\ \hline
  $I_1$ & 1215.00 gcm$^2$ \\ \hline
  $I_3$ & 482.568 gcm$^2$ 
  \end{tabular}
  \label{table:ex1Res}
\end{table}

\section{Exercise 2}
In this exercise we start with rewriting equations (21)-(23) from the lab instructions to six ordinary differential equations. The resulting equations are
\begin{align}
	\dot{\phi} &= q \\
	\dot{\psi} &= r \\
	\dot{\theta} &= s \\
	\dot{q} &= \frac{I_3}{I_1} \Bigg[ \dot{\phi}\dot{\theta}\cot{\theta} + \frac{\dot{\psi}\dot{\theta}}{\sin{\theta}} \Bigg] - 2 \dot{\theta}\dot{\psi}\cot{\theta} \\
	\dot{r} &= \dot{\psi}\dot{\theta}\sin{\theta} + 2\dot{\theta}\dot{\psi}\cot{\theta}\cos{\theta} - \frac{I_3}{I_1} \Bigg[\dot{\psi}\dot{\theta}\cos{\theta} + \dot{\psi}\dot{\theta}\Bigg]\cot{\theta}\\
	\dot{s} &= \frac{Mgl}{I_1}\sin{\theta} + \dot{\phi}^2\sin{\theta}\cos{\theta}-\frac{I_3}{I_1}\Big(\dot{\psi}+\dot{\phi}\cos{\theta}\Big)\dot{\psi}\sin{\theta}
\end{align}
The function implementing the Lagrangian ode is declared as
\begin{lstlisting}
int
topLagrange(double time, const double v[], double d[], void *params)
\end{lstlisting}
And the function code exists in the file 
\begin{verbatim}
~jeve0010/Documents/jesper/fnm/lab3/ex2/ex2.c.
\end{verbatim}
\subsection{Running the Code}
To run the code simply run the \verb+make+ command inside the directory 
\begin{verbatim}
~jeve0010/Documents/jesper/fnm/lab3/ex2+
\end{verbatim}
 then run the program with the command \verb+./ex2+. If one wants to obtain the plots showed in this document simply execute the matlab script 
\begin{verbatim}
 ~jeve0010/Documents/jesper/fnm/lab3/ex2/plots/code/topSpin.m.
\end{verbatim}
Note that the resulting tikz file produced from that script is used in this document, not the plots that matlab create itself.


\subsection{Results}
In figure \ref{fig:forwardData} and \ref{fig:backwardData} we can se the evolution of $\psi$, $\phi$ and $\theta$ going from zero to four seconds and then backward to zero again. The error is plotted in the same plot, but is clearly so low that another error plot is required to get a clearer picture of the error evolution. In figure \ref{fig:forwardDataError} and \ref{fig:backwardDataError} we can see the error when going forward and backwards in the time evolution. In the legend $c1$ and $c2$ stand for the constants given by equation 18 and 20 from the lab instructions. These errors are clearly negligebly small since it is about 10 orders of magnitude smaller then the values obtain for $\psi$, $\phi$ and $\theta$. The time step of $2\cdot10^{-3}$ was used with the \verb+gsl_odeiv2_step_rkf45+ function to get this result.
\iftikz
\begin{figure}[H]
	\centering
	\newlength\figureheight 
	\newlength\figurewidth 
	\setlength\figureheight{7cm} 
	\setlength\figurewidth{14cm}
	% This file was created by matlab2tikz.
% Minimal pgfplots version: 1.3
%
%The latest updates can be retrieved from
%  http://www.mathworks.com/matlabcentral/fileexchange/22022-matlab2tikz
%where you can also make suggestions and rate matlab2tikz.
%
\definecolor{mycolor1}{rgb}{0.00000,0.44700,0.74100}%
\definecolor{mycolor2}{rgb}{0.85000,0.32500,0.09800}%
\definecolor{mycolor3}{rgb}{0.92900,0.69400,0.12500}%
\definecolor{mycolor4}{rgb}{0.49400,0.18400,0.55600}%
\definecolor{mycolor5}{rgb}{0.46600,0.67400,0.18800}%
\definecolor{mycolor6}{rgb}{0.30100,0.74500,0.93300}%
%
\begin{tikzpicture}

\begin{axis}[%
width=0.95092\figurewidth,
height=\figureheight,
at={(0\figurewidth,0\figureheight)},
scale only axis,
xmin=0,
xmax=4,
xlabel={Time (s)},
ymin=-8.73115e-11,
ymax=40,
ylabel={Degrees},
title style={font=\bfseries},
title={Top Spin [0,4] (s)},
legend style={at={(0.03,0.97)},anchor=north west,legend cell align=left,align=left,draw=white!15!black},
title style={font=\small},ticklabel style={font=\tiny}
]
\addplot [color=mycolor1,solid]
  table[row sep=crcr]{%
0	0\\
0.002	0\\
0.004	0\\
0.006	1e-06\\
0.008	2e-06\\
0.01	3e-06\\
0.012	4e-06\\
0.014	7e-06\\
0.016	9e-06\\
0.018	1.3e-05\\
0.02	1.7e-05\\
0.022	2.2e-05\\
0.024	2.8e-05\\
0.026	3.6e-05\\
0.028	4.4e-05\\
0.03	5.3e-05\\
0.032	6.4e-05\\
0.034	7.6e-05\\
0.036	8.9e-05\\
0.038	0.000104\\
0.04	0.00012\\
0.042	0.000138\\
0.044	0.000157\\
0.046	0.000179\\
0.048	0.000202\\
0.05	0.000227\\
0.052	0.000255\\
0.054	0.000284\\
0.056	0.000315\\
0.058	0.000349\\
0.06	0.000385\\
0.062	0.000423\\
0.064	0.000464\\
0.066	0.000507\\
0.068	0.000553\\
0.07	0.000602\\
0.072	0.000653\\
0.074	0.000707\\
0.076	0.000764\\
0.078	0.000824\\
0.08	0.000887\\
0.082	0.000953\\
0.084	0.001023\\
0.086	0.001095\\
0.088	0.001171\\
0.09	0.001251\\
0.092	0.001333\\
0.094	0.00142\\
0.096	0.00151\\
0.098	0.001603\\
0.1	0.0017\\
0.102	0.001802\\
0.104	0.001907\\
0.106	0.002016\\
0.108	0.002129\\
0.11	0.002246\\
0.112	0.002367\\
0.114	0.002492\\
0.116	0.002622\\
0.118	0.002756\\
0.12	0.002895\\
0.122	0.003038\\
0.124	0.003185\\
0.126	0.003337\\
0.128	0.003494\\
0.13	0.003656\\
0.132	0.003822\\
0.134	0.003993\\
0.136	0.004169\\
0.138	0.00435\\
0.14	0.004537\\
0.142	0.004728\\
0.144	0.004924\\
0.146	0.005126\\
0.148	0.005333\\
0.15	0.005546\\
0.152	0.005763\\
0.154	0.005987\\
0.156	0.006215\\
0.158	0.00645\\
0.16	0.00669\\
0.162	0.006935\\
0.164	0.007187\\
0.166	0.007444\\
0.168	0.007707\\
0.17	0.007976\\
0.172	0.008252\\
0.174	0.008533\\
0.176	0.00882\\
0.178	0.009113\\
0.18	0.009413\\
0.182	0.009718\\
0.184	0.010031\\
0.186	0.010349\\
0.188	0.010674\\
0.19	0.011005\\
0.192	0.011343\\
0.194	0.011687\\
0.196	0.012038\\
0.198	0.012396\\
0.2	0.01276\\
0.202	0.013131\\
0.204	0.013509\\
0.206	0.013893\\
0.208	0.014285\\
0.21	0.014683\\
0.212	0.015089\\
0.214	0.015501\\
0.216	0.015921\\
0.218	0.016347\\
0.22	0.016781\\
0.222	0.017222\\
0.224	0.01767\\
0.226	0.018126\\
0.228	0.018588\\
0.23	0.019058\\
0.232	0.019536\\
0.234	0.020021\\
0.236	0.020513\\
0.238	0.021013\\
0.24	0.021521\\
0.242	0.022036\\
0.244	0.022558\\
0.246	0.023089\\
0.248	0.023627\\
0.25	0.024172\\
0.252	0.024726\\
0.254	0.025287\\
0.256	0.025856\\
0.258	0.026433\\
0.26	0.027018\\
0.262	0.027611\\
0.264	0.028212\\
0.266	0.028821\\
0.268	0.029437\\
0.27	0.030062\\
0.272	0.030695\\
0.274	0.031336\\
0.276	0.031985\\
0.278	0.032643\\
0.28	0.033308\\
0.282	0.033982\\
0.284	0.034664\\
0.286	0.035355\\
0.288	0.036053\\
0.29	0.03676\\
0.292	0.037476\\
0.294	0.038199\\
0.296	0.038931\\
0.298	0.039672\\
0.3	0.040421\\
0.302	0.041179\\
0.304	0.041945\\
0.306	0.042719\\
0.308	0.043502\\
0.31	0.044294\\
0.312	0.045094\\
0.314	0.045903\\
0.316	0.04672\\
0.318	0.047546\\
0.32	0.048381\\
0.322	0.049224\\
0.324	0.050076\\
0.326	0.050937\\
0.328	0.051806\\
0.33	0.052685\\
0.332	0.053572\\
0.334	0.054467\\
0.336	0.055372\\
0.338	0.056285\\
0.34	0.057207\\
0.342	0.058138\\
0.344	0.059078\\
0.346	0.060027\\
0.348	0.060984\\
0.35	0.061951\\
0.352	0.062926\\
0.354	0.06391\\
0.356	0.064903\\
0.358	0.065905\\
0.36	0.066916\\
0.362	0.067936\\
0.364	0.068964\\
0.366	0.070002\\
0.368	0.071049\\
0.37	0.072104\\
0.372	0.073169\\
0.374	0.074242\\
0.376	0.075325\\
0.378	0.076416\\
0.38	0.077517\\
0.382	0.078626\\
0.384	0.079745\\
0.386	0.080872\\
0.388	0.082009\\
0.39	0.083154\\
0.392	0.084309\\
0.394	0.085472\\
0.396	0.086645\\
0.398	0.087826\\
0.4	0.089017\\
0.402	0.090216\\
0.404	0.091424\\
0.406	0.092642\\
0.408	0.093868\\
0.41	0.095104\\
0.412	0.096348\\
0.414	0.097602\\
0.416	0.098864\\
0.418	0.100136\\
0.42	0.101416\\
0.422	0.102706\\
0.424	0.104004\\
0.426	0.105312\\
0.428	0.106628\\
0.43	0.107953\\
0.432	0.109287\\
0.434	0.110631\\
0.436	0.111983\\
0.438	0.113344\\
0.44	0.114714\\
0.442	0.116093\\
0.444	0.11748\\
0.446	0.118877\\
0.448	0.120283\\
0.45	0.121697\\
0.452	0.12312\\
0.454	0.124553\\
0.456	0.125994\\
0.458	0.127443\\
0.46	0.128902\\
0.462	0.130369\\
0.464	0.131846\\
0.466	0.133331\\
0.468	0.134824\\
0.47	0.136327\\
0.472	0.137838\\
0.474	0.139358\\
0.476	0.140887\\
0.478	0.142424\\
0.48	0.14397\\
0.482	0.145525\\
0.484	0.147088\\
0.486	0.14866\\
0.488	0.15024\\
0.49	0.151829\\
0.492	0.153427\\
0.494	0.155033\\
0.496	0.156648\\
0.498	0.158271\\
0.5	0.159903\\
0.502	0.161543\\
0.504	0.163191\\
0.506	0.164848\\
0.508	0.166514\\
0.51	0.168188\\
0.512	0.16987\\
0.514	0.17156\\
0.516	0.173259\\
0.518	0.174966\\
0.52	0.176681\\
0.522	0.178405\\
0.524	0.180136\\
0.526	0.181876\\
0.528	0.183624\\
0.53	0.18538\\
0.532	0.187145\\
0.534	0.188917\\
0.536	0.190697\\
0.538	0.192486\\
0.54	0.194282\\
0.542	0.196087\\
0.544	0.197899\\
0.546	0.199719\\
0.548	0.201547\\
0.55	0.203383\\
0.552	0.205227\\
0.554	0.207078\\
0.556	0.208938\\
0.558	0.210804\\
0.56	0.212679\\
0.562	0.214561\\
0.564	0.216451\\
0.566	0.218349\\
0.568	0.220254\\
0.57	0.222166\\
0.572	0.224086\\
0.574	0.226014\\
0.576	0.227949\\
0.578	0.229891\\
0.58	0.231841\\
0.582	0.233798\\
0.584	0.235762\\
0.586	0.237733\\
0.588	0.239712\\
0.59	0.241697\\
0.592	0.24369\\
0.594	0.24569\\
0.596	0.247697\\
0.598	0.249711\\
0.6	0.251732\\
0.602	0.25376\\
0.604	0.255795\\
0.606	0.257836\\
0.608	0.259884\\
0.61	0.261939\\
0.612	0.264001\\
0.614	0.26607\\
0.616	0.268145\\
0.618	0.270226\\
0.62	0.272314\\
0.622	0.274409\\
0.624	0.27651\\
0.626	0.278618\\
0.628	0.280732\\
0.63	0.282852\\
0.632	0.284978\\
0.634	0.287111\\
0.636	0.28925\\
0.638	0.291395\\
0.64	0.293546\\
0.642	0.295703\\
0.644	0.297866\\
0.646	0.300035\\
0.648	0.302209\\
0.65	0.30439\\
0.652	0.306576\\
0.654	0.308769\\
0.656	0.310966\\
0.658	0.31317\\
0.66	0.315379\\
0.662	0.317594\\
0.664	0.319814\\
0.666	0.322039\\
0.668	0.32427\\
0.67	0.326506\\
0.672	0.328748\\
0.674	0.330994\\
0.676	0.333246\\
0.678	0.335503\\
0.68	0.337765\\
0.682	0.340032\\
0.684	0.342304\\
0.686	0.344581\\
0.688	0.346863\\
0.69	0.349149\\
0.692	0.35144\\
0.694	0.353736\\
0.696	0.356037\\
0.698	0.358342\\
0.7	0.360651\\
0.702	0.362965\\
0.704	0.365283\\
0.706	0.367606\\
0.708	0.369933\\
0.71	0.372264\\
0.712	0.374599\\
0.714	0.376938\\
0.716	0.379282\\
0.718	0.381629\\
0.72	0.38398\\
0.722	0.386335\\
0.724	0.388694\\
0.726	0.391057\\
0.728	0.393423\\
0.73	0.395793\\
0.732	0.398166\\
0.734	0.400543\\
0.736	0.402923\\
0.738	0.405307\\
0.74	0.407694\\
0.742	0.410084\\
0.744	0.412477\\
0.746	0.414874\\
0.748	0.417273\\
0.75	0.419675\\
0.752	0.422081\\
0.754	0.424489\\
0.756	0.4269\\
0.758	0.429314\\
0.76	0.43173\\
0.762	0.434149\\
0.764	0.43657\\
0.766	0.438994\\
0.768	0.44142\\
0.77	0.443849\\
0.772	0.44628\\
0.774	0.448713\\
0.776	0.451148\\
0.778	0.453585\\
0.78	0.456025\\
0.782	0.458466\\
0.784	0.460909\\
0.786	0.463354\\
0.788	0.465801\\
0.79	0.468249\\
0.792	0.470699\\
0.794	0.47315\\
0.796	0.475603\\
0.798	0.478057\\
0.8	0.480513\\
0.802	0.48297\\
0.804	0.485428\\
0.806	0.487887\\
0.808	0.490347\\
0.81	0.492808\\
0.812	0.49527\\
0.814	0.497733\\
0.816	0.500197\\
0.818	0.502662\\
0.82	0.505127\\
0.822	0.507592\\
0.824	0.510059\\
0.826	0.512525\\
0.828	0.514992\\
0.83	0.51746\\
0.832	0.519927\\
0.834	0.522395\\
0.836	0.524863\\
0.838	0.527331\\
0.84	0.529799\\
0.842	0.532267\\
0.844	0.534734\\
0.846	0.537202\\
0.848	0.539669\\
0.85	0.542135\\
0.852	0.544602\\
0.854	0.547067\\
0.856	0.549533\\
0.858	0.551997\\
0.86	0.554461\\
0.862	0.556924\\
0.864	0.559386\\
0.866	0.561847\\
0.868	0.564307\\
0.87	0.566767\\
0.872	0.569225\\
0.874	0.571682\\
0.876	0.574137\\
0.878	0.576591\\
0.88	0.579044\\
0.882	0.581496\\
0.884	0.583946\\
0.886	0.586394\\
0.888	0.588841\\
0.89	0.591285\\
0.892	0.593729\\
0.894	0.59617\\
0.896	0.598609\\
0.898	0.601046\\
0.9	0.603482\\
0.902	0.605915\\
0.904	0.608346\\
0.906	0.610774\\
0.908	0.613201\\
0.91	0.615625\\
0.912	0.618046\\
0.914	0.620465\\
0.916	0.622881\\
0.918	0.625295\\
0.92	0.627706\\
0.922	0.630114\\
0.924	0.63252\\
0.926	0.634922\\
0.928	0.637322\\
0.93	0.639718\\
0.932	0.642111\\
0.934	0.644502\\
0.936	0.646888\\
0.938	0.649272\\
0.94	0.651652\\
0.942	0.654029\\
0.944	0.656403\\
0.946	0.658773\\
0.948	0.661139\\
0.95	0.663502\\
0.952	0.66586\\
0.954	0.668216\\
0.956	0.670567\\
0.958	0.672914\\
0.96	0.675258\\
0.962	0.677597\\
0.964	0.679932\\
0.966	0.682263\\
0.968	0.68459\\
0.97	0.686913\\
0.972	0.689231\\
0.974	0.691545\\
0.976	0.693855\\
0.978	0.69616\\
0.98	0.69846\\
0.982	0.700756\\
0.984	0.703048\\
0.986	0.705334\\
0.988	0.707616\\
0.99	0.709893\\
0.992	0.712165\\
0.994	0.714432\\
0.996	0.716694\\
0.998	0.718951\\
1	0.721203\\
1.002	0.72345\\
1.004	0.725691\\
1.006	0.727928\\
1.008	0.730158\\
1.01	0.732384\\
1.012	0.734604\\
1.014	0.736819\\
1.016	0.739028\\
1.018	0.741232\\
1.02	0.74343\\
1.022	0.745622\\
1.024	0.747808\\
1.026	0.749989\\
1.028	0.752164\\
1.03	0.754333\\
1.032	0.756496\\
1.034	0.758653\\
1.036	0.760804\\
1.038	0.762949\\
1.04	0.765088\\
1.042	0.767221\\
1.044	0.769348\\
1.046	0.771468\\
1.048	0.773582\\
1.05	0.775689\\
1.052	0.77779\\
1.054	0.779885\\
1.056	0.781974\\
1.058	0.784055\\
1.06	0.78613\\
1.062	0.788199\\
1.064	0.790261\\
1.066	0.792316\\
1.068	0.794364\\
1.07	0.796406\\
1.072	0.798441\\
1.074	0.800469\\
1.076	0.80249\\
1.078	0.804504\\
1.08	0.806511\\
1.082	0.808511\\
1.084	0.810504\\
1.086	0.81249\\
1.088	0.814468\\
1.09	0.81644\\
1.092	0.818404\\
1.094	0.820361\\
1.096	0.822311\\
1.098	0.824253\\
1.1	0.826188\\
1.102	0.828116\\
1.104	0.830036\\
1.106	0.831949\\
1.108	0.833854\\
1.11	0.835751\\
1.112	0.837641\\
1.114	0.839524\\
1.116	0.841398\\
1.118	0.843266\\
1.12	0.845125\\
1.122	0.846977\\
1.124	0.84882\\
1.126	0.850656\\
1.128	0.852485\\
1.13	0.854305\\
1.132	0.856117\\
1.134	0.857922\\
1.136	0.859718\\
1.138	0.861507\\
1.14	0.863287\\
1.142	0.86506\\
1.144	0.866824\\
1.146	0.86858\\
1.148	0.870329\\
1.15	0.872069\\
1.152	0.8738\\
1.154	0.875524\\
1.156	0.877239\\
1.158	0.878947\\
1.16	0.880645\\
1.162	0.882336\\
1.164	0.884018\\
1.166	0.885692\\
1.168	0.887358\\
1.17	0.889015\\
1.172	0.890664\\
1.174	0.892304\\
1.176	0.893936\\
1.178	0.895559\\
1.18	0.897174\\
1.182	0.89878\\
1.184	0.900378\\
1.186	0.901967\\
1.188	0.903548\\
1.19	0.90512\\
1.192	0.906683\\
1.194	0.908238\\
1.196	0.909784\\
1.198	0.911321\\
1.2	0.91285\\
1.202	0.91437\\
1.204	0.915882\\
1.206	0.917384\\
1.208	0.918878\\
1.21	0.920363\\
1.212	0.921839\\
1.214	0.923307\\
1.216	0.924766\\
1.218	0.926216\\
1.22	0.927657\\
1.222	0.929089\\
1.224	0.930513\\
1.226	0.931927\\
1.228	0.933333\\
1.23	0.93473\\
1.232	0.936118\\
1.234	0.937497\\
1.236	0.938867\\
1.238	0.940228\\
1.24	0.94158\\
1.242	0.942923\\
1.244	0.944258\\
1.246	0.945583\\
1.248	0.9469\\
1.25	0.948207\\
1.252	0.949506\\
1.254	0.950795\\
1.256	0.952076\\
1.258	0.953347\\
1.26	0.95461\\
1.262	0.955864\\
1.264	0.957108\\
1.266	0.958344\\
1.268	0.959571\\
1.27	0.960788\\
1.272	0.961997\\
1.274	0.963196\\
1.276	0.964387\\
1.278	0.965569\\
1.28	0.966741\\
1.282	0.967905\\
1.284	0.969059\\
1.286	0.970205\\
1.288	0.971342\\
1.29	0.972469\\
1.292	0.973588\\
1.294	0.974697\\
1.296	0.975798\\
1.298	0.97689\\
1.3	0.977972\\
1.302	0.979046\\
1.304	0.980111\\
1.306	0.981166\\
1.308	0.982213\\
1.31	0.983251\\
1.312	0.98428\\
1.314	0.9853\\
1.316	0.986311\\
1.318	0.987313\\
1.32	0.988306\\
1.322	0.98929\\
1.324	0.990266\\
1.326	0.991232\\
1.328	0.99219\\
1.33	0.993139\\
1.332	0.994079\\
1.334	0.99501\\
1.336	0.995932\\
1.338	0.996845\\
1.34	0.99775\\
1.342	0.998646\\
1.344	0.999533\\
1.346	1.000411\\
1.348	1.001281\\
1.35	1.002142\\
1.352	1.002994\\
1.354	1.003837\\
1.356	1.004672\\
1.358	1.005498\\
1.36	1.006316\\
1.362	1.007125\\
1.364	1.007925\\
1.366	1.008717\\
1.368	1.0095\\
1.37	1.010275\\
1.372	1.011041\\
1.374	1.011798\\
1.376	1.012548\\
1.378	1.013288\\
1.38	1.01402\\
1.382	1.014744\\
1.384	1.01546\\
1.386	1.016167\\
1.388	1.016866\\
1.39	1.017556\\
1.392	1.018238\\
1.394	1.018912\\
1.396	1.019578\\
1.398	1.020235\\
1.4	1.020885\\
1.402	1.021526\\
1.404	1.022159\\
1.406	1.022784\\
1.408	1.023401\\
1.41	1.02401\\
1.412	1.024611\\
1.414	1.025204\\
1.416	1.025789\\
1.418	1.026366\\
1.42	1.026935\\
1.422	1.027497\\
1.424	1.02805\\
1.426	1.028596\\
1.428	1.029134\\
1.43	1.029665\\
1.432	1.030187\\
1.434	1.030703\\
1.436	1.03121\\
1.438	1.03171\\
1.44	1.032203\\
1.442	1.032688\\
1.444	1.033165\\
1.446	1.033635\\
1.448	1.034098\\
1.45	1.034554\\
1.452	1.035002\\
1.454	1.035443\\
1.456	1.035877\\
1.458	1.036304\\
1.46	1.036723\\
1.462	1.037136\\
1.464	1.037541\\
1.466	1.03794\\
1.468	1.038331\\
1.47	1.038716\\
1.472	1.039094\\
1.474	1.039465\\
1.476	1.03983\\
1.478	1.040187\\
1.48	1.040538\\
1.482	1.040883\\
1.484	1.041221\\
1.486	1.041552\\
1.488	1.041877\\
1.49	1.042195\\
1.492	1.042508\\
1.494	1.042813\\
1.496	1.043113\\
1.498	1.043406\\
1.5	1.043694\\
1.502	1.043975\\
1.504	1.04425\\
1.506	1.044519\\
1.508	1.044783\\
1.51	1.04504\\
1.512	1.045291\\
1.514	1.045537\\
1.516	1.045777\\
1.518	1.046012\\
1.52	1.046241\\
1.522	1.046464\\
1.524	1.046682\\
1.526	1.046894\\
1.528	1.047101\\
1.53	1.047303\\
1.532	1.0475\\
1.534	1.047691\\
1.536	1.047877\\
1.538	1.048059\\
1.54	1.048235\\
1.542	1.048406\\
1.544	1.048573\\
1.546	1.048734\\
1.548	1.048891\\
1.55	1.049043\\
1.552	1.049191\\
1.554	1.049334\\
1.556	1.049472\\
1.558	1.049607\\
1.56	1.049736\\
1.562	1.049862\\
1.564	1.049983\\
1.566	1.0501\\
1.568	1.050213\\
1.57	1.050322\\
1.572	1.050427\\
1.574	1.050529\\
1.576	1.050626\\
1.578	1.05072\\
1.58	1.05081\\
1.582	1.050896\\
1.584	1.050979\\
1.586	1.051058\\
1.588	1.051134\\
1.59	1.051207\\
1.592	1.051276\\
1.594	1.051342\\
1.596	1.051405\\
1.598	1.051465\\
1.6	1.051523\\
1.602	1.051577\\
1.604	1.051628\\
1.606	1.051677\\
1.608	1.051723\\
1.61	1.051766\\
1.612	1.051807\\
1.614	1.051845\\
1.616	1.051881\\
1.618	1.051915\\
1.62	1.051946\\
1.622	1.051976\\
1.624	1.052003\\
1.626	1.052028\\
1.628	1.052051\\
1.63	1.052073\\
1.632	1.052092\\
1.634	1.05211\\
1.636	1.052127\\
1.638	1.052142\\
1.64	1.052155\\
1.642	1.052167\\
1.644	1.052177\\
1.646	1.052187\\
1.648	1.052195\\
1.65	1.052202\\
1.652	1.052208\\
1.654	1.052213\\
1.656	1.052218\\
1.658	1.052221\\
1.66	1.052224\\
1.662	1.052226\\
1.664	1.052228\\
1.666	1.052229\\
1.668	1.05223\\
1.67	1.05223\\
1.672	1.05223\\
1.674	1.052231\\
1.676	1.052231\\
1.678	1.052231\\
1.68	1.052231\\
1.682	1.052231\\
1.684	1.052231\\
1.686	1.052232\\
1.688	1.052233\\
1.69	1.052235\\
1.692	1.052237\\
1.694	1.05224\\
1.696	1.052244\\
1.698	1.052248\\
1.7	1.052253\\
1.702	1.052259\\
1.704	1.052266\\
1.706	1.052274\\
1.708	1.052284\\
1.71	1.052294\\
1.712	1.052306\\
1.714	1.052319\\
1.716	1.052334\\
1.718	1.05235\\
1.72	1.052368\\
1.722	1.052388\\
1.724	1.052409\\
1.726	1.052432\\
1.728	1.052458\\
1.73	1.052485\\
1.732	1.052514\\
1.734	1.052545\\
1.736	1.052579\\
1.738	1.052615\\
1.74	1.052653\\
1.742	1.052694\\
1.744	1.052737\\
1.746	1.052783\\
1.748	1.052832\\
1.75	1.052883\\
1.752	1.052937\\
1.754	1.052994\\
1.756	1.053054\\
1.758	1.053117\\
1.76	1.053183\\
1.762	1.053252\\
1.764	1.053325\\
1.766	1.053401\\
1.768	1.05348\\
1.77	1.053563\\
1.772	1.053649\\
1.774	1.053739\\
1.776	1.053832\\
1.778	1.05393\\
1.78	1.054031\\
1.782	1.054136\\
1.784	1.054245\\
1.786	1.054357\\
1.788	1.054474\\
1.79	1.054596\\
1.792	1.054721\\
1.794	1.054851\\
1.796	1.054985\\
1.798	1.055123\\
1.8	1.055266\\
1.802	1.055413\\
1.804	1.055566\\
1.806	1.055722\\
1.808	1.055884\\
1.81	1.05605\\
1.812	1.056221\\
1.814	1.056397\\
1.816	1.056578\\
1.818	1.056764\\
1.82	1.056956\\
1.822	1.057152\\
1.824	1.057354\\
1.826	1.057561\\
1.828	1.057773\\
1.83	1.057991\\
1.832	1.058214\\
1.834	1.058442\\
1.836	1.058677\\
1.838	1.058917\\
1.84	1.059162\\
1.842	1.059414\\
1.844	1.059671\\
1.846	1.059934\\
1.848	1.060203\\
1.85	1.060478\\
1.852	1.060759\\
1.854	1.061046\\
1.856	1.061339\\
1.858	1.061639\\
1.86	1.061945\\
1.862	1.062257\\
1.864	1.062575\\
1.866	1.0629\\
1.868	1.063231\\
1.87	1.063568\\
1.872	1.063913\\
1.874	1.064263\\
1.876	1.064621\\
1.878	1.064985\\
1.88	1.065356\\
1.882	1.065734\\
1.884	1.066118\\
1.886	1.06651\\
1.888	1.066908\\
1.89	1.067313\\
1.892	1.067726\\
1.894	1.068145\\
1.896	1.068572\\
1.898	1.069005\\
1.9	1.069446\\
1.902	1.069894\\
1.904	1.070349\\
1.906	1.070812\\
1.908	1.071282\\
1.91	1.071759\\
1.912	1.072244\\
1.914	1.072736\\
1.916	1.073236\\
1.918	1.073744\\
1.92	1.074258\\
1.922	1.074781\\
1.924	1.075311\\
1.926	1.075849\\
1.928	1.076395\\
1.93	1.076948\\
1.932	1.077509\\
1.934	1.078078\\
1.936	1.078655\\
1.938	1.07924\\
1.94	1.079833\\
1.942	1.080433\\
1.944	1.081042\\
1.946	1.081659\\
1.948	1.082284\\
1.95	1.082916\\
1.952	1.083557\\
1.954	1.084206\\
1.956	1.084864\\
1.958	1.085529\\
1.96	1.086203\\
1.962	1.086885\\
1.964	1.087575\\
1.966	1.088274\\
1.968	1.08898\\
1.97	1.089696\\
1.972	1.090419\\
1.974	1.091151\\
1.976	1.091892\\
1.978	1.092641\\
1.98	1.093398\\
1.982	1.094164\\
1.984	1.094938\\
1.986	1.095721\\
1.988	1.096513\\
1.99	1.097313\\
1.992	1.098121\\
1.994	1.098938\\
1.996	1.099764\\
1.998	1.100599\\
2	1.101442\\
2.002	1.102294\\
2.004	1.103155\\
2.006	1.104024\\
2.008	1.104902\\
2.01	1.105789\\
2.012	1.106685\\
2.014	1.107589\\
2.016	1.108502\\
2.018	1.109424\\
2.02	1.110355\\
2.022	1.111295\\
2.024	1.112243\\
2.026	1.113201\\
2.028	1.114167\\
2.03	1.115142\\
2.032	1.116126\\
2.034	1.117119\\
2.036	1.118121\\
2.038	1.119131\\
2.04	1.120151\\
2.042	1.12118\\
2.044	1.122217\\
2.046	1.123264\\
2.048	1.124319\\
2.05	1.125384\\
2.052	1.126457\\
2.054	1.127539\\
2.056	1.128631\\
2.058	1.129731\\
2.06	1.130841\\
2.062	1.131959\\
2.064	1.133086\\
2.066	1.134222\\
2.068	1.135368\\
2.07	1.136522\\
2.072	1.137685\\
2.074	1.138858\\
2.076	1.140039\\
2.078	1.141229\\
2.08	1.142429\\
2.082	1.143637\\
2.084	1.144855\\
2.086	1.146081\\
2.088	1.147316\\
2.09	1.148561\\
2.092	1.149814\\
2.094	1.151076\\
2.096	1.152348\\
2.098	1.153628\\
2.1	1.154917\\
2.102	1.156216\\
2.104	1.157523\\
2.106	1.158839\\
2.108	1.160164\\
2.11	1.161498\\
2.112	1.162841\\
2.114	1.164193\\
2.116	1.165554\\
2.118	1.166924\\
2.12	1.168303\\
2.122	1.16969\\
2.124	1.171087\\
2.126	1.172492\\
2.128	1.173907\\
2.13	1.17533\\
2.132	1.176762\\
2.134	1.178203\\
2.136	1.179652\\
2.138	1.181111\\
2.14	1.182578\\
2.142	1.184054\\
2.144	1.185539\\
2.146	1.187033\\
2.148	1.188535\\
2.15	1.190046\\
2.152	1.191566\\
2.154	1.193095\\
2.156	1.194632\\
2.158	1.196178\\
2.16	1.197732\\
2.162	1.199295\\
2.164	1.200867\\
2.166	1.202447\\
2.168	1.204036\\
2.17	1.205634\\
2.172	1.20724\\
2.174	1.208854\\
2.176	1.210478\\
2.178	1.212109\\
2.18	1.213749\\
2.182	1.215398\\
2.184	1.217054\\
2.186	1.21872\\
2.188	1.220393\\
2.19	1.222075\\
2.192	1.223766\\
2.194	1.225464\\
2.196	1.227171\\
2.198	1.228886\\
2.2	1.23061\\
2.202	1.232341\\
2.204	1.234081\\
2.206	1.235829\\
2.208	1.237585\\
2.21	1.239349\\
2.212	1.241121\\
2.214	1.242902\\
2.216	1.24469\\
2.218	1.246486\\
2.22	1.248291\\
2.222	1.250103\\
2.224	1.251923\\
2.226	1.253751\\
2.228	1.255587\\
2.23	1.25743\\
2.232	1.259281\\
2.234	1.261141\\
2.236	1.263007\\
2.238	1.264882\\
2.24	1.266764\\
2.242	1.268654\\
2.244	1.270551\\
2.246	1.272456\\
2.248	1.274369\\
2.25	1.276289\\
2.252	1.278216\\
2.254	1.280151\\
2.256	1.282093\\
2.258	1.284042\\
2.26	1.285999\\
2.262	1.287963\\
2.264	1.289935\\
2.266	1.291913\\
2.268	1.293899\\
2.27	1.295891\\
2.272	1.297891\\
2.274	1.299898\\
2.276	1.301912\\
2.278	1.303933\\
2.28	1.30596\\
2.282	1.307995\\
2.284	1.310036\\
2.286	1.312085\\
2.288	1.31414\\
2.29	1.316201\\
2.292	1.31827\\
2.294	1.320345\\
2.296	1.322426\\
2.298	1.324514\\
2.3	1.326609\\
2.302	1.32871\\
2.304	1.330817\\
2.306	1.332931\\
2.308	1.335051\\
2.31	1.337177\\
2.312	1.33931\\
2.314	1.341449\\
2.316	1.343593\\
2.318	1.345744\\
2.32	1.347901\\
2.322	1.350064\\
2.324	1.352233\\
2.326	1.354408\\
2.328	1.356588\\
2.33	1.358775\\
2.332	1.360967\\
2.334	1.363165\\
2.336	1.365368\\
2.338	1.367577\\
2.34	1.369791\\
2.342	1.372011\\
2.344	1.374237\\
2.346	1.376468\\
2.348	1.378704\\
2.35	1.380945\\
2.352	1.383192\\
2.354	1.385444\\
2.356	1.3877\\
2.358	1.389962\\
2.36	1.392229\\
2.362	1.394501\\
2.364	1.396778\\
2.366	1.39906\\
2.368	1.401346\\
2.37	1.403637\\
2.372	1.405933\\
2.374	1.408233\\
2.376	1.410538\\
2.378	1.412848\\
2.38	1.415161\\
2.382	1.41748\\
2.384	1.419802\\
2.386	1.422129\\
2.388	1.42446\\
2.39	1.426795\\
2.392	1.429134\\
2.394	1.431478\\
2.396	1.433825\\
2.398	1.436176\\
2.4	1.438531\\
2.402	1.44089\\
2.404	1.443252\\
2.406	1.445619\\
2.408	1.447988\\
2.41	1.450362\\
2.412	1.452738\\
2.414	1.455119\\
2.416	1.457502\\
2.418	1.459889\\
2.42	1.462279\\
2.422	1.464672\\
2.424	1.467069\\
2.426	1.469468\\
2.428	1.471871\\
2.43	1.474276\\
2.432	1.476684\\
2.434	1.479095\\
2.436	1.481509\\
2.438	1.483925\\
2.44	1.486344\\
2.442	1.488765\\
2.444	1.491189\\
2.446	1.493615\\
2.448	1.496044\\
2.45	1.498475\\
2.452	1.500908\\
2.454	1.503343\\
2.456	1.50578\\
2.458	1.508219\\
2.46	1.510661\\
2.462	1.513104\\
2.464	1.515548\\
2.466	1.517995\\
2.468	1.520443\\
2.47	1.522893\\
2.472	1.525344\\
2.474	1.527797\\
2.476	1.530251\\
2.478	1.532707\\
2.48	1.535164\\
2.482	1.537622\\
2.484	1.540081\\
2.486	1.542541\\
2.488	1.545002\\
2.49	1.547465\\
2.492	1.549928\\
2.494	1.552391\\
2.496	1.554856\\
2.498	1.557321\\
2.5	1.559787\\
2.502	1.562253\\
2.504	1.56472\\
2.506	1.567187\\
2.508	1.569654\\
2.51	1.572122\\
2.512	1.574589\\
2.514	1.577057\\
2.516	1.579525\\
2.518	1.581993\\
2.52	1.584461\\
2.522	1.586928\\
2.524	1.589396\\
2.526	1.591863\\
2.528	1.59433\\
2.53	1.596796\\
2.532	1.599262\\
2.534	1.601727\\
2.536	1.604191\\
2.538	1.606655\\
2.54	1.609118\\
2.542	1.61158\\
2.544	1.614042\\
2.546	1.616502\\
2.548	1.618961\\
2.55	1.621419\\
2.552	1.623876\\
2.554	1.626332\\
2.556	1.628786\\
2.558	1.631239\\
2.56	1.63369\\
2.562	1.63614\\
2.564	1.638588\\
2.566	1.641035\\
2.568	1.64348\\
2.57	1.645923\\
2.572	1.648364\\
2.574	1.650804\\
2.576	1.653241\\
2.578	1.655676\\
2.58	1.658109\\
2.582	1.66054\\
2.584	1.662969\\
2.586	1.665395\\
2.588	1.667819\\
2.59	1.670241\\
2.592	1.67266\\
2.594	1.675076\\
2.596	1.67749\\
2.598	1.679901\\
2.6	1.682309\\
2.602	1.684715\\
2.604	1.687117\\
2.606	1.689517\\
2.608	1.691913\\
2.61	1.694307\\
2.612	1.696697\\
2.614	1.699084\\
2.616	1.701468\\
2.618	1.703848\\
2.62	1.706225\\
2.622	1.708598\\
2.624	1.710968\\
2.626	1.713335\\
2.628	1.715697\\
2.63	1.718056\\
2.632	1.720411\\
2.634	1.722763\\
2.636	1.72511\\
2.638	1.727454\\
2.64	1.729793\\
2.642	1.732128\\
2.644	1.73446\\
2.646	1.736787\\
2.648	1.739109\\
2.65	1.741428\\
2.652	1.743742\\
2.654	1.746051\\
2.656	1.748357\\
2.658	1.750657\\
2.66	1.752953\\
2.662	1.755244\\
2.664	1.757531\\
2.666	1.759813\\
2.668	1.76209\\
2.67	1.764362\\
2.672	1.766629\\
2.674	1.768891\\
2.676	1.771148\\
2.678	1.7734\\
2.68	1.775647\\
2.682	1.777889\\
2.684	1.780125\\
2.686	1.782356\\
2.688	1.784582\\
2.69	1.786802\\
2.692	1.789017\\
2.694	1.791226\\
2.696	1.79343\\
2.698	1.795628\\
2.7	1.79782\\
2.702	1.800007\\
2.704	1.802188\\
2.706	1.804363\\
2.708	1.806532\\
2.71	1.808695\\
2.712	1.810852\\
2.714	1.813003\\
2.716	1.815148\\
2.718	1.817287\\
2.72	1.81942\\
2.722	1.821547\\
2.724	1.823667\\
2.726	1.825781\\
2.728	1.827889\\
2.73	1.82999\\
2.732	1.832085\\
2.734	1.834173\\
2.736	1.836255\\
2.738	1.83833\\
2.74	1.840399\\
2.742	1.842461\\
2.744	1.844516\\
2.746	1.846565\\
2.748	1.848606\\
2.75	1.850641\\
2.752	1.852669\\
2.754	1.854691\\
2.756	1.856705\\
2.758	1.858712\\
2.76	1.860712\\
2.762	1.862705\\
2.764	1.864691\\
2.766	1.86667\\
2.768	1.868641\\
2.77	1.870606\\
2.772	1.872563\\
2.774	1.874513\\
2.776	1.876455\\
2.778	1.87839\\
2.78	1.880318\\
2.782	1.882238\\
2.784	1.884151\\
2.786	1.886056\\
2.788	1.887954\\
2.79	1.889844\\
2.792	1.891727\\
2.794	1.893601\\
2.796	1.895469\\
2.798	1.897328\\
2.8	1.89918\\
2.802	1.901024\\
2.804	1.90286\\
2.806	1.904688\\
2.808	1.906509\\
2.81	1.908321\\
2.812	1.910126\\
2.814	1.911922\\
2.816	1.913711\\
2.818	1.915492\\
2.82	1.917264\\
2.822	1.919029\\
2.824	1.920785\\
2.826	1.922533\\
2.828	1.924274\\
2.83	1.926006\\
2.832	1.927729\\
2.834	1.929445\\
2.836	1.931152\\
2.838	1.932851\\
2.84	1.934542\\
2.842	1.936224\\
2.844	1.937898\\
2.846	1.939564\\
2.848	1.941221\\
2.85	1.94287\\
2.852	1.94451\\
2.854	1.946142\\
2.856	1.947766\\
2.858	1.949381\\
2.86	1.950987\\
2.862	1.952585\\
2.864	1.954174\\
2.866	1.955755\\
2.868	1.957327\\
2.87	1.958891\\
2.872	1.960445\\
2.874	1.961992\\
2.876	1.963529\\
2.878	1.965058\\
2.88	1.966578\\
2.882	1.96809\\
2.884	1.969593\\
2.886	1.971087\\
2.888	1.972572\\
2.89	1.974048\\
2.892	1.975516\\
2.894	1.976975\\
2.896	1.978425\\
2.898	1.979866\\
2.9	1.981299\\
2.902	1.982722\\
2.904	1.984137\\
2.906	1.985543\\
2.908	1.98694\\
2.91	1.988328\\
2.912	1.989707\\
2.914	1.991077\\
2.916	1.992438\\
2.918	1.993791\\
2.92	1.995134\\
2.922	1.996469\\
2.924	1.997794\\
2.926	1.999111\\
2.928	2.000419\\
2.93	2.001717\\
2.932	2.003007\\
2.934	2.004288\\
2.936	2.005559\\
2.938	2.006822\\
2.94	2.008076\\
2.942	2.009321\\
2.944	2.010556\\
2.946	2.011783\\
2.948	2.013001\\
2.95	2.01421\\
2.952	2.015409\\
2.954	2.0166\\
2.956	2.017782\\
2.958	2.018955\\
2.96	2.020118\\
2.962	2.021273\\
2.964	2.022419\\
2.966	2.023555\\
2.968	2.024683\\
2.97	2.025802\\
2.972	2.026912\\
2.974	2.028012\\
2.976	2.029104\\
2.978	2.030187\\
2.98	2.031261\\
2.982	2.032326\\
2.984	2.033381\\
2.986	2.034428\\
2.988	2.035466\\
2.99	2.036495\\
2.992	2.037515\\
2.994	2.038527\\
2.996	2.039529\\
2.998	2.040522\\
3	2.041506\\
3.002	2.042482\\
3.004	2.043449\\
3.006	2.044406\\
3.008	2.045355\\
3.01	2.046295\\
3.012	2.047227\\
3.014	2.048149\\
3.016	2.049063\\
3.018	2.049967\\
3.02	2.050863\\
3.022	2.051751\\
3.024	2.052629\\
3.026	2.053499\\
3.028	2.05436\\
3.03	2.055212\\
3.032	2.056056\\
3.034	2.056891\\
3.036	2.057717\\
3.038	2.058535\\
3.04	2.059344\\
3.042	2.060144\\
3.044	2.060936\\
3.046	2.061719\\
3.048	2.062494\\
3.05	2.06326\\
3.052	2.064018\\
3.054	2.064767\\
3.056	2.065508\\
3.058	2.06624\\
3.06	2.066964\\
3.062	2.06768\\
3.064	2.068387\\
3.066	2.069086\\
3.068	2.069777\\
3.07	2.070459\\
3.072	2.071133\\
3.074	2.071799\\
3.076	2.072456\\
3.078	2.073106\\
3.08	2.073747\\
3.082	2.07438\\
3.084	2.075005\\
3.086	2.075623\\
3.088	2.076232\\
3.09	2.076833\\
3.092	2.077426\\
3.094	2.078011\\
3.096	2.078588\\
3.098	2.079157\\
3.1	2.079719\\
3.102	2.080273\\
3.104	2.080819\\
3.106	2.081357\\
3.108	2.081887\\
3.11	2.08241\\
3.112	2.082926\\
3.114	2.083433\\
3.116	2.083933\\
3.118	2.084426\\
3.12	2.084911\\
3.122	2.085389\\
3.124	2.085859\\
3.126	2.086322\\
3.128	2.086778\\
3.13	2.087226\\
3.132	2.087667\\
3.134	2.088101\\
3.136	2.088528\\
3.138	2.088948\\
3.14	2.08936\\
3.142	2.089766\\
3.144	2.090165\\
3.146	2.090556\\
3.148	2.090941\\
3.15	2.091319\\
3.152	2.09169\\
3.154	2.092055\\
3.156	2.092413\\
3.158	2.092764\\
3.16	2.093108\\
3.162	2.093446\\
3.164	2.093778\\
3.166	2.094103\\
3.168	2.094421\\
3.17	2.094734\\
3.172	2.09504\\
3.174	2.095339\\
3.176	2.095633\\
3.178	2.09592\\
3.18	2.096201\\
3.182	2.096477\\
3.184	2.096746\\
3.186	2.097009\\
3.188	2.097267\\
3.19	2.097518\\
3.192	2.097764\\
3.194	2.098004\\
3.196	2.098239\\
3.198	2.098468\\
3.2	2.098691\\
3.202	2.098909\\
3.204	2.099122\\
3.206	2.099329\\
3.208	2.099531\\
3.21	2.099728\\
3.212	2.099919\\
3.214	2.100105\\
3.216	2.100287\\
3.218	2.100463\\
3.22	2.100634\\
3.222	2.100801\\
3.224	2.100962\\
3.226	2.101119\\
3.228	2.101272\\
3.23	2.101419\\
3.232	2.101562\\
3.234	2.101701\\
3.236	2.101835\\
3.238	2.101965\\
3.24	2.102091\\
3.242	2.102212\\
3.244	2.102329\\
3.246	2.102442\\
3.248	2.102551\\
3.25	2.102657\\
3.252	2.102758\\
3.254	2.102855\\
3.256	2.102949\\
3.258	2.103039\\
3.26	2.103125\\
3.262	2.103208\\
3.264	2.103288\\
3.266	2.103364\\
3.268	2.103436\\
3.27	2.103506\\
3.272	2.103572\\
3.274	2.103635\\
3.276	2.103695\\
3.278	2.103752\\
3.28	2.103807\\
3.282	2.103858\\
3.284	2.103907\\
3.286	2.103953\\
3.288	2.103996\\
3.29	2.104037\\
3.292	2.104075\\
3.294	2.104111\\
3.296	2.104145\\
3.298	2.104176\\
3.3	2.104206\\
3.302	2.104233\\
3.304	2.104258\\
3.306	2.104282\\
3.308	2.104303\\
3.31	2.104323\\
3.312	2.104341\\
3.314	2.104357\\
3.316	2.104372\\
3.318	2.104385\\
3.32	2.104397\\
3.322	2.104408\\
3.324	2.104417\\
3.326	2.104425\\
3.328	2.104432\\
3.33	2.104439\\
3.332	2.104444\\
3.334	2.104448\\
3.336	2.104452\\
3.338	2.104454\\
3.34	2.104457\\
3.342	2.104458\\
3.344	2.10446\\
3.346	2.10446\\
3.348	2.104461\\
3.35	2.104461\\
3.352	2.104461\\
3.354	2.104461\\
3.356	2.104461\\
3.358	2.104461\\
3.36	2.104462\\
3.362	2.104462\\
3.364	2.104463\\
3.366	2.104464\\
3.368	2.104466\\
3.37	2.104468\\
3.372	2.104471\\
3.374	2.104474\\
3.376	2.104478\\
3.378	2.104483\\
3.38	2.104489\\
3.382	2.104497\\
3.384	2.104505\\
3.386	2.104514\\
3.388	2.104525\\
3.39	2.104536\\
3.392	2.10455\\
3.394	2.104564\\
3.396	2.104581\\
3.398	2.104598\\
3.4	2.104618\\
3.402	2.104639\\
3.404	2.104663\\
3.406	2.104688\\
3.408	2.104715\\
3.41	2.104744\\
3.412	2.104775\\
3.414	2.104809\\
3.416	2.104845\\
3.418	2.104883\\
3.42	2.104924\\
3.422	2.104967\\
3.424	2.105013\\
3.426	2.105061\\
3.428	2.105113\\
3.43	2.105167\\
3.432	2.105224\\
3.434	2.105284\\
3.436	2.105347\\
3.438	2.105413\\
3.44	2.105482\\
3.442	2.105554\\
3.444	2.10563\\
3.446	2.105709\\
3.448	2.105792\\
3.45	2.105878\\
3.452	2.105968\\
3.454	2.106061\\
3.456	2.106159\\
3.458	2.10626\\
3.46	2.106365\\
3.462	2.106473\\
3.464	2.106586\\
3.466	2.106703\\
3.468	2.106824\\
3.47	2.10695\\
3.472	2.107079\\
3.474	2.107213\\
3.476	2.107352\\
3.478	2.107494\\
3.48	2.107642\\
3.482	2.107794\\
3.484	2.10795\\
3.486	2.108112\\
3.488	2.108278\\
3.49	2.108449\\
3.492	2.108625\\
3.494	2.108806\\
3.496	2.108992\\
3.498	2.109183\\
3.5	2.10938\\
3.502	2.109581\\
3.504	2.109788\\
3.506	2.11\\
3.508	2.110218\\
3.51	2.110441\\
3.512	2.11067\\
3.514	2.110904\\
3.516	2.111144\\
3.518	2.111389\\
3.52	2.111641\\
3.522	2.111898\\
3.524	2.112161\\
3.526	2.11243\\
3.528	2.112705\\
3.53	2.112985\\
3.532	2.113272\\
3.534	2.113566\\
3.536	2.113865\\
3.538	2.114171\\
3.54	2.114482\\
3.542	2.114801\\
3.544	2.115125\\
3.546	2.115456\\
3.548	2.115794\\
3.55	2.116138\\
3.552	2.116489\\
3.554	2.116846\\
3.556	2.11721\\
3.558	2.117581\\
3.56	2.117959\\
3.562	2.118343\\
3.564	2.118734\\
3.566	2.119133\\
3.568	2.119538\\
3.57	2.11995\\
3.572	2.120369\\
3.574	2.120796\\
3.576	2.121229\\
3.578	2.12167\\
3.58	2.122118\\
3.582	2.122573\\
3.584	2.123036\\
3.586	2.123506\\
3.588	2.123983\\
3.59	2.124468\\
3.592	2.12496\\
3.594	2.125459\\
3.596	2.125967\\
3.598	2.126481\\
3.6	2.127004\\
3.602	2.127534\\
3.604	2.128072\\
3.606	2.128617\\
3.608	2.12917\\
3.61	2.129732\\
3.612	2.1303\\
3.614	2.130877\\
3.616	2.131462\\
3.618	2.132055\\
3.62	2.132655\\
3.622	2.133264\\
3.624	2.13388\\
3.626	2.134505\\
3.628	2.135138\\
3.63	2.135778\\
3.632	2.136427\\
3.634	2.137084\\
3.636	2.13775\\
3.638	2.138423\\
3.64	2.139105\\
3.642	2.139795\\
3.644	2.140494\\
3.646	2.1412\\
3.648	2.141916\\
3.65	2.142639\\
3.652	2.143371\\
3.654	2.144111\\
3.656	2.14486\\
3.658	2.145617\\
3.66	2.146383\\
3.662	2.147157\\
3.664	2.14794\\
3.666	2.148731\\
3.668	2.149531\\
3.67	2.15034\\
3.672	2.151157\\
3.674	2.151983\\
3.676	2.152817\\
3.678	2.15366\\
3.68	2.154512\\
3.682	2.155373\\
3.684	2.156242\\
3.686	2.15712\\
3.688	2.158006\\
3.69	2.158902\\
3.692	2.159806\\
3.694	2.160719\\
3.696	2.161641\\
3.698	2.162572\\
3.7	2.163511\\
3.702	2.16446\\
3.704	2.165417\\
3.706	2.166383\\
3.708	2.167358\\
3.71	2.168342\\
3.712	2.169335\\
3.714	2.170336\\
3.716	2.171347\\
3.718	2.172367\\
3.72	2.173395\\
3.722	2.174432\\
3.724	2.175479\\
3.726	2.176534\\
3.728	2.177598\\
3.73	2.178672\\
3.732	2.179754\\
3.734	2.180845\\
3.736	2.181945\\
3.738	2.183055\\
3.74	2.184173\\
3.742	2.1853\\
3.744	2.186436\\
3.746	2.187581\\
3.748	2.188736\\
3.75	2.189899\\
3.752	2.191071\\
3.754	2.192252\\
3.756	2.193442\\
3.758	2.194642\\
3.76	2.19585\\
3.762	2.197067\\
3.764	2.198293\\
3.766	2.199529\\
3.768	2.200773\\
3.77	2.202026\\
3.772	2.203288\\
3.774	2.204559\\
3.776	2.20584\\
3.778	2.207129\\
3.78	2.208427\\
3.782	2.209734\\
3.784	2.21105\\
3.786	2.212375\\
3.788	2.213709\\
3.79	2.215052\\
3.792	2.216404\\
3.794	2.217765\\
3.796	2.219134\\
3.798	2.220513\\
3.8	2.221901\\
3.802	2.223297\\
3.804	2.224702\\
3.806	2.226116\\
3.808	2.227539\\
3.81	2.228971\\
3.812	2.230412\\
3.814	2.231862\\
3.816	2.23332\\
3.818	2.234787\\
3.82	2.236263\\
3.822	2.237748\\
3.824	2.239241\\
3.826	2.240743\\
3.828	2.242254\\
3.83	2.243774\\
3.832	2.245303\\
3.834	2.24684\\
3.836	2.248385\\
3.838	2.24994\\
3.84	2.251503\\
3.842	2.253074\\
3.844	2.254655\\
3.846	2.256243\\
3.848	2.257841\\
3.85	2.259447\\
3.852	2.261061\\
3.854	2.262684\\
3.856	2.264316\\
3.858	2.265955\\
3.86	2.267604\\
3.862	2.26926\\
3.864	2.270926\\
3.866	2.272599\\
3.868	2.274281\\
3.87	2.275971\\
3.872	2.27767\\
3.874	2.279377\\
3.876	2.281092\\
3.878	2.282815\\
3.88	2.284546\\
3.882	2.286286\\
3.884	2.288034\\
3.886	2.28979\\
3.888	2.291554\\
3.89	2.293326\\
3.892	2.295106\\
3.894	2.296894\\
3.896	2.29869\\
3.898	2.300494\\
3.9	2.302306\\
3.902	2.304126\\
3.904	2.305954\\
3.906	2.30779\\
3.908	2.309633\\
3.91	2.311485\\
3.912	2.313344\\
3.914	2.31521\\
3.916	2.317085\\
3.918	2.318967\\
3.92	2.320857\\
3.922	2.322754\\
3.924	2.324659\\
3.926	2.326571\\
3.928	2.328491\\
3.93	2.330418\\
3.932	2.332353\\
3.934	2.334295\\
3.936	2.336244\\
3.938	2.338201\\
3.94	2.340165\\
3.942	2.342136\\
3.944	2.344114\\
3.946	2.3461\\
3.948	2.348092\\
3.95	2.350092\\
3.952	2.352099\\
3.954	2.354113\\
3.956	2.356133\\
3.958	2.358161\\
3.96	2.360195\\
3.962	2.362237\\
3.964	2.364285\\
3.966	2.36634\\
3.968	2.368401\\
3.97	2.37047\\
3.972	2.372544\\
3.974	2.374626\\
3.976	2.376714\\
3.978	2.378808\\
3.98	2.380909\\
3.982	2.383017\\
3.984	2.38513\\
3.986	2.38725\\
3.988	2.389376\\
3.99	2.391509\\
3.992	2.393648\\
3.994	2.395792\\
3.996	2.397943\\
3.998	2.4001\\
4	2.402263\\
};
\addlegendentry{$\phi$};

\addplot [color=mycolor2,solid]
  table[row sep=crcr]{%
0	0.02\\
0.002	0.04\\
0.004	0.06\\
0.006	0.08\\
0.008	0.099999\\
0.01	0.119999\\
0.012	0.139998\\
0.014	0.159997\\
0.016	0.179996\\
0.018	0.199995\\
0.02	0.219993\\
0.022	0.239991\\
0.024	0.259988\\
0.026	0.279985\\
0.028	0.299982\\
0.03	0.319978\\
0.032	0.339974\\
0.034	0.359969\\
0.036	0.379964\\
0.038	0.399958\\
0.04	0.419951\\
0.042	0.439944\\
0.044	0.459936\\
0.046	0.479927\\
0.048	0.499918\\
0.05	0.519908\\
0.052	0.539897\\
0.054	0.559885\\
0.056	0.579872\\
0.058	0.599858\\
0.06	0.619844\\
0.062	0.639828\\
0.064	0.659812\\
0.066	0.679794\\
0.068	0.699776\\
0.07	0.719756\\
0.072	0.739736\\
0.074	0.759714\\
0.076	0.779691\\
0.078	0.799667\\
0.08	0.819642\\
0.082	0.839615\\
0.084	0.859587\\
0.086	0.879558\\
0.088	0.899528\\
0.09	0.919496\\
0.092	0.939463\\
0.094	0.959429\\
0.096	0.979393\\
0.098	0.999356\\
0.1	1.019317\\
0.102	1.039277\\
0.104	1.059235\\
0.106	1.079192\\
0.108	1.099147\\
0.11	1.119101\\
0.112	1.139053\\
0.114	1.159003\\
0.116	1.178952\\
0.118	1.198899\\
0.12	1.218845\\
0.122	1.238789\\
0.124	1.258731\\
0.126	1.278671\\
0.128	1.29861\\
0.13	1.318547\\
0.132	1.338482\\
0.134	1.358415\\
0.136	1.378346\\
0.138	1.398276\\
0.14	1.418204\\
0.142	1.43813\\
0.144	1.458053\\
0.146	1.477976\\
0.148	1.497896\\
0.15	1.517814\\
0.152	1.53773\\
0.154	1.557644\\
0.156	1.577557\\
0.158	1.597467\\
0.16	1.617375\\
0.162	1.637281\\
0.164	1.657185\\
0.166	1.677088\\
0.168	1.696988\\
0.17	1.716886\\
0.172	1.736782\\
0.174	1.756675\\
0.176	1.776567\\
0.178	1.796457\\
0.18	1.816344\\
0.182	1.836229\\
0.184	1.856113\\
0.186	1.875994\\
0.188	1.895873\\
0.19	1.915749\\
0.192	1.935624\\
0.194	1.955496\\
0.196	1.975366\\
0.198	1.995234\\
0.2	2.0151\\
0.202	2.034964\\
0.204	2.054825\\
0.206	2.074684\\
0.208	2.094541\\
0.21	2.114396\\
0.212	2.134248\\
0.214	2.154099\\
0.216	2.173947\\
0.218	2.193793\\
0.22	2.213636\\
0.222	2.233478\\
0.224	2.253317\\
0.226	2.273154\\
0.228	2.292988\\
0.23	2.312821\\
0.232	2.332651\\
0.234	2.352479\\
0.236	2.372305\\
0.238	2.392128\\
0.24	2.41195\\
0.242	2.431769\\
0.244	2.451586\\
0.246	2.4714\\
0.248	2.491213\\
0.25	2.511023\\
0.252	2.530831\\
0.254	2.550637\\
0.256	2.570441\\
0.258	2.590242\\
0.26	2.610041\\
0.262	2.629839\\
0.264	2.649634\\
0.266	2.669426\\
0.268	2.689217\\
0.27	2.709006\\
0.272	2.728792\\
0.274	2.748576\\
0.276	2.768358\\
0.278	2.788138\\
0.28	2.807916\\
0.282	2.827692\\
0.284	2.847466\\
0.286	2.867238\\
0.288	2.887007\\
0.29	2.906775\\
0.292	2.926541\\
0.294	2.946304\\
0.296	2.966066\\
0.298	2.985825\\
0.3	3.005583\\
0.302	3.025338\\
0.304	3.045092\\
0.306	3.064844\\
0.308	3.084593\\
0.31	3.104341\\
0.312	3.124087\\
0.314	3.143831\\
0.316	3.163574\\
0.318	3.183314\\
0.32	3.203052\\
0.322	3.222789\\
0.324	3.242524\\
0.326	3.262257\\
0.328	3.281988\\
0.33	3.301718\\
0.332	3.321445\\
0.334	3.341171\\
0.336	3.360896\\
0.338	3.380618\\
0.34	3.400339\\
0.342	3.420059\\
0.344	3.439776\\
0.346	3.459492\\
0.348	3.479207\\
0.35	3.498919\\
0.352	3.518631\\
0.354	3.53834\\
0.356	3.558049\\
0.358	3.577755\\
0.36	3.59746\\
0.362	3.617164\\
0.364	3.636866\\
0.366	3.656567\\
0.368	3.676267\\
0.37	3.695965\\
0.372	3.715661\\
0.374	3.735357\\
0.376	3.755051\\
0.378	3.774743\\
0.38	3.794435\\
0.382	3.814125\\
0.384	3.833814\\
0.386	3.853501\\
0.388	3.873188\\
0.39	3.892873\\
0.392	3.912557\\
0.394	3.93224\\
0.396	3.951922\\
0.398	3.971603\\
0.4	3.991282\\
0.402	4.010961\\
0.404	4.030639\\
0.406	4.050315\\
0.408	4.069991\\
0.41	4.089666\\
0.412	4.109339\\
0.414	4.129012\\
0.416	4.148684\\
0.418	4.168355\\
0.42	4.188025\\
0.422	4.207694\\
0.424	4.227363\\
0.426	4.247031\\
0.428	4.266698\\
0.43	4.286364\\
0.432	4.30603\\
0.434	4.325694\\
0.436	4.345359\\
0.438	4.365022\\
0.44	4.384685\\
0.442	4.404347\\
0.444	4.424009\\
0.446	4.44367\\
0.448	4.463331\\
0.45	4.482991\\
0.452	4.502651\\
0.454	4.52231\\
0.456	4.541968\\
0.458	4.561627\\
0.46	4.581285\\
0.462	4.600942\\
0.464	4.620599\\
0.466	4.640256\\
0.468	4.659912\\
0.47	4.679569\\
0.472	4.699225\\
0.474	4.71888\\
0.476	4.738536\\
0.478	4.758191\\
0.48	4.777846\\
0.482	4.797501\\
0.484	4.817155\\
0.486	4.83681\\
0.488	4.856464\\
0.49	4.876119\\
0.492	4.895773\\
0.494	4.915427\\
0.496	4.935082\\
0.498	4.954736\\
0.5	4.97439\\
0.502	4.994044\\
0.504	5.013699\\
0.506	5.033353\\
0.508	5.053008\\
0.51	5.072662\\
0.512	5.092317\\
0.514	5.111972\\
0.516	5.131627\\
0.518	5.151282\\
0.52	5.170938\\
0.522	5.190594\\
0.524	5.210249\\
0.526	5.229906\\
0.528	5.249562\\
0.53	5.269219\\
0.532	5.288876\\
0.534	5.308533\\
0.536	5.328191\\
0.538	5.347849\\
0.54	5.367508\\
0.542	5.387167\\
0.544	5.406826\\
0.546	5.426486\\
0.548	5.446146\\
0.55	5.465806\\
0.552	5.485468\\
0.554	5.505129\\
0.556	5.524791\\
0.558	5.544454\\
0.56	5.564117\\
0.562	5.583781\\
0.564	5.603445\\
0.566	5.62311\\
0.568	5.642775\\
0.57	5.662441\\
0.572	5.682107\\
0.574	5.701775\\
0.576	5.721442\\
0.578	5.741111\\
0.58	5.76078\\
0.582	5.78045\\
0.584	5.80012\\
0.586	5.819791\\
0.588	5.839463\\
0.59	5.859136\\
0.592	5.878809\\
0.594	5.898483\\
0.596	5.918158\\
0.598	5.937833\\
0.6	5.95751\\
0.602	5.977187\\
0.604	5.996864\\
0.606	6.016543\\
0.608	6.036222\\
0.61	6.055903\\
0.612	6.075583\\
0.614	6.095265\\
0.616	6.114948\\
0.618	6.134631\\
0.62	6.154315\\
0.622	6.174001\\
0.624	6.193686\\
0.626	6.213373\\
0.628	6.233061\\
0.63	6.252749\\
0.632	6.272439\\
0.634	6.292129\\
0.636	6.31182\\
0.638	6.331512\\
0.64	6.351205\\
0.642	6.370898\\
0.644	6.390593\\
0.646	6.410288\\
0.648	6.429985\\
0.65	6.449682\\
0.652	6.46938\\
0.654	6.489079\\
0.656	6.508779\\
0.658	6.52848\\
0.66	6.548181\\
0.662	6.567884\\
0.664	6.587587\\
0.666	6.607291\\
0.668	6.626997\\
0.67	6.646703\\
0.672	6.66641\\
0.674	6.686118\\
0.676	6.705826\\
0.678	6.725536\\
0.68	6.745246\\
0.682	6.764958\\
0.684	6.78467\\
0.686	6.804383\\
0.688	6.824097\\
0.69	6.843812\\
0.692	6.863528\\
0.694	6.883244\\
0.696	6.902962\\
0.698	6.92268\\
0.7	6.942399\\
0.702	6.962119\\
0.704	6.98184\\
0.706	7.001561\\
0.708	7.021283\\
0.71	7.041007\\
0.712	7.060731\\
0.714	7.080455\\
0.716	7.100181\\
0.718	7.119907\\
0.72	7.139634\\
0.722	7.159362\\
0.724	7.179091\\
0.726	7.19882\\
0.728	7.21855\\
0.73	7.238281\\
0.732	7.258012\\
0.734	7.277745\\
0.736	7.297478\\
0.738	7.317211\\
0.74	7.336945\\
0.742	7.35668\\
0.744	7.376416\\
0.746	7.396152\\
0.748	7.415889\\
0.75	7.435627\\
0.752	7.455365\\
0.754	7.475104\\
0.756	7.494843\\
0.758	7.514583\\
0.76	7.534323\\
0.762	7.554064\\
0.764	7.573806\\
0.766	7.593548\\
0.768	7.613291\\
0.77	7.633034\\
0.772	7.652777\\
0.774	7.672522\\
0.776	7.692266\\
0.778	7.712011\\
0.78	7.731757\\
0.782	7.751503\\
0.784	7.771249\\
0.786	7.790996\\
0.788	7.810743\\
0.79	7.83049\\
0.792	7.850238\\
0.794	7.869986\\
0.796	7.889735\\
0.798	7.909483\\
0.8	7.929232\\
0.802	7.948982\\
0.804	7.968732\\
0.806	7.988481\\
0.808	8.008232\\
0.81	8.027982\\
0.812	8.047732\\
0.814	8.067483\\
0.816	8.087234\\
0.818	8.106985\\
0.82	8.126737\\
0.822	8.146488\\
0.824	8.166239\\
0.826	8.185991\\
0.828	8.205743\\
0.83	8.225494\\
0.832	8.245246\\
0.834	8.264998\\
0.836	8.28475\\
0.838	8.304502\\
0.84	8.324254\\
0.842	8.344006\\
0.844	8.363757\\
0.846	8.383509\\
0.848	8.403261\\
0.85	8.423012\\
0.852	8.442764\\
0.854	8.462515\\
0.856	8.482267\\
0.858	8.502018\\
0.86	8.521769\\
0.862	8.541519\\
0.864	8.56127\\
0.866	8.58102\\
0.868	8.600771\\
0.87	8.62052\\
0.872	8.64027\\
0.874	8.660019\\
0.876	8.679769\\
0.878	8.699517\\
0.88	8.719266\\
0.882	8.739014\\
0.884	8.758762\\
0.886	8.778509\\
0.888	8.798256\\
0.89	8.818003\\
0.892	8.837749\\
0.894	8.857495\\
0.896	8.877241\\
0.898	8.896986\\
0.9	8.91673\\
0.902	8.936475\\
0.904	8.956218\\
0.906	8.975961\\
0.908	8.995704\\
0.91	9.015446\\
0.912	9.035188\\
0.914	9.054929\\
0.916	9.074669\\
0.918	9.094409\\
0.92	9.114149\\
0.922	9.133887\\
0.924	9.153625\\
0.926	9.173363\\
0.928	9.1931\\
0.93	9.212836\\
0.932	9.232572\\
0.934	9.252307\\
0.936	9.272041\\
0.938	9.291775\\
0.94	9.311508\\
0.942	9.33124\\
0.944	9.350971\\
0.946	9.370702\\
0.948	9.390432\\
0.95	9.410162\\
0.952	9.42989\\
0.954	9.449618\\
0.956	9.469345\\
0.958	9.489071\\
0.96	9.508797\\
0.962	9.528522\\
0.964	9.548246\\
0.966	9.567969\\
0.968	9.587691\\
0.97	9.607413\\
0.972	9.627134\\
0.974	9.646853\\
0.976	9.666573\\
0.978	9.686291\\
0.98	9.706008\\
0.982	9.725725\\
0.984	9.74544\\
0.986	9.765155\\
0.988	9.784869\\
0.99	9.804582\\
0.992	9.824295\\
0.994	9.844006\\
0.996	9.863717\\
0.998	9.883426\\
1	9.903135\\
1.002	9.922843\\
1.004	9.94255\\
1.006	9.962256\\
1.008	9.981961\\
1.01	10.001665\\
1.012	10.021369\\
1.014	10.041071\\
1.016	10.060773\\
1.018	10.080474\\
1.02	10.100174\\
1.022	10.119873\\
1.024	10.139571\\
1.026	10.159268\\
1.028	10.178964\\
1.03	10.19866\\
1.032	10.218355\\
1.034	10.238048\\
1.036	10.257741\\
1.038	10.277433\\
1.04	10.297124\\
1.042	10.316814\\
1.044	10.336504\\
1.046	10.356192\\
1.048	10.37588\\
1.05	10.395566\\
1.052	10.415252\\
1.054	10.434937\\
1.056	10.454622\\
1.058	10.474305\\
1.06	10.493988\\
1.062	10.51367\\
1.064	10.53335\\
1.066	10.553031\\
1.068	10.57271\\
1.07	10.592389\\
1.072	10.612066\\
1.074	10.631743\\
1.076	10.65142\\
1.078	10.671095\\
1.08	10.69077\\
1.082	10.710444\\
1.084	10.730117\\
1.086	10.74979\\
1.088	10.769462\\
1.09	10.789133\\
1.092	10.808803\\
1.094	10.828473\\
1.096	10.848142\\
1.098	10.867811\\
1.1	10.887479\\
1.102	10.907146\\
1.104	10.926812\\
1.106	10.946478\\
1.108	10.966144\\
1.11	10.985808\\
1.112	11.005473\\
1.114	11.025136\\
1.116	11.044799\\
1.118	11.064462\\
1.12	11.084124\\
1.122	11.103786\\
1.124	11.123447\\
1.126	11.143107\\
1.128	11.162768\\
1.13	11.182427\\
1.132	11.202087\\
1.134	11.221746\\
1.136	11.241404\\
1.138	11.261062\\
1.14	11.28072\\
1.142	11.300377\\
1.144	11.320034\\
1.146	11.339691\\
1.148	11.359348\\
1.15	11.379004\\
1.152	11.39866\\
1.154	11.418316\\
1.156	11.437971\\
1.158	11.457626\\
1.16	11.477281\\
1.162	11.496936\\
1.164	11.516591\\
1.166	11.536246\\
1.168	11.5559\\
1.17	11.575555\\
1.172	11.595209\\
1.174	11.614863\\
1.176	11.634518\\
1.178	11.654172\\
1.18	11.673826\\
1.182	11.69348\\
1.184	11.713135\\
1.186	11.732789\\
1.188	11.752443\\
1.19	11.772098\\
1.192	11.791753\\
1.194	11.811408\\
1.196	11.831063\\
1.198	11.850718\\
1.2	11.870373\\
1.202	11.890029\\
1.204	11.909685\\
1.206	11.929341\\
1.208	11.948997\\
1.21	11.968654\\
1.212	11.988311\\
1.214	12.007969\\
1.216	12.027627\\
1.218	12.047285\\
1.22	12.066944\\
1.222	12.086603\\
1.224	12.106262\\
1.226	12.125922\\
1.228	12.145583\\
1.23	12.165244\\
1.232	12.184906\\
1.234	12.204568\\
1.236	12.224231\\
1.238	12.243895\\
1.24	12.263559\\
1.242	12.283224\\
1.244	12.302889\\
1.246	12.322555\\
1.248	12.342222\\
1.25	12.36189\\
1.252	12.381559\\
1.254	12.401228\\
1.256	12.420898\\
1.258	12.440569\\
1.26	12.460241\\
1.262	12.479914\\
1.264	12.499588\\
1.266	12.519262\\
1.268	12.538938\\
1.27	12.558614\\
1.272	12.578292\\
1.274	12.597971\\
1.276	12.61765\\
1.278	12.637331\\
1.28	12.657013\\
1.282	12.676696\\
1.284	12.69638\\
1.286	12.716065\\
1.288	12.735752\\
1.29	12.755439\\
1.292	12.775128\\
1.294	12.794818\\
1.296	12.81451\\
1.298	12.834202\\
1.3	12.853896\\
1.302	12.873591\\
1.304	12.893288\\
1.306	12.912986\\
1.308	12.932685\\
1.31	12.952386\\
1.312	12.972089\\
1.314	12.991792\\
1.316	13.011497\\
1.318	13.031204\\
1.32	13.050912\\
1.322	13.070622\\
1.324	13.090333\\
1.326	13.110046\\
1.328	13.12976\\
1.33	13.149476\\
1.332	13.169194\\
1.334	13.188913\\
1.336	13.208634\\
1.338	13.228357\\
1.34	13.248081\\
1.342	13.267807\\
1.344	13.287535\\
1.346	13.307264\\
1.348	13.326995\\
1.35	13.346728\\
1.352	13.366463\\
1.354	13.3862\\
1.356	13.405938\\
1.358	13.425679\\
1.36	13.445421\\
1.362	13.465165\\
1.364	13.484911\\
1.366	13.504658\\
1.368	13.524408\\
1.37	13.54416\\
1.372	13.563913\\
1.374	13.583669\\
1.376	13.603427\\
1.378	13.623186\\
1.38	13.642948\\
1.382	13.662711\\
1.384	13.682477\\
1.386	13.702244\\
1.388	13.722014\\
1.39	13.741786\\
1.392	13.761559\\
1.394	13.781335\\
1.396	13.801113\\
1.398	13.820893\\
1.4	13.840675\\
1.402	13.86046\\
1.404	13.880246\\
1.406	13.900034\\
1.408	13.919825\\
1.41	13.939618\\
1.412	13.959413\\
1.414	13.97921\\
1.416	13.999009\\
1.418	14.018811\\
1.42	14.038614\\
1.422	14.05842\\
1.424	14.078228\\
1.426	14.098038\\
1.428	14.117851\\
1.43	14.137665\\
1.432	14.157482\\
1.434	14.177301\\
1.436	14.197123\\
1.438	14.216946\\
1.44	14.236772\\
1.442	14.2566\\
1.444	14.27643\\
1.446	14.296262\\
1.448	14.316097\\
1.45	14.335934\\
1.452	14.355773\\
1.454	14.375614\\
1.456	14.395458\\
1.458	14.415304\\
1.46	14.435152\\
1.462	14.455002\\
1.464	14.474854\\
1.466	14.494709\\
1.468	14.514566\\
1.47	14.534425\\
1.472	14.554287\\
1.474	14.57415\\
1.476	14.594016\\
1.478	14.613884\\
1.48	14.633754\\
1.482	14.653626\\
1.484	14.673501\\
1.486	14.693378\\
1.488	14.713256\\
1.49	14.733137\\
1.492	14.753021\\
1.494	14.772906\\
1.496	14.792793\\
1.498	14.812683\\
1.5	14.832574\\
1.502	14.852468\\
1.504	14.872364\\
1.506	14.892262\\
1.508	14.912162\\
1.51	14.932064\\
1.512	14.951968\\
1.514	14.971875\\
1.516	14.991783\\
1.518	15.011693\\
1.52	15.031605\\
1.522	15.051519\\
1.524	15.071436\\
1.526	15.091354\\
1.528	15.111274\\
1.53	15.131196\\
1.532	15.15112\\
1.534	15.171046\\
1.536	15.190973\\
1.538	15.210903\\
1.54	15.230834\\
1.542	15.250768\\
1.544	15.270703\\
1.546	15.290639\\
1.548	15.310578\\
1.55	15.330518\\
1.552	15.35046\\
1.554	15.370404\\
1.556	15.39035\\
1.558	15.410297\\
1.56	15.430246\\
1.562	15.450196\\
1.564	15.470148\\
1.566	15.490102\\
1.568	15.510057\\
1.57	15.530014\\
1.572	15.549972\\
1.574	15.569932\\
1.576	15.589893\\
1.578	15.609856\\
1.58	15.62982\\
1.582	15.649786\\
1.584	15.669753\\
1.586	15.689721\\
1.588	15.709691\\
1.59	15.729661\\
1.592	15.749634\\
1.594	15.769607\\
1.596	15.789582\\
1.598	15.809558\\
1.6	15.829535\\
1.602	15.849513\\
1.604	15.869492\\
1.606	15.889473\\
1.608	15.909454\\
1.61	15.929437\\
1.612	15.94942\\
1.614	15.969405\\
1.616	15.98939\\
1.618	16.009376\\
1.62	16.029364\\
1.622	16.049352\\
1.624	16.069341\\
1.626	16.089331\\
1.628	16.109321\\
1.63	16.129312\\
1.632	16.149304\\
1.634	16.169297\\
1.636	16.189291\\
1.638	16.209285\\
1.64	16.229279\\
1.642	16.249274\\
1.644	16.26927\\
1.646	16.289266\\
1.648	16.309263\\
1.65	16.32926\\
1.652	16.349257\\
1.654	16.369255\\
1.656	16.389254\\
1.658	16.409252\\
1.66	16.429251\\
1.662	16.44925\\
1.664	16.469249\\
1.666	16.489249\\
1.668	16.509249\\
1.67	16.529248\\
1.672	16.549248\\
1.674	16.569248\\
1.676	16.589248\\
1.678	16.609248\\
1.68	16.629248\\
1.682	16.649248\\
1.684	16.669248\\
1.686	16.689248\\
1.688	16.709247\\
1.69	16.729246\\
1.692	16.749246\\
1.694	16.769244\\
1.696	16.789243\\
1.698	16.809241\\
1.7	16.829239\\
1.702	16.849237\\
1.704	16.869234\\
1.706	16.88923\\
1.708	16.909227\\
1.71	16.929222\\
1.712	16.949218\\
1.714	16.969212\\
1.716	16.989206\\
1.718	17.0092\\
1.72	17.029192\\
1.722	17.049184\\
1.724	17.069176\\
1.726	17.089166\\
1.728	17.109156\\
1.73	17.129145\\
1.732	17.149133\\
1.734	17.16912\\
1.736	17.189107\\
1.738	17.209092\\
1.74	17.229077\\
1.742	17.24906\\
1.744	17.269043\\
1.746	17.289024\\
1.748	17.309005\\
1.75	17.328984\\
1.752	17.348962\\
1.754	17.36894\\
1.756	17.388915\\
1.758	17.40889\\
1.76	17.428864\\
1.762	17.448836\\
1.764	17.468807\\
1.766	17.488777\\
1.768	17.508745\\
1.77	17.528712\\
1.772	17.548677\\
1.774	17.568642\\
1.776	17.588604\\
1.778	17.608566\\
1.78	17.628526\\
1.782	17.648484\\
1.784	17.668441\\
1.786	17.688396\\
1.788	17.70835\\
1.79	17.728302\\
1.792	17.748252\\
1.794	17.768201\\
1.796	17.788148\\
1.798	17.808094\\
1.8	17.828038\\
1.802	17.84798\\
1.804	17.86792\\
1.806	17.887859\\
1.808	17.907796\\
1.81	17.927731\\
1.812	17.947664\\
1.814	17.967596\\
1.816	17.987525\\
1.818	18.007453\\
1.82	18.027379\\
1.822	18.047303\\
1.824	18.067225\\
1.826	18.087145\\
1.828	18.107063\\
1.83	18.12698\\
1.832	18.146894\\
1.834	18.166806\\
1.836	18.186716\\
1.838	18.206625\\
1.84	18.226531\\
1.842	18.246435\\
1.844	18.266337\\
1.846	18.286237\\
1.848	18.306135\\
1.85	18.326031\\
1.852	18.345925\\
1.854	18.365817\\
1.856	18.385707\\
1.858	18.405594\\
1.86	18.425479\\
1.862	18.445363\\
1.864	18.465244\\
1.866	18.485123\\
1.868	18.504999\\
1.87	18.524874\\
1.872	18.544746\\
1.874	18.564617\\
1.876	18.584485\\
1.878	18.60435\\
1.88	18.624214\\
1.882	18.644075\\
1.884	18.663935\\
1.886	18.683792\\
1.888	18.703646\\
1.89	18.723499\\
1.892	18.743349\\
1.894	18.763197\\
1.896	18.783043\\
1.898	18.802887\\
1.9	18.822728\\
1.902	18.842567\\
1.904	18.862404\\
1.906	18.882239\\
1.908	18.902072\\
1.91	18.921902\\
1.912	18.94173\\
1.914	18.961556\\
1.916	18.981379\\
1.918	19.001201\\
1.92	19.02102\\
1.922	19.040837\\
1.924	19.060651\\
1.926	19.080464\\
1.928	19.100274\\
1.93	19.120082\\
1.932	19.139888\\
1.934	19.159692\\
1.936	19.179493\\
1.938	19.199293\\
1.94	19.21909\\
1.942	19.238885\\
1.944	19.258678\\
1.946	19.278468\\
1.948	19.298257\\
1.95	19.318043\\
1.952	19.337828\\
1.954	19.35761\\
1.956	19.37739\\
1.958	19.397168\\
1.96	19.416944\\
1.962	19.436718\\
1.964	19.456489\\
1.966	19.476259\\
1.968	19.496027\\
1.97	19.515792\\
1.972	19.535556\\
1.974	19.555317\\
1.976	19.575077\\
1.978	19.594835\\
1.98	19.61459\\
1.982	19.634344\\
1.984	19.654096\\
1.986	19.673845\\
1.988	19.693593\\
1.99	19.713339\\
1.992	19.733083\\
1.994	19.752826\\
1.996	19.772566\\
1.998	19.792304\\
2	19.812041\\
2.002	19.831776\\
2.004	19.851509\\
2.006	19.87124\\
2.008	19.89097\\
2.01	19.910698\\
2.012	19.930424\\
2.014	19.950148\\
2.016	19.969871\\
2.018	19.989592\\
2.02	20.009311\\
2.022	20.029029\\
2.024	20.048745\\
2.026	20.068459\\
2.028	20.088172\\
2.03	20.107883\\
2.032	20.127593\\
2.034	20.147301\\
2.036	20.167008\\
2.038	20.186713\\
2.04	20.206417\\
2.042	20.226119\\
2.044	20.24582\\
2.046	20.265519\\
2.048	20.285217\\
2.05	20.304914\\
2.052	20.324609\\
2.054	20.344303\\
2.056	20.363996\\
2.058	20.383687\\
2.06	20.403378\\
2.062	20.423067\\
2.064	20.442754\\
2.066	20.462441\\
2.068	20.482126\\
2.07	20.50181\\
2.072	20.521493\\
2.074	20.541175\\
2.076	20.560856\\
2.078	20.580535\\
2.08	20.600214\\
2.082	20.619892\\
2.084	20.639568\\
2.086	20.659244\\
2.088	20.678919\\
2.09	20.698592\\
2.092	20.718265\\
2.094	20.737937\\
2.096	20.757608\\
2.098	20.777278\\
2.1	20.796948\\
2.102	20.816616\\
2.104	20.836284\\
2.106	20.855951\\
2.108	20.875617\\
2.11	20.895283\\
2.112	20.914948\\
2.114	20.934612\\
2.116	20.954275\\
2.118	20.973938\\
2.12	20.993601\\
2.122	21.013262\\
2.124	21.032924\\
2.126	21.052584\\
2.128	21.072244\\
2.13	21.091904\\
2.132	21.111563\\
2.134	21.131222\\
2.136	21.15088\\
2.138	21.170538\\
2.14	21.190195\\
2.142	21.209853\\
2.144	21.229509\\
2.146	21.249166\\
2.148	21.268822\\
2.15	21.288478\\
2.152	21.308134\\
2.154	21.327789\\
2.156	21.347444\\
2.158	21.367099\\
2.16	21.386754\\
2.162	21.406409\\
2.164	21.426063\\
2.166	21.445718\\
2.168	21.465372\\
2.17	21.485026\\
2.172	21.504681\\
2.174	21.524335\\
2.176	21.543989\\
2.178	21.563644\\
2.18	21.583298\\
2.182	21.602952\\
2.184	21.622607\\
2.186	21.642261\\
2.188	21.661916\\
2.19	21.68157\\
2.192	21.701225\\
2.194	21.72088\\
2.196	21.740536\\
2.198	21.760191\\
2.2	21.779847\\
2.202	21.799503\\
2.204	21.819159\\
2.206	21.838815\\
2.208	21.858472\\
2.21	21.878129\\
2.212	21.897787\\
2.214	21.917444\\
2.216	21.937103\\
2.218	21.956761\\
2.22	21.97642\\
2.222	21.996079\\
2.224	22.015739\\
2.226	22.035399\\
2.228	22.05506\\
2.23	22.074721\\
2.232	22.094382\\
2.234	22.114044\\
2.236	22.133707\\
2.238	22.15337\\
2.24	22.173034\\
2.242	22.192698\\
2.244	22.212363\\
2.246	22.232028\\
2.248	22.251694\\
2.25	22.271361\\
2.252	22.291028\\
2.254	22.310696\\
2.256	22.330364\\
2.258	22.350033\\
2.26	22.369703\\
2.262	22.389373\\
2.264	22.409045\\
2.266	22.428716\\
2.268	22.448389\\
2.27	22.468062\\
2.272	22.487736\\
2.274	22.507411\\
2.276	22.527086\\
2.278	22.546763\\
2.28	22.56644\\
2.282	22.586117\\
2.284	22.605796\\
2.286	22.625475\\
2.288	22.645155\\
2.29	22.664836\\
2.292	22.684518\\
2.294	22.704201\\
2.296	22.723884\\
2.298	22.743568\\
2.3	22.763253\\
2.302	22.782939\\
2.304	22.802626\\
2.306	22.822314\\
2.308	22.842002\\
2.31	22.861691\\
2.312	22.881382\\
2.314	22.901073\\
2.316	22.920765\\
2.318	22.940457\\
2.32	22.960151\\
2.322	22.979846\\
2.324	22.999541\\
2.326	23.019237\\
2.328	23.038935\\
2.33	23.058633\\
2.332	23.078332\\
2.334	23.098031\\
2.336	23.117732\\
2.338	23.137434\\
2.34	23.157136\\
2.342	23.17684\\
2.344	23.196544\\
2.346	23.216249\\
2.348	23.235955\\
2.35	23.255662\\
2.352	23.27537\\
2.354	23.295079\\
2.356	23.314789\\
2.358	23.334499\\
2.36	23.35421\\
2.362	23.373923\\
2.364	23.393636\\
2.366	23.41335\\
2.368	23.433065\\
2.37	23.45278\\
2.372	23.472497\\
2.374	23.492214\\
2.376	23.511932\\
2.378	23.531651\\
2.38	23.551371\\
2.382	23.571092\\
2.384	23.590813\\
2.386	23.610536\\
2.388	23.630259\\
2.39	23.649983\\
2.392	23.669708\\
2.394	23.689433\\
2.396	23.709159\\
2.398	23.728887\\
2.4	23.748614\\
2.402	23.768343\\
2.404	23.788072\\
2.406	23.807802\\
2.408	23.827533\\
2.41	23.847265\\
2.412	23.866997\\
2.414	23.88673\\
2.416	23.906463\\
2.418	23.926198\\
2.42	23.945933\\
2.422	23.965668\\
2.424	23.985404\\
2.426	24.005141\\
2.428	24.024879\\
2.43	24.044617\\
2.432	24.064356\\
2.434	24.084095\\
2.436	24.103835\\
2.438	24.123575\\
2.44	24.143316\\
2.442	24.163058\\
2.444	24.1828\\
2.446	24.202543\\
2.448	24.222286\\
2.45	24.24203\\
2.452	24.261774\\
2.454	24.281518\\
2.456	24.301263\\
2.458	24.321009\\
2.46	24.340755\\
2.462	24.360501\\
2.464	24.380248\\
2.466	24.399995\\
2.468	24.419742\\
2.47	24.43949\\
2.472	24.459238\\
2.474	24.478987\\
2.476	24.498735\\
2.478	24.518484\\
2.48	24.538234\\
2.482	24.557983\\
2.484	24.577733\\
2.486	24.597484\\
2.488	24.617234\\
2.49	24.636984\\
2.492	24.656735\\
2.494	24.676486\\
2.496	24.696237\\
2.498	24.715989\\
2.5	24.73574\\
2.502	24.755491\\
2.504	24.775243\\
2.506	24.794995\\
2.508	24.814746\\
2.51	24.834498\\
2.512	24.85425\\
2.514	24.874002\\
2.516	24.893754\\
2.518	24.913506\\
2.52	24.933258\\
2.522	24.953009\\
2.524	24.972761\\
2.526	24.992513\\
2.528	25.012264\\
2.53	25.032016\\
2.532	25.051767\\
2.534	25.071519\\
2.536	25.09127\\
2.538	25.111021\\
2.54	25.130771\\
2.542	25.150522\\
2.544	25.170272\\
2.546	25.190023\\
2.548	25.209772\\
2.55	25.229522\\
2.552	25.249271\\
2.554	25.269021\\
2.556	25.288769\\
2.558	25.308518\\
2.56	25.328266\\
2.562	25.348014\\
2.564	25.367761\\
2.566	25.387508\\
2.568	25.407255\\
2.57	25.427001\\
2.572	25.446747\\
2.574	25.466493\\
2.576	25.486238\\
2.578	25.505982\\
2.58	25.525727\\
2.582	25.54547\\
2.584	25.565213\\
2.586	25.584956\\
2.588	25.604698\\
2.59	25.62444\\
2.592	25.644181\\
2.594	25.663921\\
2.596	25.683661\\
2.598	25.703401\\
2.6	25.723139\\
2.602	25.742878\\
2.604	25.762615\\
2.606	25.782352\\
2.608	25.802088\\
2.61	25.821824\\
2.612	25.841559\\
2.614	25.861293\\
2.616	25.881027\\
2.618	25.90076\\
2.62	25.920492\\
2.622	25.940224\\
2.624	25.959954\\
2.626	25.979685\\
2.628	25.999414\\
2.63	26.019142\\
2.632	26.03887\\
2.634	26.058597\\
2.636	26.078324\\
2.638	26.098049\\
2.64	26.117774\\
2.642	26.137498\\
2.644	26.157221\\
2.646	26.176944\\
2.648	26.196665\\
2.65	26.216386\\
2.652	26.236106\\
2.654	26.255825\\
2.656	26.275543\\
2.658	26.295261\\
2.66	26.314977\\
2.662	26.334693\\
2.664	26.354408\\
2.666	26.374122\\
2.668	26.393835\\
2.67	26.413547\\
2.672	26.433259\\
2.674	26.452969\\
2.676	26.472679\\
2.678	26.492388\\
2.68	26.512095\\
2.682	26.531802\\
2.684	26.551509\\
2.686	26.571214\\
2.688	26.590918\\
2.69	26.610622\\
2.692	26.630324\\
2.694	26.650026\\
2.696	26.669727\\
2.698	26.689427\\
2.7	26.709126\\
2.702	26.728824\\
2.704	26.748521\\
2.706	26.768217\\
2.708	26.787913\\
2.71	26.807607\\
2.712	26.827301\\
2.714	26.846994\\
2.716	26.866686\\
2.718	26.886377\\
2.72	26.906067\\
2.722	26.925756\\
2.724	26.945445\\
2.726	26.965133\\
2.728	26.984819\\
2.73	27.004505\\
2.732	27.02419\\
2.734	27.043875\\
2.736	27.063558\\
2.738	27.083241\\
2.74	27.102922\\
2.742	27.122603\\
2.744	27.142284\\
2.746	27.161963\\
2.748	27.181642\\
2.75	27.201319\\
2.752	27.220996\\
2.754	27.240673\\
2.756	27.260348\\
2.758	27.280023\\
2.76	27.299697\\
2.762	27.31937\\
2.764	27.339043\\
2.766	27.358715\\
2.768	27.378386\\
2.77	27.398056\\
2.772	27.417726\\
2.774	27.437395\\
2.776	27.457064\\
2.778	27.476732\\
2.78	27.496399\\
2.782	27.516066\\
2.784	27.535732\\
2.786	27.555397\\
2.788	27.575062\\
2.79	27.594726\\
2.792	27.61439\\
2.794	27.634053\\
2.796	27.653715\\
2.798	27.673377\\
2.8	27.693039\\
2.802	27.7127\\
2.804	27.732361\\
2.806	27.752021\\
2.808	27.771681\\
2.81	27.79134\\
2.812	27.810999\\
2.814	27.830657\\
2.816	27.850315\\
2.818	27.869973\\
2.82	27.889631\\
2.822	27.909288\\
2.824	27.928945\\
2.826	27.948601\\
2.828	27.968257\\
2.83	27.987913\\
2.832	28.007569\\
2.834	28.027224\\
2.836	28.04688\\
2.838	28.066535\\
2.84	28.08619\\
2.842	28.105844\\
2.844	28.125499\\
2.846	28.145154\\
2.848	28.164808\\
2.85	28.184462\\
2.852	28.204117\\
2.854	28.223771\\
2.856	28.243425\\
2.858	28.263079\\
2.86	28.282734\\
2.862	28.302388\\
2.864	28.322042\\
2.866	28.341697\\
2.868	28.361351\\
2.87	28.381006\\
2.872	28.400661\\
2.874	28.420316\\
2.876	28.439971\\
2.878	28.459627\\
2.88	28.479282\\
2.882	28.498938\\
2.884	28.518594\\
2.886	28.538251\\
2.888	28.557907\\
2.89	28.577565\\
2.892	28.597222\\
2.894	28.61688\\
2.896	28.636538\\
2.898	28.656197\\
2.9	28.675856\\
2.902	28.695516\\
2.904	28.715176\\
2.906	28.734836\\
2.908	28.754497\\
2.91	28.774159\\
2.912	28.793821\\
2.914	28.813484\\
2.916	28.833148\\
2.918	28.852812\\
2.92	28.872477\\
2.922	28.892142\\
2.924	28.911809\\
2.926	28.931476\\
2.928	28.951143\\
2.93	28.970812\\
2.932	28.990481\\
2.934	29.010151\\
2.936	29.029822\\
2.938	29.049494\\
2.94	29.069167\\
2.942	29.088841\\
2.944	29.108515\\
2.946	29.128191\\
2.948	29.147867\\
2.95	29.167545\\
2.952	29.187224\\
2.954	29.206903\\
2.956	29.226584\\
2.958	29.246266\\
2.96	29.265949\\
2.962	29.285633\\
2.964	29.305318\\
2.966	29.325004\\
2.968	29.344692\\
2.97	29.364381\\
2.972	29.384071\\
2.974	29.403762\\
2.976	29.423455\\
2.978	29.443149\\
2.98	29.462844\\
2.982	29.482541\\
2.984	29.502239\\
2.986	29.521938\\
2.988	29.541639\\
2.99	29.561341\\
2.992	29.581045\\
2.994	29.60075\\
2.996	29.620457\\
2.998	29.640165\\
3	29.659874\\
3.002	29.679586\\
3.004	29.699298\\
3.006	29.719013\\
3.008	29.738729\\
3.01	29.758446\\
3.012	29.778166\\
3.014	29.797886\\
3.016	29.817609\\
3.018	29.837333\\
3.02	29.857059\\
3.022	29.876787\\
3.024	29.896516\\
3.026	29.916248\\
3.028	29.935981\\
3.03	29.955715\\
3.032	29.975452\\
3.034	29.99519\\
3.036	30.014931\\
3.038	30.034673\\
3.04	30.054417\\
3.042	30.074163\\
3.044	30.09391\\
3.046	30.11366\\
3.048	30.133412\\
3.05	30.153165\\
3.052	30.172921\\
3.054	30.192678\\
3.056	30.212438\\
3.058	30.232199\\
3.06	30.251963\\
3.062	30.271728\\
3.064	30.291496\\
3.066	30.311266\\
3.068	30.331037\\
3.07	30.350811\\
3.072	30.370587\\
3.074	30.390365\\
3.076	30.410145\\
3.078	30.429927\\
3.08	30.449711\\
3.082	30.469497\\
3.084	30.489286\\
3.086	30.509076\\
3.088	30.528869\\
3.09	30.548664\\
3.092	30.568461\\
3.094	30.58826\\
3.096	30.608062\\
3.098	30.627865\\
3.1	30.647671\\
3.102	30.667479\\
3.104	30.687289\\
3.106	30.707102\\
3.108	30.726916\\
3.11	30.746733\\
3.112	30.766552\\
3.114	30.786373\\
3.116	30.806197\\
3.118	30.826023\\
3.12	30.845851\\
3.122	30.865681\\
3.124	30.885513\\
3.126	30.905348\\
3.128	30.925185\\
3.13	30.945024\\
3.132	30.964865\\
3.134	30.984708\\
3.136	31.004554\\
3.138	31.024402\\
3.14	31.044252\\
3.142	31.064105\\
3.144	31.08396\\
3.146	31.103816\\
3.148	31.123676\\
3.15	31.143537\\
3.152	31.1634\\
3.154	31.183266\\
3.156	31.203134\\
3.158	31.223004\\
3.16	31.242876\\
3.162	31.262751\\
3.164	31.282628\\
3.166	31.302506\\
3.168	31.322387\\
3.17	31.342271\\
3.172	31.362156\\
3.174	31.382043\\
3.176	31.401933\\
3.178	31.421824\\
3.18	31.441718\\
3.182	31.461614\\
3.184	31.481512\\
3.186	31.501412\\
3.188	31.521314\\
3.19	31.541218\\
3.192	31.561124\\
3.194	31.581032\\
3.196	31.600943\\
3.198	31.620855\\
3.2	31.640769\\
3.202	31.660685\\
3.204	31.680603\\
3.206	31.700523\\
3.208	31.720445\\
3.21	31.740369\\
3.212	31.760295\\
3.214	31.780223\\
3.216	31.800152\\
3.218	31.820084\\
3.22	31.840017\\
3.222	31.859952\\
3.224	31.879889\\
3.226	31.899827\\
3.228	31.919768\\
3.23	31.93971\\
3.232	31.959653\\
3.234	31.979599\\
3.236	31.999546\\
3.238	32.019495\\
3.24	32.039445\\
3.242	32.059397\\
3.244	32.079351\\
3.246	32.099306\\
3.248	32.119263\\
3.25	32.139221\\
3.252	32.159181\\
3.254	32.179142\\
3.256	32.199105\\
3.258	32.219069\\
3.26	32.239034\\
3.262	32.259001\\
3.264	32.27897\\
3.266	32.298939\\
3.268	32.31891\\
3.27	32.338882\\
3.272	32.358856\\
3.274	32.37883\\
3.276	32.398806\\
3.278	32.418783\\
3.28	32.438762\\
3.282	32.458741\\
3.284	32.478721\\
3.286	32.498703\\
3.288	32.518685\\
3.29	32.538669\\
3.292	32.558653\\
3.294	32.578639\\
3.296	32.598625\\
3.298	32.618612\\
3.3	32.6386\\
3.302	32.658589\\
3.304	32.678579\\
3.306	32.69857\\
3.308	32.718561\\
3.31	32.738553\\
3.312	32.758546\\
3.314	32.778539\\
3.316	32.798533\\
3.318	32.818527\\
3.32	32.838523\\
3.322	32.858518\\
3.324	32.878514\\
3.326	32.898511\\
3.328	32.918508\\
3.33	32.938506\\
3.332	32.958504\\
3.334	32.978502\\
3.336	32.9985\\
3.338	33.018499\\
3.34	33.038498\\
3.342	33.058498\\
3.344	33.078497\\
3.346	33.098497\\
3.348	33.118497\\
3.35	33.138497\\
3.352	33.158497\\
3.354	33.178497\\
3.356	33.198497\\
3.358	33.218497\\
3.36	33.238496\\
3.362	33.258496\\
3.364	33.278496\\
3.366	33.298495\\
3.368	33.318495\\
3.37	33.338494\\
3.372	33.358493\\
3.374	33.378491\\
3.376	33.39849\\
3.378	33.418487\\
3.38	33.438485\\
3.382	33.458482\\
3.384	33.478479\\
3.386	33.498475\\
3.388	33.518471\\
3.39	33.538466\\
3.392	33.558461\\
3.394	33.578455\\
3.396	33.598448\\
3.398	33.618441\\
3.4	33.638433\\
3.402	33.658424\\
3.404	33.678415\\
3.406	33.698404\\
3.408	33.718393\\
3.41	33.738382\\
3.412	33.758369\\
3.414	33.778355\\
3.416	33.798341\\
3.418	33.818325\\
3.42	33.838309\\
3.422	33.858292\\
3.424	33.878273\\
3.426	33.898253\\
3.428	33.918233\\
3.43	33.938211\\
3.432	33.958188\\
3.434	33.978164\\
3.436	33.998139\\
3.438	34.018112\\
3.44	34.038085\\
3.442	34.058056\\
3.444	34.078025\\
3.446	34.097994\\
3.448	34.117961\\
3.45	34.137926\\
3.452	34.15789\\
3.454	34.177853\\
3.456	34.197815\\
3.458	34.217774\\
3.46	34.237733\\
3.462	34.25769\\
3.464	34.277645\\
3.466	34.297599\\
3.468	34.317551\\
3.47	34.337501\\
3.472	34.35745\\
3.474	34.377397\\
3.476	34.397343\\
3.478	34.417287\\
3.48	34.437229\\
3.482	34.457169\\
3.484	34.477108\\
3.486	34.497045\\
3.488	34.51698\\
3.49	34.536913\\
3.492	34.556845\\
3.494	34.576775\\
3.496	34.596702\\
3.498	34.616628\\
3.5	34.636552\\
3.502	34.656474\\
3.504	34.676395\\
3.506	34.696313\\
3.508	34.716229\\
3.51	34.736143\\
3.512	34.756056\\
3.514	34.775966\\
3.516	34.795874\\
3.518	34.815781\\
3.52	34.835685\\
3.522	34.855587\\
3.524	34.875487\\
3.526	34.895385\\
3.528	34.915281\\
3.53	34.935175\\
3.532	34.955067\\
3.534	34.974957\\
3.536	34.994844\\
3.538	35.014729\\
3.54	35.034613\\
3.542	35.054494\\
3.544	35.074373\\
3.546	35.09425\\
3.548	35.114124\\
3.55	35.133997\\
3.552	35.153867\\
3.554	35.173735\\
3.556	35.193601\\
3.558	35.213464\\
3.56	35.233326\\
3.562	35.253185\\
3.564	35.273042\\
3.566	35.292897\\
3.568	35.312749\\
3.57	35.3326\\
3.572	35.352448\\
3.574	35.372294\\
3.576	35.392137\\
3.578	35.411979\\
3.58	35.431818\\
3.582	35.451655\\
3.584	35.47149\\
3.586	35.491322\\
3.588	35.511153\\
3.59	35.530981\\
3.592	35.550807\\
3.594	35.57063\\
3.596	35.590451\\
3.598	35.610271\\
3.6	35.630088\\
3.602	35.649902\\
3.604	35.669715\\
3.606	35.689525\\
3.608	35.709333\\
3.61	35.729139\\
3.612	35.748943\\
3.614	35.768745\\
3.616	35.788544\\
3.618	35.808341\\
3.62	35.828136\\
3.622	35.847929\\
3.624	35.86772\\
3.626	35.887508\\
3.628	35.907295\\
3.63	35.927079\\
3.632	35.946861\\
3.634	35.966641\\
3.636	35.986419\\
3.638	36.006195\\
3.64	36.025969\\
3.642	36.045741\\
3.644	36.065511\\
3.646	36.085278\\
3.648	36.105044\\
3.65	36.124808\\
3.652	36.144569\\
3.654	36.164329\\
3.656	36.184087\\
3.658	36.203842\\
3.66	36.223596\\
3.662	36.243348\\
3.664	36.263097\\
3.666	36.282845\\
3.668	36.302591\\
3.67	36.322335\\
3.672	36.342078\\
3.674	36.361818\\
3.676	36.381557\\
3.678	36.401293\\
3.68	36.421028\\
3.682	36.440761\\
3.684	36.460493\\
3.686	36.480222\\
3.688	36.49995\\
3.69	36.519676\\
3.692	36.5394\\
3.694	36.559123\\
3.696	36.578844\\
3.698	36.598563\\
3.7	36.618281\\
3.702	36.637997\\
3.704	36.657712\\
3.706	36.677424\\
3.708	36.697136\\
3.71	36.716846\\
3.712	36.736554\\
3.714	36.75626\\
3.716	36.775966\\
3.718	36.795669\\
3.72	36.815372\\
3.722	36.835073\\
3.724	36.854772\\
3.726	36.87447\\
3.728	36.894167\\
3.73	36.913862\\
3.732	36.933556\\
3.734	36.953249\\
3.736	36.97294\\
3.738	36.99263\\
3.74	37.012319\\
3.742	37.032007\\
3.744	37.051694\\
3.746	37.071379\\
3.748	37.091063\\
3.75	37.110746\\
3.752	37.130428\\
3.754	37.150109\\
3.756	37.169788\\
3.758	37.189467\\
3.76	37.209145\\
3.762	37.228821\\
3.764	37.248497\\
3.766	37.268172\\
3.768	37.287845\\
3.77	37.307518\\
3.772	37.32719\\
3.774	37.346861\\
3.776	37.366531\\
3.778	37.386201\\
3.78	37.405869\\
3.782	37.425537\\
3.784	37.445204\\
3.786	37.46487\\
3.788	37.484536\\
3.79	37.504201\\
3.792	37.523865\\
3.794	37.543529\\
3.796	37.563192\\
3.798	37.582854\\
3.8	37.602516\\
3.802	37.622177\\
3.804	37.641837\\
3.806	37.661498\\
3.808	37.681157\\
3.81	37.700816\\
3.812	37.720475\\
3.814	37.740133\\
3.816	37.759791\\
3.818	37.779449\\
3.82	37.799106\\
3.822	37.818763\\
3.824	37.838419\\
3.826	37.858075\\
3.828	37.877731\\
3.83	37.897387\\
3.832	37.917042\\
3.834	37.936698\\
3.836	37.956353\\
3.838	37.976007\\
3.84	37.995662\\
3.842	38.015317\\
3.844	38.034971\\
3.846	38.054626\\
3.848	38.07428\\
3.85	38.093934\\
3.852	38.113588\\
3.854	38.133243\\
3.856	38.152897\\
3.858	38.172551\\
3.86	38.192206\\
3.862	38.21186\\
3.864	38.231514\\
3.866	38.251169\\
3.868	38.270824\\
3.87	38.290479\\
3.872	38.310134\\
3.874	38.329789\\
3.876	38.349445\\
3.878	38.3691\\
3.88	38.388756\\
3.882	38.408412\\
3.884	38.428069\\
3.886	38.447726\\
3.888	38.467383\\
3.89	38.48704\\
3.892	38.506698\\
3.894	38.526356\\
3.896	38.546014\\
3.898	38.565673\\
3.9	38.585333\\
3.902	38.604992\\
3.904	38.624652\\
3.906	38.644313\\
3.908	38.663974\\
3.91	38.683636\\
3.912	38.703298\\
3.914	38.72296\\
3.916	38.742623\\
3.918	38.762287\\
3.92	38.781951\\
3.922	38.801616\\
3.924	38.821281\\
3.926	38.840947\\
3.928	38.860614\\
3.93	38.880281\\
3.932	38.899949\\
3.934	38.919617\\
3.936	38.939286\\
3.938	38.958956\\
3.94	38.978627\\
3.942	38.998298\\
3.944	39.01797\\
3.946	39.037642\\
3.948	39.057315\\
3.95	39.076989\\
3.952	39.096664\\
3.954	39.116339\\
3.956	39.136016\\
3.958	39.155693\\
3.96	39.17537\\
3.962	39.195049\\
3.964	39.214728\\
3.966	39.234408\\
3.968	39.254089\\
3.97	39.273771\\
3.972	39.293454\\
3.974	39.313137\\
3.976	39.332821\\
3.978	39.352506\\
3.98	39.372192\\
3.982	39.391879\\
3.984	39.411567\\
3.986	39.431255\\
3.988	39.450944\\
3.99	39.470634\\
3.992	39.490326\\
3.994	39.510017\\
3.996	39.52971\\
3.998	39.549404\\
4	39.569098\\
};
\addlegendentry{$\psi$};

\addplot [color=mycolor3,solid]
  table[row sep=crcr]{%
0	20.000004\\
0.002	20.000018\\
0.004	20.00004\\
0.006	20.000072\\
0.008	20.000112\\
0.01	20.000161\\
0.012	20.000219\\
0.014	20.000286\\
0.016	20.000362\\
0.018	20.000447\\
0.02	20.000541\\
0.022	20.000644\\
0.024	20.000756\\
0.026	20.000876\\
0.028	20.001006\\
0.03	20.001144\\
0.032	20.001291\\
0.034	20.001447\\
0.036	20.001612\\
0.038	20.001786\\
0.04	20.001969\\
0.042	20.00216\\
0.044	20.002361\\
0.046	20.00257\\
0.048	20.002788\\
0.05	20.003015\\
0.052	20.00325\\
0.054	20.003494\\
0.056	20.003747\\
0.058	20.004009\\
0.06	20.00428\\
0.062	20.004559\\
0.064	20.004846\\
0.066	20.005143\\
0.068	20.005448\\
0.07	20.005762\\
0.072	20.006084\\
0.074	20.006415\\
0.076	20.006755\\
0.078	20.007103\\
0.08	20.007459\\
0.082	20.007824\\
0.084	20.008198\\
0.086	20.00858\\
0.088	20.008971\\
0.09	20.00937\\
0.092	20.009777\\
0.094	20.010193\\
0.096	20.010617\\
0.098	20.011049\\
0.1	20.01149\\
0.102	20.011939\\
0.104	20.012396\\
0.106	20.012861\\
0.108	20.013335\\
0.11	20.013817\\
0.112	20.014307\\
0.114	20.014805\\
0.116	20.015311\\
0.118	20.015825\\
0.12	20.016347\\
0.122	20.016877\\
0.124	20.017416\\
0.126	20.017962\\
0.128	20.018516\\
0.13	20.019078\\
0.132	20.019647\\
0.134	20.020225\\
0.136	20.02081\\
0.138	20.021403\\
0.14	20.022004\\
0.142	20.022612\\
0.144	20.023228\\
0.146	20.023852\\
0.148	20.024483\\
0.15	20.025121\\
0.152	20.025767\\
0.154	20.026421\\
0.156	20.027082\\
0.158	20.02775\\
0.16	20.028426\\
0.162	20.029109\\
0.164	20.029799\\
0.166	20.030497\\
0.168	20.031202\\
0.17	20.031913\\
0.172	20.032632\\
0.174	20.033358\\
0.176	20.034091\\
0.178	20.034831\\
0.18	20.035578\\
0.182	20.036332\\
0.184	20.037092\\
0.186	20.03786\\
0.188	20.038634\\
0.19	20.039415\\
0.192	20.040202\\
0.194	20.040997\\
0.196	20.041797\\
0.198	20.042605\\
0.2	20.043418\\
0.202	20.044239\\
0.204	20.045065\\
0.206	20.045898\\
0.208	20.046738\\
0.21	20.047583\\
0.212	20.048435\\
0.214	20.049293\\
0.216	20.050157\\
0.218	20.051027\\
0.22	20.051903\\
0.222	20.052785\\
0.224	20.053673\\
0.226	20.054567\\
0.228	20.055467\\
0.23	20.056372\\
0.232	20.057283\\
0.234	20.0582\\
0.236	20.059123\\
0.238	20.060051\\
0.24	20.060984\\
0.242	20.061923\\
0.244	20.062868\\
0.246	20.063817\\
0.248	20.064772\\
0.25	20.065733\\
0.252	20.066698\\
0.254	20.067669\\
0.256	20.068645\\
0.258	20.069625\\
0.26	20.070611\\
0.262	20.071602\\
0.264	20.072597\\
0.266	20.073598\\
0.268	20.074603\\
0.27	20.075613\\
0.272	20.076627\\
0.274	20.077646\\
0.276	20.07867\\
0.278	20.079698\\
0.28	20.08073\\
0.282	20.081767\\
0.284	20.082808\\
0.286	20.083854\\
0.288	20.084904\\
0.29	20.085957\\
0.292	20.087015\\
0.294	20.088077\\
0.296	20.089143\\
0.298	20.090213\\
0.3	20.091286\\
0.302	20.092364\\
0.304	20.093445\\
0.306	20.09453\\
0.308	20.095618\\
0.31	20.09671\\
0.312	20.097805\\
0.314	20.098904\\
0.316	20.100006\\
0.318	20.101112\\
0.32	20.102221\\
0.322	20.103333\\
0.324	20.104448\\
0.326	20.105566\\
0.328	20.106687\\
0.33	20.107811\\
0.332	20.108938\\
0.334	20.110068\\
0.336	20.111201\\
0.338	20.112336\\
0.34	20.113474\\
0.342	20.114614\\
0.344	20.115757\\
0.346	20.116903\\
0.348	20.118051\\
0.35	20.119201\\
0.352	20.120353\\
0.354	20.121508\\
0.356	20.122665\\
0.358	20.123824\\
0.36	20.124984\\
0.362	20.126147\\
0.364	20.127312\\
0.366	20.128478\\
0.368	20.129647\\
0.37	20.130817\\
0.372	20.131988\\
0.374	20.133162\\
0.376	20.134336\\
0.378	20.135512\\
0.38	20.13669\\
0.382	20.137869\\
0.384	20.139049\\
0.386	20.14023\\
0.388	20.141412\\
0.39	20.142596\\
0.392	20.14378\\
0.394	20.144965\\
0.396	20.146152\\
0.398	20.147339\\
0.4	20.148526\\
0.402	20.149715\\
0.404	20.150904\\
0.406	20.152093\\
0.408	20.153283\\
0.41	20.154474\\
0.412	20.155664\\
0.414	20.156855\\
0.416	20.158047\\
0.418	20.159238\\
0.42	20.160429\\
0.422	20.161621\\
0.424	20.162812\\
0.426	20.164004\\
0.428	20.165195\\
0.43	20.166386\\
0.432	20.167576\\
0.434	20.168767\\
0.436	20.169956\\
0.438	20.171146\\
0.44	20.172334\\
0.442	20.173523\\
0.444	20.17471\\
0.446	20.175897\\
0.448	20.177082\\
0.45	20.178267\\
0.452	20.179451\\
0.454	20.180634\\
0.456	20.181816\\
0.458	20.182997\\
0.46	20.184176\\
0.462	20.185354\\
0.464	20.186531\\
0.466	20.187707\\
0.468	20.188881\\
0.47	20.190053\\
0.472	20.191224\\
0.474	20.192393\\
0.476	20.193561\\
0.478	20.194726\\
0.48	20.19589\\
0.482	20.197052\\
0.484	20.198212\\
0.486	20.199369\\
0.488	20.200525\\
0.49	20.201678\\
0.492	20.20283\\
0.494	20.203979\\
0.496	20.205125\\
0.498	20.206269\\
0.5	20.207411\\
0.502	20.20855\\
0.504	20.209686\\
0.506	20.21082\\
0.508	20.211951\\
0.51	20.213079\\
0.512	20.214204\\
0.514	20.215326\\
0.516	20.216446\\
0.518	20.217562\\
0.52	20.218675\\
0.522	20.219785\\
0.524	20.220891\\
0.526	20.221995\\
0.528	20.223095\\
0.53	20.224191\\
0.532	20.225284\\
0.534	20.226373\\
0.536	20.227459\\
0.538	20.228541\\
0.54	20.229619\\
0.542	20.230694\\
0.544	20.231765\\
0.546	20.232831\\
0.548	20.233894\\
0.55	20.234952\\
0.552	20.236007\\
0.554	20.237057\\
0.556	20.238103\\
0.558	20.239145\\
0.56	20.240183\\
0.562	20.241216\\
0.564	20.242244\\
0.566	20.243268\\
0.568	20.244288\\
0.57	20.245302\\
0.572	20.246312\\
0.574	20.247318\\
0.576	20.248318\\
0.578	20.249314\\
0.58	20.250304\\
0.582	20.25129\\
0.584	20.252271\\
0.586	20.253246\\
0.588	20.254217\\
0.59	20.255182\\
0.592	20.256142\\
0.594	20.257096\\
0.596	20.258045\\
0.598	20.258989\\
0.6	20.259927\\
0.602	20.26086\\
0.604	20.261787\\
0.606	20.262708\\
0.608	20.263624\\
0.61	20.264534\\
0.612	20.265438\\
0.614	20.266336\\
0.616	20.267228\\
0.618	20.268114\\
0.62	20.268995\\
0.622	20.269869\\
0.624	20.270737\\
0.626	20.271599\\
0.628	20.272454\\
0.63	20.273304\\
0.632	20.274147\\
0.634	20.274983\\
0.636	20.275814\\
0.638	20.276637\\
0.64	20.277454\\
0.642	20.278265\\
0.644	20.279069\\
0.646	20.279866\\
0.648	20.280657\\
0.65	20.281441\\
0.652	20.282218\\
0.654	20.282988\\
0.656	20.283751\\
0.658	20.284507\\
0.66	20.285257\\
0.662	20.285999\\
0.664	20.286734\\
0.666	20.287462\\
0.668	20.288183\\
0.67	20.288896\\
0.672	20.289603\\
0.674	20.290302\\
0.676	20.290993\\
0.678	20.291678\\
0.68	20.292354\\
0.682	20.293024\\
0.684	20.293686\\
0.686	20.29434\\
0.688	20.294987\\
0.69	20.295626\\
0.692	20.296257\\
0.694	20.296881\\
0.696	20.297497\\
0.698	20.298105\\
0.7	20.298706\\
0.702	20.299298\\
0.704	20.299883\\
0.706	20.30046\\
0.708	20.301028\\
0.71	20.301589\\
0.712	20.302142\\
0.714	20.302687\\
0.716	20.303223\\
0.718	20.303752\\
0.72	20.304272\\
0.722	20.304784\\
0.724	20.305288\\
0.726	20.305784\\
0.728	20.306271\\
0.73	20.30675\\
0.732	20.307221\\
0.734	20.307683\\
0.736	20.308137\\
0.738	20.308582\\
0.74	20.309019\\
0.742	20.309448\\
0.744	20.309868\\
0.746	20.310279\\
0.748	20.310682\\
0.75	20.311077\\
0.752	20.311462\\
0.754	20.311839\\
0.756	20.312208\\
0.758	20.312568\\
0.76	20.312919\\
0.762	20.313261\\
0.764	20.313595\\
0.766	20.313919\\
0.768	20.314235\\
0.77	20.314543\\
0.772	20.314841\\
0.774	20.315131\\
0.776	20.315411\\
0.778	20.315683\\
0.78	20.315946\\
0.782	20.3162\\
0.784	20.316445\\
0.786	20.316681\\
0.788	20.316908\\
0.79	20.317126\\
0.792	20.317336\\
0.794	20.317536\\
0.796	20.317727\\
0.798	20.317909\\
0.8	20.318082\\
0.802	20.318246\\
0.804	20.318401\\
0.806	20.318547\\
0.808	20.318684\\
0.81	20.318812\\
0.812	20.318931\\
0.814	20.31904\\
0.816	20.319141\\
0.818	20.319232\\
0.82	20.319315\\
0.822	20.319388\\
0.824	20.319452\\
0.826	20.319507\\
0.828	20.319553\\
0.83	20.319589\\
0.832	20.319617\\
0.834	20.319635\\
0.836	20.319644\\
0.838	20.319644\\
0.84	20.319635\\
0.842	20.319617\\
0.844	20.31959\\
0.846	20.319553\\
0.848	20.319507\\
0.85	20.319453\\
0.852	20.319389\\
0.854	20.319316\\
0.856	20.319234\\
0.858	20.319142\\
0.86	20.319042\\
0.862	20.318933\\
0.864	20.318814\\
0.866	20.318686\\
0.868	20.318549\\
0.87	20.318404\\
0.872	20.318249\\
0.874	20.318085\\
0.876	20.317912\\
0.878	20.31773\\
0.88	20.317539\\
0.882	20.317339\\
0.884	20.31713\\
0.886	20.316912\\
0.888	20.316685\\
0.89	20.316449\\
0.892	20.316204\\
0.894	20.31595\\
0.896	20.315687\\
0.898	20.315415\\
0.9	20.315135\\
0.902	20.314845\\
0.904	20.314547\\
0.906	20.31424\\
0.908	20.313924\\
0.91	20.313599\\
0.912	20.313266\\
0.914	20.312924\\
0.916	20.312573\\
0.918	20.312213\\
0.92	20.311845\\
0.922	20.311468\\
0.924	20.311082\\
0.926	20.310688\\
0.928	20.310285\\
0.93	20.309874\\
0.932	20.309454\\
0.934	20.309026\\
0.936	20.308589\\
0.938	20.308144\\
0.94	20.30769\\
0.942	20.307228\\
0.944	20.306757\\
0.946	20.306278\\
0.948	20.305791\\
0.95	20.305295\\
0.952	20.304792\\
0.954	20.30428\\
0.956	20.303759\\
0.958	20.303231\\
0.96	20.302695\\
0.962	20.30215\\
0.964	20.301597\\
0.966	20.301037\\
0.968	20.300468\\
0.97	20.299892\\
0.972	20.299307\\
0.974	20.298715\\
0.976	20.298114\\
0.978	20.297506\\
0.98	20.29689\\
0.982	20.296267\\
0.984	20.295635\\
0.986	20.294996\\
0.988	20.29435\\
0.99	20.293695\\
0.992	20.293034\\
0.994	20.292364\\
0.996	20.291688\\
0.998	20.291003\\
1	20.290312\\
1.002	20.289613\\
1.004	20.288907\\
1.006	20.288193\\
1.008	20.287473\\
1.01	20.286745\\
1.012	20.28601\\
1.014	20.285268\\
1.016	20.284518\\
1.018	20.283762\\
1.02	20.282999\\
1.022	20.282229\\
1.024	20.281452\\
1.026	20.280669\\
1.028	20.279878\\
1.03	20.279081\\
1.032	20.278277\\
1.034	20.277466\\
1.036	20.276649\\
1.038	20.275826\\
1.04	20.274996\\
1.042	20.274159\\
1.044	20.273316\\
1.046	20.272467\\
1.048	20.271611\\
1.05	20.27075\\
1.052	20.269882\\
1.054	20.269008\\
1.056	20.268127\\
1.058	20.267241\\
1.06	20.266349\\
1.062	20.265451\\
1.064	20.264547\\
1.066	20.263637\\
1.068	20.262722\\
1.07	20.2618\\
1.072	20.260873\\
1.074	20.259941\\
1.076	20.259003\\
1.078	20.258059\\
1.08	20.25711\\
1.082	20.256156\\
1.084	20.255196\\
1.086	20.254231\\
1.088	20.253261\\
1.09	20.252285\\
1.092	20.251305\\
1.094	20.250319\\
1.096	20.249328\\
1.098	20.248333\\
1.1	20.247333\\
1.102	20.246327\\
1.104	20.245317\\
1.106	20.244303\\
1.108	20.243283\\
1.11	20.242259\\
1.112	20.241231\\
1.114	20.240198\\
1.116	20.239161\\
1.118	20.238119\\
1.12	20.237073\\
1.122	20.236023\\
1.124	20.234968\\
1.126	20.23391\\
1.128	20.232847\\
1.13	20.23178\\
1.132	20.23071\\
1.134	20.229635\\
1.136	20.228557\\
1.138	20.227475\\
1.14	20.226389\\
1.142	20.2253\\
1.144	20.224207\\
1.146	20.223111\\
1.148	20.222011\\
1.15	20.220908\\
1.152	20.219801\\
1.154	20.218691\\
1.156	20.217578\\
1.158	20.216462\\
1.16	20.215343\\
1.162	20.214221\\
1.164	20.213096\\
1.166	20.211967\\
1.168	20.210837\\
1.17	20.209703\\
1.172	20.208567\\
1.174	20.207428\\
1.176	20.206286\\
1.178	20.205142\\
1.18	20.203996\\
1.182	20.202847\\
1.184	20.201695\\
1.186	20.200542\\
1.188	20.199386\\
1.19	20.198229\\
1.192	20.197069\\
1.194	20.195907\\
1.196	20.194743\\
1.198	20.193578\\
1.2	20.19241\\
1.202	20.191241\\
1.204	20.19007\\
1.206	20.188898\\
1.208	20.187724\\
1.21	20.186549\\
1.212	20.185372\\
1.214	20.184194\\
1.216	20.183014\\
1.218	20.181834\\
1.22	20.180652\\
1.222	20.179469\\
1.224	20.178285\\
1.226	20.1771\\
1.228	20.175914\\
1.23	20.174727\\
1.232	20.17354\\
1.234	20.172352\\
1.236	20.171163\\
1.238	20.169974\\
1.24	20.168784\\
1.242	20.167594\\
1.244	20.166403\\
1.246	20.165212\\
1.248	20.164021\\
1.25	20.16283\\
1.252	20.161639\\
1.254	20.160447\\
1.256	20.159256\\
1.258	20.158064\\
1.26	20.156873\\
1.262	20.155682\\
1.264	20.154491\\
1.266	20.153301\\
1.268	20.152111\\
1.27	20.150921\\
1.272	20.149732\\
1.274	20.148544\\
1.276	20.147356\\
1.278	20.146169\\
1.28	20.144983\\
1.282	20.143798\\
1.284	20.142613\\
1.286	20.14143\\
1.288	20.140247\\
1.29	20.139066\\
1.292	20.137886\\
1.294	20.136707\\
1.296	20.13553\\
1.298	20.134354\\
1.3	20.133179\\
1.302	20.132006\\
1.304	20.130834\\
1.306	20.129664\\
1.308	20.128496\\
1.31	20.127329\\
1.312	20.126164\\
1.314	20.125002\\
1.316	20.123841\\
1.318	20.122682\\
1.32	20.121525\\
1.322	20.12037\\
1.324	20.119218\\
1.326	20.118068\\
1.328	20.11692\\
1.33	20.115774\\
1.332	20.114631\\
1.334	20.113491\\
1.336	20.112353\\
1.338	20.111217\\
1.34	20.110085\\
1.342	20.108955\\
1.344	20.107828\\
1.346	20.106704\\
1.348	20.105582\\
1.35	20.104464\\
1.352	20.103349\\
1.354	20.102237\\
1.356	20.101128\\
1.358	20.100023\\
1.36	20.09892\\
1.362	20.097821\\
1.364	20.096726\\
1.366	20.095634\\
1.368	20.094546\\
1.37	20.093461\\
1.372	20.09238\\
1.374	20.091302\\
1.376	20.090229\\
1.378	20.089159\\
1.38	20.088093\\
1.382	20.087031\\
1.384	20.085973\\
1.386	20.084919\\
1.388	20.083869\\
1.39	20.082824\\
1.392	20.081783\\
1.394	20.080746\\
1.396	20.079713\\
1.398	20.078685\\
1.4	20.077661\\
1.402	20.076642\\
1.404	20.075628\\
1.406	20.074618\\
1.408	20.073613\\
1.41	20.072612\\
1.412	20.071616\\
1.414	20.070626\\
1.416	20.06964\\
1.418	20.068659\\
1.42	20.067683\\
1.422	20.066712\\
1.424	20.065747\\
1.426	20.064787\\
1.428	20.063831\\
1.43	20.062882\\
1.432	20.061937\\
1.434	20.060998\\
1.436	20.060064\\
1.438	20.059136\\
1.44	20.058214\\
1.442	20.057297\\
1.444	20.056386\\
1.446	20.05548\\
1.448	20.05458\\
1.45	20.053686\\
1.452	20.052798\\
1.454	20.051916\\
1.456	20.05104\\
1.458	20.05017\\
1.46	20.049305\\
1.462	20.048447\\
1.464	20.047596\\
1.466	20.04675\\
1.468	20.045911\\
1.47	20.045078\\
1.472	20.044251\\
1.474	20.04343\\
1.476	20.042617\\
1.478	20.041809\\
1.48	20.041008\\
1.482	20.040214\\
1.484	20.039426\\
1.486	20.038645\\
1.488	20.037871\\
1.49	20.037104\\
1.492	20.036343\\
1.494	20.035589\\
1.496	20.034842\\
1.498	20.034102\\
1.5	20.033369\\
1.502	20.032643\\
1.504	20.031924\\
1.506	20.031212\\
1.508	20.030507\\
1.51	20.02981\\
1.512	20.029119\\
1.514	20.028436\\
1.516	20.02776\\
1.518	20.027092\\
1.52	20.026431\\
1.522	20.025777\\
1.524	20.025131\\
1.526	20.024492\\
1.528	20.023861\\
1.53	20.023237\\
1.532	20.022621\\
1.534	20.022013\\
1.536	20.021412\\
1.538	20.020819\\
1.54	20.020233\\
1.542	20.019656\\
1.544	20.019086\\
1.546	20.018524\\
1.548	20.01797\\
1.55	20.017424\\
1.552	20.016885\\
1.554	20.016355\\
1.556	20.015833\\
1.558	20.015318\\
1.56	20.014812\\
1.562	20.014314\\
1.564	20.013824\\
1.566	20.013342\\
1.568	20.012868\\
1.57	20.012403\\
1.572	20.011945\\
1.574	20.011496\\
1.576	20.011056\\
1.578	20.010623\\
1.58	20.010199\\
1.582	20.009783\\
1.584	20.009376\\
1.586	20.008977\\
1.588	20.008586\\
1.59	20.008204\\
1.592	20.00783\\
1.594	20.007465\\
1.596	20.007108\\
1.598	20.00676\\
1.6	20.00642\\
1.602	20.006089\\
1.604	20.005767\\
1.606	20.005453\\
1.608	20.005147\\
1.61	20.004851\\
1.612	20.004563\\
1.614	20.004284\\
1.616	20.004013\\
1.618	20.003751\\
1.62	20.003498\\
1.622	20.003254\\
1.624	20.003018\\
1.626	20.002791\\
1.628	20.002573\\
1.63	20.002364\\
1.632	20.002163\\
1.634	20.001972\\
1.636	20.001789\\
1.638	20.001615\\
1.64	20.00145\\
1.642	20.001293\\
1.644	20.001146\\
1.646	20.001008\\
1.648	20.000878\\
1.65	20.000757\\
1.652	20.000645\\
1.654	20.000543\\
1.656	20.000449\\
1.658	20.000363\\
1.66	20.000287\\
1.662	20.00022\\
1.664	20.000162\\
1.666	20.000113\\
1.668	20.000072\\
1.67	20.000041\\
1.672	20.000018\\
1.674	20.000005\\
1.676	20\\
1.678	20.000004\\
1.68	20.000018\\
1.682	20.00004\\
1.684	20.000071\\
1.686	20.000111\\
1.688	20.00016\\
1.69	20.000218\\
1.692	20.000285\\
1.694	20.000361\\
1.696	20.000446\\
1.698	20.00054\\
1.7	20.000642\\
1.702	20.000754\\
1.704	20.000874\\
1.706	20.001004\\
1.708	20.001142\\
1.71	20.001289\\
1.712	20.001445\\
1.714	20.00161\\
1.716	20.001784\\
1.718	20.001966\\
1.72	20.002158\\
1.722	20.002358\\
1.724	20.002567\\
1.726	20.002785\\
1.728	20.003011\\
1.73	20.003247\\
1.732	20.003491\\
1.734	20.003744\\
1.736	20.004005\\
1.738	20.004275\\
1.74	20.004554\\
1.742	20.004842\\
1.744	20.005139\\
1.746	20.005444\\
1.748	20.005757\\
1.75	20.006079\\
1.752	20.00641\\
1.754	20.00675\\
1.756	20.007098\\
1.758	20.007454\\
1.76	20.007819\\
1.762	20.008193\\
1.764	20.008574\\
1.766	20.008965\\
1.768	20.009364\\
1.77	20.009771\\
1.772	20.010186\\
1.774	20.01061\\
1.776	20.011043\\
1.778	20.011483\\
1.78	20.011932\\
1.782	20.012389\\
1.784	20.012854\\
1.786	20.013328\\
1.788	20.01381\\
1.79	20.014299\\
1.792	20.014797\\
1.794	20.015303\\
1.796	20.015817\\
1.798	20.01634\\
1.8	20.01687\\
1.802	20.017408\\
1.804	20.017954\\
1.806	20.018507\\
1.808	20.019069\\
1.81	20.019639\\
1.812	20.020216\\
1.814	20.020801\\
1.816	20.021394\\
1.818	20.021995\\
1.82	20.022603\\
1.822	20.023219\\
1.824	20.023842\\
1.826	20.024473\\
1.828	20.025112\\
1.83	20.025758\\
1.832	20.026411\\
1.834	20.027072\\
1.836	20.02774\\
1.838	20.028416\\
1.84	20.029099\\
1.842	20.029789\\
1.844	20.030487\\
1.846	20.031191\\
1.848	20.031903\\
1.85	20.032622\\
1.852	20.033347\\
1.854	20.03408\\
1.856	20.03482\\
1.858	20.035567\\
1.86	20.036321\\
1.862	20.037081\\
1.864	20.037848\\
1.866	20.038622\\
1.868	20.039403\\
1.87	20.040191\\
1.872	20.040985\\
1.874	20.041785\\
1.876	20.042593\\
1.878	20.043406\\
1.88	20.044226\\
1.882	20.045053\\
1.884	20.045886\\
1.886	20.046725\\
1.888	20.047571\\
1.89	20.048422\\
1.892	20.04928\\
1.894	20.050144\\
1.896	20.051014\\
1.898	20.05189\\
1.9	20.052772\\
1.902	20.05366\\
1.904	20.054554\\
1.906	20.055453\\
1.908	20.056359\\
1.91	20.05727\\
1.912	20.058187\\
1.914	20.059109\\
1.916	20.060037\\
1.918	20.06097\\
1.92	20.061909\\
1.922	20.062854\\
1.924	20.063803\\
1.926	20.064758\\
1.928	20.065718\\
1.93	20.066684\\
1.932	20.067654\\
1.934	20.06863\\
1.936	20.069611\\
1.938	20.070597\\
1.94	20.071587\\
1.942	20.072583\\
1.944	20.073583\\
1.946	20.074588\\
1.948	20.075598\\
1.95	20.076612\\
1.952	20.077631\\
1.954	20.078655\\
1.956	20.079683\\
1.958	20.080715\\
1.96	20.081752\\
1.962	20.082793\\
1.964	20.083838\\
1.966	20.084888\\
1.968	20.085942\\
1.97	20.087\\
1.972	20.088061\\
1.974	20.089127\\
1.976	20.090197\\
1.978	20.09127\\
1.98	20.092348\\
1.982	20.093429\\
1.984	20.094513\\
1.986	20.095602\\
1.988	20.096694\\
1.99	20.097789\\
1.992	20.098888\\
1.994	20.09999\\
1.996	20.101096\\
1.998	20.102204\\
2	20.103316\\
2.002	20.104431\\
2.004	20.105549\\
2.006	20.106671\\
2.008	20.107795\\
2.01	20.108922\\
2.012	20.110051\\
2.014	20.111184\\
2.016	20.112319\\
2.018	20.113457\\
2.02	20.114598\\
2.022	20.11574\\
2.024	20.116886\\
2.026	20.118034\\
2.028	20.119184\\
2.03	20.120336\\
2.032	20.121491\\
2.034	20.122648\\
2.036	20.123806\\
2.038	20.124967\\
2.04	20.12613\\
2.042	20.127295\\
2.044	20.128461\\
2.046	20.129629\\
2.048	20.130799\\
2.05	20.131971\\
2.052	20.133144\\
2.054	20.134319\\
2.056	20.135495\\
2.058	20.136672\\
2.06	20.137851\\
2.062	20.139031\\
2.064	20.140212\\
2.066	20.141395\\
2.068	20.142578\\
2.07	20.143763\\
2.072	20.144948\\
2.074	20.146134\\
2.076	20.147321\\
2.078	20.148509\\
2.08	20.149697\\
2.082	20.150886\\
2.084	20.152076\\
2.086	20.153266\\
2.088	20.154456\\
2.09	20.155647\\
2.092	20.156838\\
2.094	20.158029\\
2.096	20.15922\\
2.098	20.160412\\
2.1	20.161603\\
2.102	20.162795\\
2.104	20.163986\\
2.106	20.165177\\
2.108	20.166368\\
2.11	20.167559\\
2.112	20.168749\\
2.114	20.169939\\
2.116	20.171128\\
2.118	20.172317\\
2.12	20.173505\\
2.122	20.174692\\
2.124	20.175879\\
2.126	20.177065\\
2.128	20.17825\\
2.13	20.179434\\
2.132	20.180617\\
2.134	20.181799\\
2.136	20.182979\\
2.138	20.184159\\
2.14	20.185337\\
2.142	20.186514\\
2.144	20.187689\\
2.146	20.188863\\
2.148	20.190036\\
2.15	20.191207\\
2.152	20.192376\\
2.154	20.193543\\
2.156	20.194709\\
2.158	20.195873\\
2.16	20.197035\\
2.162	20.198194\\
2.164	20.199352\\
2.166	20.200508\\
2.168	20.201661\\
2.17	20.202813\\
2.172	20.203962\\
2.174	20.205108\\
2.176	20.206252\\
2.178	20.207394\\
2.18	20.208533\\
2.182	20.209669\\
2.184	20.210803\\
2.186	20.211934\\
2.188	20.213062\\
2.19	20.214188\\
2.192	20.21531\\
2.194	20.216429\\
2.196	20.217545\\
2.198	20.218659\\
2.2	20.219768\\
2.202	20.220875\\
2.204	20.221978\\
2.206	20.223078\\
2.208	20.224175\\
2.21	20.225268\\
2.212	20.226357\\
2.214	20.227443\\
2.216	20.228525\\
2.218	20.229604\\
2.22	20.230678\\
2.222	20.231749\\
2.224	20.232815\\
2.226	20.233878\\
2.228	20.234937\\
2.23	20.235991\\
2.232	20.237042\\
2.234	20.238088\\
2.236	20.23913\\
2.238	20.240167\\
2.24	20.2412\\
2.242	20.242229\\
2.244	20.243253\\
2.246	20.244273\\
2.248	20.245287\\
2.25	20.246298\\
2.252	20.247303\\
2.254	20.248303\\
2.256	20.249299\\
2.258	20.25029\\
2.26	20.251276\\
2.262	20.252256\\
2.264	20.253232\\
2.266	20.254202\\
2.268	20.255168\\
2.27	20.256127\\
2.272	20.257082\\
2.274	20.258031\\
2.276	20.258975\\
2.278	20.259913\\
2.28	20.260846\\
2.282	20.261773\\
2.284	20.262695\\
2.286	20.26361\\
2.288	20.26452\\
2.29	20.265424\\
2.292	20.266323\\
2.294	20.267215\\
2.296	20.268101\\
2.298	20.268982\\
2.3	20.269856\\
2.302	20.270724\\
2.304	20.271586\\
2.306	20.272442\\
2.308	20.273291\\
2.31	20.274134\\
2.312	20.274971\\
2.314	20.275801\\
2.316	20.276625\\
2.318	20.277442\\
2.32	20.278253\\
2.322	20.279057\\
2.324	20.279855\\
2.326	20.280645\\
2.328	20.281429\\
2.33	20.282206\\
2.332	20.282976\\
2.334	20.28374\\
2.336	20.284496\\
2.338	20.285246\\
2.34	20.285988\\
2.342	20.286723\\
2.344	20.287451\\
2.346	20.288172\\
2.348	20.288886\\
2.35	20.289592\\
2.352	20.290291\\
2.354	20.290983\\
2.356	20.291668\\
2.358	20.292345\\
2.36	20.293014\\
2.362	20.293676\\
2.364	20.29433\\
2.366	20.294977\\
2.368	20.295617\\
2.37	20.296248\\
2.372	20.296872\\
2.374	20.297488\\
2.376	20.298096\\
2.378	20.298697\\
2.38	20.29929\\
2.382	20.299874\\
2.384	20.300451\\
2.386	20.30102\\
2.388	20.301581\\
2.39	20.302134\\
2.392	20.302679\\
2.394	20.303215\\
2.396	20.303744\\
2.398	20.304264\\
2.4	20.304777\\
2.402	20.305281\\
2.404	20.305776\\
2.406	20.306264\\
2.408	20.306743\\
2.41	20.307214\\
2.412	20.307676\\
2.414	20.30813\\
2.416	20.308576\\
2.418	20.309013\\
2.42	20.309442\\
2.422	20.309862\\
2.424	20.310273\\
2.426	20.310676\\
2.428	20.311071\\
2.43	20.311457\\
2.432	20.311834\\
2.434	20.312203\\
2.436	20.312562\\
2.438	20.312914\\
2.44	20.313256\\
2.442	20.31359\\
2.444	20.313915\\
2.446	20.314231\\
2.448	20.314538\\
2.45	20.314837\\
2.452	20.315126\\
2.454	20.315407\\
2.456	20.315679\\
2.458	20.315942\\
2.46	20.316196\\
2.462	20.316441\\
2.464	20.316678\\
2.466	20.316905\\
2.468	20.317123\\
2.47	20.317333\\
2.472	20.317533\\
2.474	20.317724\\
2.476	20.317907\\
2.478	20.31808\\
2.48	20.318244\\
2.482	20.318399\\
2.484	20.318545\\
2.486	20.318682\\
2.488	20.31881\\
2.49	20.318929\\
2.492	20.319039\\
2.494	20.31914\\
2.496	20.319231\\
2.498	20.319314\\
2.5	20.319387\\
2.502	20.319451\\
2.504	20.319506\\
2.506	20.319552\\
2.508	20.319589\\
2.51	20.319616\\
2.512	20.319635\\
2.514	20.319644\\
2.516	20.319644\\
2.518	20.319635\\
2.52	20.319617\\
2.522	20.31959\\
2.524	20.319554\\
2.526	20.319508\\
2.528	20.319454\\
2.53	20.31939\\
2.532	20.319317\\
2.534	20.319235\\
2.536	20.319144\\
2.538	20.319044\\
2.54	20.318934\\
2.542	20.318816\\
2.544	20.318688\\
2.546	20.318552\\
2.548	20.318406\\
2.55	20.318251\\
2.552	20.318087\\
2.554	20.317914\\
2.556	20.317733\\
2.558	20.317542\\
2.56	20.317342\\
2.562	20.317133\\
2.564	20.316915\\
2.566	20.316688\\
2.568	20.316452\\
2.57	20.316207\\
2.572	20.315954\\
2.574	20.315691\\
2.576	20.315419\\
2.578	20.315139\\
2.58	20.31485\\
2.582	20.314552\\
2.584	20.314245\\
2.586	20.313929\\
2.588	20.313604\\
2.59	20.313271\\
2.592	20.312929\\
2.594	20.312578\\
2.596	20.312219\\
2.598	20.31185\\
2.6	20.311474\\
2.602	20.311088\\
2.604	20.310694\\
2.606	20.310291\\
2.608	20.30988\\
2.61	20.30946\\
2.612	20.309032\\
2.614	20.308595\\
2.616	20.30815\\
2.618	20.307697\\
2.62	20.307234\\
2.622	20.306764\\
2.624	20.306285\\
2.626	20.305798\\
2.628	20.305303\\
2.63	20.304799\\
2.632	20.304287\\
2.634	20.303767\\
2.636	20.303239\\
2.638	20.302703\\
2.64	20.302158\\
2.642	20.301606\\
2.644	20.301045\\
2.646	20.300477\\
2.648	20.2999\\
2.65	20.299316\\
2.652	20.298723\\
2.654	20.298123\\
2.656	20.297515\\
2.658	20.296899\\
2.66	20.296276\\
2.662	20.295645\\
2.664	20.295006\\
2.666	20.294359\\
2.668	20.293705\\
2.67	20.293044\\
2.672	20.292374\\
2.674	20.291698\\
2.676	20.291014\\
2.678	20.290322\\
2.68	20.289623\\
2.682	20.288917\\
2.684	20.288204\\
2.686	20.287483\\
2.688	20.286755\\
2.69	20.286021\\
2.692	20.285279\\
2.694	20.284529\\
2.696	20.283773\\
2.698	20.28301\\
2.7	20.282241\\
2.702	20.281464\\
2.704	20.28068\\
2.706	20.27989\\
2.708	20.279093\\
2.71	20.278289\\
2.712	20.277478\\
2.714	20.276662\\
2.716	20.275838\\
2.718	20.275008\\
2.72	20.274172\\
2.722	20.273329\\
2.724	20.27248\\
2.726	20.271624\\
2.728	20.270762\\
2.73	20.269895\\
2.732	20.269021\\
2.734	20.268141\\
2.736	20.267254\\
2.738	20.266362\\
2.74	20.265464\\
2.742	20.26456\\
2.744	20.263651\\
2.746	20.262735\\
2.748	20.261814\\
2.75	20.260887\\
2.752	20.259955\\
2.754	20.259017\\
2.756	20.258073\\
2.758	20.257124\\
2.76	20.25617\\
2.762	20.25521\\
2.764	20.254245\\
2.766	20.253275\\
2.768	20.2523\\
2.77	20.251319\\
2.772	20.250334\\
2.774	20.249343\\
2.776	20.248348\\
2.778	20.247347\\
2.78	20.246342\\
2.782	20.245332\\
2.784	20.244318\\
2.786	20.243298\\
2.788	20.242274\\
2.79	20.241246\\
2.792	20.240213\\
2.794	20.239176\\
2.796	20.238134\\
2.798	20.237088\\
2.8	20.236038\\
2.802	20.234984\\
2.804	20.233925\\
2.806	20.232863\\
2.808	20.231796\\
2.81	20.230726\\
2.812	20.229651\\
2.814	20.228573\\
2.816	20.227491\\
2.818	20.226405\\
2.82	20.225316\\
2.822	20.224223\\
2.824	20.223127\\
2.826	20.222027\\
2.828	20.220924\\
2.83	20.219818\\
2.832	20.218708\\
2.834	20.217595\\
2.836	20.216479\\
2.838	20.21536\\
2.84	20.214237\\
2.842	20.213112\\
2.844	20.211984\\
2.846	20.210853\\
2.848	20.20972\\
2.85	20.208583\\
2.852	20.207444\\
2.854	20.206303\\
2.856	20.205159\\
2.858	20.204012\\
2.86	20.202864\\
2.862	20.201712\\
2.864	20.200559\\
2.866	20.199404\\
2.868	20.198246\\
2.87	20.197086\\
2.872	20.195924\\
2.874	20.194761\\
2.876	20.193595\\
2.878	20.192428\\
2.88	20.191259\\
2.882	20.190088\\
2.884	20.188915\\
2.886	20.187741\\
2.888	20.186566\\
2.89	20.185389\\
2.892	20.184211\\
2.894	20.183032\\
2.896	20.181851\\
2.898	20.180669\\
2.9	20.179486\\
2.902	20.178302\\
2.904	20.177117\\
2.906	20.175932\\
2.908	20.174745\\
2.91	20.173558\\
2.912	20.17237\\
2.914	20.171181\\
2.916	20.169992\\
2.918	20.168802\\
2.92	20.167612\\
2.922	20.166421\\
2.924	20.16523\\
2.926	20.164039\\
2.928	20.162848\\
2.93	20.161656\\
2.932	20.160465\\
2.934	20.159273\\
2.936	20.158082\\
2.938	20.15689\\
2.94	20.155699\\
2.942	20.154509\\
2.944	20.153318\\
2.946	20.152128\\
2.948	20.150939\\
2.95	20.14975\\
2.952	20.148561\\
2.954	20.147374\\
2.956	20.146187\\
2.958	20.145\\
2.96	20.143815\\
2.962	20.142631\\
2.964	20.141447\\
2.966	20.140265\\
2.968	20.139084\\
2.97	20.137903\\
2.972	20.136725\\
2.974	20.135547\\
2.976	20.134371\\
2.978	20.133196\\
2.98	20.132023\\
2.982	20.130851\\
2.984	20.129681\\
2.986	20.128513\\
2.988	20.127346\\
2.99	20.126182\\
2.992	20.125019\\
2.994	20.123858\\
2.996	20.122699\\
2.998	20.121542\\
3	20.120387\\
3.002	20.119235\\
3.004	20.118085\\
3.006	20.116937\\
3.008	20.115791\\
3.01	20.114648\\
3.012	20.113508\\
3.014	20.11237\\
3.016	20.111234\\
3.018	20.110101\\
3.02	20.108972\\
3.022	20.107844\\
3.024	20.10672\\
3.026	20.105599\\
3.028	20.104481\\
3.03	20.103366\\
3.032	20.102253\\
3.034	20.101145\\
3.036	20.100039\\
3.038	20.098937\\
3.04	20.097838\\
3.042	20.096742\\
3.044	20.09565\\
3.046	20.094562\\
3.048	20.093477\\
3.05	20.092396\\
3.052	20.091318\\
3.054	20.090244\\
3.056	20.089174\\
3.058	20.088109\\
3.06	20.087047\\
3.062	20.085989\\
3.064	20.084935\\
3.066	20.083885\\
3.068	20.082839\\
3.07	20.081798\\
3.072	20.080761\\
3.074	20.079728\\
3.076	20.0787\\
3.078	20.077676\\
3.08	20.076657\\
3.082	20.075643\\
3.084	20.074633\\
3.086	20.073627\\
3.088	20.072627\\
3.09	20.071631\\
3.092	20.07064\\
3.094	20.069654\\
3.096	20.068673\\
3.098	20.067698\\
3.1	20.066727\\
3.102	20.065761\\
3.104	20.064801\\
3.106	20.063845\\
3.108	20.062896\\
3.11	20.061951\\
3.112	20.061012\\
3.114	20.060078\\
3.116	20.05915\\
3.118	20.058227\\
3.12	20.05731\\
3.122	20.056399\\
3.124	20.055493\\
3.126	20.054593\\
3.128	20.053699\\
3.13	20.052811\\
3.132	20.051929\\
3.134	20.051053\\
3.136	20.050182\\
3.138	20.049318\\
3.14	20.04846\\
3.142	20.047608\\
3.144	20.046762\\
3.146	20.045923\\
3.148	20.04509\\
3.15	20.044263\\
3.152	20.043443\\
3.154	20.042629\\
3.156	20.041821\\
3.158	20.04102\\
3.16	20.040226\\
3.162	20.039438\\
3.164	20.038657\\
3.166	20.037883\\
3.168	20.037115\\
3.17	20.036354\\
3.172	20.0356\\
3.174	20.034853\\
3.176	20.034113\\
3.178	20.03338\\
3.18	20.032654\\
3.182	20.031934\\
3.184	20.031222\\
3.186	20.030518\\
3.188	20.02982\\
3.19	20.029129\\
3.192	20.028446\\
3.194	20.02777\\
3.196	20.027102\\
3.198	20.02644\\
3.2	20.025787\\
3.202	20.02514\\
3.204	20.024501\\
3.206	20.02387\\
3.208	20.023246\\
3.21	20.02263\\
3.212	20.022021\\
3.214	20.021421\\
3.216	20.020827\\
3.218	20.020242\\
3.22	20.019664\\
3.222	20.019094\\
3.224	20.018532\\
3.226	20.017978\\
3.228	20.017432\\
3.23	20.016893\\
3.232	20.016363\\
3.234	20.01584\\
3.236	20.015326\\
3.238	20.01482\\
3.24	20.014321\\
3.242	20.013831\\
3.244	20.013349\\
3.246	20.012875\\
3.248	20.01241\\
3.25	20.011952\\
3.252	20.011503\\
3.254	20.011062\\
3.256	20.010629\\
3.258	20.010205\\
3.26	20.009789\\
3.262	20.009382\\
3.264	20.008982\\
3.266	20.008592\\
3.268	20.008209\\
3.27	20.007835\\
3.272	20.00747\\
3.274	20.007113\\
3.276	20.006765\\
3.278	20.006425\\
3.28	20.006094\\
3.282	20.005771\\
3.284	20.005457\\
3.286	20.005152\\
3.288	20.004855\\
3.29	20.004567\\
3.292	20.004288\\
3.294	20.004017\\
3.296	20.003755\\
3.298	20.003502\\
3.3	20.003257\\
3.302	20.003021\\
3.304	20.002794\\
3.306	20.002576\\
3.308	20.002367\\
3.31	20.002166\\
3.312	20.001974\\
3.314	20.001791\\
3.316	20.001617\\
3.318	20.001452\\
3.32	20.001296\\
3.322	20.001148\\
3.324	20.00101\\
3.326	20.00088\\
3.328	20.000759\\
3.33	20.000647\\
3.332	20.000544\\
3.334	20.00045\\
3.336	20.000365\\
3.338	20.000288\\
3.34	20.000221\\
3.342	20.000163\\
3.344	20.000113\\
3.346	20.000073\\
3.348	20.000041\\
3.35	20.000018\\
3.352	20.000005\\
3.354	20\\
3.356	20.000004\\
3.358	20.000017\\
3.36	20.000039\\
3.362	20.000071\\
3.364	20.000111\\
3.366	20.000159\\
3.368	20.000217\\
3.37	20.000284\\
3.372	20.00036\\
3.374	20.000445\\
3.376	20.000538\\
3.378	20.000641\\
3.38	20.000752\\
3.382	20.000872\\
3.384	20.001002\\
3.386	20.00114\\
3.388	20.001287\\
3.39	20.001443\\
3.392	20.001607\\
3.394	20.001781\\
3.396	20.001963\\
3.398	20.002155\\
3.4	20.002355\\
3.402	20.002564\\
3.404	20.002781\\
3.406	20.003008\\
3.408	20.003243\\
3.41	20.003487\\
3.412	20.00374\\
3.414	20.004001\\
3.416	20.004271\\
3.418	20.00455\\
3.42	20.004838\\
3.422	20.005134\\
3.424	20.005439\\
3.426	20.005752\\
3.428	20.006075\\
3.43	20.006405\\
3.432	20.006745\\
3.434	20.007092\\
3.436	20.007449\\
3.438	20.007814\\
3.44	20.008187\\
3.442	20.008569\\
3.444	20.008959\\
3.446	20.009358\\
3.448	20.009765\\
3.45	20.01018\\
3.452	20.010604\\
3.454	20.011036\\
3.456	20.011477\\
3.458	20.011925\\
3.46	20.012382\\
3.462	20.012848\\
3.464	20.013321\\
3.466	20.013802\\
3.468	20.014292\\
3.47	20.01479\\
3.472	20.015296\\
3.474	20.01581\\
3.476	20.016332\\
3.478	20.016862\\
3.48	20.0174\\
3.482	20.017945\\
3.484	20.018499\\
3.486	20.019061\\
3.488	20.01963\\
3.49	20.020208\\
3.492	20.020793\\
3.494	20.021385\\
3.496	20.021986\\
3.498	20.022594\\
3.5	20.02321\\
3.502	20.023833\\
3.504	20.024464\\
3.506	20.025102\\
3.508	20.025748\\
3.51	20.026402\\
3.512	20.027062\\
3.514	20.027731\\
3.516	20.028406\\
3.518	20.029089\\
3.52	20.029779\\
3.522	20.030476\\
3.524	20.031181\\
3.526	20.031892\\
3.528	20.032611\\
3.53	20.033337\\
3.532	20.034069\\
3.534	20.034809\\
3.536	20.035556\\
3.538	20.036309\\
3.54	20.03707\\
3.542	20.037837\\
3.544	20.038611\\
3.546	20.039392\\
3.548	20.040179\\
3.55	20.040973\\
3.552	20.041774\\
3.554	20.042581\\
3.556	20.043394\\
3.558	20.044214\\
3.56	20.045041\\
3.562	20.045874\\
3.564	20.046713\\
3.566	20.047558\\
3.568	20.04841\\
3.57	20.049267\\
3.572	20.050131\\
3.574	20.051001\\
3.576	20.051877\\
3.578	20.052759\\
3.58	20.053647\\
3.582	20.054541\\
3.584	20.05544\\
3.586	20.056345\\
3.588	20.057256\\
3.59	20.058173\\
3.592	20.059095\\
3.594	20.060023\\
3.596	20.060957\\
3.598	20.061895\\
3.6	20.06284\\
3.602	20.063789\\
3.604	20.064744\\
3.606	20.065704\\
3.608	20.06667\\
3.61	20.06764\\
3.612	20.068616\\
3.614	20.069596\\
3.616	20.070582\\
3.618	20.071572\\
3.62	20.072568\\
3.622	20.073568\\
3.624	20.074573\\
3.626	20.075583\\
3.628	20.076597\\
3.63	20.077616\\
3.632	20.07864\\
3.634	20.079667\\
3.636	20.0807\\
3.638	20.081737\\
3.64	20.082778\\
3.642	20.083823\\
3.644	20.084873\\
3.646	20.085926\\
3.648	20.086984\\
3.65	20.088046\\
3.652	20.089111\\
3.654	20.090181\\
3.656	20.091255\\
3.658	20.092332\\
3.66	20.093413\\
3.662	20.094497\\
3.664	20.095586\\
3.666	20.096678\\
3.668	20.097773\\
3.67	20.098872\\
3.672	20.099974\\
3.674	20.101079\\
3.676	20.102188\\
3.678	20.1033\\
3.68	20.104415\\
3.682	20.105533\\
3.684	20.106654\\
3.686	20.107778\\
3.688	20.108905\\
3.69	20.110035\\
3.692	20.111167\\
3.694	20.112302\\
3.696	20.11344\\
3.698	20.114581\\
3.7	20.115724\\
3.702	20.116869\\
3.704	20.118017\\
3.706	20.119167\\
3.708	20.120319\\
3.71	20.121474\\
3.712	20.122631\\
3.714	20.123789\\
3.716	20.12495\\
3.718	20.126113\\
3.72	20.127278\\
3.722	20.128444\\
3.724	20.129612\\
3.726	20.130782\\
3.728	20.131954\\
3.73	20.133127\\
3.732	20.134301\\
3.734	20.135478\\
3.736	20.136655\\
3.738	20.137834\\
3.74	20.139014\\
3.742	20.140195\\
3.744	20.141377\\
3.746	20.142561\\
3.748	20.143745\\
3.75	20.14493\\
3.752	20.146117\\
3.754	20.147303\\
3.756	20.148491\\
3.758	20.14968\\
3.76	20.150868\\
3.762	20.152058\\
3.764	20.153248\\
3.766	20.154438\\
3.768	20.155629\\
3.77	20.15682\\
3.772	20.158011\\
3.774	20.159203\\
3.776	20.160394\\
3.778	20.161586\\
3.78	20.162777\\
3.782	20.163969\\
3.784	20.16516\\
3.786	20.166351\\
3.788	20.167541\\
3.79	20.168731\\
3.792	20.169921\\
3.794	20.171111\\
3.796	20.172299\\
3.798	20.173487\\
3.8	20.174675\\
3.802	20.175862\\
3.804	20.177047\\
3.806	20.178232\\
3.808	20.179416\\
3.81	20.180599\\
3.812	20.181781\\
3.814	20.182962\\
3.816	20.184141\\
3.818	20.18532\\
3.82	20.186497\\
3.822	20.187672\\
3.824	20.188846\\
3.826	20.190019\\
3.828	20.191189\\
3.83	20.192359\\
3.832	20.193526\\
3.834	20.194692\\
3.836	20.195856\\
3.838	20.197017\\
3.84	20.198177\\
3.842	20.199335\\
3.844	20.200491\\
3.846	20.201644\\
3.848	20.202796\\
3.85	20.203945\\
3.852	20.205091\\
3.854	20.206235\\
3.856	20.207377\\
3.858	20.208516\\
3.86	20.209653\\
3.862	20.210786\\
3.864	20.211917\\
3.866	20.213046\\
3.868	20.214171\\
3.87	20.215293\\
3.872	20.216413\\
3.874	20.217529\\
3.876	20.218642\\
3.878	20.219752\\
3.88	20.220859\\
3.882	20.221962\\
3.884	20.223062\\
3.886	20.224159\\
3.888	20.225252\\
3.89	20.226341\\
3.892	20.227427\\
3.894	20.228509\\
3.896	20.229588\\
3.898	20.230662\\
3.9	20.231733\\
3.902	20.2328\\
3.904	20.233863\\
3.906	20.234921\\
3.908	20.235976\\
3.91	20.237026\\
3.912	20.238073\\
3.914	20.239114\\
3.916	20.240152\\
3.918	20.241185\\
3.92	20.242214\\
3.922	20.243238\\
3.924	20.244258\\
3.926	20.245272\\
3.928	20.246283\\
3.93	20.247288\\
3.932	20.248289\\
3.934	20.249284\\
3.936	20.250275\\
3.938	20.251261\\
3.94	20.252242\\
3.942	20.253218\\
3.944	20.254188\\
3.946	20.255153\\
3.948	20.256113\\
3.95	20.257068\\
3.952	20.258017\\
3.954	20.258961\\
3.956	20.259899\\
3.958	20.260832\\
3.96	20.261759\\
3.962	20.262681\\
3.964	20.263597\\
3.966	20.264507\\
3.968	20.265411\\
3.97	20.266309\\
3.972	20.267202\\
3.974	20.268088\\
3.976	20.268969\\
3.978	20.269843\\
3.98	20.270711\\
3.982	20.271573\\
3.984	20.272429\\
3.986	20.273279\\
3.988	20.274122\\
3.99	20.274959\\
3.992	20.275789\\
3.994	20.276613\\
3.996	20.27743\\
3.998	20.278241\\
4	20.279045\\
};
\addlegendentry{$\theta$};

\addplot [color=mycolor4,solid]
  table[row sep=crcr]{%
0	3.41061e-12\\
0.002	6.82121e-13\\
0.004	2.95586e-12\\
0.006	-1.13687e-12\\
0.008	6.59384e-12\\
0.01	-1.59162e-12\\
0.012	5.68434e-12\\
0.014	-4.54747e-13\\
0.016	-2.27374e-12\\
0.018	3.63798e-12\\
0.02	4.54747e-12\\
0.022	4.54747e-13\\
0.024	-4.09273e-12\\
0.026	3.18323e-12\\
0.028	5.00222e-12\\
0.03	0\\
0.032	-6.36646e-12\\
0.034	-1.27329e-11\\
0.036	-1.31877e-11\\
0.038	-2.00089e-11\\
0.04	-1.43245e-11\\
0.042	-2.04636e-11\\
0.044	-2.75122e-11\\
0.046	-3.56977e-11\\
0.048	-3.25144e-11\\
0.05	-3.81988e-11\\
0.052	-4.06999e-11\\
0.054	-4.18368e-11\\
0.056	-4.43379e-11\\
0.058	-4.502e-11\\
0.06	-4.82032e-11\\
0.062	-4.91127e-11\\
0.064	-4.91127e-11\\
0.066	-5.36602e-11\\
0.068	-5.00222e-11\\
0.07	-4.38831e-11\\
0.072	-3.68345e-11\\
0.074	-3.27418e-11\\
0.076	-4.04725e-11\\
0.078	-3.36513e-11\\
0.08	-3.70619e-11\\
0.082	-3.45608e-11\\
0.084	-3.04681e-11\\
0.086	-2.41016e-11\\
0.088	-1.90994e-11\\
0.09	-1.18234e-11\\
0.092	-1.65983e-11\\
0.094	-2.13731e-11\\
0.096	-2.45564e-11\\
0.098	-2.6148e-11\\
0.1	-3.41061e-11\\
0.102	-3.00133e-11\\
0.104	-3.22871e-11\\
0.106	-2.81943e-11\\
0.108	-3.11502e-11\\
0.11	-3.29692e-11\\
0.112	-2.91038e-11\\
0.114	-3.09228e-11\\
0.116	-2.84217e-11\\
0.118	-3.09228e-11\\
0.12	-3.02407e-11\\
0.122	-2.6148e-11\\
0.124	-3.06954e-11\\
0.126	-2.72848e-11\\
0.128	-2.54659e-11\\
0.13	-3.29692e-11\\
0.132	-3.36513e-11\\
0.134	-2.95586e-11\\
0.136	-3.20597e-11\\
0.138	-2.72848e-11\\
0.14	-3.20597e-11\\
0.142	-2.50111e-11\\
0.144	-2.251e-11\\
0.146	-2.45564e-11\\
0.148	-2.47837e-11\\
0.15	-2.11458e-11\\
0.152	-2.56932e-11\\
0.154	-2.36469e-11\\
0.156	-1.8872e-11\\
0.158	-2.11458e-11\\
0.16	-2.11458e-11\\
0.162	-1.68257e-11\\
0.164	-1.77351e-11\\
0.166	-1.5234e-11\\
0.168	-7.7307e-12\\
0.17	-1.81899e-12\\
0.172	-4.54747e-13\\
0.174	-7.04858e-12\\
0.176	-1.29603e-11\\
0.178	-1.29603e-11\\
0.18	-1.47793e-11\\
0.182	-2.251e-11\\
0.184	-2.56932e-11\\
0.186	-2.36469e-11\\
0.188	-3.04681e-11\\
0.19	-3.50155e-11\\
0.192	-3.11502e-11\\
0.194	-2.9786e-11\\
0.196	-2.54659e-11\\
0.198	-2.13731e-11\\
0.2	-1.97815e-11\\
0.202	-1.75078e-11\\
0.204	-1.36424e-11\\
0.206	-1.38698e-11\\
0.208	-1.65983e-11\\
0.21	-2.11458e-11\\
0.212	-2.75122e-11\\
0.214	-2.59206e-11\\
0.216	-2.13731e-11\\
0.218	-2.93312e-11\\
0.22	-3.68345e-11\\
0.222	-3.04681e-11\\
0.224	-2.88765e-11\\
0.226	-2.70575e-11\\
0.228	-3.29692e-11\\
0.23	-2.63753e-11\\
0.232	-3.09228e-11\\
0.234	-3.70619e-11\\
0.236	-3.29692e-11\\
0.238	-3.13776e-11\\
0.24	-2.81943e-11\\
0.242	-2.84217e-11\\
0.244	-2.13731e-11\\
0.246	-1.93268e-11\\
0.248	-1.47793e-11\\
0.25	-1.15961e-11\\
0.252	-4.09273e-12\\
0.254	-4.54747e-13\\
0.256	-4.3201e-12\\
0.258	-8.6402e-12\\
0.26	-1.11413e-11\\
0.262	-6.59384e-12\\
0.264	1.36424e-12\\
0.266	8.18545e-12\\
0.268	1.81899e-12\\
0.27	-5.91172e-12\\
0.272	-1.31877e-11\\
0.274	-1.97815e-11\\
0.276	-2.251e-11\\
0.278	-2.75122e-11\\
0.28	-2.16005e-11\\
0.282	-2.13731e-11\\
0.284	-1.84173e-11\\
0.286	-1.77351e-11\\
0.288	-1.75078e-11\\
0.29	-2.38742e-11\\
0.292	-3.04681e-11\\
0.294	-2.34195e-11\\
0.296	-2.95586e-11\\
0.298	-2.95586e-11\\
0.3	-2.70575e-11\\
0.302	-2.88765e-11\\
0.304	-2.6148e-11\\
0.306	-2.56932e-11\\
0.308	-2.27374e-11\\
0.31	-2.29647e-11\\
0.312	-2.38742e-11\\
0.314	-2.45564e-11\\
0.316	-2.11458e-11\\
0.318	-2.52385e-11\\
0.32	-1.95541e-11\\
0.322	-2.31921e-11\\
0.324	-2.52385e-11\\
0.326	-2.04636e-11\\
0.328	-1.77351e-11\\
0.33	-1.31877e-11\\
0.332	-1.86446e-11\\
0.334	-2.09184e-11\\
0.336	-1.47793e-11\\
0.338	-1.36424e-11\\
0.34	-1.36424e-11\\
0.342	-1.43245e-11\\
0.344	-7.7307e-12\\
0.346	-1.11413e-11\\
0.348	-7.04858e-12\\
0.35	-9.54969e-12\\
0.352	-5.45697e-12\\
0.354	-7.27596e-12\\
0.356	-2.04636e-12\\
0.358	3.86535e-12\\
0.36	1.06866e-11\\
0.362	3.63798e-12\\
0.364	6.59384e-12\\
0.366	-9.09495e-13\\
0.368	4.09273e-12\\
0.37	1.13687e-11\\
0.372	1.18234e-11\\
0.374	8.6402e-12\\
0.376	9.32232e-12\\
0.378	5.68434e-12\\
0.38	-2.27374e-13\\
0.382	-2.50111e-12\\
0.384	-2.04636e-12\\
0.386	-2.95586e-12\\
0.388	-9.09495e-13\\
0.39	-6.82121e-12\\
0.392	-1.59162e-12\\
0.394	-5.68434e-12\\
0.396	-5.91172e-12\\
0.398	-2.27374e-13\\
0.4	7.27596e-12\\
0.402	1.45519e-11\\
0.404	1.36424e-11\\
0.406	1.7053e-11\\
0.408	1.63709e-11\\
0.41	1.8872e-11\\
0.412	2.04636e-11\\
0.414	1.5234e-11\\
0.416	1.8872e-11\\
0.418	2.251e-11\\
0.42	2.84217e-11\\
0.422	3.52429e-11\\
0.424	3.81988e-11\\
0.426	3.5925e-11\\
0.428	4.11546e-11\\
0.43	4.41105e-11\\
0.432	4.09273e-11\\
0.434	4.20641e-11\\
0.436	3.81988e-11\\
0.438	3.70619e-11\\
0.44	3.47882e-11\\
0.442	3.11502e-11\\
0.444	3.52429e-11\\
0.446	3.9563e-11\\
0.448	4.1382e-11\\
0.45	4.41105e-11\\
0.452	4.6839e-11\\
0.454	4.3201e-11\\
0.456	4.63842e-11\\
0.458	4.91127e-11\\
0.46	4.34284e-11\\
0.462	3.97904e-11\\
0.464	4.43379e-11\\
0.466	4.8658e-11\\
0.468	5.16138e-11\\
0.47	4.36557e-11\\
0.472	4.25189e-11\\
0.474	4.04725e-11\\
0.476	4.52474e-11\\
0.478	3.86535e-11\\
0.48	3.38787e-11\\
0.482	2.63753e-11\\
0.484	2.251e-11\\
0.486	2.18279e-11\\
0.488	1.84173e-11\\
0.49	1.04592e-11\\
0.492	5.22959e-12\\
0.494	8.6402e-12\\
0.496	1.04592e-11\\
0.498	5.68434e-12\\
0.5	4.3201e-12\\
0.502	4.54747e-13\\
0.504	-5.11591e-12\\
0.506	-1.13687e-11\\
0.508	-1.6712e-11\\
0.51	-1.50067e-11\\
0.512	-1.90994e-11\\
0.514	-1.51203e-11\\
0.516	-8.98126e-12\\
0.518	-5.22959e-12\\
0.52	-1.02318e-11\\
0.522	-4.43379e-12\\
0.524	1.93268e-12\\
0.526	4.66116e-12\\
0.528	7.50333e-12\\
0.53	7.50333e-12\\
0.532	2.84217e-12\\
0.534	8.75389e-12\\
0.536	1.60298e-11\\
0.538	1.61435e-11\\
0.54	2.08047e-11\\
0.542	2.69438e-11\\
0.544	3.35376e-11\\
0.546	3.78577e-11\\
0.548	4.21778e-11\\
0.55	3.60387e-11\\
0.552	4.08136e-11\\
0.554	3.46745e-11\\
0.556	3.46745e-11\\
0.558	3.52429e-11\\
0.56	3.76303e-11\\
0.562	3.30829e-11\\
0.564	3.17186e-11\\
0.566	3.04681e-11\\
0.568	3.1946e-11\\
0.57	3.3765e-11\\
0.572	3.87672e-11\\
0.574	3.94493e-11\\
0.576	4.28599e-11\\
0.578	3.71756e-11\\
0.58	3.5584e-11\\
0.582	3.38787e-11\\
0.584	2.84217e-11\\
0.586	3.41061e-11\\
0.588	3.71756e-11\\
0.59	2.9786e-11\\
0.592	3.20597e-11\\
0.594	3.7403e-11\\
0.596	3.36513e-11\\
0.598	3.89946e-11\\
0.6	3.20597e-11\\
0.602	2.72848e-11\\
0.604	2.77396e-11\\
0.606	2.26237e-11\\
0.608	1.8531e-11\\
0.61	1.21645e-11\\
0.612	1.73941e-11\\
0.614	1.73941e-11\\
0.616	1.20508e-11\\
0.618	8.52651e-12\\
0.62	1.58025e-11\\
0.622	2.09184e-11\\
0.624	1.80762e-11\\
0.626	1.03455e-11\\
0.628	5.45697e-12\\
0.63	4.66116e-12\\
0.632	4.3201e-12\\
0.634	7.95808e-13\\
0.636	7.27596e-12\\
0.638	3.41061e-13\\
0.64	5.68434e-13\\
0.642	3.06954e-12\\
0.644	-1.13687e-12\\
0.646	-1.25056e-12\\
0.648	-6.70752e-12\\
0.65	-2.6148e-12\\
0.652	-6.70752e-12\\
0.654	-8.41283e-12\\
0.656	-3.52429e-12\\
0.658	-1.05729e-11\\
0.66	-1.11413e-11\\
0.662	-3.63798e-12\\
0.664	-7.50333e-12\\
0.666	-2.16005e-12\\
0.668	2.38742e-12\\
0.67	8.6402e-12\\
0.672	1.59162e-12\\
0.674	3.75167e-12\\
0.676	3.41061e-13\\
0.678	5.00222e-12\\
0.68	-2.38742e-12\\
0.682	-3.41061e-12\\
0.684	-6.36646e-12\\
0.686	-6.0254e-12\\
0.688	-7.7307e-12\\
0.69	-4.77485e-12\\
0.692	1.81899e-12\\
0.694	4.43379e-12\\
0.696	2.95586e-12\\
0.698	-1.7053e-12\\
0.7	2.84217e-12\\
0.702	5.00222e-12\\
0.704	8.6402e-12\\
0.706	1.43245e-11\\
0.708	1.4893e-11\\
0.71	8.86757e-12\\
0.712	1.39835e-11\\
0.714	9.54969e-12\\
0.716	1.73941e-11\\
0.718	1.45519e-11\\
0.72	1.18234e-11\\
0.722	1.90994e-11\\
0.724	1.68257e-11\\
0.726	1.8872e-11\\
0.728	1.58025e-11\\
0.73	1.58025e-11\\
0.732	2.30784e-11\\
0.734	1.87583e-11\\
0.736	1.64846e-11\\
0.738	1.4893e-11\\
0.74	1.3415e-11\\
0.742	9.89075e-12\\
0.744	1.81899e-12\\
0.746	-2.72848e-12\\
0.748	-1.93268e-12\\
0.75	-3.97904e-12\\
0.752	-3.41061e-13\\
0.754	-2.04636e-12\\
0.756	-8.6402e-12\\
0.758	-4.20641e-12\\
0.76	-8.75389e-12\\
0.762	-2.84217e-12\\
0.764	-5.91172e-12\\
0.766	-2.50111e-12\\
0.768	-1.36424e-12\\
0.77	-4.94538e-12\\
0.772	-8.12861e-12\\
0.774	-1.45519e-11\\
0.776	-1.66551e-11\\
0.778	-1.86446e-11\\
0.78	-2.10321e-11\\
0.782	-2.55795e-11\\
0.784	-2.95017e-11\\
0.786	-2.58638e-11\\
0.788	-2.96723e-11\\
0.79	-3.16618e-11\\
0.792	-3.6664e-11\\
0.794	-4.23483e-11\\
0.796	-3.54703e-11\\
0.798	-3.60956e-11\\
0.8	-3.15481e-11\\
0.802	-3.07523e-11\\
0.804	-2.48406e-11\\
0.806	-1.87015e-11\\
0.808	-2.37605e-11\\
0.81	-3.13207e-11\\
0.812	-3.63798e-11\\
0.814	-3.89377e-11\\
0.816	-3.44471e-11\\
0.818	-3.89946e-11\\
0.82	-3.89377e-11\\
0.822	-4.11546e-11\\
0.824	-4.18368e-11\\
0.826	-4.83169e-11\\
0.828	-5.12728e-11\\
0.83	-4.67821e-11\\
0.832	-4.35989e-11\\
0.834	-4.90559e-11\\
0.836	-4.45652e-11\\
0.838	-3.72893e-11\\
0.84	-3.00133e-11\\
0.842	-2.3249e-11\\
0.844	-1.79057e-11\\
0.846	-1.11413e-11\\
0.848	-1.68825e-11\\
0.85	-1.08571e-11\\
0.852	-1.26192e-11\\
0.854	-1.77351e-11\\
0.856	-1.25624e-11\\
0.858	-5.91172e-12\\
0.86	2.38742e-12\\
0.862	1.13687e-13\\
0.864	-6.13909e-12\\
0.866	-6.70752e-12\\
0.868	-1.36424e-11\\
0.87	-1.8872e-11\\
0.872	-1.89857e-11\\
0.874	-1.62572e-11\\
0.876	-1.17097e-11\\
0.878	-9.15179e-12\\
0.88	-1.19371e-11\\
0.882	-1.76783e-11\\
0.884	-1.02318e-11\\
0.886	-1.27898e-11\\
0.888	-1.77351e-11\\
0.89	-2.54659e-11\\
0.892	-2.58069e-11\\
0.894	-3.17755e-11\\
0.896	-2.60911e-11\\
0.898	-2.51248e-11\\
0.9	-2.88765e-11\\
0.902	-3.00702e-11\\
0.904	-3.72893e-11\\
0.906	-3.54703e-11\\
0.908	-2.91038e-11\\
0.91	-2.66027e-11\\
0.912	-2.87628e-11\\
0.914	-3.00133e-11\\
0.916	-3.08091e-11\\
0.918	-2.59206e-11\\
0.92	-1.95541e-11\\
0.922	-1.50067e-11\\
0.924	-1.81899e-11\\
0.926	-1.3415e-11\\
0.928	-1.10276e-11\\
0.93	-1.58025e-11\\
0.932	-2.23963e-11\\
0.934	-1.60298e-11\\
0.936	-2.34195e-11\\
0.938	-2.72848e-11\\
0.94	-2.03499e-11\\
0.942	-2.20552e-11\\
0.944	-1.50067e-11\\
0.946	-1.47793e-11\\
0.948	-1.25056e-11\\
0.95	-8.98126e-12\\
0.952	-1.08002e-11\\
0.954	-8.98126e-12\\
0.956	-4.3201e-12\\
0.958	-1.09139e-11\\
0.96	-1.03455e-11\\
0.962	-3.18323e-12\\
0.964	-4.88853e-12\\
0.966	3.18323e-12\\
0.968	7.38964e-12\\
0.97	5.91172e-12\\
0.972	1.47793e-12\\
0.974	9.54969e-12\\
0.976	8.18545e-12\\
0.978	4.43379e-12\\
0.98	8.86757e-12\\
0.982	7.27596e-12\\
0.984	9.09495e-12\\
0.986	6.70752e-12\\
0.988	6.70752e-12\\
0.99	1.15961e-11\\
0.992	6.48015e-12\\
0.994	1.3074e-11\\
0.996	2.01226e-11\\
0.998	1.69393e-11\\
1	1.39835e-11\\
1.002	1.44382e-11\\
1.004	1.26192e-11\\
1.006	4.54747e-12\\
1.008	1.47793e-12\\
1.01	3.18323e-12\\
1.012	-5.00222e-12\\
1.014	1.81899e-12\\
1.016	-5.79803e-12\\
1.018	-1.00044e-11\\
1.02	-9.66338e-12\\
1.022	-1.05729e-11\\
1.024	-1.14824e-11\\
1.026	-1.78488e-11\\
1.028	-2.26237e-11\\
1.03	-2.16005e-11\\
1.032	-2.12594e-11\\
1.034	-1.8872e-11\\
1.036	-1.13687e-11\\
1.038	-3.63798e-12\\
1.04	2.04636e-12\\
1.042	9.32232e-12\\
1.044	1.28466e-11\\
1.046	1.93268e-11\\
1.048	1.28466e-11\\
1.05	1.39835e-11\\
1.052	1.54614e-11\\
1.054	8.98126e-12\\
1.056	2.84217e-12\\
1.058	7.27596e-12\\
1.06	3.63798e-12\\
1.062	-1.25056e-12\\
1.064	2.38742e-12\\
1.066	-1.13687e-13\\
1.068	6.25278e-12\\
1.07	1.19371e-11\\
1.072	9.54969e-12\\
1.074	1.65983e-11\\
1.076	1.75078e-11\\
1.078	2.39879e-11\\
1.08	1.77351e-11\\
1.082	1.13687e-11\\
1.084	1.4893e-11\\
1.086	1.02318e-11\\
1.088	1.17097e-11\\
1.09	1.03455e-11\\
1.092	1.60298e-11\\
1.094	2.20552e-11\\
1.096	2.22826e-11\\
1.098	2.4329e-11\\
1.1	1.79625e-11\\
1.102	1.54614e-11\\
1.104	1.04592e-11\\
1.106	1.14824e-11\\
1.108	1.63709e-11\\
1.11	1.44382e-11\\
1.112	7.7307e-12\\
1.114	5.22959e-12\\
1.116	7.16227e-12\\
1.118	4.09273e-12\\
1.12	-2.38742e-12\\
1.122	1.36424e-12\\
1.124	-5.68434e-12\\
1.126	-1.21645e-11\\
1.128	-1.23919e-11\\
1.13	-6.9349e-12\\
1.132	-9.09495e-13\\
1.134	-1.13687e-12\\
1.136	-5.34328e-12\\
1.138	-1.22782e-11\\
1.14	-9.54969e-12\\
1.142	-1.43245e-11\\
1.144	-1.21645e-11\\
1.146	-1.60298e-11\\
1.148	-2.18279e-11\\
1.15	-2.41016e-11\\
1.152	-2.11458e-11\\
1.154	-1.95541e-11\\
1.156	-1.46656e-11\\
1.158	-1.33014e-11\\
1.16	-6.36646e-12\\
1.162	-1.3415e-11\\
1.164	-1.47793e-11\\
1.166	-1.86446e-11\\
1.168	-1.39835e-11\\
1.17	-1.50067e-11\\
1.172	-2.27374e-11\\
1.174	-2.60343e-11\\
1.176	-3.29692e-11\\
1.178	-3.43334e-11\\
1.18	-3.70619e-11\\
1.182	-3.50155e-11\\
1.184	-3.47882e-11\\
1.186	-4.20641e-11\\
1.188	-4.63842e-11\\
1.19	-4.41105e-11\\
1.192	-3.9563e-11\\
1.194	-3.84262e-11\\
1.196	-3.43334e-11\\
1.198	-3.54703e-11\\
1.2	-3.22871e-11\\
1.202	-3.72893e-11\\
1.204	-3.66072e-11\\
1.206	-3.84262e-11\\
1.208	-3.68345e-11\\
1.21	-3.63798e-11\\
1.212	-3.81988e-11\\
1.214	-4.16094e-11\\
1.216	-4.84306e-11\\
1.218	-5.27507e-11\\
1.22	-4.77485e-11\\
1.222	-4.36557e-11\\
1.224	-4.43379e-11\\
1.226	-4.57021e-11\\
1.228	-5.18412e-11\\
1.23	-5.38876e-11\\
1.232	-5.11591e-11\\
1.234	-5.0477e-11\\
1.236	-4.34284e-11\\
1.238	-3.86535e-11\\
1.24	-3.79714e-11\\
1.242	-3.5925e-11\\
1.244	-3.34239e-11\\
1.246	-3.43334e-11\\
1.248	-3.41061e-11\\
1.25	-3.34239e-11\\
1.252	-3.25144e-11\\
1.254	-2.63753e-11\\
1.256	-3.11502e-11\\
1.258	-2.4329e-11\\
1.26	-2.75122e-11\\
1.262	-2.59206e-11\\
1.264	-2.11458e-11\\
1.266	-1.40972e-11\\
1.268	-1.20508e-11\\
1.27	-1.00044e-11\\
1.272	-6.82121e-12\\
1.274	-1.18234e-11\\
1.276	-1.79625e-11\\
1.278	-1.38698e-11\\
1.28	-2.0691e-11\\
1.282	-2.38742e-11\\
1.284	-2.13731e-11\\
1.286	-1.50067e-11\\
1.288	-1.68257e-11\\
1.29	-1.20508e-11\\
1.292	-6.36646e-12\\
1.294	-1.22782e-11\\
1.296	-4.77485e-12\\
1.298	1.59162e-12\\
1.3	8.86757e-12\\
1.302	9.77707e-12\\
1.304	1.11413e-11\\
1.306	6.59384e-12\\
1.308	1.06866e-11\\
1.31	9.32232e-12\\
1.312	1.29603e-11\\
1.314	1.00044e-11\\
1.316	3.18323e-12\\
1.318	6.82121e-13\\
1.32	7.50333e-12\\
1.322	6.82121e-13\\
1.324	-3.86535e-12\\
1.326	1.13687e-12\\
1.328	8.6402e-12\\
1.33	3.86535e-12\\
1.332	-2.50111e-12\\
1.334	9.09495e-13\\
1.336	-6.36646e-12\\
1.338	-9.77707e-12\\
1.34	-4.3201e-12\\
1.342	-2.72848e-12\\
1.344	2.04636e-12\\
1.346	5.22959e-12\\
1.348	3.86535e-12\\
1.35	4.3201e-12\\
1.352	2.95586e-12\\
1.354	5.91172e-12\\
1.356	5.22959e-12\\
1.358	4.54747e-13\\
1.36	8.18545e-12\\
1.362	3.18323e-12\\
1.364	-4.3201e-12\\
1.366	-6.36646e-12\\
1.368	-9.54969e-12\\
1.37	-7.7307e-12\\
1.372	-1.31877e-11\\
1.374	-8.6402e-12\\
1.376	-2.50111e-12\\
1.378	-1.13687e-12\\
1.38	6.13909e-12\\
1.382	8.6402e-12\\
1.384	1.47793e-11\\
1.386	7.27596e-12\\
1.388	1.18234e-11\\
1.39	1.84173e-11\\
1.392	1.11413e-11\\
1.394	1.27329e-11\\
1.396	1.59162e-11\\
1.398	2.16005e-11\\
1.4	1.7053e-11\\
1.402	2.16005e-11\\
1.404	2.81943e-11\\
1.406	3.09228e-11\\
1.408	3.50155e-11\\
1.41	3.11502e-11\\
1.412	3.31966e-11\\
1.414	3.97904e-11\\
1.416	3.31966e-11\\
1.418	4.04725e-11\\
1.42	4.25189e-11\\
1.422	3.9563e-11\\
1.424	3.79714e-11\\
1.426	4.11546e-11\\
1.428	3.31966e-11\\
1.43	3.29692e-11\\
1.432	3.86535e-11\\
1.434	3.22871e-11\\
1.436	2.59206e-11\\
1.438	2.41016e-11\\
1.44	2.9786e-11\\
1.442	2.31921e-11\\
1.444	3.02407e-11\\
1.446	2.251e-11\\
1.448	2.72848e-11\\
1.45	3.45608e-11\\
1.452	3.88809e-11\\
1.454	4.00178e-11\\
1.456	4.06999e-11\\
1.458	4.52474e-11\\
1.46	5.09317e-11\\
1.462	5.88898e-11\\
1.464	6.29825e-11\\
1.466	6.59384e-11\\
1.468	7.04858e-11\\
1.47	6.79847e-11\\
1.472	7.1168e-11\\
1.474	7.52607e-11\\
1.476	7.66249e-11\\
1.478	7.27596e-11\\
1.48	7.68523e-11\\
1.482	7.54881e-11\\
1.484	7.25322e-11\\
1.486	7.77618e-11\\
1.488	7.70797e-11\\
1.49	7.59428e-11\\
1.492	8.13998e-11\\
1.494	7.84439e-11\\
1.496	8.57199e-11\\
1.498	8.07177e-11\\
1.5	8.75389e-11\\
1.502	8.18545e-11\\
1.504	7.88987e-11\\
1.506	8.61746e-11\\
1.508	9.11768e-11\\
1.51	8.54925e-11\\
1.512	8.04903e-11\\
1.514	8.11724e-11\\
1.516	7.95808e-11\\
1.518	8.61746e-11\\
1.52	9.34506e-11\\
1.522	9.91349e-11\\
1.524	9.75433e-11\\
1.526	1.01181e-10\\
1.528	9.34506e-11\\
1.53	9.73159e-11\\
1.532	9.11768e-11\\
1.534	8.50378e-11\\
1.536	8.61746e-11\\
1.538	8.66294e-11\\
1.54	8.89031e-11\\
1.542	8.93579e-11\\
1.544	9.52696e-11\\
1.546	9.09495e-11\\
1.548	9.34506e-11\\
1.55	9.73159e-11\\
1.552	9.07221e-11\\
1.554	8.91305e-11\\
1.556	8.57199e-11\\
1.558	8.29914e-11\\
1.56	9.02673e-11\\
1.562	8.61746e-11\\
1.564	9.16316e-11\\
1.566	8.66294e-11\\
1.568	8.77662e-11\\
1.57	8.16271e-11\\
1.572	8.36735e-11\\
1.574	7.57154e-11\\
1.576	6.82121e-11\\
1.578	6.32099e-11\\
1.58	5.77529e-11\\
1.582	5.70708e-11\\
1.584	5.59339e-11\\
1.586	5.77529e-11\\
1.588	5.61613e-11\\
1.59	5.07043e-11\\
1.592	4.75211e-11\\
1.594	4.84306e-11\\
1.596	4.502e-11\\
1.598	4.57021e-11\\
1.6	4.38831e-11\\
1.602	3.79714e-11\\
1.604	4.18368e-11\\
1.606	4.77485e-11\\
1.608	4.8658e-11\\
1.61	4.3201e-11\\
1.612	3.63798e-11\\
1.614	4.27463e-11\\
1.616	3.97904e-11\\
1.618	4.38831e-11\\
1.62	5.18412e-11\\
1.622	5.68434e-11\\
1.624	5.02496e-11\\
1.626	4.8658e-11\\
1.628	5.27507e-11\\
1.63	4.70664e-11\\
1.632	5.29781e-11\\
1.634	6.09361e-11\\
1.636	5.97993e-11\\
1.638	5.68434e-11\\
1.64	5.97993e-11\\
1.642	6.52562e-11\\
1.644	7.20775e-11\\
1.646	7.77618e-11\\
1.648	7.02585e-11\\
1.65	7.59428e-11\\
1.652	7.7307e-11\\
1.654	7.98082e-11\\
1.656	8.70841e-11\\
1.658	9.1859e-11\\
1.66	9.52696e-11\\
1.662	1.02773e-10\\
1.664	1.10276e-10\\
1.666	1.05501e-10\\
1.668	1.09821e-10\\
1.67	1.05047e-10\\
1.672	1.10049e-10\\
1.674	1.16188e-10\\
1.676	1.18007e-10\\
1.678	1.14142e-10\\
1.68	1.09821e-10\\
1.682	1.05047e-10\\
1.684	9.9817e-11\\
1.686	1.02773e-10\\
1.688	1.04592e-10\\
1.69	1.07775e-10\\
1.692	1.05956e-10\\
1.694	9.89075e-11\\
1.696	9.64064e-11\\
1.698	9.29958e-11\\
1.7	9.34506e-11\\
1.702	9.54969e-11\\
1.704	9.79981e-11\\
1.706	9.41327e-11\\
1.708	9.07221e-11\\
1.71	8.77662e-11\\
1.712	8.39009e-11\\
1.714	8.79936e-11\\
1.716	8.70841e-11\\
1.718	8.0945e-11\\
1.72	8.77662e-11\\
1.722	8.61746e-11\\
1.724	8.29914e-11\\
1.726	8.77662e-11\\
1.728	9.41327e-11\\
1.73	9.32232e-11\\
1.732	9.59517e-11\\
1.734	1.03682e-10\\
1.736	9.91349e-11\\
1.738	9.9817e-11\\
1.74	9.89075e-11\\
1.742	1.01409e-10\\
1.744	9.43601e-11\\
1.746	8.98126e-11\\
1.748	9.14042e-11\\
1.75	8.41283e-11\\
1.752	8.36735e-11\\
1.754	8.54925e-11\\
1.756	9.16316e-11\\
1.758	9.09495e-11\\
1.76	8.98126e-11\\
1.762	8.70841e-11\\
1.764	8.34461e-11\\
1.766	8.93579e-11\\
1.768	8.25366e-11\\
1.77	8.50378e-11\\
1.772	7.84439e-11\\
1.774	7.13953e-11\\
1.776	6.91216e-11\\
1.778	6.61657e-11\\
1.78	6.66205e-11\\
1.782	7.38964e-11\\
1.784	7.75344e-11\\
1.786	8.07177e-11\\
1.788	8.32188e-11\\
1.79	7.59428e-11\\
1.792	7.88987e-11\\
1.794	8.48104e-11\\
1.796	7.86713e-11\\
1.798	8.48104e-11\\
1.8	9.09495e-11\\
1.802	9.20863e-11\\
1.804	8.75389e-11\\
1.806	9.14042e-11\\
1.808	8.36735e-11\\
1.81	7.61702e-11\\
1.812	7.75344e-11\\
1.814	7.1168e-11\\
1.816	6.43468e-11\\
1.818	6.63931e-11\\
1.82	6.61657e-11\\
1.822	6.5711e-11\\
1.824	7.13953e-11\\
1.826	7.38964e-11\\
1.828	8.00355e-11\\
1.83	7.86713e-11\\
1.832	7.63976e-11\\
1.834	7.41238e-11\\
1.836	8.00355e-11\\
1.838	8.39009e-11\\
1.84	8.34461e-11\\
1.842	8.07177e-11\\
1.844	7.82165e-11\\
1.846	7.29869e-11\\
1.848	7.00311e-11\\
1.85	6.5711e-11\\
1.852	5.8435e-11\\
1.854	6.13909e-11\\
1.856	5.70708e-11\\
1.858	5.13865e-11\\
1.86	4.38831e-11\\
1.862	3.81988e-11\\
1.864	3.38787e-11\\
1.866	3.84262e-11\\
1.868	4.57021e-11\\
1.87	4.95675e-11\\
1.872	5.32054e-11\\
1.874	5.02496e-11\\
1.876	5.11591e-11\\
1.878	4.47926e-11\\
1.88	4.88853e-11\\
1.882	4.25189e-11\\
1.884	3.72893e-11\\
1.886	3.06954e-11\\
1.888	2.93312e-11\\
1.89	3.25144e-11\\
1.892	2.81943e-11\\
1.894	2.93312e-11\\
1.896	3.31966e-11\\
1.898	2.86491e-11\\
1.9	3.02407e-11\\
1.902	2.81943e-11\\
1.904	3.41061e-11\\
1.906	3.66072e-11\\
1.908	3.16049e-11\\
1.91	2.81943e-11\\
1.912	3.04681e-11\\
1.914	3.66072e-11\\
1.916	3.06954e-11\\
1.918	3.54703e-11\\
1.92	2.91038e-11\\
1.922	2.6148e-11\\
1.924	2.6148e-11\\
1.926	2.59206e-11\\
1.928	3.20597e-11\\
1.93	3.72893e-11\\
1.932	4.27463e-11\\
1.934	4.36557e-11\\
1.936	3.68345e-11\\
1.938	3.45608e-11\\
1.94	2.77396e-11\\
1.942	3.04681e-11\\
1.944	3.41061e-11\\
1.946	3.79714e-11\\
1.948	4.00178e-11\\
1.95	4.52474e-11\\
1.952	4.57021e-11\\
1.954	5.29781e-11\\
1.956	5.68434e-11\\
1.958	5.18412e-11\\
1.96	4.95675e-11\\
1.962	4.22915e-11\\
1.964	3.88809e-11\\
1.966	3.86535e-11\\
1.968	3.52429e-11\\
1.97	3.13776e-11\\
1.972	3.52429e-11\\
1.974	2.88765e-11\\
1.976	2.41016e-11\\
1.978	1.95541e-11\\
1.98	2.50111e-11\\
1.982	3.09228e-11\\
1.984	3.86535e-11\\
1.986	4.1382e-11\\
1.988	4.75211e-11\\
1.99	4.00178e-11\\
1.992	3.66072e-11\\
1.994	4.02451e-11\\
1.996	4.52474e-11\\
1.998	3.9563e-11\\
2	4.29736e-11\\
2.002	5.0477e-11\\
2.004	4.66116e-11\\
2.006	5.27507e-11\\
2.008	4.8658e-11\\
2.01	5.34328e-11\\
2.012	5.32054e-11\\
2.014	4.97948e-11\\
2.016	4.57021e-11\\
2.018	5.09317e-11\\
2.02	4.70664e-11\\
2.022	4.38831e-11\\
2.024	4.63842e-11\\
2.026	5.07043e-11\\
2.028	5.79803e-11\\
2.03	5.38876e-11\\
2.032	4.8658e-11\\
2.034	5.0477e-11\\
2.036	5.29781e-11\\
2.038	5.93445e-11\\
2.04	5.88898e-11\\
2.042	5.52518e-11\\
2.044	5.57066e-11\\
2.046	5.11591e-11\\
2.048	4.91127e-11\\
2.05	4.6839e-11\\
2.052	4.00178e-11\\
2.054	4.70664e-11\\
2.056	4.25189e-11\\
2.058	4.61569e-11\\
2.06	5.13865e-11\\
2.062	5.00222e-11\\
2.064	4.70664e-11\\
2.066	4.502e-11\\
2.068	4.88853e-11\\
2.07	5.13865e-11\\
2.072	5.47971e-11\\
2.074	5.16138e-11\\
2.076	5.93445e-11\\
2.078	6.07088e-11\\
2.08	6.11635e-11\\
2.082	6.5711e-11\\
2.084	7.00311e-11\\
2.086	7.16227e-11\\
2.088	7.43512e-11\\
2.09	7.32143e-11\\
2.092	6.79847e-11\\
2.094	7.18501e-11\\
2.096	7.86713e-11\\
2.098	8.54925e-11\\
2.1	9.16316e-11\\
2.102	8.52651e-11\\
2.104	8.2764e-11\\
2.106	8.93579e-11\\
2.108	8.43556e-11\\
2.11	8.91305e-11\\
2.112	9.16316e-11\\
2.114	9.75433e-11\\
2.116	9.45874e-11\\
2.118	8.73115e-11\\
2.12	9.3678e-11\\
2.122	9.25411e-11\\
2.124	8.86757e-11\\
2.126	9.07221e-11\\
2.128	9.57243e-11\\
2.13	9.70886e-11\\
2.132	9.48148e-11\\
2.134	9.73159e-11\\
2.136	9.09495e-11\\
2.138	9.41327e-11\\
2.14	1.00727e-10\\
2.142	1.03228e-10\\
2.144	9.75433e-11\\
2.146	9.1859e-11\\
2.148	9.32232e-11\\
2.15	9.20863e-11\\
2.152	9.32232e-11\\
2.154	9.25411e-11\\
2.156	9.86802e-11\\
2.158	1.00044e-10\\
2.16	1.06638e-10\\
2.162	1.04137e-10\\
2.164	9.82254e-11\\
2.166	9.9817e-11\\
2.168	9.32232e-11\\
2.17	9.77707e-11\\
2.172	1.01181e-10\\
2.174	1.06638e-10\\
2.176	1.05956e-10\\
2.178	1.08344e-10\\
2.18	1.09594e-10\\
2.182	1.15165e-10\\
2.184	1.19371e-10\\
2.186	1.22213e-10\\
2.188	1.17666e-10\\
2.19	1.19599e-10\\
2.192	1.12777e-10\\
2.194	1.17552e-10\\
2.196	1.19257e-10\\
2.198	1.25169e-10\\
2.2	1.18348e-10\\
2.202	1.24942e-10\\
2.204	1.2983e-10\\
2.206	1.36083e-10\\
2.208	1.41085e-10\\
2.21	1.42222e-10\\
2.212	1.39494e-10\\
2.214	1.38812e-10\\
2.216	1.35174e-10\\
2.218	1.30171e-10\\
2.22	1.30058e-10\\
2.222	1.329e-10\\
2.224	1.38016e-10\\
2.226	1.4245e-10\\
2.228	1.36879e-10\\
2.23	1.34378e-10\\
2.232	1.29262e-10\\
2.234	1.33696e-10\\
2.236	1.3074e-10\\
2.238	1.36311e-10\\
2.24	1.35628e-10\\
2.242	1.29717e-10\\
2.244	1.34378e-10\\
2.246	1.37902e-10\\
2.248	1.35969e-10\\
2.25	1.29376e-10\\
2.252	1.24032e-10\\
2.254	1.23123e-10\\
2.256	1.29035e-10\\
2.258	1.22213e-10\\
2.26	1.29603e-10\\
2.262	1.34719e-10\\
2.264	1.41199e-10\\
2.266	1.43018e-10\\
2.268	1.44723e-10\\
2.27	1.49157e-10\\
2.272	1.54046e-10\\
2.274	1.55183e-10\\
2.276	1.52681e-10\\
2.278	1.59389e-10\\
2.28	1.6621e-10\\
2.282	1.59503e-10\\
2.284	1.63027e-10\\
2.286	1.63368e-10\\
2.288	1.57797e-10\\
2.29	1.50976e-10\\
2.292	1.56888e-10\\
2.294	1.63595e-10\\
2.296	1.59162e-10\\
2.298	1.54387e-10\\
2.3	1.6064e-10\\
2.302	1.56888e-10\\
2.304	1.51658e-10\\
2.306	1.49726e-10\\
2.308	1.5541e-10\\
2.31	1.58138e-10\\
2.312	1.52227e-10\\
2.314	1.55865e-10\\
2.316	1.54273e-10\\
2.318	1.51431e-10\\
2.32	1.437e-10\\
2.322	1.45178e-10\\
2.324	1.44269e-10\\
2.326	1.44837e-10\\
2.328	1.45405e-10\\
2.33	1.44837e-10\\
2.332	1.39494e-10\\
2.334	1.36879e-10\\
2.336	1.36879e-10\\
2.338	1.43132e-10\\
2.34	1.3938e-10\\
2.342	1.36197e-10\\
2.344	1.3074e-10\\
2.346	1.33468e-10\\
2.348	1.30058e-10\\
2.35	1.28239e-10\\
2.352	1.36083e-10\\
2.354	1.29717e-10\\
2.356	1.36652e-10\\
2.358	1.38698e-10\\
2.36	1.44269e-10\\
2.362	1.36538e-10\\
2.364	1.29603e-10\\
2.366	1.26306e-10\\
2.368	1.2335e-10\\
2.37	1.21531e-10\\
2.372	1.27784e-10\\
2.374	1.31877e-10\\
2.376	1.33127e-10\\
2.378	1.32104e-10\\
2.38	1.25965e-10\\
2.382	1.29944e-10\\
2.384	1.24373e-10\\
2.386	1.28807e-10\\
2.388	1.32104e-10\\
2.39	1.39835e-10\\
2.392	1.41199e-10\\
2.394	1.44951e-10\\
2.396	1.51431e-10\\
2.398	1.4802e-10\\
2.4	1.545e-10\\
2.402	1.60298e-10\\
2.404	1.66324e-10\\
2.406	1.61435e-10\\
2.408	1.58593e-10\\
2.41	1.5541e-10\\
2.412	1.60867e-10\\
2.414	1.68939e-10\\
2.416	1.71212e-10\\
2.418	1.70758e-10\\
2.42	1.77351e-10\\
2.422	1.74282e-10\\
2.424	1.73713e-10\\
2.426	1.81331e-10\\
2.428	1.74737e-10\\
2.43	1.73713e-10\\
2.432	1.71099e-10\\
2.434	1.74623e-10\\
2.436	1.7701e-10\\
2.438	1.84059e-10\\
2.44	1.85764e-10\\
2.442	1.91676e-10\\
2.444	1.86674e-10\\
2.446	1.82467e-10\\
2.448	1.80478e-10\\
2.45	1.85764e-10\\
2.452	1.86276e-10\\
2.454	1.8855e-10\\
2.456	1.87697e-10\\
2.458	1.84002e-10\\
2.46	1.85366e-10\\
2.462	1.82524e-10\\
2.464	1.90255e-10\\
2.466	1.87754e-10\\
2.468	1.91051e-10\\
2.47	1.97531e-10\\
2.472	2.03784e-10\\
2.474	2.02135e-10\\
2.476	2.02135e-10\\
2.478	2.09468e-10\\
2.48	2.15721e-10\\
2.482	2.2186e-10\\
2.484	2.15039e-10\\
2.486	2.22201e-10\\
2.488	2.20609e-10\\
2.49	2.22656e-10\\
2.492	2.2186e-10\\
2.494	2.16403e-10\\
2.496	2.10719e-10\\
2.498	2.15209e-10\\
2.5	2.22713e-10\\
2.502	2.21803e-10\\
2.504	2.18051e-10\\
2.506	2.10377e-10\\
2.508	2.12708e-10\\
2.51	2.13618e-10\\
2.512	2.1322e-10\\
2.514	2.16403e-10\\
2.516	2.09525e-10\\
2.518	2.15152e-10\\
2.52	2.09468e-10\\
2.522	2.1123e-10\\
2.524	2.14754e-10\\
2.526	2.19472e-10\\
2.528	2.12822e-10\\
2.53	2.12822e-10\\
2.532	2.06228e-10\\
2.534	2.08445e-10\\
2.536	2.10832e-10\\
2.538	2.11116e-10\\
2.54	2.04238e-10\\
2.542	2.06626e-10\\
2.544	2.0674e-10\\
2.546	2.02476e-10\\
2.548	2.03329e-10\\
2.55	2.04579e-10\\
2.552	1.98952e-10\\
2.554	1.94234e-10\\
2.556	1.90198e-10\\
2.558	1.94632e-10\\
2.56	1.9304e-10\\
2.562	2.01112e-10\\
2.564	2.05887e-10\\
2.566	2.04636e-10\\
2.568	2.07194e-10\\
2.57	2.04579e-10\\
2.572	2.09752e-10\\
2.574	2.1538e-10\\
2.576	2.18506e-10\\
2.578	2.2203e-10\\
2.58	2.20894e-10\\
2.582	2.2402e-10\\
2.584	2.26407e-10\\
2.586	2.26692e-10\\
2.588	2.23963e-10\\
2.59	2.25896e-10\\
2.592	2.27033e-10\\
2.594	2.28624e-10\\
2.596	2.25782e-10\\
2.598	2.28738e-10\\
2.6	2.30784e-10\\
2.602	2.38174e-10\\
2.604	2.42721e-10\\
2.606	2.49656e-10\\
2.608	2.44881e-10\\
2.61	2.42039e-10\\
2.612	2.49202e-10\\
2.614	2.45677e-10\\
2.616	2.49884e-10\\
2.618	2.45109e-10\\
2.62	2.43404e-10\\
2.622	2.46246e-10\\
2.624	2.39538e-10\\
2.626	2.47496e-10\\
2.628	2.47041e-10\\
2.63	2.50793e-10\\
2.632	2.45564e-10\\
2.634	2.4329e-10\\
2.636	2.46587e-10\\
2.638	2.39993e-10\\
2.64	2.37037e-10\\
2.642	2.28624e-10\\
2.644	2.35445e-10\\
2.646	2.36923e-10\\
2.648	2.42608e-10\\
2.65	2.42039e-10\\
2.652	2.34877e-10\\
2.654	2.33626e-10\\
2.656	2.27828e-10\\
2.658	2.22713e-10\\
2.66	2.2419e-10\\
2.662	2.23963e-10\\
2.664	2.16573e-10\\
2.666	2.19302e-10\\
2.668	2.13618e-10\\
2.67	2.16232e-10\\
2.672	2.16801e-10\\
2.674	2.16573e-10\\
2.676	2.19529e-10\\
2.678	2.11571e-10\\
2.68	2.17142e-10\\
2.682	2.10889e-10\\
2.684	2.143e-10\\
2.686	2.20439e-10\\
2.688	2.18847e-10\\
2.69	2.22599e-10\\
2.692	2.23622e-10\\
2.694	2.28852e-10\\
2.696	2.30443e-10\\
2.698	2.38174e-10\\
2.7	2.3897e-10\\
2.702	2.43972e-10\\
2.704	2.41243e-10\\
2.706	2.46132e-10\\
2.708	2.5284e-10\\
2.71	2.59092e-10\\
2.712	2.53408e-10\\
2.714	2.51703e-10\\
2.716	2.59547e-10\\
2.718	2.63867e-10\\
2.72	2.68756e-10\\
2.722	2.63753e-10\\
2.724	2.62844e-10\\
2.726	2.64549e-10\\
2.728	2.69324e-10\\
2.73	2.73076e-10\\
2.732	2.80693e-10\\
2.734	2.77055e-10\\
2.736	2.71029e-10\\
2.738	2.70006e-10\\
2.74	2.6796e-10\\
2.742	2.71143e-10\\
2.744	2.70916e-10\\
2.746	2.63071e-10\\
2.748	2.64322e-10\\
2.75	2.67505e-10\\
2.752	2.70802e-10\\
2.754	2.63753e-10\\
2.756	2.5716e-10\\
2.758	2.58865e-10\\
2.76	2.55227e-10\\
2.762	2.55341e-10\\
2.764	2.50679e-10\\
2.766	2.48292e-10\\
2.768	2.41585e-10\\
2.77	2.39083e-10\\
2.772	2.46814e-10\\
2.774	2.51703e-10\\
2.776	2.53181e-10\\
2.778	2.467e-10\\
2.78	2.48519e-10\\
2.782	2.41698e-10\\
2.784	2.3806e-10\\
2.786	2.40561e-10\\
2.788	2.43858e-10\\
2.79	2.42267e-10\\
2.792	2.39197e-10\\
2.794	2.32603e-10\\
2.796	2.30557e-10\\
2.798	2.23508e-10\\
2.8	2.21007e-10\\
2.802	2.18733e-10\\
2.804	2.19529e-10\\
2.806	2.14186e-10\\
2.808	2.0782e-10\\
2.81	2.12253e-10\\
2.812	2.14982e-10\\
2.814	2.0782e-10\\
2.816	2.07592e-10\\
2.818	2.07706e-10\\
2.82	2.00657e-10\\
2.822	2.07706e-10\\
2.824	2.07365e-10\\
2.826	2.07933e-10\\
2.828	2.07024e-10\\
2.83	2.1214e-10\\
2.832	2.11003e-10\\
2.834	2.05773e-10\\
2.836	2.05205e-10\\
2.838	2.08274e-10\\
2.84	2.05546e-10\\
2.842	2.03272e-10\\
2.844	2.02363e-10\\
2.846	1.98725e-10\\
2.848	2.03727e-10\\
2.85	2.05432e-10\\
2.852	1.98952e-10\\
2.854	2.01908e-10\\
2.856	2.01453e-10\\
2.858	1.9395e-10\\
2.86	1.89857e-10\\
2.862	1.86674e-10\\
2.864	1.8008e-10\\
2.866	1.85537e-10\\
2.868	1.80989e-10\\
2.87	1.86219e-10\\
2.872	1.91221e-10\\
2.874	1.87129e-10\\
2.876	1.87583e-10\\
2.878	1.81672e-10\\
2.88	1.7485e-10\\
2.882	1.74623e-10\\
2.884	1.7485e-10\\
2.886	1.78034e-10\\
2.888	1.82808e-10\\
2.89	1.88493e-10\\
2.892	1.91903e-10\\
2.894	1.96906e-10\\
2.896	1.98497e-10\\
2.898	2.05318e-10\\
2.9	2.06228e-10\\
2.902	2.12822e-10\\
2.904	2.1123e-10\\
2.906	2.11003e-10\\
2.908	2.05773e-10\\
2.91	2.1214e-10\\
2.912	2.11685e-10\\
2.914	2.17369e-10\\
2.916	2.13959e-10\\
2.918	2.1214e-10\\
2.92	2.04636e-10\\
2.922	2.10548e-10\\
2.924	2.07365e-10\\
2.926	2.00544e-10\\
2.928	1.9827e-10\\
2.93	1.99179e-10\\
2.932	1.98497e-10\\
2.934	1.95996e-10\\
2.936	1.88265e-10\\
2.938	1.84855e-10\\
2.94	1.92131e-10\\
2.942	1.96451e-10\\
2.944	1.91903e-10\\
2.946	1.92358e-10\\
2.948	1.87583e-10\\
2.95	1.89402e-10\\
2.952	1.94632e-10\\
2.954	1.95314e-10\\
2.956	1.99861e-10\\
2.958	1.95996e-10\\
2.96	2.0259e-10\\
2.962	2.06228e-10\\
2.964	2.03045e-10\\
2.966	2.04864e-10\\
2.968	2.01453e-10\\
2.97	2.05773e-10\\
2.972	2.04182e-10\\
2.974	2.03499e-10\\
2.976	1.98952e-10\\
2.978	1.97815e-10\\
2.98	1.97815e-10\\
2.982	2.05546e-10\\
2.984	2.11912e-10\\
2.986	2.12367e-10\\
2.988	2.16005e-10\\
2.99	2.21689e-10\\
2.992	2.20098e-10\\
2.994	2.27374e-10\\
2.996	2.3374e-10\\
2.998	2.34422e-10\\
3	2.27146e-10\\
3.002	2.28283e-10\\
3.004	2.26919e-10\\
3.006	2.24873e-10\\
3.008	2.26692e-10\\
3.01	2.29193e-10\\
3.012	2.22599e-10\\
3.014	2.24873e-10\\
3.016	2.20325e-10\\
3.018	2.26464e-10\\
3.02	2.21689e-10\\
3.022	2.29647e-10\\
3.024	2.23054e-10\\
3.026	2.19188e-10\\
3.028	2.21462e-10\\
3.03	2.16914e-10\\
3.032	2.23963e-10\\
3.034	2.26919e-10\\
3.036	2.23281e-10\\
3.038	2.18961e-10\\
3.04	2.16005e-10\\
3.042	2.13504e-10\\
3.044	2.12594e-10\\
3.046	2.20098e-10\\
3.048	2.17369e-10\\
3.05	2.15778e-10\\
3.052	2.1646e-10\\
3.054	2.1214e-10\\
3.056	2.18279e-10\\
3.058	2.23963e-10\\
3.06	2.30784e-10\\
3.062	2.28738e-10\\
3.064	2.34422e-10\\
3.066	2.28056e-10\\
3.068	2.34877e-10\\
3.07	2.42835e-10\\
3.072	2.46473e-10\\
3.074	2.43517e-10\\
3.076	2.38515e-10\\
3.078	2.36014e-10\\
3.08	2.41926e-10\\
3.082	2.47383e-10\\
3.084	2.42835e-10\\
3.086	2.43517e-10\\
3.088	2.46928e-10\\
3.09	2.53067e-10\\
3.092	2.45791e-10\\
3.094	2.46473e-10\\
3.096	2.43745e-10\\
3.098	2.48519e-10\\
3.1	2.41471e-10\\
3.102	2.48519e-10\\
3.104	2.41926e-10\\
3.106	2.44654e-10\\
3.108	2.45564e-10\\
3.11	2.52385e-10\\
3.112	2.50566e-10\\
3.114	2.46246e-10\\
3.116	2.47383e-10\\
3.118	2.51021e-10\\
3.12	2.54204e-10\\
3.122	2.6148e-10\\
3.124	2.58979e-10\\
3.126	2.65572e-10\\
3.128	2.68756e-10\\
3.13	2.66255e-10\\
3.132	2.68756e-10\\
3.134	2.63071e-10\\
3.136	2.61025e-10\\
3.138	2.57614e-10\\
3.14	2.53976e-10\\
3.142	2.55795e-10\\
3.144	2.48292e-10\\
3.146	2.54659e-10\\
3.148	2.61707e-10\\
3.15	2.53976e-10\\
3.152	2.50338e-10\\
3.154	2.44199e-10\\
3.156	2.40561e-10\\
3.158	2.47155e-10\\
3.16	2.40334e-10\\
3.162	2.33513e-10\\
3.164	2.39424e-10\\
3.166	2.39424e-10\\
3.168	2.41698e-10\\
3.17	2.49429e-10\\
3.172	2.5284e-10\\
3.174	2.53067e-10\\
3.176	2.53294e-10\\
3.178	2.49202e-10\\
3.18	2.46928e-10\\
3.182	2.40561e-10\\
3.184	2.44881e-10\\
3.186	2.52612e-10\\
3.188	2.56023e-10\\
3.19	2.53294e-10\\
3.192	2.48747e-10\\
3.194	2.48974e-10\\
3.196	2.52612e-10\\
3.198	2.54659e-10\\
3.2	2.57387e-10\\
3.202	2.51475e-10\\
3.204	2.47155e-10\\
3.206	2.48065e-10\\
3.208	2.41471e-10\\
3.21	2.41926e-10\\
3.212	2.40107e-10\\
3.214	2.3465e-10\\
3.216	2.39424e-10\\
3.218	2.3806e-10\\
3.22	2.44654e-10\\
3.222	2.48519e-10\\
3.224	2.49429e-10\\
3.226	2.51021e-10\\
3.228	2.54886e-10\\
3.23	2.61252e-10\\
3.232	2.59888e-10\\
3.234	2.53976e-10\\
3.236	2.55341e-10\\
3.238	2.60115e-10\\
3.24	2.65572e-10\\
3.242	2.62162e-10\\
3.244	2.65572e-10\\
3.246	2.67164e-10\\
3.248	2.66255e-10\\
3.25	2.66709e-10\\
3.252	2.65345e-10\\
3.254	2.64208e-10\\
3.256	2.5716e-10\\
3.258	2.63071e-10\\
3.26	2.65572e-10\\
3.262	2.6921e-10\\
3.264	2.68074e-10\\
3.266	2.6489e-10\\
3.268	2.66027e-10\\
3.27	2.67164e-10\\
3.272	2.72394e-10\\
3.274	2.78987e-10\\
3.276	2.75122e-10\\
3.278	2.70347e-10\\
3.28	2.74213e-10\\
3.282	2.77169e-10\\
3.284	2.76259e-10\\
3.286	2.72621e-10\\
3.288	2.76032e-10\\
3.29	2.78305e-10\\
3.292	2.74213e-10\\
3.294	2.80806e-10\\
3.296	2.7444e-10\\
3.298	2.76714e-10\\
3.3	2.78987e-10\\
3.302	2.7876e-10\\
3.304	2.76259e-10\\
3.306	2.70575e-10\\
3.308	2.75804e-10\\
3.31	2.83535e-10\\
3.312	2.75577e-10\\
3.314	2.75122e-10\\
3.316	2.70575e-10\\
3.318	2.68074e-10\\
3.32	2.63753e-10\\
3.322	2.62389e-10\\
3.324	2.60798e-10\\
3.326	2.57387e-10\\
3.328	2.57614e-10\\
3.33	2.55795e-10\\
3.332	2.58979e-10\\
3.334	2.51703e-10\\
3.336	2.48292e-10\\
3.338	2.55341e-10\\
3.34	2.56932e-10\\
3.342	2.60798e-10\\
3.344	2.64436e-10\\
3.346	2.59433e-10\\
3.348	2.62389e-10\\
3.35	2.67619e-10\\
3.352	2.61707e-10\\
3.354	2.60343e-10\\
3.356	2.63071e-10\\
3.358	2.62844e-10\\
3.36	2.63753e-10\\
3.362	2.62389e-10\\
3.364	2.55568e-10\\
3.366	2.53067e-10\\
3.368	2.59661e-10\\
3.37	2.59661e-10\\
3.372	2.66482e-10\\
3.374	2.65118e-10\\
3.376	2.70575e-10\\
3.378	2.76714e-10\\
3.38	2.76486e-10\\
3.382	2.82853e-10\\
3.384	2.84899e-10\\
3.386	2.81261e-10\\
3.388	2.80124e-10\\
3.39	2.76714e-10\\
3.392	2.68983e-10\\
3.394	2.71712e-10\\
3.396	2.63981e-10\\
3.398	2.58979e-10\\
3.4	2.66255e-10\\
3.402	2.64663e-10\\
3.404	2.62844e-10\\
3.406	2.65572e-10\\
3.408	2.71939e-10\\
3.41	2.75577e-10\\
3.412	2.79215e-10\\
3.414	2.74667e-10\\
3.416	2.69438e-10\\
3.418	2.76714e-10\\
3.42	2.71712e-10\\
3.422	2.65118e-10\\
3.424	2.68528e-10\\
3.426	2.7444e-10\\
3.428	2.72394e-10\\
3.43	2.77623e-10\\
3.432	2.73076e-10\\
3.434	2.66027e-10\\
3.436	2.73076e-10\\
3.438	2.74895e-10\\
3.44	2.6921e-10\\
3.442	2.71712e-10\\
3.444	2.75122e-10\\
3.446	2.72394e-10\\
3.448	2.77623e-10\\
3.45	2.83308e-10\\
3.452	2.89901e-10\\
3.454	2.89447e-10\\
3.456	2.9172e-10\\
3.458	2.86491e-10\\
3.46	2.79215e-10\\
3.462	2.82398e-10\\
3.464	2.80806e-10\\
3.466	2.79897e-10\\
3.468	2.87173e-10\\
3.47	2.87173e-10\\
3.472	2.84444e-10\\
3.474	2.89219e-10\\
3.476	2.9172e-10\\
3.478	2.95358e-10\\
3.48	2.99679e-10\\
3.482	3.0218e-10\\
3.484	3.0559e-10\\
3.486	3.04681e-10\\
3.488	3.06272e-10\\
3.49	3.11957e-10\\
3.492	3.11275e-10\\
3.494	3.03544e-10\\
3.496	3.06272e-10\\
3.498	3.04453e-10\\
3.5	2.98087e-10\\
3.502	3.01043e-10\\
3.504	2.98087e-10\\
3.506	3.00815e-10\\
3.508	3.09001e-10\\
3.51	3.04908e-10\\
3.512	3.07637e-10\\
3.514	3.06272e-10\\
3.516	3.01952e-10\\
3.518	3.0127e-10\\
3.52	2.97632e-10\\
3.522	3.02407e-10\\
3.524	3.03089e-10\\
3.526	2.99451e-10\\
3.528	3.00361e-10\\
3.53	3.01497e-10\\
3.532	3.03089e-10\\
3.534	3.09228e-10\\
3.536	3.05818e-10\\
3.538	3.08091e-10\\
3.54	3.08319e-10\\
3.542	3.04681e-10\\
3.544	3.07182e-10\\
3.546	3.03544e-10\\
3.548	3.06954e-10\\
3.55	3.12411e-10\\
3.552	3.16732e-10\\
3.554	3.09683e-10\\
3.556	3.14913e-10\\
3.558	3.1514e-10\\
3.56	3.1082e-10\\
3.562	3.13548e-10\\
3.564	3.13321e-10\\
3.566	3.1946e-10\\
3.568	3.11957e-10\\
3.57	3.15595e-10\\
3.572	3.13548e-10\\
3.574	3.12411e-10\\
3.576	3.19915e-10\\
3.578	3.18096e-10\\
3.58	3.20142e-10\\
3.582	3.22416e-10\\
3.584	3.25372e-10\\
3.586	3.29237e-10\\
3.588	3.25144e-10\\
3.59	3.24235e-10\\
3.592	3.2469e-10\\
3.594	3.18551e-10\\
3.596	3.16049e-10\\
3.598	3.18778e-10\\
3.6	3.12639e-10\\
3.602	3.07182e-10\\
3.604	2.99679e-10\\
3.606	3.05818e-10\\
3.608	3.01725e-10\\
3.61	2.94222e-10\\
3.612	2.91493e-10\\
3.614	2.9172e-10\\
3.616	2.8399e-10\\
3.618	2.87173e-10\\
3.62	2.87173e-10\\
3.622	2.79897e-10\\
3.624	2.76259e-10\\
3.626	2.71029e-10\\
3.628	2.70802e-10\\
3.63	2.71712e-10\\
3.632	2.75804e-10\\
3.634	2.81261e-10\\
3.636	2.75804e-10\\
3.638	2.83308e-10\\
3.64	2.90356e-10\\
3.642	2.86946e-10\\
3.644	2.83535e-10\\
3.646	2.84899e-10\\
3.648	2.85581e-10\\
3.65	2.89674e-10\\
3.652	2.83308e-10\\
3.654	2.82171e-10\\
3.656	2.86491e-10\\
3.658	2.88537e-10\\
3.66	2.88082e-10\\
3.662	2.92175e-10\\
3.664	2.95813e-10\\
3.666	3.03316e-10\\
3.668	3.04681e-10\\
3.67	3.02407e-10\\
3.672	2.95813e-10\\
3.674	2.98087e-10\\
3.676	2.94449e-10\\
3.678	2.94904e-10\\
3.68	2.93767e-10\\
3.682	2.96041e-10\\
3.684	2.90811e-10\\
3.686	2.93085e-10\\
3.688	2.86263e-10\\
3.69	2.91038e-10\\
3.692	2.85354e-10\\
3.694	2.86036e-10\\
3.696	2.93085e-10\\
3.698	2.96041e-10\\
3.7	2.88537e-10\\
3.702	2.93994e-10\\
3.704	3.01043e-10\\
3.706	3.08319e-10\\
3.708	3.02634e-10\\
3.71	2.98087e-10\\
3.712	2.96723e-10\\
3.714	3.03544e-10\\
3.716	3.06954e-10\\
3.718	3.00133e-10\\
3.72	3.07409e-10\\
3.722	3.14685e-10\\
3.724	3.1082e-10\\
3.726	3.03089e-10\\
3.728	3.07409e-10\\
3.73	3.09001e-10\\
3.732	3.04453e-10\\
3.734	3.11275e-10\\
3.736	3.15367e-10\\
3.738	3.13094e-10\\
3.74	3.15367e-10\\
3.742	3.20142e-10\\
3.744	3.24462e-10\\
3.746	3.30829e-10\\
3.748	3.34694e-10\\
3.75	3.42652e-10\\
3.752	3.40151e-10\\
3.754	3.40606e-10\\
3.756	3.44016e-10\\
3.758	3.44698e-10\\
3.76	3.39924e-10\\
3.762	3.40833e-10\\
3.764	3.46745e-10\\
3.766	3.53793e-10\\
3.768	3.54021e-10\\
3.77	3.53339e-10\\
3.772	3.57204e-10\\
3.774	3.59933e-10\\
3.776	3.65617e-10\\
3.778	3.72438e-10\\
3.78	3.74484e-10\\
3.782	3.71301e-10\\
3.784	3.76076e-10\\
3.786	3.81533e-10\\
3.788	3.74712e-10\\
3.79	3.74484e-10\\
3.792	3.71983e-10\\
3.794	3.77895e-10\\
3.796	3.71301e-10\\
3.798	3.69027e-10\\
3.8	3.64707e-10\\
3.802	3.57659e-10\\
3.804	3.64025e-10\\
3.806	3.57204e-10\\
3.808	3.51974e-10\\
3.81	3.53793e-10\\
3.812	3.61069e-10\\
3.814	3.68345e-10\\
3.816	3.71074e-10\\
3.818	3.74484e-10\\
3.82	3.688e-10\\
3.822	3.69255e-10\\
3.824	3.74712e-10\\
3.826	3.72893e-10\\
3.828	3.71756e-10\\
3.83	3.70846e-10\\
3.832	3.65844e-10\\
3.834	3.59705e-10\\
3.836	3.58796e-10\\
3.838	3.53793e-10\\
3.84	3.53566e-10\\
3.842	3.50383e-10\\
3.844	3.56067e-10\\
3.846	3.50383e-10\\
3.848	3.49019e-10\\
3.85	3.48791e-10\\
3.852	3.46063e-10\\
3.854	3.44016e-10\\
3.856	3.47086e-10\\
3.858	3.50497e-10\\
3.86	3.45267e-10\\
3.862	3.472e-10\\
3.864	3.51179e-10\\
3.866	3.52543e-10\\
3.868	3.56863e-10\\
3.87	3.63684e-10\\
3.872	3.69937e-10\\
3.874	3.62547e-10\\
3.876	3.59478e-10\\
3.878	3.66867e-10\\
3.88	3.66981e-10\\
3.882	3.59023e-10\\
3.884	3.61297e-10\\
3.886	3.65958e-10\\
3.888	3.67436e-10\\
3.89	3.61297e-10\\
3.892	3.67095e-10\\
3.894	3.63684e-10\\
3.896	3.71415e-10\\
3.898	3.70619e-10\\
3.9	3.68686e-10\\
3.902	3.68118e-10\\
3.904	3.69937e-10\\
3.906	3.62093e-10\\
3.908	3.63571e-10\\
3.91	3.65958e-10\\
3.912	3.71415e-10\\
3.914	3.65276e-10\\
3.916	3.58114e-10\\
3.918	3.58e-10\\
3.92	3.52316e-10\\
3.922	3.58e-10\\
3.924	3.59023e-10\\
3.926	3.6448e-10\\
3.928	3.67777e-10\\
3.93	3.69027e-10\\
3.932	3.65389e-10\\
3.934	3.68459e-10\\
3.936	3.69369e-10\\
3.938	3.76644e-10\\
3.94	3.72665e-10\\
3.942	3.78122e-10\\
3.944	3.75962e-10\\
3.946	3.77327e-10\\
3.948	3.82215e-10\\
3.95	3.88013e-10\\
3.952	3.95403e-10\\
3.954	3.99382e-10\\
3.956	3.94152e-10\\
3.958	3.93243e-10\\
3.96	3.9131e-10\\
3.962	3.987e-10\\
3.964	3.9222e-10\\
3.966	3.98472e-10\\
3.968	3.94152e-10\\
3.97	3.87786e-10\\
3.972	3.84489e-10\\
3.974	3.79373e-10\\
3.976	3.83125e-10\\
3.978	3.80851e-10\\
3.98	3.7835e-10\\
3.982	3.7187e-10\\
3.984	3.74598e-10\\
3.986	3.74371e-10\\
3.988	3.81988e-10\\
3.99	3.86649e-10\\
3.992	3.91083e-10\\
3.994	3.87217e-10\\
3.996	3.88468e-10\\
3.998	3.90287e-10\\
4	3.93243e-10\\
};
\addlegendentry{c1};

\addplot [color=mycolor5,solid]
  table[row sep=crcr]{%
0	0\\
0.002	0\\
0.004	0\\
0.006	9.09495e-13\\
0.008	9.09495e-13\\
0.01	0\\
0.012	0\\
0.014	0\\
0.016	-9.09495e-13\\
0.018	0\\
0.02	-9.09495e-13\\
0.022	-9.09495e-13\\
0.024	-9.09495e-13\\
0.026	-9.09495e-13\\
0.028	-9.09495e-13\\
0.03	0\\
0.032	0\\
0.034	0\\
0.036	0\\
0.038	9.09495e-13\\
0.04	9.09495e-13\\
0.042	9.09495e-13\\
0.044	9.09495e-13\\
0.046	0\\
0.048	0\\
0.05	0\\
0.052	0\\
0.054	0\\
0.056	0\\
0.058	0\\
0.06	0\\
0.062	0\\
0.064	0\\
0.066	0\\
0.068	0\\
0.07	-9.09495e-13\\
0.072	-9.09495e-13\\
0.074	-9.09495e-13\\
0.076	-9.09495e-13\\
0.078	-9.09495e-13\\
0.08	-1.81899e-12\\
0.082	-1.81899e-12\\
0.084	-1.81899e-12\\
0.086	-2.72848e-12\\
0.088	-1.81899e-12\\
0.09	-1.81899e-12\\
0.092	-1.81899e-12\\
0.094	-2.72848e-12\\
0.096	-2.72848e-12\\
0.098	-2.72848e-12\\
0.1	-3.63798e-12\\
0.102	-3.63798e-12\\
0.104	-3.63798e-12\\
0.106	-2.72848e-12\\
0.108	-2.72848e-12\\
0.11	-2.72848e-12\\
0.112	-1.81899e-12\\
0.114	-1.81899e-12\\
0.116	-2.72848e-12\\
0.118	-2.72848e-12\\
0.12	-2.72848e-12\\
0.122	-2.72848e-12\\
0.124	-3.63798e-12\\
0.126	-2.72848e-12\\
0.128	-1.81899e-12\\
0.13	-2.72848e-12\\
0.132	-2.72848e-12\\
0.134	-2.72848e-12\\
0.136	-2.72848e-12\\
0.138	-2.72848e-12\\
0.14	-2.72848e-12\\
0.142	-2.72848e-12\\
0.144	-2.72848e-12\\
0.146	-1.81899e-12\\
0.148	-2.72848e-12\\
0.15	-2.72848e-12\\
0.152	-2.72848e-12\\
0.154	-2.72848e-12\\
0.156	-1.81899e-12\\
0.158	-2.72848e-12\\
0.16	-2.72848e-12\\
0.162	-2.72848e-12\\
0.164	-2.72848e-12\\
0.166	-2.72848e-12\\
0.168	-2.72848e-12\\
0.17	-1.81899e-12\\
0.172	-1.81899e-12\\
0.174	-1.81899e-12\\
0.176	-2.72848e-12\\
0.178	-2.72848e-12\\
0.18	-2.72848e-12\\
0.182	-2.72848e-12\\
0.184	-2.72848e-12\\
0.186	-2.72848e-12\\
0.188	-2.72848e-12\\
0.19	-2.72848e-12\\
0.192	-2.72848e-12\\
0.194	-2.72848e-12\\
0.196	-2.72848e-12\\
0.198	-2.72848e-12\\
0.2	-2.72848e-12\\
0.202	-1.81899e-12\\
0.204	-2.72848e-12\\
0.206	-1.81899e-12\\
0.208	-1.81899e-12\\
0.21	-1.81899e-12\\
0.212	-1.81899e-12\\
0.214	-1.81899e-12\\
0.216	-1.81899e-12\\
0.218	-2.72848e-12\\
0.22	-2.72848e-12\\
0.222	-2.72848e-12\\
0.224	-1.81899e-12\\
0.226	-1.81899e-12\\
0.228	-2.72848e-12\\
0.23	-2.72848e-12\\
0.232	-2.72848e-12\\
0.234	-3.63798e-12\\
0.236	-2.72848e-12\\
0.238	-2.72848e-12\\
0.24	-1.81899e-12\\
0.242	-1.81899e-12\\
0.244	-1.81899e-12\\
0.246	-9.09495e-13\\
0.248	-9.09495e-13\\
0.25	-9.09495e-13\\
0.252	0\\
0.254	0\\
0.256	0\\
0.258	0\\
0.26	0\\
0.262	0\\
0.264	0\\
0.266	9.09495e-13\\
0.268	0\\
0.27	-9.09495e-13\\
0.272	0\\
0.274	0\\
0.276	-9.09495e-13\\
0.278	-9.09495e-13\\
0.28	-9.09495e-13\\
0.282	-9.09495e-13\\
0.284	-9.09495e-13\\
0.286	-9.09495e-13\\
0.288	0\\
0.29	-9.09495e-13\\
0.292	-9.09495e-13\\
0.294	0\\
0.296	-9.09495e-13\\
0.298	-9.09495e-13\\
0.3	0\\
0.302	-9.09495e-13\\
0.304	-9.09495e-13\\
0.306	-9.09495e-13\\
0.308	0\\
0.31	0\\
0.312	-9.09495e-13\\
0.314	0\\
0.316	0\\
0.318	0\\
0.32	0\\
0.322	0\\
0.324	0\\
0.326	0\\
0.328	0\\
0.33	0\\
0.332	0\\
0.334	0\\
0.336	0\\
0.338	0\\
0.34	0\\
0.342	0\\
0.344	9.09495e-13\\
0.346	0\\
0.348	0\\
0.35	0\\
0.352	0\\
0.354	0\\
0.356	9.09495e-13\\
0.358	1.81899e-12\\
0.36	1.81899e-12\\
0.362	1.81899e-12\\
0.364	1.81899e-12\\
0.366	1.81899e-12\\
0.368	1.81899e-12\\
0.37	2.72848e-12\\
0.372	2.72848e-12\\
0.374	2.72848e-12\\
0.376	2.72848e-12\\
0.378	2.72848e-12\\
0.38	2.72848e-12\\
0.382	1.81899e-12\\
0.384	2.72848e-12\\
0.386	2.72848e-12\\
0.388	2.72848e-12\\
0.39	2.72848e-12\\
0.392	2.72848e-12\\
0.394	1.81899e-12\\
0.396	1.81899e-12\\
0.398	1.81899e-12\\
0.4	1.81899e-12\\
0.402	2.72848e-12\\
0.404	2.72848e-12\\
0.406	2.72848e-12\\
0.408	2.72848e-12\\
0.41	2.72848e-12\\
0.412	2.72848e-12\\
0.414	1.81899e-12\\
0.416	1.81899e-12\\
0.418	2.72848e-12\\
0.42	2.72848e-12\\
0.422	2.72848e-12\\
0.424	3.63798e-12\\
0.426	3.63798e-12\\
0.428	3.63798e-12\\
0.43	4.54747e-12\\
0.432	4.54747e-12\\
0.434	4.54747e-12\\
0.436	4.54747e-12\\
0.438	4.54747e-12\\
0.44	4.54747e-12\\
0.442	4.54747e-12\\
0.444	4.54747e-12\\
0.446	4.54747e-12\\
0.448	5.45697e-12\\
0.45	5.45697e-12\\
0.452	5.45697e-12\\
0.454	5.45697e-12\\
0.456	5.45697e-12\\
0.458	5.45697e-12\\
0.46	5.45697e-12\\
0.462	5.45697e-12\\
0.464	5.45697e-12\\
0.466	5.45697e-12\\
0.468	6.36646e-12\\
0.47	5.45697e-12\\
0.472	5.45697e-12\\
0.474	5.45697e-12\\
0.476	6.36646e-12\\
0.478	5.45697e-12\\
0.48	5.45697e-12\\
0.482	5.45697e-12\\
0.484	4.54747e-12\\
0.486	4.54747e-12\\
0.488	4.54747e-12\\
0.49	4.54747e-12\\
0.492	3.63798e-12\\
0.494	3.63798e-12\\
0.496	3.63798e-12\\
0.498	2.72848e-12\\
0.5	2.72848e-12\\
0.502	2.72848e-12\\
0.504	1.81899e-12\\
0.506	1.81899e-12\\
0.508	9.09495e-13\\
0.51	1.81899e-12\\
0.512	9.09495e-13\\
0.514	9.09495e-13\\
0.516	1.81899e-12\\
0.518	1.81899e-12\\
0.52	1.81899e-12\\
0.522	1.81899e-12\\
0.524	2.72848e-12\\
0.526	2.72848e-12\\
0.528	2.72848e-12\\
0.53	3.63798e-12\\
0.532	3.63798e-12\\
0.534	3.63798e-12\\
0.536	4.54747e-12\\
0.538	5.45697e-12\\
0.54	5.45697e-12\\
0.542	5.45697e-12\\
0.544	5.45697e-12\\
0.546	6.36646e-12\\
0.548	6.36646e-12\\
0.55	5.45697e-12\\
0.552	5.45697e-12\\
0.554	5.45697e-12\\
0.556	5.45697e-12\\
0.558	5.45697e-12\\
0.56	6.36646e-12\\
0.562	5.45697e-12\\
0.564	5.45697e-12\\
0.566	5.45697e-12\\
0.568	5.45697e-12\\
0.57	5.45697e-12\\
0.572	6.36646e-12\\
0.574	6.36646e-12\\
0.576	7.27596e-12\\
0.578	6.36646e-12\\
0.58	6.36646e-12\\
0.582	6.36646e-12\\
0.584	5.45697e-12\\
0.586	5.45697e-12\\
0.588	5.45697e-12\\
0.59	5.45697e-12\\
0.592	5.45697e-12\\
0.594	5.45697e-12\\
0.596	5.45697e-12\\
0.598	5.45697e-12\\
0.6	5.45697e-12\\
0.602	5.45697e-12\\
0.604	5.45697e-12\\
0.606	5.45697e-12\\
0.608	4.54747e-12\\
0.61	4.54747e-12\\
0.612	5.45697e-12\\
0.614	5.45697e-12\\
0.616	4.54747e-12\\
0.618	4.54747e-12\\
0.62	5.45697e-12\\
0.622	5.45697e-12\\
0.624	5.45697e-12\\
0.626	4.54747e-12\\
0.628	3.63798e-12\\
0.63	3.63798e-12\\
0.632	3.63798e-12\\
0.634	2.72848e-12\\
0.636	3.63798e-12\\
0.638	2.72848e-12\\
0.64	2.72848e-12\\
0.642	2.72848e-12\\
0.644	1.81899e-12\\
0.646	1.81899e-12\\
0.648	1.81899e-12\\
0.65	1.81899e-12\\
0.652	1.81899e-12\\
0.654	1.81899e-12\\
0.656	2.72848e-12\\
0.658	1.81899e-12\\
0.66	9.09495e-13\\
0.662	2.72848e-12\\
0.664	1.81899e-12\\
0.666	2.72848e-12\\
0.668	3.63798e-12\\
0.67	3.63798e-12\\
0.672	2.72848e-12\\
0.674	3.63798e-12\\
0.676	2.72848e-12\\
0.678	2.72848e-12\\
0.68	1.81899e-12\\
0.682	1.81899e-12\\
0.684	1.81899e-12\\
0.686	1.81899e-12\\
0.688	1.81899e-12\\
0.69	1.81899e-12\\
0.692	2.72848e-12\\
0.694	2.72848e-12\\
0.696	2.72848e-12\\
0.698	1.81899e-12\\
0.7	2.72848e-12\\
0.702	3.63798e-12\\
0.704	3.63798e-12\\
0.706	4.54747e-12\\
0.708	4.54747e-12\\
0.71	3.63798e-12\\
0.712	4.54747e-12\\
0.714	3.63798e-12\\
0.716	4.54747e-12\\
0.718	4.54747e-12\\
0.72	4.54747e-12\\
0.722	4.54747e-12\\
0.724	4.54747e-12\\
0.726	4.54747e-12\\
0.728	4.54747e-12\\
0.73	3.63798e-12\\
0.732	5.45697e-12\\
0.734	5.45697e-12\\
0.736	4.54747e-12\\
0.738	4.54747e-12\\
0.74	4.54747e-12\\
0.742	3.63798e-12\\
0.744	2.72848e-12\\
0.746	1.81899e-12\\
0.748	1.81899e-12\\
0.75	1.81899e-12\\
0.752	1.81899e-12\\
0.754	1.81899e-12\\
0.756	9.09495e-13\\
0.758	1.81899e-12\\
0.76	9.09495e-13\\
0.762	1.81899e-12\\
0.764	9.09495e-13\\
0.766	1.81899e-12\\
0.768	2.72848e-12\\
0.77	2.72848e-12\\
0.772	2.72848e-12\\
0.774	1.81899e-12\\
0.776	9.09495e-13\\
0.778	9.09495e-13\\
0.78	9.09495e-13\\
0.782	0\\
0.784	0\\
0.786	9.09495e-13\\
0.788	0\\
0.79	0\\
0.792	-9.09495e-13\\
0.794	-1.81899e-12\\
0.796	-9.09495e-13\\
0.798	-9.09495e-13\\
0.8	0\\
0.802	0\\
0.804	9.09495e-13\\
0.806	1.81899e-12\\
0.808	9.09495e-13\\
0.81	0\\
0.812	-9.09495e-13\\
0.814	-9.09495e-13\\
0.816	0\\
0.818	0\\
0.82	-9.09495e-13\\
0.822	-9.09495e-13\\
0.824	-9.09495e-13\\
0.826	-1.81899e-12\\
0.828	-1.81899e-12\\
0.83	-1.81899e-12\\
0.832	-1.81899e-12\\
0.834	-2.72848e-12\\
0.836	-1.81899e-12\\
0.838	-9.09495e-13\\
0.84	0\\
0.842	1.81899e-12\\
0.844	1.81899e-12\\
0.846	2.72848e-12\\
0.848	1.81899e-12\\
0.85	3.63798e-12\\
0.852	2.72848e-12\\
0.854	1.81899e-12\\
0.856	2.72848e-12\\
0.858	3.63798e-12\\
0.86	4.54747e-12\\
0.862	4.54747e-12\\
0.864	4.54747e-12\\
0.866	4.54747e-12\\
0.868	3.63798e-12\\
0.87	2.72848e-12\\
0.872	2.72848e-12\\
0.874	2.72848e-12\\
0.876	3.63798e-12\\
0.878	3.63798e-12\\
0.88	3.63798e-12\\
0.882	2.72848e-12\\
0.884	3.63798e-12\\
0.886	3.63798e-12\\
0.888	2.72848e-12\\
0.89	9.09495e-13\\
0.892	9.09495e-13\\
0.894	0\\
0.896	9.09495e-13\\
0.898	1.81899e-12\\
0.9	9.09495e-13\\
0.902	0\\
0.904	0\\
0.906	0\\
0.908	9.09495e-13\\
0.91	9.09495e-13\\
0.912	0\\
0.914	0\\
0.916	-9.09495e-13\\
0.918	9.09495e-13\\
0.92	1.81899e-12\\
0.922	1.81899e-12\\
0.924	1.81899e-12\\
0.926	1.81899e-12\\
0.928	2.72848e-12\\
0.93	1.81899e-12\\
0.932	9.09495e-13\\
0.934	1.81899e-12\\
0.936	9.09495e-13\\
0.938	0\\
0.94	1.81899e-12\\
0.942	1.81899e-12\\
0.944	2.72848e-12\\
0.946	2.72848e-12\\
0.948	3.63798e-12\\
0.95	3.63798e-12\\
0.952	3.63798e-12\\
0.954	3.63798e-12\\
0.956	4.54747e-12\\
0.958	3.63798e-12\\
0.96	4.54747e-12\\
0.962	5.45697e-12\\
0.964	4.54747e-12\\
0.966	5.45697e-12\\
0.968	5.45697e-12\\
0.97	5.45697e-12\\
0.972	5.45697e-12\\
0.974	6.36646e-12\\
0.976	5.45697e-12\\
0.978	5.45697e-12\\
0.98	6.36646e-12\\
0.982	6.36646e-12\\
0.984	6.36646e-12\\
0.986	6.36646e-12\\
0.988	6.36646e-12\\
0.99	7.27596e-12\\
0.992	6.36646e-12\\
0.994	7.27596e-12\\
0.996	8.18545e-12\\
0.998	8.18545e-12\\
1	7.27596e-12\\
1.002	7.27596e-12\\
1.004	7.27596e-12\\
1.006	5.45697e-12\\
1.008	5.45697e-12\\
1.01	5.45697e-12\\
1.012	5.45697e-12\\
1.014	5.45697e-12\\
1.016	5.45697e-12\\
1.018	4.54747e-12\\
1.02	4.54747e-12\\
1.022	4.54747e-12\\
1.024	4.54747e-12\\
1.026	3.63798e-12\\
1.028	2.72848e-12\\
1.03	2.72848e-12\\
1.032	3.63798e-12\\
1.034	3.63798e-12\\
1.036	4.54747e-12\\
1.038	5.45697e-12\\
1.04	5.45697e-12\\
1.042	6.36646e-12\\
1.044	6.36646e-12\\
1.046	7.27596e-12\\
1.048	6.36646e-12\\
1.05	7.27596e-12\\
1.052	7.27596e-12\\
1.054	6.36646e-12\\
1.056	5.45697e-12\\
1.058	6.36646e-12\\
1.06	6.36646e-12\\
1.062	5.45697e-12\\
1.064	6.36646e-12\\
1.066	6.36646e-12\\
1.068	6.36646e-12\\
1.07	7.27596e-12\\
1.072	7.27596e-12\\
1.074	8.18545e-12\\
1.076	8.18545e-12\\
1.078	1.00044e-11\\
1.08	9.09495e-12\\
1.082	8.18545e-12\\
1.084	8.18545e-12\\
1.086	8.18545e-12\\
1.088	8.18545e-12\\
1.09	7.27596e-12\\
1.092	8.18545e-12\\
1.094	9.09495e-12\\
1.096	9.09495e-12\\
1.098	9.09495e-12\\
1.1	8.18545e-12\\
1.102	8.18545e-12\\
1.104	7.27596e-12\\
1.106	7.27596e-12\\
1.108	7.27596e-12\\
1.11	7.27596e-12\\
1.112	7.27596e-12\\
1.114	6.36646e-12\\
1.116	6.36646e-12\\
1.118	6.36646e-12\\
1.12	5.45697e-12\\
1.122	6.36646e-12\\
1.124	5.45697e-12\\
1.126	5.45697e-12\\
1.128	4.54747e-12\\
1.13	5.45697e-12\\
1.132	5.45697e-12\\
1.134	5.45697e-12\\
1.136	5.45697e-12\\
1.138	4.54747e-12\\
1.14	4.54747e-12\\
1.142	3.63798e-12\\
1.144	3.63798e-12\\
1.146	2.72848e-12\\
1.148	1.81899e-12\\
1.15	1.81899e-12\\
1.152	1.81899e-12\\
1.154	1.81899e-12\\
1.156	2.72848e-12\\
1.158	3.63798e-12\\
1.16	3.63798e-12\\
1.162	2.72848e-12\\
1.164	3.63798e-12\\
1.166	2.72848e-12\\
1.168	2.72848e-12\\
1.17	2.72848e-12\\
1.172	1.81899e-12\\
1.174	1.81899e-12\\
1.176	9.09495e-13\\
1.178	1.81899e-12\\
1.18	9.09495e-13\\
1.182	1.81899e-12\\
1.184	9.09495e-13\\
1.186	0\\
1.188	0\\
1.19	0\\
1.192	9.09495e-13\\
1.194	9.09495e-13\\
1.196	1.81899e-12\\
1.198	1.81899e-12\\
1.2	1.81899e-12\\
1.202	1.81899e-12\\
1.204	1.81899e-12\\
1.206	1.81899e-12\\
1.208	1.81899e-12\\
1.21	1.81899e-12\\
1.212	9.09495e-13\\
1.214	9.09495e-13\\
1.216	0\\
1.218	0\\
1.22	9.09495e-13\\
1.222	9.09495e-13\\
1.224	9.09495e-13\\
1.226	9.09495e-13\\
1.228	0\\
1.23	9.09495e-13\\
1.232	9.09495e-13\\
1.234	9.09495e-13\\
1.236	9.09495e-13\\
1.238	1.81899e-12\\
1.24	2.72848e-12\\
1.242	2.72848e-12\\
1.244	3.63798e-12\\
1.246	2.72848e-12\\
1.248	2.72848e-12\\
1.25	2.72848e-12\\
1.252	2.72848e-12\\
1.254	3.63798e-12\\
1.256	2.72848e-12\\
1.258	3.63798e-12\\
1.26	3.63798e-12\\
1.262	4.54747e-12\\
1.264	4.54747e-12\\
1.266	5.45697e-12\\
1.268	5.45697e-12\\
1.27	5.45697e-12\\
1.272	5.45697e-12\\
1.274	5.45697e-12\\
1.276	5.45697e-12\\
1.278	5.45697e-12\\
1.28	5.45697e-12\\
1.282	5.45697e-12\\
1.284	5.45697e-12\\
1.286	5.45697e-12\\
1.288	5.45697e-12\\
1.29	5.45697e-12\\
1.292	5.45697e-12\\
1.294	5.45697e-12\\
1.296	6.36646e-12\\
1.298	6.36646e-12\\
1.3	7.27596e-12\\
1.302	7.27596e-12\\
1.304	7.27596e-12\\
1.306	7.27596e-12\\
1.308	7.27596e-12\\
1.31	7.27596e-12\\
1.312	7.27596e-12\\
1.314	7.27596e-12\\
1.316	7.27596e-12\\
1.318	6.36646e-12\\
1.32	7.27596e-12\\
1.322	7.27596e-12\\
1.324	6.36646e-12\\
1.326	7.27596e-12\\
1.328	7.27596e-12\\
1.33	7.27596e-12\\
1.332	6.36646e-12\\
1.334	7.27596e-12\\
1.336	6.36646e-12\\
1.338	6.36646e-12\\
1.34	6.36646e-12\\
1.342	6.36646e-12\\
1.344	6.36646e-12\\
1.346	6.36646e-12\\
1.348	6.36646e-12\\
1.35	6.36646e-12\\
1.352	6.36646e-12\\
1.354	7.27596e-12\\
1.356	7.27596e-12\\
1.358	6.36646e-12\\
1.36	7.27596e-12\\
1.362	7.27596e-12\\
1.364	7.27596e-12\\
1.366	6.36646e-12\\
1.368	6.36646e-12\\
1.37	6.36646e-12\\
1.372	6.36646e-12\\
1.374	5.45697e-12\\
1.376	6.36646e-12\\
1.378	6.36646e-12\\
1.38	7.27596e-12\\
1.382	7.27596e-12\\
1.384	8.18545e-12\\
1.386	8.18545e-12\\
1.388	8.18545e-12\\
1.39	8.18545e-12\\
1.392	8.18545e-12\\
1.394	8.18545e-12\\
1.396	9.09495e-12\\
1.398	9.09495e-12\\
1.4	9.09495e-12\\
1.402	9.09495e-12\\
1.404	9.09495e-12\\
1.406	9.09495e-12\\
1.408	1.00044e-11\\
1.41	9.09495e-12\\
1.412	9.09495e-12\\
1.414	9.09495e-12\\
1.416	9.09495e-12\\
1.418	9.09495e-12\\
1.42	9.09495e-12\\
1.422	9.09495e-12\\
1.424	9.09495e-12\\
1.426	1.00044e-11\\
1.428	9.09495e-12\\
1.43	9.09495e-12\\
1.432	9.09495e-12\\
1.434	9.09495e-12\\
1.436	9.09495e-12\\
1.438	9.09495e-12\\
1.44	9.09495e-12\\
1.442	9.09495e-12\\
1.444	8.18545e-12\\
1.446	8.18545e-12\\
1.448	8.18545e-12\\
1.45	8.18545e-12\\
1.452	9.09495e-12\\
1.454	9.09495e-12\\
1.456	9.09495e-12\\
1.458	9.09495e-12\\
1.46	8.18545e-12\\
1.462	9.09495e-12\\
1.464	9.09495e-12\\
1.466	9.09495e-12\\
1.468	9.09495e-12\\
1.47	9.09495e-12\\
1.472	1.00044e-11\\
1.474	9.09495e-12\\
1.476	1.00044e-11\\
1.478	9.09495e-12\\
1.48	9.09495e-12\\
1.482	9.09495e-12\\
1.484	1.00044e-11\\
1.486	1.00044e-11\\
1.488	1.00044e-11\\
1.49	1.00044e-11\\
1.492	1.00044e-11\\
1.494	9.09495e-12\\
1.496	9.09495e-12\\
1.498	9.09495e-12\\
1.5	9.09495e-12\\
1.502	9.09495e-12\\
1.504	9.09495e-12\\
1.506	1.00044e-11\\
1.508	1.00044e-11\\
1.51	1.00044e-11\\
1.512	9.09495e-12\\
1.514	9.09495e-12\\
1.516	1.00044e-11\\
1.518	1.00044e-11\\
1.52	1.00044e-11\\
1.522	1.00044e-11\\
1.524	1.00044e-11\\
1.526	1.00044e-11\\
1.528	9.09495e-12\\
1.53	9.09495e-12\\
1.532	9.09495e-12\\
1.534	9.09495e-12\\
1.536	1.00044e-11\\
1.538	9.09495e-12\\
1.54	1.00044e-11\\
1.542	1.00044e-11\\
1.544	9.09495e-12\\
1.546	9.09495e-12\\
1.548	9.09495e-12\\
1.55	9.09495e-12\\
1.552	1.00044e-11\\
1.554	9.09495e-12\\
1.556	1.00044e-11\\
1.558	9.09495e-12\\
1.56	9.09495e-12\\
1.562	9.09495e-12\\
1.564	9.09495e-12\\
1.566	8.18545e-12\\
1.568	8.18545e-12\\
1.57	8.18545e-12\\
1.572	9.09495e-12\\
1.574	9.09495e-12\\
1.576	9.09495e-12\\
1.578	9.09495e-12\\
1.58	9.09495e-12\\
1.582	9.09495e-12\\
1.584	9.09495e-12\\
1.586	9.09495e-12\\
1.588	9.09495e-12\\
1.59	1.00044e-11\\
1.592	9.09495e-12\\
1.594	9.09495e-12\\
1.596	9.09495e-12\\
1.598	9.09495e-12\\
1.6	9.09495e-12\\
1.602	9.09495e-12\\
1.604	9.09495e-12\\
1.606	9.09495e-12\\
1.608	9.09495e-12\\
1.61	9.09495e-12\\
1.612	9.09495e-12\\
1.614	9.09495e-12\\
1.616	9.09495e-12\\
1.618	9.09495e-12\\
1.62	9.09495e-12\\
1.622	1.00044e-11\\
1.624	1.00044e-11\\
1.626	1.00044e-11\\
1.628	1.00044e-11\\
1.63	1.00044e-11\\
1.632	1.00044e-11\\
1.634	1.00044e-11\\
1.636	1.00044e-11\\
1.638	1.00044e-11\\
1.64	1.00044e-11\\
1.642	1.00044e-11\\
1.644	1.00044e-11\\
1.646	1.00044e-11\\
1.648	1.00044e-11\\
1.65	1.00044e-11\\
1.652	1.00044e-11\\
1.654	1.09139e-11\\
1.656	1.09139e-11\\
1.658	1.09139e-11\\
1.66	1.09139e-11\\
1.662	1.09139e-11\\
1.664	1.09139e-11\\
1.666	1.00044e-11\\
1.668	1.00044e-11\\
1.67	1.09139e-11\\
1.672	1.09139e-11\\
1.674	1.09139e-11\\
1.676	1.09139e-11\\
1.678	1.09139e-11\\
1.68	1.09139e-11\\
1.682	1.09139e-11\\
1.684	1.00044e-11\\
1.686	1.00044e-11\\
1.688	1.00044e-11\\
1.69	1.00044e-11\\
1.692	1.09139e-11\\
1.694	1.09139e-11\\
1.696	1.00044e-11\\
1.698	1.00044e-11\\
1.7	1.09139e-11\\
1.702	1.00044e-11\\
1.704	1.00044e-11\\
1.706	1.00044e-11\\
1.708	1.00044e-11\\
1.71	1.00044e-11\\
1.712	1.09139e-11\\
1.714	1.00044e-11\\
1.716	1.00044e-11\\
1.718	1.00044e-11\\
1.72	1.00044e-11\\
1.722	1.00044e-11\\
1.724	1.09139e-11\\
1.726	1.00044e-11\\
1.728	1.00044e-11\\
1.73	1.00044e-11\\
1.732	1.00044e-11\\
1.734	1.00044e-11\\
1.736	1.00044e-11\\
1.738	1.00044e-11\\
1.74	9.09495e-12\\
1.742	9.09495e-12\\
1.744	9.09495e-12\\
1.746	9.09495e-12\\
1.748	9.09495e-12\\
1.75	9.09495e-12\\
1.752	8.18545e-12\\
1.754	8.18545e-12\\
1.756	8.18545e-12\\
1.758	8.18545e-12\\
1.76	8.18545e-12\\
1.762	7.27596e-12\\
1.764	7.27596e-12\\
1.766	7.27596e-12\\
1.768	8.18545e-12\\
1.77	8.18545e-12\\
1.772	8.18545e-12\\
1.774	8.18545e-12\\
1.776	8.18545e-12\\
1.778	8.18545e-12\\
1.78	7.27596e-12\\
1.782	8.18545e-12\\
1.784	7.27596e-12\\
1.786	7.27596e-12\\
1.788	8.18545e-12\\
1.79	8.18545e-12\\
1.792	7.27596e-12\\
1.794	8.18545e-12\\
1.796	8.18545e-12\\
1.798	8.18545e-12\\
1.8	8.18545e-12\\
1.802	8.18545e-12\\
1.804	8.18545e-12\\
1.806	8.18545e-12\\
1.808	8.18545e-12\\
1.81	8.18545e-12\\
1.812	8.18545e-12\\
1.814	8.18545e-12\\
1.816	8.18545e-12\\
1.818	8.18545e-12\\
1.82	7.27596e-12\\
1.822	8.18545e-12\\
1.824	8.18545e-12\\
1.826	8.18545e-12\\
1.828	7.27596e-12\\
1.83	8.18545e-12\\
1.832	8.18545e-12\\
1.834	7.27596e-12\\
1.836	8.18545e-12\\
1.838	8.18545e-12\\
1.84	8.18545e-12\\
1.842	8.18545e-12\\
1.844	8.18545e-12\\
1.846	8.18545e-12\\
1.848	8.18545e-12\\
1.85	8.18545e-12\\
1.852	8.18545e-12\\
1.854	8.18545e-12\\
1.856	8.18545e-12\\
1.858	8.18545e-12\\
1.86	7.27596e-12\\
1.862	8.18545e-12\\
1.864	8.18545e-12\\
1.866	7.27596e-12\\
1.868	8.18545e-12\\
1.87	8.18545e-12\\
1.872	8.18545e-12\\
1.874	8.18545e-12\\
1.876	8.18545e-12\\
1.878	8.18545e-12\\
1.88	8.18545e-12\\
1.882	8.18545e-12\\
1.884	7.27596e-12\\
1.886	7.27596e-12\\
1.888	7.27596e-12\\
1.89	6.36646e-12\\
1.892	6.36646e-12\\
1.894	7.27596e-12\\
1.896	7.27596e-12\\
1.898	7.27596e-12\\
1.9	7.27596e-12\\
1.902	7.27596e-12\\
1.904	8.18545e-12\\
1.906	8.18545e-12\\
1.908	7.27596e-12\\
1.91	7.27596e-12\\
1.912	7.27596e-12\\
1.914	8.18545e-12\\
1.916	8.18545e-12\\
1.918	8.18545e-12\\
1.92	8.18545e-12\\
1.922	8.18545e-12\\
1.924	8.18545e-12\\
1.926	8.18545e-12\\
1.928	8.18545e-12\\
1.93	8.18545e-12\\
1.932	8.18545e-12\\
1.934	8.18545e-12\\
1.936	8.18545e-12\\
1.938	8.18545e-12\\
1.94	8.18545e-12\\
1.942	8.18545e-12\\
1.944	8.18545e-12\\
1.946	8.18545e-12\\
1.948	9.09495e-12\\
1.95	9.09495e-12\\
1.952	9.09495e-12\\
1.954	9.09495e-12\\
1.956	9.09495e-12\\
1.958	9.09495e-12\\
1.96	9.09495e-12\\
1.962	8.18545e-12\\
1.964	8.18545e-12\\
1.966	8.18545e-12\\
1.968	7.27596e-12\\
1.97	8.18545e-12\\
1.972	8.18545e-12\\
1.974	7.27596e-12\\
1.976	7.27596e-12\\
1.978	7.27596e-12\\
1.98	7.27596e-12\\
1.982	7.27596e-12\\
1.984	8.18545e-12\\
1.986	8.18545e-12\\
1.988	8.18545e-12\\
1.99	8.18545e-12\\
1.992	8.18545e-12\\
1.994	9.09495e-12\\
1.996	9.09495e-12\\
1.998	9.09495e-12\\
2	1.00044e-11\\
2.002	1.09139e-11\\
2.004	1.09139e-11\\
2.006	1.09139e-11\\
2.008	1.00044e-11\\
2.01	1.09139e-11\\
2.012	1.00044e-11\\
2.014	1.00044e-11\\
2.016	1.00044e-11\\
2.018	1.00044e-11\\
2.02	1.00044e-11\\
2.022	9.09495e-12\\
2.024	9.09495e-12\\
2.026	1.00044e-11\\
2.028	1.00044e-11\\
2.03	1.09139e-11\\
2.032	1.00044e-11\\
2.034	1.00044e-11\\
2.036	1.00044e-11\\
2.038	1.09139e-11\\
2.04	1.09139e-11\\
2.042	1.09139e-11\\
2.044	1.18234e-11\\
2.046	1.18234e-11\\
2.048	1.18234e-11\\
2.05	1.18234e-11\\
2.052	1.18234e-11\\
2.054	1.27329e-11\\
2.056	1.27329e-11\\
2.058	1.27329e-11\\
2.06	1.27329e-11\\
2.062	1.27329e-11\\
2.064	1.27329e-11\\
2.066	1.27329e-11\\
2.068	1.27329e-11\\
2.07	1.36424e-11\\
2.072	1.36424e-11\\
2.074	1.36424e-11\\
2.076	1.45519e-11\\
2.078	1.45519e-11\\
2.08	1.36424e-11\\
2.082	1.36424e-11\\
2.084	1.45519e-11\\
2.086	1.45519e-11\\
2.088	1.54614e-11\\
2.09	1.45519e-11\\
2.092	1.45519e-11\\
2.094	1.45519e-11\\
2.096	1.54614e-11\\
2.098	1.63709e-11\\
2.1	1.72804e-11\\
2.102	1.72804e-11\\
2.104	1.72804e-11\\
2.106	1.72804e-11\\
2.108	1.63709e-11\\
2.11	1.63709e-11\\
2.112	1.72804e-11\\
2.114	1.72804e-11\\
2.116	1.72804e-11\\
2.118	1.63709e-11\\
2.12	1.72804e-11\\
2.122	1.72804e-11\\
2.124	1.72804e-11\\
2.126	1.72804e-11\\
2.128	1.72804e-11\\
2.13	1.72804e-11\\
2.132	1.72804e-11\\
2.134	1.81899e-11\\
2.136	1.81899e-11\\
2.138	1.81899e-11\\
2.14	1.90994e-11\\
2.142	1.90994e-11\\
2.144	1.81899e-11\\
2.146	1.72804e-11\\
2.148	1.81899e-11\\
2.15	1.81899e-11\\
2.152	1.90994e-11\\
2.154	1.90994e-11\\
2.156	2.00089e-11\\
2.158	2.00089e-11\\
2.16	2.00089e-11\\
2.162	2.00089e-11\\
2.164	2.00089e-11\\
2.166	2.00089e-11\\
2.168	2.00089e-11\\
2.17	2.00089e-11\\
2.172	2.00089e-11\\
2.174	2.00089e-11\\
2.176	2.00089e-11\\
2.178	2.09184e-11\\
2.18	2.09184e-11\\
2.182	2.18279e-11\\
2.184	2.18279e-11\\
2.186	2.27374e-11\\
2.188	2.27374e-11\\
2.19	2.27374e-11\\
2.192	2.18279e-11\\
2.194	2.18279e-11\\
2.196	2.18279e-11\\
2.198	2.27374e-11\\
2.2	2.18279e-11\\
2.202	2.27374e-11\\
2.204	2.36469e-11\\
2.206	2.36469e-11\\
2.208	2.45564e-11\\
2.21	2.45564e-11\\
2.212	2.45564e-11\\
2.214	2.45564e-11\\
2.216	2.36469e-11\\
2.218	2.27374e-11\\
2.22	2.27374e-11\\
2.222	2.36469e-11\\
2.224	2.36469e-11\\
2.226	2.45564e-11\\
2.228	2.36469e-11\\
2.23	2.36469e-11\\
2.232	2.27374e-11\\
2.234	2.27374e-11\\
2.236	2.27374e-11\\
2.238	2.36469e-11\\
2.24	2.36469e-11\\
2.242	2.36469e-11\\
2.244	2.45564e-11\\
2.246	2.54659e-11\\
2.248	2.54659e-11\\
2.25	2.45564e-11\\
2.252	2.45564e-11\\
2.254	2.45564e-11\\
2.256	2.45564e-11\\
2.258	2.36469e-11\\
2.26	2.45564e-11\\
2.262	2.45564e-11\\
2.264	2.54659e-11\\
2.266	2.54659e-11\\
2.268	2.54659e-11\\
2.27	2.63753e-11\\
2.272	2.63753e-11\\
2.274	2.72848e-11\\
2.276	2.72848e-11\\
2.278	2.81943e-11\\
2.28	2.91038e-11\\
2.282	2.81943e-11\\
2.284	2.81943e-11\\
2.286	2.81943e-11\\
2.288	2.81943e-11\\
2.29	2.72848e-11\\
2.292	2.81943e-11\\
2.294	2.91038e-11\\
2.296	2.81943e-11\\
2.298	2.81943e-11\\
2.3	2.91038e-11\\
2.302	2.81943e-11\\
2.304	2.81943e-11\\
2.306	2.72848e-11\\
2.308	2.81943e-11\\
2.31	2.81943e-11\\
2.312	2.81943e-11\\
2.314	2.91038e-11\\
2.316	2.91038e-11\\
2.318	2.91038e-11\\
2.32	2.81943e-11\\
2.322	2.81943e-11\\
2.324	2.81943e-11\\
2.326	2.81943e-11\\
2.328	2.81943e-11\\
2.33	2.81943e-11\\
2.332	2.81943e-11\\
2.334	2.72848e-11\\
2.336	2.72848e-11\\
2.338	2.91038e-11\\
2.34	2.91038e-11\\
2.342	2.81943e-11\\
2.344	2.72848e-11\\
2.346	2.72848e-11\\
2.348	2.72848e-11\\
2.35	2.72848e-11\\
2.352	2.81943e-11\\
2.354	2.72848e-11\\
2.356	2.72848e-11\\
2.358	2.72848e-11\\
2.36	2.81943e-11\\
2.362	2.81943e-11\\
2.364	2.72848e-11\\
2.366	2.72848e-11\\
2.368	2.63753e-11\\
2.37	2.63753e-11\\
2.372	2.72848e-11\\
2.374	2.72848e-11\\
2.376	2.81943e-11\\
2.378	2.81943e-11\\
2.38	2.72848e-11\\
2.382	2.72848e-11\\
2.384	2.63753e-11\\
2.386	2.72848e-11\\
2.388	2.72848e-11\\
2.39	2.91038e-11\\
2.392	2.91038e-11\\
2.394	3.00133e-11\\
2.396	3.09228e-11\\
2.398	3.09228e-11\\
2.4	3.09228e-11\\
2.402	3.18323e-11\\
2.404	3.27418e-11\\
2.406	3.18323e-11\\
2.408	3.18323e-11\\
2.41	3.18323e-11\\
2.412	3.27418e-11\\
2.414	3.36513e-11\\
2.416	3.36513e-11\\
2.418	3.36513e-11\\
2.42	3.45608e-11\\
2.422	3.36513e-11\\
2.424	3.27418e-11\\
2.426	3.45608e-11\\
2.428	3.36513e-11\\
2.43	3.36513e-11\\
2.432	3.27418e-11\\
2.434	3.36513e-11\\
2.436	3.36513e-11\\
2.438	3.45608e-11\\
2.44	3.45608e-11\\
2.442	3.54703e-11\\
2.444	3.45608e-11\\
2.446	3.36513e-11\\
2.448	3.36513e-11\\
2.45	3.45608e-11\\
2.452	3.45608e-11\\
2.454	3.45608e-11\\
2.456	3.45608e-11\\
2.458	3.36513e-11\\
2.46	3.36513e-11\\
2.462	3.36513e-11\\
2.464	3.45608e-11\\
2.466	3.45608e-11\\
2.468	3.45608e-11\\
2.47	3.54703e-11\\
2.472	3.54703e-11\\
2.474	3.54703e-11\\
2.476	3.54703e-11\\
2.478	3.63798e-11\\
2.48	3.72893e-11\\
2.482	3.81988e-11\\
2.484	3.72893e-11\\
2.486	3.81988e-11\\
2.488	3.72893e-11\\
2.49	3.72893e-11\\
2.492	3.72893e-11\\
2.494	3.63798e-11\\
2.496	3.54703e-11\\
2.498	3.54703e-11\\
2.5	3.63798e-11\\
2.502	3.63798e-11\\
2.504	3.54703e-11\\
2.506	3.54703e-11\\
2.508	3.54703e-11\\
2.51	3.54703e-11\\
2.512	3.54703e-11\\
2.514	3.54703e-11\\
2.516	3.54703e-11\\
2.518	3.54703e-11\\
2.52	3.54703e-11\\
2.522	3.54703e-11\\
2.524	3.54703e-11\\
2.526	3.54703e-11\\
2.528	3.54703e-11\\
2.53	3.54703e-11\\
2.532	3.45608e-11\\
2.534	3.54703e-11\\
2.536	3.54703e-11\\
2.538	3.54703e-11\\
2.54	3.45608e-11\\
2.542	3.54703e-11\\
2.544	3.54703e-11\\
2.546	3.45608e-11\\
2.548	3.54703e-11\\
2.55	3.45608e-11\\
2.552	3.36513e-11\\
2.554	3.36513e-11\\
2.556	3.27418e-11\\
2.558	3.27418e-11\\
2.56	3.27418e-11\\
2.562	3.45608e-11\\
2.564	3.54703e-11\\
2.566	3.54703e-11\\
2.568	3.54703e-11\\
2.57	3.54703e-11\\
2.572	3.54703e-11\\
2.574	3.54703e-11\\
2.576	3.63798e-11\\
2.578	3.63798e-11\\
2.58	3.63798e-11\\
2.582	3.63798e-11\\
2.584	3.72893e-11\\
2.586	3.72893e-11\\
2.588	3.72893e-11\\
2.59	3.72893e-11\\
2.592	3.72893e-11\\
2.594	3.72893e-11\\
2.596	3.72893e-11\\
2.598	3.81988e-11\\
2.6	3.81988e-11\\
2.602	4.00178e-11\\
2.604	4.09273e-11\\
2.606	4.18368e-11\\
2.608	4.09273e-11\\
2.61	4.00178e-11\\
2.612	4.09273e-11\\
2.614	4.00178e-11\\
2.616	4.09273e-11\\
2.618	4.00178e-11\\
2.62	4.00178e-11\\
2.622	4.00178e-11\\
2.624	3.91083e-11\\
2.626	4.00178e-11\\
2.628	4.09273e-11\\
2.63	4.09273e-11\\
2.632	4.00178e-11\\
2.634	4.00178e-11\\
2.636	4.00178e-11\\
2.638	4.00178e-11\\
2.64	3.91083e-11\\
2.642	3.72893e-11\\
2.644	3.81988e-11\\
2.646	3.81988e-11\\
2.648	3.91083e-11\\
2.65	3.91083e-11\\
2.652	3.81988e-11\\
2.654	3.81988e-11\\
2.656	3.81988e-11\\
2.658	3.72893e-11\\
2.66	3.72893e-11\\
2.662	3.72893e-11\\
2.664	3.54703e-11\\
2.666	3.63798e-11\\
2.668	3.54703e-11\\
2.67	3.54703e-11\\
2.672	3.54703e-11\\
2.674	3.54703e-11\\
2.676	3.63798e-11\\
2.678	3.54703e-11\\
2.68	3.63798e-11\\
2.682	3.54703e-11\\
2.684	3.63798e-11\\
2.686	3.63798e-11\\
2.688	3.63798e-11\\
2.69	3.72893e-11\\
2.692	3.72893e-11\\
2.694	3.81988e-11\\
2.696	3.81988e-11\\
2.698	3.91083e-11\\
2.7	3.81988e-11\\
2.702	3.91083e-11\\
2.704	3.91083e-11\\
2.706	3.91083e-11\\
2.708	4.00178e-11\\
2.71	4.09273e-11\\
2.712	4.00178e-11\\
2.714	4.00178e-11\\
2.716	4.09273e-11\\
2.718	4.18368e-11\\
2.72	4.27463e-11\\
2.722	4.18368e-11\\
2.724	4.18368e-11\\
2.726	4.18368e-11\\
2.728	4.18368e-11\\
2.73	4.18368e-11\\
2.732	4.27463e-11\\
2.734	4.27463e-11\\
2.736	4.18368e-11\\
2.738	4.18368e-11\\
2.74	4.09273e-11\\
2.742	4.18368e-11\\
2.744	4.18368e-11\\
2.746	4.09273e-11\\
2.748	4.09273e-11\\
2.75	4.18368e-11\\
2.752	4.18368e-11\\
2.754	4.09273e-11\\
2.756	4.00178e-11\\
2.758	4.00178e-11\\
2.76	3.91083e-11\\
2.762	4.00178e-11\\
2.764	3.91083e-11\\
2.766	3.91083e-11\\
2.768	3.72893e-11\\
2.77	3.72893e-11\\
2.772	3.81988e-11\\
2.774	3.91083e-11\\
2.776	3.91083e-11\\
2.778	3.91083e-11\\
2.78	3.91083e-11\\
2.782	3.81988e-11\\
2.784	3.72893e-11\\
2.786	3.81988e-11\\
2.788	3.81988e-11\\
2.79	3.72893e-11\\
2.792	3.72893e-11\\
2.794	3.63798e-11\\
2.796	3.63798e-11\\
2.798	3.54703e-11\\
2.8	3.54703e-11\\
2.802	3.54703e-11\\
2.804	3.54703e-11\\
2.806	3.54703e-11\\
2.808	3.45608e-11\\
2.81	3.45608e-11\\
2.812	3.45608e-11\\
2.814	3.36513e-11\\
2.816	3.36513e-11\\
2.818	3.36513e-11\\
2.82	3.36513e-11\\
2.822	3.36513e-11\\
2.824	3.45608e-11\\
2.826	3.36513e-11\\
2.828	3.45608e-11\\
2.83	3.45608e-11\\
2.832	3.45608e-11\\
2.834	3.45608e-11\\
2.836	3.45608e-11\\
2.838	3.45608e-11\\
2.84	3.36513e-11\\
2.842	3.36513e-11\\
2.844	3.36513e-11\\
2.846	3.36513e-11\\
2.848	3.36513e-11\\
2.85	3.36513e-11\\
2.852	3.27418e-11\\
2.854	3.27418e-11\\
2.856	3.36513e-11\\
2.858	3.27418e-11\\
2.86	3.27418e-11\\
2.862	3.18323e-11\\
2.864	3.09228e-11\\
2.866	3.18323e-11\\
2.868	3.18323e-11\\
2.87	3.27418e-11\\
2.872	3.27418e-11\\
2.874	3.27418e-11\\
2.876	3.27418e-11\\
2.878	3.18323e-11\\
2.88	3.18323e-11\\
2.882	3.18323e-11\\
2.884	3.09228e-11\\
2.886	3.18323e-11\\
2.888	3.27418e-11\\
2.89	3.27418e-11\\
2.892	3.36513e-11\\
2.894	3.36513e-11\\
2.896	3.36513e-11\\
2.898	3.45608e-11\\
2.9	3.45608e-11\\
2.902	3.54703e-11\\
2.904	3.54703e-11\\
2.906	3.54703e-11\\
2.908	3.54703e-11\\
2.91	3.54703e-11\\
2.912	3.54703e-11\\
2.914	3.63798e-11\\
2.916	3.63798e-11\\
2.918	3.63798e-11\\
2.92	3.54703e-11\\
2.922	3.63798e-11\\
2.924	3.63798e-11\\
2.926	3.54703e-11\\
2.928	3.54703e-11\\
2.93	3.63798e-11\\
2.932	3.63798e-11\\
2.934	3.63798e-11\\
2.936	3.54703e-11\\
2.938	3.63798e-11\\
2.94	3.63798e-11\\
2.942	3.72893e-11\\
2.944	3.72893e-11\\
2.946	3.72893e-11\\
2.948	3.63798e-11\\
2.95	3.72893e-11\\
2.952	3.72893e-11\\
2.954	3.72893e-11\\
2.956	3.72893e-11\\
2.958	3.72893e-11\\
2.96	3.72893e-11\\
2.962	3.81988e-11\\
2.964	3.72893e-11\\
2.966	3.72893e-11\\
2.968	3.81988e-11\\
2.97	3.81988e-11\\
2.972	3.72893e-11\\
2.974	3.72893e-11\\
2.976	3.81988e-11\\
2.978	3.72893e-11\\
2.98	3.81988e-11\\
2.982	3.91083e-11\\
2.984	3.91083e-11\\
2.986	3.91083e-11\\
2.988	4.00178e-11\\
2.99	4.00178e-11\\
2.992	4.00178e-11\\
2.994	4.00178e-11\\
2.996	4.09273e-11\\
2.998	4.09273e-11\\
3	4.00178e-11\\
3.002	4.00178e-11\\
3.004	4.00178e-11\\
3.006	3.91083e-11\\
3.008	3.91083e-11\\
3.01	3.91083e-11\\
3.012	3.91083e-11\\
3.014	4.00178e-11\\
3.016	3.91083e-11\\
3.018	3.91083e-11\\
3.02	3.91083e-11\\
3.022	4.00178e-11\\
3.024	3.91083e-11\\
3.026	3.91083e-11\\
3.028	3.91083e-11\\
3.03	3.91083e-11\\
3.032	4.00178e-11\\
3.034	4.00178e-11\\
3.036	3.91083e-11\\
3.038	3.91083e-11\\
3.04	3.81988e-11\\
3.042	3.81988e-11\\
3.044	3.81988e-11\\
3.046	3.91083e-11\\
3.048	3.81988e-11\\
3.05	3.81988e-11\\
3.052	3.81988e-11\\
3.054	3.81988e-11\\
3.056	3.81988e-11\\
3.058	3.81988e-11\\
3.06	3.81988e-11\\
3.062	3.81988e-11\\
3.064	3.81988e-11\\
3.066	3.72893e-11\\
3.068	3.81988e-11\\
3.07	3.91083e-11\\
3.072	3.91083e-11\\
3.074	3.91083e-11\\
3.076	3.81988e-11\\
3.078	3.81988e-11\\
3.08	3.81988e-11\\
3.082	3.91083e-11\\
3.084	3.91083e-11\\
3.086	3.91083e-11\\
3.088	3.91083e-11\\
3.09	3.91083e-11\\
3.092	3.91083e-11\\
3.094	3.81988e-11\\
3.096	3.81988e-11\\
3.098	3.81988e-11\\
3.1	3.91083e-11\\
3.102	3.91083e-11\\
3.104	3.81988e-11\\
3.106	3.81988e-11\\
3.108	3.91083e-11\\
3.11	3.81988e-11\\
3.112	3.81988e-11\\
3.114	3.81988e-11\\
3.116	3.91083e-11\\
3.118	3.91083e-11\\
3.12	3.81988e-11\\
3.122	3.91083e-11\\
3.124	3.91083e-11\\
3.126	3.91083e-11\\
3.128	4.00178e-11\\
3.13	4.00178e-11\\
3.132	4.00178e-11\\
3.134	4.00178e-11\\
3.136	3.91083e-11\\
3.138	4.00178e-11\\
3.14	4.00178e-11\\
3.142	4.00178e-11\\
3.144	4.00178e-11\\
3.146	4.00178e-11\\
3.148	4.00178e-11\\
3.15	4.00178e-11\\
3.152	4.00178e-11\\
3.154	3.91083e-11\\
3.156	4.00178e-11\\
3.158	4.00178e-11\\
3.16	4.00178e-11\\
3.162	4.00178e-11\\
3.164	4.00178e-11\\
3.166	4.00178e-11\\
3.168	4.00178e-11\\
3.17	4.00178e-11\\
3.172	4.00178e-11\\
3.174	4.00178e-11\\
3.176	4.00178e-11\\
3.178	4.00178e-11\\
3.18	4.00178e-11\\
3.182	4.00178e-11\\
3.184	4.00178e-11\\
3.186	4.00178e-11\\
3.188	4.00178e-11\\
3.19	4.00178e-11\\
3.192	4.00178e-11\\
3.194	4.00178e-11\\
3.196	4.00178e-11\\
3.198	4.09273e-11\\
3.2	4.00178e-11\\
3.202	4.00178e-11\\
3.204	4.00178e-11\\
3.206	4.00178e-11\\
3.208	3.91083e-11\\
3.21	3.91083e-11\\
3.212	3.91083e-11\\
3.214	3.91083e-11\\
3.216	3.91083e-11\\
3.218	3.91083e-11\\
3.22	4.00178e-11\\
3.222	4.00178e-11\\
3.224	4.00178e-11\\
3.226	4.00178e-11\\
3.228	4.00178e-11\\
3.23	4.00178e-11\\
3.232	4.00178e-11\\
3.234	4.00178e-11\\
3.236	4.00178e-11\\
3.238	4.00178e-11\\
3.24	4.00178e-11\\
3.242	4.00178e-11\\
3.244	4.09273e-11\\
3.246	4.09273e-11\\
3.248	4.00178e-11\\
3.25	4.00178e-11\\
3.252	4.00178e-11\\
3.254	4.00178e-11\\
3.256	4.00178e-11\\
3.258	3.91083e-11\\
3.26	4.00178e-11\\
3.262	3.91083e-11\\
3.264	4.00178e-11\\
3.266	4.00178e-11\\
3.268	4.00178e-11\\
3.27	4.00178e-11\\
3.272	4.00178e-11\\
3.274	3.91083e-11\\
3.276	4.00178e-11\\
3.278	4.00178e-11\\
3.28	4.00178e-11\\
3.282	4.00178e-11\\
3.284	4.00178e-11\\
3.286	4.00178e-11\\
3.288	4.00178e-11\\
3.29	4.09273e-11\\
3.292	4.09273e-11\\
3.294	4.09273e-11\\
3.296	4.09273e-11\\
3.298	4.09273e-11\\
3.3	4.09273e-11\\
3.302	4.09273e-11\\
3.304	4.09273e-11\\
3.306	4.00178e-11\\
3.308	4.00178e-11\\
3.31	4.00178e-11\\
3.312	3.91083e-11\\
3.314	3.91083e-11\\
3.316	3.91083e-11\\
3.318	4.00178e-11\\
3.32	4.00178e-11\\
3.322	4.00178e-11\\
3.324	4.00178e-11\\
3.326	4.00178e-11\\
3.328	4.00178e-11\\
3.33	4.00178e-11\\
3.332	4.00178e-11\\
3.334	4.00178e-11\\
3.336	4.00178e-11\\
3.338	4.00178e-11\\
3.34	4.09273e-11\\
3.342	4.00178e-11\\
3.344	4.00178e-11\\
3.346	4.00178e-11\\
3.348	4.00178e-11\\
3.35	4.09273e-11\\
3.352	4.09273e-11\\
3.354	4.18368e-11\\
3.356	4.09273e-11\\
3.358	4.09273e-11\\
3.36	4.09273e-11\\
3.362	4.09273e-11\\
3.364	4.09273e-11\\
3.366	4.09273e-11\\
3.368	4.00178e-11\\
3.37	4.00178e-11\\
3.372	4.09273e-11\\
3.374	4.09273e-11\\
3.376	4.09273e-11\\
3.378	4.00178e-11\\
3.38	4.00178e-11\\
3.382	3.91083e-11\\
3.384	4.00178e-11\\
3.386	4.00178e-11\\
3.388	4.00178e-11\\
3.39	4.00178e-11\\
3.392	4.00178e-11\\
3.394	4.00178e-11\\
3.396	4.00178e-11\\
3.398	3.91083e-11\\
3.4	4.00178e-11\\
3.402	4.00178e-11\\
3.404	3.91083e-11\\
3.406	4.00178e-11\\
3.408	4.00178e-11\\
3.41	4.00178e-11\\
3.412	4.09273e-11\\
3.414	4.09273e-11\\
3.416	4.00178e-11\\
3.418	4.09273e-11\\
3.42	4.09273e-11\\
3.422	4.00178e-11\\
3.424	4.09273e-11\\
3.426	4.09273e-11\\
3.428	4.09273e-11\\
3.43	4.09273e-11\\
3.432	4.09273e-11\\
3.434	4.09273e-11\\
3.436	4.09273e-11\\
3.438	4.09273e-11\\
3.44	4.00178e-11\\
3.442	4.09273e-11\\
3.444	4.09273e-11\\
3.446	4.09273e-11\\
3.448	4.18368e-11\\
3.45	4.09273e-11\\
3.452	4.09273e-11\\
3.454	4.09273e-11\\
3.456	4.09273e-11\\
3.458	4.09273e-11\\
3.46	4.09273e-11\\
3.462	4.09273e-11\\
3.464	4.09273e-11\\
3.466	4.09273e-11\\
3.468	4.09273e-11\\
3.47	4.00178e-11\\
3.472	4.09273e-11\\
3.474	4.09273e-11\\
3.476	4.09273e-11\\
3.478	4.18368e-11\\
3.48	4.09273e-11\\
3.482	4.09273e-11\\
3.484	4.09273e-11\\
3.486	4.18368e-11\\
3.488	4.18368e-11\\
3.49	4.18368e-11\\
3.492	4.18368e-11\\
3.494	4.18368e-11\\
3.496	4.18368e-11\\
3.498	4.27463e-11\\
3.5	4.18368e-11\\
3.502	4.27463e-11\\
3.504	4.27463e-11\\
3.506	4.27463e-11\\
3.508	4.36557e-11\\
3.51	4.36557e-11\\
3.512	4.36557e-11\\
3.514	4.36557e-11\\
3.516	4.36557e-11\\
3.518	4.36557e-11\\
3.52	4.36557e-11\\
3.522	4.45652e-11\\
3.524	4.45652e-11\\
3.526	4.36557e-11\\
3.528	4.36557e-11\\
3.53	4.36557e-11\\
3.532	4.36557e-11\\
3.534	4.36557e-11\\
3.536	4.36557e-11\\
3.538	4.45652e-11\\
3.54	4.45652e-11\\
3.542	4.36557e-11\\
3.544	4.36557e-11\\
3.546	4.27463e-11\\
3.548	4.36557e-11\\
3.55	4.36557e-11\\
3.552	4.36557e-11\\
3.554	4.36557e-11\\
3.556	4.45652e-11\\
3.558	4.45652e-11\\
3.56	4.36557e-11\\
3.562	4.45652e-11\\
3.564	4.45652e-11\\
3.566	4.45652e-11\\
3.568	4.36557e-11\\
3.57	4.36557e-11\\
3.572	4.36557e-11\\
3.574	4.36557e-11\\
3.576	4.36557e-11\\
3.578	4.36557e-11\\
3.58	4.36557e-11\\
3.582	4.36557e-11\\
3.584	4.36557e-11\\
3.586	4.36557e-11\\
3.588	4.36557e-11\\
3.59	4.36557e-11\\
3.592	4.45652e-11\\
3.594	4.45652e-11\\
3.596	4.45652e-11\\
3.598	4.45652e-11\\
3.6	4.45652e-11\\
3.602	4.45652e-11\\
3.604	4.45652e-11\\
3.606	4.45652e-11\\
3.608	4.45652e-11\\
3.61	4.45652e-11\\
3.612	4.45652e-11\\
3.614	4.45652e-11\\
3.616	4.45652e-11\\
3.618	4.45652e-11\\
3.62	4.45652e-11\\
3.622	4.45652e-11\\
3.624	4.45652e-11\\
3.626	4.45652e-11\\
3.628	4.36557e-11\\
3.63	4.36557e-11\\
3.632	4.36557e-11\\
3.634	4.45652e-11\\
3.636	4.36557e-11\\
3.638	4.45652e-11\\
3.64	4.45652e-11\\
3.642	4.45652e-11\\
3.644	4.45652e-11\\
3.646	4.54747e-11\\
3.648	4.45652e-11\\
3.65	4.54747e-11\\
3.652	4.45652e-11\\
3.654	4.45652e-11\\
3.656	4.45652e-11\\
3.658	4.54747e-11\\
3.66	4.54747e-11\\
3.662	4.54747e-11\\
3.664	4.63842e-11\\
3.666	4.63842e-11\\
3.668	4.63842e-11\\
3.67	4.63842e-11\\
3.672	4.63842e-11\\
3.674	4.63842e-11\\
3.676	4.54747e-11\\
3.678	4.63842e-11\\
3.68	4.54747e-11\\
3.682	4.63842e-11\\
3.684	4.63842e-11\\
3.686	4.63842e-11\\
3.688	4.63842e-11\\
3.69	4.63842e-11\\
3.692	4.54747e-11\\
3.694	4.54747e-11\\
3.696	4.63842e-11\\
3.698	4.63842e-11\\
3.7	4.63842e-11\\
3.702	4.63842e-11\\
3.704	4.63842e-11\\
3.706	4.72937e-11\\
3.708	4.72937e-11\\
3.71	4.63842e-11\\
3.712	4.63842e-11\\
3.714	4.63842e-11\\
3.716	4.63842e-11\\
3.718	4.63842e-11\\
3.72	4.72937e-11\\
3.722	4.72937e-11\\
3.724	4.63842e-11\\
3.726	4.54747e-11\\
3.728	4.63842e-11\\
3.73	4.63842e-11\\
3.732	4.54747e-11\\
3.734	4.63842e-11\\
3.736	4.63842e-11\\
3.738	4.63842e-11\\
3.74	4.54747e-11\\
3.742	4.54747e-11\\
3.744	4.63842e-11\\
3.746	4.63842e-11\\
3.748	4.63842e-11\\
3.75	4.72937e-11\\
3.752	4.72937e-11\\
3.754	4.72937e-11\\
3.756	4.72937e-11\\
3.758	4.72937e-11\\
3.76	4.72937e-11\\
3.762	4.72937e-11\\
3.764	4.82032e-11\\
3.766	4.82032e-11\\
3.768	4.82032e-11\\
3.77	4.91127e-11\\
3.772	5.00222e-11\\
3.774	5.00222e-11\\
3.776	5.00222e-11\\
3.778	5.09317e-11\\
3.78	5.09317e-11\\
3.782	5.00222e-11\\
3.784	5.00222e-11\\
3.786	5.00222e-11\\
3.788	5.00222e-11\\
3.79	5.00222e-11\\
3.792	5.00222e-11\\
3.794	5.00222e-11\\
3.796	5.00222e-11\\
3.798	5.00222e-11\\
3.8	5.00222e-11\\
3.802	4.91127e-11\\
3.804	4.91127e-11\\
3.806	4.91127e-11\\
3.808	4.91127e-11\\
3.81	4.91127e-11\\
3.812	5.00222e-11\\
3.814	5.00222e-11\\
3.816	5.00222e-11\\
3.818	5.00222e-11\\
3.82	5.00222e-11\\
3.822	4.91127e-11\\
3.824	5.00222e-11\\
3.826	5.00222e-11\\
3.828	4.91127e-11\\
3.83	5.00222e-11\\
3.832	5.00222e-11\\
3.834	4.91127e-11\\
3.836	4.91127e-11\\
3.838	4.91127e-11\\
3.84	4.91127e-11\\
3.842	4.82032e-11\\
3.844	4.91127e-11\\
3.846	4.82032e-11\\
3.848	4.82032e-11\\
3.85	4.72937e-11\\
3.852	4.72937e-11\\
3.854	4.72937e-11\\
3.856	4.72937e-11\\
3.858	4.72937e-11\\
3.86	4.63842e-11\\
3.862	4.72937e-11\\
3.864	4.72937e-11\\
3.866	4.72937e-11\\
3.868	4.82032e-11\\
3.87	4.91127e-11\\
3.872	5.00222e-11\\
3.874	4.91127e-11\\
3.876	4.82032e-11\\
3.878	4.91127e-11\\
3.88	4.82032e-11\\
3.882	4.72937e-11\\
3.884	4.82032e-11\\
3.886	4.91127e-11\\
3.888	4.91127e-11\\
3.89	4.82032e-11\\
3.892	4.82032e-11\\
3.894	4.82032e-11\\
3.896	4.91127e-11\\
3.898	4.91127e-11\\
3.9	4.91127e-11\\
3.902	4.91127e-11\\
3.904	4.91127e-11\\
3.906	4.82032e-11\\
3.908	4.82032e-11\\
3.91	4.82032e-11\\
3.912	4.82032e-11\\
3.914	4.72937e-11\\
3.916	4.72937e-11\\
3.918	4.72937e-11\\
3.92	4.63842e-11\\
3.922	4.72937e-11\\
3.924	4.72937e-11\\
3.926	4.82032e-11\\
3.928	4.91127e-11\\
3.93	4.82032e-11\\
3.932	4.82032e-11\\
3.934	4.82032e-11\\
3.936	4.91127e-11\\
3.938	4.91127e-11\\
3.94	4.91127e-11\\
3.942	4.91127e-11\\
3.944	4.91127e-11\\
3.946	4.91127e-11\\
3.948	5.00222e-11\\
3.95	5.00222e-11\\
3.952	5.09317e-11\\
3.954	5.18412e-11\\
3.956	5.00222e-11\\
3.958	5.09317e-11\\
3.96	5.00222e-11\\
3.962	5.09317e-11\\
3.964	5.00222e-11\\
3.966	5.09317e-11\\
3.968	5.00222e-11\\
3.97	5.00222e-11\\
3.972	5.00222e-11\\
3.974	4.91127e-11\\
3.976	4.91127e-11\\
3.978	4.91127e-11\\
3.98	4.82032e-11\\
3.982	4.82032e-11\\
3.984	4.82032e-11\\
3.986	4.82032e-11\\
3.988	4.91127e-11\\
3.99	5.00222e-11\\
3.992	5.00222e-11\\
3.994	5.00222e-11\\
3.996	5.00222e-11\\
3.998	5.00222e-11\\
4	5.00222e-11\\
};
\addlegendentry{c2};

\addplot [color=mycolor6,solid]
  table[row sep=crcr]{%
0	3.63798e-12\\
0.002	0\\
0.004	3.63798e-12\\
0.006	3.63798e-12\\
0.008	7.27596e-12\\
0.01	-3.63798e-12\\
0.012	3.63798e-12\\
0.014	0\\
0.016	-1.45519e-11\\
0.018	3.63798e-12\\
0.02	-1.09139e-11\\
0.022	-1.09139e-11\\
0.024	-1.45519e-11\\
0.026	-1.09139e-11\\
0.028	-7.27596e-12\\
0.03	0\\
0.032	-7.27596e-12\\
0.034	-7.27596e-12\\
0.036	-7.27596e-12\\
0.038	-1.09139e-11\\
0.04	-7.27596e-12\\
0.042	-1.09139e-11\\
0.044	-1.81899e-11\\
0.046	-2.18279e-11\\
0.048	-2.18279e-11\\
0.05	-2.54659e-11\\
0.052	-2.91038e-11\\
0.054	-2.91038e-11\\
0.056	-2.91038e-11\\
0.058	-2.91038e-11\\
0.06	-3.27418e-11\\
0.062	-3.27418e-11\\
0.064	-3.27418e-11\\
0.066	-3.63798e-11\\
0.068	-3.27418e-11\\
0.07	-4.36557e-11\\
0.072	-4.00178e-11\\
0.074	-3.63798e-11\\
0.076	-4.00178e-11\\
0.078	-3.63798e-11\\
0.08	-4.72937e-11\\
0.082	-4.36557e-11\\
0.084	-4.00178e-11\\
0.086	-4.72937e-11\\
0.088	-3.27418e-11\\
0.09	-2.91038e-11\\
0.092	-3.27418e-11\\
0.094	-4.36557e-11\\
0.096	-4.72937e-11\\
0.098	-4.72937e-11\\
0.1	-5.82077e-11\\
0.102	-5.82077e-11\\
0.104	-5.82077e-11\\
0.106	-5.09317e-11\\
0.108	-5.09317e-11\\
0.11	-5.09317e-11\\
0.112	-4.00178e-11\\
0.114	-4.00178e-11\\
0.116	-5.09317e-11\\
0.118	-5.09317e-11\\
0.12	-5.09317e-11\\
0.122	-5.09317e-11\\
0.124	-5.82077e-11\\
0.126	-5.09317e-11\\
0.128	-4.00178e-11\\
0.13	-5.45697e-11\\
0.132	-5.45697e-11\\
0.134	-5.09317e-11\\
0.136	-5.09317e-11\\
0.138	-5.09317e-11\\
0.14	-5.09317e-11\\
0.142	-5.09317e-11\\
0.144	-4.72937e-11\\
0.146	-4.00178e-11\\
0.148	-5.09317e-11\\
0.15	-4.72937e-11\\
0.152	-5.09317e-11\\
0.154	-5.09317e-11\\
0.156	-3.63798e-11\\
0.158	-4.72937e-11\\
0.16	-4.72937e-11\\
0.162	-4.36557e-11\\
0.164	-4.72937e-11\\
0.166	-4.00178e-11\\
0.168	-4.00178e-11\\
0.17	-2.18279e-11\\
0.172	-2.18279e-11\\
0.174	-2.91038e-11\\
0.176	-4.00178e-11\\
0.178	-4.36557e-11\\
0.18	-4.36557e-11\\
0.182	-4.72937e-11\\
0.184	-5.09317e-11\\
0.186	-5.09317e-11\\
0.188	-5.09317e-11\\
0.19	-5.82077e-11\\
0.192	-5.45697e-11\\
0.194	-5.09317e-11\\
0.196	-5.09317e-11\\
0.198	-4.72937e-11\\
0.2	-4.72937e-11\\
0.202	-3.63798e-11\\
0.204	-4.36557e-11\\
0.206	-3.27418e-11\\
0.208	-3.63798e-11\\
0.21	-3.63798e-11\\
0.212	-4.00178e-11\\
0.214	-4.00178e-11\\
0.216	-3.63798e-11\\
0.218	-5.09317e-11\\
0.22	-5.82077e-11\\
0.222	-5.45697e-11\\
0.224	-4.36557e-11\\
0.226	-4.00178e-11\\
0.228	-5.82077e-11\\
0.23	-5.09317e-11\\
0.232	-5.45697e-11\\
0.234	-6.54836e-11\\
0.236	-5.45697e-11\\
0.238	-5.45697e-11\\
0.24	-4.36557e-11\\
0.242	-4.00178e-11\\
0.244	-3.63798e-11\\
0.246	-2.91038e-11\\
0.248	-2.91038e-11\\
0.25	-2.54659e-11\\
0.252	-7.27596e-12\\
0.254	-7.27596e-12\\
0.256	-1.09139e-11\\
0.258	-1.45519e-11\\
0.26	-1.45519e-11\\
0.262	-1.09139e-11\\
0.264	-3.63798e-12\\
0.266	0\\
0.268	-7.27596e-12\\
0.27	-2.18279e-11\\
0.272	-1.45519e-11\\
0.274	-2.18279e-11\\
0.276	-3.27418e-11\\
0.278	-3.63798e-11\\
0.28	-3.27418e-11\\
0.282	-3.27418e-11\\
0.284	-3.27418e-11\\
0.286	-3.27418e-11\\
0.288	-2.18279e-11\\
0.29	-3.27418e-11\\
0.292	-3.63798e-11\\
0.294	-2.18279e-11\\
0.296	-4.00178e-11\\
0.298	-4.00178e-11\\
0.3	-2.54659e-11\\
0.302	-3.63798e-11\\
0.304	-3.63798e-11\\
0.306	-3.27418e-11\\
0.308	-2.18279e-11\\
0.31	-2.54659e-11\\
0.312	-3.63798e-11\\
0.314	-2.54659e-11\\
0.316	-2.18279e-11\\
0.318	-2.54659e-11\\
0.32	-2.18279e-11\\
0.322	-2.18279e-11\\
0.324	-2.54659e-11\\
0.326	-2.54659e-11\\
0.328	-2.18279e-11\\
0.33	-1.81899e-11\\
0.332	-2.18279e-11\\
0.334	-2.18279e-11\\
0.336	-1.81899e-11\\
0.338	-1.81899e-11\\
0.34	-1.81899e-11\\
0.342	-1.81899e-11\\
0.344	-1.09139e-11\\
0.346	-1.81899e-11\\
0.348	-1.45519e-11\\
0.35	-1.45519e-11\\
0.352	-1.09139e-11\\
0.354	-1.45519e-11\\
0.356	-7.27596e-12\\
0.358	7.27596e-12\\
0.36	1.09139e-11\\
0.362	7.27596e-12\\
0.364	1.09139e-11\\
0.366	3.63798e-12\\
0.368	3.63798e-12\\
0.37	1.81899e-11\\
0.372	1.81899e-11\\
0.374	1.45519e-11\\
0.376	1.81899e-11\\
0.378	1.45519e-11\\
0.38	1.09139e-11\\
0.382	0\\
0.384	1.09139e-11\\
0.386	7.27596e-12\\
0.388	7.27596e-12\\
0.39	7.27596e-12\\
0.392	1.09139e-11\\
0.394	0\\
0.396	0\\
0.398	0\\
0.4	7.27596e-12\\
0.402	2.18279e-11\\
0.404	1.81899e-11\\
0.406	1.81899e-11\\
0.408	1.81899e-11\\
0.41	2.54659e-11\\
0.412	2.54659e-11\\
0.414	1.45519e-11\\
0.416	1.45519e-11\\
0.418	2.18279e-11\\
0.42	2.91038e-11\\
0.422	3.27418e-11\\
0.424	4.72937e-11\\
0.426	4.36557e-11\\
0.428	4.72937e-11\\
0.43	5.45697e-11\\
0.432	5.45697e-11\\
0.434	5.09317e-11\\
0.436	5.09317e-11\\
0.438	5.09317e-11\\
0.44	5.09317e-11\\
0.442	4.72937e-11\\
0.444	5.09317e-11\\
0.446	5.45697e-11\\
0.448	6.18456e-11\\
0.45	6.54836e-11\\
0.452	7.63976e-11\\
0.454	7.27596e-11\\
0.456	7.27596e-11\\
0.458	7.63976e-11\\
0.46	6.18456e-11\\
0.462	6.18456e-11\\
0.464	6.18456e-11\\
0.466	7.63976e-11\\
0.468	8.36735e-11\\
0.47	7.27596e-11\\
0.472	7.27596e-11\\
0.474	6.91216e-11\\
0.476	8.00355e-11\\
0.478	6.91216e-11\\
0.48	6.54836e-11\\
0.482	5.09317e-11\\
0.484	4.00178e-11\\
0.486	4.00178e-11\\
0.488	3.63798e-11\\
0.49	2.91038e-11\\
0.492	1.81899e-11\\
0.494	2.18279e-11\\
0.496	2.18279e-11\\
0.498	1.09139e-11\\
0.5	7.27596e-12\\
0.502	7.27596e-12\\
0.504	-3.63798e-12\\
0.506	-7.27596e-12\\
0.508	-2.18279e-11\\
0.51	-1.09139e-11\\
0.512	-2.54659e-11\\
0.514	-2.18279e-11\\
0.516	-7.27596e-12\\
0.518	-7.27596e-12\\
0.52	-7.27596e-12\\
0.522	-3.63798e-12\\
0.524	7.27596e-12\\
0.526	1.09139e-11\\
0.528	1.09139e-11\\
0.53	2.18279e-11\\
0.532	1.45519e-11\\
0.534	2.18279e-11\\
0.536	3.27418e-11\\
0.538	4.36557e-11\\
0.54	4.72937e-11\\
0.542	5.09317e-11\\
0.544	6.18456e-11\\
0.546	7.27596e-11\\
0.548	7.63976e-11\\
0.55	6.18456e-11\\
0.552	6.91216e-11\\
0.554	6.18456e-11\\
0.556	5.45697e-11\\
0.558	6.54836e-11\\
0.56	7.27596e-11\\
0.562	6.18456e-11\\
0.564	5.82077e-11\\
0.566	5.09317e-11\\
0.568	5.09317e-11\\
0.57	5.82077e-11\\
0.572	7.27596e-11\\
0.574	7.27596e-11\\
0.576	8.00355e-11\\
0.578	7.27596e-11\\
0.58	6.91216e-11\\
0.582	6.91216e-11\\
0.584	5.45697e-11\\
0.586	6.18456e-11\\
0.588	6.18456e-11\\
0.59	5.82077e-11\\
0.592	6.18456e-11\\
0.594	6.18456e-11\\
0.596	6.18456e-11\\
0.598	6.54836e-11\\
0.6	6.18456e-11\\
0.602	5.82077e-11\\
0.604	4.72937e-11\\
0.606	4.72937e-11\\
0.608	3.27418e-11\\
0.61	2.91038e-11\\
0.612	4.36557e-11\\
0.614	4.36557e-11\\
0.616	2.91038e-11\\
0.618	2.54659e-11\\
0.62	4.36557e-11\\
0.622	5.09317e-11\\
0.624	4.00178e-11\\
0.626	2.54659e-11\\
0.628	1.81899e-11\\
0.63	1.45519e-11\\
0.632	1.45519e-11\\
0.634	3.63798e-12\\
0.636	1.81899e-11\\
0.638	3.63798e-12\\
0.64	3.63798e-12\\
0.642	3.63798e-12\\
0.644	-7.27596e-12\\
0.646	-7.27596e-12\\
0.648	-1.09139e-11\\
0.65	-7.27596e-12\\
0.652	-7.27596e-12\\
0.654	-1.45519e-11\\
0.656	-3.63798e-12\\
0.658	-1.45519e-11\\
0.66	-2.54659e-11\\
0.662	-3.63798e-12\\
0.664	-1.09139e-11\\
0.666	0\\
0.668	1.45519e-11\\
0.67	1.81899e-11\\
0.672	3.63798e-12\\
0.674	1.45519e-11\\
0.676	0\\
0.678	3.63798e-12\\
0.68	-7.27596e-12\\
0.682	-1.09139e-11\\
0.684	-7.27596e-12\\
0.686	-1.09139e-11\\
0.688	-1.09139e-11\\
0.69	-1.09139e-11\\
0.692	0\\
0.694	0\\
0.696	3.63798e-12\\
0.698	-7.27596e-12\\
0.7	0\\
0.702	1.45519e-11\\
0.704	1.45519e-11\\
0.706	2.54659e-11\\
0.708	2.54659e-11\\
0.71	1.45519e-11\\
0.712	2.54659e-11\\
0.714	1.45519e-11\\
0.716	2.91038e-11\\
0.718	2.91038e-11\\
0.72	2.18279e-11\\
0.722	2.91038e-11\\
0.724	2.54659e-11\\
0.726	2.91038e-11\\
0.728	2.54659e-11\\
0.73	2.18279e-11\\
0.732	4.36557e-11\\
0.734	4.00178e-11\\
0.736	2.54659e-11\\
0.738	2.54659e-11\\
0.74	2.54659e-11\\
0.742	1.45519e-11\\
0.744	0\\
0.746	-1.09139e-11\\
0.748	-1.09139e-11\\
0.75	-1.45519e-11\\
0.752	-1.09139e-11\\
0.754	-1.09139e-11\\
0.756	-2.54659e-11\\
0.758	-1.09139e-11\\
0.76	-2.54659e-11\\
0.762	-1.45519e-11\\
0.764	-2.18279e-11\\
0.766	-1.09139e-11\\
0.768	-3.63798e-12\\
0.77	-7.27596e-12\\
0.772	-7.27596e-12\\
0.774	-2.18279e-11\\
0.776	-3.27418e-11\\
0.778	-3.27418e-11\\
0.78	-3.63798e-11\\
0.782	-4.00178e-11\\
0.784	-4.36557e-11\\
0.786	-3.63798e-11\\
0.788	-4.36557e-11\\
0.79	-4.36557e-11\\
0.792	-5.82077e-11\\
0.794	-6.91216e-11\\
0.796	-5.82077e-11\\
0.798	-6.18456e-11\\
0.8	-4.72937e-11\\
0.802	-4.72937e-11\\
0.804	-3.63798e-11\\
0.806	-2.54659e-11\\
0.808	-4.00178e-11\\
0.81	-4.72937e-11\\
0.812	-6.18456e-11\\
0.814	-6.18456e-11\\
0.816	-4.72937e-11\\
0.818	-5.09317e-11\\
0.82	-6.18456e-11\\
0.822	-6.54836e-11\\
0.824	-6.54836e-11\\
0.826	-7.63976e-11\\
0.828	-7.63976e-11\\
0.83	-7.27596e-11\\
0.832	-7.27596e-11\\
0.834	-8.73115e-11\\
0.836	-7.63976e-11\\
0.838	-6.18456e-11\\
0.84	-4.72937e-11\\
0.842	-2.54659e-11\\
0.844	-2.54659e-11\\
0.846	-1.09139e-11\\
0.848	-2.54659e-11\\
0.85	-3.63798e-12\\
0.852	-1.45519e-11\\
0.854	-2.54659e-11\\
0.856	-1.45519e-11\\
0.858	0\\
0.86	1.09139e-11\\
0.862	1.09139e-11\\
0.864	7.27596e-12\\
0.866	7.27596e-12\\
0.868	-7.27596e-12\\
0.87	-1.81899e-11\\
0.872	-2.18279e-11\\
0.874	-1.81899e-11\\
0.876	-3.63798e-12\\
0.878	0\\
0.88	-7.27596e-12\\
0.882	-2.18279e-11\\
0.884	-3.63798e-12\\
0.886	-7.27596e-12\\
0.888	-1.81899e-11\\
0.89	-4.36557e-11\\
0.892	-4.36557e-11\\
0.894	-5.09317e-11\\
0.896	-4.36557e-11\\
0.898	-3.27418e-11\\
0.9	-4.36557e-11\\
0.902	-5.09317e-11\\
0.904	-5.45697e-11\\
0.906	-5.09317e-11\\
0.908	-4.72937e-11\\
0.91	-4.36557e-11\\
0.912	-5.09317e-11\\
0.914	-5.09317e-11\\
0.916	-6.18456e-11\\
0.918	-4.36557e-11\\
0.92	-2.91038e-11\\
0.922	-2.91038e-11\\
0.924	-2.54659e-11\\
0.926	-2.54659e-11\\
0.928	-1.45519e-11\\
0.93	-2.91038e-11\\
0.932	-4.00178e-11\\
0.934	-2.54659e-11\\
0.936	-4.36557e-11\\
0.938	-4.72937e-11\\
0.94	-2.91038e-11\\
0.942	-3.27418e-11\\
0.944	-1.81899e-11\\
0.946	-1.81899e-11\\
0.948	-7.27596e-12\\
0.95	-7.27596e-12\\
0.952	-7.27596e-12\\
0.954	-7.27596e-12\\
0.956	3.63798e-12\\
0.958	-1.09139e-11\\
0.96	0\\
0.962	1.81899e-11\\
0.964	0\\
0.966	1.81899e-11\\
0.968	2.91038e-11\\
0.97	2.91038e-11\\
0.972	2.54659e-11\\
0.974	3.63798e-11\\
0.976	2.91038e-11\\
0.978	2.54659e-11\\
0.98	3.63798e-11\\
0.982	3.63798e-11\\
0.984	3.63798e-11\\
0.986	3.63798e-11\\
0.988	3.63798e-11\\
0.99	4.36557e-11\\
0.992	3.27418e-11\\
0.994	4.72937e-11\\
0.996	5.82077e-11\\
0.998	5.82077e-11\\
1	4.72937e-11\\
1.002	4.72937e-11\\
1.004	4.72937e-11\\
1.006	2.54659e-11\\
1.008	2.54659e-11\\
1.01	2.54659e-11\\
1.012	1.09139e-11\\
1.014	2.18279e-11\\
1.016	1.45519e-11\\
1.018	0\\
1.02	0\\
1.022	-3.63798e-12\\
1.024	-3.63798e-12\\
1.026	-1.45519e-11\\
1.028	-2.91038e-11\\
1.03	-2.54659e-11\\
1.032	-1.45519e-11\\
1.034	-1.45519e-11\\
1.036	-3.63798e-12\\
1.038	1.45519e-11\\
1.04	2.54659e-11\\
1.042	3.27418e-11\\
1.044	3.63798e-11\\
1.046	4.72937e-11\\
1.048	3.63798e-11\\
1.05	4.36557e-11\\
1.052	4.36557e-11\\
1.054	3.27418e-11\\
1.056	2.54659e-11\\
1.058	3.27418e-11\\
1.06	2.91038e-11\\
1.062	1.81899e-11\\
1.064	2.91038e-11\\
1.066	2.91038e-11\\
1.068	2.91038e-11\\
1.07	4.36557e-11\\
1.072	4.00178e-11\\
1.074	5.82077e-11\\
1.076	5.82077e-11\\
1.078	8.00355e-11\\
1.08	6.54836e-11\\
1.082	5.45697e-11\\
1.084	5.45697e-11\\
1.086	5.09317e-11\\
1.088	5.09317e-11\\
1.09	4.36557e-11\\
1.092	5.82077e-11\\
1.094	6.54836e-11\\
1.096	6.54836e-11\\
1.098	6.54836e-11\\
1.1	5.45697e-11\\
1.102	5.45697e-11\\
1.104	4.00178e-11\\
1.106	4.00178e-11\\
1.108	4.36557e-11\\
1.11	4.36557e-11\\
1.112	4.00178e-11\\
1.114	2.91038e-11\\
1.116	3.27418e-11\\
1.118	2.91038e-11\\
1.12	1.81899e-11\\
1.122	2.54659e-11\\
1.124	1.45519e-11\\
1.126	3.63798e-12\\
1.128	-7.27596e-12\\
1.13	7.27596e-12\\
1.132	1.09139e-11\\
1.134	1.09139e-11\\
1.136	7.27596e-12\\
1.138	-7.27596e-12\\
1.14	-7.27596e-12\\
1.142	-1.45519e-11\\
1.144	-1.45519e-11\\
1.146	-2.91038e-11\\
1.148	-4.00178e-11\\
1.15	-4.00178e-11\\
1.152	-4.00178e-11\\
1.154	-3.63798e-11\\
1.156	-2.91038e-11\\
1.158	-1.81899e-11\\
1.16	-1.09139e-11\\
1.162	-2.54659e-11\\
1.164	-1.81899e-11\\
1.166	-2.91038e-11\\
1.168	-2.91038e-11\\
1.17	-2.91038e-11\\
1.172	-4.00178e-11\\
1.174	-4.36557e-11\\
1.176	-5.82077e-11\\
1.178	-4.72937e-11\\
1.18	-6.18456e-11\\
1.182	-5.09317e-11\\
1.184	-6.18456e-11\\
1.186	-6.54836e-11\\
1.188	-6.91216e-11\\
1.19	-6.91216e-11\\
1.192	-6.18456e-11\\
1.194	-6.54836e-11\\
1.196	-4.72937e-11\\
1.198	-4.72937e-11\\
1.2	-5.09317e-11\\
1.202	-5.09317e-11\\
1.204	-5.09317e-11\\
1.206	-5.45697e-11\\
1.208	-5.45697e-11\\
1.21	-5.09317e-11\\
1.212	-6.18456e-11\\
1.214	-6.54836e-11\\
1.216	-7.27596e-11\\
1.218	-8.00355e-11\\
1.22	-6.91216e-11\\
1.222	-6.54836e-11\\
1.224	-6.91216e-11\\
1.226	-6.91216e-11\\
1.228	-7.63976e-11\\
1.23	-7.63976e-11\\
1.232	-7.27596e-11\\
1.234	-7.27596e-11\\
1.236	-6.54836e-11\\
1.238	-5.45697e-11\\
1.24	-4.72937e-11\\
1.242	-4.36557e-11\\
1.244	-3.27418e-11\\
1.246	-4.36557e-11\\
1.248	-4.72937e-11\\
1.25	-4.36557e-11\\
1.252	-4.36557e-11\\
1.254	-2.91038e-11\\
1.256	-4.36557e-11\\
1.258	-2.54659e-11\\
1.26	-3.27418e-11\\
1.262	-2.18279e-11\\
1.264	-1.81899e-11\\
1.266	-3.63798e-12\\
1.268	-3.63798e-12\\
1.27	0\\
1.272	0\\
1.274	7.27596e-12\\
1.276	-3.63798e-12\\
1.278	-3.63798e-12\\
1.28	-7.27596e-12\\
1.282	-1.09139e-11\\
1.284	-7.27596e-12\\
1.286	0\\
1.288	-7.27596e-12\\
1.29	3.63798e-12\\
1.292	7.27596e-12\\
1.294	3.63798e-12\\
1.296	1.81899e-11\\
1.298	2.18279e-11\\
1.3	2.91038e-11\\
1.302	3.27418e-11\\
1.304	3.27418e-11\\
1.306	3.27418e-11\\
1.308	3.27418e-11\\
1.31	3.27418e-11\\
1.312	3.27418e-11\\
1.314	3.27418e-11\\
1.316	2.54659e-11\\
1.318	1.81899e-11\\
1.32	3.27418e-11\\
1.322	2.54659e-11\\
1.324	1.45519e-11\\
1.326	2.54659e-11\\
1.328	2.91038e-11\\
1.33	2.91038e-11\\
1.332	1.81899e-11\\
1.334	2.54659e-11\\
1.336	1.45519e-11\\
1.338	1.09139e-11\\
1.34	1.45519e-11\\
1.342	1.81899e-11\\
1.344	1.81899e-11\\
1.346	2.18279e-11\\
1.348	1.81899e-11\\
1.35	2.18279e-11\\
1.352	1.81899e-11\\
1.354	2.54659e-11\\
1.356	2.91038e-11\\
1.358	1.81899e-11\\
1.36	2.91038e-11\\
1.362	2.54659e-11\\
1.364	2.18279e-11\\
1.366	1.09139e-11\\
1.368	1.09139e-11\\
1.37	1.09139e-11\\
1.372	7.27596e-12\\
1.374	3.63798e-12\\
1.376	1.45519e-11\\
1.378	1.45519e-11\\
1.38	2.91038e-11\\
1.382	2.91038e-11\\
1.384	4.36557e-11\\
1.386	4.00178e-11\\
1.388	4.00178e-11\\
1.39	4.36557e-11\\
1.392	4.36557e-11\\
1.394	4.00178e-11\\
1.396	5.09317e-11\\
1.398	5.09317e-11\\
1.4	5.45697e-11\\
1.402	5.45697e-11\\
1.404	5.82077e-11\\
1.406	5.82077e-11\\
1.408	7.27596e-11\\
1.41	6.18456e-11\\
1.412	6.18456e-11\\
1.414	6.54836e-11\\
1.416	6.18456e-11\\
1.418	6.91216e-11\\
1.42	6.91216e-11\\
1.422	6.54836e-11\\
1.424	6.18456e-11\\
1.426	8.00355e-11\\
1.428	5.82077e-11\\
1.43	6.18456e-11\\
1.432	6.54836e-11\\
1.434	5.82077e-11\\
1.436	5.45697e-11\\
1.438	5.45697e-11\\
1.44	6.18456e-11\\
1.442	5.45697e-11\\
1.444	5.09317e-11\\
1.446	4.36557e-11\\
1.448	5.09317e-11\\
1.45	5.45697e-11\\
1.452	6.54836e-11\\
1.454	6.54836e-11\\
1.456	6.54836e-11\\
1.458	6.54836e-11\\
1.46	6.54836e-11\\
1.462	8.00355e-11\\
1.464	8.00355e-11\\
1.466	8.36735e-11\\
1.468	8.36735e-11\\
1.47	8.00355e-11\\
1.472	9.45874e-11\\
1.474	8.73115e-11\\
1.476	9.82254e-11\\
1.478	8.36735e-11\\
1.48	8.73115e-11\\
1.482	8.73115e-11\\
1.484	9.45874e-11\\
1.486	9.82254e-11\\
1.488	9.82254e-11\\
1.49	9.82254e-11\\
1.492	1.01863e-10\\
1.494	8.73115e-11\\
1.496	9.45874e-11\\
1.498	9.09495e-11\\
1.5	9.45874e-11\\
1.502	9.09495e-11\\
1.504	9.09495e-11\\
1.506	1.05501e-10\\
1.508	1.09139e-10\\
1.51	1.05501e-10\\
1.512	9.09495e-11\\
1.514	9.09495e-11\\
1.516	9.82254e-11\\
1.518	1.05501e-10\\
1.52	1.09139e-10\\
1.522	1.12777e-10\\
1.524	1.12777e-10\\
1.526	1.12777e-10\\
1.528	9.82254e-11\\
1.53	1.01863e-10\\
1.532	9.82254e-11\\
1.534	9.09495e-11\\
1.536	1.01863e-10\\
1.538	9.45874e-11\\
1.54	1.05501e-10\\
1.542	1.05501e-10\\
1.544	9.82254e-11\\
1.546	9.45874e-11\\
1.548	9.82254e-11\\
1.55	1.01863e-10\\
1.552	1.05501e-10\\
1.554	9.45874e-11\\
1.556	1.05501e-10\\
1.558	9.09495e-11\\
1.56	9.45874e-11\\
1.562	9.45874e-11\\
1.564	9.82254e-11\\
1.566	8.36735e-11\\
1.568	8.73115e-11\\
1.57	8.36735e-11\\
1.572	9.09495e-11\\
1.574	8.73115e-11\\
1.576	8.00355e-11\\
1.578	8.00355e-11\\
1.58	7.63976e-11\\
1.582	7.27596e-11\\
1.584	7.27596e-11\\
1.586	7.27596e-11\\
1.588	7.27596e-11\\
1.59	8.00355e-11\\
1.592	6.91216e-11\\
1.594	6.54836e-11\\
1.596	6.54836e-11\\
1.598	6.54836e-11\\
1.6	6.54836e-11\\
1.602	6.18456e-11\\
1.604	6.18456e-11\\
1.606	6.91216e-11\\
1.608	6.54836e-11\\
1.61	6.54836e-11\\
1.612	5.82077e-11\\
1.614	6.18456e-11\\
1.616	6.18456e-11\\
1.618	6.54836e-11\\
1.62	6.54836e-11\\
1.622	8.36735e-11\\
1.624	8.00355e-11\\
1.626	7.63976e-11\\
1.628	8.00355e-11\\
1.63	7.63976e-11\\
1.632	8.00355e-11\\
1.634	8.36735e-11\\
1.636	8.36735e-11\\
1.638	8.36735e-11\\
1.64	8.36735e-11\\
1.642	8.73115e-11\\
1.644	9.09495e-11\\
1.646	9.45874e-11\\
1.648	8.73115e-11\\
1.65	9.09495e-11\\
1.652	9.45874e-11\\
1.654	1.05501e-10\\
1.656	1.09139e-10\\
1.658	1.09139e-10\\
1.66	1.09139e-10\\
1.662	1.16415e-10\\
1.664	1.23691e-10\\
1.666	1.09139e-10\\
1.668	1.16415e-10\\
1.67	1.20053e-10\\
1.672	1.23691e-10\\
1.674	1.23691e-10\\
1.676	1.23691e-10\\
1.678	1.20053e-10\\
1.68	1.20053e-10\\
1.682	1.16415e-10\\
1.684	1.05501e-10\\
1.686	1.09139e-10\\
1.688	1.09139e-10\\
1.69	1.12777e-10\\
1.692	1.16415e-10\\
1.694	1.12777e-10\\
1.696	1.01863e-10\\
1.698	1.01863e-10\\
1.7	1.09139e-10\\
1.702	1.01863e-10\\
1.704	1.05501e-10\\
1.706	1.01863e-10\\
1.708	9.82254e-11\\
1.71	9.82254e-11\\
1.712	1.01863e-10\\
1.714	9.82254e-11\\
1.716	9.82254e-11\\
1.718	9.45874e-11\\
1.72	9.82254e-11\\
1.722	9.45874e-11\\
1.724	1.01863e-10\\
1.726	9.82254e-11\\
1.728	1.01863e-10\\
1.73	1.01863e-10\\
1.732	1.01863e-10\\
1.734	1.05501e-10\\
1.736	1.01863e-10\\
1.738	1.05501e-10\\
1.74	9.45874e-11\\
1.742	9.45874e-11\\
1.744	9.09495e-11\\
1.746	8.73115e-11\\
1.748	8.73115e-11\\
1.75	8.36735e-11\\
1.752	7.63976e-11\\
1.754	8.00355e-11\\
1.756	8.00355e-11\\
1.758	8.00355e-11\\
1.76	8.00355e-11\\
1.762	6.91216e-11\\
1.764	6.54836e-11\\
1.766	6.91216e-11\\
1.768	7.27596e-11\\
1.77	7.63976e-11\\
1.772	7.27596e-11\\
1.774	6.91216e-11\\
1.776	6.54836e-11\\
1.778	6.54836e-11\\
1.78	5.45697e-11\\
1.782	6.91216e-11\\
1.784	5.82077e-11\\
1.786	6.18456e-11\\
1.788	7.63976e-11\\
1.79	6.91216e-11\\
1.792	6.18456e-11\\
1.794	7.63976e-11\\
1.796	7.27596e-11\\
1.798	7.63976e-11\\
1.8	8.00355e-11\\
1.802	8.00355e-11\\
1.804	8.00355e-11\\
1.806	8.00355e-11\\
1.808	7.63976e-11\\
1.81	6.91216e-11\\
1.812	6.91216e-11\\
1.814	6.54836e-11\\
1.816	6.54836e-11\\
1.818	6.54836e-11\\
1.82	5.45697e-11\\
1.822	6.18456e-11\\
1.824	6.54836e-11\\
1.826	6.91216e-11\\
1.828	6.18456e-11\\
1.83	6.91216e-11\\
1.832	6.91216e-11\\
1.834	5.82077e-11\\
1.836	7.27596e-11\\
1.838	7.63976e-11\\
1.84	7.63976e-11\\
1.842	7.27596e-11\\
1.844	6.91216e-11\\
1.846	6.54836e-11\\
1.848	6.54836e-11\\
1.85	6.18456e-11\\
1.852	5.82077e-11\\
1.854	5.82077e-11\\
1.856	5.45697e-11\\
1.858	5.45697e-11\\
1.86	3.63798e-11\\
1.862	4.36557e-11\\
1.864	4.00178e-11\\
1.866	3.63798e-11\\
1.868	5.09317e-11\\
1.87	5.09317e-11\\
1.872	5.45697e-11\\
1.874	5.09317e-11\\
1.876	5.09317e-11\\
1.878	4.72937e-11\\
1.88	5.09317e-11\\
1.882	4.72937e-11\\
1.884	3.27418e-11\\
1.886	2.91038e-11\\
1.888	2.91038e-11\\
1.89	2.18279e-11\\
1.892	2.18279e-11\\
1.894	2.91038e-11\\
1.896	2.91038e-11\\
1.898	2.91038e-11\\
1.9	2.91038e-11\\
1.902	2.91038e-11\\
1.904	4.36557e-11\\
1.906	4.36557e-11\\
1.908	2.91038e-11\\
1.91	2.54659e-11\\
1.912	2.54659e-11\\
1.914	4.36557e-11\\
1.916	4.00178e-11\\
1.918	4.00178e-11\\
1.92	4.00178e-11\\
1.922	3.63798e-11\\
1.924	3.27418e-11\\
1.926	3.63798e-11\\
1.928	4.00178e-11\\
1.93	4.36557e-11\\
1.932	4.72937e-11\\
1.934	4.72937e-11\\
1.936	4.36557e-11\\
1.938	4.00178e-11\\
1.94	3.63798e-11\\
1.942	4.00178e-11\\
1.944	4.00178e-11\\
1.946	4.36557e-11\\
1.948	5.09317e-11\\
1.95	5.82077e-11\\
1.952	5.45697e-11\\
1.954	5.82077e-11\\
1.956	6.18456e-11\\
1.958	5.82077e-11\\
1.96	5.45697e-11\\
1.962	4.72937e-11\\
1.964	4.36557e-11\\
1.966	4.36557e-11\\
1.968	2.91038e-11\\
1.97	3.63798e-11\\
1.972	4.00178e-11\\
1.974	2.54659e-11\\
1.976	2.18279e-11\\
1.978	1.81899e-11\\
1.98	2.18279e-11\\
1.982	2.54659e-11\\
1.984	4.36557e-11\\
1.986	4.36557e-11\\
1.988	4.72937e-11\\
1.99	4.36557e-11\\
1.992	4.00178e-11\\
1.994	5.09317e-11\\
1.996	5.45697e-11\\
1.998	5.09317e-11\\
2	6.18456e-11\\
2.002	7.63976e-11\\
2.004	7.27596e-11\\
2.006	7.63976e-11\\
2.008	6.54836e-11\\
2.01	7.63976e-11\\
2.012	6.91216e-11\\
2.014	6.54836e-11\\
2.016	6.54836e-11\\
2.018	6.54836e-11\\
2.02	6.54836e-11\\
2.022	5.09317e-11\\
2.024	5.09317e-11\\
2.026	6.91216e-11\\
2.028	6.91216e-11\\
2.03	7.63976e-11\\
2.032	6.54836e-11\\
2.034	6.91216e-11\\
2.036	6.91216e-11\\
2.038	8.00355e-11\\
2.04	8.00355e-11\\
2.042	7.63976e-11\\
2.044	8.73115e-11\\
2.046	8.36735e-11\\
2.048	8.36735e-11\\
2.05	8.36735e-11\\
2.052	8.00355e-11\\
2.054	8.36735e-11\\
2.056	8.36735e-11\\
2.058	8.73115e-11\\
2.06	8.73115e-11\\
2.062	8.73115e-11\\
2.064	8.73115e-11\\
2.066	8.36735e-11\\
2.068	8.73115e-11\\
2.07	9.82254e-11\\
2.072	1.01863e-10\\
2.074	9.82254e-11\\
2.076	1.12777e-10\\
2.078	1.12777e-10\\
2.08	1.05501e-10\\
2.082	1.09139e-10\\
2.084	1.16415e-10\\
2.086	1.20053e-10\\
2.088	1.30967e-10\\
2.09	1.16415e-10\\
2.092	1.16415e-10\\
2.094	1.20053e-10\\
2.096	1.30967e-10\\
2.098	1.45519e-10\\
2.1	1.60071e-10\\
2.102	1.56433e-10\\
2.104	1.52795e-10\\
2.106	1.56433e-10\\
2.108	1.41881e-10\\
2.11	1.45519e-10\\
2.112	1.60071e-10\\
2.114	1.60071e-10\\
2.116	1.60071e-10\\
2.118	1.45519e-10\\
2.12	1.56433e-10\\
2.122	1.60071e-10\\
2.124	1.56433e-10\\
2.126	1.56433e-10\\
2.128	1.60071e-10\\
2.13	1.60071e-10\\
2.132	1.60071e-10\\
2.134	1.67347e-10\\
2.136	1.63709e-10\\
2.138	1.63709e-10\\
2.14	1.78261e-10\\
2.142	1.81899e-10\\
2.144	1.67347e-10\\
2.146	1.56433e-10\\
2.148	1.67347e-10\\
2.15	1.67347e-10\\
2.152	1.70985e-10\\
2.154	1.70985e-10\\
2.156	1.81899e-10\\
2.158	1.85537e-10\\
2.16	1.96451e-10\\
2.162	1.85537e-10\\
2.164	1.81899e-10\\
2.166	1.85537e-10\\
2.168	1.78261e-10\\
2.17	1.92813e-10\\
2.172	1.81899e-10\\
2.174	2.00089e-10\\
2.176	2.00089e-10\\
2.178	2.03727e-10\\
2.18	2.07365e-10\\
2.182	2.18279e-10\\
2.184	2.21917e-10\\
2.186	2.32831e-10\\
2.188	2.32831e-10\\
2.19	2.32831e-10\\
2.192	2.21917e-10\\
2.194	2.21917e-10\\
2.196	2.25555e-10\\
2.198	2.36469e-10\\
2.2	2.21917e-10\\
2.202	2.32831e-10\\
2.204	2.47383e-10\\
2.206	2.51021e-10\\
2.208	2.58296e-10\\
2.21	2.58296e-10\\
2.212	2.58296e-10\\
2.214	2.58296e-10\\
2.216	2.51021e-10\\
2.218	2.36469e-10\\
2.22	2.40107e-10\\
2.222	2.47383e-10\\
2.224	2.54659e-10\\
2.226	2.61934e-10\\
2.228	2.51021e-10\\
2.23	2.51021e-10\\
2.232	2.36469e-10\\
2.234	2.40107e-10\\
2.236	2.36469e-10\\
2.238	2.51021e-10\\
2.24	2.51021e-10\\
2.242	2.47383e-10\\
2.244	2.54659e-10\\
2.246	2.65572e-10\\
2.248	2.65572e-10\\
2.25	2.51021e-10\\
2.252	2.43745e-10\\
2.254	2.43745e-10\\
2.256	2.51021e-10\\
2.258	2.43745e-10\\
2.26	2.51021e-10\\
2.262	2.51021e-10\\
2.264	2.65572e-10\\
2.266	2.6921e-10\\
2.268	2.6921e-10\\
2.27	2.76486e-10\\
2.272	2.80124e-10\\
2.274	2.91038e-10\\
2.276	2.91038e-10\\
2.278	3.01952e-10\\
2.28	3.20142e-10\\
2.282	3.01952e-10\\
2.284	3.0559e-10\\
2.286	3.0559e-10\\
2.288	3.01952e-10\\
2.29	2.91038e-10\\
2.292	3.01952e-10\\
2.294	3.16504e-10\\
2.296	3.01952e-10\\
2.298	2.98314e-10\\
2.3	3.12866e-10\\
2.302	3.01952e-10\\
2.304	2.98314e-10\\
2.306	2.91038e-10\\
2.308	3.01952e-10\\
2.31	3.01952e-10\\
2.312	2.98314e-10\\
2.314	3.09228e-10\\
2.316	3.09228e-10\\
2.318	3.09228e-10\\
2.32	2.91038e-10\\
2.322	2.94676e-10\\
2.324	2.91038e-10\\
2.326	2.94676e-10\\
2.328	2.94676e-10\\
2.33	2.91038e-10\\
2.332	2.91038e-10\\
2.334	2.80124e-10\\
2.336	2.80124e-10\\
2.338	3.01952e-10\\
2.34	2.98314e-10\\
2.342	2.874e-10\\
2.344	2.76486e-10\\
2.346	2.76486e-10\\
2.348	2.72848e-10\\
2.35	2.76486e-10\\
2.352	2.83762e-10\\
2.354	2.76486e-10\\
2.356	2.80124e-10\\
2.358	2.80124e-10\\
2.36	2.91038e-10\\
2.362	2.874e-10\\
2.364	2.72848e-10\\
2.366	2.72848e-10\\
2.368	2.58296e-10\\
2.37	2.58296e-10\\
2.372	2.72848e-10\\
2.374	2.76486e-10\\
2.376	2.83762e-10\\
2.378	2.83762e-10\\
2.38	2.72848e-10\\
2.382	2.72848e-10\\
2.384	2.61934e-10\\
2.386	2.76486e-10\\
2.388	2.76486e-10\\
2.39	2.98314e-10\\
2.392	2.98314e-10\\
2.394	3.09228e-10\\
2.396	3.20142e-10\\
2.398	3.20142e-10\\
2.4	3.20142e-10\\
2.402	3.31056e-10\\
2.404	3.45608e-10\\
2.406	3.31056e-10\\
2.408	3.31056e-10\\
2.41	3.31056e-10\\
2.412	3.4197e-10\\
2.414	3.56522e-10\\
2.416	3.56522e-10\\
2.418	3.56522e-10\\
2.42	3.67436e-10\\
2.422	3.56522e-10\\
2.424	3.49246e-10\\
2.426	3.71074e-10\\
2.428	3.6016e-10\\
2.43	3.56522e-10\\
2.432	3.49246e-10\\
2.434	3.56522e-10\\
2.436	3.6016e-10\\
2.438	3.74712e-10\\
2.44	3.74712e-10\\
2.442	3.85626e-10\\
2.444	3.74712e-10\\
2.446	3.6016e-10\\
2.448	3.6016e-10\\
2.45	3.74712e-10\\
2.452	3.74712e-10\\
2.454	3.7835e-10\\
2.456	3.7835e-10\\
2.458	3.63798e-10\\
2.46	3.6016e-10\\
2.462	3.6016e-10\\
2.464	3.7835e-10\\
2.466	3.74712e-10\\
2.468	3.74712e-10\\
2.47	3.89264e-10\\
2.472	4.00178e-10\\
2.474	4.03816e-10\\
2.476	4.00178e-10\\
2.478	4.11092e-10\\
2.48	4.22006e-10\\
2.482	4.32919e-10\\
2.484	4.25644e-10\\
2.486	4.32919e-10\\
2.488	4.29281e-10\\
2.49	4.29281e-10\\
2.492	4.25644e-10\\
2.494	4.11092e-10\\
2.496	4.03816e-10\\
2.498	4.11092e-10\\
2.5	4.1473e-10\\
2.502	4.1473e-10\\
2.504	4.11092e-10\\
2.506	3.92902e-10\\
2.508	3.9654e-10\\
2.51	3.9654e-10\\
2.512	3.9654e-10\\
2.514	4.07454e-10\\
2.516	3.92902e-10\\
2.518	4.07454e-10\\
2.52	3.92902e-10\\
2.522	3.92902e-10\\
2.524	4.07454e-10\\
2.526	4.11092e-10\\
2.528	3.9654e-10\\
2.53	3.9654e-10\\
2.532	3.81988e-10\\
2.534	3.92902e-10\\
2.536	3.92902e-10\\
2.538	4.03816e-10\\
2.54	3.81988e-10\\
2.542	3.92902e-10\\
2.544	3.92902e-10\\
2.546	3.81988e-10\\
2.548	3.85626e-10\\
2.55	3.7835e-10\\
2.552	3.67436e-10\\
2.554	3.63798e-10\\
2.556	3.52884e-10\\
2.558	3.56522e-10\\
2.56	3.56522e-10\\
2.562	3.7835e-10\\
2.564	3.89264e-10\\
2.566	3.89264e-10\\
2.568	3.89264e-10\\
2.57	3.85626e-10\\
2.572	3.89264e-10\\
2.574	4.07454e-10\\
2.576	4.11092e-10\\
2.578	4.11092e-10\\
2.58	4.11092e-10\\
2.582	4.11092e-10\\
2.584	4.25644e-10\\
2.586	4.25644e-10\\
2.588	4.22006e-10\\
2.59	4.25644e-10\\
2.592	4.25644e-10\\
2.594	4.25644e-10\\
2.596	4.25644e-10\\
2.598	4.36557e-10\\
2.6	4.36557e-10\\
2.602	4.58385e-10\\
2.604	4.69299e-10\\
2.606	4.83851e-10\\
2.608	4.72937e-10\\
2.61	4.58385e-10\\
2.612	4.72937e-10\\
2.614	4.62023e-10\\
2.616	4.72937e-10\\
2.618	4.58385e-10\\
2.62	4.58385e-10\\
2.622	4.62023e-10\\
2.624	4.51109e-10\\
2.626	4.62023e-10\\
2.628	4.72937e-10\\
2.63	4.72937e-10\\
2.632	4.58385e-10\\
2.634	4.58385e-10\\
2.636	4.62023e-10\\
2.638	4.54747e-10\\
2.64	4.51109e-10\\
2.642	4.25644e-10\\
2.644	4.36557e-10\\
2.646	4.36557e-10\\
2.648	4.51109e-10\\
2.65	4.51109e-10\\
2.652	4.36557e-10\\
2.654	4.36557e-10\\
2.656	4.32919e-10\\
2.658	4.22006e-10\\
2.66	4.22006e-10\\
2.662	4.22006e-10\\
2.664	4.00178e-10\\
2.666	4.07454e-10\\
2.668	4.00178e-10\\
2.67	4.00178e-10\\
2.672	4.03816e-10\\
2.674	4.00178e-10\\
2.676	4.03816e-10\\
2.678	3.9654e-10\\
2.68	4.03816e-10\\
2.682	3.9654e-10\\
2.684	4.03816e-10\\
2.686	4.07454e-10\\
2.688	4.07454e-10\\
2.69	4.18368e-10\\
2.692	4.18368e-10\\
2.694	4.29281e-10\\
2.696	4.29281e-10\\
2.698	4.43833e-10\\
2.7	4.36557e-10\\
2.702	4.47471e-10\\
2.704	4.47471e-10\\
2.706	4.51109e-10\\
2.708	4.65661e-10\\
2.71	4.76575e-10\\
2.712	4.62023e-10\\
2.714	4.58385e-10\\
2.716	4.76575e-10\\
2.718	4.87489e-10\\
2.72	4.98403e-10\\
2.722	4.87489e-10\\
2.724	4.87489e-10\\
2.726	4.87489e-10\\
2.728	4.91127e-10\\
2.73	4.91127e-10\\
2.732	5.05679e-10\\
2.734	5.02041e-10\\
2.736	4.87489e-10\\
2.738	4.91127e-10\\
2.74	4.80213e-10\\
2.742	4.91127e-10\\
2.744	4.91127e-10\\
2.746	4.80213e-10\\
2.748	4.80213e-10\\
2.75	4.87489e-10\\
2.752	4.91127e-10\\
2.754	4.80213e-10\\
2.756	4.62023e-10\\
2.758	4.65661e-10\\
2.76	4.54747e-10\\
2.762	4.62023e-10\\
2.764	4.51109e-10\\
2.766	4.51109e-10\\
2.768	4.29281e-10\\
2.77	4.29281e-10\\
2.772	4.40195e-10\\
2.774	4.51109e-10\\
2.776	4.54747e-10\\
2.778	4.47471e-10\\
2.78	4.51109e-10\\
2.782	4.36557e-10\\
2.784	4.25644e-10\\
2.786	4.32919e-10\\
2.788	4.36557e-10\\
2.79	4.25644e-10\\
2.792	4.25644e-10\\
2.794	4.11092e-10\\
2.796	4.07454e-10\\
2.798	4.00178e-10\\
2.8	3.9654e-10\\
2.802	3.89264e-10\\
2.804	3.89264e-10\\
2.806	3.81988e-10\\
2.808	3.71074e-10\\
2.81	3.7835e-10\\
2.812	3.7835e-10\\
2.814	3.6016e-10\\
2.816	3.63798e-10\\
2.818	3.6016e-10\\
2.82	3.6016e-10\\
2.822	3.6016e-10\\
2.824	3.71074e-10\\
2.826	3.63798e-10\\
2.828	3.71074e-10\\
2.83	3.74712e-10\\
2.832	3.71074e-10\\
2.834	3.71074e-10\\
2.836	3.71074e-10\\
2.838	3.71074e-10\\
2.84	3.6016e-10\\
2.842	3.6016e-10\\
2.844	3.56522e-10\\
2.846	3.52884e-10\\
2.848	3.56522e-10\\
2.85	3.6016e-10\\
2.852	3.49246e-10\\
2.854	3.49246e-10\\
2.856	3.56522e-10\\
2.858	3.4197e-10\\
2.86	3.4197e-10\\
2.862	3.27418e-10\\
2.864	3.20142e-10\\
2.866	3.27418e-10\\
2.868	3.2378e-10\\
2.87	3.38332e-10\\
2.872	3.4197e-10\\
2.874	3.38332e-10\\
2.876	3.4197e-10\\
2.878	3.2378e-10\\
2.88	3.20142e-10\\
2.882	3.20142e-10\\
2.884	3.16504e-10\\
2.886	3.2378e-10\\
2.888	3.34694e-10\\
2.89	3.4197e-10\\
2.892	3.49246e-10\\
2.894	3.52884e-10\\
2.896	3.52884e-10\\
2.898	3.67436e-10\\
2.9	3.67436e-10\\
2.902	3.81988e-10\\
2.904	3.7835e-10\\
2.906	3.85626e-10\\
2.908	3.85626e-10\\
2.91	3.89264e-10\\
2.912	3.89264e-10\\
2.914	3.9654e-10\\
2.916	3.9654e-10\\
2.918	3.92902e-10\\
2.92	3.85626e-10\\
2.922	3.89264e-10\\
2.924	3.89264e-10\\
2.926	3.81988e-10\\
2.928	3.7835e-10\\
2.93	3.85626e-10\\
2.932	3.85626e-10\\
2.934	3.81988e-10\\
2.936	3.74712e-10\\
2.938	3.74712e-10\\
2.94	3.7835e-10\\
2.942	3.92902e-10\\
2.944	3.92902e-10\\
2.946	3.92902e-10\\
2.948	3.74712e-10\\
2.95	3.89264e-10\\
2.952	3.92902e-10\\
2.954	3.92902e-10\\
2.956	3.9654e-10\\
2.958	3.92902e-10\\
2.96	3.9654e-10\\
2.962	4.07454e-10\\
2.964	3.9654e-10\\
2.966	3.9654e-10\\
2.968	4.03816e-10\\
2.97	4.03816e-10\\
2.972	3.9654e-10\\
2.974	3.9654e-10\\
2.976	4.00178e-10\\
2.978	3.92902e-10\\
2.98	3.9654e-10\\
2.982	4.1473e-10\\
2.984	4.22006e-10\\
2.986	4.18368e-10\\
2.988	4.29281e-10\\
2.99	4.32919e-10\\
2.992	4.29281e-10\\
2.994	4.36557e-10\\
2.996	4.51109e-10\\
2.998	4.51109e-10\\
3	4.36557e-10\\
3.002	4.36557e-10\\
3.004	4.36557e-10\\
3.006	4.25644e-10\\
3.008	4.29281e-10\\
3.01	4.29281e-10\\
3.012	4.22006e-10\\
3.014	4.32919e-10\\
3.016	4.22006e-10\\
3.018	4.25644e-10\\
3.02	4.25644e-10\\
3.022	4.36557e-10\\
3.024	4.25644e-10\\
3.026	4.22006e-10\\
3.028	4.22006e-10\\
3.03	4.18368e-10\\
3.032	4.32919e-10\\
3.034	4.32919e-10\\
3.036	4.25644e-10\\
3.038	4.22006e-10\\
3.04	4.07454e-10\\
3.042	4.07454e-10\\
3.044	4.07454e-10\\
3.046	4.25644e-10\\
3.048	4.11092e-10\\
3.05	4.07454e-10\\
3.052	4.11092e-10\\
3.054	4.07454e-10\\
3.056	4.11092e-10\\
3.058	4.1473e-10\\
3.06	4.18368e-10\\
3.062	4.18368e-10\\
3.064	4.22006e-10\\
3.066	4.11092e-10\\
3.068	4.22006e-10\\
3.07	4.36557e-10\\
3.072	4.36557e-10\\
3.074	4.36557e-10\\
3.076	4.22006e-10\\
3.078	4.22006e-10\\
3.08	4.25644e-10\\
3.082	4.40195e-10\\
3.084	4.36557e-10\\
3.086	4.36557e-10\\
3.088	4.40195e-10\\
3.09	4.43833e-10\\
3.092	4.36557e-10\\
3.094	4.29281e-10\\
3.096	4.29281e-10\\
3.098	4.29281e-10\\
3.1	4.36557e-10\\
3.102	4.40195e-10\\
3.104	4.22006e-10\\
3.106	4.29281e-10\\
3.108	4.40195e-10\\
3.11	4.32919e-10\\
3.112	4.29281e-10\\
3.114	4.29281e-10\\
3.116	4.36557e-10\\
3.118	4.40195e-10\\
3.12	4.32919e-10\\
3.122	4.47471e-10\\
3.124	4.47471e-10\\
3.126	4.47471e-10\\
3.128	4.58385e-10\\
3.13	4.58385e-10\\
3.132	4.58385e-10\\
3.134	4.54747e-10\\
3.136	4.47471e-10\\
3.138	4.54747e-10\\
3.14	4.47471e-10\\
3.142	4.51109e-10\\
3.144	4.43833e-10\\
3.146	4.51109e-10\\
3.148	4.54747e-10\\
3.15	4.47471e-10\\
3.152	4.47471e-10\\
3.154	4.36557e-10\\
3.156	4.40195e-10\\
3.158	4.47471e-10\\
3.16	4.40195e-10\\
3.162	4.36557e-10\\
3.164	4.40195e-10\\
3.166	4.40195e-10\\
3.168	4.43833e-10\\
3.17	4.47471e-10\\
3.172	4.47471e-10\\
3.174	4.51109e-10\\
3.176	4.51109e-10\\
3.178	4.47471e-10\\
3.18	4.43833e-10\\
3.182	4.40195e-10\\
3.184	4.43833e-10\\
3.186	4.47471e-10\\
3.188	4.47471e-10\\
3.19	4.51109e-10\\
3.192	4.43833e-10\\
3.194	4.47471e-10\\
3.196	4.47471e-10\\
3.198	4.62023e-10\\
3.2	4.51109e-10\\
3.202	4.47471e-10\\
3.204	4.43833e-10\\
3.206	4.43833e-10\\
3.208	4.32919e-10\\
3.21	4.32919e-10\\
3.212	4.29281e-10\\
3.214	4.29281e-10\\
3.216	4.32919e-10\\
3.218	4.32919e-10\\
3.22	4.40195e-10\\
3.222	4.43833e-10\\
3.224	4.43833e-10\\
3.226	4.43833e-10\\
3.228	4.51109e-10\\
3.23	4.51109e-10\\
3.232	4.51109e-10\\
3.234	4.47471e-10\\
3.236	4.51109e-10\\
3.238	4.51109e-10\\
3.24	4.54747e-10\\
3.242	4.51109e-10\\
3.244	4.65661e-10\\
3.246	4.65661e-10\\
3.248	4.54747e-10\\
3.25	4.58385e-10\\
3.252	4.54747e-10\\
3.254	4.54747e-10\\
3.256	4.47471e-10\\
3.258	4.47471e-10\\
3.26	4.54747e-10\\
3.262	4.51109e-10\\
3.264	4.54747e-10\\
3.266	4.54747e-10\\
3.268	4.54747e-10\\
3.27	4.54747e-10\\
3.272	4.58385e-10\\
3.274	4.54747e-10\\
3.276	4.62023e-10\\
3.278	4.58385e-10\\
3.28	4.58385e-10\\
3.282	4.58385e-10\\
3.284	4.58385e-10\\
3.286	4.58385e-10\\
3.288	4.58385e-10\\
3.29	4.72937e-10\\
3.292	4.69299e-10\\
3.294	4.72937e-10\\
3.296	4.69299e-10\\
3.298	4.72937e-10\\
3.3	4.72937e-10\\
3.302	4.72937e-10\\
3.304	4.69299e-10\\
3.306	4.54747e-10\\
3.308	4.62023e-10\\
3.31	4.62023e-10\\
3.312	4.54747e-10\\
3.314	4.51109e-10\\
3.316	4.47471e-10\\
3.318	4.54747e-10\\
3.32	4.51109e-10\\
3.322	4.51109e-10\\
3.324	4.51109e-10\\
3.326	4.47471e-10\\
3.328	4.47471e-10\\
3.33	4.47471e-10\\
3.332	4.47471e-10\\
3.334	4.43833e-10\\
3.336	4.40195e-10\\
3.338	4.43833e-10\\
3.34	4.58385e-10\\
3.342	4.47471e-10\\
3.344	4.51109e-10\\
3.346	4.47471e-10\\
3.348	4.47471e-10\\
3.35	4.62023e-10\\
3.352	4.58385e-10\\
3.354	4.65661e-10\\
3.356	4.62023e-10\\
3.358	4.62023e-10\\
3.36	4.58385e-10\\
3.362	4.58385e-10\\
3.364	4.54747e-10\\
3.366	4.54747e-10\\
3.368	4.47471e-10\\
3.37	4.47471e-10\\
3.372	4.62023e-10\\
3.374	4.58385e-10\\
3.376	4.62023e-10\\
3.378	4.54747e-10\\
3.38	4.54747e-10\\
3.382	4.51109e-10\\
3.384	4.62023e-10\\
3.386	4.58385e-10\\
3.388	4.58385e-10\\
3.39	4.54747e-10\\
3.392	4.51109e-10\\
3.394	4.54747e-10\\
3.396	4.47471e-10\\
3.398	4.40195e-10\\
3.4	4.51109e-10\\
3.402	4.47471e-10\\
3.404	4.40195e-10\\
3.406	4.47471e-10\\
3.408	4.51109e-10\\
3.41	4.54747e-10\\
3.412	4.69299e-10\\
3.414	4.65661e-10\\
3.416	4.51109e-10\\
3.418	4.65661e-10\\
3.42	4.65661e-10\\
3.422	4.51109e-10\\
3.424	4.58385e-10\\
3.426	4.65661e-10\\
3.428	4.62023e-10\\
3.43	4.65661e-10\\
3.432	4.62023e-10\\
3.434	4.58385e-10\\
3.436	4.62023e-10\\
3.438	4.62023e-10\\
3.44	4.47471e-10\\
3.442	4.62023e-10\\
3.444	4.62023e-10\\
3.446	4.62023e-10\\
3.448	4.72937e-10\\
3.45	4.69299e-10\\
3.452	4.72937e-10\\
3.454	4.69299e-10\\
3.456	4.72937e-10\\
3.458	4.72937e-10\\
3.46	4.65661e-10\\
3.462	4.69299e-10\\
3.464	4.65661e-10\\
3.466	4.65661e-10\\
3.468	4.69299e-10\\
3.47	4.62023e-10\\
3.472	4.69299e-10\\
3.474	4.69299e-10\\
3.476	4.72937e-10\\
3.478	4.80213e-10\\
3.48	4.76575e-10\\
3.482	4.80213e-10\\
3.484	4.80213e-10\\
3.486	4.87489e-10\\
3.488	4.91127e-10\\
3.49	4.94765e-10\\
3.492	4.94765e-10\\
3.494	4.87489e-10\\
3.496	4.91127e-10\\
3.498	4.94765e-10\\
3.5	4.83851e-10\\
3.502	4.91127e-10\\
3.504	4.87489e-10\\
3.506	4.94765e-10\\
3.508	5.02041e-10\\
3.51	5.02041e-10\\
3.512	5.02041e-10\\
3.514	5.02041e-10\\
3.516	4.98403e-10\\
3.518	4.98403e-10\\
3.52	4.98403e-10\\
3.522	5.09317e-10\\
3.524	5.12955e-10\\
3.526	4.98403e-10\\
3.528	4.98403e-10\\
3.53	5.02041e-10\\
3.532	4.98403e-10\\
3.534	5.02041e-10\\
3.536	5.02041e-10\\
3.538	5.12955e-10\\
3.54	5.12955e-10\\
3.542	4.98403e-10\\
3.544	5.02041e-10\\
3.546	4.91127e-10\\
3.548	5.02041e-10\\
3.55	5.05679e-10\\
3.552	5.09317e-10\\
3.554	5.02041e-10\\
3.556	5.16593e-10\\
3.558	5.16593e-10\\
3.56	5.05679e-10\\
3.562	5.16593e-10\\
3.564	5.16593e-10\\
3.566	5.20231e-10\\
3.568	5.05679e-10\\
3.57	5.05679e-10\\
3.572	5.05679e-10\\
3.574	5.05679e-10\\
3.576	5.09317e-10\\
3.578	5.09317e-10\\
3.58	5.09317e-10\\
3.582	5.09317e-10\\
3.584	5.12955e-10\\
3.586	5.16593e-10\\
3.588	5.12955e-10\\
3.59	5.09317e-10\\
3.592	5.23869e-10\\
3.594	5.20231e-10\\
3.596	5.16593e-10\\
3.598	5.20231e-10\\
3.6	5.16593e-10\\
3.602	5.12955e-10\\
3.604	5.05679e-10\\
3.606	5.12955e-10\\
3.608	5.09317e-10\\
3.61	5.05679e-10\\
3.612	5.02041e-10\\
3.614	5.02041e-10\\
3.616	4.98403e-10\\
3.618	4.98403e-10\\
3.62	4.98403e-10\\
3.622	4.94765e-10\\
3.624	4.94765e-10\\
3.626	4.87489e-10\\
3.628	4.76575e-10\\
3.63	4.76575e-10\\
3.632	4.80213e-10\\
3.634	4.94765e-10\\
3.636	4.80213e-10\\
3.638	4.98403e-10\\
3.64	4.98403e-10\\
3.642	4.98403e-10\\
3.644	4.94765e-10\\
3.646	5.05679e-10\\
3.648	4.94765e-10\\
3.65	5.05679e-10\\
3.652	4.94765e-10\\
3.654	4.94765e-10\\
3.656	4.98403e-10\\
3.658	5.05679e-10\\
3.66	5.05679e-10\\
3.662	5.09317e-10\\
3.664	5.20231e-10\\
3.666	5.23869e-10\\
3.668	5.27507e-10\\
3.67	5.27507e-10\\
3.672	5.23869e-10\\
3.674	5.23869e-10\\
3.676	5.09317e-10\\
3.678	5.20231e-10\\
3.68	5.09317e-10\\
3.682	5.23869e-10\\
3.684	5.16593e-10\\
3.686	5.20231e-10\\
3.688	5.16593e-10\\
3.69	5.20231e-10\\
3.692	5.02041e-10\\
3.694	5.02041e-10\\
3.696	5.20231e-10\\
3.698	5.20231e-10\\
3.7	5.16593e-10\\
3.702	5.16593e-10\\
3.704	5.20231e-10\\
3.706	5.34783e-10\\
3.708	5.31145e-10\\
3.71	5.20231e-10\\
3.712	5.20231e-10\\
3.714	5.27507e-10\\
3.716	5.27507e-10\\
3.718	5.23869e-10\\
3.72	5.34783e-10\\
3.722	5.38421e-10\\
3.724	5.27507e-10\\
3.726	5.12955e-10\\
3.728	5.23869e-10\\
3.73	5.27507e-10\\
3.732	5.12955e-10\\
3.734	5.27507e-10\\
3.736	5.31145e-10\\
3.738	5.27507e-10\\
3.74	5.20231e-10\\
3.742	5.23869e-10\\
3.744	5.34783e-10\\
3.746	5.42059e-10\\
3.748	5.42059e-10\\
3.75	5.56611e-10\\
3.752	5.56611e-10\\
3.754	5.52973e-10\\
3.756	5.56611e-10\\
3.758	5.56611e-10\\
3.76	5.52973e-10\\
3.762	5.56611e-10\\
3.764	5.71163e-10\\
3.766	5.71163e-10\\
3.768	5.74801e-10\\
3.77	5.74801e-10\\
3.772	5.89353e-10\\
3.774	5.89353e-10\\
3.776	6.03904e-10\\
3.778	6.18456e-10\\
3.78	6.18456e-10\\
3.782	6.07542e-10\\
3.784	6.07542e-10\\
3.786	6.1118e-10\\
3.788	6.07542e-10\\
3.79	6.07542e-10\\
3.792	6.03904e-10\\
3.794	6.1118e-10\\
3.796	6.03904e-10\\
3.798	5.96629e-10\\
3.8	5.92991e-10\\
3.802	5.78439e-10\\
3.804	5.82077e-10\\
3.806	5.74801e-10\\
3.808	5.74801e-10\\
3.81	5.74801e-10\\
3.812	5.89353e-10\\
3.814	6.03904e-10\\
3.816	6.03904e-10\\
3.818	6.03904e-10\\
3.82	5.96629e-10\\
3.822	5.85715e-10\\
3.824	5.96629e-10\\
3.826	5.96629e-10\\
3.828	5.85715e-10\\
3.83	5.96629e-10\\
3.832	5.92991e-10\\
3.834	5.74801e-10\\
3.836	5.78439e-10\\
3.838	5.74801e-10\\
3.84	5.71163e-10\\
3.842	5.67525e-10\\
3.844	5.74801e-10\\
3.846	5.67525e-10\\
3.848	5.67525e-10\\
3.85	5.56611e-10\\
3.852	5.52973e-10\\
3.854	5.52973e-10\\
3.856	5.52973e-10\\
3.858	5.60249e-10\\
3.86	5.49335e-10\\
3.862	5.52973e-10\\
3.864	5.56611e-10\\
3.866	5.56611e-10\\
3.868	5.71163e-10\\
3.87	5.82077e-10\\
3.872	5.92991e-10\\
3.874	5.78439e-10\\
3.876	5.71163e-10\\
3.878	5.82077e-10\\
3.88	5.78439e-10\\
3.882	5.63887e-10\\
3.884	5.74801e-10\\
3.886	5.82077e-10\\
3.888	5.78439e-10\\
3.89	5.74801e-10\\
3.892	5.78439e-10\\
3.894	5.74801e-10\\
3.896	5.82077e-10\\
3.898	5.82077e-10\\
3.9	5.82077e-10\\
3.902	5.78439e-10\\
3.904	5.82077e-10\\
3.906	5.74801e-10\\
3.908	5.74801e-10\\
3.91	5.74801e-10\\
3.912	5.78439e-10\\
3.914	5.63887e-10\\
3.916	5.60249e-10\\
3.918	5.60249e-10\\
3.92	5.45697e-10\\
3.922	5.60249e-10\\
3.924	5.60249e-10\\
3.926	5.74801e-10\\
3.928	5.78439e-10\\
3.93	5.74801e-10\\
3.932	5.74801e-10\\
3.934	5.74801e-10\\
3.936	5.82077e-10\\
3.938	5.85715e-10\\
3.94	5.85715e-10\\
3.942	5.85715e-10\\
3.944	5.82077e-10\\
3.946	5.85715e-10\\
3.948	5.96629e-10\\
3.95	6.07542e-10\\
3.952	6.25732e-10\\
3.954	6.33008e-10\\
3.956	6.1118e-10\\
3.958	6.22094e-10\\
3.96	6.1118e-10\\
3.962	6.25732e-10\\
3.964	6.1118e-10\\
3.966	6.25732e-10\\
3.968	6.1118e-10\\
3.97	6.03904e-10\\
3.972	6.00267e-10\\
3.974	5.85715e-10\\
3.976	5.89353e-10\\
3.978	5.85715e-10\\
3.98	5.82077e-10\\
3.982	5.78439e-10\\
3.984	5.78439e-10\\
3.986	5.78439e-10\\
3.988	5.85715e-10\\
3.99	6.00267e-10\\
3.992	6.07542e-10\\
3.994	6.00267e-10\\
3.996	6.00267e-10\\
3.998	6.00267e-10\\
4	6.1118e-10\\
};
\addlegendentry{Energy Diff};

\end{axis}
\end{tikzpicture}%
	\caption{Showing the angle evolution for the spinning top problem over 4 seconds.}
	\label{fig:forwardData}
\end{figure}
\fi
\iftikz
\begin{figure}[H]
	\centering
	\setlength\figureheight{7cm} 
	\setlength\figurewidth{14cm}
	% This file was created by matlab2tikz.
% Minimal pgfplots version: 1.3
%
%The latest updates can be retrieved from
%  http://www.mathworks.com/matlabcentral/fileexchange/22022-matlab2tikz
%where you can also make suggestions and rate matlab2tikz.
%
\definecolor{mycolor1}{rgb}{0.00000,0.44700,0.74100}%
\definecolor{mycolor2}{rgb}{0.85000,0.32500,0.09800}%
\definecolor{mycolor3}{rgb}{0.92900,0.69400,0.12500}%
%
\begin{tikzpicture}

\begin{axis}[%
width=0.95092\figurewidth,
height=\figureheight,
at={(0\figurewidth,0\figureheight)},
scale only axis,
xmin=0,
xmax=4,
xtick={0,0.5,1,1.5,2,2.5,3,3.5,4},
xticklabels={{4},{3.5},{3},{2.5},{2},{1.5},{1},{0.5},{0}},
xlabel={Time (s)},
ymin=0,
ymax=14000,
ylabel={Degrees},
title style={font=\bfseries},
title={Top Spin [4,0] (s)},
legend style={legend cell align=left,align=left,draw=white!15!black},
title style={font=\small},ticklabel style={font=\tiny}
]
\addplot [color=mycolor1,solid]
  table[row sep=crcr]{%
0	2457.65834446344\\
0.00200300450676014	2457.47832112421\\
0.00400600901352028	2457.32448195622\\
0.00600901352028042	2457.19545186075\\
0.00801201802704056	2457.08939737288\\
0.0100150225338007	2457.00454232342\\
0.0120180270405608	2456.9388813601\\
0.014021031547321	2456.89012265173\\
0.0160240360540811	2456.85603166292\\
0.0180270405608413	2456.83403008359\\
0.0200300450676014	2456.82159689943\\
0.0220330495743615	2456.81603920882\\
0.0240360540811217	2456.81460681433\\
0.0260390585878818	2456.81449222277\\
0.028042063094642	2456.81288794095\\
0.0300450676014021	2456.80704377144\\
0.0320480721081622	2456.79415222105\\
0.0340510766149224	2456.7716349797\\
0.0360540811216825	2456.73679914575\\
0.0380570856284427	2456.68718100069\\
0.0400600901352028	2456.62054600912\\
0.0420630946419629	2456.53460233985\\
0.0440660991487231	2456.42740193638\\
0.0460691036554832	2456.29699674221\\
0.0480721081622434	2456.14183977129\\
0.0500751126690035	2455.96038403757\\
0.0520781171757636	2455.75131173813\\
0.0540811216825238	2455.51353425315\\
0.0560841261892839	2455.24613485016\\
0.0580871306960441	2454.94831138825\\
0.0600901352028042	2454.61949090963\\
0.0620931397095643	2454.25915775227\\
0.0640961442163245	2453.86708273306\\
0.0660991487230846	2453.4429793731\\
0.0681021532298448	2452.98690496818\\
0.0701051577366049	2452.49885951829\\
0.0721081622433651	2451.97895761499\\
0.0741111667501252	2451.42748573717\\
0.0761141712568853	2450.84467306797\\
0.0781171757636455	2450.23092067782\\
0.0801201802704056	2449.5866296372\\
0.0821231847771658	2448.91220101655\\
0.0841261892839259	2448.20820777367\\
0.086129193790686	2447.47516557058\\
0.0881321982974462	2446.71353277351\\
0.0901352028042063	2445.92388234026\\
0.0921382073109664	2445.10684452441\\
0.0941412118177266	2444.26287769218\\
0.0961442163244867	2443.39255480138\\
0.0981472208312469	2442.49650610557\\
0.100150225338007	2441.575189971\\
0.102153229844767	2440.62912205968\\
0.104156234351527	2439.65887532941\\
0.106159238858287	2438.66490814641\\
0.108162243365048	2437.64773617272\\
0.110165247871808	2436.60776047878\\
0.112168252378568	2435.54538213504\\
0.114171256885328	2434.46111680354\\
0.116174261392088	2433.35525096316\\
0.118177265898848	2432.22824298013\\
0.120180270405608	2431.08032203759\\
0.122183274912369	2429.9118319102\\
0.124186279419129	2428.72311637264\\
0.126189283925889	2427.51434731225\\
0.128192288432649	2426.28581120794\\
0.130195292939409	2425.03767994702\\
0.132198297446169	2423.77018271263\\
0.134201301952929	2422.48343409633\\
0.13620430645969	2421.17760598545\\
0.13820731096645	2419.85281297155\\
0.14021031547321	2418.5091123504\\
0.14221331997997	2417.14650412202\\
0.14421632448673	2415.76516017374\\
0.14621932899349	2414.36502320978\\
0.14822233350025	2412.94609323014\\
0.150225338007011	2411.50837023482\\
0.152228342513771	2410.05185422382\\
0.154231347020531	2408.5763733098\\
0.156234351527291	2407.08192749276\\
0.158237356034051	2405.56840218114\\
0.160240360540811	2404.03574007916\\
0.162243365047571	2402.48376929949\\
0.164246369554332	2400.91243254635\\
0.166249374061092	2399.32144334083\\
0.168252378567852	2397.71080168293\\
0.170255383074612	2396.08033568533\\
0.172258387581372	2394.4298161649\\
0.174261392088132	2392.75912853008\\
0.176264396594892	2391.06810089353\\
0.178267401101652	2389.35661866369\\
0.180270405608413	2387.62445265745\\
0.182273410115173	2385.87160287481\\
0.184276414621933	2384.09778283686\\
0.186279419128693	2382.30293524784\\
0.188282423635453	2380.48706010773\\
0.190285428142213	2378.6499855292\\
0.192288432648973	2376.79171151225\\
0.194291437155734	2374.9121807611\\
0.196294441662494	2373.01150786732\\
0.198297446169254	2371.08969283089\\
0.200300450676014	2369.14685024338\\
0.202303455182774	2367.18315199213\\
0.204306459689534	2365.19882726025\\
0.206309464196294	2363.19404793509\\
0.208312468703055	2361.16927238288\\
0.210315473209815	2359.12478708251\\
0.212318477716575	2357.06105040023\\
0.214321482223335	2354.97863529383\\
0.216324486730095	2352.8781147211\\
0.218327491236855	2350.7601762314\\
0.220330495743615	2348.62550737408\\
0.222333500250376	2346.47496758583\\
0.224336504757136	2344.30947359914\\
0.226339509263896	2342.12994214646\\
0.228342513770656	2339.93746184761\\
0.230345518277416	2337.7331213224\\
0.232348522784176	2335.51812378221\\
0.234351527290936	2333.29361514261\\
0.236354531797697	2331.06097050233\\
0.238357536304457	2328.82156496006\\
0.240360540811217	2326.57665902296\\
0.242363545317977	2324.32774238129\\
0.244366549824737	2322.07619013376\\
0.246369554331497	2319.82354926642\\
0.248372558838257	2317.57119487799\\
0.250375563345018	2315.32067395449\\
0.252378567851778	2313.07336159465\\
0.254381572358538	2310.83074748873\\
0.256384576865298	2308.59414943966\\
0.258387581372058	2306.36494254614\\
0.260390585878818	2304.14455920267\\
0.262393590385578	2301.93408802906\\
0.264396594892339	2299.734904124\\
0.266399599399099	2297.54803881155\\
0.268402603905859	2295.37463800728\\
0.270405608412619	2293.21567573945\\
0.272408612919379	2291.0721260363\\
0.274411617426139	2288.94479103876\\
0.276414621932899	2286.83453018352\\
0.27841762643966	2284.74197372414\\
0.28042063094642	2282.66780920999\\
0.28242363545318	2280.61260959885\\
0.28442663995994	2278.57683325697\\
0.2864296444667	2276.5608812548\\
0.28843264897346	2274.56521195858\\
0.29043565348022	2272.58999725565\\
0.29243865798698	2270.63558092068\\
0.294441662493741	2268.70207754523\\
0.296444667000501	2266.78965901664\\
0.298447671507261	2264.8983826307\\
0.300450676014021	2263.02836297895\\
0.302453680520781	2261.17948546984\\
0.304456685027541	2259.35186469493\\
0.306459689534301	2257.54532876688\\
0.308462694041062	2255.7598776857\\
0.310465698547822	2253.99539685981\\
0.312468703054582	2252.25177169767\\
0.314471707561342	2250.52883031193\\
0.316474712068102	2248.82645811104\\
0.318477716574862	2247.14454050343\\
0.320480721081622	2245.482848306\\
0.322483725588383	2243.84126692717\\
0.324486730095143	2242.21962447961\\
0.326489734601903	2240.61774907598\\
0.328492739108663	2239.03552612473\\
0.330495743615423	2237.47284103429\\
0.332498748122183	2235.92946462154\\
0.334501752628943	2234.40533959072\\
0.336504757135704	2232.90035135025\\
0.338507761642464	2231.41432801279\\
0.340510766149224	2229.94732687414\\
0.342513770655984	2228.49911875117\\
0.344516775162744	2227.06976093966\\
0.346519779669504	2225.65919614382\\
0.348522784176264	2224.26742436367\\
0.350525788683025	2222.8944455992\\
0.352528793189785	2221.54025985041\\
0.354531797696545	2220.20492441308\\
0.356534802203305	2218.88855387876\\
0.358537806710065	2217.59120554325\\
0.360540811216825	2216.31305129387\\
0.362543815723585	2215.05414842641\\
0.364546820230346	2213.81472612398\\
0.366549824737106	2212.59495627393\\
0.368552829243866	2211.39506805936\\
0.370555833750626	2210.21534795919\\
0.372558838257386	2209.05596786074\\
0.374561842764146	2207.91732883448\\
0.376564847270906	2206.79966006352\\
0.378567851777666	2205.70336261831\\
0.380570856284427	2204.62878027355\\
0.382573860791187	2203.57631409967\\
0.384576865297947	2202.54642246292\\
0.386579869804707	2201.53950643376\\
0.388582874311467	2200.5560816742\\
0.390585878818227	2199.59649195891\\
0.392588883324987	2198.66142483726\\
0.394591887831748	2197.75128137969\\
0.396594892338508	2196.86657724823\\
0.398597896845268	2196.00794269645\\
0.400600901352028	2195.17589338636\\
0.402603905858788	2194.37094497998\\
0.404606910365548	2193.59372773089\\
0.406609914872308	2192.84475730109\\
0.408612919379069	2192.12454935261\\
0.410615923885829	2191.43367684324\\
0.412618928392589	2190.77259813922\\
0.414621932899349	2190.14182890256\\
0.416624937406109	2189.5418274995\\
0.418627941912869	2188.97288040893\\
0.420630946419629	2188.4354459971\\
0.42263395092639	2187.92969615134\\
0.42463695543315	2187.45586005477\\
0.42663995993991	2187.01405229894\\
0.42864296444667	2186.60421558809\\
0.43064596895343	2186.22629262642\\
0.43264897346019	2185.88005423082\\
0.43465197796695	2185.56504203506\\
0.436654982473711	2185.28085496867\\
0.438657986980471	2185.02674818653\\
0.440660991487231	2184.80186225194\\
0.442663995993991	2184.60528043243\\
0.444667000500751	2184.43568492507\\
0.446670005007511	2184.29170063116\\
0.448673009514271	2184.1718951562\\
0.450676014021032	2184.07437773946\\
0.452679018527792	2183.99725762024\\
0.454682023034552	2183.93847215046\\
0.456685027541312	2183.89578679472\\
0.458688032048072	2183.86673783451\\
0.460691036554832	2183.84880425552\\
0.462694041061592	2183.83940774768\\
0.464697045568353	2183.83574081779\\
0.466700050075113	2183.83516786\\
0.468703054581873	2183.83482408532\\
0.470706059088633	2183.83190200057\\
0.472709063595393	2183.82359411254\\
0.474712068102153	2183.80726481537\\
0.476715072608913	2183.78016391166\\
0.478718077115673	2183.73988497867\\
0.480721081622434	2183.6838497063\\
0.482724086129194	2183.60993815073\\
0.484727090635954	2183.51585848077\\
0.486730095142714	2183.39983452726\\
0.488733099649474	2183.25991823369\\
0.490736104156234	2183.09461990979\\
0.492739108662994	2182.90250716108\\
0.494742113169755	2182.68231948041\\
0.496745117676515	2182.43302554375\\
0.498748122183275	2182.15382321019\\
0.500751126690035	2181.8439103388\\
0.502754131196795	2181.50277126758\\
0.504757135703555	2181.13006222185\\
0.506760140210315	2180.7254967227\\
0.508763144717076	2180.28890288281\\
0.510766149223836	2179.82028070218\\
0.512769153730596	2179.31974477235\\
0.514772158237356	2178.78746698067\\
0.516775162744116	2178.22367651027\\
0.518778167250876	2177.62865984002\\
0.520781171757636	2177.00293263196\\
0.522784176264397	2176.34678136498\\
0.524787180771157	2175.66072170109\\
0.526790185277917	2174.94526930231\\
0.528793189784677	2174.20099712643\\
0.530796194291437	2173.42830624392\\
0.532799198798197	2172.62788420412\\
0.534802203304957	2171.80024666905\\
0.536805207811718	2170.94590930073\\
0.538808212318478	2170.06544505696\\
0.540811216825238	2169.15942689552\\
0.542814221331998	2168.22837047843\\
0.544817225838758	2167.27279146771\\
0.546820230345518	2166.29320552537\\
0.548823234852278	2165.29007101766\\
0.550826239359039	2164.26390360658\\
0.552829243865799	2163.21510436259\\
0.554832248372559	2162.14413165193\\
0.556835252879319	2161.05138654506\\
0.558838257386079	2159.93715552087\\
0.560841261892839	2158.8018969456\\
0.562844266399599	2157.64589729814\\
0.56484727090636	2156.46950035318\\
0.56685027541312	2155.27287799805\\
0.56885327991988	2154.05637400743\\
0.57085628442664	2152.82010297287\\
0.5728592889334	2151.56440866906\\
0.57486229344016	2150.28934839178\\
0.57686529794692	2148.99515132414\\
0.57886830245368	2147.68187476192\\
0.580871306960441	2146.34963329668\\
0.582874311467201	2144.99854151998\\
0.584877315973961	2143.62859943183\\
0.586880320480721	2142.23992162377\\
0.588883324987481	2140.83239350425\\
0.590886329494241	2139.40612966483\\
0.592889334001002	2137.96107280973\\
0.594892338507762	2136.49710834739\\
0.596895343014522	2135.01423627781\\
0.598898347521282	2133.51234200944\\
0.600901352028042	2131.9913109507\\
0.602904356534802	2130.45114310161\\
0.604907361041562	2128.89155198327\\
0.606910365548322	2127.31253759566\\
0.608913370055083	2125.71392805147\\
0.610916374561843	2124.09549416756\\
0.612919379068603	2122.45717864817\\
0.614922383575363	2120.79880960594\\
0.616925388082123	2119.12021515355\\
0.618928392588883	2117.42116610786\\
0.620931397095643	2115.70160517312\\
0.622934401602404	2113.96136046197\\
0.624937406109164	2112.20031738285\\
0.626940410615924	2110.41830404844\\
0.628943415122684	2108.61520586716\\
0.630946419629444	2106.79102283902\\
0.632949424136204	2104.94564037247\\
0.634952428642964	2103.07905846749\\
0.636955433149725	2101.19127712409\\
0.638958437656485	2099.28229634227\\
0.640961442163245	2097.3522307136\\
0.642964446670005	2095.40119482962\\
0.644967451176765	2093.42941787345\\
0.646970455683525	2091.43701443667\\
0.648973460190285	2089.42438558971\\
0.650976464697046	2087.39181781148\\
0.652979469203806	2085.33976946822\\
0.654982473710566	2083.26869892616\\
0.656985478217326	2081.17906455154\\
0.658988482724086	2079.07166848527\\
0.660991487230846	2076.94708368515\\
0.662994491737606	2074.80611229208\\
0.664997496244367	2072.64961374277\\
0.667000500751127	2070.47850476968\\
0.669003505257887	2068.29375940107\\
0.671006509764647	2066.09652355252\\
0.673009514271407	2063.88782854807\\
0.675012518778167	2061.66893489486\\
0.677015523284927	2059.44116039584\\
0.679018527791687	2057.20565096657\\
0.681021532298448	2054.96389629735\\
0.683024536805208	2052.71727148686\\
0.685027541311968	2050.46715163382\\
0.687030545818728	2048.2150264285\\
0.689033550325488	2045.96232826538\\
0.691036554832248	2043.71054683474\\
0.693039559339009	2041.46105723528\\
0.695042563845769	2039.21540645304\\
0.697045568352529	2036.97496958674\\
0.699048572859289	2034.74112173509\\
0.701051577366049	2032.51523799678\\
0.703054581872809	2030.29857887898\\
0.705057586379569	2028.09240488883\\
0.707060590886329	2025.8979192377\\
0.70906359539309	2023.7162105454\\
0.71106659989985	2021.54831013596\\
0.71306960440661	2019.3953066292\\
0.71507260891337	2017.25800216602\\
0.71707561342013	2015.13725618313\\
0.71907861792689	2013.03381352564\\
0.72108162243365	2010.9484190387\\
0.723084626940411	2008.88158838433\\
0.725087631447171	2006.83395181609\\
0.727090635953931	2004.80585310867\\
0.729093640460691	2002.79780792407\\
0.731096644967451	2000.81016003698\\
0.733099649474211	1998.84302403896\\
0.735102653980971	1996.89680100046\\
0.737105658487732	1994.97154821726\\
0.739108662994492	1993.06738028092\\
0.741111667501252	1991.184411783\\
0.743114672008012	1989.32264272351\\
0.745117676514772	1987.48213039821\\
0.747120681021532	1985.66276021555\\
0.749123685528292	1983.86453217553\\
0.751126690035053	1982.08727439081\\
0.753129694541813	1980.3309868614\\
0.755132699048573	1978.59544040417\\
0.757135703555333	1976.88057772334\\
0.759138708062093	1975.18622693158\\
0.761141712568853	1973.51221614155\\
0.763144717075613	1971.85843076168\\
0.765147721582374	1970.22464160887\\
0.767150726089134	1968.61079138732\\
0.769153730595894	1967.01665091393\\
0.771156735102654	1965.44210559713\\
0.773159739609414	1963.88692625381\\
0.775162744116174	1962.35111288396\\
0.777165748622934	1960.83449360025\\
0.779168753129695	1959.33695381111\\
0.781171757636455	1957.858378925\\
0.783174762143215	1956.3987689419\\
0.785177766649975	1954.95795197449\\
0.787180771156735	1953.53598531853\\
0.789183775663495	1952.13275438248\\
0.791186780170255	1950.7483164621\\
0.793189784677016	1949.38267155741\\
0.795192789183776	1948.03587696417\\
0.797195793690536	1946.70798997818\\
0.799198798197296	1945.39901059942\\
0.801201802704056	1944.10916801102\\
0.803204807210816	1942.83851950876\\
0.805207811717576	1941.58723697998\\
0.807210816224337	1940.355492312\\
0.809213820731097	1939.14345739218\\
0.811216825237857	1937.95141869942\\
0.813219829744617	1936.77966271259\\
0.815222834251377	1935.62836131906\\
0.817225838758137	1934.49785829348\\
0.819228843264897	1933.38849741055\\
0.821231847771657	1932.30067974072\\
0.823234852278418	1931.23474905866\\
0.825237856785178	1930.19104913905\\
0.827240861291938	1929.1700956439\\
0.829243865798698	1928.17228964368\\
0.831246870305458	1927.1981468004\\
0.833249874812218	1926.24812548029\\
0.835252879318979	1925.32274134538\\
0.837255883825739	1924.42251005767\\
0.839258888332499	1923.54794727918\\
0.841261892839259	1922.69962596771\\
0.843264897346019	1921.87811908105\\
0.845267901852779	1921.083999577\\
0.847270906359539	1920.31772582179\\
0.849273910866299	1919.57998536478\\
0.85127691537306	1918.87117927643\\
0.85327991987982	1918.1919378103\\
0.85528292438658	1917.54266203686\\
0.85728592889334	1916.92386761811\\
0.8592889334001	1916.33601292031\\
0.86129193790686	1915.77932712656\\
0.86329494241362	1915.2542686031\\
0.865297946920381	1914.76100923727\\
0.867300951427141	1914.29972091641\\
0.869303955933901	1913.87040364052\\
0.871306960440661	1913.47311470538\\
0.873309964947421	1913.10762492787\\
0.875312969454181	1912.77364782908\\
0.877315973960941	1912.47078233858\\
0.879318978467702	1912.19845549855\\
0.881321982974462	1911.95592246387\\
0.883324987481222	1911.74220920629\\
0.885327991987982	1911.55634169755\\
0.887330996494742	1911.39694483894\\
0.889334001001502	1911.26264353176\\
0.891337005508262	1911.15171890263\\
0.893340010015023	1911.06245207815\\
0.895343014521783	1910.99278041026\\
0.897346019028543	1910.94058395512\\
0.899349023535303	1910.90351358578\\
0.901352028042063	1910.8791055837\\
0.903355032548823	1910.86483893461\\
0.905358037055583	1910.85802073684\\
0.907361041562344	1910.855900793\\
0.909364046069104	1910.85578620144\\
0.911367050575864	1910.85469758163\\
0.913370055082624	1910.84999932771\\
0.915373059589384	1910.83888394649\\
0.917376064096144	1910.81865853632\\
0.919379068602904	1910.78674478713\\
0.921382073109664	1910.74056438884\\
0.923385077616425	1910.6777682145\\
0.925388082123185	1910.59623632025\\
0.927391086629945	1910.49379146648\\
0.929394091136705	1910.36860018824\\
0.931397095643465	1910.21894361215\\
0.933400100150225	1910.04327475217\\
0.935403104656986	1909.84033310113\\
0.937406109163746	1909.6088581519\\
0.939409113670506	1909.347933172\\
0.941412118177266	1909.05681331629\\
0.943415122684026	1908.73469644387\\
0.945418127190786	1908.38123878005\\
0.947421131697546	1907.99609655017\\
0.949424136204306	1907.57898327531\\
0.951427140711067	1907.12984165971\\
0.953430145217827	1906.64872899914\\
0.955433149724587	1906.13575988515\\
0.957436154231347	1905.5911062051\\
0.959439158738107	1905.01505443788\\
0.961442163244867	1904.40794835816\\
0.963445167751627	1903.77024633218\\
0.965448172258388	1903.10234943039\\
0.967451176765148	1902.40471601904\\
0.969454181271908	1901.67791905592\\
0.971457185778668	1900.92247420304\\
0.973460190285428	1900.13883982664\\
0.975463194792188	1899.32770347607\\
0.977466199298948	1898.48952351757\\
0.979469203805709	1897.62493020472\\
0.981472208312469	1896.73438190375\\
0.983475212819229	1895.81850886823\\
0.985478217325989	1894.87776946441\\
0.987481221832749	1893.91273665007\\
0.989484226339509	1892.92386879145\\
0.991487230846269	1891.91168155058\\
0.993490235353029	1890.87663329367\\
0.99549323985979	1889.81906779542\\
0.99749624436655	1888.73955801361\\
0.99949924887331	1887.63833313137\\
1.00150225338007	1886.51585151493\\
1.00350525788683	1885.37245693897\\
1.00550826239359	1884.20843588238\\
1.00751126690035	1883.02407482407\\
1.00951427140711	1881.81960294714\\
1.01151727591387	1880.59536402629\\
1.01352028042063	1879.35147265306\\
1.01552328492739	1878.08815801057\\
1.01752628943415	1876.80559198617\\
1.01952929394091	1875.50388917141\\
1.02153229844767	1874.18322145364\\
1.02353530295443	1872.84358883284\\
1.02553830746119	1871.48510590059\\
1.02754131196795	1870.10782995265\\
1.02954431647471	1868.71176098904\\
1.03154732098147	1867.29689900974\\
1.03355032548823	1865.86330131054\\
1.03555332999499	1864.41079600411\\
1.03755633450175	1862.93944038621\\
1.03955933900851	1861.4491198653\\
1.04156234351527	1859.93977714558\\
1.04356534802203	1858.41129763551\\
1.04556835252879	1856.86350944775\\
1.04757135703555	1855.2963552865\\
1.04957436154231	1853.70966326445\\
1.05157736604907	1852.10331879002\\
1.05358037055583	1850.47714997588\\
1.05558337506259	1848.83098493469\\
1.05758637956935	1847.16470907489\\
1.05958938407611	1845.47815050914\\
1.06159238858287	1843.77108005433\\
1.06359539308963	1842.04344041467\\
1.06559839759639	1840.29505970283\\
1.06760140210315	1838.52582332725\\
1.06960440660991	1836.73555940058\\
1.07160741111667	1834.92421062705\\
1.07361041562344	1833.09177700667\\
1.0756134201302	1831.23808665208\\
1.07761642463696	1829.36319685907\\
1.07961942914372	1827.46710762765\\
1.08162243365048	1825.54987625358\\
1.08362543815724	1823.61156003265\\
1.085628442664	1821.65238814798\\
1.08763144717076	1819.67247519113\\
1.08963445167752	1817.67216493677\\
1.09163745618428	1815.65168656802\\
1.09364046069104	1813.61144115534\\
1.0956434651978	1811.55182976918\\
1.09764646970456	1809.47342536734\\
1.09964947421132	1807.37680090762\\
1.10165247871808	1805.26258664359\\
1.10365548322484	1803.1314701246\\
1.1056584877316	1800.98431078734\\
1.10766149223836	1798.82196806852\\
1.10966449674512	1796.64541599638\\
1.11166750125188	1794.45562859917\\
1.11367050575864	1792.25380908826\\
1.1156735102654	1790.04098878768\\
1.11767651477216	1787.81842820459\\
1.11967951927892	1785.58744514191\\
1.12168252378568	1783.34935740257\\
1.12368552829244	1781.10554008528\\
1.1256885327992	1778.85736828875\\
1.12769153730596	1776.60627440746\\
1.12969454181272	1774.35374813168\\
1.13169754631948	1772.1012218559\\
1.13370055082624	1769.85018527039\\
1.135703555333	1767.60207076964\\
1.13770655983976	1765.35825345235\\
1.13970956434652	1763.12022300879\\
1.14171256885328	1760.88935453766\\
1.14371557336004	1758.66690854613\\
1.1457185778668	1756.45420283712\\
1.14772158237356	1754.25249791777\\
1.14972458688032	1752.06282511212\\
1.15172759138708	1749.88644492731\\
1.15373059589384	1747.72427409583\\
1.1557336004006	1745.57728664591\\
1.15773660490736	1743.44634201426\\
1.15973960941412	1741.33229963757\\
1.16174261392088	1739.23584706518\\
1.16374561842764	1737.15767184646\\
1.1657486229344	1735.09828964343\\
1.16775162744116	1733.05827341386\\
1.16975463194792	1731.03802422823\\
1.17175763645468	1729.03788586121\\
1.17376064096144	1727.05820208747\\
1.1757636454682	1725.09925938592\\
1.17776664997496	1723.16122964389\\
1.17976965448172	1721.24422745294\\
1.18177265898848	1719.34836740463\\
1.18377566349524	1717.47370679474\\
1.185778668002	1715.62024562328\\
1.18778167250876	1713.78798389023\\
1.18978467701552	1711.9769215956\\
1.19178768152228	1710.18688685205\\
1.19379068602904	1708.41787965958\\
1.1957936905358	1706.66972813086\\
1.19779669504256	1704.94226037854\\
1.19979969954932	1703.23547640263\\
1.20180270405608	1701.54908972422\\
1.20380570856284	1699.88304304754\\
1.2058087130696	1698.23710718946\\
1.20781171757636	1696.61111026266\\
1.20981472208312	1695.00499497135\\
1.21181772658988	1693.41853213241\\
1.21382073109664	1691.85160715429\\
1.21582373560341	1690.30404814964\\
1.21782674011017	1688.77574052691\\
1.21982974461693	1687.26662699031\\
1.22183274912369	1685.77647835674\\
1.22383575363045	1684.30535192196\\
1.22583875813721	1682.85307579864\\
1.22784176264397	1681.41964998678\\
1.22984476715073	1680.00501719061\\
1.23184777165749	1678.60917741011\\
1.23385077616425	1677.23207334951\\
1.23585378067101	1675.87381960037\\
1.23785678517777	1674.5344161627\\
1.23985978968453	1673.21392033226\\
1.24186279419129	1671.91238940484\\
1.24386579869805	1670.63005256356\\
1.24586880320481	1669.36696710419\\
1.24787180771157	1668.12330491408\\
1.24987481221833	1666.89923788056\\
1.25187781672509	1665.69505248254\\
1.25388082123185	1664.51092060734\\
1.25588382573861	1663.34707143809\\
1.25788683024537	1662.20390604524\\
1.25988983475213	1661.08165361192\\
1.26189283925889	1659.98071520858\\
1.26389584376565	1658.90137731411\\
1.26589884827241	1657.84409829476\\
1.26790185277917	1656.80927922097\\
1.26990485728593	1655.79732116321\\
1.27190786179269	1654.80873978349\\
1.27391086629945	1653.84393615227\\
1.27591387080621	1652.90348322734\\
1.27791687531297	1651.98789667072\\
1.27991987981973	1651.09763484865\\
1.28192288432649	1650.23332801469\\
1.28392588883325	1649.3954345351\\
1.28592889334001	1648.58458466343\\
1.28793189784677	1647.8012940617\\
1.28993490235353	1647.04613568772\\
1.29193790686029	1646.31962520349\\
1.29394091136705	1645.62233556682\\
1.29594391587381	1644.95478243971\\
1.29794692038057	1644.31736689263\\
1.29994992488733	1643.71066188337\\
1.30195292939409	1643.13495389082\\
1.30395593390085	1642.59064398544\\
1.30595893840761	1642.07796135036\\
1.30796194291437	1641.59719246447\\
1.30996494742113	1641.14845191932\\
1.31196795192789	1640.73168241914\\
1.31397095643465	1640.34682666815\\
1.31597396094141	1639.99371277901\\
1.31797696544817	1639.67199697705\\
1.31997996995493	1639.38116360024\\
1.32198297446169	1639.12058239501\\
1.32398597896845	1638.88945122046\\
1.32598898347521	1638.68679604832\\
1.32799198798197	1638.51147096301\\
1.32999499248873	1638.36204357004\\
1.33199799699549	1638.2371387707\\
1.33400100150225	1638.13492310005\\
1.33600400600901	1638.05356309314\\
1.33800701051577	1637.99099610191\\
1.34001001502253	1637.94498759096\\
1.34201301952929	1637.91318843334\\
1.34401602403605	1637.89307761473\\
1.34601902854281	1637.88201952928\\
1.34802203304957	1637.87737857114\\
1.35002503755633	1637.87640454289\\
1.35202804206309	1637.87623265555\\
1.35403104656985	1637.87405541593\\
1.35603405107661	1637.86717992239\\
1.35803705558337	1637.85279868173\\
1.36004006009014	1637.82833338388\\
1.3620430645969	1637.79109112719\\
1.36404606910366	1637.73866548894\\
1.36604907361042	1637.66882193371\\
1.36805207811718	1637.57932592611\\
1.37005508262394	1637.46817211386\\
1.3720580871307	1637.33358432778\\
1.37406109163746	1637.17395828606\\
1.37606409614422	1636.98774700264\\
1.37806710065098	1636.77374726616\\
1.38007010515774	1636.5308704568\\
1.3820731096645	1636.2581998421\\
1.38407611417126	1635.9550478727\\
1.38607911867802	1635.62072699924\\
1.38808212318478	1635.25489344705\\
1.39008512769154	1634.85720344145\\
1.3920881321983	1634.42759968666\\
1.39409113670506	1633.96591029534\\
1.39609414121182	1633.47230715484\\
1.39809714571858	1632.9469048567\\
1.40010015022534	1632.38987528827\\
1.4021031547321	1631.80167681579\\
1.40410615923886	1631.18253862237\\
1.40610916374562	1630.53297637003\\
1.40811216825238	1629.85339112923\\
1.41011517275914	1629.1442412662\\
1.4121181772659	1628.40615703451\\
1.41412118177266	1627.63965409618\\
1.41612418627942	1626.84519081745\\
1.41812719078618	1626.0233974519\\
1.42013019529294	1625.17478966153\\
1.4221331997997	1624.29994040414\\
1.42413620430646	1623.39942263754\\
1.42613920881322	1622.47375202372\\
1.42814221331998	1621.52344422472\\
1.43014521782674	1620.54907219832\\
1.4321482223335	1619.5509797192\\
1.43415122684026	1618.52979704094\\
1.43615423134702	1617.48586793821\\
1.43815723585378	1616.41970807303\\
1.44016024036054	1615.3316039243\\
1.4421632448673	1614.22201385825\\
1.44416624937406	1613.09133894534\\
1.44616925388082	1611.93980836869\\
1.44817225838758	1610.76776590297\\
1.45017526289434	1609.57555532286\\
1.4521782674011	1608.36329121992\\
1.45418127190786	1607.13131736883\\
1.45618427641462	1605.8798629527\\
1.45818728092138	1604.60898526732\\
1.46019028542814	1603.31891349581\\
1.4621932899349	1602.00976222971\\
1.46419629444166	1600.6816460606\\
1.46619929894842	1599.33462228425\\
1.46820230345518	1597.96880549222\\
1.47020530796194	1596.58413838872\\
1.4722083124687	1595.18073556533\\
1.47421131697546	1593.75853972626\\
1.47621432148222	1592.3175508715\\
1.47821732598898	1590.85765440951\\
1.48022033049574	1589.37890763605\\
1.4822233350025	1587.8811386638\\
1.48422633950926	1586.36434749275\\
1.48622934401602	1584.82830493979\\
1.48823234852278	1583.27295370912\\
1.49023535302954	1581.69817920921\\
1.4922383575363	1580.1038095527\\
1.49424136204306	1578.48973014803\\
1.49624436654982	1576.8557118121\\
1.49824737105658	1575.20169724912\\
1.50025037556334	1573.52751457174\\
1.50225338007011	1571.83293459687\\
1.50425638457687	1570.11784273292\\
1.50625938908363	1568.38206709257\\
1.50826239359039	1566.62555038004\\
1.51026539809715	1564.84806341221\\
1.51226840260391	1563.04960618907\\
1.51427140711067	1561.23000682329\\
1.51627441161743	1559.3892080191\\
1.51827741612419	1557.52726707226\\
1.52028042063095	1555.64406939122\\
1.52228342513771	1553.73967227177\\
1.52428642964447	1551.81419030545\\
1.52628943415123	1549.86768078805\\
1.52829243865799	1547.90037290269\\
1.53029544316475	1545.91243853671\\
1.53229844767151	1543.90422146477\\
1.53430145217827	1541.87595087001\\
1.53630445668503	1539.82802782287\\
1.53830746119179	1537.76102528116\\
1.54031046569855	1535.6754016111\\
1.54231347020531	1533.57178706628\\
1.54431647471207	1531.45086919605\\
1.54631947921883	1529.31339284553\\
1.54832248372559	1527.16021745143\\
1.55032548823235	1524.99214515465\\
1.55232849273911	1522.8103218708\\
1.55433149724587	1520.61566433233\\
1.55633450175263	1518.40937575062\\
1.55833750625939	1516.19260204126\\
1.56034051076615	1513.96660371139\\
1.56234351527291	1511.73269856396\\
1.56434651977967	1509.49220440188\\
1.56634952428643	1507.24649632386\\
1.56835252879319	1504.9970067244\\
1.57035553329995	1502.74516799798\\
1.57235853780671	1500.49246983486\\
1.57436154231347	1498.24034462954\\
1.57636454682023	1495.99028207228\\
1.57836755132699	1493.74371455757\\
1.58037055583375	1491.50201718412\\
1.58237356034051	1489.26662234642\\
1.58437656484727	1487.03884784739\\
1.58637956935403	1484.82006878575\\
1.58838257386079	1482.61154566864\\
1.59038557836755	1480.41442441165\\
1.59238858287431	1478.22985093037\\
1.59439158738107	1476.05885654884\\
1.59639459188783	1473.90252988687\\
1.59839759639459	1471.76173038114\\
1.60040060090135	1469.63731746836\\
1.60240360540811	1467.53009328942\\
1.60440660991487	1465.4407453937\\
1.60640961442163	1463.36984673898\\
1.60841261892839	1461.31797028306\\
1.61041562343515	1459.28563168795\\
1.61241862794191	1457.27323202411\\
1.61442163244867	1455.28105777044\\
1.61642463695543	1453.30945270162\\
1.61842764146219	1451.35870329653\\
1.62043064596895	1449.42886685097\\
1.62243365047571	1447.52011525228\\
1.62443665498247	1445.632563092\\
1.62643965948923	1443.76621037014\\
1.62844266399599	1441.9210570867\\
1.63044566850275	1440.09710324168\\
1.63244867300951	1438.2942915393\\
1.63445167751627	1436.512507388\\
1.63645468202303	1434.75163619623\\
1.63845768652979	1433.01162066819\\
1.64046069103655	1431.29228891656\\
1.64246369554331	1429.593469054\\
1.64446670005008	1427.91510378473\\
1.64646970455684	1426.25690662984\\
1.6484727090636	1424.61882029356\\
1.65047571357036	1423.00061559277\\
1.65247871807712	1421.40223523169\\
1.65448172258388	1419.82339273143\\
1.65648472709064	1418.26408809198\\
1.6584877315974	1416.72409213023\\
1.66049073610416	1415.20329025461\\
1.66249374061092	1413.70162516936\\
1.66449674511768	1412.21892498712\\
1.66649974962444	1410.7551897079\\
1.6685027541312	1409.31030474013\\
1.67050575863796	1407.88427008383\\
1.67250876314472	1406.47697114743\\
1.67451176765148	1405.08846522671\\
1.67651477215824	1403.71875232167\\
1.678517776665	1402.36783243231\\
1.68052078117176	1401.03582015019\\
1.68252378567852	1399.72277277109\\
1.68452679018528	1398.42874759079\\
1.68652979469204	1397.15391649662\\
1.6885327991988	1395.89839408015\\
1.69053580370556	1394.6624095245\\
1.69253880821232	1393.44607742121\\
1.69454181271908	1392.2496842492\\
1.69654481722584	1391.07345919158\\
1.6985478217326	1389.91768872724\\
1.70055082623936	1388.78265933509\\
1.70255383074612	1387.66871478979\\
1.70455683525288	1386.57619886604\\
1.70655983975964	1385.5054553385\\
1.7085628442664	1384.45688527763\\
1.71056584877316	1383.43100434545\\
1.71256885327992	1382.42809902085\\
1.71457185778668	1381.44879955741\\
1.71657486229344	1380.49344972981\\
1.7185778668002	1379.56267979162\\
1.72058087130696	1378.65694810908\\
1.72258387581372	1377.7767703442\\
1.72458688032048	1376.92271945477\\
1.72658988482724	1376.09536839861\\
1.728592889334	1375.29523283771\\
1.73059589384076	1374.52288572987\\
1.73259889834752	1373.77890003289\\
1.73460190285428	1373.06373411301\\
1.73660490736104	1372.3780182238\\
1.7386079118678	1371.72221073149\\
1.74061091637456	1371.09677000233\\
1.74261392088132	1370.50215440254\\
1.74461692538808	1369.93870770681\\
1.74661992989484	1369.40677368981\\
1.7486229344016	1368.90658153466\\
1.75062593890836	1368.4383031287\\
1.75262894341512	1368.00205306349\\
1.75463194792188	1367.59783133902\\
1.75663495242864	1367.22546606797\\
1.7586379569354	1366.88467077142\\
1.76064096144216	1366.57510167471\\
1.76264396594892	1366.29618582004\\
1.76464697045568	1366.04729295384\\
1.76664997496244	1365.82739175207\\
1.7686529794692	1365.63556548226\\
1.77065598397596	1365.47055363726\\
1.77265898848272	1365.33092382259\\
1.77466199298948	1365.21512905219\\
1.77666499749624	1365.12133586113\\
1.778668002003	1365.04759619289\\
1.78067100650976	1364.99173280787\\
1.78267401101652	1364.95162576221\\
1.78467701552328	1364.92469674584\\
1.78668002003005	1364.90842474446\\
1.78868302453681	1364.90023144799\\
1.79068602904357	1364.89730936323\\
1.79268903355033	1364.89696558855\\
1.79469203805709	1364.89639263076\\
1.79669504256385	1364.89272570087\\
1.79869804707061	1364.88321460147\\
1.80070105157737	1364.86516643092\\
1.80270405608413	1364.83600287915\\
1.80470706059089	1364.7930883403\\
1.80671006509765	1364.73413098318\\
1.80871306960441	1364.65683897661\\
1.81071607411117	1364.55909237676\\
1.81271907861793	1364.43900042291\\
1.81472208312469	1364.29478694587\\
1.81672508763145	1364.12490495961\\
1.81872809213821	1363.92797936543\\
1.82073109664497	1363.70280695194\\
1.82273410115173	1363.44835639512\\
1.82473710565849	1363.16382555406\\
1.82674011016525	1362.84846958362\\
1.82874311467201	1362.50188741335\\
1.83074611917877	1362.123620677\\
1.83274912368553	1361.71344019147\\
1.83475212819229	1361.27128866097\\
1.83675513269905	1360.79710878972\\
1.83875813720581	1360.29101516928\\
1.84076114171257	1359.75317968699\\
1.84276414621933	1359.18394611752\\
1.84476715072609	1358.58354364401\\
1.84677015523285	1357.95248792845\\
1.84877315973961	1357.29106544975\\
1.85077616424637	1356.5998491657\\
1.85277916875313	1355.87935473833\\
1.85478217325989	1355.13004053386\\
1.85678517776665	1354.35253680586\\
1.85878818227341	1353.54730192059\\
1.86079118678017	1352.7149661316\\
1.86279419128693	1351.85604510092\\
1.86479719579369	1350.97105449056\\
1.86680020030045	1350.0606245541\\
1.86880320480721	1349.12527095355\\
1.87080620931397	1348.16545205514\\
1.87280921382073	1347.18174081668\\
1.87481221832749	1346.1745956044\\
1.87681522283425	1345.14441748876\\
1.87881822734101	1344.09172213176\\
1.88082123184777	1343.01691060388\\
1.88282423635453	1341.92038397556\\
1.88482724086129	1340.80248602148\\
1.88683024536805	1339.66361781209\\
1.88883324987481	1338.50400853053\\
1.89083625438157	1337.32405924724\\
1.89283925888833	1336.12394184956\\
1.89484226339509	1334.90400011216\\
1.89684526790185	1333.66434862662\\
1.89884827240861	1332.40521657604\\
1.90085127691537	1331.12683314354\\
1.90285428142213	1329.82931292069\\
1.90485728592889	1328.51271320326\\
1.90686029043565	1327.17714858281\\
1.90886329494241	1325.82279094668\\
1.91086629944917	1324.44958299909\\
1.91286930395593	1323.05758203582\\
1.91487230846269	1321.64684535265\\
1.91687531296945	1320.21725835802\\
1.91887831747621	1318.7688783477\\
1.92088132198297	1317.30164802593\\
1.92288432648973	1315.81545280114\\
1.92488733099649	1314.31023537756\\
1.92689033550325	1312.78593845939\\
1.92889334001002	1311.24233286353\\
1.93089634451678	1309.67941858997\\
1.93289934902354	1308.0969664556\\
1.9349023535303	1306.49491916463\\
1.93690535803706	1304.87304753396\\
1.93890836254382	1303.23123697201\\
1.94091136705058	1301.56931559145\\
1.94291437155734	1299.88716880073\\
1.9449173760641	1298.18456741672\\
1.94692038057086	1296.46139684786\\
1.94892338507762	1294.7175425026\\
1.95092638958438	1292.9528324936\\
1.95292939409114	1291.16715222929\\
1.9549323985979	1289.36038711813\\
1.95693540310466	1287.5325371601\\
1.95893840761142	1285.68348776365\\
1.96094141211818	1283.81318163301\\
1.96294441662494	1281.92167606394\\
1.9649474211317	1280.00902835224\\
1.96695042563846	1278.07529579367\\
1.96895343014522	1276.12065027558\\
1.97095643465198	1274.14520638953\\
1.97295943915874	1272.14930791019\\
1.9749624436655	1270.13318402069\\
1.97696544817226	1268.09717849569\\
1.97896845267902	1266.04174970144\\
1.98097145718578	1263.96735600417\\
1.98297446169254	1261.87462765745\\
1.9849774661993	1259.76413761909\\
1.98698047070606	1257.63663073421\\
1.98898347521282	1255.49290914372\\
1.99098647971958	1253.33377498855\\
1.99298948422634	1251.16025959272\\
1.9949924887331	1248.97327968871\\
1.99699549323986	1246.77392389632\\
1.99899849774662	1244.56339542693\\
2.00100150225338	1242.34284019612\\
2.00300450676014	1240.11357600682\\
2.0050075112669	1237.87692066197\\
2.00701051577366	1235.63424926027\\
2.00901352028042	1233.38687960465\\
2.01101652478718	1231.13630138537\\
2.01301952929394	1228.88394699694\\
2.0150225338007	1226.6313061296\\
2.01702553830746	1224.37981117785\\
2.01902854281422	1222.13095183196\\
2.02103154732098	1219.88610319064\\
2.02303455182774	1217.64669764837\\
2.0250375563345	1215.41411030387\\
2.02704056084126	1213.18977355161\\
2.02904356534802	1210.97483330719\\
2.03104656985478	1208.77060737354\\
2.03304957436154	1206.57824166626\\
2.0350525788683	1204.39888210092\\
2.03705558337506	1202.23356000156\\
2.03905858788182	1200.08319210065\\
2.04106159238858	1197.94869513067\\
2.04306459689534	1195.83092852831\\
2.0450676014021	1193.7306371387\\
2.04707060590886	1191.64839391963\\
2.04907361041562	1189.58488642047\\
2.05107661492238	1187.54063030322\\
2.05307961942914	1185.51602663835\\
2.0550826239359	1183.51153379208\\
2.05708562844266	1181.52738094755\\
2.05908863294942	1179.56391187941\\
2.06109163745618	1177.62129847502\\
2.06309464196294	1175.69971262171\\
2.0650976464697	1173.79926891104\\
2.06710065097646	1171.92002463879\\
2.06910365548322	1170.06197980496\\
2.07110665998998	1168.22513440955\\
2.07310966449675	1166.40943115678\\
2.07511266900351	1164.61487004665\\
2.07711567351027	1162.84127919183\\
2.07911867801703	1161.08860129652\\
2.08112168252379	1159.35672176918\\
2.08312468703055	1157.64541142668\\
2.08512769153731	1155.95461297325\\
2.08713069604407	1154.28415452155\\
2.08913370055083	1152.63386418423\\
2.09113670505759	1151.00357007397\\
2.09313970956435	1149.39315759919\\
2.09514271407111	1147.80245487257\\
2.09714571857787	1146.23129000676\\
2.09914872308463	1144.67954841021\\
2.10115172759139	1143.14705819557\\
2.10315473209815	1141.63376206707\\
2.10515773660491	1140.13954543315\\
2.10716074111167	1138.66429370225\\
2.10916374561843	1137.20794957859\\
2.11116675012519	1135.7703984706\\
2.11316975463195	1134.35169767408\\
2.11517275913871	1132.95178989324\\
2.11717576364547	1131.57061783229\\
2.11917876815223	1130.20823878703\\
2.12118177265899	1128.86476734901\\
2.12318477716575	1127.54014622245\\
2.12518778167251	1126.2344899989\\
2.12719078617927	1124.94797056572\\
2.12919379068603	1123.68070251445\\
2.13119679519279	1122.43280043665\\
2.13319979969955	1121.20449351545\\
2.13520280420631	1119.99595363818\\
2.13720580871307	1118.80740998796\\
2.13920881321983	1117.63914904369\\
2.14121181772659	1116.49151458004\\
2.14321482223335	1115.36467848436\\
2.14521782674011	1114.25904182709\\
2.14722083124687	1113.17500567871\\
2.14922383575363	1112.11291381387\\
2.15122684026039	1111.07316730305\\
2.15322984476715	1110.05622451247\\
2.15523284927391	1109.06254380838\\
2.15723585378067	1108.09252626122\\
2.15923885828743	1107.1467448288\\
2.16124186279419	1106.22571517312\\
2.16324486730095	1105.32989566044\\
2.16524787180771	1104.45991654431\\
2.16725087631447	1103.61623619098\\
2.16925388082123	1102.79948485402\\
2.17125688532799	1102.01012089967\\
2.17325988983475	1101.24883187728\\
2.17526289434151	1100.51607615309\\
2.17726589884827	1099.81236938911\\
2.17926890335503	1099.13834183891\\
2.18127190786179	1098.49433727719\\
2.18327491236855	1097.88092866172\\
2.18527791687531	1097.29845976719\\
2.18728092138207	1096.74733166405\\
2.18928392588883	1096.22777353543\\
2.19128693039559	1095.74007186021\\
2.19328993490235	1095.28434122997\\
2.19529293940911	1094.86063894047\\
2.19729594391587	1094.46890769594\\
2.19929894842263	1094.10891831326\\
2.20130195292939	1093.78044160931\\
2.20330495743615	1093.4829046263\\
2.20530796194291	1093.21584899799\\
2.20731096644967	1092.97841528768\\
2.20931397095643	1092.76962946714\\
2.21131697546319	1092.58846021232\\
2.21331997996995	1092.43358972029\\
2.21532298447672	1092.30347100502\\
2.21732598898348	1092.19649978467\\
2.21932899349024	1092.11078529852\\
2.221331997997	1092.04432219428\\
2.22333500250376	1091.99493323234\\
2.22533800701052	1091.96021198996\\
2.22734101151728	1091.93780934017\\
2.22934401602404	1091.92503238134\\
2.2313470205308	1091.91918821183\\
2.23335002503756	1091.91764122578\\
2.23535302954432	1091.91752663422\\
2.23735603405108	1091.91603694395\\
2.23935903855784	1091.91042195756\\
2.2413620430646	1091.89793147763\\
2.24336504757136	1091.87581530673\\
2.24536805207812	1091.84155243058\\
2.24737105658488	1091.79267913066\\
2.24937406109164	1091.72678898422\\
2.2513770655984	1091.64170475164\\
2.25338007010516	1091.53542108065\\
2.25538307461192	1091.40610450629\\
2.25738607911868	1091.25203615517\\
2.25938908362544	1091.07166904127\\
2.2613920881322	1090.86385724897\\
2.26339509263896	1090.62734027114\\
2.26539809714572	1090.36120137531\\
2.26740110165248	1090.06469571632\\
2.26940410615924	1089.73713574485\\
2.271407110666	1089.3781776862\\
2.27341011517276	1088.9874777657\\
2.27541311967952	1088.56474950445\\
2.27741612418628	1088.11005019824\\
2.27941912869304	1087.62337984705\\
2.2814221331998	1087.10485304246\\
2.28342513770656	1086.55469896757\\
2.28542814221332	1085.9732041013\\
2.28743114672008	1085.36076951408\\
2.28943415122684	1084.71773898061\\
2.2914371557336	1084.04462816289\\
2.29344016024036	1083.34189542716\\
2.29544316474712	1082.61005643544\\
2.29744616925388	1081.84962684974\\
2.29944917376064	1081.06117962786\\
2.3014521782674	1080.24528772759\\
2.30345518277416	1079.40246681096\\
2.30545818728092	1078.53328983574\\
2.30746119178768	1077.63827246397\\
2.30946419629444	1076.71804494921\\
2.3114672008012	1075.77300836192\\
2.31347020530796	1074.80379295568\\
2.31547320981472	1073.81085709672\\
2.31747621432148	1072.79460185549\\
2.31947921882824	1071.7556001898\\
2.321482223335	1070.69419587432\\
2.32348522784176	1069.61084727529\\
2.32548823234852	1068.50589816738\\
2.32749123685528	1067.37974962105\\
2.32949424136204	1066.23274541098\\
2.3314972458688	1065.06511472028\\
2.33350025037556	1063.87725861941\\
2.33550325488232	1062.66934899572\\
2.33750625938908	1061.44161503231\\
2.33950926389584	1060.19434320809\\
2.3415122684026	1058.9277054104\\
2.34351527290936	1057.64175893501\\
2.34551827741612	1056.33673296504\\
2.34752128192288	1055.01274209205\\
2.34952428642964	1053.66984361182\\
2.3515272909364	1052.30809482013\\
2.35353029544316	1050.92749571699\\
2.35553329994992	1049.52816089394\\
2.35753630445668	1048.11003305521\\
2.35953930896345	1046.6731122008\\
2.36154231347021	1045.21739833071\\
2.36354531797697	1043.7427195576\\
2.36554832248373	1042.24907588148\\
2.36755132699049	1040.73641000655\\
2.36955433149725	1039.20455004549\\
2.37155733600401	1037.65343870251\\
2.37356034051077	1036.0828467945\\
2.37556334501753	1034.49277432145\\
2.37756634952429	1032.88299210025\\
2.37956935403105	1031.25332824356\\
2.38157235853781	1029.60366815982\\
2.38357536304457	1027.93383996169\\
2.38557836755133	1026.24372905762\\
2.38758137205809	1024.53310626447\\
2.38958437656485	1022.80185699071\\
2.39158738107161	1021.04980934898\\
2.39359038557837	1019.2769060435\\
2.39559339008513	1017.48297518695\\
2.39759639459189	1015.66795948353\\
2.39959939909865	1013.83180163748\\
2.40160240360541	1011.974444353\\
2.40360540811217	1010.09583033433\\
2.40560841261893	1008.19607417301\\
2.40761141712569	1006.27511857328\\
2.40961442163245	1004.33319271824\\
2.41161742613921	1002.37041119946\\
2.41362043064597	1000.38694590427\\
2.41562343515273	998.383026015805\\
2.41762643965949	996.359109900285\\
2.41962944416625	994.315426740833\\
2.42163244867301	992.252549495244\\
2.42363545317977	990.170879233974\\
2.42563845768653	988.071160802158\\
2.42764146219329	985.953909861811\\
2.42964446670005	983.819928553847\\
2.43164747120681	981.670076314957\\
2.43365047571357	979.505212581834\\
2.43565348022033	977.326254086952\\
2.43765648472709	975.13428945012\\
2.43965948923385	972.930464586929\\
2.44166249374061	970.715868117189\\
2.44366549824737	968.49176054805\\
2.44566850275413	966.259402386662\\
2.44767150726089	964.020226027511\\
2.44967451176765	961.775549273527\\
2.45167751627441	959.526747223418\\
2.45368052078117	957.275309567451\\
2.45568352528793	955.022668700115\\
2.45768652979469	952.770314311676\\
2.45968953430145	950.519678796623\\
2.46169253880821	948.272194549442\\
2.46369554331497	946.029293964624\\
2.46569854782173	943.792466732433\\
2.46770155232849	941.56291606424\\
2.46970455683525	939.342074354534\\
2.47170756134201	937.131202110462\\
2.47371056584877	934.931502543396\\
2.47571357035553	932.744121568925\\
2.47771657486229	930.57014780686\\
2.47971957936905	928.410555285453\\
2.48172258387581	926.266260737176\\
2.48372558838257	924.138238190281\\
2.48572859288933	922.0272324899\\
2.48773159739609	919.93393118539\\
2.48973460190285	917.858964530324\\
2.49173760640961	915.802905482497\\
2.49374061091637	913.766326999704\\
2.49574361542313	911.749515560844\\
2.49774661992989	909.752986827931\\
2.49974962443665	907.776912688304\\
2.50175262894342	905.821579620861\\
2.50375563345018	903.887159512941\\
2.50575863795694	901.97388154766\\
2.5077616424637	900.081688429241\\
2.50976464697046	898.210752045021\\
2.51176765147722	896.360957803441\\
2.51377065598398	894.53242029606\\
2.51577366049074	892.72502493132\\
2.5177766649975	890.938657117661\\
2.51977966950426	889.173316855084\\
2.52178267401102	887.428774960469\\
2.52378567851778	885.704974138039\\
2.52578868302454	884.001685204674\\
2.5277916875313	882.318908160374\\
2.52979469203806	880.656356526243\\
2.53179769654482	879.013915710721\\
2.53380070105158	877.39141382647\\
2.53580370555834	875.78873628193\\
2.5378067100651	874.205653893983\\
2.53980971457186	872.642109366851\\
2.54181271907862	871.097930813194\\
2.54381572358538	869.573003641453\\
2.54581872809214	868.06715596429\\
2.5478217325989	866.580387781705\\
2.54982473710566	865.11252720636\\
2.55182774161242	863.663574238253\\
2.55383074611918	862.233414285828\\
2.55583375062594	860.822047349082\\
2.5578367551327	859.429473428016\\
2.55983975963946	858.055692522631\\
2.56184276414622	856.700704632926\\
2.56384576865298	855.364567054681\\
2.56584877315974	854.047394379455\\
2.5678517776665	852.749186607247\\
2.56985478217326	851.470172921177\\
2.57185778668002	850.210467912802\\
2.57386079118678	848.970243469462\\
2.57586379569354	847.749614182715\\
2.5778668002003	846.54886653146\\
2.57986980470706	845.368286994593\\
2.58187280921382	844.208047459453\\
2.58387581372058	843.068491700717\\
2.58587881822734	841.949906197283\\
2.5878818227341	840.852692019608\\
2.58988482724086	839.777192942367\\
2.59188783174762	838.723810036019\\
2.59389083625438	837.69294437102\\
2.59589384076114	836.685054313605\\
2.5978968452679	835.700540934232\\
2.59989984977466	834.739977190695\\
2.60190285428142	833.803821449231\\
2.60390585878818	832.892589371855\\
2.60590886329494	832.006853916362\\
2.6079118678017	831.147073448989\\
2.60991487230846	830.31387822331\\
2.61191787681522	829.50778390134\\
2.61392088132198	828.729363440875\\
2.61592388582874	827.979132503931\\
2.6179268903355	827.257721344082\\
2.61992989484226	826.565588327564\\
2.62193289934902	825.903249116393\\
2.62393590385578	825.271162076804\\
2.62593890836254	824.669785575035\\
2.6279419128693	824.099577977321\\
2.62994491737606	823.56076846678\\
2.63194792188282	823.053643522309\\
2.63395092638958	822.578432327028\\
2.63595393089634	822.135249472494\\
2.6379569354031	821.724094958709\\
2.63995993990986	821.344796898332\\
2.64196294441662	820.997183404026\\
2.64396594892338	820.680910701114\\
2.64596895343015	820.3954058318\\
2.64797195793691	820.139981246731\\
2.64997496244367	819.913892100772\\
2.65197796695043	819.716107069893\\
2.65398097145719	819.545365646944\\
2.65598397596395	819.400407324776\\
2.65798698047071	819.279513230003\\
2.65998998497747	819.181136376579\\
2.66199298948423	819.103214116442\\
2.66399599399099	819.043683801527\\
2.66599899849775	819.000310896436\\
2.66800200300451	818.970746274207\\
2.67000500751127	818.952411624763\\
2.67200801201803	818.942671342246\\
2.67401101652479	818.938832525018\\
2.67601402103155	818.938202271444\\
2.67801702553831	818.937858496767\\
2.68002003004507	818.93510829935\\
2.68202303455183	818.927086890218\\
2.68402603905859	818.911158663514\\
2.68602904356535	818.884630717599\\
2.68803204807211	818.844924742397\\
2.69003505257887	818.789634315166\\
2.69203805708563	818.716467604728\\
2.69404106159239	818.623361963019\\
2.69604406609915	818.508254741978\\
2.69804707060591	818.369427068218\\
2.70005007511267	818.205217364133\\
2.70205307961943	818.014250531016\\
2.70405608412619	817.795323357496\\
2.70605908863295	817.547289927984\\
2.70806209313971	817.269348101566\\
2.71006509764647	816.960810328888\\
2.71206810215323	816.620989060596\\
2.71407110665999	816.249655113572\\
2.71607411116675	815.846407417359\\
2.71807711567351	815.411188676178\\
2.72008012018027	814.943941594248\\
2.72208312468703	814.44478076313\\
2.72408612919379	813.913820774383\\
2.72608913370055	813.351348106903\\
2.72809213820731	812.757706535368\\
2.73009514271407	812.133239834454\\
2.73209814722083	811.4784063704\\
2.73410115172759	810.793607213659\\
2.73610415623435	810.079415322029\\
2.73810716074111	809.336346357524\\
2.74011016524787	808.56485868638\\
2.74211316975463	807.765639857952\\
2.74411617426139	806.939090942696\\
2.74611917876815	806.085899489967\\
2.74812218327491	805.20658116178\\
2.75012518778167	804.30159432437\\
2.75212819228843	803.371626527093\\
2.75413119679519	802.417078840406\\
2.75613420130195	801.438466926322\\
2.75813720580871	800.436363742638\\
2.76014021031547	799.411170359811\\
2.76214321482223	798.363345144075\\
2.76414621932899	797.293289165889\\
2.76614922383575	796.201460791488\\
2.76815222834251	795.088146499769\\
2.77015523284927	793.953804656969\\
2.77215823735603	792.798664446206\\
2.77416124186279	791.623126937936\\
2.77616424636955	790.427364019498\\
2.77816725087631	789.21166216979\\
2.78017025538307	787.976307867708\\
2.78217325988983	786.721415704812\\
2.78417626439659	785.447157568441\\
2.78617926890335	784.153762641713\\
2.78818227341012	782.841288220407\\
2.79018527791688	781.509906191861\\
2.79218828242364	780.159616556076\\
2.7941912869304	778.790476608832\\
2.79619429143716	777.402543645907\\
2.79819729594392	775.995874963081\\
2.80020030045068	774.570413264575\\
2.80220330495744	773.12610125461\\
2.8042063094642	771.662938933184\\
2.80620931397096	770.180869004519\\
2.80821231847772	768.679834172835\\
2.81021532298448	767.159662550794\\
2.81221832749124	765.620239546837\\
2.814221331998	764.061507865183\\
2.81622433650476	762.483352914275\\
2.81822734101152	760.885545510994\\
2.82023034551828	759.26802835956\\
2.82223335002504	757.630514981076\\
2.8242363545318	755.973005375542\\
2.82623935903856	754.29527035984\\
2.82824236354532	752.597080750852\\
2.83024536805208	750.878436548577\\
2.83224837255884	749.139051274119\\
2.8342513770656	747.378867631698\\
2.83625438157236	745.597771029754\\
2.83825738607912	743.795589580949\\
2.84026039058588	741.972323285284\\
2.84226339509264	740.127857551199\\
2.8442663995994	738.262192378694\\
2.84626940410616	736.37527047199\\
2.84827240861292	734.467206422645\\
2.85027541311968	732.53805752644\\
2.85227841762644	730.587938374932\\
2.8542814221332	728.616963559682\\
2.85628442663996	726.625476855367\\
2.85828743114672	724.613707445103\\
2.86029043565348	722.581999103569\\
2.86229344016024	720.530752901222\\
2.864296444667	718.460427204296\\
2.86629944917376	716.371652266367\\
2.86830245368052	714.26494374945\\
2.87030545818728	712.141103794459\\
2.87230846269404	710.000819950748\\
2.8743114672008	707.844951655009\\
2.87631447170756	705.674415639715\\
2.87831747621432	703.490300524677\\
2.88032048072108	701.293523042366\\
2.88232348522784	699.085343699931\\
2.8843264897346	696.866851117185\\
2.88632949424136	694.639363097055\\
2.88833249874812	692.40419744247\\
2.89033550325488	690.162671956359\\
2.89233850776164	687.91621903321\\
2.8943415122684	685.666213771731\\
2.89634451677516	683.41414586219\\
2.89834752128192	681.161447699074\\
2.90035052578868	678.909608972651\\
2.90235353029544	676.660004781629\\
2.9043565348022	674.414182112055\\
2.90635953930896	672.173458766857\\
2.90836254381572	669.939324436303\\
2.91036554832248	667.713039627543\\
2.91236855282924	665.495979439284\\
2.914371557336	663.289347082897\\
2.91637456184276	661.094288473971\\
2.91837756634952	658.912006823877\\
2.92038057085628	656.743533456646\\
2.92238357536304	654.589842400528\\
2.9243865798698	652.451850387998\\
2.92638958437656	650.330416855746\\
2.92839258888332	648.226229353128\\
2.93039559339008	646.140032725277\\
2.93239859789685	644.072399929989\\
2.93440160240361	642.023903925058\\
2.93640460691037	639.99500307672\\
2.93840761141713	637.986098455432\\
2.94041061592389	635.997533835872\\
2.94241362043065	634.029595696936\\
2.94441662493741	632.082398630184\\
2.94641962944417	630.156286410292\\
2.94842263395093	628.251201741482\\
2.95042563845769	626.367316511092\\
2.95242864296445	624.504688014901\\
2.95443164747121	622.66325895713\\
2.95643465197797	620.842972042\\
2.95843765648473	619.04382726951\\
2.96044066099149	617.265710048101\\
2.96244366549825	615.508505786214\\
2.96444667000501	613.77209989229\\
2.96644967451177	612.056377774771\\
2.96845267901853	610.361110250538\\
2.97045568352529	608.686240023811\\
2.97245868803205	607.031595207253\\
2.97446169253881	605.397003913525\\
2.97646469704557	603.782294255287\\
2.97846770155233	602.187294345202\\
2.98047070605909	600.61188959171\\
2.98247371056585	599.055908107473\\
2.98447671507261	597.519292596712\\
2.98647971957937	596.001871172087\\
2.98848272408613	594.503471946261\\
2.99048572859289	593.024094919233\\
2.99248873309965	591.563682795225\\
2.99449173760641	590.122063686896\\
2.99649474211317	588.699294890027\\
2.99849774661993	587.295261813059\\
3.00050075112669	585.910021751771\\
3.00250375563345	584.543574706164\\
3.00450676014021	583.195977972016\\
3.00650976464697	581.867288845108\\
3.00851276915373	580.557507325438\\
3.01051577366049	579.266805300347\\
3.01251877816725	577.995354657172\\
3.01452178267401	576.743212691694\\
3.01652478718077	575.510665882808\\
3.01852779168753	574.297771526296\\
3.02053079619429	573.104930692613\\
3.02253380070105	571.932257973319\\
3.02453680520781	570.78003984731\\
3.02653980971457	569.648677385045\\
3.02854281422133	568.53845706542\\
3.03054581872809	567.449665367333\\
3.03254882323485	566.38276065702\\
3.03455182774161	565.338144004937\\
3.03655483224837	564.316216481542\\
3.03855783675513	563.31743645307\\
3.04056084126189	562.342262285757\\
3.04256384576865	561.39115234584\\
3.04456685027541	560.464736886893\\
3.04656985478217	559.563416979373\\
3.04857285928893	558.687765581074\\
3.05057586379569	557.838298354014\\
3.05257886830245	557.015645551765\\
3.05458187280921	556.220322836344\\
3.05658487731597	555.452903165545\\
3.05858788182273	554.713902201386\\
3.06059088632949	554.00383560588\\
3.06259389083625	553.323333632603\\
3.06459689534301	552.672797352012\\
3.06659989984977	552.052685130342\\
3.06860290435653	551.463455333829\\
3.07060590886329	550.905451737151\\
3.07260891337005	550.379075410764\\
3.07461191787682	549.884440946228\\
3.07661492238358	549.42172023088\\
3.07861792689034	548.99108515206\\
3.0806209313971	548.592421118208\\
3.08262393590386	548.225556241986\\
3.08462694041062	547.890261340275\\
3.08662994491738	547.58607804684\\
3.08863294942414	547.312490699665\\
3.0906359539309	547.068697157837\\
3.09263895843766	546.853780688883\\
3.09464196294442	546.666709968773\\
3.09664496745118	546.506224490357\\
3.09864797195794	546.370891859147\\
3.1006509764647	546.259050497538\\
3.10265398097146	546.168866940584\\
3.10465698547822	546.098393131783\\
3.10665998998498	546.045509127292\\
3.10866299449174	546.007808504373\\
3.1106659989985	545.982942136064\\
3.11266900350526	545.968331712288\\
3.11467200801202	545.961227035629\\
3.11667501251878	545.958992500228\\
3.11867801702554	545.95876331711\\
3.1206810215323	545.957846584637\\
3.12268402603906	545.953377513835\\
3.12468703054582	545.942548611507\\
3.12669003505258	545.922781567575\\
3.12869303955934	545.891383480402\\
3.1306960440661	545.845890631469\\
3.13269904857286	545.783839302256\\
3.13470205307962	545.703109548922\\
3.13670505758638	545.601638723404\\
3.13870806209314	545.47742147342\\
3.1407110665999	545.328853517143\\
3.14271407110666	545.154330572746\\
3.14471707561342	544.952534837301\\
3.14672008012018	544.722377690997\\
3.14872308462694	544.462713218243\\
3.1507260891337	544.172853869687\\
3.15272909364046	543.852112095973\\
3.15473209814722	543.500029530865\\
3.15673510265398	543.116205103907\\
3.15873810716074	542.70046692776\\
3.1607411116675	542.252700410865\\
3.16274411617426	541.772905553222\\
3.16474712068102	541.26131153795\\
3.16675012518778	540.717975660828\\
3.16875312969454	540.143298992311\\
3.1707561342013	539.537510715519\\
3.17275913870806	538.901069196688\\
3.17476214321482	538.234490097833\\
3.17676514772158	537.538117193631\\
3.17876815222834	536.812523441877\\
3.1807711567351	536.058281800367\\
3.18277416124186	535.275907931116\\
3.18477716574862	534.465917496139\\
3.18678017025538	533.628883453233\\
3.18878317476214	532.765378760191\\
3.1907861792689	531.87591907903\\
3.19278918377566	530.961134663324\\
3.19479218828242	530.02142658353\\
3.19679519278918	529.057425093222\\
3.19879819729594	528.069588558637\\
3.2008012018027	527.058375346011\\
3.20280420630946	526.024243821579\\
3.20480721081622	524.967709647358\\
3.20681021532298	523.889059302245\\
3.20881321982974	522.788751152476\\
3.2108162243365	521.667186268507\\
3.21281922884326	520.524651129237\\
3.21482223335002	519.361489509342\\
3.21682523785678	518.177987887719\\
3.21882824236355	516.974432743268\\
3.22083124687031	515.750995963325\\
3.22283425137707	514.507964026788\\
3.22483725588383	513.245508820997\\
3.22684026039059	511.96374493751\\
3.22884326489735	510.662844263665\\
3.23084626940411	509.342978686802\\
3.23284927391087	508.00414820692\\
3.23485227841763	506.646467415578\\
3.23685528292439	505.269993608556\\
3.23885828743115	503.874726785853\\
3.24086129193791	502.46066694747\\
3.24286429644467	501.027871389186\\
3.24486730095143	499.576168223664\\
3.24687030545819	498.105614746681\\
3.24887330996495	496.616153662459\\
3.25087631447171	495.107613083659\\
3.25287931897847	493.579935714501\\
3.25488232348523	492.032949667648\\
3.25688532799199	490.466654943099\\
3.25888833249875	488.880822357736\\
3.26089133700551	487.27528002422\\
3.26289434151227	485.649970646773\\
3.26489734601903	484.004665042275\\
3.26690035052579	482.339248619168\\
3.26890335503255	480.653549490114\\
3.27090635953931	478.947338471994\\
3.27290936404607	477.220615564808\\
3.27491236855283	475.473094289659\\
3.27691537305959	473.704717350767\\
3.27891837756635	471.915370156574\\
3.28092138207311	470.104995411299\\
3.28292438657987	468.273421227604\\
3.28492739108663	466.42064760549\\
3.28693039559339	464.546617249176\\
3.28893340010015	462.651444750222\\
3.29093640460691	460.735130108627\\
3.29293940911367	458.797730620172\\
3.29494241362043	456.839418172194\\
3.29694541812719	454.860421947812\\
3.29894842263395	452.860971130144\\
3.30095142714071	450.841294902308\\
3.30295443164747	448.801908926319\\
3.30495743615423	446.743156976855\\
3.30696044066099	444.665497420152\\
3.30896344516775	442.569617805563\\
3.31096644967451	440.456148386664\\
3.31296945418127	438.325776712809\\
3.31497245868803	436.179247629131\\
3.31697546319479	434.017535163882\\
3.31897846770155	431.841556049534\\
3.32098147220831	429.652341610118\\
3.32298447671507	427.450980465446\\
3.32498748122183	425.238618531108\\
3.32699048572859	423.016459018472\\
3.32899349023535	420.785819730469\\
3.33099649474211	418.548018470026\\
3.33299949924887	416.304373040074\\
3.33500250375563	414.056373130878\\
3.33700550826239	411.805393841147\\
3.33900851276915	409.55286756537\\
3.34101151727591	407.300341289593\\
3.34301452178267	405.049190112524\\
3.34501752628943	402.80090372443\\
3.34702053079619	400.556914519801\\
3.34902353530295	398.318597597342\\
3.35102653980971	396.087385351544\\
3.35302954431647	393.864595585334\\
3.35503254882323	391.651431510082\\
3.35703555332999	389.449210928717\\
3.35903855783675	387.25902246105\\
3.36104156234352	385.082069318451\\
3.36304456685028	382.919268233391\\
3.36504757135704	380.771593234122\\
3.3670505758638	378.639961053118\\
3.36905358037056	376.52517383129\\
3.37105658487732	374.427976413772\\
3.37305958938408	372.34899905414\\
3.37506259389084	370.288814710188\\
3.3770655983976	368.247939043932\\
3.37906860290436	366.226830421608\\
3.38107160741112	364.225832617894\\
3.38307461191788	362.245289407465\\
3.38507761642464	360.28542997344\\
3.3870806209314	358.346483498938\\
3.38908362543816	356.428564575517\\
3.39108662994492	354.531845090516\\
3.39308963445168	352.656267748155\\
3.39509263895844	350.801889844214\\
3.3970956434652	348.968768674473\\
3.39909864797196	347.156789647372\\
3.40110165247872	345.365838171352\\
3.40310465698548	343.595914246413\\
3.40510766149224	341.846903280997\\
3.407110665999	340.118576091985\\
3.40911367050576	338.410875383597\\
3.41111667501252	336.723686564275\\
3.41311967951928	335.056723155122\\
3.41512268402604	333.409927860357\\
3.4171256885328	331.783128792642\\
3.41912869303956	330.176154064638\\
3.42113169754632	328.588831789008\\
3.42313470205308	327.021047374192\\
3.42513770655984	325.47268622863\\
3.4271407110666	323.943576464985\\
3.42914371557336	322.433603491697\\
3.43114672008012	320.942710012987\\
3.43314972458688	319.470781437296\\
3.43515272909364	318.017703173065\\
3.4371557336004	316.583475220293\\
3.43915873810716	315.168040283202\\
3.44116174261392	313.771341066012\\
3.44316474712068	312.393492160281\\
3.44516775162744	311.034436270231\\
3.4471707561342	309.694173395861\\
3.44917376064096	308.37287542451\\
3.45117676514772	307.070599651957\\
3.45317976965448	305.787403373982\\
3.45518277416124	304.523458477923\\
3.457185778668	303.278994146899\\
3.45918878317476	302.054124972468\\
3.46119178768152	300.84902284197\\
3.46319479218828	299.66403153008\\
3.46519779669504	298.499322924138\\
3.4672008012018	297.3552980946\\
3.46920380570856	296.232128928806\\
3.47120681021532	295.130273792989\\
3.47320981472208	294.05001916605\\
3.47521281922884	292.991766118443\\
3.4772158237356	291.955973016406\\
3.47921882824236	290.943040930394\\
3.48122183274912	289.953428226644\\
3.48322483725588	288.987650567171\\
3.48522784176264	288.046166318212\\
3.4872308462694	287.129433846003\\
3.48923385077616	286.238140699898\\
3.49123685528292	285.372687950353\\
3.49323985978968	284.533648555163\\
3.49524286429644	283.721595472124\\
3.4972458688032	282.93715895481\\
3.49924887330996	282.180797369458\\
3.50125187781672	281.453026378083\\
3.50325488232349	280.754533530039\\
3.50525788683025	280.085662600003\\
3.50726089133701	279.446986545771\\
3.50926389584377	278.838906437799\\
3.51126690035053	278.261880642322\\
3.51326990485729	277.716252934019\\
3.51527290936405	277.202252496007\\
3.51727591387081	276.720108511405\\
3.51927891837757	276.269935571771\\
3.52128192288433	275.851848268664\\
3.52328492739109	275.465617418966\\
3.52528793189785	275.111185726898\\
3.52729093640461	274.788152122003\\
3.52929394091137	274.496000942266\\
3.53129694541813	274.234159229891\\
3.53329994992489	274.001767548186\\
3.53530295443165	273.797909164679\\
3.53730595893841	273.621438163778\\
3.53930896344517	273.470979446777\\
3.54131196795193	273.344986027628\\
3.54331497245869	273.241853624504\\
3.54531797696545	273.159691476682\\
3.54732098147221	273.096322344541\\
3.54932398597897	273.049683580017\\
3.55132699048573	273.017311464592\\
3.55332999499249	272.996742279747\\
3.55533299949925	272.985397715404\\
3.55733600400601	272.980527574145\\
3.55933900851277	272.979438954334\\
3.56134201301953	272.979267066996\\
3.56334501752629	272.977261714713\\
3.56534802203305	272.970615404289\\
3.56735102653981	272.956635234088\\
3.56935403104657	272.932628302472\\
3.57135703555333	272.895959003584\\
3.57336004006009	272.844278210463\\
3.57536304456685	272.77517950037\\
3.57736604907361	272.686600225243\\
3.57936905358037	272.576420441239\\
3.58137205808713	272.442863979194\\
3.58337506259389	272.284326557282\\
3.58537806710065	272.099261189454\\
3.58738107160741	271.886464664343\\
3.58938407611417	271.644848362136\\
3.59138708062093	271.373495550362\\
3.59339008512769	271.071604088108\\
3.59539308963445	270.738658313357\\
3.59739609414121	270.374142564095\\
3.59939909864797	269.977884952982\\
3.60140210315473	269.549599001122\\
3.60340510766149	269.089284708514\\
3.60540811216825	268.597056666717\\
3.60741111667501	268.072972171511\\
3.60941412118177	267.517317701793\\
3.61141712568853	266.930437032241\\
3.61342013019529	266.312616641751\\
3.61542313470205	265.664314896561\\
3.61742613920881	264.985990162905\\
3.61942914371557	264.2781581028\\
3.62143214822233	263.541277082483\\
3.62343515272909	262.775920059747\\
3.62543815723585	261.982659992388\\
3.62744116174261	261.162012542423\\
3.62944416624937	260.314550667644\\
3.63144717075613	259.44084732585\\
3.63345017526289	258.541418179053\\
3.63545317976965	257.616778889271\\
3.63745618427641	256.667559710078\\
3.63945918878317	255.69416171193\\
3.64146219328993	254.697100556843\\
3.64346519779669	253.676834611054\\
3.64546820230346	252.633879536577\\
3.64747120681022	251.56863640387\\
3.64947421131698	250.481506283389\\
3.65147721582374	249.372832949811\\
3.6534802203305	248.243017473592\\
3.65548322483726	247.092403629411\\
3.65748622934402	245.921220600384\\
3.65948923385078	244.729812161189\\
3.66149223835754	243.518464790723\\
3.6634952428643	242.287350376325\\
3.66549824737106	241.036640805334\\
3.66750125187782	239.766622556647\\
3.66950425638458	238.477352926044\\
3.67150726089134	237.169003800863\\
3.6735102653981	235.841689772663\\
3.67551326990486	234.495525433003\\
3.67751627441162	233.130453486104\\
3.67951927891838	231.746645819304\\
3.68152228342514	230.344045136823\\
3.6835252879319	228.922651438663\\
3.68552829243866	227.482407429042\\
3.68753129694542	226.023370403742\\
3.68953430145218	224.545425771202\\
3.69153730595894	223.048458939864\\
3.6935403104657	221.532469909727\\
3.69554331497246	219.997229497674\\
3.69754631947922	218.442737703704\\
3.69954932398598	216.868765344701\\
3.70155232849274	215.275255124883\\
3.7035553329995	213.662035156912\\
3.70555833750626	212.028876257671\\
3.70756134201302	210.375721131381\\
3.70956434651978	208.702397890701\\
3.71156735102654	207.008677352515\\
3.7135703555333	205.294444925263\\
3.71557336004006	203.559586017386\\
3.71757636454682	201.803928741546\\
3.71957936905358	200.027358506184\\
3.72158237356034	198.22981801552\\
3.7235853780671	196.411078086436\\
3.72558838257386	194.571253310492\\
3.72759138708062	192.710171800348\\
3.72959439158738	190.827833556005\\
3.73159739609414	188.924410464801\\
3.7336004006009	186.999787935177\\
3.73560340510766	185.054195150251\\
3.73760640961442	183.087746701582\\
3.73960941412118	181.100729068069\\
3.74161241862794	179.093314137048\\
3.7436154231347	177.065902978978\\
3.74561842764146	175.018839368534\\
3.74762143214822	172.952638967734\\
3.74962443665498	170.867817438591\\
3.75162744116174	168.764947738902\\
3.7536304456685	166.64471741802\\
3.75563345017526	164.50792861686\\
3.75763645468202	162.355383476333\\
3.75963945918878	160.187998728912\\
3.76164246369554	158.006691107069\\
3.7636454682023	155.812606526395\\
3.76564847270906	153.606776310921\\
3.76765147721582	151.390403672016\\
3.76965448172258	149.16480641261\\
3.77165748622934	146.931187744072\\
3.7736604907361	144.690922765111\\
3.77566349524286	142.445443870214\\
3.77766649974962	140.196068862309\\
3.77966950425638	137.944344727445\\
3.78167250876314	135.691646564329\\
3.7836755132699	133.439464063229\\
3.78567851777666	131.189229618632\\
3.78768152228343	128.942490216586\\
3.78968452679019	126.700563660018\\
3.79168753129695	124.464882343418\\
3.79369053580371	122.236764069713\\
3.79569354031047	120.017583937613\\
3.79769654481723	117.808545158486\\
3.79969954932399	115.610908239482\\
3.80170255383075	113.425819096191\\
3.80370555833751	111.254251756866\\
3.80570856284427	109.097237545537\\
3.80771156735103	106.955807786236\\
3.80971457185779	104.830650028316\\
3.81171757636455	102.72268100425\\
3.81372058087131	100.632530967613\\
3.81572358537807	98.5608301719788\\
3.81772658988483	96.5081515751432\\
3.81972959439159	94.4749535433419\\
3.82173259889835	92.4616944428113\\
3.82373560340511	90.4686607524487\\
3.82573860791187	88.4961962469313\\
3.82774161241863	86.5445301093772\\
3.82974461692539	84.6137769313453\\
3.83174762143215	82.7041658959538\\
3.83375062593891	80.8156970032026\\
3.83575363044567	78.9484275488713\\
3.83775663495243	77.1023575329597\\
3.83975963945919	75.2774869554681\\
3.84176264396595	73.4737585206167\\
3.84376564847271	71.6911149326262\\
3.84576865297947	69.929384304158\\
3.84777165748623	68.1884520436529\\
3.84977466199299	66.4682608553317\\
3.85177766649975	64.7685815560761\\
3.85378067100651	63.0892995543271\\
3.85578367551327	61.4303002585259\\
3.85778668002003	59.7912971897746\\
3.85978968452679	58.1722903480735\\
3.86179268903355	56.5730505503043\\
3.86379569354031	54.9934059091286\\
3.86579869804707	53.4332418329874\\
3.86780170255383	51.8924437303216\\
3.86980470706059	50.3708397137926\\
3.87180771156735	48.8683151918416\\
3.87381071607411	47.3848128686889\\
3.87581372058087	45.920275448555\\
3.87781672508763	44.4745883398809\\
3.87981972959439	43.0477515426666\\
3.88182273410115	41.6396504653531\\
3.88382573860791	40.2503424037198\\
3.88582874311467	38.8798273577669\\
3.88783174762143	37.5281053274943\\
3.88983475212819	36.195290904461\\
3.89183775663495	34.881384088667\\
3.89384076114171	33.5865567674508\\
3.89584376564847	32.3109235323715\\
3.89784677015523	31.0545989749882\\
3.89984977466199	29.8177549826393\\
3.90185277916875	28.6005634426634\\
3.90385578367551	27.4033681297375\\
3.90585878818227	26.2262836354207\\
3.90786179268903	25.0695964386106\\
3.90986479719579	23.9337076097638\\
3.91186780170255	22.8188463319982\\
3.91387080620931	21.725356379991\\
3.91587381071607	20.6536961199783\\
3.91787681522283	19.6042093266372\\
3.91987981972959	18.5772970704242\\
3.92188282423635	17.573475013355\\
3.92388582874311	16.5930869301067\\
3.92588883324987	15.6367630742538\\
3.92789183775663	14.7049045162531\\
3.9298948422634	13.7980842138995\\
3.93189784677016	12.9168178292088\\
3.93390085127692	12.0616210241965\\
3.93590385578368	11.2331240524374\\
3.93790686029044	10.4318425759469\\
3.9399098647972	9.65829225674077\\
3.94191286930396	8.91304605261411\\
3.94391587381072	8.19667692136204\\
3.94591887831748	7.50970052500019\\
3.94792188282424	6.85263252554416\\
3.949924887331	6.22587399345055\\
3.95192789183776	5.62994059073498\\
3.95393089634452	5.06511879629502\\
3.95593390085128	4.53180968058725\\
3.95793690535804	4.03029972250924\\
3.9599399098648	3.5606462178405\\
3.96194291437156	3.12307834969909\\
3.96394591887832	2.71748152652598\\
3.96594892338508	2.34374115676215\\
3.96795192789184	2.00162805728953\\
3.9699549323986	1.69074115765155\\
3.97195793690536	1.41050750005306\\
3.97396094141212	1.16029683091943\\
3.97596394591888	0.939249713557959\\
3.97796695042564	0.746220232378384\\
3.9799699549324	0.580119767569959\\
3.98197295943916	0.439401333085828\\
3.98397596394592	0.322632534438167\\
3.98597896845268	0.227922610903041\\
3.98798197295944	0.153380801756521\\
3.9899849774662	0.0968298673771091\\
3.99198798197296	0.0560352723637945\\
3.99399098647972	0.0285905939770281\\
3.99599399098648	0.0119748179182342\\
3.99799699549324	0.00343774677078494\\
4	0.000401070456591576\\
};
\addlegendentry{$\phi$};

\addplot [color=mycolor2,solid]
  table[row sep=crcr]{%
0	13165.5679716269\\
0.00200300450676014	13158.5356595368\\
0.00400600901352028	13151.4791113319\\
0.00600901352028042	13144.3996448153\\
0.00801201802704056	13137.2988069729\\
0.0100150225338007	13130.1782020866\\
0.0120180270405608	13123.039720917\\
0.014021031547321	13115.8854261124\\
0.0160240360540811	13108.7174376164\\
0.0180270405608413	13101.5380472603\\
0.0200300450676014	13094.3497187626\\
0.0220330495743615	13087.1549731375\\
0.0240360540811217	13079.9563313995\\
0.0260390585878818	13072.7564291543\\
0.028042063094642	13065.5579020079\\
0.0300450676014021	13058.3634428618\\
0.0320480721081622	13051.1755154345\\
0.0340510766149224	13043.9966980362\\
0.0360540811216825	13036.8293397938\\
0.0380570856284427	13029.67584713\\
0.0400600901352028	13022.5382826929\\
0.0420630946419629	13015.4187091307\\
0.0440660991487231	13008.3189599081\\
0.0460691036554832	13001.2406966028\\
0.0480721081622434	12994.1854662009\\
0.0500751126690035	12987.154414618\\
0.0520781171757636	12980.1488023611\\
0.0540811216825238	12973.1694888671\\
0.0560841261892839	12966.2172189809\\
0.0580871306960441	12959.2925656605\\
0.0600901352028042	12952.3959872721\\
0.0620931397095643	12945.5275984072\\
0.0640961442163245	12938.6876282489\\
0.0660991487230846	12931.8759622057\\
0.0681021532298448	12925.0924283903\\
0.0701051577366049	12918.3367976195\\
0.0721081622433651	12911.6086688228\\
0.0741111667501252	12904.9075836341\\
0.0761141712568853	12898.2329690955\\
0.0781171757636455	12891.5842522492\\
0.0801201802704056	12884.9607455459\\
0.0821231847771658	12878.3617041405\\
0.0841261892839259	12871.7863831878\\
0.086129193790686	12865.2339805469\\
0.0881321982974462	12858.7036367811\\
0.0901352028042063	12852.1944924538\\
0.0921382073109664	12845.7056881281\\
0.0941412118177266	12839.2363643675\\
0.0961442163244867	12832.7856044395\\
0.0981472208312469	12826.3525489073\\
0.100150225338007	12819.9362237428\\
0.102153229844767	12813.5357695092\\
0.104156234351527	12807.1503267698\\
0.106159238858287	12800.7789787921\\
0.108162243365048	12794.4208088438\\
0.110165247871808	12788.0750147838\\
0.112168252378568	12781.7406798797\\
0.114171256885328	12775.4169446949\\
0.116174261392088	12769.1029497925\\
0.118177265898848	12762.7979503276\\
0.120180270405608	12756.5010295675\\
0.122183274912369	12750.2113853715\\
0.124186279419129	12743.9282155985\\
0.126189283925889	12737.6507181077\\
0.128192288432649	12731.378148054\\
0.130195292939409	12725.1096460006\\
0.132198297446169	12718.8445243982\\
0.134201301952929	12712.5819811058\\
0.13620430645969	12706.3213285742\\
0.13820731096645	12700.0617073666\\
0.14021031547321	12693.8024872295\\
0.14221331997997	12687.5428660219\\
0.14421632448673	12681.2822134903\\
0.14621932899349	12675.0197274937\\
0.14822233350025	12668.7547204829\\
0.150225338007011	12662.4865049084\\
0.152228342513771	12656.2143932208\\
0.154231347020531	12649.937697871\\
0.156234351527291	12643.6557313094\\
0.158237356034051	12637.3678059867\\
0.160240360540811	12631.0732916494\\
0.162243365047571	12624.7715580439\\
0.164246369554332	12618.461860325\\
0.166249374061092	12612.143625535\\
0.168252378567852	12605.8162234202\\
0.170255383074612	12599.479081023\\
0.172258387581372	12593.1315107938\\
0.174261392088132	12586.7729970708\\
0.176264396594892	12580.4029096003\\
0.178267401101652	12574.0207327204\\
0.180270405608413	12567.6259507689\\
0.182273410115173	12561.2180480839\\
0.184276414621933	12554.7965662992\\
0.186279419128693	12548.3609897528\\
0.188282423635453	12541.9109173741\\
0.190285428142213	12535.4459480927\\
0.192288432648973	12528.9657954298\\
0.194291437155734	12522.4701156106\\
0.196294441662494	12515.9586794521\\
0.198297446169254	12509.4312004753\\
0.200300450676014	12502.8875640886\\
0.202303455182774	12496.3276557006\\
0.204306459689534	12489.7514180154\\
0.206309464196294	12483.1588510331\\
0.208312468703055	12476.5501266409\\
0.210315473209815	12469.9253021347\\
0.212318477716575	12463.2846639934\\
0.214321482223335	12456.6284414\\
0.216324486730095	12449.9571500166\\
0.218327491236855	12443.2710763221\\
0.220330495743615	12436.5709078659\\
0.222333500250376	12429.8571603099\\
0.224336504757136	12423.1305784993\\
0.226339509263896	12416.3919645749\\
0.228342513770656	12409.6421206779\\
0.230345518277416	12402.8819635404\\
0.232348522784176	12396.1125244868\\
0.234351527290936	12389.3348348408\\
0.236354531797697	12382.5499832224\\
0.238357536304457	12375.759172843\\
0.240360540811217	12368.9635496181\\
0.242363545317977	12362.1644313506\\
0.244366549824737	12355.3630212519\\
0.246369554331497	12348.5605798292\\
0.248372558838257	12341.7584248854\\
0.250375563345018	12334.9578742235\\
0.252378567851778	12328.1601310547\\
0.254381572358538	12321.366513182\\
0.256384576865298	12314.5782238169\\
0.258387581372058	12307.7964088748\\
0.260390585878818	12301.0222715672\\
0.262393590385578	12294.2568432181\\
0.264396594892339	12287.5011551515\\
0.266399599399099	12280.7561813956\\
0.268402603905859	12274.022838683\\
0.270405608412619	12267.3018718588\\
0.272408612919379	12260.5940830639\\
0.274411617426139	12253.9001025518\\
0.276414621932899	12247.2205605762\\
0.27841762643966	12240.5559155032\\
0.28042063094642	12233.9066256992\\
0.28242363545318	12227.2729776429\\
0.28442663995994	12220.6552578134\\
0.2864296444667	12214.0537526894\\
0.28843264897346	12207.4684622711\\
0.29043565348022	12200.8995011499\\
0.29243865798698	12194.3468120301\\
0.294441662493741	12187.8103949117\\
0.296444667000501	12181.2900779074\\
0.298447671507261	12174.7856318339\\
0.300450676014021	12168.2968848041\\
0.302453680520781	12161.8235503389\\
0.304456685027541	12155.3652846637\\
0.306459689534301	12148.9217440039\\
0.308462694041062	12142.4924699932\\
0.310465698547822	12136.0770615611\\
0.312468703054582	12129.6750603414\\
0.314471707561342	12123.2859506722\\
0.316474712068102	12116.909274187\\
0.318477716574862	12110.5444006325\\
0.320480721081622	12104.1908143465\\
0.322483725588383	12097.8479423713\\
0.324486730095143	12091.515211749\\
0.326489734601903	12085.1919922262\\
0.328492739108663	12078.8776535492\\
0.330495743615423	12072.5715654644\\
0.332498748122183	12066.2730977183\\
0.334501752628943	12059.9816200573\\
0.336504757135704	12053.6964449321\\
0.338507761642464	12047.4168847932\\
0.340510766149224	12041.1423093872\\
0.342513770655984	12034.8720311646\\
0.344516775162744	12028.6053052804\\
0.346519779669504	12022.3415587767\\
0.348522784176264	12016.0799322168\\
0.350525788683025	12009.8197953472\\
0.352528793189785	12003.5604606185\\
0.354531797696545	11997.3011831856\\
0.356534802203305	11991.0411609076\\
0.358537806710065	11984.7797635308\\
0.360540811216825	11978.5161889144\\
0.362543815723585	11972.2496349175\\
0.364546820230346	11965.9794139907\\
0.366549824737106	11959.7047812889\\
0.368552829243866	11953.4248773754\\
0.370555833750626	11947.1389001092\\
0.372558838257386	11940.8461046453\\
0.374561842764146	11934.5456888427\\
0.376564847270906	11928.2367932647\\
0.378567851777666	11921.9185584747\\
0.380570856284427	11915.5902396275\\
0.382573860791187	11909.2508626948\\
0.384576865297947	11902.8996828316\\
0.386579869804707	11896.5357833053\\
0.388582874311467	11890.1582473835\\
0.390585878818227	11883.7662156294\\
0.392588883324987	11877.3588286065\\
0.394591887831748	11870.9351695822\\
0.396594892338508	11864.494321824\\
0.398597896845268	11858.0354258952\\
0.400600901352028	11851.5575077677\\
0.402603905858788	11845.0598225963\\
0.404606910365548	11838.5413390568\\
0.406609914872308	11832.0012550085\\
0.408612919379069	11825.4387683104\\
0.410615923885829	11818.8529622301\\
0.412618928392589	11812.2431492181\\
0.414621932899349	11805.6084698378\\
0.416624937406109	11798.9483511315\\
0.418627941912869	11792.2619909581\\
0.420630946419629	11785.5489882472\\
0.42263395092639	11778.8086554495\\
0.42463695543315	11772.0406487903\\
0.42663995993991	11765.2446244949\\
0.42864296444667	11758.4203533802\\
0.43064596895343	11751.5677208547\\
0.43264897346019	11744.6867842141\\
0.43465197796695	11737.7776580499\\
0.436654982473711	11730.8406288411\\
0.438657986980471	11723.8762695455\\
0.440660991487231	11716.8851531209\\
0.442663995993991	11709.8680817082\\
0.444667000500751	11702.8261439271\\
0.446670005007511	11695.7604283976\\
0.448673009514271	11688.6724248098\\
0.450676014021032	11681.563622854\\
0.452679018527792	11674.4358559953\\
0.454682023034552	11667.2909004026\\
0.456685027541312	11660.1309906116\\
0.458688032048072	11652.9582465659\\
0.460691036554832	11645.7750746883\\
0.462694041061592	11638.5839386974\\
0.464697045568353	11631.3873023117\\
0.466700050075113	11624.1878584327\\
0.468703054581873	11616.9882426665\\
0.470706059088633	11609.7909760272\\
0.472709063595393	11602.5987514164\\
0.474712068102153	11595.4140898486\\
0.476715072608913	11588.239512338\\
0.478718077115673	11581.0773107157\\
0.480721081622434	11573.929776813\\
0.482724086129194	11566.7990305737\\
0.484727090635954	11559.6870200543\\
0.486730095142714	11552.595521424\\
0.488733099649474	11545.5260816688\\
0.490736104156234	11538.4801331832\\
0.492739108662994	11531.4589364743\\
0.494742113169755	11524.4635228662\\
0.496745117676515	11517.4947517956\\
0.498748122183275	11510.5531962202\\
0.500751126690035	11503.6394863937\\
0.502754131196795	11496.753908795\\
0.504757135703555	11489.8966926071\\
0.506760140210315	11483.0677805342\\
0.508763144717076	11476.2671725765\\
0.510766149223836	11469.4945822549\\
0.512769153730596	11462.7497803864\\
0.514772158237356	11456.0323086048\\
0.516775162744116	11449.3416512479\\
0.518778167250876	11442.6772926538\\
0.520781171757636	11436.0384879774\\
0.522784176264397	11429.4246642609\\
0.524787180771157	11422.8350193633\\
0.526790185277917	11416.2687511438\\
0.528793189784677	11409.7250574614\\
0.530796194291437	11403.2031361752\\
0.532799198798197	11396.7020705527\\
0.534802203304957	11390.2210011573\\
0.536805207811718	11383.7590112565\\
0.538808212318478	11377.3152987093\\
0.540811216825238	11370.8888894876\\
0.542814221331998	11364.4788668588\\
0.544817225838758	11358.084428682\\
0.546820230345518	11351.704600929\\
0.548823234852278	11345.3385241631\\
0.550826239359039	11338.9852816518\\
0.552829243865799	11332.6440712542\\
0.554832248372559	11326.3139762378\\
0.556835252879319	11319.9941944617\\
0.558838257386079	11313.6838091935\\
0.560841261892839	11307.3820755879\\
0.562844266399599	11301.0880769127\\
0.56484727090636	11294.8010683225\\
0.56685027541312	11288.5201903807\\
0.56885327991988	11282.244755538\\
0.57085628442664	11275.9738470618\\
0.5728592889334	11269.7067774029\\
0.57486229344016	11263.4427444203\\
0.57686529794692	11257.1810605646\\
0.57886830245368	11250.9208663993\\
0.580871306960441	11244.6615316706\\
0.582874311467201	11238.4021969419\\
0.584877315973961	11232.1422892554\\
0.586880320480721	11225.8809491744\\
0.588883324987481	11219.6175464454\\
0.590886329494241	11213.3513935189\\
0.592889334001002	11207.0817455499\\
0.594892338507762	11200.807914989\\
0.596895343014522	11194.5292142869\\
0.598898347521282	11188.2450131899\\
0.600901352028042	11181.9546241487\\
0.602904356534802	11175.657359614\\
0.604907361041562	11169.3525320364\\
0.606910365548322	11163.0396257538\\
0.608913370055083	11156.7178959213\\
0.610916374561843	11150.3867122851\\
0.612919379068603	11144.0455591832\\
0.614922383575363	11137.6937490664\\
0.616925388082123	11131.3307662726\\
0.618928392588883	11124.956037844\\
0.620931397095643	11118.5689908227\\
0.622934401602404	11112.1691668427\\
0.624937406109164	11105.755992946\\
0.626940410615924	11099.3290107665\\
0.628943415122684	11092.8878192336\\
0.630946419629444	11086.4319599812\\
0.632949424136204	11079.9610892345\\
0.634952428642964	11073.4749205148\\
0.636955433149725	11066.9730527514\\
0.638958437656485	11060.4553713529\\
0.640961442163245	11053.9215898403\\
0.642964446670005	11047.3715936222\\
0.644967451176765	11040.8052681069\\
0.646970455683525	11034.2226705902\\
0.648973460190285	11027.6237437763\\
0.650976464697046	11021.0087168484\\
0.652979469203806	11014.3776471022\\
0.654982473710566	11007.7308783125\\
0.656985478217326	11001.068696958\\
0.658988482724086	10994.3915614051\\
0.660991487230846	10987.69993002\\
0.662994491737606	10980.9943757605\\
0.664997496244367	10974.2755288801\\
0.667000500751127	10967.544134224\\
0.669003505257887	10960.8009939331\\
0.671006509764647	10954.0469674441\\
0.673009514271407	10947.283086081\\
0.675012518778167	10940.5102665764\\
0.677015523284927	10933.7296548457\\
0.679018527791687	10926.9423395088\\
0.681021532298448	10920.1494664813\\
0.683024536805208	10913.3523535661\\
0.685027541311968	10906.5521466788\\
0.687030545818728	10899.7502209181\\
0.689033550325488	10892.9477221996\\
0.691036554832248	10886.1460829178\\
0.693039559339009	10879.3464489883\\
0.695042563845769	10872.5501955098\\
0.697045568352529	10865.7585256937\\
0.699048572859289	10858.9725854555\\
0.701051577366049	10852.1936925982\\
0.703054581872809	10845.42282115\\
0.705057586379569	10838.6610597308\\
0.707060590886329	10831.9094396645\\
0.70906359539309	10825.1689349795\\
0.71106659989985	10818.4403478166\\
0.71306960440661	10811.7244803167\\
0.71507260891337	10805.0220200293\\
0.71707561342013	10798.3336545036\\
0.71907861792689	10791.6598421059\\
0.72108162243365	10785.001155794\\
0.723084626940411	10778.3579393426\\
0.725087631447171	10771.7305365263\\
0.727090635953931	10765.1191192325\\
0.729093640460691	10758.5239739401\\
0.731096644967451	10751.9450433533\\
0.733099649474211	10745.3824420637\\
0.735102653980971	10738.8361700712\\
0.737105658487732	10732.3060554885\\
0.739108662994492	10725.7919837241\\
0.741111667501252	10719.2937255949\\
0.743114672008012	10712.8110519176\\
0.745117676514772	10706.3436762135\\
0.747120681021532	10699.8911974121\\
0.749123685528292	10693.4532717387\\
0.751126690035053	10687.0294981228\\
0.753129694541813	10680.6194181982\\
0.755132699048573	10674.2225163029\\
0.757135703555333	10667.8383340707\\
0.759138708062093	10661.4662985437\\
0.761141712568853	10655.1059513557\\
0.763144717075613	10648.7566049574\\
0.765147721582374	10642.4178009825\\
0.767150726089134	10636.0888518818\\
0.769153730595894	10629.7691846972\\
0.771156735102654	10623.4581118797\\
0.773159739609414	10617.155117767\\
0.775162744116174	10610.8594575141\\
0.777165748622934	10604.5705008674\\
0.779168753129695	10598.2876175733\\
0.781171757636455	10592.0101200825\\
0.783174762143215	10585.7373208457\\
0.785177766649975	10579.4685323133\\
0.787180771156735	10573.2030096405\\
0.789183775663495	10566.9401798692\\
0.791186780170255	10560.6792408587\\
0.793189784677016	10554.4195050596\\
0.795192789183776	10548.1602849225\\
0.797195793690536	10541.9007783064\\
0.799198798197296	10535.6402976622\\
0.801201802704056	10529.3781554403\\
0.803204807210816	10523.1134922041\\
0.805207811717576	10516.8456204043\\
0.807210816224337	10510.5737378999\\
0.809213820731097	10504.2971571415\\
0.811216825237857	10498.0149613968\\
0.813219829744617	10491.7264058206\\
0.815222834251377	10485.4307455677\\
0.817225838758137	10479.1270639057\\
0.819228843264897	10472.8146159894\\
0.821231847771657	10466.4925423821\\
0.823234852278418	10460.160040943\\
0.825237856785178	10453.8161376437\\
0.827240861291938	10447.4600876392\\
0.829243865798698	10441.0909169012\\
0.831246870305458	10434.7078232888\\
0.833249874812218	10428.3098900695\\
0.835252879318979	10421.8962578065\\
0.837255883825739	10415.4659524718\\
0.839258888332499	10409.0181719243\\
0.841261892839259	10402.5519421358\\
0.843264897346019	10396.0664036696\\
0.845267901852779	10389.5606397932\\
0.847270906359539	10383.03379107\\
0.849273910866299	10376.485055359\\
0.85127691537306	10369.9135159277\\
0.85327991987982	10363.3184279311\\
0.85528292438658	10356.6989319324\\
0.85728592889334	10350.054397678\\
0.8592889334001	10343.3840803229\\
0.86129193790686	10336.6873496134\\
0.86329494241362	10329.9636898875\\
0.865297946920381	10323.2125854833\\
0.867300951427141	10316.4336353302\\
0.869303955933901	10309.6266102451\\
0.871306960440661	10302.7912237492\\
0.873309964947421	10295.9275331382\\
0.875312969454181	10289.0354811164\\
0.877315973960941	10282.1154114584\\
0.879318978467702	10275.1676106431\\
0.881321982974462	10268.192708924\\
0.883324987481222	10261.1913365549\\
0.885327991987982	10254.1643529723\\
0.887330996494742	10247.1129613876\\
0.889334001001502	10240.0383650122\\
0.891337005508262	10232.9420535364\\
0.893340010015023	10225.8255739462\\
0.895343014521783	10218.6908170023\\
0.897346019028543	10211.5397880571\\
0.899349023535303	10204.3745497585\\
0.901352028042063	10197.1974512336\\
0.903355032548823	10190.0108416093\\
0.905358037055583	10182.8172991956\\
0.907361041562344	10175.6192304154\\
0.909364046069104	10168.419385466\\
0.911367050575864	10161.2203999533\\
0.913370055082624	10154.0247948916\\
0.915373059589384	10146.8352058867\\
0.917376064096144	10139.6542112488\\
0.919379068602904	10132.4842174005\\
0.921382073109664	10125.3275161731\\
0.923385077616425	10118.1863421019\\
0.925388082123185	10111.0627005393\\
0.927391086629945	10103.9584822459\\
0.929394091136705	10096.8754060952\\
0.931397095643465	10089.8150190731\\
0.933400100150225	10082.7786389827\\
0.935403104656986	10075.7674690352\\
0.937406109163746	10068.7823686674\\
0.939409113670506	10061.8241400202\\
0.941412118177266	10054.893470643\\
0.943415122684026	10047.9907616062\\
0.945418127190786	10041.1162420928\\
0.947421131697546	10034.2700839904\\
0.949424136204306	10027.4522872988\\
0.951427140711067	10020.6626801307\\
0.953430145217827	10013.901033303\\
0.955433149724587	10007.1669457453\\
0.957436154231347	10000.460016387\\
0.959439158738107	9993.77972956628\\
0.961442163244867	9987.12545502941\\
0.963445167751627	9980.49656252286\\
0.965448172258388	9973.89224990572\\
0.967451176765148	9967.31188692443\\
0.969454181271908	9960.75455684649\\
0.971457185778668	9954.21945753101\\
0.973460190285428	9947.70578683707\\
0.975463194792188	9941.21268532797\\
0.977466199298948	9934.73917897546\\
0.979469203805709	9928.28446563863\\
0.981472208312469	9921.84762858502\\
0.983475212819229	9915.42775108214\\
0.985478217325989	9909.02391639752\\
0.987481221832749	9902.63532239025\\
0.989484226339509	9896.26099503208\\
0.991487230846269	9889.90001759054\\
0.993490235353029	9883.55158792471\\
0.99549323985979	9877.2148465979\\
0.99749624436655	9870.88887687764\\
0.99949924887331	9864.57287662301\\
1.00150225338007	9858.26598639733\\
1.00350525788683	9851.9673467639\\
1.00550826239359	9845.67621287759\\
1.00751126690035	9839.39172530169\\
1.00951427140711	9833.11302459953\\
1.01151727591387	9826.83948051753\\
1.01352028042063	9820.5701763232\\
1.01552328492739	9814.30436717143\\
1.01752628943415	9808.04136551286\\
1.01952929394091	9801.78031191079\\
1.02153229844767	9795.52057611164\\
1.02353530295443	9789.26129867874\\
1.02553830746119	9783.00184935849\\
1.02754131196795	9776.74148330578\\
1.02954431647471	9770.47939837967\\
1.03154732098147	9764.21502162239\\
1.03355032548823	9757.94755089301\\
1.03555332999499	9751.67635593797\\
1.03755633450175	9745.40069191212\\
1.03955933900851	9739.1199285619\\
1.04156234351527	9732.83337833794\\
1.04356534802203	9726.5403536909\\
1.04556835252879	9720.2402243672\\
1.04757135703555	9713.93230281749\\
1.04957436154231	9707.61601608397\\
1.05157736604907	9701.29067661728\\
1.05358037055583	9694.95565416386\\
1.05558337506259	9688.61043306168\\
1.05758637956935	9682.2543257614\\
1.05958938407611	9675.88681660099\\
1.06159238858287	9669.50733262266\\
1.06359539308963	9663.11530086863\\
1.06559839759639	9656.71032026842\\
1.06760140210315	9650.29176056846\\
1.06960440660991	9643.85927799409\\
1.07160741111667	9637.41241417906\\
1.07361041562344	9630.95071075713\\
1.0756134201302	9624.47388124941\\
1.07761642463696	9617.98158188123\\
1.07961942914372	9611.47358346945\\
1.08162243365048	9604.9495995352\\
1.08362543815724	9598.40945819112\\
1.085628442664	9591.85310214143\\
1.08763144717076	9585.28041679459\\
1.08963445167752	9578.69140215059\\
1.09163745618428	9572.08617280098\\
1.09364046069104	9565.46484333733\\
1.0956434651978	9558.82758564698\\
1.09764646970456	9552.17474350459\\
1.09964947421132	9545.50666068486\\
1.10165247871808	9538.82379555402\\
1.10365548322484	9532.12660647829\\
1.1056584877316	9525.41578100704\\
1.10766149223836	9518.69189209806\\
1.10966449674512	9511.95585648383\\
1.11166750125188	9505.20836171369\\
1.11367050575864	9498.45032452012\\
1.1156735102654	9491.68283352293\\
1.11767651477216	9484.9068054546\\
1.11967951927892	9478.12344352649\\
1.12168252378568	9471.33383635841\\
1.12368552829244	9464.53912986595\\
1.1256885327992	9457.74069914782\\
1.12769153730596	9450.93963282384\\
1.12969454181272	9444.13730599271\\
1.13169754631948	9437.33503645736\\
1.13370055082624	9430.53402742916\\
1.135703555333	9423.73553941525\\
1.13770655983976	9416.94094751435\\
1.13970956434652	9410.15139764205\\
1.14171256885328	9403.36809300972\\
1.14371557336004	9396.59217953295\\
1.1457185778668	9389.82474583153\\
1.14772158237356	9383.06682322953\\
1.14972458688032	9376.31950034673\\
1.15172759138708	9369.58352202827\\
1.15373059589384	9362.85980500663\\
1.1557336004006	9356.14909412694\\
1.15773660490736	9349.45207693856\\
1.15973960941412	9342.76938369505\\
1.16174261392088	9336.10147276265\\
1.16374561842764	9329.44880250761\\
1.1657486229344	9322.81171670459\\
1.16775162744116	9316.19055912828\\
1.16975463194792	9309.58550166602\\
1.17175763645468	9302.99665890935\\
1.17376064096144	9296.42414544984\\
1.1757636454682	9289.8679612875\\
1.17776664997496	9283.32799183076\\
1.17976965448172	9276.80417978384\\
1.18177265898848	9270.29635325941\\
1.18377566349524	9263.80422577856\\
1.185778668002	9257.32756815818\\
1.18778167250876	9250.86609391937\\
1.18978467701552	9244.4193446959\\
1.19178768152228	9237.98697671308\\
1.19379068602904	9231.56864619625\\
1.1957936905358	9225.1637801876\\
1.19779669504256	9218.7719203209\\
1.19979969954932	9212.39255093413\\
1.20180270405608	9206.02515636528\\
1.20380570856284	9199.66916365656\\
1.2058087130696	9193.32405714594\\
1.20781171757636	9186.98914928408\\
1.20981472208312	9180.66392440895\\
1.21181772658988	9174.34769497121\\
1.21382073109664	9168.03988801305\\
1.21582373560341	9161.73981598513\\
1.21782674011017	9155.44684863387\\
1.21982974461693	9149.1603557057\\
1.22183274912369	9142.87964965125\\
1.22383575363045	9136.60410021696\\
1.22583875813721	9130.33290526192\\
1.22784176264397	9124.06549182832\\
1.22984476715073	9117.80111507104\\
1.23184777165749	9111.53908744071\\
1.23385077616425	9105.278721388\\
1.23585378067101	9099.01927206775\\
1.23785678517777	9092.76005193063\\
1.23985978968453	9086.5002588357\\
1.24186279419129	9080.23920523363\\
1.24386579869805	9073.97614627928\\
1.24586880320481	9067.71033712751\\
1.24787180771157	9061.44097563741\\
1.24987481221833	9055.16737425962\\
1.25187781672509	9048.88867355746\\
1.25388082123185	9042.60407139001\\
1.25588382573861	9036.31282291213\\
1.25788683024537	9030.01412598292\\
1.25988983475213	9023.70712116568\\
1.26189283925889	9017.3910063195\\
1.26389584376565	9011.06492200768\\
1.26589884827241	9004.72806608931\\
1.26790185277917	8998.37952183192\\
1.26990485728593	8992.01842979882\\
1.27190786179269	8985.64393055331\\
1.27391086629945	8979.25516465871\\
1.27591387080621	8972.85115808675\\
1.27791687531297	8966.43110869653\\
1.27991987981973	8959.99409975557\\
1.28192288432649	8953.53915723563\\
1.28392588883325	8947.06547899579\\
1.28592889334001	8940.57214830356\\
1.28793189784677	8934.05824842651\\
1.28993490235353	8927.5229199279\\
1.29193790686029	8920.96536066685\\
1.29394091136705	8914.38465391088\\
1.29594391587381	8907.78011211062\\
1.29794692038057	8901.15093312518\\
1.29994992488733	8894.49637210941\\
1.30195292939409	8887.81579880975\\
1.30395593390085	8881.10858297261\\
1.30595893840761	8874.37420893598\\
1.30796194291437	8867.6122756294\\
1.30996494742113	8860.82238198243\\
1.31196795192789	8854.00424151615\\
1.31397095643465	8847.15779693479\\
1.31597396094141	8840.2829336468\\
1.31797696544817	8833.37993813106\\
1.31997996995493	8826.44892497914\\
1.32198297446169	8819.49046714884\\
1.32398597896845	8812.50508030216\\
1.32598898347521	8805.49356658003\\
1.32799198798197	8798.45690001069\\
1.32999499248873	8791.39628380551\\
1.33199799699549	8784.31297847165\\
1.33400100150225	8777.20853099514\\
1.33600400600901	8770.08471754516\\
1.33800701051577	8762.94337158665\\
1.34001001502253	8755.78649847189\\
1.34201301952929	8748.61633273629\\
1.34401602403605	8741.43522350679\\
1.34601902854281	8734.24557720615\\
1.34802203304957	8727.04997214444\\
1.35002503755633	8719.85087204019\\
1.35202804206309	8712.65108438657\\
1.35403104656985	8705.45307290212\\
1.35603405107661	8698.25953048848\\
1.35803705558337	8691.07303545571\\
1.36004006009014	8683.89605152234\\
1.3620430645969	8676.73098511111\\
1.36404606910366	8669.58007075742\\
1.36604907361042	8662.44554299667\\
1.36805207811718	8655.32929258959\\
1.37005508262394	8648.23315300112\\
1.3720580871307	8641.15884310464\\
1.37406109163746	8634.10768070308\\
1.37606409614422	8627.08104089515\\
1.37806710065098	8620.07995500489\\
1.38007010515774	8613.10528246898\\
1.3820731096645	8606.15782542834\\
1.38407611417126	8599.23804224921\\
1.38607911867802	8592.34633400204\\
1.38808212318478	8585.48287257417\\
1.39008512769154	8578.64782985293\\
1.3920881321983	8571.84109124678\\
1.39409113670506	8565.06248486837\\
1.39609414121182	8558.31166694302\\
1.39809714571858	8551.58829369606\\
1.40010015022534	8544.89184946546\\
1.4021031547321	8538.22181858923\\
1.40410615923886	8531.57757081377\\
1.40610916374562	8524.95836129396\\
1.40811216825238	8518.36355977623\\
1.41011517275914	8511.79224952809\\
1.4121181772659	8505.2437430002\\
1.41412118177266	8498.71718075587\\
1.41612418627942	8492.21164606262\\
1.41812719078618	8485.72627948375\\
1.42013019529294	8479.26027887836\\
1.4221331997997	8472.81267021819\\
1.42413620430646	8466.38253677078\\
1.42613920881322	8459.9690763952\\
1.42814221331998	8453.57131506321\\
1.43014521782674	8447.18839333812\\
1.4321482223335	8440.81939448744\\
1.43415122684026	8434.4634590745\\
1.43615423134702	8428.11972766259\\
1.43815723585378	8421.78728351925\\
1.44016024036054	8415.46532450355\\
1.4421632448673	8409.15299117881\\
1.44416624937406	8402.84942410834\\
1.44616925388082	8396.55382115123\\
1.44817225838758	8390.26538016655\\
1.45017526289434	8383.98324171762\\
1.4521782674011	8377.7066609593\\
1.45418127190786	8371.43483575068\\
1.45618427641462	8365.16702124662\\
1.45818728092138	8358.90241530622\\
1.46019028542814	8352.64027308434\\
1.4621932899349	8346.37979244006\\
1.46419629444166	8340.12034311982\\
1.46619929894842	8333.86106568691\\
1.46820230345518	8327.60132988777\\
1.47020530796194	8321.34039087726\\
1.4722083124687	8315.07750381024\\
1.47421131697546	8308.81198113737\\
1.47621432148222	8302.54313530928\\
1.47821732598898	8296.27027877663\\
1.48022033049574	8289.99272399006\\
1.4822233350025	8283.70978340021\\
1.48422633950926	8277.42076945774\\
1.48622934401602	8271.12505190906\\
1.48823234852278	8264.82194320483\\
1.49023535302954	8258.51081309146\\
1.4922383575363	8252.19103131539\\
1.49424136204306	8245.86196762303\\
1.49624436654982	8239.5230490566\\
1.49824737105658	8233.17358806674\\
1.50025037556334	8226.81312628722\\
1.50225338007011	8220.44097616867\\
1.50425638457687	8214.05662204909\\
1.50625938908363	8207.65960556223\\
1.50826239359039	8201.24935375031\\
1.51026539809715	8194.82540824708\\
1.51226840260391	8188.38736798209\\
1.51427140711067	8181.93471729333\\
1.51627441161743	8175.46716970189\\
1.51827741612419	8168.98432413732\\
1.52028042063095	8162.48589412073\\
1.52228342513771	8155.97165046899\\
1.52428642964447	8149.44136399899\\
1.52628943415123	8142.89492011916\\
1.52829243865799	8136.33214694217\\
1.53029544316475	8129.75304446803\\
1.53229844767151	8123.15766999249\\
1.53430145217827	8116.54613810714\\
1.53630445668503	8109.91850610775\\
1.53830746119179	8103.27511776898\\
1.54031046569855	8096.61625956975\\
1.54231347020531	8089.94233258051\\
1.54431647471207	8083.25379516749\\
1.54631947921883	8076.55122028849\\
1.54832248372559	8069.8351809013\\
1.55032548823235	8063.10647914685\\
1.55232849273911	8056.36580257447\\
1.55433149724587	8049.61406791665\\
1.55633450175263	8042.85224920164\\
1.55833750625939	8036.0812631619\\
1.56034051076615	8029.30225571303\\
1.56234351527291	8022.51625817906\\
1.56434651977967	8015.72453106714\\
1.56634952428643	8008.92822029285\\
1.56835252879319	8002.12858636336\\
1.57035553329995	7995.3268897858\\
1.57235853780671	7988.52444836311\\
1.57436154231347	7981.72252260243\\
1.57636454682023	7974.92231571514\\
1.57836755132699	7968.12526009573\\
1.58037055583375	7961.33244436399\\
1.58237356034051	7954.54518632288\\
1.58437656484727	7947.76463188796\\
1.58637956935403	7940.99192697484\\
1.58838257386079	7934.22816020332\\
1.59038557836755	7927.47424830587\\
1.59238858287431	7920.73122260654\\
1.59439158738107	7913.99999983778\\
1.59639459188783	7907.28126754896\\
1.59839759639459	7900.57588517676\\
1.60040060090135	7893.88436838321\\
1.60240360540811	7887.20740471765\\
1.60440660991487	7880.54539525055\\
1.60640961442163	7873.89879834813\\
1.60841261892839	7867.26790048931\\
1.61041562343515	7860.65304544874\\
1.61241862794191	7854.05429052222\\
1.61442163244867	7847.47186489286\\
1.61642463695543	7840.90576856066\\
1.61842764146219	7834.35594422984\\
1.62043064596895	7827.82233460463\\
1.62243365047571	7821.30476779767\\
1.62443665498247	7814.80312921743\\
1.62643965948923	7808.31707508921\\
1.62844266399599	7801.8463762299\\
1.63044566850275	7795.3907461606\\
1.63244867300951	7788.9496692193\\
1.63445167751627	7782.5228589271\\
1.63645468202303	7776.10985691776\\
1.63845768652979	7769.71009023348\\
1.64046069103655	7763.3232150996\\
1.64246369554331	7756.94860126253\\
1.64446670005008	7750.58573306027\\
1.64646970455684	7744.23409483079\\
1.6484727090636	7737.89299902474\\
1.65047571357036	7731.56198727588\\
1.65247871807712	7725.24037203486\\
1.65448172258388	7718.92752304811\\
1.65648472709064	7712.62281006205\\
1.6584877315974	7706.32560282311\\
1.66049073610416	7700.0352710777\\
1.66249374061092	7693.75112727649\\
1.66449674511768	7687.4724838701\\
1.66649974962444	7681.19871060498\\
1.6685027541312	7674.92911993176\\
1.67050575863796	7668.66296700532\\
1.67250876314472	7662.39962157206\\
1.67451176765148	7656.13828149109\\
1.67651477215824	7649.87837380461\\
1.678517776665	7643.6190963717\\
1.68052078117176	7637.35970434724\\
1.68252378567852	7631.09951018186\\
1.68452679018528	7624.83776903043\\
1.68652979469204	7618.57373604783\\
1.6885327991988	7612.30660909313\\
1.69053580370556	7606.03570061698\\
1.69253880821232	7599.76015118269\\
1.69454181271908	7593.47927324091\\
1.69654481722584	7587.19220735494\\
1.6985478217326	7580.89809408809\\
1.70055082623936	7574.596245891\\
1.70255383074612	7568.28580332699\\
1.70455683525288	7561.96590695936\\
1.70655983975964	7555.63569735142\\
1.7085628442664	7549.29431506647\\
1.71056584877316	7542.9409579636\\
1.71256885327992	7536.57476660612\\
1.71457185778668	7530.19476696578\\
1.71657486229344	7523.80015690167\\
1.7185778668002	7517.38996238552\\
1.72058087130696	7510.96338127643\\
1.72258387581372	7504.51943954616\\
1.72458688032048	7498.057277758\\
1.72658988482724	7491.57603647526\\
1.728592889334	7485.07474166968\\
1.73059589384076	7478.55253390459\\
1.73259889834752	7472.00861103906\\
1.73460190285428	7465.44211363641\\
1.73660490736104	7458.85218225993\\
1.7386079118678	7452.2380720645\\
1.74061091637456	7445.59903820498\\
1.74261392088132	7438.93433583624\\
1.74461692538808	7432.24344929626\\
1.74661992989484	7425.52563373991\\
1.7486229344016	7418.78054539251\\
1.75062593890836	7412.00766859205\\
1.75262894341512	7405.20671685963\\
1.75463194792188	7398.37751830791\\
1.75663495242864	7391.51995834532\\
1.7586379569354	7384.63409426766\\
1.76064096144216	7377.72004066648\\
1.76264396594892	7370.77825590801\\
1.76464697045568	7363.80914106272\\
1.76664997496244	7356.81344097573\\
1.7686529794692	7349.79195778796\\
1.77065598397596	7342.74578011922\\
1.77265898848272	7335.67611118088\\
1.77466199298948	7328.58432607165\\
1.77666499749624	7321.47208636913\\
1.778668002003	7314.34116824249\\
1.78067100650976	7307.19346245246\\
1.78267401101652	7300.03114623864\\
1.78467701552328	7292.85639684068\\
1.78668002003005	7285.67167797707\\
1.78868302453681	7278.47939607058\\
1.79068602904357	7271.28207213548\\
1.79268903355033	7264.08239907343\\
1.79469203805709	7256.88301249027\\
1.79669504256385	7249.68643340031\\
1.79869804707061	7242.49535470518\\
1.80070105157737	7235.31229741918\\
1.80270405608413	7228.13966796506\\
1.80470706059089	7220.97987276555\\
1.80671006509765	7213.83514635604\\
1.80871306960441	7206.70755138462\\
1.81071607411117	7199.59897861199\\
1.81271907861793	7192.51120420732\\
1.81472208312469	7185.44577515667\\
1.81672508763145	7178.40406655873\\
1.81872809213821	7171.38728162488\\
1.82073109664497	7164.39645167915\\
1.82273410115173	7157.43237886245\\
1.82473710565849	7150.49569342836\\
1.82674011016525	7143.58685374312\\
1.82874311467201	7136.70620358139\\
1.83074611917877	7129.85391483052\\
1.83274912368553	7123.02993019474\\
1.83475212819229	7116.23424967403\\
1.83675513269905	7109.46652949372\\
1.83875813720581	7102.7265404707\\
1.84076114171257	7096.01376694295\\
1.84276414621933	7089.32775054423\\
1.84476715072609	7082.66786102097\\
1.84677015523285	7076.03352541537\\
1.84877315973961	7069.42394158652\\
1.85077616424637	7062.83842198506\\
1.85277916875313	7056.27616447009\\
1.85478217325989	7049.73630960491\\
1.85678517776665	7043.2180552486\\
1.85878818227341	7036.72054196448\\
1.86079118678017	7030.24291031585\\
1.86279419128693	7023.78418627446\\
1.86479719579369	7017.34351040361\\
1.86680020030045	7010.92002326662\\
1.86880320480721	7004.51280813101\\
1.87080620931397	6998.12100556009\\
1.87280921382073	6991.74358422983\\
1.87481221832749	6985.37985659087\\
1.87681522283425	6979.02879131918\\
1.87881822734101	6972.68952897808\\
1.88082123184777	6966.36132472242\\
1.88282423635453	6960.04320452395\\
1.88482724086129	6953.73442353754\\
1.88683024536805	6947.43406503072\\
1.88883324987481	6941.14138415836\\
1.89083625438157	6934.85546418798\\
1.89283925888833	6928.57561757023\\
1.89484226339509	6922.30098486842\\
1.89684526790185	6916.0308212374\\
1.89884827240861	6909.76438183205\\
1.90085127691537	6903.50080721568\\
1.90285428142213	6897.23940983894\\
1.90485728592889	6890.97944485667\\
1.90686029043565	6884.72011012799\\
1.90886329494241	6878.4607753993\\
1.91086629944917	6872.2006385297\\
1.91286930395593	6865.93901196984\\
1.91487230846269	6859.67520817035\\
1.91687531296945	6853.4084822861\\
1.91887831747621	6847.13814676775\\
1.92088132198297	6840.86351406594\\
1.92288432648973	6834.5838966313\\
1.92488733099649	6828.29866421028\\
1.92689033550325	6822.0071292535\\
1.92889334001002	6815.70854691585\\
1.93089634451678	6809.4024015353\\
1.93289934902354	6803.08794826672\\
1.9349023535303	6796.76461415232\\
1.93690535803706	6790.43176893852\\
1.93890836254382	6784.08883966752\\
1.94091136705058	6777.73513878998\\
1.94291437155734	6771.37015064387\\
1.9449173760641	6764.9933022714\\
1.94692038057086	6758.60402071478\\
1.94892338507762	6752.20190490355\\
1.95092638958438	6745.78632458413\\
1.95292939409114	6739.35693598185\\
1.9549323985979	6732.91316613891\\
1.95693540310466	6726.4547858722\\
1.95893840761142	6719.98127951969\\
1.96094141211818	6713.4923606025\\
1.96294441662494	6706.98774264171\\
1.9649474211317	6700.46725375001\\
1.96695042563846	6693.93066474426\\
1.96895343014522	6687.37780373713\\
1.97095643465198	6680.80867072861\\
1.97295943915874	6674.22320842294\\
1.9749624436655	6667.62147411588\\
1.97696544817226	6661.003582399\\
1.97896845267902	6654.3698197512\\
1.98097145718578	6647.72030076403\\
1.98297446169254	6641.05548380373\\
1.9849774661993	6634.37576994075\\
1.98698047070606	6627.68167483712\\
1.98898347521282	6620.97377145063\\
1.99098647971958	6614.25263273907\\
1.99298948422634	6607.51917543491\\
1.9949924887331	6600.77408708751\\
1.99699549323986	6594.01828442934\\
1.99899849774662	6587.25274148866\\
2.00100150225338	6580.47848958949\\
2.00300450676014	6573.69661735164\\
2.0050075112669	6566.90821339493\\
2.00701051577366	6560.11453822651\\
2.00901352028042	6553.31679505774\\
2.01101652478718	6546.51624439577\\
2.01301952929394	6539.71408945198\\
2.0150225338007	6532.91164802929\\
2.01702553830746	6526.11023793063\\
2.01902854281422	6519.31111966315\\
2.02103154732098	6512.51561102978\\
2.02303455182774	6505.72480065033\\
2.0250375563345	6498.94006362351\\
2.02704056084126	6492.16243127335\\
2.02904356534802	6485.39310681122\\
2.03104656985478	6478.63306426536\\
2.03304957436154	6471.88333495983\\
2.0350525788683	6465.14483562707\\
2.03705558337506	6458.41836840802\\
2.03905858788182	6451.70479273935\\
2.04106159238858	6445.00473887465\\
2.04306459689534	6438.31889436327\\
2.0450676014021	6431.64771757144\\
2.04707060590886	6424.99166686541\\
2.04907361041562	6418.3512006114\\
2.05107661492238	6411.72654799254\\
2.05307961942914	6405.11799548772\\
2.0550826239359	6398.5256576885\\
2.05708562844266	6391.94959189067\\
2.05908863294942	6385.38985539\\
2.06109163745618	6378.8463908907\\
2.06309464196294	6372.31908380124\\
2.0650976464697	6365.80776223425\\
2.06710065097646	6359.31225430241\\
2.06910365548322	6352.83227352682\\
2.07110665998998	6346.36753342858\\
2.07310966449675	6339.91763293723\\
2.07511266900351	6333.48217098232\\
2.07711567351027	6327.06080378917\\
2.07911867801703	6320.65307299155\\
2.08112168252379	6314.25840563165\\
2.08312468703055	6307.87640063903\\
2.08512769153731	6301.50648505588\\
2.08713069604407	6295.14802862864\\
2.08913370055083	6288.80063028684\\
2.09113670505759	6282.46354518536\\
2.09313970956435	6276.13625766217\\
2.09514271407111	6269.8181374637\\
2.09714571857787	6263.50855433638\\
2.09914872308463	6257.20687802663\\
2.10115172759139	6250.91242098511\\
2.10315473209815	6244.624610254\\
2.10515773660491	6238.34270098819\\
2.10716074111167	6232.06606293409\\
2.10916374561843	6225.79400854235\\
2.11116675012519	6219.52579296784\\
2.11316975463195	6213.26084325276\\
2.11517275913871	6206.9983572562\\
2.11717576364547	6200.73770472459\\
2.11917876815223	6194.478083517\\
2.12118177265899	6188.21886337988\\
2.12318477716575	6181.95924217229\\
2.12518778167251	6175.69858964068\\
2.12719078617927	6169.43598905256\\
2.12919379068603	6163.17086745014\\
2.13119679519279	6156.90236539673\\
2.13319979969955	6150.62968075142\\
2.13520280420631	6144.35218326063\\
2.13720580871307	6138.06895619189\\
2.13920881321983	6131.77919740428\\
2.14121181772659	6125.48221934845\\
2.14321482223335	6119.17704799616\\
2.14521782674011	6112.86299579804\\
2.14722083124687	6106.53914602162\\
2.14922383575363	6100.20469652599\\
2.15122684026039	6093.85878787446\\
2.15322984476715	6087.50044603877\\
2.15523284927391	6081.12898346958\\
2.15723585378067	6074.74336884285\\
2.15923885828743	6068.34274272188\\
2.16124186279419	6061.92624566999\\
2.16324486730095	6055.4929609547\\
2.16524787180771	6049.04202913933\\
2.16725087631447	6042.57247619561\\
2.16925388082123	6036.08349998263\\
2.17125688532799	6029.57412647215\\
2.17325988983475	6023.04355352325\\
2.17526289434151	6016.49086440345\\
2.17726589884827	6009.91525697186\\
2.17926890335503	6003.31598638332\\
2.18127190786179	5996.69219320115\\
2.18327491236855	5990.043189876\\
2.18527791687531	5983.36834615428\\
2.18728092138207	5976.66691719087\\
2.18928392588883	5969.93850191531\\
2.19128693039559	5963.18258466559\\
2.19328993490235	5956.39876437124\\
2.19529293940911	5949.58675455337\\
2.19729594391587	5942.7464979162\\
2.19929894842263	5935.87782257239\\
2.20130195292939	5928.98090040928\\
2.20330495743615	5922.05590331421\\
2.20530796194291	5915.1033469492\\
2.20731096644967	5908.12374697625\\
2.20931397095643	5901.11784824051\\
2.21131697546319	5894.08656747445\\
2.21331997996995	5887.03105059365\\
2.21532298447672	5879.95250080948\\
2.21732598898348	5872.85257969956\\
2.21932899349024	5865.73277695415\\
2.221331997997	5858.59504062974\\
2.22333500250376	5851.44137607864\\
2.22533800701052	5844.27390324467\\
2.22734101151728	5837.0949712548\\
2.22934401602404	5829.90698653176\\
2.2313470205308	5822.71241279409\\
2.23335002503756	5815.51388564762\\
2.23535302954432	5808.31398340245\\
2.23735603405108	5801.1153989602\\
2.23935903855784	5793.92065333519\\
2.2413620430646	5786.73238213326\\
2.24336504757136	5779.55316366449\\
2.24536805207812	5772.38528976006\\
2.24737105658488	5765.23110954694\\
2.24937406109164	5758.09285756052\\
2.2513770655984	5750.97248185732\\
2.25338007010516	5743.87187319804\\
2.25538307461192	5736.79263586452\\
2.25738607911868	5729.73637413859\\
2.25938908362544	5722.70429123161\\
2.2613920881322	5715.69759035495\\
2.26339509263896	5708.71713094531\\
2.26539809714572	5701.76365784783\\
2.26740110165248	5694.83785861184\\
2.26940410615924	5687.94001971626\\
2.271407110666	5681.07042763998\\
2.27341011517276	5674.22925427034\\
2.27541311967952	5667.41638501578\\
2.27741612418628	5660.63164798896\\
2.27941912869304	5653.87481400676\\
2.2814221331998	5647.14553929451\\
2.28342513770656	5640.44330819019\\
2.28542814221332	5633.76760503178\\
2.28743114672008	5627.11779956571\\
2.28943415122684	5620.49326153841\\
2.2914371557336	5613.89318880896\\
2.29344016024036	5607.31683653223\\
2.29544316474712	5600.76345986308\\
2.29744616925388	5594.23219936482\\
2.29944917376064	5587.72219560077\\
2.3014521782674	5581.23253183844\\
2.30345518277416	5574.76240593692\\
2.30545818728092	5568.31084386797\\
2.30746119178768	5561.87704349067\\
2.30946419629444	5555.46003077676\\
2.3114672008012	5549.05894628956\\
2.31347020530796	5542.67287329659\\
2.31547320981472	5536.30089506538\\
2.31747621432148	5529.94220945502\\
2.31947921882824	5523.59589973303\\
2.321482223335	5517.26104916695\\
2.32348522784176	5510.93691291163\\
2.32548823234852	5504.62257423461\\
2.32749123685528	5498.3171736992\\
2.32949424136204	5492.01990916448\\
2.3314972458688	5485.72997848953\\
2.33350025037556	5479.44652223767\\
2.33550325488232	5473.16879556376\\
2.33750625938908	5466.89605362267\\
2.33950926389584	5460.6274369777\\
2.3415122684026	5454.36214348794\\
2.34351527290936	5448.09954289982\\
2.34551827741612	5441.83877577665\\
2.34752128192288	5435.57915456907\\
2.34952428642964	5429.31993443194\\
2.3515272909364	5423.06037052014\\
2.35353029544316	5416.79971798852\\
2.35553329994992	5410.53734658352\\
2.35753630445668	5404.27245416422\\
2.35953930896345	5398.00441047705\\
2.36154231347021	5391.73247067687\\
2.36354531797697	5385.45594721433\\
2.36554832248373	5379.17420983585\\
2.36755132699049	5372.88657099209\\
2.36955433149725	5366.59240042946\\
2.37155733600401	5360.29095330283\\
2.37356034051077	5353.98159935863\\
2.37556334501753	5347.66376563906\\
2.37756634952429	5341.33676459477\\
2.37956935403105	5335.00002326796\\
2.38157235853781	5328.65291140506\\
2.38357536304457	5322.29491334405\\
2.38557836755133	5315.92534153558\\
2.38758137205809	5309.54368031764\\
2.38958437656485	5303.14947132398\\
2.39158738107161	5296.74214159681\\
2.39359038557837	5290.32117547412\\
2.39559339008513	5283.88622918122\\
2.39759639459189	5277.43678705611\\
2.39959939909865	5270.97250532411\\
2.40160240360541	5264.49298291475\\
2.40360540811217	5257.99799064493\\
2.40560841261893	5251.48718473996\\
2.40761141712569	5244.96039331251\\
2.40961442163245	5238.41744447523\\
2.41161742613921	5231.85822363657\\
2.41362043064597	5225.28267350075\\
2.41562343515273	5218.69085136355\\
2.41762643965949	5212.08281452075\\
2.41962944416625	5205.4586775639\\
2.42163244867301	5198.81866967613\\
2.42363545317977	5192.16313463211\\
2.42563845768653	5185.4924735023\\
2.42764146219329	5178.80703006138\\
2.42964446670005	5172.10743456291\\
2.43164747120681	5165.3942599647\\
2.43365047571357	5158.66825111188\\
2.43565348022033	5151.93009555381\\
2.43765648472709	5145.18071002295\\
2.43965948923385	5138.42095395599\\
2.44166249374061	5131.65191597276\\
2.44366549824737	5124.87451280571\\
2.44566850275413	5118.08994766623\\
2.44767150726089	5111.2993664699\\
2.44967451176765	5104.50397242809\\
2.45167751627441	5097.70496875217\\
2.45368052078117	5090.90361594929\\
2.45568352528793	5084.1011745266\\
2.45768652979469	5077.29901958281\\
2.45968953430145	5070.49835432928\\
2.46169253880821	5063.70049656895\\
2.46369554331497	5056.90664951319\\
2.46569854782173	5050.11813096492\\
2.46770155232849	5043.33602954395\\
2.46970455683525	5036.56149116588\\
2.47170756134201	5029.7956617463\\
2.47371056584877	5023.03957260924\\
2.47571357035553	5016.29408319138\\
2.47771657486229	5009.56022481677\\
2.47971957936905	5002.83874233053\\
2.48172258387581	4996.1303805778\\
2.48372558838257	4989.43582710793\\
2.48572859288933	4982.75565487872\\
2.48773159739609	4976.09032225641\\
2.48973460190285	4969.440344903\\
2.49173760640961	4962.80606659318\\
2.49374061091637	4956.18765921429\\
2.49574361542313	4949.585466541\\
2.49774661992989	4942.9994885733\\
2.49974962443665	4936.429782607\\
2.50175262894342	4929.87646323363\\
2.50375563345018	4923.33930127008\\
2.50575863795694	4916.81829671636\\
2.5077616424637	4910.31322038934\\
2.50976464697046	4903.82384310591\\
2.51176765147722	4897.34982109139\\
2.51377065598398	4890.89092516266\\
2.51577366049074	4884.44675424926\\
2.5177766649975	4878.01684998497\\
2.51977966950426	4871.60086859509\\
2.52178267401102	4865.19829441762\\
2.52378567851778	4858.80866908633\\
2.52578868302454	4852.43147693918\\
2.5277916875313	4846.06608772262\\
2.52979469203806	4839.71204307039\\
2.53179769654482	4833.36871272894\\
2.53380070105158	4827.03558103624\\
2.53580370555834	4820.71196044294\\
2.5378067100651	4814.39722069547\\
2.53980971457186	4808.0908461318\\
2.54181271907862	4801.79203461103\\
2.54381572358538	4795.50027047114\\
2.54581872809214	4789.21486616278\\
2.5478217325989	4782.9350768408\\
2.54982473710566	4776.66032954743\\
2.55182774161242	4770.38987943752\\
2.55383074611918	4764.12303896171\\
2.55583375062594	4757.85912057067\\
2.5578367551327	4751.59743671502\\
2.55983975963946	4745.33724254964\\
2.56184276414622	4739.07790782096\\
2.56384576865298	4732.81863038805\\
2.56584877315974	4726.55866540579\\
2.5678517776665	4720.29732532482\\
2.56985478217326	4714.03386530001\\
2.57185778668002	4707.76748319044\\
2.57386079118678	4701.49743415099\\
2.57586379569354	4695.22297333651\\
2.5778668002003	4688.94329860609\\
2.57986980470706	4682.65760781883\\
2.58187280921382	4676.36515612959\\
2.58387581372058	4670.06502680589\\
2.58587881822734	4663.75653229838\\
2.5878818227341	4657.43869857881\\
2.58988482724086	4651.11083809783\\
2.59188783174762	4644.77197682718\\
2.59389083625438	4638.42131262595\\
2.59589384076114	4632.05792876166\\
2.5978968452679	4625.68102309342\\
2.59989984977466	4619.28962159295\\
2.60190285428142	4612.88292211936\\
2.60390585878818	4606.45995064438\\
2.60590886329494	4600.01984773134\\
2.6079118678017	4593.56175394352\\
2.60991487230846	4587.08469525268\\
2.61191787681522	4580.58781222213\\
2.61392088132198	4574.07024541518\\
2.61592388582874	4567.53113539513\\
2.6179268903355	4560.96956542952\\
2.61992989484226	4554.3847906732\\
2.62193289934902	4547.77600898526\\
2.62393590385578	4541.14247552057\\
2.62593890836254	4534.48338813822\\
2.6279419128693	4527.7982311762\\
2.62994491737606	4521.08631708514\\
2.63194792188282	4514.34718749881\\
2.63395092638958	4507.58038405097\\
2.63595393089634	4500.78556296696\\
2.6379569354031	4493.96249506364\\
2.63995993990986	4487.11106574947\\
2.64196294441662	4480.23133232022\\
2.64396594892338	4473.32340936744\\
2.64596895343015	4466.38758337004\\
2.64797195793691	4459.42431269426\\
2.64997496244367	4452.43434218522\\
2.65197796695043	4445.41835939229\\
2.65398097145719	4438.37745293526\\
2.65598397596395	4431.31276872974\\
2.65798698047071	4424.22568187443\\
2.65998998497747	4417.11773935537\\
2.66199298948423	4409.99071734174\\
2.66399599399099	4402.84644929848\\
2.66599899849775	4395.6871124652\\
2.66800200300451	4388.51488408153\\
2.67000500751127	4381.33211327443\\
2.67200801201803	4374.14126376242\\
2.67401101652479	4366.9448565598\\
2.67601402103155	4359.74546997664\\
2.67801702553831	4352.54573961881\\
2.68002003004507	4345.34835838793\\
2.68202303455183	4338.15584729832\\
2.68402603905859	4330.97084195582\\
2.68602904356535	4323.79574878317\\
2.68803204807211	4316.63297420313\\
2.69003505257887	4309.48481004685\\
2.69203805708563	4302.35331896242\\
2.69404106159239	4295.24044900633\\
2.69604406609915	4288.14803364352\\
2.69804707060591	4281.07761986005\\
2.70005007511267	4274.03064005039\\
2.70205307961943	4267.00841201749\\
2.70405608412619	4260.01185249381\\
2.70605908863295	4253.04187821182\\
2.70806209313971	4246.09923401666\\
2.71006509764647	4239.18432097879\\
2.71206810215323	4232.29754016866\\
2.71407110665999	4225.4390634736\\
2.71607411116675	4218.60894818941\\
2.71807711567351	4211.80713702029\\
2.72008012018027	4205.03334348736\\
2.72208312468703	4198.28739570326\\
2.72408612919379	4191.56872071022\\
2.72608913370055	4184.87697473355\\
2.72809213820731	4178.2114702239\\
2.73009514271407	4171.57163422346\\
2.73209814722083	4164.95672188712\\
2.73410115172759	4158.36604566551\\
2.73610415623435	4151.79880341772\\
2.73810716074111	4145.25419300284\\
2.74011016524787	4138.73129768839\\
2.74211316975463	4132.22937262925\\
2.74411617426139	4125.74750109293\\
2.74611917876815	4119.2847090512\\
2.74812218327491	4112.8402516589\\
2.75012518778167	4106.41309759202\\
2.75212819228843	4100.00244470964\\
2.75413119679519	4093.60731898351\\
2.75613420130195	4087.22686097693\\
2.75813720580871	4080.86021125322\\
2.76014021031547	4074.50651037567\\
2.76214321482223	4068.16478431605\\
2.76414621932899	4061.83423093343\\
2.76614922383575	4055.51399079112\\
2.76815222834251	4049.20326174821\\
2.77015523284927	4042.90112707223\\
2.77215823735603	4036.60684191804\\
2.77416124186279	4030.31954684895\\
2.77616424636955	4024.03843972405\\
2.77816725087631	4017.7627756982\\
2.78017025538307	4011.49169533471\\
2.78217325988983	4005.22451108423\\
2.78417626439659	3998.96036351007\\
2.78617926890335	3992.69856506286\\
2.78818227341012	3986.4383136017\\
2.79018527791688	3980.17892157724\\
2.79218828242364	3973.91964414433\\
2.7941912869304	3967.65973645785\\
2.79619429143716	3961.39851096844\\
2.79819729594392	3955.13522283097\\
2.80020030045068	3948.86918449608\\
2.80220330495744	3942.59970841442\\
2.8042063094642	3936.32604974085\\
2.80620931397096	3930.04763551759\\
2.80821231847772	3923.76366360371\\
2.81021532298448	3917.47350374565\\
2.81221832749124	3911.17652568982\\
2.814221331998	3904.87209918266\\
2.81622433650476	3898.55953667481\\
2.81822734101152	3892.23815061691\\
2.82023034551828	3885.90742534695\\
2.82223335002504	3879.56667331557\\
2.8242363545318	3873.21537886077\\
2.82623935903856	3866.85291172896\\
2.82824236354532	3860.47864166657\\
2.83024536805208	3854.09216760314\\
2.83224837255884	3847.69285928511\\
2.8342513770656	3841.28025834622\\
2.83625438157236	3834.85390642026\\
2.83825738607912	3828.41328784519\\
2.84026039058588	3821.95805884635\\
2.84226339509264	3815.48787564906\\
2.8442663995994	3809.00233718285\\
2.84626940410616	3802.50115696884\\
2.84827240861292	3795.98416311968\\
2.85027541311968	3789.45106915648\\
2.85227841762644	3782.90176048768\\
2.8542814221332	3776.33612252172\\
2.85628442663996	3769.75421255437\\
2.85828743114672	3763.15603058565\\
2.86029043565348	3756.54163391132\\
2.86229344016024	3749.91125171451\\
2.864296444667	3743.26517047411\\
2.86629944917376	3736.60361937324\\
2.86830245368052	3729.92711407392\\
2.87030545818728	3723.23611294238\\
2.87230846269404	3716.53113164064\\
2.8743114672008	3709.81280042228\\
2.87631447170756	3703.0819214282\\
2.87831747621432	3696.33929679932\\
2.88032048072108	3689.58572867655\\
2.88232348522784	3682.82224838393\\
2.8843264897346	3676.04977265392\\
2.88632949424136	3669.26944740213\\
2.88833249874812	3662.48241854412\\
2.89033550325488	3655.68977469973\\
2.89233850776164	3648.89283367187\\
2.8943415122684	3642.09274137614\\
2.89634451677516	3635.29081561547\\
2.89834752128192	3628.488316897\\
2.90035052578868	3621.68662031944\\
2.90235353029544	3614.88687179839\\
2.9043565348022	3608.09044643254\\
2.90635953930896	3601.29854743328\\
2.90836254381572	3594.51237801198\\
2.91036554832248	3587.73314137999\\
2.91236855282924	3580.96186886135\\
2.914371557336	3574.19976366744\\
2.91637456184276	3567.44768523494\\
2.91837756634952	3560.70666488789\\
2.92038057085628	3553.97756206297\\
2.92238357536304	3547.26112160533\\
2.9243865798698	3540.5580883601\\
2.92638958437656	3533.86909258084\\
2.92839258888332	3527.19464992958\\
2.93039559339008	3520.53533336412\\
2.93239859789685	3513.89142936334\\
2.93440160240361	3507.26333899771\\
2.93640460691037	3500.65129145034\\
2.93840761141713	3494.05540131279\\
2.94041061592389	3487.47578317663\\
2.94241362043065	3480.91249433762\\
2.94441662493741	3474.36553479578\\
2.94641962944417	3467.83473266376\\
2.94842263395093	3461.31997335001\\
2.95042563845769	3454.82108496718\\
2.95242864296445	3448.33772374059\\
2.95443164747121	3441.86966048714\\
2.95643465197797	3435.41660872792\\
2.95843765648473	3428.97805280092\\
2.96044066099149	3422.55364893146\\
2.96244366549825	3416.14299604908\\
2.96444667000501	3409.74557849176\\
2.96644967451177	3403.36082330172\\
2.96845267901853	3396.9883294085\\
2.97045568352529	3390.62746655851\\
2.97245868803205	3384.27766179398\\
2.97446169253881	3377.93839945287\\
2.97646469704557	3371.60904928162\\
2.97846770155233	3365.28898102665\\
2.98047070605909	3358.97762173017\\
2.98247371056585	3352.67428384281\\
2.98447671507261	3346.37833711102\\
2.98647971957937	3340.08909398543\\
2.98848272408613	3333.80598150824\\
2.99048572859289	3327.52825483433\\
2.99248873309965	3321.25522641434\\
2.99449173760641	3314.9862659947\\
2.99649474211317	3308.72068602604\\
2.99849774661993	3302.45774166325\\
3.00050075112669	3296.19674535696\\
3.00250375563345	3289.93695226203\\
3.00450676014021	3283.67767482913\\
3.00650976464697	3277.41822550888\\
3.00851276915373	3271.15785945616\\
3.01051577366049	3264.89571723428\\
3.01251877816725	3258.63122588544\\
3.01452178267401	3252.36352597294\\
3.01652478718077	3246.09181535588\\
3.01852779168753	3239.8154064849\\
3.02053079619429	3233.53349721909\\
3.02253380070105	3227.24522812175\\
3.02453680520781	3220.94991164353\\
3.02653980971457	3214.64657375618\\
3.02854281422133	3208.33452691034\\
3.03054581872809	3202.01291166932\\
3.03254882323485	3195.68081130065\\
3.03455182774161	3189.33742366342\\
3.03655483224837	3182.98188932093\\
3.03855783675513	3176.61334883649\\
3.04056084126189	3170.23082818185\\
3.04256384576865	3163.83358251188\\
3.04456685027541	3157.42058050254\\
3.04656985478217	3150.9910773087\\
3.04857285928893	3144.54398431055\\
3.05057586379569	3138.07855666296\\
3.05257886830245	3131.59387763344\\
3.05458187280921	3125.08903048954\\
3.05658487731597	3118.56309849878\\
3.05858788182273	3112.01527952025\\
3.06059088632949	3105.44477141303\\
3.06259389083625	3098.85065744465\\
3.06459689534301	3092.23225006575\\
3.06659989984977	3085.58880443121\\
3.06860290435653	3078.91957569589\\
3.07060590886329	3072.22404819777\\
3.07260891337005	3065.5015343875\\
3.07461191787682	3058.75157589884\\
3.07661492238358	3051.97382895712\\
3.07861792689034	3045.16800708344\\
3.0806209313971	3038.3338237989\\
3.08262393590386	3031.47133639928\\
3.08462694041062	3024.5804875888\\
3.08662994491738	3017.66162114214\\
3.08863294942414	3010.71502353819\\
3.0906359539309	3003.74121043898\\
3.09263895843766	2996.74092668963\\
3.09464196294442	2989.71508902261\\
3.09664496745118	2982.66472876197\\
3.09864797195794	2975.59104911906\\
3.1006509764647	2968.49559707994\\
3.10265398097146	2961.37997692644\\
3.10465698547822	2954.2459648277\\
3.10665998998498	2947.09556613603\\
3.10866299449174	2939.93090079526\\
3.1106659989985	2932.75426063657\\
3.11266900350526	2925.5680520827\\
3.11467200801202	2918.37468155639\\
3.11667501251878	2911.1767846635\\
3.11867801702554	2903.97699700988\\
3.1206810215323	2896.77783960984\\
3.12268402603906	2889.5820626608\\
3.12468703054582	2882.39218717704\\
3.12669003505258	2875.21079146865\\
3.12869303955934	2868.04022466259\\
3.1306960440661	2860.88295047737\\
3.13269904857286	2853.74103156107\\
3.13470205307962	2846.61658785751\\
3.13670505758638	2839.51156742321\\
3.13870806209314	2832.42757453999\\
3.1407110665999	2825.3661561939\\
3.14271407110666	2818.32874477943\\
3.14471707561342	2811.31642891638\\
3.14672008012018	2804.33018263301\\
3.14872308462694	2797.37086536602\\
3.1507260891337	2790.43899277741\\
3.15272909364046	2783.5350805292\\
3.15473209814722	2776.65935780451\\
3.15673510265398	2769.81199649068\\
3.15873810716074	2762.99293929193\\
3.1607411116675	2756.20212891248\\
3.16274411617426	2749.43933616921\\
3.16474712068102	2742.70410269589\\
3.16675012518778	2735.99602742206\\
3.16875312969454	2729.3145946857\\
3.1707561342013	2722.65923152902\\
3.17275913870806	2716.02925040266\\
3.17476214321482	2709.4239064615\\
3.17676514772158	2702.84251215617\\
3.17876815222834	2696.28420804998\\
3.1807711567351	2689.74824929781\\
3.18277416124186	2683.23366187139\\
3.18477716574862	2676.7397009256\\
3.18678017025538	2670.26539243218\\
3.18878317476214	2663.80987695444\\
3.1907861792689	2657.37229505569\\
3.19278918377566	2650.95173000345\\
3.19479218828242	2644.54720776948\\
3.19679519278918	2638.15798350864\\
3.19879819729594	2631.78302589689\\
3.2008012018027	2625.42153279334\\
3.20280420630946	2619.07258746549\\
3.20480721081622	2612.73538777245\\
3.20681021532298	2606.40895968595\\
3.20881321982974	2600.09255836087\\
3.2108162243365	2593.78526706473\\
3.21281922884326	2587.4863409524\\
3.21482223335002	2581.19486329141\\
3.21682523785678	2574.91008923662\\
3.21882824236355	2568.63121664712\\
3.22083124687031	2562.35744338199\\
3.22283425137707	2556.08796730034\\
3.22483725588383	2549.822043557\\
3.22684026039059	2543.55892730687\\
3.22884326489735	2537.29781640902\\
3.23084626940411	2531.0380233141\\
3.23284927391087	2524.77874588119\\
3.23485227841763	2518.51929656095\\
3.23685528292439	2512.25898780401\\
3.23885828743115	2505.99701746946\\
3.24086129193791	2499.73275530374\\
3.24286429644467	2493.46539916592\\
3.24486730095143	2487.19437609821\\
3.24687030545819	2480.91894125549\\
3.24887330996495	2474.63840708838\\
3.25087631447171	2468.35214334332\\
3.25287931897847	2462.05940517518\\
3.25488232348523	2455.75956233038\\
3.25688532799199	2449.45198455534\\
3.25888833249875	2443.1360415965\\
3.26089133700551	2436.81110320027\\
3.26289434151227	2430.47653911308\\
3.26489734601903	2424.13171908136\\
3.26690035052579	2417.77612744309\\
3.26890335503255	2411.40907664892\\
3.27090635953931	2405.03010833261\\
3.27290936404607	2398.63864953637\\
3.27491236855283	2392.23424189396\\
3.27691537305959	2385.8162551518\\
3.27891837756635	2379.384402831\\
3.28092138207311	2372.93811197376\\
3.28292438657987	2366.47709610119\\
3.28492739108663	2360.00095414282\\
3.28693039559339	2353.50928502821\\
3.28893340010015	2347.00197416579\\
3.29093640460691	2340.47867778089\\
3.29293940911367	2333.93922398617\\
3.29494241362043	2327.38355548584\\
3.29694541812719	2320.81155768835\\
3.29894842263395	2314.2232305937\\
3.30095142714071	2307.61868879345\\
3.30295443164747	2300.99804687915\\
3.30495743615423	2294.36147673815\\
3.30696044066099	2287.70926484934\\
3.30896344516775	2281.04186957896\\
3.31096644967451	2274.35957740591\\
3.31296945418127	2267.66301858376\\
3.31497245868803	2260.95276607031\\
3.31697546319479	2254.22945011912\\
3.31897846770155	2247.49387287113\\
3.32098147220831	2240.74683646722\\
3.32298447671507	2233.98925763989\\
3.32498748122183	2227.22211041738\\
3.32699048572859	2220.4464834195\\
3.32899349023535	2213.66340797029\\
3.33099649474211	2206.87397268955\\
3.33299949924887	2200.07949538021\\
3.33500250375563	2193.28117925364\\
3.33700550826239	2186.48022752122\\
3.33900851276915	2179.67795798587\\
3.34101151727591	2172.87563115473\\
3.34301452178267	2166.07450753497\\
3.34501752628943	2159.27590492951\\
3.34702053079619	2152.48114114127\\
3.34902353530295	2145.69130479008\\
3.35102653980971	2138.90771367885\\
3.35302954431647	2132.13145642739\\
3.35503254882323	2125.36367895131\\
3.35703555332999	2118.60535527884\\
3.35903855783675	2111.85751673403\\
3.36104156234352	2105.12108004933\\
3.36304456685028	2098.3967900699\\
3.36504757135704	2091.68556352819\\
3.3670505758638	2084.98791608623\\
3.36905358037056	2078.30459258915\\
3.37105658487732	2071.6359941074\\
3.37305958938408	2064.98269359878\\
3.37506259389084	2058.34497754219\\
3.3770655983976	2051.72307512075\\
3.37906860290436	2045.11733010913\\
3.38107160741112	2038.52779980311\\
3.38307461191788	2031.95459879425\\
3.38507761642464	2025.39772708255\\
3.3870806209314	2018.85707007645\\
3.38908362543816	2012.33257048018\\
3.39108662994492	2005.82405640639\\
3.39308963445168	1999.33129867197\\
3.39509263895844	1992.85395350224\\
3.3970956434652	1986.39184900985\\
3.39909864797196	1979.9444695328\\
3.40110165247872	1973.51152859219\\
3.40310465698548	1967.09256782178\\
3.40510766149224	1960.68712885534\\
3.407110665999	1954.29475332662\\
3.40911367050576	1947.91486827784\\
3.41111667501252	1941.54695804698\\
3.41311967951928	1935.19050697202\\
3.41512268402604	1928.84494209517\\
3.4171256885328	1922.50963316284\\
3.41912869303956	1916.18394992148\\
3.42113169754632	1909.86737670906\\
3.42313470205308	1903.55922597623\\
3.42513770655984	1897.25886746941\\
3.4271407110666	1890.96561363926\\
3.42914371557336	1884.67883423218\\
3.43114672008012	1878.39789899462\\
3.43314972458688	1872.12212037721\\
3.43515272909364	1865.85075353483\\
3.4371557336004	1859.58322550967\\
3.43915873810716	1853.31873416083\\
3.44116174261392	1847.05664923473\\
3.44316474712068	1840.79616859045\\
3.44516775162744	1834.53671927021\\
3.4471707561342	1828.27749913308\\
3.44917376064096	1822.01770603816\\
3.45117676514772	1815.75676702764\\
3.45317976965448	1809.49382266485\\
3.45518277416124	1803.22812810464\\
3.457185778668	1796.95893850187\\
3.45918878317476	1790.68550901143\\
3.46119178768152	1784.40703749238\\
3.46319479218828	1778.12272180383\\
3.46519779669504	1771.83181710063\\
3.4672008012018	1765.53340665032\\
3.46920380570856	1759.22680290353\\
3.47120681021532	1752.91108912781\\
3.47320981472208	1746.58546318222\\
3.47521281922884	1740.24906563009\\
3.4772158237356	1733.90103703472\\
3.47921882824236	1727.54046066363\\
3.48122183274912	1721.1665916717\\
3.48322483725588	1714.77839873489\\
3.48522784176264	1708.37507971229\\
3.4872308462694	1701.9557751672\\
3.48923385077616	1695.5194537756\\
3.49123685528292	1689.06531339657\\
3.49323985978968	1682.59243729764\\
3.49524286429644	1676.09996604211\\
3.4972458688032	1669.58698289752\\
3.49924887330996	1663.05257113139\\
3.50125187781672	1656.49598589859\\
3.50325488232349	1649.91631046665\\
3.50525788683025	1643.31279999043\\
3.50726089133701	1636.68465232902\\
3.50926389584377	1630.03123722884\\
3.51126690035053	1623.35175254898\\
3.51326990485729	1616.64568262743\\
3.51527290936405	1609.91251180217\\
3.51727591387081	1603.15172441119\\
3.51927891837757	1596.36303397558\\
3.52128192288433	1589.54609672067\\
3.52328492739109	1582.70085535069\\
3.52528793189785	1575.82725256984\\
3.52729093640461	1568.92540296969\\
3.52929394091137	1561.99559302915\\
3.53129694541813	1555.03828111443\\
3.53329994992489	1548.05404018335\\
3.53530295443165	1541.04361508102\\
3.53730595893841	1534.00803713149\\
3.53930896344517	1526.94839495457\\
3.54131196795193	1519.86600635318\\
3.54331497245869	1512.76247560915\\
3.54531797696545	1505.6394070043\\
3.54732098147221	1498.49874859514\\
3.54932398597897	1491.34250573396\\
3.55132699048573	1484.17285566037\\
3.55332999499249	1476.99220479711\\
3.55533299949925	1469.80284497537\\
3.55733600400601	1462.607411801\\
3.55933900851277	1455.40848358408\\
3.56134201301953	1448.20863863468\\
3.56334501752629	1441.01051255868\\
3.56534802203305	1433.81668366613\\
3.56735102653981	1426.62984485869\\
3.56935403104657	1419.45240255908\\
3.57135703555333	1412.28676319006\\
3.57336004006009	1405.1352185828\\
3.57536304456685	1397.99994597691\\
3.57736604907361	1390.88289342892\\
3.57936905358037	1383.78589440375\\
3.58137205808713	1376.71055318324\\
3.58337506259389	1369.65841675343\\
3.58537806710065	1362.63068832569\\
3.58738107160741	1355.62845651984\\
3.58938407611417	1348.65269536412\\
3.59138708062093	1341.70403511211\\
3.59339008512769	1334.78304872161\\
3.59539308963445	1327.89013726307\\
3.59739609414121	1321.02552991961\\
3.59939909864797	1314.189283987\\
3.60140210315473	1307.38134216948\\
3.60340510766149	1300.60147528392\\
3.60540811216825	1293.84951144298\\
3.60741111667501	1287.12499228042\\
3.60941412118177	1280.42740213424\\
3.61141712568853	1273.7562826382\\
3.61342013019529	1267.11088894715\\
3.61542313470205	1260.49064810331\\
3.61742613920881	1253.89481526154\\
3.61942914371557	1247.32253098516\\
3.62143214822233	1240.77305042902\\
3.62343515272909	1234.24557145221\\
3.62543815723585	1227.73917732226\\
3.62744116174261	1221.25300860249\\
3.62944416624937	1214.7861485604\\
3.63144717075613	1208.3377950551\\
3.63345017526289	1201.90691676255\\
3.63545317976965	1195.49276883762\\
3.63745618427641	1189.09437725206\\
3.63945918878317	1182.71082527339\\
3.64146219328993	1176.34125346492\\
3.64346519779669	1169.98474509418\\
3.64546820230346	1163.64049802025\\
3.64747120681022	1157.30765280645\\
3.64947421131698	1150.98523542452\\
3.65147721582374	1144.67250102933\\
3.6534802203305	1138.36859018418\\
3.65548322483726	1132.07264345239\\
3.65748622934402	1125.78391598881\\
3.65948923385078	1119.50154835676\\
3.66149223835754	1113.22473841532\\
3.6634952428643	1106.95274131937\\
3.66549824737106	1100.68475492797\\
3.66750125187782	1094.42003439601\\
3.66950425638458	1088.15777758257\\
3.67150726089134	1081.89723964251\\
3.6735102653981	1075.63779032227\\
3.67551326990486	1069.37851288936\\
3.67751627441162	1063.118834386\\
3.67951927891838	1056.85795267127\\
3.68152228342514	1050.59512290003\\
3.6835252879319	1044.32977211449\\
3.68552829243866	1038.06109817375\\
3.68753129694542	1031.78841352844\\
3.68953430145218	1025.5110306292\\
3.69153730595894	1019.22831922247\\
3.6935403104657	1012.9395917589\\
3.69554331497246	1006.64416068912\\
3.69754631947922	1000.34139575956\\
3.69954932398598	994.030609420873\\
3.70155232849274	987.711228715257\\
3.7035553329995	981.38262338914\\
3.70555833750626	975.044105893166\\
3.70756134201302	968.695103269542\\
3.70956434651978	962.335099856252\\
3.71156735102654	955.96346539972\\
3.7135703555333	949.579684237931\\
3.71557336004006	943.183183413091\\
3.71757636454682	936.773504558962\\
3.71957936905358	930.350189309309\\
3.72158237356034	923.912722002117\\
3.7235853780671	917.460701566928\\
3.72558838257386	910.993784229066\\
3.72759138708062	904.511626213854\\
3.72959439158738	898.013883746614\\
3.73159739609414	891.500270348448\\
3.7336004006009	884.970671427799\\
3.73560340510766	878.424915097327\\
3.73760640961442	871.862829469694\\
3.73960941412118	865.284471840679\\
3.74161241862794	858.689784914502\\
3.7436154231347	852.078940578503\\
3.74561842764146	845.451996128461\\
3.74762143214822	838.809295339053\\
3.74962443665498	832.151067393397\\
3.75162744116174	825.477770657728\\
3.7536304456685	818.789863498284\\
3.75563345017526	812.08786157708\\
3.75763645468202	805.37239514769\\
3.75963945918878	798.644151759469\\
3.76164246369554	791.903990849109\\
3.7636454682023	785.152771853303\\
3.76564847270906	778.391296912964\\
3.76765147721582	771.620711943683\\
3.76965448172258	764.841990973711\\
3.77165748622934	758.056337214417\\
3.7736604907361	751.264781989834\\
3.77566349524286	744.46864310289\\
3.77766649974962	737.669123764954\\
3.77966950425638	730.867541778956\\
3.78167250876314	724.065043060486\\
3.7836755132699	717.263060004031\\
3.78567851777666	710.462795820962\\
3.78768152228343	703.665568314207\\
3.78968452679019	696.872580695136\\
3.79168753129695	690.085036175119\\
3.79369053580371	683.304195261304\\
3.79569354031047	676.531089277725\\
3.79769654481723	669.766921435749\\
3.79969954932399	663.012551172069\\
3.80170255383075	656.269067106498\\
3.80370555833751	649.537328675726\\
3.80570856284427	642.818080724888\\
3.80771156735103	636.112068099118\\
3.80971457185779	629.41997834777\\
3.81171757636455	622.742327132859\\
3.81372058087131	616.07968741218\\
3.81572358537807	609.43246025619\\
3.81772658988483	602.800874848007\\
3.81972959439159	596.185332258088\\
3.82173259889835	589.585889782212\\
3.82373560340511	583.002776603498\\
3.82573860791187	576.435935426165\\
3.82774161241863	569.885423545993\\
3.82974461692539	563.351126371424\\
3.83174762143215	556.832929310897\\
3.83375062593891	550.330545885517\\
3.83575363044567	543.843861503723\\
3.83775663495243	537.372532390838\\
3.83975963945919	530.916214772186\\
3.84176264396595	524.474564873089\\
3.84376564847271	518.047181623091\\
3.84576865297947	511.633549360176\\
3.84777165748623	505.233267013888\\
3.84977466199299	498.84581892221\\
3.85177766649975	492.470689423128\\
3.85378067100651	486.107362854626\\
3.85578367551327	479.755266258908\\
3.85778668002003	473.4137693824\\
3.85978968452679	467.082299267307\\
3.86179268903355	460.760282955833\\
3.86379569354031	454.447032898625\\
3.86579869804707	448.142033433667\\
3.86780170255383	441.844482420046\\
3.86980470706059	435.553864195745\\
3.87180771156735	429.269491211412\\
3.87381071607411	422.990618621912\\
3.87581372058087	416.71667346945\\
3.87781672508763	410.446910908892\\
3.87981972959439	404.180643390885\\
3.88182273410115	397.917183366074\\
3.88382573860791	391.655785989325\\
3.88582874311467	385.395821007064\\
3.88783174762143	379.136543574158\\
3.88983475212819	372.877151549692\\
3.89183775663495	366.617014680093\\
3.89384076114171	360.355388120226\\
3.89584376564847	354.091412433399\\
3.89784677015523	347.824457366038\\
3.89984977466199	341.553720777228\\
3.90185277916875	335.278400526058\\
3.90385578367551	328.997751767393\\
3.90585878818227	322.71091506454\\
3.90786179268903	316.417202868146\\
3.90986479719579	310.115698445739\\
3.91186780170255	303.805599656404\\
3.91387080620931	297.486104359229\\
3.91587381071607	291.156353117521\\
3.91787681522283	284.815486494588\\
3.91987981972959	278.462702349516\\
3.92188282423635	272.097026654053\\
3.92388582874311	265.717657267287\\
3.92588883324987	259.323677456745\\
3.92789183775663	252.914170489955\\
3.9298948422634	246.488276930225\\
3.93189784677016	240.04513734086\\
3.93390085127692	233.583777693611\\
3.93590385578368	227.103338551784\\
3.93790686029044	220.602903182907\\
3.9399098647972	214.081612150287\\
3.94191286930396	207.538663313011\\
3.94391587381072	200.973139938607\\
3.94591887831748	194.384239886161\\
3.94792188282424	187.771161014761\\
3.949924887331	181.133215775053\\
3.95192789183776	174.469602026122\\
3.95393089634452	167.779804105954\\
3.95593390085128	161.063191760974\\
3.95793690535804	154.319249329166\\
3.9599399098648	147.547575740074\\
3.96194291437156	140.7478845148\\
3.96394591887832	133.919889174446\\
3.96594892338508	127.063532423234\\
3.96795192789184	120.178814261162\\
3.9699549323986	113.266021167129\\
3.97195793690536	106.325382324253\\
3.97396094141212	99.3574133945492\\
3.97596394591888	92.3628019273717\\
3.97796695042564	85.3424073594132\\
3.9799699549324	78.2972610147051\\
3.98197295943916	71.2285661046171\\
3.98397596394592	64.1376977278575\\
3.98597896845268	57.0263174620323\\
3.98798197295944	49.8961441805268\\
3.9899849774662	42.7491259398444\\
3.99198798197296	35.5873826838272\\
3.99399098647972	28.4131489478761\\
3.99599399098648	21.2287738589512\\
3.99799699549324	14.0367211355715\\
4	6.83962638359469\\
};
\addlegendentry{$\psi$};

\addplot [color=mycolor3,solid]
  table[row sep=crcr]{%
0	21.4674107806231\\
0.00200300450676014	21.2450458603329\\
0.00400600901352028	21.0401561527941\\
0.00600901352028042	20.8530854326839\\
0.00801201802704056	20.6839482915613\\
0.0100150225338007	20.5330885041033\\
0.0120180270405608	20.4006779576486\\
0.014021031547321	20.2868885395356\\
0.0160240360540811	20.1918921371029\\
0.0180270405608413	20.1158033419095\\
0.0200300450676014	20.0587367455145\\
0.0220330495743615	20.0208642352564\\
0.0240360540811217	20.0021285153556\\
0.0260390585878818	20.0026441773712\\
0.028042063094642	20.0224112213032\\
0.0300450676014021	20.0613150555926\\
0.0320480721081622	20.1194129760189\\
0.0340510766149224	20.196475799464\\
0.0360540811216825	20.2925035259279\\
0.0380570856284427	20.4073242680721\\
0.0400600901352028	20.5407088427786\\
0.0420630946419629	20.6925426584882\\
0.0440660991487231	20.8626538278626\\
0.0460691036554832	21.0506985762245\\
0.0480721081622434	21.2565050162355\\
0.0500751126690035	21.4798439647775\\
0.0520781171757636	21.7203716471734\\
0.0540811216825238	21.9779161760847\\
0.0560841261892839	22.2521910726138\\
0.0580871306960441	22.5429098578632\\
0.0600901352028042	22.8497860529353\\
0.0620931397095643	23.1725331789325\\
0.0640961442163245	23.5109220527367\\
0.0660991487230846	23.8646661954505\\
0.0681021532298448	24.2334791281762\\
0.0701051577366049	24.6172462593548\\
0.0721081622433651	25.0155665185298\\
0.0741111667501252	25.4282680183625\\
0.0761141712568853	25.8551788715145\\
0.0781171757636455	26.2960125990882\\
0.0801201802704056	26.750597313745\\
0.0821231847771658	27.2187038323668\\
0.0841261892839259	27.7001602676153\\
0.086129193790686	28.1947947321517\\
0.0881321982974462	28.7024353386376\\
0.0901352028042063	29.2229101997345\\
0.0921382073109664	29.7560474281037\\
0.0941412118177266	30.3016751364068\\
0.0961442163244867	30.8596787330847\\
0.0981472208312469	31.4299436265784\\
0.100150225338007	32.0122406337698\\
0.102153229844767	32.6064551631\\
0.104156234351527	33.2125299187894\\
0.106159238858287	33.8302357177199\\
0.108162243365048	34.4594579683326\\
0.110165247871808	35.1000247832889\\
0.112168252378568	35.7518788668092\\
0.114171256885328	36.4147910357756\\
0.116174261392088	37.0885894028494\\
0.118177265898848	37.7732166722512\\
0.120180270405608	38.4683863650835\\
0.122183274912369	39.1739838897871\\
0.124186279419129	39.889780063244\\
0.126189283925889	40.6156602938953\\
0.128192288432649	41.3512808070637\\
0.130195292939409	42.0965270111904\\
0.132198297446169	42.8511124273777\\
0.134201301952929	43.614750576728\\
0.13620430645969	44.3872122761234\\
0.13820731096645	45.1681537508867\\
0.14021031547321	45.9572885221204\\
0.14221331997997	46.7542155193679\\
0.14421632448673	47.5586482637315\\
0.14621932899349	48.3701283889753\\
0.14822233350025	49.1882548246426\\
0.150225338007011	50.0126265002769\\
0.152228342513771	50.8426704580829\\
0.154231347020531	51.6779856276041\\
0.156234351527291	52.5179417552659\\
0.158237356034051	53.3620231790526\\
0.160240360540811	54.2095996453897\\
0.162243365047571	55.0600409007023\\
0.164246369554332	55.912602099857\\
0.166249374061092	56.766652989279\\
0.168252378567852	57.6214487238347\\
0.170255383074612	58.4760725710518\\
0.172258387581372	59.3297796857968\\
0.174261392088132	60.1816533355973\\
0.176264396594892	61.0307767879811\\
0.178267401101652	61.8762333104762\\
0.180270405608413	62.7169915790512\\
0.182273410115173	63.5519629738953\\
0.184276414621933	64.3801161709774\\
0.186279419128693	65.2003625504867\\
0.188282423635453	66.0114989010534\\
0.190285428142213	66.8123220113077\\
0.192288432648973	67.6015713741004\\
0.194291437155734	68.3781010738412\\
0.196294441662494	69.1405360118218\\
0.198297446169254	69.8875583851134\\
0.200300450676014	70.6178503907872\\
0.202303455182774	71.3300369301348\\
0.204306459689534	72.0228002002274\\
0.206309464196294	72.6947651023569\\
0.208312468703055	73.3444419462557\\
0.210315473209815	73.9705702247747\\
0.212318477716575	74.5717175434259\\
0.214321482223335	75.1465660992807\\
0.216324486730095	75.6937980894101\\
0.218327491236855	76.212038415106\\
0.220330495743615	76.7000838649984\\
0.222333500250376	77.1567312277177\\
0.224336504757136	77.580777291894\\
0.226339509263896	77.9711334377166\\
0.228342513770656	78.3267683411543\\
0.230345518277416	78.6467652697349\\
0.232348522784176	78.9301501952066\\
0.234351527290936	79.1761782724358\\
0.236354531797697	79.3841619520683\\
0.238357536304457	79.5535855720884\\
0.240360540811217	79.6838188789217\\
0.242363545317977	79.7745753936704\\
0.244366549824737	79.825568637437\\
0.246369554331497	79.8366267228831\\
0.248372558838257	79.8077496500085\\
0.250375563345018	79.7390520103723\\
0.252378567851778	79.630705691313\\
0.254381572358538	79.4829971717283\\
0.256384576865298	79.2963275220747\\
0.258387581372058	79.0713269959268\\
0.260390585878818	78.8085685510798\\
0.262393590385578	78.5087970326674\\
0.264396594892339	78.1728718773822\\
0.266399599399099	77.8016525219169\\
0.268402603905859	77.3961702903028\\
0.270405608412619	76.9574565065712\\
0.272408612919379	76.4867143820917\\
0.274411617426139	75.9850325366751\\
0.276414621932899	75.4537287732503\\
0.27841762643966	74.8940635989665\\
0.28042063094642	74.3073548167526\\
0.28242363545318	73.6949202295372\\
0.28442663995994	73.0581349360288\\
0.2864296444667	72.3983740349357\\
0.28843264897346	71.7169553291866\\
0.29043565348022	71.0153685090489\\
0.29243865798698	70.2949886732309\\
0.294441662493741	69.5570763288819\\
0.296444667000501	68.8030065747103\\
0.298447671507261	68.0341545094242\\
0.300450676014021	67.2518379359526\\
0.302453680520781	66.4572600656652\\
0.304456685027541	65.6516814057112\\
0.306459689534301	64.8363624632401\\
0.308462694041062	64.0124491538419\\
0.310465698547822	63.1810873931071\\
0.312468703054582	62.3434230966258\\
0.314471707561342	61.5004875884294\\
0.316474712068102	60.6532548967694\\
0.318477716574862	59.8027563456772\\
0.320480721081622	58.9499659634045\\
0.322483725588383	58.095743186644\\
0.324486730095143	57.2409474520883\\
0.326489734601903	56.3864381964302\\
0.328492739108663	55.5329029690238\\
0.330495743615423	54.6812012065618\\
0.332498748122183	53.8319631626189\\
0.334501752628943	52.9858763865492\\
0.336504757135704	52.1435711319274\\
0.338507761642464	51.3055630607691\\
0.340510766149224	50.4725397224284\\
0.342513770655984	49.6449594831414\\
0.344516775162744	48.8233380049238\\
0.346519779669504	48.0080763582322\\
0.348522784176264	47.1996902050821\\
0.350525788683025	46.3985806159302\\
0.352528793189785	45.6050913654535\\
0.354531797696545	44.8195662283291\\
0.356534802203305	44.0423489792342\\
0.358537806710065	43.2737833928457\\
0.360540811216825	42.5141559480612\\
0.362543815723585	41.7636958279989\\
0.364546820230346	41.0226895115562\\
0.366549824737106	40.2913661818512\\
0.368552829243866	39.569955022002\\
0.370555833750626	38.8586852151266\\
0.372558838257386	38.157671352784\\
0.374561842764146	37.4671999138719\\
0.376564847270906	36.7873281941696\\
0.378567851777666	36.1183426725749\\
0.380570856284427	35.4603579406467\\
0.382573860791187	34.8135458857235\\
0.384576865297947	34.1779638035848\\
0.386579869804707	33.5538981731283\\
0.388582874311467	32.941348994354\\
0.390585878818227	32.3405454503798\\
0.392588883324987	31.7516021327648\\
0.394591887831748	31.1746336330681\\
0.396594892338508	30.6098118386281\\
0.398597896845268	30.0573086367835\\
0.400600901352028	29.5171813233136\\
0.402603905858788	28.9896590813367\\
0.404606910365548	28.4749137981912\\
0.406609914872308	27.9730600654361\\
0.408612919379069	27.4842124746305\\
0.410615923885829	27.0087148004514\\
0.412618928392589	26.5465670428989\\
0.414621932899349	26.09811297665\\
0.416624937406109	25.6634671932637\\
0.418627941912869	25.2428588758582\\
0.420630946419629	24.8364599117719\\
0.42263395092639	24.4446140756819\\
0.42463695543315	24.0674932549268\\
0.42663995993991	23.7053266326246\\
0.42864296444667	23.3584006876729\\
0.43064596895343	23.0269446031897\\
0.43264897346019	22.7111875622931\\
0.43465197796695	22.4115306354397\\
0.436654982473711	22.1280884141885\\
0.438657986980471	21.8612619689961\\
0.440660991487231	21.611223187201\\
0.442663995993991	21.3783158434803\\
0.444667000500751	21.1627118251726\\
0.446670005007511	20.9648122027344\\
0.448673009514271	20.7847315677248\\
0.450676014021032	20.6226991032618\\
0.452679018527792	20.4790585840225\\
0.454682023034552	20.3539246015659\\
0.456685027541312	20.2474690432306\\
0.458688032048072	20.1598065005756\\
0.460691036554832	20.0911661567189\\
0.462694041061592	20.0416626032196\\
0.464697045568353	20.0112385442982\\
0.466700050075113	20.0000658672931\\
0.468703054581873	20.0080872764249\\
0.470706059088633	20.0353600674732\\
0.472709063595393	20.0817696488788\\
0.474712068102153	20.1473160206417\\
0.476715072608913	20.2318272954235\\
0.478718077115673	20.3352461774446\\
0.480721081622434	20.457343483587\\
0.482724086129194	20.5980046222916\\
0.484727090635954	20.7570004104404\\
0.486730095142714	20.9341016649154\\
0.488733099649474	21.1291364983779\\
0.490736104156234	21.34187572771\\
0.492739108662994	21.5719755782345\\
0.494742113169755	21.8192068668335\\
0.496745117676515	22.0833404103888\\
0.498748122183275	22.3640324342234\\
0.500751126690035	22.6611110509987\\
0.502754131196795	22.9742324860377\\
0.504757135703555	23.3031102604428\\
0.506760140210315	23.6475724868754\\
0.508763144717076	24.0072753906586\\
0.510766149223836	24.3819897886741\\
0.512769153730596	24.7714292020245\\
0.514772158237356	25.1754217433713\\
0.516775162744116	25.5937382295963\\
0.518778167250876	26.026092181802\\
0.520781171757636	26.4723117126499\\
0.522784176264397	26.9322249348014\\
0.524787180771157	27.4056026651385\\
0.526790185277917	27.8922157205431\\
0.528793189784677	28.3920068052357\\
0.530796194291437	28.9046321445393\\
0.532799198798197	29.4300917384538\\
0.534802203304957	29.9681564038611\\
0.536805207811718	30.5186542534228\\
0.538808212318478	31.0814706955798\\
0.540811216825238	31.6564338429936\\
0.542814221331998	32.2434291041051\\
0.544817225838758	32.8423418873554\\
0.546820230345518	33.4529430096263\\
0.548823234852278	34.0751751751384\\
0.550826239359039	34.7088664965531\\
0.552829243865799	35.3539023823113\\
0.554832248372559	36.0100536492952\\
0.556835252879319	36.677263001725\\
0.558838257386079	37.3553012564828\\
0.560841261892839	38.0440538220096\\
0.562844266399599	38.7433488109668\\
0.56484727090636	39.4530143360158\\
0.56685027541312	40.1727639182591\\
0.56885327991988	40.9024829661377\\
0.57085628442664	41.6419422965336\\
0.5728592889334	42.3908554305491\\
0.57486229344016	43.1489931850662\\
0.57686529794692	43.9161263769669\\
0.57886830245368	44.691911231574\\
0.580871306960441	45.4761185657695\\
0.582874311467201	46.2683473090969\\
0.584877315973961	47.0682536868791\\
0.586880320480721	47.8754939244389\\
0.588883324987481	48.6896096555403\\
0.590886329494241	49.5102571055062\\
0.592889334001002	50.3369206123209\\
0.594892338507762	51.1690845139689\\
0.596895343014522	52.0062904442141\\
0.598898347521282	52.8479081494817\\
0.600901352028042	53.6934219677563\\
0.602904356534802	54.5422016454631\\
0.604907361041562	55.393559633248\\
0.606910365548322	56.2468083817568\\
0.608913370055083	57.1012603416354\\
0.610916374561843	57.9560560761911\\
0.612919379068603	58.8103934445107\\
0.614922383575363	59.6635276014605\\
0.616925388082123	60.5144272230092\\
0.618928392588883	61.3622328724643\\
0.620931397095643	62.205970521574\\
0.622934401602404	63.0446088463069\\
0.624937406109164	63.8770592268525\\
0.626940410615924	64.7022330433999\\
0.628943415122684	65.5190416761384\\
0.630946419629444	66.3262819136983\\
0.632949424136204	67.1227505447096\\
0.634952428642964	67.9071870620232\\
0.636955433149725	68.6783309584898\\
0.638958437656485	69.4348644311805\\
0.640961442163245	70.1755269729461\\
0.642964446670005	70.8988861892988\\
0.644967451176765	71.6036242773097\\
0.646970455683525	72.2884234340501\\
0.648973460190285	72.9517939692525\\
0.650976464697046	73.5924180799883\\
0.652979469203806	74.2089779633286\\
0.654982473710566	74.7999839290061\\
0.656985478217326	75.3641754698714\\
0.658988482724086	75.9001774872163\\
0.660991487230846	76.4067867696709\\
0.662994491737606	76.8826855143066\\
0.664997496244367	77.3266705097535\\
0.667000500751127	77.7376531362008\\
0.669003505257887	78.1145447738379\\
0.671006509764647	78.4563713944129\\
0.673009514271407	78.7621016738947\\
0.675012518778167	79.0309334713701\\
0.677015523284927	79.2621792374849\\
0.679018527791687	79.455151422885\\
0.681021532298448	79.6092197739956\\
0.683024536805208	79.7240405161399\\
0.685027541311968	79.7992698746405\\
0.687030545818728	79.8346213706001\\
0.689033550325488	79.8300950040186\\
0.691036554832248	79.7856334791164\\
0.693039559339009	79.7013513874527\\
0.695042563845769	79.5775352079249\\
0.697045568352529	79.4145287152102\\
0.699048572859289	79.2127902755446\\
0.701051577366049	78.9729501425029\\
0.703054581872809	78.69558127388\\
0.705057586379569	78.3815431063688\\
0.707060590886329	78.0316950766619\\
0.70906359539309	77.6469539172316\\
0.71106659989985	77.2282936563295\\
0.71306960440661	76.7769175053254\\
0.71507260891337	76.2939140840302\\
0.71707561342013	75.7804866038134\\
0.71907861792689	75.2379528676041\\
0.72108162243365	74.6675160867718\\
0.723084626940411	74.070551360025\\
0.725087631447171	73.4484337860719\\
0.727090635953931	72.8024811678415\\
0.729093640460691	72.1341258998214\\
0.731096644967451	71.4446857849404\\
0.733099649474211	70.735535921907\\
0.735102653980971	70.0081087052089\\
0.737105658487732	69.263721937775\\
0.739108662994492	68.5037507183134\\
0.741111667501252	67.7294555539736\\
0.743114672008012	66.9421542476844\\
0.745117676514772	66.1431073065949\\
0.747120681021532	65.3336325336341\\
0.749123685528292	64.5148185486127\\
0.751126690035053	63.6878685629003\\
0.753129694541813	62.8538711963079\\
0.755132699048573	62.0140296602051\\
0.757135703555333	61.1692606870642\\
0.759138708062093	60.320652896696\\
0.761141712568853	59.4691803173521\\
0.763144717075613	58.6157023857252\\
0.765147721582374	57.7611358342876\\
0.767150726089134	56.9063400997319\\
0.769153730595894	56.0521173229714\\
0.771156735102654	55.1992696449191\\
0.773159739609414	54.3484273191499\\
0.775162744116174	53.5003351907972\\
0.777165748622934	52.6556235134358\\
0.779168753129695	51.8149798364199\\
0.781171757636455	50.9788625259855\\
0.783174762143215	50.1479018357072\\
0.785177766649975	49.3226134276008\\
0.787180771156735	48.5033983721228\\
0.789183775663495	47.6908296270682\\
0.791186780170255	46.8851936713348\\
0.793189784677016	46.087006166938\\
0.795192789183776	45.2966108885551\\
0.797195793690536	44.5142943150834\\
0.799198798197296	43.7404575169797\\
0.801201802704056	42.975329677362\\
0.803204807210816	42.2192545709074\\
0.805207811717576	41.4724613807339\\
0.807210816224337	40.7352365857391\\
0.809213820731097	40.0077520732615\\
0.811216825237857	39.2902943221986\\
0.813219829744617	38.5830352198891\\
0.815222834251377	37.886089357892\\
0.817225838758137	37.1997432151048\\
0.819228843264897	36.524168678866\\
0.821231847771657	35.8594803407348\\
0.823234852278418	35.2058500880495\\
0.825237856785178	34.5633925123693\\
0.827240861291938	33.9322795010327\\
0.829243865798698	33.3126829413783\\
0.831246870305458	32.704717424965\\
0.833249874812218	32.1085548391313\\
0.835252879318979	31.5243097754364\\
0.837255883825739	30.9520395296598\\
0.839258888332499	30.3920305806989\\
0.841261892839259	29.8443402243333\\
0.843264897346019	29.3091976436812\\
0.845267901852779	28.7866601345218\\
0.847270906359539	28.276899584194\\
0.849273910866299	27.7801451758155\\
0.85127691537306	27.2965115009456\\
0.85327991987982	26.8261704469227\\
0.85528292438658	26.3694084926444\\
0.85728592889334	25.9262829338902\\
0.8592889334001	25.4970802495577\\
0.86129193790686	25.082029622765\\
0.86329494241362	24.68136023663\\
0.865297946920381	24.2952439784913\\
0.867300951427141	23.923910031467\\
0.869303955933901	23.5677021702342\\
0.871306960440661	23.2267922821314\\
0.873309964947421	22.9014095502766\\
0.875312969454181	22.5919550451264\\
0.877315973960941	22.2986006540194\\
0.879318978467702	22.0216328558532\\
0.881321982974462	21.7613381295253\\
0.883324987481222	21.5180029539332\\
0.885327991987982	21.2917992164155\\
0.887330996494742	21.0831279874289\\
0.889334001001502	20.8921611543118\\
0.891337005508262	20.7191279001823\\
0.893340010015023	20.5643147039379\\
0.895343014521783	20.4278361571378\\
0.897346019028543	20.3099787386794\\
0.899349023535303	20.2109143359013\\
0.901352028042063	20.1307002445829\\
0.903355032548823	20.069508352063\\
0.905358037055583	20.0274532499004\\
0.907361041562344	20.0045922338746\\
0.909364046069104	20.0009253039858\\
0.911367050575864	20.0165097560134\\
0.913370055082624	20.0512309983983\\
0.915373059589384	20.1051463269201\\
0.917376064096144	20.1781411500198\\
0.919379068602904	20.2700435803588\\
0.921382073109664	20.3807963221575\\
0.923385077616425	20.5101701922981\\
0.925388082123185	20.6580505992213\\
0.927391086629945	20.8242083598093\\
0.929394091136705	21.0083569951643\\
0.931397095643465	21.2103246179479\\
0.933400100150225	21.4298820450421\\
0.935403104656986	21.6667427975492\\
0.937406109163746	21.9206203965716\\
0.939409113670506	22.1912856589914\\
0.941412118177266	22.4783948101315\\
0.943415122684026	22.7817759626533\\
0.945418127190786	23.1011426376592\\
0.947421131697546	23.4361510604722\\
0.949424136204306	23.7865720479742\\
0.951427140711067	24.1521764170472\\
0.953430145217827	24.5326776887935\\
0.955433149724587	24.9278466800953\\
0.957436154231347	25.3375115036138\\
0.959439158738107	25.7613283846721\\
0.961442163244867	26.1991254359315\\
0.963445167751627	26.6507307700537\\
0.965448172258388	27.1159152039204\\
0.967451176765148	27.5945068501931\\
0.969454181271908	28.0863338215334\\
0.971457185778668	28.5911669348232\\
0.973460190285428	29.1088343027239\\
0.975463194792188	29.6392213336765\\
0.977466199298948	30.1821561403425\\
0.979469203805709	30.7375241311628\\
0.981472208312469	31.3050961230194\\
0.983475212819229	31.8848148201327\\
0.985478217325989	32.4764510393848\\
0.987481221832749	33.0799474849961\\
0.989484226339509	33.6951322696281\\
0.991487230846269	34.3218908017217\\
0.993490235353029	34.9600511939384\\
0.99549323985979	35.6094415589397\\
0.99749624436655	36.2699473051665\\
0.99949924887331	36.9413965452803\\
1.00150225338007	37.6236746877221\\
1.00350525788683	38.3166098451533\\
1.00550826239359	39.0199155386764\\
1.00751126690035	39.7335344725118\\
1.00951427140711	40.4572374635416\\
1.01151727591387	41.1907953286476\\
1.01352028042063	41.9339788847118\\
1.01552328492739	42.6865589486161\\
1.01752628943415	43.448249041463\\
1.01952929394091	44.2188199801345\\
1.02153229844767	44.9979279899534\\
1.02353530295443	45.7852865920221\\
1.02553830746119	46.5806093074432\\
1.02754131196795	47.3834377699805\\
1.02954431647471	48.1933709091775\\
1.03154732098147	49.0101222461365\\
1.03355032548823	49.8331761188419\\
1.03555332999499	50.6620168652781\\
1.03755633450175	51.4962434149886\\
1.03955933900851	52.3352255143987\\
1.04156234351527	53.1785047972722\\
1.04356534802203	54.0253364184756\\
1.04556835252879	54.8752047159931\\
1.04757135703555	55.7274221404707\\
1.04957436154231	56.5811865509952\\
1.05157736604907	57.4358676939918\\
1.05358037055583	58.290606132768\\
1.05558337506259	59.1445997264105\\
1.05758637956935	59.9969317424471\\
1.05958938407611	60.8467427441851\\
1.06159238858287	61.6930587033728\\
1.06359539308963	62.5349055917586\\
1.06559839759639	63.371252085311\\
1.06760140210315	64.2010095642195\\
1.06960440660991	65.0230321128937\\
1.07160741111667	65.8362311115229\\
1.07361041562344	66.6394033487372\\
1.0756134201302	67.4312883173876\\
1.07761642463696	68.210682806104\\
1.07961942914372	68.9762690119578\\
1.08162243365048	69.7267864277997\\
1.08362543815724	70.4608026591418\\
1.085628442664	71.1770571988348\\
1.08763144717076	71.8741749481705\\
1.08963445167752	72.550723512661\\
1.09163745618428	73.2053850893775\\
1.09364046069104	73.8367272838321\\
1.0956434651978	74.4434322930961\\
1.09764646970456	75.0241250184612\\
1.09964947421132	75.5774303612191\\
1.10165247871808	76.1020878142204\\
1.10365548322484	76.5967795745363\\
1.1056584877316	77.0603597265766\\
1.10766149223836	77.4915677631921\\
1.10966449674512	77.8893150645719\\
1.11166750125188	78.2525703066849\\
1.11367050575864	78.5803594612792\\
1.1156735102654	78.8717657958828\\
1.11767651477216	79.1259871695823\\
1.11967951927892	79.3422787372442\\
1.12168252378568	79.5201248368528\\
1.12368552829244	79.6588952148335\\
1.1256885327992	79.7582460965092\\
1.12769153730596	79.8178910029823\\
1.12969454181272	79.8376007511348\\
1.13169754631948	79.8174326367462\\
1.13370055082624	79.757329364037\\
1.135703555333	79.6575774119047\\
1.13770655983976	79.5183486676879\\
1.13970956434652	79.3401587934022\\
1.14171256885328	79.1234088595042\\
1.14371557336004	78.8687864153481\\
1.1457185778668	78.5769790102879\\
1.14772158237356	78.248788785237\\
1.14972458688032	77.885189768447\\
1.15172759138708	77.4870413966106\\
1.15373059589384	77.0554322895385\\
1.1557336004006	76.5915656586006\\
1.15773660490736	76.0964728278281\\
1.15973960941412	75.5715288959292\\
1.16174261392088	75.0178797784943\\
1.16374561842764	74.4369005742317\\
1.1657486229344	73.8299663818496\\
1.16775162744116	73.1983377084973\\
1.16975463194792	72.5434469486628\\
1.17175763645468	71.8666692010543\\
1.17376064096144	71.1693222686006\\
1.1757636454682	70.452895841569\\
1.17776664997496	69.7186504271088\\
1.17976965448172	68.9680184197079\\
1.18177265898848	68.2022603265156\\
1.18377566349524	67.4226939504606\\
1.185778668002	66.6306943902513\\
1.18778167250876	65.8274075614779\\
1.18978467701552	65.0140939712896\\
1.19178768152228	64.1919568310564\\
1.19379068602904	63.3621420563684\\
1.1957936905358	62.5257382670365\\
1.19779669504256	61.6838340828713\\
1.19979969954932	60.837460827904\\
1.20180270405608	59.9875925303864\\
1.20380570856284	59.1352605143498\\
1.2058087130696	58.2812669207074\\
1.20781171757636	57.4265284819312\\
1.20981472208312	56.5718473389345\\
1.21181772658988	55.7180829284101\\
1.21382073109664	54.865922799712\\
1.21582373560341	54.0160545021945\\
1.21782674011017	53.1692228809911\\
1.21982974461693	52.3260581896766\\
1.22183274912369	51.4870760902665\\
1.22383575363045	50.6529641321151\\
1.22583875813721	49.8241233856788\\
1.22784176264397	49.0011841045324\\
1.22984476715073	48.1844900633529\\
1.23184777165749	47.3746142199355\\
1.23385077616425	46.5718430531777\\
1.23585378067101	45.7766349293157\\
1.23785678517777	44.9893909188059\\
1.23985978968453	44.2103402047665\\
1.24186279419129	43.4398838576541\\
1.24386579869805	42.6782510605867\\
1.24586880320481	41.9257855882414\\
1.24787180771157	41.1827166237362\\
1.24987481221833	40.4492733501893\\
1.25187781672509	39.7256849507186\\
1.25388082123185	39.0121806084421\\
1.25588382573861	38.3089322106986\\
1.25788683024537	37.6161689406059\\
1.25988983475213	36.9340053897231\\
1.26189283925889	36.2626707411683\\
1.26389584376565	35.6022795865006\\
1.26589884827241	34.9530038130583\\
1.26790185277917	34.3149580124006\\
1.26990485728593	33.6883713676456\\
1.27190786179269	33.0733011745726\\
1.27391086629945	32.4699193205203\\
1.27591387080621	31.8783976928273\\
1.27791687531297	31.2988508830525\\
1.27991987981973	30.7313934827549\\
1.28192288432649	30.1761973792731\\
1.28392588883325	29.6333771641662\\
1.28592889334001	29.1031047247726\\
1.28793189784677	28.5855519484309\\
1.28993490235353	28.0808907224797\\
1.29193790686029	27.5892356384779\\
1.29394091136705	27.1107585837642\\
1.29594391587381	26.645746037236\\
1.29794692038057	26.1942552946729\\
1.29994992488733	25.756630130752\\
1.30195292939409	25.3329278412528\\
1.30395593390085	24.9234922008523\\
1.30595893840761	24.5284378011096\\
1.30796194291437	24.1481084167017\\
1.30996494742113	23.7826759349673\\
1.31196795192789	23.4323695390243\\
1.31397095643465	23.0975330035499\\
1.31597396094141	22.778395511662\\
1.31797696544817	22.4751862464787\\
1.31997996995493	22.1881916868977\\
1.32198297446169	21.917755607596\\
1.32398597896845	21.664049895912\\
1.32598898347521	21.4273610307435\\
1.32799198798197	21.2080327867674\\
1.32999499248873	21.0062370513223\\
1.33199799699549	20.8222603033058\\
1.33400100150225	20.656331725836\\
1.33600400600901	20.5086805020307\\
1.33800701051577	20.3794785192287\\
1.34001001502253	20.268954960548\\
1.34201301952929	20.1772244175476\\
1.34401602403605	20.1044587775659\\
1.34601902854281	20.0507726321622\\
1.34802203304957	20.0162232771158\\
1.35002503755633	20.0008680082063\\
1.35202804206309	20.0047068254337\\
1.35403104656985	20.0277970245774\\
1.35603405107661	20.0700813098581\\
1.35803705558337	20.1314450897166\\
1.36004006009014	20.211888364153\\
1.3620430645969	20.3111819500491\\
1.36404606910366	20.4292685516256\\
1.36604907361042	20.5658616899848\\
1.36805207811718	20.7209040693472\\
1.37005508262394	20.8941665065948\\
1.3720580871307	21.0853052270504\\
1.37406109163746	21.2942056391551\\
1.37606409614422	21.5205239682318\\
1.37806710065098	21.7640883269419\\
1.38007010515774	22.0245549406084\\
1.3820731096645	22.3017519218926\\
1.38407611417126	22.5952782003382\\
1.38607911867802	22.9049045928269\\
1.38808212318478	23.2304019162407\\
1.39008512769154	23.5714836916821\\
1.3920881321983	23.927920736033\\
1.39409113670506	24.2993692746163\\
1.39609414121182	24.6856574200935\\
1.39809714571858	25.086498693567\\
1.40010015022534	25.5017212076983\\
1.4021031547321	25.9310384835898\\
1.40410615923886	26.374278633903\\
1.40610916374562	26.8312697712994\\
1.40811216825238	27.3017254168813\\
1.41011517275914	27.7854736833102\\
1.4121181772659	28.2823999790272\\
1.41412118177266	28.7922751209141\\
1.41612418627942	29.314984517412\\
1.41812719078618	29.8502989854027\\
1.42013019529294	30.3981039333273\\
1.4221331997997	30.9582847696267\\
1.42413620430646	31.5306123111829\\
1.42613920881322	32.1150292622163\\
1.42814221331998	32.711306439609\\
1.43014521782674	33.3193865475813\\
1.4321482223335	33.9391549945743\\
1.43415122684026	34.5703253016904\\
1.43615423134702	35.2128974689296\\
1.43815723585378	35.8666996089534\\
1.44016024036054	36.5315025386437\\
1.4421632448673	37.207248962221\\
1.44416624937406	37.8936524007877\\
1.44616925388082	38.5907128543439\\
1.44817225838758	39.2980865482124\\
1.45017526289434	40.0156588908343\\
1.4521782674011	40.7432579948709\\
1.45418127190786	41.4805973814247\\
1.45618427641462	42.2275051631573\\
1.45818728092138	42.9836375653914\\
1.46019028542814	43.7488799965682\\
1.4621932899349	44.5228313862309\\
1.46419629444166	45.305205255482\\
1.46619929894842	46.095715125424\\
1.46820230345518	46.8939599256003\\
1.47020530796194	47.6996531771132\\
1.4722083124687	48.5123365137268\\
1.47421131697546	49.3316088649844\\
1.47621432148222	50.1569545688703\\
1.47821732598898	50.9879725549281\\
1.48022033049574	51.824147161142\\
1.4822233350025	52.6648481339374\\
1.48422633950926	53.5096171070783\\
1.48622934401602	54.357709235431\\
1.48823234852278	55.2085515612002\\
1.49023535302954	56.061456535032\\
1.4922383575363	56.9157366075721\\
1.49424136204306	57.7704750463482\\
1.49624436654982	58.6250415977858\\
1.49824737105658	59.4784622336332\\
1.50025037556334	60.3299348129771\\
1.50225338007011	61.1785426033454\\
1.50425638457687	62.0231969849272\\
1.50625938908363	62.86303852103\\
1.50826239359039	63.6969212960634\\
1.51026539809715	64.5238139859962\\
1.51226840260391	65.3425133794586\\
1.51427140711067	66.1519308566399\\
1.51627441161743	66.9508059103908\\
1.51827741612419	67.7379926251211\\
1.52028042063095	68.5121159021223\\
1.52228342513771	69.2719725300248\\
1.52428642964447	70.0161301143408\\
1.52628943415123	70.7433854437003\\
1.52829243865799	71.4523061236157\\
1.53029544316475	72.1415170553785\\
1.53229844767151	72.8097004360601\\
1.53430145217827	73.455366575393\\
1.53630445668503	74.077254966228\\
1.53830746119179	74.6739332140773\\
1.54031046569855	75.2440262202324\\
1.54231347020531	75.7862734775442\\
1.54431647471207	76.2993571830839\\
1.54631947921883	76.7820168297021\\
1.54832248372559	77.2330492060291\\
1.55032548823235	77.6513083964746\\
1.55232849273911	78.0357057812279\\
1.55433149724587	78.3852100362577\\
1.55633450175263	78.6988471333123\\
1.55833750625939	78.975757635699\\
1.56034051076615	79.2151966982842\\
1.56234351527291	79.4165340674931\\
1.56434651977967	79.5790821939718\\
1.56634952428643	79.7024973030429\\
1.56835252879319	79.7863210284706\\
1.57035553329995	79.8303814829161\\
1.57235853780671	79.8344494832616\\
1.57436154231347	79.7986396210659\\
1.57636454682023	79.7230091921086\\
1.57836755132699	79.6077300837283\\
1.58037055583375	79.4532033663815\\
1.58237356034051	79.2598301105249\\
1.58437656484727	79.028240569733\\
1.58637956935403	78.7589504060215\\
1.58838257386079	78.4528190560831\\
1.59038557836755	78.110648660831\\
1.59238858287431	77.7333559527373\\
1.59439158738107	77.3220295516129\\
1.59639459188783	76.8776434857095\\
1.59839759639459	76.4014009663967\\
1.60040060090135	75.894447909265\\
1.60240360540811	75.358102117243\\
1.60440660991487	74.7936240974801\\
1.60640961442163	74.2023316529051\\
1.60841261892839	73.5855425864468\\
1.61041562343515	72.9446892925929\\
1.61241862794191	72.2810322784929\\
1.61442163244867	71.596061234414\\
1.61642463695543	70.891093963285\\
1.61842764146219	70.1675055638143\\
1.62043064596895	69.4266711347102\\
1.62243365047571	68.6699657746809\\
1.62443665498247	67.8986499908758\\
1.62643965948923	67.1140988820031\\
1.62844266399599	66.3175156594328\\
1.63044566850275	65.5101608303139\\
1.63244867300951	64.6932376060164\\
1.63445167751627	63.8680064936895\\
1.63645468202303	63.0354415215849\\
1.63845768652979	62.1967459010723\\
1.64046069103655	61.3530082519627\\
1.64246369554331	60.5051453067281\\
1.64446670005008	59.6541883893998\\
1.64646970455684	58.80105423245\\
1.6484727090636	57.9467168641304\\
1.65047571357036	57.0918638337953\\
1.65247871807712	56.2374691696962\\
1.65448172258388	55.3842204211874\\
1.65648472709064	54.532919729182\\
1.6584877315974	53.6841400514752\\
1.66049073610416	52.8386835289801\\
1.66249374061092	51.9970658237125\\
1.66449674511768	51.1599171892468\\
1.66649974962444	50.3278105833783\\
1.6685027541312	49.5012616681226\\
1.67050575863796	48.6806715139362\\
1.67250876314472	47.8666130786144\\
1.67451176765148	47.0594874326136\\
1.67651477215824	46.2596383506109\\
1.678517776665	45.4674669030631\\
1.68052078117176	44.6833741604265\\
1.68252378567852	43.9077038973784\\
1.68452679018528	43.1406280012573\\
1.68652979469204	42.3826048382992\\
1.6885327991988	41.6337490000632\\
1.69053580370556	40.8944615570059\\
1.69253880821232	40.1648571006863\\
1.69454181271908	39.4451648142225\\
1.69654481722584	38.735671176512\\
1.6985478217326	38.0364907791139\\
1.70055082623936	37.3478528051461\\
1.70255383074612	36.6698718461678\\
1.70455683525288	36.0028343810765\\
1.70655983975964	35.3467404098722\\
1.7085628442664	34.7018764114525\\
1.71056584877316	34.0682996815968\\
1.71256885327992	33.4461821076438\\
1.71457185778668	32.8356955769319\\
1.71657486229344	32.2369546810201\\
1.7185778668002	31.6500740114676\\
1.72058087130696	31.0752254556129\\
1.72258387581372	30.5125809007944\\
1.72458688032048	29.9621976427918\\
1.72658988482724	29.4243048647229\\
1.728592889334	28.8989598623675\\
1.73059589384076	28.386449114623\\
1.73259889834752	27.8868299172689\\
1.73460190285428	27.4003314534233\\
1.73660490736104	26.9271256104248\\
1.7386079118678	26.4673842756118\\
1.74061091637456	26.0212793363229\\
1.74261392088132	25.5890972714557\\
1.74461692538808	25.1709526725693\\
1.74661992989484	24.7670747227816\\
1.7486229344016	24.3778071967697\\
1.75062593890836	24.0032646860927\\
1.75262894341512	23.6437336696481\\
1.75463194792188	23.299443330554\\
1.75663495242864	22.9707374434874\\
1.7586379569354	22.6577878957869\\
1.76064096144216	22.3608811663502\\
1.76264396594892	22.0803610298541\\
1.76464697045568	21.8163993736373\\
1.76664997496244	21.5693399723769\\
1.7686529794692	21.3394120091909\\
1.77065598397596	21.1269592587564\\
1.77265898848272	20.9320963126324\\
1.77466199298948	20.755166945496\\
1.77666499749624	20.5963430446858\\
1.778668002003	20.4559110890992\\
1.78067100650976	20.3339856702954\\
1.78267401101652	20.2307959713923\\
1.78467701552328	20.1465138797285\\
1.78668002003005	20.0811966910836\\
1.78868302453681	20.0349589970166\\
1.79068602904357	20.0079153890864\\
1.79268903355033	20.0000658672931\\
1.79469203805709	20.0114677274162\\
1.79669504256385	20.0420636736762\\
1.79869804707061	20.0918537060731\\
1.80070105157737	20.1606659372683\\
1.80270405608413	20.2485003672618\\
1.80470706059089	20.3551851087152\\
1.80671006509765	20.4805482742898\\
1.80871306960441	20.6244179766472\\
1.81071607411117	20.7865650326692\\
1.81271907861793	20.9668748507969\\
1.81472208312469	21.1650036563531\\
1.81672508763145	21.3807222662199\\
1.81872809213821	21.6138587930586\\
1.82073109664497	21.8640694621922\\
1.82273410115173	22.1311250905027\\
1.82473710565849	22.4146819033129\\
1.82674011016525	22.7145680132844\\
1.82874311467201	23.03043964574\\
1.83074611917877	23.3620676175618\\
1.83274912368553	23.7092227456315\\
1.83475212819229	24.0715612552723\\
1.83675513269905	24.4488539633659\\
1.83875813720581	24.8408716867944\\
1.84076114171257	25.2473852424397\\
1.84276414621933	25.6681081514043\\
1.84476715072609	26.1029258221291\\
1.84677015523285	26.5515517757165\\
1.84877315973961	27.0138141248281\\
1.85077616424637	27.4895409821252\\
1.85277916875313	27.9784458687103\\
1.85478217325989	28.4804714888039\\
1.85678517776665	28.995388659288\\
1.85878818227341	29.523025492824\\
1.86079118678017	30.0632673978528\\
1.86279419128693	30.615942487036\\
1.86479719579369	31.180878873035\\
1.86680020030045	31.7579619642908\\
1.86880320480721	32.3470198734648\\
1.87080620931397	32.9479953047775\\
1.87280921382073	33.5606590751109\\
1.87481221832749	34.1848965929059\\
1.87681522283425	34.8205359708241\\
1.87881822734101	35.4675199130858\\
1.88082123184777	36.1256192365731\\
1.88282423635453	36.7947193497268\\
1.88482724086129	37.4746483652086\\
1.88683024536805	38.1652916914593\\
1.88883324987481	38.8663628495813\\
1.89083625438157	39.5778045437953\\
1.89283925888833	40.2993302952035\\
1.89484226339509	41.0307682164675\\
1.89684526790185	41.7718318286897\\
1.89884827240861	42.5224065403111\\
1.90085127691537	43.2821485766546\\
1.90285428142213	44.0508287546021\\
1.90485728592889	44.8281032994766\\
1.90686029043565	45.6136857323805\\
1.90886329494241	46.4072895744162\\
1.91086629944917	47.2085137551271\\
1.91286930395593	48.0169572040567\\
1.91487230846269	48.8322761465279\\
1.91687531296945	49.6540122163045\\
1.91887831747621	50.481649751371\\
1.92088132198297	51.3147303854912\\
1.92288432648973	52.1527384566495\\
1.92488733099649	52.9951010070508\\
1.92689033550325	53.8412450789\\
1.92889334001002	54.6905404186225\\
1.93089634451678	55.5422421810844\\
1.93289934902354	56.3957774084908\\
1.9349023535303	57.2502866641489\\
1.93690535803706	58.1050823987046\\
1.93890836254382	58.9593051754652\\
1.94091136705058	59.8120955577379\\
1.94291437155734	60.6625368130506\\
1.9449173760641	61.509712208931\\
1.94692038057086	62.3526477171275\\
1.94892338507762	63.1902547178292\\
1.95092638958438	64.021501887005\\
1.95292939409114	64.8453579006236\\
1.9549323985979	65.6605622515358\\
1.95693540310466	66.4659690241511\\
1.95893840761142	67.2604323028796\\
1.96094141211818	68.0426342847921\\
1.96294441662494	68.8113144627397\\
1.9649474211317	69.5652123295728\\
1.96695042563846	70.3029527865832\\
1.96895343014522	71.0231607350627\\
1.97095643465198	71.7245183720823\\
1.97295943915874	72.4057078947134\\
1.9749624436655	73.0651823169089\\
1.97696544817226	73.7017384272993\\
1.97896845267902	74.3138865356171\\
1.98097145718578	74.9003088389335\\
1.98297446169254	75.4596875343197\\
1.9849774661993	75.9907048188469\\
1.98698047070606	76.4919855938069\\
1.98898347521282	76.9624412393888\\
1.99098647971958	77.4007539526639\\
1.99298948422634	77.8058924096009\\
1.9949924887331	78.1767106946096\\
1.99699549323986	78.5122920752177\\
1.99899849774662	78.8116625231735\\
2.00100150225338	79.0740198975639\\
2.00300450676014	79.2986193532552\\
2.0050075112669	79.4848306366727\\
2.00701051577366	79.6320807900214\\
2.00901352028042	79.740026038624\\
2.01101652478718	79.8083226078036\\
2.01301952929394	79.8367413144421\\
2.0150225338007	79.82522486276\\
2.01702553830746	79.7737732527572\\
2.01902854281422	79.6826156675519\\
2.02103154732098	79.5519239944826\\
2.02303455182774	79.3820993040058\\
2.0250375563345	79.1737145539167\\
2.02704056084126	78.9272281104514\\
2.02904356534802	78.6434421145231\\
2.03104656985478	78.3231014112655\\
2.03304957436154	77.9670654373712\\
2.0350525788683	77.576308221092\\
2.03705558337506	77.1519183822386\\
2.03905858788182	76.6949272448422\\
2.04106159238858	76.2065380202727\\
2.04306459689534	75.6879539198998\\
2.0450676014021	75.1404354508728\\
2.04707060590886	74.5653004161205\\
2.04907361041562	73.9638666185716\\
2.05107661492238	73.3374518611551\\
2.05307961942914	72.6874885383587\\
2.0550826239359	72.0153517488907\\
2.05708562844266	71.32235929568\\
2.05908863294942	70.6099435732143\\
2.06109163745618	69.8794796802021\\
2.06309464196294	69.1322854195719\\
2.0650976464697	68.3696785942528\\
2.06710065097646	67.593034302953\\
2.06910365548322	66.8036130528217\\
2.07110665998998	66.0026753510084\\
2.07310966449675	65.1914244088826\\
2.07511266900351	64.3711207335938\\
2.07711567351027	63.5428529449527\\
2.07911867801703	62.707824254329\\
2.08112168252379	61.8670086899746\\
2.08312468703055	61.0214948717\\
2.08512769153731	60.1723141235366\\
2.08713069604407	59.3204404737361\\
2.08913370055083	58.4667333589912\\
2.09113670505759	57.6120522159945\\
2.09313970956435	56.7573137772184\\
2.09514271407111	55.9032628877964\\
2.09714571857787	55.0507016886417\\
2.09914872308463	54.2003177291085\\
2.10115172759139	53.3527412627715\\
2.10315473209815	52.5087171347643\\
2.10515773660491	51.668818302882\\
2.10716074111167	50.8335604291403\\
2.10916374561843	50.0035737671138\\
2.11116675012519	49.1792593872591\\
2.11316975463195	48.3611902473713\\
2.11517275913871	47.5498247136865\\
2.11717576364547	46.7454492651024\\
2.11917876815223	45.9485795636344\\
2.12118177265899	45.1595593839598\\
2.12318477716575	44.3787325007555\\
2.12518778167251	43.6063853929191\\
2.12719078617927	42.8428045393483\\
2.12919379068603	42.08833371472\\
2.13119679519279	41.3432021021524\\
2.13319979969955	40.6076388847634\\
2.13520280420631	39.8819305414507\\
2.13720580871307	39.1661916637733\\
2.13920881321983	38.4607087306287\\
2.14121181772659	37.7656536293555\\
2.14321482223335	37.0811409515127\\
2.14521782674011	36.4074571759979\\
2.14722083124687	35.7446595985906\\
2.14922383575363	35.0929774024088\\
2.15122684026039	34.4525251790115\\
2.15322984476715	33.8234175199579\\
2.15523284927391	33.2058263125864\\
2.15723585378067	32.5999234442355\\
2.15923885828743	32.0058235064644\\
2.16124186279419	31.4236410908319\\
2.16324486730095	30.8534907888973\\
2.16524787180771	30.2956590795579\\
2.16725087631447	29.7501459628138\\
2.16925388082123	29.2171233260036\\
2.17125688532799	28.6968203522453\\
2.17325988983475	28.1892943373185\\
2.17526289434151	27.6948317601206\\
2.17726589884827	27.2134899164311\\
2.17926890335503	26.7455552851478\\
2.18127190786179	26.2911424578295\\
2.18327491236855	25.8504233218149\\
2.18527791687531	25.4236843560015\\
2.18728092138207	25.0111547435073\\
2.18928392588883	24.6129490758914\\
2.19128693039559	24.2294111278308\\
2.19328993490235	23.8607127866641\\
2.19529293940911	23.5071405312889\\
2.19729594391587	23.1689235448232\\
2.19929894842263	22.846291010385\\
2.20130195292939	22.539643998431\\
2.20330495743615	22.2490971005201\\
2.20530796194291	21.9750513871091\\
2.20731096644967	21.7176787455363\\
2.20931397095643	21.4772656546994\\
2.21131697546319	21.2541558892755\\
2.21331997996995	21.048521336603\\
2.21532298447672	20.8606484755796\\
2.21732598898348	20.6908237851028\\
2.21932899349024	20.5391618567317\\
2.221331997997	20.4059491693638\\
2.22333500250376	20.2913576103376\\
2.22533800701052	20.1955590669917\\
2.22734101151728	20.1186681308852\\
2.22934401602404	20.060799393577\\
2.2313470205308	20.0220674466261\\
2.23335002503756	20.0025868815917\\
2.23535302954432	20.0022431069146\\
2.23735603405108	20.0211507141539\\
2.23935903855784	20.0592524075301\\
2.2413620430646	20.1165481870432\\
2.24336504757136	20.1928088695751\\
2.24536805207812	20.2880344551259\\
2.24737105658488	20.4020530563569\\
2.24937406109164	20.5346354901502\\
2.2513770655984	20.6857244607262\\
2.25338007010516	20.8550334891873\\
2.25538307461192	21.0422760966361\\
2.25738607911868	21.2473949872929\\
2.25938908362544	21.4699317949217\\
2.2613920881322	21.7097719279635\\
2.26339509263896	21.9666289075206\\
2.26539809714572	22.2401589589161\\
2.26740110165248	22.5301901948113\\
2.26940410615924	22.8363215447497\\
2.271407110666	23.1584384171723\\
2.27341011517276	23.4961397416224\\
2.27541311967952	23.8492536307615\\
2.27741612418628	24.2174363099126\\
2.27941912869304	24.6005731875165\\
2.2814221331998	24.9982631931168\\
2.28342513770656	25.4103917351544\\
2.28542814221332	25.8366723347318\\
2.28743114672008	26.2769331045103\\
2.28943415122684	26.7308875655925\\
2.2914371557336	27.1984211264192\\
2.29344016024036	27.679361899652\\
2.29544316474712	28.1734234063933\\
2.29744616925388	28.6804910550841\\
2.29944917376064	29.2004502541653\\
2.3014521782674	29.7330145247394\\
2.30345518277416	30.2781265710269\\
2.30545818728092	30.8356145056892\\
2.30746119178768	31.4053637371673\\
2.30946419629444	31.9871450823431\\
2.3114672008012	32.5809012454372\\
2.31347020530796	33.1864030433314\\
2.31547320981472	33.8036504760259\\
2.31747621432148	34.4323570646229\\
2.31947921882824	35.0724655133431\\
2.321482223335	35.7238039348478\\
2.32348522784176	36.3862577375781\\
2.32548823234852	37.0595977384158\\
2.32749123685528	37.7437666415815\\
2.32949424136204	38.4384779681776\\
2.3314972458688	39.1436744224247\\
2.33350025037556	39.859069525425\\
2.33550325488232	40.5844913898401\\
2.33750625938908	41.319710832552\\
2.33950926389584	42.0645559662221\\
2.3415122684026	42.8187403119528\\
2.34351527290936	43.5819773908466\\
2.34551827741612	44.3540953155649\\
2.34752128192288	45.1346930156511\\
2.34952428642964	45.9234267164282\\
2.3515272909364	46.7200672347781\\
2.35353029544316	47.5241562044647\\
2.35553329994992	48.3353498508109\\
2.35753630445668	49.1532471033601\\
2.35953930896345	49.9773323000968\\
2.36154231347021	50.8071470747848\\
2.36354531797697	51.6422330611879\\
2.36554832248373	52.4820173015112\\
2.36755132699049	53.3259268379594\\
2.36955433149725	54.1733887127374\\
2.37155733600401	55.0236580807115\\
2.37356034051077	55.8762192798662\\
2.37556334501753	56.7302128735087\\
2.37756634952429	57.5849513122848\\
2.37956935403105	58.4396324552815\\
2.38157235853781	59.2933968658059\\
2.38357536304457	60.145327811386\\
2.38557836755133	60.9946231511084\\
2.38758137205809	61.840251560942\\
2.38958437656485	62.681239012635\\
2.39158738107161	63.5164968863767\\
2.39359038557837	64.3449365623563\\
2.39559339008513	65.1655267165427\\
2.39759639459189	65.977064137566\\
2.39959939909865	66.7783456140565\\
2.40160240360541	67.5681679346443\\
2.40360540811217	68.3452705921802\\
2.40560841261893	69.108278487956\\
2.40761141712569	69.8559884106017\\
2.40961442163245	70.5870252614091\\
2.41161742613921	71.3000712374494\\
2.41362043064597	71.9936939442348\\
2.41562343515273	72.6665182830569\\
2.41762643965949	73.317226450987\\
2.41962944416625	73.9443287577577\\
2.42163244867301	74.5466219919992\\
2.42363545317977	75.1226164634442\\
2.42563845768653	75.6709943691639\\
2.42764146219329	76.1905524977886\\
2.42964446670005	76.6799157506098\\
2.43164747120681	77.1379382120374\\
2.43365047571357	77.563359374922\\
2.43565348022033	77.955205211012\\
2.43765648472709	78.312329804717\\
2.43965948923385	78.6338164235649\\
2.44166249374061	78.918805630863\\
2.44366549824737	79.1664379899185\\
2.44566850275413	79.3760832471569\\
2.44767150726089	79.5471111490035\\
2.44967451176765	79.6790633292221\\
2.45167751627441	79.7714814215767\\
2.45368052078117	79.8241935387287\\
2.45568352528793	79.8369704975601\\
2.45768652979469	79.8098122980709\\
2.45968953430145	79.7428335318202\\
2.46169253880821	79.6361487903668\\
2.46369554331497	79.490101848388\\
2.46569854782173	79.3050937763402\\
2.46770155232849	79.0816975320187\\
2.46970455683525	78.8205433689981\\
2.47170756134201	78.522376132412\\
2.47371056584877	78.1879406673941\\
2.47571357035553	77.8182110021962\\
2.47771657486229	77.4141611650699\\
2.47971957936905	76.9768224800466\\
2.48172258387581	76.5073981584959\\
2.48372558838257	76.0070341160082\\
2.48572859288933	75.4769908597326\\
2.48773159739609	74.9185288968186\\
2.48973460190285	74.3329087344154\\
2.49173760640961	73.7215627670108\\
2.49374061091637	73.0857515017541\\
2.49574361542313	72.4269646289127\\
2.49774661992989	71.7464626556359\\
2.49974962443665	71.0457352721908\\
2.50175262894342	70.3261002815065\\
2.50375563345018	69.5888754865117\\
2.50575863795694	68.8354932816942\\
2.5077616424637	68.0672714699828\\
2.50976464697046	67.2854705585268\\
2.51176765147722	66.4913510544754\\
2.51377065598398	65.6862880565371\\
2.51577366049074	64.871312888743\\
2.5177766649975	64.0478006498015\\
2.51977966950426	63.2167253679643\\
2.52178267401102	62.379290254601\\
2.52378567851778	61.5365266337431\\
2.52578868302454	60.6894658294217\\
2.5277916875313	59.839139165668\\
2.52979469203806	58.9864060791748\\
2.53179769654482	58.1322405981938\\
2.53380070105158	57.2774448636381\\
2.53580370555834	56.4228783122005\\
2.5378067100651	55.5693430847941\\
2.53980971457186	54.7175267307731\\
2.54181271907862	53.8681167994917\\
2.54381572358538	53.021915431863\\
2.54581872809214	52.1793809941231\\
2.5478217325989	51.3412583314057\\
2.54982473710566	50.508005809947\\
2.55182774161242	49.680196387542\\
2.55383074611918	48.8582884304268\\
2.55583375062594	48.0427403048376\\
2.5578367551327	47.2340676727899\\
2.55983975963946	46.4326143089609\\
2.56184276414622	45.6387812838072\\
2.56384576865298	44.8529123720057\\
2.56584877315974	44.0753513482337\\
2.5678517776665	43.3063846913886\\
2.56985478217326	42.5463561761476\\
2.57185778668002	41.7954949856287\\
2.57386079118678	41.0540875987294\\
2.57586379569354	40.3223631985678\\
2.5778668002003	39.600550968262\\
2.57986980470706	38.8888227951505\\
2.58187280921382	38.1873505665718\\
2.58387581372058	37.4964207614235\\
2.58587881822734	36.8161479712647\\
2.5878818227341	36.1467040834339\\
2.58988482724086	35.48820368949\\
2.59188783174762	34.8408759725512\\
2.59389083625438	34.204892819956\\
2.59589384076114	33.5802542317044\\
2.5978968452679	32.9672466866939\\
2.59989984977466	32.3659274807041\\
2.60190285428142	31.7764685010735\\
2.60390585878818	31.1989843393612\\
2.60590886329494	30.6336468829056\\
2.6079118678017	30.0806280190453\\
2.60991487230846	29.5399850435599\\
2.61191787681522	29.0119471395673\\
2.61392088132198	28.4966288986266\\
2.61592388582874	27.9942022080764\\
2.6179268903355	27.5048389552552\\
2.61992989484226	27.0287110275015\\
2.62193289934902	26.5660476079333\\
2.62393590385578	26.1169632881098\\
2.62593890836254	25.6816872511489\\
2.6279419128693	25.2605059759482\\
2.62994491737606	24.8535340540668\\
2.63194792188282	24.4610579644022\\
2.63395092638958	24.0832495942929\\
2.63595393089634	23.7204527184161\\
2.6379569354031	23.3728965198897\\
2.63995993990986	23.0407528860524\\
2.64196294441662	22.7243655915811\\
2.64396594892338	22.4239638195941\\
2.64596895343015	22.1398340489887\\
2.64797195793691	21.8722627586626\\
2.64997496244367	21.6215364275133\\
2.65197796695043	21.387884238659\\
2.65398097145719	21.1715926709971\\
2.65598397596395	20.9728909076457\\
2.65798698047071	20.7920081317229\\
2.65998998497747	20.6292881179058\\
2.66199298948423	20.4847881619738\\
2.66399599399099	20.358852038604\\
2.66599899849775	20.2515943393555\\
2.66800200300451	20.1631869515669\\
2.67000500751127	20.093744466797\\
2.67200801201803	20.043381476605\\
2.67401101652479	20.0121552767704\\
2.67601402103155	20.0001231630726\\
2.67801702553831	20.0073424312913\\
2.68002003004507	20.0338130814263\\
2.68202303455183	20.0794205219187\\
2.68402603905859	20.144107456989\\
2.68602904356535	20.2278165908576\\
2.68803204807211	20.3304333319655\\
2.69003505257887	20.4517284971947\\
2.69203805708563	20.5915874949862\\
2.69404106159239	20.7498384380013\\
2.69604406609915	20.9261948473426\\
2.69804707060591	21.1204848356714\\
2.70005007511267	21.3324219240903\\
2.70205307961943	21.5617769294812\\
2.70405608412619	21.808320668726\\
2.70605908863295	22.0717093671476\\
2.70806209313971	22.3517711374076\\
2.71006509764647	22.6481049090492\\
2.71206810215323	22.9605387947341\\
2.71407110665999	23.2887863155645\\
2.71607411116675	23.632560992643\\
2.71807711567351	23.991576347072\\
2.72008012018027	24.3656604915129\\
2.72208312468703	24.7545269470682\\
2.72408612919379	25.1578892348403\\
2.72608913370055	25.5755754674907\\
2.72809213820731	26.0073564619012\\
2.73009514271407	26.453003034954\\
2.73209814722083	26.9123432993104\\
2.73410115172759	27.3851480718523\\
2.73610415623435	27.8711881694618\\
2.73810716074111	28.3704062963593\\
2.74011016524787	28.8825159736472\\
2.74211316975463	29.4074026097666\\
2.74411617426139	29.9449516131583\\
2.74611917876815	30.4949338007044\\
2.74812218327491	31.0571772850663\\
2.75012518778167	31.631682066244\\
2.75212819228843	32.2181616653399\\
2.75413119679519	32.8165587865745\\
2.75613420130195	33.4266442468298\\
2.75813720580871	34.0484180461058\\
2.76014021031547	34.6815937055048\\
2.76214321482223	35.3261139292475\\
2.76414621932899	35.9818068299952\\
2.76614922383575	36.648557816189\\
2.76815222834251	37.3261950004902\\
2.77015523284927	38.0144891997808\\
2.77215823735603	38.7133258225019\\
2.77416124186279	39.4225329813148\\
2.77616424636955	40.1418241973221\\
2.77816725087631	40.8711421747441\\
2.78017025538307	41.6101431389038\\
2.78217325988983	42.3587124982423\\
2.78417626439659	43.1164491823028\\
2.78617926890335	43.8832385995263\\
2.78818227341012	44.6586796794564\\
2.79018527791688	45.4424859431954\\
2.79218828242364	46.2343709118457\\
2.7941912869304	47.0339908107303\\
2.79619429143716	47.840887273613\\
2.79819729594392	48.6547738215963\\
2.80020030045068	49.4751347926646\\
2.80220330495744	50.3015118205818\\
2.8042063094642	51.1334465391118\\
2.80620931397096	51.9704232862389\\
2.80821231847772	52.8119263999475\\
2.81021532298448	53.6572683308835\\
2.81221832749124	54.5059334170313\\
2.814221331998	55.3571768132572\\
2.81622433650476	56.2103682659865\\
2.81822734101152	57.0647629300856\\
2.82023034551828	57.9195586646413\\
2.82223335002504	58.7740106245198\\
2.8242363545318	59.6271447814696\\
2.82623935903856	60.4781589945775\\
2.82824236354532	61.3261365313711\\
2.83024536805208	62.1700460678193\\
2.83224837255884	63.0089135756703\\
2.8342513770656	63.8416504351134\\
2.83625438157236	64.6671680263379\\
2.83825738607912	65.484377729533\\
2.84026039058588	66.2920190375494\\
2.84226339509264	67.0890033305764\\
2.8442663995994	67.8739555099056\\
2.84626940410616	68.6456723641673\\
2.84827240861292	69.4028933862122\\
2.85027541311968	70.144243477332\\
2.85227841762644	70.8684048345979\\
2.8542814221332	71.5739450635219\\
2.85628442663996	72.259603656955\\
2.85828743114672	72.9239482204092\\
2.86029043565348	73.5656036551762\\
2.86229344016024	74.1831375667682\\
2.864296444667	74.7752894480359\\
2.86629944917376	75.3406269044915\\
2.86830245368052	75.8778894289857\\
2.87030545818728	76.3857592185896\\
2.87230846269404	76.8630330619336\\
2.8743114672008	77.3083931560888\\
2.87631447170756	77.720808177024\\
2.87831747621432	78.0991895049284\\
2.88032048072108	78.4425058157707\\
2.88232348522784	78.7497830812994\\
2.8843264897346	79.0202191606012\\
2.88632949424136	79.2530692085423\\
2.88833249874812	79.4477029715483\\
2.89033550325488	79.6034329002648\\
2.89233850776164	79.7199725157944\\
2.8943415122684	79.796863451901\\
2.89634451677516	79.8339338212459\\
2.89834752128192	79.8311263280498\\
2.90035052578868	79.7883263807535\\
2.90235353029544	79.7057631624752\\
2.9043565348022	79.5836085605533\\
2.90635953930896	79.4222636454444\\
2.90836254381572	79.2221867833848\\
2.91036554832248	78.9839509321694\\
2.91236855282924	78.7081863453729\\
2.914371557336	78.3956951639086\\
2.91637456184276	78.047336824469\\
2.91837756634952	77.6640280595265\\
2.92038057085628	77.2468574888917\\
2.92238357536304	76.796856436596\\
2.9243865798698	76.3151708182295\\
2.92638958437656	75.8030038451621\\
2.92839258888332	75.2616733203225\\
2.93039559339008	74.69243975086\\
2.93239859789685	74.0965636439239\\
2.93440160240361	73.4754773940021\\
2.93640460691037	72.8304988040233\\
2.93840761141713	72.1630602684755\\
2.94041061592389	71.4744795902872\\
2.94241362043065	70.7661891639465\\
2.94441662493741	70.0395067923821\\
2.94641962944417	69.2958075743023\\
2.94842263395093	68.5364666084154\\
2.95042563845769	67.7627444018707\\
2.95242864296445	66.9759587575971\\
2.95443164747121	66.1774274785233\\
2.95643465197797	65.368353776019\\
2.95843765648473	64.5498835656746\\
2.96044066099149	63.7232773546394\\
2.96244366549825	62.8896237627241\\
2.96444667000501	62.0499541139598\\
2.96644967451177	61.205414323937\\
2.96845267901853	60.3569211251278\\
2.97045568352529	59.5055631373429\\
2.97245868803205	58.6521425014955\\
2.97446169253881	57.7976332458374\\
2.97646469704557	56.9428375112817\\
2.97846770155233	56.0885574387417\\
2.98047070605909	55.2355951691304\\
2.98247371056585	54.3846955475816\\
2.98447671507261	53.53648882767\\
2.98647971957937	52.6916052629701\\
2.98848272408613	51.8507324028361\\
2.99048572859289	51.0144432050631\\
2.99248873309965	50.1832533316668\\
2.99449173760641	49.3576784446628\\
2.99649474211317	48.5382342060667\\
2.99849774661993	47.7253216863351\\
3.00050075112669	46.9194565474836\\
3.00250375563345	46.1209252684098\\
3.00450676014021	45.3301862153497\\
3.00650976464697	44.547525867201\\
3.00851276915373	43.7732879986407\\
3.01051577366049	43.007816384346\\
3.01251877816725	42.2513402074347\\
3.01452178267401	41.5041459468046\\
3.01652478718077	40.7665200813532\\
3.01852779168753	40.038634498419\\
3.02053079619429	39.3207183811201\\
3.02253380070105	38.6130009125745\\
3.02453680520781	37.9156539801208\\
3.02653980971457	37.2288494710974\\
3.02854281422133	36.5527592728431\\
3.03054581872809	35.8876125684757\\
3.03254882323485	35.2334666537748\\
3.03455182774161	34.5905507118585\\
3.03655483224837	33.9589793342858\\
3.03855783675513	33.3389244083953\\
3.04056084126189	32.7304432299663\\
3.04256384576865	32.1337649821171\\
3.04456685027541	31.5489469606271\\
3.04656985478217	30.9762183486143\\
3.04857285928893	30.4156937376378\\
3.05057586379569	29.8674877192566\\
3.05257886830245	29.3317721808093\\
3.05458187280921	28.8086617138549\\
3.05658487731597	28.2983855015114\\
3.05858788182273	27.8010581353378\\
3.06059088632949	27.3168515026727\\
3.06259389083625	26.8459947866342\\
3.06459689534301	26.3886025787813\\
3.06659989984977	25.944904062232\\
3.06860290435653	25.5151284201044\\
3.07060590886329	25.0994475397369\\
3.07260891337005	24.6981479000273\\
3.07461191787682	24.311401388314\\
3.07661492238358	23.9394371877151\\
3.07861792689034	23.5825417771281\\
3.0806209313971	23.2410016354506\\
3.08262393590386	22.9149886500212\\
3.08462694041062	22.6048465955169\\
3.08662994491738	22.3108046550557\\
3.08863294942414	22.0330920117558\\
3.0906359539309	21.7721097360737\\
3.09263895843766	21.528029715348\\
3.09464196294442	21.3011384284762\\
3.09664496745118	21.0916650585764\\
3.09864797195794	20.8999533803256\\
3.1006509764647	20.7261179852829\\
3.10265398097146	20.5705026481254\\
3.10465698547822	20.4332792561915\\
3.10665998998498	20.3146196968199\\
3.10866299449174	20.2147531531286\\
3.1106659989985	20.1337369208971\\
3.11266900350526	20.071742887464\\
3.11467200801202	20.0288856443882\\
3.11667501251878	20.0051651916698\\
3.11867801702554	20.0006961208678\\
3.1206810215323	20.0154211362026\\
3.12268402603906	20.0493975334539\\
3.12468703054582	20.102453425283\\
3.12669003505258	20.1746461074695\\
3.12869303955934	20.2657463968953\\
3.1306960440661	20.3756969977809\\
3.13269904857286	20.5042687270082\\
3.13470205307962	20.6513469930183\\
3.13670505758638	20.8167026126931\\
3.13870806209314	21.0001064029144\\
3.1407110665999	21.2013291805644\\
3.14271407110666	21.4201417625249\\
3.14471707561342	21.6562576698983\\
3.14672008012018	21.9094477195666\\
3.14872308462694	22.1793681368527\\
3.1507260891337	22.4658470344181\\
3.15272909364046	22.7685406375857\\
3.15473209814722	23.087162467458\\
3.15673510265398	23.4215406366963\\
3.15873810716074	23.7713313706237\\
3.1607411116675	24.1362481903425\\
3.16274411617426	24.5161192085143\\
3.16474712068102	24.9107152420209\\
3.16675012518778	25.3197498119648\\
3.16875312969454	25.7429364394484\\
3.1707561342013	26.1801605329127\\
3.17275913870806	26.6311929092397\\
3.17476214321482	27.0958043853113\\
3.17676514772158	27.5738230737889\\
3.17876815222834	28.0650770873341\\
3.1807711567351	28.5693372428287\\
3.18277416124186	29.0864889487138\\
3.18477716574862	29.6163603176508\\
3.18678017025538	30.1587794623011\\
3.18878317476214	30.7135744953263\\
3.1907861792689	31.2806308251673\\
3.19278918377566	31.859833860265\\
3.19479218828242	32.451011713281\\
3.19679519278918	33.0539924968767\\
3.19879819729594	33.6686616194931\\
3.2008012018027	34.294904489571\\
3.20280420630946	34.9325492197721\\
3.20480721081622	35.5814812185373\\
3.20681021532298	36.241528598528\\
3.20881321982974	36.9125194724057\\
3.2108162243365	37.5943392486114\\
3.21281922884326	38.2868160398065\\
3.21482223335002	38.989720662873\\
3.21682523785678	39.7028812304723\\
3.21882824236355	40.4261831510455\\
3.22083124687031	41.1592826499154\\
3.22283425137707	41.902065135523\\
3.22483725588383	42.6542441289707\\
3.22684026039059	43.4155904471406\\
3.22884326489735	44.1857603153554\\
3.23084626940411	44.9645245504972\\
3.23284927391087	45.7515393778889\\
3.23485227841763	46.5465183186329\\
3.23685528292439	47.3490030064932\\
3.23885828743115	48.1587069625721\\
3.24086129193791	48.9751718206335\\
3.24286429644467	49.7979392144413\\
3.24486730095143	50.6265507777595\\
3.24687030545819	51.460548144352\\
3.24887330996495	52.2993583564235\\
3.25087631447171	53.142408456179\\
3.25287931897847	53.9891254858233\\
3.25488232348523	54.8388791917818\\
3.25688532799199	55.6909820247004\\
3.25888833249875	56.5447464352248\\
3.26089133700551	57.399370282442\\
3.26289434151227	58.2541660169976\\
3.26489734601903	59.1081596106402\\
3.26690035052579	59.9606062182358\\
3.26890335503255	60.8105318115328\\
3.27090635953931	61.6570769538386\\
3.27290936404607	62.4990957295629\\
3.27491236855283	63.3357287020129\\
3.27691537305959	64.1657726598189\\
3.27891837756635	64.9881389831702\\
3.28092138207311	65.801739052256\\
3.28292438657987	66.6053696557065\\
3.28492739108663	67.3977702863724\\
3.28693039559339	68.177737732884\\
3.28893340010015	68.9438968965329\\
3.29093640460691	69.695101861729\\
3.29293940911367	70.4298629382047\\
3.29494241362043	71.1468623230314\\
3.29694541812719	71.8448395090598\\
3.29894842263395	72.5223048060225\\
3.30095142714071	73.1778831152112\\
3.30295443164747	73.8103139294766\\
3.30495743615423	74.4180502627718\\
3.30696044066099	74.9998889037272\\
3.30896344516775	75.5543974578548\\
3.31096644967451	76.0802581222259\\
3.31296945418127	76.5763249812501\\
3.31497245868803	77.0412229362193\\
3.31697546319479	77.4738633673226\\
3.31897846770155	77.8730430631902\\
3.32098147220831	78.2377879955705\\
3.32298447671507	78.5671241362117\\
3.32498748122183	78.8600774568621\\
3.32699048572859	79.115903112388\\
3.32899349023535	79.3338562576558\\
3.33099649474211	79.5133066390907\\
3.33299949924887	79.6537385946773\\
3.33500250375563	79.7548083497384\\
3.33700550826239	79.8161148338174\\
3.33900851276915	79.8376007511348\\
3.34101151727591	79.8190942143521\\
3.34301452178267	79.7607098150282\\
3.34501752628943	79.6626194405018\\
3.34702053079619	79.5251095696704\\
3.34902353530295	79.3485239772111\\
3.35102653980971	79.133435620919\\
3.35302954431647	78.8804174585892\\
3.35503254882323	78.5902143353555\\
3.35703555332999	78.2635710963514\\
3.35903855783675	77.9014044740492\\
3.36104156234352	77.5046884967006\\
3.36304456685028	77.0745117841164\\
3.36504757135704	76.6119629561073\\
3.3670505758638	76.1182452240431\\
3.36905358037056	75.594504503514\\
3.37105658487732	75.0420585974488\\
3.37305958938408	74.4622826045559\\
3.37506259389084	73.8563797362051\\
3.3770655983976	73.2257823868841\\
3.37906860290436	72.5718656553013\\
3.38107160741112	71.8959473443855\\
3.38307461191788	71.199517144404\\
3.38507761642464	70.483835562506\\
3.3870806209314	69.7503349931796\\
3.38908362543816	69.0003332393533\\
3.39108662994492	68.2352053997356\\
3.39308963445168	67.4562692772553\\
3.39509263895844	66.664728083282\\
3.3970956434652	65.8618996207447\\
3.39909864797196	65.0489871010131\\
3.40110165247872	64.227193735457\\
3.40310465698548	63.3977227354461\\
3.40510766149224	62.5615481292321\\
3.407110665999	61.719873128185\\
3.40911367050576	60.8736717605563\\
3.41111667501252	60.0239180545977\\
3.41311967951928	59.1716433343406\\
3.41512268402604	58.3177070364777\\
3.4171256885328	57.4629685977015\\
3.41912869303956	56.6082874547049\\
3.42113169754632	55.7544657484009\\
3.42313470205308	54.9022483239233\\
3.42513770655984	54.0522654348467\\
3.4271407110666	53.2053192220843\\
3.42914371557336	52.3619253476518\\
3.43114672008012	51.5227713609032\\
3.43314972458688	50.6884302196337\\
3.43515272909364	49.8593602900794\\
3.4371557336004	49.0361345300354\\
3.43915873810716	48.2192113057379\\
3.44116174261392	47.4089916876434\\
3.44316474712068	46.605934041988\\
3.44516775162744	45.8104394392284\\
3.4471707561342	45.022794358262\\
3.44917376064096	44.2433998695456\\
3.45117676514772	43.4725424519766\\
3.45317976965448	42.7105658802321\\
3.45518277416124	41.9576993374302\\
3.457185778668	41.2142293024684\\
3.45918878317476	40.4803849584649\\
3.46119178768152	39.7563954885376\\
3.46319479218828	39.042432780025\\
3.46519779669504	38.3387260160454\\
3.4672008012018	37.6455043797166\\
3.46920380570856	36.9628824625977\\
3.47120681021532	36.2910894478068\\
3.47320981472208	35.6302399269029\\
3.47521281922884	34.9804484914451\\
3.4772158237356	34.3419443245513\\
3.47921882824236	33.7148420177806\\
3.48122183274912	33.0993134584716\\
3.48322483725588	32.4954732381832\\
3.48522784176264	31.903378652695\\
3.4872308462694	31.3233161809045\\
3.48923385077616	30.7553431185914\\
3.49123685528292	30.199631353094\\
3.49323985978968	29.6562954759714\\
3.49524286429644	29.1254500787827\\
3.4972458688032	28.6073816404254\\
3.49924887330996	28.1021474566791\\
3.50125187781672	27.6099194148822\\
3.50325488232349	27.1309266981528\\
3.50525788683025	26.66528389805\\
3.50726089133701	26.2132774934713\\
3.50926389584377	25.7750220759757\\
3.51126690035053	25.3507468286813\\
3.51326990485729	24.9406236389267\\
3.51527290936405	24.5449962813889\\
3.51727591387081	24.1640366434064\\
3.51927891837757	23.7979166123178\\
3.52128192288433	23.4470372585797\\
3.52328492739109	23.111513173751\\
3.52528793189785	22.791688132509\\
3.52729093640461	22.4877913179716\\
3.52929394091137	22.2001092090364\\
3.53129694541813	21.928928284601\\
3.53329994992489	21.6745350235629\\
3.53530295443165	21.4371013132607\\
3.53730595893841	21.217028224151\\
3.53930896344517	21.0144876435722\\
3.54131196795193	20.8297660504221\\
3.54331497245869	20.663035332039\\
3.54531797696545	20.5145819673206\\
3.54732098147221	20.3846351393849\\
3.54932398597897	20.273309439791\\
3.55132699048573	20.1807767558774\\
3.55332999499249	20.1072089749826\\
3.55533299949925	20.0526633928861\\
3.55733600400601	20.0173118969265\\
3.55933900851277	20.0010971913243\\
3.56134201301953	20.004191163418\\
3.56334501752629	20.0264219258691\\
3.56534802203305	20.0679040702366\\
3.56735102653981	20.1284657091819\\
3.56935403104657	20.2080495469256\\
3.57135703555333	20.3065409919086\\
3.57336004006009	20.4238254525719\\
3.57536304456685	20.5596737457974\\
3.57736604907361	20.7139139842466\\
3.57936905358037	20.886374280581\\
3.58137205808713	21.076768155903\\
3.58337506259389	21.284923722874\\
3.58537806710065	21.5105545025965\\
3.58738107160741	21.7533740161729\\
3.58938407611417	22.0130957847057\\
3.59138708062093	22.2895479208564\\
3.59339008512769	22.5823866499477\\
3.59539308963445	22.8913827888618\\
3.59739609414121	23.2161925629214\\
3.59939909864797	23.5566440847882\\
3.60140210315473	23.9123935797849\\
3.60340510766149	24.2832118647936\\
3.60540811216825	24.6688697566961\\
3.60741111667501	25.0691380723745\\
3.60941412118177	25.4837303329312\\
3.61141712568853	25.9124746510276\\
3.61342013019529	26.3550845477661\\
3.61542313470205	26.8115027273674\\
3.61742613920881	27.2813854151542\\
3.61942914371557	27.764560723788\\
3.62143214822233	28.2609140617098\\
3.62343515272909	28.7702735415811\\
3.62543815723585	29.2924099802838\\
3.62744116174261	29.8272087862589\\
3.62944416624937	30.3744980721679\\
3.63144717075613	30.9341059506722\\
3.63345017526289	31.5059178302127\\
3.63545317976965	32.0898191192306\\
3.63745618427641	32.6856379303871\\
3.63945918878317	33.2932023763438\\
3.64146219328993	33.9124551613212\\
3.64346519779669	34.5431671022012\\
3.64546820230346	35.1852809032043\\
3.64747120681022	35.8385673812125\\
3.64947421131698	36.5029119446667\\
3.65147721582374	37.1781427062284\\
3.6534802203305	37.8641450743385\\
3.65548322483726	38.5607471616586\\
3.65748622934402	39.2677197850705\\
3.65948923385078	39.9848337614562\\
3.66149223835754	40.7120317950363\\
3.6634952428643	41.448912815354\\
3.66549824737106	42.19541952663\\
3.66750125187782	42.951208154187\\
3.66950425638458	43.7160495149072\\
3.67150726089134	44.4895998341133\\
3.6735102653981	45.2716299286874\\
3.67551326990486	46.0617960239523\\
3.67751627441162	46.859754345231\\
3.67951927891838	47.6651611178464\\
3.68152228342514	48.4775006797828\\
3.6835252879319	49.2964865521429\\
3.68552829243866	50.1216603686903\\
3.68753129694542	50.9524491716299\\
3.68953430145218	51.7883372989463\\
3.69153730595894	52.6289236801827\\
3.6935403104657	53.4734634702056\\
3.69554331497246	54.3214983027787\\
3.69754631947922	55.172226036989\\
3.69954932398598	56.0250737150412\\
3.70155232849274	56.8792391960222\\
3.7035553329995	57.7340349305779\\
3.70555833750626	58.5886014820155\\
3.70756134201302	59.4420794136424\\
3.70956434651978	60.2936665845453\\
3.71156735102654	61.1423889664726\\
3.7135703555333	61.9872725311725\\
3.71557336004006	62.8273432503933\\
3.71757636454682	63.6614552085448\\
3.71957936905358	64.4886916731547\\
3.72158237356034	65.3077348412942\\
3.7235853780671	66.1176106847116\\
3.72558838257386	66.9169441046986\\
3.72759138708062	67.7046464814445\\
3.72959439158738	68.4793427162409\\
3.73159739609414	69.239829597718\\
3.7336004006009	69.9847320271676\\
3.73560340510766	70.7127322016608\\
3.73760640961442	71.4224550224894\\
3.73960941412118	72.1125253909449\\
3.74161241862794	72.7816255040987\\
3.7436154231347	73.4283229674629\\
3.74561842764146	74.0512426823291\\
3.74762143214822	74.6490095499891\\
3.74962443665498	75.2202484717345\\
3.75162744116174	75.7637562361956\\
3.7536304456685	76.2781004488846\\
3.75563345017526	76.7620778984315\\
3.75763645468202	77.2144853734669\\
3.75963945918878	77.6341769584002\\
3.76164246369554	78.0200067376413\\
3.7636454682023	78.3710006829384\\
3.76564847270906	78.6861847660399\\
3.76765147721582	78.964699550253\\
3.76965448172258	79.205800190444\\
3.77165748622934	79.4087418414794\\
3.7736604907361	79.5729515455639\\
3.77566349524286	79.6980282322409\\
3.77766649974962	79.783570831054\\
3.77966950425638	79.8292928631054\\
3.78167250876314	79.8351370326157\\
3.7836755132699	79.8009887480259\\
3.78567851777666	79.727077192454\\
3.78768152228343	79.6134596616796\\
3.78968452679019	79.4605945219387\\
3.79168753129695	79.2688828436879\\
3.79369053580371	79.0388975847224\\
3.79569354031047	78.7712117028373\\
3.79769654481723	78.4666273389458\\
3.79969954932399	78.125946633961\\
3.80170255383075	77.7501436161347\\
3.80370555833751	77.3402496094981\\
3.80570856284427	76.8972386423029\\
3.80771156735103	76.4223712216985\\
3.80971457185779	75.9167359674955\\
3.81171757636455	75.3815933868433\\
3.81372058087131	74.8182612826707\\
3.81572358537807	74.228114753686\\
3.81772658988483	73.6123570112589\\
3.81972959439159	72.9724777456568\\
3.82173259889835	72.3097947598085\\
3.82373560340511	71.6256831524223\\
3.82573860791187	70.9215180222065\\
3.82774161241863	70.198731763649\\
3.82974461692539	69.4586421796785\\
3.83174762143215	68.7025670732238\\
3.83375062593891	67.9318242472139\\
3.83575363044567	67.1477888003568\\
3.83775663495243	66.3517212398021\\
3.83975963945919	65.5448247769193\\
3.84176264396595	64.7283026230784\\
3.84376564847271	63.903357989649\\
3.84576865297947	63.071079496442\\
3.84777165748623	62.2326703548271\\
3.84977466199299	61.3891045930559\\
3.85177766649975	60.5414135351599\\
3.85378067100651	59.6905712093906\\
3.85578367551327	58.8374943482203\\
3.85778668002003	57.9831569799008\\
3.85978968452679	57.1283612453451\\
3.86179268903355	56.2739092854665\\
3.86379569354031	55.4206032411782\\
3.86579869804707	54.5691879576138\\
3.86780170255383	53.7202936883479\\
3.86980470706059	52.8746652785143\\
3.87180771156735	52.0328756859081\\
3.87381071607411	51.195555164104\\
3.87581372058087	50.3632193751174\\
3.87781672508763	49.5363839809641\\
3.87981972959439	48.7155646436597\\
3.88182273410115	47.9012197294403\\
3.88382573860791	47.0937503087624\\
3.88582874311467	46.2936147478622\\
3.88783174762143	45.5011568214168\\
3.88983475212819	44.7166630083236\\
3.89183775663495	43.9405916748189\\
3.89384076114171	43.1732292998002\\
3.89584376564847	42.414747770606\\
3.89784677015523	41.665548157693\\
3.89984977466199	40.9258023483996\\
3.90185277916875	40.1957968216234\\
3.90385578367551	39.475703464703\\
3.90585878818227	38.7657514607564\\
3.90786179268903	38.0661126971221\\
3.90986479719579	37.3770163569183\\
3.91186780170255	36.6986343274834\\
3.91387080620931	36.0310812003765\\
3.91587381071607	35.3745288629361\\
3.91787681522283	34.7291492025007\\
3.91987981972959	34.0951141064089\\
3.92188282423635	33.4725381662198\\
3.92388582874311	32.8615359734923\\
3.92588883324987	32.2622794155649\\
3.92789183775663	31.6748830839968\\
3.9298948422634	31.0995188661264\\
3.93189784677016	30.5363013535128\\
3.93390085127692	29.9854024334946\\
3.93590385578368	29.4469939934101\\
3.93790686029044	28.9211333290391\\
3.9399098647972	28.4080496234994\\
3.94191286930396	27.9079147641297\\
3.94391587381072	27.420843342489\\
3.94591887831748	26.9470645416953\\
3.94792188282424	26.4866929533077\\
3.949924887331	26.0400723520032\\
3.95192789183776	25.6072600335614\\
3.95393089634452	25.1884851811003\\
3.95593390085128	24.7840342735174\\
3.95793690535804	24.3940791981514\\
3.9599399098648	24.0189064338997\\
3.96194291437156	23.6587451638805\\
3.96394591887832	23.3138245712117\\
3.96594892338508	22.984431134791\\
3.96795192789184	22.6707940377364\\
3.9699549323986	22.373257054725\\
3.97195793690536	22.0919920730953\\
3.97396094141212	21.8273428675243\\
3.97596394591888	21.5795386211303\\
3.97796695042564	21.3489231085901\\
3.9799699549324	21.1356682172424\\
3.98197295943916	20.9400604259847\\
3.98397596394592	20.7623289179352\\
3.98597896845268	20.6027601719912\\
3.98798197295944	20.4615260754915\\
3.9899849774662	20.3387985157744\\
3.99198798197296	20.2348066759582\\
3.99399098647972	20.1497224433813\\
3.99599399098648	20.0835458180437\\
3.99799699549324	20.0365632788429\\
4	20.0086602342201\\
};
\addlegendentry{$\theta$};

\end{axis}
\end{tikzpicture}%
	\caption{Showing the angle evolution for the spinning top problem when going backwards in time.}
	\label{fig:backwardData}
\end{figure}
\fi
\iftikz
\begin{figure}[H]
	\centering
	\setlength\figureheight{7cm} 
	\setlength\figurewidth{14cm}
	% This file was created by matlab2tikz.
% Minimal pgfplots version: 1.3
%
%The latest updates can be retrieved from
%  http://www.mathworks.com/matlabcentral/fileexchange/22022-matlab2tikz
%where you can also make suggestions and rate matlab2tikz.
%
\definecolor{mycolor1}{rgb}{0.00000,0.44700,0.74100}%
\definecolor{mycolor2}{rgb}{0.85000,0.32500,0.09800}%
\definecolor{mycolor3}{rgb}{0.92900,0.69400,0.12500}%
%
\begin{tikzpicture}

\begin{axis}[%
width=0.95092\figurewidth,
height=\figureheight,
at={(0\figurewidth,0\figureheight)},
scale only axis,
xmin=0,
xmax=4,
xlabel={Time (s)},
ymin=-6e-07,
ymax=3e-07,
ylabel={Degrees},
title style={font=\bfseries},
title={Top Spin [0,4] (s) Errors},
legend style={at={(0.03,0.97)},anchor=north west,legend cell align=left,align=left,draw=white!15!black},
title style={font=\small},ticklabel style={font=\tiny}
]
\addplot [color=mycolor1,solid]
  table[row sep=crcr]{%
0	0\\
0.00200100050025012	4.16881799906007e-10\\
0.00400200100050025	6.25320408027829e-10\\
0.00600300150075038	4.16881799906007e-10\\
0.0080040020010005	2.08440899953003e-10\\
0.0100050025012506	6.25320408027829e-10\\
0.0120060030015008	4.16881799906007e-10\\
0.0140070035017509	1.04220449976502e-09\\
0.016008004002001	1.66752490779285e-09\\
0.0180090045022511	1.4590885915022e-09\\
0.0200100050025013	1.4590885915022e-09\\
0.0220110055027514	8.33762453896423e-10\\
0.0240120060030015	8.33762453896423e-10\\
0.0260130065032516	6.25320408027829e-10\\
0.0280140070035018	2.08440899953003e-10\\
0.0300150075037519	0\\
0.032016008004002	-2.08440899953003e-10\\
0.0340170085042521	0\\
0.0360180090045022	-2.08440899953003e-10\\
0.0380190095047524	0\\
0.0400200100050025	-2.08440899953003e-10\\
0.0420210105052526	-2.08440899953003e-10\\
0.0440220110055028	-2.08440899953003e-10\\
0.0460230115057529	0\\
0.048024012006003	-2.08440899953003e-10\\
0.0500250125062531	-2.08440899953003e-10\\
0.0520260130065033	-2.08440899953003e-10\\
0.0540270135067534	-4.16881799906007e-10\\
0.0560280140070035	-2.08440899953003e-10\\
0.0580290145072536	-2.08440899953003e-10\\
0.0600300150075038	-4.16881799906007e-10\\
0.0620310155077539	-4.16881799906007e-10\\
0.064032016008004	-8.33762453896423e-10\\
0.0660330165082541	-8.33762453896423e-10\\
0.0680340170085043	-4.16881799906007e-10\\
0.0700350175087544	-2.08440899953003e-10\\
0.0720360180090045	0\\
0.0740370185092546	-2.08440899953003e-10\\
0.0760380190095048	0\\
0.0780390195097549	4.16881799906007e-10\\
0.080040020010005	-2.08440899953003e-10\\
0.0820410205102551	-6.25320408027829e-10\\
0.0840420210105053	-4.16881799906007e-10\\
0.0860430215107554	-8.33762453896423e-10\\
0.0880440220110055	-1.04220449976502e-09\\
0.0900450225112556	-1.25064654563361e-09\\
0.0920460230115058	-1.04220449976502e-09\\
0.0940470235117559	-1.04220449976502e-09\\
0.096048024012006	-8.33762453896423e-10\\
0.0980490245122561	-6.25320408027829e-10\\
0.100050025012506	-6.25320408027829e-10\\
0.102051025512756	-1.04220449976502e-09\\
0.104052026013007	-1.4590885915022e-09\\
0.106053026513257	-1.25064654563361e-09\\
0.108054027013507	-1.25064654563361e-09\\
0.110055027513757	-1.25064654563361e-09\\
0.112056028014007	-1.04220449976502e-09\\
0.114057028514257	-1.25064654563361e-09\\
0.116058029014507	-8.33762453896423e-10\\
0.118059029514757	-1.25064654563361e-09\\
0.120060030015008	-1.4590885915022e-09\\
0.122061030515258	-1.25064654563361e-09\\
0.124062031015508	-1.66752490779285e-09\\
0.126063031515758	-1.87596695366144e-09\\
0.128064032016008	-1.66752490779285e-09\\
0.130065032516258	-2.08440899953003e-09\\
0.132066033016508	-2.08440899953003e-09\\
0.134067033516758	-2.29285104539863e-09\\
0.136068034017009	-2.70972940755786e-09\\
0.138069034517259	-2.50128736168927e-09\\
0.140070035017509	-2.29285104539863e-09\\
0.142071035517759	-2.70972940755786e-09\\
0.144072036018009	-2.08440899953003e-09\\
0.146073036518259	-2.70972940755786e-09\\
0.148074037018509	-2.29285104539863e-09\\
0.150075037518759	-2.70972940755786e-09\\
0.15207603801901	-2.70972940755786e-09\\
0.15407703851926	-2.08440899953003e-09\\
0.15607803901951	-2.29285104539863e-09\\
0.15807903951976	-2.70972940755786e-09\\
0.16008004002001	-2.50128736168927e-09\\
0.16208104052026	-2.29285104539863e-09\\
0.16408204102051	-2.50128736168927e-09\\
0.16608304152076	-2.50128736168927e-09\\
0.168084042021011	-2.50128736168927e-09\\
0.170085042521261	-1.66752490779285e-09\\
0.172086043021511	-2.08440899953003e-09\\
0.174087043521761	-2.08440899953003e-09\\
0.176088044022011	-2.08440899953003e-09\\
0.178089044522261	-2.29285104539863e-09\\
0.180090045022511	-2.70972940755786e-09\\
0.182091045522761	-2.91817145342646e-09\\
0.184092046023012	-2.70972940755786e-09\\
0.186093046523262	-2.29285104539863e-09\\
0.188094047023512	-3.12661349929505e-09\\
0.190095047523762	-2.91817145342646e-09\\
0.192096048024012	-2.91817145342646e-09\\
0.194097048524262	-3.12661349929505e-09\\
0.196098049024512	-2.70972940755786e-09\\
0.198099049524762	-2.91817145342646e-09\\
0.200100050025012	-3.33505554516364e-09\\
0.202101050525263	-2.91817145342646e-09\\
0.204102051025513	-2.91817145342646e-09\\
0.206103051525763	-2.91817145342646e-09\\
0.208104052026013	-2.70972940755786e-09\\
0.210105052526263	-2.29285104539863e-09\\
0.212106053026513	-2.70972940755786e-09\\
0.214107053526763	-2.70972940755786e-09\\
0.216108054027013	-2.91817145342646e-09\\
0.218109054527264	-3.12661349929505e-09\\
0.220110055027514	-2.50128736168927e-09\\
0.222111055527764	-2.70972940755786e-09\\
0.224112056028014	-2.29285104539863e-09\\
0.226113056528264	-2.70972940755786e-09\\
0.228114057028514	-3.12661349929505e-09\\
0.230115057528764	-2.91817145342646e-09\\
0.232116058029014	-2.91817145342646e-09\\
0.234117058529265	-3.12661349929505e-09\\
0.236118059029515	-3.12661349929505e-09\\
0.238119059529765	-3.12661349929505e-09\\
0.240120060030015	-3.33505554516364e-09\\
0.242121060530265	-2.91817145342646e-09\\
0.244122061030515	-3.54349186145428e-09\\
0.246123061530765	-3.12661349929505e-09\\
0.248124062031016	-3.12661349929505e-09\\
0.250125062531266	-3.12661349929505e-09\\
0.252126063031516	-2.91817145342646e-09\\
0.254127063531766	-3.33505554516364e-09\\
0.256128064032016	-4.16881799906006e-09\\
0.258129064532266	-3.75193390732288e-09\\
0.260130065032516	-3.96037595319147e-09\\
0.262131065532766	-3.54349186145428e-09\\
0.264132066033017	-3.75193390732288e-09\\
0.266133066533267	-3.33505554516364e-09\\
0.268134067033517	-3.33505554516364e-09\\
0.270135067533767	-3.12661349929505e-09\\
0.272136068034017	-3.33505554516364e-09\\
0.274137068534267	-3.12661349929505e-09\\
0.276138069034517	-2.91817145342646e-09\\
0.278139069534767	-3.12661349929505e-09\\
0.280140070035018	-3.12661349929505e-09\\
0.282141070535268	-3.33505554516364e-09\\
0.284142071035518	-3.75193390732288e-09\\
0.286143071535768	-3.54349186145428e-09\\
0.288144072036018	-3.54349186145428e-09\\
0.290145072536268	-3.75193390732288e-09\\
0.292146073036518	-3.12661349929505e-09\\
0.294147073536768	-3.33505554516364e-09\\
0.296148074037018	-3.33505554516364e-09\\
0.298149074537269	-3.33505554516364e-09\\
0.300150075037519	-3.54349186145428e-09\\
0.302151075537769	-3.33505554516364e-09\\
0.304152076038019	-3.54349186145428e-09\\
0.306153076538269	-3.54349186145428e-09\\
0.308154077038519	-3.75193390732288e-09\\
0.310155077538769	-3.96037595319147e-09\\
0.31215607803902	-3.75193390732288e-09\\
0.31415707853927	-3.96037595319147e-09\\
0.31615807903952	-3.96037595319147e-09\\
0.31815907953977	-3.54349186145428e-09\\
0.32016008004002	-3.75193390732288e-09\\
0.32216108054027	-3.75193390732288e-09\\
0.32416208104052	-3.75193390732288e-09\\
0.32616308154077	-4.16881799906006e-09\\
0.32816408204102	-3.75193390732288e-09\\
0.330165082541271	-3.75193390732288e-09\\
0.332166083041521	-4.16881799906006e-09\\
0.334167083541771	-3.96037595319147e-09\\
0.336168084042021	-4.16881799906006e-09\\
0.338169084542271	-4.37726004492866e-09\\
0.340170085042521	-3.96037595319147e-09\\
0.342171085542771	-3.54349186145428e-09\\
0.344172086043022	-3.96037595319147e-09\\
0.346173086543272	-3.96037595319147e-09\\
0.348174087043522	-4.16881799906006e-09\\
0.350175087543772	-3.75193390732288e-09\\
0.352176088044022	-3.96037595319147e-09\\
0.354177088544272	-3.75193390732288e-09\\
0.356178089044522	-3.75193390732288e-09\\
0.358179089544772	-3.96037595319147e-09\\
0.360180090045022	-3.96037595319147e-09\\
0.362181090545273	-3.75193390732288e-09\\
0.364182091045523	-3.12661349929505e-09\\
0.366183091545773	-2.91817145342646e-09\\
0.368184092046023	-3.33505554516364e-09\\
0.370185092546273	-3.33505554516364e-09\\
0.372186093046523	-3.96037595319147e-09\\
0.374187093546773	-3.54349186145428e-09\\
0.376188094047024	-3.75193390732288e-09\\
0.378189094547274	-3.54349186145428e-09\\
0.380190095047524	-3.54349186145428e-09\\
0.382191095547774	-3.12661349929505e-09\\
0.384192096048024	-2.70972940755786e-09\\
0.386193096548274	-2.29285104539863e-09\\
0.388194097048524	-1.87596695366144e-09\\
0.390195097548774	-1.87596695366144e-09\\
0.392196098049024	-1.66752490779285e-09\\
0.394197098549275	-1.4590885915022e-09\\
0.396198099049525	-1.66752490779285e-09\\
0.398199099549775	-1.66752490779285e-09\\
0.400200100050025	-2.08440899953003e-09\\
0.402201100550275	-1.87596695366144e-09\\
0.404202101050525	-2.08440899953003e-09\\
0.406203101550775	-1.66752490779285e-09\\
0.408204102051026	-1.66752490779285e-09\\
0.410205102551276	-1.87596695366144e-09\\
0.412206103051526	-2.08440899953003e-09\\
0.414207103551776	-1.66752490779285e-09\\
0.416208104052026	-1.87596695366144e-09\\
0.418209104552276	-1.87596695366144e-09\\
0.420210105052526	-1.66752490779285e-09\\
0.422211105552776	-1.66752490779285e-09\\
0.424212106053027	-1.04220449976502e-09\\
0.426213106553277	-1.04220449976502e-09\\
0.428214107053527	-6.25320408027829e-10\\
0.430215107553777	-8.33762453896423e-10\\
0.432216108054027	-8.33762453896423e-10\\
0.434217108554277	-8.33762453896423e-10\\
0.436218109054527	-8.33762453896423e-10\\
0.438219109554777	-4.16881799906007e-10\\
0.440220110055028	-2.08440899953003e-10\\
0.442221110555278	-4.16881799906007e-10\\
0.444222111055528	-6.25320408027829e-10\\
0.446223111555778	-6.25320408027829e-10\\
0.448224112056028	-1.25064654563361e-09\\
0.450225112556278	-6.25320408027829e-10\\
0.452226113056528	-6.25320408027829e-10\\
0.454227113556778	-8.33762453896423e-10\\
0.456228114057029	-2.08440899953003e-10\\
0.458229114557279	0\\
0.460230115057529	-2.08440899953003e-10\\
0.462231115557779	-6.25320408027829e-10\\
0.464232116058029	-1.25064654563361e-09\\
0.466233116558279	-1.87596695366144e-09\\
0.468234117058529	-1.66752490779285e-09\\
0.470235117558779	-1.87596695366144e-09\\
0.47223611805903	-1.87596695366144e-09\\
0.47423711855928	-1.87596695366144e-09\\
0.47623811905953	-1.66752490779285e-09\\
0.47823911955978	-2.08440899953003e-09\\
0.48024012006003	-2.29285104539863e-09\\
0.48224112056028	-2.08440899953003e-09\\
0.48424212106053	-1.87596695366144e-09\\
0.48624312156078	-1.66752490779285e-09\\
0.488244122061031	-2.50128736168927e-09\\
0.490245122561281	-2.29285104539863e-09\\
0.492246123061531	-2.29285104539863e-09\\
0.494247123561781	-2.50128736168927e-09\\
0.496248124062031	-2.70972940755786e-09\\
0.498249124562281	-2.08440899953003e-09\\
0.500250125062531	-2.08440899953003e-09\\
0.502251125562781	-2.50128736168927e-09\\
0.504252126063031	-2.29285104539863e-09\\
0.506253126563282	-1.87596695366144e-09\\
0.508254127063532	-1.87596695366144e-09\\
0.510255127563782	-1.87596695366144e-09\\
0.512256128064032	-1.87596695366144e-09\\
0.514257128564282	-1.04220449976502e-09\\
0.516258129064532	-1.4590885915022e-09\\
0.518259129564782	-1.04220449976502e-09\\
0.520260130065032	-8.33762453896423e-10\\
0.522261130565283	-8.33762453896423e-10\\
0.524262131065533	-8.33762453896423e-10\\
0.526263131565783	-6.25320408027829e-10\\
0.528264132066033	-4.16881799906007e-10\\
0.530265132566283	-2.08440899953003e-10\\
0.532266133066533	-4.16881799906007e-10\\
0.534267133566783	-2.08440899953003e-10\\
0.536268134067034	-4.16881799906007e-10\\
0.538269134567284	-4.16881799906007e-10\\
0.540270135067534	-4.16881799906007e-10\\
0.542271135567784	-4.16881799906007e-10\\
0.544272136068034	-8.33762453896423e-10\\
0.546273136568284	-4.16881799906007e-10\\
0.548274137068534	-2.08440899953003e-10\\
0.550275137568784	-4.16881799906007e-10\\
0.552276138069035	-4.16881799906007e-10\\
0.554277138569285	2.08440899953003e-10\\
0.556278139069535	2.08440899953003e-10\\
0.558279139569785	6.25320408027829e-10\\
0.560280140070035	8.33762453896423e-10\\
0.562281140570285	4.16881799906007e-10\\
0.564282141070535	2.08440899953003e-10\\
0.566283141570785	2.08440899953003e-10\\
0.568284142071036	0\\
0.570285142571286	-2.08440899953003e-10\\
0.572286143071536	-2.08440899953003e-10\\
0.574287143571786	0\\
0.576288144072036	0\\
0.578289144572286	0\\
0.580290145072536	-4.16881799906007e-10\\
0.582291145572786	2.08440899953003e-10\\
0.584292146073036	-2.08440899953003e-10\\
0.586293146573287	-2.08440899953003e-10\\
0.588294147073537	2.08440899953003e-10\\
0.590295147573787	2.08440899953003e-10\\
0.592296148074037	6.25320408027829e-10\\
0.594297148574287	6.25320408027829e-10\\
0.596298149074537	2.08440899953003e-10\\
0.598299149574787	6.25320408027829e-10\\
0.600300150075038	4.16881799906007e-10\\
0.602301150575288	6.25320408027829e-10\\
0.604302151075538	2.08440899953003e-10\\
0.606303151575788	0\\
0.608304152076038	2.08440899953003e-10\\
0.610305152576288	6.25320408027829e-10\\
0.612306153076538	8.33762453896423e-10\\
0.614307153576788	1.25064654563361e-09\\
0.616308154077039	1.25064654563361e-09\\
0.618309154577289	1.4590885915022e-09\\
0.620310155077539	1.04220449976502e-09\\
0.622311155577789	4.16881799906007e-10\\
0.624312156078039	6.25320408027829e-10\\
0.626313156578289	8.33762453896423e-10\\
0.628314157078539	6.25320408027829e-10\\
0.630315157578789	6.25320408027829e-10\\
0.63231615807904	4.16881799906007e-10\\
0.63431715857929	0\\
0.63631815907954	2.08440899953003e-10\\
0.63831915957979	0\\
0.64032016008004	-8.33762453896423e-10\\
0.64232116058029	-4.16881799906007e-10\\
0.64432216108054	-8.33762453896423e-10\\
0.64632316158079	-6.25320408027829e-10\\
0.64832416208104	-8.33762453896423e-10\\
0.650325162581291	-4.16881799906007e-10\\
0.652326163081541	-6.25320408027829e-10\\
0.654327163581791	-8.33762453896423e-10\\
0.656328164082041	-4.16881799906007e-10\\
0.658329164582291	0\\
0.660330165082541	-2.08440899953003e-10\\
0.662331165582791	-4.16881799906007e-10\\
0.664332166083042	-4.16881799906007e-10\\
0.666333166583292	-4.16881799906007e-10\\
0.668334167083542	-4.16881799906007e-10\\
0.670335167583792	-4.16881799906007e-10\\
0.672336168084042	-2.08440899953003e-10\\
0.674337168584292	-6.25320408027829e-10\\
0.676338169084542	-6.25320408027829e-10\\
0.678339169584792	-6.25320408027829e-10\\
0.680340170085043	-4.16881799906007e-10\\
0.682341170585293	-2.08440899953003e-10\\
0.684342171085543	-4.16881799906007e-10\\
0.686343171585793	-4.16881799906007e-10\\
0.688344172086043	-2.08440899953003e-10\\
0.690345172586293	0\\
0.692346173086543	-2.08440899953003e-10\\
0.694347173586793	-2.08440899953003e-10\\
0.696348174087044	0\\
0.698349174587294	2.08440899953003e-10\\
0.700350175087544	2.08440899953003e-10\\
0.702351175587794	0\\
0.704352176088044	-2.08440899953003e-10\\
0.706353176588294	0\\
0.708354177088544	-2.08440899953003e-10\\
0.710355177588794	-2.08440899953003e-10\\
0.712356178089045	2.08440899953003e-10\\
0.714357178589295	0\\
0.716358179089545	-2.08440899953003e-10\\
0.718359179589795	-2.08440899953003e-10\\
0.720360180090045	-2.08440899953003e-10\\
0.722361180590295	-2.08440899953003e-10\\
0.724362181090545	2.08440899953003e-10\\
0.726363181590795	-4.16881799906007e-10\\
0.728364182091045	-6.25320408027829e-10\\
0.730365182591296	-4.16881799906007e-10\\
0.732366183091546	-8.33762453896423e-10\\
0.734367183591796	-4.16881799906007e-10\\
0.736368184092046	-4.16881799906007e-10\\
0.738369184592296	-2.08440899953003e-10\\
0.740370185092546	0\\
0.742371185592796	-4.16881799906007e-10\\
0.744372186093047	-4.16881799906007e-10\\
0.746373186593297	-6.25320408027829e-10\\
0.748374187093547	-2.08440899953003e-10\\
0.750375187593797	2.08440899953003e-10\\
0.752376188094047	-2.08440899953003e-10\\
0.754377188594297	-2.08440899953003e-10\\
0.756378189094547	-4.16881799906007e-10\\
0.758379189594797	-4.16881799906007e-10\\
0.760380190095048	-8.33762453896423e-10\\
0.762381190595298	-4.16881799906007e-10\\
0.764382191095548	-6.25320408027829e-10\\
0.766383191595798	-2.08440899953003e-10\\
0.768384192096048	-4.16881799906007e-10\\
0.770385192596298	0\\
0.772386193096548	-4.16881799906007e-10\\
0.774387193596798	-4.16881799906007e-10\\
0.776388194097049	-8.33762453896423e-10\\
0.778389194597299	-1.04220449976502e-09\\
0.780390195097549	-1.25064654563361e-09\\
0.782391195597799	-1.25064654563361e-09\\
0.784392196098049	-6.25320408027829e-10\\
0.786393196598299	-1.04220449976502e-09\\
0.788394197098549	-4.16881799906007e-10\\
0.790395197598799	-6.25320408027829e-10\\
0.792396198099049	-2.08440899953003e-10\\
0.7943971985993	-8.33762453896423e-10\\
0.79639819909955	-6.25320408027829e-10\\
0.7983991995998	-4.16881799906007e-10\\
0.80040020010005	-4.16881799906007e-10\\
0.8024012006003	-6.25320408027829e-10\\
0.80440220110055	-8.33762453896423e-10\\
0.8064032016008	-1.04220449976502e-09\\
0.808404202101051	-8.33762453896423e-10\\
0.810405202601301	-1.04220449976502e-09\\
0.812406203101551	-8.33762453896423e-10\\
0.814407203601801	-6.25320408027829e-10\\
0.816408204102051	-6.25320408027829e-10\\
0.818409204602301	-4.16881799906007e-10\\
0.820410205102551	-6.25320408027829e-10\\
0.822411205602801	-8.33762453896423e-10\\
0.824412206103052	-6.25320408027829e-10\\
0.826413206603302	-4.16881799906007e-10\\
0.828414207103552	-6.25320408027829e-10\\
0.830415207603802	-6.25320408027829e-10\\
0.832416208104052	-8.33762453896423e-10\\
0.834417208604302	-1.04220449976502e-09\\
0.836418209104552	-8.33762453896423e-10\\
0.838419209604802	-1.04220449976502e-09\\
0.840420210105053	-8.33762453896423e-10\\
0.842421210605303	-6.25320408027829e-10\\
0.844422211105553	-6.25320408027829e-10\\
0.846423211605803	-6.25320408027829e-10\\
0.848424212106053	-2.08440899953003e-10\\
0.850425212606303	0\\
0.852426213106553	2.08440899953003e-10\\
0.854427213606803	4.16881799906007e-10\\
0.856428214107053	8.33762453896423e-10\\
0.858429214607304	8.33762453896423e-10\\
0.860430215107554	1.25064654563361e-09\\
0.862431215607804	1.25064654563361e-09\\
0.864432216108054	1.4590885915022e-09\\
0.866433216608304	1.4590885915022e-09\\
0.868434217108554	1.04220449976502e-09\\
0.870435217608804	1.4590885915022e-09\\
0.872436218109054	1.25064654563361e-09\\
0.874437218609305	1.66752490779285e-09\\
0.876438219109555	1.87596695366144e-09\\
0.878439219609805	1.66752490779285e-09\\
0.880440220110055	1.04220449976502e-09\\
0.882441220610305	1.4590885915022e-09\\
0.884442221110555	1.25064654563361e-09\\
0.886443221610805	4.16881799906007e-10\\
0.888444222111056	6.25320408027829e-10\\
0.890445222611306	4.16881799906007e-10\\
0.892446223111556	2.08440899953003e-10\\
0.894447223611806	0\\
0.896448224112056	4.16881799906007e-10\\
0.898449224612306	4.16881799906007e-10\\
0.900450225112556	2.08440899953003e-10\\
0.902451225612806	4.16881799906007e-10\\
0.904452226113057	2.08440899953003e-10\\
0.906453226613307	2.08440899953003e-10\\
0.908454227113557	0\\
0.910455227613807	-2.08440899953003e-10\\
0.912456228114057	-6.25320408027829e-10\\
0.914457228614307	-6.25320408027829e-10\\
0.916458229114557	-6.25320408027829e-10\\
0.918459229614807	-6.25320408027829e-10\\
0.920460230115058	-4.16881799906007e-10\\
0.922461230615308	-2.08440899953003e-10\\
0.924462231115558	-4.16881799906007e-10\\
0.926463231615808	2.08440899953003e-10\\
0.928464232116058	0\\
0.930465232616308	0\\
0.932466233116558	2.08440899953003e-10\\
0.934467233616808	4.16881799906007e-10\\
0.936468234117058	4.16881799906007e-10\\
0.938469234617309	4.16881799906007e-10\\
0.940470235117559	2.08440899953003e-10\\
0.942471235617809	0\\
0.944472236118059	4.16881799906007e-10\\
0.946473236618309	8.33762453896423e-10\\
0.948474237118559	8.33762453896423e-10\\
0.950475237618809	8.33762453896423e-10\\
0.95247623811906	1.04220449976502e-09\\
0.95447723861931	8.33762453896423e-10\\
0.95647823911956	8.33762453896423e-10\\
0.95847923961981	1.04220449976502e-09\\
0.96048024012006	4.16881799906007e-10\\
0.96248124062031	2.08440899953003e-10\\
0.96448224112056	0\\
0.96648324162081	-2.08440899953003e-10\\
0.968484242121061	-4.16881799906007e-10\\
0.970485242621311	-4.16881799906007e-10\\
0.972486243121561	-4.16881799906007e-10\\
0.974487243621811	-6.25320408027829e-10\\
0.976488244122061	-2.08440899953003e-10\\
0.978489244622311	-2.08440899953003e-10\\
0.980490245122561	4.16881799906007e-10\\
0.982491245622811	6.25320408027829e-10\\
0.984492246123062	2.08440899953003e-10\\
0.986493246623312	2.08440899953003e-10\\
0.988494247123562	4.16881799906007e-10\\
0.990495247623812	6.25320408027829e-10\\
0.992496248124062	4.16881799906007e-10\\
0.994497248624312	2.08440899953003e-10\\
0.996498249124562	2.08440899953003e-10\\
0.998499249624812	4.16881799906007e-10\\
1.00050025012506	8.33762453896423e-10\\
1.00250125062531	8.33762453896423e-10\\
1.00450225112556	8.33762453896423e-10\\
1.00650325162581	8.33762453896423e-10\\
1.00850425212606	1.04220449976502e-09\\
1.01050525262631	8.33762453896423e-10\\
1.01250625312656	1.25064654563361e-09\\
1.01450725362681	1.4590885915022e-09\\
1.01650825412706	1.25064654563361e-09\\
1.01850925462731	1.4590885915022e-09\\
1.02051025512756	1.25064654563361e-09\\
1.02251125562781	1.66752490779285e-09\\
1.02451225612806	2.08440899953003e-09\\
1.02651325662831	1.25064654563361e-09\\
1.02851425712856	1.25064654563361e-09\\
1.03051525762881	8.33762453896423e-10\\
1.03251625812906	1.25064654563361e-09\\
1.03451725862931	1.04220449976502e-09\\
1.03651825912956	1.04220449976502e-09\\
1.03851925962981	1.04220449976502e-09\\
1.04052026013006	1.4590885915022e-09\\
1.04252126063032	1.4590885915022e-09\\
1.04452226113057	1.66752490779285e-09\\
1.04652326163082	1.4590885915022e-09\\
1.04852426213107	1.87596695366144e-09\\
1.05052526263132	1.25064654563361e-09\\
1.05252626313157	1.87596695366144e-09\\
1.05452726363182	1.4590885915022e-09\\
1.05652826413207	1.66752490779285e-09\\
1.05852926463232	1.4590885915022e-09\\
1.06053026513257	1.66752490779285e-09\\
1.06253126563282	2.29285104539863e-09\\
1.06453226613307	2.50128736168927e-09\\
1.06653326663332	2.91817145342646e-09\\
1.06853426713357	2.91817145342646e-09\\
1.07053526763382	2.70972940755786e-09\\
1.07253626813407	3.12661349929505e-09\\
1.07453726863432	3.12661349929505e-09\\
1.07653826913457	3.75193390732288e-09\\
1.07853926963482	3.33505554516364e-09\\
1.08054027013507	3.12661349929505e-09\\
1.08254127063532	3.33505554516364e-09\\
1.08454227113557	2.91817145342646e-09\\
1.08654327163582	2.50128736168927e-09\\
1.08854427213607	2.50128736168927e-09\\
1.09054527263632	2.50128736168927e-09\\
1.09254627313657	1.87596695366144e-09\\
1.09454727363682	2.08440899953003e-09\\
1.09654827413707	2.50128736168927e-09\\
1.09854927463732	2.70972940755786e-09\\
1.10055027513757	2.29285104539863e-09\\
1.10255127563782	2.29285104539863e-09\\
1.10455227613807	2.08440899953003e-09\\
1.10655327663832	2.08440899953003e-09\\
1.10855427713857	2.08440899953003e-09\\
1.11055527763882	1.87596695366144e-09\\
1.11255627813907	1.87596695366144e-09\\
1.11455727863932	1.87596695366144e-09\\
1.11655827913957	1.87596695366144e-09\\
1.11855927963982	1.87596695366144e-09\\
1.12056028014007	1.66752490779285e-09\\
1.12256128064032	2.08440899953003e-09\\
1.12456228114057	1.4590885915022e-09\\
1.12656328164082	1.25064654563361e-09\\
1.12856428214107	1.87596695366144e-09\\
1.13056528264132	1.4590885915022e-09\\
1.13256628314157	1.66752490779285e-09\\
1.13456728364182	1.4590885915022e-09\\
1.13656828414207	1.25064654563361e-09\\
1.13856928464232	1.87596695366144e-09\\
1.14057028514257	1.66752490779285e-09\\
1.14257128564282	1.66752490779285e-09\\
1.14457228614307	1.4590885915022e-09\\
1.14657328664332	1.25064654563361e-09\\
1.14857428714357	1.04220449976502e-09\\
1.15057528764382	1.04220449976502e-09\\
1.15257628814407	1.25064654563361e-09\\
1.15457728864432	1.25064654563361e-09\\
1.15657828914457	1.4590885915022e-09\\
1.15857928964482	1.66752490779285e-09\\
1.16058029014507	1.87596695366144e-09\\
1.16258129064532	2.50128736168927e-09\\
1.16458229114557	2.91817145342646e-09\\
1.16658329164582	2.50128736168927e-09\\
1.16858429214607	2.29285104539863e-09\\
1.17058529264632	2.08440899953003e-09\\
1.17258629314657	2.08440899953003e-09\\
1.17458729364682	2.50128736168927e-09\\
1.17658829414707	2.91817145342646e-09\\
1.17858929464732	3.12661349929505e-09\\
1.18059029514757	3.54349186145428e-09\\
1.18259129564782	3.54349186145428e-09\\
1.18459229614807	3.33505554516364e-09\\
1.18659329664832	3.54349186145428e-09\\
1.18859429714857	3.33505554516364e-09\\
1.19059529764882	3.54349186145428e-09\\
1.19259629814907	3.33505554516364e-09\\
1.19459729864932	3.12661349929505e-09\\
1.19659829914957	2.70972940755786e-09\\
1.19859929964982	3.12661349929505e-09\\
1.20060030015008	2.50128736168927e-09\\
1.20260130065033	2.91817145342646e-09\\
1.20460230115058	2.50128736168927e-09\\
1.20660330165083	2.70972940755786e-09\\
1.20860430215108	2.91817145342646e-09\\
1.21060530265133	2.70972940755786e-09\\
1.21260630315158	2.08440899953003e-09\\
1.21460730365183	2.50128736168927e-09\\
1.21660830415208	2.50128736168927e-09\\
1.21860930465233	2.29285104539863e-09\\
1.22061030515258	2.08440899953003e-09\\
1.22261130565283	2.50128736168927e-09\\
1.22461230615308	2.50128736168927e-09\\
1.22661330665333	2.50128736168927e-09\\
1.22861430715358	2.70972940755786e-09\\
1.23061530765383	3.12661349929505e-09\\
1.23261630815408	2.70972940755786e-09\\
1.23461730865433	2.50128736168927e-09\\
1.23661830915458	3.12661349929505e-09\\
1.23861930965483	3.33505554516364e-09\\
1.24062031015508	3.33505554516364e-09\\
1.24262131065533	3.33505554516364e-09\\
1.24462231115558	3.33505554516364e-09\\
1.24662331165583	3.75193390732288e-09\\
1.24862431215608	3.54349186145428e-09\\
1.25062531265633	3.54349186145428e-09\\
1.25262631315658	3.75193390732288e-09\\
1.25462731365683	3.96037595319147e-09\\
1.25662831415708	3.96037595319147e-09\\
1.25862931465733	4.16881799906006e-09\\
1.26063031515758	4.37726004492866e-09\\
1.26263131565783	3.96037595319147e-09\\
1.26463231615808	3.75193390732288e-09\\
1.26663331665833	3.96037595319147e-09\\
1.26863431715858	4.37726004492866e-09\\
1.27063531765883	3.96037595319147e-09\\
1.27263631815908	4.37726004492866e-09\\
1.27463731865933	4.37726004492866e-09\\
1.27663831915958	4.79413840708789e-09\\
1.27863931965983	5.00258045295649e-09\\
1.28064032016008	4.5856963612193e-09\\
1.28264132066033	4.79413840708789e-09\\
1.28464232116058	4.5856963612193e-09\\
1.28664332166083	4.5856963612193e-09\\
1.28864432216108	4.37726004492866e-09\\
1.29064532266133	4.16881799906006e-09\\
1.29264632316158	4.5856963612193e-09\\
1.29464732366183	4.5856963612193e-09\\
1.29664832416208	4.79413840708789e-09\\
1.29864932466233	4.37726004492866e-09\\
1.30065032516258	4.16881799906006e-09\\
1.30265132566283	4.37726004492866e-09\\
1.30465232616308	4.5856963612193e-09\\
1.30665332666333	4.16881799906006e-09\\
1.30865432716358	4.5856963612193e-09\\
1.31065532766383	4.37726004492866e-09\\
1.31265632816408	4.5856963612193e-09\\
1.31465732866433	4.5856963612193e-09\\
1.31665832916458	4.5856963612193e-09\\
1.31865932966483	4.37726004492866e-09\\
1.32066033016508	4.79413840708789e-09\\
1.32266133066533	4.5856963612193e-09\\
1.32466233116558	4.16881799906006e-09\\
1.32666333166583	4.37726004492866e-09\\
1.32866433216608	4.37726004492866e-09\\
1.33066533266633	3.96037595319147e-09\\
1.33266633316658	3.75193390732288e-09\\
1.33466733366683	3.75193390732288e-09\\
1.33666833416708	3.54349186145428e-09\\
1.33866933466733	3.96037595319147e-09\\
1.34067033516758	3.75193390732288e-09\\
1.34267133566783	3.96037595319147e-09\\
1.34467233616808	4.37726004492866e-09\\
1.34667333666833	4.16881799906006e-09\\
1.34867433716858	3.96037595319147e-09\\
1.35067533766883	3.96037595319147e-09\\
1.35267633816908	4.16881799906006e-09\\
1.35467733866933	3.96037595319147e-09\\
1.35667833916958	3.75193390732288e-09\\
1.35867933966983	3.54349186145428e-09\\
1.36068034017009	3.33505554516364e-09\\
1.36268134067034	3.54349186145428e-09\\
1.36468234117059	3.54349186145428e-09\\
1.36668334167084	3.12661349929505e-09\\
1.36868434217109	2.91817145342646e-09\\
1.37068534267134	3.33505554516364e-09\\
1.37268634317159	3.75193390732288e-09\\
1.37468734367184	3.54349186145428e-09\\
1.37668834417209	3.96037595319147e-09\\
1.37868934467234	4.16881799906006e-09\\
1.38069034517259	3.96037595319147e-09\\
1.38269134567284	4.16881799906006e-09\\
1.38469234617309	4.16881799906006e-09\\
1.38669334667334	3.54349186145428e-09\\
1.38869434717359	3.75193390732288e-09\\
1.39069534767384	4.16881799906006e-09\\
1.39269634817409	4.37726004492866e-09\\
1.39469734867434	4.5856963612193e-09\\
1.39669834917459	4.16881799906006e-09\\
1.39869934967484	4.16881799906006e-09\\
1.40070035017509	3.96037595319147e-09\\
1.40270135067534	3.96037595319147e-09\\
1.40470235117559	3.96037595319147e-09\\
1.40670335167584	4.16881799906006e-09\\
1.40870435217609	3.96037595319147e-09\\
1.41070535267634	3.54349186145428e-09\\
1.41270635317659	3.33505554516364e-09\\
1.41470735367684	3.33505554516364e-09\\
1.41670835417709	3.54349186145428e-09\\
1.41870935467734	3.33505554516364e-09\\
1.42071035517759	3.33505554516364e-09\\
1.42271135567784	3.75193390732288e-09\\
1.42471235617809	3.33505554516364e-09\\
1.42671335667834	3.54349186145428e-09\\
1.42871435717859	2.91817145342646e-09\\
1.43071535767884	2.91817145342646e-09\\
1.43271635817909	2.91817145342646e-09\\
1.43471735867934	2.70972940755786e-09\\
1.43671835917959	2.29285104539863e-09\\
1.43871935967984	2.50128736168927e-09\\
1.44072036018009	2.29285104539863e-09\\
1.44272136068034	2.50128736168927e-09\\
1.44472236118059	2.70972940755786e-09\\
1.44672336168084	3.12661349929505e-09\\
1.44872436218109	2.91817145342646e-09\\
1.45072536268134	2.91817145342646e-09\\
1.45272636318159	3.12661349929505e-09\\
1.45472736368184	3.12661349929505e-09\\
1.45672836418209	2.70972940755786e-09\\
1.45872936468234	2.91817145342646e-09\\
1.46073036518259	2.50128736168927e-09\\
1.46273136568284	2.08440899953003e-09\\
1.46473236618309	2.08440899953003e-09\\
1.46673336668334	1.87596695366144e-09\\
1.46873436718359	1.87596695366144e-09\\
1.47073536768384	1.4590885915022e-09\\
1.47273636818409	1.87596695366144e-09\\
1.47473736868434	1.87596695366144e-09\\
1.47673836918459	1.66752490779285e-09\\
1.47873936968484	1.4590885915022e-09\\
1.48074037018509	1.04220449976502e-09\\
1.48274137068534	8.33762453896423e-10\\
1.48474237118559	1.04220449976502e-09\\
1.48674337168584	1.04220449976502e-09\\
1.48874437218609	6.25320408027829e-10\\
1.49074537268634	6.25320408027829e-10\\
1.49274637318659	6.25320408027829e-10\\
1.49474737368684	8.33762453896423e-10\\
1.49674837418709	1.04220449976502e-09\\
1.49874937468734	1.04220449976502e-09\\
1.50075037518759	8.33762453896423e-10\\
1.50275137568784	1.04220449976502e-09\\
1.50475237618809	8.33762453896423e-10\\
1.50675337668834	8.33762453896423e-10\\
1.50875437718859	6.25320408027829e-10\\
1.51075537768884	6.25320408027829e-10\\
1.51275637818909	4.16881799906007e-10\\
1.51475737868934	-2.08440899953003e-10\\
1.51675837918959	2.08440899953003e-10\\
1.51875937968984	2.08440899953003e-10\\
1.5207603801901	4.16881799906007e-10\\
1.52276138069035	4.16881799906007e-10\\
1.5247623811906	2.08440899953003e-10\\
1.52676338169085	6.25320408027829e-10\\
1.5287643821911	4.16881799906007e-10\\
1.53076538269135	2.08440899953003e-10\\
1.5327663831916	6.25320408027829e-10\\
1.53476738369185	2.08440899953003e-10\\
1.5367683841921	4.16881799906007e-10\\
1.53876938469235	2.08440899953003e-10\\
1.5407703851926	-2.08440899953003e-10\\
1.54277138569285	-2.08440899953003e-10\\
1.5447723861931	-4.16881799906007e-10\\
1.54677338669335	-8.33762453896423e-10\\
1.5487743871936	-8.33762453896423e-10\\
1.55077538769385	-1.25064654563361e-09\\
1.5527763881941	-1.4590885915022e-09\\
1.55477738869435	-1.4590885915022e-09\\
1.5567783891946	-1.25064654563361e-09\\
1.55877938969485	-1.66752490779285e-09\\
1.5607803901951	-1.87596695366144e-09\\
1.56278139069535	-2.08440899953003e-09\\
1.5647823911956	-2.08440899953003e-09\\
1.56678339169585	-2.29285104539863e-09\\
1.5687843921961	-2.29285104539863e-09\\
1.57078539269635	-2.08440899953003e-09\\
1.5727863931966	-2.50128736168927e-09\\
1.57478739369685	-2.70972940755786e-09\\
1.5767883941971	-2.91817145342646e-09\\
1.57878939469735	-3.12661349929505e-09\\
1.5807903951976	-3.12661349929505e-09\\
1.58279139569785	-2.70972940755786e-09\\
1.5847923961981	-3.12661349929505e-09\\
1.58679339669835	-3.12661349929505e-09\\
1.5887943971986	-2.91817145342646e-09\\
1.59079539769885	-2.91817145342646e-09\\
1.5927963981991	-2.91817145342646e-09\\
1.59479739869935	-3.12661349929505e-09\\
1.5967983991996	-2.70972940755786e-09\\
1.59879939969985	-2.29285104539863e-09\\
1.6008004002001	-2.08440899953003e-09\\
1.60280140070035	-1.87596695366144e-09\\
1.6048024012006	-1.25064654563361e-09\\
1.60680340170085	-1.4590885915022e-09\\
1.6088044022011	-1.4590885915022e-09\\
1.61080540270135	-1.25064654563361e-09\\
1.6128064032016	-1.4590885915022e-09\\
1.61480740370185	-1.4590885915022e-09\\
1.6168084042021	-8.33762453896423e-10\\
1.61880940470235	-8.33762453896423e-10\\
1.6208104052026	-6.25320408027829e-10\\
1.62281140570285	-2.08440899953003e-10\\
1.6248124062031	-4.16881799906007e-10\\
1.62681340670335	-4.16881799906007e-10\\
1.6288144072036	-6.25320408027829e-10\\
1.63081540770385	-8.33762453896423e-10\\
1.6328164082041	-6.25320408027829e-10\\
1.63481740870435	-6.25320408027829e-10\\
1.6368184092046	-8.33762453896423e-10\\
1.63881940970485	-8.33762453896423e-10\\
1.6408204102051	-1.04220449976502e-09\\
1.64282141070535	-8.33762453896423e-10\\
1.6448224112056	-8.33762453896423e-10\\
1.64682341170585	-2.08440899953003e-10\\
1.6488244122061	0\\
1.65082541270635	0\\
1.6528264132066	-2.08440899953003e-10\\
1.65482741370685	-4.16881799906007e-10\\
1.6568284142071	-2.08440899953003e-10\\
1.65882941470735	0\\
1.6608304152076	0\\
1.66283141570785	2.08440899953003e-10\\
1.6648324162081	4.16881799906007e-10\\
1.66683341670835	4.16881799906007e-10\\
1.6688344172086	2.08440899953003e-10\\
1.67083541770885	2.08440899953003e-10\\
1.6728364182091	2.08440899953003e-10\\
1.67483741870935	4.16881799906007e-10\\
1.6768384192096	2.08440899953003e-10\\
1.67883941970985	4.16881799906007e-10\\
1.68084042021011	4.16881799906007e-10\\
1.68284142071036	4.16881799906007e-10\\
1.68484242121061	-2.08440899953003e-10\\
1.68684342171086	0\\
1.68884442221111	0\\
1.69084542271136	0\\
1.69284642321161	0\\
1.69484742371186	4.16881799906007e-10\\
1.69684842421211	6.25320408027829e-10\\
1.69884942471236	1.04220449976502e-09\\
1.70085042521261	1.25064654563361e-09\\
1.70285142571286	1.4590885915022e-09\\
1.70485242621311	1.4590885915022e-09\\
1.70685342671336	1.66752490779285e-09\\
1.70885442721361	1.87596695366144e-09\\
1.71085542771386	1.66752490779285e-09\\
1.71285642821411	1.87596695366144e-09\\
1.71485742871436	2.50128736168927e-09\\
1.71685842921461	2.29285104539863e-09\\
1.71885942971486	1.87596695366144e-09\\
1.72086043021511	1.87596695366144e-09\\
1.72286143071536	2.29285104539863e-09\\
1.72486243121561	1.66752490779285e-09\\
1.72686343171586	2.29285104539863e-09\\
1.72886443221611	2.08440899953003e-09\\
1.73086543271636	1.87596695366144e-09\\
1.73286643321661	1.87596695366144e-09\\
1.73486743371686	1.4590885915022e-09\\
1.73686843421711	1.66752490779285e-09\\
1.73886943471736	1.4590885915022e-09\\
1.74087043521761	1.25064654563361e-09\\
1.74287143571786	1.04220449976502e-09\\
1.74487243621811	1.4590885915022e-09\\
1.74687343671836	1.25064654563361e-09\\
1.74887443721861	1.04220449976502e-09\\
1.75087543771886	1.4590885915022e-09\\
1.75287643821911	1.25064654563361e-09\\
1.75487743871936	8.33762453896423e-10\\
1.75687843921961	4.16881799906007e-10\\
1.75887943971986	6.25320408027829e-10\\
1.76088044022011	6.25320408027829e-10\\
1.76288144072036	6.25320408027829e-10\\
1.76488244122061	6.25320408027829e-10\\
1.76688344172086	8.33762453896423e-10\\
1.76888444222111	6.25320408027829e-10\\
1.77088544272136	1.04220449976502e-09\\
1.77288644322161	1.04220449976502e-09\\
1.77488744372186	8.33762453896423e-10\\
1.77688844422211	8.33762453896423e-10\\
1.77888944472236	8.33762453896423e-10\\
1.78089044522261	8.33762453896423e-10\\
1.78289144572286	4.16881799906007e-10\\
1.78489244622311	4.16881799906007e-10\\
1.78689344672336	4.16881799906007e-10\\
1.78889444722361	6.25320408027829e-10\\
1.79089544772386	2.08440899953003e-10\\
1.79289644822411	4.16881799906007e-10\\
1.79489744872436	4.16881799906007e-10\\
1.79689844922461	6.25320408027829e-10\\
1.79889944972486	2.08440899953003e-10\\
1.80090045022511	2.08440899953003e-10\\
1.80290145072536	6.25320408027829e-10\\
1.80490245122561	6.25320408027829e-10\\
1.80690345172586	6.25320408027829e-10\\
1.80890445222611	6.25320408027829e-10\\
1.81090545272636	4.16881799906007e-10\\
1.81290645322661	2.08440899953003e-10\\
1.81490745372686	2.08440899953003e-10\\
1.81690845422711	2.08440899953003e-10\\
1.81890945472736	2.08440899953003e-10\\
1.82091045522761	4.16881799906007e-10\\
1.82291145572786	6.25320408027829e-10\\
1.82491245622811	4.16881799906007e-10\\
1.82691345672836	4.16881799906007e-10\\
1.82891445722861	4.16881799906007e-10\\
1.83091545772886	4.16881799906007e-10\\
1.83291645822911	8.33762453896423e-10\\
1.83491745872936	6.25320408027829e-10\\
1.83691845922961	8.33762453896423e-10\\
1.83891945972987	8.33762453896423e-10\\
1.84092046023012	6.25320408027829e-10\\
1.84292146073037	6.25320408027829e-10\\
1.84492246123062	1.04220449976502e-09\\
1.84692346173087	8.33762453896423e-10\\
1.84892446223112	6.25320408027829e-10\\
1.85092546273137	8.33762453896423e-10\\
1.85292646323162	8.33762453896423e-10\\
1.85492746373187	1.04220449976502e-09\\
1.85692846423212	4.16881799906007e-10\\
1.85892946473237	6.25320408027829e-10\\
1.86093046523262	6.25320408027829e-10\\
1.86293146573287	6.25320408027829e-10\\
1.86493246623312	6.25320408027829e-10\\
1.86693346673337	4.16881799906007e-10\\
1.86893446723362	8.33762453896423e-10\\
1.87093546773387	8.33762453896423e-10\\
1.87293646823412	4.16881799906007e-10\\
1.87493746873437	4.16881799906007e-10\\
1.87693846923462	6.25320408027829e-10\\
1.87893946973487	8.33762453896423e-10\\
1.88094047023512	8.33762453896423e-10\\
1.88294147073537	1.04220449976502e-09\\
1.88494247123562	1.25064654563361e-09\\
1.88694347173587	1.04220449976502e-09\\
1.88894447223612	6.25320408027829e-10\\
1.89094547273637	4.16881799906007e-10\\
1.89294647323662	2.08440899953003e-10\\
1.89494747373687	6.25320408027829e-10\\
1.89694847423712	6.25320408027829e-10\\
1.89894947473737	6.25320408027829e-10\\
1.90095047523762	8.33762453896423e-10\\
1.90295147573787	1.4590885915022e-09\\
1.90495247623812	1.04220449976502e-09\\
1.90695347673837	8.33762453896423e-10\\
1.90895447723862	8.33762453896423e-10\\
1.91095547773887	8.33762453896423e-10\\
1.91295647823912	6.25320408027829e-10\\
1.91495747873937	2.08440899953003e-10\\
1.91695847923962	0\\
1.91895947973987	2.08440899953003e-10\\
1.92096048024012	0\\
1.92296148074037	-2.08440899953003e-10\\
1.92496248124062	-2.08440899953003e-10\\
1.92696348174087	0\\
1.92896448224112	-2.08440899953003e-10\\
1.93096548274137	0\\
1.93296648324162	0\\
1.93496748374187	2.08440899953003e-10\\
1.93696848424212	4.16881799906007e-10\\
1.93896948474237	4.16881799906007e-10\\
1.94097048524262	0\\
1.94297148574287	-2.08440899953003e-10\\
1.94497248624312	-2.08440899953003e-10\\
1.94697348674337	-2.08440899953003e-10\\
1.94897448724362	2.08440899953003e-10\\
1.95097548774387	2.08440899953003e-10\\
1.95297648824412	4.16881799906007e-10\\
1.95497748874437	4.16881799906007e-10\\
1.95697848924462	4.16881799906007e-10\\
1.95897948974487	4.16881799906007e-10\\
1.96098049024512	4.16881799906007e-10\\
1.96298149074537	4.16881799906007e-10\\
1.96498249124562	4.16881799906007e-10\\
1.96698349174587	2.08440899953003e-10\\
1.96898449224612	0\\
1.97098549274637	-2.08440899953003e-10\\
1.97298649324662	-2.08440899953003e-10\\
1.97498749374687	0\\
1.97698849424712	-4.16881799906007e-10\\
1.97898949474737	-4.16881799906007e-10\\
1.98099049524762	-2.08440899953003e-10\\
1.98299149574787	-6.25320408027829e-10\\
1.98499249624812	0\\
1.98699349674837	2.08440899953003e-10\\
1.98899449724862	0\\
1.99099549774887	-4.16881799906007e-10\\
1.99299649824912	-4.16881799906007e-10\\
1.99499749874937	-4.16881799906007e-10\\
1.99699849924962	-4.16881799906007e-10\\
1.99899949974988	-2.08440899953003e-10\\
2.00100050025013	4.16881799906007e-10\\
2.00300150075038	4.16881799906007e-10\\
2.00500250125063	6.25320408027829e-10\\
2.00700350175088	4.16881799906007e-10\\
2.00900450225113	6.25320408027829e-10\\
2.01100550275138	6.25320408027829e-10\\
2.01300650325163	6.25320408027829e-10\\
2.01500750375188	6.25320408027829e-10\\
2.01700850425213	4.16881799906007e-10\\
2.01900950475238	2.08440899953003e-10\\
2.02101050525263	4.16881799906007e-10\\
2.02301150575288	4.16881799906007e-10\\
2.02501250625313	4.16881799906007e-10\\
2.02701350675338	4.16881799906007e-10\\
2.02901450725363	6.25320408027829e-10\\
2.03101550775388	4.16881799906007e-10\\
2.03301650825413	8.33762453896423e-10\\
2.03501750875438	6.25320408027829e-10\\
2.03701850925463	4.16881799906007e-10\\
2.03901950975488	2.08440899953003e-10\\
2.04102051025513	-4.16881799906007e-10\\
2.04302151075538	-2.08440899953003e-10\\
2.04502251125563	0\\
2.04702351175588	0\\
2.04902451225613	0\\
2.05102551275638	4.16881799906007e-10\\
2.05302651325663	6.25320408027829e-10\\
2.05502751375688	6.25320408027829e-10\\
2.05702851425713	2.08440899953003e-10\\
2.05902951475738	4.16881799906007e-10\\
2.06103051525763	6.25320408027829e-10\\
2.06303151575788	6.25320408027829e-10\\
2.06503251625813	2.08440899953003e-10\\
2.06703351675838	4.16881799906007e-10\\
2.06903451725863	8.33762453896423e-10\\
2.07103551775888	8.33762453896423e-10\\
2.07303651825913	8.33762453896423e-10\\
2.07503751875938	6.25320408027829e-10\\
2.07703851925963	8.33762453896423e-10\\
2.07903951975988	6.25320408027829e-10\\
2.08104052026013	6.25320408027829e-10\\
2.08304152076038	4.16881799906007e-10\\
2.08504252126063	4.16881799906007e-10\\
2.08704352176088	0\\
2.08904452226113	4.16881799906007e-10\\
2.09104552276138	6.25320408027829e-10\\
2.09304652326163	6.25320408027829e-10\\
2.09504752376188	1.04220449976502e-09\\
2.09704852426213	1.04220449976502e-09\\
2.09904952476238	1.04220449976502e-09\\
2.10105052526263	1.25064654563361e-09\\
2.10305152576288	1.25064654563361e-09\\
2.10505252626313	1.25064654563361e-09\\
2.10705352676338	1.25064654563361e-09\\
2.10905452726363	1.04220449976502e-09\\
2.11105552776388	1.25064654563361e-09\\
2.11305652826413	8.33762453896423e-10\\
2.11505752876438	8.33762453896423e-10\\
2.11705852926463	6.25320408027829e-10\\
2.11905952976488	4.16881799906007e-10\\
2.12106053026513	6.25320408027829e-10\\
2.12306153076538	2.08440899953003e-10\\
2.12506253126563	0\\
2.12706353176588	6.25320408027829e-10\\
2.12906453226613	6.25320408027829e-10\\
2.13106553276638	6.25320408027829e-10\\
2.13306653326663	1.04220449976502e-09\\
2.13506753376688	1.25064654563361e-09\\
2.13706853426713	1.66752490779285e-09\\
2.13906953476738	1.66752490779285e-09\\
2.14107053526763	1.4590885915022e-09\\
2.14307153576788	1.25064654563361e-09\\
2.14507253626813	8.33762453896423e-10\\
2.14707353676838	4.16881799906007e-10\\
2.14907453726863	4.16881799906007e-10\\
2.15107553776888	1.04220449976502e-09\\
2.15307653826913	1.04220449976502e-09\\
2.15507753876938	1.04220449976502e-09\\
2.15707853926963	1.25064654563361e-09\\
2.15907953976988	8.33762453896423e-10\\
2.16108054027013	6.25320408027829e-10\\
2.16308154077039	4.16881799906007e-10\\
2.16508254127064	4.16881799906007e-10\\
2.16708354177089	4.16881799906007e-10\\
2.16908454227114	2.08440899953003e-10\\
2.17108554277139	2.08440899953003e-10\\
2.17308654327164	0\\
2.17508754377189	2.08440899953003e-10\\
2.17708854427214	0\\
2.17908954477239	-2.08440899953003e-10\\
2.18109054527264	-2.08440899953003e-10\\
2.18309154577289	0\\
2.18509254627314	0\\
2.18709354677339	0\\
2.18909454727364	-2.08440899953003e-10\\
2.19109554777389	-4.16881799906007e-10\\
2.19309654827414	4.16881799906007e-10\\
2.19509754877439	6.25320408027829e-10\\
2.19709854927464	8.33762453896423e-10\\
2.19909954977489	6.25320408027829e-10\\
2.20110055027514	8.33762453896423e-10\\
2.20310155077539	1.4590885915022e-09\\
2.20510255127564	1.4590885915022e-09\\
2.20710355177589	1.66752490779285e-09\\
2.20910455227614	1.87596695366144e-09\\
2.21110555277639	1.4590885915022e-09\\
2.21310655327664	2.08440899953003e-09\\
2.21510755377689	2.08440899953003e-09\\
2.21710855427714	2.29285104539863e-09\\
2.21910955477739	1.87596695366144e-09\\
2.22111055527764	1.87596695366144e-09\\
2.22311155577789	1.87596695366144e-09\\
2.22511255627814	2.08440899953003e-09\\
2.22711355677839	1.4590885915022e-09\\
2.22911455727864	1.4590885915022e-09\\
2.23111555777889	1.66752490779285e-09\\
2.23311655827914	1.4590885915022e-09\\
2.23511755877939	1.04220449976502e-09\\
2.23711855927964	1.04220449976502e-09\\
2.23911955977989	1.4590885915022e-09\\
2.24112056028014	1.04220449976502e-09\\
2.24312156078039	4.16881799906007e-10\\
2.24512256128064	1.25064654563361e-09\\
2.24712356178089	1.25064654563361e-09\\
2.24912456228114	1.25064654563361e-09\\
2.25112556278139	1.4590885915022e-09\\
2.25312656328164	1.25064654563361e-09\\
2.25512756378189	1.66752490779285e-09\\
2.25712856428214	1.87596695366144e-09\\
2.25912956478239	1.87596695366144e-09\\
2.26113056528264	1.87596695366144e-09\\
2.26313156578289	2.08440899953003e-09\\
2.26513256628314	1.87596695366144e-09\\
2.26713356678339	1.87596695366144e-09\\
2.26913456728364	1.87596695366144e-09\\
2.27113556778389	1.66752490779285e-09\\
2.27313656828414	1.4590885915022e-09\\
2.27513756878439	1.87596695366144e-09\\
2.27713856928464	2.08440899953003e-09\\
2.27913956978489	1.66752490779285e-09\\
2.28114057028514	1.87596695366144e-09\\
2.28314157078539	1.87596695366144e-09\\
2.28514257128564	1.66752490779285e-09\\
2.28714357178589	1.04220449976502e-09\\
2.28914457228614	6.25320408027829e-10\\
2.29114557278639	8.33762453896423e-10\\
2.29314657328664	8.33762453896423e-10\\
2.29514757378689	6.25320408027829e-10\\
2.29714857428714	8.33762453896423e-10\\
2.29914957478739	8.33762453896423e-10\\
2.30115057528764	1.04220449976502e-09\\
2.30315157578789	1.66752490779285e-09\\
2.30515257628814	1.4590885915022e-09\\
2.30715357678839	1.04220449976502e-09\\
2.30915457728864	8.33762453896423e-10\\
2.31115557778889	1.04220449976502e-09\\
2.31315657828914	1.4590885915022e-09\\
2.31515757878939	2.08440899953003e-09\\
2.31715857928964	2.50128736168927e-09\\
2.31915957978989	2.50128736168927e-09\\
2.32116058029015	2.29285104539863e-09\\
2.3231615807904	2.29285104539863e-09\\
2.32516258129065	1.4590885915022e-09\\
2.3271635817909	1.66752490779285e-09\\
2.32916458229115	1.87596695366144e-09\\
2.3311655827914	1.66752490779285e-09\\
2.33316658329165	1.87596695366144e-09\\
2.3351675837919	1.87596695366144e-09\\
2.33716858429215	2.29285104539863e-09\\
2.3391695847924	1.87596695366144e-09\\
2.34117058529265	1.87596695366144e-09\\
2.3431715857929	1.4590885915022e-09\\
2.34517258629315	1.87596695366144e-09\\
2.3471735867934	2.08440899953003e-09\\
2.34917458729365	2.50128736168927e-09\\
2.3511755877939	2.70972940755786e-09\\
2.35317658829415	2.91817145342646e-09\\
2.3551775887944	2.50128736168927e-09\\
2.35717858929465	2.70972940755786e-09\\
2.3591795897949	3.12661349929505e-09\\
2.36118059029515	3.54349186145428e-09\\
2.3631815907954	3.54349186145428e-09\\
2.36518259129565	3.54349186145428e-09\\
2.3671835917959	4.16881799906006e-09\\
2.36918459229615	3.75193390732288e-09\\
2.3711855927964	4.16881799906006e-09\\
2.37318659329665	3.75193390732288e-09\\
2.3751875937969	3.54349186145428e-09\\
2.37718859429715	3.75193390732288e-09\\
2.3791895947974	3.75193390732288e-09\\
2.38119059529765	3.54349186145428e-09\\
2.3831915957979	3.75193390732288e-09\\
2.38519259629815	4.16881799906006e-09\\
2.3871935967984	4.16881799906006e-09\\
2.38919459729865	4.16881799906006e-09\\
2.3911955977989	4.5856963612193e-09\\
2.39319659829915	4.37726004492866e-09\\
2.3951975987994	4.5856963612193e-09\\
2.39719859929965	4.79413840708789e-09\\
2.3991995997999	4.5856963612193e-09\\
2.40120060030015	4.37726004492866e-09\\
2.4032016008004	3.96037595319147e-09\\
2.40520260130065	3.75193390732288e-09\\
2.4072036018009	3.75193390732288e-09\\
2.40920460230115	3.54349186145428e-09\\
2.4112056028014	2.91817145342646e-09\\
2.41320660330165	3.33505554516364e-09\\
2.4152076038019	3.54349186145428e-09\\
2.41720860430215	3.33505554516364e-09\\
2.4192096048024	2.91817145342646e-09\\
2.42121060530265	2.91817145342646e-09\\
2.4232116058029	2.91817145342646e-09\\
2.42521260630315	2.70972940755786e-09\\
2.4272136068034	2.70972940755786e-09\\
2.42921460730365	2.70972940755786e-09\\
2.4312156078039	3.12661349929505e-09\\
2.43321660830415	2.70972940755786e-09\\
2.4352176088044	2.70972940755786e-09\\
2.43721860930465	2.70972940755786e-09\\
2.4392196098049	2.70972940755786e-09\\
2.44122061030515	2.70972940755786e-09\\
2.4432216108054	3.12661349929505e-09\\
2.44522261130565	3.12661349929505e-09\\
2.4472236118059	3.54349186145428e-09\\
2.44922461230615	3.33505554516364e-09\\
2.4512256128064	3.12661349929505e-09\\
2.45322661330665	3.12661349929505e-09\\
2.4552276138069	2.70972940755786e-09\\
2.45722861430715	2.70972940755786e-09\\
2.4592296148074	2.70972940755786e-09\\
2.46123061530765	2.91817145342646e-09\\
2.4632316158079	2.91817145342646e-09\\
2.46523261630815	3.12661349929505e-09\\
2.4672336168084	2.50128736168927e-09\\
2.46923461730865	2.70972940755786e-09\\
2.4712356178089	2.91817145342646e-09\\
2.47323661830915	2.91817145342646e-09\\
2.4752376188094	2.70972940755786e-09\\
2.47723861930965	2.08440899953003e-09\\
2.4792396198099	2.70972940755786e-09\\
2.48124062031015	2.50128736168927e-09\\
2.48324162081041	2.08440899953003e-09\\
2.48524262131066	2.08440899953003e-09\\
2.48724362181091	1.87596695366144e-09\\
2.48924462231116	1.66752490779285e-09\\
2.49124562281141	1.87596695366144e-09\\
2.49324662331166	1.4590885915022e-09\\
2.49524762381191	1.4590885915022e-09\\
2.49724862431216	1.25064654563361e-09\\
2.49924962481241	1.04220449976502e-09\\
2.50125062531266	1.25064654563361e-09\\
2.50325162581291	1.04220449976502e-09\\
2.50525262631316	1.04220449976502e-09\\
2.50725362681341	1.04220449976502e-09\\
2.50925462731366	1.4590885915022e-09\\
2.51125562781391	1.25064654563361e-09\\
2.51325662831416	1.04220449976502e-09\\
2.51525762881441	1.04220449976502e-09\\
2.51725862931466	1.25064654563361e-09\\
2.51925962981491	8.33762453896423e-10\\
2.52126063031516	1.04220449976502e-09\\
2.52326163081541	1.04220449976502e-09\\
2.52526263131566	1.04220449976502e-09\\
2.52726363181591	1.04220449976502e-09\\
2.52926463231616	1.04220449976502e-09\\
2.53126563281641	8.33762453896423e-10\\
2.53326663331666	8.33762453896423e-10\\
2.53526763381691	8.33762453896423e-10\\
2.53726863431716	6.25320408027829e-10\\
2.53926963481741	1.04220449976502e-09\\
2.54127063531766	6.25320408027829e-10\\
2.54327163581791	6.25320408027829e-10\\
2.54527263631816	4.16881799906007e-10\\
2.54727363681841	4.16881799906007e-10\\
2.54927463731866	2.08440899953003e-10\\
2.55127563781891	2.08440899953003e-10\\
2.55327663831916	4.16881799906007e-10\\
2.55527763881941	-2.08440899953003e-10\\
2.55727863931966	-2.08440899953003e-10\\
2.55927963981991	-4.16881799906007e-10\\
2.56128064032016	-4.16881799906007e-10\\
2.56328164082041	-2.08440899953003e-10\\
2.56528264132066	-6.25320408027829e-10\\
2.56728364182091	-4.16881799906007e-10\\
2.56928464232116	-2.08440899953003e-10\\
2.57128564282141	2.08440899953003e-10\\
2.57328664332166	6.25320408027829e-10\\
2.57528764382191	0\\
2.57728864432216	4.16881799906007e-10\\
2.57928964482241	4.16881799906007e-10\\
2.58129064532266	6.25320408027829e-10\\
2.58329164582291	4.16881799906007e-10\\
2.58529264632316	2.08440899953003e-10\\
2.58729364682341	6.25320408027829e-10\\
2.58929464732366	4.16881799906007e-10\\
2.59129564782391	2.08440899953003e-10\\
2.59329664832416	4.16881799906007e-10\\
2.59529764882441	0\\
2.59729864932466	2.08440899953003e-10\\
2.59929964982491	6.25320408027829e-10\\
2.60130065032516	4.16881799906007e-10\\
2.60330165082541	6.25320408027829e-10\\
2.60530265132566	6.25320408027829e-10\\
2.60730365182591	6.25320408027829e-10\\
2.60930465232616	8.33762453896423e-10\\
2.61130565282641	4.16881799906007e-10\\
2.61330665332666	6.25320408027829e-10\\
2.61530765382691	8.33762453896423e-10\\
2.61730865432716	8.33762453896423e-10\\
2.61930965482741	1.25064654563361e-09\\
2.62131065532766	6.25320408027829e-10\\
2.62331165582791	8.33762453896423e-10\\
2.62531265632816	8.33762453896423e-10\\
2.62731365682841	8.33762453896423e-10\\
2.62931465732866	1.04220449976502e-09\\
2.63131565782891	1.04220449976502e-09\\
2.63331665832916	8.33762453896423e-10\\
2.63531765882941	6.25320408027829e-10\\
2.63731865932966	8.33762453896423e-10\\
2.63931965982992	6.25320408027829e-10\\
2.64132066033017	8.33762453896423e-10\\
2.64332166083042	1.04220449976502e-09\\
2.64532266133067	2.08440899953003e-10\\
2.64732366183092	6.25320408027829e-10\\
2.64932466233117	6.25320408027829e-10\\
2.65132566283142	8.33762453896423e-10\\
2.65332666333167	8.33762453896423e-10\\
2.65532766383192	8.33762453896423e-10\\
2.65732866433217	6.25320408027829e-10\\
2.65932966483242	4.16881799906007e-10\\
2.66133066533267	2.08440899953003e-10\\
2.66333166583292	-2.08440899953003e-10\\
2.66533266633317	0\\
2.66733366683342	0\\
2.66933466733367	-2.08440899953003e-10\\
2.67133566783392	-6.25320408027829e-10\\
2.67333666833417	-6.25320408027829e-10\\
2.67533766883442	-6.25320408027829e-10\\
2.67733866933467	-6.25320408027829e-10\\
2.67933966983492	-6.25320408027829e-10\\
2.68134067033517	-4.16881799906007e-10\\
2.68334167083542	-2.08440899953003e-10\\
2.68534267133567	-4.16881799906007e-10\\
2.68734367183592	-6.25320408027829e-10\\
2.68934467233617	-6.25320408027829e-10\\
2.69134567283642	-8.33762453896423e-10\\
2.69334667333667	-6.25320408027829e-10\\
2.69534767383692	-8.33762453896423e-10\\
2.69734867433717	0\\
2.69934967483742	0\\
2.70135067533767	-2.08440899953003e-10\\
2.70335167583792	-2.08440899953003e-10\\
2.70535267633817	-6.25320408027829e-10\\
2.70735367683842	-6.25320408027829e-10\\
2.70935467733867	-6.25320408027829e-10\\
2.71135567783892	-6.25320408027829e-10\\
2.71335667833917	-2.08440899953003e-10\\
2.71535767883942	2.08440899953003e-10\\
2.71735867933967	-4.16881799906007e-10\\
2.71935967983992	-2.08440899953003e-10\\
2.72136068034017	2.08440899953003e-10\\
2.72336168084042	-2.08440899953003e-10\\
2.72536268134067	4.16881799906007e-10\\
2.72736368184092	0\\
2.72936468234117	-2.08440899953003e-10\\
2.73136568284142	-8.33762453896423e-10\\
2.73336668334167	-6.25320408027829e-10\\
2.73536768384192	-4.16881799906007e-10\\
2.73736868434217	-6.25320408027829e-10\\
2.73936968484242	-8.33762453896423e-10\\
2.74137068534267	-8.33762453896423e-10\\
2.74337168584292	-1.04220449976502e-09\\
2.74537268634317	-1.25064654563361e-09\\
2.74737368684342	-1.25064654563361e-09\\
2.74937468734367	-1.25064654563361e-09\\
2.75137568784392	-1.25064654563361e-09\\
2.75337668834417	-1.4590885915022e-09\\
2.75537768884442	-1.4590885915022e-09\\
2.75737868934467	-1.4590885915022e-09\\
2.75937968984492	-1.66752490779285e-09\\
2.76138069034517	-1.4590885915022e-09\\
2.76338169084542	-1.25064654563361e-09\\
2.76538269134567	-6.25320408027829e-10\\
2.76738369184592	-1.04220449976502e-09\\
2.76938469234617	-1.04220449976502e-09\\
2.77138569284642	-1.25064654563361e-09\\
2.77338669334667	-1.04220449976502e-09\\
2.77538769384692	-1.04220449976502e-09\\
2.77738869434717	-1.4590885915022e-09\\
2.77938969484742	-6.25320408027829e-10\\
2.78139069534767	-6.25320408027829e-10\\
2.78339169584792	-4.16881799906007e-10\\
2.78539269634817	0\\
2.78739369684842	-2.08440899953003e-10\\
2.78939469734867	0\\
2.79139569784892	4.16881799906007e-10\\
2.79339669834917	4.16881799906007e-10\\
2.79539769884942	4.16881799906007e-10\\
2.79739869934967	4.16881799906007e-10\\
2.79939969984992	4.16881799906007e-10\\
2.80140070035018	4.16881799906007e-10\\
2.80340170085043	8.33762453896423e-10\\
2.80540270135068	1.04220449976502e-09\\
2.80740370185093	1.25064654563361e-09\\
2.80940470235118	1.4590885915022e-09\\
2.81140570285143	1.25064654563361e-09\\
2.81340670335168	1.4590885915022e-09\\
2.81540770385193	1.66752490779285e-09\\
2.81740870435218	1.4590885915022e-09\\
2.81940970485243	1.4590885915022e-09\\
2.82141070535268	1.66752490779285e-09\\
2.82341170585293	1.4590885915022e-09\\
2.82541270635318	8.33762453896423e-10\\
2.82741370685343	8.33762453896423e-10\\
2.82941470735368	8.33762453896423e-10\\
2.83141570785393	8.33762453896423e-10\\
2.83341670835418	1.25064654563361e-09\\
2.83541770885443	1.04220449976502e-09\\
2.83741870935468	1.04220449976502e-09\\
2.83941970985493	1.25064654563361e-09\\
2.84142071035518	1.4590885915022e-09\\
2.84342171085543	1.4590885915022e-09\\
2.84542271135568	1.25064654563361e-09\\
2.84742371185593	1.4590885915022e-09\\
2.84942471235618	1.66752490779285e-09\\
2.85142571285643	1.66752490779285e-09\\
2.85342671335668	1.25064654563361e-09\\
2.85542771385693	1.4590885915022e-09\\
2.85742871435718	1.4590885915022e-09\\
2.85942971485743	1.4590885915022e-09\\
2.86143071535768	2.08440899953003e-09\\
2.86343171585793	1.66752490779285e-09\\
2.86543271635818	1.25064654563361e-09\\
2.86743371685843	1.66752490779285e-09\\
2.86943471735868	1.66752490779285e-09\\
2.87143571785893	1.66752490779285e-09\\
2.87343671835918	2.29285104539863e-09\\
2.87543771885943	1.87596695366144e-09\\
2.87743871935968	1.25064654563361e-09\\
2.87943971985993	1.25064654563361e-09\\
2.88144072036018	1.4590885915022e-09\\
2.88344172086043	2.08440899953003e-09\\
2.88544272136068	2.08440899953003e-09\\
2.88744372186093	2.08440899953003e-09\\
2.88944472236118	2.50128736168927e-09\\
2.89144572286143	2.29285104539863e-09\\
2.89344672336168	2.29285104539863e-09\\
2.89544772386193	1.87596695366144e-09\\
2.89744872436218	1.87596695366144e-09\\
2.89944972486243	1.87596695366144e-09\\
2.90145072536268	2.50128736168927e-09\\
2.90345172586293	1.87596695366144e-09\\
2.90545272636318	2.08440899953003e-09\\
2.90745372686343	1.87596695366144e-09\\
2.90945472736368	2.08440899953003e-09\\
2.91145572786393	2.08440899953003e-09\\
2.91345672836418	2.08440899953003e-09\\
2.91545772886443	2.29285104539863e-09\\
2.91745872936468	1.87596695366144e-09\\
2.91945972986493	2.08440899953003e-09\\
2.92146073036518	2.50128736168927e-09\\
2.92346173086543	2.70972940755786e-09\\
2.92546273136568	3.12661349929505e-09\\
2.92746373186593	2.70972940755786e-09\\
2.92946473236618	2.70972940755786e-09\\
2.93146573286643	2.70972940755786e-09\\
2.93346673336668	2.70972940755786e-09\\
2.93546773386693	2.50128736168927e-09\\
2.93746873436718	2.08440899953003e-09\\
2.93946973486743	1.87596695366144e-09\\
2.94147073536768	1.25064654563361e-09\\
2.94347173586793	1.4590885915022e-09\\
2.94547273636818	1.4590885915022e-09\\
2.94747373686843	1.87596695366144e-09\\
2.94947473736868	1.66752490779285e-09\\
2.95147573786893	1.25064654563361e-09\\
2.95347673836918	1.4590885915022e-09\\
2.95547773886943	1.4590885915022e-09\\
2.95747873936968	1.04220449976502e-09\\
2.95947973986994	1.66752490779285e-09\\
2.96148074037019	1.66752490779285e-09\\
2.96348174087044	1.87596695366144e-09\\
2.96548274137069	2.08440899953003e-09\\
2.96748374187094	1.87596695366144e-09\\
2.96948474237119	1.66752490779285e-09\\
2.97148574287144	1.4590885915022e-09\\
2.97348674337169	1.25064654563361e-09\\
2.97548774387194	1.25064654563361e-09\\
2.97748874437219	1.04220449976502e-09\\
2.97948974487244	6.25320408027829e-10\\
2.98149074537269	2.08440899953003e-10\\
2.98349174587294	8.33762453896423e-10\\
2.98549274637319	4.16881799906007e-10\\
2.98749374687344	6.25320408027829e-10\\
2.98949474737369	4.16881799906007e-10\\
2.99149574787394	6.25320408027829e-10\\
2.99349674837419	4.16881799906007e-10\\
2.99549774887444	6.25320408027829e-10\\
2.99749874937469	1.04220449976502e-09\\
2.99949974987494	1.25064654563361e-09\\
3.00150075037519	1.66752490779285e-09\\
3.00350175087544	1.87596695366144e-09\\
3.00550275137569	1.4590885915022e-09\\
3.00750375187594	1.4590885915022e-09\\
3.00950475237619	1.25064654563361e-09\\
3.01150575287644	8.33762453896423e-10\\
3.01350675337669	6.25320408027829e-10\\
3.01550775387694	4.16881799906007e-10\\
3.01750875437719	2.08440899953003e-10\\
3.01950975487744	4.16881799906007e-10\\
3.02151075537769	2.08440899953003e-10\\
3.02351175587794	-2.08440899953003e-10\\
3.02551275637819	0\\
3.02751375687844	2.08440899953003e-10\\
3.02951475737869	-2.08440899953003e-10\\
3.03151575787894	-4.16881799906007e-10\\
3.03351675837919	0\\
3.03551775887944	4.16881799906007e-10\\
3.03751875937969	2.08440899953003e-10\\
3.03951975987994	2.08440899953003e-10\\
3.04152076038019	-2.08440899953003e-10\\
3.04352176088044	2.08440899953003e-10\\
3.04552276138069	6.25320408027829e-10\\
3.04752376188094	6.25320408027829e-10\\
3.04952476238119	4.16881799906007e-10\\
3.05152576288144	4.16881799906007e-10\\
3.05352676338169	2.08440899953003e-10\\
3.05552776388194	4.16881799906007e-10\\
3.05752876438219	-2.08440899953003e-10\\
3.05952976488244	-4.16881799906007e-10\\
3.06153076538269	-4.16881799906007e-10\\
3.06353176588294	-4.16881799906007e-10\\
3.06553276638319	-8.33762453896423e-10\\
3.06753376688344	-8.33762453896423e-10\\
3.06953476738369	-2.08440899953003e-10\\
3.07153576788394	-2.08440899953003e-10\\
3.07353676838419	-2.08440899953003e-10\\
3.07553776888444	-2.08440899953003e-10\\
3.07753876938469	-2.08440899953003e-10\\
3.07953976988494	-6.25320408027829e-10\\
3.08154077038519	-6.25320408027829e-10\\
3.08354177088544	-1.25064654563361e-09\\
3.08554277138569	-1.04220449976502e-09\\
3.08754377188594	-1.4590885915022e-09\\
3.08954477238619	-1.66752490779285e-09\\
3.09154577288644	-1.87596695366144e-09\\
3.09354677338669	-1.87596695366144e-09\\
3.09554777388694	-1.66752490779285e-09\\
3.09754877438719	-1.4590885915022e-09\\
3.09954977488744	-2.08440899953003e-09\\
3.10155077538769	-1.87596695366144e-09\\
3.10355177588794	-2.08440899953003e-09\\
3.10555277638819	-1.4590885915022e-09\\
3.10755377688844	-1.4590885915022e-09\\
3.10955477738869	-1.04220449976502e-09\\
3.11155577788894	-1.04220449976502e-09\\
3.11355677838919	-1.04220449976502e-09\\
3.11555777888944	-1.04220449976502e-09\\
3.11755877938969	-1.4590885915022e-09\\
3.11955977988994	-1.4590885915022e-09\\
3.1215607803902	-1.4590885915022e-09\\
3.12356178089045	-1.4590885915022e-09\\
3.1255627813907	-1.66752490779285e-09\\
3.12756378189095	-1.25064654563361e-09\\
3.1295647823912	-1.25064654563361e-09\\
3.13156578289145	-1.4590885915022e-09\\
3.1335667833917	-1.66752490779285e-09\\
3.13556778389195	-1.87596695366144e-09\\
3.1375687843922	-1.87596695366144e-09\\
3.13956978489245	-1.87596695366144e-09\\
3.1415707853927	-2.50128736168927e-09\\
3.14357178589295	-2.08440899953003e-09\\
3.1455727863932	-2.50128736168927e-09\\
3.14757378689345	-2.91817145342646e-09\\
3.1495747873937	-2.91817145342646e-09\\
3.15157578789395	-3.12661349929505e-09\\
3.1535767883942	-2.91817145342646e-09\\
3.15557778889445	-2.70972940755786e-09\\
3.1575787893947	-2.50128736168927e-09\\
3.15957978989495	-2.70972940755786e-09\\
3.1615807903952	-3.12661349929505e-09\\
3.16358179089545	-2.70972940755786e-09\\
3.1655827913957	-3.12661349929505e-09\\
3.16758379189595	-3.12661349929505e-09\\
3.1695847923962	-3.33505554516364e-09\\
3.17158579289645	-3.54349186145428e-09\\
3.1735867933967	-3.33505554516364e-09\\
3.17558779389695	-3.12661349929505e-09\\
3.1775887943972	-2.70972940755786e-09\\
3.17958979489745	-3.33505554516364e-09\\
3.1815907953977	-3.33505554516364e-09\\
3.18359179589795	-3.33505554516364e-09\\
3.1855927963982	-3.12661349929505e-09\\
3.18759379689845	-3.33505554516364e-09\\
3.1895947973987	-3.54349186145428e-09\\
3.19159579789895	-4.16881799906006e-09\\
3.1935967983992	-3.75193390732288e-09\\
3.19559779889945	-4.16881799906006e-09\\
3.1975987993997	-3.96037595319147e-09\\
3.19959979989995	-4.16881799906006e-09\\
3.2016008004002	-3.96037595319147e-09\\
3.20360180090045	-3.54349186145428e-09\\
3.2056028014007	-3.75193390732288e-09\\
3.20760380190095	-3.75193390732288e-09\\
3.2096048024012	-3.54349186145428e-09\\
3.21160580290145	-3.75193390732288e-09\\
3.2136068034017	-3.75193390732288e-09\\
3.21560780390195	-3.96037595319147e-09\\
3.2176088044022	-3.96037595319147e-09\\
3.21960980490245	-3.75193390732288e-09\\
3.2216108054027	-4.37726004492866e-09\\
3.22361180590295	-4.5856963612193e-09\\
3.2256128064032	-4.79413840708789e-09\\
3.22761380690345	-5.41945881511572e-09\\
3.2296148074037	-5.00258045295649e-09\\
3.23161580790395	-5.21102249882508e-09\\
3.2336168084042	-5.21102249882508e-09\\
3.23561780890445	-5.00258045295649e-09\\
3.2376188094047	-4.5856963612193e-09\\
3.23961980990495	-4.79413840708789e-09\\
3.2416208104052	-4.79413840708789e-09\\
3.24362181090545	-5.00258045295649e-09\\
3.2456228114057	-4.5856963612193e-09\\
3.24762381190595	-5.00258045295649e-09\\
3.2496248124062	-5.00258045295649e-09\\
3.25162581290645	-4.5856963612193e-09\\
3.2536268134067	-3.96037595319147e-09\\
3.25562781390695	-4.16881799906006e-09\\
3.2576288144072	-3.96037595319147e-09\\
3.25962981490745	-4.5856963612193e-09\\
3.2616308154077	-5.00258045295649e-09\\
3.26363181590795	-4.5856963612193e-09\\
3.2656328164082	-4.79413840708789e-09\\
3.26763381690845	-4.5856963612193e-09\\
3.2696348174087	-4.79413840708789e-09\\
3.27163581790895	-5.21102249882508e-09\\
3.2736368184092	-5.00258045295649e-09\\
3.27563781890945	-5.00258045295649e-09\\
3.27763881940971	-4.79413840708789e-09\\
3.27963981990996	-5.00258045295649e-09\\
3.2816408204102	-5.41945881511572e-09\\
3.28364182091046	-5.21102249882508e-09\\
3.28564282141071	-5.00258045295649e-09\\
3.28764382191096	-5.21102249882508e-09\\
3.28964482241121	-5.41945881511572e-09\\
3.29164582291146	-4.5856963612193e-09\\
3.29364682341171	-4.79413840708789e-09\\
3.29564782391196	-5.21102249882508e-09\\
3.29764882441221	-5.00258045295649e-09\\
3.29964982491246	-4.79413840708789e-09\\
3.30165082541271	-4.5856963612193e-09\\
3.30365182591296	-4.5856963612193e-09\\
3.30565282641321	-4.5856963612193e-09\\
3.30765382691346	-5.00258045295649e-09\\
3.30965482741371	-4.16881799906006e-09\\
3.31165582791396	-4.37726004492866e-09\\
3.31365682841421	-5.00258045295649e-09\\
3.31565782891446	-4.79413840708789e-09\\
3.31765882941471	-4.5856963612193e-09\\
3.31965982991496	-4.79413840708789e-09\\
3.32166083041521	-4.79413840708789e-09\\
3.32366183091546	-4.5856963612193e-09\\
3.32566283141571	-4.16881799906006e-09\\
3.32766383191596	-4.16881799906006e-09\\
3.32966483241621	-4.79413840708789e-09\\
3.33166583291646	-4.79413840708789e-09\\
3.33366683341671	-4.79413840708789e-09\\
3.33566783391696	-4.79413840708789e-09\\
3.33766883441721	-4.37726004492866e-09\\
3.33966983491746	-4.5856963612193e-09\\
3.34167083541771	-4.16881799906006e-09\\
3.34367183591796	-4.37726004492866e-09\\
3.34567283641821	-4.37726004492866e-09\\
3.34767383691846	-4.5856963612193e-09\\
3.34967483741871	-4.37726004492866e-09\\
3.35167583791896	-4.5856963612193e-09\\
3.35367683841921	-5.00258045295649e-09\\
3.35567783891946	-5.41945881511572e-09\\
3.35767883941971	-5.62790086098432e-09\\
3.35967983991996	-4.79413840708789e-09\\
3.36168084042021	-5.00258045295649e-09\\
3.36368184092046	-4.79413840708789e-09\\
3.36568284142071	-4.37726004492866e-09\\
3.36768384192096	-4.5856963612193e-09\\
3.36968484242121	-4.5856963612193e-09\\
3.37168584292146	-5.00258045295649e-09\\
3.37368684342171	-5.00258045295649e-09\\
3.37568784392196	-5.00258045295649e-09\\
3.37768884442221	-5.41945881511572e-09\\
3.37968984492246	-5.21102249882508e-09\\
3.38169084542271	-5.00258045295649e-09\\
3.38369184592296	-4.79413840708789e-09\\
3.38569284642321	-4.5856963612193e-09\\
3.38769384692346	-5.00258045295649e-09\\
3.38969484742371	-5.21102249882508e-09\\
3.39169584792396	-5.00258045295649e-09\\
3.39369684842421	-5.00258045295649e-09\\
3.39569784892446	-5.21102249882508e-09\\
3.39769884942471	-4.79413840708789e-09\\
3.39969984992496	-4.79413840708789e-09\\
3.40170085042521	-4.79413840708789e-09\\
3.40370185092546	-4.37726004492866e-09\\
3.40570285142571	-4.16881799906006e-09\\
3.40770385192596	-4.16881799906006e-09\\
3.40970485242621	-3.75193390732288e-09\\
3.41170585292646	-3.75193390732288e-09\\
3.41370685342671	-3.96037595319147e-09\\
3.41570785392696	-4.37726004492866e-09\\
3.41770885442721	-4.5856963612193e-09\\
3.41970985492746	-5.21102249882508e-09\\
3.42171085542771	-5.21102249882508e-09\\
3.42371185592796	-5.41945881511572e-09\\
3.42571285642821	-5.00258045295649e-09\\
3.42771385692846	-5.21102249882508e-09\\
3.42971485742871	-5.21102249882508e-09\\
3.43171585792896	-5.62790086098432e-09\\
3.43371685842921	-5.21102249882508e-09\\
3.43571785892946	-5.41945881511572e-09\\
3.43771885942971	-5.41945881511572e-09\\
3.43971985992996	-5.41945881511572e-09\\
3.44172086043022	-5.8363199885411e-09\\
3.44372186093047	-5.8363199885411e-09\\
3.44572286143072	-5.8363199885411e-09\\
3.44772386193097	-6.0447620344097e-09\\
3.44972486243122	-5.8363199885411e-09\\
3.45172586293147	-6.0447620344097e-09\\
3.45372686343172	-5.62790086098432e-09\\
3.45572786393197	-5.62790086098432e-09\\
3.45772886443222	-5.62790086098432e-09\\
3.45972986493247	-5.62790086098432e-09\\
3.46173086543272	-5.62790086098432e-09\\
3.46373186593297	-5.41945881511572e-09\\
3.46573286643322	-5.8363199885411e-09\\
3.46773386693347	-5.8363199885411e-09\\
3.46973486743372	-5.8363199885411e-09\\
3.47173586793397	-5.8363199885411e-09\\
3.47373686843422	-6.0447620344097e-09\\
3.47573786893447	-6.0447620344097e-09\\
3.47773886943472	-6.0447620344097e-09\\
3.47973986993497	-6.25320408027829e-09\\
3.48174087043522	-6.25320408027829e-09\\
3.48374187093547	-6.25320408027829e-09\\
3.48574287143572	-6.46164612614688e-09\\
3.48774387193597	-6.46164612614688e-09\\
3.48974487243622	-6.67008817201548e-09\\
3.49174587293647	-6.67008817201548e-09\\
3.49374687343672	-6.87853021788407e-09\\
3.49574787393697	-6.87853021788407e-09\\
3.49774887443722	-6.67008817201548e-09\\
3.49974987493747	-7.08697226375267e-09\\
3.50175087543772	-6.87853021788407e-09\\
3.50375187593797	-6.87853021788407e-09\\
3.50575287643822	-6.87853021788407e-09\\
3.50775387693847	-6.87853021788407e-09\\
3.50975487743872	-7.08697226375267e-09\\
3.51175587793897	-6.87853021788407e-09\\
3.51375687843922	-6.46164612614688e-09\\
3.51575787893947	-6.67008817201548e-09\\
3.51775887943972	-6.87853021788407e-09\\
3.51975987993997	-6.87853021788407e-09\\
3.52176088044022	-6.67008817201548e-09\\
3.52376188094047	-7.08697226375267e-09\\
3.52576288144072	-7.08697226375267e-09\\
3.52776388194097	-7.08697226375267e-09\\
3.52976488244122	-6.46164612614688e-09\\
3.53176588294147	-6.87853021788407e-09\\
3.53376688344172	-6.87853021788407e-09\\
3.53576788394197	-6.87853021788407e-09\\
3.53776888444222	-6.87853021788407e-09\\
3.53976988494247	-6.87853021788407e-09\\
3.54177088544272	-6.46164612614688e-09\\
3.54377188594297	-6.25320408027829e-09\\
3.54577288644322	-6.87853021788407e-09\\
3.54777388694347	-6.67008817201548e-09\\
3.54977488744372	-7.29541430962126e-09\\
3.55177588794397	-7.29541430962126e-09\\
3.55377688844422	-7.71229840135845e-09\\
3.55577788894447	-7.71229840135845e-09\\
3.55777888944472	-7.92074044722704e-09\\
3.55977988994497	-7.71229840135845e-09\\
3.56178089044522	-7.71229840135845e-09\\
3.56378189094547	-7.92074044722704e-09\\
3.56578289144572	-8.54606658483282e-09\\
3.56778389194597	-8.54606658483282e-09\\
3.56978489244622	-8.54606658483282e-09\\
3.57178589294647	-8.75450863070141e-09\\
3.57378689344672	-8.54606658483282e-09\\
3.57578789394697	-8.75450863070141e-09\\
3.57778889444722	-8.33762453896423e-09\\
3.57978989494747	-8.54606658483282e-09\\
3.58179089544772	-8.96295067657001e-09\\
3.58379189594797	-9.1713927224386e-09\\
3.58579289644822	-8.96295067657001e-09\\
3.58779389694847	-8.75450863070141e-09\\
3.58979489744872	-8.75450863070141e-09\\
3.59179589794897	-8.75450863070141e-09\\
3.59379689844922	-8.54606658483282e-09\\
3.59579789894947	-8.33762453896423e-09\\
3.59779889944972	-8.54606658483282e-09\\
3.59979989994997	-8.12918249309563e-09\\
3.60180090045022	-8.75450863070141e-09\\
3.60380190095048	-8.54606658483282e-09\\
3.60580290145073	-8.33762453896423e-09\\
3.60780390195098	-8.33762453896423e-09\\
3.60980490245123	-8.12918249309563e-09\\
3.61180590295148	-7.92074044722704e-09\\
3.61380690345173	-8.12918249309563e-09\\
3.61580790395198	-7.71229840135845e-09\\
3.61780890445223	-7.92074044722704e-09\\
3.61980990495248	-7.71229840135845e-09\\
3.62181090545273	-7.71229840135845e-09\\
3.62381190595298	-7.29541430962126e-09\\
3.62581290645323	-7.29541430962126e-09\\
3.62781390695348	-7.50385635548985e-09\\
3.62981490745373	-7.08697226375267e-09\\
3.63181590795398	-7.08697226375267e-09\\
3.63381690845423	-6.87853021788407e-09\\
3.63581790895448	-6.46164612614688e-09\\
3.63781890945473	-7.08697226375267e-09\\
3.63981990995498	-7.71229840135845e-09\\
3.64182091045523	-7.92074044722704e-09\\
3.64382191095548	-7.92074044722704e-09\\
3.64582291145573	-7.29541430962126e-09\\
3.64782391195598	-7.50385635548985e-09\\
3.64982491245623	-7.29541430962126e-09\\
3.65182591295648	-7.08697226375267e-09\\
3.65382691345673	-7.08697226375267e-09\\
3.65582791395698	-7.50385635548985e-09\\
3.65782891445723	-7.29541430962126e-09\\
3.65982991495748	-7.08697226375267e-09\\
3.66183091545773	-7.29541430962126e-09\\
3.66383191595798	-7.29541430962126e-09\\
3.66583291645823	-7.29541430962126e-09\\
3.66783391695848	-7.50385635548985e-09\\
3.66983491745873	-7.71229840135845e-09\\
3.67183591795898	-7.29541430962126e-09\\
3.67383691845923	-7.71229840135845e-09\\
3.67583791895948	-7.50385635548985e-09\\
3.67783891945973	-7.50385635548985e-09\\
3.67983991995998	-6.87853021788407e-09\\
3.68184092046023	-6.87853021788407e-09\\
3.68384192096048	-6.87853021788407e-09\\
3.68584292146073	-7.29541430962126e-09\\
3.68784392196098	-7.29541430962126e-09\\
3.68984492246123	-7.29541430962126e-09\\
3.69184592296148	-7.29541430962126e-09\\
3.69384692346173	-7.08697226375267e-09\\
3.69584792396198	-7.08697226375267e-09\\
3.69784892446223	-6.87853021788407e-09\\
3.69984992496248	-6.67008817201548e-09\\
3.70185092546273	-6.25320408027829e-09\\
3.70385192596298	-6.46164612614688e-09\\
3.70585292646323	-6.25320408027829e-09\\
3.70785392696348	-6.46164612614688e-09\\
3.70985492746373	-6.0447620344097e-09\\
3.71185592796398	-6.25320408027829e-09\\
3.71385692846423	-6.0447620344097e-09\\
3.71585792896448	-5.21102249882508e-09\\
3.71785892946473	-5.8363199885411e-09\\
3.71985992996498	-6.25320408027829e-09\\
3.72186093046523	-6.46164612614688e-09\\
3.72386193096548	-6.0447620344097e-09\\
3.72586293146573	-6.0447620344097e-09\\
3.72786393196598	-6.0447620344097e-09\\
3.72986493246623	-5.62790086098432e-09\\
3.73186593296648	-5.41945881511572e-09\\
3.73386693346673	-5.8363199885411e-09\\
3.73586793396698	-5.8363199885411e-09\\
3.73786893446723	-5.8363199885411e-09\\
3.73986993496748	-6.0447620344097e-09\\
3.74187093546773	-6.0447620344097e-09\\
3.74387193596798	-5.8363199885411e-09\\
3.74587293646823	-5.41945881511572e-09\\
3.74787393696848	-5.41945881511572e-09\\
3.74987493746873	-5.62790086098432e-09\\
3.75187593796898	-6.0447620344097e-09\\
3.75387693846923	-5.8363199885411e-09\\
3.75587793896948	-5.62790086098432e-09\\
3.75787893946973	-6.25320408027829e-09\\
3.75987993996999	-6.25320408027829e-09\\
3.76188094047024	-6.46164612614688e-09\\
3.76388194097049	-6.67008817201548e-09\\
3.76588294147074	-6.25320408027829e-09\\
3.76788394197099	-5.62790086098432e-09\\
3.76988494247124	-5.62790086098432e-09\\
3.77188594297149	-6.0447620344097e-09\\
3.77388694347174	-6.0447620344097e-09\\
3.77588794397199	-5.8363199885411e-09\\
3.77788894447224	-6.25320408027829e-09\\
3.77988994497249	-6.25320408027829e-09\\
3.78189094547274	-6.25320408027829e-09\\
3.78389194597299	-6.67008817201548e-09\\
3.78589294647324	-6.25320408027829e-09\\
3.78789394697349	-5.8363199885411e-09\\
3.78989494747374	-5.62790086098432e-09\\
3.79189594797399	-5.41945881511572e-09\\
3.79389694847424	-4.79413840708789e-09\\
3.79589794897449	-4.5856963612193e-09\\
3.79789894947474	-5.00258045295649e-09\\
3.79989994997499	-5.62790086098432e-09\\
3.80190095047524	-5.62790086098432e-09\\
3.80390195097549	-5.62790086098432e-09\\
3.80590295147574	-6.25320408027829e-09\\
3.80790395197599	-5.62790086098432e-09\\
3.80990495247624	-6.0447620344097e-09\\
3.81190595297649	-6.0447620344097e-09\\
3.81390695347674	-6.0447620344097e-09\\
3.81590795397699	-6.25320408027829e-09\\
3.81790895447724	-6.87853021788407e-09\\
3.81990995497749	-6.67008817201548e-09\\
3.82191095547774	-6.46164612614688e-09\\
3.82391195597799	-6.67008817201548e-09\\
3.82591295647824	-7.08697226375267e-09\\
3.82791395697849	-7.29541430962126e-09\\
3.82991495747874	-7.29541430962126e-09\\
3.83191595797899	-7.29541430962126e-09\\
3.83391695847924	-6.46164612614688e-09\\
3.83591795897949	-6.87853021788407e-09\\
3.83791895947974	-7.29541430962126e-09\\
3.83991995997999	-6.67008817201548e-09\\
3.84192096048024	-7.08697226375267e-09\\
3.84392196098049	-6.67008817201548e-09\\
3.84592296148074	-7.08697226375267e-09\\
3.84792396198099	-6.87853021788407e-09\\
3.84992496248124	-6.25320408027829e-09\\
3.85192596298149	-7.08697226375267e-09\\
3.85392696348174	-7.08697226375267e-09\\
3.85592796398199	-7.29541430962126e-09\\
3.85792896448224	-7.08697226375267e-09\\
3.85992996498249	-7.08697226375267e-09\\
3.86193096548274	-7.29541430962126e-09\\
3.86393196598299	-7.29541430962126e-09\\
3.86593296648324	-7.92074044722704e-09\\
3.86793396698349	-7.71229840135845e-09\\
3.86993496748374	-7.71229840135845e-09\\
3.87193596798399	-7.50385635548985e-09\\
3.87393696848424	-7.29541430962126e-09\\
3.87593796898449	-7.29541430962126e-09\\
3.87793896948474	-7.50385635548985e-09\\
3.87993996998499	-7.29541430962126e-09\\
3.88194097048524	-6.67008817201548e-09\\
3.88394197098549	-6.46164612614688e-09\\
3.88594297148574	-6.25320408027829e-09\\
3.88794397198599	-6.46164612614688e-09\\
3.88994497248624	-6.87853021788407e-09\\
3.89194597298649	-6.46164612614688e-09\\
3.89394697348674	-7.08697226375267e-09\\
3.89594797398699	-6.67008817201548e-09\\
3.89794897448724	-6.46164612614688e-09\\
3.89994997498749	-6.87853021788407e-09\\
3.90195097548774	-6.67008817201548e-09\\
3.90395197598799	-6.46164612614688e-09\\
3.90595297648824	-6.25320408027829e-09\\
3.90795397698849	-6.46164612614688e-09\\
3.90995497748874	-6.67008817201548e-09\\
3.91195597798899	-6.46164612614688e-09\\
3.91395697848924	-6.0447620344097e-09\\
3.91595797898949	-5.62790086098432e-09\\
3.91795897948974	-5.41945881511572e-09\\
3.91995997998999	-5.41945881511572e-09\\
3.92196098049025	-5.00258045295649e-09\\
3.9239619809905	-5.21102249882508e-09\\
3.92596298149075	-4.5856963612193e-09\\
3.927963981991	-4.5856963612193e-09\\
3.92996498249125	-4.79413840708789e-09\\
3.9319659829915	-5.00258045295649e-09\\
3.93396698349175	-5.41945881511572e-09\\
3.935967983992	-5.21102249882508e-09\\
3.93796898449225	-5.21102249882508e-09\\
3.9399699849925	-5.21102249882508e-09\\
3.94197098549275	-4.37726004492866e-09\\
3.943971985993	-3.96037595319147e-09\\
3.94597298649325	-3.96037595319147e-09\\
3.9479739869935	-3.54349186145428e-09\\
3.94997498749375	-3.75193390732288e-09\\
3.951975987994	-3.96037595319147e-09\\
3.95397698849425	-4.37726004492866e-09\\
3.9559779889945	-3.96037595319147e-09\\
3.95797898949475	-4.16881799906006e-09\\
3.959979989995	-3.75193390732288e-09\\
3.96198099049525	-3.12661349929505e-09\\
3.9639819909955	-3.12661349929505e-09\\
3.96598299149575	-3.33505554516364e-09\\
3.967983991996	-3.12661349929505e-09\\
3.96998499249625	-3.12661349929505e-09\\
3.9719859929965	-2.91817145342646e-09\\
3.97398699349675	-3.33505554516364e-09\\
3.975987993997	-3.54349186145428e-09\\
3.97798899449725	-3.96037595319147e-09\\
3.9799899949975	-4.16881799906006e-09\\
3.98199099549775	-4.16881799906006e-09\\
3.983991995998	-3.96037595319147e-09\\
3.98599299649825	-3.75193390732288e-09\\
3.9879939969985	-3.54349186145428e-09\\
3.98999499749875	-3.33505554516364e-09\\
3.991995997999	-3.54349186145428e-09\\
3.99399699849925	-3.33505554516364e-09\\
3.9959979989995	-3.33505554516364e-09\\
3.99799899949975	-3.12661349929505e-09\\
4	-2.70972940755786e-09\\
};
\addlegendentry{c1};

\addplot [color=mycolor2,solid]
  table[row sep=crcr]{%
0	0\\
0.00200100050025012	4.16881799906007e-10\\
0.00400200100050025	6.25320408027829e-10\\
0.00600300150075038	4.16881799906007e-10\\
0.0080040020010005	4.16881799906007e-10\\
0.0100050025012506	6.25320408027829e-10\\
0.0120060030015008	4.16881799906007e-10\\
0.0140070035017509	1.04220449976502e-09\\
0.016008004002001	1.66752490779285e-09\\
0.0180090045022511	1.4590885915022e-09\\
0.0200100050025013	1.4590885915022e-09\\
0.0220110055027514	1.04220449976502e-09\\
0.0240120060030015	6.25320408027829e-10\\
0.0260130065032516	6.25320408027829e-10\\
0.0280140070035018	0\\
0.0300150075037519	-2.08440899953003e-10\\
0.032016008004002	-2.08440899953003e-10\\
0.0340170085042521	-2.08440899953003e-10\\
0.0360180090045022	-2.08440899953003e-10\\
0.0380190095047524	-2.08440899953003e-10\\
0.0400200100050025	-2.08440899953003e-10\\
0.0420210105052526	-2.08440899953003e-10\\
0.0440220110055028	-2.08440899953003e-10\\
0.0460230115057529	-2.08440899953003e-10\\
0.048024012006003	-4.16881799906007e-10\\
0.0500250125062531	-4.16881799906007e-10\\
0.0520260130065033	-2.08440899953003e-10\\
0.0540270135067534	-6.25320408027829e-10\\
0.0560280140070035	-4.16881799906007e-10\\
0.0580290145072536	-2.08440899953003e-10\\
0.0600300150075038	-4.16881799906007e-10\\
0.0620310155077539	-6.25320408027829e-10\\
0.064032016008004	-8.33762453896423e-10\\
0.0660330165082541	-8.33762453896423e-10\\
0.0680340170085043	-6.25320408027829e-10\\
0.0700350175087544	-4.16881799906007e-10\\
0.0720360180090045	-2.08440899953003e-10\\
0.0740370185092546	-2.08440899953003e-10\\
0.0760380190095048	-2.08440899953003e-10\\
0.0780390195097549	0\\
0.080040020010005	-4.16881799906007e-10\\
0.0820410205102551	-8.33762453896423e-10\\
0.0840420210105053	-8.33762453896423e-10\\
0.0860430215107554	-1.04220449976502e-09\\
0.0880440220110055	-1.4590885915022e-09\\
0.0900450225112556	-1.4590885915022e-09\\
0.0920460230115058	-1.25064654563361e-09\\
0.0940470235117559	-1.25064654563361e-09\\
0.096048024012006	-1.04220449976502e-09\\
0.0980490245122561	-8.33762453896423e-10\\
0.100050025012506	-8.33762453896423e-10\\
0.102051025512756	-1.4590885915022e-09\\
0.104052026013007	-1.66752490779285e-09\\
0.106053026513257	-1.66752490779285e-09\\
0.108054027013507	-1.66752490779285e-09\\
0.110055027513757	-1.66752490779285e-09\\
0.112056028014007	-1.4590885915022e-09\\
0.114057028514257	-1.4590885915022e-09\\
0.116058029014507	-1.04220449976502e-09\\
0.118059029514757	-1.4590885915022e-09\\
0.120060030015008	-1.4590885915022e-09\\
0.122061030515258	-1.4590885915022e-09\\
0.124062031015508	-1.66752490779285e-09\\
0.126063031515758	-2.08440899953003e-09\\
0.128064032016008	-1.87596695366144e-09\\
0.130065032516258	-2.29285104539863e-09\\
0.132066033016508	-2.08440899953003e-09\\
0.134067033516758	-2.50128736168927e-09\\
0.136068034017009	-2.91817145342646e-09\\
0.138069034517259	-2.70972940755786e-09\\
0.140070035017509	-2.70972940755786e-09\\
0.142071035517759	-2.91817145342646e-09\\
0.144072036018009	-2.29285104539863e-09\\
0.146073036518259	-2.91817145342646e-09\\
0.148074037018509	-2.50128736168927e-09\\
0.150075037518759	-2.91817145342646e-09\\
0.15207603801901	-2.91817145342646e-09\\
0.15407703851926	-2.29285104539863e-09\\
0.15607803901951	-2.29285104539863e-09\\
0.15807903951976	-2.70972940755786e-09\\
0.16008004002001	-2.50128736168927e-09\\
0.16208104052026	-2.50128736168927e-09\\
0.16408204102051	-2.50128736168927e-09\\
0.16608304152076	-2.70972940755786e-09\\
0.168084042021011	-2.50128736168927e-09\\
0.170085042521261	-1.87596695366144e-09\\
0.172086043021511	-2.08440899953003e-09\\
0.174087043521761	-2.08440899953003e-09\\
0.176088044022011	-2.08440899953003e-09\\
0.178089044522261	-2.29285104539863e-09\\
0.180090045022511	-2.50128736168927e-09\\
0.182091045522761	-2.70972940755786e-09\\
0.184092046023012	-2.50128736168927e-09\\
0.186093046523262	-2.29285104539863e-09\\
0.188094047023512	-2.91817145342646e-09\\
0.190095047523762	-2.70972940755786e-09\\
0.192096048024012	-2.91817145342646e-09\\
0.194097048524262	-2.91817145342646e-09\\
0.196098049024512	-2.70972940755786e-09\\
0.198099049524762	-2.91817145342646e-09\\
0.200100050025012	-2.91817145342646e-09\\
0.202101050525263	-2.91817145342646e-09\\
0.204102051025513	-2.70972940755786e-09\\
0.206103051525763	-2.70972940755786e-09\\
0.208104052026013	-2.50128736168927e-09\\
0.210105052526263	-1.87596695366144e-09\\
0.212106053026513	-2.50128736168927e-09\\
0.214107053526763	-2.50128736168927e-09\\
0.216108054027013	-2.91817145342646e-09\\
0.218109054527264	-2.91817145342646e-09\\
0.220110055027514	-2.50128736168927e-09\\
0.222111055527764	-2.50128736168927e-09\\
0.224112056028014	-2.29285104539863e-09\\
0.226113056528264	-2.70972940755786e-09\\
0.228114057028514	-3.12661349929505e-09\\
0.230115057528764	-3.12661349929505e-09\\
0.232116058029014	-2.91817145342646e-09\\
0.234117058529265	-3.12661349929505e-09\\
0.236118059029515	-3.12661349929505e-09\\
0.238119059529765	-3.12661349929505e-09\\
0.240120060030015	-3.33505554516364e-09\\
0.242121060530265	-2.91817145342646e-09\\
0.244122061030515	-3.54349186145428e-09\\
0.246123061530765	-3.12661349929505e-09\\
0.248124062031016	-3.12661349929505e-09\\
0.250125062531266	-3.33505554516364e-09\\
0.252126063031516	-2.91817145342646e-09\\
0.254127063531766	-3.33505554516364e-09\\
0.256128064032016	-3.96037595319147e-09\\
0.258129064532266	-3.96037595319147e-09\\
0.260130065032516	-3.96037595319147e-09\\
0.262131065532766	-3.75193390732288e-09\\
0.264132066033017	-3.75193390732288e-09\\
0.266133066533267	-3.54349186145428e-09\\
0.268134067033517	-3.54349186145428e-09\\
0.270135067533767	-3.33505554516364e-09\\
0.272136068034017	-3.54349186145428e-09\\
0.274137068534267	-3.12661349929505e-09\\
0.276138069034517	-3.12661349929505e-09\\
0.278139069534767	-3.33505554516364e-09\\
0.280140070035018	-3.33505554516364e-09\\
0.282141070535268	-3.54349186145428e-09\\
0.284142071035518	-3.75193390732288e-09\\
0.286143071535768	-3.54349186145428e-09\\
0.288144072036018	-3.54349186145428e-09\\
0.290145072536268	-3.75193390732288e-09\\
0.292146073036518	-3.33505554516364e-09\\
0.294147073536768	-3.33505554516364e-09\\
0.296148074037018	-3.33505554516364e-09\\
0.298149074537269	-3.12661349929505e-09\\
0.300150075037519	-3.33505554516364e-09\\
0.302151075537769	-3.33505554516364e-09\\
0.304152076038019	-3.33505554516364e-09\\
0.306153076538269	-3.54349186145428e-09\\
0.308154077038519	-3.54349186145428e-09\\
0.310155077538769	-3.75193390732288e-09\\
0.31215607803902	-3.75193390732288e-09\\
0.31415707853927	-3.75193390732288e-09\\
0.31615807903952	-3.75193390732288e-09\\
0.31815907953977	-3.54349186145428e-09\\
0.32016008004002	-3.75193390732288e-09\\
0.32216108054027	-3.75193390732288e-09\\
0.32416208104052	-3.75193390732288e-09\\
0.32616308154077	-3.96037595319147e-09\\
0.32816408204102	-3.75193390732288e-09\\
0.330165082541271	-3.75193390732288e-09\\
0.332166083041521	-4.16881799906006e-09\\
0.334167083541771	-3.96037595319147e-09\\
0.336168084042021	-4.37726004492866e-09\\
0.338169084542271	-4.37726004492866e-09\\
0.340170085042521	-3.96037595319147e-09\\
0.342171085542771	-3.75193390732288e-09\\
0.344172086043022	-4.16881799906006e-09\\
0.346173086543272	-4.16881799906006e-09\\
0.348174087043522	-4.16881799906006e-09\\
0.350175087543772	-3.75193390732288e-09\\
0.352176088044022	-3.75193390732288e-09\\
0.354177088544272	-3.75193390732288e-09\\
0.356178089044522	-3.75193390732288e-09\\
0.358179089544772	-3.75193390732288e-09\\
0.360180090045022	-3.75193390732288e-09\\
0.362181090545273	-3.54349186145428e-09\\
0.364182091045523	-3.12661349929505e-09\\
0.366183091545773	-2.91817145342646e-09\\
0.368184092046023	-3.33505554516364e-09\\
0.370185092546273	-3.54349186145428e-09\\
0.372186093046523	-3.75193390732288e-09\\
0.374187093546773	-3.75193390732288e-09\\
0.376188094047024	-3.54349186145428e-09\\
0.378189094547274	-3.54349186145428e-09\\
0.380190095047524	-3.54349186145428e-09\\
0.382191095547774	-3.12661349929505e-09\\
0.384192096048024	-2.70972940755786e-09\\
0.386193096548274	-2.29285104539863e-09\\
0.388194097048524	-1.66752490779285e-09\\
0.390195097548774	-1.66752490779285e-09\\
0.392196098049024	-1.4590885915022e-09\\
0.394197098549275	-1.04220449976502e-09\\
0.396198099049525	-1.4590885915022e-09\\
0.398199099549775	-1.4590885915022e-09\\
0.400200100050025	-1.87596695366144e-09\\
0.402201100550275	-1.66752490779285e-09\\
0.404202101050525	-1.87596695366144e-09\\
0.406203101550775	-1.66752490779285e-09\\
0.408204102051026	-1.4590885915022e-09\\
0.410205102551276	-1.66752490779285e-09\\
0.412206103051526	-1.87596695366144e-09\\
0.414207103551776	-1.4590885915022e-09\\
0.416208104052026	-1.4590885915022e-09\\
0.418209104552276	-1.4590885915022e-09\\
0.420210105052526	-1.25064654563361e-09\\
0.422211105552776	-1.25064654563361e-09\\
0.424212106053027	-6.25320408027829e-10\\
0.426213106553277	-6.25320408027829e-10\\
0.428214107053527	-2.08440899953003e-10\\
0.430215107553777	-2.08440899953003e-10\\
0.432216108054027	-2.08440899953003e-10\\
0.434217108554277	-4.16881799906007e-10\\
0.436218109054527	-2.08440899953003e-10\\
0.438219109554777	0\\
0.440220110055028	2.08440899953003e-10\\
0.442221110555278	0\\
0.444222111055528	-2.08440899953003e-10\\
0.446223111555778	-2.08440899953003e-10\\
0.448224112056028	-6.25320408027829e-10\\
0.450225112556278	-2.08440899953003e-10\\
0.452226113056528	-2.08440899953003e-10\\
0.454227113556778	-4.16881799906007e-10\\
0.456228114057029	0\\
0.458229114557279	4.16881799906007e-10\\
0.460230115057529	2.08440899953003e-10\\
0.462231115557779	-2.08440899953003e-10\\
0.464232116058029	-8.33762453896423e-10\\
0.466233116558279	-1.25064654563361e-09\\
0.468234117058529	-1.04220449976502e-09\\
0.470235117558779	-1.4590885915022e-09\\
0.47223611805903	-1.4590885915022e-09\\
0.47423711855928	-1.25064654563361e-09\\
0.47623811905953	-1.25064654563361e-09\\
0.47823911955978	-1.66752490779285e-09\\
0.48024012006003	-2.08440899953003e-09\\
0.48224112056028	-1.87596695366144e-09\\
0.48424212106053	-1.66752490779285e-09\\
0.48624312156078	-1.4590885915022e-09\\
0.488244122061031	-2.08440899953003e-09\\
0.490245122561281	-1.87596695366144e-09\\
0.492246123061531	-2.08440899953003e-09\\
0.494247123561781	-2.08440899953003e-09\\
0.496248124062031	-2.50128736168927e-09\\
0.498249124562281	-1.87596695366144e-09\\
0.500250125062531	-1.87596695366144e-09\\
0.502251125562781	-2.29285104539863e-09\\
0.504252126063031	-2.08440899953003e-09\\
0.506253126563282	-1.87596695366144e-09\\
0.508254127063532	-1.87596695366144e-09\\
0.510255127563782	-1.66752490779285e-09\\
0.512256128064032	-1.66752490779285e-09\\
0.514257128564282	-1.04220449976502e-09\\
0.516258129064532	-1.25064654563361e-09\\
0.518259129564782	-8.33762453896423e-10\\
0.520260130065032	-6.25320408027829e-10\\
0.522261130565283	-6.25320408027829e-10\\
0.524262131065533	-6.25320408027829e-10\\
0.526263131565783	-6.25320408027829e-10\\
0.528264132066033	-4.16881799906007e-10\\
0.530265132566283	-2.08440899953003e-10\\
0.532266133066533	-2.08440899953003e-10\\
0.534267133566783	-2.08440899953003e-10\\
0.536268134067034	-2.08440899953003e-10\\
0.538269134567284	-2.08440899953003e-10\\
0.540270135067534	-2.08440899953003e-10\\
0.542271135567784	-4.16881799906007e-10\\
0.544272136068034	-8.33762453896423e-10\\
0.546273136568284	-2.08440899953003e-10\\
0.548274137068534	-2.08440899953003e-10\\
0.550275137568784	-2.08440899953003e-10\\
0.552276138069035	-2.08440899953003e-10\\
0.554277138569285	2.08440899953003e-10\\
0.556278139069535	4.16881799906007e-10\\
0.558279139569785	6.25320408027829e-10\\
0.560280140070035	8.33762453896423e-10\\
0.562281140570285	6.25320408027829e-10\\
0.564282141070535	2.08440899953003e-10\\
0.566283141570785	2.08440899953003e-10\\
0.568284142071036	0\\
0.570285142571286	-2.08440899953003e-10\\
0.572286143071536	-2.08440899953003e-10\\
0.574287143571786	-2.08440899953003e-10\\
0.576288144072036	-2.08440899953003e-10\\
0.578289144572286	-2.08440899953003e-10\\
0.580290145072536	-6.25320408027829e-10\\
0.582291145572786	-2.08440899953003e-10\\
0.584292146073036	-4.16881799906007e-10\\
0.586293146573287	-4.16881799906007e-10\\
0.588294147073537	-2.08440899953003e-10\\
0.590295147573787	-2.08440899953003e-10\\
0.592296148074037	2.08440899953003e-10\\
0.594297148574287	2.08440899953003e-10\\
0.596298149074537	0\\
0.598299149574787	2.08440899953003e-10\\
0.600300150075038	2.08440899953003e-10\\
0.602301150575288	2.08440899953003e-10\\
0.604302151075538	0\\
0.606303151575788	-2.08440899953003e-10\\
0.608304152076038	0\\
0.610305152576288	2.08440899953003e-10\\
0.612306153076538	6.25320408027829e-10\\
0.614307153576788	1.04220449976502e-09\\
0.616308154077039	1.25064654563361e-09\\
0.618309154577289	1.4590885915022e-09\\
0.620310155077539	1.04220449976502e-09\\
0.622311155577789	4.16881799906007e-10\\
0.624312156078039	6.25320408027829e-10\\
0.626313156578289	6.25320408027829e-10\\
0.628314157078539	6.25320408027829e-10\\
0.630315157578789	4.16881799906007e-10\\
0.63231615807904	2.08440899953003e-10\\
0.63431715857929	0\\
0.63631815907954	0\\
0.63831915957979	-2.08440899953003e-10\\
0.64032016008004	-8.33762453896423e-10\\
0.64232116058029	-6.25320408027829e-10\\
0.64432216108054	-1.04220449976502e-09\\
0.64632316158079	-8.33762453896423e-10\\
0.64832416208104	-8.33762453896423e-10\\
0.650325162581291	-4.16881799906007e-10\\
0.652326163081541	-6.25320408027829e-10\\
0.654327163581791	-8.33762453896423e-10\\
0.656328164082041	-2.08440899953003e-10\\
0.658329164582291	0\\
0.660330165082541	-2.08440899953003e-10\\
0.662331165582791	-4.16881799906007e-10\\
0.664332166083042	-6.25320408027829e-10\\
0.666333166583292	-6.25320408027829e-10\\
0.668334167083542	-8.33762453896423e-10\\
0.670335167583792	-6.25320408027829e-10\\
0.672336168084042	-2.08440899953003e-10\\
0.674337168584292	-6.25320408027829e-10\\
0.676338169084542	-6.25320408027829e-10\\
0.678339169584792	-6.25320408027829e-10\\
0.680340170085043	-4.16881799906007e-10\\
0.682341170585293	-4.16881799906007e-10\\
0.684342171085543	-6.25320408027829e-10\\
0.686343171585793	-6.25320408027829e-10\\
0.688344172086043	-4.16881799906007e-10\\
0.690345172586293	-2.08440899953003e-10\\
0.692346173086543	-4.16881799906007e-10\\
0.694347173586793	-2.08440899953003e-10\\
0.696348174087044	0\\
0.698349174587294	0\\
0.700350175087544	2.08440899953003e-10\\
0.702351175587794	0\\
0.704352176088044	-2.08440899953003e-10\\
0.706353176588294	-2.08440899953003e-10\\
0.708354177088544	-2.08440899953003e-10\\
0.710355177588794	-2.08440899953003e-10\\
0.712356178089045	0\\
0.714357178589295	-2.08440899953003e-10\\
0.716358179089545	-2.08440899953003e-10\\
0.718359179589795	-2.08440899953003e-10\\
0.720360180090045	-4.16881799906007e-10\\
0.722361180590295	-4.16881799906007e-10\\
0.724362181090545	-2.08440899953003e-10\\
0.726363181590795	-6.25320408027829e-10\\
0.728364182091045	-1.04220449976502e-09\\
0.730365182591296	-8.33762453896423e-10\\
0.732366183091546	-1.04220449976502e-09\\
0.734367183591796	-6.25320408027829e-10\\
0.736368184092046	-6.25320408027829e-10\\
0.738369184592296	-6.25320408027829e-10\\
0.740370185092546	-2.08440899953003e-10\\
0.742371185592796	-6.25320408027829e-10\\
0.744372186093047	-6.25320408027829e-10\\
0.746373186593297	-1.04220449976502e-09\\
0.748374187093547	-6.25320408027829e-10\\
0.750375187593797	-2.08440899953003e-10\\
0.752376188094047	-4.16881799906007e-10\\
0.754377188594297	-4.16881799906007e-10\\
0.756378189094547	-6.25320408027829e-10\\
0.758379189594797	-8.33762453896423e-10\\
0.760380190095048	-1.04220449976502e-09\\
0.762381190595298	-8.33762453896423e-10\\
0.764382191095548	-8.33762453896423e-10\\
0.766383191595798	-6.25320408027829e-10\\
0.768384192096048	-8.33762453896423e-10\\
0.770385192596298	-2.08440899953003e-10\\
0.772386193096548	-6.25320408027829e-10\\
0.774387193596798	-8.33762453896423e-10\\
0.776388194097049	-1.04220449976502e-09\\
0.778389194597299	-1.4590885915022e-09\\
0.780390195097549	-1.4590885915022e-09\\
0.782391195597799	-1.4590885915022e-09\\
0.784392196098049	-1.04220449976502e-09\\
0.786393196598299	-1.25064654563361e-09\\
0.788394197098549	-8.33762453896423e-10\\
0.790395197598799	-1.04220449976502e-09\\
0.792396198099049	-6.25320408027829e-10\\
0.7943971985993	-1.04220449976502e-09\\
0.79639819909955	-1.04220449976502e-09\\
0.7983991995998	-8.33762453896423e-10\\
0.80040020010005	-8.33762453896423e-10\\
0.8024012006003	-1.04220449976502e-09\\
0.80440220110055	-1.25064654563361e-09\\
0.8064032016008	-1.4590885915022e-09\\
0.808404202101051	-1.25064654563361e-09\\
0.810405202601301	-1.25064654563361e-09\\
0.812406203101551	-1.25064654563361e-09\\
0.814407203601801	-1.25064654563361e-09\\
0.816408204102051	-1.04220449976502e-09\\
0.818409204602301	-1.04220449976502e-09\\
0.820410205102551	-1.04220449976502e-09\\
0.822411205602801	-1.25064654563361e-09\\
0.824412206103052	-1.25064654563361e-09\\
0.826413206603302	-1.04220449976502e-09\\
0.828414207103552	-1.25064654563361e-09\\
0.830415207603802	-1.04220449976502e-09\\
0.832416208104052	-1.25064654563361e-09\\
0.834417208604302	-1.4590885915022e-09\\
0.836418209104552	-1.4590885915022e-09\\
0.838419209604802	-1.4590885915022e-09\\
0.840420210105053	-1.25064654563361e-09\\
0.842421210605303	-1.04220449976502e-09\\
0.844422211105553	-1.04220449976502e-09\\
0.846423211605803	-8.33762453896423e-10\\
0.848424212106053	-6.25320408027829e-10\\
0.850425212606303	-4.16881799906007e-10\\
0.852426213106553	-2.08440899953003e-10\\
0.854427213606803	0\\
0.856428214107053	2.08440899953003e-10\\
0.858429214607304	2.08440899953003e-10\\
0.860430215107554	6.25320408027829e-10\\
0.862431215607804	8.33762453896423e-10\\
0.864432216108054	8.33762453896423e-10\\
0.866433216608304	1.04220449976502e-09\\
0.868434217108554	6.25320408027829e-10\\
0.870435217608804	8.33762453896423e-10\\
0.872436218109054	8.33762453896423e-10\\
0.874437218609305	1.04220449976502e-09\\
0.876438219109555	1.25064654563361e-09\\
0.878439219609805	1.04220449976502e-09\\
0.880440220110055	6.25320408027829e-10\\
0.882441220610305	1.04220449976502e-09\\
0.884442221110555	8.33762453896423e-10\\
0.886443221610805	0\\
0.888444222111056	0\\
0.890445222611306	0\\
0.892446223111556	-2.08440899953003e-10\\
0.894447223611806	-4.16881799906007e-10\\
0.896448224112056	-2.08440899953003e-10\\
0.898449224612306	-2.08440899953003e-10\\
0.900450225112556	-2.08440899953003e-10\\
0.902451225612806	-2.08440899953003e-10\\
0.904452226113057	-2.08440899953003e-10\\
0.906453226613307	-2.08440899953003e-10\\
0.908454227113557	-6.25320408027829e-10\\
0.910455227613807	-8.33762453896423e-10\\
0.912456228114057	-1.25064654563361e-09\\
0.914457228614307	-1.04220449976502e-09\\
0.916458229114557	-1.25064654563361e-09\\
0.918459229614807	-1.04220449976502e-09\\
0.920460230115058	-8.33762453896423e-10\\
0.922461230615308	-8.33762453896423e-10\\
0.924462231115558	-8.33762453896423e-10\\
0.926463231615808	-4.16881799906007e-10\\
0.928464232116058	-6.25320408027829e-10\\
0.930465232616308	-6.25320408027829e-10\\
0.932466233116558	-2.08440899953003e-10\\
0.934467233616808	-2.08440899953003e-10\\
0.936468234117058	-2.08440899953003e-10\\
0.938469234617309	-2.08440899953003e-10\\
0.940470235117559	-2.08440899953003e-10\\
0.942471235617809	-6.25320408027829e-10\\
0.944472236118059	-2.08440899953003e-10\\
0.946473236618309	0\\
0.948474237118559	2.08440899953003e-10\\
0.950475237618809	2.08440899953003e-10\\
0.95247623811906	2.08440899953003e-10\\
0.95447723861931	0\\
0.95647823911956	2.08440899953003e-10\\
0.95847923961981	2.08440899953003e-10\\
0.96048024012006	-4.16881799906007e-10\\
0.96248124062031	-6.25320408027829e-10\\
0.96448224112056	-8.33762453896423e-10\\
0.96648324162081	-1.04220449976502e-09\\
0.968484242121061	-1.25064654563361e-09\\
0.970485242621311	-1.25064654563361e-09\\
0.972486243121561	-1.25064654563361e-09\\
0.974487243621811	-1.4590885915022e-09\\
0.976488244122061	-1.25064654563361e-09\\
0.978489244622311	-1.04220449976502e-09\\
0.980490245122561	-6.25320408027829e-10\\
0.982491245622811	-6.25320408027829e-10\\
0.984492246123062	-1.04220449976502e-09\\
0.986493246623312	-1.04220449976502e-09\\
0.988494247123562	-8.33762453896423e-10\\
0.990495247623812	-4.16881799906007e-10\\
0.992496248124062	-6.25320408027829e-10\\
0.994497248624312	-8.33762453896423e-10\\
0.996498249124562	-1.04220449976502e-09\\
0.998499249624812	-6.25320408027829e-10\\
1.00050025012506	-2.08440899953003e-10\\
1.00250125062531	-2.08440899953003e-10\\
1.00450225112556	-2.08440899953003e-10\\
1.00650325162581	-2.08440899953003e-10\\
1.00850425212606	0\\
1.01050525262631	-2.08440899953003e-10\\
1.01250625312656	2.08440899953003e-10\\
1.01450725362681	4.16881799906007e-10\\
1.01650825412706	2.08440899953003e-10\\
1.01850925462731	4.16881799906007e-10\\
1.02051025512756	4.16881799906007e-10\\
1.02251125562781	6.25320408027829e-10\\
1.02451225612806	1.04220449976502e-09\\
1.02651325662831	4.16881799906007e-10\\
1.02851425712856	2.08440899953003e-10\\
1.03051525762881	-2.08440899953003e-10\\
1.03251625812906	2.08440899953003e-10\\
1.03451725862931	-2.08440899953003e-10\\
1.03651825912956	0\\
1.03851925962981	0\\
1.04052026013006	4.16881799906007e-10\\
1.04252126063032	2.08440899953003e-10\\
1.04452226113057	6.25320408027829e-10\\
1.04652326163082	6.25320408027829e-10\\
1.04852426213107	8.33762453896423e-10\\
1.05052526263132	4.16881799906007e-10\\
1.05252626313157	1.04220449976502e-09\\
1.05452726363182	4.16881799906007e-10\\
1.05652826413207	6.25320408027829e-10\\
1.05852926463232	6.25320408027829e-10\\
1.06053026513257	8.33762453896423e-10\\
1.06253126563282	1.25064654563361e-09\\
1.06453226613307	1.66752490779285e-09\\
1.06653326663332	2.08440899953003e-09\\
1.06853426713357	2.08440899953003e-09\\
1.07053526763382	1.87596695366144e-09\\
1.07253626813407	2.29285104539863e-09\\
1.07453726863432	2.29285104539863e-09\\
1.07653826913457	2.91817145342646e-09\\
1.07853926963482	2.50128736168927e-09\\
1.08054027013507	2.29285104539863e-09\\
1.08254127063532	2.29285104539863e-09\\
1.08454227113557	1.87596695366144e-09\\
1.08654327163582	1.66752490779285e-09\\
1.08854427213607	1.66752490779285e-09\\
1.09054527263632	1.4590885915022e-09\\
1.09254627313657	8.33762453896423e-10\\
1.09454727363682	1.25064654563361e-09\\
1.09654827413707	1.66752490779285e-09\\
1.09854927463732	1.87596695366144e-09\\
1.10055027513757	1.4590885915022e-09\\
1.10255127563782	1.4590885915022e-09\\
1.10455227613807	1.25064654563361e-09\\
1.10655327663832	1.25064654563361e-09\\
1.10855427713857	1.25064654563361e-09\\
1.11055527763882	1.04220449976502e-09\\
1.11255627813907	1.04220449976502e-09\\
1.11455727863932	1.04220449976502e-09\\
1.11655827913957	1.25064654563361e-09\\
1.11855927963982	1.25064654563361e-09\\
1.12056028014007	1.04220449976502e-09\\
1.12256128064032	1.25064654563361e-09\\
1.12456228114057	8.33762453896423e-10\\
1.12656328164082	6.25320408027829e-10\\
1.12856428214107	1.04220449976502e-09\\
1.13056528264132	8.33762453896423e-10\\
1.13256628314157	8.33762453896423e-10\\
1.13456728364182	6.25320408027829e-10\\
1.13656828414207	4.16881799906007e-10\\
1.13856928464232	1.04220449976502e-09\\
1.14057028514257	1.04220449976502e-09\\
1.14257128564282	8.33762453896423e-10\\
1.14457228614307	6.25320408027829e-10\\
1.14657328664332	4.16881799906007e-10\\
1.14857428714357	2.08440899953003e-10\\
1.15057528764382	2.08440899953003e-10\\
1.15257628814407	4.16881799906007e-10\\
1.15457728864432	6.25320408027829e-10\\
1.15657828914457	8.33762453896423e-10\\
1.15857928964482	8.33762453896423e-10\\
1.16058029014507	1.04220449976502e-09\\
1.16258129064532	1.87596695366144e-09\\
1.16458229114557	2.08440899953003e-09\\
1.16658329164582	1.66752490779285e-09\\
1.16858429214607	1.66752490779285e-09\\
1.17058529264632	1.25064654563361e-09\\
1.17258629314657	1.25064654563361e-09\\
1.17458729364682	1.4590885915022e-09\\
1.17658829414707	2.08440899953003e-09\\
1.17858929464732	2.29285104539863e-09\\
1.18059029514757	2.70972940755786e-09\\
1.18259129564782	2.91817145342646e-09\\
1.18459229614807	2.50128736168927e-09\\
1.18659329664832	2.70972940755786e-09\\
1.18859429714857	2.70972940755786e-09\\
1.19059529764882	2.91817145342646e-09\\
1.19259629814907	2.50128736168927e-09\\
1.19459729864932	2.29285104539863e-09\\
1.19659829914957	2.29285104539863e-09\\
1.19859929964982	2.29285104539863e-09\\
1.20060030015008	1.87596695366144e-09\\
1.20260130065033	2.08440899953003e-09\\
1.20460230115058	1.87596695366144e-09\\
1.20660330165083	2.08440899953003e-09\\
1.20860430215108	2.08440899953003e-09\\
1.21060530265133	1.87596695366144e-09\\
1.21260630315158	1.25064654563361e-09\\
1.21460730365183	1.66752490779285e-09\\
1.21660830415208	1.66752490779285e-09\\
1.21860930465233	1.4590885915022e-09\\
1.22061030515258	1.4590885915022e-09\\
1.22261130565283	1.66752490779285e-09\\
1.22461230615308	1.87596695366144e-09\\
1.22661330665333	1.66752490779285e-09\\
1.22861430715358	1.87596695366144e-09\\
1.23061530765383	2.08440899953003e-09\\
1.23261630815408	1.87596695366144e-09\\
1.23461730865433	1.66752490779285e-09\\
1.23661830915458	2.08440899953003e-09\\
1.23861930965483	2.50128736168927e-09\\
1.24062031015508	2.50128736168927e-09\\
1.24262131065533	2.29285104539863e-09\\
1.24462231115558	2.29285104539863e-09\\
1.24662331165583	2.70972940755786e-09\\
1.24862431215608	2.50128736168927e-09\\
1.25062531265633	2.50128736168927e-09\\
1.25262631315658	2.50128736168927e-09\\
1.25462731365683	2.70972940755786e-09\\
1.25662831415708	2.91817145342646e-09\\
1.25862931465733	3.12661349929505e-09\\
1.26063031515758	3.12661349929505e-09\\
1.26263131565783	2.91817145342646e-09\\
1.26463231615808	2.91817145342646e-09\\
1.26663331665833	2.91817145342646e-09\\
1.26863431715858	3.33505554516364e-09\\
1.27063531765883	3.12661349929505e-09\\
1.27263631815908	3.33505554516364e-09\\
1.27463731865933	3.33505554516364e-09\\
1.27663831915958	3.54349186145428e-09\\
1.27863931965983	3.75193390732288e-09\\
1.28064032016008	3.54349186145428e-09\\
1.28264132066033	3.75193390732288e-09\\
1.28464232116058	3.54349186145428e-09\\
1.28664332166083	3.33505554516364e-09\\
1.28864432216108	3.33505554516364e-09\\
1.29064532266133	3.12661349929505e-09\\
1.29264632316158	3.54349186145428e-09\\
1.29464732366183	3.54349186145428e-09\\
1.29664832416208	3.75193390732288e-09\\
1.29864932466233	3.33505554516364e-09\\
1.30065032516258	3.12661349929505e-09\\
1.30265132566283	3.33505554516364e-09\\
1.30465232616308	3.75193390732288e-09\\
1.30665332666333	3.33505554516364e-09\\
1.30865432716358	3.75193390732288e-09\\
1.31065532766383	3.54349186145428e-09\\
1.31265632816408	3.75193390732288e-09\\
1.31465732866433	3.75193390732288e-09\\
1.31665832916458	3.75193390732288e-09\\
1.31865932966483	3.54349186145428e-09\\
1.32066033016508	3.75193390732288e-09\\
1.32266133066533	3.75193390732288e-09\\
1.32466233116558	3.33505554516364e-09\\
1.32666333166583	3.54349186145428e-09\\
1.32866433216608	3.54349186145428e-09\\
1.33066533266633	3.12661349929505e-09\\
1.33266633316658	3.12661349929505e-09\\
1.33466733366683	2.91817145342646e-09\\
1.33666833416708	2.91817145342646e-09\\
1.33866933466733	3.12661349929505e-09\\
1.34067033516758	3.12661349929505e-09\\
1.34267133566783	3.12661349929505e-09\\
1.34467233616808	3.54349186145428e-09\\
1.34667333666833	3.54349186145428e-09\\
1.34867433716858	3.12661349929505e-09\\
1.35067533766883	3.12661349929505e-09\\
1.35267633816908	3.12661349929505e-09\\
1.35467733866933	3.12661349929505e-09\\
1.35667833916958	2.70972940755786e-09\\
1.35867933966983	2.70972940755786e-09\\
1.36068034017009	2.29285104539863e-09\\
1.36268134067034	2.50128736168927e-09\\
1.36468234117059	2.50128736168927e-09\\
1.36668334167084	2.29285104539863e-09\\
1.36868434217109	2.08440899953003e-09\\
1.37068534267134	2.29285104539863e-09\\
1.37268634317159	2.70972940755786e-09\\
1.37468734367184	2.70972940755786e-09\\
1.37668834417209	2.91817145342646e-09\\
1.37868934467234	3.12661349929505e-09\\
1.38069034517259	3.12661349929505e-09\\
1.38269134567284	3.12661349929505e-09\\
1.38469234617309	3.33505554516364e-09\\
1.38669334667334	2.70972940755786e-09\\
1.38869434717359	2.91817145342646e-09\\
1.39069534767384	3.12661349929505e-09\\
1.39269634817409	3.54349186145428e-09\\
1.39469734867434	3.54349186145428e-09\\
1.39669834917459	3.33505554516364e-09\\
1.39869934967484	3.12661349929505e-09\\
1.40070035017509	3.12661349929505e-09\\
1.40270135067534	3.12661349929505e-09\\
1.40470235117559	3.12661349929505e-09\\
1.40670335167584	3.33505554516364e-09\\
1.40870435217609	3.12661349929505e-09\\
1.41070535267634	2.70972940755786e-09\\
1.41270635317659	2.70972940755786e-09\\
1.41470735367684	2.50128736168927e-09\\
1.41670835417709	2.70972940755786e-09\\
1.41870935467734	2.50128736168927e-09\\
1.42071035517759	2.70972940755786e-09\\
1.42271135567784	2.91817145342646e-09\\
1.42471235617809	2.70972940755786e-09\\
1.42671335667834	2.70972940755786e-09\\
1.42871435717859	2.29285104539863e-09\\
1.43071535767884	2.08440899953003e-09\\
1.43271635817909	2.29285104539863e-09\\
1.43471735867934	2.08440899953003e-09\\
1.43671835917959	1.66752490779285e-09\\
1.43871935967984	1.87596695366144e-09\\
1.44072036018009	1.87596695366144e-09\\
1.44272136068034	1.87596695366144e-09\\
1.44472236118059	2.08440899953003e-09\\
1.44672336168084	2.29285104539863e-09\\
1.44872436218109	2.29285104539863e-09\\
1.45072536268134	2.08440899953003e-09\\
1.45272636318159	2.29285104539863e-09\\
1.45472736368184	2.08440899953003e-09\\
1.45672836418209	1.87596695366144e-09\\
1.45872936468234	2.08440899953003e-09\\
1.46073036518259	1.66752490779285e-09\\
1.46273136568284	1.4590885915022e-09\\
1.46473236618309	1.25064654563361e-09\\
1.46673336668334	8.33762453896423e-10\\
1.46873436718359	1.04220449976502e-09\\
1.47073536768384	8.33762453896423e-10\\
1.47273636818409	1.04220449976502e-09\\
1.47473736868434	8.33762453896423e-10\\
1.47673836918459	6.25320408027829e-10\\
1.47873936968484	4.16881799906007e-10\\
1.48074037018509	0\\
1.48274137068534	-2.08440899953003e-10\\
1.48474237118559	0\\
1.48674337168584	0\\
1.48874437218609	-2.08440899953003e-10\\
1.49074537268634	-2.08440899953003e-10\\
1.49274637318659	-4.16881799906007e-10\\
1.49474737368684	-2.08440899953003e-10\\
1.49674837418709	0\\
1.49874937468734	2.08440899953003e-10\\
1.50075037518759	0\\
1.50275137568784	2.08440899953003e-10\\
1.50475237618809	0\\
1.50675337668834	-2.08440899953003e-10\\
1.50875437718859	-4.16881799906007e-10\\
1.51075537768884	-4.16881799906007e-10\\
1.51275637818909	-6.25320408027829e-10\\
1.51475737868934	-1.04220449976502e-09\\
1.51675837918959	-8.33762453896423e-10\\
1.51875937968984	-6.25320408027829e-10\\
1.5207603801901	-4.16881799906007e-10\\
1.52276138069035	-4.16881799906007e-10\\
1.5247623811906	-6.25320408027829e-10\\
1.52676338169085	-4.16881799906007e-10\\
1.5287643821911	-4.16881799906007e-10\\
1.53076538269135	-4.16881799906007e-10\\
1.5327663831916	-2.08440899953003e-10\\
1.53476738369185	-6.25320408027829e-10\\
1.5367683841921	-4.16881799906007e-10\\
1.53876938469235	-6.25320408027829e-10\\
1.5407703851926	-1.04220449976502e-09\\
1.54277138569285	-1.25064654563361e-09\\
1.5447723861931	-1.25064654563361e-09\\
1.54677338669335	-1.4590885915022e-09\\
1.5487743871936	-1.66752490779285e-09\\
1.55077538769385	-2.08440899953003e-09\\
1.5527763881941	-2.29285104539863e-09\\
1.55477738869435	-2.29285104539863e-09\\
1.5567783891946	-2.29285104539863e-09\\
1.55877938969485	-2.50128736168927e-09\\
1.5607803901951	-2.70972940755786e-09\\
1.56278139069535	-2.91817145342646e-09\\
1.5647823911956	-2.91817145342646e-09\\
1.56678339169585	-3.12661349929505e-09\\
1.5687843921961	-3.12661349929505e-09\\
1.57078539269635	-2.91817145342646e-09\\
1.5727863931966	-3.33505554516364e-09\\
1.57478739369685	-3.54349186145428e-09\\
1.5767883941971	-3.75193390732288e-09\\
1.57878939469735	-3.75193390732288e-09\\
1.5807903951976	-3.96037595319147e-09\\
1.58279139569785	-3.75193390732288e-09\\
1.5847923961981	-3.96037595319147e-09\\
1.58679339669835	-3.96037595319147e-09\\
1.5887943971986	-3.75193390732288e-09\\
1.59079539769885	-3.75193390732288e-09\\
1.5927963981991	-3.75193390732288e-09\\
1.59479739869935	-3.75193390732288e-09\\
1.5967983991996	-3.33505554516364e-09\\
1.59879939969985	-2.91817145342646e-09\\
1.6008004002001	-2.70972940755786e-09\\
1.60280140070035	-2.50128736168927e-09\\
1.6048024012006	-2.08440899953003e-09\\
1.60680340170085	-2.29285104539863e-09\\
1.6088044022011	-2.08440899953003e-09\\
1.61080540270135	-2.08440899953003e-09\\
1.6128064032016	-2.08440899953003e-09\\
1.61480740370185	-2.08440899953003e-09\\
1.6168084042021	-1.4590885915022e-09\\
1.61880940470235	-1.25064654563361e-09\\
1.6208104052026	-1.04220449976502e-09\\
1.62281140570285	-8.33762453896423e-10\\
1.6248124062031	-1.04220449976502e-09\\
1.62681340670335	-8.33762453896423e-10\\
1.6288144072036	-1.25064654563361e-09\\
1.63081540770385	-1.25064654563361e-09\\
1.6328164082041	-1.04220449976502e-09\\
1.63481740870435	-1.04220449976502e-09\\
1.6368184092046	-1.25064654563361e-09\\
1.63881940970485	-1.4590885915022e-09\\
1.6408204102051	-1.4590885915022e-09\\
1.64282141070535	-1.25064654563361e-09\\
1.6448224112056	-1.4590885915022e-09\\
1.64682341170585	-8.33762453896423e-10\\
1.6488244122061	-6.25320408027829e-10\\
1.65082541270635	-6.25320408027829e-10\\
1.6528264132066	-6.25320408027829e-10\\
1.65482741370685	-8.33762453896423e-10\\
1.6568284142071	-6.25320408027829e-10\\
1.65882941470735	-6.25320408027829e-10\\
1.6608304152076	-4.16881799906007e-10\\
1.66283141570785	-2.08440899953003e-10\\
1.6648324162081	0\\
1.66683341670835	0\\
1.6688344172086	-2.08440899953003e-10\\
1.67083541770885	-2.08440899953003e-10\\
1.6728364182091	-2.08440899953003e-10\\
1.67483741870935	-2.08440899953003e-10\\
1.6768384192096	-2.08440899953003e-10\\
1.67883941970985	-2.08440899953003e-10\\
1.68084042021011	-2.08440899953003e-10\\
1.68284142071036	-2.08440899953003e-10\\
1.68484242121061	-6.25320408027829e-10\\
1.68684342171086	-4.16881799906007e-10\\
1.68884442221111	-4.16881799906007e-10\\
1.69084542271136	-6.25320408027829e-10\\
1.69284642321161	-6.25320408027829e-10\\
1.69484742371186	-2.08440899953003e-10\\
1.69684842421211	0\\
1.69884942471236	4.16881799906007e-10\\
1.70085042521261	8.33762453896423e-10\\
1.70285142571286	6.25320408027829e-10\\
1.70485242621311	8.33762453896423e-10\\
1.70685342671336	8.33762453896423e-10\\
1.70885442721361	1.04220449976502e-09\\
1.71085542771386	1.04220449976502e-09\\
1.71285642821411	1.25064654563361e-09\\
1.71485742871436	1.66752490779285e-09\\
1.71685842921461	1.4590885915022e-09\\
1.71885942971486	1.25064654563361e-09\\
1.72086043021511	1.25064654563361e-09\\
1.72286143071536	1.4590885915022e-09\\
1.72486243121561	1.04220449976502e-09\\
1.72686343171586	1.4590885915022e-09\\
1.72886443221611	1.4590885915022e-09\\
1.73086543271636	1.04220449976502e-09\\
1.73286643321661	1.25064654563361e-09\\
1.73486743371686	8.33762453896423e-10\\
1.73686843421711	8.33762453896423e-10\\
1.73886943471736	8.33762453896423e-10\\
1.74087043521761	6.25320408027829e-10\\
1.74287143571786	4.16881799906007e-10\\
1.74487243621811	6.25320408027829e-10\\
1.74687343671836	4.16881799906007e-10\\
1.74887443721861	2.08440899953003e-10\\
1.75087543771886	6.25320408027829e-10\\
1.75287643821911	4.16881799906007e-10\\
1.75487743871936	-2.08440899953003e-10\\
1.75687843921961	-4.16881799906007e-10\\
1.75887943971986	-2.08440899953003e-10\\
1.76088044022011	-4.16881799906007e-10\\
1.76288144072036	-2.08440899953003e-10\\
1.76488244122061	-2.08440899953003e-10\\
1.76688344172086	-2.08440899953003e-10\\
1.76888444222111	-2.08440899953003e-10\\
1.77088544272136	0\\
1.77288644322161	0\\
1.77488744372186	-2.08440899953003e-10\\
1.77688844422211	-2.08440899953003e-10\\
1.77888944472236	-2.08440899953003e-10\\
1.78089044522261	-4.16881799906007e-10\\
1.78289144572286	-6.25320408027829e-10\\
1.78489244622311	-4.16881799906007e-10\\
1.78689344672336	-6.25320408027829e-10\\
1.78889444722361	-4.16881799906007e-10\\
1.79089544772386	-6.25320408027829e-10\\
1.79289644822411	-6.25320408027829e-10\\
1.79489744872436	-8.33762453896423e-10\\
1.79689844922461	-6.25320408027829e-10\\
1.79889944972486	-8.33762453896423e-10\\
1.80090045022511	-8.33762453896423e-10\\
1.80290145072536	-4.16881799906007e-10\\
1.80490245122561	-2.08440899953003e-10\\
1.80690345172586	-2.08440899953003e-10\\
1.80890445222611	-2.08440899953003e-10\\
1.81090545272636	-6.25320408027829e-10\\
1.81290645322661	-6.25320408027829e-10\\
1.81490745372686	-6.25320408027829e-10\\
1.81690845422711	-6.25320408027829e-10\\
1.81890945472736	-6.25320408027829e-10\\
1.82091045522761	-2.08440899953003e-10\\
1.82291145572786	-2.08440899953003e-10\\
1.82491245622811	-4.16881799906007e-10\\
1.82691345672836	-4.16881799906007e-10\\
1.82891445722861	-4.16881799906007e-10\\
1.83091545772886	-4.16881799906007e-10\\
1.83291645822911	0\\
1.83491745872936	-2.08440899953003e-10\\
1.83691845922961	0\\
1.83891945972987	-2.08440899953003e-10\\
1.84092046023012	-4.16881799906007e-10\\
1.84292146073037	-2.08440899953003e-10\\
1.84492246123062	0\\
1.84692346173087	0\\
1.84892446223112	-2.08440899953003e-10\\
1.85092546273137	0\\
1.85292646323162	-2.08440899953003e-10\\
1.85492746373187	2.08440899953003e-10\\
1.85692846423212	-2.08440899953003e-10\\
1.85892946473237	-2.08440899953003e-10\\
1.86093046523262	-2.08440899953003e-10\\
1.86293146573287	-2.08440899953003e-10\\
1.86493246623312	-2.08440899953003e-10\\
1.86693346673337	-4.16881799906007e-10\\
1.86893446723362	-2.08440899953003e-10\\
1.87093546773387	-2.08440899953003e-10\\
1.87293646823412	-6.25320408027829e-10\\
1.87493746873437	-8.33762453896423e-10\\
1.87693846923462	-4.16881799906007e-10\\
1.87893946973487	-2.08440899953003e-10\\
1.88094047023512	-2.08440899953003e-10\\
1.88294147073537	0\\
1.88494247123562	0\\
1.88694347173587	-2.08440899953003e-10\\
1.88894447223612	-4.16881799906007e-10\\
1.89094547273637	-6.25320408027829e-10\\
1.89294647323662	-8.33762453896423e-10\\
1.89494747373687	-4.16881799906007e-10\\
1.89694847423712	-4.16881799906007e-10\\
1.89894947473737	-4.16881799906007e-10\\
1.90095047523762	-2.08440899953003e-10\\
1.90295147573787	4.16881799906007e-10\\
1.90495247623812	0\\
1.90695347673837	-2.08440899953003e-10\\
1.90895447723862	-2.08440899953003e-10\\
1.91095547773887	-2.08440899953003e-10\\
1.91295647823912	-4.16881799906007e-10\\
1.91495747873937	-8.33762453896423e-10\\
1.91695847923962	-1.04220449976502e-09\\
1.91895947973987	-8.33762453896423e-10\\
1.92096048024012	-1.25064654563361e-09\\
1.92296148074037	-1.25064654563361e-09\\
1.92496248124062	-1.25064654563361e-09\\
1.92696348174087	-1.04220449976502e-09\\
1.92896448224112	-1.04220449976502e-09\\
1.93096548274137	-1.04220449976502e-09\\
1.93296648324162	-1.04220449976502e-09\\
1.93496748374187	-8.33762453896423e-10\\
1.93696848424212	-4.16881799906007e-10\\
1.93896948474237	-6.25320408027829e-10\\
1.94097048524262	-1.04220449976502e-09\\
1.94297148574287	-1.25064654563361e-09\\
1.94497248624312	-1.25064654563361e-09\\
1.94697348674337	-1.25064654563361e-09\\
1.94897448724362	-1.04220449976502e-09\\
1.95097548774387	-8.33762453896423e-10\\
1.95297648824412	-8.33762453896423e-10\\
1.95497748874437	-6.25320408027829e-10\\
1.95697848924462	-6.25320408027829e-10\\
1.95897948974487	-8.33762453896423e-10\\
1.96098049024512	-6.25320408027829e-10\\
1.96298149074537	-4.16881799906007e-10\\
1.96498249124562	-4.16881799906007e-10\\
1.96698349174587	-6.25320408027829e-10\\
1.96898449224612	-8.33762453896423e-10\\
1.97098549274637	-1.25064654563361e-09\\
1.97298649324662	-1.04220449976502e-09\\
1.97498749374687	-8.33762453896423e-10\\
1.97698849424712	-1.25064654563361e-09\\
1.97898949474737	-1.25064654563361e-09\\
1.98099049524762	-1.04220449976502e-09\\
1.98299149574787	-1.4590885915022e-09\\
1.98499249624812	-8.33762453896423e-10\\
1.98699349674837	-6.25320408027829e-10\\
1.98899449724862	-6.25320408027829e-10\\
1.99099549774887	-1.04220449976502e-09\\
1.99299649824912	-1.04220449976502e-09\\
1.99499749874937	-1.04220449976502e-09\\
1.99699849924962	-8.33762453896423e-10\\
1.99899949974988	-8.33762453896423e-10\\
2.00100050025013	-4.16881799906007e-10\\
2.00300150075038	-2.08440899953003e-10\\
2.00500250125063	0\\
2.00700350175088	-2.08440899953003e-10\\
2.00900450225113	0\\
2.01100550275138	0\\
2.01300650325163	0\\
2.01500750375188	0\\
2.01700850425213	-2.08440899953003e-10\\
2.01900950475238	-4.16881799906007e-10\\
2.02101050525263	-2.08440899953003e-10\\
2.02301150575288	-2.08440899953003e-10\\
2.02501250625313	-2.08440899953003e-10\\
2.02701350675338	-2.08440899953003e-10\\
2.02901450725363	0\\
2.03101550775388	-2.08440899953003e-10\\
2.03301650825413	0\\
2.03501750875438	-2.08440899953003e-10\\
2.03701850925463	-4.16881799906007e-10\\
2.03901950975488	-4.16881799906007e-10\\
2.04102051025513	-1.04220449976502e-09\\
2.04302151075538	-8.33762453896423e-10\\
2.04502251125563	-8.33762453896423e-10\\
2.04702351175588	-8.33762453896423e-10\\
2.04902451225613	-6.25320408027829e-10\\
2.05102551275638	-2.08440899953003e-10\\
2.05302651325663	-2.08440899953003e-10\\
2.05502751375688	-2.08440899953003e-10\\
2.05702851425713	-4.16881799906007e-10\\
2.05902951475738	-4.16881799906007e-10\\
2.06103051525763	-2.08440899953003e-10\\
2.06303151575788	-2.08440899953003e-10\\
2.06503251625813	-6.25320408027829e-10\\
2.06703351675838	-2.08440899953003e-10\\
2.06903451725863	2.08440899953003e-10\\
2.07103551775888	0\\
2.07303651825913	2.08440899953003e-10\\
2.07503751875938	0\\
2.07703851925963	0\\
2.07903951975988	-2.08440899953003e-10\\
2.08104052026013	-2.08440899953003e-10\\
2.08304152076038	-2.08440899953003e-10\\
2.08504252126063	-2.08440899953003e-10\\
2.08704352176088	-6.25320408027829e-10\\
2.08904452226113	-2.08440899953003e-10\\
2.09104552276138	-2.08440899953003e-10\\
2.09304652326163	-2.08440899953003e-10\\
2.09504752376188	2.08440899953003e-10\\
2.09704852426213	2.08440899953003e-10\\
2.09904952476238	2.08440899953003e-10\\
2.10105052526263	6.25320408027829e-10\\
2.10305152576288	6.25320408027829e-10\\
2.10505252626313	6.25320408027829e-10\\
2.10705352676338	6.25320408027829e-10\\
2.10905452726363	4.16881799906007e-10\\
2.11105552776388	4.16881799906007e-10\\
2.11305652826413	2.08440899953003e-10\\
2.11505752876438	0\\
2.11705852926463	-2.08440899953003e-10\\
2.11905952976488	-4.16881799906007e-10\\
2.12106053026513	-2.08440899953003e-10\\
2.12306153076538	-6.25320408027829e-10\\
2.12506253126563	-6.25320408027829e-10\\
2.12706353176588	-4.16881799906007e-10\\
2.12906453226613	-2.08440899953003e-10\\
2.13106553276638	0\\
2.13306653326663	4.16881799906007e-10\\
2.13506753376688	6.25320408027829e-10\\
2.13706853426713	1.04220449976502e-09\\
2.13906953476738	1.04220449976502e-09\\
2.14107053526763	8.33762453896423e-10\\
2.14307153576788	8.33762453896423e-10\\
2.14507253626813	4.16881799906007e-10\\
2.14707353676838	0\\
2.14907453726863	0\\
2.15107553776888	6.25320408027829e-10\\
2.15307653826913	6.25320408027829e-10\\
2.15507753876938	6.25320408027829e-10\\
2.15707853926963	6.25320408027829e-10\\
2.15907953976988	4.16881799906007e-10\\
2.16108054027013	2.08440899953003e-10\\
2.16308154077039	-2.08440899953003e-10\\
2.16508254127064	-2.08440899953003e-10\\
2.16708354177089	-2.08440899953003e-10\\
2.16908454227114	-2.08440899953003e-10\\
2.17108554277139	-2.08440899953003e-10\\
2.17308654327164	-6.25320408027829e-10\\
2.17508754377189	-4.16881799906007e-10\\
2.17708854427214	-4.16881799906007e-10\\
2.17908954477239	-8.33762453896423e-10\\
2.18109054527264	-6.25320408027829e-10\\
2.18309154577289	-6.25320408027829e-10\\
2.18509254627314	-8.33762453896423e-10\\
2.18709354677339	-6.25320408027829e-10\\
2.18909454727364	-1.04220449976502e-09\\
2.19109554777389	-1.04220449976502e-09\\
2.19309654827414	-2.08440899953003e-10\\
2.19509754877439	2.08440899953003e-10\\
2.19709854927464	4.16881799906007e-10\\
2.19909954977489	2.08440899953003e-10\\
2.20110055027514	4.16881799906007e-10\\
2.20310155077539	1.04220449976502e-09\\
2.20510255127564	1.04220449976502e-09\\
2.20710355177589	1.25064654563361e-09\\
2.20910455227614	1.25064654563361e-09\\
2.21110555277639	1.04220449976502e-09\\
2.21310655327664	1.66752490779285e-09\\
2.21510755377689	1.66752490779285e-09\\
2.21710855427714	1.66752490779285e-09\\
2.21910955477739	1.4590885915022e-09\\
2.22111055527764	1.66752490779285e-09\\
2.22311155577789	1.4590885915022e-09\\
2.22511255627814	1.66752490779285e-09\\
2.22711355677839	1.04220449976502e-09\\
2.22911455727864	1.25064654563361e-09\\
2.23111555777889	1.4590885915022e-09\\
2.23311655827914	1.25064654563361e-09\\
2.23511755877939	8.33762453896423e-10\\
2.23711855927964	1.04220449976502e-09\\
2.23911955977989	1.25064654563361e-09\\
2.24112056028014	1.04220449976502e-09\\
2.24312156078039	4.16881799906007e-10\\
2.24512256128064	1.04220449976502e-09\\
2.24712356178089	8.33762453896423e-10\\
2.24912456228114	8.33762453896423e-10\\
2.25112556278139	1.04220449976502e-09\\
2.25312656328164	1.04220449976502e-09\\
2.25512756378189	1.25064654563361e-09\\
2.25712856428214	1.4590885915022e-09\\
2.25912956478239	1.4590885915022e-09\\
2.26113056528264	1.4590885915022e-09\\
2.26313156578289	1.4590885915022e-09\\
2.26513256628314	1.4590885915022e-09\\
2.26713356678339	1.4590885915022e-09\\
2.26913456728364	1.4590885915022e-09\\
2.27113556778389	1.25064654563361e-09\\
2.27313656828414	1.04220449976502e-09\\
2.27513756878439	1.25064654563361e-09\\
2.27713856928464	1.25064654563361e-09\\
2.27913956978489	1.25064654563361e-09\\
2.28114057028514	1.4590885915022e-09\\
2.28314157078539	1.4590885915022e-09\\
2.28514257128564	1.04220449976502e-09\\
2.28714357178589	4.16881799906007e-10\\
2.28914457228614	0\\
2.29114557278639	2.08440899953003e-10\\
2.29314657328664	0\\
2.29514757378689	-2.08440899953003e-10\\
2.29714857428714	0\\
2.29914957478739	4.16881799906007e-10\\
2.30115057528764	4.16881799906007e-10\\
2.30315157578789	8.33762453896423e-10\\
2.30515257628814	6.25320408027829e-10\\
2.30715357678839	4.16881799906007e-10\\
2.30915457728864	0\\
2.31115557778889	4.16881799906007e-10\\
2.31315657828914	8.33762453896423e-10\\
2.31515757878939	1.25064654563361e-09\\
2.31715857928964	1.66752490779285e-09\\
2.31915957978989	1.66752490779285e-09\\
2.32116058029015	1.4590885915022e-09\\
2.3231615807904	1.25064654563361e-09\\
2.32516258129065	6.25320408027829e-10\\
2.3271635817909	6.25320408027829e-10\\
2.32916458229115	8.33762453896423e-10\\
2.3311655827914	6.25320408027829e-10\\
2.33316658329165	1.04220449976502e-09\\
2.3351675837919	1.04220449976502e-09\\
2.33716858429215	1.25064654563361e-09\\
2.3391695847924	1.04220449976502e-09\\
2.34117058529265	8.33762453896423e-10\\
2.3431715857929	4.16881799906007e-10\\
2.34517258629315	8.33762453896423e-10\\
2.3471735867934	1.04220449976502e-09\\
2.34917458729365	1.4590885915022e-09\\
2.3511755877939	1.66752490779285e-09\\
2.35317658829415	1.87596695366144e-09\\
2.3551775887944	1.4590885915022e-09\\
2.35717858929465	1.66752490779285e-09\\
2.3591795897949	2.08440899953003e-09\\
2.36118059029515	2.50128736168927e-09\\
2.3631815907954	2.50128736168927e-09\\
2.36518259129565	2.50128736168927e-09\\
2.3671835917959	2.91817145342646e-09\\
2.36918459229615	2.70972940755786e-09\\
2.3711855927964	2.91817145342646e-09\\
2.37318659329665	2.29285104539863e-09\\
2.3751875937969	2.29285104539863e-09\\
2.37718859429715	2.29285104539863e-09\\
2.3791895947974	2.50128736168927e-09\\
2.38119059529765	2.29285104539863e-09\\
2.3831915957979	2.70972940755786e-09\\
2.38519259629815	3.12661349929505e-09\\
2.3871935967984	2.70972940755786e-09\\
2.38919459729865	2.91817145342646e-09\\
2.3911955977989	3.12661349929505e-09\\
2.39319659829915	3.12661349929505e-09\\
2.3951975987994	3.33505554516364e-09\\
2.39719859929965	3.33505554516364e-09\\
2.3991995997999	3.12661349929505e-09\\
2.40120060030015	2.91817145342646e-09\\
2.4032016008004	2.70972940755786e-09\\
2.40520260130065	2.50128736168927e-09\\
2.4072036018009	2.29285104539863e-09\\
2.40920460230115	2.08440899953003e-09\\
2.4112056028014	1.66752490779285e-09\\
2.41320660330165	2.08440899953003e-09\\
2.4152076038019	2.29285104539863e-09\\
2.41720860430215	1.87596695366144e-09\\
2.4192096048024	1.66752490779285e-09\\
2.42121060530265	1.66752490779285e-09\\
2.4232116058029	1.66752490779285e-09\\
2.42521260630315	1.66752490779285e-09\\
2.4272136068034	1.66752490779285e-09\\
2.42921460730365	1.66752490779285e-09\\
2.4312156078039	1.87596695366144e-09\\
2.43321660830415	1.66752490779285e-09\\
2.4352176088044	1.66752490779285e-09\\
2.43721860930465	1.66752490779285e-09\\
2.4392196098049	1.66752490779285e-09\\
2.44122061030515	1.66752490779285e-09\\
2.4432216108054	2.08440899953003e-09\\
2.44522261130565	2.08440899953003e-09\\
2.4472236118059	2.50128736168927e-09\\
2.44922461230615	2.29285104539863e-09\\
2.4512256128064	2.29285104539863e-09\\
2.45322661330665	2.08440899953003e-09\\
2.4552276138069	1.87596695366144e-09\\
2.45722861430715	1.66752490779285e-09\\
2.4592296148074	1.66752490779285e-09\\
2.46123061530765	1.87596695366144e-09\\
2.4632316158079	2.08440899953003e-09\\
2.46523261630815	2.08440899953003e-09\\
2.4672336168084	1.66752490779285e-09\\
2.46923461730865	1.66752490779285e-09\\
2.4712356178089	1.87596695366144e-09\\
2.47323661830915	1.87596695366144e-09\\
2.4752376188094	1.4590885915022e-09\\
2.47723861930965	8.33762453896423e-10\\
2.4792396198099	1.25064654563361e-09\\
2.48124062031015	1.4590885915022e-09\\
2.48324162081041	1.04220449976502e-09\\
2.48524262131066	8.33762453896423e-10\\
2.48724362181091	6.25320408027829e-10\\
2.48924462231116	4.16881799906007e-10\\
2.49124562281141	6.25320408027829e-10\\
2.49324662331166	2.08440899953003e-10\\
2.49524762381191	2.08440899953003e-10\\
2.49724862431216	0\\
2.49924962481241	-2.08440899953003e-10\\
2.50125062531266	-2.08440899953003e-10\\
2.50325162581291	-2.08440899953003e-10\\
2.50525262631316	0\\
2.50725362681341	0\\
2.50925462731366	4.16881799906007e-10\\
2.51125562781391	2.08440899953003e-10\\
2.51325662831416	0\\
2.51525762881441	0\\
2.51725862931466	2.08440899953003e-10\\
2.51925962981491	-2.08440899953003e-10\\
2.52126063031516	-2.08440899953003e-10\\
2.52326163081541	-2.08440899953003e-10\\
2.52526263131566	-2.08440899953003e-10\\
2.52726363181591	-2.08440899953003e-10\\
2.52926463231616	-2.08440899953003e-10\\
2.53126563281641	-2.08440899953003e-10\\
2.53326663331666	0\\
2.53526763381691	0\\
2.53726863431716	-2.08440899953003e-10\\
2.53926963481741	-2.08440899953003e-10\\
2.54127063531766	-4.16881799906007e-10\\
2.54327163581791	-4.16881799906007e-10\\
2.54527263631816	-8.33762453896423e-10\\
2.54727363681841	-6.25320408027829e-10\\
2.54927463731866	-8.33762453896423e-10\\
2.55127563781891	-8.33762453896423e-10\\
2.55327663831916	-1.04220449976502e-09\\
2.55527763881941	-1.4590885915022e-09\\
2.55727863931966	-1.4590885915022e-09\\
2.55927963981991	-1.4590885915022e-09\\
2.56128064032016	-1.66752490779285e-09\\
2.56328164082041	-1.4590885915022e-09\\
2.56528264132066	-1.66752490779285e-09\\
2.56728364182091	-1.4590885915022e-09\\
2.56928464232116	-1.25064654563361e-09\\
2.57128564282141	-8.33762453896423e-10\\
2.57328664332166	-4.16881799906007e-10\\
2.57528764382191	-1.04220449976502e-09\\
2.57728864432216	-6.25320408027829e-10\\
2.57928964482241	-6.25320408027829e-10\\
2.58129064532266	-4.16881799906007e-10\\
2.58329164582291	-6.25320408027829e-10\\
2.58529264632316	-8.33762453896423e-10\\
2.58729364682341	-4.16881799906007e-10\\
2.58929464732366	-6.25320408027829e-10\\
2.59129564782391	-8.33762453896423e-10\\
2.59329664832416	-6.25320408027829e-10\\
2.59529764882441	-1.04220449976502e-09\\
2.59729864932466	-6.25320408027829e-10\\
2.59929964982491	-2.08440899953003e-10\\
2.60130065032516	-4.16881799906007e-10\\
2.60330165082541	-2.08440899953003e-10\\
2.60530265132566	-2.08440899953003e-10\\
2.60730365182591	-4.16881799906007e-10\\
2.60930465232616	-2.08440899953003e-10\\
2.61130565282641	-4.16881799906007e-10\\
2.61330665332666	-2.08440899953003e-10\\
2.61530765382691	0\\
2.61730865432716	0\\
2.61930965482741	4.16881799906007e-10\\
2.62131065532766	-2.08440899953003e-10\\
2.62331165582791	-2.08440899953003e-10\\
2.62531265632816	-2.08440899953003e-10\\
2.62731365682841	-2.08440899953003e-10\\
2.62931465732866	0\\
2.63131565782891	0\\
2.63331665832916	-2.08440899953003e-10\\
2.63531765882941	-4.16881799906007e-10\\
2.63731865932966	-2.08440899953003e-10\\
2.63931965982992	-4.16881799906007e-10\\
2.64132066033017	-2.08440899953003e-10\\
2.64332166083042	-2.08440899953003e-10\\
2.64532266133067	-8.33762453896423e-10\\
2.64732366183092	-4.16881799906007e-10\\
2.64932466233117	-4.16881799906007e-10\\
2.65132566283142	-2.08440899953003e-10\\
2.65332666333167	-2.08440899953003e-10\\
2.65532766383192	-2.08440899953003e-10\\
2.65732866433217	-4.16881799906007e-10\\
2.65932966483242	-6.25320408027829e-10\\
2.66133066533267	-8.33762453896423e-10\\
2.66333166583292	-1.4590885915022e-09\\
2.66533266633317	-1.25064654563361e-09\\
2.66733366683342	-1.25064654563361e-09\\
2.66933466733367	-1.4590885915022e-09\\
2.67133566783392	-1.66752490779285e-09\\
2.67333666833417	-1.66752490779285e-09\\
2.67533766883442	-1.66752490779285e-09\\
2.67733866933467	-1.87596695366144e-09\\
2.67933966983492	-1.66752490779285e-09\\
2.68134067033517	-1.25064654563361e-09\\
2.68334167083542	-1.25064654563361e-09\\
2.68534267133567	-1.4590885915022e-09\\
2.68734367183592	-1.66752490779285e-09\\
2.68934467233617	-1.87596695366144e-09\\
2.69134567283642	-1.66752490779285e-09\\
2.69334667333667	-1.4590885915022e-09\\
2.69534767383692	-1.66752490779285e-09\\
2.69734867433717	-8.33762453896423e-10\\
2.69934967483742	-8.33762453896423e-10\\
2.70135067533767	-1.04220449976502e-09\\
2.70335167583792	-1.25064654563361e-09\\
2.70535267633817	-1.4590885915022e-09\\
2.70735367683842	-1.4590885915022e-09\\
2.70935467733867	-1.4590885915022e-09\\
2.71135567783892	-1.4590885915022e-09\\
2.71335667833917	-8.33762453896423e-10\\
2.71535767883942	-6.25320408027829e-10\\
2.71735867933967	-1.25064654563361e-09\\
2.71935967983992	-1.04220449976502e-09\\
2.72136068034017	-8.33762453896423e-10\\
2.72336168084042	-1.04220449976502e-09\\
2.72536268134067	-6.25320408027829e-10\\
2.72736368184092	-8.33762453896423e-10\\
2.72936468234117	-1.04220449976502e-09\\
2.73136568284142	-1.66752490779285e-09\\
2.73336668334167	-1.4590885915022e-09\\
2.73536768384192	-1.4590885915022e-09\\
2.73736868434217	-1.4590885915022e-09\\
2.73936968484242	-1.87596695366144e-09\\
2.74137068534267	-1.66752490779285e-09\\
2.74337168584292	-1.87596695366144e-09\\
2.74537268634317	-2.08440899953003e-09\\
2.74737368684342	-2.08440899953003e-09\\
2.74937468734367	-2.08440899953003e-09\\
2.75137568784392	-2.08440899953003e-09\\
2.75337668834417	-2.29285104539863e-09\\
2.75537768884442	-2.29285104539863e-09\\
2.75737868934467	-2.29285104539863e-09\\
2.75937968984492	-2.50128736168927e-09\\
2.76138069034517	-2.29285104539863e-09\\
2.76338169084542	-2.08440899953003e-09\\
2.76538269134567	-1.4590885915022e-09\\
2.76738369184592	-1.87596695366144e-09\\
2.76938469234617	-1.66752490779285e-09\\
2.77138569284642	-1.87596695366144e-09\\
2.77338669334667	-1.87596695366144e-09\\
2.77538769384692	-1.66752490779285e-09\\
2.77738869434717	-2.08440899953003e-09\\
2.77938969484742	-1.4590885915022e-09\\
2.78139069534767	-1.25064654563361e-09\\
2.78339169584792	-1.04220449976502e-09\\
2.78539269634817	-6.25320408027829e-10\\
2.78739369684842	-6.25320408027829e-10\\
2.78939469734867	-4.16881799906007e-10\\
2.79139569784892	-2.08440899953003e-10\\
2.79339669834917	-2.08440899953003e-10\\
2.79539769884942	0\\
2.79739869934967	-2.08440899953003e-10\\
2.79939969984992	-2.08440899953003e-10\\
2.80140070035018	-2.08440899953003e-10\\
2.80340170085043	2.08440899953003e-10\\
2.80540270135068	4.16881799906007e-10\\
2.80740370185093	6.25320408027829e-10\\
2.80940470235118	8.33762453896423e-10\\
2.81140570285143	6.25320408027829e-10\\
2.81340670335168	8.33762453896423e-10\\
2.81540770385193	8.33762453896423e-10\\
2.81740870435218	6.25320408027829e-10\\
2.81940970485243	6.25320408027829e-10\\
2.82141070535268	8.33762453896423e-10\\
2.82341170585293	6.25320408027829e-10\\
2.82541270635318	0\\
2.82741370685343	2.08440899953003e-10\\
2.82941470735368	2.08440899953003e-10\\
2.83141570785393	2.08440899953003e-10\\
2.83341670835418	6.25320408027829e-10\\
2.83541770885443	6.25320408027829e-10\\
2.83741870935468	6.25320408027829e-10\\
2.83941970985493	6.25320408027829e-10\\
2.84142071035518	1.04220449976502e-09\\
2.84342171085543	1.25064654563361e-09\\
2.84542271135568	8.33762453896423e-10\\
2.84742371185593	1.04220449976502e-09\\
2.84942471235618	1.25064654563361e-09\\
2.85142571285643	1.25064654563361e-09\\
2.85342671335668	8.33762453896423e-10\\
2.85542771385693	1.04220449976502e-09\\
2.85742871435718	1.04220449976502e-09\\
2.85942971485743	1.04220449976502e-09\\
2.86143071535768	1.4590885915022e-09\\
2.86343171585793	1.25064654563361e-09\\
2.86543271635818	1.04220449976502e-09\\
2.86743371685843	1.4590885915022e-09\\
2.86943471735868	1.4590885915022e-09\\
2.87143571785893	1.25064654563361e-09\\
2.87343671835918	1.87596695366144e-09\\
2.87543771885943	1.66752490779285e-09\\
2.87743871935968	1.04220449976502e-09\\
2.87943971985993	1.04220449976502e-09\\
2.88144072036018	1.04220449976502e-09\\
2.88344172086043	1.87596695366144e-09\\
2.88544272136068	1.87596695366144e-09\\
2.88744372186093	1.87596695366144e-09\\
2.88944472236118	2.08440899953003e-09\\
2.89144572286143	1.87596695366144e-09\\
2.89344672336168	1.87596695366144e-09\\
2.89544772386193	1.66752490779285e-09\\
2.89744872436218	1.66752490779285e-09\\
2.89944972486243	1.66752490779285e-09\\
2.90145072536268	2.08440899953003e-09\\
2.90345172586293	1.66752490779285e-09\\
2.90545272636318	1.87596695366144e-09\\
2.90745372686343	1.66752490779285e-09\\
2.90945472736368	1.87596695366144e-09\\
2.91145572786393	1.66752490779285e-09\\
2.91345672836418	1.66752490779285e-09\\
2.91545772886443	2.08440899953003e-09\\
2.91745872936468	1.66752490779285e-09\\
2.91945972986493	1.87596695366144e-09\\
2.92146073036518	2.08440899953003e-09\\
2.92346173086543	2.50128736168927e-09\\
2.92546273136568	2.70972940755786e-09\\
2.92746373186593	2.29285104539863e-09\\
2.92946473236618	2.29285104539863e-09\\
2.93146573286643	2.29285104539863e-09\\
2.93346673336668	2.08440899953003e-09\\
2.93546773386693	1.87596695366144e-09\\
2.93746873436718	1.4590885915022e-09\\
2.93946973486743	1.25064654563361e-09\\
2.94147073536768	8.33762453896423e-10\\
2.94347173586793	8.33762453896423e-10\\
2.94547273636818	1.04220449976502e-09\\
2.94747373686843	1.25064654563361e-09\\
2.94947473736868	1.04220449976502e-09\\
2.95147573786893	8.33762453896423e-10\\
2.95347673836918	8.33762453896423e-10\\
2.95547773886943	8.33762453896423e-10\\
2.95747873936968	6.25320408027829e-10\\
2.95947973986994	1.25064654563361e-09\\
2.96148074037019	1.25064654563361e-09\\
2.96348174087044	1.25064654563361e-09\\
2.96548274137069	1.66752490779285e-09\\
2.96748374187094	1.25064654563361e-09\\
2.96948474237119	1.25064654563361e-09\\
2.97148574287144	1.25064654563361e-09\\
2.97348674337169	8.33762453896423e-10\\
2.97548774387194	8.33762453896423e-10\\
2.97748874437219	6.25320408027829e-10\\
2.97948974487244	2.08440899953003e-10\\
2.98149074537269	-2.08440899953003e-10\\
2.98349174587294	2.08440899953003e-10\\
2.98549274637319	-2.08440899953003e-10\\
2.98749374687344	-2.08440899953003e-10\\
2.98949474737369	-2.08440899953003e-10\\
2.99149574787394	-2.08440899953003e-10\\
2.99349674837419	-2.08440899953003e-10\\
2.99549774887444	0\\
2.99749874937469	4.16881799906007e-10\\
2.99949974987494	6.25320408027829e-10\\
3.00150075037519	1.04220449976502e-09\\
3.00350175087544	1.25064654563361e-09\\
3.00550275137569	1.04220449976502e-09\\
3.00750375187594	8.33762453896423e-10\\
3.00950475237619	6.25320408027829e-10\\
3.01150575287644	4.16881799906007e-10\\
3.01350675337669	2.08440899953003e-10\\
3.01550775387694	0\\
3.01750875437719	-4.16881799906007e-10\\
3.01950975487744	-2.08440899953003e-10\\
3.02151075537769	-4.16881799906007e-10\\
3.02351175587794	-8.33762453896423e-10\\
3.02551275637819	-6.25320408027829e-10\\
3.02751375687844	-6.25320408027829e-10\\
3.02951475737869	-8.33762453896423e-10\\
3.03151575787894	-1.04220449976502e-09\\
3.03351675837919	-8.33762453896423e-10\\
3.03551775887944	-6.25320408027829e-10\\
3.03751875937969	-6.25320408027829e-10\\
3.03951975987994	-6.25320408027829e-10\\
3.04152076038019	-8.33762453896423e-10\\
3.04352176088044	-6.25320408027829e-10\\
3.04552276138069	-2.08440899953003e-10\\
3.04752376188094	-2.08440899953003e-10\\
3.04952476238119	-4.16881799906007e-10\\
3.05152576288144	-6.25320408027829e-10\\
3.05352676338169	-6.25320408027829e-10\\
3.05552776388194	-6.25320408027829e-10\\
3.05752876438219	-1.25064654563361e-09\\
3.05952976488244	-1.4590885915022e-09\\
3.06153076538269	-1.25064654563361e-09\\
3.06353176588294	-1.25064654563361e-09\\
3.06553276638319	-1.66752490779285e-09\\
3.06753376688344	-1.66752490779285e-09\\
3.06953476738369	-1.25064654563361e-09\\
3.07153576788394	-1.25064654563361e-09\\
3.07353676838419	-1.25064654563361e-09\\
3.07553776888444	-1.25064654563361e-09\\
3.07753876938469	-1.04220449976502e-09\\
3.07953976988494	-1.4590885915022e-09\\
3.08154077038519	-1.4590885915022e-09\\
3.08354177088544	-2.08440899953003e-09\\
3.08554277138569	-1.87596695366144e-09\\
3.08754377188594	-2.29285104539863e-09\\
3.08954477238619	-2.50128736168927e-09\\
3.09154577288644	-2.70972940755786e-09\\
3.09354677338669	-2.50128736168927e-09\\
3.09554777388694	-2.29285104539863e-09\\
3.09754877438719	-2.29285104539863e-09\\
3.09954977488744	-2.70972940755786e-09\\
3.10155077538769	-2.70972940755786e-09\\
3.10355177588794	-2.50128736168927e-09\\
3.10555277638819	-2.08440899953003e-09\\
3.10755377688844	-2.08440899953003e-09\\
3.10955477738869	-1.66752490779285e-09\\
3.11155577788894	-1.66752490779285e-09\\
3.11355677838919	-1.66752490779285e-09\\
3.11555777888944	-1.66752490779285e-09\\
3.11755877938969	-1.87596695366144e-09\\
3.11955977988994	-1.87596695366144e-09\\
3.1215607803902	-1.87596695366144e-09\\
3.12356178089045	-2.08440899953003e-09\\
3.1255627813907	-2.29285104539863e-09\\
3.12756378189095	-1.66752490779285e-09\\
3.1295647823912	-1.4590885915022e-09\\
3.13156578289145	-1.66752490779285e-09\\
3.1335667833917	-1.87596695366144e-09\\
3.13556778389195	-2.29285104539863e-09\\
3.1375687843922	-2.29285104539863e-09\\
3.13956978489245	-2.29285104539863e-09\\
3.1415707853927	-2.91817145342646e-09\\
3.14357178589295	-2.29285104539863e-09\\
3.1455727863932	-2.70972940755786e-09\\
3.14757378689345	-3.12661349929505e-09\\
3.1495747873937	-3.33505554516364e-09\\
3.15157578789395	-3.33505554516364e-09\\
3.1535767883942	-3.12661349929505e-09\\
3.15557778889445	-2.91817145342646e-09\\
3.1575787893947	-2.70972940755786e-09\\
3.15957978989495	-2.91817145342646e-09\\
3.1615807903952	-3.33505554516364e-09\\
3.16358179089545	-3.12661349929505e-09\\
3.1655827913957	-3.33505554516364e-09\\
3.16758379189595	-3.54349186145428e-09\\
3.1695847923962	-3.75193390732288e-09\\
3.17158579289645	-3.75193390732288e-09\\
3.1735867933967	-3.75193390732288e-09\\
3.17558779389695	-3.54349186145428e-09\\
3.1775887943972	-3.12661349929505e-09\\
3.17958979489745	-3.54349186145428e-09\\
3.1815907953977	-3.75193390732288e-09\\
3.18359179589795	-3.75193390732288e-09\\
3.1855927963982	-3.54349186145428e-09\\
3.18759379689845	-3.75193390732288e-09\\
3.1895947973987	-3.96037595319147e-09\\
3.19159579789895	-4.37726004492866e-09\\
3.1935967983992	-4.16881799906006e-09\\
3.19559779889945	-4.37726004492866e-09\\
3.1975987993997	-4.37726004492866e-09\\
3.19959979989995	-4.16881799906006e-09\\
3.2016008004002	-4.16881799906006e-09\\
3.20360180090045	-3.75193390732288e-09\\
3.2056028014007	-3.75193390732288e-09\\
3.20760380190095	-3.75193390732288e-09\\
3.2096048024012	-3.75193390732288e-09\\
3.21160580290145	-3.75193390732288e-09\\
3.2136068034017	-3.75193390732288e-09\\
3.21560780390195	-3.96037595319147e-09\\
3.2176088044022	-3.96037595319147e-09\\
3.21960980490245	-3.75193390732288e-09\\
3.2216108054027	-4.5856963612193e-09\\
3.22361180590295	-4.79413840708789e-09\\
3.2256128064032	-5.21102249882508e-09\\
3.22761380690345	-5.41945881511572e-09\\
3.2296148074037	-5.00258045295649e-09\\
3.23161580790395	-5.21102249882508e-09\\
3.2336168084042	-5.21102249882508e-09\\
3.23561780890445	-5.00258045295649e-09\\
3.2376188094047	-4.5856963612193e-09\\
3.23961980990495	-4.79413840708789e-09\\
3.2416208104052	-4.79413840708789e-09\\
3.24362181090545	-5.00258045295649e-09\\
3.2456228114057	-4.5856963612193e-09\\
3.24762381190595	-4.79413840708789e-09\\
3.2496248124062	-5.00258045295649e-09\\
3.25162581290645	-4.5856963612193e-09\\
3.2536268134067	-4.16881799906006e-09\\
3.25562781390695	-4.16881799906006e-09\\
3.2576288144072	-4.16881799906006e-09\\
3.25962981490745	-4.5856963612193e-09\\
3.2616308154077	-5.00258045295649e-09\\
3.26363181590795	-4.79413840708789e-09\\
3.2656328164082	-4.79413840708789e-09\\
3.26763381690845	-4.5856963612193e-09\\
3.2696348174087	-5.00258045295649e-09\\
3.27163581790895	-5.21102249882508e-09\\
3.2736368184092	-5.21102249882508e-09\\
3.27563781890945	-5.21102249882508e-09\\
3.27763881940971	-5.00258045295649e-09\\
3.27963981990996	-5.21102249882508e-09\\
3.2816408204102	-5.41945881511572e-09\\
3.28364182091046	-5.21102249882508e-09\\
3.28564282141071	-5.00258045295649e-09\\
3.28764382191096	-5.21102249882508e-09\\
3.28964482241121	-5.21102249882508e-09\\
3.29164582291146	-4.5856963612193e-09\\
3.29364682341171	-4.79413840708789e-09\\
3.29564782391196	-5.21102249882508e-09\\
3.29764882441221	-5.00258045295649e-09\\
3.29964982491246	-4.79413840708789e-09\\
3.30165082541271	-4.5856963612193e-09\\
3.30365182591296	-4.37726004492866e-09\\
3.30565282641321	-4.5856963612193e-09\\
3.30765382691346	-4.79413840708789e-09\\
3.30965482741371	-3.96037595319147e-09\\
3.31165582791396	-4.16881799906006e-09\\
3.31365682841421	-4.79413840708789e-09\\
3.31565782891446	-4.79413840708789e-09\\
3.31765882941471	-4.5856963612193e-09\\
3.31965982991496	-4.79413840708789e-09\\
3.32166083041521	-4.79413840708789e-09\\
3.32366183091546	-4.5856963612193e-09\\
3.32566283141571	-4.37726004492866e-09\\
3.32766383191596	-4.37726004492866e-09\\
3.32966483241621	-4.79413840708789e-09\\
3.33166583291646	-4.79413840708789e-09\\
3.33366683341671	-4.79413840708789e-09\\
3.33566783391696	-4.5856963612193e-09\\
3.33766883441721	-4.37726004492866e-09\\
3.33966983491746	-4.5856963612193e-09\\
3.34167083541771	-4.16881799906006e-09\\
3.34367183591796	-4.37726004492866e-09\\
3.34567283641821	-4.37726004492866e-09\\
3.34767383691846	-4.5856963612193e-09\\
3.34967483741871	-4.37726004492866e-09\\
3.35167583791896	-4.5856963612193e-09\\
3.35367683841921	-5.00258045295649e-09\\
3.35567783891946	-5.21102249882508e-09\\
3.35767883941971	-5.62790086098432e-09\\
3.35967983991996	-5.00258045295649e-09\\
3.36168084042021	-5.00258045295649e-09\\
3.36368184092046	-4.79413840708789e-09\\
3.36568284142071	-4.37726004492866e-09\\
3.36768384192096	-4.5856963612193e-09\\
3.36968484242121	-4.5856963612193e-09\\
3.37168584292146	-4.79413840708789e-09\\
3.37368684342171	-5.00258045295649e-09\\
3.37568784392196	-4.79413840708789e-09\\
3.37768884442221	-5.21102249882508e-09\\
3.37968984492246	-5.21102249882508e-09\\
3.38169084542271	-4.79413840708789e-09\\
3.38369184592296	-4.79413840708789e-09\\
3.38569284642321	-4.5856963612193e-09\\
3.38769384692346	-5.00258045295649e-09\\
3.38969484742371	-5.00258045295649e-09\\
3.39169584792396	-4.79413840708789e-09\\
3.39369684842421	-4.79413840708789e-09\\
3.39569784892446	-5.00258045295649e-09\\
3.39769884942471	-4.79413840708789e-09\\
3.39969984992496	-4.79413840708789e-09\\
3.40170085042521	-4.79413840708789e-09\\
3.40370185092546	-4.37726004492866e-09\\
3.40570285142571	-4.37726004492866e-09\\
3.40770385192596	-4.16881799906006e-09\\
3.40970485242621	-3.75193390732288e-09\\
3.41170585292646	-3.75193390732288e-09\\
3.41370685342671	-4.16881799906006e-09\\
3.41570785392696	-4.37726004492866e-09\\
3.41770885442721	-4.79413840708789e-09\\
3.41970985492746	-5.41945881511572e-09\\
3.42171085542771	-5.41945881511572e-09\\
3.42371185592796	-5.62790086098432e-09\\
3.42571285642821	-5.21102249882508e-09\\
3.42771385692846	-5.62790086098432e-09\\
3.42971485742871	-5.62790086098432e-09\\
3.43171585792896	-5.8363199885411e-09\\
3.43371685842921	-5.62790086098432e-09\\
3.43571785892946	-5.62790086098432e-09\\
3.43771885942971	-5.8363199885411e-09\\
3.43971985992996	-5.8363199885411e-09\\
3.44172086043022	-6.46164612614688e-09\\
3.44372186093047	-6.25320408027829e-09\\
3.44572286143072	-6.25320408027829e-09\\
3.44772386193097	-6.25320408027829e-09\\
3.44972486243122	-6.25320408027829e-09\\
3.45172586293147	-6.25320408027829e-09\\
3.45372686343172	-6.0447620344097e-09\\
3.45572786393197	-6.0447620344097e-09\\
3.45772886443222	-6.0447620344097e-09\\
3.45972986493247	-5.8363199885411e-09\\
3.46173086543272	-5.8363199885411e-09\\
3.46373186593297	-5.8363199885411e-09\\
3.46573286643322	-6.25320408027829e-09\\
3.46773386693347	-6.0447620344097e-09\\
3.46973486743372	-6.25320408027829e-09\\
3.47173586793397	-6.25320408027829e-09\\
3.47373686843422	-6.25320408027829e-09\\
3.47573786893447	-6.46164612614688e-09\\
3.47773886943472	-6.25320408027829e-09\\
3.47973986993497	-6.46164612614688e-09\\
3.48174087043522	-6.46164612614688e-09\\
3.48374187093547	-6.46164612614688e-09\\
3.48574287143572	-6.87853021788407e-09\\
3.48774387193597	-6.87853021788407e-09\\
3.48974487243622	-6.87853021788407e-09\\
3.49174587293647	-6.87853021788407e-09\\
3.49374687343672	-7.08697226375267e-09\\
3.49574787393697	-7.08697226375267e-09\\
3.49774887443722	-7.08697226375267e-09\\
3.49974987493747	-7.08697226375267e-09\\
3.50175087543772	-7.08697226375267e-09\\
3.50375187593797	-7.08697226375267e-09\\
3.50575287643822	-7.08697226375267e-09\\
3.50775387693847	-7.08697226375267e-09\\
3.50975487743872	-7.08697226375267e-09\\
3.51175587793897	-7.08697226375267e-09\\
3.51375687843922	-6.46164612614688e-09\\
3.51575787893947	-6.87853021788407e-09\\
3.51775887943972	-7.08697226375267e-09\\
3.51975987993997	-7.08697226375267e-09\\
3.52176088044022	-6.87853021788407e-09\\
3.52376188094047	-7.08697226375267e-09\\
3.52576288144072	-7.08697226375267e-09\\
3.52776388194097	-6.87853021788407e-09\\
3.52976488244122	-6.46164612614688e-09\\
3.53176588294147	-6.87853021788407e-09\\
3.53376688344172	-6.67008817201548e-09\\
3.53576788394197	-6.87853021788407e-09\\
3.53776888444222	-6.87853021788407e-09\\
3.53976988494247	-6.87853021788407e-09\\
3.54177088544272	-6.46164612614688e-09\\
3.54377188594297	-6.25320408027829e-09\\
3.54577288644322	-6.67008817201548e-09\\
3.54777388694347	-6.67008817201548e-09\\
3.54977488744372	-7.08697226375267e-09\\
3.55177588794397	-7.29541430962126e-09\\
3.55377688844422	-7.50385635548985e-09\\
3.55577788894447	-7.50385635548985e-09\\
3.55777888944472	-7.71229840135845e-09\\
3.55977988994497	-7.71229840135845e-09\\
3.56178089044522	-7.71229840135845e-09\\
3.56378189094547	-7.92074044722704e-09\\
3.56578289144572	-8.33762453896423e-09\\
3.56778389194597	-8.54606658483282e-09\\
3.56978489244622	-8.54606658483282e-09\\
3.57178589294647	-8.54606658483282e-09\\
3.57378689344672	-8.54606658483282e-09\\
3.57578789394697	-8.75450863070141e-09\\
3.57778889444722	-8.54606658483282e-09\\
3.57978989494747	-8.75450863070141e-09\\
3.58179089544772	-8.96295067657001e-09\\
3.58379189594797	-9.1713927224386e-09\\
3.58579289644822	-8.96295067657001e-09\\
3.58779389694847	-8.75450863070141e-09\\
3.58979489744872	-8.96295067657001e-09\\
3.59179589794897	-8.96295067657001e-09\\
3.59379689844922	-8.75450863070141e-09\\
3.59579789894947	-8.75450863070141e-09\\
3.59779889944972	-8.75450863070141e-09\\
3.59979989994997	-8.54606658483282e-09\\
3.60180090045022	-8.96295067657001e-09\\
3.60380190095048	-8.75450863070141e-09\\
3.60580290145073	-8.54606658483282e-09\\
3.60780390195098	-8.54606658483282e-09\\
3.60980490245123	-8.54606658483282e-09\\
3.61180590295148	-8.54606658483282e-09\\
3.61380690345173	-8.54606658483282e-09\\
3.61580790395198	-8.12918249309563e-09\\
3.61780890445223	-8.33762453896423e-09\\
3.61980990495248	-8.33762453896423e-09\\
3.62181090545273	-8.33762453896423e-09\\
3.62381190595298	-7.71229840135845e-09\\
3.62581290645323	-7.71229840135845e-09\\
3.62781390695348	-7.92074044722704e-09\\
3.62981490745373	-7.92074044722704e-09\\
3.63181590795398	-7.71229840135845e-09\\
3.63381690845423	-7.29541430962126e-09\\
3.63581790895448	-7.08697226375267e-09\\
3.63781890945473	-7.50385635548985e-09\\
3.63981990995498	-8.12918249309563e-09\\
3.64182091045523	-8.33762453896423e-09\\
3.64382191095548	-8.33762453896423e-09\\
3.64582291145573	-7.92074044722704e-09\\
3.64782391195598	-7.92074044722704e-09\\
3.64982491245623	-7.92074044722704e-09\\
3.65182591295648	-7.71229840135845e-09\\
3.65382691345673	-7.71229840135845e-09\\
3.65582791395698	-7.92074044722704e-09\\
3.65782891445723	-7.92074044722704e-09\\
3.65982991495748	-7.71229840135845e-09\\
3.66183091545773	-7.92074044722704e-09\\
3.66383191595798	-7.71229840135845e-09\\
3.66583291645823	-7.71229840135845e-09\\
3.66783391695848	-7.92074044722704e-09\\
3.66983491745873	-8.12918249309563e-09\\
3.67183591795898	-7.92074044722704e-09\\
3.67383691845923	-8.12918249309563e-09\\
3.67583791895948	-7.92074044722704e-09\\
3.67783891945973	-7.92074044722704e-09\\
3.67983991995998	-7.50385635548985e-09\\
3.68184092046023	-7.50385635548985e-09\\
3.68384192096048	-7.71229840135845e-09\\
3.68584292146073	-7.92074044722704e-09\\
3.68784392196098	-8.12918249309563e-09\\
3.68984492246123	-7.92074044722704e-09\\
3.69184592296148	-8.12918249309563e-09\\
3.69384692346173	-7.92074044722704e-09\\
3.69584792396198	-7.71229840135845e-09\\
3.69784892446223	-7.50385635548985e-09\\
3.69984992496248	-7.29541430962126e-09\\
3.70185092546273	-7.08697226375267e-09\\
3.70385192596298	-7.08697226375267e-09\\
3.70585292646323	-7.08697226375267e-09\\
3.70785392696348	-7.08697226375267e-09\\
3.70985492746373	-6.87853021788407e-09\\
3.71185592796398	-6.67008817201548e-09\\
3.71385692846423	-6.67008817201548e-09\\
3.71585792896448	-5.8363199885411e-09\\
3.71785892946473	-6.46164612614688e-09\\
3.71985992996498	-6.87853021788407e-09\\
3.72186093046523	-7.08697226375267e-09\\
3.72386193096548	-6.67008817201548e-09\\
3.72586293146573	-6.87853021788407e-09\\
3.72786393196598	-6.67008817201548e-09\\
3.72986493246623	-6.0447620344097e-09\\
3.73186593296648	-6.0447620344097e-09\\
3.73386693346673	-6.25320408027829e-09\\
3.73586793396698	-6.25320408027829e-09\\
3.73786893446723	-6.25320408027829e-09\\
3.73986993496748	-6.67008817201548e-09\\
3.74187093546773	-6.67008817201548e-09\\
3.74387193596798	-6.25320408027829e-09\\
3.74587293646823	-6.0447620344097e-09\\
3.74787393696848	-6.0447620344097e-09\\
3.74987493746873	-6.25320408027829e-09\\
3.75187593796898	-6.46164612614688e-09\\
3.75387693846923	-6.46164612614688e-09\\
3.75587793896948	-6.25320408027829e-09\\
3.75787893946973	-6.67008817201548e-09\\
3.75987993996999	-6.67008817201548e-09\\
3.76188094047024	-7.08697226375267e-09\\
3.76388194097049	-7.08697226375267e-09\\
3.76588294147074	-6.87853021788407e-09\\
3.76788394197099	-6.25320408027829e-09\\
3.76988494247124	-6.0447620344097e-09\\
3.77188594297149	-6.46164612614688e-09\\
3.77388694347174	-6.46164612614688e-09\\
3.77588794397199	-6.25320408027829e-09\\
3.77788894447224	-6.67008817201548e-09\\
3.77988994497249	-6.46164612614688e-09\\
3.78189094547274	-6.67008817201548e-09\\
3.78389194597299	-6.67008817201548e-09\\
3.78589294647324	-6.46164612614688e-09\\
3.78789394697349	-6.0447620344097e-09\\
3.78989494747374	-5.8363199885411e-09\\
3.79189594797399	-5.62790086098432e-09\\
3.79389694847424	-4.79413840708789e-09\\
3.79589794897449	-4.5856963612193e-09\\
3.79789894947474	-5.00258045295649e-09\\
3.79989994997499	-5.41945881511572e-09\\
3.80190095047524	-5.62790086098432e-09\\
3.80390195097549	-5.62790086098432e-09\\
3.80590295147574	-6.0447620344097e-09\\
3.80790395197599	-5.62790086098432e-09\\
3.80990495247624	-5.8363199885411e-09\\
3.81190595297649	-6.0447620344097e-09\\
3.81390695347674	-6.0447620344097e-09\\
3.81590795397699	-6.25320408027829e-09\\
3.81790895447724	-6.87853021788407e-09\\
3.81990995497749	-6.46164612614688e-09\\
3.82191095547774	-6.25320408027829e-09\\
3.82391195597799	-6.46164612614688e-09\\
3.82591295647824	-6.67008817201548e-09\\
3.82791395697849	-7.08697226375267e-09\\
3.82991495747874	-7.08697226375267e-09\\
3.83191595797899	-6.87853021788407e-09\\
3.83391695847924	-6.25320408027829e-09\\
3.83591795897949	-6.67008817201548e-09\\
3.83791895947974	-7.08697226375267e-09\\
3.83991995997999	-6.46164612614688e-09\\
3.84192096048024	-7.08697226375267e-09\\
3.84392196098049	-6.46164612614688e-09\\
3.84592296148074	-6.87853021788407e-09\\
3.84792396198099	-6.67008817201548e-09\\
3.84992496248124	-6.25320408027829e-09\\
3.85192596298149	-6.87853021788407e-09\\
3.85392696348174	-6.87853021788407e-09\\
3.85592796398199	-7.08697226375267e-09\\
3.85792896448224	-7.08697226375267e-09\\
3.85992996498249	-7.08697226375267e-09\\
3.86193096548274	-7.08697226375267e-09\\
3.86393196598299	-7.08697226375267e-09\\
3.86593296648324	-7.71229840135845e-09\\
3.86793396698349	-7.50385635548985e-09\\
3.86993496748374	-7.50385635548985e-09\\
3.87193596798399	-7.29541430962126e-09\\
3.87393696848424	-7.08697226375267e-09\\
3.87593796898449	-7.08697226375267e-09\\
3.87793896948474	-7.08697226375267e-09\\
3.87993996998499	-7.08697226375267e-09\\
3.88194097048524	-6.46164612614688e-09\\
3.88394197098549	-6.25320408027829e-09\\
3.88594297148574	-5.8363199885411e-09\\
3.88794397198599	-6.0447620344097e-09\\
3.88994497248624	-6.25320408027829e-09\\
3.89194597298649	-6.0447620344097e-09\\
3.89394697348674	-6.46164612614688e-09\\
3.89594797398699	-6.25320408027829e-09\\
3.89794897448724	-6.0447620344097e-09\\
3.89994997498749	-6.46164612614688e-09\\
3.90195097548774	-6.25320408027829e-09\\
3.90395197598799	-6.0447620344097e-09\\
3.90595297648824	-5.8363199885411e-09\\
3.90795397698849	-6.0447620344097e-09\\
3.90995497748874	-6.25320408027829e-09\\
3.91195597798899	-5.8363199885411e-09\\
3.91395697848924	-5.41945881511572e-09\\
3.91595797898949	-5.00258045295649e-09\\
3.91795897948974	-4.79413840708789e-09\\
3.91995997998999	-4.79413840708789e-09\\
3.92196098049025	-4.37726004492866e-09\\
3.9239619809905	-4.37726004492866e-09\\
3.92596298149075	-3.75193390732288e-09\\
3.927963981991	-3.75193390732288e-09\\
3.92996498249125	-4.16881799906006e-09\\
3.9319659829915	-4.37726004492866e-09\\
3.93396698349175	-4.5856963612193e-09\\
3.935967983992	-4.37726004492866e-09\\
3.93796898449225	-4.37726004492866e-09\\
3.9399699849925	-4.16881799906006e-09\\
3.94197098549275	-3.75193390732288e-09\\
3.943971985993	-3.12661349929505e-09\\
3.94597298649325	-3.12661349929505e-09\\
3.9479739869935	-2.91817145342646e-09\\
3.94997498749375	-2.91817145342646e-09\\
3.951975987994	-3.33505554516364e-09\\
3.95397698849425	-3.54349186145428e-09\\
3.9559779889945	-3.33505554516364e-09\\
3.95797898949475	-3.54349186145428e-09\\
3.959979989995	-3.33505554516364e-09\\
3.96198099049525	-2.70972940755786e-09\\
3.9639819909955	-2.70972940755786e-09\\
3.96598299149575	-2.70972940755786e-09\\
3.967983991996	-2.50128736168927e-09\\
3.96998499249625	-2.50128736168927e-09\\
3.9719859929965	-2.50128736168927e-09\\
3.97398699349675	-2.70972940755786e-09\\
3.975987993997	-2.91817145342646e-09\\
3.97798899449725	-3.54349186145428e-09\\
3.9799899949975	-3.33505554516364e-09\\
3.98199099549775	-3.33505554516364e-09\\
3.983991995998	-3.33505554516364e-09\\
3.98599299649825	-3.12661349929505e-09\\
3.9879939969985	-2.91817145342646e-09\\
3.98999499749875	-2.70972940755786e-09\\
3.991995997999	-2.91817145342646e-09\\
3.99399699849925	-2.91817145342646e-09\\
3.9959979989995	-2.91817145342646e-09\\
3.99799899949975	-2.50128736168927e-09\\
4	-2.50128736168927e-09\\
};
\addlegendentry{c2};

\addplot [color=mycolor3,solid]
  table[row sep=crcr]{%
0	-6.67008817201548e-09\\
0.00200100050025012	2.66804099838414e-08\\
0.00400200100050025	3.33505554516364e-08\\
0.00600300150075038	2.66804099838414e-08\\
0.0080040020010005	2.66804099838414e-08\\
0.0100050025012506	4.00206436236519e-08\\
0.0120060030015008	2.66804099838414e-08\\
0.0140070035017509	6.67008817201548e-08\\
0.016008004002001	1.0005160905913e-07\\
0.0180090045022511	8.67114327150988e-08\\
0.0200100050025013	8.67114327150988e-08\\
0.0220110055027514	6.67008817201548e-08\\
0.0240120060030015	4.00206436236519e-08\\
0.0260130065032516	4.00206436236519e-08\\
0.0280140070035018	0\\
0.0300150075037519	-1.33402336398105e-08\\
0.032016008004002	-2.66804099838414e-08\\
0.0340170085042521	-2.00103218118259e-08\\
0.0360180090045022	-1.33402336398105e-08\\
0.0380190095047524	-2.00103218118259e-08\\
0.0400200100050025	-1.33402336398105e-08\\
0.0420210105052526	-1.33402336398105e-08\\
0.0440220110055028	-2.66804099838414e-08\\
0.0460230115057529	-2.00103218118259e-08\\
0.048024012006003	-4.00206436236519e-08\\
0.0500250125062531	-4.00206436236519e-08\\
0.0520260130065033	-2.00103218118259e-08\\
0.0540270135067534	-5.33608772634624e-08\\
0.0560280140070035	-3.33505554516364e-08\\
0.0580290145072536	-2.66804099838414e-08\\
0.0600300150075038	-3.33505554516364e-08\\
0.0620310155077539	-4.66907317956674e-08\\
0.064032016008004	-6.00310800270369e-08\\
0.0660330165082541	-6.67008817201548e-08\\
0.0680340170085043	-4.66907317956674e-08\\
0.0700350175087544	-4.00206436236519e-08\\
0.0720360180090045	-2.00103218118259e-08\\
0.0740370185092546	-2.00103218118259e-08\\
0.0760380190095048	-2.00103218118259e-08\\
0.0780390195097549	0\\
0.080040020010005	-4.00206436236519e-08\\
0.0820410205102551	-6.00310800270369e-08\\
0.0840420210105053	-6.00310800270369e-08\\
0.0860430215107554	-7.33712563710678e-08\\
0.0880440220110055	-9.33812344082167e-08\\
0.0900450225112556	-9.33812344082167e-08\\
0.0920460230115058	-8.00410580641858e-08\\
0.0940470235117559	-8.67114327150988e-08\\
0.096048024012006	-7.33712563710678e-08\\
0.0980490245122561	-6.67008817201548e-08\\
0.100050025012506	-6.67008817201548e-08\\
0.102051025512756	-9.33812344082167e-08\\
0.104052026013007	-1.13391785403161e-07\\
0.106053026513257	-1.13391785403161e-07\\
0.108054027013507	-1.06721983710043e-07\\
0.110055027513757	-1.13391785403161e-07\\
0.112056028014007	-1.0005160905913e-07\\
0.114057028514257	-1.0005160905913e-07\\
0.116058029014507	-7.33712563710678e-08\\
0.118059029514757	-1.0005160905913e-07\\
0.120060030015008	-9.33812344082167e-08\\
0.122061030515258	-1.0005160905913e-07\\
0.124062031015508	-1.13391785403161e-07\\
0.126063031515758	-1.26731961747192e-07\\
0.128064032016008	-1.20062160054074e-07\\
0.130065032516258	-1.46742512742136e-07\\
0.132066033016508	-1.33402336398105e-07\\
0.134067033516758	-1.60082689086167e-07\\
0.136068034017009	-1.80092667123315e-07\\
0.138069034517259	-1.66752490779285e-07\\
0.140070035017509	-1.73422865430198e-07\\
0.142071035517759	-1.86763041774229e-07\\
0.144072036018009	-1.40072138091223e-07\\
0.146073036518259	-1.80092667123315e-07\\
0.148074037018509	-1.60082689086167e-07\\
0.150075037518759	-1.86763041774229e-07\\
0.15207603801901	-1.80092667123315e-07\\
0.15407703851926	-1.46742512742136e-07\\
0.15607803901951	-1.40072138091223e-07\\
0.15807903951976	-1.66752490779285e-07\\
0.16008004002001	-1.53412314435254e-07\\
0.16208104052026	-1.53412314435254e-07\\
0.16408204102051	-1.53412314435254e-07\\
0.16608304152076	-1.66752490779285e-07\\
0.168084042021011	-1.53412314435254e-07\\
0.170085042521261	-1.26731961747192e-07\\
0.172086043021511	-1.33402336398105e-07\\
0.174087043521761	-1.33402336398105e-07\\
0.176088044022011	-1.33402336398105e-07\\
0.178089044522261	-1.46742512742136e-07\\
0.180090045022511	-1.53412314435254e-07\\
0.182091045522761	-1.73422865430198e-07\\
0.184092046023012	-1.53412314435254e-07\\
0.186093046523262	-1.46742512742136e-07\\
0.188094047023512	-1.80092667123315e-07\\
0.190095047523762	-1.73422865430198e-07\\
0.192096048024012	-1.80092667123315e-07\\
0.194097048524262	-1.86763041774229e-07\\
0.196098049024512	-1.73422865430198e-07\\
0.198099049524762	-1.86763041774229e-07\\
0.200100050025012	-1.80092667123315e-07\\
0.202101050525263	-1.86763041774229e-07\\
0.204102051025513	-1.73422865430198e-07\\
0.206103051525763	-1.73422865430198e-07\\
0.208104052026013	-1.60082689086167e-07\\
0.210105052526263	-1.20062160054074e-07\\
0.212106053026513	-1.60082689086167e-07\\
0.214107053526763	-1.60082689086167e-07\\
0.216108054027013	-1.86763041774229e-07\\
0.218109054527264	-1.86763041774229e-07\\
0.220110055027514	-1.53412314435254e-07\\
0.222111055527764	-1.60082689086167e-07\\
0.224112056028014	-1.40072138091223e-07\\
0.226113056528264	-1.66752490779285e-07\\
0.228114057028514	-1.93432843467346e-07\\
0.230115057528764	-1.93432843467346e-07\\
0.232116058029014	-1.86763041774229e-07\\
0.234117058529265	-1.93432843467346e-07\\
0.236118059029515	-1.93432843467346e-07\\
0.238119059529765	-1.93432843467346e-07\\
0.240120060030015	-2.06773019811377e-07\\
0.242121060530265	-1.86763041774229e-07\\
0.244122061030515	-2.26783570806321e-07\\
0.246123061530765	-2.0010321811826e-07\\
0.248124062031016	-1.93432843467346e-07\\
0.250125062531266	-2.06773019811377e-07\\
0.252126063031516	-1.86763041774229e-07\\
0.254127063531766	-2.1344339446229e-07\\
0.256128064032016	-2.53463923494383e-07\\
0.258129064532266	-2.53463923494383e-07\\
0.260130065032516	-2.60134298145296e-07\\
0.262131065532766	-2.46794121801265e-07\\
0.264132066033017	-2.40123747150352e-07\\
0.266133066533267	-2.26783570806321e-07\\
0.268134067033517	-2.20113769113203e-07\\
0.270135067533767	-2.06773019811377e-07\\
0.272136068034017	-2.26783570806321e-07\\
0.274137068534267	-1.93432843467346e-07\\
0.276138069034517	-1.93432843467346e-07\\
0.278139069534767	-2.06773019811377e-07\\
0.280140070035018	-2.06773019811377e-07\\
0.282141070535268	-2.20113769113203e-07\\
0.284142071035518	-2.40123747150352e-07\\
0.286143071535768	-2.20113769113203e-07\\
0.288144072036018	-2.26783570806321e-07\\
0.290145072536268	-2.26783570806321e-07\\
0.292146073036518	-2.1344339446229e-07\\
0.294147073536768	-2.06773019811377e-07\\
0.296148074037018	-2.06773019811377e-07\\
0.298149074537269	-2.0010321811826e-07\\
0.300150075037519	-2.06773019811377e-07\\
0.302151075537769	-2.06773019811377e-07\\
0.304152076038019	-2.06773019811377e-07\\
0.306153076538269	-2.26783570806321e-07\\
0.308154077038519	-2.20113769113203e-07\\
0.310155077538769	-2.26783570806321e-07\\
0.31215607803902	-2.33453945457234e-07\\
0.31415707853927	-2.33453945457234e-07\\
0.31615807903952	-2.33453945457234e-07\\
0.31815907953977	-2.20113769113203e-07\\
0.32016008004002	-2.40123747150352e-07\\
0.32216108054027	-2.40123747150352e-07\\
0.32416208104052	-2.46794121801265e-07\\
0.32616308154077	-2.60134298145296e-07\\
0.32816408204102	-2.40123747150352e-07\\
0.330165082541271	-2.33453945457234e-07\\
0.332166083041521	-2.73474474489327e-07\\
0.334167083541771	-2.53463923494383e-07\\
0.336168084042021	-2.80144276182445e-07\\
0.338169084542271	-2.86814650833358e-07\\
0.340170085042521	-2.53463923494383e-07\\
0.342171085542771	-2.40123747150352e-07\\
0.344172086043022	-2.73474474489327e-07\\
0.346173086543272	-2.73474474489327e-07\\
0.348174087043522	-2.66804099838414e-07\\
0.350175087543772	-2.40123747150352e-07\\
0.352176088044022	-2.33453945457234e-07\\
0.354177088544272	-2.26783570806321e-07\\
0.356178089044522	-2.40123747150352e-07\\
0.358179089544772	-2.40123747150352e-07\\
0.360180090045022	-2.33453945457234e-07\\
0.362181090545273	-2.26783570806321e-07\\
0.364182091045523	-1.93432843467346e-07\\
0.366183091545773	-1.80092667123315e-07\\
0.368184092046023	-2.1344339446229e-07\\
0.370185092546273	-2.26783570806321e-07\\
0.372186093046523	-2.33453945457234e-07\\
0.374187093546773	-2.26783570806321e-07\\
0.376188094047024	-2.20113769113203e-07\\
0.378189094547274	-2.26783570806321e-07\\
0.380190095047524	-2.20113769113203e-07\\
0.382191095547774	-1.93432843467346e-07\\
0.384192096048024	-1.66752490779285e-07\\
0.386193096548274	-1.46742512742136e-07\\
0.388194097048524	-1.13391785403161e-07\\
0.390195097548774	-1.06721983710043e-07\\
0.392196098049024	-1.0005160905913e-07\\
0.394197098549275	-8.00410580641858e-08\\
0.396198099049525	-9.33812344082167e-08\\
0.398199099549775	-9.33812344082167e-08\\
0.400200100050025	-1.20062160054074e-07\\
0.402201100550275	-1.13391785403161e-07\\
0.404202101050525	-1.26731961747192e-07\\
0.406203101550775	-1.06721983710043e-07\\
0.408204102051026	-9.33812344082167e-08\\
0.410205102551276	-1.13391785403161e-07\\
0.412206103051526	-1.26731961747192e-07\\
0.414207103551776	-1.0005160905913e-07\\
0.416208104052026	-1.0005160905913e-07\\
0.418209104552276	-9.33812344082167e-08\\
0.420210105052526	-8.00410580641858e-08\\
0.422211105552776	-8.00410580641858e-08\\
0.424212106053027	-4.66907317956674e-08\\
0.426213106553277	-4.66907317956674e-08\\
0.428214107053527	-2.00103218118259e-08\\
0.430215107553777	-2.00103218118259e-08\\
0.432216108054027	-2.00103218118259e-08\\
0.434217108554277	-4.00206436236519e-08\\
0.436218109054527	-2.00103218118259e-08\\
0.438219109554777	0\\
0.440220110055028	6.67008817201548e-09\\
0.442221110555278	-6.67008817201548e-09\\
0.444222111055528	-1.33402336398105e-08\\
0.446223111555778	-2.00103218118259e-08\\
0.448224112056028	-4.66907317956674e-08\\
0.450225112556278	-2.00103218118259e-08\\
0.452226113056528	-2.00103218118259e-08\\
0.454227113556778	-3.33505554516364e-08\\
0.456228114057029	0\\
0.458229114557279	2.00103218118259e-08\\
0.460230115057529	1.33402336398105e-08\\
0.462231115557779	-2.66804099838414e-08\\
0.464232116058029	-6.67008817201548e-08\\
0.466233116558279	-8.67114327150988e-08\\
0.468234117058529	-7.33712563710678e-08\\
0.470235117558779	-9.33812344082167e-08\\
0.47223611805903	-9.33812344082167e-08\\
0.47423711855928	-8.00410580641858e-08\\
0.47623811905953	-8.00410580641858e-08\\
0.47823911955978	-1.13391785403161e-07\\
0.48024012006003	-1.26731961747192e-07\\
0.48224112056028	-1.20062160054074e-07\\
0.48424212106053	-1.13391785403161e-07\\
0.48624312156078	-1.0005160905913e-07\\
0.488244122061031	-1.26731961747192e-07\\
0.490245122561281	-1.20062160054074e-07\\
0.492246123061531	-1.26731961747192e-07\\
0.494247123561781	-1.26731961747192e-07\\
0.496248124062031	-1.60082689086167e-07\\
0.498249124562281	-1.20062160054074e-07\\
0.500250125062531	-1.26731961747192e-07\\
0.502251125562781	-1.40072138091223e-07\\
0.504252126063031	-1.33402336398105e-07\\
0.506253126563282	-1.20062160054074e-07\\
0.508254127063532	-1.20062160054074e-07\\
0.510255127563782	-1.13391785403161e-07\\
0.512256128064032	-1.13391785403161e-07\\
0.514257128564282	-7.33712563710678e-08\\
0.516258129064532	-8.00410580641858e-08\\
0.518259129564782	-6.00310800270369e-08\\
0.520260130065032	-4.66907317956674e-08\\
0.522261130565283	-4.66907317956674e-08\\
0.524262131065533	-4.66907317956674e-08\\
0.526263131565783	-4.66907317956674e-08\\
0.528264132066033	-3.33505554516364e-08\\
0.530265132566283	-2.00103218118259e-08\\
0.532266133066533	-2.66804099838414e-08\\
0.534267133566783	-2.00103218118259e-08\\
0.536268134067034	-2.66804099838414e-08\\
0.538269134567284	-2.00103218118259e-08\\
0.540270135067534	-2.66804099838414e-08\\
0.542271135567784	-4.00206436236519e-08\\
0.544272136068034	-6.00310800270369e-08\\
0.546273136568284	-2.66804099838414e-08\\
0.548274137068534	-1.33402336398105e-08\\
0.550275137568784	-2.00103218118259e-08\\
0.552276138069035	-2.00103218118259e-08\\
0.554277138569285	6.67008817201548e-09\\
0.556278139069535	2.66804099838414e-08\\
0.558279139569785	3.33505554516364e-08\\
0.560280140070035	5.33608772634624e-08\\
0.562281140570285	4.00206436236519e-08\\
0.564282141070535	6.67008817201548e-09\\
0.566283141570785	1.33402336398105e-08\\
0.568284142071036	-6.67008817201548e-09\\
0.570285142571286	-1.33402336398105e-08\\
0.572286143071536	-2.00103218118259e-08\\
0.574287143571786	-1.33402336398105e-08\\
0.576288144072036	-2.66804099838414e-08\\
0.578289144572286	-2.00103218118259e-08\\
0.580290145072536	-4.66907317956674e-08\\
0.582291145572786	-2.66804099838414e-08\\
0.584292146073036	-3.33505554516364e-08\\
0.586293146573287	-3.33505554516364e-08\\
0.588294147073537	-2.66804099838414e-08\\
0.590295147573787	-2.00103218118259e-08\\
0.592296148074037	6.67008817201548e-09\\
0.594297148574287	1.33402336398105e-08\\
0.596298149074537	0\\
0.598299149574787	1.33402336398105e-08\\
0.600300150075038	6.67008817201548e-09\\
0.602301150575288	1.33402336398105e-08\\
0.604302151075538	-6.67008817201548e-09\\
0.606303151575788	-2.00103218118259e-08\\
0.608304152076038	-6.67008817201548e-09\\
0.610305152576288	1.33402336398105e-08\\
0.612306153076538	3.33505554516364e-08\\
0.614307153576788	6.00310800270369e-08\\
0.616308154077039	7.33712563710678e-08\\
0.618309154577289	9.33812344082167e-08\\
0.620310155077539	6.00310800270369e-08\\
0.622311155577789	2.66804099838414e-08\\
0.624312156078039	3.33505554516364e-08\\
0.626313156578289	3.33505554516364e-08\\
0.628314157078539	3.33505554516364e-08\\
0.630315157578789	2.00103218118259e-08\\
0.63231615807904	1.33402336398105e-08\\
0.63431715857929	-6.67008817201548e-09\\
0.63631815907954	0\\
0.63831915957979	-2.66804099838414e-08\\
0.64032016008004	-6.67008817201548e-08\\
0.64232116058029	-4.66907317956674e-08\\
0.64432216108054	-8.00410580641858e-08\\
0.64632316158079	-6.67008817201548e-08\\
0.64832416208104	-6.67008817201548e-08\\
0.650325162581291	-4.00206436236519e-08\\
0.652326163081541	-5.33608772634624e-08\\
0.654327163581791	-6.67008817201548e-08\\
0.656328164082041	-2.00103218118259e-08\\
0.658329164582291	0\\
0.660330165082541	-2.00103218118259e-08\\
0.662331165582791	-4.00206436236519e-08\\
0.664332166083042	-4.66907317956674e-08\\
0.666333166583292	-4.66907317956674e-08\\
0.668334167083542	-6.00310800270369e-08\\
0.670335167583792	-4.66907317956674e-08\\
0.672336168084042	-2.00103218118259e-08\\
0.674337168584292	-4.66907317956674e-08\\
0.676338169084542	-4.66907317956674e-08\\
0.678339169584792	-4.66907317956674e-08\\
0.680340170085043	-4.00206436236519e-08\\
0.682341170585293	-4.00206436236519e-08\\
0.684342171085543	-4.66907317956674e-08\\
0.686343171585793	-4.66907317956674e-08\\
0.688344172086043	-4.00206436236519e-08\\
0.690345172586293	-2.66804099838414e-08\\
0.692346173086543	-3.33505554516364e-08\\
0.694347173586793	-2.00103218118259e-08\\
0.696348174087044	0\\
0.698349174587294	0\\
0.700350175087544	6.67008817201548e-09\\
0.702351175587794	0\\
0.704352176088044	-2.66804099838414e-08\\
0.706353176588294	-2.00103218118259e-08\\
0.708354177088544	-2.66804099838414e-08\\
0.710355177588794	-2.00103218118259e-08\\
0.712356178089045	0\\
0.714357178589295	-2.00103218118259e-08\\
0.716358179089545	-2.00103218118259e-08\\
0.718359179589795	-1.33402336398105e-08\\
0.720360180090045	-3.33505554516364e-08\\
0.722361180590295	-3.33505554516364e-08\\
0.724362181090545	-2.00103218118259e-08\\
0.726363181590795	-5.33608772634624e-08\\
0.728364182091045	-8.00410580641858e-08\\
0.730365182591296	-6.67008817201548e-08\\
0.732366183091546	-7.33712563710678e-08\\
0.734367183591796	-5.33608772634624e-08\\
0.736368184092046	-4.66907317956674e-08\\
0.738369184592296	-4.66907317956674e-08\\
0.740370185092546	-2.00103218118259e-08\\
0.742371185592796	-4.66907317956674e-08\\
0.744372186093047	-5.33608772634624e-08\\
0.746373186593297	-8.00410580641858e-08\\
0.748374187093547	-4.66907317956674e-08\\
0.750375187593797	-1.33402336398105e-08\\
0.752376188094047	-4.00206436236519e-08\\
0.754377188594297	-4.00206436236519e-08\\
0.756378189094547	-4.66907317956674e-08\\
0.758379189594797	-6.00310800270369e-08\\
0.760380190095048	-7.33712563710678e-08\\
0.762381190595298	-6.00310800270369e-08\\
0.764382191095548	-6.00310800270369e-08\\
0.766383191595798	-4.66907317956674e-08\\
0.768384192096048	-6.00310800270369e-08\\
0.770385192596298	-2.66804099838414e-08\\
0.772386193096548	-4.66907317956674e-08\\
0.774387193596798	-6.67008817201548e-08\\
0.776388194097049	-7.33712563710678e-08\\
0.778389194597299	-1.0005160905913e-07\\
0.780390195097549	-9.33812344082167e-08\\
0.782391195597799	-1.0005160905913e-07\\
0.784392196098049	-8.00410580641858e-08\\
0.786393196598299	-8.67114327150988e-08\\
0.788394197098549	-6.00310800270369e-08\\
0.790395197598799	-7.33712563710678e-08\\
0.792396198099049	-5.33608772634624e-08\\
0.7943971985993	-7.33712563710678e-08\\
0.79639819909955	-8.00410580641858e-08\\
0.7983991995998	-6.00310800270369e-08\\
0.80040020010005	-6.67008817201548e-08\\
0.8024012006003	-8.00410580641858e-08\\
0.80440220110055	-8.00410580641858e-08\\
0.8064032016008	-1.0005160905913e-07\\
0.808404202101051	-8.00410580641858e-08\\
0.810405202601301	-8.67114327150988e-08\\
0.812406203101551	-8.67114327150988e-08\\
0.814407203601801	-8.00410580641858e-08\\
0.816408204102051	-7.33712563710678e-08\\
0.818409204602301	-7.33712563710678e-08\\
0.820410205102551	-7.33712563710678e-08\\
0.822411205602801	-8.00410580641858e-08\\
0.824412206103052	-8.00410580641858e-08\\
0.826413206603302	-8.00410580641858e-08\\
0.828414207103552	-8.00410580641858e-08\\
0.830415207603802	-7.33712563710678e-08\\
0.832416208104052	-8.67114327150988e-08\\
0.834417208604302	-1.0005160905913e-07\\
0.836418209104552	-9.33812344082167e-08\\
0.838419209604802	-9.33812344082167e-08\\
0.840420210105053	-8.67114327150988e-08\\
0.842421210605303	-8.00410580641858e-08\\
0.844422211105553	-7.33712563710678e-08\\
0.846423211605803	-6.67008817201548e-08\\
0.848424212106053	-5.33608772634624e-08\\
0.850425212606303	-3.33505554516364e-08\\
0.852426213106553	-2.00103218118259e-08\\
0.854427213606803	0\\
0.856428214107053	1.33402336398105e-08\\
0.858429214607304	6.67008817201548e-09\\
0.860430215107554	3.33505554516364e-08\\
0.862431215607804	5.33608772634624e-08\\
0.864432216108054	4.66907317956674e-08\\
0.866433216608304	6.67008817201548e-08\\
0.868434217108554	3.33505554516364e-08\\
0.870435217608804	5.33608772634624e-08\\
0.872436218109054	4.66907317956674e-08\\
0.874437218609305	6.00310800270369e-08\\
0.876438219109555	8.00410580641858e-08\\
0.878439219609805	6.00310800270369e-08\\
0.880440220110055	4.00206436236519e-08\\
0.882441220610305	6.00310800270369e-08\\
0.884442221110555	4.66907317956674e-08\\
0.886443221610805	0\\
0.888444222111056	0\\
0.890445222611306	0\\
0.892446223111556	-2.66804099838414e-08\\
0.894447223611806	-3.33505554516364e-08\\
0.896448224112056	-2.66804099838414e-08\\
0.898449224612306	-2.00103218118259e-08\\
0.900450225112556	-2.00103218118259e-08\\
0.902451225612806	-2.66804099838414e-08\\
0.904452226113057	-2.00103218118259e-08\\
0.906453226613307	-2.66804099838414e-08\\
0.908454227113557	-5.33608772634624e-08\\
0.910455227613807	-6.00310800270369e-08\\
0.912456228114057	-8.67114327150988e-08\\
0.914457228614307	-8.00410580641858e-08\\
0.916458229114557	-8.00410580641858e-08\\
0.918459229614807	-8.00410580641858e-08\\
0.920460230115058	-6.00310800270369e-08\\
0.922461230615308	-6.00310800270369e-08\\
0.924462231115558	-6.67008817201548e-08\\
0.926463231615808	-4.00206436236519e-08\\
0.928464232116058	-4.66907317956674e-08\\
0.930465232616308	-4.66907317956674e-08\\
0.932466233116558	-2.00103218118259e-08\\
0.934467233616808	-2.00103218118259e-08\\
0.936468234117058	-1.33402336398105e-08\\
0.938469234617309	-1.33402336398105e-08\\
0.940470235117559	-2.00103218118259e-08\\
0.942471235617809	-4.66907317956674e-08\\
0.944472236118059	-2.66804099838414e-08\\
0.946473236618309	-6.67008817201548e-09\\
0.948474237118559	6.67008817201548e-09\\
0.950475237618809	1.33402336398105e-08\\
0.95247623811906	1.33402336398105e-08\\
0.95447723861931	0\\
0.95647823911956	1.33402336398105e-08\\
0.95847923961981	6.67008817201548e-09\\
0.96048024012006	-4.00206436236519e-08\\
0.96248124062031	-4.66907317956674e-08\\
0.96448224112056	-6.00310800270369e-08\\
0.96648324162081	-7.33712563710678e-08\\
0.968484242121061	-8.00410580641858e-08\\
0.970485242621311	-8.00410580641858e-08\\
0.972486243121561	-8.00410580641858e-08\\
0.974487243621811	-1.0005160905913e-07\\
0.976488244122061	-8.67114327150988e-08\\
0.978489244622311	-7.33712563710678e-08\\
0.980490245122561	-5.33608772634624e-08\\
0.982491245622811	-5.33608772634624e-08\\
0.984492246123062	-7.33712563710678e-08\\
0.986493246623312	-7.33712563710678e-08\\
0.988494247123562	-6.00310800270369e-08\\
0.990495247623812	-3.33505554516364e-08\\
0.992496248124062	-4.66907317956674e-08\\
0.994497248624312	-6.67008817201548e-08\\
0.996498249124562	-7.33712563710678e-08\\
0.998499249624812	-5.33608772634624e-08\\
1.00050025012506	-2.66804099838414e-08\\
1.00250125062531	-2.00103218118259e-08\\
1.00450225112556	-1.33402336398105e-08\\
1.00650325162581	-2.00103218118259e-08\\
1.00850425212606	0\\
1.01050525262631	-2.00103218118259e-08\\
1.01250625312656	1.33402336398105e-08\\
1.01450725362681	2.66804099838414e-08\\
1.01650825412706	1.33402336398105e-08\\
1.01850925462731	2.66804099838414e-08\\
1.02051025512756	2.66804099838414e-08\\
1.02251125562781	3.33505554516364e-08\\
1.02451225612806	6.67008817201548e-08\\
1.02651325662831	2.66804099838414e-08\\
1.02851425712856	6.67008817201548e-09\\
1.03051525762881	-2.00103218118259e-08\\
1.03251625812906	1.33402336398105e-08\\
1.03451725862931	-2.00103218118259e-08\\
1.03651825912956	-6.67008817201548e-09\\
1.03851925962981	-6.67008817201548e-09\\
1.04052026013006	2.66804099838414e-08\\
1.04252126063032	1.33402336398105e-08\\
1.04452226113057	4.00206436236519e-08\\
1.04652326163082	4.00206436236519e-08\\
1.04852426213107	4.66907317956674e-08\\
1.05052526263132	2.66804099838414e-08\\
1.05252626313157	6.00310800270369e-08\\
1.05452726363182	2.66804099838414e-08\\
1.05652826413207	4.00206436236519e-08\\
1.05852926463232	4.00206436236519e-08\\
1.06053026513257	4.66907317956674e-08\\
1.06253126563282	8.00410580641858e-08\\
1.06453226613307	9.33812344082167e-08\\
1.06653326663332	1.26731961747192e-07\\
1.06853426713357	1.26731961747192e-07\\
1.07053526763382	1.13391785403161e-07\\
1.07253626813407	1.40072138091223e-07\\
1.07453726863432	1.33402336398105e-07\\
1.07653826913457	1.80092667123315e-07\\
1.07853926963482	1.53412314435254e-07\\
1.08054027013507	1.40072138091223e-07\\
1.08254127063532	1.40072138091223e-07\\
1.08454227113557	1.13391785403161e-07\\
1.08654327163582	9.33812344082167e-08\\
1.08854427213607	1.0005160905913e-07\\
1.09054527263632	8.67114327150988e-08\\
1.09254627313657	4.66907317956674e-08\\
1.09454727363682	7.33712563710678e-08\\
1.09654827413707	1.0005160905913e-07\\
1.09854927463732	1.13391785403161e-07\\
1.10055027513757	8.67114327150988e-08\\
1.10255127563782	9.33812344082167e-08\\
1.10455227613807	7.33712563710678e-08\\
1.10655327663832	7.33712563710678e-08\\
1.10855427713857	8.00410580641858e-08\\
1.11055527763882	6.00310800270369e-08\\
1.11255627813907	6.00310800270369e-08\\
1.11455727863932	6.67008817201548e-08\\
1.11655827913957	8.00410580641858e-08\\
1.11855927963982	7.33712563710678e-08\\
1.12056028014007	6.00310800270369e-08\\
1.12256128064032	8.00410580641858e-08\\
1.12456228114057	4.66907317956674e-08\\
1.12656328164082	4.00206436236519e-08\\
1.12856428214107	6.67008817201548e-08\\
1.13056528264132	5.33608772634624e-08\\
1.13256628314157	4.66907317956674e-08\\
1.13456728364182	4.00206436236519e-08\\
1.13656828414207	2.00103218118259e-08\\
1.13856928464232	6.00310800270369e-08\\
1.14057028514257	6.67008817201548e-08\\
1.14257128564282	5.33608772634624e-08\\
1.14457228614307	4.00206436236519e-08\\
1.14657328664332	2.66804099838414e-08\\
1.14857428714357	6.67008817201548e-09\\
1.15057528764382	6.67008817201548e-09\\
1.15257628814407	2.00103218118259e-08\\
1.15457728864432	4.00206436236519e-08\\
1.15657828914457	5.33608772634624e-08\\
1.15857928964482	4.66907317956674e-08\\
1.16058029014507	6.67008817201548e-08\\
1.16258129064532	1.13391785403161e-07\\
1.16458229114557	1.20062160054074e-07\\
1.16658329164582	9.33812344082167e-08\\
1.16858429214607	1.0005160905913e-07\\
1.17058529264632	8.00410580641858e-08\\
1.17258629314657	7.33712563710678e-08\\
1.17458729364682	8.67114327150988e-08\\
1.17658829414707	1.20062160054074e-07\\
1.17858929464732	1.40072138091223e-07\\
1.18059029514757	1.66752490779285e-07\\
1.18259129564782	1.80092667123315e-07\\
1.18459229614807	1.46742512742136e-07\\
1.18659329664832	1.60082689086167e-07\\
1.18859429714857	1.60082689086167e-07\\
1.19059529764882	1.80092667123315e-07\\
1.19259629814907	1.53412314435254e-07\\
1.19459729864932	1.40072138091223e-07\\
1.19659829914957	1.40072138091223e-07\\
1.19859929964982	1.40072138091223e-07\\
1.20060030015008	1.13391785403161e-07\\
1.20260130065033	1.20062160054074e-07\\
1.20460230115058	1.06721983710043e-07\\
1.20660330165083	1.20062160054074e-07\\
1.20860430215108	1.20062160054074e-07\\
1.21060530265133	1.13391785403161e-07\\
1.21260630315158	8.00410580641858e-08\\
1.21460730365183	9.33812344082167e-08\\
1.21660830415208	1.0005160905913e-07\\
1.21860930465233	8.67114327150988e-08\\
1.22061030515258	9.33812344082167e-08\\
1.22261130565283	1.0005160905913e-07\\
1.22461230615308	1.13391785403161e-07\\
1.22661330665333	9.33812344082167e-08\\
1.22861430715358	1.13391785403161e-07\\
1.23061530765383	1.20062160054074e-07\\
1.23261630815408	1.13391785403161e-07\\
1.23461730865433	9.33812344082167e-08\\
1.23661830915458	1.20062160054074e-07\\
1.23861930965483	1.53412314435254e-07\\
1.24062031015508	1.53412314435254e-07\\
1.24262131065533	1.33402336398105e-07\\
1.24462231115558	1.40072138091223e-07\\
1.24662331165583	1.60082689086167e-07\\
1.24862431215608	1.46742512742136e-07\\
1.25062531265633	1.53412314435254e-07\\
1.25262631315658	1.53412314435254e-07\\
1.25462731365683	1.60082689086167e-07\\
1.25662831415708	1.73422865430198e-07\\
1.25862931465733	1.86763041774229e-07\\
1.26063031515758	1.86763041774229e-07\\
1.26263131565783	1.80092667123315e-07\\
1.26463231615808	1.80092667123315e-07\\
1.26663331665833	1.80092667123315e-07\\
1.26863431715858	2.06773019811377e-07\\
1.27063531765883	2.0010321811826e-07\\
1.27263631815908	2.06773019811377e-07\\
1.27463731865933	2.1344339446229e-07\\
1.27663831915958	2.26783570806321e-07\\
1.27863931965983	2.33453945457234e-07\\
1.28064032016008	2.26783570806321e-07\\
1.28264132066033	2.33453945457234e-07\\
1.28464232116058	2.20113769113203e-07\\
1.28664332166083	2.1344339446229e-07\\
1.28864432216108	2.1344339446229e-07\\
1.29064532266133	2.06773019811377e-07\\
1.29264632316158	2.20113769113203e-07\\
1.29464732366183	2.26783570806321e-07\\
1.29664832416208	2.33453945457234e-07\\
1.29864932466233	2.06773019811377e-07\\
1.30065032516258	2.06773019811377e-07\\
1.30265132566283	2.06773019811377e-07\\
1.30465232616308	2.40123747150352e-07\\
1.30665332666333	2.1344339446229e-07\\
1.30865432716358	2.40123747150352e-07\\
1.31065532766383	2.20113769113203e-07\\
1.31265632816408	2.40123747150352e-07\\
1.31465732866433	2.40123747150352e-07\\
1.31665832916458	2.33453945457234e-07\\
1.31865932966483	2.20113769113203e-07\\
1.32066033016508	2.40123747150352e-07\\
1.32266133066533	2.40123747150352e-07\\
1.32466233116558	2.06773019811377e-07\\
1.32666333166583	2.26783570806321e-07\\
1.32866433216608	2.20113769113203e-07\\
1.33066533266633	2.0010321811826e-07\\
1.33266633316658	1.86763041774229e-07\\
1.33466733366683	1.73422865430198e-07\\
1.33666833416708	1.73422865430198e-07\\
1.33866933466733	2.06773019811377e-07\\
1.34067033516758	1.86763041774229e-07\\
1.34267133566783	2.06773019811377e-07\\
1.34467233616808	2.26783570806321e-07\\
1.34667333666833	2.26783570806321e-07\\
1.34867433716858	2.06773019811377e-07\\
1.35067533766883	1.93432843467346e-07\\
1.35267633816908	1.86763041774229e-07\\
1.35467733866933	1.86763041774229e-07\\
1.35667833916958	1.66752490779285e-07\\
1.35867933966983	1.66752490779285e-07\\
1.36068034017009	1.40072138091223e-07\\
1.36268134067034	1.53412314435254e-07\\
1.36468234117059	1.46742512742136e-07\\
1.36668334167084	1.40072138091223e-07\\
1.36868434217109	1.26731961747192e-07\\
1.37068534267134	1.33402336398105e-07\\
1.37268634317159	1.60082689086167e-07\\
1.37468734367184	1.66752490779285e-07\\
1.37668834417209	1.80092667123315e-07\\
1.37868934467234	1.86763041774229e-07\\
1.38069034517259	1.86763041774229e-07\\
1.38269134567284	1.86763041774229e-07\\
1.38469234617309	2.1344339446229e-07\\
1.38669334667334	1.66752490779285e-07\\
1.38869434717359	1.80092667123315e-07\\
1.39069534767384	1.86763041774229e-07\\
1.39269634817409	2.26783570806321e-07\\
1.39469734867434	2.26783570806321e-07\\
1.39669834917459	2.1344339446229e-07\\
1.39869934967484	2.06773019811377e-07\\
1.40070035017509	1.86763041774229e-07\\
1.40270135067534	2.0010321811826e-07\\
1.40470235117559	2.06773019811377e-07\\
1.40670335167584	2.06773019811377e-07\\
1.40870435217609	1.93432843467346e-07\\
1.41070535267634	1.66752490779285e-07\\
1.41270635317659	1.66752490779285e-07\\
1.41470735367684	1.46742512742136e-07\\
1.41670835417709	1.60082689086167e-07\\
1.41870935467734	1.46742512742136e-07\\
1.42071035517759	1.60082689086167e-07\\
1.42271135567784	1.80092667123315e-07\\
1.42471235617809	1.60082689086167e-07\\
1.42671335667834	1.66752490779285e-07\\
1.42871435717859	1.40072138091223e-07\\
1.43071535767884	1.20062160054074e-07\\
1.43271635817909	1.33402336398105e-07\\
1.43471735867934	1.20062160054074e-07\\
1.43671835917959	1.0005160905913e-07\\
1.43871935967984	1.06721983710043e-07\\
1.44072036018009	1.13391785403161e-07\\
1.44272136068034	1.06721983710043e-07\\
1.44472236118059	1.20062160054074e-07\\
1.44672336168084	1.40072138091223e-07\\
1.44872436218109	1.40072138091223e-07\\
1.45072536268134	1.26731961747192e-07\\
1.45272636318159	1.40072138091223e-07\\
1.45472736368184	1.20062160054074e-07\\
1.45672836418209	1.13391785403161e-07\\
1.45872936468234	1.26731961747192e-07\\
1.46073036518259	1.0005160905913e-07\\
1.46273136568284	9.33812344082167e-08\\
1.46473236618309	7.33712563710678e-08\\
1.46673336668334	5.33608772634624e-08\\
1.46873436718359	6.67008817201548e-08\\
1.47073536768384	4.66907317956674e-08\\
1.47273636818409	6.00310800270369e-08\\
1.47473736868434	5.33608772634624e-08\\
1.47673836918459	4.00206436236519e-08\\
1.47873936968484	2.66804099838414e-08\\
1.48074037018509	-6.67008817201548e-09\\
1.48274137068534	-1.33402336398105e-08\\
1.48474237118559	-6.67008817201548e-09\\
1.48674337168584	0\\
1.48874437218609	-2.00103218118259e-08\\
1.49074537268634	-2.00103218118259e-08\\
1.49274637318659	-4.00206436236519e-08\\
1.49474737368684	-2.00103218118259e-08\\
1.49674837418709	0\\
1.49874937468734	6.67008817201548e-09\\
1.50075037518759	-6.67008817201548e-09\\
1.50275137568784	1.33402336398105e-08\\
1.50475237618809	-6.67008817201548e-09\\
1.50675337668834	-2.00103218118259e-08\\
1.50875437718859	-4.00206436236519e-08\\
1.51075537768884	-3.33505554516364e-08\\
1.51275637818909	-5.33608772634624e-08\\
1.51475737868934	-8.00410580641858e-08\\
1.51675837918959	-6.67008817201548e-08\\
1.51875937968984	-5.33608772634624e-08\\
1.5207603801901	-3.33505554516364e-08\\
1.52276138069035	-3.33505554516364e-08\\
1.5247623811906	-4.66907317956674e-08\\
1.52676338169085	-3.33505554516364e-08\\
1.5287643821911	-3.33505554516364e-08\\
1.53076538269135	-3.33505554516364e-08\\
1.5327663831916	-2.00103218118259e-08\\
1.53476738369185	-4.66907317956674e-08\\
1.5367683841921	-3.33505554516364e-08\\
1.53876938469235	-5.33608772634624e-08\\
1.5407703851926	-8.00410580641858e-08\\
1.54277138569285	-8.00410580641858e-08\\
1.5447723861931	-8.00410580641858e-08\\
1.54677338669335	-9.33812344082167e-08\\
1.5487743871936	-1.06721983710043e-07\\
1.55077538769385	-1.33402336398105e-07\\
1.5527763881941	-1.46742512742136e-07\\
1.55477738869435	-1.40072138091223e-07\\
1.5567783891946	-1.40072138091223e-07\\
1.55877938969485	-1.60082689086167e-07\\
1.5607803901951	-1.66752490779285e-07\\
1.56278139069535	-1.80092667123315e-07\\
1.5647823911956	-1.86763041774229e-07\\
1.56678339169585	-1.93432843467346e-07\\
1.5687843921961	-2.0010321811826e-07\\
1.57078539269635	-1.86763041774229e-07\\
1.5727863931966	-2.1344339446229e-07\\
1.57478739369685	-2.26783570806321e-07\\
1.5767883941971	-2.26783570806321e-07\\
1.57878939469735	-2.40123747150352e-07\\
1.5807903951976	-2.60134298145296e-07\\
1.58279139569785	-2.46794121801265e-07\\
1.5847923961981	-2.60134298145296e-07\\
1.58679339669835	-2.60134298145296e-07\\
1.5887943971986	-2.26783570806321e-07\\
1.59079539769885	-2.26783570806321e-07\\
1.5927963981991	-2.46794121801265e-07\\
1.59479739869935	-2.26783570806321e-07\\
1.5967983991996	-2.1344339446229e-07\\
1.59879939969985	-1.86763041774229e-07\\
1.6008004002001	-1.73422865430198e-07\\
1.60280140070035	-1.53412314435254e-07\\
1.6048024012006	-1.26731961747192e-07\\
1.60680340170085	-1.46742512742136e-07\\
1.6088044022011	-1.33402336398105e-07\\
1.61080540270135	-1.33402336398105e-07\\
1.6128064032016	-1.26731961747192e-07\\
1.61480740370185	-1.26731961747192e-07\\
1.6168084042021	-9.33812344082167e-08\\
1.61880940470235	-8.67114327150988e-08\\
1.6208104052026	-7.33712563710678e-08\\
1.62281140570285	-6.67008817201548e-08\\
1.6248124062031	-8.00410580641858e-08\\
1.62681340670335	-6.67008817201548e-08\\
1.6288144072036	-8.67114327150988e-08\\
1.63081540770385	-8.67114327150988e-08\\
1.6328164082041	-7.33712563710678e-08\\
1.63481740870435	-8.00410580641858e-08\\
1.6368184092046	-8.00410580641858e-08\\
1.63881940970485	-9.33812344082167e-08\\
1.6408204102051	-1.0005160905913e-07\\
1.64282141070535	-8.67114327150988e-08\\
1.6448224112056	-9.33812344082167e-08\\
1.64682341170585	-6.00310800270369e-08\\
1.6488244122061	-5.33608772634624e-08\\
1.65082541270635	-5.33608772634624e-08\\
1.6528264132066	-4.66907317956674e-08\\
1.65482741370685	-6.67008817201548e-08\\
1.6568284142071	-4.66907317956674e-08\\
1.65882941470735	-4.66907317956674e-08\\
1.6608304152076	-3.33505554516364e-08\\
1.66283141570785	-1.33402336398105e-08\\
1.6648324162081	0\\
1.66683341670835	-6.67008817201548e-09\\
1.6688344172086	-2.00103218118259e-08\\
1.67083541770885	-2.66804099838414e-08\\
1.6728364182091	-2.00103218118259e-08\\
1.67483741870935	-2.00103218118259e-08\\
1.6768384192096	-2.00103218118259e-08\\
1.67883941970985	-2.00103218118259e-08\\
1.68084042021011	-2.00103218118259e-08\\
1.68284142071036	-2.66804099838414e-08\\
1.68484242121061	-5.33608772634624e-08\\
1.68684342171086	-4.00206436236519e-08\\
1.68884442221111	-4.00206436236519e-08\\
1.69084542271136	-5.33608772634624e-08\\
1.69284642321161	-4.66907317956674e-08\\
1.69484742371186	-2.00103218118259e-08\\
1.69684842421211	0\\
1.69884942471236	2.66804099838414e-08\\
1.70085042521261	5.33608772634624e-08\\
1.70285142571286	3.33505554516364e-08\\
1.70485242621311	5.33608772634624e-08\\
1.70685342671336	5.33608772634624e-08\\
1.70885442721361	6.67008817201548e-08\\
1.71085542771386	6.00310800270369e-08\\
1.71285642821411	7.33712563710678e-08\\
1.71485742871436	1.0005160905913e-07\\
1.71685842921461	9.33812344082167e-08\\
1.71885942971486	7.33712563710678e-08\\
1.72086043021511	7.33712563710678e-08\\
1.72286143071536	9.33812344082167e-08\\
1.72486243121561	6.67008817201548e-08\\
1.72686343171586	8.67114327150988e-08\\
1.72886443221611	8.67114327150988e-08\\
1.73086543271636	6.00310800270369e-08\\
1.73286643321661	8.00410580641858e-08\\
1.73486743371686	4.66907317956674e-08\\
1.73686843421711	5.33608772634624e-08\\
1.73886943471736	5.33608772634624e-08\\
1.74087043521761	4.00206436236519e-08\\
1.74287143571786	2.00103218118259e-08\\
1.74487243621811	4.00206436236519e-08\\
1.74687343671836	2.00103218118259e-08\\
1.74887443721861	6.67008817201548e-09\\
1.75087543771886	3.33505554516364e-08\\
1.75287643821911	2.66804099838414e-08\\
1.75487743871936	-1.33402336398105e-08\\
1.75687843921961	-3.33505554516364e-08\\
1.75887943971986	-2.00103218118259e-08\\
1.76088044022011	-4.00206436236519e-08\\
1.76288144072036	-2.00103218118259e-08\\
1.76488244122061	-2.66804099838414e-08\\
1.76688344172086	-2.00103218118259e-08\\
1.76888444222111	-2.00103218118259e-08\\
1.77088544272136	0\\
1.77288644322161	0\\
1.77488744372186	-2.00103218118259e-08\\
1.77688844422211	-2.00103218118259e-08\\
1.77888944472236	-2.66804099838414e-08\\
1.78089044522261	-3.33505554516364e-08\\
1.78289144572286	-4.66907317956674e-08\\
1.78489244622311	-3.33505554516364e-08\\
1.78689344672336	-5.33608772634624e-08\\
1.78889444722361	-3.33505554516364e-08\\
1.79089544772386	-5.33608772634624e-08\\
1.79289644822411	-4.66907317956674e-08\\
1.79489744872436	-6.67008817201548e-08\\
1.79689844922461	-5.33608772634624e-08\\
1.79889944972486	-6.00310800270369e-08\\
1.80090045022511	-6.00310800270369e-08\\
1.80290145072536	-4.00206436236519e-08\\
1.80490245122561	-2.00103218118259e-08\\
1.80690345172586	-2.00103218118259e-08\\
1.80890445222611	-2.66804099838414e-08\\
1.81090545272636	-5.33608772634624e-08\\
1.81290645322661	-4.66907317956674e-08\\
1.81490745372686	-4.66907317956674e-08\\
1.81690845422711	-4.66907317956674e-08\\
1.81890945472736	-5.33608772634624e-08\\
1.82091045522761	-2.00103218118259e-08\\
1.82291145572786	-1.33402336398105e-08\\
1.82491245622811	-4.00206436236519e-08\\
1.82691345672836	-3.33505554516364e-08\\
1.82891445722861	-4.00206436236519e-08\\
1.83091545772886	-3.33505554516364e-08\\
1.83291645822911	0\\
1.83491745872936	-1.33402336398105e-08\\
1.83691845922961	0\\
1.83891945972987	-2.66804099838414e-08\\
1.84092046023012	-4.00206436236519e-08\\
1.84292146073037	-2.00103218118259e-08\\
1.84492246123062	-6.67008817201548e-09\\
1.84692346173087	0\\
1.84892446223112	-2.00103218118259e-08\\
1.85092546273137	-6.67008817201548e-09\\
1.85292646323162	-1.33402336398105e-08\\
1.85492746373187	1.33402336398105e-08\\
1.85692846423212	-2.00103218118259e-08\\
1.85892946473237	-2.00103218118259e-08\\
1.86093046523262	-2.66804099838414e-08\\
1.86293146573287	-2.00103218118259e-08\\
1.86493246623312	-2.66804099838414e-08\\
1.86693346673337	-3.33505554516364e-08\\
1.86893446723362	-2.00103218118259e-08\\
1.87093546773387	-2.66804099838414e-08\\
1.87293646823412	-4.66907317956674e-08\\
1.87493746873437	-6.00310800270369e-08\\
1.87693846923462	-4.00206436236519e-08\\
1.87893946973487	-2.00103218118259e-08\\
1.88094047023512	-2.66804099838414e-08\\
1.88294147073537	0\\
1.88494247123562	-6.67008817201548e-09\\
1.88694347173587	-1.33402336398105e-08\\
1.88894447223612	-4.00206436236519e-08\\
1.89094547273637	-5.33608772634624e-08\\
1.89294647323662	-6.67008817201548e-08\\
1.89494747373687	-4.00206436236519e-08\\
1.89694847423712	-4.00206436236519e-08\\
1.89894947473737	-4.00206436236519e-08\\
1.90095047523762	-1.33402336398105e-08\\
1.90295147573787	2.00103218118259e-08\\
1.90495247623812	-6.67008817201548e-09\\
1.90695347673837	-1.33402336398105e-08\\
1.90895447723862	-2.00103218118259e-08\\
1.91095547773887	-2.00103218118259e-08\\
1.91295647823912	-3.33505554516364e-08\\
1.91495747873937	-6.67008817201548e-08\\
1.91695847923962	-7.33712563710678e-08\\
1.91895947973987	-6.00310800270369e-08\\
1.92096048024012	-8.67114327150988e-08\\
1.92296148074037	-8.00410580641858e-08\\
1.92496248124062	-8.67114327150988e-08\\
1.92696348174087	-8.00410580641858e-08\\
1.92896448224112	-8.00410580641858e-08\\
1.93096548274137	-7.33712563710678e-08\\
1.93296648324162	-8.00410580641858e-08\\
1.93496748374187	-6.67008817201548e-08\\
1.93696848424212	-3.33505554516364e-08\\
1.93896948474237	-5.33608772634624e-08\\
1.94097048524262	-8.00410580641858e-08\\
1.94297148574287	-8.00410580641858e-08\\
1.94497248624312	-8.00410580641858e-08\\
1.94697348674337	-8.67114327150988e-08\\
1.94897448724362	-8.00410580641858e-08\\
1.95097548774387	-6.67008817201548e-08\\
1.95297648824412	-6.67008817201548e-08\\
1.95497748874437	-4.66907317956674e-08\\
1.95697848924462	-4.66907317956674e-08\\
1.95897948974487	-6.67008817201548e-08\\
1.96098049024512	-4.66907317956674e-08\\
1.96298149074537	-4.00206436236519e-08\\
1.96498249124562	-3.33505554516364e-08\\
1.96698349174587	-4.66907317956674e-08\\
1.96898449224612	-6.67008817201548e-08\\
1.97098549274637	-8.00410580641858e-08\\
1.97298649324662	-8.00410580641858e-08\\
1.97498749374687	-6.00310800270369e-08\\
1.97698849424712	-8.00410580641858e-08\\
1.97898949474737	-8.00410580641858e-08\\
1.98099049524762	-7.33712563710678e-08\\
1.98299149574787	-9.33812344082167e-08\\
1.98499249624812	-6.00310800270369e-08\\
1.98699349674837	-5.33608772634624e-08\\
1.98899449724862	-4.66907317956674e-08\\
1.99099549774887	-7.33712563710678e-08\\
1.99299649824912	-7.33712563710678e-08\\
1.99499749874937	-7.33712563710678e-08\\
1.99699849924962	-6.00310800270369e-08\\
1.99899949974988	-6.00310800270369e-08\\
2.00100050025013	-4.00206436236519e-08\\
2.00300150075038	-2.00103218118259e-08\\
2.00500250125063	-6.67008817201548e-09\\
2.00700350175088	-1.33402336398105e-08\\
2.00900450225113	-6.67008817201548e-09\\
2.01100550275138	0\\
2.01300650325163	0\\
2.01500750375188	0\\
2.01700850425213	-2.00103218118259e-08\\
2.01900950475238	-3.33505554516364e-08\\
2.02101050525263	-2.00103218118259e-08\\
2.02301150575288	-1.33402336398105e-08\\
2.02501250625313	-1.33402336398105e-08\\
2.02701350675338	-1.33402336398105e-08\\
2.02901450725363	-6.67008817201548e-09\\
2.03101550775388	-1.33402336398105e-08\\
2.03301650825413	0\\
2.03501750875438	-1.33402336398105e-08\\
2.03701850925463	-3.33505554516364e-08\\
2.03901950975488	-3.33505554516364e-08\\
2.04102051025513	-7.33712563710678e-08\\
2.04302151075538	-6.00310800270369e-08\\
2.04502251125563	-6.00310800270369e-08\\
2.04702351175588	-6.67008817201548e-08\\
2.04902451225613	-4.66907317956674e-08\\
2.05102551275638	-2.00103218118259e-08\\
2.05302651325663	-1.33402336398105e-08\\
2.05502751375688	-1.33402336398105e-08\\
2.05702851425713	-4.00206436236519e-08\\
2.05902951475738	-3.33505554516364e-08\\
2.06103051525763	-2.00103218118259e-08\\
2.06303151575788	-2.00103218118259e-08\\
2.06503251625813	-5.33608772634624e-08\\
2.06703351675838	-2.00103218118259e-08\\
2.06903451725863	1.33402336398105e-08\\
2.07103551775888	-6.67008817201548e-09\\
2.07303651825913	1.33402336398105e-08\\
2.07503751875938	0\\
2.07703851925963	0\\
2.07903951975988	-2.00103218118259e-08\\
2.08104052026013	-1.33402336398105e-08\\
2.08304152076038	-1.33402336398105e-08\\
2.08504252126063	-2.66804099838414e-08\\
2.08704352176088	-4.66907317956674e-08\\
2.08904452226113	-2.66804099838414e-08\\
2.09104552276138	-2.00103218118259e-08\\
2.09304652326163	-1.33402336398105e-08\\
2.09504752376188	1.33402336398105e-08\\
2.09704852426213	1.33402336398105e-08\\
2.09904952476238	1.33402336398105e-08\\
2.10105052526263	4.00206436236519e-08\\
2.10305152576288	3.33505554516364e-08\\
2.10505252626313	4.00206436236519e-08\\
2.10705352676338	3.33505554516364e-08\\
2.10905452726363	2.66804099838414e-08\\
2.11105552776388	2.66804099838414e-08\\
2.11305652826413	1.33402336398105e-08\\
2.11505752876438	0\\
2.11705852926463	-2.00103218118259e-08\\
2.11905952976488	-3.33505554516364e-08\\
2.12106053026513	-2.00103218118259e-08\\
2.12306153076538	-5.33608772634624e-08\\
2.12506253126563	-4.66907317956674e-08\\
2.12706353176588	-4.00206436236519e-08\\
2.12906453226613	-1.33402336398105e-08\\
2.13106553276638	0\\
2.13306653326663	2.66804099838414e-08\\
2.13506753376688	4.00206436236519e-08\\
2.13706853426713	6.67008817201548e-08\\
2.13906953476738	6.67008817201548e-08\\
2.14107053526763	4.66907317956674e-08\\
2.14307153576788	4.66907317956674e-08\\
2.14507253626813	2.00103218118259e-08\\
2.14707353676838	0\\
2.14907453726863	-6.67008817201548e-09\\
2.15107553776888	4.00206436236519e-08\\
2.15307653826913	4.00206436236519e-08\\
2.15507753876938	3.33505554516364e-08\\
2.15707853926963	3.33505554516364e-08\\
2.15907953976988	2.66804099838414e-08\\
2.16108054027013	1.33402336398105e-08\\
2.16308154077039	-1.33402336398105e-08\\
2.16508254127064	-2.66804099838414e-08\\
2.16708354177089	-1.33402336398105e-08\\
2.16908454227114	-2.66804099838414e-08\\
2.17108554277139	-1.33402336398105e-08\\
2.17308654327164	-4.66907317956674e-08\\
2.17508754377189	-3.33505554516364e-08\\
2.17708854427214	-4.00206436236519e-08\\
2.17908954477239	-6.00310800270369e-08\\
2.18109054527264	-4.66907317956674e-08\\
2.18309154577289	-4.66907317956674e-08\\
2.18509254627314	-6.00310800270369e-08\\
2.18709354677339	-5.33608772634624e-08\\
2.18909454727364	-8.00410580641858e-08\\
2.19109554777389	-8.00410580641858e-08\\
2.19309654827414	-2.66804099838414e-08\\
2.19509754877439	6.67008817201548e-09\\
2.19709854927464	2.00103218118259e-08\\
2.19909954977489	1.33402336398105e-08\\
2.20110055027514	2.66804099838414e-08\\
2.20310155077539	6.67008817201548e-08\\
2.20510255127564	6.67008817201548e-08\\
2.20710355177589	7.33712563710678e-08\\
2.20910455227614	7.33712563710678e-08\\
2.21110555277639	6.67008817201548e-08\\
2.21310655327664	9.33812344082167e-08\\
2.21510755377689	9.33812344082167e-08\\
2.21710855427714	9.33812344082167e-08\\
2.21910955477739	8.67114327150988e-08\\
2.22111055527764	1.0005160905913e-07\\
2.22311155577789	8.67114327150988e-08\\
2.22511255627814	1.0005160905913e-07\\
2.22711355677839	6.00310800270369e-08\\
2.22911455727864	7.33712563710678e-08\\
2.23111555777889	8.67114327150988e-08\\
2.23311655827914	8.00410580641858e-08\\
2.23511755877939	4.66907317956674e-08\\
2.23711855927964	6.67008817201548e-08\\
2.23911955977989	7.33712563710678e-08\\
2.24112056028014	6.00310800270369e-08\\
2.24312156078039	2.00103218118259e-08\\
2.24512256128064	6.67008817201548e-08\\
2.24712356178089	4.66907317956674e-08\\
2.24912456228114	4.66907317956674e-08\\
2.25112556278139	6.00310800270369e-08\\
2.25312656328164	6.67008817201548e-08\\
2.25512756378189	8.00410580641858e-08\\
2.25712856428214	8.67114327150988e-08\\
2.25912956478239	8.67114327150988e-08\\
2.26113056528264	9.33812344082167e-08\\
2.26313156578289	9.33812344082167e-08\\
2.26513256628314	9.33812344082167e-08\\
2.26713356678339	9.33812344082167e-08\\
2.26913456728364	9.33812344082167e-08\\
2.27113556778389	7.33712563710678e-08\\
2.27313656828414	6.67008817201548e-08\\
2.27513756878439	8.00410580641858e-08\\
2.27713856928464	8.00410580641858e-08\\
2.27913956978489	7.33712563710678e-08\\
2.28114057028514	9.33812344082167e-08\\
2.28314157078539	8.67114327150988e-08\\
2.28514257128564	6.67008817201548e-08\\
2.28714357178589	2.00103218118259e-08\\
2.28914457228614	-6.67008817201548e-09\\
2.29114557278639	6.67008817201548e-09\\
2.29314657328664	0\\
2.29514757378689	-1.33402336398105e-08\\
2.29714857428714	-6.67008817201548e-09\\
2.29914957478739	2.66804099838414e-08\\
2.30115057528764	2.00103218118259e-08\\
2.30315157578789	5.33608772634624e-08\\
2.30515257628814	3.33505554516364e-08\\
2.30715357678839	2.00103218118259e-08\\
2.30915457728864	-6.67008817201548e-09\\
2.31115557778889	2.00103218118259e-08\\
2.31315657828914	5.33608772634624e-08\\
2.31515757878939	8.00410580641858e-08\\
2.31715857928964	9.33812344082167e-08\\
2.31915957978989	9.33812344082167e-08\\
2.32116058029015	8.67114327150988e-08\\
2.3231615807904	7.33712563710678e-08\\
2.32516258129065	3.33505554516364e-08\\
2.3271635817909	4.00206436236519e-08\\
2.32916458229115	5.33608772634624e-08\\
2.3311655827914	3.33505554516364e-08\\
2.33316658329165	6.67008817201548e-08\\
2.3351675837919	6.67008817201548e-08\\
2.33716858429215	7.33712563710678e-08\\
2.3391695847924	6.67008817201548e-08\\
2.34117058529265	4.66907317956674e-08\\
2.3431715857929	2.00103218118259e-08\\
2.34517258629315	4.66907317956674e-08\\
2.3471735867934	6.67008817201548e-08\\
2.34917458729365	8.67114327150988e-08\\
2.3511755877939	1.0005160905913e-07\\
2.35317658829415	1.13391785403161e-07\\
2.3551775887944	8.67114327150988e-08\\
2.35717858929465	9.33812344082167e-08\\
2.3591795897949	1.20062160054074e-07\\
2.36118059029515	1.46742512742136e-07\\
2.3631815907954	1.46742512742136e-07\\
2.36518259129565	1.53412314435254e-07\\
2.3671835917959	1.80092667123315e-07\\
2.36918459229615	1.60082689086167e-07\\
2.3711855927964	1.73422865430198e-07\\
2.37318659329665	1.40072138091223e-07\\
2.3751875937969	1.33402336398105e-07\\
2.37718859429715	1.40072138091223e-07\\
2.3791895947974	1.53412314435254e-07\\
2.38119059529765	1.40072138091223e-07\\
2.3831915957979	1.60082689086167e-07\\
2.38519259629815	2.0010321811826e-07\\
2.3871935967984	1.60082689086167e-07\\
2.38919459729865	1.80092667123315e-07\\
2.3911955977989	2.06773019811377e-07\\
2.39319659829915	2.0010321811826e-07\\
2.3951975987994	2.06773019811377e-07\\
2.39719859929965	2.06773019811377e-07\\
2.3991995997999	1.86763041774229e-07\\
2.40120060030015	1.73422865430198e-07\\
2.4032016008004	1.66752490779285e-07\\
2.40520260130065	1.46742512742136e-07\\
2.4072036018009	1.40072138091223e-07\\
2.40920460230115	1.26731961747192e-07\\
2.4112056028014	1.0005160905913e-07\\
2.41320660330165	1.26731961747192e-07\\
2.4152076038019	1.40072138091223e-07\\
2.41720860430215	1.13391785403161e-07\\
2.4192096048024	1.0005160905913e-07\\
2.42121060530265	9.33812344082167e-08\\
2.4232116058029	1.0005160905913e-07\\
2.42521260630315	9.33812344082167e-08\\
2.4272136068034	1.0005160905913e-07\\
2.42921460730365	9.33812344082167e-08\\
2.4312156078039	1.06721983710043e-07\\
2.43321660830415	1.0005160905913e-07\\
2.4352176088044	1.0005160905913e-07\\
2.43721860930465	9.33812344082167e-08\\
2.4392196098049	9.33812344082167e-08\\
2.44122061030515	1.0005160905913e-07\\
2.4432216108054	1.20062160054074e-07\\
2.44522261130565	1.26731961747192e-07\\
2.4472236118059	1.53412314435254e-07\\
2.44922461230615	1.33402336398105e-07\\
2.4512256128064	1.40072138091223e-07\\
2.45322661330665	1.26731961747192e-07\\
2.4552276138069	1.13391785403161e-07\\
2.45722861430715	1.0005160905913e-07\\
2.4592296148074	1.0005160905913e-07\\
2.46123061530765	1.06721983710043e-07\\
2.4632316158079	1.26731961747192e-07\\
2.46523261630815	1.26731961747192e-07\\
2.4672336168084	1.0005160905913e-07\\
2.46923461730865	9.33812344082167e-08\\
2.4712356178089	1.06721983710043e-07\\
2.47323661830915	1.13391785403161e-07\\
2.4752376188094	8.67114327150988e-08\\
2.47723861930965	5.33608772634624e-08\\
2.4792396198099	8.00410580641858e-08\\
2.48124062031015	9.33812344082167e-08\\
2.48324162081041	6.00310800270369e-08\\
2.48524262131066	4.66907317956674e-08\\
2.48724362181091	4.00206436236519e-08\\
2.48924462231116	2.00103218118259e-08\\
2.49124562281141	4.00206436236519e-08\\
2.49324662331166	1.33402336398105e-08\\
2.49524762381191	1.33402336398105e-08\\
2.49724862431216	0\\
2.49924962481241	-2.00103218118259e-08\\
2.50125062531266	-2.00103218118259e-08\\
2.50325162581291	-2.00103218118259e-08\\
2.50525262631316	0\\
2.50725362681341	0\\
2.50925462731366	2.00103218118259e-08\\
2.51125562781391	6.67008817201548e-09\\
2.51325662831416	0\\
2.51525762881441	-6.67008817201548e-09\\
2.51725862931466	1.33402336398105e-08\\
2.51925962981491	-1.33402336398105e-08\\
2.52126063031516	-2.00103218118259e-08\\
2.52326163081541	-1.33402336398105e-08\\
2.52526263131566	-2.00103218118259e-08\\
2.52726363181591	-1.33402336398105e-08\\
2.52926463231616	-2.00103218118259e-08\\
2.53126563281641	-2.00103218118259e-08\\
2.53326663331666	0\\
2.53526763381691	-6.67008817201548e-09\\
2.53726863431716	-2.00103218118259e-08\\
2.53926963481741	-1.33402336398105e-08\\
2.54127063531766	-3.33505554516364e-08\\
2.54327163581791	-3.33505554516364e-08\\
2.54527263631816	-6.67008817201548e-08\\
2.54727363681841	-5.33608772634624e-08\\
2.54927463731866	-6.67008817201548e-08\\
2.55127563781891	-6.00310800270369e-08\\
2.55327663831916	-8.00410580641858e-08\\
2.55527763881941	-9.33812344082167e-08\\
2.55727863931966	-1.0005160905913e-07\\
2.55927963981991	-1.0005160905913e-07\\
2.56128064032016	-1.13391785403161e-07\\
2.56328164082041	-1.0005160905913e-07\\
2.56528264132066	-1.06721983710043e-07\\
2.56728364182091	-9.33812344082167e-08\\
2.56928464232116	-8.00410580641858e-08\\
2.57128564282141	-6.00310800270369e-08\\
2.57328664332166	-4.00206436236519e-08\\
2.57528764382191	-8.00410580641858e-08\\
2.57728864432216	-4.66907317956674e-08\\
2.57928964482241	-4.66907317956674e-08\\
2.58129064532266	-4.00206436236519e-08\\
2.58329164582291	-4.66907317956674e-08\\
2.58529264632316	-6.67008817201548e-08\\
2.58729364682341	-4.00206436236519e-08\\
2.58929464732366	-4.66907317956674e-08\\
2.59129564782391	-6.00310800270369e-08\\
2.59329664832416	-4.66907317956674e-08\\
2.59529764882441	-7.33712563710678e-08\\
2.59729864932466	-5.33608772634624e-08\\
2.59929964982491	-2.00103218118259e-08\\
2.60130065032516	-3.33505554516364e-08\\
2.60330165082541	-2.00103218118259e-08\\
2.60530265132566	-1.33402336398105e-08\\
2.60730365182591	-4.00206436236519e-08\\
2.60930465232616	-2.66804099838414e-08\\
2.61130565282641	-3.33505554516364e-08\\
2.61330665332666	-1.33402336398105e-08\\
2.61530765382691	0\\
2.61730865432716	0\\
2.61930965482741	2.00103218118259e-08\\
2.62131065532766	-2.00103218118259e-08\\
2.62331165582791	-1.33402336398105e-08\\
2.62531265632816	-2.00103218118259e-08\\
2.62731365682841	-1.33402336398105e-08\\
2.62931465732866	-6.67008817201548e-09\\
2.63131565782891	-6.67008817201548e-09\\
2.63331665832916	-1.33402336398105e-08\\
2.63531765882941	-3.33505554516364e-08\\
2.63731865932966	-2.00103218118259e-08\\
2.63931965982992	-4.00206436236519e-08\\
2.64132066033017	-1.33402336398105e-08\\
2.64332166083042	-1.33402336398105e-08\\
2.64532266133067	-6.00310800270369e-08\\
2.64732366183092	-3.33505554516364e-08\\
2.64932466233117	-3.33505554516364e-08\\
2.65132566283142	-2.00103218118259e-08\\
2.65332666333167	-1.33402336398105e-08\\
2.65532766383192	-2.00103218118259e-08\\
2.65732866433217	-3.33505554516364e-08\\
2.65932966483242	-5.33608772634624e-08\\
2.66133066533267	-6.67008817201548e-08\\
2.66333166583292	-1.0005160905913e-07\\
2.66533266633317	-8.67114327150988e-08\\
2.66733366683342	-8.00410580641858e-08\\
2.66933466733367	-1.0005160905913e-07\\
2.67133566783392	-1.06721983710043e-07\\
2.67333666833417	-1.13391785403161e-07\\
2.67533766883442	-1.06721983710043e-07\\
2.67733866933467	-1.26731961747192e-07\\
2.67933966983492	-1.13391785403161e-07\\
2.68134067033517	-8.67114327150988e-08\\
2.68334167083542	-8.00410580641858e-08\\
2.68534267133567	-1.0005160905913e-07\\
2.68734367183592	-1.13391785403161e-07\\
2.68934467233617	-1.26731961747192e-07\\
2.69134567283642	-1.13391785403161e-07\\
2.69334667333667	-9.33812344082167e-08\\
2.69534767383692	-1.06721983710043e-07\\
2.69734867433717	-6.00310800270369e-08\\
2.69934967483742	-6.67008817201548e-08\\
2.70135067533767	-7.33712563710678e-08\\
2.70335167583792	-8.67114327150988e-08\\
2.70535267633817	-9.33812344082167e-08\\
2.70735367683842	-9.33812344082167e-08\\
2.70935467733867	-9.33812344082167e-08\\
2.71135567783892	-1.0005160905913e-07\\
2.71335667833917	-6.67008817201548e-08\\
2.71535767883942	-5.33608772634624e-08\\
2.71735867933967	-8.67114327150988e-08\\
2.71935967983992	-7.33712563710678e-08\\
2.72136068034017	-6.67008817201548e-08\\
2.72336168084042	-7.33712563710678e-08\\
2.72536268134067	-4.66907317956674e-08\\
2.72736368184092	-6.00310800270369e-08\\
2.72936468234117	-7.33712563710678e-08\\
2.73136568284142	-1.13391785403161e-07\\
2.73336668334167	-9.33812344082167e-08\\
2.73536768384192	-9.33812344082167e-08\\
2.73736868434217	-1.0005160905913e-07\\
2.73936968484242	-1.20062160054074e-07\\
2.74137068534267	-1.06721983710043e-07\\
2.74337168584292	-1.20062160054074e-07\\
2.74537268634317	-1.26731961747192e-07\\
2.74737368684342	-1.26731961747192e-07\\
2.74937468734367	-1.33402336398105e-07\\
2.75137568784392	-1.33402336398105e-07\\
2.75337668834417	-1.46742512742136e-07\\
2.75537768884442	-1.40072138091223e-07\\
2.75737868934467	-1.46742512742136e-07\\
2.75937968984492	-1.60082689086167e-07\\
2.76138069034517	-1.40072138091223e-07\\
2.76338169084542	-1.26731961747192e-07\\
2.76538269134567	-9.33812344082167e-08\\
2.76738369184592	-1.26731961747192e-07\\
2.76938469234617	-1.13391785403161e-07\\
2.77138569284642	-1.20062160054074e-07\\
2.77338669334667	-1.26731961747192e-07\\
2.77538769384692	-1.06721983710043e-07\\
2.77738869434717	-1.26731961747192e-07\\
2.77938969484742	-9.33812344082167e-08\\
2.78139069534767	-8.67114327150988e-08\\
2.78339169584792	-7.33712563710678e-08\\
2.78539269634817	-5.33608772634624e-08\\
2.78739369684842	-4.66907317956674e-08\\
2.78939469734867	-3.33505554516364e-08\\
2.79139569784892	-2.00103218118259e-08\\
2.79339669834917	-2.00103218118259e-08\\
2.79539769884942	-6.67008817201548e-09\\
2.79739869934967	-2.00103218118259e-08\\
2.79939969984992	-1.33402336398105e-08\\
2.80140070035018	-2.00103218118259e-08\\
2.80340170085043	1.33402336398105e-08\\
2.80540270135068	2.66804099838414e-08\\
2.80740370185093	4.00206436236519e-08\\
2.80940470235118	5.33608772634624e-08\\
2.81140570285143	4.00206436236519e-08\\
2.81340670335168	5.33608772634624e-08\\
2.81540770385193	4.66907317956674e-08\\
2.81740870435218	4.00206436236519e-08\\
2.81940970485243	4.00206436236519e-08\\
2.82141070535268	5.33608772634624e-08\\
2.82341170585293	4.00206436236519e-08\\
2.82541270635318	0\\
2.82741370685343	6.67008817201548e-09\\
2.82941470735368	1.33402336398105e-08\\
2.83141570785393	1.33402336398105e-08\\
2.83341670835418	3.33505554516364e-08\\
2.83541770885443	4.00206436236519e-08\\
2.83741870935468	3.33505554516364e-08\\
2.83941970985493	4.00206436236519e-08\\
2.84142071035518	6.00310800270369e-08\\
2.84342171085543	8.00410580641858e-08\\
2.84542271135568	5.33608772634624e-08\\
2.84742371185593	6.67008817201548e-08\\
2.84942471235618	7.33712563710678e-08\\
2.85142571285643	7.33712563710678e-08\\
2.85342671335668	4.66907317956674e-08\\
2.85542771385693	6.00310800270369e-08\\
2.85742871435718	6.67008817201548e-08\\
2.85942971485743	6.00310800270369e-08\\
2.86143071535768	9.33812344082167e-08\\
2.86343171585793	8.00410580641858e-08\\
2.86543271635818	6.67008817201548e-08\\
2.86743371685843	9.33812344082167e-08\\
2.86943471735868	8.67114327150988e-08\\
2.87143571785893	8.00410580641858e-08\\
2.87343671835918	1.13391785403161e-07\\
2.87543771885943	9.33812344082167e-08\\
2.87743871935968	6.67008817201548e-08\\
2.87943971985993	6.00310800270369e-08\\
2.88144072036018	6.00310800270369e-08\\
2.88344172086043	1.06721983710043e-07\\
2.88544272136068	1.06721983710043e-07\\
2.88744372186093	1.06721983710043e-07\\
2.88944472236118	1.26731961747192e-07\\
2.89144572286143	1.13391785403161e-07\\
2.89344672336168	1.06721983710043e-07\\
2.89544772386193	9.33812344082167e-08\\
2.89744872436218	9.33812344082167e-08\\
2.89944972486243	1.0005160905913e-07\\
2.90145072536268	1.20062160054074e-07\\
2.90345172586293	9.33812344082167e-08\\
2.90545272636318	1.06721983710043e-07\\
2.90745372686343	1.0005160905913e-07\\
2.90945472736368	1.13391785403161e-07\\
2.91145572786393	9.33812344082167e-08\\
2.91345672836418	1.0005160905913e-07\\
2.91545772886443	1.26731961747192e-07\\
2.91745872936468	1.0005160905913e-07\\
2.91945972986493	1.06721983710043e-07\\
2.92146073036518	1.20062160054074e-07\\
2.92346173086543	1.53412314435254e-07\\
2.92546273136568	1.66752490779285e-07\\
2.92746373186593	1.40072138091223e-07\\
2.92946473236618	1.33402336398105e-07\\
2.93146573286643	1.40072138091223e-07\\
2.93346673336668	1.20062160054074e-07\\
2.93546773386693	1.13391785403161e-07\\
2.93746873436718	9.33812344082167e-08\\
2.93946973486743	7.33712563710678e-08\\
2.94147073536768	4.66907317956674e-08\\
2.94347173586793	5.33608772634624e-08\\
2.94547273636818	6.00310800270369e-08\\
2.94747373686843	8.00410580641858e-08\\
2.94947473736868	6.67008817201548e-08\\
2.95147573786893	4.66907317956674e-08\\
2.95347673836918	5.33608772634624e-08\\
2.95547773886943	5.33608772634624e-08\\
2.95747873936968	3.33505554516364e-08\\
2.95947973986994	7.33712563710678e-08\\
2.96148074037019	7.33712563710678e-08\\
2.96348174087044	8.00410580641858e-08\\
2.96548274137069	1.0005160905913e-07\\
2.96748374187094	7.33712563710678e-08\\
2.96948474237119	7.33712563710678e-08\\
2.97148574287144	7.33712563710678e-08\\
2.97348674337169	4.66907317956674e-08\\
2.97548774387194	4.66907317956674e-08\\
2.97748874437219	3.33505554516364e-08\\
2.97948974487244	1.33402336398105e-08\\
2.98149074537269	-1.33402336398105e-08\\
2.98349174587294	1.33402336398105e-08\\
2.98549274637319	-2.00103218118259e-08\\
2.98749374687344	-1.33402336398105e-08\\
2.98949474737369	-1.33402336398105e-08\\
2.99149574787394	-1.33402336398105e-08\\
2.99349674837419	-1.33402336398105e-08\\
2.99549774887444	0\\
2.99749874937469	2.66804099838414e-08\\
2.99949974987494	4.00206436236519e-08\\
3.00150075037519	6.67008817201548e-08\\
3.00350175087544	8.00410580641858e-08\\
3.00550275137569	6.00310800270369e-08\\
3.00750375187594	4.66907317956674e-08\\
3.00950475237619	4.00206436236519e-08\\
3.01150575287644	2.00103218118259e-08\\
3.01350675337669	6.67008817201548e-09\\
3.01550775387694	0\\
3.01750875437719	-3.33505554516364e-08\\
3.01950975487744	-2.00103218118259e-08\\
3.02151075537769	-3.33505554516364e-08\\
3.02351175587794	-6.00310800270369e-08\\
3.02551275637819	-4.66907317956674e-08\\
3.02751375687844	-4.66907317956674e-08\\
3.02951475737869	-6.67008817201548e-08\\
3.03151575787894	-8.00410580641858e-08\\
3.03351675837919	-6.67008817201548e-08\\
3.03551775887944	-4.66907317956674e-08\\
3.03751875937969	-4.66907317956674e-08\\
3.03951975987994	-4.66907317956674e-08\\
3.04152076038019	-6.00310800270369e-08\\
3.04352176088044	-4.66907317956674e-08\\
3.04552276138069	-2.66804099838414e-08\\
3.04752376188094	-2.66804099838414e-08\\
3.04952476238119	-3.33505554516364e-08\\
3.05152576288144	-5.33608772634624e-08\\
3.05352676338169	-4.66907317956674e-08\\
3.05552776388194	-5.33608772634624e-08\\
3.05752876438219	-8.00410580641858e-08\\
3.05952976488244	-9.33812344082167e-08\\
3.06153076538269	-8.67114327150988e-08\\
3.06353176588294	-8.67114327150988e-08\\
3.06553276638319	-1.13391785403161e-07\\
3.06753376688344	-1.13391785403161e-07\\
3.06953476738369	-8.00410580641858e-08\\
3.07153576788394	-8.67114327150988e-08\\
3.07353676838419	-8.00410580641858e-08\\
3.07553776888444	-8.67114327150988e-08\\
3.07753876938469	-8.00410580641858e-08\\
3.07953976988494	-1.0005160905913e-07\\
3.08154077038519	-1.0005160905913e-07\\
3.08354177088544	-1.33402336398105e-07\\
3.08554277138569	-1.20062160054074e-07\\
3.08754377188594	-1.40072138091223e-07\\
3.08954477238619	-1.53412314435254e-07\\
3.09154577288644	-1.66752490779285e-07\\
3.09354677338669	-1.53412314435254e-07\\
3.09554777388694	-1.46742512742136e-07\\
3.09754877438719	-1.40072138091223e-07\\
3.09954977488744	-1.66752490779285e-07\\
3.10155077538769	-1.66752490779285e-07\\
3.10355177588794	-1.60082689086167e-07\\
3.10555277638819	-1.26731961747192e-07\\
3.10755377688844	-1.26731961747192e-07\\
3.10955477738869	-1.13391785403161e-07\\
3.11155577788894	-1.13391785403161e-07\\
3.11355677838919	-1.06721983710043e-07\\
3.11555777888944	-1.13391785403161e-07\\
3.11755877938969	-1.26731961747192e-07\\
3.11955977988994	-1.26731961747192e-07\\
3.1215607803902	-1.20062160054074e-07\\
3.12356178089045	-1.33402336398105e-07\\
3.1255627813907	-1.40072138091223e-07\\
3.12756378189095	-1.06721983710043e-07\\
3.1295647823912	-1.0005160905913e-07\\
3.13156578289145	-1.13391785403161e-07\\
3.1335667833917	-1.20062160054074e-07\\
3.13556778389195	-1.40072138091223e-07\\
3.1375687843922	-1.46742512742136e-07\\
3.13956978489245	-1.46742512742136e-07\\
3.1415707853927	-1.86763041774229e-07\\
3.14357178589295	-1.46742512742136e-07\\
3.1455727863932	-1.73422865430198e-07\\
3.14757378689345	-2.0010321811826e-07\\
3.1495747873937	-2.1344339446229e-07\\
3.15157578789395	-2.1344339446229e-07\\
3.1535767883942	-2.0010321811826e-07\\
3.15557778889445	-1.80092667123315e-07\\
3.1575787893947	-1.73422865430198e-07\\
3.15957978989495	-1.86763041774229e-07\\
3.1615807903952	-2.1344339446229e-07\\
3.16358179089545	-2.0010321811826e-07\\
3.1655827913957	-2.06773019811377e-07\\
3.16758379189595	-2.26783570806321e-07\\
3.1695847923962	-2.26783570806321e-07\\
3.17158579289645	-2.33453945457234e-07\\
3.1735867933967	-2.40123747150352e-07\\
3.17558779389695	-2.20113769113203e-07\\
3.1775887943972	-2.0010321811826e-07\\
3.17958979489745	-2.26783570806321e-07\\
3.1815907953977	-2.40123747150352e-07\\
3.18359179589795	-2.46794121801265e-07\\
3.1855927963982	-2.20113769113203e-07\\
3.18759379689845	-2.26783570806321e-07\\
3.1895947973987	-2.53463923494383e-07\\
3.19159579789895	-2.86814650833358e-07\\
3.1935967983992	-2.66804099838414e-07\\
3.19559779889945	-2.80144276182445e-07\\
3.1975987993997	-2.80144276182445e-07\\
3.19959979989995	-2.66804099838414e-07\\
3.2016008004002	-2.66804099838414e-07\\
3.20360180090045	-2.33453945457234e-07\\
3.2056028014007	-2.33453945457234e-07\\
3.20760380190095	-2.46794121801265e-07\\
3.2096048024012	-2.33453945457234e-07\\
3.21160580290145	-2.33453945457234e-07\\
3.2136068034017	-2.46794121801265e-07\\
3.21560780390195	-2.53463923494383e-07\\
3.2176088044022	-2.53463923494383e-07\\
3.21960980490245	-2.46794121801265e-07\\
3.2216108054027	-3.00154827177389e-07\\
3.22361180590295	-3.06824628870507e-07\\
3.2256128064032	-3.33505554516364e-07\\
3.22761380690345	-3.40175356209482e-07\\
3.2296148074037	-3.20165378172333e-07\\
3.23161580790395	-3.40175356209482e-07\\
3.2336168084042	-3.40175356209482e-07\\
3.23561780890445	-3.20165378172333e-07\\
3.2376188094047	-2.93484452526476e-07\\
3.23961980990495	-3.1349500352142e-07\\
3.2416208104052	-3.06824628870507e-07\\
3.24362181090545	-3.20165378172333e-07\\
3.2456228114057	-3.00154827177389e-07\\
3.24762381190595	-3.1349500352142e-07\\
3.2496248124062	-3.26835179865451e-07\\
3.25162581290645	-3.00154827177389e-07\\
3.2536268134067	-2.73474474489327e-07\\
3.25562781390695	-2.73474474489327e-07\\
3.2576288144072	-2.73474474489327e-07\\
3.25962981490745	-3.00154827177389e-07\\
3.2616308154077	-3.20165378172333e-07\\
3.26363181590795	-3.06824628870507e-07\\
3.2656328164082	-3.06824628870507e-07\\
3.26763381690845	-3.00154827177389e-07\\
3.2696348174087	-3.20165378172333e-07\\
3.27163581790895	-3.40175356209482e-07\\
3.2736368184092	-3.33505554516364e-07\\
3.27563781890945	-3.40175356209482e-07\\
3.27763881940971	-3.20165378172333e-07\\
3.27963981990996	-3.33505554516364e-07\\
3.2816408204102	-3.46845730860395e-07\\
3.28364182091046	-3.40175356209482e-07\\
3.28564282141071	-3.26835179865451e-07\\
3.28764382191096	-3.33505554516364e-07\\
3.28964482241121	-3.33505554516364e-07\\
3.29164582291146	-2.93484452526476e-07\\
3.29364682341171	-3.1349500352142e-07\\
3.29564782391196	-3.40175356209482e-07\\
3.29764882441221	-3.20165378172333e-07\\
3.29964982491246	-3.06824628870507e-07\\
3.30165082541271	-3.00154827177389e-07\\
3.30365182591296	-2.80144276182445e-07\\
3.30565282641321	-3.00154827177389e-07\\
3.30765382691346	-3.06824628870507e-07\\
3.30965482741371	-2.60134298145296e-07\\
3.31165582791396	-2.73474474489327e-07\\
3.31365682841421	-3.1349500352142e-07\\
3.31565782891446	-3.1349500352142e-07\\
3.31765882941471	-2.93484452526476e-07\\
3.31965982991496	-3.06824628870507e-07\\
3.32166083041521	-3.06824628870507e-07\\
3.32366183091546	-3.00154827177389e-07\\
3.32566283141571	-2.86814650833358e-07\\
3.32766383191596	-2.86814650833358e-07\\
3.32966483241621	-3.06824628870507e-07\\
3.33166583291646	-3.1349500352142e-07\\
3.33366683341671	-3.1349500352142e-07\\
3.33566783391696	-2.93484452526476e-07\\
3.33766883441721	-2.80144276182445e-07\\
3.33966983491746	-3.00154827177389e-07\\
3.34167083541771	-2.73474474489327e-07\\
3.34367183591796	-2.86814650833358e-07\\
3.34567283641821	-2.86814650833358e-07\\
3.34767383691846	-2.93484452526476e-07\\
3.34967483741871	-2.86814650833358e-07\\
3.35167583791896	-2.93484452526476e-07\\
3.35367683841921	-3.20165378172333e-07\\
3.35567783891946	-3.40175356209482e-07\\
3.35767883941971	-3.60185907204426e-07\\
3.35967983991996	-3.26835179865451e-07\\
3.36168084042021	-3.20165378172333e-07\\
3.36368184092046	-3.1349500352142e-07\\
3.36568284142071	-2.80144276182445e-07\\
3.36768384192096	-2.93484452526476e-07\\
3.36968484242121	-2.93484452526476e-07\\
3.37168584292146	-3.06824628870507e-07\\
3.37368684342171	-3.20165378172333e-07\\
3.37568784392196	-3.06824628870507e-07\\
3.37768884442221	-3.33505554516364e-07\\
3.37968984492246	-3.40175356209482e-07\\
3.38169084542271	-3.06824628870507e-07\\
3.38369184592296	-3.06824628870507e-07\\
3.38569284642321	-3.00154827177389e-07\\
3.38769384692346	-3.20165378172333e-07\\
3.38969484742371	-3.20165378172333e-07\\
3.39169584792396	-3.1349500352142e-07\\
3.39369684842421	-3.1349500352142e-07\\
3.39569784892446	-3.20165378172333e-07\\
3.39769884942471	-3.1349500352142e-07\\
3.39969984992496	-3.06824628870507e-07\\
3.40170085042521	-3.06824628870507e-07\\
3.40370185092546	-2.80144276182445e-07\\
3.40570285142571	-2.80144276182445e-07\\
3.40770385192596	-2.73474474489327e-07\\
3.40970485242621	-2.46794121801265e-07\\
3.41170585292646	-2.40123747150352e-07\\
3.41370685342671	-2.73474474489327e-07\\
3.41570785392696	-2.86814650833358e-07\\
3.41770885442721	-3.06824628870507e-07\\
3.41970985492746	-3.46845730860395e-07\\
3.42171085542771	-3.40175356209482e-07\\
3.42371185592796	-3.53515532553513e-07\\
3.42571285642821	-3.33505554516364e-07\\
3.42771385692846	-3.60185907204426e-07\\
3.42971485742871	-3.53515532553513e-07\\
3.43171585792896	-3.73526083548457e-07\\
3.43371685842921	-3.53515532553513e-07\\
3.43571785892946	-3.60185907204426e-07\\
3.43771885942971	-3.66855708897544e-07\\
3.43971985992996	-3.66855708897544e-07\\
3.44172086043022	-4.06876237929637e-07\\
3.44372186093047	-4.00206436236519e-07\\
3.44572286143072	-4.00206436236519e-07\\
3.44772386193097	-4.00206436236519e-07\\
3.44972486243122	-3.93536061585606e-07\\
3.45172586293147	-3.93536061585606e-07\\
3.45372686343172	-3.86866259892488e-07\\
3.45572786393197	-3.80195885241575e-07\\
3.45772886443222	-3.80195885241575e-07\\
3.45972986493247	-3.73526083548457e-07\\
3.46173086543272	-3.73526083548457e-07\\
3.46373186593297	-3.66855708897544e-07\\
3.46573286643322	-4.00206436236519e-07\\
3.46773386693347	-3.86866259892488e-07\\
3.46973486743372	-3.93536061585606e-07\\
3.47173586793397	-4.00206436236519e-07\\
3.47373686843422	-4.00206436236519e-07\\
3.47573786893447	-4.06876237929637e-07\\
3.47773886943472	-3.93536061585606e-07\\
3.47973986993497	-4.1354661258055e-07\\
3.48174087043522	-4.06876237929637e-07\\
3.48374187093547	-4.06876237929637e-07\\
3.48574287143572	-4.26886788924581e-07\\
3.48774387193597	-4.26886788924581e-07\\
3.48974487243622	-4.26886788924581e-07\\
3.49174587293647	-4.33557163575494e-07\\
3.49374687343672	-4.53567141612643e-07\\
3.49574787393697	-4.40226965268612e-07\\
3.49774887443722	-4.46897339919525e-07\\
3.49974987493747	-4.40226965268612e-07\\
3.50175087543772	-4.53567141612643e-07\\
3.50375187593797	-4.46897339919525e-07\\
3.50575287643822	-4.46897339919525e-07\\
3.50775387693847	-4.53567141612643e-07\\
3.50975487743872	-4.40226965268612e-07\\
3.51175587793897	-4.40226965268612e-07\\
3.51375687843922	-4.1354661258055e-07\\
3.51575787893947	-4.26886788924581e-07\\
3.51775887943972	-4.40226965268612e-07\\
3.51975987993997	-4.46897339919525e-07\\
3.52176088044022	-4.33557163575494e-07\\
3.52376188094047	-4.40226965268612e-07\\
3.52576288144072	-4.40226965268612e-07\\
3.52776388194097	-4.26886788924581e-07\\
3.52976488244122	-4.06876237929637e-07\\
3.53176588294147	-4.33557163575494e-07\\
3.53376688344172	-4.20216987231463e-07\\
3.53576788394197	-4.26886788924581e-07\\
3.53776888444222	-4.33557163575494e-07\\
3.53976988494247	-4.26886788924581e-07\\
3.54177088544272	-4.1354661258055e-07\\
3.54377188594297	-3.93536061585606e-07\\
3.54577288644322	-4.26886788924581e-07\\
3.54777388694347	-4.26886788924581e-07\\
3.54977488744372	-4.40226965268612e-07\\
3.55177588794397	-4.66907317956674e-07\\
3.55377688844422	-4.80247494300705e-07\\
3.55577788894447	-4.80247494300705e-07\\
3.55777888944472	-4.86917868951618e-07\\
3.55977988994497	-4.86917868951618e-07\\
3.56178089044522	-4.93587670644736e-07\\
3.56378189094547	-5.00258045295649e-07\\
3.56578289144572	-5.26938397983711e-07\\
3.56778389194597	-5.40278574327742e-07\\
3.56978489244622	-5.40278574327742e-07\\
3.57178589294647	-5.40278574327742e-07\\
3.57378689344672	-5.46948948978655e-07\\
3.57578789394697	-5.60289125322686e-07\\
3.57778889444722	-5.46948948978655e-07\\
3.57978989494747	-5.53618750671773e-07\\
3.58179089544772	-5.66958927015804e-07\\
3.58379189594797	-5.80297384486449e-07\\
3.58579289644822	-5.60289125322686e-07\\
3.58779389694847	-5.53618750671773e-07\\
3.58979489744872	-5.60289125322686e-07\\
3.59179589794897	-5.60289125322686e-07\\
3.59379689844922	-5.53618750671773e-07\\
3.59579789894947	-5.60289125322686e-07\\
3.59779889944972	-5.60289125322686e-07\\
3.59979989994997	-5.46948948978655e-07\\
3.60180090045022	-5.60289125322686e-07\\
3.60380190095048	-5.60289125322686e-07\\
3.60580290145073	-5.40278574327742e-07\\
3.60780390195098	-5.46948948978655e-07\\
3.60980490245123	-5.40278574327742e-07\\
3.61180590295148	-5.46948948978655e-07\\
3.61380690345173	-5.40278574327742e-07\\
3.61580790395198	-5.20268596290593e-07\\
3.61780890445223	-5.33608772634624e-07\\
3.61980990495248	-5.26938397983711e-07\\
3.62181090545273	-5.26938397983711e-07\\
3.62381190595298	-4.86917868951618e-07\\
3.62581290645323	-4.86917868951618e-07\\
3.62781390695348	-5.06927846988767e-07\\
3.62981490745373	-5.00258045295649e-07\\
3.63181590795398	-4.93587670644736e-07\\
3.63381690845423	-4.60237516263556e-07\\
3.63581790895448	-4.53567141612643e-07\\
3.63781890945473	-4.73577692607587e-07\\
3.63981990995498	-5.1359822163968e-07\\
3.64182091045523	-5.26938397983711e-07\\
3.64382191095548	-5.26938397983711e-07\\
3.64582291145573	-5.00258045295649e-07\\
3.64782391195598	-5.00258045295649e-07\\
3.64982491245623	-5.06927846988767e-07\\
3.65182591295648	-4.86917868951618e-07\\
3.65382691345673	-4.93587670644736e-07\\
3.65582791395698	-5.06927846988767e-07\\
3.65782891445723	-5.06927846988767e-07\\
3.65982991495748	-4.86917868951618e-07\\
3.66183091545773	-5.00258045295649e-07\\
3.66383191595798	-4.93587670644736e-07\\
3.66583291645823	-4.93587670644736e-07\\
3.66783391695848	-5.06927846988767e-07\\
3.66983491745873	-5.1359822163968e-07\\
3.67183591795898	-5.00258045295649e-07\\
3.67383691845923	-5.1359822163968e-07\\
3.67583791895948	-5.00258045295649e-07\\
3.67783891945973	-5.06927846988767e-07\\
3.67983991995998	-4.80247494300705e-07\\
3.68184092046023	-4.73577692607587e-07\\
3.68384192096048	-4.93587670644736e-07\\
3.68584292146073	-5.00258045295649e-07\\
3.68784392196098	-5.20268596290593e-07\\
3.68984492246123	-5.00258045295649e-07\\
3.69184592296148	-5.20268596290593e-07\\
3.69384692346173	-5.00258045295649e-07\\
3.69584792396198	-4.93587670644736e-07\\
3.69784892446223	-4.73577692607587e-07\\
3.69984992496248	-4.60237516263556e-07\\
3.70185092546273	-4.53567141612643e-07\\
3.70385192596298	-4.46897339919525e-07\\
3.70585292646323	-4.40226965268612e-07\\
3.70785392696348	-4.46897339919525e-07\\
3.70985492746373	-4.33557163575494e-07\\
3.71185592796398	-4.20216987231463e-07\\
3.71385692846423	-4.26886788924581e-07\\
3.71585792896448	-3.66855708897544e-07\\
3.71785892946473	-4.1354661258055e-07\\
3.71985992996498	-4.33557163575494e-07\\
3.72186093046523	-4.46897339919525e-07\\
3.72386193096548	-4.26886788924581e-07\\
3.72586293146573	-4.33557163575494e-07\\
3.72786393196598	-4.26886788924581e-07\\
3.72986493246623	-3.80195885241575e-07\\
3.73186593296648	-3.86866259892488e-07\\
3.73386693346673	-3.93536061585606e-07\\
3.73586793396698	-4.00206436236519e-07\\
3.73786893446723	-3.93536061585606e-07\\
3.73986993496748	-4.26886788924581e-07\\
3.74187093546773	-4.20216987231463e-07\\
3.74387193596798	-3.93536061585606e-07\\
3.74587293646823	-3.80195885241575e-07\\
3.74787393696848	-3.80195885241575e-07\\
3.74987493746873	-4.00206436236519e-07\\
3.75187593796898	-4.06876237929637e-07\\
3.75387693846923	-4.1354661258055e-07\\
3.75587793896948	-3.93536061585606e-07\\
3.75787893946973	-4.26886788924581e-07\\
3.75987993996999	-4.26886788924581e-07\\
3.76188094047024	-4.40226965268612e-07\\
3.76388194097049	-4.40226965268612e-07\\
3.76588294147074	-4.26886788924581e-07\\
3.76788394197099	-4.00206436236519e-07\\
3.76988494247124	-3.86866259892488e-07\\
3.77188594297149	-4.06876237929637e-07\\
3.77388694347174	-4.1354661258055e-07\\
3.77588794397199	-3.93536061585606e-07\\
3.77788894447224	-4.26886788924581e-07\\
3.77988994497249	-4.06876237929637e-07\\
3.78189094547274	-4.20216987231463e-07\\
3.78389194597299	-4.20216987231463e-07\\
3.78589294647324	-4.1354661258055e-07\\
3.78789394697349	-3.86866259892488e-07\\
3.78989494747374	-3.73526083548457e-07\\
3.79189594797399	-3.60185907204426e-07\\
3.79389694847424	-3.06824628870507e-07\\
3.79589794897449	-3.00154827177389e-07\\
3.79789894947474	-3.20165378172333e-07\\
3.79989994997499	-3.46845730860395e-07\\
3.80190095047524	-3.60185907204426e-07\\
3.80390195097549	-3.53515532553513e-07\\
3.80590295147574	-3.80195885241575e-07\\
3.80790395197599	-3.60185907204426e-07\\
3.80990495247624	-3.73526083548457e-07\\
3.81190595297649	-3.80195885241575e-07\\
3.81390695347674	-3.86866259892488e-07\\
3.81590795397699	-3.93536061585606e-07\\
3.81790895447724	-4.33557163575494e-07\\
3.81990995497749	-4.06876237929637e-07\\
3.82191095547774	-3.93536061585606e-07\\
3.82391195597799	-4.1354661258055e-07\\
3.82591295647824	-4.26886788924581e-07\\
3.82791395697849	-4.40226965268612e-07\\
3.82991495747874	-4.40226965268612e-07\\
3.83191595797899	-4.33557163575494e-07\\
3.83391695847924	-3.93536061585606e-07\\
3.83591795897949	-4.26886788924581e-07\\
3.83791895947974	-4.40226965268612e-07\\
3.83991995997999	-4.1354661258055e-07\\
3.84192096048024	-4.46897339919525e-07\\
3.84392196098049	-4.06876237929637e-07\\
3.84592296148074	-4.26886788924581e-07\\
3.84792396198099	-4.20216987231463e-07\\
3.84992496248124	-4.00206436236519e-07\\
3.85192596298149	-4.26886788924581e-07\\
3.85392696348174	-4.33557163575494e-07\\
3.85592796398199	-4.40226965268612e-07\\
3.85792896448224	-4.46897339919525e-07\\
3.85992996498249	-4.53567141612643e-07\\
3.86193096548274	-4.53567141612643e-07\\
3.86393196598299	-4.53567141612643e-07\\
3.86593296648324	-4.93587670644736e-07\\
3.86793396698349	-4.80247494300705e-07\\
3.86993496748374	-4.80247494300705e-07\\
3.87193596798399	-4.60237516263556e-07\\
3.87393696848424	-4.53567141612643e-07\\
3.87593796898449	-4.46897339919525e-07\\
3.87793896948474	-4.46897339919525e-07\\
3.87993996998499	-4.40226965268612e-07\\
3.88194097048524	-4.06876237929637e-07\\
3.88394197098549	-4.00206436236519e-07\\
3.88594297148574	-3.66855708897544e-07\\
3.88794397198599	-3.80195885241575e-07\\
3.88994497248624	-4.00206436236519e-07\\
3.89194597298649	-3.80195885241575e-07\\
3.89394697348674	-4.06876237929637e-07\\
3.89594797398699	-4.00206436236519e-07\\
3.89794897448724	-3.80195885241575e-07\\
3.89994997498749	-4.1354661258055e-07\\
3.90195097548774	-4.00206436236519e-07\\
3.90395197598799	-3.86866259892488e-07\\
3.90595297648824	-3.66855708897544e-07\\
3.90795397698849	-3.80195885241575e-07\\
3.90995497748874	-3.93536061585606e-07\\
3.91195597798899	-3.73526083548457e-07\\
3.91395697848924	-3.40175356209482e-07\\
3.91595797898949	-3.20165378172333e-07\\
3.91795897948974	-3.06824628870507e-07\\
3.91995997998999	-3.1349500352142e-07\\
3.92196098049025	-2.80144276182445e-07\\
3.9239619809905	-2.86814650833358e-07\\
3.92596298149075	-2.40123747150352e-07\\
3.927963981991	-2.46794121801265e-07\\
3.92996498249125	-2.66804099838414e-07\\
3.9319659829915	-2.86814650833358e-07\\
3.93396698349175	-2.93484452526476e-07\\
3.935967983992	-2.80144276182445e-07\\
3.93796898449225	-2.80144276182445e-07\\
3.9399699849925	-2.66804099838414e-07\\
3.94197098549275	-2.33453945457234e-07\\
3.943971985993	-2.0010321811826e-07\\
3.94597298649325	-1.93432843467346e-07\\
3.9479739869935	-1.80092667123315e-07\\
3.94997498749375	-1.80092667123315e-07\\
3.951975987994	-2.1344339446229e-07\\
3.95397698849425	-2.20113769113203e-07\\
3.9559779889945	-2.1344339446229e-07\\
3.95797898949475	-2.26783570806321e-07\\
3.959979989995	-2.06773019811377e-07\\
3.96198099049525	-1.73422865430198e-07\\
3.9639819909955	-1.73422865430198e-07\\
3.96598299149575	-1.66752490779285e-07\\
3.967983991996	-1.53412314435254e-07\\
3.96998499249625	-1.60082689086167e-07\\
3.9719859929965	-1.60082689086167e-07\\
3.97398699349675	-1.66752490779285e-07\\
3.975987993997	-1.80092667123315e-07\\
3.97798899449725	-2.26783570806321e-07\\
3.9799899949975	-2.06773019811377e-07\\
3.98199099549775	-2.06773019811377e-07\\
3.983991995998	-2.1344339446229e-07\\
3.98599299649825	-2.0010321811826e-07\\
3.9879939969985	-1.80092667123315e-07\\
3.98999499749875	-1.73422865430198e-07\\
3.991995997999	-1.86763041774229e-07\\
3.99399699849925	-1.86763041774229e-07\\
3.9959979989995	-1.86763041774229e-07\\
3.99799899949975	-1.53412314435254e-07\\
4	-1.60082689086167e-07\\
};
\addlegendentry{Energy Diff};

\end{axis}
\end{tikzpicture}%
	\caption{A zoom in on the error when stepping forward in time.}
	\label{fig:forwardDataError}
\end{figure}
\fi
oe
\iftikz
\begin{figure}[H]
	\centering
	\setlength\figureheight{7cm} 
	\setlength\figurewidth{14cm}
	% This file was created by matlab2tikz.
% Minimal pgfplots version: 1.3
%
%The latest updates can be retrieved from
%  http://www.mathworks.com/matlabcentral/fileexchange/22022-matlab2tikz
%where you can also make suggestions and rate matlab2tikz.
%
\definecolor{mycolor1}{rgb}{0.00000,0.44700,0.74100}%
\definecolor{mycolor2}{rgb}{0.85000,0.32500,0.09800}%
\definecolor{mycolor3}{rgb}{0.92900,0.69400,0.12500}%
%
\begin{tikzpicture}

\begin{axis}[%
width=0.95092\figurewidth,
height=\figureheight,
at={(0\figurewidth,0\figureheight)},
scale only axis,
xmin=0,
xmax=4,
xtick={0,0.5,1,1.5,2,2.5,3,3.5,4},
xticklabels={{4},{3.5},{3},{2.5},{2},{1.5},{1},{0.5},{0}},
xlabel={Time (s)},
ymin=-1e-10,
ymax=7e-10,
ylabel={Degrees},
title style={font=\bfseries},
title={Top Spin [4,0] (s) Errors},
legend style={legend cell align=left,align=left,draw=white!15!black},
title style={font=\small},ticklabel style={font=\tiny}
]
\addplot [color=mycolor1,solid]
  table[row sep=crcr]{%
0	3.89377e-10\\
0.00199900049975012	3.87558e-10\\
0.00399800099950025	3.86308e-10\\
0.00599700149925037	3.89718e-10\\
0.0079960019990005	3.85512e-10\\
0.00999500249875063	3.80396e-10\\
0.0119940029985007	3.72779e-10\\
0.0139930034982509	3.72779e-10\\
0.015992003998001	3.69823e-10\\
0.0179910044977511	3.76076e-10\\
0.0199900049975013	3.78122e-10\\
0.0219890054972514	3.80624e-10\\
0.0239880059970015	3.76644e-10\\
0.0259870064967516	3.81533e-10\\
0.0279860069965017	3.8483e-10\\
0.0299850074962519	3.90969e-10\\
0.031984007996002	3.95289e-10\\
0.0339830084957521	3.89264e-10\\
0.0359820089955022	3.95744e-10\\
0.0379810094952524	3.88127e-10\\
0.0399800099950025	3.89377e-10\\
0.0419790104947526	3.90742e-10\\
0.0439780109945027	3.95517e-10\\
0.0459770114942529	3.91765e-10\\
0.047976011994003	3.84603e-10\\
0.0499750124937531	3.7835e-10\\
0.0519740129935032	3.73802e-10\\
0.0539730134932534	3.72097e-10\\
0.0559720139930035	3.74143e-10\\
0.0579710144927536	3.68686e-10\\
0.0599700149925037	3.72779e-10\\
0.0619690154922539	3.65162e-10\\
0.063968015992004	3.64366e-10\\
0.0659670164917541	3.61069e-10\\
0.0679660169915042	3.64821e-10\\
0.0699650174912544	3.63343e-10\\
0.0719640179910045	3.60046e-10\\
0.0739630184907546	3.54817e-10\\
0.0759620189905048	3.53566e-10\\
0.0779610194902549	3.47882e-10\\
0.079960019990005	3.53566e-10\\
0.0819590204897551	3.5368e-10\\
0.0839580209895052	3.60956e-10\\
0.0859570214892554	3.66981e-10\\
0.0879560219890055	3.61183e-10\\
0.0899550224887556	3.58909e-10\\
0.0919540229885057	3.57659e-10\\
0.0939530234882559	3.65048e-10\\
0.095952023988006	3.63229e-10\\
0.0979510244877561	3.63684e-10\\
0.0999500249875062	3.65731e-10\\
0.101949025487256	3.66526e-10\\
0.103948025987006	3.58796e-10\\
0.105947026486757	3.62093e-10\\
0.107946026986507	3.56408e-10\\
0.109945027486257	3.62434e-10\\
0.111944027986007	3.61069e-10\\
0.113943028485757	3.56408e-10\\
0.115942028985507	3.54362e-10\\
0.117941029485257	3.62093e-10\\
0.119940029985007	3.62093e-10\\
0.121939030484758	3.54817e-10\\
0.123938030984508	3.57545e-10\\
0.125937031484258	3.64935e-10\\
0.127936031984008	3.58796e-10\\
0.129935032483758	3.52088e-10\\
0.131934032983508	3.47995e-10\\
0.133933033483258	3.46745e-10\\
0.135932033983008	3.42652e-10\\
0.137931034482759	3.40719e-10\\
0.139930034982509	3.45949e-10\\
0.141929035482259	3.42538e-10\\
0.143928035982009	3.39469e-10\\
0.145927036481759	3.41743e-10\\
0.147926036981509	3.44244e-10\\
0.149925037481259	3.44471e-10\\
0.15192403798101	3.4629e-10\\
0.15392303848076	3.51747e-10\\
0.15592203898051	3.46517e-10\\
0.15792103948026	3.49246e-10\\
0.15992003998001	3.49701e-10\\
0.16191904047976	3.5493e-10\\
0.16391804097951	3.55385e-10\\
0.16591704147926	3.61979e-10\\
0.16791604197901	3.66526e-10\\
0.169915042478761	3.68118e-10\\
0.171914042978511	3.688e-10\\
0.173913043478261	3.70846e-10\\
0.175912043978011	3.65617e-10\\
0.177911044477761	3.64935e-10\\
0.179910044977511	3.70846e-10\\
0.181909045477261	3.67663e-10\\
0.183908045977011	3.65162e-10\\
0.185907046476762	3.57886e-10\\
0.187906046976512	3.50383e-10\\
0.189905047476262	3.49019e-10\\
0.191904047976012	3.54248e-10\\
0.193903048475762	3.61297e-10\\
0.195902048975512	3.55158e-10\\
0.197901049475262	3.62434e-10\\
0.199900049975012	3.66754e-10\\
0.201899050474763	3.69255e-10\\
0.203898050974513	3.76303e-10\\
0.205897051474263	3.69937e-10\\
0.207896051974013	3.72893e-10\\
0.209895052473763	3.72893e-10\\
0.211894052973513	3.80396e-10\\
0.213893053473263	3.74484e-10\\
0.215892053973014	3.70164e-10\\
0.217891054472764	3.73348e-10\\
0.219890054972514	3.71074e-10\\
0.221889055472264	3.6448e-10\\
0.223888055972014	3.59023e-10\\
0.225887056471764	3.56522e-10\\
0.227886056971514	3.52884e-10\\
0.229885057471264	3.53566e-10\\
0.231884057971014	3.53339e-10\\
0.233883058470765	3.46517e-10\\
0.235882058970515	3.40833e-10\\
0.237881059470265	3.39924e-10\\
0.239880059970015	3.44926e-10\\
0.241879060469765	3.44244e-10\\
0.243878060969515	3.41288e-10\\
0.245877061469265	3.40833e-10\\
0.247876061969015	3.43334e-10\\
0.249875062468766	3.35604e-10\\
0.251874062968516	3.32193e-10\\
0.253873063468266	3.26054e-10\\
0.255872063968016	3.21734e-10\\
0.257871064467766	3.17186e-10\\
0.259870064967516	3.1514e-10\\
0.261869065467266	3.17414e-10\\
0.263868065967017	3.13321e-10\\
0.265867066466767	3.06954e-10\\
0.267866066966517	3.11729e-10\\
0.269865067466267	3.10138e-10\\
0.271864067966017	3.06272e-10\\
0.273863068465767	2.98542e-10\\
0.275862068965517	3.02634e-10\\
0.277861069465267	2.95131e-10\\
0.279860069965017	2.87855e-10\\
0.281859070464768	2.95586e-10\\
0.283858070964518	2.9263e-10\\
0.285857071464268	2.85809e-10\\
0.287856071964018	2.86946e-10\\
0.289855072463768	2.9172e-10\\
0.291854072963518	2.9786e-10\\
0.293853073463268	2.90356e-10\\
0.295852073963018	2.83308e-10\\
0.297851074462769	2.7876e-10\\
0.299850074962519	2.70802e-10\\
0.301849075462269	2.68301e-10\\
0.303848075962019	2.6148e-10\\
0.305847076461769	2.61025e-10\\
0.307846076961519	2.66709e-10\\
0.309845077461269	2.62389e-10\\
0.311844077961019	2.68983e-10\\
0.31384307846077	2.67164e-10\\
0.31584207896052	2.72394e-10\\
0.31784107946027	2.70802e-10\\
0.31984007996002	2.72166e-10\\
0.32183908045977	2.71712e-10\\
0.32383808095952	2.75577e-10\\
0.32583708145927	2.73303e-10\\
0.32783608195902	2.80579e-10\\
0.329835082458771	2.82853e-10\\
0.331834082958521	2.82171e-10\\
0.333833083458271	2.7444e-10\\
0.335832083958021	2.71257e-10\\
0.337831084457771	2.67619e-10\\
0.339830084957521	2.68074e-10\\
0.341829085457271	2.66709e-10\\
0.343828085957021	2.61934e-10\\
0.345827086456772	2.63981e-10\\
0.347826086956522	2.7012e-10\\
0.349825087456272	2.66709e-10\\
0.351824087956022	2.65572e-10\\
0.353823088455772	2.65118e-10\\
0.355822088955522	2.68528e-10\\
0.357821089455272	2.72394e-10\\
0.359820089955023	2.65572e-10\\
0.361819090454773	2.58524e-10\\
0.363818090954523	2.63753e-10\\
0.365817091454273	2.58751e-10\\
0.367816091954023	2.54659e-10\\
0.369815092453773	2.54431e-10\\
0.371814092953523	2.54886e-10\\
0.373813093453273	2.60343e-10\\
0.375812093953024	2.64208e-10\\
0.377811094452774	2.72166e-10\\
0.379810094952524	2.72394e-10\\
0.381809095452274	2.69438e-10\\
0.383808095952024	2.61934e-10\\
0.385807096451774	2.62162e-10\\
0.387806096951524	2.65572e-10\\
0.389805097451274	2.73076e-10\\
0.391804097951024	2.77396e-10\\
0.393803098450775	2.71257e-10\\
0.395802098950525	2.63981e-10\\
0.397801099450275	2.69893e-10\\
0.399800099950025	2.76259e-10\\
0.401799100449775	2.73985e-10\\
0.403798100949525	2.77169e-10\\
0.405797101449275	2.8308e-10\\
0.407796101949025	2.8308e-10\\
0.409795102448776	2.84217e-10\\
0.411794102948526	2.88992e-10\\
0.413793103448276	2.85354e-10\\
0.415792103948026	2.82853e-10\\
0.417791104447776	2.80806e-10\\
0.419790104947526	2.78987e-10\\
0.421789105447276	2.81261e-10\\
0.423788105947026	2.74213e-10\\
0.425787106446777	2.7535e-10\\
0.427786106946527	2.78305e-10\\
0.429785107446277	2.74667e-10\\
0.431784107946027	2.82398e-10\\
0.433783108445777	2.77169e-10\\
0.435782108945527	2.77851e-10\\
0.437781109445277	2.75577e-10\\
0.439780109945027	2.79897e-10\\
0.441779110444778	2.79897e-10\\
0.443778110944528	2.75577e-10\\
0.445777111444278	2.82853e-10\\
0.447776111944028	2.78987e-10\\
0.449775112443778	2.73303e-10\\
0.451774112943528	2.71029e-10\\
0.453773113443278	2.74667e-10\\
0.455772113943028	2.71712e-10\\
0.457771114442779	2.75804e-10\\
0.459770114942529	2.7535e-10\\
0.461769115442279	2.74213e-10\\
0.463768115942029	2.78078e-10\\
0.465767116441779	2.72166e-10\\
0.467766116941529	2.70802e-10\\
0.469765117441279	2.7012e-10\\
0.471764117941029	2.69893e-10\\
0.47376311844078	2.73531e-10\\
0.47576211894053	2.73076e-10\\
0.47776111944028	2.68983e-10\\
0.47976011994003	2.72848e-10\\
0.48175912043978	2.73985e-10\\
0.48375812093953	2.79215e-10\\
0.48575712143928	2.81034e-10\\
0.487756121939031	2.78533e-10\\
0.489755122438781	2.82853e-10\\
0.491754122938531	2.75577e-10\\
0.493753123438281	2.73303e-10\\
0.495752123938031	2.76259e-10\\
0.497751124437781	2.73985e-10\\
0.499750124937531	2.80806e-10\\
0.501749125437281	2.8308e-10\\
0.503748125937031	2.81034e-10\\
0.505747126436782	2.73076e-10\\
0.507746126936532	2.74667e-10\\
0.509745127436282	2.68983e-10\\
0.511744127936032	2.67846e-10\\
0.513743128435782	2.69665e-10\\
0.515742128935532	2.66709e-10\\
0.517741129435282	2.64663e-10\\
0.519740129935032	2.6057e-10\\
0.521739130434783	2.57387e-10\\
0.523738130934533	2.55341e-10\\
0.525737131434283	2.51021e-10\\
0.527736131934033	2.54886e-10\\
0.529735132433783	2.54204e-10\\
0.531734132933533	2.4761e-10\\
0.533733133433283	2.48747e-10\\
0.535732133933034	2.51021e-10\\
0.537731134432784	2.48519e-10\\
0.539730134932534	2.40789e-10\\
0.541729135432284	2.46473e-10\\
0.543728135932034	2.44427e-10\\
0.545727136431784	2.45336e-10\\
0.547726136931534	2.3897e-10\\
0.549725137431284	2.33513e-10\\
0.551724137931034	2.28965e-10\\
0.553723138430785	2.32149e-10\\
0.555722138930535	2.29193e-10\\
0.557721139430285	2.27374e-10\\
0.559720139930035	2.33285e-10\\
0.561719140429785	2.31694e-10\\
0.563718140929535	2.25327e-10\\
0.565717141429285	2.32831e-10\\
0.567716141929036	2.37833e-10\\
0.569715142428786	2.32603e-10\\
0.571714142928536	2.35559e-10\\
0.573713143428286	2.29875e-10\\
0.575712143928036	2.27601e-10\\
0.577711144427786	2.3374e-10\\
0.579710144927536	2.39197e-10\\
0.581709145427286	2.32603e-10\\
0.583708145927036	2.37833e-10\\
0.585707146426787	2.42608e-10\\
0.587706146926537	2.39879e-10\\
0.589705147426287	2.36696e-10\\
0.591704147926037	2.30102e-10\\
0.593703148425787	2.28511e-10\\
0.595702148925537	2.3033e-10\\
0.597701149425287	2.31921e-10\\
0.599700149925038	2.25327e-10\\
0.601699150424788	2.30784e-10\\
0.603698150924538	2.23054e-10\\
0.605697151424288	2.20552e-10\\
0.607696151924038	2.13049e-10\\
0.609695152423788	2.16914e-10\\
0.611694152923538	2.18279e-10\\
0.613693153423288	2.22599e-10\\
0.615692153923038	2.21007e-10\\
0.617691154422789	2.14868e-10\\
0.619690154922539	2.15778e-10\\
0.621689155422289	2.09184e-10\\
0.623688155922039	2.03954e-10\\
0.625687156421789	2.05546e-10\\
0.627686156921539	1.99407e-10\\
0.629685157421289	1.99861e-10\\
0.631684157921039	1.93495e-10\\
0.63368315842079	1.96451e-10\\
0.63568215892054	2.03727e-10\\
0.63768115942029	2.05318e-10\\
0.63968015992004	2.04864e-10\\
0.64167916041979	2.06001e-10\\
0.64367816091954	2.03045e-10\\
0.64567716141929	2.04409e-10\\
0.64767616191904	2.11003e-10\\
0.649675162418791	2.06455e-10\\
0.651674162918541	2.03727e-10\\
0.653673163418291	2.08956e-10\\
0.655672163918041	2.05773e-10\\
0.657671164417791	2.01453e-10\\
0.659670164917541	2.00771e-10\\
0.661669165417291	1.9395e-10\\
0.663668165917041	1.97588e-10\\
0.665667166416792	2.05318e-10\\
0.667666166916542	2.0259e-10\\
0.669665167416292	2.04409e-10\\
0.671664167916042	2.05091e-10\\
0.673663168415792	2.08729e-10\\
0.675662168915542	2.10548e-10\\
0.677661169415292	2.1214e-10\\
0.679660169915043	2.16914e-10\\
0.681659170414793	2.19643e-10\\
0.683658170914543	2.2419e-10\\
0.685657171414293	2.251e-10\\
0.687656171914043	2.17142e-10\\
0.689655172413793	2.10093e-10\\
0.691654172913543	2.05091e-10\\
0.693653173413293	2.11003e-10\\
0.695652173913043	2.13277e-10\\
0.697651174412794	2.14413e-10\\
0.699650174912544	2.1214e-10\\
0.701649175412294	2.10321e-10\\
0.703648175912044	2.16232e-10\\
0.705647176411794	2.10321e-10\\
0.707646176911544	2.14641e-10\\
0.709645177411294	2.12594e-10\\
0.711644177911045	2.09411e-10\\
0.713643178410795	2.13504e-10\\
0.715642178910545	2.14868e-10\\
0.717641179410295	2.11912e-10\\
0.719640179910045	2.0782e-10\\
0.721639180409795	2.12822e-10\\
0.723638180909545	2.17142e-10\\
0.725637181409295	2.09866e-10\\
0.727636181909045	2.05318e-10\\
0.729635182408796	2.04636e-10\\
0.731634182908546	2.03272e-10\\
0.733633183408296	2.06228e-10\\
0.735632183908046	2.08274e-10\\
0.737631184407796	2.03727e-10\\
0.739630184907546	2.02135e-10\\
0.741629185407296	1.95996e-10\\
0.743628185907046	2.02817e-10\\
0.745627186406797	2.04409e-10\\
0.747626186906547	2.06001e-10\\
0.749625187406297	2.05318e-10\\
0.751624187906047	2.06228e-10\\
0.753623188405797	2.04409e-10\\
0.755622188905547	2.01453e-10\\
0.757621189405297	2.04864e-10\\
0.759620189905047	1.99634e-10\\
0.761619190404798	1.94632e-10\\
0.763618190904548	1.93268e-10\\
0.765617191404298	1.99407e-10\\
0.767616191904048	2.00998e-10\\
0.769615192403798	1.94404e-10\\
0.771614192903548	1.90312e-10\\
0.773613193403298	1.88948e-10\\
0.775612193903048	1.88038e-10\\
0.777611194402799	1.83945e-10\\
0.779610194902549	1.77579e-10\\
0.781609195402299	1.78261e-10\\
0.783608195902049	1.73713e-10\\
0.785607196401799	1.7917e-10\\
0.787606196901549	1.80762e-10\\
0.789605197401299	1.80307e-10\\
0.79160419790105	1.86446e-10\\
0.7936031984008	1.8531e-10\\
0.79560219890055	1.8963e-10\\
0.7976011994003	1.95541e-10\\
0.79960019990005	1.92586e-10\\
0.8015992003998	1.90312e-10\\
0.80359820089955	1.86901e-10\\
0.8055972013993	1.85992e-10\\
0.80759620189905	1.90767e-10\\
0.809595202398801	1.93268e-10\\
0.811594202898551	1.8963e-10\\
0.813593203398301	1.9736e-10\\
0.815592203898051	1.92586e-10\\
0.817591204397801	1.98725e-10\\
0.819590204897551	2.00998e-10\\
0.821589205397301	2.04636e-10\\
0.823588205897052	2.04182e-10\\
0.825587206396802	2.03954e-10\\
0.827586206896552	2.00089e-10\\
0.829585207396302	2.07592e-10\\
0.831584207896052	2.05546e-10\\
0.833583208395802	2.05091e-10\\
0.835582208895552	1.98725e-10\\
0.837581209395302	2.05318e-10\\
0.839580209895052	2.1214e-10\\
0.841579210394803	2.04636e-10\\
0.843578210894553	2.08274e-10\\
0.845577211394303	2.13731e-10\\
0.847576211894053	2.17369e-10\\
0.849575212393803	2.24645e-10\\
0.851574212893553	2.32603e-10\\
0.853573213393303	2.25782e-10\\
0.855572213893053	2.33285e-10\\
0.857571214392804	2.31466e-10\\
0.859570214892554	2.34195e-10\\
0.861569215392304	2.37605e-10\\
0.863568215892054	2.3897e-10\\
0.865567216391804	2.44199e-10\\
0.867566216891554	2.41471e-10\\
0.869565217391304	2.4329e-10\\
0.871564217891054	2.39879e-10\\
0.873563218390805	2.33058e-10\\
0.875562218890555	2.34877e-10\\
0.877561219390305	2.27601e-10\\
0.879560219890055	2.23736e-10\\
0.881559220389805	2.1987e-10\\
0.883558220889555	2.18506e-10\\
0.885557221389305	2.22144e-10\\
0.887556221889055	2.23736e-10\\
0.889555222388806	2.31239e-10\\
0.891554222888556	2.3033e-10\\
0.893553223388306	2.26919e-10\\
0.895552223888056	2.33058e-10\\
0.897551224387806	2.25327e-10\\
0.899550224887556	2.31921e-10\\
0.901549225387306	2.26919e-10\\
0.903548225887057	2.29193e-10\\
0.905547226386807	2.27374e-10\\
0.907546226886557	2.34422e-10\\
0.909545227386307	2.27828e-10\\
0.911544227886057	2.23736e-10\\
0.913543228385807	2.22826e-10\\
0.915542228885557	2.26464e-10\\
0.917541229385307	2.2078e-10\\
0.919540229885057	2.14413e-10\\
0.921539230384808	2.15778e-10\\
0.923538230884558	2.20325e-10\\
0.925537231384308	2.22599e-10\\
0.927536231884058	2.18279e-10\\
0.929535232383808	2.25555e-10\\
0.931534232883558	2.18733e-10\\
0.933533233383308	2.24645e-10\\
0.935532233883059	2.17597e-10\\
0.937531234382809	2.18961e-10\\
0.939530234882559	2.11912e-10\\
0.941529235382309	2.05773e-10\\
0.943528235882059	1.98725e-10\\
0.945527236381809	2.02363e-10\\
0.947526236881559	2.00998e-10\\
0.949525237381309	2.02135e-10\\
0.951524237881059	2.04409e-10\\
0.95352323838081	2.11458e-10\\
0.95552223888056	2.11912e-10\\
0.95752123938031	2.13731e-10\\
0.95952023988006	2.1555e-10\\
0.96151924037981	2.19643e-10\\
0.96351824087956	2.22371e-10\\
0.96551724137931	2.18733e-10\\
0.96751624187906	2.26237e-10\\
0.969515242378811	2.30557e-10\\
0.971514242878561	2.27601e-10\\
0.973513243378311	2.30557e-10\\
0.975512243878061	2.36241e-10\\
0.977511244377811	2.4329e-10\\
0.979510244877561	2.47383e-10\\
0.981509245377311	2.40107e-10\\
0.983508245877061	2.44199e-10\\
0.985507246376812	2.41016e-10\\
0.987506246876562	2.47383e-10\\
0.989505247376312	2.44199e-10\\
0.991504247876062	2.41698e-10\\
0.993503248375812	2.43062e-10\\
0.995502248875562	2.43062e-10\\
0.997501249375312	2.41471e-10\\
0.999500249875062	2.48065e-10\\
1.00149925037481	2.467e-10\\
1.00349825087456	2.39424e-10\\
1.00549725137431	2.467e-10\\
1.00749625187406	2.47837e-10\\
1.00949525237381	2.41471e-10\\
1.01149425287356	2.37605e-10\\
1.01349325337331	2.35787e-10\\
1.01549225387306	2.28511e-10\\
1.01749125437281	2.35559e-10\\
1.01949025487256	2.34877e-10\\
1.02148925537231	2.35104e-10\\
1.02348825587206	2.39424e-10\\
1.02548725637181	2.39424e-10\\
1.02748625687156	2.40107e-10\\
1.02948525737131	2.34877e-10\\
1.03148425787106	2.37605e-10\\
1.03348325837081	2.35104e-10\\
1.03548225887056	2.37378e-10\\
1.03748125937031	2.32831e-10\\
1.03948025987006	2.251e-10\\
1.04147926036982	2.28511e-10\\
1.04347826086957	2.22599e-10\\
1.04547726136932	2.21462e-10\\
1.04747626186907	2.15323e-10\\
1.04947526236882	2.13277e-10\\
1.05147426286857	2.16914e-10\\
1.05347326336832	2.15778e-10\\
1.05547226386807	2.19188e-10\\
1.05747126436782	2.14186e-10\\
1.05947026486757	2.21462e-10\\
1.06146926536732	2.24645e-10\\
1.06346826586707	2.31239e-10\\
1.06546726636682	2.33285e-10\\
1.06746626686657	2.33058e-10\\
1.06946526736632	2.31239e-10\\
1.07146426786607	2.32603e-10\\
1.07346326836582	2.38515e-10\\
1.07546226886557	2.40789e-10\\
1.07746126936532	2.34195e-10\\
1.07946026986507	2.40789e-10\\
1.08145927036482	2.42153e-10\\
1.08345827086457	2.44199e-10\\
1.08545727136432	2.37833e-10\\
1.08745627186407	2.37605e-10\\
1.08945527236382	2.30102e-10\\
1.09145427286357	2.34422e-10\\
1.09345327336332	2.33513e-10\\
1.09545227386307	2.34422e-10\\
1.09745127436282	2.27374e-10\\
1.09945027486257	2.25327e-10\\
1.10144927536232	2.17597e-10\\
1.10344827586207	2.15095e-10\\
1.10544727636182	2.09184e-10\\
1.10744627686157	2.05318e-10\\
1.10944527736132	1.98725e-10\\
1.11144427786107	1.92586e-10\\
1.11344327836082	1.89175e-10\\
1.11544227886057	1.88038e-10\\
1.11744127936032	1.86901e-10\\
1.11944027986007	1.9304e-10\\
1.12143928035982	1.97588e-10\\
1.12343828085957	1.96678e-10\\
1.12543728135932	1.99634e-10\\
1.12743628185907	1.93722e-10\\
1.12943528235882	1.87356e-10\\
1.13143428285857	1.91221e-10\\
1.13343328335832	1.8531e-10\\
1.13543228385807	1.90312e-10\\
1.13743128435782	1.92813e-10\\
1.13943028485757	1.96223e-10\\
1.14142928535732	2.0259e-10\\
1.14342828585707	2.0259e-10\\
1.14542728635682	1.98611e-10\\
1.14742628685657	2.04523e-10\\
1.14942528735632	2.01453e-10\\
1.15142428785607	1.95769e-10\\
1.15342328835582	1.98725e-10\\
1.15542228885557	1.98497e-10\\
1.15742128935532	1.99975e-10\\
1.15942028985507	2.01567e-10\\
1.16141929035482	1.97815e-10\\
1.16341829085457	1.97588e-10\\
1.16541729135432	2.01794e-10\\
1.16741629185407	2.0168e-10\\
1.16941529235382	1.95769e-10\\
1.17141429285357	1.9611e-10\\
1.17341329335332	1.94291e-10\\
1.17541229385307	1.9395e-10\\
1.17741129435282	2.0168e-10\\
1.17941029485257	2.0782e-10\\
1.18140929535232	2.06796e-10\\
1.18340829585207	2.06228e-10\\
1.18540729635182	2.12481e-10\\
1.18740629685157	2.08843e-10\\
1.18940529735132	2.03613e-10\\
1.19140429785107	2.0907e-10\\
1.19340329835082	2.13731e-10\\
1.19540229885057	2.11799e-10\\
1.19740129935032	2.13163e-10\\
1.19940029985008	2.15437e-10\\
1.20139930034983	2.21007e-10\\
1.20339830084958	2.22258e-10\\
1.20539730134933	2.28169e-10\\
1.20739630184908	2.30443e-10\\
1.20939530234883	2.30898e-10\\
1.21139430284858	2.26805e-10\\
1.21339330334833	2.23395e-10\\
1.21539230384808	2.26237e-10\\
1.21739130434783	2.32376e-10\\
1.21939030484758	2.29306e-10\\
1.22138930534733	2.34877e-10\\
1.22338830584708	2.3249e-10\\
1.22538730634683	2.27033e-10\\
1.22738630684658	2.34081e-10\\
1.22938530734633	2.35787e-10\\
1.23138430784608	2.41812e-10\\
1.23338330834583	2.43517e-10\\
1.23538230884558	2.467e-10\\
1.23738130934533	2.45677e-10\\
1.23938030984508	2.48747e-10\\
1.24137931034483	2.46132e-10\\
1.24337831084458	2.51703e-10\\
1.24537731134433	2.5841e-10\\
1.24737631184408	2.54317e-10\\
1.24937531234383	2.50225e-10\\
1.25137431284358	2.48406e-10\\
1.25337331334333	2.55227e-10\\
1.25537231384308	2.54772e-10\\
1.25737131434283	2.50679e-10\\
1.25937031484258	2.51475e-10\\
1.26136931534233	2.51589e-10\\
1.26336831584208	2.56705e-10\\
1.26536731634183	2.59661e-10\\
1.26736631684158	2.67391e-10\\
1.26936531734133	2.62276e-10\\
1.27136431784108	2.56819e-10\\
1.27336331834083	2.54204e-10\\
1.27536231884058	2.54431e-10\\
1.27736131934033	2.58296e-10\\
1.27936031984008	2.52498e-10\\
1.28135932033983	2.47837e-10\\
1.28335832083958	2.55e-10\\
1.28535732133933	2.56023e-10\\
1.28735632183908	2.60684e-10\\
1.28935532233883	2.53635e-10\\
1.29135432283858	2.46359e-10\\
1.29335332333833	2.40334e-10\\
1.29535232383808	2.42267e-10\\
1.29735132433783	2.3681e-10\\
1.29935032483758	2.34763e-10\\
1.30134932533733	2.42267e-10\\
1.30334832583708	2.39766e-10\\
1.30534732633683	2.34081e-10\\
1.30734632683658	2.31694e-10\\
1.30934532733633	2.27033e-10\\
1.31134432783608	2.27828e-10\\
1.31334332833583	2.20666e-10\\
1.31534232883558	2.16801e-10\\
1.31734132933533	2.2203e-10\\
1.31934032983508	2.16005e-10\\
1.32133933033483	2.23508e-10\\
1.32333833083458	2.19416e-10\\
1.32533733133433	2.1862e-10\\
1.32733633183408	2.17824e-10\\
1.32933533233383	2.14072e-10\\
1.33133433283358	2.18847e-10\\
1.33333333333333	2.1555e-10\\
1.33533233383308	2.21803e-10\\
1.33733133433283	2.21803e-10\\
1.33933033483258	2.19075e-10\\
1.34132933533233	2.23054e-10\\
1.34332833583208	2.28965e-10\\
1.34532733633183	2.29306e-10\\
1.34732633683158	2.35673e-10\\
1.34932533733133	2.35559e-10\\
1.35132433783108	2.29079e-10\\
1.35332333833083	2.26919e-10\\
1.35532233883058	2.19302e-10\\
1.35732133933033	2.26464e-10\\
1.35932033983009	2.28965e-10\\
1.36131934032984	2.3465e-10\\
1.36331834082959	2.31012e-10\\
1.36531734132934	2.32603e-10\\
1.36731634182909	2.36923e-10\\
1.36931534232884	2.32376e-10\\
1.37131434282859	2.32262e-10\\
1.37331334332834	2.39652e-10\\
1.37531234382809	2.45905e-10\\
1.37731134432784	2.42608e-10\\
1.37931034482759	2.43404e-10\\
1.38130934532734	2.47269e-10\\
1.38330834582709	2.42835e-10\\
1.38530734632684	2.44995e-10\\
1.38730634682659	2.37378e-10\\
1.38930534732634	2.39652e-10\\
1.39130434782609	2.43858e-10\\
1.39330334832584	2.36241e-10\\
1.39530234882559	2.31239e-10\\
1.39730134932534	2.23281e-10\\
1.39930034982509	2.20211e-10\\
1.40129935032484	2.16914e-10\\
1.40329835082459	2.18961e-10\\
1.40529735132434	2.16801e-10\\
1.40729635182409	2.15209e-10\\
1.40929535232384	2.13277e-10\\
1.41129435282359	2.14641e-10\\
1.41329335332334	2.14015e-10\\
1.41529235382309	2.11116e-10\\
1.41729135432284	2.07706e-10\\
1.41929035482259	2.08161e-10\\
1.42128935532234	2.04068e-10\\
1.42328835582209	2.0043e-10\\
1.42528735632184	1.94063e-10\\
1.42728635682159	1.88322e-10\\
1.42928535732134	1.9071e-10\\
1.43128435782109	1.8747e-10\\
1.43328335832084	1.87924e-10\\
1.43528235882059	1.83149e-10\\
1.43728135932034	1.90369e-10\\
1.43928035982009	1.9196e-10\\
1.44127936031984	1.86446e-10\\
1.44327836081959	1.90141e-10\\
1.44527736131934	1.94461e-10\\
1.44727636181909	1.99691e-10\\
1.44927536231884	1.9736e-10\\
1.45127436281859	1.96565e-10\\
1.45327336331834	2.006e-10\\
1.45527236381809	1.99577e-10\\
1.45727136431784	1.97133e-10\\
1.45927036481759	2.03386e-10\\
1.46126936531734	2.02533e-10\\
1.46326836581709	1.99691e-10\\
1.46526736631684	1.97304e-10\\
1.46726636681659	2.03727e-10\\
1.46926536731634	2.02931e-10\\
1.47126436781609	2.09525e-10\\
1.47326336831584	2.04523e-10\\
1.47526236881559	2.0043e-10\\
1.47726136931534	1.9844e-10\\
1.47926036981509	2.03784e-10\\
1.48125937031484	1.97588e-10\\
1.48325837081459	2.04523e-10\\
1.48525737131434	2.00544e-10\\
1.48725637181409	2.00544e-10\\
1.48925537231384	1.99861e-10\\
1.49125437281359	1.9736e-10\\
1.49325337331334	2.04636e-10\\
1.49525237381309	2.08445e-10\\
1.49725137431284	2.08843e-10\\
1.49925037481259	2.01112e-10\\
1.50124937531234	1.96394e-10\\
1.50324837581209	2.02022e-10\\
1.50524737631184	2.07081e-10\\
1.50724637681159	2.07876e-10\\
1.50924537731134	2.05318e-10\\
1.51124437781109	2.06626e-10\\
1.51324337831084	1.99236e-10\\
1.51524237881059	2.06171e-10\\
1.51724137931034	2.00032e-10\\
1.51924037981009	1.93324e-10\\
1.52123938030985	1.86333e-10\\
1.5232383808096	1.8656e-10\\
1.52523738130935	1.87981e-10\\
1.5272363818091	1.82297e-10\\
1.52923538230885	1.74907e-10\\
1.5312343828086	1.72065e-10\\
1.53323338330835	1.74623e-10\\
1.5352323838081	1.66608e-10\\
1.53723138430785	1.69678e-10\\
1.5392303848076	1.68143e-10\\
1.54122938530735	1.72008e-10\\
1.5432283858071	1.72861e-10\\
1.54522738630685	1.70587e-10\\
1.5472263868066	1.69848e-10\\
1.54922538730635	1.64562e-10\\
1.5512243878061	1.66779e-10\\
1.55322338830585	1.70985e-10\\
1.5552223888056	1.76101e-10\\
1.55722138930535	1.69962e-10\\
1.5592203898051	1.68939e-10\\
1.56121939030485	1.62345e-10\\
1.5632183908046	1.59616e-10\\
1.56521739130435	1.56774e-10\\
1.5672163918041	1.59503e-10\\
1.56921539230385	1.60412e-10\\
1.5712143928036	1.67802e-10\\
1.57321339330335	1.60412e-10\\
1.5752123938031	1.61094e-10\\
1.57721139430285	1.64277e-10\\
1.5792103948026	1.58138e-10\\
1.58120939530235	1.58479e-10\\
1.5832083958021	1.56206e-10\\
1.58520739630185	1.48816e-10\\
1.5872063968016	1.43359e-10\\
1.58920539730135	1.46997e-10\\
1.5912043978011	1.49839e-10\\
1.59320339830085	1.55183e-10\\
1.5952023988006	1.49271e-10\\
1.59720139930035	1.44155e-10\\
1.5992003998001	1.37902e-10\\
1.60119940029985	1.41767e-10\\
1.6031984007996	1.35969e-10\\
1.60519740129935	1.32673e-10\\
1.6071964017991	1.31195e-10\\
1.60919540229885	1.24373e-10\\
1.6111944027986	1.21531e-10\\
1.61319340329835	1.17439e-10\\
1.6151924037981	1.23919e-10\\
1.61719140429785	1.20281e-10\\
1.6191904047976	1.26875e-10\\
1.62118940529735	1.28352e-10\\
1.6231884057971	1.27898e-10\\
1.62518740629685	1.24032e-10\\
1.6271864067966	1.18462e-10\\
1.62918540729635	1.20622e-10\\
1.6311844077961	1.24373e-10\\
1.63318340829585	1.28239e-10\\
1.6351824087956	1.35401e-10\\
1.63718140929535	1.27557e-10\\
1.6391804097951	1.22668e-10\\
1.64117941029485	1.21304e-10\\
1.6431784107946	1.15051e-10\\
1.64517741129435	1.21759e-10\\
1.6471764117941	1.14596e-10\\
1.64917541229385	1.17097e-10\\
1.6511744127936	1.21872e-10\\
1.65317341329335	1.19371e-10\\
1.6551724137931	1.25283e-10\\
1.65717141429285	1.29148e-10\\
1.6591704147926	1.33468e-10\\
1.66116941529235	1.28011e-10\\
1.6631684157921	1.29148e-10\\
1.66516741629185	1.31877e-10\\
1.6671664167916	1.38357e-10\\
1.66916541729135	1.39835e-10\\
1.6711644177911	1.39607e-10\\
1.67316341829085	1.40062e-10\\
1.6751624187906	1.41767e-10\\
1.67716141929035	1.41085e-10\\
1.6791604197901	1.33582e-10\\
1.68115942028985	1.37561e-10\\
1.68315842078961	1.39721e-10\\
1.68515742128936	1.37447e-10\\
1.68715642178911	1.43928e-10\\
1.68915542228886	1.42109e-10\\
1.69115442278861	1.37675e-10\\
1.69315342328836	1.40631e-10\\
1.69515242378811	1.46429e-10\\
1.69715142428786	1.51203e-10\\
1.69915042478761	1.45747e-10\\
1.70114942528736	1.5109e-10\\
1.70314842578711	1.56774e-10\\
1.70514742628686	1.50976e-10\\
1.70714642678661	1.45633e-10\\
1.70914542728636	1.53477e-10\\
1.71114442778611	1.59844e-10\\
1.71314342828586	1.61094e-10\\
1.71514242878561	1.57911e-10\\
1.71714142928536	1.50521e-10\\
1.71914042978511	1.44269e-10\\
1.72113943028486	1.39039e-10\\
1.72313843078461	1.42336e-10\\
1.72513743128436	1.42336e-10\\
1.72713643178411	1.38698e-10\\
1.72913543228386	1.35174e-10\\
1.73113443278361	1.34719e-10\\
1.73313343328336	1.33696e-10\\
1.73513243378311	1.28125e-10\\
1.73713143428286	1.2426e-10\\
1.73913043478261	1.18007e-10\\
1.74112943528236	1.10276e-10\\
1.74312843578211	1.05274e-10\\
1.74512743628186	1.07093e-10\\
1.74712643678161	1.13687e-10\\
1.74912543728136	1.21304e-10\\
1.75112443778111	1.24714e-10\\
1.75312343828086	1.22554e-10\\
1.75512243878061	1.1903e-10\\
1.75712143928036	1.26192e-10\\
1.75912043978011	1.27898e-10\\
1.76111944027986	1.2335e-10\\
1.76311844077961	1.27784e-10\\
1.76511744127936	1.24032e-10\\
1.76711644177911	1.30171e-10\\
1.76911544227886	1.34492e-10\\
1.77111444277861	1.41426e-10\\
1.77311344327836	1.38016e-10\\
1.77511244377811	1.34264e-10\\
1.77711144427786	1.33355e-10\\
1.77911044477761	1.3415e-10\\
1.78110944527736	1.40062e-10\\
1.78310844577711	1.45292e-10\\
1.78510744627686	1.47338e-10\\
1.78710644677661	1.51545e-10\\
1.78910544727636	1.51658e-10\\
1.79110444777611	1.47907e-10\\
1.79310344827586	1.42563e-10\\
1.79510244877561	1.39266e-10\\
1.79710144927536	1.33696e-10\\
1.79910044977511	1.26192e-10\\
1.80109945027486	1.21531e-10\\
1.80309845077461	1.2119e-10\\
1.80509745127436	1.1778e-10\\
1.80709645177411	1.10276e-10\\
1.80909545227386	1.09821e-10\\
1.81109445277361	1.15733e-10\\
1.81309345327336	1.14369e-10\\
1.81509245377311	1.11413e-10\\
1.81709145427286	1.07207e-10\\
1.81909045477261	1.07207e-10\\
1.82108945527236	1.06638e-10\\
1.82308845577211	1.08685e-10\\
1.82508745627186	1.04819e-10\\
1.82708645677161	1.02318e-10\\
1.82908545727136	9.91349e-11\\
1.83108445777111	9.1859e-11\\
1.83308345827086	9.09495e-11\\
1.83508245877061	9.89075e-11\\
1.83708145927036	1.02773e-10\\
1.83908045977011	9.75433e-11\\
1.84107946026987	9.70886e-11\\
1.84307846076962	9.27685e-11\\
1.84507746126937	9.45874e-11\\
1.84707646176912	9.54969e-11\\
1.84907546226887	9.77707e-11\\
1.85107446276862	9.77707e-11\\
1.85307346326837	1.04592e-10\\
1.85507246376812	1.11186e-10\\
1.85707146426787	1.10504e-10\\
1.85907046476762	1.05501e-10\\
1.86106946526737	1.03682e-10\\
1.86306846576712	9.57243e-11\\
1.86506746626687	9.50422e-11\\
1.86706646676662	9.84528e-11\\
1.86906546726637	9.86802e-11\\
1.87106446776612	9.50422e-11\\
1.87306346826587	9.45874e-11\\
1.87506246876562	9.95897e-11\\
1.87706146926537	1.02546e-10\\
1.87906046976512	9.77707e-11\\
1.88105947026487	9.11768e-11\\
1.88305847076462	9.52696e-11\\
1.88505747126437	9.07221e-11\\
1.88705647176412	9.004e-11\\
1.88905547226387	8.6402e-11\\
1.89105447276362	9.29958e-11\\
1.89305347326337	8.77662e-11\\
1.89505247376312	9.1859e-11\\
1.89705147426287	8.39009e-11\\
1.89905047476262	7.95808e-11\\
1.90104947526237	7.41238e-11\\
1.90304847576212	6.9349e-11\\
1.90504747626187	6.63931e-11\\
1.90704647676162	7.32143e-11\\
1.90904547726137	7.59428e-11\\
1.91104447776112	7.48059e-11\\
1.91304347826087	7.45786e-11\\
1.91504247876062	7.16227e-11\\
1.91704147926037	6.86668e-11\\
1.91904047976012	6.9349e-11\\
1.92103948025987	6.98037e-11\\
1.92303848075962	6.36646e-11\\
1.92503748125937	6.82121e-11\\
1.92703648175912	6.61657e-11\\
1.92903548225887	6.48015e-11\\
1.93103448275862	6.23004e-11\\
1.93303348325837	6.61657e-11\\
1.93503248375812	7.07132e-11\\
1.93703148425787	7.38964e-11\\
1.93903048475762	6.98037e-11\\
1.94102948525737	6.79847e-11\\
1.94302848575712	7.36691e-11\\
1.94502748625687	6.86668e-11\\
1.94702648675662	6.13909e-11\\
1.94902548725637	6.50289e-11\\
1.95102448775612	6.84395e-11\\
1.95302348825587	7.41238e-11\\
1.95502248875562	7.59428e-11\\
1.95702148925537	8.13998e-11\\
1.95902048975512	8.23093e-11\\
1.96101949025487	7.75344e-11\\
1.96301849075462	7.70797e-11\\
1.96501749125437	7.66249e-11\\
1.96701649175412	8.34461e-11\\
1.96901549225387	8.86757e-11\\
1.97101449275362	8.29914e-11\\
1.97301349325337	8.04903e-11\\
1.97501249375312	7.95808e-11\\
1.97701149425287	8.41283e-11\\
1.97901049475262	8.91305e-11\\
1.98100949525237	8.52651e-11\\
1.98300849575212	9.14042e-11\\
1.98500749625187	9.64064e-11\\
1.98700649675162	9.77707e-11\\
1.98900549725137	9.45874e-11\\
1.99100449775112	9.95897e-11\\
1.99300349825087	9.50422e-11\\
1.99500249875062	1.00727e-10\\
1.99700149925037	9.41327e-11\\
1.99900049975012	9.27685e-11\\
2.00099950024988	8.48104e-11\\
2.00299850074963	8.11724e-11\\
2.00499750124938	7.9126e-11\\
2.00699650174913	8.43556e-11\\
2.00899550224888	7.75344e-11\\
2.01099450274863	7.29869e-11\\
2.01299350324838	7.18501e-11\\
2.01499250374813	6.54836e-11\\
2.01699150424788	6.11635e-11\\
2.01899050474763	5.6616e-11\\
2.02098950524738	6.27551e-11\\
2.02298850574713	6.9349e-11\\
2.02498750624688	7.66249e-11\\
2.02698650674663	7.43512e-11\\
2.02898550724638	7.98082e-11\\
2.03098450774613	8.43556e-11\\
2.03298350824588	8.54925e-11\\
2.03498250874563	9.02673e-11\\
2.03698150924538	8.36735e-11\\
2.03898050974513	8.68567e-11\\
2.04097951024488	9.34506e-11\\
2.04297851074463	9.07221e-11\\
2.04497751124438	8.48104e-11\\
2.04697651174413	8.59472e-11\\
2.04897551224388	8.18545e-11\\
2.05097451274363	8.13998e-11\\
2.05297351324338	7.88987e-11\\
2.05497251374313	7.68523e-11\\
2.05697151424288	7.48059e-11\\
2.05897051474263	6.79847e-11\\
2.06096951524238	7.13953e-11\\
2.06296851574213	6.43468e-11\\
2.06496751624188	6.48015e-11\\
2.06696651674163	6.04814e-11\\
2.06896551724138	5.63887e-11\\
2.07096451774113	5.16138e-11\\
2.07296351824088	5.29781e-11\\
2.07496251874063	5.41149e-11\\
2.07696151924038	5.86624e-11\\
2.07896051974013	5.07043e-11\\
2.08095952023988	4.72937e-11\\
2.08295852073963	5.50244e-11\\
2.08495752123938	5.00222e-11\\
2.08695652173913	4.91127e-11\\
2.08895552223888	5.36602e-11\\
2.09095452273863	5.93445e-11\\
2.09295352323838	5.86624e-11\\
2.09495252373813	5.41149e-11\\
2.09695152423788	5.77529e-11\\
2.09895052473763	5.70708e-11\\
2.10094952523738	6.29825e-11\\
2.10294852573713	6.00267e-11\\
2.10494752623688	6.07088e-11\\
2.10694652673663	6.66205e-11\\
2.10894552723638	6.43468e-11\\
2.11094452773613	6.66205e-11\\
2.11294352823588	7.43512e-11\\
2.11494252873563	8.07177e-11\\
2.11694152923538	8.79936e-11\\
2.11894052973513	8.48104e-11\\
2.12093953023488	7.75344e-11\\
2.12293853073463	7.75344e-11\\
2.12493753123438	8.13998e-11\\
2.12693653173413	7.9126e-11\\
2.12893553223388	7.63976e-11\\
2.13093453273363	7.02585e-11\\
2.13293353323338	6.73026e-11\\
2.13493253373313	7.23048e-11\\
2.13693153423288	7.95808e-11\\
2.13893053473263	7.25322e-11\\
2.14092953523238	7.9126e-11\\
2.14292853573213	8.4583e-11\\
2.14492753623188	8.29914e-11\\
2.14692653673163	7.59428e-11\\
2.14892553723138	8.11724e-11\\
2.15092453773113	8.54925e-11\\
2.15292353823088	9.20863e-11\\
2.15492253873063	9.57243e-11\\
2.15692153923038	9.95897e-11\\
2.15892053973013	1.00727e-10\\
2.16091954022989	9.84528e-11\\
2.16291854072964	9.41327e-11\\
2.16491754122939	9.66338e-11\\
2.16691654172914	9.9817e-11\\
2.16891554222889	1.03e-10\\
2.17091454272864	9.70886e-11\\
2.17291354322839	9.59517e-11\\
2.17491254372814	9.07221e-11\\
2.17691154422789	9.29958e-11\\
2.17891054472764	9.39053e-11\\
2.18090954522739	9.29958e-11\\
2.18290854572714	8.4583e-11\\
2.18490754622689	9.20863e-11\\
2.18690654672664	9.16316e-11\\
2.18890554722639	8.50378e-11\\
2.19090454772614	7.82165e-11\\
2.19290354822589	7.50333e-11\\
2.19490254872564	8.00355e-11\\
2.19690154922539	8.02629e-11\\
2.19890054972514	7.52607e-11\\
2.20089955022489	6.98037e-11\\
2.20289855072464	7.68523e-11\\
2.20489755122439	7.25322e-11\\
2.20689655172414	6.98037e-11\\
2.20889555222389	6.23004e-11\\
2.21089455272364	6.09361e-11\\
2.21289355322339	5.91172e-11\\
2.21489255372314	5.59339e-11\\
2.21689155422289	4.95675e-11\\
2.21889055472264	4.97948e-11\\
2.22088955522239	5.34328e-11\\
2.22288855572214	5.6616e-11\\
2.22488755622189	4.88853e-11\\
2.22688655672164	5.61613e-11\\
2.22888555722139	5.43423e-11\\
2.23088455772114	4.70664e-11\\
2.23288355822089	4.18368e-11\\
2.23488255872064	4.61569e-11\\
2.23688155922039	4.95675e-11\\
2.23888055972014	5.11591e-11\\
2.24087956021989	5.22959e-11\\
2.24287856071964	4.72937e-11\\
2.24487756121939	4.61569e-11\\
2.24687656171914	4.6839e-11\\
2.24887556221889	5.45697e-11\\
2.25087456271864	5.43423e-11\\
2.25287356321839	5.91172e-11\\
2.25487256371814	6.66205e-11\\
2.25687156421789	6.48015e-11\\
2.25887056471764	6.5711e-11\\
2.26086956521739	6.59384e-11\\
2.26286856571714	7.13953e-11\\
2.26486756621689	6.43468e-11\\
2.26686656671664	6.2073e-11\\
2.26886556721639	6.3892e-11\\
2.27086456771614	5.79803e-11\\
2.27286356821589	5.41149e-11\\
2.27486256871564	5.77529e-11\\
2.27686156921539	5.97993e-11\\
2.27886056971514	5.36602e-11\\
2.28085957021489	6.04814e-11\\
2.28285857071464	6.16183e-11\\
2.28485757121439	5.75255e-11\\
2.28685657171414	6.2073e-11\\
2.28885557221389	6.54836e-11\\
2.29085457271364	6.98037e-11\\
2.29285357321339	7.41238e-11\\
2.29485257371314	7.18501e-11\\
2.29685157421289	7.02585e-11\\
2.29885057471264	7.02585e-11\\
2.30084957521239	7.43512e-11\\
2.30284857571214	7.70797e-11\\
2.30484757621189	8.43556e-11\\
2.30684657671164	8.66294e-11\\
2.30884557721139	8.39009e-11\\
2.31084457771114	8.20819e-11\\
2.31284357821089	7.98082e-11\\
2.31484257871064	8.48104e-11\\
2.31684157921039	9.02673e-11\\
2.31884057971015	9.45874e-11\\
2.3208395802099	9.91349e-11\\
2.32283858070965	9.75433e-11\\
2.3248375812094	9.16316e-11\\
2.32683658170915	8.68567e-11\\
2.3288355822089	9.1859e-11\\
2.33083458270865	8.75389e-11\\
2.3328335832084	9.25411e-11\\
2.33483258370815	8.50378e-11\\
2.3368315842079	7.75344e-11\\
2.33883058470765	7.43512e-11\\
2.3408295852074	6.98037e-11\\
2.34282858570715	6.2073e-11\\
2.3448275862069	6.00267e-11\\
2.34682658670665	5.8435e-11\\
2.3488255872064	5.29781e-11\\
2.35082458770615	6.04814e-11\\
2.3528235882059	5.50244e-11\\
2.35482258870565	4.82032e-11\\
2.3568215892054	4.27463e-11\\
2.35882058970515	3.97904e-11\\
2.3608195902049	4.25189e-11\\
2.36281859070465	4.29736e-11\\
2.3648175912044	3.52429e-11\\
2.36681659170415	2.95586e-11\\
2.3688155922039	3.50155e-11\\
2.37081459270365	3.06954e-11\\
2.3728135932034	3.22871e-11\\
2.37481259370315	3.86535e-11\\
2.3768115942029	3.36513e-11\\
2.37881059470265	2.56932e-11\\
2.3808095952024	2.16005e-11\\
2.38280859570215	2.4329e-11\\
2.3848075962019	1.72804e-11\\
2.38680659670165	2.41016e-11\\
2.3888055972014	2.86491e-11\\
2.39080459770115	2.77396e-11\\
2.3928035982009	2.13731e-11\\
2.39480259870065	1.7053e-11\\
2.3968015992004	2.22826e-11\\
2.39880059970015	2.41016e-11\\
2.4007996001999	2.29647e-11\\
2.40279860069965	2.6148e-11\\
2.4047976011994	2.50111e-11\\
2.40679660169915	2.72848e-11\\
2.4087956021989	3.22871e-11\\
2.41079460269865	3.34239e-11\\
2.4127936031984	3.11502e-11\\
2.41479260369815	3.18323e-11\\
2.4167916041979	3.20597e-11\\
2.41879060469765	3.70619e-11\\
2.4207896051974	4.16094e-11\\
2.42278860569715	4.84306e-11\\
2.4247876061969	5.57066e-11\\
2.42678660669665	5.34328e-11\\
2.4287856071964	5.93445e-11\\
2.43078460769615	5.75255e-11\\
2.4327836081959	6.2073e-11\\
2.43478260869565	5.54792e-11\\
2.4367816091954	5.91172e-11\\
2.43878060969515	6.66205e-11\\
2.4407796101949	6.82121e-11\\
2.44277861069465	7.18501e-11\\
2.4447776111944	7.20775e-11\\
2.44677661169415	7.84439e-11\\
2.4487756121939	7.41238e-11\\
2.45077461269365	7.04858e-11\\
2.4527736131934	7.38964e-11\\
2.45477261369315	6.66205e-11\\
2.4567716141929	6.54836e-11\\
2.45877061469265	6.29825e-11\\
2.4607696151924	6.09361e-11\\
2.46276861569215	6.0254e-11\\
2.4647676161919	6.52562e-11\\
2.46676661669165	7.02585e-11\\
2.4687656171914	6.54836e-11\\
2.47076461769115	7.16227e-11\\
2.4727636181909	6.70752e-11\\
2.47476261869065	6.77574e-11\\
2.4767616191904	6.09361e-11\\
2.47876061969016	6.82121e-11\\
2.48075962018991	6.09361e-11\\
2.48275862068966	6.23004e-11\\
2.48475762118941	6.0254e-11\\
2.48675662168916	6.36646e-11\\
2.48875562218891	6.86668e-11\\
2.49075462268866	6.2073e-11\\
2.49275362318841	7.00311e-11\\
2.49475262368816	7.20775e-11\\
2.49675162418791	7.66249e-11\\
2.49875062468766	8.39009e-11\\
2.50074962518741	8.77662e-11\\
2.50274862568716	9.52696e-11\\
2.50474762618691	9.61791e-11\\
2.50674662668666	8.98126e-11\\
2.50874562718641	9.004e-11\\
2.51074462768616	8.86757e-11\\
2.51274362818591	8.25366e-11\\
2.51474262868566	8.43556e-11\\
2.51674162918541	8.43556e-11\\
2.51874062968516	7.95808e-11\\
2.52073963018491	8.16271e-11\\
2.52273863068466	7.93534e-11\\
2.52473763118441	7.38964e-11\\
2.52673663168416	6.91216e-11\\
2.52873563218391	7.07132e-11\\
2.53073463268366	6.50289e-11\\
2.53273363318341	6.07088e-11\\
2.53473263368316	5.52518e-11\\
2.53673163418291	6.16183e-11\\
2.53873063468266	6.91216e-11\\
2.54072963518241	6.29825e-11\\
2.54272863568216	6.16183e-11\\
2.54472763618191	5.8435e-11\\
2.54672663668166	5.29781e-11\\
2.54872563718141	5.95719e-11\\
2.55072463768116	5.34328e-11\\
2.55272363818091	5.95719e-11\\
2.55472263868066	6.61657e-11\\
2.55672163918041	7.09406e-11\\
2.55872063968016	6.36646e-11\\
2.56071964017991	6.3892e-11\\
2.56271864067966	6.88942e-11\\
2.56471764117941	7.34417e-11\\
2.56671664167916	6.59384e-11\\
2.56871564217891	6.48015e-11\\
2.57071464267866	7.07132e-11\\
2.57271364317841	6.61657e-11\\
2.57471264367816	6.59384e-11\\
2.57671164417791	6.753e-11\\
2.57871064467766	6.34373e-11\\
2.58070964517741	6.95763e-11\\
2.58270864567716	7.50333e-11\\
2.58470764617691	8.20819e-11\\
2.58670664667666	7.84439e-11\\
2.58870564717641	8.00355e-11\\
2.59070464767616	7.45786e-11\\
2.59270364817591	7.00311e-11\\
2.59470264867566	7.70797e-11\\
2.59670164917541	7.02585e-11\\
2.59870064967516	7.32143e-11\\
2.60069965017491	6.5711e-11\\
2.60269865067466	6.11635e-11\\
2.60469765117441	5.79803e-11\\
2.60669665167416	6.36646e-11\\
2.60869565217391	7.07132e-11\\
2.61069465267366	6.3892e-11\\
2.61269365317341	6.98037e-11\\
2.61469265367316	6.23004e-11\\
2.61669165417291	5.77529e-11\\
2.61869065467266	6.43468e-11\\
2.62068965517241	6.09361e-11\\
2.62268865567216	6.86668e-11\\
2.62468765617191	6.18456e-11\\
2.62668665667166	6.54836e-11\\
2.62868565717141	6.13909e-11\\
2.63068465767116	6.32099e-11\\
2.63268365817091	6.32099e-11\\
2.63468265867066	6.86668e-11\\
2.63668165917041	7.18501e-11\\
2.63868065967017	7.79892e-11\\
2.64067966016992	8.00355e-11\\
2.64267866066967	7.84439e-11\\
2.64467766116942	7.45786e-11\\
2.64667666166917	7.36691e-11\\
2.64867566216892	7.13953e-11\\
2.65067466266867	7.07132e-11\\
2.65267366316842	6.5711e-11\\
2.65467266366817	5.86624e-11\\
2.65667166416792	5.50244e-11\\
2.65867066466767	4.77485e-11\\
2.66066966516742	4.91127e-11\\
2.66266866566717	5.43423e-11\\
2.66466766616692	4.93401e-11\\
2.66666666666667	5.34328e-11\\
2.66866566716642	5.61613e-11\\
2.67066466766617	6.2073e-11\\
2.67266366816592	5.50244e-11\\
2.67466266866567	5.72982e-11\\
2.67666166916542	6.2073e-11\\
2.67866066966517	6.88942e-11\\
2.68065967016492	6.88942e-11\\
2.68265867066467	7.43512e-11\\
2.68465767116442	7.50333e-11\\
2.68665667166417	6.9349e-11\\
2.68865567216392	6.84395e-11\\
2.69065467266367	6.18456e-11\\
2.69265367316342	6.48015e-11\\
2.69465267366317	6.09361e-11\\
2.69665167416292	5.75255e-11\\
2.69865067466267	6.36646e-11\\
2.70064967516242	7.1168e-11\\
2.70264867566217	7.7307e-11\\
2.70464767616192	8.16271e-11\\
2.70664667666167	8.86757e-11\\
2.70864567716142	8.18545e-11\\
2.71064467766117	8.13998e-11\\
2.71264367816092	8.8221e-11\\
2.71464267866067	8.36735e-11\\
2.71664167916042	8.4583e-11\\
2.71864067966017	8.95852e-11\\
2.72063968015992	8.32188e-11\\
2.72263868065967	8.73115e-11\\
2.72463768115942	9.004e-11\\
2.72663668165917	8.48104e-11\\
2.72863568215892	8.04903e-11\\
2.73063468265867	7.61702e-11\\
2.73263368315842	8.25366e-11\\
2.73463268365817	7.54881e-11\\
2.73663168415792	7.18501e-11\\
2.73863068465767	7.32143e-11\\
2.74062968515742	7.95808e-11\\
2.74262868565717	8.20819e-11\\
2.74462768615692	7.45786e-11\\
2.74662668665667	7.1168e-11\\
2.74862568715642	6.84395e-11\\
2.75062468765617	6.5711e-11\\
2.75262368815592	6.41194e-11\\
2.75462268865567	6.00267e-11\\
2.75662168915542	5.52518e-11\\
2.75862068965517	5.25233e-11\\
2.76061969015492	4.57021e-11\\
2.76261869065467	5.22959e-11\\
2.76461769115442	4.93401e-11\\
2.76661669165417	4.43379e-11\\
2.76861569215392	4.41105e-11\\
2.77061469265367	4.82032e-11\\
2.77261369315342	4.72937e-11\\
2.77461269365317	4.57021e-11\\
2.77661169415292	3.97904e-11\\
2.77861069465267	3.20597e-11\\
2.78060969515242	3.45608e-11\\
2.78260869565217	3.93356e-11\\
2.78460769615192	4.04725e-11\\
2.78660669665167	4.02451e-11\\
2.78860569715142	3.75167e-11\\
2.79060469765117	3.36513e-11\\
2.79260369815092	3.31966e-11\\
2.79460269865067	3.00133e-11\\
2.79660169915043	3.29692e-11\\
2.79860069965018	2.77396e-11\\
2.80059970014993	2.66027e-11\\
2.80259870064968	2.0691e-11\\
2.80459770114943	1.75078e-11\\
2.80659670164918	1.06866e-11\\
2.80859570214893	6.36646e-12\\
2.81059470264868	8.6402e-12\\
2.81259370314843	1.3415e-11\\
2.81459270364818	1.11413e-11\\
2.81659170414793	7.50333e-12\\
2.81859070464768	7.95808e-12\\
2.82058970514743	6.82121e-12\\
2.82258870564718	1.18234e-11\\
2.82458770614693	1.3074e-11\\
2.82658670664668	1.86446e-11\\
2.82858570714643	1.73941e-11\\
2.83058470764618	1.06866e-11\\
2.83258370814593	1.21645e-11\\
2.83458270864568	1.18234e-11\\
2.83658170914543	1.69393e-11\\
2.83858070964518	2.31921e-11\\
2.84057971014493	1.98952e-11\\
2.84257871064468	1.26192e-11\\
2.84457771114443	9.20863e-12\\
2.84657671164418	3.97904e-12\\
2.84857571214393	4.09273e-12\\
2.85057471264368	8.29914e-12\\
2.85257371314343	9.54969e-12\\
2.85457271364318	5.34328e-12\\
2.85657171414293	8.41283e-12\\
2.85857071464268	3.41061e-12\\
2.86056971514243	7.95808e-12\\
2.86256871564218	1.00044e-11\\
2.86456771614193	8.29914e-12\\
2.86656671664168	1.59162e-11\\
2.86856571714143	8.52651e-12\\
2.87056471764118	6.25278e-12\\
2.87256371814093	1.11413e-11\\
2.87456271864068	1.60298e-11\\
2.87656171914043	1.02318e-11\\
2.87856071964018	1.46656e-11\\
2.88055972013993	1.53477e-11\\
2.88255872063968	1.18234e-11\\
2.88455772113943	1.23919e-11\\
2.88655672163918	1.72804e-11\\
2.88855572213893	1.69393e-11\\
2.89055472263868	9.77707e-12\\
2.89255372313843	6.82121e-12\\
2.89455272363818	9.66338e-12\\
2.89655172413793	1.01181e-11\\
2.89855072463768	1.46656e-11\\
2.90054972513743	1.08002e-11\\
2.90254872563718	8.41283e-12\\
2.90454772613693	1.63709e-11\\
2.90654672663668	8.86757e-12\\
2.90854572713643	8.07177e-12\\
2.91054472763618	5.11591e-12\\
2.91254372813593	7.84439e-12\\
2.91454272863568	2.50111e-12\\
2.91654172913543	6.82121e-12\\
2.91854072963518	1.15961e-11\\
2.92053973013493	1.89857e-11\\
2.92253873063468	1.64846e-11\\
2.92453773113443	2.28511e-11\\
2.92653673163418	2.36469e-11\\
2.92853573213393	1.59162e-11\\
2.93053473263368	2.37605e-11\\
2.93253373313343	2.4329e-11\\
2.93453273363318	1.86446e-11\\
2.93653173413293	2.21689e-11\\
2.93853073463268	2.37605e-11\\
2.94052973513243	1.78488e-11\\
2.94252873563218	2.22826e-11\\
2.94452773613193	2.73985e-11\\
2.94652673663168	2.37605e-11\\
2.94852573713143	2.10321e-11\\
2.95052473763118	2.60343e-11\\
2.95252373813093	3.36513e-11\\
2.95452273863068	2.81943e-11\\
2.95652173913043	3.47882e-11\\
2.95852073963018	2.76259e-11\\
2.96051974012994	3.3765e-11\\
2.96251874062969	4.03588e-11\\
2.96451774112944	3.66072e-11\\
2.96651674162919	3.43334e-11\\
2.96851574212894	3.20597e-11\\
2.97051474262869	3.49019e-11\\
2.97251374312844	3.93356e-11\\
2.97451274362819	3.89946e-11\\
2.97651174412794	3.85398e-11\\
2.97851074462769	3.63798e-11\\
2.98050974512744	3.88809e-11\\
2.98250874562719	4.52474e-11\\
2.98450774612694	5.27507e-11\\
2.98650674662669	5.94582e-11\\
2.98850574712644	5.65024e-11\\
2.99050474762619	5.77529e-11\\
2.99250374812594	6.42331e-11\\
2.99450274862569	6.45741e-11\\
2.99650174912544	6.25278e-11\\
2.99850074962519	6.3892e-11\\
3.00049975012494	6.5711e-11\\
3.00249875062469	7.3328e-11\\
3.00449775112444	8.10587e-11\\
3.00649675162419	8.53788e-11\\
3.00849575212394	7.92397e-11\\
3.01049475262369	7.76481e-11\\
3.01249375312344	7.93534e-11\\
3.01449275362319	7.60565e-11\\
3.01649175412294	7.63976e-11\\
3.01849075462269	7.01448e-11\\
3.02048975512244	7.27596e-11\\
3.02248875562219	7.29869e-11\\
3.02448775612194	7.98082e-11\\
3.02648675662169	8.2764e-11\\
3.02848575712144	8.33325e-11\\
3.03048475762119	7.74207e-11\\
3.03248375812094	8.48104e-11\\
3.03448275862069	8.50378e-11\\
3.03648175912044	9.25411e-11\\
3.03848075962019	9.07221e-11\\
3.04047976011994	9.64064e-11\\
3.04247876061969	9.11768e-11\\
3.04447776111944	8.77662e-11\\
3.04647676161919	8.89031e-11\\
3.04847576211894	8.41283e-11\\
3.05047476261869	8.0945e-11\\
3.05247376311844	7.93534e-11\\
3.05447276361819	7.12816e-11\\
3.05647176411794	7.20775e-11\\
3.05847076461769	6.41194e-11\\
3.06046976511744	6.69615e-11\\
3.06246876561719	7.3328e-11\\
3.06446776611694	6.5711e-11\\
3.06646676661669	7.1509e-11\\
3.06846576711644	7.53744e-11\\
3.07046476761619	7.23048e-11\\
3.07246376811594	6.66205e-11\\
3.07446276861569	6.94627e-11\\
3.07646176911544	6.37783e-11\\
3.07846076961519	5.63887e-11\\
3.08045977011494	5.11591e-11\\
3.08245877061469	5.09317e-11\\
3.08445777111444	5.17275e-11\\
3.08645677161419	5.28644e-11\\
3.08845577211394	4.96811e-11\\
3.09045477261369	4.28599e-11\\
3.09245377311344	4.03588e-11\\
3.09445277361319	4.6839e-11\\
3.09645177411294	4.72369e-11\\
3.09845077461269	5.08749e-11\\
3.10044977511244	4.88853e-11\\
3.10244877561219	4.23483e-11\\
3.10444777611194	4.80895e-11\\
3.10644677661169	4.75211e-11\\
3.10844577711144	5.47971e-11\\
3.11044477761119	5.93445e-11\\
3.11244377811094	6.16751e-11\\
3.11444277861069	6.94627e-11\\
3.11644177911044	7.44649e-11\\
3.11844077961019	7.69091e-11\\
3.12043978010994	7.4067e-11\\
3.12243878060969	6.90079e-11\\
3.12443778110945	6.60521e-11\\
3.1264367816092	6.55973e-11\\
3.12843578210895	7.06564e-11\\
3.1304347826087	7.70228e-11\\
3.13243378310845	7.71934e-11\\
3.1344327836082	8.36735e-11\\
3.13643178410795	8.54925e-11\\
3.1384307846077	7.72502e-11\\
3.14042978510745	7.03153e-11\\
3.1424287856072	6.47447e-11\\
3.14442778610695	6.94058e-11\\
3.1464267866067	7.09406e-11\\
3.14842578710645	6.50289e-11\\
3.1504247876062	7.03153e-11\\
3.15242378810595	6.37783e-11\\
3.1544227886057	5.8435e-11\\
3.15642178910545	5.16707e-11\\
3.1584207896052	4.43947e-11\\
3.16041979010495	3.68914e-11\\
3.1624187906047	3.28555e-11\\
3.16441779110445	3.80851e-11\\
3.1664167916042	3.46745e-11\\
3.16841579210395	3.06386e-11\\
3.1704147926037	3.38218e-11\\
3.17241379310345	4.0302e-11\\
3.1744127936032	4.07567e-11\\
3.17641179410295	4.3201e-11\\
3.1784107946027	4.35989e-11\\
3.18040979510245	4.83737e-11\\
3.1824087956022	4.38831e-11\\
3.18440779610195	4.6839e-11\\
3.1864067966017	5.22391e-11\\
3.18840579710145	6.00835e-11\\
3.1904047976012	6.58247e-11\\
3.19240379810095	5.93445e-11\\
3.1944027986007	5.39444e-11\\
3.19640179910045	5.3376e-11\\
3.1984007996002	4.91127e-11\\
3.20039980009995	5.04201e-11\\
3.2023988005997	4.34852e-11\\
3.20439780109945	4.95675e-11\\
3.2063968015992	5.54223e-11\\
3.20839580209895	5.81508e-11\\
3.2103948025987	6.24709e-11\\
3.21239380309845	5.96287e-11\\
3.2143928035982	6.40057e-11\\
3.21639180409795	6.92353e-11\\
3.2183908045977	7.17364e-11\\
3.22038980509745	7.48628e-11\\
3.2223888055972	7.70228e-11\\
3.22438780609695	8.44125e-11\\
3.2263868065967	8.78799e-11\\
3.22838580709645	9.24842e-11\\
3.2303848075962	9.19727e-11\\
3.23238380809595	8.91305e-11\\
3.2343828085957	9.31095e-11\\
3.23638180909545	8.79936e-11\\
3.2383808095952	9.33369e-11\\
3.24037981009495	8.98126e-11\\
3.2423788105947	9.72022e-11\\
3.24437781109445	9.9476e-11\\
3.2463768115942	9.72022e-11\\
3.24837581209395	1.00272e-10\\
3.2503748125937	1.00499e-10\\
3.25237381309345	1.06411e-10\\
3.2543728135932	9.92486e-11\\
3.25637181409295	1.04137e-10\\
3.2583708145927	1.06297e-10\\
3.26036981509245	1.08571e-10\\
3.2623688155922	1.11868e-10\\
3.26436781609195	1.17552e-10\\
3.2663668165917	1.11072e-10\\
3.26836581709145	1.11982e-10\\
3.2703648175912	1.16529e-10\\
3.27236381809095	1.15961e-10\\
3.2743628185907	1.19144e-10\\
3.27636181909045	1.13459e-10\\
3.2783608195902	1.17552e-10\\
3.28035982008995	1.21531e-10\\
3.2823588205897	1.15165e-10\\
3.28435782108946	1.20849e-10\\
3.28635682158921	1.17097e-10\\
3.28835582208896	1.24714e-10\\
3.29035482258871	1.25056e-10\\
3.29235382308846	1.21076e-10\\
3.29435282358821	1.18348e-10\\
3.29635182408796	1.17893e-10\\
3.29835082458771	1.15278e-10\\
3.30034982508746	1.21531e-10\\
3.30234882558721	1.24601e-10\\
3.30434782608696	1.23464e-10\\
3.30634682658671	1.18803e-10\\
3.30834582708646	1.1687e-10\\
3.31034482758621	1.20622e-10\\
3.31234382808596	1.21645e-10\\
3.31434282858571	1.26306e-10\\
3.31634182908546	1.28921e-10\\
3.31834082958521	1.22327e-10\\
3.32033983008496	1.19371e-10\\
3.32233883058471	1.24373e-10\\
3.32433783108446	1.24032e-10\\
3.32633683158421	1.17325e-10\\
3.32833583208396	1.12664e-10\\
3.33033483258371	1.10163e-10\\
3.33233383308346	1.06297e-10\\
3.33433283358321	1.12436e-10\\
3.33633183408296	1.06525e-10\\
3.33833083458271	1.08571e-10\\
3.34032983508246	1.01522e-10\\
3.34232883558221	9.83391e-11\\
3.34432783608196	1.02659e-10\\
3.34632683658171	1.0823e-10\\
3.34832583708146	1.06184e-10\\
3.35032483758121	1.13346e-10\\
3.35232383808096	1.15733e-10\\
3.35432283858071	1.21759e-10\\
3.35632183908046	1.21418e-10\\
3.35832083958021	1.23009e-10\\
3.36031984007996	1.15961e-10\\
3.36231884057971	1.11754e-10\\
3.36431784107946	1.1778e-10\\
3.36631684157921	1.19599e-10\\
3.36831584207896	1.22782e-10\\
3.37031484257871	1.29603e-10\\
3.37231384307846	1.23805e-10\\
3.37431284357821	1.28694e-10\\
3.37631184407796	1.25397e-10\\
3.37831084457771	1.20281e-10\\
3.38030984507746	1.25965e-10\\
3.38230884557721	1.33468e-10\\
3.38430784607696	1.35742e-10\\
3.38630684657671	1.329e-10\\
3.38830584707646	1.25283e-10\\
3.39030484757621	1.3199e-10\\
3.39230384807596	1.39494e-10\\
3.39430284857571	1.41199e-10\\
3.39630184907546	1.48361e-10\\
3.39830084957521	1.42109e-10\\
3.40029985007496	1.38584e-10\\
3.40229885057471	1.45405e-10\\
3.40429785107446	1.41654e-10\\
3.40629685157421	1.42109e-10\\
3.40829585207396	1.36083e-10\\
3.41029485257371	1.35515e-10\\
3.41229385307346	1.3199e-10\\
3.41429285357321	1.39835e-10\\
3.41629185407296	1.44041e-10\\
3.41829085457271	1.47793e-10\\
3.42028985507246	1.40062e-10\\
3.42228885557221	1.39607e-10\\
3.42428785607196	1.41313e-10\\
3.42628685657171	1.39039e-10\\
3.42828585707146	1.39266e-10\\
3.43028485757121	1.40744e-10\\
3.43228385807096	1.44723e-10\\
3.43428285857071	1.48589e-10\\
3.43628185907046	1.55296e-10\\
3.43828085957021	1.55751e-10\\
3.44027986006996	1.57456e-10\\
3.44227886056971	1.60753e-10\\
3.44427786106947	1.53818e-10\\
3.44627686156922	1.51431e-10\\
3.44827586206897	1.437e-10\\
3.45027486256872	1.42222e-10\\
3.45227386306847	1.40858e-10\\
3.45427286356822	1.37106e-10\\
3.45627186406797	1.33127e-10\\
3.45827086456772	1.31081e-10\\
3.46026986506747	1.33582e-10\\
3.46226886556722	1.29376e-10\\
3.46426786606697	1.25965e-10\\
3.46626686656672	1.33468e-10\\
3.46826586706647	1.36993e-10\\
3.47026486756622	1.36311e-10\\
3.47226386806597	1.36197e-10\\
3.47426286856572	1.33127e-10\\
3.47626186906547	1.29489e-10\\
3.47826086956522	1.37106e-10\\
3.48025987006497	1.36197e-10\\
3.48225887056472	1.329e-10\\
3.48425787106447	1.3199e-10\\
3.48625687156422	1.38584e-10\\
3.48825587206397	1.39835e-10\\
3.49025487256372	1.32331e-10\\
3.49225387306347	1.25624e-10\\
3.49425287356322	1.18234e-10\\
3.49625187406297	1.24828e-10\\
3.49825087456272	1.29148e-10\\
3.50024987506247	1.36424e-10\\
3.50224887556222	1.37334e-10\\
3.50424787606197	1.37106e-10\\
3.50624687656172	1.29603e-10\\
3.50824587706147	1.24601e-10\\
3.51024487756122	1.3074e-10\\
3.51224387806097	1.34605e-10\\
3.51424287856072	1.40972e-10\\
3.51624187906047	1.35969e-10\\
3.51824087956022	1.43018e-10\\
3.52023988005997	1.37334e-10\\
3.52223888055972	1.35287e-10\\
3.52423788105947	1.40062e-10\\
3.52623688155922	1.44382e-10\\
3.52823588205897	1.3938e-10\\
3.53023488255872	1.39607e-10\\
3.53223388305847	1.37561e-10\\
3.53423288355822	1.35969e-10\\
3.53623188405797	1.42563e-10\\
3.53823088455772	1.35742e-10\\
3.54022988505747	1.35515e-10\\
3.54222888555722	1.35287e-10\\
3.54422788605697	1.42109e-10\\
3.54622688655672	1.41426e-10\\
3.54822588705647	1.41881e-10\\
3.55022488755622	1.42791e-10\\
3.55222388805597	1.40972e-10\\
3.55422288855572	1.39153e-10\\
3.55622188905547	1.46201e-10\\
3.55822088955522	1.51203e-10\\
3.56021989005497	1.55524e-10\\
3.56221889055472	1.61663e-10\\
3.56421789105447	1.63709e-10\\
3.56621689155422	1.70076e-10\\
3.56821589205397	1.69848e-10\\
3.57021489255372	1.66892e-10\\
3.57221389305347	1.72122e-10\\
3.57421289355322	1.72122e-10\\
3.57621189405297	1.68257e-10\\
3.57821089455272	1.65073e-10\\
3.58020989505247	1.64391e-10\\
3.58220889555222	1.63936e-10\\
3.58420789605197	1.55978e-10\\
3.58620689655172	1.5757e-10\\
3.58820589705147	1.5757e-10\\
3.59020489755122	1.61208e-10\\
3.59220389805097	1.60526e-10\\
3.59420289855072	1.64164e-10\\
3.59620189905047	1.60298e-10\\
3.59820089955022	1.55069e-10\\
3.60019990004997	1.52568e-10\\
3.60219890054973	1.55751e-10\\
3.60419790104948	1.62117e-10\\
3.60619690154923	1.59162e-10\\
3.60819590204898	1.52795e-10\\
3.61019490254873	1.53477e-10\\
3.61219390304848	1.5757e-10\\
3.61419290354823	1.59389e-10\\
3.61619190404798	1.64619e-10\\
3.61819090454773	1.57797e-10\\
3.62018990504748	1.64164e-10\\
3.62218890554723	1.66665e-10\\
3.62418790604698	1.72577e-10\\
3.62618690654673	1.7485e-10\\
3.62818590704648	1.7053e-10\\
3.63018490754623	1.68029e-10\\
3.63218390804598	1.62572e-10\\
3.63418290854573	1.62572e-10\\
3.63618190904548	1.57343e-10\\
3.63818090954523	1.53022e-10\\
3.64017991004498	1.50067e-10\\
3.64217891054473	1.47793e-10\\
3.64417791104448	1.5234e-10\\
3.64617691154423	1.50976e-10\\
3.64817591204398	1.55978e-10\\
3.65017491254373	1.54387e-10\\
3.65217391304348	1.60071e-10\\
3.65417291354323	1.56888e-10\\
3.65617191404298	1.60298e-10\\
3.65817091454273	1.628e-10\\
3.66016991504248	1.63936e-10\\
3.66216891554223	1.60298e-10\\
3.66416791604198	1.65073e-10\\
3.66616691654173	1.58025e-10\\
3.66816591704148	1.55978e-10\\
3.67016491754123	1.55751e-10\\
3.67216391804098	1.53477e-10\\
3.67416291854073	1.58025e-10\\
3.67616191904048	1.64391e-10\\
3.67816091954023	1.60981e-10\\
3.68015992003998	1.68029e-10\\
3.68215892053973	1.66892e-10\\
3.68415792103948	1.71212e-10\\
3.68615692153923	1.73713e-10\\
3.68815592203898	1.77124e-10\\
3.69015492253873	1.76442e-10\\
3.69215392303848	1.78488e-10\\
3.69415292353823	1.78034e-10\\
3.69615192403798	1.82126e-10\\
3.69815092453773	1.81899e-10\\
3.70014992503748	1.84855e-10\\
3.70214892553723	1.77806e-10\\
3.70414792603698	1.73259e-10\\
3.70614692653673	1.66665e-10\\
3.70814592703648	1.59844e-10\\
3.71014492753623	1.62572e-10\\
3.71214392803598	1.64391e-10\\
3.71414292853573	1.63709e-10\\
3.71614192903548	1.6621e-10\\
3.71814092953523	1.62345e-10\\
3.72013993003498	1.69621e-10\\
3.72213893053473	1.74396e-10\\
3.72413793103448	1.68257e-10\\
3.72613693153423	1.62345e-10\\
3.72813593203398	1.56888e-10\\
3.73013493253373	1.50294e-10\\
3.73213393303348	1.45519e-10\\
3.73413293353323	1.4029e-10\\
3.73613193403298	1.37788e-10\\
3.73813093453273	1.42791e-10\\
3.74012993503248	1.49157e-10\\
3.74212893553223	1.55524e-10\\
3.74412793603198	1.53932e-10\\
3.74612693653173	1.4893e-10\\
3.74812593703148	1.48475e-10\\
3.75012493753123	1.46201e-10\\
3.75212393803098	1.46429e-10\\
3.75412293853073	1.41426e-10\\
3.75612193903048	1.43928e-10\\
3.75812093953023	1.43245e-10\\
3.76011994002998	1.44155e-10\\
3.76211894052974	1.42563e-10\\
3.76411794102949	1.34605e-10\\
3.76611694152924	1.41654e-10\\
3.76811594202899	1.37106e-10\\
3.77011494252874	1.4461e-10\\
3.77211394302849	1.44382e-10\\
3.77411294352824	1.45519e-10\\
3.77611194402799	1.41199e-10\\
3.77811094452774	1.3506e-10\\
3.78010994502749	1.29148e-10\\
3.78210894552724	1.26647e-10\\
3.78410794602699	1.26875e-10\\
3.78610694652674	1.19826e-10\\
3.78810594702649	1.2642e-10\\
3.79010494752624	1.31422e-10\\
3.79210394802599	1.33923e-10\\
3.79410294852574	1.31877e-10\\
3.79610194902549	1.31422e-10\\
3.79810094952524	1.31877e-10\\
3.80009995002499	1.29603e-10\\
3.80209895052474	1.26875e-10\\
3.80409795102449	1.27102e-10\\
3.80609695152424	1.25056e-10\\
3.80809595202399	1.31422e-10\\
3.81009495252374	1.24601e-10\\
3.81209395302349	1.24146e-10\\
3.81409295352324	1.29603e-10\\
3.81609195402299	1.23464e-10\\
3.81809095452274	1.2642e-10\\
3.82008995502249	1.28466e-10\\
3.82208895552224	1.35969e-10\\
3.82408795602199	1.28466e-10\\
3.82608695652174	1.28921e-10\\
3.82808595702149	1.25056e-10\\
3.83008495752124	1.19144e-10\\
3.83208395802099	1.18234e-10\\
3.83408295852074	1.20281e-10\\
3.83608195902049	1.17552e-10\\
3.83808095952024	1.19144e-10\\
3.84007996001999	1.22554e-10\\
3.84207896051974	1.19371e-10\\
3.84407796101949	1.19371e-10\\
3.84607696151924	1.2551e-10\\
3.84807596201899	1.22782e-10\\
3.85007496251874	1.24373e-10\\
3.85207396301849	1.28466e-10\\
3.85407296351824	1.27329e-10\\
3.85607196401799	1.21645e-10\\
3.85807096451774	1.27784e-10\\
3.86006996501749	1.24146e-10\\
3.86206896551724	1.28239e-10\\
3.86406796601699	1.25965e-10\\
3.86606696651674	1.27784e-10\\
3.86806596701649	1.20281e-10\\
3.87006496751624	1.20281e-10\\
3.87206396801599	1.18689e-10\\
3.87406296851574	1.23919e-10\\
3.87606196901549	1.20963e-10\\
3.87806096951524	1.21645e-10\\
3.88005997001499	1.24828e-10\\
3.88205897051474	1.22782e-10\\
3.88405797101449	1.26192e-10\\
3.88605697151424	1.23919e-10\\
3.88805597201399	1.27329e-10\\
3.89005497251374	1.31195e-10\\
3.89205397301349	1.28239e-10\\
3.89405297351324	1.31649e-10\\
3.89605197401299	1.28694e-10\\
3.89805097451274	1.21418e-10\\
3.90004997501249	1.23691e-10\\
3.90204897551224	1.27557e-10\\
3.90404797601199	1.32786e-10\\
3.90604697651174	1.38471e-10\\
3.90804597701149	1.31649e-10\\
3.91004497751124	1.28239e-10\\
3.91204397801099	1.22327e-10\\
3.91404297851074	1.19144e-10\\
3.91604197901049	1.17325e-10\\
3.91804097951024	1.21418e-10\\
3.92003998000999	1.15506e-10\\
3.92203898050975	1.07548e-10\\
3.9240379810095	1.04365e-10\\
3.92603698150925	9.82254e-11\\
3.928035982009	9.25411e-11\\
3.93003498250875	8.89031e-11\\
3.9320339830085	9.41327e-11\\
3.93403298350825	9.45874e-11\\
3.936031984008	9.61791e-11\\
3.93803098450775	1.00044e-10\\
3.9400299850075	1.01181e-10\\
3.94202898550725	1.0391e-10\\
3.944027986007	1.05729e-10\\
3.94602698650675	1.08912e-10\\
3.9480259870065	1.14596e-10\\
3.95002498750625	1.1255e-10\\
3.952023988006	1.04365e-10\\
3.95402298850575	1.12323e-10\\
3.9560219890055	1.18462e-10\\
3.95802098950525	1.13232e-10\\
3.960019990005	1.20735e-10\\
3.96201899050475	1.2119e-10\\
3.9640179910045	1.28011e-10\\
3.96601699150425	1.34605e-10\\
3.968015992004	1.40062e-10\\
3.97001499250375	1.38243e-10\\
3.9720139930035	1.30967e-10\\
3.97401299350325	1.35287e-10\\
3.976011994003	1.39607e-10\\
3.97801099450275	1.38471e-10\\
3.9800099950025	1.33241e-10\\
3.98200899550225	1.34378e-10\\
3.984007996002	1.40744e-10\\
3.98600699650175	1.33468e-10\\
3.9880059970015	1.40972e-10\\
3.99000499750125	1.33696e-10\\
3.992003998001	1.37788e-10\\
3.99400299850075	1.35515e-10\\
3.9960019990005	1.38016e-10\\
3.99800099950025	1.34378e-10\\
4	1.37334e-10\\
};
\addlegendentry{c1};

\addplot [color=mycolor2,solid]
  table[row sep=crcr]{%
0	5.00222e-11\\
0.00199900049975012	5.00222e-11\\
0.00399800099950025	5.00222e-11\\
0.00599700149925037	5.00222e-11\\
0.0079960019990005	5.00222e-11\\
0.00999500249875063	4.91127e-11\\
0.0119940029985007	4.82032e-11\\
0.0139930034982509	4.82032e-11\\
0.015992003998001	4.82032e-11\\
0.0179910044977511	4.82032e-11\\
0.0199900049975013	4.91127e-11\\
0.0219890054972514	4.91127e-11\\
0.0239880059970015	4.91127e-11\\
0.0259870064967516	5.00222e-11\\
0.0279860069965017	5.00222e-11\\
0.0299850074962519	5.00222e-11\\
0.031984007996002	5.09317e-11\\
0.0339830084957521	5.00222e-11\\
0.0359820089955022	5.09317e-11\\
0.0379810094952524	5.09317e-11\\
0.0399800099950025	5.09317e-11\\
0.0419790104947526	5.09317e-11\\
0.0439780109945027	5.18412e-11\\
0.0459770114942529	5.09317e-11\\
0.047976011994003	5.09317e-11\\
0.0499750124937531	5.00222e-11\\
0.0519740129935032	5.00222e-11\\
0.0539730134932534	5.00222e-11\\
0.0559720139930035	5.00222e-11\\
0.0579710144927536	5.00222e-11\\
0.0599700149925037	5.00222e-11\\
0.0619690154922539	5.00222e-11\\
0.063968015992004	5.00222e-11\\
0.0659670164917541	4.91127e-11\\
0.0679660169915042	5.00222e-11\\
0.0699650174912544	5.00222e-11\\
0.0719640179910045	5.00222e-11\\
0.0739630184907546	4.91127e-11\\
0.0759620189905048	4.91127e-11\\
0.0779610194902549	4.82032e-11\\
0.079960019990005	4.91127e-11\\
0.0819590204897551	4.91127e-11\\
0.0839580209895052	5.00222e-11\\
0.0859570214892554	5.00222e-11\\
0.0879560219890055	5.00222e-11\\
0.0899550224887556	5.00222e-11\\
0.0919540229885057	5.00222e-11\\
0.0939530234882559	5.09317e-11\\
0.095952023988006	5.09317e-11\\
0.0979510244877561	5.00222e-11\\
0.0999500249875062	5.09317e-11\\
0.101949025487256	5.00222e-11\\
0.103948025987006	5.00222e-11\\
0.105947026486757	5.00222e-11\\
0.107946026986507	5.00222e-11\\
0.109945027486257	5.00222e-11\\
0.111944027986007	5.00222e-11\\
0.113943028485757	5.00222e-11\\
0.115942028985507	5.00222e-11\\
0.117941029485257	5.00222e-11\\
0.119940029985007	5.00222e-11\\
0.121939030484758	5.00222e-11\\
0.123938030984508	5.00222e-11\\
0.125937031484258	5.09317e-11\\
0.127936031984008	5.00222e-11\\
0.129935032483758	5.00222e-11\\
0.131934032983508	5.00222e-11\\
0.133933033483258	5.00222e-11\\
0.135932033983008	4.91127e-11\\
0.137931034482759	4.91127e-11\\
0.139930034982509	4.91127e-11\\
0.141929035482259	4.91127e-11\\
0.143928035982009	4.91127e-11\\
0.145927036481759	5.00222e-11\\
0.147926036981509	5.00222e-11\\
0.149925037481259	5.00222e-11\\
0.15192403798101	5.00222e-11\\
0.15392303848076	5.00222e-11\\
0.15592203898051	5.00222e-11\\
0.15792103948026	5.00222e-11\\
0.15992003998001	5.00222e-11\\
0.16191904047976	5.09317e-11\\
0.16391804097951	5.00222e-11\\
0.16591704147926	5.09317e-11\\
0.16791604197901	5.09317e-11\\
0.169915042478761	5.09317e-11\\
0.171914042978511	5.09317e-11\\
0.173913043478261	5.09317e-11\\
0.175912043978011	5.09317e-11\\
0.177911044477761	5.09317e-11\\
0.179910044977511	5.18412e-11\\
0.181909045477261	5.18412e-11\\
0.183908045977011	5.18412e-11\\
0.185907046476762	5.09317e-11\\
0.187906046976512	5.00222e-11\\
0.189905047476262	5.00222e-11\\
0.191904047976012	5.00222e-11\\
0.193903048475762	5.09317e-11\\
0.195902048975512	5.09317e-11\\
0.197901049475262	5.18412e-11\\
0.199900049975012	5.18412e-11\\
0.201899050474763	5.27507e-11\\
0.203898050974513	5.27507e-11\\
0.205897051474263	5.27507e-11\\
0.207896051974013	5.27507e-11\\
0.209895052473763	5.27507e-11\\
0.211894052973513	5.27507e-11\\
0.213893053473263	5.27507e-11\\
0.215892053973014	5.27507e-11\\
0.217891054472764	5.27507e-11\\
0.219890054972514	5.27507e-11\\
0.221889055472264	5.27507e-11\\
0.223888055972014	5.18412e-11\\
0.225887056471764	5.09317e-11\\
0.227886056971514	5.09317e-11\\
0.229885057471264	5.00222e-11\\
0.231884057971014	5.00222e-11\\
0.233883058470765	5.00222e-11\\
0.235882058970515	5.00222e-11\\
0.237881059470265	5.00222e-11\\
0.239880059970015	5.00222e-11\\
0.241879060469765	5.00222e-11\\
0.243878060969515	5.00222e-11\\
0.245877061469265	5.00222e-11\\
0.247876061969015	5.00222e-11\\
0.249875062468766	4.91127e-11\\
0.251874062968516	4.91127e-11\\
0.253873063468266	4.91127e-11\\
0.255872063968016	4.82032e-11\\
0.257871064467766	4.82032e-11\\
0.259870064967516	4.91127e-11\\
0.261869065467266	4.91127e-11\\
0.263868065967017	4.91127e-11\\
0.265867066466767	4.82032e-11\\
0.267866066966517	4.91127e-11\\
0.269865067466267	4.91127e-11\\
0.271864067966017	4.91127e-11\\
0.273863068465767	4.91127e-11\\
0.275862068965517	4.91127e-11\\
0.277861069465267	4.91127e-11\\
0.279860069965017	4.82032e-11\\
0.281859070464768	4.91127e-11\\
0.283858070964518	4.91127e-11\\
0.285857071464268	4.91127e-11\\
0.287856071964018	4.91127e-11\\
0.289855072463768	5.00222e-11\\
0.291854072963518	5.00222e-11\\
0.293853073463268	4.91127e-11\\
0.295852073963018	4.91127e-11\\
0.297851074462769	4.91127e-11\\
0.299850074962519	4.82032e-11\\
0.301849075462269	4.82032e-11\\
0.303848075962019	4.72937e-11\\
0.305847076461769	4.82032e-11\\
0.307846076961519	4.82032e-11\\
0.309845077461269	4.82032e-11\\
0.311844077961019	4.82032e-11\\
0.31384307846077	4.82032e-11\\
0.31584207896052	4.82032e-11\\
0.31784107946027	4.82032e-11\\
0.31984007996002	4.82032e-11\\
0.32183908045977	4.82032e-11\\
0.32383808095952	4.82032e-11\\
0.32583708145927	4.82032e-11\\
0.32783608195902	4.91127e-11\\
0.329835082458771	4.91127e-11\\
0.331834082958521	4.91127e-11\\
0.333833083458271	4.82032e-11\\
0.335832083958021	4.82032e-11\\
0.337831084457771	4.82032e-11\\
0.339830084957521	4.82032e-11\\
0.341829085457271	4.82032e-11\\
0.343828085957021	4.72937e-11\\
0.345827086456772	4.82032e-11\\
0.347826086956522	4.82032e-11\\
0.349825087456272	4.82032e-11\\
0.351824087956022	4.82032e-11\\
0.353823088455772	4.82032e-11\\
0.355822088955522	4.82032e-11\\
0.357821089455272	4.82032e-11\\
0.359820089955023	4.82032e-11\\
0.361819090454773	4.72937e-11\\
0.363818090954523	4.72937e-11\\
0.365817091454273	4.72937e-11\\
0.367816091954023	4.72937e-11\\
0.369815092453773	4.72937e-11\\
0.371814092953523	4.82032e-11\\
0.373813093453273	4.82032e-11\\
0.375812093953024	4.82032e-11\\
0.377811094452774	4.91127e-11\\
0.379810094952524	4.91127e-11\\
0.381809095452274	4.82032e-11\\
0.383808095952024	4.82032e-11\\
0.385807096451774	4.82032e-11\\
0.387806096951524	4.82032e-11\\
0.389805097451274	4.82032e-11\\
0.391804097951024	4.82032e-11\\
0.393803098450775	4.82032e-11\\
0.395802098950525	4.91127e-11\\
0.397801099450275	4.91127e-11\\
0.399800099950025	4.91127e-11\\
0.401799100449775	4.91127e-11\\
0.403798100949525	4.91127e-11\\
0.405797101449275	4.82032e-11\\
0.407796101949025	4.82032e-11\\
0.409795102448776	4.82032e-11\\
0.411794102948526	4.82032e-11\\
0.413793103448276	4.82032e-11\\
0.415792103948026	4.82032e-11\\
0.417791104447776	4.82032e-11\\
0.419790104947526	4.82032e-11\\
0.421789105447276	4.91127e-11\\
0.423788105947026	4.82032e-11\\
0.425787106446777	4.82032e-11\\
0.427786106946527	4.91127e-11\\
0.429785107446277	4.91127e-11\\
0.431784107946027	5.00222e-11\\
0.433783108445777	5.00222e-11\\
0.435782108945527	5.00222e-11\\
0.437781109445277	5.00222e-11\\
0.439780109945027	5.00222e-11\\
0.441779110444778	5.00222e-11\\
0.443778110944528	5.00222e-11\\
0.445777111444278	5.00222e-11\\
0.447776111944028	5.00222e-11\\
0.449775112443778	4.91127e-11\\
0.451774112943528	4.91127e-11\\
0.453773113443278	4.91127e-11\\
0.455772113943028	4.82032e-11\\
0.457771114442779	4.82032e-11\\
0.459770114942529	4.82032e-11\\
0.461769115442279	4.82032e-11\\
0.463768115942029	4.82032e-11\\
0.465767116441779	4.82032e-11\\
0.467766116941529	4.72937e-11\\
0.469765117441279	4.72937e-11\\
0.471764117941029	4.72937e-11\\
0.47376311844078	4.82032e-11\\
0.47576211894053	4.82032e-11\\
0.47776111944028	4.82032e-11\\
0.47976011994003	4.82032e-11\\
0.48175912043978	4.82032e-11\\
0.48375812093953	4.82032e-11\\
0.48575712143928	4.82032e-11\\
0.487756121939031	4.82032e-11\\
0.489755122438781	4.82032e-11\\
0.491754122938531	4.82032e-11\\
0.493753123438281	4.82032e-11\\
0.495752123938031	4.72937e-11\\
0.497751124437781	4.72937e-11\\
0.499750124937531	4.82032e-11\\
0.501749125437281	4.72937e-11\\
0.503748125937031	4.82032e-11\\
0.505747126436782	4.72937e-11\\
0.507746126936532	4.82032e-11\\
0.509745127436282	4.72937e-11\\
0.511744127936032	4.72937e-11\\
0.513743128435782	4.72937e-11\\
0.515742128935532	4.72937e-11\\
0.517741129435282	4.72937e-11\\
0.519740129935032	4.82032e-11\\
0.521739130434783	4.72937e-11\\
0.523738130934533	4.72937e-11\\
0.525737131434283	4.72937e-11\\
0.527736131934033	4.72937e-11\\
0.529735132433783	4.63842e-11\\
0.531734132933533	4.72937e-11\\
0.533733133433283	4.72937e-11\\
0.535732133933034	4.72937e-11\\
0.537731134432784	4.72937e-11\\
0.539730134932534	4.72937e-11\\
0.541729135432284	4.72937e-11\\
0.543728135932034	4.72937e-11\\
0.545727136431784	4.72937e-11\\
0.547726136931534	4.72937e-11\\
0.549725137431284	4.82032e-11\\
0.551724137931034	4.72937e-11\\
0.553723138430785	4.72937e-11\\
0.555722138930535	4.72937e-11\\
0.557721139430285	4.72937e-11\\
0.559720139930035	4.72937e-11\\
0.561719140429785	4.72937e-11\\
0.563718140929535	4.72937e-11\\
0.565717141429285	4.72937e-11\\
0.567716141929036	4.72937e-11\\
0.569715142428786	4.72937e-11\\
0.571714142928536	4.72937e-11\\
0.573713143428286	4.72937e-11\\
0.575712143928036	4.72937e-11\\
0.577711144427786	4.72937e-11\\
0.579710144927536	4.72937e-11\\
0.581709145427286	4.63842e-11\\
0.583708145927036	4.63842e-11\\
0.585707146426787	4.63842e-11\\
0.587706146926537	4.63842e-11\\
0.589705147426287	4.63842e-11\\
0.591704147926037	4.54747e-11\\
0.593703148425787	4.54747e-11\\
0.595702148925537	4.54747e-11\\
0.597701149425287	4.45652e-11\\
0.599700149925038	4.45652e-11\\
0.601699150424788	4.54747e-11\\
0.603698150924538	4.45652e-11\\
0.605697151424288	4.45652e-11\\
0.607696151924038	4.45652e-11\\
0.609695152423788	4.45652e-11\\
0.611694152923538	4.45652e-11\\
0.613693153423288	4.45652e-11\\
0.615692153923038	4.45652e-11\\
0.617691154422789	4.45652e-11\\
0.619690154922539	4.45652e-11\\
0.621689155422289	4.45652e-11\\
0.623688155922039	4.36557e-11\\
0.625687156421789	4.36557e-11\\
0.627686156921539	4.36557e-11\\
0.629685157421289	4.36557e-11\\
0.631684157921039	4.36557e-11\\
0.63368315842079	4.36557e-11\\
0.63568215892054	4.45652e-11\\
0.63768115942029	4.45652e-11\\
0.63968015992004	4.45652e-11\\
0.64167916041979	4.45652e-11\\
0.64367816091954	4.45652e-11\\
0.64567716141929	4.36557e-11\\
0.64767616191904	4.36557e-11\\
0.649675162418791	4.36557e-11\\
0.651674162918541	4.36557e-11\\
0.653673163418291	4.36557e-11\\
0.655672163918041	4.36557e-11\\
0.657671164417791	4.36557e-11\\
0.659670164917541	4.36557e-11\\
0.661669165417291	4.36557e-11\\
0.663668165917041	4.36557e-11\\
0.665667166416792	4.36557e-11\\
0.667666166916542	4.36557e-11\\
0.669665167416292	4.36557e-11\\
0.671664167916042	4.36557e-11\\
0.673663168415792	4.36557e-11\\
0.675662168915542	4.36557e-11\\
0.677661169415292	4.36557e-11\\
0.679660169915043	4.36557e-11\\
0.681659170414793	4.36557e-11\\
0.683658170914543	4.27463e-11\\
0.685657171414293	4.36557e-11\\
0.687656171914043	4.36557e-11\\
0.689655172413793	4.36557e-11\\
0.691654172913543	4.36557e-11\\
0.693653173413293	4.45652e-11\\
0.695652173913043	4.45652e-11\\
0.697651174412794	4.54747e-11\\
0.699650174912544	4.54747e-11\\
0.701649175412294	4.54747e-11\\
0.703648175912044	4.45652e-11\\
0.705647176411794	4.45652e-11\\
0.707646176911544	4.45652e-11\\
0.709645177411294	4.45652e-11\\
0.711644177911045	4.45652e-11\\
0.713643178410795	4.54747e-11\\
0.715642178910545	4.54747e-11\\
0.717641179410295	4.54747e-11\\
0.719640179910045	4.45652e-11\\
0.721639180409795	4.45652e-11\\
0.723638180909545	4.45652e-11\\
0.725637181409295	4.45652e-11\\
0.727636181909045	4.45652e-11\\
0.729635182408796	4.45652e-11\\
0.731634182908546	4.45652e-11\\
0.733633183408296	4.45652e-11\\
0.735632183908046	4.45652e-11\\
0.737631184407796	4.45652e-11\\
0.739630184907546	4.45652e-11\\
0.741629185407296	4.45652e-11\\
0.743628185907046	4.45652e-11\\
0.745627186406797	4.45652e-11\\
0.747626186906547	4.45652e-11\\
0.749625187406297	4.45652e-11\\
0.751624187906047	4.54747e-11\\
0.753623188405797	4.45652e-11\\
0.755622188905547	4.54747e-11\\
0.757621189405297	4.54747e-11\\
0.759620189905047	4.54747e-11\\
0.761619190404798	4.54747e-11\\
0.763618190904548	4.54747e-11\\
0.765617191404298	4.54747e-11\\
0.767616191904048	4.54747e-11\\
0.769615192403798	4.54747e-11\\
0.771614192903548	4.54747e-11\\
0.773613193403298	4.54747e-11\\
0.775612193903048	4.54747e-11\\
0.777611194402799	4.54747e-11\\
0.779610194902549	4.45652e-11\\
0.781609195402299	4.45652e-11\\
0.783608195902049	4.45652e-11\\
0.785607196401799	4.45652e-11\\
0.787606196901549	4.45652e-11\\
0.789605197401299	4.45652e-11\\
0.79160419790105	4.45652e-11\\
0.7936031984008	4.45652e-11\\
0.79560219890055	4.45652e-11\\
0.7976011994003	4.45652e-11\\
0.79960019990005	4.54747e-11\\
0.8015992003998	4.45652e-11\\
0.80359820089955	4.45652e-11\\
0.8055972013993	4.45652e-11\\
0.80759620189905	4.45652e-11\\
0.809595202398801	4.45652e-11\\
0.811594202898551	4.45652e-11\\
0.813593203398301	4.45652e-11\\
0.815592203898051	4.45652e-11\\
0.817591204397801	4.45652e-11\\
0.819590204897551	4.45652e-11\\
0.821589205397301	4.45652e-11\\
0.823588205897052	4.45652e-11\\
0.825587206396802	4.45652e-11\\
0.827586206896552	4.45652e-11\\
0.829585207396302	4.45652e-11\\
0.831584207896052	4.45652e-11\\
0.833583208395802	4.45652e-11\\
0.835582208895552	4.45652e-11\\
0.837581209395302	4.45652e-11\\
0.839580209895052	4.45652e-11\\
0.841579210394803	4.36557e-11\\
0.843578210894553	4.36557e-11\\
0.845577211394303	4.36557e-11\\
0.847576211894053	4.36557e-11\\
0.849575212393803	4.36557e-11\\
0.851574212893553	4.45652e-11\\
0.853573213393303	4.36557e-11\\
0.855572213893053	4.45652e-11\\
0.857571214392804	4.45652e-11\\
0.859570214892554	4.36557e-11\\
0.861569215392304	4.36557e-11\\
0.863568215892054	4.36557e-11\\
0.865567216391804	4.36557e-11\\
0.867566216891554	4.36557e-11\\
0.869565217391304	4.27463e-11\\
0.871564217891054	4.27463e-11\\
0.873563218390805	4.27463e-11\\
0.875562218890555	4.27463e-11\\
0.877561219390305	4.27463e-11\\
0.879560219890055	4.27463e-11\\
0.881559220389805	4.27463e-11\\
0.883558220889555	4.27463e-11\\
0.885557221389305	4.27463e-11\\
0.887556221889055	4.27463e-11\\
0.889555222388806	4.27463e-11\\
0.891554222888556	4.27463e-11\\
0.893553223388306	4.27463e-11\\
0.895552223888056	4.27463e-11\\
0.897551224387806	4.27463e-11\\
0.899550224887556	4.18368e-11\\
0.901549225387306	4.18368e-11\\
0.903548225887057	4.18368e-11\\
0.905547226386807	4.18368e-11\\
0.907546226886557	4.27463e-11\\
0.909545227386307	4.27463e-11\\
0.911544227886057	4.27463e-11\\
0.913543228385807	4.27463e-11\\
0.915542228885557	4.27463e-11\\
0.917541229385307	4.27463e-11\\
0.919540229885057	4.18368e-11\\
0.921539230384808	4.18368e-11\\
0.923538230884558	4.27463e-11\\
0.925537231384308	4.18368e-11\\
0.927536231884058	4.18368e-11\\
0.929535232383808	4.18368e-11\\
0.931534232883558	4.18368e-11\\
0.933533233383308	4.27463e-11\\
0.935532233883059	4.18368e-11\\
0.937531234382809	4.18368e-11\\
0.939530234882559	4.18368e-11\\
0.941529235382309	4.18368e-11\\
0.943528235882059	4.18368e-11\\
0.945527236381809	4.18368e-11\\
0.947526236881559	4.18368e-11\\
0.949525237381309	4.18368e-11\\
0.951524237881059	4.27463e-11\\
0.95352323838081	4.27463e-11\\
0.95552223888056	4.27463e-11\\
0.95752123938031	4.27463e-11\\
0.95952023988006	4.27463e-11\\
0.96151924037981	4.36557e-11\\
0.96351824087956	4.36557e-11\\
0.96551724137931	4.36557e-11\\
0.96751624187906	4.36557e-11\\
0.969515242378811	4.36557e-11\\
0.971514242878561	4.36557e-11\\
0.973513243378311	4.36557e-11\\
0.975512243878061	4.45652e-11\\
0.977511244377811	4.36557e-11\\
0.979510244877561	4.45652e-11\\
0.981509245377311	4.36557e-11\\
0.983508245877061	4.45652e-11\\
0.985507246376812	4.36557e-11\\
0.987506246876562	4.45652e-11\\
0.989505247376312	4.45652e-11\\
0.991504247876062	4.45652e-11\\
0.993503248375812	4.45652e-11\\
0.995502248875562	4.45652e-11\\
0.997501249375312	4.54747e-11\\
0.999500249875062	4.54747e-11\\
1.00149925037481	4.54747e-11\\
1.00349825087456	4.45652e-11\\
1.00549725137431	4.54747e-11\\
1.00749625187406	4.54747e-11\\
1.00949525237381	4.54747e-11\\
1.01149425287356	4.54747e-11\\
1.01349325337331	4.45652e-11\\
1.01549225387306	4.45652e-11\\
1.01749125437281	4.45652e-11\\
1.01949025487256	4.45652e-11\\
1.02148925537231	4.45652e-11\\
1.02348825587206	4.54747e-11\\
1.02548725637181	4.54747e-11\\
1.02748625687156	4.54747e-11\\
1.02948525737131	4.54747e-11\\
1.03148425787106	4.63842e-11\\
1.03348325837081	4.63842e-11\\
1.03548225887056	4.63842e-11\\
1.03748125937031	4.54747e-11\\
1.03948025987006	4.45652e-11\\
1.04147926036982	4.45652e-11\\
1.04347826086957	4.45652e-11\\
1.04547726136932	4.54747e-11\\
1.04747626186907	4.45652e-11\\
1.04947526236882	4.54747e-11\\
1.05147426286857	4.54747e-11\\
1.05347326336832	4.54747e-11\\
1.05547226386807	4.45652e-11\\
1.05747126436782	4.45652e-11\\
1.05947026486757	4.54747e-11\\
1.06146926536732	4.54747e-11\\
1.06346826586707	4.63842e-11\\
1.06546726636682	4.63842e-11\\
1.06746626686657	4.54747e-11\\
1.06946526736632	4.54747e-11\\
1.07146426786607	4.54747e-11\\
1.07346326836582	4.54747e-11\\
1.07546226886557	4.54747e-11\\
1.07746126936532	4.45652e-11\\
1.07946026986507	4.54747e-11\\
1.08145927036482	4.54747e-11\\
1.08345827086457	4.54747e-11\\
1.08545727136432	4.45652e-11\\
1.08745627186407	4.45652e-11\\
1.08945527236382	4.36557e-11\\
1.09145427286357	4.36557e-11\\
1.09345327336332	4.36557e-11\\
1.09545227386307	4.36557e-11\\
1.09745127436282	4.27463e-11\\
1.09945027486257	4.18368e-11\\
1.10144927536232	4.18368e-11\\
1.10344827586207	4.09273e-11\\
1.10544727636182	4.09273e-11\\
1.10744627686157	4.00178e-11\\
1.10944527736132	4.00178e-11\\
1.11144427786107	3.91083e-11\\
1.11344327836082	3.81988e-11\\
1.11544227886057	3.91083e-11\\
1.11744127936032	3.91083e-11\\
1.11944027986007	3.91083e-11\\
1.12143928035982	3.91083e-11\\
1.12343828085957	3.91083e-11\\
1.12543728135932	3.91083e-11\\
1.12743628185907	3.91083e-11\\
1.12943528235882	3.81988e-11\\
1.13143428285857	3.81988e-11\\
1.13343328335832	3.72893e-11\\
1.13543228385807	3.81988e-11\\
1.13743128435782	3.91083e-11\\
1.13943028485757	3.91083e-11\\
1.14142928535732	4.00178e-11\\
1.14342828585707	3.91083e-11\\
1.14542728635682	3.91083e-11\\
1.14742628685657	4.00178e-11\\
1.14942528735632	4.00178e-11\\
1.15142428785607	3.91083e-11\\
1.15342328835582	4.00178e-11\\
1.15542228885557	4.00178e-11\\
1.15742128935532	4.00178e-11\\
1.15942028985507	4.00178e-11\\
1.16141929035482	4.00178e-11\\
1.16341829085457	4.00178e-11\\
1.16541729135432	4.09273e-11\\
1.16741629185407	4.00178e-11\\
1.16941529235382	4.00178e-11\\
1.17141429285357	4.00178e-11\\
1.17341329335332	4.00178e-11\\
1.17541229385307	3.91083e-11\\
1.17741129435282	4.00178e-11\\
1.17941029485257	4.09273e-11\\
1.18140929535232	4.09273e-11\\
1.18340829585207	4.09273e-11\\
1.18540729635182	4.18368e-11\\
1.18740629685157	4.18368e-11\\
1.18940529735132	4.09273e-11\\
1.19140429785107	4.18368e-11\\
1.19340329835082	4.27463e-11\\
1.19540229885057	4.27463e-11\\
1.19740129935032	4.27463e-11\\
1.19940029985008	4.36557e-11\\
1.20139930034983	4.36557e-11\\
1.20339830084958	4.36557e-11\\
1.20539730134933	4.45652e-11\\
1.20739630184908	4.54747e-11\\
1.20939530234883	4.54747e-11\\
1.21139430284858	4.45652e-11\\
1.21339330334833	4.45652e-11\\
1.21539230384808	4.45652e-11\\
1.21739130434783	4.54747e-11\\
1.21939030484758	4.54747e-11\\
1.22138930534733	4.54747e-11\\
1.22338830584708	4.54747e-11\\
1.22538730634683	4.45652e-11\\
1.22738630684658	4.54747e-11\\
1.22938530734633	4.63842e-11\\
1.23138430784608	4.72937e-11\\
1.23338330834583	4.72937e-11\\
1.23538230884558	4.82032e-11\\
1.23738130934533	4.72937e-11\\
1.23938030984508	4.82032e-11\\
1.24137931034483	4.82032e-11\\
1.24337831084458	4.91127e-11\\
1.24537731134433	5.00222e-11\\
1.24737631184408	5.00222e-11\\
1.24937531234383	4.91127e-11\\
1.25137431284358	4.91127e-11\\
1.25337331334333	5.00222e-11\\
1.25537231384308	5.00222e-11\\
1.25737131434283	4.91127e-11\\
1.25937031484258	5.00222e-11\\
1.26136931534233	5.00222e-11\\
1.26336831584208	5.00222e-11\\
1.26536731634183	5.00222e-11\\
1.26736631684158	5.09317e-11\\
1.26936531734133	5.00222e-11\\
1.27136431784108	5.00222e-11\\
1.27336331834083	5.00222e-11\\
1.27536231884058	5.00222e-11\\
1.27736131934033	5.09317e-11\\
1.27936031984008	5.00222e-11\\
1.28135932033983	5.00222e-11\\
1.28335832083958	5.09317e-11\\
1.28535732133933	5.09317e-11\\
1.28735632183908	5.09317e-11\\
1.28935532233883	5.00222e-11\\
1.29135432283858	5.00222e-11\\
1.29335332333833	4.82032e-11\\
1.29535232383808	4.82032e-11\\
1.29735132433783	4.82032e-11\\
1.29935032483758	4.72937e-11\\
1.30134932533733	4.82032e-11\\
1.30334832583708	4.82032e-11\\
1.30534732633683	4.72937e-11\\
1.30734632683658	4.72937e-11\\
1.30934532733633	4.63842e-11\\
1.31134432783608	4.63842e-11\\
1.31334332833583	4.63842e-11\\
1.31534232883558	4.54747e-11\\
1.31734132933533	4.63842e-11\\
1.31934032983508	4.54747e-11\\
1.32133933033483	4.63842e-11\\
1.32333833083458	4.54747e-11\\
1.32533733133433	4.54747e-11\\
1.32733633183408	4.54747e-11\\
1.32933533233383	4.45652e-11\\
1.33133433283358	4.54747e-11\\
1.33333333333333	4.54747e-11\\
1.33533233383308	4.54747e-11\\
1.33733133433283	4.54747e-11\\
1.33933033483258	4.54747e-11\\
1.34132933533233	4.63842e-11\\
1.34332833583208	4.63842e-11\\
1.34532733633183	4.63842e-11\\
1.34732633683158	4.82032e-11\\
1.34932533733133	4.82032e-11\\
1.35132433783108	4.63842e-11\\
1.35332333833083	4.72937e-11\\
1.35532233883058	4.63842e-11\\
1.35732133933033	4.63842e-11\\
1.35932033983009	4.63842e-11\\
1.36131934032984	4.72937e-11\\
1.36331834082959	4.72937e-11\\
1.36531734132934	4.72937e-11\\
1.36731634182909	4.82032e-11\\
1.36931534232884	4.72937e-11\\
1.37131434282859	4.72937e-11\\
1.37331334332834	4.82032e-11\\
1.37531234382809	4.91127e-11\\
1.37731134432784	4.82032e-11\\
1.37931034482759	4.82032e-11\\
1.38130934532734	4.91127e-11\\
1.38330834582709	4.82032e-11\\
1.38530734632684	4.91127e-11\\
1.38730634682659	4.72937e-11\\
1.38930534732634	4.82032e-11\\
1.39130434782609	4.82032e-11\\
1.39330334832584	4.72937e-11\\
1.39530234882559	4.63842e-11\\
1.39730134932534	4.45652e-11\\
1.39930034982509	4.45652e-11\\
1.40129935032484	4.45652e-11\\
1.40329835082459	4.45652e-11\\
1.40529735132434	4.45652e-11\\
1.40729635182409	4.45652e-11\\
1.40929535232384	4.36557e-11\\
1.41129435282359	4.45652e-11\\
1.41329335332334	4.36557e-11\\
1.41529235382309	4.36557e-11\\
1.41729135432284	4.27463e-11\\
1.41929035482259	4.27463e-11\\
1.42128935532234	4.27463e-11\\
1.42328835582209	4.27463e-11\\
1.42528735632184	4.18368e-11\\
1.42728635682159	4.09273e-11\\
1.42928535732134	4.09273e-11\\
1.43128435782109	4.09273e-11\\
1.43328335832084	4.00178e-11\\
1.43528235882059	3.91083e-11\\
1.43728135932034	4.00178e-11\\
1.43928035982009	4.00178e-11\\
1.44127936031984	4.00178e-11\\
1.44327836081959	4.00178e-11\\
1.44527736131934	4.09273e-11\\
1.44727636181909	4.18368e-11\\
1.44927536231884	4.18368e-11\\
1.45127436281859	4.09273e-11\\
1.45327336331834	4.18368e-11\\
1.45527236381809	4.18368e-11\\
1.45727136431784	4.09273e-11\\
1.45927036481759	4.18368e-11\\
1.46126936531734	4.09273e-11\\
1.46326836581709	4.09273e-11\\
1.46526736631684	4.00178e-11\\
1.46726636681659	4.09273e-11\\
1.46926536731634	4.09273e-11\\
1.47126436781609	4.18368e-11\\
1.47326336831584	4.18368e-11\\
1.47526236881559	4.09273e-11\\
1.47726136931534	4.09273e-11\\
1.47926036981509	4.18368e-11\\
1.48125937031484	4.09273e-11\\
1.48325837081459	4.18368e-11\\
1.48525737131434	4.09273e-11\\
1.48725637181409	4.09273e-11\\
1.48925537231384	4.09273e-11\\
1.49125437281359	4.09273e-11\\
1.49325337331334	4.18368e-11\\
1.49525237381309	4.27463e-11\\
1.49725137431284	4.27463e-11\\
1.49925037481259	4.18368e-11\\
1.50124937531234	4.18368e-11\\
1.50324837581209	4.27463e-11\\
1.50524737631184	4.27463e-11\\
1.50724637681159	4.36557e-11\\
1.50924537731134	4.36557e-11\\
1.51124437781109	4.36557e-11\\
1.51324337831084	4.27463e-11\\
1.51524237881059	4.45652e-11\\
1.51724137931034	4.36557e-11\\
1.51924037981009	4.27463e-11\\
1.52123938030985	4.09273e-11\\
1.5232383808096	4.09273e-11\\
1.52523738130935	4.09273e-11\\
1.5272363818091	4.00178e-11\\
1.52923538230885	3.91083e-11\\
1.5312343828086	3.91083e-11\\
1.53323338330835	3.91083e-11\\
1.5352323838081	3.81988e-11\\
1.53723138430785	3.81988e-11\\
1.5392303848076	3.81988e-11\\
1.54122938530735	3.91083e-11\\
1.5432283858071	3.91083e-11\\
1.54522738630685	3.91083e-11\\
1.5472263868066	3.91083e-11\\
1.54922538730635	3.81988e-11\\
1.5512243878061	3.81988e-11\\
1.55322338830585	3.91083e-11\\
1.5552223888056	3.91083e-11\\
1.55722138930535	3.81988e-11\\
1.5592203898051	3.91083e-11\\
1.56121939030485	3.81988e-11\\
1.5632183908046	3.72893e-11\\
1.56521739130435	3.72893e-11\\
1.5672163918041	3.72893e-11\\
1.56921539230385	3.72893e-11\\
1.5712143928036	3.81988e-11\\
1.57321339330335	3.72893e-11\\
1.5752123938031	3.72893e-11\\
1.57721139430285	3.81988e-11\\
1.5792103948026	3.72893e-11\\
1.58120939530235	3.72893e-11\\
1.5832083958021	3.72893e-11\\
1.58520739630185	3.63798e-11\\
1.5872063968016	3.54703e-11\\
1.58920539730135	3.54703e-11\\
1.5912043978011	3.54703e-11\\
1.59320339830085	3.63798e-11\\
1.5952023988006	3.54703e-11\\
1.59720139930035	3.54703e-11\\
1.5992003998001	3.45608e-11\\
1.60119940029985	3.54703e-11\\
1.6031984007996	3.45608e-11\\
1.60519740129935	3.45608e-11\\
1.6071964017991	3.45608e-11\\
1.60919540229885	3.36513e-11\\
1.6111944027986	3.36513e-11\\
1.61319340329835	3.36513e-11\\
1.6151924037981	3.45608e-11\\
1.61719140429785	3.36513e-11\\
1.6191904047976	3.45608e-11\\
1.62118940529735	3.45608e-11\\
1.6231884057971	3.45608e-11\\
1.62518740629685	3.45608e-11\\
1.6271864067966	3.36513e-11\\
1.62918540729635	3.36513e-11\\
1.6311844077961	3.45608e-11\\
1.63318340829585	3.54703e-11\\
1.6351824087956	3.54703e-11\\
1.63718140929535	3.45608e-11\\
1.6391804097951	3.36513e-11\\
1.64117941029485	3.36513e-11\\
1.6431784107946	3.27418e-11\\
1.64517741129435	3.36513e-11\\
1.6471764117941	3.27418e-11\\
1.64917541229385	3.27418e-11\\
1.6511744127936	3.45608e-11\\
1.65317341329335	3.36513e-11\\
1.6551724137931	3.45608e-11\\
1.65717141429285	3.45608e-11\\
1.6591704147926	3.54703e-11\\
1.66116941529235	3.54703e-11\\
1.6631684157921	3.54703e-11\\
1.66516741629185	3.54703e-11\\
1.6671664167916	3.54703e-11\\
1.66916541729135	3.54703e-11\\
1.6711644177911	3.54703e-11\\
1.67316341829085	3.54703e-11\\
1.6751624187906	3.54703e-11\\
1.67716141929035	3.54703e-11\\
1.6791604197901	3.54703e-11\\
1.68115942028985	3.54703e-11\\
1.68315842078961	3.54703e-11\\
1.68515742128936	3.54703e-11\\
1.68715642178911	3.54703e-11\\
1.68915542228886	3.54703e-11\\
1.69115442278861	3.54703e-11\\
1.69315342328836	3.54703e-11\\
1.69515242378811	3.54703e-11\\
1.69715142428786	3.63798e-11\\
1.69915042478761	3.54703e-11\\
1.70114942528736	3.54703e-11\\
1.70314842578711	3.63798e-11\\
1.70514742628686	3.54703e-11\\
1.70714642678661	3.54703e-11\\
1.70914542728636	3.63798e-11\\
1.71114442778611	3.72893e-11\\
1.71314342828586	3.72893e-11\\
1.71514242878561	3.63798e-11\\
1.71714142928536	3.63798e-11\\
1.71914042978511	3.54703e-11\\
1.72113943028486	3.45608e-11\\
1.72313843078461	3.45608e-11\\
1.72513743128436	3.45608e-11\\
1.72713643178411	3.45608e-11\\
1.72913543228386	3.36513e-11\\
1.73113443278361	3.27418e-11\\
1.73313343328336	3.27418e-11\\
1.73513243378311	3.27418e-11\\
1.73713143428286	3.18323e-11\\
1.73913043478261	3.18323e-11\\
1.74112943528236	3.09228e-11\\
1.74312843578211	3.09228e-11\\
1.74512743628186	3.00133e-11\\
1.74712643678161	3.09228e-11\\
1.74912543728136	3.09228e-11\\
1.75112443778111	3.09228e-11\\
1.75312343828086	3.09228e-11\\
1.75512243878061	3.09228e-11\\
1.75712143928036	3.09228e-11\\
1.75912043978011	3.09228e-11\\
1.76111944027986	3.00133e-11\\
1.76311844077961	3.00133e-11\\
1.76511744127936	2.91038e-11\\
1.76711644177911	3.00133e-11\\
1.76911544227886	3.09228e-11\\
1.77111444277861	3.18323e-11\\
1.77311344327836	3.18323e-11\\
1.77511244377811	3.18323e-11\\
1.77711144427786	3.18323e-11\\
1.77911044477761	3.18323e-11\\
1.78110944527736	3.27418e-11\\
1.78310844577711	3.36513e-11\\
1.78510744627686	3.36513e-11\\
1.78710644677661	3.36513e-11\\
1.78910544727636	3.36513e-11\\
1.79110444777611	3.36513e-11\\
1.79310344827586	3.27418e-11\\
1.79510244877561	3.27418e-11\\
1.79710144927536	3.18323e-11\\
1.79910044977511	3.09228e-11\\
1.80109945027486	3.00133e-11\\
1.80309845077461	3.00133e-11\\
1.80509745127436	3.00133e-11\\
1.80709645177411	2.91038e-11\\
1.80909545227386	2.91038e-11\\
1.81109445277361	3.00133e-11\\
1.81309345327336	3.00133e-11\\
1.81509245377311	2.91038e-11\\
1.81709145427286	2.91038e-11\\
1.81909045477261	2.81943e-11\\
1.82108945527236	2.91038e-11\\
1.82308845577211	2.91038e-11\\
1.82508745627186	2.91038e-11\\
1.82708645677161	3.00133e-11\\
1.82908545727136	2.91038e-11\\
1.83108445777111	2.81943e-11\\
1.83308345827086	2.72848e-11\\
1.83508245877061	2.81943e-11\\
1.83708145927036	2.91038e-11\\
1.83908045977011	2.81943e-11\\
1.84107946026987	2.81943e-11\\
1.84307846076962	2.72848e-11\\
1.84507746126937	2.72848e-11\\
1.84707646176912	2.72848e-11\\
1.84907546226887	2.72848e-11\\
1.85107446276862	2.63753e-11\\
1.85307346326837	2.72848e-11\\
1.85507246376812	2.81943e-11\\
1.85707146426787	2.81943e-11\\
1.85907046476762	2.72848e-11\\
1.86106946526737	2.72848e-11\\
1.86306846576712	2.63753e-11\\
1.86506746626687	2.63753e-11\\
1.86706646676662	2.63753e-11\\
1.86906546726637	2.63753e-11\\
1.87106446776612	2.63753e-11\\
1.87306346826587	2.63753e-11\\
1.87506246876562	2.63753e-11\\
1.87706146926537	2.63753e-11\\
1.87906046976512	2.63753e-11\\
1.88105947026487	2.63753e-11\\
1.88305847076462	2.63753e-11\\
1.88505747126437	2.54659e-11\\
1.88705647176412	2.54659e-11\\
1.88905547226387	2.54659e-11\\
1.89105447276362	2.63753e-11\\
1.89305347326337	2.63753e-11\\
1.89505247376312	2.63753e-11\\
1.89705147426287	2.54659e-11\\
1.89905047476262	2.54659e-11\\
1.90104947526237	2.45564e-11\\
1.90304847576212	2.45564e-11\\
1.90504747626187	2.36469e-11\\
1.90704647676162	2.45564e-11\\
1.90904547726137	2.45564e-11\\
1.91104447776112	2.45564e-11\\
1.91304347826087	2.45564e-11\\
1.91504247876062	2.45564e-11\\
1.91704147926037	2.36469e-11\\
1.91904047976012	2.36469e-11\\
1.92103948025987	2.45564e-11\\
1.92303848075962	2.36469e-11\\
1.92503748125937	2.45564e-11\\
1.92703648175912	2.36469e-11\\
1.92903548225887	2.36469e-11\\
1.93103448275862	2.36469e-11\\
1.93303348325837	2.36469e-11\\
1.93503248375812	2.45564e-11\\
1.93703148425787	2.45564e-11\\
1.93903048475762	2.45564e-11\\
1.94102948525737	2.45564e-11\\
1.94302848575712	2.45564e-11\\
1.94502748625687	2.45564e-11\\
1.94702648675662	2.36469e-11\\
1.94902548725637	2.36469e-11\\
1.95102448775612	2.36469e-11\\
1.95302348825587	2.36469e-11\\
1.95502248875562	2.36469e-11\\
1.95702148925537	2.45564e-11\\
1.95902048975512	2.36469e-11\\
1.96101949025487	2.36469e-11\\
1.96301849075462	2.36469e-11\\
1.96501749125437	2.36469e-11\\
1.96701649175412	2.45564e-11\\
1.96901549225387	2.45564e-11\\
1.97101449275362	2.36469e-11\\
1.97301349325337	2.45564e-11\\
1.97501249375312	2.45564e-11\\
1.97701149425287	2.54659e-11\\
1.97901049475262	2.54659e-11\\
1.98100949525237	2.54659e-11\\
1.98300849575212	2.63753e-11\\
1.98500749625187	2.63753e-11\\
1.98700649675162	2.63753e-11\\
1.98900549725137	2.63753e-11\\
1.99100449775112	2.63753e-11\\
1.99300349825087	2.63753e-11\\
1.99500249875062	2.63753e-11\\
1.99700149925037	2.54659e-11\\
1.99900049975012	2.54659e-11\\
2.00099950024988	2.54659e-11\\
2.00299850074963	2.45564e-11\\
2.00499750124938	2.54659e-11\\
2.00699650174913	2.54659e-11\\
2.00899550224888	2.45564e-11\\
2.01099450274863	2.45564e-11\\
2.01299350324838	2.54659e-11\\
2.01499250374813	2.45564e-11\\
2.01699150424788	2.45564e-11\\
2.01899050474763	2.36469e-11\\
2.02098950524738	2.45564e-11\\
2.02298850574713	2.45564e-11\\
2.02498750624688	2.45564e-11\\
2.02698650674663	2.45564e-11\\
2.02898550724638	2.45564e-11\\
2.03098450774613	2.45564e-11\\
2.03298350824588	2.45564e-11\\
2.03498250874563	2.45564e-11\\
2.03698150924538	2.36469e-11\\
2.03898050974513	2.36469e-11\\
2.04097951024488	2.45564e-11\\
2.04297851074463	2.36469e-11\\
2.04497751124438	2.36469e-11\\
2.04697651174413	2.36469e-11\\
2.04897551224388	2.36469e-11\\
2.05097451274363	2.36469e-11\\
2.05297351324338	2.36469e-11\\
2.05497251374313	2.36469e-11\\
2.05697151424288	2.27374e-11\\
2.05897051474263	2.27374e-11\\
2.06096951524238	2.27374e-11\\
2.06296851574213	2.27374e-11\\
2.06496751624188	2.27374e-11\\
2.06696651674163	2.18279e-11\\
2.06896551724138	2.18279e-11\\
2.07096451774113	2.18279e-11\\
2.07296351824088	2.09184e-11\\
2.07496251874063	2.09184e-11\\
2.07696151924038	2.09184e-11\\
2.07896051974013	2.09184e-11\\
2.08095952023988	2.09184e-11\\
2.08295852073963	2.18279e-11\\
2.08495752123938	2.18279e-11\\
2.08695652173913	2.18279e-11\\
2.08895552223888	2.09184e-11\\
2.09095452273863	2.09184e-11\\
2.09295352323838	2.18279e-11\\
2.09495252373813	2.09184e-11\\
2.09695152423788	2.18279e-11\\
2.09895052473763	2.09184e-11\\
2.10094952523738	2.09184e-11\\
2.10294852573713	2.09184e-11\\
2.10494752623688	2.09184e-11\\
2.10694652673663	2.18279e-11\\
2.10894552723638	2.09184e-11\\
2.11094452773613	2.09184e-11\\
2.11294352823588	2.09184e-11\\
2.11494252873563	2.09184e-11\\
2.11694152923538	2.09184e-11\\
2.11894052973513	2.09184e-11\\
2.12093953023488	2.09184e-11\\
2.12293853073463	2.09184e-11\\
2.12493753123438	2.00089e-11\\
2.12693653173413	2.00089e-11\\
2.12893553223388	2.00089e-11\\
2.13093453273363	2.00089e-11\\
2.13293353323338	2.00089e-11\\
2.13493253373313	2.00089e-11\\
2.13693153423288	2.00089e-11\\
2.13893053473263	2.00089e-11\\
2.14092953523238	2.00089e-11\\
2.14292853573213	2.00089e-11\\
2.14492753623188	2.00089e-11\\
2.14692653673163	2.00089e-11\\
2.14892553723138	2.00089e-11\\
2.15092453773113	2.09184e-11\\
2.15292353823088	2.09184e-11\\
2.15492253873063	2.09184e-11\\
2.15692153923038	2.09184e-11\\
2.15892053973013	2.09184e-11\\
2.16091954022989	2.09184e-11\\
2.16291854072964	2.18279e-11\\
2.16491754122939	2.09184e-11\\
2.16691654172914	2.09184e-11\\
2.16891554222889	2.09184e-11\\
2.17091454272864	2.09184e-11\\
2.17291354322839	2.09184e-11\\
2.17491254372814	2.09184e-11\\
2.17691154422789	2.09184e-11\\
2.17891054472764	2.09184e-11\\
2.18090954522739	2.09184e-11\\
2.18290854572714	2.00089e-11\\
2.18490754622689	2.00089e-11\\
2.18690654672664	2.00089e-11\\
2.18890554722639	2.09184e-11\\
2.19090454772614	2.09184e-11\\
2.19290354822589	2.00089e-11\\
2.19490254872564	2.00089e-11\\
2.19690154922539	2.00089e-11\\
2.19890054972514	2.09184e-11\\
2.20089955022489	2.00089e-11\\
2.20289855072464	2.00089e-11\\
2.20489755122439	2.00089e-11\\
2.20689655172414	2.00089e-11\\
2.20889555222389	2.00089e-11\\
2.21089455272364	2.00089e-11\\
2.21289355322339	2.09184e-11\\
2.21489255372314	2.00089e-11\\
2.21689155422289	2.00089e-11\\
2.21889055472264	2.00089e-11\\
2.22088955522239	2.00089e-11\\
2.22288855572214	2.00089e-11\\
2.22488755622189	2.00089e-11\\
2.22688655672164	2.00089e-11\\
2.22888555722139	2.00089e-11\\
2.23088455772114	2.00089e-11\\
2.23288355822089	2.00089e-11\\
2.23488255872064	2.00089e-11\\
2.23688155922039	2.00089e-11\\
2.23888055972014	2.09184e-11\\
2.24087956021989	2.00089e-11\\
2.24287856071964	2.00089e-11\\
2.24487756121939	2.00089e-11\\
2.24687656171914	2.00089e-11\\
2.24887556221889	2.00089e-11\\
2.25087456271864	2.09184e-11\\
2.25287356321839	2.09184e-11\\
2.25487256371814	2.00089e-11\\
2.25687156421789	2.00089e-11\\
2.25887056471764	2.00089e-11\\
2.26086956521739	2.00089e-11\\
2.26286856571714	2.09184e-11\\
2.26486756621689	2.09184e-11\\
2.26686656671664	2.09184e-11\\
2.26886556721639	2.09184e-11\\
2.27086456771614	2.09184e-11\\
2.27286356821589	2.18279e-11\\
2.27486256871564	2.18279e-11\\
2.27686156921539	2.18279e-11\\
2.27886056971514	2.18279e-11\\
2.28085957021489	2.18279e-11\\
2.28285857071464	2.18279e-11\\
2.28485757121439	2.18279e-11\\
2.28685657171414	2.09184e-11\\
2.28885557221389	2.09184e-11\\
2.29085457271364	2.18279e-11\\
2.29285357321339	2.18279e-11\\
2.29485257371314	2.18279e-11\\
2.29685157421289	2.27374e-11\\
2.29885057471264	2.18279e-11\\
2.30084957521239	2.27374e-11\\
2.30284857571214	2.27374e-11\\
2.30484757621189	2.27374e-11\\
2.30684657671164	2.27374e-11\\
2.30884557721139	2.27374e-11\\
2.31084457771114	2.27374e-11\\
2.31284357821089	2.27374e-11\\
2.31484257871064	2.27374e-11\\
2.31684157921039	2.36469e-11\\
2.31884057971015	2.27374e-11\\
2.3208395802099	2.36469e-11\\
2.32283858070965	2.36469e-11\\
2.3248375812094	2.36469e-11\\
2.32683658170915	2.36469e-11\\
2.3288355822089	2.27374e-11\\
2.33083458270865	2.27374e-11\\
2.3328335832084	2.36469e-11\\
2.33483258370815	2.36469e-11\\
2.3368315842079	2.36469e-11\\
2.33883058470765	2.36469e-11\\
2.3408295852074	2.36469e-11\\
2.34282858570715	2.36469e-11\\
2.3448275862069	2.27374e-11\\
2.34682658670665	2.27374e-11\\
2.3488255872064	2.27374e-11\\
2.35082458770615	2.27374e-11\\
2.3528235882059	2.27374e-11\\
2.35482258870565	2.27374e-11\\
2.3568215892054	2.27374e-11\\
2.35882058970515	2.27374e-11\\
2.3608195902049	2.27374e-11\\
2.36281859070465	2.18279e-11\\
2.3648175912044	2.18279e-11\\
2.36681659170415	2.27374e-11\\
2.3688155922039	2.27374e-11\\
2.37081459270365	2.27374e-11\\
2.3728135932034	2.27374e-11\\
2.37481259370315	2.27374e-11\\
2.3768115942029	2.27374e-11\\
2.37881059470265	2.27374e-11\\
2.3808095952024	2.27374e-11\\
2.38280859570215	2.27374e-11\\
2.3848075962019	2.18279e-11\\
2.38680659670165	2.18279e-11\\
2.3888055972014	2.18279e-11\\
2.39080459770115	2.18279e-11\\
2.3928035982009	2.18279e-11\\
2.39480259870065	2.18279e-11\\
2.3968015992004	2.09184e-11\\
2.39880059970015	2.09184e-11\\
2.4007996001999	2.09184e-11\\
2.40279860069965	2.18279e-11\\
2.4047976011994	2.18279e-11\\
2.40679660169915	2.18279e-11\\
2.4087956021989	2.09184e-11\\
2.41079460269865	2.09184e-11\\
2.4127936031984	2.09184e-11\\
2.41479260369815	2.09184e-11\\
2.4167916041979	2.09184e-11\\
2.41879060469765	2.09184e-11\\
2.4207896051974	2.09184e-11\\
2.42278860569715	2.09184e-11\\
2.4247876061969	2.09184e-11\\
2.42678660669665	2.09184e-11\\
2.4287856071964	2.18279e-11\\
2.43078460769615	2.18279e-11\\
2.4327836081959	2.18279e-11\\
2.43478260869565	2.18279e-11\\
2.4367816091954	2.18279e-11\\
2.43878060969515	2.18279e-11\\
2.4407796101949	2.18279e-11\\
2.44277861069465	2.27374e-11\\
2.4447776111944	2.27374e-11\\
2.44677661169415	2.27374e-11\\
2.4487756121939	2.36469e-11\\
2.45077461269365	2.27374e-11\\
2.4527736131934	2.27374e-11\\
2.45477261369315	2.27374e-11\\
2.4567716141929	2.27374e-11\\
2.45877061469265	2.27374e-11\\
2.4607696151924	2.27374e-11\\
2.46276861569215	2.36469e-11\\
2.4647676161919	2.36469e-11\\
2.46676661669165	2.36469e-11\\
2.4687656171914	2.27374e-11\\
2.47076461769115	2.27374e-11\\
2.4727636181909	2.27374e-11\\
2.47476261869065	2.27374e-11\\
2.4767616191904	2.27374e-11\\
2.47876061969016	2.27374e-11\\
2.48075962018991	2.36469e-11\\
2.48275862068966	2.36469e-11\\
2.48475762118941	2.36469e-11\\
2.48675662168916	2.36469e-11\\
2.48875562218891	2.36469e-11\\
2.49075462268866	2.36469e-11\\
2.49275362318841	2.45564e-11\\
2.49475262368816	2.45564e-11\\
2.49675162418791	2.45564e-11\\
2.49875062468766	2.45564e-11\\
2.50074962518741	2.45564e-11\\
2.50274862568716	2.45564e-11\\
2.50474762618691	2.45564e-11\\
2.50674662668666	2.45564e-11\\
2.50874562718641	2.54659e-11\\
2.51074462768616	2.45564e-11\\
2.51274362818591	2.45564e-11\\
2.51474262868566	2.45564e-11\\
2.51674162918541	2.45564e-11\\
2.51874062968516	2.45564e-11\\
2.52073963018491	2.45564e-11\\
2.52273863068466	2.45564e-11\\
2.52473763118441	2.45564e-11\\
2.52673663168416	2.45564e-11\\
2.52873563218391	2.54659e-11\\
2.53073463268366	2.54659e-11\\
2.53273363318341	2.54659e-11\\
2.53473263368316	2.54659e-11\\
2.53673163418291	2.54659e-11\\
2.53873063468266	2.54659e-11\\
2.54072963518241	2.45564e-11\\
2.54272863568216	2.54659e-11\\
2.54472763618191	2.45564e-11\\
2.54672663668166	2.45564e-11\\
2.54872563718141	2.45564e-11\\
2.55072463768116	2.45564e-11\\
2.55272363818091	2.45564e-11\\
2.55472263868066	2.45564e-11\\
2.55672163918041	2.45564e-11\\
2.55872063968016	2.45564e-11\\
2.56071964017991	2.36469e-11\\
2.56271864067966	2.45564e-11\\
2.56471764117941	2.36469e-11\\
2.56671664167916	2.36469e-11\\
2.56871564217891	2.27374e-11\\
2.57071464267866	2.36469e-11\\
2.57271364317841	2.36469e-11\\
2.57471264367816	2.36469e-11\\
2.57671164417791	2.36469e-11\\
2.57871064467766	2.27374e-11\\
2.58070964517741	2.27374e-11\\
2.58270864567716	2.27374e-11\\
2.58470764617691	2.36469e-11\\
2.58670664667666	2.36469e-11\\
2.58870564717641	2.36469e-11\\
2.59070464767616	2.36469e-11\\
2.59270364817591	2.36469e-11\\
2.59470264867566	2.36469e-11\\
2.59670164917541	2.27374e-11\\
2.59870064967516	2.27374e-11\\
2.60069965017491	2.27374e-11\\
2.60269865067466	2.27374e-11\\
2.60469765117441	2.27374e-11\\
2.60669665167416	2.36469e-11\\
2.60869565217391	2.36469e-11\\
2.61069465267366	2.27374e-11\\
2.61269365317341	2.27374e-11\\
2.61469265367316	2.27374e-11\\
2.61669165417291	2.27374e-11\\
2.61869065467266	2.36469e-11\\
2.62068965517241	2.36469e-11\\
2.62268865567216	2.36469e-11\\
2.62468765617191	2.36469e-11\\
2.62668665667166	2.36469e-11\\
2.62868565717141	2.36469e-11\\
2.63068465767116	2.36469e-11\\
2.63268365817091	2.36469e-11\\
2.63468265867066	2.45564e-11\\
2.63668165917041	2.45564e-11\\
2.63868065967017	2.45564e-11\\
2.64067966016992	2.36469e-11\\
2.64267866066967	2.36469e-11\\
2.64467766116942	2.45564e-11\\
2.64667666166917	2.36469e-11\\
2.64867566216892	2.36469e-11\\
2.65067466266867	2.36469e-11\\
2.65267366316842	2.36469e-11\\
2.65467266366817	2.36469e-11\\
2.65667166416792	2.36469e-11\\
2.65867066466767	2.36469e-11\\
2.66066966516742	2.36469e-11\\
2.66266866566717	2.36469e-11\\
2.66466766616692	2.45564e-11\\
2.66666666666667	2.45564e-11\\
2.66866566716642	2.45564e-11\\
2.67066466766617	2.45564e-11\\
2.67266366816592	2.45564e-11\\
2.67466266866567	2.45564e-11\\
2.67666166916542	2.36469e-11\\
2.67866066966517	2.45564e-11\\
2.68065967016492	2.45564e-11\\
2.68265867066467	2.54659e-11\\
2.68465767116442	2.54659e-11\\
2.68665667166417	2.54659e-11\\
2.68865567216392	2.45564e-11\\
2.69065467266367	2.45564e-11\\
2.69265367316342	2.45564e-11\\
2.69465267366317	2.45564e-11\\
2.69665167416292	2.36469e-11\\
2.69865067466267	2.45564e-11\\
2.70064967516242	2.54659e-11\\
2.70264867566217	2.54659e-11\\
2.70464767616192	2.63753e-11\\
2.70664667666167	2.63753e-11\\
2.70864567716142	2.63753e-11\\
2.71064467766117	2.63753e-11\\
2.71264367816092	2.72848e-11\\
2.71464267866067	2.63753e-11\\
2.71664167916042	2.63753e-11\\
2.71864067966017	2.63753e-11\\
2.72063968015992	2.54659e-11\\
2.72263868065967	2.63753e-11\\
2.72463768115942	2.63753e-11\\
2.72663668165917	2.63753e-11\\
2.72863568215892	2.54659e-11\\
2.73063468265867	2.54659e-11\\
2.73263368315842	2.54659e-11\\
2.73463268365817	2.54659e-11\\
2.73663168415792	2.45564e-11\\
2.73863068465767	2.45564e-11\\
2.74062968515742	2.54659e-11\\
2.74262868565717	2.63753e-11\\
2.74462768615692	2.54659e-11\\
2.74662668665667	2.45564e-11\\
2.74862568715642	2.45564e-11\\
2.75062468765617	2.45564e-11\\
2.75262368815592	2.45564e-11\\
2.75462268865567	2.45564e-11\\
2.75662168915542	2.36469e-11\\
2.75862068965517	2.27374e-11\\
2.76061969015492	2.18279e-11\\
2.76261869065467	2.27374e-11\\
2.76461769115442	2.27374e-11\\
2.76661669165417	2.27374e-11\\
2.76861569215392	2.18279e-11\\
2.77061469265367	2.18279e-11\\
2.77261369315342	2.18279e-11\\
2.77461269365317	2.18279e-11\\
2.77661169415292	2.18279e-11\\
2.77861069465267	2.09184e-11\\
2.78060969515242	2.09184e-11\\
2.78260869565217	2.18279e-11\\
2.78460769615192	2.18279e-11\\
2.78660669665167	2.18279e-11\\
2.78860569715142	2.18279e-11\\
2.79060469765117	2.18279e-11\\
2.79260369815092	2.09184e-11\\
2.79460269865067	2.09184e-11\\
2.79660169915043	2.09184e-11\\
2.79860069965018	2.09184e-11\\
2.80059970014993	2.09184e-11\\
2.80259870064968	2.00089e-11\\
2.80459770114943	2.00089e-11\\
2.80659670164918	1.90994e-11\\
2.80859570214893	1.90994e-11\\
2.81059470264868	2.00089e-11\\
2.81259370314843	2.00089e-11\\
2.81459270364818	2.00089e-11\\
2.81659170414793	2.00089e-11\\
2.81859070464768	2.00089e-11\\
2.82058970514743	1.90994e-11\\
2.82258870564718	2.00089e-11\\
2.82458770614693	2.00089e-11\\
2.82658670664668	2.00089e-11\\
2.82858570714643	2.00089e-11\\
2.83058470764618	2.00089e-11\\
2.83258370814593	2.00089e-11\\
2.83458270864568	2.00089e-11\\
2.83658170914543	2.09184e-11\\
2.83858070964518	2.09184e-11\\
2.84057971014493	2.09184e-11\\
2.84257871064468	2.00089e-11\\
2.84457771114443	2.00089e-11\\
2.84657671164418	2.00089e-11\\
2.84857571214393	2.00089e-11\\
2.85057471264368	2.09184e-11\\
2.85257371314343	2.00089e-11\\
2.85457271364318	2.00089e-11\\
2.85657171414293	2.00089e-11\\
2.85857071464268	2.00089e-11\\
2.86056971514243	2.00089e-11\\
2.86256871564218	2.00089e-11\\
2.86456771614193	2.00089e-11\\
2.86656671664168	2.00089e-11\\
2.86856571714143	2.00089e-11\\
2.87056471764118	2.00089e-11\\
2.87256371814093	2.00089e-11\\
2.87456271864068	2.00089e-11\\
2.87656171914043	1.90994e-11\\
2.87856071964018	1.90994e-11\\
2.88055972013993	2.00089e-11\\
2.88255872063968	1.90994e-11\\
2.88455772113943	2.00089e-11\\
2.88655672163918	2.00089e-11\\
2.88855572213893	2.00089e-11\\
2.89055472263868	1.90994e-11\\
2.89255372313843	1.81899e-11\\
2.89455272363818	1.81899e-11\\
2.89655172413793	1.81899e-11\\
2.89855072463768	1.90994e-11\\
2.90054972513743	1.81899e-11\\
2.90254872563718	1.72804e-11\\
2.90454772613693	1.81899e-11\\
2.90654672663668	1.72804e-11\\
2.90854572713643	1.72804e-11\\
2.91054472763618	1.72804e-11\\
2.91254372813593	1.72804e-11\\
2.91454272863568	1.72804e-11\\
2.91654172913543	1.81899e-11\\
2.91854072963518	1.90994e-11\\
2.92053973013493	2.00089e-11\\
2.92253873063468	1.90994e-11\\
2.92453773113443	2.00089e-11\\
2.92653673163418	2.00089e-11\\
2.92853573213393	1.90994e-11\\
2.93053473263368	2.00089e-11\\
2.93253373313343	2.00089e-11\\
2.93453273363318	1.90994e-11\\
2.93653173413293	2.00089e-11\\
2.93853073463268	2.00089e-11\\
2.94052973513243	1.90994e-11\\
2.94252873563218	2.00089e-11\\
2.94452773613193	2.00089e-11\\
2.94652673663168	2.00089e-11\\
2.94852573713143	2.00089e-11\\
2.95052473763118	2.09184e-11\\
2.95252373813093	2.18279e-11\\
2.95452273863068	2.00089e-11\\
2.95652173913043	2.09184e-11\\
2.95852073963018	2.09184e-11\\
2.96051974012994	2.09184e-11\\
2.96251874062969	2.18279e-11\\
2.96451774112944	2.09184e-11\\
2.96651674162919	2.09184e-11\\
2.96851574212894	2.09184e-11\\
2.97051474262869	2.09184e-11\\
2.97251374312844	2.09184e-11\\
2.97451274362819	2.09184e-11\\
2.97651174412794	2.09184e-11\\
2.97851074462769	2.09184e-11\\
2.98050974512744	2.09184e-11\\
2.98250874562719	2.18279e-11\\
2.98450774612694	2.27374e-11\\
2.98650674662669	2.36469e-11\\
2.98850574712644	2.36469e-11\\
2.99050474762619	2.36469e-11\\
2.99250374812594	2.45564e-11\\
2.99450274862569	2.36469e-11\\
2.99650174912544	2.36469e-11\\
2.99850074962519	2.36469e-11\\
3.00049975012494	2.36469e-11\\
3.00249875062469	2.54659e-11\\
3.00449775112444	2.54659e-11\\
3.00649675162419	2.63753e-11\\
3.00849575212394	2.54659e-11\\
3.01049475262369	2.54659e-11\\
3.01249375312344	2.63753e-11\\
3.01449275362319	2.54659e-11\\
3.01649175412294	2.63753e-11\\
3.01849075462269	2.54659e-11\\
3.02048975512244	2.54659e-11\\
3.02248875562219	2.54659e-11\\
3.02448775612194	2.54659e-11\\
3.02648675662169	2.63753e-11\\
3.02848575712144	2.63753e-11\\
3.03048475762119	2.54659e-11\\
3.03248375812094	2.63753e-11\\
3.03448275862069	2.63753e-11\\
3.03648175912044	2.72848e-11\\
3.03848075962019	2.63753e-11\\
3.04047976011994	2.72848e-11\\
3.04247876061969	2.72848e-11\\
3.04447776111944	2.63753e-11\\
3.04647676161919	2.72848e-11\\
3.04847576211894	2.63753e-11\\
3.05047476261869	2.63753e-11\\
3.05247376311844	2.63753e-11\\
3.05447276361819	2.54659e-11\\
3.05647176411794	2.54659e-11\\
3.05847076461769	2.36469e-11\\
3.06046976511744	2.36469e-11\\
3.06246876561719	2.54659e-11\\
3.06446776611694	2.36469e-11\\
3.06646676661669	2.54659e-11\\
3.06846576711644	2.54659e-11\\
3.07046476761619	2.54659e-11\\
3.07246376811594	2.45564e-11\\
3.07446276861569	2.54659e-11\\
3.07646176911544	2.45564e-11\\
3.07846076961519	2.36469e-11\\
3.08045977011494	2.27374e-11\\
3.08245877061469	2.36469e-11\\
3.08445777111444	2.36469e-11\\
3.08645677161419	2.45564e-11\\
3.08845577211394	2.45564e-11\\
3.09045477261369	2.36469e-11\\
3.09245377311344	2.36469e-11\\
3.09445277361319	2.45564e-11\\
3.09645177411294	2.36469e-11\\
3.09845077461269	2.45564e-11\\
3.10044977511244	2.45564e-11\\
3.10244877561219	2.27374e-11\\
3.10444777611194	2.36469e-11\\
3.10644677661169	2.36469e-11\\
3.10844577711144	2.54659e-11\\
3.11044477761119	2.63753e-11\\
3.11244377811094	2.63753e-11\\
3.11444277861069	2.72848e-11\\
3.11644177911044	2.81943e-11\\
3.11844077961019	2.91038e-11\\
3.12043978010994	2.81943e-11\\
3.12243878060969	2.81943e-11\\
3.12443778110945	2.81943e-11\\
3.1264367816092	2.81943e-11\\
3.12843578210895	2.91038e-11\\
3.1304347826087	2.91038e-11\\
3.13243378310845	3.00133e-11\\
3.1344327836082	3.00133e-11\\
3.13643178410795	3.00133e-11\\
3.1384307846077	2.91038e-11\\
3.14042978510745	2.72848e-11\\
3.1424287856072	2.72848e-11\\
3.14442778610695	2.81943e-11\\
3.1464267866067	2.72848e-11\\
3.14842578710645	2.63753e-11\\
3.1504247876062	2.72848e-11\\
3.15242378810595	2.63753e-11\\
3.1544227886057	2.63753e-11\\
3.15642178910545	2.45564e-11\\
3.1584207896052	2.36469e-11\\
3.16041979010495	2.27374e-11\\
3.1624187906047	2.18279e-11\\
3.16441779110445	2.27374e-11\\
3.1664167916042	2.27374e-11\\
3.16841579210395	2.27374e-11\\
3.1704147926037	2.27374e-11\\
3.17241379310345	2.36469e-11\\
3.1744127936032	2.36469e-11\\
3.17641179410295	2.36469e-11\\
3.1784107946027	2.45564e-11\\
3.18040979510245	2.45564e-11\\
3.1824087956022	2.36469e-11\\
3.18440779610195	2.36469e-11\\
3.1864067966017	2.45564e-11\\
3.18840579710145	2.63753e-11\\
3.1904047976012	2.72848e-11\\
3.19240379810095	2.54659e-11\\
3.1944027986007	2.54659e-11\\
3.19640179910045	2.54659e-11\\
3.1984007996002	2.45564e-11\\
3.20039980009995	2.45564e-11\\
3.2023988005997	2.36469e-11\\
3.20439780109945	2.36469e-11\\
3.2063968015992	2.45564e-11\\
3.20839580209895	2.54659e-11\\
3.2103948025987	2.63753e-11\\
3.21239380309845	2.63753e-11\\
3.2143928035982	2.63753e-11\\
3.21639180409795	2.72848e-11\\
3.2183908045977	2.72848e-11\\
3.22038980509745	2.81943e-11\\
3.2223888055972	2.81943e-11\\
3.22438780609695	3.00133e-11\\
3.2263868065967	3.00133e-11\\
3.22838580709645	3.09228e-11\\
3.2303848075962	3.09228e-11\\
3.23238380809595	3.09228e-11\\
3.2343828085957	3.18323e-11\\
3.23638180909545	3.09228e-11\\
3.2383808095952	3.09228e-11\\
3.24037981009495	3.00133e-11\\
3.2423788105947	3.09228e-11\\
3.24437781109445	3.09228e-11\\
3.2463768115942	3.09228e-11\\
3.24837581209395	3.18323e-11\\
3.2503748125937	3.18323e-11\\
3.25237381309345	3.27418e-11\\
3.2543728135932	3.18323e-11\\
3.25637181409295	3.27418e-11\\
3.2583708145927	3.27418e-11\\
3.26036981509245	3.27418e-11\\
3.2623688155922	3.27418e-11\\
3.26436781609195	3.45608e-11\\
3.2663668165917	3.36513e-11\\
3.26836581709145	3.36513e-11\\
3.2703648175912	3.45608e-11\\
3.27236381809095	3.45608e-11\\
3.2743628185907	3.54703e-11\\
3.27636181909045	3.45608e-11\\
3.2783608195902	3.54703e-11\\
3.28035982008995	3.54703e-11\\
3.2823588205897	3.45608e-11\\
3.28435782108946	3.54703e-11\\
3.28635682158921	3.45608e-11\\
3.28835582208896	3.54703e-11\\
3.29035482258871	3.54703e-11\\
3.29235382308846	3.45608e-11\\
3.29435282358821	3.45608e-11\\
3.29635182408796	3.45608e-11\\
3.29835082458771	3.45608e-11\\
3.30034982508746	3.54703e-11\\
3.30234882558721	3.54703e-11\\
3.30434782608696	3.54703e-11\\
3.30634682658671	3.45608e-11\\
3.30834582708646	3.45608e-11\\
3.31034482758621	3.45608e-11\\
3.31234382808596	3.45608e-11\\
3.31434282858571	3.54703e-11\\
3.31634182908546	3.54703e-11\\
3.31834082958521	3.54703e-11\\
3.32033983008496	3.54703e-11\\
3.32233883058471	3.54703e-11\\
3.32433783108446	3.54703e-11\\
3.32633683158421	3.45608e-11\\
3.32833583208396	3.45608e-11\\
3.33033483258371	3.36513e-11\\
3.33233383308346	3.36513e-11\\
3.33433283358321	3.45608e-11\\
3.33633183408296	3.36513e-11\\
3.33833083458271	3.36513e-11\\
3.34032983508246	3.27418e-11\\
3.34232883558221	3.27418e-11\\
3.34432783608196	3.36513e-11\\
3.34632683658171	3.36513e-11\\
3.34832583708146	3.36513e-11\\
3.35032483758121	3.54703e-11\\
3.35232383808096	3.54703e-11\\
3.35432283858071	3.54703e-11\\
3.35632183908046	3.54703e-11\\
3.35832083958021	3.54703e-11\\
3.36031984007996	3.54703e-11\\
3.36231884057971	3.45608e-11\\
3.36431784107946	3.45608e-11\\
3.36631684157921	3.45608e-11\\
3.36831584207896	3.54703e-11\\
3.37031484257871	3.54703e-11\\
3.37231384307846	3.54703e-11\\
3.37431284357821	3.54703e-11\\
3.37631184407796	3.54703e-11\\
3.37831084457771	3.45608e-11\\
3.38030984507746	3.54703e-11\\
3.38230884557721	3.54703e-11\\
3.38430784607696	3.54703e-11\\
3.38630684657671	3.54703e-11\\
3.38830584707646	3.54703e-11\\
3.39030484757621	3.54703e-11\\
3.39230384807596	3.63798e-11\\
3.39430284857571	3.63798e-11\\
3.39630184907546	3.72893e-11\\
3.39830084957521	3.63798e-11\\
3.40029985007496	3.54703e-11\\
3.40229885057471	3.63798e-11\\
3.40429785107446	3.63798e-11\\
3.40629685157421	3.63798e-11\\
3.40829585207396	3.54703e-11\\
3.41029485257371	3.54703e-11\\
3.41229385307346	3.54703e-11\\
3.41429285357321	3.54703e-11\\
3.41629185407296	3.63798e-11\\
3.41829085457271	3.63798e-11\\
3.42028985507246	3.54703e-11\\
3.42228885557221	3.63798e-11\\
3.42428785607196	3.63798e-11\\
3.42628685657171	3.72893e-11\\
3.42828585707146	3.72893e-11\\
3.43028485757121	3.72893e-11\\
3.43228385807096	3.81988e-11\\
3.43428285857071	3.81988e-11\\
3.43628185907046	3.81988e-11\\
3.43828085957021	3.91083e-11\\
3.44027986006996	3.91083e-11\\
3.44227886056971	4.00178e-11\\
3.44427786106947	3.91083e-11\\
3.44627686156922	3.91083e-11\\
3.44827586206897	3.72893e-11\\
3.45027486256872	3.72893e-11\\
3.45227386306847	3.72893e-11\\
3.45427286356822	3.63798e-11\\
3.45627186406797	3.63798e-11\\
3.45827086456772	3.54703e-11\\
3.46026986506747	3.63798e-11\\
3.46226886556722	3.63798e-11\\
3.46426786606697	3.63798e-11\\
3.46626686656672	3.72893e-11\\
3.46826586706647	3.81988e-11\\
3.47026486756622	3.81988e-11\\
3.47226386806597	3.81988e-11\\
3.47426286856572	3.81988e-11\\
3.47626186906547	3.72893e-11\\
3.47826086956522	3.81988e-11\\
3.48025987006497	3.91083e-11\\
3.48225887056472	3.81988e-11\\
3.48425787106447	3.91083e-11\\
3.48625687156422	3.91083e-11\\
3.48825587206397	4.00178e-11\\
3.49025487256372	3.91083e-11\\
3.49225387306347	3.81988e-11\\
3.49425287356322	3.72893e-11\\
3.49625187406297	3.72893e-11\\
3.49825087456272	3.81988e-11\\
3.50024987506247	3.81988e-11\\
3.50224887556222	3.81988e-11\\
3.50424787606197	3.91083e-11\\
3.50624687656172	3.81988e-11\\
3.50824587706147	3.81988e-11\\
3.51024487756122	3.81988e-11\\
3.51224387806097	3.81988e-11\\
3.51424287856072	3.91083e-11\\
3.51624187906047	3.91083e-11\\
3.51824087956022	3.91083e-11\\
3.52023988005997	3.91083e-11\\
3.52223888055972	3.91083e-11\\
3.52423788105947	3.91083e-11\\
3.52623688155922	4.00178e-11\\
3.52823588205897	3.91083e-11\\
3.53023488255872	3.91083e-11\\
3.53223388305847	3.81988e-11\\
3.53423288355822	3.81988e-11\\
3.53623188405797	3.81988e-11\\
3.53823088455772	3.81988e-11\\
3.54022988505747	3.81988e-11\\
3.54222888555722	3.91083e-11\\
3.54422788605697	3.91083e-11\\
3.54622688655672	3.91083e-11\\
3.54822588705647	3.91083e-11\\
3.55022488755622	3.81988e-11\\
3.55222388805597	3.81988e-11\\
3.55422288855572	3.72893e-11\\
3.55622188905547	3.81988e-11\\
3.55822088955522	3.81988e-11\\
3.56021989005497	3.81988e-11\\
3.56221889055472	3.91083e-11\\
3.56421789105447	3.91083e-11\\
3.56621689155422	3.91083e-11\\
3.56821589205397	3.91083e-11\\
3.57021489255372	3.91083e-11\\
3.57221389305347	3.91083e-11\\
3.57421289355322	3.91083e-11\\
3.57621189405297	3.91083e-11\\
3.57821089455272	3.91083e-11\\
3.58020989505247	3.91083e-11\\
3.58220889555222	4.00178e-11\\
3.58420789605197	3.81988e-11\\
3.58620689655172	3.91083e-11\\
3.58820589705147	3.91083e-11\\
3.59020489755122	3.91083e-11\\
3.59220389805097	3.91083e-11\\
3.59420289855072	4.00178e-11\\
3.59620189905047	4.00178e-11\\
3.59820089955022	3.91083e-11\\
3.60019990004997	3.91083e-11\\
3.60219890054973	4.00178e-11\\
3.60419790104948	4.09273e-11\\
3.60619690154923	4.00178e-11\\
3.60819590204898	4.00178e-11\\
3.61019490254873	4.00178e-11\\
3.61219390304848	4.09273e-11\\
3.61419290354823	4.00178e-11\\
3.61619190404798	4.00178e-11\\
3.61819090454773	4.00178e-11\\
3.62018990504748	4.00178e-11\\
3.62218890554723	4.09273e-11\\
3.62418790604698	4.09273e-11\\
3.62618690654673	4.09273e-11\\
3.62818590704648	4.09273e-11\\
3.63018490754623	4.09273e-11\\
3.63218390804598	4.09273e-11\\
3.63418290854573	4.09273e-11\\
3.63618190904548	4.09273e-11\\
3.63818090954523	4.09273e-11\\
3.64017991004498	4.09273e-11\\
3.64217891054473	4.09273e-11\\
3.64417791104448	4.09273e-11\\
3.64617691154423	4.09273e-11\\
3.64817591204398	4.09273e-11\\
3.65017491254373	4.09273e-11\\
3.65217391304348	4.09273e-11\\
3.65417291354323	4.18368e-11\\
3.65617191404298	4.18368e-11\\
3.65817091454273	4.18368e-11\\
3.66016991504248	4.18368e-11\\
3.66216891554223	4.18368e-11\\
3.66416791604198	4.18368e-11\\
3.66616691654173	4.18368e-11\\
3.66816591704148	4.09273e-11\\
3.67016491754123	4.09273e-11\\
3.67216391804098	4.09273e-11\\
3.67416291854073	4.09273e-11\\
3.67616191904048	4.18368e-11\\
3.67816091954023	4.09273e-11\\
3.68015992003998	4.18368e-11\\
3.68215892053973	4.18368e-11\\
3.68415792103948	4.27463e-11\\
3.68615692153923	4.27463e-11\\
3.68815592203898	4.27463e-11\\
3.69015492253873	4.27463e-11\\
3.69215392303848	4.36557e-11\\
3.69415292353823	4.27463e-11\\
3.69615192403798	4.27463e-11\\
3.69815092453773	4.27463e-11\\
3.70014992503748	4.36557e-11\\
3.70214892553723	4.27463e-11\\
3.70414792603698	4.27463e-11\\
3.70614692653673	4.18368e-11\\
3.70814592703648	4.18368e-11\\
3.71014492753623	4.27463e-11\\
3.71214392803598	4.27463e-11\\
3.71414292853573	4.27463e-11\\
3.71614192903548	4.27463e-11\\
3.71814092953523	4.27463e-11\\
3.72013993003498	4.27463e-11\\
3.72213893053473	4.27463e-11\\
3.72413793103448	4.27463e-11\\
3.72613693153423	4.27463e-11\\
3.72813593203398	4.27463e-11\\
3.73013493253373	4.27463e-11\\
3.73213393303348	4.18368e-11\\
3.73413293353323	4.18368e-11\\
3.73613193403298	4.18368e-11\\
3.73813093453273	4.18368e-11\\
3.74012993503248	4.18368e-11\\
3.74212893553223	4.27463e-11\\
3.74412793603198	4.27463e-11\\
3.74612693653173	4.18368e-11\\
3.74812593703148	4.27463e-11\\
3.75012493753123	4.27463e-11\\
3.75212393803098	4.27463e-11\\
3.75412293853073	4.27463e-11\\
3.75612193903048	4.36557e-11\\
3.75812093953023	4.36557e-11\\
3.76011994002998	4.36557e-11\\
3.76211894052974	4.36557e-11\\
3.76411794102949	4.36557e-11\\
3.76611694152924	4.36557e-11\\
3.76811594202899	4.36557e-11\\
3.77011494252874	4.36557e-11\\
3.77211394302849	4.27463e-11\\
3.77411294352824	4.36557e-11\\
3.77611194402799	4.36557e-11\\
3.77811094452774	4.27463e-11\\
3.78010994502749	4.27463e-11\\
3.78210894552724	4.27463e-11\\
3.78410794602699	4.27463e-11\\
3.78610694652674	4.27463e-11\\
3.78810594702649	4.27463e-11\\
3.79010494752624	4.36557e-11\\
3.79210394802599	4.36557e-11\\
3.79410294852574	4.36557e-11\\
3.79610194902549	4.36557e-11\\
3.79810094952524	4.36557e-11\\
3.80009995002499	4.36557e-11\\
3.80209895052474	4.27463e-11\\
3.80409795102449	4.27463e-11\\
3.80609695152424	4.18368e-11\\
3.80809595202399	4.27463e-11\\
3.81009495252374	4.27463e-11\\
3.81209395302349	4.27463e-11\\
3.81409295352324	4.36557e-11\\
3.81609195402299	4.36557e-11\\
3.81809095452274	4.27463e-11\\
3.82008995502249	4.27463e-11\\
3.82208895552224	4.27463e-11\\
3.82408795602199	4.27463e-11\\
3.82608695652174	4.27463e-11\\
3.82808595702149	4.27463e-11\\
3.83008495752124	4.27463e-11\\
3.83208395802099	4.27463e-11\\
3.83408295852074	4.27463e-11\\
3.83608195902049	4.27463e-11\\
3.83808095952024	4.27463e-11\\
3.84007996001999	4.27463e-11\\
3.84207896051974	4.27463e-11\\
3.84407796101949	4.27463e-11\\
3.84607696151924	4.27463e-11\\
3.84807596201899	4.18368e-11\\
3.85007496251874	4.18368e-11\\
3.85207396301849	4.27463e-11\\
3.85407296351824	4.27463e-11\\
3.85607196401799	4.27463e-11\\
3.85807096451774	4.27463e-11\\
3.86006996501749	4.27463e-11\\
3.86206896551724	4.27463e-11\\
3.86406796601699	4.36557e-11\\
3.86606696651674	4.36557e-11\\
3.86806596701649	4.36557e-11\\
3.87006496751624	4.36557e-11\\
3.87206396801599	4.36557e-11\\
3.87406296851574	4.36557e-11\\
3.87606196901549	4.36557e-11\\
3.87806096951524	4.36557e-11\\
3.88005997001499	4.27463e-11\\
3.88205897051474	4.27463e-11\\
3.88405797101449	4.36557e-11\\
3.88605697151424	4.36557e-11\\
3.88805597201399	4.36557e-11\\
3.89005497251374	4.36557e-11\\
3.89205397301349	4.36557e-11\\
3.89405297351324	4.36557e-11\\
3.89605197401299	4.36557e-11\\
3.89805097451274	4.36557e-11\\
3.90004997501249	4.27463e-11\\
3.90204897551224	4.27463e-11\\
3.90404797601199	4.27463e-11\\
3.90604697651174	4.18368e-11\\
3.90804597701149	4.18368e-11\\
3.91004497751124	4.18368e-11\\
3.91204397801099	4.18368e-11\\
3.91404297851074	4.18368e-11\\
3.91604197901049	4.18368e-11\\
3.91804097951024	4.18368e-11\\
3.92003998000999	4.27463e-11\\
3.92203898050975	4.18368e-11\\
3.9240379810095	4.18368e-11\\
3.92603698150925	4.18368e-11\\
3.928035982009	4.18368e-11\\
3.93003498250875	4.09273e-11\\
3.9320339830085	4.09273e-11\\
3.93403298350825	4.09273e-11\\
3.936031984008	4.09273e-11\\
3.93803098450775	4.09273e-11\\
3.9400299850075	4.09273e-11\\
3.94202898550725	4.09273e-11\\
3.944027986007	4.09273e-11\\
3.94602698650675	4.09273e-11\\
3.9480259870065	4.09273e-11\\
3.95002498750625	4.18368e-11\\
3.952023988006	4.09273e-11\\
3.95402298850575	4.18368e-11\\
3.9560219890055	4.09273e-11\\
3.95802098950525	4.09273e-11\\
3.960019990005	4.09273e-11\\
3.96201899050475	4.00178e-11\\
3.9640179910045	4.00178e-11\\
3.96601699150425	4.09273e-11\\
3.968015992004	4.00178e-11\\
3.97001499250375	4.00178e-11\\
3.9720139930035	4.00178e-11\\
3.97401299350325	3.91083e-11\\
3.976011994003	3.91083e-11\\
3.97801099450275	4.00178e-11\\
3.9800099950025	4.00178e-11\\
3.98200899550225	3.91083e-11\\
3.984007996002	3.91083e-11\\
3.98600699650175	4.00178e-11\\
3.9880059970015	4.00178e-11\\
3.99000499750125	4.00178e-11\\
3.992003998001	4.00178e-11\\
3.99400299850075	4.00178e-11\\
3.9960019990005	4.00178e-11\\
3.99800099950025	4.00178e-11\\
4	4.00178e-11\\
};
\addlegendentry{c2};

\addplot [color=mycolor3,solid]
  table[row sep=crcr]{%
0	6.00267e-10\\
0.00199900049975012	6.00267e-10\\
0.00399800099950025	6.00267e-10\\
0.00599700149925037	6.00267e-10\\
0.0079960019990005	6.00267e-10\\
0.00999500249875063	5.82077e-10\\
0.0119940029985007	5.74801e-10\\
0.0139930034982509	5.74801e-10\\
0.015992003998001	5.74801e-10\\
0.0179910044977511	5.78439e-10\\
0.0199900049975013	5.82077e-10\\
0.0219890054972514	5.85715e-10\\
0.0239880059970015	5.82077e-10\\
0.0259870064967516	5.96629e-10\\
0.0279860069965017	5.96629e-10\\
0.0299850074962519	6.07542e-10\\
0.031984007996002	6.22094e-10\\
0.0339830084957521	6.03904e-10\\
0.0359820089955022	6.22094e-10\\
0.0379810094952524	6.18456e-10\\
0.0399800099950025	6.18456e-10\\
0.0419790104947526	6.18456e-10\\
0.0439780109945027	6.2937e-10\\
0.0459770114942529	6.18456e-10\\
0.047976011994003	6.14818e-10\\
0.0499750124937531	6.00267e-10\\
0.0519740129935032	5.89353e-10\\
0.0539730134932534	5.89353e-10\\
0.0559720139930035	5.96629e-10\\
0.0579710144927536	5.89353e-10\\
0.0599700149925037	5.96629e-10\\
0.0619690154922539	5.85715e-10\\
0.063968015992004	5.82077e-10\\
0.0659670164917541	5.71163e-10\\
0.0679660169915042	5.82077e-10\\
0.0699650174912544	5.82077e-10\\
0.0719640179910045	5.82077e-10\\
0.0739630184907546	5.63887e-10\\
0.0759620189905048	5.63887e-10\\
0.0779610194902549	5.56611e-10\\
0.079960019990005	5.67525e-10\\
0.0819590204897551	5.67525e-10\\
0.0839580209895052	5.82077e-10\\
0.0859570214892554	5.92991e-10\\
0.0879560219890055	5.89353e-10\\
0.0899550224887556	5.85715e-10\\
0.0919540229885057	5.85715e-10\\
0.0939530234882559	6.00267e-10\\
0.095952023988006	6.00267e-10\\
0.0979510244877561	5.89353e-10\\
0.0999500249875062	6.00267e-10\\
0.101949025487256	5.89353e-10\\
0.103948025987006	5.85715e-10\\
0.105947026486757	5.89353e-10\\
0.107946026986507	5.78439e-10\\
0.109945027486257	5.85715e-10\\
0.111944027986007	5.85715e-10\\
0.113943028485757	5.78439e-10\\
0.115942028985507	5.74801e-10\\
0.117941029485257	5.85715e-10\\
0.119940029985007	5.89353e-10\\
0.121939030484758	5.82077e-10\\
0.123938030984508	5.82077e-10\\
0.125937031484258	6.00267e-10\\
0.127936031984008	5.82077e-10\\
0.129935032483758	5.74801e-10\\
0.131934032983508	5.71163e-10\\
0.133933033483258	5.67525e-10\\
0.135932033983008	5.56611e-10\\
0.137931034482759	5.56611e-10\\
0.139930034982509	5.60249e-10\\
0.141929035482259	5.56611e-10\\
0.143928035982009	5.52973e-10\\
0.145927036481759	5.63887e-10\\
0.147926036981509	5.67525e-10\\
0.149925037481259	5.67525e-10\\
0.15192403798101	5.74801e-10\\
0.15392303848076	5.78439e-10\\
0.15592203898051	5.74801e-10\\
0.15792103948026	5.74801e-10\\
0.15992003998001	5.78439e-10\\
0.16191904047976	5.92991e-10\\
0.16391804097951	5.78439e-10\\
0.16591704147926	5.96629e-10\\
0.16791604197901	6.00267e-10\\
0.169915042478761	6.00267e-10\\
0.171914042978511	6.00267e-10\\
0.173913043478261	6.00267e-10\\
0.175912043978011	6.00267e-10\\
0.177911044477761	6.00267e-10\\
0.179910044977511	6.07542e-10\\
0.181909045477261	6.07542e-10\\
0.183908045977011	6.03904e-10\\
0.185907046476762	5.92991e-10\\
0.187906046976512	5.78439e-10\\
0.189905047476262	5.74801e-10\\
0.191904047976012	5.78439e-10\\
0.193903048475762	5.96629e-10\\
0.195902048975512	5.89353e-10\\
0.197901049475262	6.00267e-10\\
0.199900049975012	6.07542e-10\\
0.201899050474763	6.14818e-10\\
0.203898050974513	6.22094e-10\\
0.205897051474263	6.14818e-10\\
0.207896051974013	6.22094e-10\\
0.209895052473763	6.18456e-10\\
0.211894052973513	6.22094e-10\\
0.213893053473263	6.18456e-10\\
0.215892053973014	6.18456e-10\\
0.217891054472764	6.18456e-10\\
0.219890054972514	6.18456e-10\\
0.221889055472264	6.14818e-10\\
0.223888055972014	5.96629e-10\\
0.225887056471764	5.89353e-10\\
0.227886056971514	5.85715e-10\\
0.229885057471264	5.78439e-10\\
0.231884057971014	5.74801e-10\\
0.233883058470765	5.74801e-10\\
0.235882058970515	5.63887e-10\\
0.237881059470265	5.60249e-10\\
0.239880059970015	5.63887e-10\\
0.241879060469765	5.63887e-10\\
0.243878060969515	5.60249e-10\\
0.245877061469265	5.60249e-10\\
0.247876061969015	5.60249e-10\\
0.249875062468766	5.45697e-10\\
0.251874062968516	5.45697e-10\\
0.253873063468266	5.38421e-10\\
0.255872063968016	5.34783e-10\\
0.257871064467766	5.31145e-10\\
0.259870064967516	5.31145e-10\\
0.261869065467266	5.34783e-10\\
0.263868065967017	5.31145e-10\\
0.265867066466767	5.23869e-10\\
0.267866066966517	5.31145e-10\\
0.269865067466267	5.31145e-10\\
0.271864067966017	5.27507e-10\\
0.273863068465767	5.20231e-10\\
0.275862068965517	5.23869e-10\\
0.277861069465267	5.20231e-10\\
0.279860069965017	5.12955e-10\\
0.281859070464768	5.20231e-10\\
0.283858070964518	5.20231e-10\\
0.285857071464268	5.16593e-10\\
0.287856071964018	5.12955e-10\\
0.289855072463768	5.27507e-10\\
0.291854072963518	5.31145e-10\\
0.293853073463268	5.16593e-10\\
0.295852073963018	5.09317e-10\\
0.297851074462769	5.09317e-10\\
0.299850074962519	4.98403e-10\\
0.301849075462269	4.98403e-10\\
0.303848075962019	4.83851e-10\\
0.305847076461769	4.94765e-10\\
0.307846076961519	4.98403e-10\\
0.309845077461269	4.94765e-10\\
0.311844077961019	4.98403e-10\\
0.31384307846077	4.98403e-10\\
0.31584207896052	5.02041e-10\\
0.31784107946027	4.98403e-10\\
0.31984007996002	5.02041e-10\\
0.32183908045977	5.02041e-10\\
0.32383808095952	5.02041e-10\\
0.32583708145927	5.02041e-10\\
0.32783608195902	5.09317e-10\\
0.329835082458771	5.09317e-10\\
0.331834082958521	5.09317e-10\\
0.333833083458271	5.02041e-10\\
0.335832083958021	4.98403e-10\\
0.337831084457771	4.98403e-10\\
0.339830084957521	4.98403e-10\\
0.341829085457271	4.98403e-10\\
0.343828085957021	4.80213e-10\\
0.345827086456772	4.94765e-10\\
0.347826086956522	4.98403e-10\\
0.349825087456272	4.98403e-10\\
0.351824087956022	4.94765e-10\\
0.353823088455772	4.94765e-10\\
0.355822088955522	4.94765e-10\\
0.357821089455272	5.02041e-10\\
0.359820089955023	4.94765e-10\\
0.361819090454773	4.80213e-10\\
0.363818090954523	4.80213e-10\\
0.365817091454273	4.80213e-10\\
0.367816091954023	4.80213e-10\\
0.369815092453773	4.76575e-10\\
0.371814092953523	4.87489e-10\\
0.373813093453273	4.91127e-10\\
0.375812093953024	4.94765e-10\\
0.377811094452774	5.05679e-10\\
0.379810094952524	5.02041e-10\\
0.381809095452274	4.98403e-10\\
0.383808095952024	4.91127e-10\\
0.385807096451774	4.91127e-10\\
0.387806096951524	4.94765e-10\\
0.389805097451274	5.02041e-10\\
0.391804097951024	5.02041e-10\\
0.393803098450775	4.98403e-10\\
0.395802098950525	4.98403e-10\\
0.397801099450275	5.02041e-10\\
0.399800099950025	5.05679e-10\\
0.401799100449775	5.02041e-10\\
0.403798100949525	5.05679e-10\\
0.405797101449275	5.05679e-10\\
0.407796101949025	5.05679e-10\\
0.409795102448776	5.05679e-10\\
0.411794102948526	5.09317e-10\\
0.413793103448276	5.09317e-10\\
0.415792103948026	5.05679e-10\\
0.417791104447776	5.02041e-10\\
0.419790104947526	5.02041e-10\\
0.421789105447276	5.05679e-10\\
0.423788105947026	5.02041e-10\\
0.425787106446777	4.98403e-10\\
0.427786106946527	5.05679e-10\\
0.429785107446277	5.02041e-10\\
0.431784107946027	5.16593e-10\\
0.433783108445777	5.16593e-10\\
0.435782108945527	5.16593e-10\\
0.437781109445277	5.12955e-10\\
0.439780109945027	5.16593e-10\\
0.441779110444778	5.20231e-10\\
0.443778110944528	5.12955e-10\\
0.445777111444278	5.16593e-10\\
0.447776111944028	5.12955e-10\\
0.449775112443778	5.02041e-10\\
0.451774112943528	5.02041e-10\\
0.453773113443278	5.02041e-10\\
0.455772113943028	4.98403e-10\\
0.457771114442779	4.98403e-10\\
0.459770114942529	5.02041e-10\\
0.461769115442279	4.98403e-10\\
0.463768115942029	5.02041e-10\\
0.465767116441779	4.98403e-10\\
0.467766116941529	4.87489e-10\\
0.469765117441279	4.87489e-10\\
0.471764117941029	4.83851e-10\\
0.47376311844078	4.98403e-10\\
0.47576211894053	4.98403e-10\\
0.47776111944028	4.94765e-10\\
0.47976011994003	4.94765e-10\\
0.48175912043978	4.98403e-10\\
0.48375812093953	5.02041e-10\\
0.48575712143928	5.05679e-10\\
0.487756121939031	5.02041e-10\\
0.489755122438781	5.05679e-10\\
0.491754122938531	4.98403e-10\\
0.493753123438281	4.98403e-10\\
0.495752123938031	4.91127e-10\\
0.497751124437781	4.87489e-10\\
0.499750124937531	5.02041e-10\\
0.501749125437281	4.94765e-10\\
0.503748125937031	5.02041e-10\\
0.505747126436782	4.83851e-10\\
0.507746126936532	5.02041e-10\\
0.509745127436282	4.87489e-10\\
0.511744127936032	4.80213e-10\\
0.513743128435782	4.83851e-10\\
0.515742128935532	4.80213e-10\\
0.517741129435282	4.83851e-10\\
0.519740129935032	4.87489e-10\\
0.521739130434783	4.76575e-10\\
0.523738130934533	4.76575e-10\\
0.525737131434283	4.69299e-10\\
0.527736131934033	4.72937e-10\\
0.529735132433783	4.69299e-10\\
0.531734132933533	4.69299e-10\\
0.533733133433283	4.69299e-10\\
0.535732133933034	4.69299e-10\\
0.537731134432784	4.69299e-10\\
0.539730134932534	4.65661e-10\\
0.541729135432284	4.69299e-10\\
0.543728135932034	4.65661e-10\\
0.545727136431784	4.65661e-10\\
0.547726136931534	4.62023e-10\\
0.549725137431284	4.72937e-10\\
0.551724137931034	4.54747e-10\\
0.553723138430785	4.58385e-10\\
0.555722138930535	4.58385e-10\\
0.557721139430285	4.54747e-10\\
0.559720139930035	4.58385e-10\\
0.561719140429785	4.58385e-10\\
0.563718140929535	4.54747e-10\\
0.565717141429285	4.58385e-10\\
0.567716141929036	4.65661e-10\\
0.569715142428786	4.58385e-10\\
0.571714142928536	4.62023e-10\\
0.573713143428286	4.58385e-10\\
0.575712143928036	4.58385e-10\\
0.577711144427786	4.58385e-10\\
0.579710144927536	4.62023e-10\\
0.581709145427286	4.51109e-10\\
0.583708145927036	4.54747e-10\\
0.585707146426787	4.58385e-10\\
0.587706146926537	4.58385e-10\\
0.589705147426287	4.51109e-10\\
0.591704147926037	4.40195e-10\\
0.593703148425787	4.40195e-10\\
0.595702148925537	4.36557e-10\\
0.597701149425287	4.32919e-10\\
0.599700149925038	4.29281e-10\\
0.601699150424788	4.36557e-10\\
0.603698150924538	4.25644e-10\\
0.605697151424288	4.25644e-10\\
0.607696151924038	4.22006e-10\\
0.609695152423788	4.22006e-10\\
0.611694152923538	4.25644e-10\\
0.613693153423288	4.25644e-10\\
0.615692153923038	4.25644e-10\\
0.617691154422789	4.22006e-10\\
0.619690154922539	4.22006e-10\\
0.621689155422289	4.18368e-10\\
0.623688155922039	4.03816e-10\\
0.625687156421789	4.03816e-10\\
0.627686156921539	4.00178e-10\\
0.629685157421289	4.00178e-10\\
0.631684157921039	4.00178e-10\\
0.63368315842079	3.9654e-10\\
0.63568215892054	4.1473e-10\\
0.63768115942029	4.1473e-10\\
0.63968015992004	4.1473e-10\\
0.64167916041979	4.18368e-10\\
0.64367816091954	4.1473e-10\\
0.64567716141929	4.03816e-10\\
0.64767616191904	4.07454e-10\\
0.649675162418791	4.03816e-10\\
0.651674162918541	4.03816e-10\\
0.653673163418291	4.03816e-10\\
0.655672163918041	4.03816e-10\\
0.657671164417791	4.00178e-10\\
0.659670164917541	4.00178e-10\\
0.661669165417291	3.92902e-10\\
0.663668165917041	3.9654e-10\\
0.665667166416792	4.03816e-10\\
0.667666166916542	4.03816e-10\\
0.669665167416292	4.03816e-10\\
0.671664167916042	4.00178e-10\\
0.673663168415792	4.03816e-10\\
0.675662168915542	4.03816e-10\\
0.677661169415292	4.07454e-10\\
0.679660169915043	4.07454e-10\\
0.681659170414793	4.11092e-10\\
0.683658170914543	4.07454e-10\\
0.685657171414293	4.1473e-10\\
0.687656171914043	4.07454e-10\\
0.689655172413793	4.07454e-10\\
0.691654172913543	4.03816e-10\\
0.693653173413293	4.1473e-10\\
0.695652173913043	4.18368e-10\\
0.697651174412794	4.25644e-10\\
0.699650174912544	4.25644e-10\\
0.701649175412294	4.25644e-10\\
0.703648175912044	4.18368e-10\\
0.705647176411794	4.1473e-10\\
0.707646176911544	4.18368e-10\\
0.709645177411294	4.18368e-10\\
0.711644177911045	4.1473e-10\\
0.713643178410795	4.22006e-10\\
0.715642178910545	4.22006e-10\\
0.717641179410295	4.25644e-10\\
0.719640179910045	4.1473e-10\\
0.721639180409795	4.18368e-10\\
0.723638180909545	4.18368e-10\\
0.725637181409295	4.1473e-10\\
0.727636181909045	4.11092e-10\\
0.729635182408796	4.11092e-10\\
0.731634182908546	4.11092e-10\\
0.733633183408296	4.11092e-10\\
0.735632183908046	4.1473e-10\\
0.737631184407796	4.11092e-10\\
0.739630184907546	4.11092e-10\\
0.741629185407296	4.03816e-10\\
0.743628185907046	4.11092e-10\\
0.745627186406797	4.11092e-10\\
0.747626186906547	4.11092e-10\\
0.749625187406297	4.11092e-10\\
0.751624187906047	4.22006e-10\\
0.753623188405797	4.11092e-10\\
0.755622188905547	4.1473e-10\\
0.757621189405297	4.18368e-10\\
0.759620189905047	4.1473e-10\\
0.761619190404798	4.11092e-10\\
0.763618190904548	4.11092e-10\\
0.765617191404298	4.1473e-10\\
0.767616191904048	4.1473e-10\\
0.769615192403798	4.11092e-10\\
0.771614192903548	4.07454e-10\\
0.773613193403298	4.07454e-10\\
0.775612193903048	4.07454e-10\\
0.777611194402799	4.03816e-10\\
0.779610194902549	3.92902e-10\\
0.781609195402299	3.92902e-10\\
0.783608195902049	3.92902e-10\\
0.785607196401799	3.92902e-10\\
0.787606196901549	3.9654e-10\\
0.789605197401299	3.9654e-10\\
0.79160419790105	4.00178e-10\\
0.7936031984008	4.00178e-10\\
0.79560219890055	4.00178e-10\\
0.7976011994003	4.03816e-10\\
0.79960019990005	4.11092e-10\\
0.8015992003998	4.00178e-10\\
0.80359820089955	4.00178e-10\\
0.8055972013993	3.9654e-10\\
0.80759620189905	4.00178e-10\\
0.809595202398801	4.00178e-10\\
0.811594202898551	4.00178e-10\\
0.813593203398301	4.03816e-10\\
0.815592203898051	4.03816e-10\\
0.817591204397801	4.07454e-10\\
0.819590204897551	4.07454e-10\\
0.821589205397301	4.07454e-10\\
0.823588205897052	4.07454e-10\\
0.825587206396802	4.11092e-10\\
0.827586206896552	4.03816e-10\\
0.829585207396302	4.11092e-10\\
0.831584207896052	4.11092e-10\\
0.833583208395802	4.11092e-10\\
0.835582208895552	4.03816e-10\\
0.837581209395302	4.07454e-10\\
0.839580209895052	4.1473e-10\\
0.841579210394803	3.9654e-10\\
0.843578210894553	4.00178e-10\\
0.845577211394303	4.03816e-10\\
0.847576211894053	4.03816e-10\\
0.849575212393803	4.11092e-10\\
0.851574212893553	4.25644e-10\\
0.853573213393303	4.11092e-10\\
0.855572213893053	4.25644e-10\\
0.857571214392804	4.25644e-10\\
0.859570214892554	4.1473e-10\\
0.861569215392304	4.18368e-10\\
0.863568215892054	4.18368e-10\\
0.865567216391804	4.22006e-10\\
0.867566216891554	4.22006e-10\\
0.869565217391304	4.1473e-10\\
0.871564217891054	4.11092e-10\\
0.873563218390805	4.07454e-10\\
0.875562218890555	4.11092e-10\\
0.877561219390305	4.03816e-10\\
0.879560219890055	4.00178e-10\\
0.881559220389805	4.00178e-10\\
0.883558220889555	4.00178e-10\\
0.885557221389305	4.03816e-10\\
0.887556221889055	4.00178e-10\\
0.889555222388806	4.07454e-10\\
0.891554222888556	4.07454e-10\\
0.893553223388306	4.03816e-10\\
0.895552223888056	4.11092e-10\\
0.897551224387806	4.00178e-10\\
0.899550224887556	4.00178e-10\\
0.901549225387306	3.9654e-10\\
0.903548225887057	3.9654e-10\\
0.905547226386807	3.9654e-10\\
0.907546226886557	4.07454e-10\\
0.909545227386307	4.07454e-10\\
0.911544227886057	4.00178e-10\\
0.913543228385807	4.03816e-10\\
0.915542228885557	4.00178e-10\\
0.917541229385307	3.9654e-10\\
0.919540229885057	3.89264e-10\\
0.921539230384808	3.89264e-10\\
0.923538230884558	4.00178e-10\\
0.925537231384308	3.9654e-10\\
0.927536231884058	3.89264e-10\\
0.929535232383808	3.9654e-10\\
0.931534232883558	3.92902e-10\\
0.933533233383308	4.03816e-10\\
0.935532233883059	3.89264e-10\\
0.937531234382809	3.89264e-10\\
0.939530234882559	3.85626e-10\\
0.941529235382309	3.81988e-10\\
0.943528235882059	3.7835e-10\\
0.945527236381809	3.81988e-10\\
0.947526236881559	3.7835e-10\\
0.949525237381309	3.7835e-10\\
0.951524237881059	3.89264e-10\\
0.95352323838081	3.89264e-10\\
0.95552223888056	3.92902e-10\\
0.95752123938031	3.92902e-10\\
0.95952023988006	3.9654e-10\\
0.96151924037981	4.07454e-10\\
0.96351824087956	4.07454e-10\\
0.96551724137931	4.03816e-10\\
0.96751624187906	4.11092e-10\\
0.969515242378811	4.11092e-10\\
0.971514242878561	4.07454e-10\\
0.973513243378311	4.11092e-10\\
0.975512243878061	4.25644e-10\\
0.977511244377811	4.22006e-10\\
0.979510244877561	4.32919e-10\\
0.981509245377311	4.1473e-10\\
0.983508245877061	4.29281e-10\\
0.985507246376812	4.18368e-10\\
0.987506246876562	4.32919e-10\\
0.989505247376312	4.32919e-10\\
0.991504247876062	4.25644e-10\\
0.993503248375812	4.29281e-10\\
0.995502248875562	4.29281e-10\\
0.997501249375312	4.36557e-10\\
0.999500249875062	4.40195e-10\\
1.00149925037481	4.40195e-10\\
1.00349825087456	4.29281e-10\\
1.00549725137431	4.36557e-10\\
1.00749625187406	4.40195e-10\\
1.00949525237381	4.32919e-10\\
1.01149425287356	4.32919e-10\\
1.01349325337331	4.25644e-10\\
1.01549225387306	4.18368e-10\\
1.01749125437281	4.22006e-10\\
1.01949025487256	4.22006e-10\\
1.02148925537231	4.25644e-10\\
1.02348825587206	4.36557e-10\\
1.02548725637181	4.32919e-10\\
1.02748625687156	4.32919e-10\\
1.02948525737131	4.29281e-10\\
1.03148425787106	4.43833e-10\\
1.03348325837081	4.40195e-10\\
1.03548225887056	4.43833e-10\\
1.03748125937031	4.29281e-10\\
1.03948025987006	4.1473e-10\\
1.04147926036982	4.18368e-10\\
1.04347826086957	4.1473e-10\\
1.04547726136932	4.22006e-10\\
1.04747626186907	4.11092e-10\\
1.04947526236882	4.1473e-10\\
1.05147426286857	4.1473e-10\\
1.05347326336832	4.1473e-10\\
1.05547226386807	4.11092e-10\\
1.05747126436782	4.07454e-10\\
1.05947026486757	4.18368e-10\\
1.06146926536732	4.22006e-10\\
1.06346826586707	4.36557e-10\\
1.06546726636682	4.36557e-10\\
1.06746626686657	4.29281e-10\\
1.06946526736632	4.25644e-10\\
1.07146426786607	4.29281e-10\\
1.07346326836582	4.32919e-10\\
1.07546226886557	4.32919e-10\\
1.07746126936532	4.22006e-10\\
1.07946026986507	4.32919e-10\\
1.08145927036482	4.32919e-10\\
1.08345827086457	4.32919e-10\\
1.08545727136432	4.22006e-10\\
1.08745627186407	4.22006e-10\\
1.08945527236382	4.07454e-10\\
1.09145427286357	4.07454e-10\\
1.09345327336332	4.11092e-10\\
1.09545227386307	4.07454e-10\\
1.09745127436282	4.00178e-10\\
1.09945027486257	3.89264e-10\\
1.10144927536232	3.85626e-10\\
1.10344827586207	3.7835e-10\\
1.10544727636182	3.71074e-10\\
1.10744627686157	3.56522e-10\\
1.10944527736132	3.52884e-10\\
1.11144427786107	3.4197e-10\\
1.11344327836082	3.27418e-10\\
1.11544227886057	3.38332e-10\\
1.11744127936032	3.38332e-10\\
1.11944027986007	3.4197e-10\\
1.12143928035982	3.45608e-10\\
1.12343828085957	3.45608e-10\\
1.12543728135932	3.45608e-10\\
1.12743628185907	3.4197e-10\\
1.12943528235882	3.31056e-10\\
1.13143428285857	3.31056e-10\\
1.13343328335832	3.16504e-10\\
1.13543228385807	3.27418e-10\\
1.13743128435782	3.4197e-10\\
1.13943028485757	3.45608e-10\\
1.14142928535732	3.56522e-10\\
1.14342828585707	3.49246e-10\\
1.14542728635682	3.4197e-10\\
1.14742628685657	3.56522e-10\\
1.14942528735632	3.52884e-10\\
1.15142428785607	3.45608e-10\\
1.15342328835582	3.52884e-10\\
1.15542228885557	3.49246e-10\\
1.15742128935532	3.52884e-10\\
1.15942028985507	3.56522e-10\\
1.16141929035482	3.52884e-10\\
1.16341829085457	3.49246e-10\\
1.16541729135432	3.63798e-10\\
1.16741629185407	3.52884e-10\\
1.16941529235382	3.52884e-10\\
1.17141429285357	3.52884e-10\\
1.17341329335332	3.49246e-10\\
1.17541229385307	3.4197e-10\\
1.17741129435282	3.52884e-10\\
1.17941029485257	3.71074e-10\\
1.18140929535232	3.67436e-10\\
1.18340829585207	3.67436e-10\\
1.18540729635182	3.7835e-10\\
1.18740629685157	3.74712e-10\\
1.18940529735132	3.63798e-10\\
1.19140429785107	3.74712e-10\\
1.19340329835082	3.85626e-10\\
1.19540229885057	3.81988e-10\\
1.19740129935032	3.85626e-10\\
1.19940029985008	3.9654e-10\\
1.20139930034983	3.9654e-10\\
1.20339830084958	3.9654e-10\\
1.20539730134933	4.1473e-10\\
1.20739630184908	4.22006e-10\\
1.20939530234883	4.18368e-10\\
1.21139430284858	4.11092e-10\\
1.21339330334833	4.11092e-10\\
1.21539230384808	4.11092e-10\\
1.21739130434783	4.22006e-10\\
1.21939030484758	4.22006e-10\\
1.22138930534733	4.22006e-10\\
1.22338830584708	4.22006e-10\\
1.22538730634683	4.11092e-10\\
1.22738630684658	4.22006e-10\\
1.22938530734633	4.32919e-10\\
1.23138430784608	4.43833e-10\\
1.23338330834583	4.47471e-10\\
1.23538230884558	4.58385e-10\\
1.23738130934533	4.47471e-10\\
1.23938030984508	4.58385e-10\\
1.24137931034483	4.58385e-10\\
1.24337831084458	4.65661e-10\\
1.24537731134433	4.80213e-10\\
1.24737631184408	4.76575e-10\\
1.24937531234383	4.62023e-10\\
1.25137431284358	4.62023e-10\\
1.25337331334333	4.76575e-10\\
1.25537231384308	4.76575e-10\\
1.25737131434283	4.65661e-10\\
1.25937031484258	4.76575e-10\\
1.26136931534233	4.76575e-10\\
1.26336831584208	4.83851e-10\\
1.26536731634183	4.87489e-10\\
1.26736631684158	5.02041e-10\\
1.26936531734133	4.87489e-10\\
1.27136431784108	4.83851e-10\\
1.27336331834083	4.83851e-10\\
1.27536231884058	4.83851e-10\\
1.27736131934033	4.94765e-10\\
1.27936031984008	4.83851e-10\\
1.28135932033983	4.80213e-10\\
1.28335832083958	4.94765e-10\\
1.28535732133933	4.98403e-10\\
1.28735632183908	4.98403e-10\\
1.28935532233883	4.83851e-10\\
1.29135432283858	4.72937e-10\\
1.29335332333833	4.51109e-10\\
1.29535232383808	4.54747e-10\\
1.29735132433783	4.51109e-10\\
1.29935032483758	4.40195e-10\\
1.30134932533733	4.51109e-10\\
1.30334832583708	4.51109e-10\\
1.30534732633683	4.40195e-10\\
1.30734632683658	4.36557e-10\\
1.30934532733633	4.25644e-10\\
1.31134432783608	4.25644e-10\\
1.31334332833583	4.22006e-10\\
1.31534232883558	4.07454e-10\\
1.31734132933533	4.25644e-10\\
1.31934032983508	4.11092e-10\\
1.32133933033483	4.25644e-10\\
1.32333833083458	4.11092e-10\\
1.32533733133433	4.11092e-10\\
1.32733633183408	4.07454e-10\\
1.32933533233383	4.00178e-10\\
1.33133433283358	4.11092e-10\\
1.33333333333333	4.07454e-10\\
1.33533233383308	4.11092e-10\\
1.33733133433283	4.11092e-10\\
1.33933033483258	4.11092e-10\\
1.34132933533233	4.22006e-10\\
1.34332833583208	4.25644e-10\\
1.34532733633183	4.25644e-10\\
1.34732633683158	4.51109e-10\\
1.34932533733133	4.47471e-10\\
1.35132433783108	4.25644e-10\\
1.35332333833083	4.29281e-10\\
1.35532233883058	4.18368e-10\\
1.35732133933033	4.25644e-10\\
1.35932033983009	4.29281e-10\\
1.36131934032984	4.36557e-10\\
1.36331834082959	4.32919e-10\\
1.36531734132934	4.36557e-10\\
1.36731634182909	4.51109e-10\\
1.36931534232884	4.36557e-10\\
1.37131434282859	4.32919e-10\\
1.37331334332834	4.51109e-10\\
1.37531234382809	4.58385e-10\\
1.37731134432784	4.51109e-10\\
1.37931034482759	4.51109e-10\\
1.38130934532734	4.58385e-10\\
1.38330834582709	4.51109e-10\\
1.38530734632684	4.54747e-10\\
1.38730634682659	4.36557e-10\\
1.38930534732634	4.51109e-10\\
1.39130434782609	4.51109e-10\\
1.39330334832584	4.36557e-10\\
1.39530234882559	4.25644e-10\\
1.39730134932534	4.03816e-10\\
1.39930034982509	4.00178e-10\\
1.40129935032484	3.9654e-10\\
1.40329835082459	4.00178e-10\\
1.40529735132434	4.00178e-10\\
1.40729635182409	3.9654e-10\\
1.40929535232384	3.85626e-10\\
1.41129435282359	4.00178e-10\\
1.41329335332334	3.85626e-10\\
1.41529235382309	3.81988e-10\\
1.41729135432284	3.74712e-10\\
1.41929035482259	3.74712e-10\\
1.42128935532234	3.71074e-10\\
1.42328835582209	3.71074e-10\\
1.42528735632184	3.56522e-10\\
1.42728635682159	3.45608e-10\\
1.42928535732134	3.49246e-10\\
1.43128435782109	3.45608e-10\\
1.43328335832084	3.38332e-10\\
1.43528235882059	3.27418e-10\\
1.43728135932034	3.38332e-10\\
1.43928035982009	3.38332e-10\\
1.44127936031984	3.34694e-10\\
1.44327836081959	3.34694e-10\\
1.44527736131934	3.49246e-10\\
1.44727636181909	3.63798e-10\\
1.44927536231884	3.6016e-10\\
1.45127436281859	3.52884e-10\\
1.45327336331834	3.63798e-10\\
1.45527236381809	3.63798e-10\\
1.45727136431784	3.52884e-10\\
1.45927036481759	3.63798e-10\\
1.46126936531734	3.56522e-10\\
1.46326836581709	3.52884e-10\\
1.46526736631684	3.4197e-10\\
1.46726636681659	3.56522e-10\\
1.46926536731634	3.56522e-10\\
1.47126436781609	3.63798e-10\\
1.47326336831584	3.63798e-10\\
1.47526236881559	3.52884e-10\\
1.47726136931534	3.49246e-10\\
1.47926036981509	3.63798e-10\\
1.48125937031484	3.52884e-10\\
1.48325837081459	3.63798e-10\\
1.48525737131434	3.52884e-10\\
1.48725637181409	3.52884e-10\\
1.48925537231384	3.52884e-10\\
1.49125437281359	3.49246e-10\\
1.49325337331334	3.63798e-10\\
1.49525237381309	3.74712e-10\\
1.49725137431284	3.71074e-10\\
1.49925037481259	3.6016e-10\\
1.50124937531234	3.56522e-10\\
1.50324837581209	3.67436e-10\\
1.50524737631184	3.71074e-10\\
1.50724637681159	3.7835e-10\\
1.50924537731134	3.7835e-10\\
1.51124437781109	3.74712e-10\\
1.51324337831084	3.67436e-10\\
1.51524237881059	3.85626e-10\\
1.51724137931034	3.71074e-10\\
1.51924037981009	3.6016e-10\\
1.52123938030985	3.4197e-10\\
1.5232383808096	3.4197e-10\\
1.52523738130935	3.45608e-10\\
1.5272363818091	3.31056e-10\\
1.52923538230885	3.16504e-10\\
1.5312343828086	3.16504e-10\\
1.53323338330835	3.16504e-10\\
1.5352323838081	2.98314e-10\\
1.53723138430785	3.01952e-10\\
1.5392303848076	3.01952e-10\\
1.54122938530735	3.16504e-10\\
1.5432283858071	3.16504e-10\\
1.54522738630685	3.12866e-10\\
1.5472263868066	3.12866e-10\\
1.54922538730635	2.98314e-10\\
1.5512243878061	2.98314e-10\\
1.55322338830585	3.12866e-10\\
1.5552223888056	3.16504e-10\\
1.55722138930535	3.01952e-10\\
1.5592203898051	3.12866e-10\\
1.56121939030485	2.94676e-10\\
1.5632183908046	2.91038e-10\\
1.56521739130435	2.874e-10\\
1.5672163918041	2.91038e-10\\
1.56921539230385	2.91038e-10\\
1.5712143928036	2.98314e-10\\
1.57321339330335	2.874e-10\\
1.5752123938031	2.874e-10\\
1.57721139430285	2.94676e-10\\
1.5792103948026	2.874e-10\\
1.58120939530235	2.874e-10\\
1.5832083958021	2.874e-10\\
1.58520739630185	2.6921e-10\\
1.5872063968016	2.65572e-10\\
1.58920539730135	2.65572e-10\\
1.5912043978011	2.65572e-10\\
1.59320339830085	2.72848e-10\\
1.5952023988006	2.65572e-10\\
1.59720139930035	2.51021e-10\\
1.5992003998001	2.43745e-10\\
1.60119940029985	2.51021e-10\\
1.6031984007996	2.40107e-10\\
1.60519740129935	2.36469e-10\\
1.6071964017991	2.36469e-10\\
1.60919540229885	2.21917e-10\\
1.6111944027986	2.21917e-10\\
1.61319340329835	2.18279e-10\\
1.6151924037981	2.32831e-10\\
1.61719140429785	2.18279e-10\\
1.6191904047976	2.32831e-10\\
1.62118940529735	2.36469e-10\\
1.6231884057971	2.32831e-10\\
1.62518740629685	2.29193e-10\\
1.6271864067966	2.18279e-10\\
1.62918540729635	2.18279e-10\\
1.6311844077961	2.29193e-10\\
1.63318340829585	2.40107e-10\\
1.6351824087956	2.43745e-10\\
1.63718140929535	2.36469e-10\\
1.6391804097951	2.21917e-10\\
1.64117941029485	2.18279e-10\\
1.6431784107946	2.07365e-10\\
1.64517741129435	2.18279e-10\\
1.6471764117941	2.07365e-10\\
1.64917541229385	2.07365e-10\\
1.6511744127936	2.29193e-10\\
1.65317341329335	2.18279e-10\\
1.6551724137931	2.32831e-10\\
1.65717141429285	2.32831e-10\\
1.6591704147926	2.43745e-10\\
1.66116941529235	2.40107e-10\\
1.6631684157921	2.43745e-10\\
1.66516741629185	2.43745e-10\\
1.6671664167916	2.54659e-10\\
1.66916541729135	2.58296e-10\\
1.6711644177911	2.58296e-10\\
1.67316341829085	2.58296e-10\\
1.6751624187906	2.58296e-10\\
1.67716141929035	2.58296e-10\\
1.6791604197901	2.43745e-10\\
1.68115942028985	2.58296e-10\\
1.68315842078961	2.58296e-10\\
1.68515742128936	2.43745e-10\\
1.68715642178911	2.58296e-10\\
1.68915542228886	2.58296e-10\\
1.69115442278861	2.43745e-10\\
1.69315342328836	2.58296e-10\\
1.69515242378811	2.61934e-10\\
1.69715142428786	2.65572e-10\\
1.69915042478761	2.61934e-10\\
1.70114942528736	2.61934e-10\\
1.70314842578711	2.6921e-10\\
1.70514742628686	2.65572e-10\\
1.70714642678661	2.61934e-10\\
1.70914542728636	2.65572e-10\\
1.71114442778611	2.83762e-10\\
1.71314342828586	2.83762e-10\\
1.71514242878561	2.6921e-10\\
1.71714142928536	2.65572e-10\\
1.71914042978511	2.47383e-10\\
1.72113943028486	2.36469e-10\\
1.72313843078461	2.40107e-10\\
1.72513743128436	2.40107e-10\\
1.72713643178411	2.36469e-10\\
1.72913543228386	2.21917e-10\\
1.73113443278361	2.14641e-10\\
1.73313343328336	2.14641e-10\\
1.73513243378311	2.11003e-10\\
1.73713143428286	1.96451e-10\\
1.73913043478261	1.96451e-10\\
1.74112943528236	1.81899e-10\\
1.74312843578211	1.81899e-10\\
1.74512743628186	1.74623e-10\\
1.74712643678161	1.81899e-10\\
1.74912543728136	1.89175e-10\\
1.75112443778111	1.92813e-10\\
1.75312343828086	1.89175e-10\\
1.75512243878061	1.89175e-10\\
1.75712143928036	1.89175e-10\\
1.75912043978011	1.92813e-10\\
1.76111944027986	1.81899e-10\\
1.76311844077961	1.85537e-10\\
1.76511744127936	1.74623e-10\\
1.76711644177911	1.89175e-10\\
1.76911544227886	1.96451e-10\\
1.77111444277861	2.07365e-10\\
1.77311344327836	2.07365e-10\\
1.77511244377811	2.03727e-10\\
1.77711144427786	2.03727e-10\\
1.77911044477761	2.03727e-10\\
1.78110944527736	2.18279e-10\\
1.78310844577711	2.25555e-10\\
1.78510744627686	2.29193e-10\\
1.78710644677661	2.29193e-10\\
1.78910544727636	2.32831e-10\\
1.79110444777611	2.29193e-10\\
1.79310344827586	2.18279e-10\\
1.79510244877561	2.18279e-10\\
1.79710144927536	2.03727e-10\\
1.79910044977511	1.89175e-10\\
1.80109945027486	1.81899e-10\\
1.80309845077461	1.81899e-10\\
1.80509745127436	1.78261e-10\\
1.80709645177411	1.63709e-10\\
1.80909545227386	1.63709e-10\\
1.81109445277361	1.78261e-10\\
1.81309345327336	1.74623e-10\\
1.81509245377311	1.67347e-10\\
1.81709145427286	1.63709e-10\\
1.81909045477261	1.52795e-10\\
1.82108945527236	1.60071e-10\\
1.82308845577211	1.63709e-10\\
1.82508745627186	1.60071e-10\\
1.82708645677161	1.67347e-10\\
1.82908545727136	1.60071e-10\\
1.83108445777111	1.41881e-10\\
1.83308345827086	1.34605e-10\\
1.83508245877061	1.45519e-10\\
1.83708145927036	1.60071e-10\\
1.83908045977011	1.45519e-10\\
1.84107946026987	1.45519e-10\\
1.84307846076962	1.34605e-10\\
1.84507746126937	1.38243e-10\\
1.84707646176912	1.34605e-10\\
1.84907546226887	1.38243e-10\\
1.85107446276862	1.27329e-10\\
1.85307346326837	1.41881e-10\\
1.85507246376812	1.52795e-10\\
1.85707146426787	1.52795e-10\\
1.85907046476762	1.41881e-10\\
1.86106946526737	1.41881e-10\\
1.86306846576712	1.27329e-10\\
1.86506746626687	1.23691e-10\\
1.86706646676662	1.27329e-10\\
1.86906546726637	1.27329e-10\\
1.87106446776612	1.23691e-10\\
1.87306346826587	1.23691e-10\\
1.87506246876562	1.27329e-10\\
1.87706146926537	1.27329e-10\\
1.87906046976512	1.27329e-10\\
1.88105947026487	1.23691e-10\\
1.88305847076462	1.23691e-10\\
1.88505747126437	1.12777e-10\\
1.88705647176412	1.12777e-10\\
1.88905547226387	1.09139e-10\\
1.89105447276362	1.23691e-10\\
1.89305347326337	1.16415e-10\\
1.89505247376312	1.23691e-10\\
1.89705147426287	1.09139e-10\\
1.89905047476262	1.09139e-10\\
1.90104947526237	9.09495e-11\\
1.90304847576212	9.09495e-11\\
1.90504747626187	8.36735e-11\\
1.90704647676162	9.45874e-11\\
1.90904547726137	9.09495e-11\\
1.91104447776112	9.09495e-11\\
1.91304347826087	9.09495e-11\\
1.91504247876062	9.09495e-11\\
1.91704147926037	8.36735e-11\\
1.91904047976012	8.36735e-11\\
1.92103948025987	9.09495e-11\\
1.92303848075962	8.00355e-11\\
1.92503748125937	8.73115e-11\\
1.92703648175912	8.36735e-11\\
1.92903548225887	8.00355e-11\\
1.93103448275862	8.00355e-11\\
1.93303348325837	8.36735e-11\\
1.93503248375812	8.73115e-11\\
1.93703148425787	9.09495e-11\\
1.93903048475762	8.73115e-11\\
1.94102948525737	8.73115e-11\\
1.94302848575712	8.73115e-11\\
1.94502748625687	8.73115e-11\\
1.94702648675662	7.63976e-11\\
1.94902548725637	8.00355e-11\\
1.95102448775612	8.36735e-11\\
1.95302348825587	8.73115e-11\\
1.95502248875562	9.09495e-11\\
1.95702148925537	9.45874e-11\\
1.95902048975512	9.09495e-11\\
1.96101949025487	8.73115e-11\\
1.96301849075462	8.73115e-11\\
1.96501749125437	8.73115e-11\\
1.96701649175412	9.82254e-11\\
1.96901549225387	9.82254e-11\\
1.97101449275362	9.09495e-11\\
1.97301349325337	9.45874e-11\\
1.97501249375312	9.45874e-11\\
1.97701149425287	1.09139e-10\\
1.97901049475262	1.09139e-10\\
1.98100949525237	1.05501e-10\\
1.98300849575212	1.20053e-10\\
1.98500749625187	1.23691e-10\\
1.98700649675162	1.23691e-10\\
1.98900549725137	1.20053e-10\\
1.99100449775112	1.23691e-10\\
1.99300349825087	1.20053e-10\\
1.99500249875062	1.23691e-10\\
1.99700149925037	1.12777e-10\\
1.99900049975012	1.12777e-10\\
2.00099950024988	1.09139e-10\\
2.00299850074963	9.45874e-11\\
2.00499750124938	1.01863e-10\\
2.00699650174913	1.05501e-10\\
2.00899550224888	9.09495e-11\\
2.01099450274863	8.73115e-11\\
2.01299350324838	9.45874e-11\\
2.01499250374813	8.36735e-11\\
2.01699150424788	8.00355e-11\\
2.01899050474763	7.27596e-11\\
2.02098950524738	8.00355e-11\\
2.02298850574713	8.73115e-11\\
2.02498750624688	8.73115e-11\\
2.02698650674663	8.73115e-11\\
2.02898550724638	9.09495e-11\\
2.03098450774613	9.45874e-11\\
2.03298350824588	9.45874e-11\\
2.03498250874563	1.01863e-10\\
2.03698150924538	9.09495e-11\\
2.03898050974513	9.09495e-11\\
2.04097951024488	1.01863e-10\\
2.04297851074463	9.45874e-11\\
2.04497751124438	9.09495e-11\\
2.04697651174413	9.45874e-11\\
2.04897551224388	8.73115e-11\\
2.05097451274363	9.09495e-11\\
2.05297351324338	8.73115e-11\\
2.05497251374313	8.36735e-11\\
2.05697151424288	7.63976e-11\\
2.05897051474263	7.27596e-11\\
2.06096951524238	7.27596e-11\\
2.06296851574213	6.91216e-11\\
2.06496751624188	6.54836e-11\\
2.06696651674163	5.82077e-11\\
2.06896551724138	5.45697e-11\\
2.07096451774113	5.09317e-11\\
2.07296351824088	4.00178e-11\\
2.07496251874063	4.36557e-11\\
2.07696151924038	4.72937e-11\\
2.07896051974013	4.00178e-11\\
2.08095952023988	3.63798e-11\\
2.08295852073963	5.45697e-11\\
2.08495752123938	5.09317e-11\\
2.08695652173913	5.09317e-11\\
2.08895552223888	4.36557e-11\\
2.09095452273863	4.72937e-11\\
2.09295352323838	5.45697e-11\\
2.09495252373813	4.00178e-11\\
2.09695152423788	5.45697e-11\\
2.09895052473763	4.36557e-11\\
2.10094952523738	4.72937e-11\\
2.10294852573713	4.72937e-11\\
2.10494752623688	4.72937e-11\\
2.10694652673663	5.82077e-11\\
2.10894552723638	4.72937e-11\\
2.11094452773613	5.09317e-11\\
2.11294352823588	5.45697e-11\\
2.11494252873563	5.82077e-11\\
2.11694152923538	6.54836e-11\\
2.11894052973513	6.18456e-11\\
2.12093953023488	5.45697e-11\\
2.12293853073463	5.82077e-11\\
2.12493753123438	5.09317e-11\\
2.12693653173413	5.09317e-11\\
2.12893553223388	4.72937e-11\\
2.13093453273363	4.72937e-11\\
2.13293353323338	4.36557e-11\\
2.13493253373313	4.72937e-11\\
2.13693153423288	5.09317e-11\\
2.13893053473263	3.63798e-11\\
2.14092953523238	3.63798e-11\\
2.14292853573213	5.09317e-11\\
2.14492753623188	5.45697e-11\\
2.14692653673163	5.09317e-11\\
2.14892553723138	5.09317e-11\\
2.15092453773113	6.18456e-11\\
2.15292353823088	6.54836e-11\\
2.15492253873063	6.54836e-11\\
2.15692153923038	6.91216e-11\\
2.15892053973013	6.91216e-11\\
2.16091954022989	6.91216e-11\\
2.16291854072964	7.63976e-11\\
2.16491754122939	6.91216e-11\\
2.16691654172914	6.91216e-11\\
2.16891554222889	7.27596e-11\\
2.17091454272864	6.54836e-11\\
2.17291354322839	6.54836e-11\\
2.17491254372814	6.18456e-11\\
2.17691154422789	6.54836e-11\\
2.17891054472764	6.54836e-11\\
2.18090954522739	6.54836e-11\\
2.18290854572714	5.09317e-11\\
2.18490754622689	5.82077e-11\\
2.18690654672664	5.45697e-11\\
2.18890554722639	6.18456e-11\\
2.19090454772614	5.82077e-11\\
2.19290354822589	4.72937e-11\\
2.19490254872564	5.09317e-11\\
2.19690154922539	4.72937e-11\\
2.19890054972514	5.09317e-11\\
2.20089955022489	4.36557e-11\\
2.20289855072464	4.72937e-11\\
2.20489755122439	4.36557e-11\\
2.20689655172414	4.36557e-11\\
2.20889555222389	3.63798e-11\\
2.21089455272364	3.63798e-11\\
2.21289355322339	4.00178e-11\\
2.21489255372314	3.63798e-11\\
2.21689155422289	2.91038e-11\\
2.21889055472264	3.27418e-11\\
2.22088955522239	2.91038e-11\\
2.22288855572214	3.63798e-11\\
2.22488755622189	2.91038e-11\\
2.22688655672164	3.27418e-11\\
2.22888555722139	3.27418e-11\\
2.23088455772114	2.91038e-11\\
2.23288355822089	2.18279e-11\\
2.23488255872064	2.54659e-11\\
2.23688155922039	2.91038e-11\\
2.23888055972014	4.00178e-11\\
2.24087956021989	3.27418e-11\\
2.24287856071964	2.54659e-11\\
2.24487756121939	2.91038e-11\\
2.24687656171914	2.54659e-11\\
2.24887556221889	3.27418e-11\\
2.25087456271864	3.63798e-11\\
2.25287356321839	4.00178e-11\\
2.25487256371814	4.00178e-11\\
2.25687156421789	4.00178e-11\\
2.25887056471764	4.00178e-11\\
2.26086956521739	3.63798e-11\\
2.26286856571714	5.09317e-11\\
2.26486756621689	4.72937e-11\\
2.26686656671664	4.36557e-11\\
2.26886556721639	4.36557e-11\\
2.27086456771614	4.00178e-11\\
2.27286356821589	4.72937e-11\\
2.27486256871564	5.09317e-11\\
2.27686156921539	5.45697e-11\\
2.27886056971514	5.09317e-11\\
2.28085957021489	5.45697e-11\\
2.28285857071464	5.09317e-11\\
2.28485757121439	5.09317e-11\\
2.28685657171414	4.00178e-11\\
2.28885557221389	4.36557e-11\\
2.29085457271364	5.82077e-11\\
2.29285357321339	6.18456e-11\\
2.29485257371314	5.82077e-11\\
2.29685157421289	6.54836e-11\\
2.29885057471264	5.82077e-11\\
2.30084957521239	6.54836e-11\\
2.30284857571214	6.54836e-11\\
2.30484757621189	7.27596e-11\\
2.30684657671164	7.27596e-11\\
2.30884557721139	7.27596e-11\\
2.31084457771114	6.91216e-11\\
2.31284357821089	6.91216e-11\\
2.31484257871064	7.27596e-11\\
2.31684157921039	8.73115e-11\\
2.31884057971015	7.63976e-11\\
2.3208395802099	9.09495e-11\\
2.32283858070965	9.09495e-11\\
2.3248375812094	8.73115e-11\\
2.32683658170915	8.73115e-11\\
2.3288355822089	7.63976e-11\\
2.33083458270865	7.27596e-11\\
2.3328335832084	8.73115e-11\\
2.33483258370815	8.00355e-11\\
2.3368315842079	7.63976e-11\\
2.33883058470765	7.63976e-11\\
2.3408295852074	7.63976e-11\\
2.34282858570715	6.54836e-11\\
2.3448275862069	5.82077e-11\\
2.34682658670665	5.45697e-11\\
2.3488255872064	5.09317e-11\\
2.35082458770615	5.45697e-11\\
2.3528235882059	5.09317e-11\\
2.35482258870565	5.09317e-11\\
2.3568215892054	4.36557e-11\\
2.35882058970515	4.36557e-11\\
2.3608195902049	4.36557e-11\\
2.36281859070465	4.00178e-11\\
2.3648175912044	3.63798e-11\\
2.36681659170415	3.63798e-11\\
2.3688155922039	4.00178e-11\\
2.37081459270365	3.63798e-11\\
2.3728135932034	4.00178e-11\\
2.37481259370315	4.36557e-11\\
2.3768115942029	3.63798e-11\\
2.37881059470265	3.63798e-11\\
2.3808095952024	3.27418e-11\\
2.38280859570215	3.27418e-11\\
2.3848075962019	2.18279e-11\\
2.38680659670165	2.54659e-11\\
2.3888055972014	2.91038e-11\\
2.39080459770115	2.54659e-11\\
2.3928035982009	2.18279e-11\\
2.39480259870065	2.18279e-11\\
2.3968015992004	1.45519e-11\\
2.39880059970015	1.45519e-11\\
2.4007996001999	1.45519e-11\\
2.40279860069965	2.54659e-11\\
2.4047976011994	2.18279e-11\\
2.40679660169915	2.54659e-11\\
2.4087956021989	2.18279e-11\\
2.41079460269865	1.81899e-11\\
2.4127936031984	1.81899e-11\\
2.41479260369815	1.81899e-11\\
2.4167916041979	1.81899e-11\\
2.41879060469765	2.18279e-11\\
2.4207896051974	2.54659e-11\\
2.42278860569715	2.91038e-11\\
2.4247876061969	3.27418e-11\\
2.42678660669665	2.91038e-11\\
2.4287856071964	4.36557e-11\\
2.43078460769615	4.72937e-11\\
2.4327836081959	4.72937e-11\\
2.43478260869565	4.36557e-11\\
2.4367816091954	4.36557e-11\\
2.43878060969515	5.09317e-11\\
2.4407796101949	5.09317e-11\\
2.44277861069465	5.82077e-11\\
2.4447776111944	5.82077e-11\\
2.44677661169415	6.54836e-11\\
2.4487756121939	7.27596e-11\\
2.45077461269365	6.18456e-11\\
2.4527736131934	6.18456e-11\\
2.45477261369315	5.82077e-11\\
2.4567716141929	5.82077e-11\\
2.45877061469265	5.45697e-11\\
2.4607696151924	5.45697e-11\\
2.46276861569215	6.18456e-11\\
2.4647676161919	6.54836e-11\\
2.46676661669165	6.91216e-11\\
2.4687656171914	5.45697e-11\\
2.47076461769115	6.18456e-11\\
2.4727636181909	5.82077e-11\\
2.47476261869065	5.82077e-11\\
2.4767616191904	5.09317e-11\\
2.47876061969016	5.82077e-11\\
2.48075962018991	6.18456e-11\\
2.48275862068966	6.54836e-11\\
2.48475762118941	6.54836e-11\\
2.48675662168916	6.54836e-11\\
2.48875562218891	6.54836e-11\\
2.49075462268866	6.54836e-11\\
2.49275362318841	7.27596e-11\\
2.49475262368816	7.27596e-11\\
2.49675162418791	7.63976e-11\\
2.49875062468766	8.00355e-11\\
2.50074962518741	8.36735e-11\\
2.50274862568716	8.73115e-11\\
2.50474762618691	9.09495e-11\\
2.50674662668666	8.36735e-11\\
2.50874562718641	9.45874e-11\\
2.51074462768616	8.36735e-11\\
2.51274362818591	8.00355e-11\\
2.51474262868566	8.36735e-11\\
2.51674162918541	8.36735e-11\\
2.51874062968516	8.00355e-11\\
2.52073963018491	8.00355e-11\\
2.52273863068466	7.63976e-11\\
2.52473763118441	7.27596e-11\\
2.52673663168416	7.27596e-11\\
2.52873563218391	8.36735e-11\\
2.53073463268366	8.00355e-11\\
2.53273363318341	7.63976e-11\\
2.53473263368316	7.27596e-11\\
2.53673163418291	7.63976e-11\\
2.53873063468266	8.00355e-11\\
2.54072963518241	6.91216e-11\\
2.54272863568216	7.63976e-11\\
2.54472763618191	6.18456e-11\\
2.54672663668166	5.82077e-11\\
2.54872563718141	6.54836e-11\\
2.55072463768116	6.18456e-11\\
2.55272363818091	6.54836e-11\\
2.55472263868066	6.91216e-11\\
2.55672163918041	7.27596e-11\\
2.55872063968016	6.91216e-11\\
2.56071964017991	6.54836e-11\\
2.56271864067966	7.27596e-11\\
2.56471764117941	6.91216e-11\\
2.56671664167916	6.54836e-11\\
2.56871564217891	5.45697e-11\\
2.57071464267866	6.91216e-11\\
2.57271364317841	6.54836e-11\\
2.57471264367816	6.54836e-11\\
2.57671164417791	6.54836e-11\\
2.57871064467766	5.45697e-11\\
2.58070964517741	5.45697e-11\\
2.58270864567716	6.18456e-11\\
2.58470764617691	7.27596e-11\\
2.58670664667666	7.27596e-11\\
2.58870564717641	7.27596e-11\\
2.59070464767616	6.91216e-11\\
2.59270364817591	6.91216e-11\\
2.59470264867566	7.27596e-11\\
2.59670164917541	5.82077e-11\\
2.59870064967516	5.82077e-11\\
2.60069965017491	5.45697e-11\\
2.60269865067466	5.09317e-11\\
2.60469765117441	4.72937e-11\\
2.60669665167416	6.54836e-11\\
2.60869565217391	6.54836e-11\\
2.61069465267366	5.09317e-11\\
2.61269365317341	5.45697e-11\\
2.61469265367316	5.09317e-11\\
2.61669165417291	4.72937e-11\\
2.61869065467266	6.18456e-11\\
2.62068965517241	5.82077e-11\\
2.62268865567216	6.54836e-11\\
2.62468765617191	6.18456e-11\\
2.62668665667166	6.54836e-11\\
2.62868565717141	6.18456e-11\\
2.63068465767116	6.18456e-11\\
2.63268365817091	5.82077e-11\\
2.63468265867066	6.91216e-11\\
2.63668165917041	6.91216e-11\\
2.63868065967017	7.63976e-11\\
2.64067966016992	7.27596e-11\\
2.64267866066967	7.27596e-11\\
2.64467766116942	7.27596e-11\\
2.64667666166917	6.91216e-11\\
2.64867566216892	6.54836e-11\\
2.65067466266867	6.54836e-11\\
2.65267366316842	6.18456e-11\\
2.65467266366817	5.82077e-11\\
2.65667166416792	5.82077e-11\\
2.65867066466767	5.09317e-11\\
2.66066966516742	5.09317e-11\\
2.66266866566717	5.45697e-11\\
2.66466766616692	5.45697e-11\\
2.66666666666667	5.82077e-11\\
2.66866566716642	5.82077e-11\\
2.67066466766617	6.18456e-11\\
2.67266366816592	5.82077e-11\\
2.67466266866567	6.18456e-11\\
2.67666166916542	6.18456e-11\\
2.67866066966517	6.91216e-11\\
2.68065967016492	6.54836e-11\\
2.68265867066467	8.00355e-11\\
2.68465767116442	8.36735e-11\\
2.68665667166417	7.63976e-11\\
2.68865567216392	6.54836e-11\\
2.69065467266367	6.18456e-11\\
2.69265367316342	6.18456e-11\\
2.69465267366317	6.18456e-11\\
2.69665167416292	5.45697e-11\\
2.69865067466267	6.18456e-11\\
2.70064967516242	8.00355e-11\\
2.70264867566217	8.36735e-11\\
2.70464767616192	9.45874e-11\\
2.70664667666167	9.82254e-11\\
2.70864567716142	9.09495e-11\\
2.71064467766117	9.09495e-11\\
2.71264367816092	1.09139e-10\\
2.71464267866067	9.45874e-11\\
2.71664167916042	9.45874e-11\\
2.71864067966017	9.82254e-11\\
2.72063968015992	8.73115e-11\\
2.72263868065967	9.82254e-11\\
2.72463768115942	9.82254e-11\\
2.72663668165917	9.45874e-11\\
2.72863568215892	8.36735e-11\\
2.73063468265867	8.00355e-11\\
2.73263368315842	8.73115e-11\\
2.73463268365817	8.00355e-11\\
2.73663168415792	6.91216e-11\\
2.73863068465767	6.54836e-11\\
2.74062968515742	8.36735e-11\\
2.74262868565717	9.09495e-11\\
2.74462768615692	8.00355e-11\\
2.74662668665667	6.91216e-11\\
2.74862568715642	6.54836e-11\\
2.75062468765617	6.54836e-11\\
2.75262368815592	6.18456e-11\\
2.75462268865567	5.82077e-11\\
2.75662168915542	5.45697e-11\\
2.75862068965517	4.00178e-11\\
2.76061969015492	2.91038e-11\\
2.76261869065467	4.00178e-11\\
2.76461769115442	4.00178e-11\\
2.76661669165417	3.63798e-11\\
2.76861569215392	2.54659e-11\\
2.77061469265367	2.91038e-11\\
2.77261369315342	2.91038e-11\\
2.77461269365317	2.91038e-11\\
2.77661169415292	2.18279e-11\\
2.77861069465267	7.27596e-12\\
2.78060969515242	1.09139e-11\\
2.78260869565217	2.54659e-11\\
2.78460769615192	2.18279e-11\\
2.78660669665167	2.18279e-11\\
2.78860569715142	2.18279e-11\\
2.79060469765117	2.18279e-11\\
2.79260369815092	7.27596e-12\\
2.79460269865067	7.27596e-12\\
2.79660169915043	7.27596e-12\\
2.79860069965018	3.63798e-12\\
2.80059970014993	3.63798e-12\\
2.80259870064968	-7.27596e-12\\
2.80459770114943	-2.18279e-11\\
2.80659670164918	-2.91038e-11\\
2.80859570214893	-3.63798e-11\\
2.81059470264868	-2.54659e-11\\
2.81259370314843	-1.09139e-11\\
2.81459270364818	-1.09139e-11\\
2.81659170414793	-2.91038e-11\\
2.81859070464768	-2.54659e-11\\
2.82058970514743	-3.63798e-11\\
2.82258870564718	-2.54659e-11\\
2.82458770614693	-2.54659e-11\\
2.82658670664668	-7.27596e-12\\
2.82858570714643	-1.09139e-11\\
2.83058470764618	-2.54659e-11\\
2.83258370814593	-1.45519e-11\\
2.83458270864568	-1.45519e-11\\
2.83658170914543	0\\
2.83858070964518	0\\
2.84057971014493	0\\
2.84257871064468	-1.45519e-11\\
2.84457771114443	-1.45519e-11\\
2.84657671164418	-3.27418e-11\\
2.84857571214393	-2.18279e-11\\
2.85057471264368	-7.27596e-12\\
2.85257371314343	-1.81899e-11\\
2.85457271364318	-1.81899e-11\\
2.85657171414293	-1.81899e-11\\
2.85857071464268	-3.27418e-11\\
2.86056971514243	-1.45519e-11\\
2.86256871564218	-1.81899e-11\\
2.86456771614193	-1.45519e-11\\
2.86656671664168	-1.45519e-11\\
2.86856571714143	-2.91038e-11\\
2.87056471764118	-3.27418e-11\\
2.87256371814093	-2.54659e-11\\
2.87456271864068	-2.18279e-11\\
2.87656171914043	-3.27418e-11\\
2.87856071964018	-3.27418e-11\\
2.88055972013993	-2.54659e-11\\
2.88255872063968	-3.63798e-11\\
2.88455772113943	-2.54659e-11\\
2.88655672163918	-2.18279e-11\\
2.88855572213893	-2.54659e-11\\
2.89055472263868	-3.27418e-11\\
2.89255372313843	-4.72937e-11\\
2.89455272363818	-4.36557e-11\\
2.89655172413793	-4.00178e-11\\
2.89855072463768	-3.27418e-11\\
2.90054972513743	-4.36557e-11\\
2.90254872563718	-5.09317e-11\\
2.90454772613693	-4.00178e-11\\
2.90654672663668	-5.09317e-11\\
2.90854572713643	-5.45697e-11\\
2.91054472763618	-5.45697e-11\\
2.91254372813593	-5.09317e-11\\
2.91454272863568	-5.45697e-11\\
2.91654172913543	-4.36557e-11\\
2.91854072963518	-3.63798e-11\\
2.92053973013493	-2.54659e-11\\
2.92253873063468	-3.27418e-11\\
2.92453773113443	-2.18279e-11\\
2.92653673163418	-1.81899e-11\\
2.92853573213393	-3.63798e-11\\
2.93053473263368	-1.81899e-11\\
2.93253373313343	-2.18279e-11\\
2.93453273363318	-3.27418e-11\\
2.93653173413293	-2.18279e-11\\
2.93853073463268	-2.18279e-11\\
2.94052973513243	-2.91038e-11\\
2.94252873563218	-2.18279e-11\\
2.94452773613193	-1.09139e-11\\
2.94652673663168	-1.09139e-11\\
2.94852573713143	-1.09139e-11\\
2.95052473763118	-3.63798e-12\\
2.95252373813093	1.45519e-11\\
2.95452273863068	-1.09139e-11\\
2.95652173913043	3.63798e-12\\
2.95852073963018	0\\
2.96051974012994	0\\
2.96251874062969	1.45519e-11\\
2.96451774112944	7.27596e-12\\
2.96651674162919	3.63798e-12\\
2.96851574212894	0\\
2.97051474262869	0\\
2.97251374312844	7.27596e-12\\
2.97451274362819	7.27596e-12\\
2.97651174412794	7.27596e-12\\
2.97851074462769	3.63798e-12\\
2.98050974512744	3.63798e-12\\
2.98250874562719	1.81899e-11\\
2.98450774612694	3.27418e-11\\
2.98650674662669	4.72937e-11\\
2.98850574712644	4.36557e-11\\
2.99050474762619	4.72937e-11\\
2.99250374812594	5.09317e-11\\
2.99450274862569	5.09317e-11\\
2.99650174912544	5.09317e-11\\
2.99850074962519	5.09317e-11\\
3.00049975012494	5.09317e-11\\
3.00249875062469	6.91216e-11\\
3.00449775112444	7.63976e-11\\
3.00649675162419	8.36735e-11\\
3.00849575212394	7.27596e-11\\
3.01049475262369	6.91216e-11\\
3.01249375312344	8.00355e-11\\
3.01449275362319	6.91216e-11\\
3.01649175412294	7.63976e-11\\
3.01849075462269	6.54836e-11\\
3.02048975512244	6.91216e-11\\
3.02248875562219	6.91216e-11\\
3.02448775612194	7.27596e-11\\
3.02648675662169	8.00355e-11\\
3.02848575712144	8.00355e-11\\
3.03048475762119	7.27596e-11\\
3.03248375812094	8.36735e-11\\
3.03448275862069	8.36735e-11\\
3.03648175912044	9.82254e-11\\
3.03848075962019	8.73115e-11\\
3.04047976011994	1.01863e-10\\
3.04247876061969	9.45874e-11\\
3.04447776111944	8.36735e-11\\
3.04647676161919	9.45874e-11\\
3.04847576211894	8.00355e-11\\
3.05047476261869	8.00355e-11\\
3.05247376311844	7.63976e-11\\
3.05447276361819	6.91216e-11\\
3.05647176411794	6.54836e-11\\
3.05847076461769	4.72937e-11\\
3.06046976511744	4.72937e-11\\
3.06246876561719	6.54836e-11\\
3.06446776611694	4.72937e-11\\
3.06646676661669	6.54836e-11\\
3.06846576711644	6.91216e-11\\
3.07046476761619	6.54836e-11\\
3.07246376811594	5.09317e-11\\
3.07446276861569	6.54836e-11\\
3.07646176911544	4.72937e-11\\
3.07846076961519	4.00178e-11\\
3.08045977011494	2.91038e-11\\
3.08245877061469	4.00178e-11\\
3.08445777111444	3.63798e-11\\
3.08645677161419	4.36557e-11\\
3.08845577211394	4.36557e-11\\
3.09045477261369	3.27418e-11\\
3.09245377311344	2.91038e-11\\
3.09445277361319	4.00178e-11\\
3.09645177411294	3.27418e-11\\
3.09845077461269	4.00178e-11\\
3.10044977511244	4.00178e-11\\
3.10244877561219	2.18279e-11\\
3.10444777611194	3.63798e-11\\
3.10644677661169	3.27418e-11\\
3.10844577711144	5.45697e-11\\
3.11044477761119	6.54836e-11\\
3.11244377811094	6.54836e-11\\
3.11444277861069	8.36735e-11\\
3.11644177911044	9.09495e-11\\
3.11844077961019	1.05501e-10\\
3.12043978010994	9.09495e-11\\
3.12243878060969	8.73115e-11\\
3.12443778110945	8.36735e-11\\
3.1264367816092	8.36735e-11\\
3.12843578210895	1.01863e-10\\
3.1304347826087	1.01863e-10\\
3.13243378310845	1.09139e-10\\
3.1344327836082	1.12777e-10\\
3.13643178410795	1.12777e-10\\
3.1384307846077	1.01863e-10\\
3.14042978510745	8.00355e-11\\
3.1424287856072	7.63976e-11\\
3.14442778610695	8.73115e-11\\
3.1464267866067	8.00355e-11\\
3.14842578710645	6.18456e-11\\
3.1504247876062	8.00355e-11\\
3.15242378810595	6.54836e-11\\
3.1544227886057	6.18456e-11\\
3.15642178910545	3.63798e-11\\
3.1584207896052	3.27418e-11\\
3.16041979010495	1.45519e-11\\
3.1624187906047	7.27596e-12\\
3.16441779110445	1.45519e-11\\
3.1664167916042	1.09139e-11\\
3.16841579210395	1.09139e-11\\
3.1704147926037	1.45519e-11\\
3.17241379310345	2.91038e-11\\
3.1744127936032	2.91038e-11\\
3.17641179410295	2.91038e-11\\
3.1784107946027	3.27418e-11\\
3.18040979510245	3.63798e-11\\
3.1824087956022	2.91038e-11\\
3.18440779610195	3.27418e-11\\
3.1864067966017	4.00178e-11\\
3.18840579710145	5.82077e-11\\
3.1904047976012	7.27596e-11\\
3.19240379810095	5.09317e-11\\
3.1944027986007	5.09317e-11\\
3.19640179910045	4.72937e-11\\
3.1984007996002	3.27418e-11\\
3.20039980009995	3.63798e-11\\
3.2023988005997	2.54659e-11\\
3.20439780109945	2.91038e-11\\
3.2063968015992	3.63798e-11\\
3.20839580209895	5.09317e-11\\
3.2103948025987	6.18456e-11\\
3.21239380309845	6.18456e-11\\
3.2143928035982	6.18456e-11\\
3.21639180409795	7.27596e-11\\
3.2183908045977	7.63976e-11\\
3.22038980509745	8.36735e-11\\
3.2223888055972	8.73115e-11\\
3.22438780609695	1.09139e-10\\
3.2263868065967	1.09139e-10\\
3.22838580709645	1.20053e-10\\
3.2303848075962	1.23691e-10\\
3.23238380809595	1.20053e-10\\
3.2343828085957	1.30967e-10\\
3.23638180909545	1.20053e-10\\
3.2383808095952	1.23691e-10\\
3.24037981009495	1.12777e-10\\
3.2423788105947	1.23691e-10\\
3.24437781109445	1.23691e-10\\
3.2463768115942	1.27329e-10\\
3.24837581209395	1.34605e-10\\
3.2503748125937	1.30967e-10\\
3.25237381309345	1.49157e-10\\
3.2543728135932	1.30967e-10\\
3.25637181409295	1.45519e-10\\
3.2583708145927	1.49157e-10\\
3.26036981509245	1.45519e-10\\
3.2623688155922	1.52795e-10\\
3.26436781609195	1.74623e-10\\
3.2663668165917	1.56433e-10\\
3.26836581709145	1.60071e-10\\
3.2703648175912	1.70985e-10\\
3.27236381809095	1.67347e-10\\
3.2743628185907	1.78261e-10\\
3.27636181909045	1.70985e-10\\
3.2783608195902	1.78261e-10\\
3.28035982008995	1.81899e-10\\
3.2823588205897	1.67347e-10\\
3.28435782108946	1.78261e-10\\
3.28635682158921	1.70985e-10\\
3.28835582208896	1.81899e-10\\
3.29035482258871	1.81899e-10\\
3.29235382308846	1.70985e-10\\
3.29435282358821	1.70985e-10\\
3.29635182408796	1.70985e-10\\
3.29835082458771	1.70985e-10\\
3.30034982508746	1.81899e-10\\
3.30234882558721	1.81899e-10\\
3.30434782608696	1.81899e-10\\
3.30634682658671	1.70985e-10\\
3.30834582708646	1.67347e-10\\
3.31034482758621	1.74623e-10\\
3.31234382808596	1.74623e-10\\
3.31434282858571	1.85537e-10\\
3.31634182908546	1.81899e-10\\
3.31834082958521	1.81899e-10\\
3.32033983008496	1.78261e-10\\
3.32233883058471	1.78261e-10\\
3.32433783108446	1.81899e-10\\
3.32633683158421	1.70985e-10\\
3.32833583208396	1.67347e-10\\
3.33033483258371	1.56433e-10\\
3.33233383308346	1.52795e-10\\
3.33433283358321	1.67347e-10\\
3.33633183408296	1.49157e-10\\
3.33833083458271	1.52795e-10\\
3.34032983508246	1.41881e-10\\
3.34232883558221	1.38243e-10\\
3.34432783608196	1.49157e-10\\
3.34632683658171	1.52795e-10\\
3.34832583708146	1.49157e-10\\
3.35032483758121	1.74623e-10\\
3.35232383808096	1.74623e-10\\
3.35432283858071	1.89175e-10\\
3.35632183908046	1.78261e-10\\
3.35832083958021	1.89175e-10\\
3.36031984007996	1.74623e-10\\
3.36231884057971	1.63709e-10\\
3.36431784107946	1.70985e-10\\
3.36631684157921	1.70985e-10\\
3.36831584207896	1.78261e-10\\
3.37031484257871	1.96451e-10\\
3.37231384307846	1.78261e-10\\
3.37431284357821	1.81899e-10\\
3.37631184407796	1.78261e-10\\
3.37831084457771	1.67347e-10\\
3.38030984507746	1.78261e-10\\
3.38230884557721	1.92813e-10\\
3.38430784607696	1.96451e-10\\
3.38630684657671	1.92813e-10\\
3.38830584707646	1.78261e-10\\
3.39030484757621	1.92813e-10\\
3.39230384807596	2.03727e-10\\
3.39430284857571	2.03727e-10\\
3.39630184907546	2.18279e-10\\
3.39830084957521	2.07365e-10\\
3.40029985007496	1.96451e-10\\
3.40229885057471	2.07365e-10\\
3.40429785107446	2.03727e-10\\
3.40629685157421	2.03727e-10\\
3.40829585207396	1.96451e-10\\
3.41029485257371	1.96451e-10\\
3.41229385307346	1.92813e-10\\
3.41429285357321	2.00089e-10\\
3.41629185407296	2.03727e-10\\
3.41829085457271	2.07365e-10\\
3.42028985507246	2.00089e-10\\
3.42228885557221	2.03727e-10\\
3.42428785607196	2.03727e-10\\
3.42628685657171	2.11003e-10\\
3.42828585707146	2.11003e-10\\
3.43028485757121	2.11003e-10\\
3.43228385807096	2.21917e-10\\
3.43428285857071	2.25555e-10\\
3.43628185907046	2.29193e-10\\
3.43828085957021	2.40107e-10\\
3.44027986006996	2.40107e-10\\
3.44227886056971	2.51021e-10\\
3.44427786106947	2.40107e-10\\
3.44627686156922	2.36469e-10\\
3.44827586206897	2.14641e-10\\
3.45027486256872	2.14641e-10\\
3.45227386306847	2.11003e-10\\
3.45427286356822	2.00089e-10\\
3.45627186406797	1.96451e-10\\
3.45827086456772	1.89175e-10\\
3.46026986506747	1.96451e-10\\
3.46226886556722	1.96451e-10\\
3.46426786606697	1.89175e-10\\
3.46626686656672	2.07365e-10\\
3.46826586706647	2.18279e-10\\
3.47026486756622	2.18279e-10\\
3.47226386806597	2.18279e-10\\
3.47426286856572	2.14641e-10\\
3.47626186906547	2.03727e-10\\
3.47826086956522	2.14641e-10\\
3.48025987006497	2.25555e-10\\
3.48225887056472	2.14641e-10\\
3.48425787106447	2.21917e-10\\
3.48625687156422	2.29193e-10\\
3.48825587206397	2.36469e-10\\
3.49025487256372	2.21917e-10\\
3.49225387306347	2.07365e-10\\
3.49425287356322	1.96451e-10\\
3.49625187406297	2.00089e-10\\
3.49825087456272	2.07365e-10\\
3.50024987506247	2.14641e-10\\
3.50224887556222	2.14641e-10\\
3.50424787606197	2.25555e-10\\
3.50624687656172	2.11003e-10\\
3.50824587706147	2.03727e-10\\
3.51024487756122	2.11003e-10\\
3.51224387806097	2.14641e-10\\
3.51424287856072	2.29193e-10\\
3.51624187906047	2.25555e-10\\
3.51824087956022	2.25555e-10\\
3.52023988005997	2.25555e-10\\
3.52223888055972	2.21917e-10\\
3.52423788105947	2.25555e-10\\
3.52623688155922	2.36469e-10\\
3.52823588205897	2.29193e-10\\
3.53023488255872	2.29193e-10\\
3.53223388305847	2.11003e-10\\
3.53423288355822	2.14641e-10\\
3.53623188405797	2.14641e-10\\
3.53823088455772	2.11003e-10\\
3.54022988505747	2.11003e-10\\
3.54222888555722	2.21917e-10\\
3.54422788605697	2.25555e-10\\
3.54622688655672	2.25555e-10\\
3.54822588705647	2.25555e-10\\
3.55022488755622	2.18279e-10\\
3.55222388805597	2.14641e-10\\
3.55422288855572	2.07365e-10\\
3.55622188905547	2.18279e-10\\
3.55822088955522	2.21917e-10\\
3.56021989005497	2.25555e-10\\
3.56221889055472	2.40107e-10\\
3.56421789105447	2.43745e-10\\
3.56621689155422	2.43745e-10\\
3.56821589205397	2.43745e-10\\
3.57021489255372	2.43745e-10\\
3.57221389305347	2.47383e-10\\
3.57421289355322	2.43745e-10\\
3.57621189405297	2.43745e-10\\
3.57821089455272	2.40107e-10\\
3.58020989505247	2.40107e-10\\
3.58220889555222	2.47383e-10\\
3.58420789605197	2.25555e-10\\
3.58620689655172	2.36469e-10\\
3.58820589705147	2.32831e-10\\
3.59020489755122	2.40107e-10\\
3.59220389805097	2.36469e-10\\
3.59420289855072	2.47383e-10\\
3.59620189905047	2.47383e-10\\
3.59820089955022	2.36469e-10\\
3.60019990004997	2.32831e-10\\
3.60219890054973	2.40107e-10\\
3.60419790104948	2.58296e-10\\
3.60619690154923	2.43745e-10\\
3.60819590204898	2.40107e-10\\
3.61019490254873	2.40107e-10\\
3.61219390304848	2.51021e-10\\
3.61419290354823	2.43745e-10\\
3.61619190404798	2.47383e-10\\
3.61819090454773	2.43745e-10\\
3.62018990504748	2.43745e-10\\
3.62218890554723	2.58296e-10\\
3.62418790604698	2.61934e-10\\
3.62618690654673	2.61934e-10\\
3.62818590704648	2.61934e-10\\
3.63018490754623	2.58296e-10\\
3.63218390804598	2.54659e-10\\
3.63418290854573	2.54659e-10\\
3.63618190904548	2.51021e-10\\
3.63818090954523	2.51021e-10\\
3.64017991004498	2.47383e-10\\
3.64217891054473	2.47383e-10\\
3.64417791104448	2.51021e-10\\
3.64617691154423	2.47383e-10\\
3.64817591204398	2.51021e-10\\
3.65017491254373	2.47383e-10\\
3.65217391304348	2.54659e-10\\
3.65417291354323	2.58296e-10\\
3.65617191404298	2.61934e-10\\
3.65817091454273	2.61934e-10\\
3.66016991504248	2.61934e-10\\
3.66216891554223	2.58296e-10\\
3.66416791604198	2.61934e-10\\
3.66616691654173	2.58296e-10\\
3.66816591704148	2.51021e-10\\
3.67016491754123	2.51021e-10\\
3.67216391804098	2.47383e-10\\
3.67416291854073	2.51021e-10\\
3.67616191904048	2.61934e-10\\
3.67816091954023	2.54659e-10\\
3.68015992003998	2.61934e-10\\
3.68215892053973	2.61934e-10\\
3.68415792103948	2.72848e-10\\
3.68615692153923	2.76486e-10\\
3.68815592203898	2.76486e-10\\
3.69015492253873	2.76486e-10\\
3.69215392303848	2.83762e-10\\
3.69415292353823	2.76486e-10\\
3.69615192403798	2.80124e-10\\
3.69815092453773	2.76486e-10\\
3.70014992503748	2.91038e-10\\
3.70214892553723	2.80124e-10\\
3.70414792603698	2.72848e-10\\
3.70614692653673	2.61934e-10\\
3.70814592703648	2.58296e-10\\
3.71014492753623	2.6921e-10\\
3.71214392803598	2.65572e-10\\
3.71414292853573	2.6921e-10\\
3.71614192903548	2.6921e-10\\
3.71814092953523	2.65572e-10\\
3.72013993003498	2.72848e-10\\
3.72213893053473	2.72848e-10\\
3.72413793103448	2.6921e-10\\
3.72613693153423	2.65572e-10\\
3.72813593203398	2.61934e-10\\
3.73013493253373	2.58296e-10\\
3.73213393303348	2.51021e-10\\
3.73413293353323	2.43745e-10\\
3.73613193403298	2.43745e-10\\
3.73813093453273	2.43745e-10\\
3.74012993503248	2.47383e-10\\
3.74212893553223	2.61934e-10\\
3.74412793603198	2.61934e-10\\
3.74612693653173	2.51021e-10\\
3.74812593703148	2.58296e-10\\
3.75012493753123	2.54659e-10\\
3.75212393803098	2.54659e-10\\
3.75412293853073	2.54659e-10\\
3.75612193903048	2.61934e-10\\
3.75812093953023	2.61934e-10\\
3.76011994002998	2.61934e-10\\
3.76211894052974	2.61934e-10\\
3.76411794102949	2.54659e-10\\
3.76611694152924	2.58296e-10\\
3.76811594202899	2.58296e-10\\
3.77011494252874	2.61934e-10\\
3.77211394302849	2.54659e-10\\
3.77411294352824	2.61934e-10\\
3.77611194402799	2.58296e-10\\
3.77811094452774	2.47383e-10\\
3.78010994502749	2.43745e-10\\
3.78210894552724	2.40107e-10\\
3.78410794602699	2.43745e-10\\
3.78610694652674	2.36469e-10\\
3.78810594702649	2.43745e-10\\
3.79010494752624	2.51021e-10\\
3.79210394802599	2.51021e-10\\
3.79410294852574	2.54659e-10\\
3.79610194902549	2.51021e-10\\
3.79810094952524	2.54659e-10\\
3.80009995002499	2.51021e-10\\
3.80209895052474	2.40107e-10\\
3.80409795102449	2.40107e-10\\
3.80609695152424	2.32831e-10\\
3.80809595202399	2.43745e-10\\
3.81009495252374	2.40107e-10\\
3.81209395302349	2.40107e-10\\
3.81409295352324	2.51021e-10\\
3.81609195402299	2.47383e-10\\
3.81809095452274	2.40107e-10\\
3.82008995502249	2.43745e-10\\
3.82208895552224	2.47383e-10\\
3.82408795602199	2.40107e-10\\
3.82608695652174	2.43745e-10\\
3.82808595702149	2.43745e-10\\
3.83008495752124	2.36469e-10\\
3.83208395802099	2.36469e-10\\
3.83408295852074	2.40107e-10\\
3.83608195902049	2.32831e-10\\
3.83808095952024	2.36469e-10\\
3.84007996001999	2.40107e-10\\
3.84207896051974	2.36469e-10\\
3.84407796101949	2.36469e-10\\
3.84607696151924	2.40107e-10\\
3.84807596201899	2.32831e-10\\
3.85007496251874	2.32831e-10\\
3.85207396301849	2.43745e-10\\
3.85407296351824	2.40107e-10\\
3.85607196401799	2.40107e-10\\
3.85807096451774	2.40107e-10\\
3.86006996501749	2.36469e-10\\
3.86206896551724	2.43745e-10\\
3.86406796601699	2.47383e-10\\
3.86606696651674	2.47383e-10\\
3.86806596701649	2.47383e-10\\
3.87006496751624	2.43745e-10\\
3.87206396801599	2.43745e-10\\
3.87406296851574	2.43745e-10\\
3.87606196901549	2.43745e-10\\
3.87806096951524	2.43745e-10\\
3.88005997001499	2.40107e-10\\
3.88205897051474	2.36469e-10\\
3.88405797101449	2.47383e-10\\
3.88605697151424	2.47383e-10\\
3.88805597201399	2.47383e-10\\
3.89005497251374	2.47383e-10\\
3.89205397301349	2.47383e-10\\
3.89405297351324	2.51021e-10\\
3.89605197401299	2.47383e-10\\
3.89805097451274	2.47383e-10\\
3.90004997501249	2.40107e-10\\
3.90204897551224	2.40107e-10\\
3.90404797601199	2.43745e-10\\
3.90604697651174	2.40107e-10\\
3.90804597701149	2.32831e-10\\
3.91004497751124	2.32831e-10\\
3.91204397801099	2.29193e-10\\
3.91404297851074	2.25555e-10\\
3.91604197901049	2.25555e-10\\
3.91804097951024	2.29193e-10\\
3.92003998000999	2.32831e-10\\
3.92203898050975	2.18279e-10\\
3.9240379810095	2.18279e-10\\
3.92603698150925	2.14641e-10\\
3.928035982009	2.11003e-10\\
3.93003498250875	2.00089e-10\\
3.9320339830085	2.03727e-10\\
3.93403298350825	2.03727e-10\\
3.936031984008	2.07365e-10\\
3.93803098450775	2.07365e-10\\
3.9400299850075	2.07365e-10\\
3.94202898550725	2.11003e-10\\
3.944027986007	2.11003e-10\\
3.94602698650675	2.14641e-10\\
3.9480259870065	2.14641e-10\\
3.95002498750625	2.21917e-10\\
3.952023988006	2.11003e-10\\
3.95402298850575	2.21917e-10\\
3.9560219890055	2.18279e-10\\
3.95802098950525	2.18279e-10\\
3.960019990005	2.18279e-10\\
3.96201899050475	2.07365e-10\\
3.9640179910045	2.14641e-10\\
3.96601699150425	2.25555e-10\\
3.968015992004	2.21917e-10\\
3.97001499250375	2.21917e-10\\
3.9720139930035	2.14641e-10\\
3.97401299350325	2.11003e-10\\
3.976011994003	2.14641e-10\\
3.97801099450275	2.18279e-10\\
3.9800099950025	2.18279e-10\\
3.98200899550225	2.11003e-10\\
3.984007996002	2.14641e-10\\
3.98600699650175	2.18279e-10\\
3.9880059970015	2.18279e-10\\
3.99000499750125	2.18279e-10\\
3.992003998001	2.18279e-10\\
3.99400299850075	2.18279e-10\\
3.9960019990005	2.18279e-10\\
3.99800099950025	2.14641e-10\\
4	2.18279e-10\\
};
\addlegendentry{Energy Diff};

\end{axis}
\end{tikzpicture}%
	\caption{A zoom in on the error when stepping backwards in time.}
	\label{fig:backwardDataError}
\end{figure}
\fi


\end{document}
