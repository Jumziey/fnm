\documentclass[11pt]{article}
\usepackage{report}
\newif\iftikz
\tikztrue

\begin{document}
\section{Exercise 1}
\subsection{How to run the code}
In the directory \verb+~jeve0010/Documents/jesper/fnm/lab3/ex1+ the written code resides. To execute the code simply write \verb+make+ in the console to be sure to use the updated code, then execute it by writing \verb+./ex1+ in the console.

\subsection{Result}
In table \ref{table:ex1Res} we can see the obtained results for \verb+top1.o+.
\begin{table}[h]\center\Large
  \caption{The obtained obtained values}
  \begin{tabular}{c|l}
  M & 100.995g \\ \hline
  l & 3.00630cm \\ \hline
  $I_1$ & 1215.00 gcm$^2$ \\ \hline
  $I_3$ & 482.568 gcm$^2$ 
  \end{tabular}
  \label{table:ex1Res}
\end{table}

\section{Exercise 2}
In this exercise we start with rewriting equations (21)-(23) from the lab instructions to six ordinary differential equations. The resulting equations are
\begin{align}
	\dot{\phi} &= q \\
	\dot{\psi} &= r \\
	\dot{\theta} &= s \\
	\dot{q} &= \frac{I_3}{I_1} \Bigg[ \dot{\phi}\dot{\theta}\cot{\theta} + \frac{\dot{\psi}\dot{\theta}}{\sin{\theta}} \Bigg] - 2 \dot{\theta}\dot{\psi}\cot{\theta} \\
	\dot{r} &= \dot{\psi}\dot{\theta}\sin{\theta} + 2\dot{\theta}\dot{\psi}\cot{\theta}\cos{\theta} - \frac{I_3}{I_1} \Bigg[\dot{\psi}\dot{\theta}\cos{\theta} + \dot{\psi}\dot{\theta}\Bigg]\cot{\theta}\\
	\dot{s} &= \frac{Mgl}{I_1}\sin{\theta} + \dot{\phi}^2\sin{\theta}\cos{\theta}-\frac{I_3}{I_1}\Big(\dot{\psi}+\dot{\phi}\cos{\theta}\Big)\dot{\psi}\sin{\theta}
\end{align}
The function implementing the Lagrangian ode is declared as
\begin{lstlisting}
int
topLagrange(double time, const double v[], double d[], void *params)
\end{lstlisting}
And the function code exists in the file 
\begin{verbatim}
~jeve0010/Documents/jesper/fnm/lab3/ex2/ex2.c.
\end{verbatim}
\subsection{Running the Code}
To run the code simply run the \verb+make+ command inside the directory 
\begin{verbatim}
~jeve0010/Documents/jesper/fnm/lab3/ex2+
\end{verbatim}
 then run the program with the command \verb+./ex2+. If one wants to obtain the plots showed in this document simply execute the matlab script 
\begin{verbatim}
 ~jeve0010/Documents/jesper/fnm/lab3/ex2/plots/code/topSpin.m.
\end{verbatim}
Note that the resulting tikz file produced from that script is used in this document, not the plots that matlab create itself.


\subsection{Results}
In figure \ref{fig:forwardData} and \ref{fig:backwardData} we can se the evolution of $\psi$, $\phi$ and $\theta$ going from zero to four seconds and then backward to zero again. The error is plotted in the same plot, but is clearly so low that another error plot is required to get a clearer picture of the error evolution. In figure \ref{fig:forwardDataError} and \ref{fig:backwardDataError} we can see the error when going forward and backwards in the time evolution. In the legend $c1$ and $c2$ stand for the constants given by equation 18 and 20 from the lab instructions. These errors are clearly negligebly small since it is about 10 orders of magnitude smaller then the values obtain for $\psi$, $\phi$ and $\theta$. The time step of $2\cdot10^{-3}$ was used with the \verb+gsl_odeiv2_step_rkf45+ function to get this result.
\iftikz
\begin{figure}[H]
	\centering
	\newlength\figureheight 
	\newlength\figurewidth 
	\setlength\figureheight{7cm} 
	\setlength\figurewidth{14cm}
	% This file was created by matlab2tikz.
% Minimal pgfplots version: 1.3
%
%The latest updates can be retrieved from
%  http://www.mathworks.com/matlabcentral/fileexchange/22022-matlab2tikz
%where you can also make suggestions and rate matlab2tikz.
%
\definecolor{mycolor1}{rgb}{0.00000,0.44700,0.74100}%
\definecolor{mycolor2}{rgb}{0.85000,0.32500,0.09800}%
\definecolor{mycolor3}{rgb}{0.92900,0.69400,0.12500}%
%
\begin{tikzpicture}

\begin{axis}[%
width=0.95092\figurewidth,
height=\figureheight,
at={(0\figurewidth,0\figureheight)},
scale only axis,
xmin=0,
xmax=4,
xlabel={Time (s)},
ymin=0,
ymax=14000,
ylabel={Degrees},
title style={font=\bfseries},
title={Top Spin [0,4] (s)},
legend style={at={(0.03,0.97)},anchor=north west,legend cell align=left,align=left,draw=white!15!black},
title style={font=\small},ticklabel style={font=\tiny}
]
\addplot [color=mycolor1,solid]
  table[row sep=crcr]{%
0	0\\
0.00200100050025012	0.000630253574643906\\
0.00400200100050025	0.00429718346348117\\
0.00600300150075038	0.0138655786421659\\
0.0080040020010005	0.0319710449682999\\
0.0100050025012506	0.061249188299485\\
0.0120060030015008	0.104221022934297\\
0.0140070035017509	0.163350267391798\\
0.016008004002001	0.240756865513972\\
0.0180090045022511	0.33867535270183\\
0.0200100050025013	0.458939193899789\\
0.0220110055027514	0.60338185405227\\
0.0240120060030015	0.773435727647098\\
0.0260130065032516	0.970590504951614\\
0.0280140070035018	1.19593480577657\\
0.0300150075037519	1.45061454571222\\
0.032016008004002	1.73537456989224\\
0.0340170085042521	2.05090242767078\\
0.0360180090045022	2.39777107684298\\
0.0380190095047524	2.7762669963064\\
0.0400200100050025	3.18667666495861\\
0.0420210105052526	3.62905737857912\\
0.0440220110055028	4.10346643294744\\
0.0460230115057529	4.60978923650455\\
0.048024012006003	5.14785390191191\\
0.0500250125062531	5.71731665449243\\
0.0520260130065033	6.31794831112807\\
0.0540270135067534	6.94923320980322\\
0.0560280140070035	7.61088487162029\\
0.0580290145072536	8.30227304300466\\
0.0600300150075038	9.02299665349972\\
0.0620310155077539	9.77254004108986\\
0.064032016008004	10.5502729522004\\
0.0660330165082541	11.3556797248158\\
0.0680340170085043	12.1882446969204\\
0.0700350175087544	13.0473376149396\\
0.0720360180090045	13.9325001126372\\
0.0740370185092546	14.8431019364386\\
0.0760380190095048	15.7786847201077\\
0.0780390195097549	16.7386182100699\\
0.080040020010005	17.7225586316481\\
0.0820410205102551	18.7298757312676\\
0.0840420210105053	19.7601684384718\\
0.0860430215107554	20.8130356828042\\
0.0880440220110055	21.8880190980287\\
0.0900450225112556	22.9847176136886\\
0.0920460230115058	24.1027301593274\\
0.0940470235117559	25.2417702560474\\
0.096048024012006	26.4015514249513\\
0.0980490245122561	27.5816725955822\\
0.100050025012506	28.7819045848223\\
0.102051025512756	30.0020182095534\\
0.104052026013007	31.2417842866574\\
0.106053026513257	32.5010309287959\\
0.108054027013507	33.7795862486304\\
0.110055027513757	35.0772783588222\\
0.112056028014007	36.3939926678123\\
0.114057028514257	37.7296718798213\\
0.116058029014507	39.0842014032901\\
0.118059029514757	40.4575239424392\\
0.120060030015008	41.8496394972685\\
0.122061030515258	43.2605480677782\\
0.124062031015508	44.6902496539681\\
0.126063031515758	46.1387442558383\\
0.128064032016008	47.6061464649479\\
0.130065032516258	49.0924562812968\\
0.132066033016508	50.5978455922235\\
0.134067033516758	52.1223143977281\\
0.136068034017009	53.666034585149\\
0.138069034517259	55.2291207460454\\
0.140070035017509	56.8116874719763\\
0.142071035517759	58.4139066502801\\
0.144072036018009	60.035892872516\\
0.146073036518259	61.6778180260224\\
0.148074037018509	63.3399112939174\\
0.150075037518759	65.0222299719805\\
0.15207603801901	66.7249459475503\\
0.15407703851926	68.4482884037448\\
0.15607803901951	70.1923146363435\\
0.15807903951976	71.9571392369054\\
0.16008004002001	73.7429913885487\\
0.16208104052026	75.5498710912733\\
0.16408204102051	77.3779502324177\\
0.16608304152076	79.2271715162024\\
0.168084042021011	81.097592238407\\
0.170085042521261	82.9892696948109\\
0.172086043021511	84.9020319980756\\
0.174087043521761	86.8359364439807\\
0.176088044022011	88.790753849408\\
0.178089044522261	90.7663123270191\\
0.180090045022511	92.7623826936959\\
0.182091045522761	94.7786784705408\\
0.184092046023012	96.8147985870972\\
0.186093046523262	98.870399268688\\
0.188094047023512	100.944907557518\\
0.190095047523762	103.037750495793\\
0.192096048024012	105.148355125716\\
0.194097048524262	107.275976602155\\
0.196098049024512	109.419812784196\\
0.198099049524762	111.579061530926\\
0.200100050025012	113.752748814093\\
0.202101050525263	115.939786013887\\
0.204102051025513	118.139199102056\\
0.206103051525763	120.349842163009\\
0.208104052026013	122.570454689597\\
0.210105052526263	124.799776174672\\
0.212106053026513	127.036488815304\\
0.214107053526763	129.279217512784\\
0.216108054027013	131.526587168405\\
0.218109054527264	133.777165387679\\
0.220110055027514	136.029519776118\\
0.222111055527764	138.282160643454\\
0.224112056028014	140.5336555952\\
0.226113056528264	142.782514941089\\
0.228114057028514	145.027363582412\\
0.230115057528764	147.266654533121\\
0.232116058029014	149.499184581849\\
0.234117058529265	151.723521334105\\
0.236118059029515	153.938346986963\\
0.238119059529765	156.142458329052\\
0.240120060030015	158.334766740561\\
0.242121060530265	160.514011714341\\
0.244122061030515	162.67921922214\\
0.246123061530765	164.829472531487\\
0.248124062031016	166.963854909908\\
0.250125062531266	169.081506920712\\
0.252126063031516	171.181683718764\\
0.254127063531766	173.26375505049\\
0.256128064032016	175.327147958094\\
0.258129064532266	177.371289483783\\
0.260130065032516	179.395721261318\\
0.262131065532766	181.400099516025\\
0.264132066033017	183.384080473224\\
0.266133066533267	185.347377654019\\
0.268134067033517	187.289819171072\\
0.270135067533767	189.211233137043\\
0.272136068034017	191.111562256153\\
0.274137068534267	192.990691936844\\
0.276138069034517	194.848564883335\\
0.278139069534767	196.685238391407\\
0.280140070035018	198.500769756838\\
0.282141070535268	200.295216275408\\
0.284142071035518	202.068635242897\\
0.286143071535768	203.821141250863\\
0.288144072036018	205.552906186646\\
0.290145072536268	207.264044641805\\
0.292146073036518	208.954671207897\\
0.294147073536768	210.624957772262\\
0.296148074037018	212.275133518019\\
0.298149074537269	213.905255740945\\
0.300150075037519	215.515553624161\\
0.302151075537769	217.106141759223\\
0.304152076038019	218.677134737692\\
0.306153076538269	220.228761742686\\
0.308154077038519	221.761080069984\\
0.310155077538769	223.274204311145\\
0.31215607803902	224.768306353508\\
0.31415707853927	226.243443492852\\
0.31615807903952	227.699673024956\\
0.31815907953977	229.137052245601\\
0.32016008004002	230.555581154786\\
0.32216108054027	231.955431639849\\
0.32416208104052	233.336431813453\\
0.32616308154077	234.698638971377\\
0.32816408204102	236.04199581784\\
0.330165082541271	237.366502352844\\
0.332166083041521	238.67198668905\\
0.334167083541771	239.958334234898\\
0.336168084042021	241.225487694609\\
0.338169084542271	242.473275180845\\
0.340170085042521	243.701410214708\\
0.342171085542771	244.909835500419\\
0.344172086043022	246.0982072633\\
0.346173086543272	247.266296320233\\
0.348174087043522	248.413816192321\\
0.350175087543772	249.540480400666\\
0.352176088044022	250.645945170591\\
0.354177088544272	251.72986672742\\
0.356178089044522	252.791786704915\\
0.358179089544772	253.831361328401\\
0.360180090045022	254.84813223164\\
0.362181090545273	255.841641048397\\
0.364182091045523	256.811486708215\\
0.366183091545773	257.757096253299\\
0.368184092046023	258.677954021633\\
0.370185092546273	259.573544351202\\
0.372186093046523	260.44335157999\\
0.374187093546773	261.286860045982\\
0.376188094047024	262.103382199822\\
0.378189094547274	262.892516971056\\
0.380190095047524	263.653634106108\\
0.382191095547774	264.386160647183\\
0.384192096048024	265.089638228044\\
0.386193096548274	265.763493890898\\
0.388194097048524	266.407211973727\\
0.390195097548774	267.020448701856\\
0.392196098049024	267.602688413268\\
0.394197098549275	268.153587333286\\
0.396198099049525	268.672916278792\\
0.398199099549775	269.16038877089\\
0.400200100050025	269.615890218019\\
0.402201100550275	270.0393633244\\
0.404202101050525	270.430865385813\\
0.406203101550775	270.790568289596\\
0.408204102051026	271.118873106206\\
0.410205102551276	271.41612361032\\
0.412206103051526	271.683007351292\\
0.414207103551776	271.920211878476\\
0.416208104052026	272.128768515904\\
0.418209104552276	272.309765883385\\
0.420210105052526	272.464464488071\\
0.422211105552776	272.594354020227\\
0.424212106053027	272.701210649019\\
0.426213106553277	272.786753247832\\
0.428214107053527	272.853101760508\\
0.430215107553777	272.902376130889\\
0.432216108054027	272.936982781715\\
0.434217108554277	272.959328135725\\
0.436218109054527	272.972047798777\\
0.438219109554777	272.977777376728\\
0.440220110055028	272.979324362775\\
0.442221110555278	272.979438954334\\
0.444222111055528	272.980928644602\\
0.446223111555778	272.986600926773\\
0.448224112056028	272.999205998266\\
0.450225112556278	273.021379464938\\
0.452226113056528	273.055699636866\\
0.454227113556778	273.104744824129\\
0.456228114057029	273.170749562128\\
0.458229114557279	273.255948386264\\
0.460230115057529	273.3624039446\\
0.462231115557779	273.491892406299\\
0.464232116058029	273.646189940528\\
0.466233116558279	273.826728941774\\
0.468234117058529	274.034769917186\\
0.470235117558779	274.271516078134\\
0.47223611805903	274.53788415709\\
0.47423711855928	274.834618999188\\
0.47623811905953	275.162350858003\\
0.47823911955978	275.521538099771\\
0.48024012006003	275.912524499168\\
0.48224112056028	276.335424647754\\
0.48424212106053	276.790353137088\\
0.48624312156078	277.27730996717\\
0.488244122061031	277.796065954881\\
0.490245122561281	278.346449212884\\
0.492246123061531	278.92817326228\\
0.494247123561781	279.540837032613\\
0.496248124062031	280.184096749207\\
0.498249124562281	280.857379454265\\
0.500250125062531	281.560341373111\\
0.502251125562781	282.29240954795\\
0.504252126063031	283.053068316765\\
0.506253126563282	283.841687425983\\
0.508254127063532	284.657808509368\\
0.510255127563782	285.500801313344\\
0.512256128064032	286.370207471675\\
0.514257128564282	287.265339435008\\
0.516258129064532	288.185796132886\\
0.518259129564782	289.130947311734\\
0.520260130065032	290.100391901095\\
0.522261130565283	291.093499647395\\
0.524262131065533	292.109869480178\\
0.526263131565783	293.149100328986\\
0.528264132066033	294.210619236025\\
0.530265132566283	295.294139722397\\
0.532266133066533	296.399260717645\\
0.534267133566783	297.525581151313\\
0.536268134067034	298.672757248724\\
0.538269134567284	299.84050253098\\
0.540270135067534	301.028530519184\\
0.542271135567784	302.236554734438\\
0.544272136068034	303.464403289403\\
0.546273136568284	304.711847000962\\
0.548274137068534	305.978656685996\\
0.550275137568784	307.264717752947\\
0.552276138069035	308.569858314476\\
0.554277138569285	309.894021074802\\
0.556278139069535	311.237091442369\\
0.558279139569785	312.598954825615\\
0.560280140070035	313.979668520321\\
0.562281140570285	315.379117934928\\
0.564282141070535	316.797417660995\\
0.566283141570785	318.234453106963\\
0.568284142071036	319.69033886439\\
0.570285142571286	321.165132229057\\
0.572286143071536	322.658890496743\\
0.574287143571786	324.171728259006\\
0.576288144072036	325.703702811627\\
0.578289144572286	327.254986041943\\
0.580290145072536	328.825692541515\\
0.582291145572786	330.415936901901\\
0.584292146073036	332.025891010439\\
0.586293146573287	333.655669458688\\
0.588294147073537	335.305501429768\\
0.590295147573787	336.975444219456\\
0.592296148074037	338.665727010871\\
0.594297148574287	340.376521691353\\
0.596298149074537	342.107942852458\\
0.598299149574787	343.860105085748\\
0.600300150075038	345.63312298278\\
0.602301150575288	347.427225726673\\
0.604302151075538	349.242356021648\\
0.606303151575788	351.078685755042\\
0.608304152076038	352.936214926856\\
0.610305152576288	354.81500083287\\
0.612306153076538	356.714928881524\\
0.614307153576788	358.635999072818\\
0.616308154077039	360.578096815193\\
0.618309154577289	362.541050221311\\
0.620310155077539	364.524687403834\\
0.622311155577789	366.528721883863\\
0.624312156078039	368.552809886721\\
0.626313156578289	370.596607637733\\
0.628314157078539	372.65965677066\\
0.630315157578789	374.741384327709\\
0.63231615807904	376.841331942643\\
0.63431715857929	378.958640178769\\
0.63631815907954	381.092736078293\\
0.63831915957979	383.242702908742\\
0.64032016008004	385.407681233423\\
0.64232116058029	387.586754319865\\
0.64432216108054	389.778776252477\\
0.64632316158079	391.982715707227\\
0.64832416208104	394.197426768525\\
0.650325162581291	396.421591633444\\
0.652326163081541	398.653949794832\\
0.654327163581791	400.893183449763\\
0.656328164082041	403.137917499526\\
0.658329164582291	405.386719549635\\
0.660330165082541	407.638157205602\\
0.662331165582791	409.890798072938\\
0.664332166083042	412.143152461377\\
0.666333166583292	414.39378797643\\
0.668334167083542	416.641272223611\\
0.670335167583792	418.88411551265\\
0.672336168084042	421.120942744841\\
0.674337168584292	423.350378821474\\
0.676338169084542	425.571105939622\\
0.678339169584792	427.781920887914\\
0.680340170085043	429.981563159201\\
0.682341170585293	432.168829542112\\
0.684342171085543	434.342746008398\\
0.686343171585793	436.502223938246\\
0.688344172086043	438.646346599184\\
0.690345172586293	440.774254554521\\
0.692346173086543	442.885145663342\\
0.694347173586793	444.978332376293\\
0.696348174087044	447.0531844398\\
0.698349174587294	449.109071600289\\
0.700350175087544	451.145535491522\\
0.702351175587794	453.162175043044\\
0.704352176088044	455.158589184398\\
0.706353176588294	457.134491436686\\
0.708354177088544	459.089652616791\\
0.710355177588794	461.023900837373\\
0.712356178089045	462.937064211094\\
0.714357178589295	464.829085442175\\
0.716358179089545	466.699907234836\\
0.718359179589795	468.549472293298\\
0.720360180090045	470.37789520912\\
0.722361180590295	472.185118686521\\
0.724362181090545	473.971314612842\\
0.726363181590795	475.73654028386\\
0.728364182091045	477.480910291136\\
0.730365182591296	479.204596522008\\
0.732366183091546	480.907656272254\\
0.734367183591796	482.590376020774\\
0.736368184092046	484.252755767567\\
0.738369184592296	485.89502469575\\
0.740370185092546	487.517411988443\\
0.742371185592796	489.119917645644\\
0.744372186093047	490.702828146252\\
0.746373186593297	492.266258081825\\
0.748374187093547	493.810322043924\\
0.750375187593797	495.335077328326\\
0.752376188094047	496.84075311815\\
0.754377188594297	498.327406709176\\
0.756378189094547	499.795152692963\\
0.758379189594797	501.24399106951\\
0.760380190095048	502.673979134597\\
0.762381190595298	504.085231479784\\
0.764382191095548	505.477633513511\\
0.766383191595798	506.851299827337\\
0.768384192096048	508.206173125483\\
0.770385192596298	509.54213881639\\
0.772386193096548	510.859196900057\\
0.774387193596798	512.157232784926\\
0.776388194097049	513.436074583658\\
0.778389194597299	514.695665000473\\
0.780390195097549	515.935774852254\\
0.782391195597799	517.156232251663\\
0.784392196098049	518.35680801558\\
0.786393196598299	519.537272960888\\
0.788394197098549	520.697397904469\\
0.790395197598799	521.836781775866\\
0.792396198099049	522.955195391961\\
0.7943971985993	524.052237682298\\
0.79639819909955	525.1275648722\\
0.7983991995998	526.180833186989\\
0.80040020010005	527.211526964649\\
0.8024012006003	528.219245134726\\
0.80440220110055	529.20358662676\\
0.8064032016008	530.163978482959\\
0.808404202101051	531.099905041305\\
0.810405202601301	532.010965231342\\
0.812406203101551	532.896528799497\\
0.814407203601801	533.756137379531\\
0.816408204102051	534.589103422093\\
0.818409204602301	535.395025856724\\
0.820410205102551	536.17321713407\\
0.822411205602801	536.923218887896\\
0.824412206103052	537.644458160407\\
0.826413206603302	538.336361993807\\
0.828414207103552	538.99847202186\\
0.830415207603802	539.630329878331\\
0.832416208104052	540.231477196982\\
0.834417208604302	540.801512907358\\
0.836418209104552	541.340035939001\\
0.838419209604802	541.846931700353\\
0.840420210105053	542.321913712517\\
0.842421210605303	542.764867383932\\
0.844422211105553	543.1757927146\\
0.846423211605803	543.554861591859\\
0.848424212106053	543.902245903046\\
0.850425212606303	544.218289422841\\
0.852426213106553	544.503565109036\\
0.854427213606803	544.758760510988\\
0.856428214107053	544.984620473828\\
0.858429214607304	545.182233617369\\
0.860430215107554	545.3527458572\\
0.862431215607804	545.497589587809\\
0.864432216108054	545.618254499463\\
0.866433216608304	545.716516761328\\
0.868434217108554	545.794324429907\\
0.870435217608804	545.853682857483\\
0.872436218109054	545.896941171015\\
0.874437218609305	545.926448497464\\
0.876438219109555	545.944668555349\\
0.878439219609805	545.954351542087\\
0.880440220110055	545.958133063535\\
0.882441220610305	545.95876331711\\
0.884442221110555	545.959107091787\\
0.886443221610805	545.961914584983\\
0.888444222111056	545.969935994115\\
0.890445222611306	545.985978812378\\
0.892446223111556	546.012621349852\\
0.894447223611806	546.052384620834\\
0.896448224112056	546.107789639623\\
0.898449224612306	546.18107094162\\
0.900450225112556	546.274405766447\\
0.902451225612806	546.389627579048\\
0.904452226113057	546.528684435926\\
0.906453226613307	546.693066027349\\
0.908454227113557	546.884204747805\\
0.910455227613807	547.103361104442\\
0.912456228114057	547.351623717073\\
0.914457228614307	547.629737430829\\
0.916458229114557	547.938561682405\\
0.918459229614807	548.278554838035\\
0.920460230115058	548.650175263957\\
0.922461230615308	549.053652143288\\
0.924462231115558	549.489100067588\\
0.926463231615808	549.956576332635\\
0.928464232116058	550.455966346871\\
0.930465232616308	550.987155518737\\
0.932466233116558	551.549857369335\\
0.934467233616808	552.143728123988\\
0.936468234117058	552.768424008019\\
0.938469234617309	553.423486655192\\
0.940470235117559	554.108457699271\\
0.942471235617809	554.82287877402\\
0.944472236118059	555.566176921643\\
0.946473236618309	556.337836480125\\
0.948474237118559	557.137341787451\\
0.950475237618809	557.964005294265\\
0.95247623811906	558.817425930113\\
0.95447723861931	559.696973441418\\
0.95647823911956	560.602074870386\\
0.95847923961981	561.532271850781\\
0.96048024012006	562.486991424807\\
0.96248124062031	563.46577522623\\
0.96448224112056	564.468050297252\\
0.96648324162081	565.493415567418\\
0.968484242121061	566.541412670492\\
0.970485242621311	567.611583240237\\
0.972486243121561	568.703640797756\\
0.974487243621811	569.817069681034\\
0.976488244122061	570.951583411173\\
0.978489244622311	572.106838213495\\
0.980490245122561	573.282547609103\\
0.982491245622811	574.47848241488\\
0.984492246123062	575.694298856147\\
0.986493246623312	576.929825045568\\
0.988494247123562	578.184831800022\\
0.990495247623812	579.459204527952\\
0.992496248124062	580.752771342019\\
0.994497248624312	582.065360354884\\
0.996498249124562	583.396914270768\\
0.998499249624812	584.747375793892\\
1.00050025012506	586.116630332695\\
1.00250125062531	587.504677887179\\
1.00450225112556	588.911518457343\\
1.00650325162581	590.337094747409\\
1.00850425212606	591.781521348933\\
1.01050525262631	593.244798261918\\
1.01250625312656	594.727040077921\\
1.01450725362681	596.228246796944\\
1.01650825412706	597.748533010544\\
1.01850925462731	599.28807060606\\
1.02051025512756	600.846974175052\\
1.02251125562781	602.425301013299\\
1.02451225612806	604.023223008139\\
1.02651325662831	605.640912046912\\
1.02851425712856	607.278540016955\\
1.03051525762881	608.936221509827\\
1.03251625812906	610.614128412868\\
1.03451725862931	612.312375317636\\
1.03651825912956	614.031248703028\\
1.03851925962981	615.770748569045\\
1.04052026013006	617.531104098805\\
1.04252126063032	619.312372588087\\
1.04452226113057	621.114668628451\\
1.04652326163082	622.938106811455\\
1.04852426213107	624.782744432878\\
1.05052526263132	626.648581492722\\
1.05252626313157	628.535617990985\\
1.05452726363182	630.443853927669\\
1.05652826413207	632.373174711213\\
1.05852926463232	634.323465750058\\
1.06053026513257	636.294497861088\\
1.06253126563282	638.286156452742\\
1.06453226613307	640.298097750344\\
1.06653326663332	642.329977979216\\
1.06853426713357	644.381338773123\\
1.07053526763382	646.451779061608\\
1.07253626813407	648.540725886876\\
1.07453726863432	650.647548995351\\
1.07653826913457	652.771503541901\\
1.07853926963482	654.911959272951\\
1.08054027013507	657.067884864469\\
1.08254127063532	659.238535471322\\
1.08454227113557	661.42276517792\\
1.08654327163582	663.61959995601\\
1.08854427213607	665.827893890004\\
1.09054527263632	668.04644376853\\
1.09254627313657	670.273989084439\\
1.09454727363682	672.509212034804\\
1.09654827413707	674.750737520915\\
1.09854927463732	676.997247739843\\
1.10055027513757	679.247253001322\\
1.10255127563782	681.499320910863\\
1.10455227613807	683.752019073979\\
1.10655327663832	686.003915096181\\
1.10855427713857	688.253461991424\\
1.11055527763882	690.499284660998\\
1.11255627813907	692.739893414637\\
1.11455727863932	694.974027745191\\
1.11655827913957	697.200197962392\\
1.11855927963982	699.417258150651\\
1.12056028014007	701.623775915479\\
1.12256128064032	703.818719932846\\
1.12456228114057	706.000886991381\\
1.12656328164082	708.169303062833\\
1.12856428214107	710.322879527391\\
1.13056528264132	712.460756948363\\
1.13256628314157	714.582075889055\\
1.13456728364182	716.686091504335\\
1.13656828414207	718.772173540626\\
1.13856928464232	720.839634448576\\
1.14057028514257	722.888015861949\\
1.14257128564282	724.916744822948\\
1.14457228614307	726.925534852677\\
1.14657328664332	728.913927584899\\
1.14857428714357	730.881751132275\\
1.15057528764382	732.828776311689\\
1.15257628814407	734.754773940021\\
1.15457728864432	736.659629425713\\
1.15657828914457	738.543342768765\\
1.15857928964482	740.405856673397\\
1.16058029014507	742.247171139609\\
1.16258129064532	744.067286167401\\
1.16458229114557	745.866259052552\\
1.16658329164582	747.644261682402\\
1.16858429214607	749.401294056951\\
1.17058529264632	751.137528063535\\
1.17258629314657	752.853135589496\\
1.17458729364682	754.54823122639\\
1.17658829414707	756.222929565778\\
1.17858929464732	757.877459790777\\
1.18059029514757	759.511879197168\\
1.18259129564782	761.126474263846\\
1.18459229614807	762.721302286593\\
1.18659329664832	764.296592448526\\
1.18859429714857	765.852402045424\\
1.19059529764882	767.388902964626\\
1.19259629814907	768.906209797692\\
1.19459729864932	770.404437136179\\
1.19659829914957	771.883642275868\\
1.19859929964982	773.343939808318\\
1.20060030015008	774.785387029308\\
1.20260130065033	776.208041234618\\
1.20460230115058	777.611959720027\\
1.20660330165083	778.997027893977\\
1.20860430215108	780.363360348025\\
1.21060530265133	781.710785194834\\
1.21260630315158	783.039417025963\\
1.21460730365183	784.349026658294\\
1.21660830415208	785.639556796046\\
1.21860930465233	786.910892847662\\
1.22061030515258	788.162862925802\\
1.22261130565283	789.395295143129\\
1.22461230615308	790.608017612303\\
1.22661330665333	791.800743854427\\
1.22861430715358	792.973244686382\\
1.23061530765383	794.125290925052\\
1.23261630815408	795.256538795758\\
1.23461730865433	796.366587228045\\
1.23661830915458	797.455207038793\\
1.23861930965483	798.521939861768\\
1.24062031015508	799.566384626512\\
1.24262131065533	800.588197558348\\
1.24462231115558	801.586805699481\\
1.24662331165583	802.561807979456\\
1.24862431215608	803.512688736255\\
1.25062531265633	804.438989603643\\
1.25262631315658	805.340080328045\\
1.25462731365683	806.215559839005\\
1.25662831415708	807.064797882948\\
1.25862931465733	807.887278797858\\
1.26063031515758	808.682372330161\\
1.26263131565783	809.449620113621\\
1.26463231615808	810.188391894662\\
1.26663331665833	810.89822930705\\
1.26863431715858	811.578559392988\\
1.27063531765883	812.228866490462\\
1.27263631815908	812.848749529014\\
1.27463731865933	813.437750142409\\
1.27663831915958	813.995467260189\\
1.27863931965983	814.521671699237\\
1.28064032016008	815.016076980655\\
1.28264132066033	815.478511217105\\
1.28464232116058	815.908917112808\\
1.28664332166083	816.307351963542\\
1.28864432216108	816.673987656646\\
1.29064532266133	817.009053375238\\
1.29264632316158	817.313007485555\\
1.29464732366183	817.586422945392\\
1.29664832416208	817.829987304102\\
1.29864932466233	818.044674589937\\
1.30065032516258	818.231516126929\\
1.30265132566283	818.391829718007\\
1.30465232616308	818.526990461878\\
1.30665332666333	818.638659936149\\
1.30865432716358	818.728671605764\\
1.31065532766383	818.799030823007\\
1.31265632816408	818.851800235938\\
1.31465732866433	818.889386267299\\
1.31665832916458	818.914195339828\\
1.31865932966483	818.928748467824\\
1.32066033016508	818.935795848704\\
1.32266133066533	818.938030384105\\
1.32466233116558	818.938202271444\\
1.32666333166583	818.939176299696\\
1.32866433216608	818.943702666277\\
1.33066533266633	818.954531568605\\
1.33266633316658	818.974355908317\\
1.33466733366683	819.005868587049\\
1.33666833416708	819.051533323321\\
1.33866933466733	819.113641948313\\
1.34067033516758	819.194543588985\\
1.34267133566783	819.296186301842\\
1.34467233616808	819.420575439165\\
1.34667333666833	819.56931528278\\
1.34867433716858	819.744010114516\\
1.35067533766883	819.945977737299\\
1.35267633816908	820.176421362501\\
1.35467733866933	820.436257722593\\
1.35667833916958	820.726346254268\\
1.35867933966983	821.0473172111\\
1.36068034017009	821.399686255105\\
1.36268134067034	821.783739865181\\
1.36468234117059	822.199707224447\\
1.36668334167084	822.647702924459\\
1.36868434217109	823.12766966944\\
1.37068534267134	823.63955016361\\
1.37268634317159	824.183115223851\\
1.37468734367184	824.758021075485\\
1.37668834417209	825.364038535395\\
1.37868934467234	826.000651941565\\
1.38069034517259	826.667517519318\\
1.38269134567284	827.364062310858\\
1.38469234617309	828.08988524573\\
1.38669334667334	828.844298774579\\
1.38869434717359	829.626959122727\\
1.39069534767384	830.437121445042\\
1.39269634817409	831.274327375287\\
1.39469734867434	832.138061251447\\
1.39669834917459	833.027692819947\\
1.39869934967484	833.942649122991\\
1.40070035017509	834.882529090124\\
1.40270135067534	835.84670246777\\
1.40470235117559	836.834710889694\\
1.40670335167584	837.846095989659\\
1.40870435217609	838.880399401429\\
1.41070535267634	839.937105462988\\
1.41270635317659	841.01592769544\\
1.41470735367684	842.116350436769\\
1.41670835417709	843.238087208076\\
1.41870935467734	844.380736938905\\
1.42071035517759	845.544070446139\\
1.42271135567784	846.7277439551\\
1.42471235617809	847.93147098689\\
1.42671335667834	849.155022358392\\
1.42871435717859	850.398226182267\\
1.43071535767884	851.660795979617\\
1.43271635817909	852.942731750442\\
1.43471735867934	854.243747015846\\
1.43671835917959	855.563784480048\\
1.43871935967984	856.902729551489\\
1.44072036018009	858.26052493439\\
1.44272136068034	859.637170628751\\
1.44472236118059	861.032552043013\\
1.44672336168084	862.446726472955\\
1.44872436218109	863.879693918577\\
1.45072536268134	865.331511675659\\
1.45272636318159	866.802179744201\\
1.45472736368184	868.291812715761\\
1.45672836418209	869.8005251819\\
1.45872936468234	871.328317142616\\
1.46073036518259	872.875417781028\\
1.46273136568284	874.441884392916\\
1.46473236618309	876.027888865618\\
1.46673336668334	877.633545790692\\
1.46873436718359	879.258969759699\\
1.47073536768384	880.904447251535\\
1.47273636818409	882.57003556198\\
1.47473736868434	884.255906578373\\
1.47673836918459	885.962232188052\\
1.47873936968484	887.689126982577\\
1.48074037018509	889.436762849285\\
1.48274137068534	891.205311675515\\
1.48474237118559	892.994773461268\\
1.48674337168584	894.805377389661\\
1.48874437218609	896.637066164914\\
1.49074537268634	898.490011674367\\
1.49274637318659	900.36415662224\\
1.49474737368684	902.259501008533\\
1.49674837418709	904.175987537466\\
1.49874937468734	906.11355891326\\
1.50075037518759	908.071985952797\\
1.50275137568784	910.051154064517\\
1.50475237618809	912.050776769524\\
1.50675337668834	914.070567588919\\
1.50875437718859	916.110125452246\\
1.51075537768884	918.168991993269\\
1.51275637818909	920.246766141532\\
1.51475737868934	922.342760347679\\
1.51675837918959	924.456401653917\\
1.51875937968984	926.586887919331\\
1.5207603801901	928.733531594568\\
1.52276138069035	930.895301355597\\
1.5247623811906	933.071395061504\\
1.52676338169085	935.260724092478\\
1.5287643821911	937.462199828709\\
1.53076538269135	939.674619058828\\
1.5327663831916	941.896835867242\\
1.53476738369185	944.127532451025\\
1.5367683841921	946.365391007247\\
1.53876938469235	948.6090364372\\
1.5407703851926	950.857093642175\\
1.54277138569285	953.108072931906\\
1.5447723861931	955.360599207683\\
1.54677338669335	957.61312548346\\
1.5487743871936	959.864276660529\\
1.55077538769385	962.112563048623\\
1.5527763881941	964.356494957473\\
1.55477738869435	966.594754584152\\
1.5567783891946	968.825909534171\\
1.55877938969485	971.048699300381\\
1.5607803901951	973.261748784074\\
1.56278139069535	975.463912069659\\
1.5647823911956	977.653928649987\\
1.56678339169585	979.830824496807\\
1.5687843921961	981.993568286088\\
1.57078539269635	984.141071398018\\
1.5727863931966	986.272588987463\\
1.57478739369685	988.387204321952\\
1.5767883941971	990.484287147911\\
1.57878939469735	992.563149915984\\
1.5807903951976	994.623219668377\\
1.58279139569785	996.663923447295\\
1.5847923961981	998.684917478059\\
1.58679339669835	1000.68574339444\\
1.5887943971986	1002.66611471753\\
1.59079539769885	1004.62580226421\\
1.5927963981991	1006.56463414715\\
1.59479739869935	1008.48238118324\\
1.5967983991996	1010.37898607668\\
1.59879939969985	1012.2543915317\\
1.6008004002001	1014.1085975483\\
1.60280140070035	1015.94160412649\\
1.6048024012006	1017.75341126625\\
1.60680340170085	1019.54419085493\\
1.6088044022011	1021.31394289253\\
1.61080540270135	1023.06283926639\\
1.6128064032016	1024.79099456806\\
1.61480740370185	1026.49852338911\\
1.6168084042021	1028.18559761687\\
1.61880940470235	1029.85238913869\\
1.6208104052026	1031.49906984189\\
1.62281140570285	1033.12569702227\\
1.6248124062031	1034.73255715872\\
1.62681340670335	1036.31970754701\\
1.6288144072036	1037.88732007448\\
1.63081540770385	1039.43556662849\\
1.6328164082041	1040.96456180057\\
1.63481740870435	1042.47436288652\\
1.6368184092046	1043.96514177367\\
1.63881940970485	1045.43695575781\\
1.6408204102051	1046.8898621347\\
1.64282141070535	1048.32397549591\\
1.6448224112056	1049.73929584144\\
1.64682341170585	1051.13576587552\\
1.6488244122061	1052.51355748547\\
1.65082541270635	1053.87244148818\\
1.6528264132066	1055.21253247521\\
1.65482741370685	1056.533715855\\
1.6568284142071	1057.835877036\\
1.65882941470735	1059.11890142663\\
1.6608304152076	1060.38267443535\\
1.66283141570785	1061.62702417482\\
1.6648324162081	1062.85177875769\\
1.66683341670835	1064.05670900085\\
1.6688344172086	1065.24158572118\\
1.67083541770885	1066.40612243978\\
1.6728364182091	1067.55003267776\\
1.67483741870935	1068.67302995622\\
1.6768384192096	1069.7747132047\\
1.67883941970985	1070.8547959443\\
1.68084042021011	1071.91287710456\\
1.68284142071036	1072.94849831926\\
1.68484242121061	1073.96125851794\\
1.68684342171086	1074.95069933435\\
1.68884442221111	1075.91630510648\\
1.69084542271136	1076.85767476388\\
1.69284642321161	1077.77412075719\\
1.69484742371186	1078.66529931174\\
1.69684842421211	1079.53058017395\\
1.69884942471236	1080.36939038602\\
1.70085042521261	1081.18121428594\\
1.70285142571286	1081.96547891591\\
1.70485242621311	1082.72166861393\\
1.70685342671336	1083.44915312641\\
1.70885442721361	1084.14747408711\\
1.71085542771386	1084.81611583403\\
1.71285642821411	1085.45456270514\\
1.71485742871436	1086.06241363\\
1.71685842921461	1086.63921024236\\
1.71885942971486	1087.18460876754\\
1.72086043021511	1087.69838002243\\
1.72286143071536	1088.18029482392\\
1.72486243121561	1088.63023858044\\
1.72686343171586	1089.04809670042\\
1.72886443221611	1089.43404107122\\
1.73086543271636	1089.78830087595\\
1.73286643321661	1090.11110529773\\
1.73486743371686	1090.40302729435\\
1.73686843421711	1090.66463982361\\
1.73886943471736	1090.89680232219\\
1.74087043521761	1091.10048881836\\
1.74287143571786	1091.27678793192\\
1.74487243621811	1091.42701746581\\
1.74687343671836	1091.55283899762\\
1.74887443721861	1091.6557995134\\
1.75087543771886	1091.73784706967\\
1.75287643821911	1091.80104431447\\
1.75487743871936	1091.84756848743\\
1.75687843921961	1091.87988330708\\
1.75887943971986	1091.90033790037\\
1.76088044022011	1091.91168246471\\
1.76288144072036	1091.91649531019\\
1.76488244122061	1091.91758393\\
1.76688344172086	1091.91769852156\\
1.76888444222111	1091.91976116962\\
1.77088544272136	1091.92640748004\\
1.77288644322161	1091.94044494602\\
1.77488744372186	1091.9645664692\\
1.77688844422211	1092.00129306387\\
1.77888944472236	1092.05314574433\\
1.78089044522261	1092.12235904598\\
1.78289144572286	1092.21111020844\\
1.78489244622311	1092.32146187979\\
1.78689344672336	1092.45519022917\\
1.78889444722361	1092.6139568342\\
1.79089544772386	1092.79919408937\\
1.79289644822411	1093.01216250182\\
1.79489744872436	1093.25400798714\\
1.79689844922461	1093.52558998203\\
1.79889944972486	1093.82771062741\\
1.80090045022511	1094.16088558527\\
1.80290145072536	1094.52563051765\\
1.80490245122561	1094.92217460766\\
1.80690345172586	1095.35063244686\\
1.80890445222611	1095.81117592259\\
1.81090545272636	1096.30369044328\\
1.81290645322661	1096.82794682583\\
1.81490745372686	1097.38383047866\\
1.81690845422711	1097.97099762711\\
1.81890945472736	1098.58898990494\\
1.82091045522761	1099.23752083325\\
1.82291145572786	1099.91607475003\\
1.82491245622811	1100.62413599325\\
1.82691345672836	1101.3611889009\\
1.82891445722861	1102.12677510676\\
1.83091545772886	1102.92026435723\\
1.83291645822911	1103.74108369454\\
1.83491745872936	1104.58871745666\\
1.83691845922961	1105.46264998157\\
1.83891945972987	1106.3622510157\\
1.84092046023012	1107.28706219282\\
1.84292146073037	1108.23651055514\\
1.84492246123062	1109.21008044062\\
1.84692346173087	1110.20731348305\\
1.84892446223112	1111.2276940204\\
1.85092546273137	1112.27082098221\\
1.85292646323162	1113.33623600226\\
1.85492746373187	1114.42353801008\\
1.85692846423212	1115.53232593521\\
1.85892946473237	1116.66231329877\\
1.86093046523262	1117.81309903029\\
1.86293146573287	1118.98439665088\\
1.86493246623312	1120.17597697741\\
1.86693346673337	1121.38749623521\\
1.86893446723362	1122.61872524117\\
1.87093546773387	1123.86954940372\\
1.87293646823412	1125.13973953975\\
1.87493746873437	1126.42912376191\\
1.87693846923462	1127.73764477443\\
1.87893946973487	1129.06507339419\\
1.88094047023512	1130.41140962119\\
1.88294147073537	1131.77659615964\\
1.88494247123562	1133.16057571378\\
1.88694347173587	1134.56329098782\\
1.88894447223612	1135.98485657332\\
1.89094547273637	1137.42515787872\\
1.89294647323662	1138.88436679136\\
1.89494747373687	1140.36248331124\\
1.89694847423712	1141.85956473414\\
1.89894947473737	1143.37572565161\\
1.90095047523762	1144.91108065522\\
1.90295147573787	1146.46568704075\\
1.90495247623812	1148.03983128709\\
1.90695347673837	1149.63345609847\\
1.90895447723862	1151.24684795378\\
1.91095547773887	1152.88012144458\\
1.91295647823912	1154.53344845821\\
1.91495747873937	1156.20694358623\\
1.91695847923962	1157.90083601175\\
1.91895947973987	1159.61518303056\\
1.92096048024012	1161.35021382578\\
1.92296148074037	1163.10598569318\\
1.92496248124062	1164.88272781588\\
1.92696348174087	1166.68044019388\\
1.92896448224112	1168.49929471452\\
1.93096548274137	1170.33929137781\\
1.93296648324162	1172.20054477529\\
1.93496748374187	1174.08299761119\\
1.93696848424212	1175.98664988551\\
1.93896948474237	1177.9113870067\\
1.94097048524262	1179.85709438318\\
1.94297148574287	1181.82371471919\\
1.94497248624312	1183.81090424004\\
1.94697348674337	1185.81843376262\\
1.94897448724362	1187.84601680803\\
1.95097548774387	1189.89319501003\\
1.95297648824412	1191.95956729817\\
1.95497748874437	1194.04456071465\\
1.95697848924462	1196.1475450059\\
1.95897948974487	1198.26783262256\\
1.96098049024512	1200.40473601528\\
1.96298149074537	1202.55739574737\\
1.96498249124562	1204.72495238213\\
1.96698349174587	1206.90631729975\\
1.96898449224612	1209.10051647198\\
1.97098549274637	1211.30640398323\\
1.97298649324662	1213.52283391792\\
1.97498749374687	1215.74854576888\\
1.97698849424712	1217.9822217332\\
1.97898949474737	1220.22248671216\\
1.98099049524762	1222.46802290284\\
1.98299149574787	1224.71739791074\\
1.98499249624812	1226.96917934139\\
1.98699349674837	1229.2218775045\\
1.98899449724862	1231.4740600056\\
1.99099549774887	1233.72423715442\\
1.99299649824912	1235.97097655647\\
1.99499749874937	1238.21284581726\\
1.99699849924962	1240.44846983808\\
1.99899949974988	1242.676530816\\
2.00100050025013	1244.89565365232\\
2.00300150075038	1247.10457783989\\
2.00500250125063	1249.30210016734\\
2.00700350175088	1251.48713201485\\
2.00900450225113	1253.65858476261\\
2.01100550275138	1255.81548438238\\
2.01300650325163	1257.9568568459\\
2.01500750375188	1260.08184271649\\
2.01700850425213	1262.18969714899\\
2.01900950475238	1264.27973259407\\
2.02101050525263	1266.35126150237\\
2.02301150575288	1268.40382550764\\
2.02501250625313	1270.43685165211\\
2.02701350675338	1272.44999616108\\
2.02901450725363	1274.4428579641\\
2.03101550775388	1276.41515058228\\
2.03301650825413	1278.36670212827\\
2.03501750875438	1280.29728341897\\
2.03701850925463	1282.2067798628\\
2.03901950975488	1284.09507686821\\
2.04102051025513	1285.96217443521\\
2.04302151075538	1287.80807256378\\
2.04502251125563	1289.63277125393\\
2.04702351175588	1291.43632780145\\
2.04902451225613	1293.21885679788\\
2.05102551275638	1294.98047283479\\
2.05302651325663	1296.72123320795\\
2.05502751375688	1298.44125250894\\
2.05702851425713	1300.14081721663\\
2.05902951475738	1301.81992733104\\
2.06103051525763	1303.47881203528\\
2.06303151575788	1305.11758592092\\
2.06503251625813	1306.73647817106\\
2.06703351675838	1308.33560337727\\
2.06903451725863	1309.91507613111\\
2.07103551775888	1311.47512561569\\
2.07303651825913	1313.0158091268\\
2.07503751875938	1314.53724125599\\
2.07703851925963	1316.0395938906\\
2.07903951975988	1317.52292432641\\
2.08104052026013	1318.98734715499\\
2.08304152076038	1320.4329196721\\
2.08504252126063	1321.85964187776\\
2.08704352176088	1323.26757106773\\
2.08904452226113	1324.65676453781\\
2.09104552276138	1326.0271649922\\
2.09304652326163	1327.37871513514\\
2.09504752376188	1328.71141496661\\
2.09704852426213	1330.02514989507\\
2.09904952476238	1331.31980532894\\
2.10105052526263	1332.59532397246\\
2.10305152576288	1333.85153393829\\
2.10505252626313	1335.08826333908\\
2.10705352676338	1336.30522569593\\
2.10905452726363	1337.50236371308\\
2.11105552776388	1338.67927632006\\
2.11305652826413	1339.83579162953\\
2.11505752876438	1340.97156586682\\
2.11705852926463	1342.08625525725\\
2.11905952976488	1343.17951602614\\
2.12106053026513	1344.25106169459\\
2.12306153076538	1345.30037660059\\
2.12506253126563	1346.32711696947\\
2.12706353176588	1347.33082443498\\
2.12906453226613	1348.31098333511\\
2.13106553276638	1349.26713530362\\
2.13306653326663	1350.19882197428\\
2.13506753376688	1351.1054703893\\
2.13706853426713	1351.98650759087\\
2.13906953476738	1352.84153250854\\
2.14107053526763	1353.66985759296\\
2.14307153576788	1354.47090988634\\
2.14507253626813	1355.24428831821\\
2.14707353676838	1355.98930533921\\
2.14907453726863	1356.70544528735\\
2.15107553776888	1357.39219250059\\
2.15307653826913	1358.04908861271\\
2.15507753876938	1358.6755606659\\
2.15707853926963	1359.27132218128\\
2.15907953976988	1359.83585749682\\
2.16108054027013	1360.36893742941\\
2.16308154077039	1360.87027550015\\
2.16508254127064	1361.33964252592\\
2.16708354177089	1361.77703850673\\
2.16908454227114	1362.18240614678\\
2.17108554277139	1362.55586003765\\
2.17308654327164	1362.897743954\\
2.17508754377189	1363.2084589663\\
2.17708854427214	1363.488406145\\
2.17908954477239	1363.73838763102\\
2.18109054527264	1363.95926286104\\
2.18309154577289	1364.1520631591\\
2.18509254627314	1364.31799173657\\
2.18709354677339	1364.45848098794\\
2.18909454727364	1364.57513519503\\
2.19109554777389	1364.66967323123\\
2.19309654827414	1364.74410044881\\
2.19509754877439	1364.80053679163\\
2.19709854927464	1364.84115949931\\
2.19909954977489	1364.86854688191\\
2.20110055027514	1364.88510536219\\
2.20310155077539	1364.89358513756\\
2.20510255127564	1364.8965645181\\
2.20710355177589	1364.89696558855\\
2.20910455227614	1364.89753854635\\
2.21110555277639	1364.9010335889\\
2.21310655327664	1364.9102582094\\
2.21510755377689	1364.92796260527\\
2.21710855427714	1364.95672508659\\
2.21910955477739	1364.99906666765\\
2.22111055527764	1365.05745106697\\
2.22311155577789	1365.13405552418\\
2.22511255627814	1365.23105727889\\
2.22711355677839	1365.35034709184\\
2.22911455727864	1365.4936438364\\
2.23111555777889	1365.66260909019\\
2.23311655827914	1365.85856065612\\
2.23511755877939	1366.08275904136\\
2.23711855927964	1366.33612097836\\
2.23911955977989	1366.61956319961\\
2.24112056028014	1366.93377325446\\
2.24312156078039	1367.27926680493\\
2.24512256128064	1367.65644492146\\
2.24712356178089	1368.06547949141\\
2.24912456228114	1368.5065424021\\
2.25112556278139	1368.97957635776\\
2.25312656328164	1369.48452406261\\
2.25512756378189	1370.02121362931\\
2.25712856428214	1370.58935857896\\
2.25912956478239	1371.18867243266\\
2.26113056528264	1371.81869682419\\
2.26313156578289	1372.47903068308\\
2.26513256628314	1373.16915834731\\
2.26713356678339	1373.88862145066\\
2.26913456728364	1374.63696162688\\
2.27113556778389	1375.41349132662\\
2.27313656828414	1376.21775218365\\
2.27513756878439	1377.04917124016\\
2.27713856928464	1377.90711824259\\
2.27913956978489	1378.79119212048\\
2.28114057028514	1379.70070532447\\
2.28314157078539	1380.63519948832\\
2.28514257128564	1381.59415895004\\
2.28714357178589	1382.57706804758\\
2.28914457228614	1383.58341111895\\
2.29114557278639	1384.6127297979\\
2.29314657328664	1385.66462301398\\
2.29514757378689	1386.73868969674\\
2.29714857428714	1387.83447147992\\
2.29914957478739	1388.95156729309\\
2.30115057528764	1390.08974795312\\
2.30315157578789	1391.24861238955\\
2.30515257628814	1392.42787412349\\
2.30715357678839	1393.62724667603\\
2.30915457728864	1394.84655815985\\
2.31115557778889	1396.08546480026\\
2.31315657828914	1397.34390930149\\
2.31515757878939	1398.62160518463\\
2.31715857928964	1399.91849515391\\
2.31915957978989	1401.23440732199\\
2.32116058029015	1402.56928439308\\
2.3231615807904	1403.92301177564\\
2.32516258129065	1405.29553217387\\
2.3271635817909	1406.68684558779\\
2.32916458229115	1408.09695201739\\
2.3311655827914	1409.52585146266\\
2.33316658329165	1410.97354392362\\
2.3351675837919	1412.44014399182\\
2.33716858429215	1413.92565166725\\
2.3391695847924	1415.43018154149\\
2.34117058529265	1416.95384820608\\
2.3431715857929	1418.49676625259\\
2.34517258629315	1420.05899297679\\
2.3471735867934	1421.64075756181\\
2.34917458729365	1423.24211730342\\
2.3511755877939	1424.86324408896\\
2.35317658829415	1426.50436710155\\
2.3551775887944	1428.16554363698\\
2.35717858929465	1429.84700287835\\
2.3591795897949	1431.54885941722\\
2.36118059029515	1433.27128514095\\
2.3631815907954	1435.01445193685\\
2.36518259129565	1436.77841710072\\
2.3671835917959	1438.56335251989\\
2.36918459229615	1440.36931549015\\
2.3711855927964	1442.19647789882\\
2.37318659329665	1444.04478245013\\
2.3751875937969	1445.91434373564\\
2.37718859429715	1447.80510445957\\
2.3791895947974	1449.71700732615\\
2.38119059529765	1451.64999503958\\
2.3831915957979	1453.60389571253\\
2.38519259629815	1455.57859475345\\
2.3871935967984	1457.57380568344\\
2.38919459729865	1459.58924202359\\
2.3911955977989	1461.62450270345\\
2.39319659829915	1463.67924394835\\
2.3951975987994	1465.75295009627\\
2.39719859929965	1467.84504818941\\
2.3991995997999	1469.9549079742\\
2.40120060030015	1472.08184190128\\
2.4032016008004	1474.22504782975\\
2.40520260130065	1476.38360902712\\
2.4072036018009	1478.55666605672\\
2.40920460230115	1480.7431875945\\
2.4112056028014	1482.94214231643\\
2.41320660330165	1485.15226971537\\
2.4152076038019	1487.3724811715\\
2.41720860430215	1489.6014588819\\
2.4192096048024	1491.83788504363\\
2.42121060530265	1494.08038455799\\
2.4232116058029	1496.32763962205\\
2.42521260630315	1498.57810324977\\
2.4272136068034	1500.83040034243\\
2.42921460730365	1503.08304120976\\
2.4312156078039	1505.33459345729\\
2.43321660830415	1507.58362469052\\
2.4352176088044	1509.82864521918\\
2.43721860930465	1512.06822264879\\
2.4392196098049	1514.30103917641\\
2.44122061030515	1516.52577699912\\
2.4432216108054	1518.74100372244\\
2.44522261130565	1520.94563072654\\
2.4472236118059	1523.13845480007\\
2.44922461230615	1525.31827273164\\
2.4512256128064	1527.4841677888\\
2.45322661330665	1529.63505135172\\
2.4552276138069	1531.77012127949\\
2.45722861430715	1533.88846083965\\
2.4592296148074	1535.98943977862\\
2.46123061530765	1538.07231325125\\
2.4632316158079	1540.13645100399\\
2.46523261630815	1542.18145196637\\
2.4672336168084	1544.2067431806\\
2.46923461730865	1546.211980872\\
2.4712356178089	1548.19682126589\\
2.47323661830915	1550.16103517916\\
2.4752376188094	1552.10433613291\\
2.47723861930965	1554.02666683135\\
2.4792396198099	1555.92791268293\\
2.48124062031015	1557.80790180032\\
2.48324162081041	1559.66669147928\\
2.48524262131066	1561.50428171982\\
2.48724362181091	1563.32072981773\\
2.48924462231116	1565.11603577299\\
2.49124562281141	1566.89037147295\\
2.49324662331166	1568.64379421339\\
2.49524762381191	1570.37641858586\\
2.49724862431216	1572.08841647771\\
2.49924962481241	1573.7799024805\\
2.50125062531266	1575.45110577734\\
2.50325162581291	1577.10208366401\\
2.50525262631316	1578.73312261941\\
2.50725362681341	1580.34422264353\\
2.50925462731366	1581.93561291951\\
2.51125562781391	1583.50746533467\\
2.51325662831416	1585.05995177636\\
2.51525762881441	1586.59307224457\\
2.51725862931466	1588.10705592242\\
2.51925962981491	1589.6019601057\\
2.52126063031516	1591.07784209018\\
2.52326163081541	1592.53487376319\\
2.52526263131566	1593.97305512475\\
2.52726363181591	1595.39244347063\\
2.52926463231616	1596.79303880083\\
2.53126563281641	1598.17484111534\\
2.53326663331666	1599.53785041418\\
2.53526763381691	1600.88206669734\\
2.53726863431716	1602.20731807747\\
2.53926963481741	1603.51366185037\\
2.54127063531766	1604.80081153713\\
2.54327163581791	1606.06882443354\\
2.54527263631816	1607.31741406069\\
2.54727363681841	1608.54640853124\\
2.54927463731866	1609.75563595787\\
2.55127563781891	1610.94486715744\\
2.55327663831916	1612.11387294685\\
2.55527763881941	1613.26225225563\\
2.55727863931966	1614.38983319644\\
2.55927963981991	1615.49621469884\\
2.56128064032016	1616.58099569236\\
2.56328164082041	1617.64388969811\\
2.56528264132066	1618.68443834985\\
2.56728364182091	1619.70218328134\\
2.56928464232116	1620.69672342213\\
2.57128564282141	1621.6675431102\\
2.57328664332166	1622.61418397931\\
2.57528764382191	1623.53613036746\\
2.57728864432216	1624.43280931684\\
2.57928964482241	1625.30376246121\\
2.58129064532266	1626.14835954702\\
2.58329164582291	1626.96608491223\\
2.58529264632316	1627.75642289483\\
2.58729364682341	1628.51874324125\\
2.58929464732366	1629.2524729937\\
2.59129564782391	1629.95721108171\\
2.59329664832416	1630.63232725171\\
2.59529764882441	1631.27736313747\\
2.59729864932466	1631.89186037275\\
2.59929964982491	1632.47547518287\\
2.60130065032516	1633.02769190581\\
2.60330165082541	1633.54839595003\\
2.60530265132566	1634.03718624505\\
2.60730365182591	1634.49406279089\\
2.60930465232616	1634.91891099598\\
2.61130565282641	1635.3117881561\\
2.61330665332666	1635.67286615859\\
2.61530765382691	1636.00248877813\\
2.61730865432716	1636.30105708518\\
2.61930965482741	1636.5692013333\\
2.62131065532766	1636.80766636763\\
2.62331165582791	1637.01742621643\\
2.62531265632816	1637.1995694995\\
2.62731365682841	1637.355356724\\
2.62931465732866	1637.48627758018\\
2.63131565782891	1637.59405094145\\
2.63331665832916	1637.68051027273\\
2.63531765882941	1637.74760363054\\
2.63731865932966	1637.79756555028\\
2.63931965982992	1637.8327451589\\
2.64132066033017	1637.85554887915\\
2.64332166083042	1637.86861231687\\
2.64532266133067	1637.87462837372\\
2.64732366183092	1637.87628995133\\
2.64932466233117	1637.87640454289\\
2.65132566283142	1637.8777796416\\
2.65332666333167	1637.88322274065\\
2.65532766383192	1637.89542674169\\
2.65732866433217	1637.91714184212\\
2.65932966483242	1637.95094635203\\
2.66133066533267	1637.99930398994\\
2.66333166583292	1638.06450658703\\
2.66533266633317	1638.14890327025\\
2.66733366683342	1638.25438480034\\
2.66933466733367	1638.38289923378\\
2.67133566783392	1638.53605085242\\
2.67333666833417	1638.71544393808\\
2.67533766883442	1638.92228170212\\
2.67733866933467	1639.15776735592\\
2.67933966983492	1639.42287492773\\
2.68134067033517	1639.71829196689\\
2.68334167083542	1640.04470602278\\
2.68534267133567	1640.40251816584\\
2.68734367183592	1640.79212946653\\
2.68934467233617	1641.21371181219\\
2.69134567283642	1641.66726520281\\
2.69334667333667	1642.15284693418\\
2.69534767383692	1642.67028511897\\
2.69734867433717	1643.21929327826\\
2.69934967483742	1643.79969952473\\
2.70135067533767	1644.41104549213\\
2.70335167583792	1645.0529874058\\
2.70535267633817	1645.72506689949\\
2.70735367683842	1646.42676831118\\
2.70935467733867	1647.15757597887\\
2.71135567783892	1647.91697424054\\
2.71335667833917	1648.70444743417\\
2.71535767883942	1649.51942260196\\
2.71735867933967	1650.36126949035\\
2.71935967983992	1651.22952973309\\
2.72136068034017	1652.12363037239\\
2.72336168084042	1653.04299845046\\
2.72536268134067	1653.98711830527\\
2.72736368184092	1654.9555315706\\
2.72936468234117	1655.94766528865\\
2.73136568284142	1656.96306109318\\
2.73336668334167	1658.00126061796\\
2.73536768384192	1659.06186279253\\
2.73736868434217	1660.14446654643\\
2.73936968484242	1661.2486708092\\
2.74137068534267	1662.3740745104\\
2.74337168584292	1663.52033387534\\
2.74537268634317	1664.6872197209\\
2.74737368684342	1665.87438827241\\
2.74937468734367	1667.08161034675\\
2.75137568784392	1668.30859946503\\
2.75337668834417	1669.55518373989\\
2.75537768884442	1670.82119128401\\
2.75737868934467	1672.10645021005\\
2.75937968984492	1673.41078863067\\
2.76138069034517	1674.7340919543\\
2.76338169084542	1676.07636018095\\
2.76538269134567	1677.43747871907\\
2.76738369184592	1678.81733297708\\
2.76938469234617	1680.21603754655\\
2.77138569284642	1681.63347783593\\
2.77338669334667	1683.06971114098\\
2.77538769384692	1684.52479475749\\
2.77738869434717	1685.99878598125\\
2.77938969484742	1687.49174210802\\
2.78139069534767	1689.00377772937\\
2.78339169584792	1690.53495014108\\
2.78539269634817	1692.0853739347\\
2.78739369684842	1693.65527829336\\
2.78939469734867	1695.24466321705\\
2.79139569784892	1696.8537578889\\
2.79339669834917	1698.48273419624\\
2.79539769884942	1700.13164943484\\
2.79739869934967	1701.80073278784\\
2.79939969984992	1703.49015614256\\
2.80140070035018	1705.20009138635\\
2.80340170085043	1706.93059581498\\
2.80540270135068	1708.68189861158\\
2.80740370185093	1710.45405707192\\
2.80940470235118	1712.24724308334\\
2.81140570285143	1714.06145664584\\
2.81340670335168	1715.89692694254\\
2.81540770385193	1717.75353938189\\
2.81740870435218	1719.63135125965\\
2.81940970485243	1721.53041987161\\
2.82141070535268	1723.45057333043\\
2.82341170585293	1725.39175434033\\
2.82541270635318	1727.35384830976\\
2.82741370685343	1729.33662605559\\
2.82941470735368	1731.33974380315\\
2.83141570785393	1733.36302966509\\
2.83341670835418	1735.40596797941\\
2.83541770885443	1737.46821497142\\
2.83741870935468	1739.54919768334\\
2.83941970985493	1741.64834315736\\
2.84142071035518	1743.76496384413\\
2.84342171085543	1745.89831489852\\
2.84542271135568	1748.0476514754\\
2.84742371185593	1750.2119995465\\
2.84942471235618	1752.39049967515\\
2.85142571285643	1754.58200594575\\
2.85342671335668	1756.78542973848\\
2.85542771385693	1758.99968243354\\
2.85742871435718	1761.22350352379\\
2.85942971485743	1763.4555179105\\
2.86143071535768	1765.69452238231\\
2.86343171585793	1767.93902724895\\
2.86543271635818	1770.1877147075\\
2.86743371685843	1772.43903777191\\
2.86943471735868	1774.69167863925\\
2.87143571785893	1776.94409032347\\
2.87343671835918	1779.1947831343\\
2.87543771885943	1781.44243926882\\
2.87743871935968	1783.68551174098\\
2.87943971985993	1785.92256815628\\
2.88144072036018	1788.15240530337\\
2.88344172086043	1790.37353349198\\
2.88544272136068	1792.58480680651\\
2.88744372186093	1794.78490744403\\
2.88944472236118	1796.97274678474\\
2.89144572286143	1799.14723620882\\
2.89344672336168	1801.30740168802\\
2.89544772386193	1803.45215460253\\
2.89744872436218	1805.580807403\\
2.89944972486243	1807.69244335696\\
2.90145072536268	1809.78637491504\\
2.90345172586293	1811.86197182368\\
2.90545272636318	1813.91871842086\\
2.90745372686343	1815.95604174879\\
2.90945472736368	1817.97348344123\\
2.91145572786393	1819.97081431505\\
2.91345672836418	1821.94757600403\\
2.91545772886443	1823.90365391661\\
2.91745872936468	1825.83876157388\\
2.91945972986493	1827.75284168008\\
2.92146073036518	1829.64577964363\\
2.92346173086543	1831.51746087298\\
2.92546273136568	1833.3679999597\\
2.92746373186593	1835.19728231221\\
2.92946473236618	1837.00542252209\\
2.93146573286643	1838.79253518088\\
2.93346673336668	1840.55862028859\\
2.93546773386693	1842.30390702834\\
2.93746873436718	1844.0284526959\\
2.93946973486743	1845.73242917862\\
2.94147073536768	1847.41595106805\\
2.94347173586793	1849.07919025154\\
2.94547273636818	1850.7223759122\\
2.94747373686843	1852.3455653458\\
2.94947473736868	1853.94893043969\\
2.95147573786893	1855.532700377\\
2.95347673836918	1857.09693245348\\
2.95547773886943	1858.64179855649\\
2.95747873936968	1860.16741327759\\
2.95947973986994	1861.67389120833\\
2.96148074037019	1863.16134694027\\
2.96348174087044	1864.62989506496\\
2.96548274137069	1866.07953558243\\
2.96748374187094	1867.51032578843\\
2.96948474237119	1868.92238027453\\
2.97148574287144	1870.31558444917\\
2.97348674337169	1871.69005290391\\
2.97548774387194	1873.04572834296\\
2.97748874437219	1874.38249617478\\
2.97948974487244	1875.70035639936\\
2.98149074537269	1876.99919442515\\
2.98349174587294	1878.27889566057\\
2.98549274637319	1879.5392882183\\
2.98749374687344	1880.78025750677\\
2.98949474737369	1882.00157434287\\
2.99149574787394	1883.20300954348\\
2.99349674837419	1884.38433392549\\
2.99549774887444	1885.54526100998\\
2.99749874937469	1886.68556161385\\
2.99949974987494	1887.80489196242\\
3.00150075037519	1888.90285098523\\
3.00350175087544	1889.97915220338\\
3.00550275137569	1891.03333725064\\
3.00750375187594	1892.06500505655\\
3.00950475237619	1893.07369725488\\
3.01150575287644	1894.05901277517\\
3.01350675337669	1895.0204359554\\
3.01550775387694	1895.95745113355\\
3.01750875437719	1896.8695999434\\
3.01950975487744	1897.75625213137\\
3.02151075537769	1898.61694933121\\
3.02351175587794	1899.45111858514\\
3.02551275637819	1900.25818693536\\
3.02751375687844	1901.03758142408\\
3.02951475737869	1901.78878638928\\
3.03151575787894	1902.51122887316\\
3.03351675837919	1903.20445050949\\
3.03551775887944	1903.86782104469\\
3.03751875937969	1904.50099670409\\
3.03951975987994	1905.10340452989\\
3.04152076038019	1905.67475804319\\
3.04352176088044	1906.21471346932\\
3.04552276138069	1906.7229270336\\
3.04752376188094	1907.19928414448\\
3.04952476238119	1907.6436129146\\
3.05152576288144	1908.05591334398\\
3.05352676338169	1908.43630002416\\
3.05552776388194	1908.78500213828\\
3.05752876438219	1909.10242075678\\
3.05952976488244	1909.38901424591\\
3.06153076538269	1909.64541285923\\
3.06353176588294	1909.87253332922\\
3.06553276638319	1910.07134968413\\
3.06753376688344	1910.24300783955\\
3.06953476738369	1910.38888289419\\
3.07153576788394	1910.5105218341\\
3.07353676838419	1910.60975812421\\
3.07553776888444	1910.68836793371\\
3.07753876938469	1910.74847120641\\
3.07953976988494	1910.79235977352\\
3.08154077038519	1910.82238276199\\
3.08354177088544	1910.84111848189\\
3.08554277138569	1910.85108794752\\
3.08754377188594	1910.85504135631\\
3.08954477238619	1910.85578620144\\
3.09154577288644	1910.85607268034\\
3.09354677338669	1910.8587082862\\
3.09554777388694	1910.86644321643\\
3.09754877438719	1910.88208496424\\
3.09954977488744	1910.9082118397\\
3.10155077538769	1910.94740215288\\
3.10355177588794	1911.00206232654\\
3.10555277638819	1911.07454148762\\
3.10755377688844	1911.16695957998\\
3.10955477738869	1911.28126466011\\
3.11155577788894	1911.41923289717\\
3.11355677838919	1911.58246857301\\
3.11555777888944	1911.77246137787\\
3.11755877938969	1911.99041452314\\
3.11955977988994	1912.23735933284\\
3.1215607803902	1912.51426983523\\
3.12356178089045	1912.8217189881\\
3.1255627813907	1913.1603943408\\
3.12756378189095	1913.53063966801\\
3.1295647823912	1913.93274144863\\
3.13156578289145	1914.36687157\\
3.1335667833917	1914.83297273634\\
3.13556778389195	1915.33104494765\\
3.1375687843922	1915.86085902081\\
3.13956978489245	1916.4221857727\\
3.1415707853927	1917.01473872442\\
3.14357178589295	1917.63811680552\\
3.1455727863932	1918.29191894555\\
3.14757378689345	1918.97562948248\\
3.1495747873937	1919.68879005008\\
3.15157578789395	1920.43088498633\\
3.1535767883942	1921.20134133344\\
3.15557778889445	1921.9996434294\\
3.1575787893947	1922.82516102062\\
3.15957978989495	1923.67743574088\\
3.1615807903952	1924.55589463237\\
3.16358179089545	1925.45996473731\\
3.1655827913957	1926.3890730979\\
3.16758379189595	1927.34276134789\\
3.1695847923962	1928.32051382528\\
3.17158579289645	1929.32175757227\\
3.1735867933967	1930.34614881419\\
3.17558779389695	1931.39322918479\\
3.1775887943972	1932.46248302206\\
3.17958979489745	1933.55356655133\\
3.1815907953977	1934.66607870213\\
3.18359179589795	1935.79973299558\\
3.1855927963982	1936.95412836121\\
3.18759379689845	1938.12897832013\\
3.1895947973987	1939.32399639343\\
3.19159579789895	1940.53901069379\\
3.1935967983992	1941.77367744651\\
3.19559779889945	1943.02788206005\\
3.1975987993997	1944.30145264707\\
3.19959979989995	1945.59416002444\\
3.2016008004002	1946.9059468964\\
3.20360180090045	1948.23669867137\\
3.2056028014007	1949.58635805358\\
3.20760380190095	1950.95481045147\\
3.2096048024012	1952.34205586504\\
3.21160580290145	1953.74803699851\\
3.2136068034017	1955.17286844344\\
3.21560780390195	1956.61649290405\\
3.2176088044022	1958.07896767612\\
3.21960980490245	1959.56035005544\\
3.2216108054027	1961.06075463355\\
3.22361180590295	1962.58029600201\\
3.2256128064032	1964.11897416084\\
3.22761380690345	1965.67701829313\\
3.2296148074037	1967.25454299047\\
3.23161580790395	1968.85160554862\\
3.2336168084042	1970.4684351507\\
3.23561780890445	1972.10520368405\\
3.2376188094047	1973.76202574023\\
3.23961980990495	1975.43907320657\\
3.2416208104052	1977.13651797043\\
3.24362181090545	1978.85447462335\\
3.2456228114057	1980.59305775689\\
3.24762381190595	1982.35255384996\\
3.2496248124062	1984.13290560677\\
3.25162581290645	1985.93434221044\\
3.2536268134067	1987.75686366097\\
3.25562781390695	1989.60058454992\\
3.2576288144072	1991.4655048773\\
3.25962981490745	1993.35168193887\\
3.2616308154077	1995.25900114308\\
3.26363181590795	1997.18740519415\\
3.2656328164082	1999.13677950052\\
3.26763381690845	2001.10695217486\\
3.2696348174087	2003.09775132982\\
3.27163581790895	2005.10883319073\\
3.2736368184092	2007.13985398291\\
3.27563781890945	2009.1904126359\\
3.27763881940971	2011.26005078348\\
3.27963981990996	2013.34819546783\\
3.2816408204102	2015.45427373117\\
3.28364182091046	2017.57754072837\\
3.28564282141071	2019.71725161428\\
3.28764382191096	2021.87260424801\\
3.28964482241121	2024.04256730551\\
3.29164582291146	2026.22628135009\\
3.29364682341171	2028.42260046616\\
3.29564782391196	2030.63043603392\\
3.29764882441221	2032.84858484199\\
3.29964982491246	2035.07572908744\\
3.30165082541271	2037.31066555891\\
3.30365182591296	2039.5519618619\\
3.30565282641321	2041.79830019349\\
3.30765382691346	2044.04819086341\\
3.30965482741371	2046.30020147717\\
3.31165582791396	2048.55289964029\\
3.31365682841421	2050.80485295827\\
3.31565782891446	2053.05451444507\\
3.31765882941471	2055.30050900199\\
3.31965982991496	2057.54140423452\\
3.32166083041521	2059.77582504397\\
3.32366183091546	2062.00239633163\\
3.32566283141571	2064.21980029457\\
3.32766383191596	2066.42683372141\\
3.32966483241621	2068.62229340079\\
3.33166583291646	2070.80503341712\\
3.33366683341671	2072.97402244637\\
3.33566783391696	2075.12828646028\\
3.33766883441721	2077.26685143061\\
3.33966983491746	2079.38891521643\\
3.34167083541771	2081.49367567685\\
3.34367183591796	2083.58050255827\\
3.34567283641821	2085.64882290292\\
3.34767383691846	2087.6980064572\\
3.34967483741871	2089.72759485489\\
3.35167583791896	2091.73718702553\\
3.35367683841921	2093.72649649023\\
3.35567783891946	2095.6951794743\\
3.35767883941971	2097.64306409041\\
3.35967983991996	2099.56997845121\\
3.36168084042021	2101.47580796515\\
3.36368184092046	2103.3603807449\\
3.36568284142071	2105.223811382\\
3.36768384192096	2107.06598528491\\
3.36968484242121	2108.88701704517\\
3.37168584292146	2110.68690666279\\
3.37368684342171	2112.46576872934\\
3.37568784392196	2114.22371783636\\
3.37768884442221	2115.96086857541\\
3.37968984492246	2117.67733553807\\
3.38169084542271	2119.37329061165\\
3.38369184592296	2121.04884838774\\
3.38569284642321	2122.70423804943\\
3.38769384692346	2124.33957418829\\
3.38969484742371	2125.95497139588\\
3.39169584792396	2127.55065885532\\
3.39369684842421	2129.12675115817\\
3.39569784892446	2130.68336289598\\
3.39769884942471	2132.22072325187\\
3.39969984992496	2133.73883222585\\
3.40170085042521	2135.23786170525\\
3.40370185092546	2136.71786898586\\
3.40570285142571	2138.179025955\\
3.40770385192596	2139.6212753169\\
3.40970485242621	2141.04473166312\\
3.41170585292646	2142.44939499367\\
3.41370685342671	2143.83532260431\\
3.41570785392696	2145.20239990349\\
3.41770885442721	2146.55068418699\\
3.41970985492746	2147.88006086326\\
3.42171085542771	2149.1904726365\\
3.42371185592796	2150.48186221094\\
3.42571285642821	2151.75400040347\\
3.42771385692846	2153.00682991831\\
3.42971485742871	2154.24006427655\\
3.43171585792896	2155.45364618241\\
3.43371685842921	2156.64723186123\\
3.43571785892946	2157.82064942566\\
3.43771885942971	2158.97355510102\\
3.43971985992996	2160.10566240842\\
3.44172086043022	2161.21662757318\\
3.44372186093047	2162.3061641164\\
3.44572286143072	2163.37387096762\\
3.44772386193097	2164.41928976062\\
3.44972486243122	2165.44201942493\\
3.45172586293147	2166.44165889009\\
3.45372686343172	2167.41763519832\\
3.45572786393197	2168.36960457493\\
3.45772886443222	2169.29693676635\\
3.45972986493247	2170.19917340634\\
3.46173086543272	2171.07574153711\\
3.46373186593297	2171.92606820086\\
3.46573286643322	2172.74969503137\\
3.46773386693347	2173.54599177504\\
3.46973486743372	2174.31444276987\\
3.47173586793397	2175.05447505806\\
3.47373686843422	2175.76551568182\\
3.47573786893447	2176.4471062749\\
3.47773886943472	2177.09873117531\\
3.47973986993497	2177.71987472101\\
3.48174087043522	2178.31019313733\\
3.48374187093547	2178.86928535382\\
3.48574287143572	2179.39686489158\\
3.48774387193597	2179.89258797592\\
3.48974487243622	2180.35639731108\\
3.49174587293647	2180.78817830549\\
3.49374687343672	2181.18798825493\\
3.49574787393697	2181.55594175097\\
3.49774887443722	2181.89238256827\\
3.49974987493747	2182.19765448151\\
3.50175087543772	2182.4723304485\\
3.50375187593797	2182.71715531436\\
3.50575287643822	2182.93304581156\\
3.50775387693847	2183.12109055993\\
3.50975487743872	2183.28249277081\\
3.51175587793897	2183.41868483872\\
3.51375687843922	2183.53132834124\\
3.51575787893947	2183.62225674333\\
3.51775887943972	2183.6933608057\\
3.51975987993997	2183.74687506377\\
3.52176088044022	2183.78503405292\\
3.52376188094047	2183.81030149169\\
3.52576288144072	2183.82531298592\\
3.52776388194097	2183.83258954992\\
3.52976488244122	2183.83493867688\\
3.53176588294147	2183.83516786\\
3.53376688344172	2183.83602729669\\
3.53576788394197	2183.84038177593\\
3.53776888444222	2183.85086690358\\
3.53976988494247	2183.87029017284\\
3.54177088544272	2183.90122989378\\
3.54377188594297	2183.94620708069\\
3.54577288644322	2184.00762815633\\
3.54777388694347	2184.08761306453\\
3.54977488744372	2184.1883963407\\
3.55177588794397	2184.31175415399\\
3.55377688844422	2184.45940537779\\
3.55577788894447	2184.63301158972\\
3.55777888944472	2184.83377600113\\
3.55977988994497	2185.06295911918\\
3.56178089044522	2185.32153497213\\
3.56378189094547	2185.61036299665\\
3.56578289144572	2185.92995885478\\
3.56778389194597	2186.28095280007\\
3.56978489244622	2186.66368860722\\
3.57178589294647	2187.07828086778\\
3.57378689344672	2187.52490146908\\
3.57578789394697	2188.00355041113\\
3.57778889444722	2188.51405580659\\
3.57978989494747	2189.05624576813\\
3.58179089544772	2189.62983381683\\
3.58379189594797	2190.23453347381\\
3.58579289644822	2190.86988637283\\
3.58779389694847	2191.53543414766\\
3.58979489744872	2192.23071843205\\
3.59179589794897	2192.95528085977\\
3.59379689844922	2193.70854847303\\
3.59579789894947	2194.48994831403\\
3.59779889944972	2195.29896472075\\
3.59979989994997	2196.13508203119\\
3.60180090045022	2196.99766999176\\
3.60380190095048	2197.88615564467\\
3.60580290145073	2198.80008062368\\
3.60780390195098	2199.73892926678\\
3.60980490245123	2200.7020713204\\
3.61180590295148	2201.68904841829\\
3.61380690345173	2202.69945949\\
3.61580790395198	2203.73278887352\\
3.61780890445223	2204.78857820261\\
3.61980990495248	2205.86642640681\\
3.62181090545273	2206.96593241566\\
3.62381190595298	2208.0867524545\\
3.62581290645323	2209.22854274864\\
3.62781390695348	2210.39101681918\\
3.62981490745373	2211.57377359567\\
3.63181590795398	2212.77664119076\\
3.63381690845423	2213.99939042135\\
3.63581790895448	2215.24173480853\\
3.63781890945473	2216.50355976075\\
3.63981990995498	2217.7845787991\\
3.64182091045523	2219.08479192359\\
3.64382191095548	2220.40402724688\\
3.64582291145573	2221.74217017741\\
3.64782391195598	2223.09922071518\\
3.64982491245623	2224.47500697285\\
3.65182591295648	2225.8695862462\\
3.65382691345673	2227.28295853522\\
3.65582791395698	2228.71518113571\\
3.65782891445723	2230.1661394561\\
3.65982991495748	2231.63600538373\\
3.66183091545773	2233.12483621438\\
3.66383191595798	2234.6327465396\\
3.66583291645823	2236.15973635941\\
3.66783391695848	2237.70597756113\\
3.66983491745873	2239.2716420321\\
3.67183591795898	2240.85678706811\\
3.67383691845923	2242.46158455649\\
3.67583791895948	2244.08620638458\\
3.67783891945973	2245.73082443973\\
3.67983991995998	2247.3954960177\\
3.68184092046023	2249.0805075974\\
3.68384192096048	2250.78597377039\\
3.68584292146073	2252.51200912822\\
3.68784392196098	2254.25878555823\\
3.68984492246123	2256.02636035621\\
3.69184592296148	2257.81496270527\\
3.69384692346173	2259.62464990119\\
3.69584792396198	2261.45547923976\\
3.69784892446223	2263.30750801674\\
3.69984992496248	2265.18073623214\\
3.70185092546273	2267.07516388596\\
3.70385192596298	2268.9907909782\\
3.70585292646323	2270.92744562152\\
3.70785392696348	2272.88495592858\\
3.70985492746373	2274.86326460361\\
3.71185592796398	2276.86202787193\\
3.71385692846423	2278.88095925463\\
3.71585792896448	2280.91965768126\\
3.71785892946473	2282.97777937715\\
3.71985992996498	2285.05469408872\\
3.72186093046523	2287.14994344974\\
3.72386193096548	2289.26283991084\\
3.72586293146573	2291.3926386269\\
3.72786393196598	2293.53859475278\\
3.72986493246623	2295.69979155602\\
3.73186593296648	2297.87525500835\\
3.73386693346673	2300.06401108153\\
3.73586793396698	2302.26497115574\\
3.73786893446723	2304.4769893154\\
3.73986993496748	2306.69880505336\\
3.74187093546773	2308.92915786247\\
3.74387193596798	2311.16678723557\\
3.74587293646823	2313.41020348241\\
3.74787393696848	2315.65808880004\\
3.74987493746873	2317.909010794\\
3.75187593796898	2320.16147977399\\
3.75387693846923	2322.41400604977\\
3.75587793896948	2324.6652718184\\
3.75787893946973	2326.91367279805\\
3.75987993996999	2329.15783389002\\
3.76188094047024	2331.3963799956\\
3.76388194097049	2333.62787872029\\
3.76588294147074	2335.85101226118\\
3.76788394197099	2338.06452011111\\
3.76988494247124	2340.26714176293\\
3.77188594297149	2342.45773130105\\
3.77388694347174	2344.63520010567\\
3.77588794397199	2346.79857414852\\
3.77788894447224	2348.94676480981\\
3.77988994497249	2351.07896994861\\
3.78189094547274	2353.19433012823\\
3.78389194597299	2355.29215779932\\
3.78589294647324	2357.37182270831\\
3.78789394697349	2359.43269460161\\
3.78989494747374	2361.47425781722\\
3.79189594797399	2363.4960539889\\
3.79389694847424	2365.49779663775\\
3.79589794897449	2367.47902739753\\
3.79789894947474	2369.43963167669\\
3.79989994997499	2371.37932299633\\
3.80190095047524	2373.29798676488\\
3.80390195097549	2375.19545109502\\
3.80590295147574	2377.07177328251\\
3.80790395197599	2378.92689603159\\
3.80990495247624	2380.76076204646\\
3.81190595297649	2382.57354321448\\
3.81390695347674	2384.36518223985\\
3.81590795397699	2386.13585100992\\
3.81790895447724	2387.88560682047\\
3.81990995497749	2389.61467885462\\
3.82191095547774	2391.32306711236\\
3.82391195597799	2393.01100077681\\
3.82591295647824	2394.67865173532\\
3.82791395697849	2396.32619187522\\
3.82991495747874	2397.95367849229\\
3.83191595797899	2399.56134076965\\
3.83391695847924	2401.14935059463\\
3.83591795897949	2402.7178225588\\
3.83791895947974	2404.26687125372\\
3.83991995997999	2405.79666856672\\
3.84192096048024	2407.30732908936\\
3.84392196098049	2408.79891011742\\
3.84592296148074	2410.27152624247\\
3.84792396198099	2411.72523476028\\
3.84992496248124	2413.1601502624\\
3.85192596298149	2414.57627274885\\
3.85392696348174	2415.97354492383\\
3.85592796398199	2417.3521386747\\
3.85792896448224	2418.71182481832\\
3.85992996498249	2420.05271794627\\
3.86193096548274	2421.37470346697\\
3.86393196598299	2422.67766678888\\
3.86593296648324	2423.96155061621\\
3.86793396698349	2425.22612576584\\
3.86993496748374	2426.471334942\\
3.87193596798399	2427.69689166578\\
3.87393696848424	2428.90268134564\\
3.87593796898449	2430.08841750266\\
3.87793896948474	2431.25381365796\\
3.87993996998499	2432.39864062841\\
3.88194097048524	2433.52249734356\\
3.88394197098549	2434.62515462028\\
3.88594297148574	2435.70615409236\\
3.88794397198599	2436.7651519851\\
3.88994497248624	2437.80174722805\\
3.89194597298649	2438.81553875075\\
3.89394697348674	2439.80595359542\\
3.89594797398699	2440.77259069158\\
3.89794897448724	2441.71499167301\\
3.89994997498749	2442.63252628614\\
3.90195097548774	2443.52479346049\\
3.90395197598799	2444.39116294251\\
3.90595297648824	2445.23111907017\\
3.90795397698849	2446.04408888568\\
3.90995497748874	2446.82955672703\\
3.91195597798899	2447.58694963641\\
3.91395697848924	2448.31569465604\\
3.91595797898949	2449.01527612389\\
3.91795897948974	2449.68517837796\\
3.91995997998999	2450.324943052\\
3.92196098049025	2450.93405448401\\
3.9239619809905	2451.51222619507\\
3.92596298149075	2452.05899981897\\
3.927963981991	2452.57408887679\\
3.92996498249125	2453.05737877698\\
3.9319659829915	2453.50864033643\\
3.93396698349175	2453.9279308509\\
3.935967983992	2454.31525032041\\
3.93796898449225	2454.67077063229\\
3.9399699849925	2454.99495015278\\
3.94197098549275	2455.28818995232\\
3.943971985993	2455.55106298873\\
3.94597298649325	2455.78448599447\\
3.9479739869935	2455.98931840623\\
3.94997498749375	2456.16676343538\\
3.951975987994	2456.31813888485\\
3.95397698849425	2456.44487714913\\
3.9559779889945	2456.54881169317\\
3.95797898949475	2456.63171868613\\
3.959979989995	2456.69566077606\\
3.96198099049525	2456.74287249838\\
3.9639819909955	2456.77570298004\\
3.96598299149575	2456.79667323534\\
3.967983991996	2456.80830427858\\
3.96998499249625	2456.81334630718\\
3.9719859929965	2456.81454951855\\
3.97398699349675	2456.81466411011\\
3.975987993997	2456.81655487083\\
3.97798899449725	2456.82297199814\\
3.9799899949975	2456.83666568944\\
3.98199099549775	2456.8602715506\\
3.983991995998	2456.89642518748\\
3.98599299649825	2456.94759031858\\
3.9879939969985	2457.01600147932\\
3.98999499749875	2457.10389320509\\
3.991995997999	2457.21327084818\\
3.99399699849925	2457.34596787354\\
3.9959979989995	2457.50358856298\\
3.99799899949975	2457.68767990255\\
4	2457.89944510363\\
};
\addlegendentry{$\phi$};

\addplot [color=mycolor2,solid]
  table[row sep=crcr]{%
0	0.719978765361392\\
0.00200100050025012	7.91942264429824\\
0.00400200100050025	15.1159444384799\\
0.00600300150075038	22.3069658378283\\
0.0080040020010005	29.4899658280449\\
0.0100050025012506	36.6624806906106\\
0.0120060030015008	43.8221612985654\\
0.0140070035017509	50.9667731165082\\
0.016008004002001	58.0942534963766\\
0.0180090045022511	65.2027116774467\\
0.0200100050025013	72.290314194774\\
0.0220110055027514	79.3555713580912\\
0.0240120060030015	86.39710806869\\
0.0260130065032516	93.4136638194211\\
0.0280140070035018	100.404321877812\\
0.0300150075037519	107.368222807171\\
0.032016008004002	114.304679058143\\
0.0340170085042521	121.213346856051\\
0.0360180090045022	128.093767834659\\
0.0380190095047524	134.945884698187\\
0.0400200100050025	141.769640150857\\
0.0420210105052526	148.565148784226\\
0.0440220110055028	155.332639781414\\
0.0460230115057529	162.072456917097\\
0.048024012006003	168.785001261732\\
0.0500250125062531	175.470845773113\\
0.0520260130065033	182.130506113257\\
0.0540270135067534	188.764669831518\\
0.0560280140070035	195.374081773029\\
0.0580290145072536	201.959429487145\\
0.0600300150075038	208.521515114778\\
0.0620310155077539	215.061198092622\\
0.064032016008004	221.579280561589\\
0.0660330165082541	228.076621958373\\
0.0680340170085043	234.554139015445\\
0.0700350175087544	241.012748465277\\
0.0720360180090045	247.453252448784\\
0.0740370185092546	253.876624994216\\
0.0760380190095048	260.283725538266\\
0.0780390195097549	266.675470813407\\
0.080040020010005	273.052720256332\\
0.0820410205102551	279.416390599511\\
0.0840420210105053	285.767398575418\\
0.0860430215107554	292.106489029187\\
0.0880440220110055	298.434693284847\\
0.0900450225112556	304.752698891755\\
0.0920460230115058	311.061422582382\\
0.0940470235117559	317.36172379342\\
0.096048024012006	323.654404665782\\
0.0980490245122561	329.940210044604\\
0.100050025012506	336.220056662355\\
0.102051025512756	342.494632068392\\
0.104052026013007	348.764738403627\\
0.106053026513257	355.031177808973\\
0.108054027013507	361.294752425343\\
0.110055027513757	367.556149802091\\
0.112056028014007	373.816114784352\\
0.114057028514257	380.075392217259\\
0.116058029014507	386.334726945945\\
0.118059029514757	392.594921111324\\
0.120060030015008	398.856547671191\\
0.122061030515258	405.120351470679\\
0.124062031015508	411.387134650702\\
0.126063031515758	417.657470169055\\
0.128064032016008	423.93216016665\\
0.130065032516258	430.211777601284\\
0.132066033016508	436.49706731809\\
0.134067033516758	442.788659570642\\
0.136068034017009	449.087241908294\\
0.138069034517259	455.393501880402\\
0.140070035017509	461.708012444759\\
0.142071035517759	468.031403854941\\
0.144072036018009	474.364306364522\\
0.146073036518259	480.707350227077\\
0.148074037018509	487.06116569618\\
0.150075037518759	493.426211138068\\
0.15207603801901	499.803174102094\\
0.15407703851926	506.192512954496\\
0.15607803901951	512.594743357287\\
0.15807903951976	519.010380972485\\
0.16008004002001	525.439884166325\\
0.16208104052026	531.883768600823\\
0.16408204102051	538.342263459097\\
0.16608304152076	544.815884403162\\
0.168084042021011	551.304917911916\\
0.170085042521261	557.809650464257\\
0.172086043021511	564.330253947523\\
0.174087043521761	570.866957544832\\
0.176088044022011	577.419933143523\\
0.178089044522261	583.989180743595\\
0.180090045022511	590.574757640829\\
0.182091045522761	597.176606539444\\
0.184092046023012	603.794612847882\\
0.186093046523262	610.428547383025\\
0.188094047023512	617.078123665975\\
0.190095047523762	623.743055217834\\
0.192096048024012	630.422883672367\\
0.194097048524262	637.117093367557\\
0.196098049024512	643.82511134561\\
0.198099049524762	650.546307352951\\
0.200100050025012	657.279879248667\\
0.202101050525263	664.025082187624\\
0.204102051025513	670.780942141571\\
0.206103051525763	677.546542378035\\
0.208104052026013	684.320851572985\\
0.210105052526263	691.10278110661\\
0.212106053026513	697.89118506332\\
0.214107053526763	704.684917527524\\
0.216108054027013	711.482660696295\\
0.218109054527264	718.283268654041\\
0.220110055027514	725.085423597835\\
0.222111055527764	731.887865020525\\
0.224112056028014	738.689217823405\\
0.226113056528264	745.488336090884\\
0.228114057028514	752.283844724254\\
0.230115057528764	759.074597807924\\
0.232116058029014	765.859277538965\\
0.234117058529265	772.636852593347\\
0.236118059029515	779.4061197597\\
0.238119059529765	786.166105009772\\
0.240120060030015	792.915777019531\\
0.242121060530265	799.654161760726\\
0.244122061030515	806.380514388223\\
0.246123061530765	813.094032761109\\
0.248124062031016	819.793972034251\\
0.250125062531266	826.479701954074\\
0.252126063031516	833.150764154341\\
0.254127063531766	839.806700268817\\
0.256128064032016	846.447051931265\\
0.258129064532266	853.071589958568\\
0.260130065032516	859.680027871827\\
0.262131065532766	866.272251079484\\
0.264132066033017	872.84820228576\\
0.266133066533267	879.407824194874\\
0.268134067033517	885.951174102606\\
0.270135067533767	892.478366600516\\
0.272136068034017	898.989516280162\\
0.274137068534267	905.484909620443\\
0.276138069034517	911.964775804474\\
0.278139069534767	918.429458606936\\
0.280140070035018	924.879244506723\\
0.282141070535268	931.314591870074\\
0.284142071035518	937.735844471664\\
0.286143071535768	944.143517973509\\
0.288144072036018	950.538070741847\\
0.290145072536268	956.919961142911\\
0.292146073036518	963.289819430278\\
0.294147073536768	969.648161265963\\
0.296148074037018	975.995502311981\\
0.298149074537269	982.332530117687\\
0.300150075037519	988.659703049317\\
0.302151075537769	994.977765952004\\
0.304152076038019	1001.28729178354\\
0.306153076538269	1007.58896809329\\
0.308154077038519	1013.88331054326\\
0.310155077538769	1020.17112127436\\
0.31215607803902	1026.4529732444\\
0.31415707853927	1032.7296112985\\
0.31615807903952	1039.00160839446\\
0.31815907953977	1045.26976667319\\
0.32016008004002	1051.53471638826\\
0.32216108054027	1057.79714508904\\
0.32416208104052	1064.05785491644\\
0.32616308154077	1070.31741882824\\
0.32816408204102	1076.57663896537\\
0.330165082541271	1082.83626017295\\
0.332166083041521	1089.09697000035\\
0.334167083541771	1095.35957058847\\
0.336168084042021	1101.62474948666\\
0.338169084542271	1107.89325154007\\
0.340170085042521	1114.16593618539\\
0.342171085542771	1120.44349097196\\
0.344172086043022	1126.72677533648\\
0.346173086543272	1133.01653412409\\
0.348174087043522	1139.31362677147\\
0.350175087543772	1145.61879812377\\
0.352176088044022	1151.93296491345\\
0.354177088544272	1158.25687198565\\
0.356178089044522	1164.59143607283\\
0.358179089544772	1170.93745931592\\
0.360180090045022	1177.29580115161\\
0.362181090545273	1183.66743560814\\
0.364182091045523	1190.05316482643\\
0.366183091545773	1196.45384824318\\
0.368184092046023	1202.87045988663\\
0.370185092546273	1209.30385919347\\
0.372186093046523	1215.75496289619\\
0.374187093546773	1222.22463043147\\
0.376188094047024	1228.71377853178\\
0.378189094547274	1235.2233239296\\
0.380190095047524	1241.75406876584\\
0.382191095547774	1248.30692977297\\
0.384192096048024	1254.88265179613\\
0.386193096548274	1261.48215156779\\
0.388194097048524	1268.10611663729\\
0.390195097548774	1274.75529184979\\
0.392196098049024	1281.43036475462\\
0.394197098549275	1288.13190830959\\
0.396198099049525	1294.86055276827\\
0.398199099549775	1301.61669920111\\
0.400200100050025	1308.4006913828\\
0.402201100550275	1315.212873088\\
0.404202101050525	1322.05341620407\\
0.406203101550775	1328.92226343522\\
0.408204102051026	1335.81941478144\\
0.410205102551276	1342.74458376385\\
0.412206103051526	1349.69736931198\\
0.414207103551776	1356.67714117227\\
0.416208104052026	1363.68326909113\\
0.418209104552276	1370.71472174453\\
0.420210105052526	1377.77041051267\\
0.422211105552776	1384.84907488839\\
0.424212106053027	1391.94922518143\\
0.426213106553277	1399.06914251841\\
0.428214107053527	1406.20699343437\\
0.430215107553777	1413.36077257703\\
0.432216108054027	1420.52830270678\\
0.434217108554277	1427.70734928821\\
0.436218109054527	1434.89539130702\\
0.438219109554777	1442.0899650447\\
0.440220110055028	1449.28854948695\\
0.442221110555278	1456.48845173212\\
0.444222111055528	1463.68703617436\\
0.446223111555778	1470.8817245036\\
0.448224112056028	1478.06988111397\\
0.450225112556278	1485.24909958274\\
0.452226113056528	1492.41685889561\\
0.454227113556778	1499.57092451717\\
0.456228114057029	1506.70906191203\\
0.458229114557279	1513.8292657279\\
0.460230115057529	1520.92975979562\\
0.462231115557779	1528.0088252418\\
0.464232116058029	1535.06491508039\\
0.466233116558279	1542.09682610003\\
0.468234117058529	1549.10335508935\\
0.470235117558779	1556.08358531587\\
0.47223611805903	1563.03682923024\\
0.47423711855928	1569.96245657888\\
0.47623811905953	1576.86006629134\\
0.47823911955978	1583.72942918451\\
0.48024012006003	1590.57043066681\\
0.48224112056028	1597.38312803403\\
0.48424212106053	1604.16763587774\\
0.48624312156078	1610.92424067682\\
0.488244122061031	1617.65334350173\\
0.490245122561281	1624.35534542293\\
0.492246123061531	1631.030876694\\
0.494247123561781	1637.68051027273\\
0.496248124062031	1644.3048764127\\
0.498249124562281	1650.90477725481\\
0.500250125062531	1657.48090034842\\
0.502251125562781	1664.03416242601\\
0.504252126063031	1670.56525103693\\
0.506253126563282	1677.07514020942\\
0.508254127063532	1683.56463208441\\
0.510255127563782	1690.03464339437\\
0.512256128064032	1696.48603357598\\
0.514257128564282	1702.91971936173\\
0.516258129064532	1709.33661748408\\
0.518259129564782	1715.73758737972\\
0.520260130065032	1722.12354578113\\
0.522261130565283	1728.49540942078\\
0.524262131065533	1734.85398043958\\
0.526263131565783	1741.20023286579\\
0.528264132066033	1747.53496884031\\
0.530265132566283	1753.85904779985\\
0.532266133066533	1760.17332918109\\
0.534267133566783	1766.47867242072\\
0.536268134067034	1772.77587965967\\
0.538269134567284	1779.06575303884\\
0.540270135067534	1785.34915199492\\
0.542271135567784	1791.62682137305\\
0.544272136068034	1797.89956331414\\
0.546273136568284	1804.16817995911\\
0.548274137068534	1810.43341615308\\
0.550275137568784	1816.6960167412\\
0.552276138069035	1822.9567265686\\
0.554277138569285	1829.21634777618\\
0.556278139069535	1835.47562520909\\
0.558279139569785	1841.73518912089\\
0.560280140070035	1847.99584165251\\
0.562281140570285	1854.25821305751\\
0.564282141070535	1860.52316277258\\
0.566283141570785	1866.79120645976\\
0.568284142071036	1873.06320355572\\
0.570285142571286	1879.33972701826\\
0.572286143071536	1885.62146439673\\
0.574287143571786	1891.90916053628\\
0.576288144072036	1898.20344569047\\
0.578289144572286	1904.50495011287\\
0.580290145072536	1910.81430405707\\
0.582291145572786	1917.13219507242\\
0.584292146073036	1923.45925341249\\
0.586293146573287	1929.79610933086\\
0.588294147073537	1936.14327848954\\
0.590295147573787	1942.50139114211\\
0.592296148074037	1948.87107754214\\
0.594297148574287	1955.25279605586\\
0.596298149074537	1961.64711964108\\
0.598299149574787	1968.05456395981\\
0.600300150075038	1974.47558737828\\
0.602301150575288	1980.91064826273\\
0.604302151075538	1987.3602049794\\
0.606303151575788	1993.82460130297\\
0.608304152076038	2000.30423830388\\
0.610305152576288	2006.79934516526\\
0.612306153076538	2013.31026566179\\
0.614307153576788	2019.8371716808\\
0.616308154077039	2026.38023510964\\
0.618309154577289	2032.93957053986\\
0.620310155077539	2039.51523526723\\
0.622311155577789	2046.10717199599\\
0.624312156078039	2052.71538072613\\
0.626313156578289	2059.33963227454\\
0.628314157078539	2065.97969745809\\
0.630315157578789	2072.63534709367\\
0.63231615807904	2079.30618011082\\
0.63431715857929	2085.99168084752\\
0.63631815907954	2092.69139093755\\
0.63831915957979	2099.40462283153\\
0.64032016008004	2106.13074627591\\
0.64232116058029	2112.86895912977\\
0.64432216108054	2119.61845925219\\
0.64632316158079	2126.37827261492\\
0.64832416208104	2133.14736789394\\
0.650325162581291	2139.92482835676\\
0.652326163081541	2146.70945079202\\
0.654327163581791	2153.50003198835\\
0.656328164082041	2160.29548332594\\
0.658329164582291	2167.09454429764\\
0.660330165082541	2173.89589710052\\
0.662331165582791	2180.69833852322\\
0.664332166083042	2187.50049346701\\
0.666333166583292	2194.30110142475\\
0.668334167083542	2201.09895918508\\
0.670335167583792	2207.89274894507\\
0.672336168084042	2214.68126749334\\
0.674337168584292	2221.46331161852\\
0.676338169084542	2228.23779270081\\
0.678339169584792	2235.00350752883\\
0.680340170085043	2241.7595966659\\
0.682341170585293	2248.50491419641\\
0.684342171085543	2255.23871527525\\
0.686343171585793	2261.96014046571\\
0.688344172086043	2268.66838762688\\
0.690345172586293	2275.36282650519\\
0.692346173086543	2282.04294143862\\
0.694347173586793	2288.70810217359\\
0.696348174087044	2295.35796493544\\
0.698349174587294	2301.99218594948\\
0.700350175087544	2308.61042144104\\
0.702351175587794	2315.21255681855\\
0.704352176088044	2321.79842019468\\
0.706353176588294	2328.36795427365\\
0.708354177088544	2334.92115905546\\
0.710355177588794	2341.45820642745\\
0.712356178089045	2347.97909638961\\
0.714357178589295	2354.48405812507\\
0.716358179089545	2360.97332081694\\
0.718359179589795	2367.4472282399\\
0.720360180090045	2373.90600957708\\
0.722361180590295	2380.35006589891\\
0.724362181090545	2386.77985557165\\
0.726363181590795	2393.19572236997\\
0.728364182091045	2399.59818195588\\
0.730365182591296	2405.98774999139\\
0.732366183091546	2412.36488484276\\
0.734367183591796	2418.73015946777\\
0.736368184092046	2425.08414682421\\
0.738369184592296	2431.4273625741\\
0.740370185092546	2437.76043697102\\
0.742371185592796	2444.08400026854\\
0.744372186093047	2450.39868272024\\
0.746373186593297	2456.70499998812\\
0.748374187093547	2463.00375421311\\
0.750375187593797	2469.29546105723\\
0.752376188094047	2475.58086536559\\
0.754377188594297	2481.86059739178\\
0.756378189094547	2488.13528738938\\
0.758379189594797	2494.40573749929\\
0.760380190095048	2500.67257797509\\
0.762381190595298	2506.93643907036\\
0.764382191095548	2513.19812292601\\
0.766383191595798	2519.45825979561\\
0.768384192096048	2525.71765182007\\
0.770385192596298	2531.97692925298\\
0.772386193096548	2538.23689423524\\
0.774387193596798	2544.49823431621\\
0.776388194097049	2550.7617516368\\
0.778389194597299	2557.02813374637\\
0.780390195097549	2563.29818278582\\
0.782391195597799	2569.57270089608\\
0.784392196098049	2575.85243292227\\
0.786393196598299	2582.13812370953\\
0.788394197098549	2588.43068999034\\
0.790395197598799	2594.73081931404\\
0.792396198099049	2601.0394284131\\
0.7943971985993	2607.35726213267\\
0.79639819909955	2613.68523720522\\
0.7983991995998	2620.02415577165\\
0.80040020010005	2626.37493456443\\
0.8024012006003	2632.7383757245\\
0.80440220110055	2639.11545328008\\
0.8064032016008	2645.50691207632\\
0.808404202101051	2651.91372614148\\
0.810405202601301	2658.33681220801\\
0.812406203101551	2664.77702971262\\
0.814407203601801	2671.23529538778\\
0.816408204102051	2677.71246867017\\
0.818409204602301	2684.20952358806\\
0.820410205102551	2690.72726228235\\
0.822411205602801	2697.26654418973\\
0.824412206103052	2703.82822874691\\
0.826413206603302	2710.41317539057\\
0.828414207103552	2717.02218626162\\
0.830415207603802	2723.65589161365\\
0.832416208104052	2730.31515088334\\
0.834417208604302	2737.00053702848\\
0.836418209104552	2743.71262300688\\
0.838419209604802	2750.45198177633\\
0.840420210105053	2757.21901440728\\
0.842421210605303	2764.01400737863\\
0.844422211105553	2770.83730446507\\
0.846423211605803	2777.68890566658\\
0.848424212106053	2784.56886827895\\
0.850425212606303	2791.47702041484\\
0.852426213106553	2798.41301829958\\
0.854427213606803	2805.3764608627\\
0.856428214107053	2812.36666055486\\
0.858429214607304	2819.38281523513\\
0.860430215107554	2826.42389357949\\
0.862431215607804	2833.48874967235\\
0.864432216108054	2840.57600841501\\
0.866433216608304	2847.68406552562\\
0.868434217108554	2854.81125942659\\
0.870435217608804	2861.95558476564\\
0.872436218109054	2869.11503619047\\
0.874437218609305	2876.2873791657\\
0.876438219109555	2883.47020726858\\
0.878439219609805	2890.66111407637\\
0.880440220110055	2897.85757857477\\
0.882441220610305	2905.05696515793\\
0.884442221110555	2912.25669551576\\
0.886443221610805	2919.45401945086\\
0.888444222111056	2926.64647324469\\
0.890445222611306	2933.83142129141\\
0.892446223111556	2941.00645716828\\
0.894447223611806	2948.16911715677\\
0.896448224112056	2955.31716672148\\
0.898449224612306	2962.44854321435\\
0.900450225112556	2969.56124128311\\
0.902451225612806	2976.65354205436\\
0.904452226113057	2983.72378395049\\
0.906453226613307	2990.77053457703\\
0.908454227113557	2997.79264801837\\
0.910455227613807	3004.78897835893\\
0.912456228114057	3011.7587234578\\
0.914457228614307	3018.7012530614\\
0.916458229114557	3025.61593691616\\
0.918459229614807	3032.50248854317\\
0.920460230115058	3039.36079335089\\
0.922461230615308	3046.19067945197\\
0.924462231115558	3052.99231873374\\
0.926463231615808	3059.76582578778\\
0.928464232116058	3066.51160168453\\
0.930465232616308	3073.23004749446\\
0.932466233116558	3079.92162158379\\
0.934467233616808	3086.58695420611\\
0.936468234117058	3093.2266183192\\
0.938469234617309	3099.84130147243\\
0.940470235117559	3106.4318058067\\
0.942471235617809	3112.99887616715\\
0.944472236118059	3119.54337199047\\
0.946473236618309	3126.0660381218\\
0.948474237118559	3132.56784858939\\
0.950475237618809	3139.04960553414\\
0.95247623811906	3145.51222568854\\
0.95447723861931	3151.95656848928\\
0.95647823911956	3158.3836079646\\
0.95847923961981	3164.79414625542\\
0.96048024012006	3171.18915738999\\
0.96248124062031	3177.56950080501\\
0.96448224112056	3183.93603593716\\
0.96648324162081	3190.28967951893\\
0.968484242121061	3196.631290987\\
0.970485242621311	3202.96178707384\\
0.972486243121561	3209.28191262459\\
0.974487243621811	3215.5926416675\\
0.976488244122061	3221.89466175192\\
0.978489244622311	3228.18888961033\\
0.980490245122561	3234.47618467942\\
0.982491245622811	3240.75723450854\\
0.984492246123062	3247.03284123861\\
0.986493246623312	3253.30392160209\\
0.988494247123562	3259.57110585257\\
0.990495247623812	3265.83519613096\\
0.992496248124062	3272.09699457816\\
0.994497248624312	3278.35718874354\\
0.996498249124562	3284.61658076801\\
0.998499249624812	3290.87585820091\\
1.00050025012506	3297.1357658874\\
1.00250125062531	3303.39704867259\\
1.00450225112556	3309.66033681006\\
1.00650325162581	3315.92637514495\\
1.00850425212606	3322.19590852239\\
1.01050525262631	3328.46956719595\\
1.01250625312656	3334.74809601078\\
1.01450725362681	3341.03206792465\\
1.01650825412706	3347.3222850785\\
1.01850925462731	3353.61926313432\\
1.02051025512756	3359.92380423304\\
1.02251125562781	3366.23642403668\\
1.02451225612806	3372.55786739036\\
1.02651325662831	3378.88864995609\\
1.02851425712856	3385.22945928325\\
1.03051525762881	3391.58086832961\\
1.03251625812906	3397.94345005298\\
1.03451725862931	3404.31777741115\\
1.03651825912956	3410.70436606613\\
1.03851925962981	3417.10373167995\\
1.04052026013006	3423.51644721039\\
1.04252126063032	3429.94291372792\\
1.04452226113057	3436.38358959876\\
1.04652326163082	3442.83893318916\\
1.04852426213107	3449.30923097802\\
1.05052526263132	3455.79488403578\\
1.05252626313157	3462.29617884135\\
1.05452726363182	3468.81334457785\\
1.05652826413207	3475.3465531326\\
1.05852926463232	3481.89597639297\\
1.06053026513257	3488.46167165471\\
1.06253126563282	3495.04375350939\\
1.06453226613307	3501.64205006968\\
1.06653326663332	3508.25656133556\\
1.06853426713357	3514.88705812394\\
1.07053526763382	3521.5332539559\\
1.07253626813407	3528.19486235254\\
1.07453726863432	3534.8715395392\\
1.07653826913457	3541.5626552623\\
1.07853926963482	3548.2677511556\\
1.08054027013507	3554.98613966974\\
1.08254127063532	3561.7170759596\\
1.08454227113557	3568.45981518004\\
1.08654327163582	3575.21344059858\\
1.08854427213607	3581.97703548277\\
1.09054527263632	3588.74951121277\\
1.09254627313657	3595.52989376035\\
1.09454727363682	3602.31697991413\\
1.09654827413707	3609.1096810543\\
1.09854927463732	3615.90667937794\\
1.10055027513757	3622.70677167367\\
1.10255127563782	3629.50869743435\\
1.10455227613807	3636.31113885704\\
1.10655327663832	3643.11289273037\\
1.10855427713857	3649.91258395565\\
1.11055527763882	3656.70900932149\\
1.11255627813907	3663.50085102497\\
1.11455727863932	3670.2869631505\\
1.11655827913957	3677.06619978249\\
1.11855927963982	3683.83735770956\\
1.12056028014007	3690.59946290348\\
1.12256128064032	3697.35142674442\\
1.12456228114057	3704.09233249991\\
1.12656328164082	3710.82137802904\\
1.12856428214107	3717.53770389513\\
1.13056528264132	3724.2406225488\\
1.13256628314157	3730.92956103228\\
1.13456728364182	3737.6038317962\\
1.13656828414207	3744.26309106589\\
1.13856928464232	3750.90682317933\\
1.14057028514257	3757.5347989534\\
1.14257128564282	3764.14678920499\\
1.14457228614307	3770.74256475098\\
1.14657328664332	3777.3220109998\\
1.14857428714357	3783.88518524725\\
1.15057528764382	3790.43203019753\\
1.15257628814407	3796.96271773799\\
1.15457728864432	3803.47736246018\\
1.15657828914457	3809.97619354724\\
1.15857928964482	3816.45938288648\\
1.16058029014507	3822.92733154837\\
1.16258129064532	3829.38032601181\\
1.16458229114557	3835.81871005148\\
1.16658329164582	3842.24299932938\\
1.16858429214607	3848.65359491598\\
1.17058529264632	3855.05095517752\\
1.17258629314657	3861.435595776\\
1.17458729364682	3867.80797507766\\
1.17658829414707	3874.16878063187\\
1.17858929464732	3880.51847080485\\
1.18059029514757	3886.85767585017\\
1.18259129564782	3893.18696872564\\
1.18459229614807	3899.50692238906\\
1.18659329664832	3905.81828168554\\
1.18859429714857	3912.12156227711\\
1.19059529764882	3918.41745171313\\
1.19259629814907	3924.70663754294\\
1.19459729864932	3930.98969272434\\
1.19659829914957	3937.26741939825\\
1.19859929964982	3943.54039052247\\
1.20060030015008	3949.80929364633\\
1.20260130065033	3956.07487361498\\
1.20460230115058	3962.33781797778\\
1.20660330165083	3968.59881428407\\
1.20860430215108	3974.85855008321\\
1.21060530265133	3981.11782751612\\
1.21260630315158	3987.37727683637\\
1.21460730365183	3993.63770018486\\
1.21660830415208	3999.89978511096\\
1.21860930465233	4006.16433375559\\
1.22061030515258	4012.43209096386\\
1.22261130565283	4018.70380158092\\
1.22461230615308	4024.98026774768\\
1.22661330665333	4031.26223430928\\
1.22861430715358	4037.55050340662\\
1.23061530765383	4043.8459344764\\
1.23261630815408	4050.14927236375\\
1.23461730865433	4056.46143380115\\
1.23661830915458	4062.78310633794\\
1.23861930965483	4069.11532129817\\
1.24062031015508	4075.45876623118\\
1.24262131065533	4081.81441516523\\
1.24462231115558	4088.18307024123\\
1.24662331165583	4094.56564819164\\
1.24862431215608	4100.96300845318\\
1.25062531265633	4107.37612505408\\
1.25262631315658	4113.80580013526\\
1.25462731365683	4120.25295042919\\
1.25662831415708	4126.71854996412\\
1.25862931465733	4133.20340088097\\
1.26063031515758	4139.70841991221\\
1.26263131565783	4146.23446649453\\
1.26463231615808	4152.7824573604\\
1.26663331665833	4159.35313735496\\
1.26863431715858	4165.94742321068\\
1.27063531765883	4172.56600247691\\
1.27263631815908	4179.20961999879\\
1.27463731865933	4185.87902062146\\
1.27663831915958	4192.57477730269\\
1.27863931965983	4199.29752029608\\
1.28064032016008	4206.04765067208\\
1.28264132066033	4212.82556950113\\
1.28464232116058	4219.63162055794\\
1.28664332166083	4226.46603302559\\
1.28864432216108	4233.32874960833\\
1.29064532266133	4240.21977030615\\
1.29264632316158	4247.13886593593\\
1.29464732366183	4254.08563542721\\
1.29664832416208	4261.05967770955\\
1.29864932466233	4268.06013334623\\
1.30065032516258	4275.08614290058\\
1.30265132566283	4282.13673234435\\
1.30465232616308	4289.21052657881\\
1.30665332666333	4296.30609320949\\
1.30865432716358	4303.42188525034\\
1.31065532766383	4310.55601194063\\
1.31265632816408	4317.70652522386\\
1.31465732866433	4324.87130515619\\
1.31665832916458	4332.04800261067\\
1.31865932966483	4339.23432575609\\
1.32066033016508	4346.4276962824\\
1.32266133066533	4353.62559317529\\
1.32466233116558	4360.82543812468\\
1.32666333166583	4368.02453822894\\
1.32866433216608	4375.22031517799\\
1.33066533266633	4382.41013336597\\
1.33266633316658	4389.59147177858\\
1.33466733366683	4396.76192399308\\
1.33666833416708	4403.91914088252\\
1.33866933466733	4411.06088791149\\
1.34067033516758	4418.1851597277\\
1.34267133566783	4425.29012286622\\
1.34467233616808	4432.3739438621\\
1.34667333666833	4439.43513302507\\
1.34867433716858	4446.47237255221\\
1.35067533766883	4453.48451652792\\
1.35267633816908	4460.47053362817\\
1.35467733866933	4467.42967900783\\
1.35667833916958	4474.3613797091\\
1.35867933966983	4481.26506277419\\
1.36068034017009	4488.14061361154\\
1.36268134067034	4494.98774574225\\
1.36468234117059	4501.80657375788\\
1.36668334167084	4508.59715495421\\
1.36868434217109	4515.35977581014\\
1.37068534267134	4522.09483739613\\
1.37268634317159	4528.80268348684\\
1.37468734367184	4535.48388704008\\
1.37668834417209	4542.13907830942\\
1.37868934467234	4548.76888754844\\
1.38069034517259	4555.37405960227\\
1.38269134567284	4561.95528202026\\
1.38469234617309	4568.51341423911\\
1.38669334667334	4575.04925839972\\
1.38869434717359	4581.5636739388\\
1.39069534767384	4588.05752029303\\
1.39269634817409	4594.53165689911\\
1.39469734867434	4600.98700048952\\
1.39669834917459	4607.42446779671\\
1.39869934967484	4613.84491825738\\
1.40070035017509	4620.2493258998\\
1.40270135067534	4626.63843556908\\
1.40470235117559	4633.01327858927\\
1.40670335167584	4639.37471439704\\
1.40870435217609	4645.72354513333\\
1.41070535267634	4652.06068753059\\
1.41270635317659	4658.38700102553\\
1.41470735367684	4664.70334505483\\
1.41670835417709	4671.01057905519\\
1.41870935467734	4677.30944787174\\
1.42071035517759	4683.60086823696\\
1.42271135567784	4689.88558499597\\
1.42471235617809	4696.16445758547\\
1.42671335667834	4702.43817355481\\
1.42871435717859	4708.70764963647\\
1.43071535767884	4714.97351608402\\
1.43271635817909	4721.23663233416\\
1.43471735867934	4727.49774323201\\
1.43671835917959	4733.75753632693\\
1.43871935967984	4740.01675646406\\
1.44072036018009	4746.2762057843\\
1.44272136068034	4752.53657183702\\
1.44472236118059	4758.79854217156\\
1.44672336168084	4765.06286163307\\
1.44872436218109	4771.33016047511\\
1.45072536268134	4777.60124083859\\
1.45272636318159	4783.8767329771\\
1.45472736368184	4790.15726714421\\
1.45672836418209	4796.44364548082\\
1.45872936468234	4802.73638364896\\
1.46073036518259	4809.03628378955\\
1.46273136568284	4815.34391886036\\
1.46473236618309	4821.65991911499\\
1.46673336668334	4827.984914807\\
1.46873436718359	4834.31959348574\\
1.47073536768384	4840.66447081324\\
1.47273636818409	4847.02017704307\\
1.47473736868434	4853.38728513302\\
1.47673836918459	4859.76631074511\\
1.47873936968484	4866.15788413291\\
1.48074037018509	4872.56240636688\\
1.48274137068534	4878.98045040482\\
1.48474237118559	4885.41247461296\\
1.48674337168584	4891.85882276597\\
1.48874437218609	4898.31995323011\\
1.49074537268634	4904.79626707581\\
1.49274637318659	4911.2879934862\\
1.49474737368684	4917.79541894018\\
1.49674837418709	4924.31888721242\\
1.49874937468734	4930.85839830292\\
1.50075037518759	4937.41418139481\\
1.50275137568784	4943.98635107964\\
1.50475237618809	4950.57473547007\\
1.50675337668834	4957.17939186188\\
1.50875437718859	4963.80020566351\\
1.51075537768884	4970.43689039607\\
1.51275637818909	4977.08915958066\\
1.51475737868934	4983.75672673838\\
1.51675837918959	4990.43907620721\\
1.51875937968984	4997.13574962092\\
1.5207603801901	5003.84611672593\\
1.52276138069035	5010.56954726867\\
1.5247623811906	5017.30518181245\\
1.52676338169085	5024.05233280791\\
1.5287643821911	5030.80996893102\\
1.53076538269135	5037.57717344931\\
1.5327663831916	5044.35285774297\\
1.53476738369185	5051.13599048797\\
1.5367683841921	5057.92542576871\\
1.53876938469235	5064.71996037383\\
1.5407703851926	5071.51833379617\\
1.54277138569285	5078.3192855286\\
1.5447723861931	5085.12155506395\\
1.54677338669335	5091.92388189508\\
1.5487743871936	5098.72500551484\\
1.55077538769385	5105.52355082452\\
1.5527763881941	5112.31831461276\\
1.55477738869435	5119.10809366818\\
1.5567783891946	5125.89162748363\\
1.55877938969485	5132.6678274393\\
1.5607803901951	5139.43554761961\\
1.56278139069535	5146.19381399629\\
1.5647823911956	5152.94153794955\\
1.56678339169585	5159.67791733846\\
1.5687843921961	5166.40209272634\\
1.57078539269635	5173.11326197226\\
1.5727863931966	5179.81073752689\\
1.57478739369685	5186.49400372819\\
1.5767883941971	5193.1624303226\\
1.57878939469735	5199.81567353544\\
1.5807903951976	5206.45327500047\\
1.58279139569785	5213.07500553458\\
1.5847923961981	5219.68063595464\\
1.58679339669835	5226.2700516691\\
1.5887943971986	5232.8431380864\\
1.59079539769885	5239.39989520654\\
1.5927963981991	5245.94043762108\\
1.59479739869935	5252.46482262579\\
1.5967983991996	5258.97322210802\\
1.59879939969985	5265.46592254666\\
1.6008004002001	5271.94309582906\\
1.60280140070035	5278.40514302566\\
1.6048024012006	5284.85235061537\\
1.60680340170085	5291.2852342602\\
1.6088044022011	5297.70408043905\\
1.61080540270135	5304.10940481394\\
1.6128064032016	5310.50172304687\\
1.61480740370185	5316.8814935041\\
1.6168084042021	5323.2492891434\\
1.61880940470235	5329.60568292258\\
1.6208104052026	5335.95119050366\\
1.62281140570285	5342.2864421402\\
1.6248124062031	5348.61201079\\
1.62681340670335	5354.92852670664\\
1.6288144072036	5361.23662014369\\
1.63081540770385	5367.53697865051\\
1.6328164082041	5373.83017518489\\
1.63481740870435	5380.1168400004\\
1.6368184092046	5386.39777523797\\
1.63881940970485	5392.67349655959\\
1.6408204102051	5398.9448061062\\
1.64282141070535	5405.21233413135\\
1.6448224112056	5411.4768254802\\
1.64682341170585	5417.7389104063\\
1.6488244122061	5423.9993337548\\
1.65082541270635	5430.25878307504\\
1.6528264132066	5436.51806050795\\
1.65482741370685	5442.77779630709\\
1.6568284142071	5449.03879261338\\
1.65882941470735	5455.30173697618\\
1.6608304152076	5461.56748883217\\
1.66283141570785	5467.83667843493\\
1.6648324162081	5474.11016522116\\
1.66683341670835	5480.3886367402\\
1.6688344172086	5486.67300972453\\
1.67083541770885	5492.96397172351\\
1.6728364182091	5499.26243946961\\
1.67483741870935	5505.56910051217\\
1.6768384192096	5511.88487158368\\
1.67883941970985	5518.21061212082\\
1.68084042021011	5524.54706696873\\
1.68284142071036	5530.89521015566\\
1.68484242121061	5537.25584382253\\
1.68684342171086	5543.62982740602\\
1.68884442221111	5550.01813493439\\
1.69084542271136	5556.42151125277\\
1.69284642321161	5562.8409876852\\
1.69484742371186	5569.27742366836\\
1.69684842421211	5575.73167863895\\
1.69884942471236	5582.20472662522\\
1.70085042521261	5588.69736976808\\
1.70285142571286	5595.21052480001\\
1.70485242621311	5601.7450511577\\
1.70685342671336	5608.30180827784\\
1.70885442721361	5614.88165559712\\
1.71085542771386	5621.48533796068\\
1.71285642821411	5628.11365750943\\
1.71485742871436	5634.76730179272\\
1.71685842921461	5641.44695835992\\
1.71885942971486	5648.15320016881\\
1.72086043021511	5654.88660017718\\
1.72286143071536	5661.64755945551\\
1.72486243121561	5668.43647907423\\
1.72686343171586	5675.25364551226\\
1.72886443221611	5682.09905876958\\
1.73086543271636	5688.97289073355\\
1.73286643321661	5695.87496951681\\
1.73486743371686	5702.8049513447\\
1.73686843421711	5709.76249244253\\
1.73886943471736	5716.74690526095\\
1.74087043521761	5723.75750225062\\
1.74287143571786	5730.79330938326\\
1.74487243621811	5737.85312344753\\
1.74687343671836	5744.93562664048\\
1.74887443721861	5752.03932927185\\
1.75087543771886	5759.16251246825\\
1.75287643821911	5766.30328546897\\
1.75487743871936	5773.45964292171\\
1.75687843921961	5780.62935029108\\
1.75887943971986	5787.8101157459\\
1.76088044022011	5794.99953286342\\
1.76288144072036	5802.19496603779\\
1.76488244122061	5809.39395155049\\
1.76688344172086	5816.59379649988\\
1.76888444222111	5823.79192257589\\
1.77088544272136	5830.98563687688\\
1.77288644322161	5838.17241838854\\
1.77488744372186	5845.34980339237\\
1.77688844422211	5852.51532816983\\
1.77888944472236	5859.66675818554\\
1.78089044522261	5866.80191619986\\
1.78289144572286	5873.9188541563\\
1.78489244622311	5881.01573858991\\
1.78689344672336	5888.0908506273\\
1.78889444722361	5895.14281516977\\
1.79089544772386	5902.17037171017\\
1.79289644822411	5909.1723743329\\
1.79489744872436	5916.14796360128\\
1.79689844922461	5923.09645196595\\
1.79889944972486	5930.01720917334\\
1.80090045022511	5936.90989144876\\
1.80290145072536	5943.77432690488\\
1.80490245122561	5950.61040095015\\
1.80690345172586	5957.41811358455\\
1.80890445222611	5964.197751287\\
1.81090545272636	5970.9495432406\\
1.81290645322661	5977.67389051581\\
1.81490745372686	5984.37125147888\\
1.81690845422711	5991.04219908759\\
1.81890945472736	5997.68736359551\\
1.82091045522761	6004.30743255201\\
1.82291145572786	6010.90309350644\\
1.82491245622811	6017.47520589549\\
1.82691345672836	6024.02451456429\\
1.82891445722861	6030.55187894954\\
1.83091545772886	6037.05810119215\\
1.83291645822911	6043.54415532037\\
1.83491745872936	6050.01084347511\\
1.83691845922961	6056.45908238885\\
1.83891945972987	6062.88978879406\\
1.84092046023012	6069.30382212743\\
1.84292146073037	6075.70215641722\\
1.84492246123062	6082.08559380433\\
1.84692346173087	6088.45505102124\\
1.84892446223112	6094.81144480042\\
1.85092546273137	6101.15557728279\\
1.85292646323162	6107.48842249659\\
1.85492746373187	6113.81072528696\\
1.85692846423212	6120.12340238637\\
1.85892946473237	6126.42725593574\\
1.86093046523262	6132.72314537176\\
1.86293146573287	6139.01181553955\\
1.86493246623312	6145.29412587583\\
1.86693346673337	6151.57087852148\\
1.86893446723362	6157.84287561744\\
1.87093546773387	6164.11080471305\\
1.87293646823412	6170.37552524502\\
1.87493746873437	6176.63778205846\\
1.87693846923462	6182.89826270273\\
1.87893946973487	6189.15776931876\\
1.88094047023512	6195.41698945588\\
1.88294147073537	6201.67672525503\\
1.88494247123562	6207.93760696976\\
1.88694347173587	6214.200436741\\
1.88894447223612	6220.46584482231\\
1.89094547273637	6226.73451876306\\
1.89294647323662	6233.00726070415\\
1.89494747373687	6239.28464360338\\
1.89694847423712	6245.56735501011\\
1.89894947473737	6251.85613976947\\
1.90095047523762	6258.15162813503\\
1.90295147573787	6264.45445036036\\
1.90495247623812	6270.76529399483\\
1.90695347673837	6277.08473199623\\
1.90895447723862	6283.4134519139\\
1.91095547773887	6289.75202670566\\
1.91295647823912	6296.10108662506\\
1.91495747873937	6302.46120462991\\
1.91695847923962	6308.83289638222\\
1.91895947973987	6315.21679213557\\
1.92096048024012	6321.61340755197\\
1.92296148074037	6328.02314370188\\
1.92496248124062	6334.44663083887\\
1.92696348174087	6340.88415544184\\
1.92896448224112	6347.33629046859\\
1.93096548274137	6353.80332239801\\
1.93296648324162	6360.28559500478\\
1.93496748374187	6366.78345206358\\
1.93696848424212	6373.2971800533\\
1.93896948474237	6379.82689356551\\
1.94097048524262	6386.37276448754\\
1.94297148574287	6392.93496470673\\
1.94497248624312	6399.51343692731\\
1.94697348674337	6406.10823844504\\
1.94897448724362	6412.71925466838\\
1.95097548774387	6419.3462564142\\
1.95297648824412	6425.98912909095\\
1.95497748874437	6432.64741433239\\
1.95697848924462	6439.32082565961\\
1.95897948974487	6446.00884741062\\
1.96098049024512	6452.71096392338\\
1.96298149074537	6459.42654494433\\
1.96498249124562	6466.15484562833\\
1.96698349174587	6472.89506383447\\
1.96898449224612	6479.64639742183\\
1.97098549274637	6486.40792965795\\
1.97298649324662	6493.17857192301\\
1.97498749374687	6499.95735018876\\
1.97698849424712	6506.74306124384\\
1.97898949474737	6513.5346737642\\
1.98099049524762	6520.33081265114\\
1.98299149574787	6527.13033198908\\
1.98499249624812	6533.93197127086\\
1.98699349674837	6540.73446998933\\
1.98899449724862	6547.53645304578\\
1.99099549774887	6554.33665993307\\
1.99299649824912	6561.13388743983\\
1.99499749874937	6567.9268750589\\
1.99699849924962	6574.71436228314\\
1.99899949974988	6581.49514590117\\
2.00100050025013	6588.26813729319\\
2.00300150075038	6595.03230513517\\
2.00500250125063	6601.78656080729\\
2.00700350175088	6608.5299302813\\
2.00900450225113	6615.26161141629\\
2.01100550275138	6621.98080207135\\
2.01300650325163	6628.68670010556\\
2.01500750375188	6635.37867526535\\
2.01700850425213	6642.0562118887\\
2.01900950475238	6648.71873701782\\
2.02101050525263	6655.36584958225\\
2.02301150575288	6661.99732039888\\
2.02501250625313	6668.61274839724\\
2.02701350675338	6675.21207628155\\
2.02901450725363	6681.79507486871\\
2.03101550775388	6688.36180145448\\
2.03301650825413	6694.9121987431\\
2.03501750875438	6701.44638132611\\
2.03701850925463	6707.96446379507\\
2.03901950975488	6714.46667533312\\
2.04102051025513	6720.95324512335\\
2.04302151075538	6727.42445964468\\
2.04502251125563	6733.88066267177\\
2.04702351175588	6740.32225527509\\
2.04902451225613	6746.74952393353\\
2.05102551275638	6753.16304160488\\
2.05302651325663	6759.56320935961\\
2.05502751375688	6765.95060015551\\
2.05702851425713	6772.32561506303\\
2.05902951475738	6778.68888433576\\
2.06103051525763	6785.04092363569\\
2.06303151575788	6791.38236321642\\
2.06503251625813	6797.71371873996\\
2.06703351675838	6804.03567775565\\
2.06903451725863	6810.34887051708\\
2.07103551775888	6816.65381268626\\
2.07303651825913	6822.95124910832\\
2.07503751875938	6829.24186733262\\
2.07703851925963	6835.52618302117\\
2.07903951975988	6841.80499831489\\
2.08104052026013	6848.07894346736\\
2.08304152076038	6854.34870602791\\
2.08504252126063	6860.61491625014\\
2.08704352176088	6866.87837627495\\
2.08904452226113	6873.13971635592\\
2.09104552276138	6879.39968133818\\
2.09304652326163	6885.65901606687\\
2.09504752376188	6891.91835079556\\
2.09704852426213	6898.17848766515\\
2.09904952476238	6904.4401715208\\
2.10105052526263	6910.70414720763\\
2.10305152576288	6916.97110227499\\
2.10505252626313	6923.24189615958\\
2.10705352676338	6929.51727370653\\
2.10905452726363	6935.79797976097\\
2.11105552776388	6942.08481646383\\
2.11305652826413	6948.378585956\\
2.11505752876438	6954.68020496997\\
2.11705852926463	6960.9903037593\\
2.11905952976488	6967.30991364803\\
2.12106053026513	6973.63972218552\\
2.12306153076538	6979.98064610423\\
2.12506253126563	6986.33354484087\\
2.12706353176588	6992.69933512789\\
2.12906453226613	6999.07881910621\\
2.13106553276638	7005.47291350831\\
2.13306653326663	7011.88253506666\\
2.13506753376688	7018.30848592217\\
2.13706853426713	7024.75179739887\\
2.13906953476738	7031.21332893346\\
2.14107053526763	7037.69388266685\\
2.14307153576788	7044.19448992306\\
2.14507253626813	7050.71589554724\\
2.14707353676838	7057.25901627186\\
2.14907453726863	7063.8247115336\\
2.15107553776888	7070.41384076916\\
2.15307653826913	7077.02703423212\\
2.15507753876938	7083.66520865495\\
2.15707853926963	7090.32893699544\\
2.15907953976988	7097.01896409873\\
2.16108054027013	7103.73574833104\\
2.16308154077039	7110.47991994597\\
2.16508254127064	7117.2517654224\\
2.16708354177089	7124.05174312657\\
2.16908454227114	7130.87991035427\\
2.17108554277139	7137.7364962886\\
2.17308654327164	7144.62138633801\\
2.17508754377189	7151.53440861516\\
2.17708854427214	7158.47527664115\\
2.17908954477239	7165.44341745819\\
2.18109054527264	7172.43820081271\\
2.18309154577289	7179.45876726801\\
2.18509254627314	7186.50408550005\\
2.18709354677339	7193.57295229748\\
2.18909454727364	7200.66399256158\\
2.19109554777389	7207.77554471474\\
2.19309654827414	7214.90583258781\\
2.19509754877439	7222.05296542005\\
2.19709854927464	7229.21482326762\\
2.19909954977489	7236.38911429935\\
2.20110055027514	7243.57354668406\\
2.20310155077539	7250.76565670322\\
2.20510255127564	7257.96280875097\\
2.20710355177589	7265.16242451725\\
2.20910455227614	7272.36192569197\\
2.21110555277639	7279.55861937349\\
2.21310655327664	7286.74992725173\\
2.21510755377689	7293.93332831241\\
2.21710855427714	7301.10635883699\\
2.21910955477739	7308.26661240274\\
2.22111055527764	7315.41191177003\\
2.22311155577789	7322.54019429081\\
2.22511255627814	7329.6494546128\\
2.22711355677839	7336.7379738626\\
2.22911455727864	7343.80420505417\\
2.23111555777889	7350.8467730888\\
2.23311655827914	7357.86447475512\\
2.23511755877939	7364.85622143332\\
2.23711855927964	7371.82126827827\\
2.23911955977989	7378.75892774061\\
2.24112056028014	7385.66874145411\\
2.24312156078039	7392.55042293987\\
2.24512256128064	7399.40374301476\\
2.24712356178089	7406.2287016788\\
2.24912456228114	7413.02541352354\\
2.25112556278139	7419.79405043632\\
2.25312656328164	7426.53507078337\\
2.25512756378189	7433.2487610436\\
2.25712856428214	7439.93575147057\\
2.25912956478239	7446.59650043052\\
2.26113056528264	7453.2317527686\\
2.26313156578289	7459.84219603414\\
2.26513256628314	7466.42857507229\\
2.26713356678339	7472.99163472817\\
2.26913456728364	7479.53223443849\\
2.27113556778389	7486.05123363993\\
2.27313656828414	7492.5494344734\\
2.27513756878439	7499.02781096717\\
2.27713856928464	7505.48716526213\\
2.27913956978489	7511.92847138655\\
2.28114057028514	7518.35253148134\\
2.28314157078539	7524.76031957474\\
2.28514257128564	7531.15269510346\\
2.28714357178589	7537.53051750418\\
2.28914457228614	7543.89476080515\\
2.29114557278639	7550.24628444308\\
2.29314657328664	7556.58589055886\\
2.29514757378689	7562.91449588498\\
2.29714857428714	7569.23295985812\\
2.29914957478739	7575.54208461921\\
2.30115057528764	7581.84272960492\\
2.30315157578789	7588.1356396604\\
2.30515257628814	7594.4217888139\\
2.30715357678839	7600.70180731899\\
2.30915457728864	7606.97661190814\\
2.31115557778889	7613.2469474265\\
2.31315657828914	7619.5135014234\\
2.31515757878939	7625.77719063133\\
2.31715857928964	7632.03864530386\\
2.31915957978989	7638.2986675819\\
2.32116058029015	7644.55794501481\\
2.3231615807904	7650.81727974349\\
2.32516258129065	7657.07741661309\\
2.3271635817909	7663.3389285814\\
2.32916458229115	7669.60267508511\\
2.3311655827914	7675.86928637779\\
2.33316658329165	7682.13950730459\\
2.3351675837919	7688.41396811906\\
2.33716858429215	7694.69335637058\\
2.3391695847924	7700.97841690427\\
2.34117058529265	7707.26972267792\\
2.3431715857929	7713.56801853668\\
2.34517258629315	7719.87387743833\\
2.3471735867934	7726.18804422801\\
2.34917458729365	7732.51103456773\\
2.3511755877939	7738.84353600686\\
2.35317658829415	7745.18612150318\\
2.3551775887944	7751.53942131026\\
2.35717858929465	7757.90400838592\\
2.3591795897949	7764.28045568793\\
2.36118059029515	7770.66922158253\\
2.3631815907954	7777.07093632331\\
2.36518259129565	7783.48600098071\\
2.3671835917959	7789.91493121676\\
2.36918459229615	7796.3581281019\\
2.3711855927964	7802.81605000238\\
2.37318659329665	7809.28898339709\\
2.3751875937969	7815.77738665227\\
2.37718859429715	7822.28137435948\\
2.3791895947974	7828.80134758917\\
2.38119059529765	7835.33736363712\\
2.3831915957979	7841.88965168646\\
2.38519259629815	7848.45821173718\\
2.3871935967984	7855.04310108506\\
2.38919459729865	7861.64426243432\\
2.3911955977989	7868.26152389762\\
2.39319659829915	7874.89477088341\\
2.3951975987994	7881.54371691279\\
2.39719859929965	7888.20801821107\\
2.3991995997999	7894.88721641203\\
2.40120060030015	7901.58079585365\\
2.4032016008004	7908.2882408739\\
2.40520260130065	7915.00892121923\\
2.4072036018009	7921.74197745293\\
2.40920460230115	7928.48666472987\\
2.4112056028014	7935.24206631758\\
2.41320660330165	7942.00732277937\\
2.4152076038019	7948.78128819964\\
2.41720860430215	7955.56287395859\\
2.4192096048024	7962.35104873218\\
2.42121060530265	7969.14455201327\\
2.4232116058029	7975.94218059048\\
2.42521260630315	7982.74273125244\\
2.4272136068034	7989.54482890046\\
2.42921460730365	7996.34727032315\\
2.4312156078039	8003.14873771759\\
2.43321660830415	8009.94791328085\\
2.4352176088044	8016.74365109733\\
2.43721860930465	8023.53457606834\\
2.4392196098049	8030.31959957406\\
2.44122061030515	8037.09746110734\\
2.4432216108054	8043.86712934415\\
2.44522261130565	8050.62751566468\\
2.4472236118059	8057.37764604067\\
2.44922461230615	8064.11654644388\\
2.4512256128064	8070.8434147334\\
2.45322661330665	8077.5574487683\\
2.4552276138069	8084.25801829502\\
2.45722861430715	8090.94437846841\\
2.4592296148074	8097.61607092225\\
2.46123061530765	8104.2726372903\\
2.4632316158079	8110.91367650211\\
2.46523261630815	8117.53890207877\\
2.4672336168084	8124.14802754138\\
2.46923461730865	8130.74093829839\\
2.4712356178089	8137.31757705402\\
2.47323661830915	8143.87788651249\\
2.4752376188094	8150.42192396957\\
2.47723861930965	8156.94980401684\\
2.4792396198099	8163.46164124584\\
2.48124062031015	8169.95772213547\\
2.48324162081041	8176.43827586886\\
2.48524262131066	8182.90353162911\\
2.48724362181091	8189.35400507826\\
2.48924462231116	8195.7899253994\\
2.49124562281141	8202.21175095879\\
2.49324662331166	8208.61999741843\\
2.49524762381191	8215.01512314456\\
2.49724862431216	8221.39758650342\\
2.49924962481241	8227.76790315702\\
2.50125062531266	8234.12670335894\\
2.50325162581291	8240.47456006698\\
2.50525262631316	8246.81198894314\\
2.50725362681341	8253.13956294523\\
2.50925462731366	8259.45802691837\\
2.51125562781391	8265.76789652459\\
2.51325662831416	8272.06985931323\\
2.51525762881441	8278.36454553788\\
2.51725862931466	8284.6525854521\\
2.51925962981491	8290.93466660525\\
2.52126063031516	8297.21153384247\\
2.52326163081541	8303.48370282577\\
2.52526263131566	8309.75203299184\\
2.52726363181591	8316.01709729847\\
2.52926463231616	8322.27964059081\\
2.53126563281641	8328.54035041821\\
2.53326663331666	8334.79997162579\\
2.53526763381691	8341.05919176292\\
2.53726863431716	8347.3188129705\\
2.53926963481741	8353.57946550212\\
2.54127063531766	8359.84195149868\\
2.54327163581791	8366.10695850953\\
2.54527263631816	8372.37534597138\\
2.54727363681841	8378.64780143358\\
2.54927463731866	8384.92518433281\\
2.55127563781891	8391.20818221844\\
2.55327663831916	8397.49765452715\\
2.55527763881941	8403.79440339985\\
2.55727863931966	8410.09923097747\\
2.55927963981991	8416.4129966967\\
2.56128064032016	8422.73644540266\\
2.56328164082041	8429.07055112361\\
2.56528264132066	8435.41605870468\\
2.56728364182091	8441.77388487835\\
2.56928464232116	8448.14494637709\\
2.57128564282141	8454.53004534181\\
2.57328664332166	8460.93009850498\\
2.57528764382191	8467.34602259907\\
2.57728864432216	8473.77867706079\\
2.57928964482241	8480.22897862259\\
2.58129064532266	8486.69784401696\\
2.58329164582291	8493.18613268057\\
2.58529264632316	8499.69476134592\\
2.58729364682341	8506.22458944969\\
2.58929464732366	8512.77647642857\\
2.59129564782391	8519.35122442348\\
2.59329664832416	8525.9496928711\\
2.59529764882441	8532.5725693208\\
2.59729864932466	8539.22065591348\\
2.59929964982491	8545.8946401985\\
2.60130065032516	8552.59503783788\\
2.60330165082541	8559.32253638097\\
2.60530265132566	8566.07747960244\\
2.60730365182591	8572.86032586854\\
2.60930465232616	8579.67130436238\\
2.61130565282641	8586.51058697129\\
2.61330665332666	8593.37828828685\\
2.61530765382691	8600.27417912593\\
2.61730865432716	8607.19814489697\\
2.61930965482741	8614.14978452951\\
2.62131065532766	8621.1284104742\\
2.62331165582791	8628.13339247747\\
2.62531265632816	8635.16381380684\\
2.62731365682841	8642.21852854673\\
2.62931465732866	8649.2962188942\\
2.63131565782891	8656.39545245477\\
2.63331665832916	8663.51456765083\\
2.63531765882941	8670.65173101744\\
2.63731865932966	8677.80487990653\\
2.63931965982992	8684.97189437426\\
2.64132066033017	8692.15048258945\\
2.64332166083042	8699.33818083359\\
2.64532266133067	8706.53252538815\\
2.64732366183092	8713.73099523884\\
2.64932466233117	8720.93084018823\\
2.65132566283142	8728.12959651781\\
2.65332666333167	8735.32445673439\\
2.65532766383192	8742.51301441522\\
2.65732866433217	8749.69263395444\\
2.65932966483242	8756.86090892932\\
2.66133066533267	8764.01560480446\\
2.66333166583292	8771.15448704445\\
2.66533266633317	8778.27549300124\\
2.66733366683342	8785.37684650565\\
2.66933466733367	8792.4568286843\\
2.67133566783392	8799.51394984692\\
2.67333666833417	8806.5468921906\\
2.67533766883442	8813.5545670955\\
2.67733866933467	8820.53594323761\\
2.67933966983492	8827.49033306757\\
2.68134067033517	8834.41716362758\\
2.68334167083542	8841.31597655142\\
2.68534267133567	8848.18654265595\\
2.68734367183592	8855.02874734962\\
2.68934467233617	8861.84264792822\\
2.69134567283642	8868.62835898329\\
2.69334667333667	8875.38616699374\\
2.69534767383692	8882.11641573424\\
2.69734867433717	8888.81956357104\\
2.69934967483742	8895.49618346192\\
2.70135067533767	8902.14690566046\\
2.70335167583792	8908.77236042023\\
2.70535267633817	8915.37329258638\\
2.70735367683842	8921.95044700402\\
2.70935467733867	8928.50462581408\\
2.71135567783892	8935.03663115747\\
2.71335667833917	8941.54743706244\\
2.71535767883942	8948.03773107834\\
2.71735867933967	8954.50854452921\\
2.71935967983992	8960.96073685174\\
2.72136068034017	8967.39516748262\\
2.72336168084042	8973.81275315432\\
2.72536268134067	8980.21441059931\\
2.72736368184092	8986.6009992543\\
2.72936468234117	8992.97343585175\\
2.73136568284142	8999.33257982834\\
2.73336668334167	9005.67934791656\\
2.73536768384192	9012.01454225733\\
2.73736868434217	9018.3390795831\\
2.73936968484242	9024.65376203479\\
2.74137068534267	9030.95944904911\\
2.74337168584292	9037.25700006273\\
2.74537268634317	9043.54715992079\\
2.74737368684342	9049.83078805999\\
2.74937468734367	9056.10868662124\\
2.75137568784392	9062.38165774545\\
2.75337668834417	9068.65038898198\\
2.75537768884442	9074.91573976751\\
2.75737868934467	9081.17845494719\\
2.75937968984492	9087.43927936614\\
2.76138069034517	9093.69895786951\\
2.76338169084542	9099.95817800663\\
2.76538269134567	9106.21774191844\\
2.76738369184592	9112.47827985849\\
2.76938469234617	9118.74059396772\\
2.77138569284642	9125.00537179546\\
2.77338669334667	9131.27335818685\\
2.77538769384692	9137.54512609969\\
2.77738869434717	9143.82142037911\\
2.77938969484742	9150.10298587025\\
2.78139069534767	9156.39039553089\\
2.78339169584792	9162.68433691041\\
2.78539269634817	9168.98555485392\\
2.78739369684842	9175.29456502344\\
2.78939469734867	9181.61211226411\\
2.79139569784892	9187.93876953373\\
2.79339669834917	9194.27516708586\\
2.79539769884942	9200.6218778783\\
2.79739869934967	9206.97953216463\\
2.79939969984992	9213.34864560687\\
2.80140070035018	9219.72984845858\\
2.80340170085043	9226.12365638178\\
2.80540270135068	9232.53052774271\\
2.80740370185093	9238.95097820339\\
2.80940470235118	9245.38540883427\\
2.81140570285143	9251.83433529736\\
2.81340670335168	9258.29810136735\\
2.81540770385193	9264.77705081891\\
2.81740870435218	9271.27152742672\\
2.81940970485243	9277.78176037389\\
2.82141070535268	9284.30797884355\\
2.82341170585293	9290.85035472303\\
2.82541270635318	9297.40900260389\\
2.82741370685343	9303.98392248614\\
2.82941470735368	9310.57522896132\\
2.83141570785393	9317.18269284633\\
2.83341670835418	9323.80625684538\\
2.83541770885443	9330.44569177536\\
2.83741870935468	9337.10065386158\\
2.83941970985493	9343.77085662516\\
2.84142071035518	9350.45578440407\\
2.84342171085543	9357.15486424052\\
2.84542271135568	9363.86758047249\\
2.84742371185593	9370.59313095907\\
2.84942471235618	9377.33088544669\\
2.85142571285643	9384.0798699071\\
2.85342671335668	9390.83928219937\\
2.85542771385693	9397.60803370371\\
2.85742871435718	9404.38509309608\\
2.85942971485743	9411.16942905244\\
2.86143071535768	9417.95983836143\\
2.86343171585793	9424.75511781169\\
2.86543271635818	9431.55400689605\\
2.86743371685843	9438.35530240315\\
2.86943471735868	9445.15768653006\\
2.87143571785893	9451.95989876963\\
2.87343671835918	9458.76062131894\\
2.87543771885943	9465.5586509666\\
2.87743871935968	9472.35261261393\\
2.87943971985993	9479.14136034531\\
2.88144072036018	9485.92374824518\\
2.88344172086043	9492.69851580636\\
2.88544272136068	9499.46468900062\\
2.88744372186093	9506.22117920814\\
2.88944472236118	9512.96701240068\\
2.89144572286143	9519.70127184574\\
2.89344672336168	9526.42321269822\\
2.89544772386193	9533.13209011297\\
2.89744872436218	9539.82715924485\\
2.89944972486243	9546.50784713607\\
2.90145072536268	9553.17363812462\\
2.90345172586293	9559.82418843583\\
2.90545272636318	9566.45903970344\\
2.90745372686343	9573.07796274435\\
2.90945472736368	9579.68078567122\\
2.91145572786393	9586.2673365967\\
2.91345672836418	9592.83761552081\\
2.91545772886443	9599.39150785197\\
2.91745872936468	9605.92924277331\\
2.91945972986493	9612.45082028483\\
2.92146073036518	9618.95646956964\\
2.92346173086543	9625.44641981087\\
2.92546273136568	9631.92090019163\\
2.92746373186593	9638.38036907815\\
2.92946473236618	9644.82505565356\\
2.93146573286643	9651.2554182841\\
2.93346673336668	9657.67191533599\\
2.93546773386693	9664.07494787969\\
2.93746873436718	9670.46503157723\\
2.93946973486743	9676.84268209061\\
2.94147073536768	9683.20847237763\\
2.94347173586793	9689.56286080453\\
2.94547273636818	9695.90653492066\\
2.94747373686843	9702.24006768381\\
2.94947473736868	9708.56403205179\\
2.95147573786893	9714.87905827816\\
2.95347673836918	9721.18571932073\\
2.95547773886943	9727.48476002461\\
2.95747873936968	9733.77675334762\\
2.95947973986994	9740.06244413488\\
2.96148074037019	9746.34234804842\\
2.96348174087044	9752.61732452491\\
2.96548274137069	9758.88788922638\\
2.96748374187094	9765.15484429374\\
2.96948474237119	9771.41887727635\\
2.97148574287144	9777.68061842777\\
2.97348674337169	9783.94081259315\\
2.97548774387194	9790.20020461762\\
2.97748874437219	9796.45948205052\\
2.97948974487244	9802.71938973701\\
2.98149074537269	9808.9806725222\\
2.98349174587294	9815.24407525123\\
2.98549274637319	9821.51034276924\\
2.98749374687344	9827.78021992135\\
2.98949474737369	9834.05450884849\\
2.99149574787394	9840.33395439579\\
2.99349674837419	9846.61941599993\\
2.99549774887444	9852.91163850606\\
2.99749874937469	9859.21148135086\\
2.99949974987494	9865.51968937947\\
3.00150075037519	9871.83712202858\\
3.00350175087544	9878.16463873489\\
3.00550275137569	9884.50309893508\\
3.00750375187594	9890.85336206586\\
3.00950475237619	9897.21623026812\\
3.01150575287644	9903.59267757013\\
3.01350675337669	9909.9835061128\\
3.01550775387694	9916.38968992438\\
3.01750875437719	9922.81203114578\\
3.01950975487744	9929.25150380526\\
3.02151075537769	9935.7089673395\\
3.02351175587794	9942.18533848098\\
3.02551275637819	9948.68153396217\\
3.02751375687844	9955.19835592399\\
3.02951475737869	9961.73672109891\\
3.03151575787894	9968.29743162783\\
3.03351675837919	9974.88134694746\\
3.03551775887944	9981.48932649448\\
3.03751875937969	9988.12200052248\\
3.03951975987994	9994.78011387657\\
3.04152076038019	10001.4643541061\\
3.04352176088044	10008.1752941689\\
3.04552276138069	10014.9135070228\\
3.04752376188094	10021.6793364424\\
3.04952476238119	10028.4731262024\\
3.05152576288144	10035.2952200774\\
3.05352676338169	10042.1456180676\\
3.05552776388194	10049.0243774686\\
3.05752876438219	10055.9313263931\\
3.05952976488244	10062.8661210665\\
3.06153076538269	10069.828417714\\
3.06353176588294	10076.8174714906\\
3.06553276638319	10083.832537551\\
3.06753376688344	10090.8725845714\\
3.06953476738369	10097.936466636\\
3.07153576788394	10105.0228086461\\
3.07353676838419	10112.1300063201\\
3.07553776888444	10119.2563980801\\
3.07753876938469	10126.4000931656\\
3.07953976988494	10133.5589143368\\
3.08154077038519	10140.7307416501\\
3.08354177088544	10147.9131686825\\
3.08554277138569	10155.1037890114\\
3.08754377188594	10162.3000816224\\
3.08954477238619	10169.4994109098\\
3.09154577288644	10176.6991412676\\
3.09354677338669	10183.8966370901\\
3.09554777388694	10191.0893773628\\
3.09754877438719	10198.2746691842\\
3.09954977488744	10205.4501634273\\
3.10155077538769	10212.6133963736\\
3.10355177588794	10219.7621334877\\
3.10555277638819	10226.8942548257\\
3.10755377688844	10234.0078123311\\
3.10955477738869	10241.1009725391\\
3.11155577788894	10248.1721884634\\
3.11355677838919	10255.2200277098\\
3.11555777888944	10262.2431724752\\
3.11755877938969	10269.2406487313\\
3.11955977988994	10276.2115397458\\
3.1215607803902	10283.155215265\\
3.12356178089045	10290.0711023311\\
3.1255627813907	10296.9588571695\\
3.12756378189095	10303.8183651886\\
3.1295647823912	10310.649454501\\
3.13156578289145	10317.4522969942\\
3.1335667833917	10324.2270072596\\
3.13556778389195	10330.9739863677\\
3.1375687843922	10337.6935780932\\
3.13956978489245	10344.3862980981\\
3.1415707853927	10351.0527193403\\
3.14357178589295	10357.6934720732\\
3.1455727863932	10364.3091865504\\
3.14757378689345	10370.9007222087\\
3.1495747873937	10377.4687665974\\
3.15157578789395	10384.0141791532\\
3.1535767883942	10390.5378193128\\
3.15557778889445	10397.0404892171\\
3.1575787893947	10403.5230483027\\
3.15957978989495	10409.9864705981\\
3.1615807903952	10416.4316155397\\
3.16358179089545	10422.8593425644\\
3.1655827913957	10429.2705684046\\
3.16758379189595	10435.6662097927\\
3.1695847923962	10442.0471834613\\
3.17158579289645	10448.4142915512\\
3.1735867933967	10454.768450795\\
3.17558779389695	10461.1105779251\\
3.1775887943972	10467.4415323782\\
3.17958979489745	10473.7621162952\\
3.1815907953977	10480.0731891128\\
3.18359179589795	10486.3755529719\\
3.1855927963982	10492.6701246049\\
3.18759379689845	10498.9576488572\\
3.1895947973987	10505.2389851652\\
3.19159579789895	10511.5148210784\\
3.1935967983992	10517.7860160334\\
3.19559779889945	10524.0533721712\\
3.1975987993997	10530.3175770412\\
3.19959979989995	10536.5794900799\\
3.2016008004002	10542.8397415411\\
3.20360180090045	10549.0991335656\\
3.2056028014007	10555.3584109985\\
3.20760380190095	10561.6182613892\\
3.2096048024012	10567.8794868786\\
3.21160580290145	10574.1426604245\\
3.2136068034017	10580.4085841678\\
3.21560780390195	10586.6779456579\\
3.2176088044022	10592.9514324441\\
3.21960980490245	10599.2297320758\\
3.2216108054027	10605.5134748066\\
3.22361180590295	10611.8033481858\\
3.2256128064032	10618.1000970585\\
3.22761380690345	10624.4042370868\\
3.2296148074037	10630.7165704115\\
3.23161580790395	10637.0375553989\\
3.2336168084042	10643.36799419\\
3.23561780890445	10649.7083451509\\
3.2376188094047	10656.059295831\\
3.23961980990495	10662.4213618924\\
3.2416208104052	10668.7951735885\\
3.24362181090545	10675.1811892857\\
3.2456228114057	10681.5800392375\\
3.24762381190595	10687.9921818102\\
3.2496248124062	10694.4180753699\\
3.25162581290645	10700.8581209872\\
3.2536268134067	10707.312834324\\
3.25562781390695	10713.7825018593\\
3.2576288144072	10720.2674673677\\
3.25962981490745	10726.7680746239\\
3.2616308154077	10733.2845528111\\
3.26363181590795	10739.8170738165\\
3.2656328164082	10746.3658095275\\
3.26763381690845	10752.9308172398\\
3.2696348174087	10759.5122115452\\
3.27163581790895	10766.1098205561\\
3.2736368184092	10772.7236442726\\
3.27563781890945	10779.3534535117\\
3.27763881940971	10785.9989617943\\
3.27963981990996	10792.6599972331\\
3.2816408204102	10799.3359868704\\
3.28364182091046	10806.0264723399\\
3.28564282141071	10812.7309952754\\
3.28764382191096	10819.4488108318\\
3.28964482241121	10826.1792887554\\
3.29164582291146	10832.9215123138\\
3.29364682341171	10839.6746793662\\
3.29564782391196	10846.4378731799\\
3.29764882441221	10853.2100051352\\
3.29964982491246	10859.9901012039\\
3.30165082541271	10866.7769008788\\
3.30365182591296	10873.5693728358\\
3.30565282641321	10880.3661992721\\
3.30765382691346	10887.1662342721\\
3.30965482741371	10893.968102737\\
3.31165582791396	10900.7705441597\\
3.31365682841421	10907.5723553288\\
3.31565782891446	10914.3721611456\\
3.31765882941471	10921.168701103\\
3.31965982991496	10927.9608292854\\
3.32166083041521	10934.747170594\\
3.32366183091546	10941.5266937049\\
3.32566283141571	10948.2982527025\\
3.32766383191596	10955.0607589668\\
3.32966483241621	10961.813181174\\
3.33166583291646	10968.5546025915\\
3.33366683341671	10975.2841637827\\
3.33566783391696	10982.0010626065\\
3.33766883441721	10988.704554218\\
3.33966983491746	10995.3940656593\\
3.34167083541771	11002.0690239726\\
3.34367183591796	11008.7288562\\
3.34567283641821	11015.3733331586\\
3.34767383691846	11022.001996482\\
3.34967483741871	11028.6146169872\\
3.35167583791896	11035.2110800825\\
3.35367683841921	11041.7912711765\\
3.35567783891946	11048.3551329733\\
3.35767883941971	11054.9026654729\\
3.35967983991996	11061.4340405627\\
3.36168084042021	11067.9493728343\\
3.36368184092046	11074.4488341749\\
3.36568284142071	11080.9327110635\\
3.36768384192096	11087.401289979\\
3.36968484242121	11093.854914696\\
3.37168584292146	11100.2939289892\\
3.37368684342171	11106.7187912249\\
3.37568784392196	11113.1299597693\\
3.37768884442221	11119.5278929887\\
3.37968984492246	11125.9130492492\\
3.38169084542271	11132.2860015086\\
3.38369184592296	11138.6472654291\\
3.38569284642321	11144.9974139683\\
3.38769384692346	11151.3370200841\\
3.38969484742371	11157.66671403\\
3.39169584792396	11163.9870687639\\
3.39369684842421	11170.298771835\\
3.39569784892446	11176.6023962013\\
3.39769884942471	11182.8985721162\\
3.39969984992496	11189.1880444249\\
3.40170085042521	11195.4713860852\\
3.40370185092546	11201.7492846464\\
3.40570285142571	11208.022427658\\
3.40770385192596	11214.2915026692\\
3.40970485242621	11220.5572545252\\
3.41170585292646	11226.8202561838\\
3.41370685342671	11233.0813097858\\
3.41570785392696	11239.3411601765\\
3.41770885442721	11245.6003803137\\
3.41970985492746	11251.8598296339\\
3.42171085542771	11258.1201956866\\
3.42371185592796	11264.382223317\\
3.42571285642821	11270.64665737\\
3.42771385692846	11276.914242691\\
3.42971485742871	11283.1857241249\\
3.43171585792896	11289.4619611085\\
3.43371685842921	11295.743698487\\
3.43571785892946	11302.0316811055\\
3.43771885942971	11308.3268256963\\
3.43971985992996	11314.629819809\\
3.44172086043022	11320.9415228802\\
3.44372186093047	11327.2628516423\\
3.44572286143072	11333.5945509405\\
3.44772386193097	11339.9374802115\\
3.44972486243122	11346.2925561878\\
3.45172586293147	11352.6606956017\\
3.45372686343172	11359.0426432986\\
3.45572786393197	11365.4393733065\\
3.45772886443222	11371.8518023581\\
3.45972986493247	11378.2807325941\\
3.46173086543272	11384.7271380429\\
3.46373186593297	11391.1918781412\\
3.46573286643322	11397.6759269171\\
3.46773386693347	11404.1800292159\\
3.46973486743372	11410.7052163615\\
3.47173586793397	11417.2522331991\\
3.47373686843422	11423.8219391654\\
3.47573786893447	11430.4151936971\\
3.47773886943472	11437.0326843435\\
3.47973986993497	11443.6752705414\\
3.48174087043522	11450.3435825442\\
3.48374187093547	11457.0381933099\\
3.48574287143572	11463.7597330919\\
3.48774387193597	11470.5087175523\\
3.48974487243622	11477.28543317\\
3.49174587293647	11484.0902810154\\
3.49374687343672	11490.9234329759\\
3.49574787393697	11497.7849463473\\
3.49774887443722	11504.6747638338\\
3.49974987493747	11511.5927135479\\
3.50175087543772	11518.5383371236\\
3.50375187593797	11525.5111761946\\
3.50575287643822	11532.5105432115\\
3.50775387693847	11539.5355214418\\
3.50975487743872	11546.5850222657\\
3.51175587793897	11553.657842472\\
3.51375687843922	11560.7525496659\\
3.51575787893947	11567.8675395659\\
3.51775887943972	11575.000921411\\
3.51975987993997	11582.1507471449\\
3.52176088044022	11589.3149541195\\
3.52376188094047	11596.4911932077\\
3.52576288144072	11603.6771725784\\
3.52776388194097	11610.8703139216\\
3.52976488244122	11618.068096223\\
3.53176588294147	11625.2678838766\\
3.53376688344172	11632.4670412766\\
3.53576788394197	11639.662990113\\
3.53776888444222	11646.8531520757\\
3.53976988494247	11654.0348915587\\
3.54177088544272	11661.2058594352\\
3.54377188594297	11668.3636492825\\
3.54577288644322	11675.5061411566\\
3.54777388694347	11682.6312151137\\
3.54977488744372	11689.7369803931\\
3.55177588794397	11696.8217754173\\
3.55377688844422	11703.8839959043\\
3.55577788894447	11710.9222667554\\
3.55777888944472	11717.935499351\\
3.55977988994497	11724.9226623668\\
3.56178089044522	11731.882953662\\
3.56378189094547	11738.8158002789\\
3.56578289144572	11745.7207438511\\
3.56778389194597	11752.5974406041\\
3.56978489244622	11759.4458332419\\
3.57178589294647	11766.2658071732\\
3.57378689344672	11773.0576488766\\
3.57578789394697	11779.8214729439\\
3.57778889444722	11786.5576231497\\
3.57978989494747	11793.2666724518\\
3.58179089544772	11799.9490219207\\
3.58379189594797	11806.6053018098\\
3.58579289644822	11813.2361423729\\
3.58779389694847	11819.8423457507\\
3.58979489744872	11826.4245994927\\
3.59179589794897	11832.9837057398\\
3.59379689844922	11839.5204666329\\
3.59579789894947	11846.0357416087\\
3.59779889944972	11852.5304473996\\
3.59979989994997	11859.0054434424\\
3.60180090045022	11865.4615891737\\
3.60380190095048	11871.899801326\\
3.60580290145073	11878.3209393361\\
3.60780390195098	11884.7259772321\\
3.60980490245123	11891.1157744507\\
3.61180590295148	11897.4911904287\\
3.61380690345173	11903.8531418985\\
3.61580790395198	11910.2025455925\\
3.61780890445223	11916.540146356\\
3.61980990495248	11922.8669182172\\
3.62181090545273	11929.183663317\\
3.62381190595298	11935.491241092\\
3.62581290645323	11941.790510979\\
3.62781390695348	11948.0822178231\\
3.62981490745373	11954.367221061\\
3.63181590795398	11960.6462655379\\
3.63381690845423	11966.9202106903\\
3.63581790895448	11973.1898586593\\
3.63781890945473	11979.4558396985\\
3.63981990995498	11985.7190705401\\
3.64182091045523	11991.9802387338\\
3.64382191095548	11998.2400891245\\
3.64582291145573	12004.4993665574\\
3.64782391195598	12010.7587585818\\
3.64982491245623	12017.0190673388\\
3.65182591295648	12023.2809230818\\
3.65382691345673	12029.5451279517\\
3.65582791395698	12035.8123122022\\
3.65782891445723	12042.0832206783\\
3.65982991495748	12048.3584836337\\
3.66183091545773	12054.6388459135\\
3.66383191595798	12060.9248804754\\
3.66583291645823	12067.2173894605\\
3.66783391695848	12073.5169458264\\
3.66983491745873	12079.8242371225\\
3.67183591795898	12086.1398936025\\
3.67383691845923	12092.464488224\\
3.67583791895948	12098.7987085365\\
3.67783891945973	12105.1431274978\\
3.67983991995998	12111.4983753614\\
3.68184092046023	12117.8649677893\\
3.68384192096048	12124.2435350352\\
3.68584292146073	12130.6345354652\\
3.68784392196098	12137.0384847413\\
3.68984492246123	12143.4559558215\\
3.69184592296148	12149.8873497761\\
3.69384692346173	12156.3331249713\\
3.69584792396198	12162.7936251818\\
3.69784892446223	12169.2692514782\\
3.69984992496248	12175.7602903392\\
3.70185092546273	12182.2670855396\\
3.70385192596298	12188.7898089667\\
3.70585292646323	12195.3286898037\\
3.70785392696348	12201.8837853462\\
3.70985492746373	12208.4552101859\\
3.71185592796398	12215.042907027\\
3.71385692846423	12221.6469331652\\
3.71585792896448	12228.2670021217\\
3.71785892946473	12234.9029993049\\
3.71985992996498	12241.5546382359\\
3.72186093046523	12248.2215751401\\
3.72386193096548	12254.9032943553\\
3.72586293146573	12261.5993948112\\
3.72786393196598	12268.3091889584\\
3.72986493246623	12275.0320465434\\
3.73186593296648	12281.7672227209\\
3.73386693346673	12288.5138580544\\
3.73586793396698	12295.2710358113\\
3.73786893446723	12302.0378392591\\
3.73986993496748	12308.8132370738\\
3.74187093546773	12315.5960833399\\
3.74387193596798	12322.3852894376\\
3.74587293646823	12329.1795948596\\
3.74787393696848	12335.9777963946\\
3.74987493746873	12342.7786908312\\
3.75187593796898	12349.5809603666\\
3.75387693846923	12356.3832871977\\
3.75587793896948	12363.1844681132\\
3.75787893946973	12369.9831280145\\
3.75987993996999	12376.7781209858\\
3.76188094047024	12383.5681292244\\
3.76388194097049	12390.3519495187\\
3.76588294147074	12397.1284932491\\
3.76788394197099	12403.8966144998\\
3.76988494247124	12410.655281947\\
3.77188594297149	12417.4035215623\\
3.77388694347174	12424.1403593174\\
3.77588794397199	12430.8650503673\\
3.77788894447224	12437.576792571\\
3.77988994497249	12444.2748983792\\
3.78189094547274	12450.9587948341\\
3.78389194597299	12457.6278516821\\
3.78589294647324	12464.2817251485\\
3.78789394697349	12470.9200141629\\
3.78989494747374	12477.5424322463\\
3.79189594797399	12484.1487502158\\
3.79389694847424	12490.7388534796\\
3.79589794897449	12497.3126274462\\
3.79789894947474	12503.8701294115\\
3.79989994997499	12510.4113593754\\
3.80190095047524	12516.9363746337\\
3.80390195097549	12523.4454616653\\
3.80590295147574	12529.9387923575\\
3.80790395197599	12536.4166531892\\
3.80990495247624	12542.8793306394\\
3.81190595297649	12549.3272257785\\
3.81390695347674	12555.7606823811\\
3.81590795397699	12562.1801015177\\
3.81790895447724	12568.5859988504\\
3.81990995497749	12574.9788900411\\
3.82191095547774	12581.3591761604\\
3.82391195597799	12587.7274874617\\
3.82591295647824	12594.0843396071\\
3.82791395697849	12600.4303055544\\
3.82991495747874	12606.7659582614\\
3.83191595797899	12613.0919279817\\
3.83391695847924	12619.4088449688\\
3.83591795897949	12625.7172821805\\
3.83791895947974	12632.0179271662\\
3.83991995997999	12638.3114101795\\
3.84192096048024	12644.5983614739\\
3.84392196098049	12650.8795258946\\
3.84592296148074	12657.1554763993\\
3.84792396198099	12663.4269578333\\
3.84992496248124	12669.6946577458\\
3.85192596298149	12675.9592063904\\
3.85392696348174	12682.2214059081\\
3.85592796398199	12688.4818865523\\
3.85792896448224	12694.7413931684\\
3.85992996498249	12701.0006133055\\
3.86193096548274	12707.2603491046\\
3.86393196598299	12713.5212308194\\
3.86593296648324	12719.7841178864\\
3.86793396698349	12726.049697855\\
3.86993496748374	12732.3187155705\\
3.87193596798399	12738.5919731736\\
3.87393696848424	12744.8702728053\\
3.87593796898449	12751.1543593107\\
3.87793896948474	12757.4450348308\\
3.87993996998499	12763.7431588022\\
3.88194097048524	12770.0494760701\\
3.88394197098549	12776.3648460711\\
3.88594297148574	12782.6901282421\\
3.88794397198599	12789.0261247237\\
3.88994497248624	12795.3737522486\\
3.89194597298649	12801.7338129577\\
3.89394697348674	12808.1072235834\\
3.89594797398699	12814.4948435624\\
3.89794897448724	12820.897646923\\
3.89994997498749	12827.3163785103\\
3.90195097548774	12833.7520696483\\
3.90395197598799	12840.2055797738\\
3.90595297648824	12846.6777683234\\
3.90795397698849	12853.1695520295\\
3.90995497748874	12859.681790329\\
3.91195597798899	12866.2153999542\\
3.91395697848924	12872.7712403419\\
3.91595797898949	12879.3500563371\\
3.91795897948974	12885.9527073767\\
3.91995997998999	12892.5799383056\\
3.92196098049025	12899.2324939691\\
3.9239619809905	12905.9110619165\\
3.92596298149075	12912.6161578098\\
3.927963981991	12919.3483546068\\
3.92996498249125	12926.1081679695\\
3.9319659829915	12932.8958843769\\
3.93396698349175	12939.7117903077\\
3.935967983992	12946.5560576495\\
3.93796898449225	12953.4286864021\\
3.9399699849925	12960.3295046782\\
3.94197098549275	12967.2583405905\\
3.943971985993	12974.2146784769\\
3.94597298649325	12981.1980026755\\
3.9479739869935	12988.2075110454\\
3.94997498749375	12995.2422295582\\
3.951975987994	13002.3010122985\\
3.95397698849425	13009.3825987589\\
3.9559779889945	13016.4854419536\\
3.95797898949475	13023.6078230091\\
3.959979989995	13030.7479084605\\
3.96198099049525	13037.9036356596\\
3.9639819909955	13045.072827367\\
3.96598299149575	13052.2531917514\\
3.967983991996	13059.4422650942\\
3.96998499249625	13066.6375263812\\
3.9719859929965	13073.8363973024\\
3.97398699349675	13081.0362422518\\
3.975987993997	13088.2344829193\\
3.97798899449725	13095.4284836992\\
3.9799899949975	13102.6156089856\\
3.98199099549775	13109.7933950598\\
3.983991995998	13116.9594927951\\
3.98599299649825	13124.1115530644\\
3.9879939969985	13131.2474559238\\
3.98999499749875	13138.3651960212\\
3.991995997999	13145.4629398915\\
3.99399699849925	13152.5390832529\\
3.9959979989995	13159.5920791194\\
3.99799899949975	13166.6206669838\\
4	13173.6238155222\\
};
\addlegendentry{$\psi$};

\addplot [color=mycolor3,solid]
  table[row sep=crcr]{%
0	20.0001231630726\\
0.00200100050025012	20.0116396147547\\
0.00400200100050025	20.0423501525738\\
0.00600300150075038	20.0922547765297\\
0.0080040020010005	20.1612388950634\\
0.0100050025012506	20.2492452123955\\
0.0120060030015008	20.3560445454079\\
0.0140070035017509	20.4815223025415\\
0.016008004002001	20.6255065964579\\
0.0180090045022511	20.7878255398185\\
0.0200100050025013	20.9682499495052\\
0.0220110055027514	21.1664933466204\\
0.0240120060030015	21.3823838438257\\
0.0260130065032516	21.6156349622235\\
0.0280140070035018	21.8659602229161\\
0.0300150075037519	22.1331304427856\\
0.032016008004002	22.4168591429344\\
0.0340170085042521	22.7168025486854\\
0.0360180090045022	23.0328460684796\\
0.0380190095047524	23.3645886318603\\
0.0400200100050025	23.7118010557096\\
0.0420210105052526	24.0742541569094\\
0.0440220110055028	24.451661456562\\
0.0460230115057529	24.8437937715496\\
0.048024012006003	25.2504219187539\\
0.0500250125062531	25.6712594192775\\
0.0520260130065033	26.1061916815613\\
0.0540270135067534	26.5548749309283\\
0.0560280140070035	27.0172518715988\\
0.0580290145072536	27.4930360246755\\
0.0600300150075038	27.9821127985991\\
0.0620310155077539	28.4841957144723\\
0.064032016008004	28.9992274765154\\
0.0660330165082541	29.5269216058309\\
0.0680340170085043	30.0672781024188\\
0.0700350175087544	30.6200104873815\\
0.0720360180090045	31.1850614649395\\
0.0740370185092546	31.7622591477543\\
0.0760380190095048	32.3513743527078\\
0.0780390195097549	32.9524070798\\
0.080040020010005	33.5651854416924\\
0.0820410205102551	34.189480255267\\
0.0840420210105053	34.8252342247441\\
0.0860430215107554	35.4722754627854\\
0.0880440220110055	36.1304893778317\\
0.0900450225112556	36.7997040825445\\
0.0920460230115058	37.4796903938057\\
0.0940470235117559	38.1703910158359\\
0.096048024012006	38.871576765517\\
0.0980490245122561	39.5830757555105\\
0.100050025012506	40.3046588026982\\
0.102051025512756	41.0361540197418\\
0.104052026013007	41.777332223523\\
0.106053026513257	42.5279642309239\\
0.108054027013507	43.2877635630469\\
0.110055027513757	44.0565010367739\\
0.112056028014007	44.8338328774279\\
0.114057028514257	45.6194726061113\\
0.116058029014507	46.4131337439265\\
0.118059029514757	47.214415220417\\
0.120060030015008	48.0229159651261\\
0.122061030515258	48.8382922033767\\
0.124062031015508	49.6600282731534\\
0.126063031515758	50.4877231039993\\
0.128064032016008	51.3208610338991\\
0.130065032516258	52.1589264008369\\
0.132066033016508	53.0012889512383\\
0.134067033516758	53.8474330230875\\
0.136068034017009	54.6967283628099\\
0.138069034517259	55.5484874210514\\
0.140070035017509	56.4020226484578\\
0.142071035517759	57.2565891998954\\
0.144072036018009	58.111384934451\\
0.146073036518259	58.9655504154321\\
0.148074037018509	59.8183407977048\\
0.150075037518759	60.6687820530175\\
0.15207603801901	61.5159001531184\\
0.15407703851926	62.3587783655354\\
0.15607803901951	63.1963853662371\\
0.15807903951976	64.0275752396334\\
0.16008004002001	64.851316661693\\
0.16208104052026	65.6664637168256\\
0.16408204102051	66.471870489441\\
0.16608304152076	67.2662191766104\\
0.168084042021011	68.0483065669639\\
0.170085042521261	68.8169294491319\\
0.172086043021511	69.5707127244061\\
0.174087043521761	70.308281294078\\
0.176088044022011	71.0283746509984\\
0.178089044522261	71.729617696459\\
0.180090045022511	72.410578035972\\
0.182091045522761	73.0699378666085\\
0.184092046023012	73.7063220896603\\
0.186093046523262	74.3182983106396\\
0.188094047023512	74.9045487266174\\
0.190095047523762	75.4636982388856\\
0.192096048024012	75.9944863402948\\
0.194097048524262	76.4955952279162\\
0.196098049024512	76.9657643946006\\
0.198099049524762	77.4038479247576\\
0.200100050025012	77.8087571985765\\
0.202101050525263	78.1793463004672\\
0.204102051025513	78.5146412021777\\
0.206103051525763	78.813725171236\\
0.208104052026013	79.0757960667289\\
0.210105052526263	79.3001090435226\\
0.212106053026513	79.4860338480425\\
0.214107053526763	79.6330548182731\\
0.216108054027013	79.7406562921987\\
0.218109054527264	79.8086663824807\\
0.220110055027514	79.8367986102216\\
0.222111055527764	79.8249956796419\\
0.224112056028014	79.7732575907416\\
0.226113056528264	79.6817562308592\\
0.228114057028514	79.5508353746718\\
0.230115057528764	79.3807242052975\\
0.232116058029014	79.1720529763108\\
0.234117058529265	78.925280053948\\
0.236118059029515	78.6412075791221\\
0.238119059529765	78.3206376927464\\
0.240120060030015	77.9643152399546\\
0.242121060530265	77.5733288405573\\
0.244122061030515	77.1487098185858\\
0.246123061530765	76.6914894980714\\
0.248124062031016	76.2028137946044\\
0.250125062531266	75.6840578068929\\
0.252126063031516	75.1363101547478\\
0.254127063531766	74.561003232657\\
0.256128064032016	73.9593402519901\\
0.258129064532266	73.3328109030145\\
0.260130065032516	72.6826756928796\\
0.262131065532766	72.0103670160731\\
0.264132066033017	71.3172026755238\\
0.266133066533267	70.6046723614991\\
0.268134067033517	69.8740365811483\\
0.270135067533767	69.1267277289592\\
0.272136068034017	68.3640636078605\\
0.274137068534267	67.5873047250017\\
0.276138069034517	66.7977688833114\\
0.278139069534767	65.9967738857185\\
0.280140070035018	65.1854656478133\\
0.282141070535268	64.365104676745\\
0.284142071035518	63.5367795923243\\
0.286143071535768	62.7016363101416\\
0.288144072036018	61.8608207457872\\
0.290145072536268	61.0153069275126\\
0.292146073036518	60.1661261793492\\
0.294147073536768	59.3141952337692\\
0.296148074037018	58.4604308232447\\
0.298149074537269	57.6058069760276\\
0.300150075037519	56.7510685372514\\
0.302151075537769	55.8970176478294\\
0.304152076038019	55.0444564486748\\
0.306153076538269	54.1940724891416\\
0.308154077038519	53.3465533185841\\
0.310155077538769	52.5025291905769\\
0.31215607803902	51.6626876544741\\
0.31415707853927	50.8274870765119\\
0.31615807903952	49.9975004144854\\
0.31815907953977	49.1732433304102\\
0.32016008004002	48.3552314863019\\
0.32216108054027	47.5438659526172\\
0.32416208104052	46.739605095592\\
0.32616308154077	45.9427926899036\\
0.32816408204102	45.1538298060085\\
0.330165082541271	44.3730029228042\\
0.332166083041521	43.6007131107473\\
0.334167083541771	42.8372468487355\\
0.336168084042021	42.0828333198868\\
0.338169084542271	41.3377590030986\\
0.340170085042521	40.6023103772687\\
0.342171085542771	39.8766593297355\\
0.344172086043022	39.1609777478376\\
0.346173086543272	38.4556094062521\\
0.348174087043522	37.7606116007584\\
0.350175087543772	37.0762135144746\\
0.352176088044022	36.4025870347393\\
0.354177088544272	35.7398467531115\\
0.356178089044522	35.0882218527092\\
0.358179089544772	34.4478269250915\\
0.360180090045022	33.8188338575968\\
0.362181090545273	33.2013572417844\\
0.364182091045523	32.595511669213\\
0.366183091545773	32.0014690272214\\
0.368184092046023	31.419401203148\\
0.370185092546273	30.8493654927723\\
0.372186093046523	30.2915910792125\\
0.374187093546773	29.7461925540274\\
0.376188094047024	29.2132845087763\\
0.378189094547274	28.6930388307975\\
0.380190095047524	28.1856274074296\\
0.382191095547774	27.6912221260112\\
0.384192096048024	27.2100521696604\\
0.386193096548274	26.7421748341565\\
0.388194097048524	26.2878193026178\\
0.390195097548774	25.8472720539417\\
0.392196098049024	25.4206476796873\\
0.394197098549275	25.0081753629726\\
0.396198099049525	24.6100842869157\\
0.398199099549775	24.2266609304142\\
0.400200100050025	23.8580771808065\\
0.402201100550275	23.5045622212108\\
0.404202101050525	23.1664598263041\\
0.406203101550775	22.8439991792045\\
0.408204102051026	22.5374667588095\\
0.410205102551276	22.2470344524577\\
0.412206103051526	21.9731033306056\\
0.414207103551776	21.7158452805919\\
0.416208104052026	21.4756040770935\\
0.418209104552276	21.2526089032286\\
0.420210105052526	21.0470889421152\\
0.422211105552776	20.8593879684303\\
0.424212106053027	20.6896205737331\\
0.426213106553277	20.5381305327005\\
0.428214107053527	20.4050324368916\\
0.430215107553777	20.2905554694244\\
0.432216108054027	20.1949288134171\\
0.434217108554277	20.1181524688696\\
0.436218109054527	20.0604556188999\\
0.438219109554777	20.0218955592876\\
0.440220110055028	20.0024722900327\\
0.442221110555278	20.0023004026941\\
0.444222111055528	20.021379897272\\
0.446223111555778	20.0596534779867\\
0.448224112056028	20.1170065532793\\
0.450225112556278	20.1934391231498\\
0.452226113056528	20.2887793002595\\
0.454227113556778	20.4029124930496\\
0.456228114057029	20.5357241099609\\
0.458229114557279	20.6868703763164\\
0.460230115057529	20.8562939963366\\
0.462231115557779	21.0437657869034\\
0.464232116058029	21.2489419733398\\
0.466233116558279	21.4716506683071\\
0.468234117058529	21.7116053929079\\
0.470235117558779	21.9685769640241\\
0.47223611805903	22.2422216069786\\
0.47423711855928	22.5323674344328\\
0.47623811905953	22.8386706717097\\
0.47823911955978	23.1608448399118\\
0.48024012006003	23.4986607559209\\
0.48224112056028	23.8518892366191\\
0.48424212106053	24.2202438031087\\
0.48624312156078	24.6034379764922\\
0.488244122061031	25.0012425736515\\
0.490245122561281	25.4134857072481\\
0.492246123061531	25.8398808983845\\
0.494247123561781	26.2801989639425\\
0.496248124062031	26.7342680165837\\
0.498249124562281	27.2019161689695\\
0.500250125062531	27.6829142379818\\
0.502251125562781	28.1770903362822\\
0.504252126063031	28.684272576532\\
0.506253126563282	29.2042890713927\\
0.508254127063532	29.7369679335258\\
0.510255127563782	30.2821945713723\\
0.512256128064032	30.8397398018141\\
0.514257128564282	31.4095463290717\\
0.516258129064532	31.9914422658066\\
0.518259129564782	32.5852557246802\\
0.520260130065032	33.190929409913\\
0.522261130565283	33.8081768426074\\
0.524262131065533	34.4369980227635\\
0.526263131565783	35.0772210630427\\
0.528264132066033	35.7286167803269\\
0.530265132566283	36.3911278788367\\
0.532266133066533	37.0645824712334\\
0.534267133566783	37.7488086701787\\
0.536268134067034	38.4436345883338\\
0.538269134567284	39.1488883383604\\
0.540270135067534	39.8643407371402\\
0.542271135567784	40.5898198973349\\
0.544272136068034	41.3251539316058\\
0.546273136568284	42.0700563610554\\
0.548274137068534	42.8242980025656\\
0.550275137568784	43.5876496730184\\
0.552276138069035	44.3597675977366\\
0.554277138569285	45.1404225936024\\
0.556278139069535	45.9292708859385\\
0.558279139569785	46.7259114042884\\
0.560280140070035	47.530114965534\\
0.562281140570285	48.3413086118803\\
0.564282141070535	49.159263160209\\
0.566283141570785	49.9833483569457\\
0.568284142071036	50.8132777231927\\
0.570285142571286	51.6483637095958\\
0.572286143071536	52.4882052456986\\
0.574287143571786	53.3321147821468\\
0.576288144072036	54.1795766569248\\
0.578289144572286	55.0299033206785\\
0.580290145072536	55.8824645198331\\
0.582291145572786	56.7364581134756\\
0.584292146073036	57.5911965522518\\
0.586293146573287	58.4458776952484\\
0.588294147073537	59.2996421057729\\
0.590295147573787	60.1515730513529\\
0.592296148074037	61.0008683910753\\
0.594297148574287	61.8464395051294\\
0.596298149074537	62.6873696610429\\
0.598299149574787	63.5225702390051\\
0.600300150075038	64.3510099149847\\
0.602301150575288	65.1715427733916\\
0.604302151075538	65.9829656028559\\
0.606303151575788	66.7841897835668\\
0.608304152076038	67.5738975125956\\
0.610305152576288	68.3508855785725\\
0.612306153076538	69.1138361785687\\
0.614307153576788	69.8614315096554\\
0.616308154077039	70.5923537689038\\
0.618309154577289	71.3052278576056\\
0.620310155077539	71.9986786770524\\
0.622311155577789	72.6713884243155\\
0.624312156078039	73.3218674091275\\
0.626313156578289	73.9488551243392\\
0.628314157078539	74.5509191754627\\
0.630315157578789	75.1267417595692\\
0.63231615807904	75.6749477779503\\
0.63431715857929	76.1942194276774\\
0.63631815907954	76.6834107931601\\
0.63831915957979	77.1411467756901\\
0.64032016008004	77.5663960512362\\
0.64232116058029	77.9579554084286\\
0.64432216108054	78.3148508190156\\
0.64632316158079	78.6360509589659\\
0.64832416208104	78.9207536873664\\
0.650325162581291	79.1681568633039\\
0.652326163081541	79.3775156416447\\
0.654327163581791	79.5482570645937\\
0.656328164082041	79.6798654701353\\
0.658329164582291	79.7720543793718\\
0.660330165082541	79.8244227218468\\
0.662331165582791	79.8369132017806\\
0.664332166083042	79.8094685233939\\
0.666333166583292	79.742145982466\\
0.668334167083542	79.6351747621151\\
0.670335167583792	79.4888413412387\\
0.672336168084042	79.3036040860729\\
0.674337168584292	79.0799213628538\\
0.676338169084542	78.8184807209356\\
0.678339169584792	78.520027005452\\
0.680340170085043	78.185362357316\\
0.682341170585293	77.8153462132205\\
0.684342171085543	77.4110671929762\\
0.686343171585793	76.9734993248348\\
0.688344172086043	76.5038458201661\\
0.690345172586293	76.0032525945603\\
0.692346173086543	75.4729801551667\\
0.694347173586793	74.9142890091346\\
0.696348174087044	74.3284969593929\\
0.698349174587294	73.7169791046498\\
0.700350175087544	73.0809959520546\\
0.702351175587794	72.4220371918746\\
0.704352176088044	71.7414206270387\\
0.706353176588294	71.0405213562552\\
0.708354177088544	70.3207144782323\\
0.710355177588794	69.583432387458\\
0.712356178089045	68.8299355910814\\
0.714357178589295	68.0615418920315\\
0.716358179089545	67.2796836847959\\
0.718359179589795	66.4855068849651\\
0.720360180090045	65.6803292954678\\
0.722361180590295	64.8653541276737\\
0.724362181090545	64.0417272971731\\
0.726363181590795	63.2105947195563\\
0.728364182091045	62.3731023104136\\
0.730365182591296	61.5303386895557\\
0.732366183091546	60.6832778852343\\
0.734367183591796	59.8328939257011\\
0.736368184092046	58.9801608392079\\
0.738369184592296	58.1259380624474\\
0.740370185092546	57.2711423278917\\
0.742371185592796	56.4166330722336\\
0.744372186093047	55.5630405490477\\
0.746373186593297	54.7112814908062\\
0.748374187093547	53.8619288553043\\
0.750375187593797	53.0157274876756\\
0.752376188094047	52.1732503457152\\
0.754377188594297	51.3351276829978\\
0.756378189094547	50.5019324573186\\
0.758379189594797	49.6741230349136\\
0.760380190095048	48.8522723735779\\
0.762381190595298	48.0367815437682\\
0.764382191095548	47.2281662075001\\
0.766383191595798	46.4267701394506\\
0.768384192096048	45.6329944100764\\
0.770385192596298	44.8471827940544\\
0.772386193096548	44.0696790660619\\
0.774387193596798	43.3007697049963\\
0.776388194097049	42.5408557813143\\
0.778389194597299	41.7900518865749\\
0.780390195097549	41.0487017954551\\
0.782391195597799	40.3170346910731\\
0.784392196098049	39.5952797565468\\
0.786393196598299	38.8836088792148\\
0.788394197098549	38.1822512421951\\
0.790395197598799	37.4913787328264\\
0.792396198099049	36.8112205342266\\
0.7943971985993	36.1418339421752\\
0.79639819909955	35.4834481397904\\
0.7983991995998	34.8361777186311\\
0.80040020010005	34.2002518618154\\
0.8024012006003	33.5757278651228\\
0.80440220110055	32.9627776158919\\
0.8064032016008	32.3615730014611\\
0.808404202101051	31.77217131761\\
0.810405202601301	31.1948017474567\\
0.812406203101551	30.6295788825601\\
0.814407203601801	30.0766173144794\\
0.816408204102051	29.536088930553\\
0.818409204602301	29.0081083223399\\
0.820410205102551	28.4928473771788\\
0.822411205602801	27.9905352781876\\
0.824412206103052	27.5012866169254\\
0.826413206603302	27.0252732807307\\
0.828414207103552	26.562667156942\\
0.830415207603802	26.1136974286775\\
0.832416208104052	25.6785932790552\\
0.834417208604302	25.257469299634\\
0.836418209104552	24.8506119693116\\
0.838419209604802	24.458250471206\\
0.840420210105053	24.0805566926558\\
0.842421210605303	23.717874408338\\
0.844422211105553	23.3703755055911\\
0.846423211605803	23.0383464633128\\
0.848424212106053	22.7220737604006\\
0.850425212606303	22.4218438757521\\
0.852426213106553	22.1378286967057\\
0.854427213606803	21.8703719979387\\
0.856428214107053	21.6197602583484\\
0.858429214607304	21.3862226610531\\
0.860430215107554	21.1700456849502\\
0.862431215607804	20.9714585131579\\
0.864432216108054	20.7907476245736\\
0.866433216608304	20.6281422023155\\
0.868434217108554	20.4838141337221\\
0.870435217608804	20.3579926019113\\
0.872436218109054	20.2509067900014\\
0.874437218609305	20.1626139937717\\
0.876438219109555	20.0932861005609\\
0.878439219609805	20.0430377019279\\
0.880440220110055	20.0119833894318\\
0.882441220610305	20.0001231630726\\
0.884442221110555	20.0075143186298\\
0.886443221610805	20.0340422645444\\
0.888444222111056	20.0798215923753\\
0.890445222611306	20.1446804147841\\
0.892446223111556	20.2285041402118\\
0.894447223611806	20.3312354728787\\
0.896448224112056	20.4527025254465\\
0.898449224612306	20.5926761147969\\
0.900450225112556	20.7510416493711\\
0.902451225612806	20.9275699460509\\
0.904452226113057	21.1219745259388\\
0.906453226613307	21.3340262059167\\
0.908454227113557	21.5635530986461\\
0.910455227613807	21.8102114294499\\
0.912456228114057	22.0737147194306\\
0.914457228614307	22.35383378547\\
0.916458229114557	22.6503394444502\\
0.918459229614807	22.9628879216941\\
0.920460230115058	23.2912500340836\\
0.922461230615308	23.6351393027211\\
0.924462231115558	23.9942692487091\\
0.926463231615808	24.368467984709\\
0.928464232116058	24.7574490318234\\
0.930465232616308	25.1609259111545\\
0.932466233116558	25.5787267353639\\
0.934467233616808	26.010565025554\\
0.936468234117058	26.4563261901658\\
0.938469234617309	26.9157237503017\\
0.940470235117559	27.3886431144026\\
0.942471235617809	27.8747978035711\\
0.944472236118059	28.3740732262481\\
0.946473236618309	28.8862974950951\\
0.948474237118559	29.4112987227735\\
0.950475237618809	29.9489050219447\\
0.95247623811906	30.4990018010498\\
0.95447723861931	31.0613598769707\\
0.95647823911956	31.6359219539279\\
0.95847923961981	32.2225161445829\\
0.96048024012006	32.820970561597\\
0.96248124062031	33.4311706134113\\
0.96448224112056	34.0530017084668\\
0.96648324162081	34.6862919594249\\
0.968484242121061	35.3308694789471\\
0.970485242621311	35.9866769712538\\
0.972486243121561	36.6534852532271\\
0.974487243621811	37.3311797333078\\
0.976488244122061	38.0195885241575\\
0.978489244622311	38.7184824426581\\
0.980490245122561	39.4277468972505\\
0.982491245622811	40.1471527048168\\
0.984492246123062	40.8765279780183\\
0.986493246623312	41.6156435337371\\
0.988494247123562	42.3642128930755\\
0.990495247623812	43.122064168695\\
0.992496248124062	43.8888535859186\\
0.994497248624312	44.6643519616282\\
0.996498249124562	45.4482728169262\\
0.998499249624812	46.240215081356\\
1.00050025012506	47.0398349802406\\
1.00250125062531	47.8468460346824\\
1.00450225112556	48.6607325826657\\
1.00650325162581	49.4811508495135\\
1.00850425212606	50.3075851732102\\
1.01050525262631	51.1395771875197\\
1.01250625312656	51.9766112304263\\
1.01450725362681	52.8181143441349\\
1.01650825412706	53.6635135708505\\
1.01850925462731	54.5121786569982\\
1.02051025512756	55.3634220532241\\
1.02251125562781	56.2166708017329\\
1.02451225612806	57.0710081700525\\
1.02651325662831	57.9258612003877\\
1.02851425712856	58.7802558644868\\
1.03051525762881	59.6333900214366\\
1.03251625812906	60.4844042345444\\
1.03451725862931	61.3323244755585\\
1.03651825912956	62.1762340120067\\
1.03851925962981	63.0150442240782\\
1.04052026013006	63.8477237877418\\
1.04252126063032	64.6732413789663\\
1.04452226113057	65.4903364906024\\
1.04652326163082	66.2979205028393\\
1.04852426213107	67.0947902043072\\
1.05052526263132	67.8796850878569\\
1.05252626313157	68.6512873505596\\
1.05452726363182	69.4083937810455\\
1.05652826413207	70.1496292806062\\
1.05852926463232	70.8736187505335\\
1.06053026513257	71.5791016836781\\
1.06253126563282	72.2645883897726\\
1.06453226613307	72.9287610658883\\
1.06653326663332	73.5701873175372\\
1.06853426713357	74.1876066375702\\
1.07053526763382	74.7795293357199\\
1.07253626813407	75.3446949048369\\
1.07453726863432	75.8817282462131\\
1.07653826913457	76.389368852699\\
1.07853926963482	76.8664135129249\\
1.08054027013507	77.311544423962\\
1.08254127063532	77.7237302617791\\
1.08454227113557	78.101825110786\\
1.08654327163582	78.4448549427308\\
1.08854427213607	78.7519030251414\\
1.09054527263632	79.0221099213251\\
1.09254627313657	79.2546734903687\\
1.09454727363682	79.4489634786976\\
1.09654827413707	79.6044642242961\\
1.09854927463732	79.7206600651486\\
1.10055027513757	79.7972645223576\\
1.10255127563782	79.8341057085845\\
1.10455227613807	79.8309544407113\\
1.10655327663832	79.7878680145174\\
1.10855427713857	79.7050183173415\\
1.11055527763882	79.5825772365221\\
1.11255627813907	79.4209458425156\\
1.11455727863932	79.2205825015584\\
1.11655827913957	78.9820601714455\\
1.11855927963982	78.7060091057514\\
1.12056028014007	78.393288741169\\
1.12256128064032	78.0446439228319\\
1.12456228114057	77.6611059747713\\
1.12656328164082	77.243648925239\\
1.12856428214107	76.7934186898252\\
1.13056528264132	76.3115038883407\\
1.13256628314157	75.7991650279347\\
1.13456728364182	75.257605319977\\
1.13656828414207	74.6881425673965\\
1.13856928464232	74.0920945731219\\
1.14057028514257	73.4708364358616\\
1.14257128564282	72.8256859585443\\
1.14457228614307	72.1580755356578\\
1.14657328664332	71.4693802659106\\
1.14857428714357	70.7609179522313\\
1.15057528764382	70.0341209891079\\
1.15257628814407	69.290307179469\\
1.15457728864432	68.5308516220231\\
1.15657828914457	67.7570148239194\\
1.15857928964482	66.9701718838663\\
1.16058029014507	66.1715260132334\\
1.16258129064532	65.3623950149497\\
1.16458229114557	64.5438675088258\\
1.16658329164582	63.717204002011\\
1.16858429214607	62.8834358185367\\
1.17058529264632	62.0437661697724\\
1.17258629314657	61.1991690839701\\
1.17458729364682	60.3506758851608\\
1.17658829414707	59.499317897376\\
1.17858929464732	58.6458972615286\\
1.18059029514757	57.791330710091\\
1.18259129564782	56.9365922713148\\
1.18459229614807	56.0823121987748\\
1.18659329664832	55.2293499291635\\
1.18859429714857	54.3784503076147\\
1.19059529764882	53.530243587703\\
1.19259629814907	52.6854173187826\\
1.19459729864932	51.8446017544282\\
1.19659829914957	51.0083125566552\\
1.19859929964982	50.1771799790384\\
1.20060030015008	49.3516623878139\\
1.20260130065033	48.5322754449974\\
1.20460230115058	47.7194202210453\\
1.20660330165083	46.9135550821937\\
1.20860430215108	46.1150810988994\\
1.21060530265133	45.3243993416189\\
1.21260630315158	44.5417962892497\\
1.21460730365183	43.7676157164689\\
1.21660830415208	43.0022013979537\\
1.21860930465233	42.2458398126015\\
1.22061030515258	41.4987028477509\\
1.22261130565283	40.761134278079\\
1.22461230615308	40.0333059909243\\
1.22661330665333	39.3154471694049\\
1.22861430715358	38.6078442924183\\
1.23061530765383	37.9105546557441\\
1.23261630815408	37.2238647382798\\
1.23461730865433	36.5478318358049\\
1.23661830915458	35.8827424272171\\
1.23861930965483	35.2287111040753\\
1.24062031015508	34.585909753718\\
1.24262131065533	33.9543956719248\\
1.24462231115558	33.3343980418137\\
1.24662331165583	32.7260314549438\\
1.24862431215608	32.1294105028741\\
1.25062531265633	31.5447070729431\\
1.25262631315658	30.9720930524893\\
1.25462731365683	30.4116257372924\\
1.25662831415708	29.8635343104702\\
1.25862931465733	29.3278760678024\\
1.26063031515758	28.804880192407\\
1.26263131565783	28.294661275843\\
1.26463231615808	27.7974485012285\\
1.26663331665833	27.3133564601224\\
1.26863431715858	26.8425570398635\\
1.27063531765883	26.3852794235695\\
1.27263631815908	25.9416954985793\\
1.27463731865933	25.5120344480107\\
1.27663831915958	25.0964681592023\\
1.27863931965983	24.6952831110517\\
1.28064032016008	24.3085938951179\\
1.28264132066033	23.9368015818575\\
1.28464232116058	23.5800207628295\\
1.28664332166083	23.2385379169315\\
1.28864432216108	22.9126395230611\\
1.29064532266133	22.6026120601158\\
1.29264632316158	22.3086847112137\\
1.29464732366183	22.0311439552524\\
1.29664832416208	21.7702762711293\\
1.29864932466233	21.5263108419626\\
1.30065032516258	21.2995341466498\\
1.30265132566283	21.090175368309\\
1.30465232616308	20.8985782816173\\
1.30665332666333	20.7249147739131\\
1.30865432716358	20.5694713240941\\
1.31065532766383	20.4323625237193\\
1.31265632816408	20.3138175559067\\
1.31465732866433	20.2140656037745\\
1.31665832916458	20.1332212588815\\
1.31865932966483	20.0713418170074\\
1.32066033016508	20.0285991654906\\
1.32266133066533	20.0050506001107\\
1.32466233116558	20.0006961208678\\
1.32666333166583	20.0155930235412\\
1.32866433216608	20.0496840123514\\
1.33066533266633	20.1029117915191\\
1.33266633316658	20.1752190652646\\
1.33466733366683	20.2664912420289\\
1.33666833416708	20.3765564344736\\
1.33866933466733	20.5053000510395\\
1.34067033516758	20.6524929086086\\
1.34267133566783	20.8180204156219\\
1.34467233616808	21.0015387974023\\
1.34667333666833	21.2028761666112\\
1.34867433716858	21.4218033401307\\
1.35067533766883	21.6580911348427\\
1.35267633816908	21.9113384802905\\
1.35467733866933	22.1814307849152\\
1.35667833916958	22.4679669782601\\
1.35867933966983	22.7707751729867\\
1.36068034017009	23.0895688901975\\
1.36268134067034	23.4240043552154\\
1.36468234117059	23.7739096807018\\
1.36668334167084	24.1389983877591\\
1.36868434217109	24.5189839974899\\
1.37068534267134	24.913637326776\\
1.37268634317159	25.3227864882789\\
1.37468734367184	25.7460877073216\\
1.37668834417209	26.183426392345\\
1.37868934467234	26.634573360231\\
1.38069034517259	27.0992994278616\\
1.38269134567284	27.5773754121187\\
1.38469234617309	28.0687440172229\\
1.38669334667334	28.5730614684971\\
1.38869434717359	29.0903277659412\\
1.39069534767384	29.6202564306577\\
1.39269634817409	30.1627901668671\\
1.39469734867434	30.7176997914513\\
1.39669834917459	31.2848707128513\\
1.39869934967484	31.8641310437285\\
1.40070035017509	32.455366192524\\
1.40270135067534	33.0584615676787\\
1.40470235117559	33.6732452818541\\
1.40670335167584	34.2995454477116\\
1.40870435217609	34.9373047694717\\
1.41070535267634	35.5862940640164\\
1.41270635317659	36.2463987397866\\
1.41470735367684	36.9175042052234\\
1.41670835417709	37.5993812772086\\
1.41870935467734	38.2919153641832\\
1.42071035517759	38.9949345788087\\
1.42271135567784	39.7081524421875\\
1.42471235617809	40.4315116585402\\
1.42671335667834	41.1647257489691\\
1.42871435717859	41.9075655303562\\
1.43071535767884	42.6598018195835\\
1.43271635817909	43.4212054335328\\
1.43471735867934	44.1914325975272\\
1.43671835917959	44.9702541284485\\
1.43871935967984	45.7573262516198\\
1.44072036018009	46.5523624881433\\
1.44272136068034	47.354904471783\\
1.44472236118059	48.1646657236414\\
1.44672336168084	48.9811305817028\\
1.44872436218109	49.8039552712902\\
1.45072536268134	50.6326241303879\\
1.45272636318159	51.4666787927599\\
1.45472736368184	52.3054890048314\\
1.45672836418209	53.1485964003664\\
1.45872936468234	53.9953707257902\\
1.46073036518259	54.8451244317488\\
1.46273136568284	55.6972272646673\\
1.46473236618309	56.5509916751918\\
1.46673336668334	57.4056728181884\\
1.46873436718359	58.2604112569646\\
1.47073536768384	59.1144048506071\\
1.47273636818409	59.9668514582027\\
1.47473736868434	60.8167770514998\\
1.47673836918459	61.6632648980261\\
1.47873936968484	62.5052836737503\\
1.48074037018509	63.3418020546413\\
1.48274137068534	64.1717887166678\\
1.48474237118559	64.9941550400191\\
1.48674337168584	65.8076978133253\\
1.48874437218609	66.6112138252168\\
1.49074537268634	67.4035571601032\\
1.49274637318659	68.1834100150558\\
1.49474737368684	68.9495118829252\\
1.49674837418709	69.7005449607827\\
1.49874937468734	70.4351914456994\\
1.50075037518759	71.1520762389671\\
1.50275137568784	71.8498815376569\\
1.50475237618809	72.5271749472811\\
1.50675337668834	73.1826386649108\\
1.50875437718859	73.8148402960581\\
1.51075537768884	74.4224047420148\\
1.51275637818909	75.0040141998521\\
1.51475737868934	75.5583508666412\\
1.51675837918959	76.0840396436737\\
1.51875937968984	76.5798200238005\\
1.5207603801901	77.0444887956515\\
1.52276138069035	77.4769000436368\\
1.5247623811906	77.8758505563864\\
1.52676338169085	78.2403663056486\\
1.5287643821911	78.5694159673922\\
1.53076538269135	78.8620828091451\\
1.5327663831916	79.1176219857734\\
1.53476738369185	79.3352886521436\\
1.5367683841921	79.5145098504605\\
1.53876938469235	79.6546553271495\\
1.5407703851926	79.7553813075335\\
1.54277138569285	79.8164586084945\\
1.5447723861931	79.8376007511348\\
1.54677338669335	79.8188077354545\\
1.5487743871936	79.7601368572331\\
1.55077538769385	79.6617600038091\\
1.5527763881941	79.5239636540802\\
1.55477738869435	79.3470915827233\\
1.5567783891946	79.1317167475336\\
1.55877938969485	78.8784121063063\\
1.5607803901951	78.587922504175\\
1.56278139069535	78.2610500820528\\
1.5647823911956	77.8985969808531\\
1.56678339169585	77.5016518203864\\
1.5687843921961	77.0712459246841\\
1.57078539269635	76.608467913557\\
1.5727863931966	76.1145209983747\\
1.57478739369685	75.5905510947276\\
1.5767883941971	75.0379333013239\\
1.57878939469735	74.4579281253129\\
1.5807903951976	73.8518533696236\\
1.58279139569785	73.221084132964\\
1.5847923961981	72.5669382182632\\
1.58679339669835	71.8909053157883\\
1.5887943971986	71.1943032284683\\
1.59079539769885	70.4785070550113\\
1.5927963981991	69.7448918941258\\
1.59479739869935	68.9947755487405\\
1.5967983991996	68.2295331175638\\
1.59879939969985	67.4504824035244\\
1.6008004002001	66.6588839137717\\
1.60280140070035	65.8559981554549\\
1.6048024012006	65.0430283399438\\
1.60680340170085	64.2211776786081\\
1.6088044022011	63.3915920870382\\
1.61080540270135	62.5554174808243\\
1.6128064032016	61.7136851839976\\
1.61480740370185	60.8674265205893\\
1.6168084042021	60.0176728146308\\
1.61880940470235	59.1653980943737\\
1.6208104052026	58.3114617965107\\
1.62281140570285	57.4567233577346\\
1.6248124062031	56.6020422147379\\
1.62681340670335	55.748220508434\\
1.6288144072036	54.8960030839564\\
1.63081540770385	54.0460774906593\\
1.6328164082041	53.1990739821174\\
1.63481740870435	52.3557946992439\\
1.6368184092046	51.5166407124953\\
1.63881940970485	50.6822995712258\\
1.6408204102051	49.8533442332305\\
1.64282141070535	49.0301184731865\\
1.6448224112056	48.2132525446685\\
1.64682341170585	47.4030902223535\\
1.6488244122061	46.6000898724777\\
1.65082541270635	45.804595269718\\
1.6528264132066	45.0170647803107\\
1.65482741370685	44.2377275873738\\
1.6568284142071	43.4669274655843\\
1.65882941470735	42.7050081896193\\
1.6608304152076	41.9522562383764\\
1.66283141570785	41.2088434991942\\
1.6648324162081	40.4750564509702\\
1.66683341670835	39.7511242768224\\
1.6688344172086	39.0372188640894\\
1.67083541770885	38.3336266916687\\
1.6728364182091	37.6404623511194\\
1.67483741870935	36.9579550255596\\
1.6768384192096	36.2862193065482\\
1.67883941970985	35.6254270814238\\
1.68084042021011	34.975750237525\\
1.68284142071036	34.3373033664107\\
1.68484242121061	33.7103156511991\\
1.68684342171086	33.0948443876695\\
1.68884442221111	32.4910614631607\\
1.69084542271136	31.8990814692315\\
1.69284642321161	31.3191335890001\\
1.69484742371186	30.7512178224664\\
1.69684842421211	30.1955633527485\\
1.69884942471236	29.652342067185\\
1.70085042521261	29.1216112615553\\
1.70285142571286	28.6036001189775\\
1.70485242621311	28.0984805267902\\
1.70685342671336	27.6063670765523\\
1.70885442721361	27.127488951382\\
1.71085542771386	26.6619607428382\\
1.71285642821411	26.210011634039\\
1.71485742871436	25.771813512323\\
1.71685842921461	25.3476528565876\\
1.71885942971486	24.9377015541715\\
1.72086043021511	24.5421314924132\\
1.72286143071536	24.1612864459897\\
1.72486243121561	23.7952810064602\\
1.72686343171586	23.4445162442811\\
1.72886443221611	23.1091067510115\\
1.73086543271636	22.7893963013285\\
1.73286643321661	22.4856140783501\\
1.73486743371686	22.198046560974\\
1.73686843421711	21.9270375238771\\
1.73886943471736	21.6727015586185\\
1.74087043521761	21.4354397356548\\
1.74287143571786	21.2154812381041\\
1.74487243621811	21.0130552490844\\
1.74687343671836	20.8284482474933\\
1.74887443721861	20.6618894164487\\
1.75087543771886	20.5136079390689\\
1.75287643821911	20.3837184069127\\
1.75487743871936	20.2725645946573\\
1.75687843921961	20.1801465023027\\
1.75887943971986	20.106693312967\\
1.76088044022011	20.052319618209\\
1.76288144072036	20.0170827138085\\
1.76488244122061	20.0010971913243\\
1.76688344172086	20.0042484591976\\
1.76888444222111	20.0266511089872\\
1.77088544272136	20.0682478449137\\
1.77288644322161	20.1289813711975\\
1.77488744372186	20.2086798005002\\
1.77688844422211	20.3073431328218\\
1.77888944472236	20.4247421850441\\
1.78089044522261	20.5607623656081\\
1.78289144572286	20.7151171956164\\
1.78489244622311	20.8876920835098\\
1.78689344672336	21.0782578461703\\
1.78889444722361	21.2865280047003\\
1.79089544772386	21.5122733759819\\
1.79289644822411	21.7552074811174\\
1.79489744872436	22.0151011369887\\
1.79689844922461	22.2916678646983\\
1.79889944972486	22.5846211853487\\
1.80090045022511	22.8936746200423\\
1.80290145072536	23.2186562814405\\
1.80490245122561	23.5591650990868\\
1.80690345172586	23.9150291856425\\
1.80890445222611	24.2859620622102\\
1.81090545272636	24.6717345456718\\
1.81290645322661	25.0721174529092\\
1.81490745372686	25.4868243050249\\
1.81690845422711	25.9156259189008\\
1.81890945472736	26.3584077029779\\
1.82091045522761	26.8148831783586\\
1.82291145572786	27.2848804577044\\
1.82491245622811	27.7681703578973\\
1.82691345672836	28.2646382873782\\
1.82891445722861	28.774055063029\\
1.83091545772886	29.2962487975112\\
1.83291645822911	29.8311621950453\\
1.83491745872936	30.3785087767338\\
1.83691845922961	30.9382312467971\\
1.83891945972987	31.5101577178967\\
1.84092046023012	32.0941735984736\\
1.84292146073037	32.6900497054096\\
1.84492246123062	33.2977287429254\\
1.84692346173087	33.9170388236823\\
1.84892446223112	34.5478653561213\\
1.85092546273137	35.1900364529039\\
1.85292646323162	35.8433802266916\\
1.85492746373187	36.5078393817048\\
1.85692846423212	37.1831847348255\\
1.85892946473237	37.8692443987152\\
1.86093046523262	38.5659037818147\\
1.86293146573287	39.2729337010062\\
1.86493246623312	39.9901622689509\\
1.86693346673337	40.717360302531\\
1.86893446723362	41.4543559144078\\
1.87093546773387	42.2009199214632\\
1.87293646823412	42.9567658447998\\
1.87493746873437	43.7216645012995\\
1.87693846923462	44.4953294120646\\
1.87893946973487	45.2774168024182\\
1.88094047023512	46.0676401934626\\
1.88294147073537	46.8656558105208\\
1.88494247123562	47.6710625831362\\
1.88694347173587	48.4835167366317\\
1.88894447223612	49.3025599047712\\
1.89094547273637	50.1276764255391\\
1.89294647323662	50.9585225242583\\
1.89494747373687	51.7945252431337\\
1.89694847423712	52.6351116243702\\
1.89894947473737	53.4797087101725\\
1.90095047523762	54.3276862469661\\
1.90295147573787	55.1784712769559\\
1.90495247623812	56.0313189550081\\
1.90695347673837	56.8854844359891\\
1.90895447723862	57.7402801705448\\
1.91095547773887	58.5948467219824\\
1.91295647823912	59.4483819493888\\
1.91495747873937	60.2999118245123\\
1.91695847923962	61.14857691066\\
1.91895947973987	61.9934604753599\\
1.92096048024012	62.8334738988013\\
1.92296148074037	63.6675858569527\\
1.92496248124062	64.4947077300036\\
1.92696348174087	65.3137508981431\\
1.92896448224112	66.1235121500015\\
1.93096548274137	66.922788274209\\
1.93296648324162	67.7103760593958\\
1.93496748374187	68.4850149984127\\
1.93696848424212	69.2453299925513\\
1.93896948474237	69.9901178304418\\
1.94097048524262	70.718003413376\\
1.94297148574287	71.4276116426455\\
1.94497248624312	72.1175101237626\\
1.94697348674337	72.7864383495778\\
1.94897448724362	73.4329639256034\\
1.95097548774387	74.0557117531311\\
1.95297648824412	74.6532494376731\\
1.95497748874437	75.2243737678594\\
1.95697848924462	75.767595053423\\
1.95897948974487	76.2817673787734\\
1.96098049024512	76.7655156452023\\
1.96298149074537	77.2176939371196\\
1.96498249124562	77.6370990431554\\
1.96698349174587	78.0226996392784\\
1.96898449224612	78.3734644014575\\
1.97098549274637	78.6883620056614\\
1.97298649324662	78.9665903109769\\
1.97498749374687	79.2074044722704\\
1.97698849424712	79.4100596444082\\
1.97898949474737	79.5740401653746\\
1.98099049524762	79.6987730773746\\
1.98299149574787	79.7840291972901\\
1.98499249624812	79.8294647504439\\
1.98699349674837	79.8350224410567\\
1.98899449724862	79.8005876775694\\
1.99099549774887	79.7263896430999\\
1.99299649824912	79.6124856334279\\
1.99499749874937	79.4593340147894\\
1.99699849924962	79.2673358576411\\
1.99899949974988	79.037064119778\\
2.00100050025013	78.7690917589953\\
2.00300150075038	78.4642209162062\\
2.00500250125063	78.1233110281034\\
2.00700350175088	77.747278827159\\
2.00900450225113	77.3370983416249\\
2.01100550275138	76.8938581913117\\
2.01300650325163	76.4187615875892\\
2.01500750375188	75.9128971502682\\
2.01700850425213	75.3775826822774\\
2.01900950475238	74.8140786907663\\
2.02101050525263	74.2237029786635\\
2.02301150575288	73.6077733488978\\
2.02501250625313	72.9677221959572\\
2.02701350675338	72.3048673227703\\
2.02901450725363	71.6205838280456\\
2.03101550775388	70.9163041062708\\
2.03301650825413	70.1933459603747\\
2.03501750875438	69.4531417848452\\
2.03701850925463	68.6969520868315\\
2.03901950975488	67.9261519650421\\
2.04102051025513	67.142001926626\\
2.04302151075538	66.3458197745122\\
2.04502251125563	65.53886601585\\
2.04702351175588	64.7222865662295\\
2.04902451225613	63.8972846370206\\
2.05102551275638	63.0650061438136\\
2.05302651325663	62.2264824106396\\
2.05502751375688	61.382859353089\\
2.05702851425713	60.535168295193\\
2.05902951475738	59.6843259694237\\
2.06103051525763	58.8312491082534\\
2.06303151575788	57.9769117399338\\
2.06503251625813	57.1221160053782\\
2.06703351675838	56.2676640454996\\
2.06903451725863	55.4143580012112\\
2.07103551775888	54.5629427176468\\
2.07303651825913	53.7141057441605\\
2.07503751875938	52.8684773343269\\
2.07703851925963	52.0267450375002\\
2.07903951975988	51.1894245156961\\
2.08104052026013	50.357146022489\\
2.08304152076038	49.5303679241153\\
2.08504252126063	48.7095485868108\\
2.08704352176088	47.8952609683709\\
2.08904452226113	47.0878488434725\\
2.09104552276138	46.2877705783519\\
2.09304652326163	45.4953699476859\\
2.09504752376188	44.7109907261518\\
2.09704852426213	43.9349766884267\\
2.09904952476238	43.1676143134079\\
2.10105052526263	42.4092473757728\\
2.10305152576288	41.6601050586392\\
2.10505252626313	40.9204165451253\\
2.10705352676338	40.1904683141287\\
2.10905452726363	39.4704322529878\\
2.11105552776388	38.7605375448207\\
2.11305652826413	38.0610133727455\\
2.11505752876438	37.3719743283211\\
2.11705852926463	36.6936495946657\\
2.11905952976488	36.0262110591179\\
2.12106053026513	35.3697733132365\\
2.12306153076538	34.7244509485806\\
2.12506253126563	34.0905304440479\\
2.12706353176588	33.4680117996382\\
2.12906453226613	32.8570669026902\\
2.13106553276638	32.2579249363219\\
2.13306653326663	31.6706431963128\\
2.13506753376688	31.095336274222\\
2.13706853426713	30.5322333531674\\
2.13906953476738	29.9814490247082\\
2.14107053526763	29.4430978804032\\
2.14307153576788	28.9173518075912\\
2.14507253626813	28.404325397831\\
2.14707353676838	27.9042478342409\\
2.14907453726863	27.4172910041592\\
2.15107553776888	26.9436267949245\\
2.15307653826913	26.4833697980959\\
2.15507753876938	26.036806492571\\
2.15707853926963	25.6041087656882\\
2.15907953976988	25.1854485047861\\
2.16108054027013	24.7811121887623\\
2.16308154077039	24.3912717049552\\
2.16508254127064	24.0162135322626\\
2.16708354177089	23.6561668538024\\
2.16908454227114	23.3113608526927\\
2.17108554277139	22.982082007831\\
2.17308654327164	22.6685595023354\\
2.17508754377189	22.371137110883\\
2.17708854427214	22.0899867208123\\
2.17908954477239	21.8254521068004\\
2.18109054527264	21.5778197477449\\
2.18309154577289	21.3472615309842\\
2.18509254627314	21.1341212311955\\
2.18709354677339	20.9386853272764\\
2.18909454727364	20.7610684107859\\
2.19109554777389	20.6016715521805\\
2.19309654827414	20.4605520472397\\
2.19509754877439	20.3379963748613\\
2.19709854927464	20.234119126604\\
2.19909954977489	20.1491494855861\\
2.20110055027514	20.0831447475871\\
2.20310155077539	20.0362767999454\\
2.20510255127564	20.008545642661\\
2.20710355177589	20.0000085715136\\
2.20910455227614	20.0107228822825\\
2.21110555277639	20.0406885749679\\
2.21310655327664	20.0897337622311\\
2.21510755377689	20.1579157398516\\
2.21710855427714	20.245062620491\\
2.21910955477739	20.3510598125903\\
2.22111055527764	20.4757927245902\\
2.22311155577789	20.6189748775934\\
2.22511255627814	20.7805489758203\\
2.22711355677839	20.9601712445938\\
2.22911455727864	21.1576697965754\\
2.23111555777889	21.372815448647\\
2.23311655827914	21.6053217219111\\
2.23511755877939	21.8549594332496\\
2.23711855927964	22.1213848079855\\
2.23911955977989	22.4043686630006\\
2.24112056028014	22.7036818151769\\
2.24312156078039	23.0190377856169\\
2.24512256128064	23.3500927996435\\
2.24712356178089	23.6966749699182\\
2.24912456228114	24.0584978175433\\
2.25112556278139	24.4352748636213\\
2.25312656328164	24.8267769250342\\
2.25512756378189	25.2327748186639\\
2.25712856428214	25.6530393613924\\
2.25912956478239	26.0873413701015\\
2.26113056528264	26.5354516616733\\
2.26313156578289	26.9972556445488\\
2.26513256628314	27.4724668398303\\
2.26713356678339	27.9609706559588\\
2.26913456728364	28.4624806140368\\
2.27113556778389	28.9769394182848\\
2.27313656828414	29.5041751813642\\
2.27513756878439	30.0439587201569\\
2.27713856928464	30.596175443104\\
2.27913956978489	31.1607107586464\\
2.28114057028514	31.7373354836661\\
2.28314157078539	32.3259923223835\\
2.28514257128564	32.9265666832396\\
2.28714357178589	33.5387720873369\\
2.28914457228614	34.1626085346753\\
2.29114557278639	34.7979041379164\\
2.29314657328664	35.4444870097215\\
2.29514757378689	36.1021852627522\\
2.29714857428714	36.7708843054494\\
2.29914957478739	37.4504695462541\\
2.30115057528764	38.1407118020482\\
2.30315157578789	38.8414391854932\\
2.30515257628814	39.5524798092505\\
2.30715357678839	40.2736617859817\\
2.30915457728864	41.0047559325686\\
2.31115557778889	41.7454757701137\\
2.31315657828914	42.495706707058\\
2.31515757878939	43.2551622645039\\
2.31715857928964	44.0234986677744\\
2.31915957978989	44.8004867337513\\
2.32116058029015	45.5857826877576\\
2.3231615807904	46.3791000508957\\
2.32516258129065	47.1800950484886\\
2.3271635817909	47.9883093143002\\
2.32916458229115	48.8033417778737\\
2.3311655827914	49.6248486645323\\
2.33316658329165	50.4523143122603\\
2.3351675837919	51.2852230590419\\
2.33716858429215	52.1230592428617\\
2.3391695847924	52.9652499059245\\
2.34117058529265	53.8112793862147\\
2.3431715857929	54.6604601343781\\
2.34517258629315	55.5121046010606\\
2.3471735867934	56.3655825326874\\
2.34917458729365	57.2200917883455\\
2.3511755877939	58.0748875229012\\
2.35317658829415	58.9291675954413\\
2.3551775887944	59.7820152734935\\
2.35717858929465	60.6325711203652\\
2.3591795897949	61.4798611078047\\
2.36118059029515	62.3229112075602\\
2.3631815907954	63.16074739138\\
2.36518259129565	63.9922810394533\\
2.3671835917959	64.81636623619\\
2.36918459229615	65.6319143617792\\
2.3711855927964	66.4377222048512\\
2.37318659329665	67.2325865540362\\
2.3751875937969	68.0152469021849\\
2.37718859429715	68.784442742148\\
2.3791895947974	69.5388562709968\\
2.38119059529765	70.2771696858024\\
2.3831915957979	70.998065183636\\
2.38519259629815	71.7001103700098\\
2.3871935967984	72.381987441995\\
2.38919459729865	73.0422640051037\\
2.3911955977989	73.6796222564072\\
2.39319659829915	74.2926870971972\\
2.3951975987994	74.8800834287653\\
2.39719859929965	75.4403788566238\\
2.3991995997999	75.9724274651822\\
2.40120060030015	76.474796859953\\
2.4032016008004	76.9463411253456\\
2.40520260130065	77.3858570499905\\
2.4072036018009	77.7921414225177\\
2.40920460230115	78.1642202146757\\
2.4112056028014	78.5010621024331\\
2.41320660330165	78.8016930575383\\
2.4152076038019	79.0653682348575\\
2.41720860430215	79.2913427892571\\
2.4192096048024	79.4788718756034\\
2.42121060530265	79.6275544234398\\
2.4232116058029	79.7368747707508\\
2.42521260630315	79.8065464386387\\
2.4272136068034	79.836397539765\\
2.42921460730365	79.8263134825707\\
2.4312156078039	79.7762942670558\\
2.43321660830415	79.6865117805588\\
2.4352176088044	79.5571952061978\\
2.43721860930465	79.3888029102088\\
2.4392196098049	79.1817359630485\\
2.44122061030515	78.936567322512\\
2.4432216108054	78.6540991295126\\
2.44522261130565	78.3350189334042\\
2.4472236118059	77.9802434666592\\
2.44922461230615	77.5907467575293\\
2.4512256128064	77.1675028342661\\
2.45322661330665	76.7116003166805\\
2.4552276138069	76.2242997119218\\
2.45722861430715	75.7067469355801\\
2.4592296148074	75.1602597905843\\
2.46123061530765	74.5860987840837\\
2.4632316158079	73.9855244232276\\
2.46523261630815	73.3600263982833\\
2.4672336168084	72.7108652164\\
2.46923461730865	72.0394732720657\\
2.4712356178089	71.3472256639887\\
2.47323661830915	70.6354401950977\\
2.4752376188094	69.9055492598805\\
2.47723861930965	69.1589279570455\\
2.4792396198099	68.3968940895215\\
2.48124062031015	67.6207081644578\\
2.48324162081041	66.8316879847832\\
2.48524262131066	66.0311513534264\\
2.48724362181091	65.2202441859777\\
2.48924462231116	64.4002269895865\\
2.49124562281141	63.5722456798429\\
2.49324662331166	62.7374461723373\\
2.49524762381191	61.8968024953214\\
2.49724862431216	61.0514605643854\\
2.49924962481241	60.202394407781\\
2.50125062531266	59.35057805376\\
2.50325162581291	58.4969282347946\\
2.50525262631316	57.6423043875774\\
2.50725362681341	56.7875086530218\\
2.50925462731366	55.9334577635998\\
2.51125562781391	55.0807819728861\\
2.51325662831416	54.2302834217939\\
2.51525762881441	53.3826496596773\\
2.51725862931466	52.5385109401111\\
2.51925962981491	51.6983829251108\\
2.52126063031516	50.86301045981\\
2.52326163081541	50.0327946146655\\
2.52526263131566	49.2083083474722\\
2.52726363181591	48.3900100244664\\
2.52926463231616	47.578358011884\\
2.53126563281641	46.7737533801818\\
2.53326663331666	45.9766544955958\\
2.53526763381691	45.1872905412441\\
2.53726863431716	44.4061771791422\\
2.53926963481741	43.6334862966288\\
2.54127063531766	42.8696189641604\\
2.54327163581791	42.1148616606346\\
2.54527263631816	41.3693862733898\\
2.54727363681841	40.6334792813238\\
2.54927463731866	39.9073698675545\\
2.55127563781891	39.1913445109795\\
2.55327663831916	38.4854605073784\\
2.55527763881941	37.7900043356486\\
2.55727863931966	37.1051478831287\\
2.55927963981991	36.4310630371573\\
2.56128064032016	35.7679216850729\\
2.56328164082041	35.115781122655\\
2.56528264132066	34.4749278288011\\
2.56728364182091	33.8454190992909\\
2.56928464232116	33.2274268214628\\
2.57128564282141	32.6211228826554\\
2.57328664332166	32.0265645786481\\
2.57528764382191	31.4439810925591\\
2.57728864432216	30.8734297201678\\
2.57928964482241	30.3151396445923\\
2.58129064532266	29.7691681616122\\
2.58329164582291	29.2357444543454\\
2.58529264632316	28.714983114351\\
2.58729364682341	28.2070560289675\\
2.58929464732366	27.712077789754\\
2.59129564782391	27.2302775798285\\
2.59329664832416	26.7618272865295\\
2.59529764882441	26.3069560929752\\
2.59729864932466	25.8657785907244\\
2.59929964982491	25.4385239628954\\
2.60130065032516	25.0254786883856\\
2.60330165082541	24.626757358754\\
2.60530265132566	24.2427037486778\\
2.60730365182591	23.8734897454955\\
2.60930465232616	23.5193445323252\\
2.61130565282641	23.1806118838438\\
2.61330665332666	22.85746368739\\
2.61530765382691	22.5501864218614\\
2.61730865432716	22.2591238619349\\
2.61930965482741	21.9844478949492\\
2.62131065532766	21.7265022955813\\
2.62331165582791	21.4855162469493\\
2.62531265632816	21.2617762279507\\
2.62731365682841	21.0555114217036\\
2.62931465732866	20.8670083071056\\
2.63131565782891	20.6964960672746\\
2.63331665832916	20.5442038853289\\
2.63531765882941	20.4103609443863\\
2.63731865932966	20.295081836006\\
2.63931965982992	20.1985957433059\\
2.64132066033017	20.1210172578452\\
2.64332166083042	20.0625182669624\\
2.64532266133067	20.0230987706574\\
2.64732366183092	20.0029306562688\\
2.64932466233117	20.001956628017\\
2.65132566283142	20.0201766859022\\
2.65332666333167	20.0575908299242\\
2.65532766383192	20.1141990600832\\
2.65732866433217	20.1897721932609\\
2.65932966483242	20.2843102294575\\
2.66133066533267	20.3976985771139\\
2.66333166583292	20.5296507573325\\
2.66533266633317	20.6800521785544\\
2.66733366683342	20.8487309534409\\
2.66933466733367	21.0354006030945\\
2.67133566783392	21.2398319443972\\
2.67333666833417	21.4617957942309\\
2.67533766883442	21.701005673698\\
2.67733866933467	21.9572323996805\\
2.67933966983492	22.2301894932808\\
2.68134067033517	22.5196477713809\\
2.68334167083542	22.8252634593037\\
2.68534267133567	23.1467500781516\\
2.68734367183592	23.4839357405861\\
2.68934467233617	23.8365339677096\\
2.69134567283642	24.204200984845\\
2.69334667333667	24.5867649046539\\
2.69534767383692	24.9839965440181\\
2.69734867433717	25.3956094240401\\
2.69934967483742	25.8213743616018\\
2.70135067533767	26.2611194693647\\
2.70335167583792	26.7146155642107\\
2.70535267633817	27.1816907588014\\
2.70735367683842	27.6621158700186\\
2.70935467733867	28.1557190105238\\
2.71135567783892	28.662385588758\\
2.71335667833917	29.1818291258236\\
2.71535767883942	29.7139923259411\\
2.71735867933967	30.2586460059924\\
2.71935967983992	30.8157328701981\\
2.72136068034017	31.3850237354401\\
2.72336168084042	31.9664040101594\\
2.72536268134067	32.5597018070173\\
2.72736368184092	33.164802534455\\
2.72936468234117	33.7816488966929\\
2.73136568284142	34.4099544148333\\
2.73336668334167	35.0496617930969\\
2.73536768384192	35.700599144145\\
2.73736868434217	36.3626518764187\\
2.73936968484242	37.0356481025793\\
2.74137068534267	37.7194159352885\\
2.74337168584292	38.4137834872075\\
2.74537268634317	39.1185788709979\\
2.74737368684342	39.8335729035417\\
2.74937468734367	40.5586509932797\\
2.75137568784392	41.2935839570941\\
2.75337668834417	42.038085316087\\
2.75537768884442	42.7919258871407\\
2.75737868934467	43.5548764871369\\
2.75937968984492	44.3266506371781\\
2.76138069034517	45.1069618583668\\
2.76338169084542	45.8954090802463\\
2.76538269134567	46.6917631196986\\
2.76738369184592	47.4956229062672\\
2.76938469234617	48.3065873694953\\
2.77138569284642	49.124198143147\\
2.77338669334667	49.9481114525451\\
2.77538769384692	50.7777543398946\\
2.77738869434717	51.6126111431797\\
2.77938969484742	52.4522807919439\\
2.78139069534767	53.2960184410536\\
2.78339169584792	54.1433657242725\\
2.78539269634817	54.9935777964672\\
2.78739369684842	55.8460244040628\\
2.78939469734867	56.7000179977053\\
2.79139569784892	57.5547564364815\\
2.79339669834917	58.4094375794781\\
2.79539769884942	59.2632019900025\\
2.79739869934967	60.1153048229211\\
2.79939969984992	60.964657458423\\
2.80140070035018	61.8104577555952\\
2.80340170085043	62.6515597988472\\
2.80540270135068	63.487046855707\\
2.80740370185093	64.3158303063637\\
2.80940470235118	65.1367069394476\\
2.81140570285143	65.948588135148\\
2.81340670335168	66.7502133863156\\
2.81540770385193	67.5404940731395\\
2.81740870435218	68.317997801132\\
2.81940970485243	69.0815786547029\\
2.82141070535268	69.8298615351437\\
2.82341170585293	70.5615286395258\\
2.82541270635318	71.2752048691407\\
2.82741370685343	71.9695151252803\\
2.82941470735368	72.6430843092361\\
2.83141570785393	73.2945946180793\\
2.83341670835418	73.9226136573222\\
2.83541770885443	74.5257663282564\\
2.83741870935468	75.1027348279532\\
2.83941970985493	75.6521440577041\\
2.84142071035518	76.1726762145805\\
2.84342171085543	76.6631280872125\\
2.84542271135568	77.1222964642303\\
2.84742371185593	77.5489208384847\\
2.84942471235618	77.9419698859445\\
2.85142571285643	78.3002976910193\\
2.85342671335668	78.623102112796\\
2.85542771385693	78.9093518272433\\
2.85742871435718	79.1583592850072\\
2.85942971485743	79.3693223451744\\
2.86143071535768	79.5417253457292\\
2.86343171585793	79.6750526246562\\
2.86543271635818	79.7689031114986\\
2.86743371685843	79.822990327359\\
2.86943471735868	79.8371996806782\\
2.87143571785893	79.8114738756768\\
2.87343671835918	79.7458702081343\\
2.87543771885943	79.6405605653893\\
2.87743871935968	79.4958887221188\\
2.87943971985993	79.3123130445588\\
2.88144072036018	79.0902346031661\\
2.88344172086043	78.8303982430743\\
2.88544272136068	78.5334915136375\\
2.88744372186093	78.2003738515485\\
2.88944472236118	77.8318473977203\\
2.89144572286143	77.4290007719638\\
2.89344672336168	76.9928652983102\\
2.89544772386193	76.5245295965703\\
2.89744872436218	76.0252541738933\\
2.89944972486243	75.4961849458695\\
2.90145072536268	74.9386970112072\\
2.90345172586293	74.3540508770557\\
2.90545272636318	73.7435643463438\\
2.90745372686343	73.1086125177799\\
2.90945472736368	72.4506277858516\\
2.91145572786393	71.7708706577084\\
2.91345672836418	71.0708308236176\\
2.91545772886443	70.3518260865079\\
2.91745872936468	69.6152315450877\\
2.91945972986493	68.8623650022858\\
2.92146073036518	68.09465885259\\
2.92346173086543	67.3133163073701\\
2.92546273136568	66.5195978737754\\
2.92746373186593	65.7148786505142\\
2.92946473236618	64.9003045531767\\
2.93146573286643	64.0770214973532\\
2.93346673336668	63.2462326944135\\
2.93546773386693	62.4089694683888\\
2.93746873436718	61.5663777348694\\
2.93946973486743	60.7194888178866\\
2.94147073536768	59.8692194499124\\
2.94347173586793	59.0165436591987\\
2.94547273636818	58.1624354739972\\
2.94747373686843	57.3076397394415\\
2.94947473736868	56.4530731880039\\
2.95147573786893	55.599480664818\\
2.95347673836918	54.747549719238\\
2.95547773886943	53.898082492177\\
2.95747873936968	53.0517665329893\\
2.95947973986994	52.2091175036904\\
2.96148074037019	51.370765657855\\
2.96348174087044	50.5373412490577\\
2.96548274137069	49.7093026435346\\
2.96748374187094	48.8872227990809\\
2.96948474237119	48.0714454903736\\
2.97148574287144	47.2625436752079\\
2.97348674337169	46.4608038324814\\
2.97548774387194	45.6666843284301\\
2.97748874437219	44.8805289377311\\
2.97948974487244	44.1026814350614\\
2.98149074537269	43.3334282993188\\
2.98349174587294	42.5730560094007\\
2.98549274637319	41.8219083399842\\
2.98749374687344	41.0801571784078\\
2.98949474737369	40.3480890035692\\
2.99149574787394	39.6258757028068\\
2.99349674837419	38.9138037550182\\
2.99549774887444	38.2119877517624\\
2.99749874937469	37.5206568761576\\
2.99949974987494	36.8399830155422\\
3.00150075037519	36.1701380572547\\
3.00350175087544	35.5112938886338\\
3.00550275137569	34.8635651012384\\
3.00750375187594	34.2271235824071\\
3.00950475237619	33.6021412194784\\
3.01150575287644	32.9886753082318\\
3.01350675337669	32.3869550317854\\
3.01550775387694	31.7970949816982\\
3.01750875437719	31.2192097495293\\
3.01950975487744	30.6534139268376\\
3.02151075537769	30.0999366967412\\
3.02351175587794	29.5588926507992\\
3.02551275637819	29.0303963805705\\
3.02751375687844	28.5146197733937\\
3.02951475737869	28.0117347166074\\
3.03151575787894	27.5219130975501\\
3.03351675837919	27.0453268035603\\
3.03551775887944	26.5821477219765\\
3.03751875937969	26.1326050359168\\
3.03951975987994	25.6968706327199\\
3.04152076038019	25.2751163997241\\
3.04352176088044	24.8676861116065\\
3.04552276138069	24.4746370641468\\
3.04752376188094	24.0963703278014\\
3.04952476238119	23.733057789909\\
3.05152576288144	23.384871337808\\
3.05352676338169	23.052212041955\\
3.05552776388194	22.7352517896886\\
3.05752876438219	22.4342770599064\\
3.05952976488244	22.1496316272854\\
3.06153076538269	21.8814300833847\\
3.06353176588294	21.6301307944403\\
3.06553276638319	21.3958483520113\\
3.06753376688344	21.1789265307748\\
3.06953476738369	20.9795945138488\\
3.07153576788394	20.7980814843513\\
3.07353676838419	20.6347312169595\\
3.07553776888444	20.4896010074529\\
3.07753876938469	20.362977334729\\
3.07953976988494	20.2550893819058\\
3.08154077038519	20.165994444763\\
3.08354177088544	20.095864410639\\
3.08554277138569	20.0448138710928\\
3.08754377188594	20.012900121904\\
3.08954477238619	20.0002377546316\\
3.09154577288644	20.0067694734961\\
3.09354677338669	20.032552574277\\
3.09554777388694	20.0774724654153\\
3.09754877438719	20.1415291469109\\
3.09954977488744	20.2245507314254\\
3.10155077538769	20.3264799231791\\
3.10355177588794	20.4471448348337\\
3.10555277638819	20.5863735790505\\
3.10755377688844	20.7439369727115\\
3.10955477738869	20.9196631284781\\
3.11155577788894	21.1133228632323\\
3.11355677838919	21.3246296980765\\
3.11555777888944	21.5534117456723\\
3.11755877938969	21.7993252313424\\
3.11955977988994	22.0621409719689\\
3.1215607803902	22.3415724886542\\
3.12356178089045	22.6373333025008\\
3.1255627813907	22.94925152617\\
3.12756378189095	23.2769260892053\\
3.1295647823912	23.6201278084887\\
3.13156578289145	23.9786847966815\\
3.1335667833917	24.3521959833273\\
3.13556778389195	24.740546776867\\
3.1375687843922	25.1433934026235\\
3.13956978489245	25.5605639732582\\
3.1415707853927	25.9918866014327\\
3.14357178589295	26.4370175124699\\
3.1455727863932	26.8958421148106\\
3.14757378689345	27.3681885211165\\
3.1495747873937	27.8538275482694\\
3.15157578789395	28.3525300131512\\
3.1535767883942	28.864181324203\\
3.15557778889445	29.3886668898658\\
3.1575787893947	29.9257002312419\\
3.15957978989495	30.4752813483314\\
3.1615807903952	31.0371237622367\\
3.16358179089545	31.6111701771783\\
3.1655827913957	32.1972487058176\\
3.16758379189595	32.7951874608161\\
3.1695847923962	33.4048718506148\\
3.17158579289645	34.0262445794342\\
3.1735867933967	34.6590191683767\\
3.17558779389695	35.3031383216628\\
3.1775887943972	35.9584874477334\\
3.17958979489745	36.6248373634705\\
3.1815907953977	37.3020161815357\\
3.18359179589795	37.9899666061492\\
3.1855927963982	38.6884594541932\\
3.18759379689845	39.3972655425496\\
3.1895947973987	40.1162702796592\\
3.19159579789895	40.8451871866247\\
3.1935967983992	41.5839016718868\\
3.19559779889945	42.3320699607687\\
3.1975987993997	43.0895201659316\\
3.19959979989995	43.8559658084781\\
3.2016008004002	44.6311204095106\\
3.20360180090045	45.414640194352\\
3.2056028014007	46.2062386841048\\
3.20760380190095	47.0055721040918\\
3.2096048024012	47.8122393838565\\
3.21160580290145	48.6258967487217\\
3.2136068034017	49.446028536672\\
3.21560780390195	50.2722336772506\\
3.2176088044022	51.1039965084421\\
3.21960980490245	51.9408013682306\\
3.2216108054027	52.7821325946007\\
3.22361180590295	53.6273599339777\\
3.2256128064032	54.4759104285665\\
3.22761380690345	55.3270965290128\\
3.2296148074037	56.1802306859626\\
3.23161580790395	57.0345680542822\\
3.2336168084042	57.8893637888378\\
3.23561780890445	58.7438157487165\\
3.2376188094047	59.5970072014458\\
3.23961980990495	60.4481360061126\\
3.2416208104052	61.2962281344652\\
3.24362181090545	62.140309558252\\
3.2456228114057	62.9794062492211\\
3.24762381190595	63.8123722917823\\
3.2496248124062	64.6381763619043\\
3.25162581290645	65.455672543997\\
3.2536268134067	66.2636576266905\\
3.25562781390695	67.061042990174\\
3.2576288144072	67.8464535357394\\
3.25962981490745	68.6186860520167\\
3.2616308154077	69.3764227360772\\
3.26363181590795	70.1183457849921\\
3.2656328164082	70.8431373958326\\
3.26763381690845	71.5493651741108\\
3.2696348174087	72.2357686126776\\
3.27163581790895	72.9008580212654\\
3.2736368184092	73.5433155969456\\
3.27563781890945	74.1617662410098\\
3.27763881940971	74.7548348547497\\
3.27963981990996	75.3211463394571\\
3.2816408204102	75.8593828922029\\
3.28364182091046	76.3683413016176\\
3.28564282141071	76.8467037647724\\
3.28764382191096	77.2932097745178\\
3.28964482241121	77.7068280068228\\
3.29164582291146	78.0864125460969\\
3.29364682341171	78.4309893640886\\
3.29564782391196	78.7395844325461\\
3.29764882441221	79.0113383147766\\
3.29964982491246	79.2455061656466\\
3.30165082541271	79.4414577315813\\
3.30365182591296	79.5986200547857\\
3.30565282641321	79.7165347690237\\
3.30765382691346	79.794858099618\\
3.30965482741371	79.8333035676713\\
3.31165582791396	79.8318711731835\\
3.31365682841421	79.790503620375\\
3.31565782891446	79.7093727965845\\
3.31765882941471	79.5885932933709\\
3.31965982991496	79.4286234769704\\
3.32166083041521	79.229921713619\\
3.32366183091546	78.9930036653324\\
3.32566283141571	78.7185568814648\\
3.32766383191596	78.4073835029292\\
3.32966483241621	78.0602283748595\\
3.33166583291646	77.6781801170662\\
3.33366683341671	77.2621554620217\\
3.33566783391696	76.8133003253162\\
3.33766883441721	76.3327033267605\\
3.33966983491746	75.8216249735038\\
3.34167083541771	75.2812684769159\\
3.34367183591796	74.7130089357052\\
3.34567283641821	74.1180495612413\\
3.34767383691846	73.4978227480122\\
3.34967483741871	72.8537035947262\\
3.35167583791896	72.1870099043119\\
3.35367683841921	71.4991740712574\\
3.35567783891946	70.7915138984913\\
3.35767883941971	70.0654617805015\\
3.35967983991996	69.3223928159964\\
3.36168084042021	68.5635102163456\\
3.36368184092046	67.7903036718165\\
3.36568284142071	67.003976393779\\
3.36768384192096	66.2058461851618\\
3.36968484242121	65.3971162573346\\
3.37168584292146	64.5789898216673\\
3.37368684342171	63.7526127937501\\
3.37568784392196	62.9191883849528\\
3.37768884442221	62.0796906235271\\
3.37968984492246	61.2353227208428\\
3.38169084542271	60.3870014093721\\
3.38369184592296	59.5356434215873\\
3.38569284642321	58.6823373772989\\
3.38769384692346	57.8278281216408\\
3.38969484742371	56.9730323870851\\
3.39169584792396	56.1187523145451\\
3.39369684842421	55.2657327491543\\
3.39569784892446	54.4147185360465\\
3.39769884942471	53.5663972245758\\
3.39969984992496	52.7213990683169\\
3.40170085042521	51.8804116166238\\
3.40370185092546	51.0438932357328\\
3.40570285142571	50.212531474998\\
3.40770385192596	49.3867847006555\\
3.40970485242621	48.5671112789413\\
3.41170585292646	47.7539695760916\\
3.41370685342671	46.9478179583426\\
3.41570785392696	46.1490002003712\\
3.41770885442721	45.3579746684136\\
3.41970985492746	44.5750278413673\\
3.42171085542771	43.8005034939095\\
3.42371185592796	43.0346881049376\\
3.42571285642821	42.2779254491288\\
3.42771385692846	41.5303874138216\\
3.42971485742871	40.7924177736931\\
3.43171585792896	40.0641884160818\\
3.43371685842921	39.3458712283263\\
3.43571785892946	38.6378099851037\\
3.43771885942971	37.9401192779728\\
3.43971985992996	37.2529136984929\\
3.44172086043022	36.5764797255615\\
3.44372186093047	35.9109319507375\\
3.44572286143072	35.2563849655801\\
3.44772386193097	34.6130679532072\\
3.44972486243122	33.9810955051779\\
3.45172586293147	33.3606395088307\\
3.45372686343172	32.7517572599452\\
3.45572786393197	32.1546206458599\\
3.45772886443222	31.5694015539132\\
3.45972986493247	30.9962718714439\\
3.46173086543272	30.4352888942313\\
3.46373186593297	29.886624509614\\
3.46573286643322	29.3504506049306\\
3.46773386693347	28.8269390675195\\
3.46973486743372	28.3162044889399\\
3.47173586793397	27.8184187565303\\
3.47373686843422	27.3337537576291\\
3.47573786893447	26.862381379575\\
3.47773886943472	26.4045308054859\\
3.47973986993497	25.9603739227005\\
3.48174087043522	25.5300826185573\\
3.48374187093547	25.1139433719538\\
3.48574287143572	24.712070774449\\
3.48774387193597	24.3248086007201\\
3.48974487243622	23.9523287381055\\
3.49174587293647	23.5949176655029\\
3.49374687343672	23.2528045660303\\
3.49574787393697	22.9262759185852\\
3.49774887443722	22.6155036105063\\
3.49974987493747	22.32088871225\\
3.50175087543772	22.0426604069345\\
3.50375187593797	21.7810478776778\\
3.50575287643822	21.5363376033774\\
3.50775387693847	21.3088160629309\\
3.50975487743872	21.098769735236\\
3.51175587793897	20.9064278034106\\
3.51375687843922	20.7319621547932\\
3.51575787893947	20.575716564061\\
3.51775887943972	20.4378056227731\\
3.51975987993997	20.3185158098268\\
3.52176088044022	20.2179617167814\\
3.52376188094047	20.1362579351957\\
3.52576288144072	20.0735763524084\\
3.52776388194097	20.0300315599785\\
3.52976488244122	20.0056808536854\\
3.53176588294147	20.0005242335292\\
3.53376688344172	20.0145616995099\\
3.53576788394197	20.047850547407\\
3.53776888444222	20.1002761856615\\
3.53976988494247	20.1717813184938\\
3.54177088544272	20.2621940585655\\
3.54377188594297	20.3715144058764\\
3.54577288644322	20.4994558815291\\
3.54777388694347	20.6458465981851\\
3.54977488744372	20.8105719642852\\
3.55177588794397	20.9933455009319\\
3.55377688844422	21.1939380250072\\
3.55577788894447	21.412120353393\\
3.55777888944472	21.6476060071918\\
3.55977988994497	21.900223099065\\
3.56178089044522	22.169570558556\\
3.56378189094547	22.4554192025467\\
3.56578289144572	22.7575398479192\\
3.56778389194597	23.0756460157759\\
3.56978489244622	23.4094512272191\\
3.57178589294647	23.7586690033513\\
3.57378689344672	24.123127456834\\
3.57578789394697	24.5024828129901\\
3.57778889444722	24.8965058887016\\
3.57978989494747	25.3050247966299\\
3.58179089544772	25.7277530578774\\
3.58379189594797	26.1645187851056\\
3.58579289644822	26.615035499417\\
3.58779389694847	27.0791886092525\\
3.58979489744872	27.5566916357145\\
3.59179589794897	28.0474872830236\\
3.59379689844922	28.5512890722821\\
3.59579789894947	29.0679824119311\\
3.59779889944972	29.597395414632\\
3.59979989994997	30.1393561930462\\
3.60180090045022	30.6937501556148\\
3.60380190095048	31.2604054149992\\
3.60580290145073	31.8391500838608\\
3.60780390195098	32.4299268664202\\
3.60980490245123	33.0324492837798\\
3.61180590295148	33.6467746317191\\
3.61380690345173	34.2725591355609\\
3.61580790395198	34.909860091085\\
3.61780890445223	35.5583910193935\\
3.61980990495248	36.2180373289277\\
3.62181090545273	36.8886271323488\\
3.62381190595298	37.5701031338774\\
3.62581290645323	38.2621788546159\\
3.62781390695348	38.9646824072258\\
3.62981490745373	39.6775564959276\\
3.63181590795398	40.4004573460441\\
3.63381690845423	41.1332130702369\\
3.63581790895448	41.8756517811675\\
3.63781890945473	42.6274869999381\\
3.63981990995498	43.3884895434309\\
3.64182091045523	44.1583729327482\\
3.64382191095548	44.9368506889924\\
3.64582291145573	45.7235790374865\\
3.64782391195598	46.518271499333\\
3.64982491245623	47.3205270040752\\
3.65182591295648	48.1299444812565\\
3.65382691345673	48.9461801561999\\
3.65582791395698	49.7687183668897\\
3.65782891445723	50.5971580428693\\
3.65982991495748	51.4309835221232\\
3.66183091545773	52.2696218468562\\
3.66383191595798	53.1125573550527\\
3.66583291645823	53.959159793138\\
3.66783391695848	54.8087989075375\\
3.66983491745873	55.6608444446765\\
3.67183591795898	56.5145515594214\\
3.67383691845923	57.3691754066386\\
3.67583791895948	58.2239138454147\\
3.67783891945973	59.0780220306163\\
3.67983991995998	59.9305259339914\\
3.68184092046023	60.7805661188475\\
3.68384192096048	61.6272258527123\\
3.68584292146073	62.4694165157751\\
3.68784392196098	63.3062213755637\\
3.68984492246123	64.1365518122673\\
3.69184592296148	64.9592619102956\\
3.69384692346173	65.773148458279\\
3.69584792396198	66.5771228364065\\
3.69784892446223	67.3699818333086\\
3.69984992496248	68.1504076460563\\
3.70185092546273	68.9170824717208\\
3.70385192596298	69.6688030989325\\
3.70585292646323	70.4041944289829\\
3.70785392696348	71.1218813631637\\
3.70985492746373	71.8204888027667\\
3.71185592796398	72.4986989448631\\
3.71385692846423	73.1551366907445\\
3.71585792896448	73.7883696459231\\
3.71785892946473	74.3970227116906\\
3.71985992996498	74.9797780851181\\
3.72186093046523	75.5352606674974\\
3.72386193096548	76.0622099516793\\
3.72586293146573	76.5593081347347\\
3.72786393196598	77.0253520052942\\
3.72986493246623	77.4591383519877\\
3.73186593296648	77.8595212592251\\
3.73386693346673	78.2255266987547\\
3.73586793396698	78.5561233465452\\
3.73786893446723	78.8503944701244\\
3.73986993496748	79.1075379285791\\
3.74187093546773	79.3268088767757\\
3.74387193596798	79.5076343569189\\
3.74587293646823	79.6494414112138\\
3.74787393696848	79.7518862649832\\
3.74987493746873	79.81462514355\\
3.75187593796898	79.8375434553553\\
3.75387693846923	79.8204693130604\\
3.75587793896948	79.7634600124449\\
3.75787893946973	79.6668020324063\\
3.75987993996999	79.5306099645037\\
3.76188094047024	79.3554567665322\\
3.76388194097049	79.1416862131689\\
3.76588294147074	78.8899858537679\\
3.76788394197099	78.6010432376835\\
3.76988494247124	78.2757178016082\\
3.77188594297149	77.9148116864553\\
3.77388694347174	77.5192989204765\\
3.77588794397199	77.0902681234825\\
3.77788894447224	76.6288652110636\\
3.77988994497249	76.1361788030306\\
3.78189094547274	75.6135267023123\\
3.78389194597299	75.0620548244989\\
3.78589294647324	74.4831955640782\\
3.78789394697349	73.8782094281996\\
3.78989494747374	73.2484715155713\\
3.79189594797399	72.5953569249017\\
3.79389694847424	71.9201834591195\\
3.79589794897449	71.2244408084921\\
3.79789894947474	70.5094467759484\\
3.79989994997499	69.776519164417\\
3.80190095047524	69.0270903683859\\
3.80390195097549	68.2624781907838\\
3.80590295147574	67.4840004345396\\
3.80790395197599	66.6929176068025\\
3.80990495247624	65.8904329189422\\
3.81190595297649	65.0778641738877\\
3.81390695347674	64.2564145830086\\
3.81590795397699	63.4271154703363\\
3.81790895447724	62.5912273430199\\
3.81990995497749	61.7496669335318\\
3.82191095547774	60.9036374532416\\
3.82391195597799	60.0539983388421\\
3.82591295647824	59.2017809143645\\
3.82791395697849	58.347901912281\\
3.82991495747874	57.4932207692844\\
3.83191595797899	56.6384823305082\\
3.83391695847924	55.7846033284248\\
3.83591795897949	54.9323286081677\\
3.83791895947974	54.0822311275321\\
3.83991995997999	53.2351703232107\\
3.84192096048024	52.3916618572191\\
3.84392196098049	51.5523359831319\\
3.84592296148074	50.7178229545239\\
3.84792396198099	49.888581137631\\
3.84992496248124	49.065126194469\\
3.85192596298149	48.2479737870534\\
3.85392696348174	47.4375249858409\\
3.85592796398199	46.634180861288\\
3.85792896448224	45.8383997796308\\
3.85992996498249	45.0504682197668\\
3.86193096548274	44.2707872521528\\
3.86393196598299	43.4996433556863\\
3.86593296648324	42.7373230092647\\
3.86793396698349	41.9841699875652\\
3.86993496748374	41.2403561779264\\
3.87193596798399	40.5061680592458\\
3.87393696848424	39.7817775188619\\
3.87593796898449	39.0674710356723\\
3.87793896948474	38.3634204970155\\
3.87993996998499	37.6697977902301\\
3.88194097048524	36.9868320984342\\
3.88394197098549	36.3146380131867\\
3.88594297148574	35.6533874218262\\
3.88794397198599	35.0032522116913\\
3.88994497248624	34.3643469743409\\
3.89194597298649	33.7368435971136\\
3.89394697348674	33.1208566715685\\
3.89594797398699	32.516558085044\\
3.89794897448724	31.9241197248787\\
3.89994997498749	31.3435988868522\\
3.90195097548774	30.7752247540824\\
3.90395197598799	30.2190546223489\\
3.90595297648824	29.6752603789902\\
3.90795397698849	29.1440139113449\\
3.90995497748874	28.6254871067515\\
3.91195597798899	28.1197945567691\\
3.91395697848924	27.6271081487361\\
3.91595797898949	27.1475997699911\\
3.91795897948974	26.6815558994317\\
3.91995997998999	26.2290338328374\\
3.92196098049025	25.7902627533262\\
3.9239619809905	25.3654718440162\\
3.92596298149075	24.9548902880254\\
3.927963981991	24.558747268472\\
3.92996498249125	24.1772146726944\\
3.9319659829915	23.8106362753697\\
3.93396698349175	23.4591266680569\\
3.935967983992	23.1230869212127\\
3.93796898449225	22.8026889221755\\
3.9399699849925	22.4982764456225\\
3.94197098549275	22.2100213788922\\
3.943971985993	21.9382674966617\\
3.94597298649325	21.6832439820489\\
3.9479739869935	21.4452373139516\\
3.94997498749375	21.2245339712672\\
3.951975987994	21.0213631371138\\
3.95397698849425	20.8359539946095\\
3.9559779889945	20.6686503184313\\
3.95797898949475	20.5195094043587\\
3.959979989995	20.3888750270689\\
3.96198099049525	20.2769190739003\\
3.9639819909955	20.1836988406325\\
3.96598299149575	20.1094435103836\\
3.967983991996	20.0542676747125\\
3.96998499249625	20.0182286293988\\
3.9719859929965	20.0013836702219\\
3.97398699349675	20.0037327971819\\
3.975987993997	20.0253333060584\\
3.97798899449725	20.0660706052922\\
3.9799899949975	20.1260019906629\\
3.98199099549775	20.2048982790524\\
3.983991995998	20.3027594704607\\
3.98599299649825	20.4193563817698\\
3.9879939969985	20.5545744214207\\
3.98999499749875	20.7081844062953\\
3.991995997999	20.8799571532755\\
3.99399699849925	21.0697780708024\\
3.9959979989995	21.2773033841987\\
3.99799899949975	21.5022466145671\\
4	21.7444931703484\\
};
\addlegendentry{$\theta$};

\end{axis}
\end{tikzpicture}%
	\caption{Showing the angle evolution for the spinning top problem over 4 seconds.}
	\label{fig:forwardData}
\end{figure}
\fi
\iftikz
\begin{figure}[H]
	\centering
	\setlength\figureheight{7cm} 
	\setlength\figurewidth{14cm}
	% This file was created by matlab2tikz.
% Minimal pgfplots version: 1.3
%
%The latest updates can be retrieved from
%  http://www.mathworks.com/matlabcentral/fileexchange/22022-matlab2tikz
%where you can also make suggestions and rate matlab2tikz.
%
\definecolor{mycolor1}{rgb}{0.00000,0.44700,0.74100}%
\definecolor{mycolor2}{rgb}{0.85000,0.32500,0.09800}%
\definecolor{mycolor3}{rgb}{0.92900,0.69400,0.12500}%
\definecolor{mycolor4}{rgb}{0.49400,0.18400,0.55600}%
\definecolor{mycolor5}{rgb}{0.46600,0.67400,0.18800}%
\definecolor{mycolor6}{rgb}{0.30100,0.74500,0.93300}%
%
\begin{tikzpicture}

\begin{axis}[%
width=0.95092\figurewidth,
height=\figureheight,
at={(0\figurewidth,0\figureheight)},
scale only axis,
xmin=0,
xmax=4,
xtick={0,0.5,1,1.5,2,2.5,3,3.5,4},
xticklabels={{4},{3.5},{3},{2.5},{2},{1.5},{1},{0.5},{0}},
xlabel={Time (s)},
ymin=-0.02,
ymax=40,
ylabel={Degrees},
title style={font=\bfseries},
title={Top Spin [4,0] (s)},
legend style={legend cell align=left,align=left,draw=white!15!black},
title style={font=\small},ticklabel style={font=\tiny}
]
\addplot [color=mycolor1,solid]
  table[row sep=crcr]{%
0	2.4001\\
0.00199900049975012	2.397943\\
0.00399800099950025	2.395792\\
0.00599700149925037	2.393648\\
0.0079960019990005	2.391509\\
0.00999500249875063	2.389376\\
0.0119940029985007	2.38725\\
0.0139930034982509	2.38513\\
0.015992003998001	2.383017\\
0.0179910044977511	2.380909\\
0.0199900049975013	2.378808\\
0.0219890054972514	2.376714\\
0.0239880059970015	2.374626\\
0.0259870064967516	2.372544\\
0.0279860069965017	2.37047\\
0.0299850074962519	2.368401\\
0.031984007996002	2.36634\\
0.0339830084957521	2.364285\\
0.0359820089955022	2.362237\\
0.0379810094952524	2.360195\\
0.0399800099950025	2.358161\\
0.0419790104947526	2.356133\\
0.0439780109945027	2.354113\\
0.0459770114942529	2.352099\\
0.047976011994003	2.350092\\
0.0499750124937531	2.348092\\
0.0519740129935032	2.3461\\
0.0539730134932534	2.344114\\
0.0559720139930035	2.342136\\
0.0579710144927536	2.340165\\
0.0599700149925037	2.338201\\
0.0619690154922539	2.336244\\
0.063968015992004	2.334295\\
0.0659670164917541	2.332353\\
0.0679660169915042	2.330418\\
0.0699650174912544	2.328491\\
0.0719640179910045	2.326571\\
0.0739630184907546	2.324659\\
0.0759620189905048	2.322754\\
0.0779610194902549	2.320857\\
0.079960019990005	2.318967\\
0.0819590204897551	2.317085\\
0.0839580209895052	2.31521\\
0.0859570214892554	2.313344\\
0.0879560219890055	2.311485\\
0.0899550224887556	2.309633\\
0.0919540229885057	2.30779\\
0.0939530234882559	2.305954\\
0.095952023988006	2.304126\\
0.0979510244877561	2.302306\\
0.0999500249875062	2.300494\\
0.101949025487256	2.29869\\
0.103948025987006	2.296894\\
0.105947026486757	2.295106\\
0.107946026986507	2.293326\\
0.109945027486257	2.291554\\
0.111944027986007	2.28979\\
0.113943028485757	2.288034\\
0.115942028985507	2.286286\\
0.117941029485257	2.284546\\
0.119940029985007	2.282815\\
0.121939030484758	2.281092\\
0.123938030984508	2.279377\\
0.125937031484258	2.27767\\
0.127936031984008	2.275971\\
0.129935032483758	2.274281\\
0.131934032983508	2.272599\\
0.133933033483258	2.270926\\
0.135932033983008	2.26926\\
0.137931034482759	2.267604\\
0.139930034982509	2.265955\\
0.141929035482259	2.264316\\
0.143928035982009	2.262684\\
0.145927036481759	2.261061\\
0.147926036981509	2.259447\\
0.149925037481259	2.257841\\
0.15192403798101	2.256243\\
0.15392303848076	2.254655\\
0.15592203898051	2.253074\\
0.15792103948026	2.251503\\
0.15992003998001	2.24994\\
0.16191904047976	2.248385\\
0.16391804097951	2.24684\\
0.16591704147926	2.245303\\
0.16791604197901	2.243774\\
0.169915042478761	2.242254\\
0.171914042978511	2.240743\\
0.173913043478261	2.239241\\
0.175912043978011	2.237748\\
0.177911044477761	2.236263\\
0.179910044977511	2.234787\\
0.181909045477261	2.23332\\
0.183908045977011	2.231862\\
0.185907046476762	2.230412\\
0.187906046976512	2.228971\\
0.189905047476262	2.227539\\
0.191904047976012	2.226116\\
0.193903048475762	2.224702\\
0.195902048975512	2.223297\\
0.197901049475262	2.221901\\
0.199900049975012	2.220513\\
0.201899050474763	2.219134\\
0.203898050974513	2.217765\\
0.205897051474263	2.216404\\
0.207896051974013	2.215052\\
0.209895052473763	2.213709\\
0.211894052973513	2.212375\\
0.213893053473263	2.21105\\
0.215892053973014	2.209734\\
0.217891054472764	2.208427\\
0.219890054972514	2.207129\\
0.221889055472264	2.20584\\
0.223888055972014	2.204559\\
0.225887056471764	2.203288\\
0.227886056971514	2.202026\\
0.229885057471264	2.200773\\
0.231884057971014	2.199529\\
0.233883058470765	2.198293\\
0.235882058970515	2.197067\\
0.237881059470265	2.19585\\
0.239880059970015	2.194642\\
0.241879060469765	2.193442\\
0.243878060969515	2.192252\\
0.245877061469265	2.191071\\
0.247876061969015	2.189899\\
0.249875062468766	2.188736\\
0.251874062968516	2.187581\\
0.253873063468266	2.186436\\
0.255872063968016	2.1853\\
0.257871064467766	2.184173\\
0.259870064967516	2.183055\\
0.261869065467266	2.181945\\
0.263868065967017	2.180845\\
0.265867066466767	2.179754\\
0.267866066966517	2.178672\\
0.269865067466267	2.177598\\
0.271864067966017	2.176534\\
0.273863068465767	2.175479\\
0.275862068965517	2.174432\\
0.277861069465267	2.173395\\
0.279860069965017	2.172367\\
0.281859070464768	2.171347\\
0.283858070964518	2.170336\\
0.285857071464268	2.169335\\
0.287856071964018	2.168342\\
0.289855072463768	2.167358\\
0.291854072963518	2.166383\\
0.293853073463268	2.165417\\
0.295852073963018	2.16446\\
0.297851074462769	2.163511\\
0.299850074962519	2.162572\\
0.301849075462269	2.161641\\
0.303848075962019	2.160719\\
0.305847076461769	2.159806\\
0.307846076961519	2.158902\\
0.309845077461269	2.158006\\
0.311844077961019	2.15712\\
0.31384307846077	2.156242\\
0.31584207896052	2.155373\\
0.31784107946027	2.154512\\
0.31984007996002	2.15366\\
0.32183908045977	2.152817\\
0.32383808095952	2.151983\\
0.32583708145927	2.151157\\
0.32783608195902	2.15034\\
0.329835082458771	2.149531\\
0.331834082958521	2.148731\\
0.333833083458271	2.14794\\
0.335832083958021	2.147157\\
0.337831084457771	2.146383\\
0.339830084957521	2.145617\\
0.341829085457271	2.14486\\
0.343828085957021	2.144111\\
0.345827086456772	2.143371\\
0.347826086956522	2.142639\\
0.349825087456272	2.141916\\
0.351824087956022	2.1412\\
0.353823088455772	2.140494\\
0.355822088955522	2.139795\\
0.357821089455272	2.139105\\
0.359820089955023	2.138423\\
0.361819090454773	2.13775\\
0.363818090954523	2.137084\\
0.365817091454273	2.136427\\
0.367816091954023	2.135778\\
0.369815092453773	2.135138\\
0.371814092953523	2.134505\\
0.373813093453273	2.13388\\
0.375812093953024	2.133264\\
0.377811094452774	2.132655\\
0.379810094952524	2.132055\\
0.381809095452274	2.131462\\
0.383808095952024	2.130877\\
0.385807096451774	2.1303\\
0.387806096951524	2.129732\\
0.389805097451274	2.12917\\
0.391804097951024	2.128617\\
0.393803098450775	2.128072\\
0.395802098950525	2.127534\\
0.397801099450275	2.127004\\
0.399800099950025	2.126481\\
0.401799100449775	2.125967\\
0.403798100949525	2.125459\\
0.405797101449275	2.12496\\
0.407796101949025	2.124468\\
0.409795102448776	2.123983\\
0.411794102948526	2.123506\\
0.413793103448276	2.123036\\
0.415792103948026	2.122573\\
0.417791104447776	2.122118\\
0.419790104947526	2.12167\\
0.421789105447276	2.121229\\
0.423788105947026	2.120796\\
0.425787106446777	2.120369\\
0.427786106946527	2.11995\\
0.429785107446277	2.119538\\
0.431784107946027	2.119133\\
0.433783108445777	2.118734\\
0.435782108945527	2.118343\\
0.437781109445277	2.117959\\
0.439780109945027	2.117581\\
0.441779110444778	2.11721\\
0.443778110944528	2.116846\\
0.445777111444278	2.116489\\
0.447776111944028	2.116138\\
0.449775112443778	2.115794\\
0.451774112943528	2.115456\\
0.453773113443278	2.115125\\
0.455772113943028	2.114801\\
0.457771114442779	2.114482\\
0.459770114942529	2.114171\\
0.461769115442279	2.113865\\
0.463768115942029	2.113566\\
0.465767116441779	2.113272\\
0.467766116941529	2.112985\\
0.469765117441279	2.112705\\
0.471764117941029	2.11243\\
0.47376311844078	2.112161\\
0.47576211894053	2.111898\\
0.47776111944028	2.111641\\
0.47976011994003	2.111389\\
0.48175912043978	2.111144\\
0.48375812093953	2.110904\\
0.48575712143928	2.11067\\
0.487756121939031	2.110441\\
0.489755122438781	2.110218\\
0.491754122938531	2.11\\
0.493753123438281	2.109788\\
0.495752123938031	2.109581\\
0.497751124437781	2.10938\\
0.499750124937531	2.109183\\
0.501749125437281	2.108992\\
0.503748125937031	2.108806\\
0.505747126436782	2.108625\\
0.507746126936532	2.108449\\
0.509745127436282	2.108278\\
0.511744127936032	2.108112\\
0.513743128435782	2.10795\\
0.515742128935532	2.107794\\
0.517741129435282	2.107642\\
0.519740129935032	2.107494\\
0.521739130434783	2.107352\\
0.523738130934533	2.107213\\
0.525737131434283	2.107079\\
0.527736131934033	2.10695\\
0.529735132433783	2.106824\\
0.531734132933533	2.106703\\
0.533733133433283	2.106586\\
0.535732133933034	2.106473\\
0.537731134432784	2.106365\\
0.539730134932534	2.10626\\
0.541729135432284	2.106159\\
0.543728135932034	2.106061\\
0.545727136431784	2.105968\\
0.547726136931534	2.105878\\
0.549725137431284	2.105792\\
0.551724137931034	2.105709\\
0.553723138430785	2.10563\\
0.555722138930535	2.105554\\
0.557721139430285	2.105482\\
0.559720139930035	2.105413\\
0.561719140429785	2.105347\\
0.563718140929535	2.105284\\
0.565717141429285	2.105224\\
0.567716141929036	2.105167\\
0.569715142428786	2.105113\\
0.571714142928536	2.105061\\
0.573713143428286	2.105013\\
0.575712143928036	2.104967\\
0.577711144427786	2.104924\\
0.579710144927536	2.104883\\
0.581709145427286	2.104845\\
0.583708145927036	2.104809\\
0.585707146426787	2.104775\\
0.587706146926537	2.104744\\
0.589705147426287	2.104715\\
0.591704147926037	2.104688\\
0.593703148425787	2.104663\\
0.595702148925537	2.104639\\
0.597701149425287	2.104618\\
0.599700149925038	2.104598\\
0.601699150424788	2.104581\\
0.603698150924538	2.104564\\
0.605697151424288	2.10455\\
0.607696151924038	2.104536\\
0.609695152423788	2.104525\\
0.611694152923538	2.104514\\
0.613693153423288	2.104505\\
0.615692153923038	2.104497\\
0.617691154422789	2.104489\\
0.619690154922539	2.104483\\
0.621689155422289	2.104478\\
0.623688155922039	2.104474\\
0.625687156421789	2.104471\\
0.627686156921539	2.104468\\
0.629685157421289	2.104466\\
0.631684157921039	2.104464\\
0.63368315842079	2.104463\\
0.63568215892054	2.104462\\
0.63768115942029	2.104462\\
0.63968015992004	2.104461\\
0.64167916041979	2.104461\\
0.64367816091954	2.104461\\
0.64567716141929	2.104461\\
0.64767616191904	2.104461\\
0.649675162418791	2.104461\\
0.651674162918541	2.10446\\
0.653673163418291	2.10446\\
0.655672163918041	2.104458\\
0.657671164417791	2.104457\\
0.659670164917541	2.104454\\
0.661669165417291	2.104452\\
0.663668165917041	2.104448\\
0.665667166416792	2.104444\\
0.667666166916542	2.104439\\
0.669665167416292	2.104432\\
0.671664167916042	2.104425\\
0.673663168415792	2.104417\\
0.675662168915542	2.104408\\
0.677661169415292	2.104397\\
0.679660169915043	2.104385\\
0.681659170414793	2.104372\\
0.683658170914543	2.104357\\
0.685657171414293	2.104341\\
0.687656171914043	2.104323\\
0.689655172413793	2.104303\\
0.691654172913543	2.104282\\
0.693653173413293	2.104258\\
0.695652173913043	2.104233\\
0.697651174412794	2.104206\\
0.699650174912544	2.104176\\
0.701649175412294	2.104145\\
0.703648175912044	2.104111\\
0.705647176411794	2.104075\\
0.707646176911544	2.104037\\
0.709645177411294	2.103996\\
0.711644177911045	2.103953\\
0.713643178410795	2.103907\\
0.715642178910545	2.103858\\
0.717641179410295	2.103807\\
0.719640179910045	2.103752\\
0.721639180409795	2.103695\\
0.723638180909545	2.103635\\
0.725637181409295	2.103572\\
0.727636181909045	2.103506\\
0.729635182408796	2.103436\\
0.731634182908546	2.103364\\
0.733633183408296	2.103288\\
0.735632183908046	2.103208\\
0.737631184407796	2.103125\\
0.739630184907546	2.103039\\
0.741629185407296	2.102949\\
0.743628185907046	2.102855\\
0.745627186406797	2.102758\\
0.747626186906547	2.102657\\
0.749625187406297	2.102551\\
0.751624187906047	2.102442\\
0.753623188405797	2.102329\\
0.755622188905547	2.102212\\
0.757621189405297	2.102091\\
0.759620189905047	2.101965\\
0.761619190404798	2.101835\\
0.763618190904548	2.101701\\
0.765617191404298	2.101562\\
0.767616191904048	2.101419\\
0.769615192403798	2.101272\\
0.771614192903548	2.101119\\
0.773613193403298	2.100962\\
0.775612193903048	2.100801\\
0.777611194402799	2.100634\\
0.779610194902549	2.100463\\
0.781609195402299	2.100287\\
0.783608195902049	2.100105\\
0.785607196401799	2.099919\\
0.787606196901549	2.099728\\
0.789605197401299	2.099531\\
0.79160419790105	2.099329\\
0.7936031984008	2.099122\\
0.79560219890055	2.098909\\
0.7976011994003	2.098691\\
0.79960019990005	2.098468\\
0.8015992003998	2.098239\\
0.80359820089955	2.098004\\
0.8055972013993	2.097764\\
0.80759620189905	2.097518\\
0.809595202398801	2.097267\\
0.811594202898551	2.097009\\
0.813593203398301	2.096746\\
0.815592203898051	2.096477\\
0.817591204397801	2.096201\\
0.819590204897551	2.09592\\
0.821589205397301	2.095633\\
0.823588205897052	2.095339\\
0.825587206396802	2.09504\\
0.827586206896552	2.094734\\
0.829585207396302	2.094421\\
0.831584207896052	2.094103\\
0.833583208395802	2.093778\\
0.835582208895552	2.093446\\
0.837581209395302	2.093108\\
0.839580209895052	2.092764\\
0.841579210394803	2.092413\\
0.843578210894553	2.092055\\
0.845577211394303	2.09169\\
0.847576211894053	2.091319\\
0.849575212393803	2.090941\\
0.851574212893553	2.090556\\
0.853573213393303	2.090165\\
0.855572213893053	2.089766\\
0.857571214392804	2.08936\\
0.859570214892554	2.088948\\
0.861569215392304	2.088528\\
0.863568215892054	2.088101\\
0.865567216391804	2.087667\\
0.867566216891554	2.087226\\
0.869565217391304	2.086778\\
0.871564217891054	2.086322\\
0.873563218390805	2.085859\\
0.875562218890555	2.085389\\
0.877561219390305	2.084911\\
0.879560219890055	2.084426\\
0.881559220389805	2.083933\\
0.883558220889555	2.083433\\
0.885557221389305	2.082926\\
0.887556221889055	2.08241\\
0.889555222388806	2.081887\\
0.891554222888556	2.081357\\
0.893553223388306	2.080819\\
0.895552223888056	2.080273\\
0.897551224387806	2.079719\\
0.899550224887556	2.079157\\
0.901549225387306	2.078588\\
0.903548225887057	2.078011\\
0.905547226386807	2.077426\\
0.907546226886557	2.076833\\
0.909545227386307	2.076232\\
0.911544227886057	2.075623\\
0.913543228385807	2.075005\\
0.915542228885557	2.07438\\
0.917541229385307	2.073747\\
0.919540229885057	2.073106\\
0.921539230384808	2.072456\\
0.923538230884558	2.071799\\
0.925537231384308	2.071133\\
0.927536231884058	2.070459\\
0.929535232383808	2.069777\\
0.931534232883558	2.069086\\
0.933533233383308	2.068387\\
0.935532233883059	2.06768\\
0.937531234382809	2.066964\\
0.939530234882559	2.06624\\
0.941529235382309	2.065508\\
0.943528235882059	2.064767\\
0.945527236381809	2.064018\\
0.947526236881559	2.06326\\
0.949525237381309	2.062494\\
0.951524237881059	2.061719\\
0.95352323838081	2.060936\\
0.95552223888056	2.060144\\
0.95752123938031	2.059344\\
0.95952023988006	2.058535\\
0.96151924037981	2.057717\\
0.96351824087956	2.056891\\
0.96551724137931	2.056056\\
0.96751624187906	2.055212\\
0.969515242378811	2.05436\\
0.971514242878561	2.053499\\
0.973513243378311	2.052629\\
0.975512243878061	2.051751\\
0.977511244377811	2.050863\\
0.979510244877561	2.049967\\
0.981509245377311	2.049063\\
0.983508245877061	2.048149\\
0.985507246376812	2.047227\\
0.987506246876562	2.046295\\
0.989505247376312	2.045355\\
0.991504247876062	2.044406\\
0.993503248375812	2.043449\\
0.995502248875562	2.042482\\
0.997501249375312	2.041506\\
0.999500249875062	2.040522\\
1.00149925037481	2.039529\\
1.00349825087456	2.038527\\
1.00549725137431	2.037515\\
1.00749625187406	2.036495\\
1.00949525237381	2.035466\\
1.01149425287356	2.034428\\
1.01349325337331	2.033381\\
1.01549225387306	2.032326\\
1.01749125437281	2.031261\\
1.01949025487256	2.030187\\
1.02148925537231	2.029104\\
1.02348825587206	2.028012\\
1.02548725637181	2.026912\\
1.02748625687156	2.025802\\
1.02948525737131	2.024683\\
1.03148425787106	2.023555\\
1.03348325837081	2.022419\\
1.03548225887056	2.021273\\
1.03748125937031	2.020118\\
1.03948025987006	2.018955\\
1.04147926036982	2.017782\\
1.04347826086957	2.0166\\
1.04547726136932	2.015409\\
1.04747626186907	2.01421\\
1.04947526236882	2.013001\\
1.05147426286857	2.011783\\
1.05347326336832	2.010556\\
1.05547226386807	2.009321\\
1.05747126436782	2.008076\\
1.05947026486757	2.006822\\
1.06146926536732	2.005559\\
1.06346826586707	2.004288\\
1.06546726636682	2.003007\\
1.06746626686657	2.001717\\
1.06946526736632	2.000419\\
1.07146426786607	1.999111\\
1.07346326836582	1.997794\\
1.07546226886557	1.996469\\
1.07746126936532	1.995134\\
1.07946026986507	1.993791\\
1.08145927036482	1.992438\\
1.08345827086457	1.991077\\
1.08545727136432	1.989707\\
1.08745627186407	1.988328\\
1.08945527236382	1.98694\\
1.09145427286357	1.985543\\
1.09345327336332	1.984137\\
1.09545227386307	1.982722\\
1.09745127436282	1.981299\\
1.09945027486257	1.979866\\
1.10144927536232	1.978425\\
1.10344827586207	1.976975\\
1.10544727636182	1.975516\\
1.10744627686157	1.974048\\
1.10944527736132	1.972572\\
1.11144427786107	1.971087\\
1.11344327836082	1.969593\\
1.11544227886057	1.96809\\
1.11744127936032	1.966578\\
1.11944027986007	1.965058\\
1.12143928035982	1.963529\\
1.12343828085957	1.961992\\
1.12543728135932	1.960445\\
1.12743628185907	1.958891\\
1.12943528235882	1.957327\\
1.13143428285857	1.955755\\
1.13343328335832	1.954174\\
1.13543228385807	1.952585\\
1.13743128435782	1.950987\\
1.13943028485757	1.949381\\
1.14142928535732	1.947766\\
1.14342828585707	1.946142\\
1.14542728635682	1.94451\\
1.14742628685657	1.94287\\
1.14942528735632	1.941221\\
1.15142428785607	1.939564\\
1.15342328835582	1.937898\\
1.15542228885557	1.936224\\
1.15742128935532	1.934542\\
1.15942028985507	1.932851\\
1.16141929035482	1.931152\\
1.16341829085457	1.929445\\
1.16541729135432	1.927729\\
1.16741629185407	1.926006\\
1.16941529235382	1.924274\\
1.17141429285357	1.922533\\
1.17341329335332	1.920785\\
1.17541229385307	1.919029\\
1.17741129435282	1.917264\\
1.17941029485257	1.915492\\
1.18140929535232	1.913711\\
1.18340829585207	1.911922\\
1.18540729635182	1.910126\\
1.18740629685157	1.908321\\
1.18940529735132	1.906509\\
1.19140429785107	1.904688\\
1.19340329835082	1.90286\\
1.19540229885057	1.901024\\
1.19740129935032	1.89918\\
1.19940029985008	1.897328\\
1.20139930034983	1.895469\\
1.20339830084958	1.893601\\
1.20539730134933	1.891727\\
1.20739630184908	1.889844\\
1.20939530234883	1.887954\\
1.21139430284858	1.886056\\
1.21339330334833	1.884151\\
1.21539230384808	1.882238\\
1.21739130434783	1.880318\\
1.21939030484758	1.87839\\
1.22138930534733	1.876455\\
1.22338830584708	1.874513\\
1.22538730634683	1.872563\\
1.22738630684658	1.870606\\
1.22938530734633	1.868641\\
1.23138430784608	1.86667\\
1.23338330834583	1.864691\\
1.23538230884558	1.862705\\
1.23738130934533	1.860712\\
1.23938030984508	1.858712\\
1.24137931034483	1.856705\\
1.24337831084458	1.854691\\
1.24537731134433	1.852669\\
1.24737631184408	1.850641\\
1.24937531234383	1.848606\\
1.25137431284358	1.846565\\
1.25337331334333	1.844516\\
1.25537231384308	1.842461\\
1.25737131434283	1.840399\\
1.25937031484258	1.83833\\
1.26136931534233	1.836255\\
1.26336831584208	1.834173\\
1.26536731634183	1.832085\\
1.26736631684158	1.82999\\
1.26936531734133	1.827889\\
1.27136431784108	1.825781\\
1.27336331834083	1.823667\\
1.27536231884058	1.821547\\
1.27736131934033	1.81942\\
1.27936031984008	1.817287\\
1.28135932033983	1.815148\\
1.28335832083958	1.813003\\
1.28535732133933	1.810852\\
1.28735632183908	1.808695\\
1.28935532233883	1.806532\\
1.29135432283858	1.804363\\
1.29335332333833	1.802188\\
1.29535232383808	1.800007\\
1.29735132433783	1.79782\\
1.29935032483758	1.795628\\
1.30134932533733	1.79343\\
1.30334832583708	1.791226\\
1.30534732633683	1.789017\\
1.30734632683658	1.786802\\
1.30934532733633	1.784582\\
1.31134432783608	1.782356\\
1.31334332833583	1.780125\\
1.31534232883558	1.777889\\
1.31734132933533	1.775647\\
1.31934032983508	1.7734\\
1.32133933033483	1.771148\\
1.32333833083458	1.768891\\
1.32533733133433	1.766629\\
1.32733633183408	1.764362\\
1.32933533233383	1.76209\\
1.33133433283358	1.759813\\
1.33333333333333	1.757531\\
1.33533233383308	1.755244\\
1.33733133433283	1.752953\\
1.33933033483258	1.750657\\
1.34132933533233	1.748357\\
1.34332833583208	1.746051\\
1.34532733633183	1.743742\\
1.34732633683158	1.741428\\
1.34932533733133	1.739109\\
1.35132433783108	1.736787\\
1.35332333833083	1.73446\\
1.35532233883058	1.732128\\
1.35732133933033	1.729793\\
1.35932033983009	1.727454\\
1.36131934032984	1.72511\\
1.36331834082959	1.722763\\
1.36531734132934	1.720411\\
1.36731634182909	1.718056\\
1.36931534232884	1.715697\\
1.37131434282859	1.713335\\
1.37331334332834	1.710968\\
1.37531234382809	1.708598\\
1.37731134432784	1.706225\\
1.37931034482759	1.703848\\
1.38130934532734	1.701468\\
1.38330834582709	1.699084\\
1.38530734632684	1.696697\\
1.38730634682659	1.694307\\
1.38930534732634	1.691913\\
1.39130434782609	1.689517\\
1.39330334832584	1.687117\\
1.39530234882559	1.684715\\
1.39730134932534	1.682309\\
1.39930034982509	1.679901\\
1.40129935032484	1.67749\\
1.40329835082459	1.675076\\
1.40529735132434	1.67266\\
1.40729635182409	1.670241\\
1.40929535232384	1.667819\\
1.41129435282359	1.665395\\
1.41329335332334	1.662969\\
1.41529235382309	1.66054\\
1.41729135432284	1.658109\\
1.41929035482259	1.655676\\
1.42128935532234	1.653241\\
1.42328835582209	1.650804\\
1.42528735632184	1.648364\\
1.42728635682159	1.645923\\
1.42928535732134	1.64348\\
1.43128435782109	1.641035\\
1.43328335832084	1.638588\\
1.43528235882059	1.63614\\
1.43728135932034	1.63369\\
1.43928035982009	1.631239\\
1.44127936031984	1.628786\\
1.44327836081959	1.626332\\
1.44527736131934	1.623876\\
1.44727636181909	1.621419\\
1.44927536231884	1.618961\\
1.45127436281859	1.616502\\
1.45327336331834	1.614042\\
1.45527236381809	1.61158\\
1.45727136431784	1.609118\\
1.45927036481759	1.606655\\
1.46126936531734	1.604191\\
1.46326836581709	1.601727\\
1.46526736631684	1.599262\\
1.46726636681659	1.596796\\
1.46926536731634	1.59433\\
1.47126436781609	1.591863\\
1.47326336831584	1.589396\\
1.47526236881559	1.586928\\
1.47726136931534	1.584461\\
1.47926036981509	1.581993\\
1.48125937031484	1.579525\\
1.48325837081459	1.577057\\
1.48525737131434	1.574589\\
1.48725637181409	1.572122\\
1.48925537231384	1.569654\\
1.49125437281359	1.567187\\
1.49325337331334	1.56472\\
1.49525237381309	1.562253\\
1.49725137431284	1.559787\\
1.49925037481259	1.557321\\
1.50124937531234	1.554856\\
1.50324837581209	1.552391\\
1.50524737631184	1.549928\\
1.50724637681159	1.547465\\
1.50924537731134	1.545002\\
1.51124437781109	1.542541\\
1.51324337831084	1.540081\\
1.51524237881059	1.537622\\
1.51724137931034	1.535164\\
1.51924037981009	1.532707\\
1.52123938030985	1.530251\\
1.5232383808096	1.527797\\
1.52523738130935	1.525344\\
1.5272363818091	1.522893\\
1.52923538230885	1.520443\\
1.5312343828086	1.517995\\
1.53323338330835	1.515548\\
1.5352323838081	1.513104\\
1.53723138430785	1.510661\\
1.5392303848076	1.508219\\
1.54122938530735	1.50578\\
1.5432283858071	1.503343\\
1.54522738630685	1.500908\\
1.5472263868066	1.498475\\
1.54922538730635	1.496044\\
1.5512243878061	1.493615\\
1.55322338830585	1.491189\\
1.5552223888056	1.488765\\
1.55722138930535	1.486344\\
1.5592203898051	1.483925\\
1.56121939030485	1.481509\\
1.5632183908046	1.479095\\
1.56521739130435	1.476684\\
1.5672163918041	1.474276\\
1.56921539230385	1.471871\\
1.5712143928036	1.469468\\
1.57321339330335	1.467069\\
1.5752123938031	1.464672\\
1.57721139430285	1.462279\\
1.5792103948026	1.459889\\
1.58120939530235	1.457502\\
1.5832083958021	1.455119\\
1.58520739630185	1.452738\\
1.5872063968016	1.450362\\
1.58920539730135	1.447988\\
1.5912043978011	1.445619\\
1.59320339830085	1.443252\\
1.5952023988006	1.44089\\
1.59720139930035	1.438531\\
1.5992003998001	1.436176\\
1.60119940029985	1.433825\\
1.6031984007996	1.431478\\
1.60519740129935	1.429134\\
1.6071964017991	1.426795\\
1.60919540229885	1.42446\\
1.6111944027986	1.422129\\
1.61319340329835	1.419802\\
1.6151924037981	1.41748\\
1.61719140429785	1.415161\\
1.6191904047976	1.412848\\
1.62118940529735	1.410538\\
1.6231884057971	1.408233\\
1.62518740629685	1.405933\\
1.6271864067966	1.403637\\
1.62918540729635	1.401346\\
1.6311844077961	1.39906\\
1.63318340829585	1.396778\\
1.6351824087956	1.394501\\
1.63718140929535	1.392229\\
1.6391804097951	1.389962\\
1.64117941029485	1.3877\\
1.6431784107946	1.385444\\
1.64517741129435	1.383192\\
1.6471764117941	1.380945\\
1.64917541229385	1.378704\\
1.6511744127936	1.376468\\
1.65317341329335	1.374237\\
1.6551724137931	1.372011\\
1.65717141429285	1.369791\\
1.6591704147926	1.367577\\
1.66116941529235	1.365368\\
1.6631684157921	1.363165\\
1.66516741629185	1.360967\\
1.6671664167916	1.358775\\
1.66916541729135	1.356588\\
1.6711644177911	1.354408\\
1.67316341829085	1.352233\\
1.6751624187906	1.350064\\
1.67716141929035	1.347901\\
1.6791604197901	1.345744\\
1.68115942028985	1.343593\\
1.68315842078961	1.341449\\
1.68515742128936	1.33931\\
1.68715642178911	1.337177\\
1.68915542228886	1.335051\\
1.69115442278861	1.332931\\
1.69315342328836	1.330817\\
1.69515242378811	1.32871\\
1.69715142428786	1.326609\\
1.69915042478761	1.324514\\
1.70114942528736	1.322426\\
1.70314842578711	1.320345\\
1.70514742628686	1.31827\\
1.70714642678661	1.316201\\
1.70914542728636	1.31414\\
1.71114442778611	1.312085\\
1.71314342828586	1.310036\\
1.71514242878561	1.307995\\
1.71714142928536	1.30596\\
1.71914042978511	1.303933\\
1.72113943028486	1.301912\\
1.72313843078461	1.299898\\
1.72513743128436	1.297891\\
1.72713643178411	1.295891\\
1.72913543228386	1.293899\\
1.73113443278361	1.291913\\
1.73313343328336	1.289935\\
1.73513243378311	1.287963\\
1.73713143428286	1.285999\\
1.73913043478261	1.284042\\
1.74112943528236	1.282093\\
1.74312843578211	1.280151\\
1.74512743628186	1.278216\\
1.74712643678161	1.276289\\
1.74912543728136	1.274369\\
1.75112443778111	1.272456\\
1.75312343828086	1.270551\\
1.75512243878061	1.268654\\
1.75712143928036	1.266764\\
1.75912043978011	1.264882\\
1.76111944027986	1.263007\\
1.76311844077961	1.261141\\
1.76511744127936	1.259281\\
1.76711644177911	1.25743\\
1.76911544227886	1.255587\\
1.77111444277861	1.253751\\
1.77311344327836	1.251923\\
1.77511244377811	1.250103\\
1.77711144427786	1.248291\\
1.77911044477761	1.246486\\
1.78110944527736	1.24469\\
1.78310844577711	1.242902\\
1.78510744627686	1.241121\\
1.78710644677661	1.239349\\
1.78910544727636	1.237585\\
1.79110444777611	1.235829\\
1.79310344827586	1.234081\\
1.79510244877561	1.232341\\
1.79710144927536	1.23061\\
1.79910044977511	1.228886\\
1.80109945027486	1.227171\\
1.80309845077461	1.225464\\
1.80509745127436	1.223766\\
1.80709645177411	1.222075\\
1.80909545227386	1.220393\\
1.81109445277361	1.21872\\
1.81309345327336	1.217054\\
1.81509245377311	1.215398\\
1.81709145427286	1.213749\\
1.81909045477261	1.212109\\
1.82108945527236	1.210478\\
1.82308845577211	1.208854\\
1.82508745627186	1.20724\\
1.82708645677161	1.205634\\
1.82908545727136	1.204036\\
1.83108445777111	1.202447\\
1.83308345827086	1.200867\\
1.83508245877061	1.199295\\
1.83708145927036	1.197732\\
1.83908045977011	1.196178\\
1.84107946026987	1.194632\\
1.84307846076962	1.193095\\
1.84507746126937	1.191566\\
1.84707646176912	1.190046\\
1.84907546226887	1.188535\\
1.85107446276862	1.187033\\
1.85307346326837	1.185539\\
1.85507246376812	1.184054\\
1.85707146426787	1.182578\\
1.85907046476762	1.181111\\
1.86106946526737	1.179652\\
1.86306846576712	1.178203\\
1.86506746626687	1.176762\\
1.86706646676662	1.17533\\
1.86906546726637	1.173907\\
1.87106446776612	1.172492\\
1.87306346826587	1.171087\\
1.87506246876562	1.16969\\
1.87706146926537	1.168303\\
1.87906046976512	1.166924\\
1.88105947026487	1.165554\\
1.88305847076462	1.164193\\
1.88505747126437	1.162841\\
1.88705647176412	1.161498\\
1.88905547226387	1.160164\\
1.89105447276362	1.158839\\
1.89305347326337	1.157523\\
1.89505247376312	1.156216\\
1.89705147426287	1.154917\\
1.89905047476262	1.153628\\
1.90104947526237	1.152348\\
1.90304847576212	1.151076\\
1.90504747626187	1.149814\\
1.90704647676162	1.148561\\
1.90904547726137	1.147316\\
1.91104447776112	1.146081\\
1.91304347826087	1.144855\\
1.91504247876062	1.143637\\
1.91704147926037	1.142429\\
1.91904047976012	1.141229\\
1.92103948025987	1.140039\\
1.92303848075962	1.138858\\
1.92503748125937	1.137685\\
1.92703648175912	1.136522\\
1.92903548225887	1.135368\\
1.93103448275862	1.134222\\
1.93303348325837	1.133086\\
1.93503248375812	1.131959\\
1.93703148425787	1.130841\\
1.93903048475762	1.129731\\
1.94102948525737	1.128631\\
1.94302848575712	1.127539\\
1.94502748625687	1.126457\\
1.94702648675662	1.125384\\
1.94902548725637	1.124319\\
1.95102448775612	1.123264\\
1.95302348825587	1.122217\\
1.95502248875562	1.12118\\
1.95702148925537	1.120151\\
1.95902048975512	1.119131\\
1.96101949025487	1.118121\\
1.96301849075462	1.117119\\
1.96501749125437	1.116126\\
1.96701649175412	1.115142\\
1.96901549225387	1.114167\\
1.97101449275362	1.113201\\
1.97301349325337	1.112243\\
1.97501249375312	1.111295\\
1.97701149425287	1.110355\\
1.97901049475262	1.109424\\
1.98100949525237	1.108502\\
1.98300849575212	1.107589\\
1.98500749625187	1.106685\\
1.98700649675162	1.105789\\
1.98900549725137	1.104902\\
1.99100449775112	1.104024\\
1.99300349825087	1.103155\\
1.99500249875062	1.102294\\
1.99700149925037	1.101442\\
1.99900049975012	1.100599\\
2.00099950024988	1.099764\\
2.00299850074963	1.098938\\
2.00499750124938	1.098121\\
2.00699650174913	1.097313\\
2.00899550224888	1.096513\\
2.01099450274863	1.095721\\
2.01299350324838	1.094938\\
2.01499250374813	1.094164\\
2.01699150424788	1.093398\\
2.01899050474763	1.092641\\
2.02098950524738	1.091892\\
2.02298850574713	1.091151\\
2.02498750624688	1.090419\\
2.02698650674663	1.089696\\
2.02898550724638	1.08898\\
2.03098450774613	1.088274\\
2.03298350824588	1.087575\\
2.03498250874563	1.086885\\
2.03698150924538	1.086203\\
2.03898050974513	1.085529\\
2.04097951024488	1.084864\\
2.04297851074463	1.084206\\
2.04497751124438	1.083557\\
2.04697651174413	1.082916\\
2.04897551224388	1.082284\\
2.05097451274363	1.081659\\
2.05297351324338	1.081042\\
2.05497251374313	1.080433\\
2.05697151424288	1.079833\\
2.05897051474263	1.07924\\
2.06096951524238	1.078655\\
2.06296851574213	1.078078\\
2.06496751624188	1.077509\\
2.06696651674163	1.076948\\
2.06896551724138	1.076395\\
2.07096451774113	1.075849\\
2.07296351824088	1.075311\\
2.07496251874063	1.074781\\
2.07696151924038	1.074258\\
2.07896051974013	1.073744\\
2.08095952023988	1.073236\\
2.08295852073963	1.072736\\
2.08495752123938	1.072244\\
2.08695652173913	1.071759\\
2.08895552223888	1.071282\\
2.09095452273863	1.070812\\
2.09295352323838	1.070349\\
2.09495252373813	1.069894\\
2.09695152423788	1.069446\\
2.09895052473763	1.069005\\
2.10094952523738	1.068572\\
2.10294852573713	1.068145\\
2.10494752623688	1.067726\\
2.10694652673663	1.067313\\
2.10894552723638	1.066908\\
2.11094452773613	1.06651\\
2.11294352823588	1.066118\\
2.11494252873563	1.065734\\
2.11694152923538	1.065356\\
2.11894052973513	1.064985\\
2.12093953023488	1.064621\\
2.12293853073463	1.064263\\
2.12493753123438	1.063913\\
2.12693653173413	1.063568\\
2.12893553223388	1.063231\\
2.13093453273363	1.0629\\
2.13293353323338	1.062575\\
2.13493253373313	1.062257\\
2.13693153423288	1.061945\\
2.13893053473263	1.061639\\
2.14092953523238	1.061339\\
2.14292853573213	1.061046\\
2.14492753623188	1.060759\\
2.14692653673163	1.060478\\
2.14892553723138	1.060203\\
2.15092453773113	1.059934\\
2.15292353823088	1.059671\\
2.15492253873063	1.059414\\
2.15692153923038	1.059162\\
2.15892053973013	1.058917\\
2.16091954022989	1.058677\\
2.16291854072964	1.058442\\
2.16491754122939	1.058214\\
2.16691654172914	1.057991\\
2.16891554222889	1.057773\\
2.17091454272864	1.057561\\
2.17291354322839	1.057354\\
2.17491254372814	1.057152\\
2.17691154422789	1.056956\\
2.17891054472764	1.056764\\
2.18090954522739	1.056578\\
2.18290854572714	1.056397\\
2.18490754622689	1.056221\\
2.18690654672664	1.05605\\
2.18890554722639	1.055884\\
2.19090454772614	1.055722\\
2.19290354822589	1.055566\\
2.19490254872564	1.055413\\
2.19690154922539	1.055266\\
2.19890054972514	1.055123\\
2.20089955022489	1.054985\\
2.20289855072464	1.054851\\
2.20489755122439	1.054721\\
2.20689655172414	1.054596\\
2.20889555222389	1.054474\\
2.21089455272364	1.054357\\
2.21289355322339	1.054245\\
2.21489255372314	1.054136\\
2.21689155422289	1.054031\\
2.21889055472264	1.05393\\
2.22088955522239	1.053832\\
2.22288855572214	1.053739\\
2.22488755622189	1.053649\\
2.22688655672164	1.053563\\
2.22888555722139	1.05348\\
2.23088455772114	1.053401\\
2.23288355822089	1.053325\\
2.23488255872064	1.053252\\
2.23688155922039	1.053183\\
2.23888055972014	1.053117\\
2.24087956021989	1.053054\\
2.24287856071964	1.052994\\
2.24487756121939	1.052937\\
2.24687656171914	1.052883\\
2.24887556221889	1.052832\\
2.25087456271864	1.052783\\
2.25287356321839	1.052737\\
2.25487256371814	1.052694\\
2.25687156421789	1.052653\\
2.25887056471764	1.052615\\
2.26086956521739	1.052579\\
2.26286856571714	1.052545\\
2.26486756621689	1.052514\\
2.26686656671664	1.052485\\
2.26886556721639	1.052458\\
2.27086456771614	1.052432\\
2.27286356821589	1.052409\\
2.27486256871564	1.052388\\
2.27686156921539	1.052368\\
2.27886056971514	1.05235\\
2.28085957021489	1.052334\\
2.28285857071464	1.052319\\
2.28485757121439	1.052306\\
2.28685657171414	1.052294\\
2.28885557221389	1.052284\\
2.29085457271364	1.052274\\
2.29285357321339	1.052266\\
2.29485257371314	1.052259\\
2.29685157421289	1.052253\\
2.29885057471264	1.052248\\
2.30084957521239	1.052244\\
2.30284857571214	1.05224\\
2.30484757621189	1.052237\\
2.30684657671164	1.052235\\
2.30884557721139	1.052233\\
2.31084457771114	1.052232\\
2.31284357821089	1.052231\\
2.31484257871064	1.052231\\
2.31684157921039	1.052231\\
2.31884057971015	1.052231\\
2.3208395802099	1.052231\\
2.32283858070965	1.052231\\
2.3248375812094	1.05223\\
2.32683658170915	1.05223\\
2.3288355822089	1.05223\\
2.33083458270865	1.052229\\
2.3328335832084	1.052228\\
2.33483258370815	1.052226\\
2.3368315842079	1.052224\\
2.33883058470765	1.052221\\
2.3408295852074	1.052218\\
2.34282858570715	1.052213\\
2.3448275862069	1.052208\\
2.34682658670665	1.052202\\
2.3488255872064	1.052195\\
2.35082458770615	1.052187\\
2.3528235882059	1.052177\\
2.35482258870565	1.052167\\
2.3568215892054	1.052155\\
2.35882058970515	1.052142\\
2.3608195902049	1.052127\\
2.36281859070465	1.05211\\
2.3648175912044	1.052092\\
2.36681659170415	1.052073\\
2.3688155922039	1.052051\\
2.37081459270365	1.052028\\
2.3728135932034	1.052003\\
2.37481259370315	1.051976\\
2.3768115942029	1.051946\\
2.37881059470265	1.051915\\
2.3808095952024	1.051881\\
2.38280859570215	1.051845\\
2.3848075962019	1.051807\\
2.38680659670165	1.051766\\
2.3888055972014	1.051723\\
2.39080459770115	1.051677\\
2.3928035982009	1.051628\\
2.39480259870065	1.051577\\
2.3968015992004	1.051523\\
2.39880059970015	1.051465\\
2.4007996001999	1.051405\\
2.40279860069965	1.051342\\
2.4047976011994	1.051276\\
2.40679660169915	1.051207\\
2.4087956021989	1.051134\\
2.41079460269865	1.051058\\
2.4127936031984	1.050979\\
2.41479260369815	1.050896\\
2.4167916041979	1.05081\\
2.41879060469765	1.05072\\
2.4207896051974	1.050626\\
2.42278860569715	1.050529\\
2.4247876061969	1.050427\\
2.42678660669665	1.050322\\
2.4287856071964	1.050213\\
2.43078460769615	1.0501\\
2.4327836081959	1.049983\\
2.43478260869565	1.049862\\
2.4367816091954	1.049736\\
2.43878060969515	1.049607\\
2.4407796101949	1.049472\\
2.44277861069465	1.049334\\
2.4447776111944	1.049191\\
2.44677661169415	1.049043\\
2.4487756121939	1.048891\\
2.45077461269365	1.048734\\
2.4527736131934	1.048573\\
2.45477261369315	1.048406\\
2.4567716141929	1.048235\\
2.45877061469265	1.048059\\
2.4607696151924	1.047877\\
2.46276861569215	1.047691\\
2.4647676161919	1.0475\\
2.46676661669165	1.047303\\
2.4687656171914	1.047101\\
2.47076461769115	1.046894\\
2.4727636181909	1.046682\\
2.47476261869065	1.046464\\
2.4767616191904	1.046241\\
2.47876061969016	1.046012\\
2.48075962018991	1.045777\\
2.48275862068966	1.045537\\
2.48475762118941	1.045291\\
2.48675662168916	1.04504\\
2.48875562218891	1.044783\\
2.49075462268866	1.044519\\
2.49275362318841	1.04425\\
2.49475262368816	1.043975\\
2.49675162418791	1.043694\\
2.49875062468766	1.043406\\
2.50074962518741	1.043113\\
2.50274862568716	1.042813\\
2.50474762618691	1.042508\\
2.50674662668666	1.042195\\
2.50874562718641	1.041877\\
2.51074462768616	1.041552\\
2.51274362818591	1.041221\\
2.51474262868566	1.040883\\
2.51674162918541	1.040538\\
2.51874062968516	1.040187\\
2.52073963018491	1.03983\\
2.52273863068466	1.039465\\
2.52473763118441	1.039094\\
2.52673663168416	1.038716\\
2.52873563218391	1.038331\\
2.53073463268366	1.03794\\
2.53273363318341	1.037541\\
2.53473263368316	1.037136\\
2.53673163418291	1.036723\\
2.53873063468266	1.036304\\
2.54072963518241	1.035877\\
2.54272863568216	1.035443\\
2.54472763618191	1.035002\\
2.54672663668166	1.034554\\
2.54872563718141	1.034098\\
2.55072463768116	1.033635\\
2.55272363818091	1.033165\\
2.55472263868066	1.032688\\
2.55672163918041	1.032203\\
2.55872063968016	1.03171\\
2.56071964017991	1.03121\\
2.56271864067966	1.030703\\
2.56471764117941	1.030187\\
2.56671664167916	1.029665\\
2.56871564217891	1.029134\\
2.57071464267866	1.028596\\
2.57271364317841	1.02805\\
2.57471264367816	1.027497\\
2.57671164417791	1.026935\\
2.57871064467766	1.026366\\
2.58070964517741	1.025789\\
2.58270864567716	1.025204\\
2.58470764617691	1.024611\\
2.58670664667666	1.02401\\
2.58870564717641	1.023401\\
2.59070464767616	1.022784\\
2.59270364817591	1.022159\\
2.59470264867566	1.021526\\
2.59670164917541	1.020885\\
2.59870064967516	1.020235\\
2.60069965017491	1.019578\\
2.60269865067466	1.018912\\
2.60469765117441	1.018238\\
2.60669665167416	1.017556\\
2.60869565217391	1.016866\\
2.61069465267366	1.016167\\
2.61269365317341	1.01546\\
2.61469265367316	1.014744\\
2.61669165417291	1.01402\\
2.61869065467266	1.013288\\
2.62068965517241	1.012548\\
2.62268865567216	1.011798\\
2.62468765617191	1.011041\\
2.62668665667166	1.010275\\
2.62868565717141	1.0095\\
2.63068465767116	1.008717\\
2.63268365817091	1.007925\\
2.63468265867066	1.007125\\
2.63668165917041	1.006316\\
2.63868065967017	1.005498\\
2.64067966016992	1.004672\\
2.64267866066967	1.003837\\
2.64467766116942	1.002994\\
2.64667666166917	1.002142\\
2.64867566216892	1.001281\\
2.65067466266867	1.000411\\
2.65267366316842	0.999533\\
2.65467266366817	0.998646\\
2.65667166416792	0.99775\\
2.65867066466767	0.996845\\
2.66066966516742	0.995932\\
2.66266866566717	0.99501\\
2.66466766616692	0.994079\\
2.66666666666667	0.993139\\
2.66866566716642	0.99219\\
2.67066466766617	0.991232\\
2.67266366816592	0.990266\\
2.67466266866567	0.98929\\
2.67666166916542	0.988306\\
2.67866066966517	0.987313\\
2.68065967016492	0.986311\\
2.68265867066467	0.9853\\
2.68465767116442	0.98428\\
2.68665667166417	0.983251\\
2.68865567216392	0.982213\\
2.69065467266367	0.981166\\
2.69265367316342	0.980111\\
2.69465267366317	0.979046\\
2.69665167416292	0.977972\\
2.69865067466267	0.97689\\
2.70064967516242	0.975798\\
2.70264867566217	0.974697\\
2.70464767616192	0.973588\\
2.70664667666167	0.972469\\
2.70864567716142	0.971342\\
2.71064467766117	0.970205\\
2.71264367816092	0.969059\\
2.71464267866067	0.967905\\
2.71664167916042	0.966741\\
2.71864067966017	0.965569\\
2.72063968015992	0.964387\\
2.72263868065967	0.963196\\
2.72463768115942	0.961997\\
2.72663668165917	0.960788\\
2.72863568215892	0.959571\\
2.73063468265867	0.958344\\
2.73263368315842	0.957108\\
2.73463268365817	0.955864\\
2.73663168415792	0.95461\\
2.73863068465767	0.953347\\
2.74062968515742	0.952076\\
2.74262868565717	0.950795\\
2.74462768615692	0.949506\\
2.74662668665667	0.948207\\
2.74862568715642	0.9469\\
2.75062468765617	0.945583\\
2.75262368815592	0.944258\\
2.75462268865567	0.942923\\
2.75662168915542	0.94158\\
2.75862068965517	0.940228\\
2.76061969015492	0.938867\\
2.76261869065467	0.937497\\
2.76461769115442	0.936118\\
2.76661669165417	0.93473\\
2.76861569215392	0.933333\\
2.77061469265367	0.931927\\
2.77261369315342	0.930513\\
2.77461269365317	0.929089\\
2.77661169415292	0.927657\\
2.77861069465267	0.926216\\
2.78060969515242	0.924766\\
2.78260869565217	0.923307\\
2.78460769615192	0.921839\\
2.78660669665167	0.920363\\
2.78860569715142	0.918878\\
2.79060469765117	0.917384\\
2.79260369815092	0.915882\\
2.79460269865067	0.91437\\
2.79660169915043	0.91285\\
2.79860069965018	0.911321\\
2.80059970014993	0.909784\\
2.80259870064968	0.908238\\
2.80459770114943	0.906683\\
2.80659670164918	0.90512\\
2.80859570214893	0.903548\\
2.81059470264868	0.901967\\
2.81259370314843	0.900378\\
2.81459270364818	0.89878\\
2.81659170414793	0.897174\\
2.81859070464768	0.895559\\
2.82058970514743	0.893936\\
2.82258870564718	0.892304\\
2.82458770614693	0.890664\\
2.82658670664668	0.889015\\
2.82858570714643	0.887358\\
2.83058470764618	0.885692\\
2.83258370814593	0.884018\\
2.83458270864568	0.882336\\
2.83658170914543	0.880645\\
2.83858070964518	0.878947\\
2.84057971014493	0.877239\\
2.84257871064468	0.875524\\
2.84457771114443	0.8738\\
2.84657671164418	0.872069\\
2.84857571214393	0.870329\\
2.85057471264368	0.86858\\
2.85257371314343	0.866824\\
2.85457271364318	0.86506\\
2.85657171414293	0.863287\\
2.85857071464268	0.861507\\
2.86056971514243	0.859718\\
2.86256871564218	0.857922\\
2.86456771614193	0.856117\\
2.86656671664168	0.854305\\
2.86856571714143	0.852485\\
2.87056471764118	0.850656\\
2.87256371814093	0.84882\\
2.87456271864068	0.846977\\
2.87656171914043	0.845125\\
2.87856071964018	0.843266\\
2.88055972013993	0.841398\\
2.88255872063968	0.839524\\
2.88455772113943	0.837641\\
2.88655672163918	0.835751\\
2.88855572213893	0.833854\\
2.89055472263868	0.831949\\
2.89255372313843	0.830036\\
2.89455272363818	0.828116\\
2.89655172413793	0.826188\\
2.89855072463768	0.824253\\
2.90054972513743	0.822311\\
2.90254872563718	0.820361\\
2.90454772613693	0.818404\\
2.90654672663668	0.81644\\
2.90854572713643	0.814468\\
2.91054472763618	0.81249\\
2.91254372813593	0.810504\\
2.91454272863568	0.808511\\
2.91654172913543	0.806511\\
2.91854072963518	0.804504\\
2.92053973013493	0.80249\\
2.92253873063468	0.800469\\
2.92453773113443	0.798441\\
2.92653673163418	0.796406\\
2.92853573213393	0.794364\\
2.93053473263368	0.792316\\
2.93253373313343	0.790261\\
2.93453273363318	0.788199\\
2.93653173413293	0.78613\\
2.93853073463268	0.784055\\
2.94052973513243	0.781974\\
2.94252873563218	0.779885\\
2.94452773613193	0.77779\\
2.94652673663168	0.775689\\
2.94852573713143	0.773582\\
2.95052473763118	0.771468\\
2.95252373813093	0.769348\\
2.95452273863068	0.767221\\
2.95652173913043	0.765088\\
2.95852073963018	0.762949\\
2.96051974012994	0.760804\\
2.96251874062969	0.758653\\
2.96451774112944	0.756496\\
2.96651674162919	0.754333\\
2.96851574212894	0.752164\\
2.97051474262869	0.749989\\
2.97251374312844	0.747808\\
2.97451274362819	0.745622\\
2.97651174412794	0.74343\\
2.97851074462769	0.741232\\
2.98050974512744	0.739028\\
2.98250874562719	0.736819\\
2.98450774612694	0.734604\\
2.98650674662669	0.732384\\
2.98850574712644	0.730158\\
2.99050474762619	0.727928\\
2.99250374812594	0.725691\\
2.99450274862569	0.72345\\
2.99650174912544	0.721203\\
2.99850074962519	0.718951\\
3.00049975012494	0.716694\\
3.00249875062469	0.714432\\
3.00449775112444	0.712165\\
3.00649675162419	0.709893\\
3.00849575212394	0.707616\\
3.01049475262369	0.705334\\
3.01249375312344	0.703048\\
3.01449275362319	0.700756\\
3.01649175412294	0.69846\\
3.01849075462269	0.69616\\
3.02048975512244	0.693855\\
3.02248875562219	0.691545\\
3.02448775612194	0.689231\\
3.02648675662169	0.686913\\
3.02848575712144	0.68459\\
3.03048475762119	0.682263\\
3.03248375812094	0.679932\\
3.03448275862069	0.677597\\
3.03648175912044	0.675258\\
3.03848075962019	0.672914\\
3.04047976011994	0.670567\\
3.04247876061969	0.668216\\
3.04447776111944	0.66586\\
3.04647676161919	0.663502\\
3.04847576211894	0.661139\\
3.05047476261869	0.658773\\
3.05247376311844	0.656403\\
3.05447276361819	0.654029\\
3.05647176411794	0.651652\\
3.05847076461769	0.649272\\
3.06046976511744	0.646888\\
3.06246876561719	0.644502\\
3.06446776611694	0.642111\\
3.06646676661669	0.639718\\
3.06846576711644	0.637322\\
3.07046476761619	0.634922\\
3.07246376811594	0.63252\\
3.07446276861569	0.630114\\
3.07646176911544	0.627706\\
3.07846076961519	0.625295\\
3.08045977011494	0.622881\\
3.08245877061469	0.620465\\
3.08445777111444	0.618046\\
3.08645677161419	0.615625\\
3.08845577211394	0.613201\\
3.09045477261369	0.610774\\
3.09245377311344	0.608346\\
3.09445277361319	0.605915\\
3.09645177411294	0.603482\\
3.09845077461269	0.601046\\
3.10044977511244	0.598609\\
3.10244877561219	0.59617\\
3.10444777611194	0.593729\\
3.10644677661169	0.591285\\
3.10844577711144	0.588841\\
3.11044477761119	0.586394\\
3.11244377811094	0.583946\\
3.11444277861069	0.581496\\
3.11644177911044	0.579044\\
3.11844077961019	0.576591\\
3.12043978010994	0.574137\\
3.12243878060969	0.571682\\
3.12443778110945	0.569225\\
3.1264367816092	0.566767\\
3.12843578210895	0.564307\\
3.1304347826087	0.561847\\
3.13243378310845	0.559386\\
3.1344327836082	0.556924\\
3.13643178410795	0.554461\\
3.1384307846077	0.551997\\
3.14042978510745	0.549533\\
3.1424287856072	0.547067\\
3.14442778610695	0.544602\\
3.1464267866067	0.542135\\
3.14842578710645	0.539669\\
3.1504247876062	0.537202\\
3.15242378810595	0.534734\\
3.1544227886057	0.532267\\
3.15642178910545	0.529799\\
3.1584207896052	0.527331\\
3.16041979010495	0.524863\\
3.1624187906047	0.522395\\
3.16441779110445	0.519927\\
3.1664167916042	0.51746\\
3.16841579210395	0.514992\\
3.1704147926037	0.512525\\
3.17241379310345	0.510059\\
3.1744127936032	0.507592\\
3.17641179410295	0.505127\\
3.1784107946027	0.502662\\
3.18040979510245	0.500197\\
3.1824087956022	0.497733\\
3.18440779610195	0.49527\\
3.1864067966017	0.492808\\
3.18840579710145	0.490347\\
3.1904047976012	0.487887\\
3.19240379810095	0.485428\\
3.1944027986007	0.48297\\
3.19640179910045	0.480513\\
3.1984007996002	0.478057\\
3.20039980009995	0.475603\\
3.2023988005997	0.47315\\
3.20439780109945	0.470699\\
3.2063968015992	0.468249\\
3.20839580209895	0.465801\\
3.2103948025987	0.463354\\
3.21239380309845	0.460909\\
3.2143928035982	0.458466\\
3.21639180409795	0.456025\\
3.2183908045977	0.453585\\
3.22038980509745	0.451148\\
3.2223888055972	0.448713\\
3.22438780609695	0.44628\\
3.2263868065967	0.443849\\
3.22838580709645	0.44142\\
3.2303848075962	0.438994\\
3.23238380809595	0.43657\\
3.2343828085957	0.434149\\
3.23638180909545	0.43173\\
3.2383808095952	0.429314\\
3.24037981009495	0.4269\\
3.2423788105947	0.424489\\
3.24437781109445	0.422081\\
3.2463768115942	0.419675\\
3.24837581209395	0.417273\\
3.2503748125937	0.414874\\
3.25237381309345	0.412477\\
3.2543728135932	0.410084\\
3.25637181409295	0.407694\\
3.2583708145927	0.405307\\
3.26036981509245	0.402923\\
3.2623688155922	0.400543\\
3.26436781609195	0.398166\\
3.2663668165917	0.395793\\
3.26836581709145	0.393423\\
3.2703648175912	0.391057\\
3.27236381809095	0.388694\\
3.2743628185907	0.386335\\
3.27636181909045	0.38398\\
3.2783608195902	0.381629\\
3.28035982008995	0.379282\\
3.2823588205897	0.376938\\
3.28435782108946	0.374599\\
3.28635682158921	0.372264\\
3.28835582208896	0.369933\\
3.29035482258871	0.367606\\
3.29235382308846	0.365283\\
3.29435282358821	0.362965\\
3.29635182408796	0.360651\\
3.29835082458771	0.358342\\
3.30034982508746	0.356037\\
3.30234882558721	0.353736\\
3.30434782608696	0.35144\\
3.30634682658671	0.349149\\
3.30834582708646	0.346863\\
3.31034482758621	0.344581\\
3.31234382808596	0.342304\\
3.31434282858571	0.340032\\
3.31634182908546	0.337765\\
3.31834082958521	0.335503\\
3.32033983008496	0.333246\\
3.32233883058471	0.330994\\
3.32433783108446	0.328748\\
3.32633683158421	0.326506\\
3.32833583208396	0.32427\\
3.33033483258371	0.322039\\
3.33233383308346	0.319814\\
3.33433283358321	0.317594\\
3.33633183408296	0.315379\\
3.33833083458271	0.31317\\
3.34032983508246	0.310966\\
3.34232883558221	0.308769\\
3.34432783608196	0.306576\\
3.34632683658171	0.30439\\
3.34832583708146	0.302209\\
3.35032483758121	0.300035\\
3.35232383808096	0.297866\\
3.35432283858071	0.295703\\
3.35632183908046	0.293546\\
3.35832083958021	0.291395\\
3.36031984007996	0.28925\\
3.36231884057971	0.287111\\
3.36431784107946	0.284978\\
3.36631684157921	0.282852\\
3.36831584207896	0.280732\\
3.37031484257871	0.278618\\
3.37231384307846	0.27651\\
3.37431284357821	0.274409\\
3.37631184407796	0.272314\\
3.37831084457771	0.270226\\
3.38030984507746	0.268145\\
3.38230884557721	0.26607\\
3.38430784607696	0.264001\\
3.38630684657671	0.261939\\
3.38830584707646	0.259884\\
3.39030484757621	0.257836\\
3.39230384807596	0.255795\\
3.39430284857571	0.25376\\
3.39630184907546	0.251732\\
3.39830084957521	0.249711\\
3.40029985007496	0.247697\\
3.40229885057471	0.24569\\
3.40429785107446	0.24369\\
3.40629685157421	0.241697\\
3.40829585207396	0.239712\\
3.41029485257371	0.237733\\
3.41229385307346	0.235762\\
3.41429285357321	0.233798\\
3.41629185407296	0.231841\\
3.41829085457271	0.229891\\
3.42028985507246	0.227949\\
3.42228885557221	0.226014\\
3.42428785607196	0.224086\\
3.42628685657171	0.222166\\
3.42828585707146	0.220254\\
3.43028485757121	0.218349\\
3.43228385807096	0.216451\\
3.43428285857071	0.214561\\
3.43628185907046	0.212679\\
3.43828085957021	0.210804\\
3.44027986006996	0.208938\\
3.44227886056971	0.207078\\
3.44427786106947	0.205227\\
3.44627686156922	0.203383\\
3.44827586206897	0.201547\\
3.45027486256872	0.199719\\
3.45227386306847	0.197899\\
3.45427286356822	0.196087\\
3.45627186406797	0.194282\\
3.45827086456772	0.192486\\
3.46026986506747	0.190697\\
3.46226886556722	0.188917\\
3.46426786606697	0.187145\\
3.46626686656672	0.18538\\
3.46826586706647	0.183624\\
3.47026486756622	0.181876\\
3.47226386806597	0.180136\\
3.47426286856572	0.178405\\
3.47626186906547	0.176681\\
3.47826086956522	0.174966\\
3.48025987006497	0.173259\\
3.48225887056472	0.17156\\
3.48425787106447	0.16987\\
3.48625687156422	0.168188\\
3.48825587206397	0.166514\\
3.49025487256372	0.164848\\
3.49225387306347	0.163191\\
3.49425287356322	0.161543\\
3.49625187406297	0.159903\\
3.49825087456272	0.158271\\
3.50024987506247	0.156648\\
3.50224887556222	0.155033\\
3.50424787606197	0.153427\\
3.50624687656172	0.151829\\
3.50824587706147	0.15024\\
3.51024487756122	0.14866\\
3.51224387806097	0.147088\\
3.51424287856072	0.145525\\
3.51624187906047	0.14397\\
3.51824087956022	0.142424\\
3.52023988005997	0.140887\\
3.52223888055972	0.139358\\
3.52423788105947	0.137838\\
3.52623688155922	0.136327\\
3.52823588205897	0.134824\\
3.53023488255872	0.133331\\
3.53223388305847	0.131846\\
3.53423288355822	0.130369\\
3.53623188405797	0.128902\\
3.53823088455772	0.127443\\
3.54022988505747	0.125994\\
3.54222888555722	0.124553\\
3.54422788605697	0.12312\\
3.54622688655672	0.121697\\
3.54822588705647	0.120283\\
3.55022488755622	0.118877\\
3.55222388805597	0.11748\\
3.55422288855572	0.116093\\
3.55622188905547	0.114714\\
3.55822088955522	0.113344\\
3.56021989005497	0.111983\\
3.56221889055472	0.110631\\
3.56421789105447	0.109287\\
3.56621689155422	0.107953\\
3.56821589205397	0.106628\\
3.57021489255372	0.105312\\
3.57221389305347	0.104004\\
3.57421289355322	0.102706\\
3.57621189405297	0.101416\\
3.57821089455272	0.100136\\
3.58020989505247	0.098864\\
3.58220889555222	0.097602\\
3.58420789605197	0.096348\\
3.58620689655172	0.095104\\
3.58820589705147	0.093868\\
3.59020489755122	0.092642\\
3.59220389805097	0.091424\\
3.59420289855072	0.090216\\
3.59620189905047	0.089017\\
3.59820089955022	0.087826\\
3.60019990004997	0.086645\\
3.60219890054973	0.085472\\
3.60419790104948	0.084309\\
3.60619690154923	0.083154\\
3.60819590204898	0.082009\\
3.61019490254873	0.080872\\
3.61219390304848	0.079745\\
3.61419290354823	0.078626\\
3.61619190404798	0.077517\\
3.61819090454773	0.076416\\
3.62018990504748	0.075325\\
3.62218890554723	0.074242\\
3.62418790604698	0.073169\\
3.62618690654673	0.072104\\
3.62818590704648	0.071049\\
3.63018490754623	0.070002\\
3.63218390804598	0.068964\\
3.63418290854573	0.067936\\
3.63618190904548	0.066916\\
3.63818090954523	0.065905\\
3.64017991004498	0.064903\\
3.64217891054473	0.06391\\
3.64417791104448	0.062926\\
3.64617691154423	0.061951\\
3.64817591204398	0.060984\\
3.65017491254373	0.060027\\
3.65217391304348	0.059078\\
3.65417291354323	0.058138\\
3.65617191404298	0.057207\\
3.65817091454273	0.056285\\
3.66016991504248	0.055372\\
3.66216891554223	0.054467\\
3.66416791604198	0.053572\\
3.66616691654173	0.052685\\
3.66816591704148	0.051806\\
3.67016491754123	0.050937\\
3.67216391804098	0.050076\\
3.67416291854073	0.049224\\
3.67616191904048	0.048381\\
3.67816091954023	0.047546\\
3.68015992003998	0.04672\\
3.68215892053973	0.045903\\
3.68415792103948	0.045094\\
3.68615692153923	0.044294\\
3.68815592203898	0.043502\\
3.69015492253873	0.042719\\
3.69215392303848	0.041945\\
3.69415292353823	0.041179\\
3.69615192403798	0.040421\\
3.69815092453773	0.039672\\
3.70014992503748	0.038931\\
3.70214892553723	0.038199\\
3.70414792603698	0.037476\\
3.70614692653673	0.03676\\
3.70814592703648	0.036053\\
3.71014492753623	0.035355\\
3.71214392803598	0.034664\\
3.71414292853573	0.033982\\
3.71614192903548	0.033308\\
3.71814092953523	0.032643\\
3.72013993003498	0.031985\\
3.72213893053473	0.031336\\
3.72413793103448	0.030695\\
3.72613693153423	0.030062\\
3.72813593203398	0.029437\\
3.73013493253373	0.028821\\
3.73213393303348	0.028212\\
3.73413293353323	0.027611\\
3.73613193403298	0.027018\\
3.73813093453273	0.026433\\
3.74012993503248	0.025856\\
3.74212893553223	0.025287\\
3.74412793603198	0.024726\\
3.74612693653173	0.024172\\
3.74812593703148	0.023627\\
3.75012493753123	0.023089\\
3.75212393803098	0.022558\\
3.75412293853073	0.022036\\
3.75612193903048	0.021521\\
3.75812093953023	0.021013\\
3.76011994002998	0.020513\\
3.76211894052974	0.020021\\
3.76411794102949	0.019536\\
3.76611694152924	0.019058\\
3.76811594202899	0.018588\\
3.77011494252874	0.018126\\
3.77211394302849	0.01767\\
3.77411294352824	0.017222\\
3.77611194402799	0.016781\\
3.77811094452774	0.016347\\
3.78010994502749	0.015921\\
3.78210894552724	0.015501\\
3.78410794602699	0.015089\\
3.78610694652674	0.014683\\
3.78810594702649	0.014285\\
3.79010494752624	0.013893\\
3.79210394802599	0.013509\\
3.79410294852574	0.013131\\
3.79610194902549	0.01276\\
3.79810094952524	0.012396\\
3.80009995002499	0.012038\\
3.80209895052474	0.011687\\
3.80409795102449	0.011343\\
3.80609695152424	0.011005\\
3.80809595202399	0.010674\\
3.81009495252374	0.010349\\
3.81209395302349	0.010031\\
3.81409295352324	0.009718\\
3.81609195402299	0.009413\\
3.81809095452274	0.009113\\
3.82008995502249	0.00882\\
3.82208895552224	0.008533\\
3.82408795602199	0.008252\\
3.82608695652174	0.007976\\
3.82808595702149	0.007707\\
3.83008495752124	0.007444\\
3.83208395802099	0.007187\\
3.83408295852074	0.006935\\
3.83608195902049	0.00669\\
3.83808095952024	0.00645\\
3.84007996001999	0.006215\\
3.84207896051974	0.005987\\
3.84407796101949	0.005763\\
3.84607696151924	0.005546\\
3.84807596201899	0.005333\\
3.85007496251874	0.005126\\
3.85207396301849	0.004924\\
3.85407296351824	0.004728\\
3.85607196401799	0.004537\\
3.85807096451774	0.00435\\
3.86006996501749	0.004169\\
3.86206896551724	0.003993\\
3.86406796601699	0.003822\\
3.86606696651674	0.003656\\
3.86806596701649	0.003494\\
3.87006496751624	0.003337\\
3.87206396801599	0.003185\\
3.87406296851574	0.003038\\
3.87606196901549	0.002895\\
3.87806096951524	0.002756\\
3.88005997001499	0.002622\\
3.88205897051474	0.002492\\
3.88405797101449	0.002367\\
3.88605697151424	0.002246\\
3.88805597201399	0.002129\\
3.89005497251374	0.002016\\
3.89205397301349	0.001907\\
3.89405297351324	0.001802\\
3.89605197401299	0.0017\\
3.89805097451274	0.001603\\
3.90004997501249	0.00151\\
3.90204897551224	0.00142\\
3.90404797601199	0.001333\\
3.90604697651174	0.001251\\
3.90804597701149	0.001171\\
3.91004497751124	0.001095\\
3.91204397801099	0.001023\\
3.91404297851074	0.000953\\
3.91604197901049	0.000887\\
3.91804097951024	0.000824\\
3.92003998000999	0.000764\\
3.92203898050975	0.000707\\
3.9240379810095	0.000653\\
3.92603698150925	0.000602\\
3.928035982009	0.000553\\
3.93003498250875	0.000507\\
3.9320339830085	0.000464\\
3.93403298350825	0.000423\\
3.936031984008	0.000385\\
3.93803098450775	0.000349\\
3.9400299850075	0.000315\\
3.94202898550725	0.000284\\
3.944027986007	0.000255\\
3.94602698650675	0.000227\\
3.9480259870065	0.000202\\
3.95002498750625	0.000179\\
3.952023988006	0.000157\\
3.95402298850575	0.000138\\
3.9560219890055	0.00012\\
3.95802098950525	0.000104\\
3.960019990005	8.9e-05\\
3.96201899050475	7.6e-05\\
3.9640179910045	6.4e-05\\
3.96601699150425	5.3e-05\\
3.968015992004	4.4e-05\\
3.97001499250375	3.6e-05\\
3.9720139930035	2.8e-05\\
3.97401299350325	2.2e-05\\
3.976011994003	1.7e-05\\
3.97801099450275	1.3e-05\\
3.9800099950025	9e-06\\
3.98200899550225	7e-06\\
3.984007996002	4e-06\\
3.98600699650175	3e-06\\
3.9880059970015	2e-06\\
3.99000499750125	1e-06\\
3.992003998001	0\\
3.99400299850075	0\\
3.9960019990005	0\\
3.99800099950025	0\\
4	-0\\
};
\addlegendentry{$\phi$};

\addplot [color=mycolor2,solid]
  table[row sep=crcr]{%
0	39.549404\\
0.00199900049975012	39.52971\\
0.00399800099950025	39.510017\\
0.00599700149925037	39.490326\\
0.0079960019990005	39.470634\\
0.00999500249875063	39.450944\\
0.0119940029985007	39.431255\\
0.0139930034982509	39.411567\\
0.015992003998001	39.391879\\
0.0179910044977511	39.372192\\
0.0199900049975013	39.352506\\
0.0219890054972514	39.332821\\
0.0239880059970015	39.313137\\
0.0259870064967516	39.293454\\
0.0279860069965017	39.273771\\
0.0299850074962519	39.254089\\
0.031984007996002	39.234408\\
0.0339830084957521	39.214728\\
0.0359820089955022	39.195049\\
0.0379810094952524	39.17537\\
0.0399800099950025	39.155693\\
0.0419790104947526	39.136016\\
0.0439780109945027	39.116339\\
0.0459770114942529	39.096664\\
0.047976011994003	39.076989\\
0.0499750124937531	39.057315\\
0.0519740129935032	39.037642\\
0.0539730134932534	39.01797\\
0.0559720139930035	38.998298\\
0.0579710144927536	38.978627\\
0.0599700149925037	38.958956\\
0.0619690154922539	38.939286\\
0.063968015992004	38.919617\\
0.0659670164917541	38.899949\\
0.0679660169915042	38.880281\\
0.0699650174912544	38.860614\\
0.0719640179910045	38.840947\\
0.0739630184907546	38.821281\\
0.0759620189905048	38.801616\\
0.0779610194902549	38.781951\\
0.079960019990005	38.762287\\
0.0819590204897551	38.742623\\
0.0839580209895052	38.72296\\
0.0859570214892554	38.703298\\
0.0879560219890055	38.683636\\
0.0899550224887556	38.663974\\
0.0919540229885057	38.644313\\
0.0939530234882559	38.624652\\
0.095952023988006	38.604992\\
0.0979510244877561	38.585333\\
0.0999500249875062	38.565673\\
0.101949025487256	38.546014\\
0.103948025987006	38.526356\\
0.105947026486757	38.506698\\
0.107946026986507	38.48704\\
0.109945027486257	38.467383\\
0.111944027986007	38.447726\\
0.113943028485757	38.428069\\
0.115942028985507	38.408412\\
0.117941029485257	38.388756\\
0.119940029985007	38.3691\\
0.121939030484758	38.349445\\
0.123938030984508	38.329789\\
0.125937031484258	38.310134\\
0.127936031984008	38.290479\\
0.129935032483758	38.270824\\
0.131934032983508	38.251169\\
0.133933033483258	38.231514\\
0.135932033983008	38.21186\\
0.137931034482759	38.192206\\
0.139930034982509	38.172551\\
0.141929035482259	38.152897\\
0.143928035982009	38.133243\\
0.145927036481759	38.113588\\
0.147926036981509	38.093934\\
0.149925037481259	38.07428\\
0.15192403798101	38.054626\\
0.15392303848076	38.034971\\
0.15592203898051	38.015317\\
0.15792103948026	37.995662\\
0.15992003998001	37.976007\\
0.16191904047976	37.956353\\
0.16391804097951	37.936698\\
0.16591704147926	37.917042\\
0.16791604197901	37.897387\\
0.169915042478761	37.877731\\
0.171914042978511	37.858075\\
0.173913043478261	37.838419\\
0.175912043978011	37.818763\\
0.177911044477761	37.799106\\
0.179910044977511	37.779449\\
0.181909045477261	37.759791\\
0.183908045977011	37.740133\\
0.185907046476762	37.720475\\
0.187906046976512	37.700816\\
0.189905047476262	37.681157\\
0.191904047976012	37.661498\\
0.193903048475762	37.641837\\
0.195902048975512	37.622177\\
0.197901049475262	37.602516\\
0.199900049975012	37.582854\\
0.201899050474763	37.563192\\
0.203898050974513	37.543529\\
0.205897051474263	37.523865\\
0.207896051974013	37.504201\\
0.209895052473763	37.484536\\
0.211894052973513	37.46487\\
0.213893053473263	37.445204\\
0.215892053973014	37.425537\\
0.217891054472764	37.405869\\
0.219890054972514	37.386201\\
0.221889055472264	37.366531\\
0.223888055972014	37.346861\\
0.225887056471764	37.32719\\
0.227886056971514	37.307518\\
0.229885057471264	37.287845\\
0.231884057971014	37.268172\\
0.233883058470765	37.248497\\
0.235882058970515	37.228821\\
0.237881059470265	37.209145\\
0.239880059970015	37.189467\\
0.241879060469765	37.169788\\
0.243878060969515	37.150109\\
0.245877061469265	37.130428\\
0.247876061969015	37.110746\\
0.249875062468766	37.091063\\
0.251874062968516	37.071379\\
0.253873063468266	37.051694\\
0.255872063968016	37.032007\\
0.257871064467766	37.012319\\
0.259870064967516	36.99263\\
0.261869065467266	36.97294\\
0.263868065967017	36.953249\\
0.265867066466767	36.933556\\
0.267866066966517	36.913862\\
0.269865067466267	36.894167\\
0.271864067966017	36.87447\\
0.273863068465767	36.854772\\
0.275862068965517	36.835073\\
0.277861069465267	36.815372\\
0.279860069965017	36.795669\\
0.281859070464768	36.775966\\
0.283858070964518	36.75626\\
0.285857071464268	36.736554\\
0.287856071964018	36.716846\\
0.289855072463768	36.697136\\
0.291854072963518	36.677424\\
0.293853073463268	36.657712\\
0.295852073963018	36.637997\\
0.297851074462769	36.618281\\
0.299850074962519	36.598563\\
0.301849075462269	36.578844\\
0.303848075962019	36.559123\\
0.305847076461769	36.5394\\
0.307846076961519	36.519676\\
0.309845077461269	36.49995\\
0.311844077961019	36.480222\\
0.31384307846077	36.460493\\
0.31584207896052	36.440761\\
0.31784107946027	36.421028\\
0.31984007996002	36.401293\\
0.32183908045977	36.381557\\
0.32383808095952	36.361818\\
0.32583708145927	36.342078\\
0.32783608195902	36.322335\\
0.329835082458771	36.302591\\
0.331834082958521	36.282845\\
0.333833083458271	36.263097\\
0.335832083958021	36.243348\\
0.337831084457771	36.223596\\
0.339830084957521	36.203842\\
0.341829085457271	36.184087\\
0.343828085957021	36.164329\\
0.345827086456772	36.144569\\
0.347826086956522	36.124808\\
0.349825087456272	36.105044\\
0.351824087956022	36.085278\\
0.353823088455772	36.065511\\
0.355822088955522	36.045741\\
0.357821089455272	36.025969\\
0.359820089955023	36.006195\\
0.361819090454773	35.986419\\
0.363818090954523	35.966641\\
0.365817091454273	35.946861\\
0.367816091954023	35.927079\\
0.369815092453773	35.907295\\
0.371814092953523	35.887508\\
0.373813093453273	35.86772\\
0.375812093953024	35.847929\\
0.377811094452774	35.828136\\
0.379810094952524	35.808341\\
0.381809095452274	35.788544\\
0.383808095952024	35.768745\\
0.385807096451774	35.748943\\
0.387806096951524	35.729139\\
0.389805097451274	35.709333\\
0.391804097951024	35.689525\\
0.393803098450775	35.669715\\
0.395802098950525	35.649902\\
0.397801099450275	35.630088\\
0.399800099950025	35.610271\\
0.401799100449775	35.590451\\
0.403798100949525	35.57063\\
0.405797101449275	35.550807\\
0.407796101949025	35.530981\\
0.409795102448776	35.511153\\
0.411794102948526	35.491322\\
0.413793103448276	35.47149\\
0.415792103948026	35.451655\\
0.417791104447776	35.431818\\
0.419790104947526	35.411979\\
0.421789105447276	35.392137\\
0.423788105947026	35.372294\\
0.425787106446777	35.352448\\
0.427786106946527	35.3326\\
0.429785107446277	35.312749\\
0.431784107946027	35.292897\\
0.433783108445777	35.273042\\
0.435782108945527	35.253185\\
0.437781109445277	35.233326\\
0.439780109945027	35.213464\\
0.441779110444778	35.193601\\
0.443778110944528	35.173735\\
0.445777111444278	35.153867\\
0.447776111944028	35.133997\\
0.449775112443778	35.114124\\
0.451774112943528	35.09425\\
0.453773113443278	35.074373\\
0.455772113943028	35.054494\\
0.457771114442779	35.034613\\
0.459770114942529	35.014729\\
0.461769115442279	34.994844\\
0.463768115942029	34.974957\\
0.465767116441779	34.955067\\
0.467766116941529	34.935175\\
0.469765117441279	34.915281\\
0.471764117941029	34.895385\\
0.47376311844078	34.875487\\
0.47576211894053	34.855587\\
0.47776111944028	34.835685\\
0.47976011994003	34.815781\\
0.48175912043978	34.795874\\
0.48375812093953	34.775966\\
0.48575712143928	34.756056\\
0.487756121939031	34.736143\\
0.489755122438781	34.716229\\
0.491754122938531	34.696313\\
0.493753123438281	34.676395\\
0.495752123938031	34.656474\\
0.497751124437781	34.636552\\
0.499750124937531	34.616628\\
0.501749125437281	34.596702\\
0.503748125937031	34.576775\\
0.505747126436782	34.556845\\
0.507746126936532	34.536913\\
0.509745127436282	34.51698\\
0.511744127936032	34.497045\\
0.513743128435782	34.477108\\
0.515742128935532	34.457169\\
0.517741129435282	34.437229\\
0.519740129935032	34.417287\\
0.521739130434783	34.397343\\
0.523738130934533	34.377397\\
0.525737131434283	34.35745\\
0.527736131934033	34.337501\\
0.529735132433783	34.317551\\
0.531734132933533	34.297599\\
0.533733133433283	34.277645\\
0.535732133933034	34.25769\\
0.537731134432784	34.237733\\
0.539730134932534	34.217774\\
0.541729135432284	34.197815\\
0.543728135932034	34.177853\\
0.545727136431784	34.15789\\
0.547726136931534	34.137926\\
0.549725137431284	34.117961\\
0.551724137931034	34.097994\\
0.553723138430785	34.078025\\
0.555722138930535	34.058056\\
0.557721139430285	34.038085\\
0.559720139930035	34.018112\\
0.561719140429785	33.998139\\
0.563718140929535	33.978164\\
0.565717141429285	33.958188\\
0.567716141929036	33.938211\\
0.569715142428786	33.918233\\
0.571714142928536	33.898253\\
0.573713143428286	33.878273\\
0.575712143928036	33.858292\\
0.577711144427786	33.838309\\
0.579710144927536	33.818325\\
0.581709145427286	33.798341\\
0.583708145927036	33.778355\\
0.585707146426787	33.758369\\
0.587706146926537	33.738382\\
0.589705147426287	33.718393\\
0.591704147926037	33.698404\\
0.593703148425787	33.678415\\
0.595702148925537	33.658424\\
0.597701149425287	33.638433\\
0.599700149925038	33.618441\\
0.601699150424788	33.598448\\
0.603698150924538	33.578455\\
0.605697151424288	33.558461\\
0.607696151924038	33.538466\\
0.609695152423788	33.518471\\
0.611694152923538	33.498475\\
0.613693153423288	33.478479\\
0.615692153923038	33.458482\\
0.617691154422789	33.438485\\
0.619690154922539	33.418487\\
0.621689155422289	33.39849\\
0.623688155922039	33.378491\\
0.625687156421789	33.358493\\
0.627686156921539	33.338494\\
0.629685157421289	33.318495\\
0.631684157921039	33.298495\\
0.63368315842079	33.278496\\
0.63568215892054	33.258496\\
0.63768115942029	33.238496\\
0.63968015992004	33.218497\\
0.64167916041979	33.198497\\
0.64367816091954	33.178497\\
0.64567716141929	33.158497\\
0.64767616191904	33.138497\\
0.649675162418791	33.118497\\
0.651674162918541	33.098497\\
0.653673163418291	33.078497\\
0.655672163918041	33.058498\\
0.657671164417791	33.038498\\
0.659670164917541	33.018499\\
0.661669165417291	32.9985\\
0.663668165917041	32.978502\\
0.665667166416792	32.958504\\
0.667666166916542	32.938506\\
0.669665167416292	32.918508\\
0.671664167916042	32.898511\\
0.673663168415792	32.878514\\
0.675662168915542	32.858518\\
0.677661169415292	32.838523\\
0.679660169915043	32.818527\\
0.681659170414793	32.798533\\
0.683658170914543	32.778539\\
0.685657171414293	32.758546\\
0.687656171914043	32.738553\\
0.689655172413793	32.718561\\
0.691654172913543	32.69857\\
0.693653173413293	32.678579\\
0.695652173913043	32.658589\\
0.697651174412794	32.6386\\
0.699650174912544	32.618612\\
0.701649175412294	32.598625\\
0.703648175912044	32.578639\\
0.705647176411794	32.558653\\
0.707646176911544	32.538669\\
0.709645177411294	32.518685\\
0.711644177911045	32.498703\\
0.713643178410795	32.478721\\
0.715642178910545	32.458741\\
0.717641179410295	32.438762\\
0.719640179910045	32.418783\\
0.721639180409795	32.398806\\
0.723638180909545	32.37883\\
0.725637181409295	32.358856\\
0.727636181909045	32.338882\\
0.729635182408796	32.31891\\
0.731634182908546	32.298939\\
0.733633183408296	32.27897\\
0.735632183908046	32.259001\\
0.737631184407796	32.239034\\
0.739630184907546	32.219069\\
0.741629185407296	32.199105\\
0.743628185907046	32.179142\\
0.745627186406797	32.159181\\
0.747626186906547	32.139221\\
0.749625187406297	32.119263\\
0.751624187906047	32.099306\\
0.753623188405797	32.079351\\
0.755622188905547	32.059397\\
0.757621189405297	32.039445\\
0.759620189905047	32.019495\\
0.761619190404798	31.999546\\
0.763618190904548	31.979599\\
0.765617191404298	31.959653\\
0.767616191904048	31.93971\\
0.769615192403798	31.919768\\
0.771614192903548	31.899827\\
0.773613193403298	31.879889\\
0.775612193903048	31.859952\\
0.777611194402799	31.840017\\
0.779610194902549	31.820084\\
0.781609195402299	31.800152\\
0.783608195902049	31.780223\\
0.785607196401799	31.760295\\
0.787606196901549	31.740369\\
0.789605197401299	31.720445\\
0.79160419790105	31.700523\\
0.7936031984008	31.680603\\
0.79560219890055	31.660685\\
0.7976011994003	31.640769\\
0.79960019990005	31.620855\\
0.8015992003998	31.600943\\
0.80359820089955	31.581032\\
0.8055972013993	31.561124\\
0.80759620189905	31.541218\\
0.809595202398801	31.521314\\
0.811594202898551	31.501412\\
0.813593203398301	31.481512\\
0.815592203898051	31.461614\\
0.817591204397801	31.441718\\
0.819590204897551	31.421824\\
0.821589205397301	31.401933\\
0.823588205897052	31.382043\\
0.825587206396802	31.362156\\
0.827586206896552	31.342271\\
0.829585207396302	31.322387\\
0.831584207896052	31.302506\\
0.833583208395802	31.282628\\
0.835582208895552	31.262751\\
0.837581209395302	31.242876\\
0.839580209895052	31.223004\\
0.841579210394803	31.203134\\
0.843578210894553	31.183266\\
0.845577211394303	31.1634\\
0.847576211894053	31.143537\\
0.849575212393803	31.123676\\
0.851574212893553	31.103816\\
0.853573213393303	31.08396\\
0.855572213893053	31.064105\\
0.857571214392804	31.044252\\
0.859570214892554	31.024402\\
0.861569215392304	31.004554\\
0.863568215892054	30.984708\\
0.865567216391804	30.964865\\
0.867566216891554	30.945024\\
0.869565217391304	30.925185\\
0.871564217891054	30.905348\\
0.873563218390805	30.885513\\
0.875562218890555	30.865681\\
0.877561219390305	30.845851\\
0.879560219890055	30.826023\\
0.881559220389805	30.806197\\
0.883558220889555	30.786373\\
0.885557221389305	30.766552\\
0.887556221889055	30.746733\\
0.889555222388806	30.726916\\
0.891554222888556	30.707102\\
0.893553223388306	30.687289\\
0.895552223888056	30.667479\\
0.897551224387806	30.647671\\
0.899550224887556	30.627865\\
0.901549225387306	30.608062\\
0.903548225887057	30.58826\\
0.905547226386807	30.568461\\
0.907546226886557	30.548664\\
0.909545227386307	30.528869\\
0.911544227886057	30.509076\\
0.913543228385807	30.489286\\
0.915542228885557	30.469497\\
0.917541229385307	30.449711\\
0.919540229885057	30.429927\\
0.921539230384808	30.410145\\
0.923538230884558	30.390365\\
0.925537231384308	30.370587\\
0.927536231884058	30.350811\\
0.929535232383808	30.331037\\
0.931534232883558	30.311266\\
0.933533233383308	30.291496\\
0.935532233883059	30.271728\\
0.937531234382809	30.251963\\
0.939530234882559	30.232199\\
0.941529235382309	30.212438\\
0.943528235882059	30.192678\\
0.945527236381809	30.172921\\
0.947526236881559	30.153165\\
0.949525237381309	30.133412\\
0.951524237881059	30.11366\\
0.95352323838081	30.09391\\
0.95552223888056	30.074163\\
0.95752123938031	30.054417\\
0.95952023988006	30.034673\\
0.96151924037981	30.014931\\
0.96351824087956	29.99519\\
0.96551724137931	29.975452\\
0.96751624187906	29.955715\\
0.969515242378811	29.935981\\
0.971514242878561	29.916248\\
0.973513243378311	29.896516\\
0.975512243878061	29.876787\\
0.977511244377811	29.857059\\
0.979510244877561	29.837333\\
0.981509245377311	29.817609\\
0.983508245877061	29.797886\\
0.985507246376812	29.778166\\
0.987506246876562	29.758446\\
0.989505247376312	29.738729\\
0.991504247876062	29.719013\\
0.993503248375812	29.699298\\
0.995502248875562	29.679586\\
0.997501249375312	29.659874\\
0.999500249875062	29.640165\\
1.00149925037481	29.620457\\
1.00349825087456	29.60075\\
1.00549725137431	29.581045\\
1.00749625187406	29.561341\\
1.00949525237381	29.541639\\
1.01149425287356	29.521938\\
1.01349325337331	29.502239\\
1.01549225387306	29.482541\\
1.01749125437281	29.462844\\
1.01949025487256	29.443149\\
1.02148925537231	29.423455\\
1.02348825587206	29.403762\\
1.02548725637181	29.384071\\
1.02748625687156	29.364381\\
1.02948525737131	29.344692\\
1.03148425787106	29.325004\\
1.03348325837081	29.305318\\
1.03548225887056	29.285633\\
1.03748125937031	29.265949\\
1.03948025987006	29.246266\\
1.04147926036982	29.226584\\
1.04347826086957	29.206903\\
1.04547726136932	29.187224\\
1.04747626186907	29.167545\\
1.04947526236882	29.147867\\
1.05147426286857	29.128191\\
1.05347326336832	29.108515\\
1.05547226386807	29.088841\\
1.05747126436782	29.069167\\
1.05947026486757	29.049494\\
1.06146926536732	29.029822\\
1.06346826586707	29.010151\\
1.06546726636682	28.990481\\
1.06746626686657	28.970812\\
1.06946526736632	28.951143\\
1.07146426786607	28.931476\\
1.07346326836582	28.911809\\
1.07546226886557	28.892142\\
1.07746126936532	28.872477\\
1.07946026986507	28.852812\\
1.08145927036482	28.833148\\
1.08345827086457	28.813484\\
1.08545727136432	28.793821\\
1.08745627186407	28.774159\\
1.08945527236382	28.754497\\
1.09145427286357	28.734836\\
1.09345327336332	28.715176\\
1.09545227386307	28.695516\\
1.09745127436282	28.675856\\
1.09945027486257	28.656197\\
1.10144927536232	28.636538\\
1.10344827586207	28.61688\\
1.10544727636182	28.597222\\
1.10744627686157	28.577565\\
1.10944527736132	28.557907\\
1.11144427786107	28.538251\\
1.11344327836082	28.518594\\
1.11544227886057	28.498938\\
1.11744127936032	28.479282\\
1.11944027986007	28.459627\\
1.12143928035982	28.439971\\
1.12343828085957	28.420316\\
1.12543728135932	28.400661\\
1.12743628185907	28.381006\\
1.12943528235882	28.361351\\
1.13143428285857	28.341697\\
1.13343328335832	28.322042\\
1.13543228385807	28.302388\\
1.13743128435782	28.282734\\
1.13943028485757	28.263079\\
1.14142928535732	28.243425\\
1.14342828585707	28.223771\\
1.14542728635682	28.204117\\
1.14742628685657	28.184462\\
1.14942528735632	28.164808\\
1.15142428785607	28.145154\\
1.15342328835582	28.125499\\
1.15542228885557	28.105844\\
1.15742128935532	28.08619\\
1.15942028985507	28.066535\\
1.16141929035482	28.04688\\
1.16341829085457	28.027224\\
1.16541729135432	28.007569\\
1.16741629185407	27.987913\\
1.16941529235382	27.968257\\
1.17141429285357	27.948601\\
1.17341329335332	27.928945\\
1.17541229385307	27.909288\\
1.17741129435282	27.889631\\
1.17941029485257	27.869973\\
1.18140929535232	27.850315\\
1.18340829585207	27.830657\\
1.18540729635182	27.810999\\
1.18740629685157	27.79134\\
1.18940529735132	27.771681\\
1.19140429785107	27.752021\\
1.19340329835082	27.732361\\
1.19540229885057	27.7127\\
1.19740129935032	27.693039\\
1.19940029985008	27.673377\\
1.20139930034983	27.653715\\
1.20339830084958	27.634053\\
1.20539730134933	27.61439\\
1.20739630184908	27.594726\\
1.20939530234883	27.575062\\
1.21139430284858	27.555397\\
1.21339330334833	27.535732\\
1.21539230384808	27.516066\\
1.21739130434783	27.496399\\
1.21939030484758	27.476732\\
1.22138930534733	27.457064\\
1.22338830584708	27.437395\\
1.22538730634683	27.417726\\
1.22738630684658	27.398056\\
1.22938530734633	27.378386\\
1.23138430784608	27.358715\\
1.23338330834583	27.339043\\
1.23538230884558	27.31937\\
1.23738130934533	27.299697\\
1.23938030984508	27.280023\\
1.24137931034483	27.260348\\
1.24337831084458	27.240673\\
1.24537731134433	27.220996\\
1.24737631184408	27.201319\\
1.24937531234383	27.181642\\
1.25137431284358	27.161963\\
1.25337331334333	27.142284\\
1.25537231384308	27.122603\\
1.25737131434283	27.102922\\
1.25937031484258	27.083241\\
1.26136931534233	27.063558\\
1.26336831584208	27.043875\\
1.26536731634183	27.02419\\
1.26736631684158	27.004505\\
1.26936531734133	26.984819\\
1.27136431784108	26.965133\\
1.27336331834083	26.945445\\
1.27536231884058	26.925756\\
1.27736131934033	26.906067\\
1.27936031984008	26.886377\\
1.28135932033983	26.866686\\
1.28335832083958	26.846994\\
1.28535732133933	26.827301\\
1.28735632183908	26.807607\\
1.28935532233883	26.787913\\
1.29135432283858	26.768217\\
1.29335332333833	26.748521\\
1.29535232383808	26.728824\\
1.29735132433783	26.709126\\
1.29935032483758	26.689427\\
1.30134932533733	26.669727\\
1.30334832583708	26.650026\\
1.30534732633683	26.630324\\
1.30734632683658	26.610622\\
1.30934532733633	26.590918\\
1.31134432783608	26.571214\\
1.31334332833583	26.551509\\
1.31534232883558	26.531802\\
1.31734132933533	26.512095\\
1.31934032983508	26.492388\\
1.32133933033483	26.472679\\
1.32333833083458	26.452969\\
1.32533733133433	26.433259\\
1.32733633183408	26.413547\\
1.32933533233383	26.393835\\
1.33133433283358	26.374122\\
1.33333333333333	26.354408\\
1.33533233383308	26.334693\\
1.33733133433283	26.314977\\
1.33933033483258	26.295261\\
1.34132933533233	26.275543\\
1.34332833583208	26.255825\\
1.34532733633183	26.236106\\
1.34732633683158	26.216386\\
1.34932533733133	26.196665\\
1.35132433783108	26.176944\\
1.35332333833083	26.157221\\
1.35532233883058	26.137498\\
1.35732133933033	26.117774\\
1.35932033983009	26.098049\\
1.36131934032984	26.078324\\
1.36331834082959	26.058597\\
1.36531734132934	26.03887\\
1.36731634182909	26.019142\\
1.36931534232884	25.999414\\
1.37131434282859	25.979685\\
1.37331334332834	25.959954\\
1.37531234382809	25.940224\\
1.37731134432784	25.920492\\
1.37931034482759	25.90076\\
1.38130934532734	25.881027\\
1.38330834582709	25.861293\\
1.38530734632684	25.841559\\
1.38730634682659	25.821824\\
1.38930534732634	25.802088\\
1.39130434782609	25.782352\\
1.39330334832584	25.762615\\
1.39530234882559	25.742878\\
1.39730134932534	25.723139\\
1.39930034982509	25.703401\\
1.40129935032484	25.683661\\
1.40329835082459	25.663921\\
1.40529735132434	25.644181\\
1.40729635182409	25.62444\\
1.40929535232384	25.604698\\
1.41129435282359	25.584956\\
1.41329335332334	25.565213\\
1.41529235382309	25.54547\\
1.41729135432284	25.525727\\
1.41929035482259	25.505982\\
1.42128935532234	25.486238\\
1.42328835582209	25.466493\\
1.42528735632184	25.446747\\
1.42728635682159	25.427001\\
1.42928535732134	25.407255\\
1.43128435782109	25.387508\\
1.43328335832084	25.367761\\
1.43528235882059	25.348014\\
1.43728135932034	25.328266\\
1.43928035982009	25.308518\\
1.44127936031984	25.288769\\
1.44327836081959	25.269021\\
1.44527736131934	25.249271\\
1.44727636181909	25.229522\\
1.44927536231884	25.209772\\
1.45127436281859	25.190023\\
1.45327336331834	25.170272\\
1.45527236381809	25.150522\\
1.45727136431784	25.130771\\
1.45927036481759	25.111021\\
1.46126936531734	25.09127\\
1.46326836581709	25.071519\\
1.46526736631684	25.051767\\
1.46726636681659	25.032016\\
1.46926536731634	25.012264\\
1.47126436781609	24.992513\\
1.47326336831584	24.972761\\
1.47526236881559	24.953009\\
1.47726136931534	24.933258\\
1.47926036981509	24.913506\\
1.48125937031484	24.893754\\
1.48325837081459	24.874002\\
1.48525737131434	24.85425\\
1.48725637181409	24.834498\\
1.48925537231384	24.814746\\
1.49125437281359	24.794995\\
1.49325337331334	24.775243\\
1.49525237381309	24.755491\\
1.49725137431284	24.73574\\
1.49925037481259	24.715989\\
1.50124937531234	24.696237\\
1.50324837581209	24.676486\\
1.50524737631184	24.656735\\
1.50724637681159	24.636984\\
1.50924537731134	24.617234\\
1.51124437781109	24.597484\\
1.51324337831084	24.577733\\
1.51524237881059	24.557983\\
1.51724137931034	24.538234\\
1.51924037981009	24.518484\\
1.52123938030985	24.498735\\
1.5232383808096	24.478987\\
1.52523738130935	24.459238\\
1.5272363818091	24.43949\\
1.52923538230885	24.419742\\
1.5312343828086	24.399995\\
1.53323338330835	24.380248\\
1.5352323838081	24.360501\\
1.53723138430785	24.340755\\
1.5392303848076	24.321009\\
1.54122938530735	24.301263\\
1.5432283858071	24.281518\\
1.54522738630685	24.261774\\
1.5472263868066	24.24203\\
1.54922538730635	24.222286\\
1.5512243878061	24.202543\\
1.55322338830585	24.1828\\
1.5552223888056	24.163058\\
1.55722138930535	24.143316\\
1.5592203898051	24.123575\\
1.56121939030485	24.103835\\
1.5632183908046	24.084095\\
1.56521739130435	24.064356\\
1.5672163918041	24.044617\\
1.56921539230385	24.024879\\
1.5712143928036	24.005141\\
1.57321339330335	23.985404\\
1.5752123938031	23.965668\\
1.57721139430285	23.945933\\
1.5792103948026	23.926198\\
1.58120939530235	23.906463\\
1.5832083958021	23.88673\\
1.58520739630185	23.866997\\
1.5872063968016	23.847265\\
1.58920539730135	23.827533\\
1.5912043978011	23.807802\\
1.59320339830085	23.788072\\
1.5952023988006	23.768343\\
1.59720139930035	23.748614\\
1.5992003998001	23.728887\\
1.60119940029985	23.709159\\
1.6031984007996	23.689433\\
1.60519740129935	23.669708\\
1.6071964017991	23.649983\\
1.60919540229885	23.630259\\
1.6111944027986	23.610536\\
1.61319340329835	23.590813\\
1.6151924037981	23.571092\\
1.61719140429785	23.551371\\
1.6191904047976	23.531651\\
1.62118940529735	23.511932\\
1.6231884057971	23.492214\\
1.62518740629685	23.472497\\
1.6271864067966	23.45278\\
1.62918540729635	23.433065\\
1.6311844077961	23.41335\\
1.63318340829585	23.393636\\
1.6351824087956	23.373923\\
1.63718140929535	23.35421\\
1.6391804097951	23.334499\\
1.64117941029485	23.314789\\
1.6431784107946	23.295079\\
1.64517741129435	23.27537\\
1.6471764117941	23.255662\\
1.64917541229385	23.235955\\
1.6511744127936	23.216249\\
1.65317341329335	23.196544\\
1.6551724137931	23.17684\\
1.65717141429285	23.157136\\
1.6591704147926	23.137434\\
1.66116941529235	23.117732\\
1.6631684157921	23.098031\\
1.66516741629185	23.078332\\
1.6671664167916	23.058633\\
1.66916541729135	23.038935\\
1.6711644177911	23.019237\\
1.67316341829085	22.999541\\
1.6751624187906	22.979846\\
1.67716141929035	22.960151\\
1.6791604197901	22.940457\\
1.68115942028985	22.920765\\
1.68315842078961	22.901073\\
1.68515742128936	22.881382\\
1.68715642178911	22.861691\\
1.68915542228886	22.842002\\
1.69115442278861	22.822314\\
1.69315342328836	22.802626\\
1.69515242378811	22.782939\\
1.69715142428786	22.763253\\
1.69915042478761	22.743568\\
1.70114942528736	22.723884\\
1.70314842578711	22.704201\\
1.70514742628686	22.684518\\
1.70714642678661	22.664836\\
1.70914542728636	22.645155\\
1.71114442778611	22.625475\\
1.71314342828586	22.605796\\
1.71514242878561	22.586117\\
1.71714142928536	22.56644\\
1.71914042978511	22.546763\\
1.72113943028486	22.527086\\
1.72313843078461	22.507411\\
1.72513743128436	22.487736\\
1.72713643178411	22.468062\\
1.72913543228386	22.448389\\
1.73113443278361	22.428716\\
1.73313343328336	22.409045\\
1.73513243378311	22.389373\\
1.73713143428286	22.369703\\
1.73913043478261	22.350033\\
1.74112943528236	22.330364\\
1.74312843578211	22.310696\\
1.74512743628186	22.291028\\
1.74712643678161	22.271361\\
1.74912543728136	22.251694\\
1.75112443778111	22.232028\\
1.75312343828086	22.212363\\
1.75512243878061	22.192698\\
1.75712143928036	22.173034\\
1.75912043978011	22.15337\\
1.76111944027986	22.133707\\
1.76311844077961	22.114044\\
1.76511744127936	22.094382\\
1.76711644177911	22.074721\\
1.76911544227886	22.05506\\
1.77111444277861	22.035399\\
1.77311344327836	22.015739\\
1.77511244377811	21.996079\\
1.77711144427786	21.97642\\
1.77911044477761	21.956761\\
1.78110944527736	21.937103\\
1.78310844577711	21.917444\\
1.78510744627686	21.897787\\
1.78710644677661	21.878129\\
1.78910544727636	21.858472\\
1.79110444777611	21.838815\\
1.79310344827586	21.819159\\
1.79510244877561	21.799503\\
1.79710144927536	21.779847\\
1.79910044977511	21.760191\\
1.80109945027486	21.740536\\
1.80309845077461	21.72088\\
1.80509745127436	21.701225\\
1.80709645177411	21.68157\\
1.80909545227386	21.661916\\
1.81109445277361	21.642261\\
1.81309345327336	21.622607\\
1.81509245377311	21.602952\\
1.81709145427286	21.583298\\
1.81909045477261	21.563644\\
1.82108945527236	21.543989\\
1.82308845577211	21.524335\\
1.82508745627186	21.504681\\
1.82708645677161	21.485026\\
1.82908545727136	21.465372\\
1.83108445777111	21.445718\\
1.83308345827086	21.426063\\
1.83508245877061	21.406409\\
1.83708145927036	21.386754\\
1.83908045977011	21.367099\\
1.84107946026987	21.347444\\
1.84307846076962	21.327789\\
1.84507746126937	21.308134\\
1.84707646176912	21.288478\\
1.84907546226887	21.268822\\
1.85107446276862	21.249166\\
1.85307346326837	21.229509\\
1.85507246376812	21.209853\\
1.85707146426787	21.190195\\
1.85907046476762	21.170538\\
1.86106946526737	21.15088\\
1.86306846576712	21.131222\\
1.86506746626687	21.111563\\
1.86706646676662	21.091904\\
1.86906546726637	21.072244\\
1.87106446776612	21.052584\\
1.87306346826587	21.032924\\
1.87506246876562	21.013262\\
1.87706146926537	20.993601\\
1.87906046976512	20.973938\\
1.88105947026487	20.954275\\
1.88305847076462	20.934612\\
1.88505747126437	20.914948\\
1.88705647176412	20.895283\\
1.88905547226387	20.875617\\
1.89105447276362	20.855951\\
1.89305347326337	20.836284\\
1.89505247376312	20.816616\\
1.89705147426287	20.796948\\
1.89905047476262	20.777278\\
1.90104947526237	20.757608\\
1.90304847576212	20.737937\\
1.90504747626187	20.718265\\
1.90704647676162	20.698592\\
1.90904547726137	20.678919\\
1.91104447776112	20.659244\\
1.91304347826087	20.639568\\
1.91504247876062	20.619892\\
1.91704147926037	20.600214\\
1.91904047976012	20.580535\\
1.92103948025987	20.560856\\
1.92303848075962	20.541175\\
1.92503748125937	20.521493\\
1.92703648175912	20.50181\\
1.92903548225887	20.482126\\
1.93103448275862	20.462441\\
1.93303348325837	20.442754\\
1.93503248375812	20.423067\\
1.93703148425787	20.403378\\
1.93903048475762	20.383687\\
1.94102948525737	20.363996\\
1.94302848575712	20.344303\\
1.94502748625687	20.324609\\
1.94702648675662	20.304914\\
1.94902548725637	20.285217\\
1.95102448775612	20.265519\\
1.95302348825587	20.24582\\
1.95502248875562	20.226119\\
1.95702148925537	20.206417\\
1.95902048975512	20.186713\\
1.96101949025487	20.167008\\
1.96301849075462	20.147301\\
1.96501749125437	20.127593\\
1.96701649175412	20.107883\\
1.96901549225387	20.088172\\
1.97101449275362	20.068459\\
1.97301349325337	20.048745\\
1.97501249375312	20.029029\\
1.97701149425287	20.009311\\
1.97901049475262	19.989592\\
1.98100949525237	19.969871\\
1.98300849575212	19.950148\\
1.98500749625187	19.930424\\
1.98700649675162	19.910698\\
1.98900549725137	19.89097\\
1.99100449775112	19.87124\\
1.99300349825087	19.851509\\
1.99500249875062	19.831776\\
1.99700149925037	19.812041\\
1.99900049975012	19.792304\\
2.00099950024988	19.772566\\
2.00299850074963	19.752826\\
2.00499750124938	19.733083\\
2.00699650174913	19.713339\\
2.00899550224888	19.693593\\
2.01099450274863	19.673845\\
2.01299350324838	19.654096\\
2.01499250374813	19.634344\\
2.01699150424788	19.61459\\
2.01899050474763	19.594835\\
2.02098950524738	19.575077\\
2.02298850574713	19.555317\\
2.02498750624688	19.535556\\
2.02698650674663	19.515792\\
2.02898550724638	19.496027\\
2.03098450774613	19.476259\\
2.03298350824588	19.456489\\
2.03498250874563	19.436718\\
2.03698150924538	19.416944\\
2.03898050974513	19.397168\\
2.04097951024488	19.37739\\
2.04297851074463	19.35761\\
2.04497751124438	19.337828\\
2.04697651174413	19.318043\\
2.04897551224388	19.298257\\
2.05097451274363	19.278468\\
2.05297351324338	19.258678\\
2.05497251374313	19.238885\\
2.05697151424288	19.21909\\
2.05897051474263	19.199293\\
2.06096951524238	19.179493\\
2.06296851574213	19.159692\\
2.06496751624188	19.139888\\
2.06696651674163	19.120082\\
2.06896551724138	19.100274\\
2.07096451774113	19.080464\\
2.07296351824088	19.060651\\
2.07496251874063	19.040837\\
2.07696151924038	19.02102\\
2.07896051974013	19.001201\\
2.08095952023988	18.981379\\
2.08295852073963	18.961556\\
2.08495752123938	18.94173\\
2.08695652173913	18.921902\\
2.08895552223888	18.902072\\
2.09095452273863	18.882239\\
2.09295352323838	18.862404\\
2.09495252373813	18.842567\\
2.09695152423788	18.822728\\
2.09895052473763	18.802887\\
2.10094952523738	18.783043\\
2.10294852573713	18.763197\\
2.10494752623688	18.743349\\
2.10694652673663	18.723499\\
2.10894552723638	18.703646\\
2.11094452773613	18.683792\\
2.11294352823588	18.663935\\
2.11494252873563	18.644075\\
2.11694152923538	18.624214\\
2.11894052973513	18.60435\\
2.12093953023488	18.584485\\
2.12293853073463	18.564617\\
2.12493753123438	18.544746\\
2.12693653173413	18.524874\\
2.12893553223388	18.504999\\
2.13093453273363	18.485123\\
2.13293353323338	18.465244\\
2.13493253373313	18.445363\\
2.13693153423288	18.425479\\
2.13893053473263	18.405594\\
2.14092953523238	18.385707\\
2.14292853573213	18.365817\\
2.14492753623188	18.345925\\
2.14692653673163	18.326031\\
2.14892553723138	18.306135\\
2.15092453773113	18.286237\\
2.15292353823088	18.266337\\
2.15492253873063	18.246435\\
2.15692153923038	18.226531\\
2.15892053973013	18.206625\\
2.16091954022989	18.186716\\
2.16291854072964	18.166806\\
2.16491754122939	18.146894\\
2.16691654172914	18.12698\\
2.16891554222889	18.107063\\
2.17091454272864	18.087145\\
2.17291354322839	18.067225\\
2.17491254372814	18.047303\\
2.17691154422789	18.027379\\
2.17891054472764	18.007453\\
2.18090954522739	17.987525\\
2.18290854572714	17.967596\\
2.18490754622689	17.947664\\
2.18690654672664	17.927731\\
2.18890554722639	17.907796\\
2.19090454772614	17.887859\\
2.19290354822589	17.86792\\
2.19490254872564	17.84798\\
2.19690154922539	17.828038\\
2.19890054972514	17.808094\\
2.20089955022489	17.788148\\
2.20289855072464	17.768201\\
2.20489755122439	17.748252\\
2.20689655172414	17.728302\\
2.20889555222389	17.70835\\
2.21089455272364	17.688396\\
2.21289355322339	17.668441\\
2.21489255372314	17.648484\\
2.21689155422289	17.628526\\
2.21889055472264	17.608566\\
2.22088955522239	17.588604\\
2.22288855572214	17.568642\\
2.22488755622189	17.548677\\
2.22688655672164	17.528712\\
2.22888555722139	17.508745\\
2.23088455772114	17.488777\\
2.23288355822089	17.468807\\
2.23488255872064	17.448836\\
2.23688155922039	17.428864\\
2.23888055972014	17.40889\\
2.24087956021989	17.388915\\
2.24287856071964	17.36894\\
2.24487756121939	17.348962\\
2.24687656171914	17.328984\\
2.24887556221889	17.309005\\
2.25087456271864	17.289024\\
2.25287356321839	17.269043\\
2.25487256371814	17.24906\\
2.25687156421789	17.229077\\
2.25887056471764	17.209092\\
2.26086956521739	17.189107\\
2.26286856571714	17.16912\\
2.26486756621689	17.149133\\
2.26686656671664	17.129145\\
2.26886556721639	17.109156\\
2.27086456771614	17.089166\\
2.27286356821589	17.069176\\
2.27486256871564	17.049184\\
2.27686156921539	17.029192\\
2.27886056971514	17.0092\\
2.28085957021489	16.989206\\
2.28285857071464	16.969212\\
2.28485757121439	16.949218\\
2.28685657171414	16.929222\\
2.28885557221389	16.909227\\
2.29085457271364	16.88923\\
2.29285357321339	16.869234\\
2.29485257371314	16.849237\\
2.29685157421289	16.829239\\
2.29885057471264	16.809241\\
2.30084957521239	16.789243\\
2.30284857571214	16.769244\\
2.30484757621189	16.749246\\
2.30684657671164	16.729246\\
2.30884557721139	16.709247\\
2.31084457771114	16.689248\\
2.31284357821089	16.669248\\
2.31484257871064	16.649248\\
2.31684157921039	16.629248\\
2.31884057971015	16.609248\\
2.3208395802099	16.589248\\
2.32283858070965	16.569248\\
2.3248375812094	16.549248\\
2.32683658170915	16.529248\\
2.3288355822089	16.509249\\
2.33083458270865	16.489249\\
2.3328335832084	16.469249\\
2.33483258370815	16.44925\\
2.3368315842079	16.429251\\
2.33883058470765	16.409252\\
2.3408295852074	16.389254\\
2.34282858570715	16.369255\\
2.3448275862069	16.349257\\
2.34682658670665	16.32926\\
2.3488255872064	16.309263\\
2.35082458770615	16.289266\\
2.3528235882059	16.26927\\
2.35482258870565	16.249274\\
2.3568215892054	16.229279\\
2.35882058970515	16.209285\\
2.3608195902049	16.189291\\
2.36281859070465	16.169297\\
2.3648175912044	16.149304\\
2.36681659170415	16.129312\\
2.3688155922039	16.109321\\
2.37081459270365	16.089331\\
2.3728135932034	16.069341\\
2.37481259370315	16.049352\\
2.3768115942029	16.029364\\
2.37881059470265	16.009376\\
2.3808095952024	15.98939\\
2.38280859570215	15.969405\\
2.3848075962019	15.94942\\
2.38680659670165	15.929437\\
2.3888055972014	15.909454\\
2.39080459770115	15.889473\\
2.3928035982009	15.869492\\
2.39480259870065	15.849513\\
2.3968015992004	15.829535\\
2.39880059970015	15.809558\\
2.4007996001999	15.789582\\
2.40279860069965	15.769607\\
2.4047976011994	15.749634\\
2.40679660169915	15.729661\\
2.4087956021989	15.709691\\
2.41079460269865	15.689721\\
2.4127936031984	15.669753\\
2.41479260369815	15.649786\\
2.4167916041979	15.62982\\
2.41879060469765	15.609856\\
2.4207896051974	15.589893\\
2.42278860569715	15.569932\\
2.4247876061969	15.549972\\
2.42678660669665	15.530014\\
2.4287856071964	15.510057\\
2.43078460769615	15.490102\\
2.4327836081959	15.470148\\
2.43478260869565	15.450196\\
2.4367816091954	15.430246\\
2.43878060969515	15.410297\\
2.4407796101949	15.39035\\
2.44277861069465	15.370404\\
2.4447776111944	15.35046\\
2.44677661169415	15.330518\\
2.4487756121939	15.310578\\
2.45077461269365	15.290639\\
2.4527736131934	15.270703\\
2.45477261369315	15.250768\\
2.4567716141929	15.230834\\
2.45877061469265	15.210903\\
2.4607696151924	15.190973\\
2.46276861569215	15.171046\\
2.4647676161919	15.15112\\
2.46676661669165	15.131196\\
2.4687656171914	15.111274\\
2.47076461769115	15.091354\\
2.4727636181909	15.071436\\
2.47476261869065	15.051519\\
2.4767616191904	15.031605\\
2.47876061969016	15.011693\\
2.48075962018991	14.991783\\
2.48275862068966	14.971875\\
2.48475762118941	14.951968\\
2.48675662168916	14.932064\\
2.48875562218891	14.912162\\
2.49075462268866	14.892262\\
2.49275362318841	14.872364\\
2.49475262368816	14.852468\\
2.49675162418791	14.832574\\
2.49875062468766	14.812683\\
2.50074962518741	14.792793\\
2.50274862568716	14.772906\\
2.50474762618691	14.753021\\
2.50674662668666	14.733137\\
2.50874562718641	14.713256\\
2.51074462768616	14.693378\\
2.51274362818591	14.673501\\
2.51474262868566	14.653626\\
2.51674162918541	14.633754\\
2.51874062968516	14.613884\\
2.52073963018491	14.594016\\
2.52273863068466	14.57415\\
2.52473763118441	14.554287\\
2.52673663168416	14.534425\\
2.52873563218391	14.514566\\
2.53073463268366	14.494709\\
2.53273363318341	14.474854\\
2.53473263368316	14.455002\\
2.53673163418291	14.435152\\
2.53873063468266	14.415304\\
2.54072963518241	14.395458\\
2.54272863568216	14.375614\\
2.54472763618191	14.355773\\
2.54672663668166	14.335934\\
2.54872563718141	14.316097\\
2.55072463768116	14.296262\\
2.55272363818091	14.27643\\
2.55472263868066	14.2566\\
2.55672163918041	14.236772\\
2.55872063968016	14.216946\\
2.56071964017991	14.197123\\
2.56271864067966	14.177301\\
2.56471764117941	14.157482\\
2.56671664167916	14.137665\\
2.56871564217891	14.117851\\
2.57071464267866	14.098038\\
2.57271364317841	14.078228\\
2.57471264367816	14.05842\\
2.57671164417791	14.038614\\
2.57871064467766	14.018811\\
2.58070964517741	13.999009\\
2.58270864567716	13.97921\\
2.58470764617691	13.959413\\
2.58670664667666	13.939618\\
2.58870564717641	13.919825\\
2.59070464767616	13.900034\\
2.59270364817591	13.880246\\
2.59470264867566	13.86046\\
2.59670164917541	13.840675\\
2.59870064967516	13.820893\\
2.60069965017491	13.801113\\
2.60269865067466	13.781335\\
2.60469765117441	13.761559\\
2.60669665167416	13.741786\\
2.60869565217391	13.722014\\
2.61069465267366	13.702244\\
2.61269365317341	13.682477\\
2.61469265367316	13.662711\\
2.61669165417291	13.642948\\
2.61869065467266	13.623186\\
2.62068965517241	13.603427\\
2.62268865567216	13.583669\\
2.62468765617191	13.563913\\
2.62668665667166	13.54416\\
2.62868565717141	13.524408\\
2.63068465767116	13.504658\\
2.63268365817091	13.484911\\
2.63468265867066	13.465165\\
2.63668165917041	13.445421\\
2.63868065967017	13.425679\\
2.64067966016992	13.405938\\
2.64267866066967	13.3862\\
2.64467766116942	13.366463\\
2.64667666166917	13.346728\\
2.64867566216892	13.326995\\
2.65067466266867	13.307264\\
2.65267366316842	13.287535\\
2.65467266366817	13.267807\\
2.65667166416792	13.248081\\
2.65867066466767	13.228357\\
2.66066966516742	13.208634\\
2.66266866566717	13.188913\\
2.66466766616692	13.169194\\
2.66666666666667	13.149476\\
2.66866566716642	13.12976\\
2.67066466766617	13.110046\\
2.67266366816592	13.090333\\
2.67466266866567	13.070622\\
2.67666166916542	13.050912\\
2.67866066966517	13.031204\\
2.68065967016492	13.011497\\
2.68265867066467	12.991792\\
2.68465767116442	12.972089\\
2.68665667166417	12.952386\\
2.68865567216392	12.932685\\
2.69065467266367	12.912986\\
2.69265367316342	12.893288\\
2.69465267366317	12.873591\\
2.69665167416292	12.853896\\
2.69865067466267	12.834202\\
2.70064967516242	12.81451\\
2.70264867566217	12.794818\\
2.70464767616192	12.775128\\
2.70664667666167	12.755439\\
2.70864567716142	12.735752\\
2.71064467766117	12.716065\\
2.71264367816092	12.69638\\
2.71464267866067	12.676696\\
2.71664167916042	12.657013\\
2.71864067966017	12.637331\\
2.72063968015992	12.61765\\
2.72263868065967	12.597971\\
2.72463768115942	12.578292\\
2.72663668165917	12.558614\\
2.72863568215892	12.538938\\
2.73063468265867	12.519262\\
2.73263368315842	12.499588\\
2.73463268365817	12.479914\\
2.73663168415792	12.460241\\
2.73863068465767	12.440569\\
2.74062968515742	12.420898\\
2.74262868565717	12.401228\\
2.74462768615692	12.381559\\
2.74662668665667	12.36189\\
2.74862568715642	12.342222\\
2.75062468765617	12.322555\\
2.75262368815592	12.302889\\
2.75462268865567	12.283224\\
2.75662168915542	12.263559\\
2.75862068965517	12.243895\\
2.76061969015492	12.224231\\
2.76261869065467	12.204568\\
2.76461769115442	12.184906\\
2.76661669165417	12.165244\\
2.76861569215392	12.145583\\
2.77061469265367	12.125922\\
2.77261369315342	12.106262\\
2.77461269365317	12.086603\\
2.77661169415292	12.066944\\
2.77861069465267	12.047285\\
2.78060969515242	12.027627\\
2.78260869565217	12.007969\\
2.78460769615192	11.988311\\
2.78660669665167	11.968654\\
2.78860569715142	11.948997\\
2.79060469765117	11.929341\\
2.79260369815092	11.909685\\
2.79460269865067	11.890029\\
2.79660169915043	11.870373\\
2.79860069965018	11.850718\\
2.80059970014993	11.831063\\
2.80259870064968	11.811408\\
2.80459770114943	11.791753\\
2.80659670164918	11.772098\\
2.80859570214893	11.752443\\
2.81059470264868	11.732789\\
2.81259370314843	11.713135\\
2.81459270364818	11.69348\\
2.81659170414793	11.673826\\
2.81859070464768	11.654172\\
2.82058970514743	11.634518\\
2.82258870564718	11.614863\\
2.82458770614693	11.595209\\
2.82658670664668	11.575555\\
2.82858570714643	11.5559\\
2.83058470764618	11.536246\\
2.83258370814593	11.516591\\
2.83458270864568	11.496936\\
2.83658170914543	11.477281\\
2.83858070964518	11.457626\\
2.84057971014493	11.437971\\
2.84257871064468	11.418316\\
2.84457771114443	11.39866\\
2.84657671164418	11.379004\\
2.84857571214393	11.359348\\
2.85057471264368	11.339691\\
2.85257371314343	11.320034\\
2.85457271364318	11.300377\\
2.85657171414293	11.28072\\
2.85857071464268	11.261062\\
2.86056971514243	11.241404\\
2.86256871564218	11.221746\\
2.86456771614193	11.202087\\
2.86656671664168	11.182427\\
2.86856571714143	11.162768\\
2.87056471764118	11.143107\\
2.87256371814093	11.123447\\
2.87456271864068	11.103786\\
2.87656171914043	11.084124\\
2.87856071964018	11.064462\\
2.88055972013993	11.044799\\
2.88255872063968	11.025136\\
2.88455772113943	11.005473\\
2.88655672163918	10.985808\\
2.88855572213893	10.966144\\
2.89055472263868	10.946478\\
2.89255372313843	10.926812\\
2.89455272363818	10.907146\\
2.89655172413793	10.887479\\
2.89855072463768	10.867811\\
2.90054972513743	10.848142\\
2.90254872563718	10.828473\\
2.90454772613693	10.808803\\
2.90654672663668	10.789133\\
2.90854572713643	10.769462\\
2.91054472763618	10.74979\\
2.91254372813593	10.730117\\
2.91454272863568	10.710444\\
2.91654172913543	10.69077\\
2.91854072963518	10.671095\\
2.92053973013493	10.65142\\
2.92253873063468	10.631743\\
2.92453773113443	10.612066\\
2.92653673163418	10.592389\\
2.92853573213393	10.57271\\
2.93053473263368	10.553031\\
2.93253373313343	10.53335\\
2.93453273363318	10.51367\\
2.93653173413293	10.493988\\
2.93853073463268	10.474305\\
2.94052973513243	10.454622\\
2.94252873563218	10.434937\\
2.94452773613193	10.415252\\
2.94652673663168	10.395566\\
2.94852573713143	10.37588\\
2.95052473763118	10.356192\\
2.95252373813093	10.336504\\
2.95452273863068	10.316814\\
2.95652173913043	10.297124\\
2.95852073963018	10.277433\\
2.96051974012994	10.257741\\
2.96251874062969	10.238048\\
2.96451774112944	10.218355\\
2.96651674162919	10.19866\\
2.96851574212894	10.178964\\
2.97051474262869	10.159268\\
2.97251374312844	10.139571\\
2.97451274362819	10.119873\\
2.97651174412794	10.100174\\
2.97851074462769	10.080474\\
2.98050974512744	10.060773\\
2.98250874562719	10.041071\\
2.98450774612694	10.021369\\
2.98650674662669	10.001665\\
2.98850574712644	9.981961\\
2.99050474762619	9.962256\\
2.99250374812594	9.94255\\
2.99450274862569	9.922843\\
2.99650174912544	9.903135\\
2.99850074962519	9.883426\\
3.00049975012494	9.863717\\
3.00249875062469	9.844006\\
3.00449775112444	9.824295\\
3.00649675162419	9.804582\\
3.00849575212394	9.784869\\
3.01049475262369	9.765155\\
3.01249375312344	9.74544\\
3.01449275362319	9.725725\\
3.01649175412294	9.706008\\
3.01849075462269	9.686291\\
3.02048975512244	9.666573\\
3.02248875562219	9.646853\\
3.02448775612194	9.627134\\
3.02648675662169	9.607413\\
3.02848575712144	9.587691\\
3.03048475762119	9.567969\\
3.03248375812094	9.548246\\
3.03448275862069	9.528522\\
3.03648175912044	9.508797\\
3.03848075962019	9.489071\\
3.04047976011994	9.469345\\
3.04247876061969	9.449618\\
3.04447776111944	9.42989\\
3.04647676161919	9.410162\\
3.04847576211894	9.390432\\
3.05047476261869	9.370702\\
3.05247376311844	9.350971\\
3.05447276361819	9.33124\\
3.05647176411794	9.311508\\
3.05847076461769	9.291775\\
3.06046976511744	9.272041\\
3.06246876561719	9.252307\\
3.06446776611694	9.232572\\
3.06646676661669	9.212836\\
3.06846576711644	9.1931\\
3.07046476761619	9.173363\\
3.07246376811594	9.153625\\
3.07446276861569	9.133887\\
3.07646176911544	9.114149\\
3.07846076961519	9.094409\\
3.08045977011494	9.074669\\
3.08245877061469	9.054929\\
3.08445777111444	9.035188\\
3.08645677161419	9.015446\\
3.08845577211394	8.995704\\
3.09045477261369	8.975961\\
3.09245377311344	8.956218\\
3.09445277361319	8.936475\\
3.09645177411294	8.91673\\
3.09845077461269	8.896986\\
3.10044977511244	8.877241\\
3.10244877561219	8.857495\\
3.10444777611194	8.837749\\
3.10644677661169	8.818003\\
3.10844577711144	8.798256\\
3.11044477761119	8.778509\\
3.11244377811094	8.758762\\
3.11444277861069	8.739014\\
3.11644177911044	8.719266\\
3.11844077961019	8.699517\\
3.12043978010994	8.679769\\
3.12243878060969	8.660019\\
3.12443778110945	8.64027\\
3.1264367816092	8.62052\\
3.12843578210895	8.600771\\
3.1304347826087	8.58102\\
3.13243378310845	8.56127\\
3.1344327836082	8.541519\\
3.13643178410795	8.521769\\
3.1384307846077	8.502018\\
3.14042978510745	8.482267\\
3.1424287856072	8.462515\\
3.14442778610695	8.442764\\
3.1464267866067	8.423012\\
3.14842578710645	8.403261\\
3.1504247876062	8.383509\\
3.15242378810595	8.363757\\
3.1544227886057	8.344006\\
3.15642178910545	8.324254\\
3.1584207896052	8.304502\\
3.16041979010495	8.28475\\
3.1624187906047	8.264998\\
3.16441779110445	8.245246\\
3.1664167916042	8.225494\\
3.16841579210395	8.205743\\
3.1704147926037	8.185991\\
3.17241379310345	8.166239\\
3.1744127936032	8.146488\\
3.17641179410295	8.126737\\
3.1784107946027	8.106985\\
3.18040979510245	8.087234\\
3.1824087956022	8.067483\\
3.18440779610195	8.047732\\
3.1864067966017	8.027982\\
3.18840579710145	8.008232\\
3.1904047976012	7.988481\\
3.19240379810095	7.968732\\
3.1944027986007	7.948982\\
3.19640179910045	7.929232\\
3.1984007996002	7.909483\\
3.20039980009995	7.889735\\
3.2023988005997	7.869986\\
3.20439780109945	7.850238\\
3.2063968015992	7.83049\\
3.20839580209895	7.810743\\
3.2103948025987	7.790996\\
3.21239380309845	7.771249\\
3.2143928035982	7.751503\\
3.21639180409795	7.731757\\
3.2183908045977	7.712011\\
3.22038980509745	7.692266\\
3.2223888055972	7.672522\\
3.22438780609695	7.652777\\
3.2263868065967	7.633034\\
3.22838580709645	7.613291\\
3.2303848075962	7.593548\\
3.23238380809595	7.573806\\
3.2343828085957	7.554064\\
3.23638180909545	7.534323\\
3.2383808095952	7.514583\\
3.24037981009495	7.494843\\
3.2423788105947	7.475104\\
3.24437781109445	7.455365\\
3.2463768115942	7.435627\\
3.24837581209395	7.415889\\
3.2503748125937	7.396152\\
3.25237381309345	7.376416\\
3.2543728135932	7.35668\\
3.25637181409295	7.336945\\
3.2583708145927	7.317211\\
3.26036981509245	7.297478\\
3.2623688155922	7.277745\\
3.26436781609195	7.258012\\
3.2663668165917	7.238281\\
3.26836581709145	7.21855\\
3.2703648175912	7.19882\\
3.27236381809095	7.179091\\
3.2743628185907	7.159362\\
3.27636181909045	7.139634\\
3.2783608195902	7.119907\\
3.28035982008995	7.100181\\
3.2823588205897	7.080455\\
3.28435782108946	7.060731\\
3.28635682158921	7.041007\\
3.28835582208896	7.021283\\
3.29035482258871	7.001561\\
3.29235382308846	6.98184\\
3.29435282358821	6.962119\\
3.29635182408796	6.942399\\
3.29835082458771	6.92268\\
3.30034982508746	6.902962\\
3.30234882558721	6.883244\\
3.30434782608696	6.863528\\
3.30634682658671	6.843812\\
3.30834582708646	6.824097\\
3.31034482758621	6.804383\\
3.31234382808596	6.78467\\
3.31434282858571	6.764958\\
3.31634182908546	6.745246\\
3.31834082958521	6.725536\\
3.32033983008496	6.705826\\
3.32233883058471	6.686118\\
3.32433783108446	6.66641\\
3.32633683158421	6.646703\\
3.32833583208396	6.626997\\
3.33033483258371	6.607291\\
3.33233383308346	6.587587\\
3.33433283358321	6.567884\\
3.33633183408296	6.548181\\
3.33833083458271	6.52848\\
3.34032983508246	6.508779\\
3.34232883558221	6.489079\\
3.34432783608196	6.46938\\
3.34632683658171	6.449682\\
3.34832583708146	6.429985\\
3.35032483758121	6.410288\\
3.35232383808096	6.390593\\
3.35432283858071	6.370898\\
3.35632183908046	6.351205\\
3.35832083958021	6.331512\\
3.36031984007996	6.31182\\
3.36231884057971	6.292129\\
3.36431784107946	6.272439\\
3.36631684157921	6.252749\\
3.36831584207896	6.233061\\
3.37031484257871	6.213373\\
3.37231384307846	6.193686\\
3.37431284357821	6.174001\\
3.37631184407796	6.154315\\
3.37831084457771	6.134631\\
3.38030984507746	6.114948\\
3.38230884557721	6.095265\\
3.38430784607696	6.075583\\
3.38630684657671	6.055903\\
3.38830584707646	6.036222\\
3.39030484757621	6.016543\\
3.39230384807596	5.996864\\
3.39430284857571	5.977187\\
3.39630184907546	5.95751\\
3.39830084957521	5.937833\\
3.40029985007496	5.918158\\
3.40229885057471	5.898483\\
3.40429785107446	5.878809\\
3.40629685157421	5.859136\\
3.40829585207396	5.839463\\
3.41029485257371	5.819791\\
3.41229385307346	5.80012\\
3.41429285357321	5.78045\\
3.41629185407296	5.76078\\
3.41829085457271	5.741111\\
3.42028985507246	5.721442\\
3.42228885557221	5.701775\\
3.42428785607196	5.682107\\
3.42628685657171	5.662441\\
3.42828585707146	5.642775\\
3.43028485757121	5.62311\\
3.43228385807096	5.603445\\
3.43428285857071	5.583781\\
3.43628185907046	5.564117\\
3.43828085957021	5.544454\\
3.44027986006996	5.524791\\
3.44227886056971	5.505129\\
3.44427786106947	5.485468\\
3.44627686156922	5.465806\\
3.44827586206897	5.446146\\
3.45027486256872	5.426486\\
3.45227386306847	5.406826\\
3.45427286356822	5.387167\\
3.45627186406797	5.367508\\
3.45827086456772	5.347849\\
3.46026986506747	5.328191\\
3.46226886556722	5.308533\\
3.46426786606697	5.288876\\
3.46626686656672	5.269219\\
3.46826586706647	5.249562\\
3.47026486756622	5.229906\\
3.47226386806597	5.210249\\
3.47426286856572	5.190594\\
3.47626186906547	5.170938\\
3.47826086956522	5.151282\\
3.48025987006497	5.131627\\
3.48225887056472	5.111972\\
3.48425787106447	5.092317\\
3.48625687156422	5.072662\\
3.48825587206397	5.053008\\
3.49025487256372	5.033353\\
3.49225387306347	5.013699\\
3.49425287356322	4.994044\\
3.49625187406297	4.97439\\
3.49825087456272	4.954736\\
3.50024987506247	4.935082\\
3.50224887556222	4.915427\\
3.50424787606197	4.895773\\
3.50624687656172	4.876119\\
3.50824587706147	4.856464\\
3.51024487756122	4.83681\\
3.51224387806097	4.817155\\
3.51424287856072	4.797501\\
3.51624187906047	4.777846\\
3.51824087956022	4.758191\\
3.52023988005997	4.738536\\
3.52223888055972	4.71888\\
3.52423788105947	4.699225\\
3.52623688155922	4.679569\\
3.52823588205897	4.659912\\
3.53023488255872	4.640256\\
3.53223388305847	4.620599\\
3.53423288355822	4.600942\\
3.53623188405797	4.581285\\
3.53823088455772	4.561627\\
3.54022988505747	4.541968\\
3.54222888555722	4.52231\\
3.54422788605697	4.502651\\
3.54622688655672	4.482991\\
3.54822588705647	4.463331\\
3.55022488755622	4.44367\\
3.55222388805597	4.424009\\
3.55422288855572	4.404347\\
3.55622188905547	4.384685\\
3.55822088955522	4.365022\\
3.56021989005497	4.345359\\
3.56221889055472	4.325694\\
3.56421789105447	4.30603\\
3.56621689155422	4.286364\\
3.56821589205397	4.266698\\
3.57021489255372	4.247031\\
3.57221389305347	4.227363\\
3.57421289355322	4.207694\\
3.57621189405297	4.188025\\
3.57821089455272	4.168355\\
3.58020989505247	4.148684\\
3.58220889555222	4.129012\\
3.58420789605197	4.109339\\
3.58620689655172	4.089666\\
3.58820589705147	4.069991\\
3.59020489755122	4.050315\\
3.59220389805097	4.030639\\
3.59420289855072	4.010961\\
3.59620189905047	3.991282\\
3.59820089955022	3.971603\\
3.60019990004997	3.951922\\
3.60219890054973	3.93224\\
3.60419790104948	3.912557\\
3.60619690154923	3.892873\\
3.60819590204898	3.873188\\
3.61019490254873	3.853501\\
3.61219390304848	3.833814\\
3.61419290354823	3.814125\\
3.61619190404798	3.794435\\
3.61819090454773	3.774743\\
3.62018990504748	3.755051\\
3.62218890554723	3.735357\\
3.62418790604698	3.715661\\
3.62618690654673	3.695965\\
3.62818590704648	3.676267\\
3.63018490754623	3.656567\\
3.63218390804598	3.636866\\
3.63418290854573	3.617164\\
3.63618190904548	3.59746\\
3.63818090954523	3.577755\\
3.64017991004498	3.558049\\
3.64217891054473	3.53834\\
3.64417791104448	3.518631\\
3.64617691154423	3.498919\\
3.64817591204398	3.479207\\
3.65017491254373	3.459492\\
3.65217391304348	3.439776\\
3.65417291354323	3.420059\\
3.65617191404298	3.400339\\
3.65817091454273	3.380618\\
3.66016991504248	3.360896\\
3.66216891554223	3.341171\\
3.66416791604198	3.321445\\
3.66616691654173	3.301718\\
3.66816591704148	3.281988\\
3.67016491754123	3.262257\\
3.67216391804098	3.242524\\
3.67416291854073	3.222789\\
3.67616191904048	3.203052\\
3.67816091954023	3.183314\\
3.68015992003998	3.163574\\
3.68215892053973	3.143831\\
3.68415792103948	3.124087\\
3.68615692153923	3.104341\\
3.68815592203898	3.084593\\
3.69015492253873	3.064844\\
3.69215392303848	3.045092\\
3.69415292353823	3.025338\\
3.69615192403798	3.005583\\
3.69815092453773	2.985825\\
3.70014992503748	2.966066\\
3.70214892553723	2.946304\\
3.70414792603698	2.926541\\
3.70614692653673	2.906775\\
3.70814592703648	2.887007\\
3.71014492753623	2.867238\\
3.71214392803598	2.847466\\
3.71414292853573	2.827692\\
3.71614192903548	2.807916\\
3.71814092953523	2.788138\\
3.72013993003498	2.768358\\
3.72213893053473	2.748576\\
3.72413793103448	2.728792\\
3.72613693153423	2.709006\\
3.72813593203398	2.689217\\
3.73013493253373	2.669426\\
3.73213393303348	2.649634\\
3.73413293353323	2.629839\\
3.73613193403298	2.610041\\
3.73813093453273	2.590242\\
3.74012993503248	2.570441\\
3.74212893553223	2.550637\\
3.74412793603198	2.530831\\
3.74612693653173	2.511023\\
3.74812593703148	2.491213\\
3.75012493753123	2.4714\\
3.75212393803098	2.451586\\
3.75412293853073	2.431769\\
3.75612193903048	2.41195\\
3.75812093953023	2.392128\\
3.76011994002998	2.372305\\
3.76211894052974	2.352479\\
3.76411794102949	2.332651\\
3.76611694152924	2.312821\\
3.76811594202899	2.292988\\
3.77011494252874	2.273154\\
3.77211394302849	2.253317\\
3.77411294352824	2.233478\\
3.77611194402799	2.213636\\
3.77811094452774	2.193793\\
3.78010994502749	2.173947\\
3.78210894552724	2.154099\\
3.78410794602699	2.134248\\
3.78610694652674	2.114396\\
3.78810594702649	2.094541\\
3.79010494752624	2.074684\\
3.79210394802599	2.054825\\
3.79410294852574	2.034964\\
3.79610194902549	2.0151\\
3.79810094952524	1.995234\\
3.80009995002499	1.975366\\
3.80209895052474	1.955496\\
3.80409795102449	1.935624\\
3.80609695152424	1.915749\\
3.80809595202399	1.895873\\
3.81009495252374	1.875994\\
3.81209395302349	1.856113\\
3.81409295352324	1.836229\\
3.81609195402299	1.816344\\
3.81809095452274	1.796457\\
3.82008995502249	1.776567\\
3.82208895552224	1.756675\\
3.82408795602199	1.736782\\
3.82608695652174	1.716886\\
3.82808595702149	1.696988\\
3.83008495752124	1.677088\\
3.83208395802099	1.657185\\
3.83408295852074	1.637281\\
3.83608195902049	1.617375\\
3.83808095952024	1.597467\\
3.84007996001999	1.577557\\
3.84207896051974	1.557644\\
3.84407796101949	1.53773\\
3.84607696151924	1.517814\\
3.84807596201899	1.497896\\
3.85007496251874	1.477976\\
3.85207396301849	1.458053\\
3.85407296351824	1.43813\\
3.85607196401799	1.418204\\
3.85807096451774	1.398276\\
3.86006996501749	1.378346\\
3.86206896551724	1.358415\\
3.86406796601699	1.338482\\
3.86606696651674	1.318547\\
3.86806596701649	1.29861\\
3.87006496751624	1.278671\\
3.87206396801599	1.258731\\
3.87406296851574	1.238789\\
3.87606196901549	1.218845\\
3.87806096951524	1.198899\\
3.88005997001499	1.178952\\
3.88205897051474	1.159003\\
3.88405797101449	1.139053\\
3.88605697151424	1.119101\\
3.88805597201399	1.099147\\
3.89005497251374	1.079192\\
3.89205397301349	1.059235\\
3.89405297351324	1.039277\\
3.89605197401299	1.019317\\
3.89805097451274	0.999356\\
3.90004997501249	0.979393\\
3.90204897551224	0.959429\\
3.90404797601199	0.939463\\
3.90604697651174	0.919496\\
3.90804597701149	0.899528\\
3.91004497751124	0.879558\\
3.91204397801099	0.859587\\
3.91404297851074	0.839615\\
3.91604197901049	0.819642\\
3.91804097951024	0.799667\\
3.92003998000999	0.779691\\
3.92203898050975	0.759714\\
3.9240379810095	0.739736\\
3.92603698150925	0.719756\\
3.928035982009	0.699776\\
3.93003498250875	0.679794\\
3.9320339830085	0.659812\\
3.93403298350825	0.639828\\
3.936031984008	0.619844\\
3.93803098450775	0.599858\\
3.9400299850075	0.579872\\
3.94202898550725	0.559885\\
3.944027986007	0.539897\\
3.94602698650675	0.519908\\
3.9480259870065	0.499918\\
3.95002498750625	0.479927\\
3.952023988006	0.459936\\
3.95402298850575	0.439944\\
3.9560219890055	0.419951\\
3.95802098950525	0.399958\\
3.960019990005	0.379964\\
3.96201899050475	0.359969\\
3.9640179910045	0.339974\\
3.96601699150425	0.319978\\
3.968015992004	0.299982\\
3.97001499250375	0.279985\\
3.9720139930035	0.259988\\
3.97401299350325	0.239991\\
3.976011994003	0.219993\\
3.97801099450275	0.199995\\
3.9800099950025	0.179996\\
3.98200899550225	0.159997\\
3.984007996002	0.139998\\
3.98600699650175	0.119999\\
3.9880059970015	0.099999\\
3.99000499750125	0.08\\
3.992003998001	0.06\\
3.99400299850075	0.04\\
3.9960019990005	0.02\\
3.99800099950025	-0\\
4	-0.02\\
};
\addlegendentry{$\psi$};

\addplot [color=mycolor3,solid]
  table[row sep=crcr]{%
0	20.278241\\
0.00199900049975012	20.27743\\
0.00399800099950025	20.276613\\
0.00599700149925037	20.275789\\
0.0079960019990005	20.274959\\
0.00999500249875063	20.274122\\
0.0119940029985007	20.273279\\
0.0139930034982509	20.272429\\
0.015992003998001	20.271573\\
0.0179910044977511	20.270711\\
0.0199900049975013	20.269843\\
0.0219890054972514	20.268969\\
0.0239880059970015	20.268088\\
0.0259870064967516	20.267202\\
0.0279860069965017	20.266309\\
0.0299850074962519	20.265411\\
0.031984007996002	20.264507\\
0.0339830084957521	20.263597\\
0.0359820089955022	20.262681\\
0.0379810094952524	20.261759\\
0.0399800099950025	20.260832\\
0.0419790104947526	20.259899\\
0.0439780109945027	20.258961\\
0.0459770114942529	20.258017\\
0.047976011994003	20.257068\\
0.0499750124937531	20.256113\\
0.0519740129935032	20.255153\\
0.0539730134932534	20.254188\\
0.0559720139930035	20.253218\\
0.0579710144927536	20.252242\\
0.0599700149925037	20.251261\\
0.0619690154922539	20.250275\\
0.063968015992004	20.249284\\
0.0659670164917541	20.248289\\
0.0679660169915042	20.247288\\
0.0699650174912544	20.246283\\
0.0719640179910045	20.245272\\
0.0739630184907546	20.244258\\
0.0759620189905048	20.243238\\
0.0779610194902549	20.242214\\
0.079960019990005	20.241185\\
0.0819590204897551	20.240152\\
0.0839580209895052	20.239114\\
0.0859570214892554	20.238073\\
0.0879560219890055	20.237026\\
0.0899550224887556	20.235976\\
0.0919540229885057	20.234921\\
0.0939530234882559	20.233863\\
0.095952023988006	20.2328\\
0.0979510244877561	20.231733\\
0.0999500249875062	20.230662\\
0.101949025487256	20.229588\\
0.103948025987006	20.228509\\
0.105947026486757	20.227427\\
0.107946026986507	20.226341\\
0.109945027486257	20.225252\\
0.111944027986007	20.224159\\
0.113943028485757	20.223062\\
0.115942028985507	20.221962\\
0.117941029485257	20.220859\\
0.119940029985007	20.219752\\
0.121939030484758	20.218642\\
0.123938030984508	20.217529\\
0.125937031484258	20.216413\\
0.127936031984008	20.215293\\
0.129935032483758	20.214171\\
0.131934032983508	20.213046\\
0.133933033483258	20.211917\\
0.135932033983008	20.210786\\
0.137931034482759	20.209653\\
0.139930034982509	20.208516\\
0.141929035482259	20.207377\\
0.143928035982009	20.206235\\
0.145927036481759	20.205091\\
0.147926036981509	20.203945\\
0.149925037481259	20.202796\\
0.15192403798101	20.201644\\
0.15392303848076	20.200491\\
0.15592203898051	20.199335\\
0.15792103948026	20.198177\\
0.15992003998001	20.197017\\
0.16191904047976	20.195856\\
0.16391804097951	20.194692\\
0.16591704147926	20.193526\\
0.16791604197901	20.192359\\
0.169915042478761	20.191189\\
0.171914042978511	20.190019\\
0.173913043478261	20.188846\\
0.175912043978011	20.187672\\
0.177911044477761	20.186497\\
0.179910044977511	20.18532\\
0.181909045477261	20.184141\\
0.183908045977011	20.182962\\
0.185907046476762	20.181781\\
0.187906046976512	20.180599\\
0.189905047476262	20.179416\\
0.191904047976012	20.178232\\
0.193903048475762	20.177047\\
0.195902048975512	20.175862\\
0.197901049475262	20.174675\\
0.199900049975012	20.173487\\
0.201899050474763	20.172299\\
0.203898050974513	20.171111\\
0.205897051474263	20.169921\\
0.207896051974013	20.168731\\
0.209895052473763	20.167541\\
0.211894052973513	20.166351\\
0.213893053473263	20.16516\\
0.215892053973014	20.163969\\
0.217891054472764	20.162777\\
0.219890054972514	20.161586\\
0.221889055472264	20.160394\\
0.223888055972014	20.159203\\
0.225887056471764	20.158011\\
0.227886056971514	20.15682\\
0.229885057471264	20.155629\\
0.231884057971014	20.154438\\
0.233883058470765	20.153248\\
0.235882058970515	20.152058\\
0.237881059470265	20.150868\\
0.239880059970015	20.14968\\
0.241879060469765	20.148491\\
0.243878060969515	20.147303\\
0.245877061469265	20.146117\\
0.247876061969015	20.14493\\
0.249875062468766	20.143745\\
0.251874062968516	20.142561\\
0.253873063468266	20.141377\\
0.255872063968016	20.140195\\
0.257871064467766	20.139014\\
0.259870064967516	20.137834\\
0.261869065467266	20.136655\\
0.263868065967017	20.135478\\
0.265867066466767	20.134301\\
0.267866066966517	20.133127\\
0.269865067466267	20.131954\\
0.271864067966017	20.130782\\
0.273863068465767	20.129612\\
0.275862068965517	20.128444\\
0.277861069465267	20.127278\\
0.279860069965017	20.126113\\
0.281859070464768	20.12495\\
0.283858070964518	20.123789\\
0.285857071464268	20.122631\\
0.287856071964018	20.121474\\
0.289855072463768	20.120319\\
0.291854072963518	20.119167\\
0.293853073463268	20.118017\\
0.295852073963018	20.116869\\
0.297851074462769	20.115724\\
0.299850074962519	20.114581\\
0.301849075462269	20.11344\\
0.303848075962019	20.112302\\
0.305847076461769	20.111167\\
0.307846076961519	20.110035\\
0.309845077461269	20.108905\\
0.311844077961019	20.107778\\
0.31384307846077	20.106654\\
0.31584207896052	20.105533\\
0.31784107946027	20.104415\\
0.31984007996002	20.1033\\
0.32183908045977	20.102188\\
0.32383808095952	20.101079\\
0.32583708145927	20.099974\\
0.32783608195902	20.098872\\
0.329835082458771	20.097773\\
0.331834082958521	20.096678\\
0.333833083458271	20.095586\\
0.335832083958021	20.094497\\
0.337831084457771	20.093413\\
0.339830084957521	20.092332\\
0.341829085457271	20.091255\\
0.343828085957021	20.090181\\
0.345827086456772	20.089111\\
0.347826086956522	20.088046\\
0.349825087456272	20.086984\\
0.351824087956022	20.085926\\
0.353823088455772	20.084873\\
0.355822088955522	20.083823\\
0.357821089455272	20.082778\\
0.359820089955023	20.081737\\
0.361819090454773	20.0807\\
0.363818090954523	20.079667\\
0.365817091454273	20.07864\\
0.367816091954023	20.077616\\
0.369815092453773	20.076597\\
0.371814092953523	20.075583\\
0.373813093453273	20.074573\\
0.375812093953024	20.073568\\
0.377811094452774	20.072568\\
0.379810094952524	20.071572\\
0.381809095452274	20.070582\\
0.383808095952024	20.069596\\
0.385807096451774	20.068616\\
0.387806096951524	20.06764\\
0.389805097451274	20.06667\\
0.391804097951024	20.065704\\
0.393803098450775	20.064744\\
0.395802098950525	20.063789\\
0.397801099450275	20.06284\\
0.399800099950025	20.061895\\
0.401799100449775	20.060957\\
0.403798100949525	20.060023\\
0.405797101449275	20.059095\\
0.407796101949025	20.058173\\
0.409795102448776	20.057256\\
0.411794102948526	20.056345\\
0.413793103448276	20.05544\\
0.415792103948026	20.054541\\
0.417791104447776	20.053647\\
0.419790104947526	20.052759\\
0.421789105447276	20.051877\\
0.423788105947026	20.051001\\
0.425787106446777	20.050131\\
0.427786106946527	20.049267\\
0.429785107446277	20.04841\\
0.431784107946027	20.047558\\
0.433783108445777	20.046713\\
0.435782108945527	20.045874\\
0.437781109445277	20.045041\\
0.439780109945027	20.044214\\
0.441779110444778	20.043394\\
0.443778110944528	20.042581\\
0.445777111444278	20.041774\\
0.447776111944028	20.040973\\
0.449775112443778	20.040179\\
0.451774112943528	20.039392\\
0.453773113443278	20.038611\\
0.455772113943028	20.037837\\
0.457771114442779	20.03707\\
0.459770114942529	20.036309\\
0.461769115442279	20.035556\\
0.463768115942029	20.034809\\
0.465767116441779	20.034069\\
0.467766116941529	20.033337\\
0.469765117441279	20.032611\\
0.471764117941029	20.031892\\
0.47376311844078	20.031181\\
0.47576211894053	20.030476\\
0.47776111944028	20.029779\\
0.47976011994003	20.029089\\
0.48175912043978	20.028406\\
0.48375812093953	20.027731\\
0.48575712143928	20.027062\\
0.487756121939031	20.026402\\
0.489755122438781	20.025748\\
0.491754122938531	20.025102\\
0.493753123438281	20.024464\\
0.495752123938031	20.023833\\
0.497751124437781	20.02321\\
0.499750124937531	20.022594\\
0.501749125437281	20.021986\\
0.503748125937031	20.021385\\
0.505747126436782	20.020793\\
0.507746126936532	20.020208\\
0.509745127436282	20.01963\\
0.511744127936032	20.019061\\
0.513743128435782	20.018499\\
0.515742128935532	20.017945\\
0.517741129435282	20.0174\\
0.519740129935032	20.016862\\
0.521739130434783	20.016332\\
0.523738130934533	20.01581\\
0.525737131434283	20.015296\\
0.527736131934033	20.01479\\
0.529735132433783	20.014292\\
0.531734132933533	20.013802\\
0.533733133433283	20.013321\\
0.535732133933034	20.012848\\
0.537731134432784	20.012382\\
0.539730134932534	20.011925\\
0.541729135432284	20.011477\\
0.543728135932034	20.011036\\
0.545727136431784	20.010604\\
0.547726136931534	20.01018\\
0.549725137431284	20.009765\\
0.551724137931034	20.009358\\
0.553723138430785	20.008959\\
0.555722138930535	20.008569\\
0.557721139430285	20.008187\\
0.559720139930035	20.007814\\
0.561719140429785	20.007449\\
0.563718140929535	20.007092\\
0.565717141429285	20.006745\\
0.567716141929036	20.006405\\
0.569715142428786	20.006075\\
0.571714142928536	20.005752\\
0.573713143428286	20.005439\\
0.575712143928036	20.005134\\
0.577711144427786	20.004838\\
0.579710144927536	20.00455\\
0.581709145427286	20.004271\\
0.583708145927036	20.004001\\
0.585707146426787	20.00374\\
0.587706146926537	20.003487\\
0.589705147426287	20.003243\\
0.591704147926037	20.003008\\
0.593703148425787	20.002781\\
0.595702148925537	20.002564\\
0.597701149425287	20.002355\\
0.599700149925038	20.002155\\
0.601699150424788	20.001963\\
0.603698150924538	20.001781\\
0.605697151424288	20.001607\\
0.607696151924038	20.001443\\
0.609695152423788	20.001287\\
0.611694152923538	20.00114\\
0.613693153423288	20.001002\\
0.615692153923038	20.000872\\
0.617691154422789	20.000752\\
0.619690154922539	20.000641\\
0.621689155422289	20.000538\\
0.623688155922039	20.000445\\
0.625687156421789	20.00036\\
0.627686156921539	20.000284\\
0.629685157421289	20.000217\\
0.631684157921039	20.000159\\
0.63368315842079	20.000111\\
0.63568215892054	20.000071\\
0.63768115942029	20.000039\\
0.63968015992004	20.000017\\
0.64167916041979	20.000004\\
0.64367816091954	20\\
0.64567716141929	20.000005\\
0.64767616191904	20.000018\\
0.649675162418791	20.000041\\
0.651674162918541	20.000073\\
0.653673163418291	20.000113\\
0.655672163918041	20.000163\\
0.657671164417791	20.000221\\
0.659670164917541	20.000288\\
0.661669165417291	20.000365\\
0.663668165917041	20.00045\\
0.665667166416792	20.000544\\
0.667666166916542	20.000647\\
0.669665167416292	20.000759\\
0.671664167916042	20.00088\\
0.673663168415792	20.00101\\
0.675662168915542	20.001148\\
0.677661169415292	20.001296\\
0.679660169915043	20.001452\\
0.681659170414793	20.001617\\
0.683658170914543	20.001791\\
0.685657171414293	20.001974\\
0.687656171914043	20.002166\\
0.689655172413793	20.002367\\
0.691654172913543	20.002576\\
0.693653173413293	20.002794\\
0.695652173913043	20.003021\\
0.697651174412794	20.003257\\
0.699650174912544	20.003502\\
0.701649175412294	20.003755\\
0.703648175912044	20.004017\\
0.705647176411794	20.004288\\
0.707646176911544	20.004567\\
0.709645177411294	20.004855\\
0.711644177911045	20.005152\\
0.713643178410795	20.005457\\
0.715642178910545	20.005771\\
0.717641179410295	20.006094\\
0.719640179910045	20.006425\\
0.721639180409795	20.006765\\
0.723638180909545	20.007113\\
0.725637181409295	20.00747\\
0.727636181909045	20.007835\\
0.729635182408796	20.008209\\
0.731634182908546	20.008592\\
0.733633183408296	20.008982\\
0.735632183908046	20.009382\\
0.737631184407796	20.009789\\
0.739630184907546	20.010205\\
0.741629185407296	20.010629\\
0.743628185907046	20.011062\\
0.745627186406797	20.011503\\
0.747626186906547	20.011952\\
0.749625187406297	20.01241\\
0.751624187906047	20.012875\\
0.753623188405797	20.013349\\
0.755622188905547	20.013831\\
0.757621189405297	20.014321\\
0.759620189905047	20.01482\\
0.761619190404798	20.015326\\
0.763618190904548	20.01584\\
0.765617191404298	20.016363\\
0.767616191904048	20.016893\\
0.769615192403798	20.017432\\
0.771614192903548	20.017978\\
0.773613193403298	20.018532\\
0.775612193903048	20.019094\\
0.777611194402799	20.019664\\
0.779610194902549	20.020242\\
0.781609195402299	20.020827\\
0.783608195902049	20.021421\\
0.785607196401799	20.022021\\
0.787606196901549	20.02263\\
0.789605197401299	20.023246\\
0.79160419790105	20.02387\\
0.7936031984008	20.024501\\
0.79560219890055	20.02514\\
0.7976011994003	20.025787\\
0.79960019990005	20.02644\\
0.8015992003998	20.027102\\
0.80359820089955	20.02777\\
0.8055972013993	20.028446\\
0.80759620189905	20.029129\\
0.809595202398801	20.02982\\
0.811594202898551	20.030518\\
0.813593203398301	20.031222\\
0.815592203898051	20.031934\\
0.817591204397801	20.032654\\
0.819590204897551	20.03338\\
0.821589205397301	20.034113\\
0.823588205897052	20.034853\\
0.825587206396802	20.0356\\
0.827586206896552	20.036354\\
0.829585207396302	20.037115\\
0.831584207896052	20.037883\\
0.833583208395802	20.038657\\
0.835582208895552	20.039438\\
0.837581209395302	20.040226\\
0.839580209895052	20.04102\\
0.841579210394803	20.041821\\
0.843578210894553	20.042629\\
0.845577211394303	20.043443\\
0.847576211894053	20.044263\\
0.849575212393803	20.04509\\
0.851574212893553	20.045923\\
0.853573213393303	20.046762\\
0.855572213893053	20.047608\\
0.857571214392804	20.04846\\
0.859570214892554	20.049318\\
0.861569215392304	20.050182\\
0.863568215892054	20.051053\\
0.865567216391804	20.051929\\
0.867566216891554	20.052811\\
0.869565217391304	20.053699\\
0.871564217891054	20.054593\\
0.873563218390805	20.055493\\
0.875562218890555	20.056399\\
0.877561219390305	20.05731\\
0.879560219890055	20.058227\\
0.881559220389805	20.05915\\
0.883558220889555	20.060078\\
0.885557221389305	20.061012\\
0.887556221889055	20.061951\\
0.889555222388806	20.062896\\
0.891554222888556	20.063845\\
0.893553223388306	20.064801\\
0.895552223888056	20.065761\\
0.897551224387806	20.066727\\
0.899550224887556	20.067698\\
0.901549225387306	20.068673\\
0.903548225887057	20.069654\\
0.905547226386807	20.07064\\
0.907546226886557	20.071631\\
0.909545227386307	20.072627\\
0.911544227886057	20.073627\\
0.913543228385807	20.074633\\
0.915542228885557	20.075643\\
0.917541229385307	20.076657\\
0.919540229885057	20.077676\\
0.921539230384808	20.0787\\
0.923538230884558	20.079728\\
0.925537231384308	20.080761\\
0.927536231884058	20.081798\\
0.929535232383808	20.082839\\
0.931534232883558	20.083885\\
0.933533233383308	20.084935\\
0.935532233883059	20.085989\\
0.937531234382809	20.087047\\
0.939530234882559	20.088109\\
0.941529235382309	20.089174\\
0.943528235882059	20.090244\\
0.945527236381809	20.091318\\
0.947526236881559	20.092396\\
0.949525237381309	20.093477\\
0.951524237881059	20.094562\\
0.95352323838081	20.09565\\
0.95552223888056	20.096742\\
0.95752123938031	20.097838\\
0.95952023988006	20.098937\\
0.96151924037981	20.100039\\
0.96351824087956	20.101145\\
0.96551724137931	20.102253\\
0.96751624187906	20.103366\\
0.969515242378811	20.104481\\
0.971514242878561	20.105599\\
0.973513243378311	20.10672\\
0.975512243878061	20.107844\\
0.977511244377811	20.108972\\
0.979510244877561	20.110101\\
0.981509245377311	20.111234\\
0.983508245877061	20.11237\\
0.985507246376812	20.113508\\
0.987506246876562	20.114648\\
0.989505247376312	20.115791\\
0.991504247876062	20.116937\\
0.993503248375812	20.118085\\
0.995502248875562	20.119235\\
0.997501249375312	20.120387\\
0.999500249875062	20.121542\\
1.00149925037481	20.122699\\
1.00349825087456	20.123858\\
1.00549725137431	20.125019\\
1.00749625187406	20.126182\\
1.00949525237381	20.127346\\
1.01149425287356	20.128513\\
1.01349325337331	20.129681\\
1.01549225387306	20.130851\\
1.01749125437281	20.132023\\
1.01949025487256	20.133196\\
1.02148925537231	20.134371\\
1.02348825587206	20.135547\\
1.02548725637181	20.136725\\
1.02748625687156	20.137903\\
1.02948525737131	20.139084\\
1.03148425787106	20.140265\\
1.03348325837081	20.141447\\
1.03548225887056	20.142631\\
1.03748125937031	20.143815\\
1.03948025987006	20.145\\
1.04147926036982	20.146187\\
1.04347826086957	20.147374\\
1.04547726136932	20.148561\\
1.04747626186907	20.14975\\
1.04947526236882	20.150939\\
1.05147426286857	20.152128\\
1.05347326336832	20.153318\\
1.05547226386807	20.154509\\
1.05747126436782	20.155699\\
1.05947026486757	20.15689\\
1.06146926536732	20.158082\\
1.06346826586707	20.159273\\
1.06546726636682	20.160465\\
1.06746626686657	20.161656\\
1.06946526736632	20.162848\\
1.07146426786607	20.164039\\
1.07346326836582	20.16523\\
1.07546226886557	20.166421\\
1.07746126936532	20.167612\\
1.07946026986507	20.168802\\
1.08145927036482	20.169992\\
1.08345827086457	20.171181\\
1.08545727136432	20.17237\\
1.08745627186407	20.173558\\
1.08945527236382	20.174745\\
1.09145427286357	20.175932\\
1.09345327336332	20.177117\\
1.09545227386307	20.178302\\
1.09745127436282	20.179486\\
1.09945027486257	20.180669\\
1.10144927536232	20.181851\\
1.10344827586207	20.183032\\
1.10544727636182	20.184211\\
1.10744627686157	20.185389\\
1.10944527736132	20.186566\\
1.11144427786107	20.187741\\
1.11344327836082	20.188915\\
1.11544227886057	20.190088\\
1.11744127936032	20.191259\\
1.11944027986007	20.192428\\
1.12143928035982	20.193595\\
1.12343828085957	20.194761\\
1.12543728135932	20.195924\\
1.12743628185907	20.197086\\
1.12943528235882	20.198246\\
1.13143428285857	20.199404\\
1.13343328335832	20.200559\\
1.13543228385807	20.201712\\
1.13743128435782	20.202864\\
1.13943028485757	20.204012\\
1.14142928535732	20.205159\\
1.14342828585707	20.206303\\
1.14542728635682	20.207444\\
1.14742628685657	20.208583\\
1.14942528735632	20.20972\\
1.15142428785607	20.210853\\
1.15342328835582	20.211984\\
1.15542228885557	20.213112\\
1.15742128935532	20.214237\\
1.15942028985507	20.21536\\
1.16141929035482	20.216479\\
1.16341829085457	20.217595\\
1.16541729135432	20.218708\\
1.16741629185407	20.219818\\
1.16941529235382	20.220924\\
1.17141429285357	20.222027\\
1.17341329335332	20.223127\\
1.17541229385307	20.224223\\
1.17741129435282	20.225316\\
1.17941029485257	20.226405\\
1.18140929535232	20.227491\\
1.18340829585207	20.228573\\
1.18540729635182	20.229651\\
1.18740629685157	20.230726\\
1.18940529735132	20.231796\\
1.19140429785107	20.232863\\
1.19340329835082	20.233925\\
1.19540229885057	20.234984\\
1.19740129935032	20.236038\\
1.19940029985008	20.237088\\
1.20139930034983	20.238134\\
1.20339830084958	20.239176\\
1.20539730134933	20.240213\\
1.20739630184908	20.241246\\
1.20939530234883	20.242274\\
1.21139430284858	20.243298\\
1.21339330334833	20.244318\\
1.21539230384808	20.245332\\
1.21739130434783	20.246342\\
1.21939030484758	20.247347\\
1.22138930534733	20.248348\\
1.22338830584708	20.249343\\
1.22538730634683	20.250334\\
1.22738630684658	20.251319\\
1.22938530734633	20.2523\\
1.23138430784608	20.253275\\
1.23338330834583	20.254245\\
1.23538230884558	20.25521\\
1.23738130934533	20.25617\\
1.23938030984508	20.257124\\
1.24137931034483	20.258073\\
1.24337831084458	20.259017\\
1.24537731134433	20.259955\\
1.24737631184408	20.260887\\
1.24937531234383	20.261814\\
1.25137431284358	20.262735\\
1.25337331334333	20.263651\\
1.25537231384308	20.26456\\
1.25737131434283	20.265464\\
1.25937031484258	20.266362\\
1.26136931534233	20.267254\\
1.26336831584208	20.268141\\
1.26536731634183	20.269021\\
1.26736631684158	20.269895\\
1.26936531734133	20.270762\\
1.27136431784108	20.271624\\
1.27336331834083	20.27248\\
1.27536231884058	20.273329\\
1.27736131934033	20.274172\\
1.27936031984008	20.275008\\
1.28135932033983	20.275838\\
1.28335832083958	20.276662\\
1.28535732133933	20.277478\\
1.28735632183908	20.278289\\
1.28935532233883	20.279093\\
1.29135432283858	20.27989\\
1.29335332333833	20.28068\\
1.29535232383808	20.281464\\
1.29735132433783	20.282241\\
1.29935032483758	20.28301\\
1.30134932533733	20.283773\\
1.30334832583708	20.284529\\
1.30534732633683	20.285279\\
1.30734632683658	20.286021\\
1.30934532733633	20.286755\\
1.31134432783608	20.287483\\
1.31334332833583	20.288204\\
1.31534232883558	20.288917\\
1.31734132933533	20.289623\\
1.31934032983508	20.290322\\
1.32133933033483	20.291014\\
1.32333833083458	20.291698\\
1.32533733133433	20.292374\\
1.32733633183408	20.293044\\
1.32933533233383	20.293705\\
1.33133433283358	20.294359\\
1.33333333333333	20.295006\\
1.33533233383308	20.295645\\
1.33733133433283	20.296276\\
1.33933033483258	20.296899\\
1.34132933533233	20.297515\\
1.34332833583208	20.298123\\
1.34532733633183	20.298723\\
1.34732633683158	20.299316\\
1.34932533733133	20.2999\\
1.35132433783108	20.300477\\
1.35332333833083	20.301045\\
1.35532233883058	20.301606\\
1.35732133933033	20.302158\\
1.35932033983009	20.302703\\
1.36131934032984	20.303239\\
1.36331834082959	20.303767\\
1.36531734132934	20.304287\\
1.36731634182909	20.304799\\
1.36931534232884	20.305303\\
1.37131434282859	20.305798\\
1.37331334332834	20.306285\\
1.37531234382809	20.306764\\
1.37731134432784	20.307234\\
1.37931034482759	20.307697\\
1.38130934532734	20.30815\\
1.38330834582709	20.308595\\
1.38530734632684	20.309032\\
1.38730634682659	20.30946\\
1.38930534732634	20.30988\\
1.39130434782609	20.310291\\
1.39330334832584	20.310694\\
1.39530234882559	20.311088\\
1.39730134932534	20.311474\\
1.39930034982509	20.31185\\
1.40129935032484	20.312219\\
1.40329835082459	20.312578\\
1.40529735132434	20.312929\\
1.40729635182409	20.313271\\
1.40929535232384	20.313604\\
1.41129435282359	20.313929\\
1.41329335332334	20.314245\\
1.41529235382309	20.314552\\
1.41729135432284	20.31485\\
1.41929035482259	20.315139\\
1.42128935532234	20.315419\\
1.42328835582209	20.315691\\
1.42528735632184	20.315954\\
1.42728635682159	20.316207\\
1.42928535732134	20.316452\\
1.43128435782109	20.316688\\
1.43328335832084	20.316915\\
1.43528235882059	20.317133\\
1.43728135932034	20.317342\\
1.43928035982009	20.317542\\
1.44127936031984	20.317733\\
1.44327836081959	20.317914\\
1.44527736131934	20.318087\\
1.44727636181909	20.318251\\
1.44927536231884	20.318406\\
1.45127436281859	20.318552\\
1.45327336331834	20.318688\\
1.45527236381809	20.318816\\
1.45727136431784	20.318934\\
1.45927036481759	20.319044\\
1.46126936531734	20.319144\\
1.46326836581709	20.319235\\
1.46526736631684	20.319317\\
1.46726636681659	20.31939\\
1.46926536731634	20.319454\\
1.47126436781609	20.319508\\
1.47326336831584	20.319554\\
1.47526236881559	20.31959\\
1.47726136931534	20.319617\\
1.47926036981509	20.319635\\
1.48125937031484	20.319644\\
1.48325837081459	20.319644\\
1.48525737131434	20.319635\\
1.48725637181409	20.319616\\
1.48925537231384	20.319589\\
1.49125437281359	20.319552\\
1.49325337331334	20.319506\\
1.49525237381309	20.319451\\
1.49725137431284	20.319387\\
1.49925037481259	20.319314\\
1.50124937531234	20.319231\\
1.50324837581209	20.31914\\
1.50524737631184	20.319039\\
1.50724637681159	20.318929\\
1.50924537731134	20.31881\\
1.51124437781109	20.318682\\
1.51324337831084	20.318545\\
1.51524237881059	20.318399\\
1.51724137931034	20.318244\\
1.51924037981009	20.31808\\
1.52123938030985	20.317907\\
1.5232383808096	20.317724\\
1.52523738130935	20.317533\\
1.5272363818091	20.317333\\
1.52923538230885	20.317123\\
1.5312343828086	20.316905\\
1.53323338330835	20.316678\\
1.5352323838081	20.316441\\
1.53723138430785	20.316196\\
1.5392303848076	20.315942\\
1.54122938530735	20.315679\\
1.5432283858071	20.315407\\
1.54522738630685	20.315126\\
1.5472263868066	20.314837\\
1.54922538730635	20.314538\\
1.5512243878061	20.314231\\
1.55322338830585	20.313915\\
1.5552223888056	20.31359\\
1.55722138930535	20.313256\\
1.5592203898051	20.312914\\
1.56121939030485	20.312562\\
1.5632183908046	20.312203\\
1.56521739130435	20.311834\\
1.5672163918041	20.311457\\
1.56921539230385	20.311071\\
1.5712143928036	20.310676\\
1.57321339330335	20.310273\\
1.5752123938031	20.309862\\
1.57721139430285	20.309442\\
1.5792103948026	20.309013\\
1.58120939530235	20.308576\\
1.5832083958021	20.30813\\
1.58520739630185	20.307676\\
1.5872063968016	20.307214\\
1.58920539730135	20.306743\\
1.5912043978011	20.306264\\
1.59320339830085	20.305776\\
1.5952023988006	20.305281\\
1.59720139930035	20.304777\\
1.5992003998001	20.304264\\
1.60119940029985	20.303744\\
1.6031984007996	20.303215\\
1.60519740129935	20.302679\\
1.6071964017991	20.302134\\
1.60919540229885	20.301581\\
1.6111944027986	20.30102\\
1.61319340329835	20.300451\\
1.6151924037981	20.299874\\
1.61719140429785	20.29929\\
1.6191904047976	20.298697\\
1.62118940529735	20.298096\\
1.6231884057971	20.297488\\
1.62518740629685	20.296872\\
1.6271864067966	20.296248\\
1.62918540729635	20.295617\\
1.6311844077961	20.294977\\
1.63318340829585	20.29433\\
1.6351824087956	20.293676\\
1.63718140929535	20.293014\\
1.6391804097951	20.292345\\
1.64117941029485	20.291668\\
1.6431784107946	20.290983\\
1.64517741129435	20.290291\\
1.6471764117941	20.289592\\
1.64917541229385	20.288886\\
1.6511744127936	20.288172\\
1.65317341329335	20.287451\\
1.6551724137931	20.286723\\
1.65717141429285	20.285988\\
1.6591704147926	20.285246\\
1.66116941529235	20.284496\\
1.6631684157921	20.28374\\
1.66516741629185	20.282976\\
1.6671664167916	20.282206\\
1.66916541729135	20.281429\\
1.6711644177911	20.280645\\
1.67316341829085	20.279855\\
1.6751624187906	20.279057\\
1.67716141929035	20.278253\\
1.6791604197901	20.277442\\
1.68115942028985	20.276625\\
1.68315842078961	20.275801\\
1.68515742128936	20.274971\\
1.68715642178911	20.274134\\
1.68915542228886	20.273291\\
1.69115442278861	20.272442\\
1.69315342328836	20.271586\\
1.69515242378811	20.270724\\
1.69715142428786	20.269856\\
1.69915042478761	20.268982\\
1.70114942528736	20.268101\\
1.70314842578711	20.267215\\
1.70514742628686	20.266323\\
1.70714642678661	20.265424\\
1.70914542728636	20.26452\\
1.71114442778611	20.26361\\
1.71314342828586	20.262695\\
1.71514242878561	20.261773\\
1.71714142928536	20.260846\\
1.71914042978511	20.259913\\
1.72113943028486	20.258975\\
1.72313843078461	20.258031\\
1.72513743128436	20.257082\\
1.72713643178411	20.256127\\
1.72913543228386	20.255168\\
1.73113443278361	20.254202\\
1.73313343328336	20.253232\\
1.73513243378311	20.252256\\
1.73713143428286	20.251276\\
1.73913043478261	20.25029\\
1.74112943528236	20.249299\\
1.74312843578211	20.248303\\
1.74512743628186	20.247303\\
1.74712643678161	20.246298\\
1.74912543728136	20.245287\\
1.75112443778111	20.244273\\
1.75312343828086	20.243253\\
1.75512243878061	20.242229\\
1.75712143928036	20.2412\\
1.75912043978011	20.240167\\
1.76111944027986	20.23913\\
1.76311844077961	20.238088\\
1.76511744127936	20.237042\\
1.76711644177911	20.235991\\
1.76911544227886	20.234937\\
1.77111444277861	20.233878\\
1.77311344327836	20.232815\\
1.77511244377811	20.231749\\
1.77711144427786	20.230678\\
1.77911044477761	20.229604\\
1.78110944527736	20.228525\\
1.78310844577711	20.227443\\
1.78510744627686	20.226357\\
1.78710644677661	20.225268\\
1.78910544727636	20.224175\\
1.79110444777611	20.223078\\
1.79310344827586	20.221978\\
1.79510244877561	20.220875\\
1.79710144927536	20.219768\\
1.79910044977511	20.218659\\
1.80109945027486	20.217545\\
1.80309845077461	20.216429\\
1.80509745127436	20.21531\\
1.80709645177411	20.214188\\
1.80909545227386	20.213062\\
1.81109445277361	20.211934\\
1.81309345327336	20.210803\\
1.81509245377311	20.209669\\
1.81709145427286	20.208533\\
1.81909045477261	20.207394\\
1.82108945527236	20.206252\\
1.82308845577211	20.205108\\
1.82508745627186	20.203962\\
1.82708645677161	20.202813\\
1.82908545727136	20.201661\\
1.83108445777111	20.200508\\
1.83308345827086	20.199352\\
1.83508245877061	20.198194\\
1.83708145927036	20.197035\\
1.83908045977011	20.195873\\
1.84107946026987	20.194709\\
1.84307846076962	20.193543\\
1.84507746126937	20.192376\\
1.84707646176912	20.191207\\
1.84907546226887	20.190036\\
1.85107446276862	20.188863\\
1.85307346326837	20.187689\\
1.85507246376812	20.186514\\
1.85707146426787	20.185337\\
1.85907046476762	20.184159\\
1.86106946526737	20.182979\\
1.86306846576712	20.181799\\
1.86506746626687	20.180617\\
1.86706646676662	20.179434\\
1.86906546726637	20.17825\\
1.87106446776612	20.177065\\
1.87306346826587	20.175879\\
1.87506246876562	20.174692\\
1.87706146926537	20.173505\\
1.87906046976512	20.172317\\
1.88105947026487	20.171128\\
1.88305847076462	20.169939\\
1.88505747126437	20.168749\\
1.88705647176412	20.167559\\
1.88905547226387	20.166368\\
1.89105447276362	20.165177\\
1.89305347326337	20.163986\\
1.89505247376312	20.162795\\
1.89705147426287	20.161603\\
1.89905047476262	20.160412\\
1.90104947526237	20.15922\\
1.90304847576212	20.158029\\
1.90504747626187	20.156838\\
1.90704647676162	20.155647\\
1.90904547726137	20.154456\\
1.91104447776112	20.153266\\
1.91304347826087	20.152076\\
1.91504247876062	20.150886\\
1.91704147926037	20.149697\\
1.91904047976012	20.148509\\
1.92103948025987	20.147321\\
1.92303848075962	20.146134\\
1.92503748125937	20.144948\\
1.92703648175912	20.143763\\
1.92903548225887	20.142578\\
1.93103448275862	20.141395\\
1.93303348325837	20.140212\\
1.93503248375812	20.139031\\
1.93703148425787	20.137851\\
1.93903048475762	20.136672\\
1.94102948525737	20.135495\\
1.94302848575712	20.134319\\
1.94502748625687	20.133144\\
1.94702648675662	20.131971\\
1.94902548725637	20.130799\\
1.95102448775612	20.129629\\
1.95302348825587	20.128461\\
1.95502248875562	20.127295\\
1.95702148925537	20.12613\\
1.95902048975512	20.124967\\
1.96101949025487	20.123806\\
1.96301849075462	20.122648\\
1.96501749125437	20.121491\\
1.96701649175412	20.120336\\
1.96901549225387	20.119184\\
1.97101449275362	20.118034\\
1.97301349325337	20.116886\\
1.97501249375312	20.11574\\
1.97701149425287	20.114598\\
1.97901049475262	20.113457\\
1.98100949525237	20.112319\\
1.98300849575212	20.111184\\
1.98500749625187	20.110051\\
1.98700649675162	20.108922\\
1.98900549725137	20.107795\\
1.99100449775112	20.106671\\
1.99300349825087	20.105549\\
1.99500249875062	20.104431\\
1.99700149925037	20.103316\\
1.99900049975012	20.102204\\
2.00099950024988	20.101096\\
2.00299850074963	20.09999\\
2.00499750124938	20.098888\\
2.00699650174913	20.097789\\
2.00899550224888	20.096694\\
2.01099450274863	20.095602\\
2.01299350324838	20.094513\\
2.01499250374813	20.093429\\
2.01699150424788	20.092348\\
2.01899050474763	20.09127\\
2.02098950524738	20.090197\\
2.02298850574713	20.089127\\
2.02498750624688	20.088061\\
2.02698650674663	20.087\\
2.02898550724638	20.085942\\
2.03098450774613	20.084888\\
2.03298350824588	20.083838\\
2.03498250874563	20.082793\\
2.03698150924538	20.081752\\
2.03898050974513	20.080715\\
2.04097951024488	20.079683\\
2.04297851074463	20.078655\\
2.04497751124438	20.077631\\
2.04697651174413	20.076612\\
2.04897551224388	20.075598\\
2.05097451274363	20.074588\\
2.05297351324338	20.073583\\
2.05497251374313	20.072583\\
2.05697151424288	20.071587\\
2.05897051474263	20.070597\\
2.06096951524238	20.069611\\
2.06296851574213	20.06863\\
2.06496751624188	20.067654\\
2.06696651674163	20.066684\\
2.06896551724138	20.065718\\
2.07096451774113	20.064758\\
2.07296351824088	20.063803\\
2.07496251874063	20.062854\\
2.07696151924038	20.061909\\
2.07896051974013	20.06097\\
2.08095952023988	20.060037\\
2.08295852073963	20.059109\\
2.08495752123938	20.058187\\
2.08695652173913	20.05727\\
2.08895552223888	20.056359\\
2.09095452273863	20.055453\\
2.09295352323838	20.054554\\
2.09495252373813	20.05366\\
2.09695152423788	20.052772\\
2.09895052473763	20.05189\\
2.10094952523738	20.051014\\
2.10294852573713	20.050144\\
2.10494752623688	20.04928\\
2.10694652673663	20.048422\\
2.10894552723638	20.047571\\
2.11094452773613	20.046725\\
2.11294352823588	20.045886\\
2.11494252873563	20.045053\\
2.11694152923538	20.044226\\
2.11894052973513	20.043406\\
2.12093953023488	20.042593\\
2.12293853073463	20.041785\\
2.12493753123438	20.040985\\
2.12693653173413	20.040191\\
2.12893553223388	20.039403\\
2.13093453273363	20.038622\\
2.13293353323338	20.037848\\
2.13493253373313	20.037081\\
2.13693153423288	20.036321\\
2.13893053473263	20.035567\\
2.14092953523238	20.03482\\
2.14292853573213	20.03408\\
2.14492753623188	20.033347\\
2.14692653673163	20.032622\\
2.14892553723138	20.031903\\
2.15092453773113	20.031191\\
2.15292353823088	20.030487\\
2.15492253873063	20.029789\\
2.15692153923038	20.029099\\
2.15892053973013	20.028416\\
2.16091954022989	20.02774\\
2.16291854072964	20.027072\\
2.16491754122939	20.026411\\
2.16691654172914	20.025758\\
2.16891554222889	20.025112\\
2.17091454272864	20.024473\\
2.17291354322839	20.023842\\
2.17491254372814	20.023219\\
2.17691154422789	20.022603\\
2.17891054472764	20.021995\\
2.18090954522739	20.021394\\
2.18290854572714	20.020801\\
2.18490754622689	20.020216\\
2.18690654672664	20.019639\\
2.18890554722639	20.019069\\
2.19090454772614	20.018507\\
2.19290354822589	20.017954\\
2.19490254872564	20.017408\\
2.19690154922539	20.01687\\
2.19890054972514	20.01634\\
2.20089955022489	20.015817\\
2.20289855072464	20.015303\\
2.20489755122439	20.014797\\
2.20689655172414	20.014299\\
2.20889555222389	20.01381\\
2.21089455272364	20.013328\\
2.21289355322339	20.012854\\
2.21489255372314	20.012389\\
2.21689155422289	20.011932\\
2.21889055472264	20.011483\\
2.22088955522239	20.011043\\
2.22288855572214	20.01061\\
2.22488755622189	20.010186\\
2.22688655672164	20.009771\\
2.22888555722139	20.009364\\
2.23088455772114	20.008965\\
2.23288355822089	20.008574\\
2.23488255872064	20.008193\\
2.23688155922039	20.007819\\
2.23888055972014	20.007454\\
2.24087956021989	20.007098\\
2.24287856071964	20.00675\\
2.24487756121939	20.00641\\
2.24687656171914	20.006079\\
2.24887556221889	20.005757\\
2.25087456271864	20.005444\\
2.25287356321839	20.005139\\
2.25487256371814	20.004842\\
2.25687156421789	20.004554\\
2.25887056471764	20.004275\\
2.26086956521739	20.004005\\
2.26286856571714	20.003744\\
2.26486756621689	20.003491\\
2.26686656671664	20.003247\\
2.26886556721639	20.003011\\
2.27086456771614	20.002785\\
2.27286356821589	20.002567\\
2.27486256871564	20.002358\\
2.27686156921539	20.002158\\
2.27886056971514	20.001966\\
2.28085957021489	20.001784\\
2.28285857071464	20.00161\\
2.28485757121439	20.001445\\
2.28685657171414	20.001289\\
2.28885557221389	20.001142\\
2.29085457271364	20.001004\\
2.29285357321339	20.000874\\
2.29485257371314	20.000754\\
2.29685157421289	20.000642\\
2.29885057471264	20.00054\\
2.30084957521239	20.000446\\
2.30284857571214	20.000361\\
2.30484757621189	20.000285\\
2.30684657671164	20.000218\\
2.30884557721139	20.00016\\
2.31084457771114	20.000111\\
2.31284357821089	20.000071\\
2.31484257871064	20.00004\\
2.31684157921039	20.000018\\
2.31884057971015	20.000004\\
2.3208395802099	20\\
2.32283858070965	20.000005\\
2.3248375812094	20.000018\\
2.32683658170915	20.000041\\
2.3288355822089	20.000072\\
2.33083458270865	20.000113\\
2.3328335832084	20.000162\\
2.33483258370815	20.00022\\
2.3368315842079	20.000287\\
2.33883058470765	20.000363\\
2.3408295852074	20.000449\\
2.34282858570715	20.000543\\
2.3448275862069	20.000645\\
2.34682658670665	20.000757\\
2.3488255872064	20.000878\\
2.35082458770615	20.001008\\
2.3528235882059	20.001146\\
2.35482258870565	20.001293\\
2.3568215892054	20.00145\\
2.35882058970515	20.001615\\
2.3608195902049	20.001789\\
2.36281859070465	20.001972\\
2.3648175912044	20.002163\\
2.36681659170415	20.002364\\
2.3688155922039	20.002573\\
2.37081459270365	20.002791\\
2.3728135932034	20.003018\\
2.37481259370315	20.003254\\
2.3768115942029	20.003498\\
2.37881059470265	20.003751\\
2.3808095952024	20.004013\\
2.38280859570215	20.004284\\
2.3848075962019	20.004563\\
2.38680659670165	20.004851\\
2.3888055972014	20.005147\\
2.39080459770115	20.005453\\
2.3928035982009	20.005767\\
2.39480259870065	20.006089\\
2.3968015992004	20.00642\\
2.39880059970015	20.00676\\
2.4007996001999	20.007108\\
2.40279860069965	20.007465\\
2.4047976011994	20.00783\\
2.40679660169915	20.008204\\
2.4087956021989	20.008586\\
2.41079460269865	20.008977\\
2.4127936031984	20.009376\\
2.41479260369815	20.009783\\
2.4167916041979	20.010199\\
2.41879060469765	20.010623\\
2.4207896051974	20.011056\\
2.42278860569715	20.011496\\
2.4247876061969	20.011945\\
2.42678660669665	20.012403\\
2.4287856071964	20.012868\\
2.43078460769615	20.013342\\
2.4327836081959	20.013824\\
2.43478260869565	20.014314\\
2.4367816091954	20.014812\\
2.43878060969515	20.015318\\
2.4407796101949	20.015833\\
2.44277861069465	20.016355\\
2.4447776111944	20.016885\\
2.44677661169415	20.017424\\
2.4487756121939	20.01797\\
2.45077461269365	20.018524\\
2.4527736131934	20.019086\\
2.45477261369315	20.019656\\
2.4567716141929	20.020233\\
2.45877061469265	20.020819\\
2.4607696151924	20.021412\\
2.46276861569215	20.022013\\
2.4647676161919	20.022621\\
2.46676661669165	20.023237\\
2.4687656171914	20.023861\\
2.47076461769115	20.024492\\
2.4727636181909	20.025131\\
2.47476261869065	20.025777\\
2.4767616191904	20.026431\\
2.47876061969016	20.027092\\
2.48075962018991	20.02776\\
2.48275862068966	20.028436\\
2.48475762118941	20.029119\\
2.48675662168916	20.02981\\
2.48875562218891	20.030507\\
2.49075462268866	20.031212\\
2.49275362318841	20.031924\\
2.49475262368816	20.032643\\
2.49675162418791	20.033369\\
2.49875062468766	20.034102\\
2.50074962518741	20.034842\\
2.50274862568716	20.035589\\
2.50474762618691	20.036343\\
2.50674662668666	20.037104\\
2.50874562718641	20.037871\\
2.51074462768616	20.038645\\
2.51274362818591	20.039426\\
2.51474262868566	20.040214\\
2.51674162918541	20.041008\\
2.51874062968516	20.041809\\
2.52073963018491	20.042617\\
2.52273863068466	20.04343\\
2.52473763118441	20.044251\\
2.52673663168416	20.045078\\
2.52873563218391	20.045911\\
2.53073463268366	20.04675\\
2.53273363318341	20.047596\\
2.53473263368316	20.048447\\
2.53673163418291	20.049305\\
2.53873063468266	20.05017\\
2.54072963518241	20.05104\\
2.54272863568216	20.051916\\
2.54472763618191	20.052798\\
2.54672663668166	20.053686\\
2.54872563718141	20.05458\\
2.55072463768116	20.05548\\
2.55272363818091	20.056386\\
2.55472263868066	20.057297\\
2.55672163918041	20.058214\\
2.55872063968016	20.059136\\
2.56071964017991	20.060064\\
2.56271864067966	20.060998\\
2.56471764117941	20.061937\\
2.56671664167916	20.062882\\
2.56871564217891	20.063831\\
2.57071464267866	20.064787\\
2.57271364317841	20.065747\\
2.57471264367816	20.066712\\
2.57671164417791	20.067683\\
2.57871064467766	20.068659\\
2.58070964517741	20.06964\\
2.58270864567716	20.070626\\
2.58470764617691	20.071616\\
2.58670664667666	20.072612\\
2.58870564717641	20.073613\\
2.59070464767616	20.074618\\
2.59270364817591	20.075628\\
2.59470264867566	20.076642\\
2.59670164917541	20.077661\\
2.59870064967516	20.078685\\
2.60069965017491	20.079713\\
2.60269865067466	20.080746\\
2.60469765117441	20.081783\\
2.60669665167416	20.082824\\
2.60869565217391	20.083869\\
2.61069465267366	20.084919\\
2.61269365317341	20.085973\\
2.61469265367316	20.087031\\
2.61669165417291	20.088093\\
2.61869065467266	20.089159\\
2.62068965517241	20.090229\\
2.62268865567216	20.091302\\
2.62468765617191	20.09238\\
2.62668665667166	20.093461\\
2.62868565717141	20.094546\\
2.63068465767116	20.095634\\
2.63268365817091	20.096726\\
2.63468265867066	20.097821\\
2.63668165917041	20.09892\\
2.63868065967017	20.100023\\
2.64067966016992	20.101128\\
2.64267866066967	20.102237\\
2.64467766116942	20.103349\\
2.64667666166917	20.104464\\
2.64867566216892	20.105582\\
2.65067466266867	20.106704\\
2.65267366316842	20.107828\\
2.65467266366817	20.108955\\
2.65667166416792	20.110085\\
2.65867066466767	20.111217\\
2.66066966516742	20.112353\\
2.66266866566717	20.113491\\
2.66466766616692	20.114631\\
2.66666666666667	20.115774\\
2.66866566716642	20.11692\\
2.67066466766617	20.118068\\
2.67266366816592	20.119218\\
2.67466266866567	20.12037\\
2.67666166916542	20.121525\\
2.67866066966517	20.122682\\
2.68065967016492	20.123841\\
2.68265867066467	20.125002\\
2.68465767116442	20.126164\\
2.68665667166417	20.127329\\
2.68865567216392	20.128496\\
2.69065467266367	20.129664\\
2.69265367316342	20.130834\\
2.69465267366317	20.132006\\
2.69665167416292	20.133179\\
2.69865067466267	20.134354\\
2.70064967516242	20.13553\\
2.70264867566217	20.136707\\
2.70464767616192	20.137886\\
2.70664667666167	20.139066\\
2.70864567716142	20.140247\\
2.71064467766117	20.14143\\
2.71264367816092	20.142613\\
2.71464267866067	20.143798\\
2.71664167916042	20.144983\\
2.71864067966017	20.146169\\
2.72063968015992	20.147356\\
2.72263868065967	20.148544\\
2.72463768115942	20.149732\\
2.72663668165917	20.150921\\
2.72863568215892	20.152111\\
2.73063468265867	20.153301\\
2.73263368315842	20.154491\\
2.73463268365817	20.155682\\
2.73663168415792	20.156873\\
2.73863068465767	20.158064\\
2.74062968515742	20.159256\\
2.74262868565717	20.160447\\
2.74462768615692	20.161639\\
2.74662668665667	20.16283\\
2.74862568715642	20.164021\\
2.75062468765617	20.165212\\
2.75262368815592	20.166403\\
2.75462268865567	20.167594\\
2.75662168915542	20.168784\\
2.75862068965517	20.169974\\
2.76061969015492	20.171163\\
2.76261869065467	20.172352\\
2.76461769115442	20.17354\\
2.76661669165417	20.174727\\
2.76861569215392	20.175914\\
2.77061469265367	20.1771\\
2.77261369315342	20.178285\\
2.77461269365317	20.179469\\
2.77661169415292	20.180652\\
2.77861069465267	20.181834\\
2.78060969515242	20.183014\\
2.78260869565217	20.184194\\
2.78460769615192	20.185372\\
2.78660669665167	20.186549\\
2.78860569715142	20.187724\\
2.79060469765117	20.188898\\
2.79260369815092	20.19007\\
2.79460269865067	20.191241\\
2.79660169915043	20.19241\\
2.79860069965018	20.193578\\
2.80059970014993	20.194743\\
2.80259870064968	20.195907\\
2.80459770114943	20.197069\\
2.80659670164918	20.198229\\
2.80859570214893	20.199386\\
2.81059470264868	20.200542\\
2.81259370314843	20.201695\\
2.81459270364818	20.202847\\
2.81659170414793	20.203996\\
2.81859070464768	20.205142\\
2.82058970514743	20.206286\\
2.82258870564718	20.207428\\
2.82458770614693	20.208567\\
2.82658670664668	20.209703\\
2.82858570714643	20.210837\\
2.83058470764618	20.211967\\
2.83258370814593	20.213096\\
2.83458270864568	20.214221\\
2.83658170914543	20.215343\\
2.83858070964518	20.216462\\
2.84057971014493	20.217578\\
2.84257871064468	20.218691\\
2.84457771114443	20.219801\\
2.84657671164418	20.220908\\
2.84857571214393	20.222011\\
2.85057471264368	20.223111\\
2.85257371314343	20.224207\\
2.85457271364318	20.2253\\
2.85657171414293	20.226389\\
2.85857071464268	20.227475\\
2.86056971514243	20.228557\\
2.86256871564218	20.229635\\
2.86456771614193	20.23071\\
2.86656671664168	20.23178\\
2.86856571714143	20.232847\\
2.87056471764118	20.23391\\
2.87256371814093	20.234968\\
2.87456271864068	20.236023\\
2.87656171914043	20.237073\\
2.87856071964018	20.238119\\
2.88055972013993	20.239161\\
2.88255872063968	20.240198\\
2.88455772113943	20.241231\\
2.88655672163918	20.242259\\
2.88855572213893	20.243283\\
2.89055472263868	20.244303\\
2.89255372313843	20.245317\\
2.89455272363818	20.246327\\
2.89655172413793	20.247333\\
2.89855072463768	20.248333\\
2.90054972513743	20.249328\\
2.90254872563718	20.250319\\
2.90454772613693	20.251305\\
2.90654672663668	20.252285\\
2.90854572713643	20.253261\\
2.91054472763618	20.254231\\
2.91254372813593	20.255196\\
2.91454272863568	20.256156\\
2.91654172913543	20.25711\\
2.91854072963518	20.258059\\
2.92053973013493	20.259003\\
2.92253873063468	20.259941\\
2.92453773113443	20.260873\\
2.92653673163418	20.2618\\
2.92853573213393	20.262722\\
2.93053473263368	20.263637\\
2.93253373313343	20.264547\\
2.93453273363318	20.265451\\
2.93653173413293	20.266349\\
2.93853073463268	20.267241\\
2.94052973513243	20.268127\\
2.94252873563218	20.269008\\
2.94452773613193	20.269882\\
2.94652673663168	20.27075\\
2.94852573713143	20.271611\\
2.95052473763118	20.272467\\
2.95252373813093	20.273316\\
2.95452273863068	20.274159\\
2.95652173913043	20.274996\\
2.95852073963018	20.275826\\
2.96051974012994	20.276649\\
2.96251874062969	20.277466\\
2.96451774112944	20.278277\\
2.96651674162919	20.279081\\
2.96851574212894	20.279878\\
2.97051474262869	20.280669\\
2.97251374312844	20.281452\\
2.97451274362819	20.282229\\
2.97651174412794	20.282999\\
2.97851074462769	20.283762\\
2.98050974512744	20.284518\\
2.98250874562719	20.285268\\
2.98450774612694	20.28601\\
2.98650674662669	20.286745\\
2.98850574712644	20.287473\\
2.99050474762619	20.288193\\
2.99250374812594	20.288907\\
2.99450274862569	20.289613\\
2.99650174912544	20.290312\\
2.99850074962519	20.291003\\
3.00049975012494	20.291688\\
3.00249875062469	20.292364\\
3.00449775112444	20.293034\\
3.00649675162419	20.293695\\
3.00849575212394	20.29435\\
3.01049475262369	20.294996\\
3.01249375312344	20.295635\\
3.01449275362319	20.296267\\
3.01649175412294	20.29689\\
3.01849075462269	20.297506\\
3.02048975512244	20.298114\\
3.02248875562219	20.298715\\
3.02448775612194	20.299307\\
3.02648675662169	20.299892\\
3.02848575712144	20.300468\\
3.03048475762119	20.301037\\
3.03248375812094	20.301597\\
3.03448275862069	20.30215\\
3.03648175912044	20.302695\\
3.03848075962019	20.303231\\
3.04047976011994	20.303759\\
3.04247876061969	20.30428\\
3.04447776111944	20.304792\\
3.04647676161919	20.305295\\
3.04847576211894	20.305791\\
3.05047476261869	20.306278\\
3.05247376311844	20.306757\\
3.05447276361819	20.307228\\
3.05647176411794	20.30769\\
3.05847076461769	20.308144\\
3.06046976511744	20.308589\\
3.06246876561719	20.309026\\
3.06446776611694	20.309454\\
3.06646676661669	20.309874\\
3.06846576711644	20.310285\\
3.07046476761619	20.310688\\
3.07246376811594	20.311082\\
3.07446276861569	20.311468\\
3.07646176911544	20.311845\\
3.07846076961519	20.312213\\
3.08045977011494	20.312573\\
3.08245877061469	20.312924\\
3.08445777111444	20.313266\\
3.08645677161419	20.313599\\
3.08845577211394	20.313924\\
3.09045477261369	20.31424\\
3.09245377311344	20.314547\\
3.09445277361319	20.314845\\
3.09645177411294	20.315135\\
3.09845077461269	20.315415\\
3.10044977511244	20.315687\\
3.10244877561219	20.31595\\
3.10444777611194	20.316204\\
3.10644677661169	20.316449\\
3.10844577711144	20.316685\\
3.11044477761119	20.316912\\
3.11244377811094	20.31713\\
3.11444277861069	20.317339\\
3.11644177911044	20.317539\\
3.11844077961019	20.31773\\
3.12043978010994	20.317912\\
3.12243878060969	20.318085\\
3.12443778110945	20.318249\\
3.1264367816092	20.318404\\
3.12843578210895	20.318549\\
3.1304347826087	20.318686\\
3.13243378310845	20.318814\\
3.1344327836082	20.318933\\
3.13643178410795	20.319042\\
3.1384307846077	20.319142\\
3.14042978510745	20.319234\\
3.1424287856072	20.319316\\
3.14442778610695	20.319389\\
3.1464267866067	20.319453\\
3.14842578710645	20.319507\\
3.1504247876062	20.319553\\
3.15242378810595	20.31959\\
3.1544227886057	20.319617\\
3.15642178910545	20.319635\\
3.1584207896052	20.319644\\
3.16041979010495	20.319644\\
3.1624187906047	20.319635\\
3.16441779110445	20.319617\\
3.1664167916042	20.319589\\
3.16841579210395	20.319553\\
3.1704147926037	20.319507\\
3.17241379310345	20.319452\\
3.1744127936032	20.319388\\
3.17641179410295	20.319315\\
3.1784107946027	20.319232\\
3.18040979510245	20.319141\\
3.1824087956022	20.31904\\
3.18440779610195	20.318931\\
3.1864067966017	20.318812\\
3.18840579710145	20.318684\\
3.1904047976012	20.318547\\
3.19240379810095	20.318401\\
3.1944027986007	20.318246\\
3.19640179910045	20.318082\\
3.1984007996002	20.317909\\
3.20039980009995	20.317727\\
3.2023988005997	20.317536\\
3.20439780109945	20.317336\\
3.2063968015992	20.317126\\
3.20839580209895	20.316908\\
3.2103948025987	20.316681\\
3.21239380309845	20.316445\\
3.2143928035982	20.3162\\
3.21639180409795	20.315946\\
3.2183908045977	20.315683\\
3.22038980509745	20.315411\\
3.2223888055972	20.315131\\
3.22438780609695	20.314841\\
3.2263868065967	20.314543\\
3.22838580709645	20.314235\\
3.2303848075962	20.313919\\
3.23238380809595	20.313595\\
3.2343828085957	20.313261\\
3.23638180909545	20.312919\\
3.2383808095952	20.312568\\
3.24037981009495	20.312208\\
3.2423788105947	20.311839\\
3.24437781109445	20.311462\\
3.2463768115942	20.311077\\
3.24837581209395	20.310682\\
3.2503748125937	20.310279\\
3.25237381309345	20.309868\\
3.2543728135932	20.309448\\
3.25637181409295	20.309019\\
3.2583708145927	20.308582\\
3.26036981509245	20.308137\\
3.2623688155922	20.307683\\
3.26436781609195	20.307221\\
3.2663668165917	20.30675\\
3.26836581709145	20.306271\\
3.2703648175912	20.305784\\
3.27236381809095	20.305288\\
3.2743628185907	20.304784\\
3.27636181909045	20.304272\\
3.2783608195902	20.303752\\
3.28035982008995	20.303223\\
3.2823588205897	20.302687\\
3.28435782108946	20.302142\\
3.28635682158921	20.301589\\
3.28835582208896	20.301028\\
3.29035482258871	20.30046\\
3.29235382308846	20.299883\\
3.29435282358821	20.299298\\
3.29635182408796	20.298706\\
3.29835082458771	20.298105\\
3.30034982508746	20.297497\\
3.30234882558721	20.296881\\
3.30434782608696	20.296257\\
3.30634682658671	20.295626\\
3.30834582708646	20.294987\\
3.31034482758621	20.29434\\
3.31234382808596	20.293686\\
3.31434282858571	20.293024\\
3.31634182908546	20.292354\\
3.31834082958521	20.291678\\
3.32033983008496	20.290993\\
3.32233883058471	20.290302\\
3.32433783108446	20.289603\\
3.32633683158421	20.288896\\
3.32833583208396	20.288183\\
3.33033483258371	20.287462\\
3.33233383308346	20.286734\\
3.33433283358321	20.285999\\
3.33633183408296	20.285257\\
3.33833083458271	20.284507\\
3.34032983508246	20.283751\\
3.34232883558221	20.282988\\
3.34432783608196	20.282218\\
3.34632683658171	20.281441\\
3.34832583708146	20.280657\\
3.35032483758121	20.279866\\
3.35232383808096	20.279069\\
3.35432283858071	20.278265\\
3.35632183908046	20.277454\\
3.35832083958021	20.276637\\
3.36031984007996	20.275814\\
3.36231884057971	20.274983\\
3.36431784107946	20.274147\\
3.36631684157921	20.273304\\
3.36831584207896	20.272454\\
3.37031484257871	20.271599\\
3.37231384307846	20.270737\\
3.37431284357821	20.269869\\
3.37631184407796	20.268995\\
3.37831084457771	20.268114\\
3.38030984507746	20.267228\\
3.38230884557721	20.266336\\
3.38430784607696	20.265438\\
3.38630684657671	20.264534\\
3.38830584707646	20.263624\\
3.39030484757621	20.262708\\
3.39230384807596	20.261787\\
3.39430284857571	20.26086\\
3.39630184907546	20.259927\\
3.39830084957521	20.258989\\
3.40029985007496	20.258045\\
3.40229885057471	20.257096\\
3.40429785107446	20.256142\\
3.40629685157421	20.255182\\
3.40829585207396	20.254217\\
3.41029485257371	20.253246\\
3.41229385307346	20.252271\\
3.41429285357321	20.25129\\
3.41629185407296	20.250304\\
3.41829085457271	20.249314\\
3.42028985507246	20.248318\\
3.42228885557221	20.247318\\
3.42428785607196	20.246312\\
3.42628685657171	20.245302\\
3.42828585707146	20.244288\\
3.43028485757121	20.243268\\
3.43228385807096	20.242244\\
3.43428285857071	20.241216\\
3.43628185907046	20.240183\\
3.43828085957021	20.239145\\
3.44027986006996	20.238103\\
3.44227886056971	20.237057\\
3.44427786106947	20.236007\\
3.44627686156922	20.234952\\
3.44827586206897	20.233894\\
3.45027486256872	20.232831\\
3.45227386306847	20.231765\\
3.45427286356822	20.230694\\
3.45627186406797	20.229619\\
3.45827086456772	20.228541\\
3.46026986506747	20.227459\\
3.46226886556722	20.226373\\
3.46426786606697	20.225284\\
3.46626686656672	20.224191\\
3.46826586706647	20.223095\\
3.47026486756622	20.221995\\
3.47226386806597	20.220891\\
3.47426286856572	20.219785\\
3.47626186906547	20.218675\\
3.47826086956522	20.217562\\
3.48025987006497	20.216446\\
3.48225887056472	20.215326\\
3.48425787106447	20.214204\\
3.48625687156422	20.213079\\
3.48825587206397	20.211951\\
3.49025487256372	20.21082\\
3.49225387306347	20.209686\\
3.49425287356322	20.20855\\
3.49625187406297	20.207411\\
3.49825087456272	20.206269\\
3.50024987506247	20.205125\\
3.50224887556222	20.203979\\
3.50424787606197	20.20283\\
3.50624687656172	20.201678\\
3.50824587706147	20.200525\\
3.51024487756122	20.199369\\
3.51224387806097	20.198212\\
3.51424287856072	20.197052\\
3.51624187906047	20.19589\\
3.51824087956022	20.194726\\
3.52023988005997	20.193561\\
3.52223888055972	20.192393\\
3.52423788105947	20.191224\\
3.52623688155922	20.190053\\
3.52823588205897	20.188881\\
3.53023488255872	20.187707\\
3.53223388305847	20.186531\\
3.53423288355822	20.185354\\
3.53623188405797	20.184176\\
3.53823088455772	20.182997\\
3.54022988505747	20.181816\\
3.54222888555722	20.180634\\
3.54422788605697	20.179451\\
3.54622688655672	20.178267\\
3.54822588705647	20.177082\\
3.55022488755622	20.175897\\
3.55222388805597	20.17471\\
3.55422288855572	20.173523\\
3.55622188905547	20.172334\\
3.55822088955522	20.171146\\
3.56021989005497	20.169956\\
3.56221889055472	20.168767\\
3.56421789105447	20.167576\\
3.56621689155422	20.166386\\
3.56821589205397	20.165195\\
3.57021489255372	20.164004\\
3.57221389305347	20.162812\\
3.57421289355322	20.161621\\
3.57621189405297	20.160429\\
3.57821089455272	20.159238\\
3.58020989505247	20.158047\\
3.58220889555222	20.156855\\
3.58420789605197	20.155664\\
3.58620689655172	20.154474\\
3.58820589705147	20.153283\\
3.59020489755122	20.152093\\
3.59220389805097	20.150904\\
3.59420289855072	20.149715\\
3.59620189905047	20.148526\\
3.59820089955022	20.147339\\
3.60019990004997	20.146152\\
3.60219890054973	20.144965\\
3.60419790104948	20.14378\\
3.60619690154923	20.142596\\
3.60819590204898	20.141412\\
3.61019490254873	20.14023\\
3.61219390304848	20.139049\\
3.61419290354823	20.137869\\
3.61619190404798	20.13669\\
3.61819090454773	20.135512\\
3.62018990504748	20.134336\\
3.62218890554723	20.133162\\
3.62418790604698	20.131988\\
3.62618690654673	20.130817\\
3.62818590704648	20.129647\\
3.63018490754623	20.128478\\
3.63218390804598	20.127312\\
3.63418290854573	20.126147\\
3.63618190904548	20.124984\\
3.63818090954523	20.123824\\
3.64017991004498	20.122665\\
3.64217891054473	20.121508\\
3.64417791104448	20.120353\\
3.64617691154423	20.119201\\
3.64817591204398	20.118051\\
3.65017491254373	20.116903\\
3.65217391304348	20.115757\\
3.65417291354323	20.114614\\
3.65617191404298	20.113474\\
3.65817091454273	20.112336\\
3.66016991504248	20.111201\\
3.66216891554223	20.110068\\
3.66416791604198	20.108938\\
3.66616691654173	20.107811\\
3.66816591704148	20.106687\\
3.67016491754123	20.105566\\
3.67216391804098	20.104448\\
3.67416291854073	20.103333\\
3.67616191904048	20.102221\\
3.67816091954023	20.101112\\
3.68015992003998	20.100006\\
3.68215892053973	20.098904\\
3.68415792103948	20.097805\\
3.68615692153923	20.09671\\
3.68815592203898	20.095618\\
3.69015492253873	20.09453\\
3.69215392303848	20.093445\\
3.69415292353823	20.092364\\
3.69615192403798	20.091286\\
3.69815092453773	20.090213\\
3.70014992503748	20.089143\\
3.70214892553723	20.088077\\
3.70414792603698	20.087015\\
3.70614692653673	20.085957\\
3.70814592703648	20.084904\\
3.71014492753623	20.083854\\
3.71214392803598	20.082808\\
3.71414292853573	20.081767\\
3.71614192903548	20.08073\\
3.71814092953523	20.079698\\
3.72013993003498	20.07867\\
3.72213893053473	20.077646\\
3.72413793103448	20.076627\\
3.72613693153423	20.075613\\
3.72813593203398	20.074603\\
3.73013493253373	20.073598\\
3.73213393303348	20.072597\\
3.73413293353323	20.071602\\
3.73613193403298	20.070611\\
3.73813093453273	20.069625\\
3.74012993503248	20.068645\\
3.74212893553223	20.067669\\
3.74412793603198	20.066698\\
3.74612693653173	20.065733\\
3.74812593703148	20.064772\\
3.75012493753123	20.063817\\
3.75212393803098	20.062868\\
3.75412293853073	20.061923\\
3.75612193903048	20.060984\\
3.75812093953023	20.060051\\
3.76011994002998	20.059123\\
3.76211894052974	20.0582\\
3.76411794102949	20.057283\\
3.76611694152924	20.056372\\
3.76811594202899	20.055467\\
3.77011494252874	20.054567\\
3.77211394302849	20.053673\\
3.77411294352824	20.052785\\
3.77611194402799	20.051903\\
3.77811094452774	20.051027\\
3.78010994502749	20.050157\\
3.78210894552724	20.049293\\
3.78410794602699	20.048435\\
3.78610694652674	20.047583\\
3.78810594702649	20.046738\\
3.79010494752624	20.045898\\
3.79210394802599	20.045065\\
3.79410294852574	20.044239\\
3.79610194902549	20.043418\\
3.79810094952524	20.042605\\
3.80009995002499	20.041797\\
3.80209895052474	20.040997\\
3.80409795102449	20.040202\\
3.80609695152424	20.039415\\
3.80809595202399	20.038634\\
3.81009495252374	20.03786\\
3.81209395302349	20.037092\\
3.81409295352324	20.036332\\
3.81609195402299	20.035578\\
3.81809095452274	20.034831\\
3.82008995502249	20.034091\\
3.82208895552224	20.033358\\
3.82408795602199	20.032632\\
3.82608695652174	20.031913\\
3.82808595702149	20.031202\\
3.83008495752124	20.030497\\
3.83208395802099	20.029799\\
3.83408295852074	20.029109\\
3.83608195902049	20.028426\\
3.83808095952024	20.02775\\
3.84007996001999	20.027082\\
3.84207896051974	20.026421\\
3.84407796101949	20.025767\\
3.84607696151924	20.025121\\
3.84807596201899	20.024483\\
3.85007496251874	20.023852\\
3.85207396301849	20.023228\\
3.85407296351824	20.022612\\
3.85607196401799	20.022004\\
3.85807096451774	20.021403\\
3.86006996501749	20.02081\\
3.86206896551724	20.020225\\
3.86406796601699	20.019647\\
3.86606696651674	20.019078\\
3.86806596701649	20.018516\\
3.87006496751624	20.017962\\
3.87206396801599	20.017416\\
3.87406296851574	20.016877\\
3.87606196901549	20.016347\\
3.87806096951524	20.015825\\
3.88005997001499	20.015311\\
3.88205897051474	20.014805\\
3.88405797101449	20.014307\\
3.88605697151424	20.013817\\
3.88805597201399	20.013335\\
3.89005497251374	20.012861\\
3.89205397301349	20.012396\\
3.89405297351324	20.011939\\
3.89605197401299	20.01149\\
3.89805097451274	20.011049\\
3.90004997501249	20.010617\\
3.90204897551224	20.010193\\
3.90404797601199	20.009777\\
3.90604697651174	20.00937\\
3.90804597701149	20.008971\\
3.91004497751124	20.00858\\
3.91204397801099	20.008198\\
3.91404297851074	20.007824\\
3.91604197901049	20.007459\\
3.91804097951024	20.007103\\
3.92003998000999	20.006755\\
3.92203898050975	20.006415\\
3.9240379810095	20.006084\\
3.92603698150925	20.005762\\
3.928035982009	20.005448\\
3.93003498250875	20.005143\\
3.9320339830085	20.004846\\
3.93403298350825	20.004559\\
3.936031984008	20.00428\\
3.93803098450775	20.004009\\
3.9400299850075	20.003747\\
3.94202898550725	20.003494\\
3.944027986007	20.00325\\
3.94602698650675	20.003015\\
3.9480259870065	20.002788\\
3.95002498750625	20.00257\\
3.952023988006	20.002361\\
3.95402298850575	20.00216\\
3.9560219890055	20.001969\\
3.95802098950525	20.001786\\
3.960019990005	20.001612\\
3.96201899050475	20.001447\\
3.9640179910045	20.001291\\
3.96601699150425	20.001144\\
3.968015992004	20.001006\\
3.97001499250375	20.000876\\
3.9720139930035	20.000756\\
3.97401299350325	20.000644\\
3.976011994003	20.000541\\
3.97801099450275	20.000447\\
3.9800099950025	20.000362\\
3.98200899550225	20.000286\\
3.984007996002	20.000219\\
3.98600699650175	20.000161\\
3.9880059970015	20.000112\\
3.99000499750125	20.000072\\
3.992003998001	20.00004\\
3.99400299850075	20.000018\\
3.9960019990005	20.000004\\
3.99800099950025	20\\
4	20.000004\\
};
\addlegendentry{$\theta$};

\addplot [color=mycolor4,solid]
  table[row sep=crcr]{%
0	3.89377e-10\\
0.00199900049975012	3.87558e-10\\
0.00399800099950025	3.86308e-10\\
0.00599700149925037	3.89718e-10\\
0.0079960019990005	3.85512e-10\\
0.00999500249875063	3.80396e-10\\
0.0119940029985007	3.72779e-10\\
0.0139930034982509	3.72779e-10\\
0.015992003998001	3.69823e-10\\
0.0179910044977511	3.76076e-10\\
0.0199900049975013	3.78122e-10\\
0.0219890054972514	3.80624e-10\\
0.0239880059970015	3.76644e-10\\
0.0259870064967516	3.81533e-10\\
0.0279860069965017	3.8483e-10\\
0.0299850074962519	3.90969e-10\\
0.031984007996002	3.95289e-10\\
0.0339830084957521	3.89264e-10\\
0.0359820089955022	3.95744e-10\\
0.0379810094952524	3.88127e-10\\
0.0399800099950025	3.89377e-10\\
0.0419790104947526	3.90742e-10\\
0.0439780109945027	3.95517e-10\\
0.0459770114942529	3.91765e-10\\
0.047976011994003	3.84603e-10\\
0.0499750124937531	3.7835e-10\\
0.0519740129935032	3.73802e-10\\
0.0539730134932534	3.72097e-10\\
0.0559720139930035	3.74143e-10\\
0.0579710144927536	3.68686e-10\\
0.0599700149925037	3.72779e-10\\
0.0619690154922539	3.65162e-10\\
0.063968015992004	3.64366e-10\\
0.0659670164917541	3.61069e-10\\
0.0679660169915042	3.64821e-10\\
0.0699650174912544	3.63343e-10\\
0.0719640179910045	3.60046e-10\\
0.0739630184907546	3.54817e-10\\
0.0759620189905048	3.53566e-10\\
0.0779610194902549	3.47882e-10\\
0.079960019990005	3.53566e-10\\
0.0819590204897551	3.5368e-10\\
0.0839580209895052	3.60956e-10\\
0.0859570214892554	3.66981e-10\\
0.0879560219890055	3.61183e-10\\
0.0899550224887556	3.58909e-10\\
0.0919540229885057	3.57659e-10\\
0.0939530234882559	3.65048e-10\\
0.095952023988006	3.63229e-10\\
0.0979510244877561	3.63684e-10\\
0.0999500249875062	3.65731e-10\\
0.101949025487256	3.66526e-10\\
0.103948025987006	3.58796e-10\\
0.105947026486757	3.62093e-10\\
0.107946026986507	3.56408e-10\\
0.109945027486257	3.62434e-10\\
0.111944027986007	3.61069e-10\\
0.113943028485757	3.56408e-10\\
0.115942028985507	3.54362e-10\\
0.117941029485257	3.62093e-10\\
0.119940029985007	3.62093e-10\\
0.121939030484758	3.54817e-10\\
0.123938030984508	3.57545e-10\\
0.125937031484258	3.64935e-10\\
0.127936031984008	3.58796e-10\\
0.129935032483758	3.52088e-10\\
0.131934032983508	3.47995e-10\\
0.133933033483258	3.46745e-10\\
0.135932033983008	3.42652e-10\\
0.137931034482759	3.40719e-10\\
0.139930034982509	3.45949e-10\\
0.141929035482259	3.42538e-10\\
0.143928035982009	3.39469e-10\\
0.145927036481759	3.41743e-10\\
0.147926036981509	3.44244e-10\\
0.149925037481259	3.44471e-10\\
0.15192403798101	3.4629e-10\\
0.15392303848076	3.51747e-10\\
0.15592203898051	3.46517e-10\\
0.15792103948026	3.49246e-10\\
0.15992003998001	3.49701e-10\\
0.16191904047976	3.5493e-10\\
0.16391804097951	3.55385e-10\\
0.16591704147926	3.61979e-10\\
0.16791604197901	3.66526e-10\\
0.169915042478761	3.68118e-10\\
0.171914042978511	3.688e-10\\
0.173913043478261	3.70846e-10\\
0.175912043978011	3.65617e-10\\
0.177911044477761	3.64935e-10\\
0.179910044977511	3.70846e-10\\
0.181909045477261	3.67663e-10\\
0.183908045977011	3.65162e-10\\
0.185907046476762	3.57886e-10\\
0.187906046976512	3.50383e-10\\
0.189905047476262	3.49019e-10\\
0.191904047976012	3.54248e-10\\
0.193903048475762	3.61297e-10\\
0.195902048975512	3.55158e-10\\
0.197901049475262	3.62434e-10\\
0.199900049975012	3.66754e-10\\
0.201899050474763	3.69255e-10\\
0.203898050974513	3.76303e-10\\
0.205897051474263	3.69937e-10\\
0.207896051974013	3.72893e-10\\
0.209895052473763	3.72893e-10\\
0.211894052973513	3.80396e-10\\
0.213893053473263	3.74484e-10\\
0.215892053973014	3.70164e-10\\
0.217891054472764	3.73348e-10\\
0.219890054972514	3.71074e-10\\
0.221889055472264	3.6448e-10\\
0.223888055972014	3.59023e-10\\
0.225887056471764	3.56522e-10\\
0.227886056971514	3.52884e-10\\
0.229885057471264	3.53566e-10\\
0.231884057971014	3.53339e-10\\
0.233883058470765	3.46517e-10\\
0.235882058970515	3.40833e-10\\
0.237881059470265	3.39924e-10\\
0.239880059970015	3.44926e-10\\
0.241879060469765	3.44244e-10\\
0.243878060969515	3.41288e-10\\
0.245877061469265	3.40833e-10\\
0.247876061969015	3.43334e-10\\
0.249875062468766	3.35604e-10\\
0.251874062968516	3.32193e-10\\
0.253873063468266	3.26054e-10\\
0.255872063968016	3.21734e-10\\
0.257871064467766	3.17186e-10\\
0.259870064967516	3.1514e-10\\
0.261869065467266	3.17414e-10\\
0.263868065967017	3.13321e-10\\
0.265867066466767	3.06954e-10\\
0.267866066966517	3.11729e-10\\
0.269865067466267	3.10138e-10\\
0.271864067966017	3.06272e-10\\
0.273863068465767	2.98542e-10\\
0.275862068965517	3.02634e-10\\
0.277861069465267	2.95131e-10\\
0.279860069965017	2.87855e-10\\
0.281859070464768	2.95586e-10\\
0.283858070964518	2.9263e-10\\
0.285857071464268	2.85809e-10\\
0.287856071964018	2.86946e-10\\
0.289855072463768	2.9172e-10\\
0.291854072963518	2.9786e-10\\
0.293853073463268	2.90356e-10\\
0.295852073963018	2.83308e-10\\
0.297851074462769	2.7876e-10\\
0.299850074962519	2.70802e-10\\
0.301849075462269	2.68301e-10\\
0.303848075962019	2.6148e-10\\
0.305847076461769	2.61025e-10\\
0.307846076961519	2.66709e-10\\
0.309845077461269	2.62389e-10\\
0.311844077961019	2.68983e-10\\
0.31384307846077	2.67164e-10\\
0.31584207896052	2.72394e-10\\
0.31784107946027	2.70802e-10\\
0.31984007996002	2.72166e-10\\
0.32183908045977	2.71712e-10\\
0.32383808095952	2.75577e-10\\
0.32583708145927	2.73303e-10\\
0.32783608195902	2.80579e-10\\
0.329835082458771	2.82853e-10\\
0.331834082958521	2.82171e-10\\
0.333833083458271	2.7444e-10\\
0.335832083958021	2.71257e-10\\
0.337831084457771	2.67619e-10\\
0.339830084957521	2.68074e-10\\
0.341829085457271	2.66709e-10\\
0.343828085957021	2.61934e-10\\
0.345827086456772	2.63981e-10\\
0.347826086956522	2.7012e-10\\
0.349825087456272	2.66709e-10\\
0.351824087956022	2.65572e-10\\
0.353823088455772	2.65118e-10\\
0.355822088955522	2.68528e-10\\
0.357821089455272	2.72394e-10\\
0.359820089955023	2.65572e-10\\
0.361819090454773	2.58524e-10\\
0.363818090954523	2.63753e-10\\
0.365817091454273	2.58751e-10\\
0.367816091954023	2.54659e-10\\
0.369815092453773	2.54431e-10\\
0.371814092953523	2.54886e-10\\
0.373813093453273	2.60343e-10\\
0.375812093953024	2.64208e-10\\
0.377811094452774	2.72166e-10\\
0.379810094952524	2.72394e-10\\
0.381809095452274	2.69438e-10\\
0.383808095952024	2.61934e-10\\
0.385807096451774	2.62162e-10\\
0.387806096951524	2.65572e-10\\
0.389805097451274	2.73076e-10\\
0.391804097951024	2.77396e-10\\
0.393803098450775	2.71257e-10\\
0.395802098950525	2.63981e-10\\
0.397801099450275	2.69893e-10\\
0.399800099950025	2.76259e-10\\
0.401799100449775	2.73985e-10\\
0.403798100949525	2.77169e-10\\
0.405797101449275	2.8308e-10\\
0.407796101949025	2.8308e-10\\
0.409795102448776	2.84217e-10\\
0.411794102948526	2.88992e-10\\
0.413793103448276	2.85354e-10\\
0.415792103948026	2.82853e-10\\
0.417791104447776	2.80806e-10\\
0.419790104947526	2.78987e-10\\
0.421789105447276	2.81261e-10\\
0.423788105947026	2.74213e-10\\
0.425787106446777	2.7535e-10\\
0.427786106946527	2.78305e-10\\
0.429785107446277	2.74667e-10\\
0.431784107946027	2.82398e-10\\
0.433783108445777	2.77169e-10\\
0.435782108945527	2.77851e-10\\
0.437781109445277	2.75577e-10\\
0.439780109945027	2.79897e-10\\
0.441779110444778	2.79897e-10\\
0.443778110944528	2.75577e-10\\
0.445777111444278	2.82853e-10\\
0.447776111944028	2.78987e-10\\
0.449775112443778	2.73303e-10\\
0.451774112943528	2.71029e-10\\
0.453773113443278	2.74667e-10\\
0.455772113943028	2.71712e-10\\
0.457771114442779	2.75804e-10\\
0.459770114942529	2.7535e-10\\
0.461769115442279	2.74213e-10\\
0.463768115942029	2.78078e-10\\
0.465767116441779	2.72166e-10\\
0.467766116941529	2.70802e-10\\
0.469765117441279	2.7012e-10\\
0.471764117941029	2.69893e-10\\
0.47376311844078	2.73531e-10\\
0.47576211894053	2.73076e-10\\
0.47776111944028	2.68983e-10\\
0.47976011994003	2.72848e-10\\
0.48175912043978	2.73985e-10\\
0.48375812093953	2.79215e-10\\
0.48575712143928	2.81034e-10\\
0.487756121939031	2.78533e-10\\
0.489755122438781	2.82853e-10\\
0.491754122938531	2.75577e-10\\
0.493753123438281	2.73303e-10\\
0.495752123938031	2.76259e-10\\
0.497751124437781	2.73985e-10\\
0.499750124937531	2.80806e-10\\
0.501749125437281	2.8308e-10\\
0.503748125937031	2.81034e-10\\
0.505747126436782	2.73076e-10\\
0.507746126936532	2.74667e-10\\
0.509745127436282	2.68983e-10\\
0.511744127936032	2.67846e-10\\
0.513743128435782	2.69665e-10\\
0.515742128935532	2.66709e-10\\
0.517741129435282	2.64663e-10\\
0.519740129935032	2.6057e-10\\
0.521739130434783	2.57387e-10\\
0.523738130934533	2.55341e-10\\
0.525737131434283	2.51021e-10\\
0.527736131934033	2.54886e-10\\
0.529735132433783	2.54204e-10\\
0.531734132933533	2.4761e-10\\
0.533733133433283	2.48747e-10\\
0.535732133933034	2.51021e-10\\
0.537731134432784	2.48519e-10\\
0.539730134932534	2.40789e-10\\
0.541729135432284	2.46473e-10\\
0.543728135932034	2.44427e-10\\
0.545727136431784	2.45336e-10\\
0.547726136931534	2.3897e-10\\
0.549725137431284	2.33513e-10\\
0.551724137931034	2.28965e-10\\
0.553723138430785	2.32149e-10\\
0.555722138930535	2.29193e-10\\
0.557721139430285	2.27374e-10\\
0.559720139930035	2.33285e-10\\
0.561719140429785	2.31694e-10\\
0.563718140929535	2.25327e-10\\
0.565717141429285	2.32831e-10\\
0.567716141929036	2.37833e-10\\
0.569715142428786	2.32603e-10\\
0.571714142928536	2.35559e-10\\
0.573713143428286	2.29875e-10\\
0.575712143928036	2.27601e-10\\
0.577711144427786	2.3374e-10\\
0.579710144927536	2.39197e-10\\
0.581709145427286	2.32603e-10\\
0.583708145927036	2.37833e-10\\
0.585707146426787	2.42608e-10\\
0.587706146926537	2.39879e-10\\
0.589705147426287	2.36696e-10\\
0.591704147926037	2.30102e-10\\
0.593703148425787	2.28511e-10\\
0.595702148925537	2.3033e-10\\
0.597701149425287	2.31921e-10\\
0.599700149925038	2.25327e-10\\
0.601699150424788	2.30784e-10\\
0.603698150924538	2.23054e-10\\
0.605697151424288	2.20552e-10\\
0.607696151924038	2.13049e-10\\
0.609695152423788	2.16914e-10\\
0.611694152923538	2.18279e-10\\
0.613693153423288	2.22599e-10\\
0.615692153923038	2.21007e-10\\
0.617691154422789	2.14868e-10\\
0.619690154922539	2.15778e-10\\
0.621689155422289	2.09184e-10\\
0.623688155922039	2.03954e-10\\
0.625687156421789	2.05546e-10\\
0.627686156921539	1.99407e-10\\
0.629685157421289	1.99861e-10\\
0.631684157921039	1.93495e-10\\
0.63368315842079	1.96451e-10\\
0.63568215892054	2.03727e-10\\
0.63768115942029	2.05318e-10\\
0.63968015992004	2.04864e-10\\
0.64167916041979	2.06001e-10\\
0.64367816091954	2.03045e-10\\
0.64567716141929	2.04409e-10\\
0.64767616191904	2.11003e-10\\
0.649675162418791	2.06455e-10\\
0.651674162918541	2.03727e-10\\
0.653673163418291	2.08956e-10\\
0.655672163918041	2.05773e-10\\
0.657671164417791	2.01453e-10\\
0.659670164917541	2.00771e-10\\
0.661669165417291	1.9395e-10\\
0.663668165917041	1.97588e-10\\
0.665667166416792	2.05318e-10\\
0.667666166916542	2.0259e-10\\
0.669665167416292	2.04409e-10\\
0.671664167916042	2.05091e-10\\
0.673663168415792	2.08729e-10\\
0.675662168915542	2.10548e-10\\
0.677661169415292	2.1214e-10\\
0.679660169915043	2.16914e-10\\
0.681659170414793	2.19643e-10\\
0.683658170914543	2.2419e-10\\
0.685657171414293	2.251e-10\\
0.687656171914043	2.17142e-10\\
0.689655172413793	2.10093e-10\\
0.691654172913543	2.05091e-10\\
0.693653173413293	2.11003e-10\\
0.695652173913043	2.13277e-10\\
0.697651174412794	2.14413e-10\\
0.699650174912544	2.1214e-10\\
0.701649175412294	2.10321e-10\\
0.703648175912044	2.16232e-10\\
0.705647176411794	2.10321e-10\\
0.707646176911544	2.14641e-10\\
0.709645177411294	2.12594e-10\\
0.711644177911045	2.09411e-10\\
0.713643178410795	2.13504e-10\\
0.715642178910545	2.14868e-10\\
0.717641179410295	2.11912e-10\\
0.719640179910045	2.0782e-10\\
0.721639180409795	2.12822e-10\\
0.723638180909545	2.17142e-10\\
0.725637181409295	2.09866e-10\\
0.727636181909045	2.05318e-10\\
0.729635182408796	2.04636e-10\\
0.731634182908546	2.03272e-10\\
0.733633183408296	2.06228e-10\\
0.735632183908046	2.08274e-10\\
0.737631184407796	2.03727e-10\\
0.739630184907546	2.02135e-10\\
0.741629185407296	1.95996e-10\\
0.743628185907046	2.02817e-10\\
0.745627186406797	2.04409e-10\\
0.747626186906547	2.06001e-10\\
0.749625187406297	2.05318e-10\\
0.751624187906047	2.06228e-10\\
0.753623188405797	2.04409e-10\\
0.755622188905547	2.01453e-10\\
0.757621189405297	2.04864e-10\\
0.759620189905047	1.99634e-10\\
0.761619190404798	1.94632e-10\\
0.763618190904548	1.93268e-10\\
0.765617191404298	1.99407e-10\\
0.767616191904048	2.00998e-10\\
0.769615192403798	1.94404e-10\\
0.771614192903548	1.90312e-10\\
0.773613193403298	1.88948e-10\\
0.775612193903048	1.88038e-10\\
0.777611194402799	1.83945e-10\\
0.779610194902549	1.77579e-10\\
0.781609195402299	1.78261e-10\\
0.783608195902049	1.73713e-10\\
0.785607196401799	1.7917e-10\\
0.787606196901549	1.80762e-10\\
0.789605197401299	1.80307e-10\\
0.79160419790105	1.86446e-10\\
0.7936031984008	1.8531e-10\\
0.79560219890055	1.8963e-10\\
0.7976011994003	1.95541e-10\\
0.79960019990005	1.92586e-10\\
0.8015992003998	1.90312e-10\\
0.80359820089955	1.86901e-10\\
0.8055972013993	1.85992e-10\\
0.80759620189905	1.90767e-10\\
0.809595202398801	1.93268e-10\\
0.811594202898551	1.8963e-10\\
0.813593203398301	1.9736e-10\\
0.815592203898051	1.92586e-10\\
0.817591204397801	1.98725e-10\\
0.819590204897551	2.00998e-10\\
0.821589205397301	2.04636e-10\\
0.823588205897052	2.04182e-10\\
0.825587206396802	2.03954e-10\\
0.827586206896552	2.00089e-10\\
0.829585207396302	2.07592e-10\\
0.831584207896052	2.05546e-10\\
0.833583208395802	2.05091e-10\\
0.835582208895552	1.98725e-10\\
0.837581209395302	2.05318e-10\\
0.839580209895052	2.1214e-10\\
0.841579210394803	2.04636e-10\\
0.843578210894553	2.08274e-10\\
0.845577211394303	2.13731e-10\\
0.847576211894053	2.17369e-10\\
0.849575212393803	2.24645e-10\\
0.851574212893553	2.32603e-10\\
0.853573213393303	2.25782e-10\\
0.855572213893053	2.33285e-10\\
0.857571214392804	2.31466e-10\\
0.859570214892554	2.34195e-10\\
0.861569215392304	2.37605e-10\\
0.863568215892054	2.3897e-10\\
0.865567216391804	2.44199e-10\\
0.867566216891554	2.41471e-10\\
0.869565217391304	2.4329e-10\\
0.871564217891054	2.39879e-10\\
0.873563218390805	2.33058e-10\\
0.875562218890555	2.34877e-10\\
0.877561219390305	2.27601e-10\\
0.879560219890055	2.23736e-10\\
0.881559220389805	2.1987e-10\\
0.883558220889555	2.18506e-10\\
0.885557221389305	2.22144e-10\\
0.887556221889055	2.23736e-10\\
0.889555222388806	2.31239e-10\\
0.891554222888556	2.3033e-10\\
0.893553223388306	2.26919e-10\\
0.895552223888056	2.33058e-10\\
0.897551224387806	2.25327e-10\\
0.899550224887556	2.31921e-10\\
0.901549225387306	2.26919e-10\\
0.903548225887057	2.29193e-10\\
0.905547226386807	2.27374e-10\\
0.907546226886557	2.34422e-10\\
0.909545227386307	2.27828e-10\\
0.911544227886057	2.23736e-10\\
0.913543228385807	2.22826e-10\\
0.915542228885557	2.26464e-10\\
0.917541229385307	2.2078e-10\\
0.919540229885057	2.14413e-10\\
0.921539230384808	2.15778e-10\\
0.923538230884558	2.20325e-10\\
0.925537231384308	2.22599e-10\\
0.927536231884058	2.18279e-10\\
0.929535232383808	2.25555e-10\\
0.931534232883558	2.18733e-10\\
0.933533233383308	2.24645e-10\\
0.935532233883059	2.17597e-10\\
0.937531234382809	2.18961e-10\\
0.939530234882559	2.11912e-10\\
0.941529235382309	2.05773e-10\\
0.943528235882059	1.98725e-10\\
0.945527236381809	2.02363e-10\\
0.947526236881559	2.00998e-10\\
0.949525237381309	2.02135e-10\\
0.951524237881059	2.04409e-10\\
0.95352323838081	2.11458e-10\\
0.95552223888056	2.11912e-10\\
0.95752123938031	2.13731e-10\\
0.95952023988006	2.1555e-10\\
0.96151924037981	2.19643e-10\\
0.96351824087956	2.22371e-10\\
0.96551724137931	2.18733e-10\\
0.96751624187906	2.26237e-10\\
0.969515242378811	2.30557e-10\\
0.971514242878561	2.27601e-10\\
0.973513243378311	2.30557e-10\\
0.975512243878061	2.36241e-10\\
0.977511244377811	2.4329e-10\\
0.979510244877561	2.47383e-10\\
0.981509245377311	2.40107e-10\\
0.983508245877061	2.44199e-10\\
0.985507246376812	2.41016e-10\\
0.987506246876562	2.47383e-10\\
0.989505247376312	2.44199e-10\\
0.991504247876062	2.41698e-10\\
0.993503248375812	2.43062e-10\\
0.995502248875562	2.43062e-10\\
0.997501249375312	2.41471e-10\\
0.999500249875062	2.48065e-10\\
1.00149925037481	2.467e-10\\
1.00349825087456	2.39424e-10\\
1.00549725137431	2.467e-10\\
1.00749625187406	2.47837e-10\\
1.00949525237381	2.41471e-10\\
1.01149425287356	2.37605e-10\\
1.01349325337331	2.35787e-10\\
1.01549225387306	2.28511e-10\\
1.01749125437281	2.35559e-10\\
1.01949025487256	2.34877e-10\\
1.02148925537231	2.35104e-10\\
1.02348825587206	2.39424e-10\\
1.02548725637181	2.39424e-10\\
1.02748625687156	2.40107e-10\\
1.02948525737131	2.34877e-10\\
1.03148425787106	2.37605e-10\\
1.03348325837081	2.35104e-10\\
1.03548225887056	2.37378e-10\\
1.03748125937031	2.32831e-10\\
1.03948025987006	2.251e-10\\
1.04147926036982	2.28511e-10\\
1.04347826086957	2.22599e-10\\
1.04547726136932	2.21462e-10\\
1.04747626186907	2.15323e-10\\
1.04947526236882	2.13277e-10\\
1.05147426286857	2.16914e-10\\
1.05347326336832	2.15778e-10\\
1.05547226386807	2.19188e-10\\
1.05747126436782	2.14186e-10\\
1.05947026486757	2.21462e-10\\
1.06146926536732	2.24645e-10\\
1.06346826586707	2.31239e-10\\
1.06546726636682	2.33285e-10\\
1.06746626686657	2.33058e-10\\
1.06946526736632	2.31239e-10\\
1.07146426786607	2.32603e-10\\
1.07346326836582	2.38515e-10\\
1.07546226886557	2.40789e-10\\
1.07746126936532	2.34195e-10\\
1.07946026986507	2.40789e-10\\
1.08145927036482	2.42153e-10\\
1.08345827086457	2.44199e-10\\
1.08545727136432	2.37833e-10\\
1.08745627186407	2.37605e-10\\
1.08945527236382	2.30102e-10\\
1.09145427286357	2.34422e-10\\
1.09345327336332	2.33513e-10\\
1.09545227386307	2.34422e-10\\
1.09745127436282	2.27374e-10\\
1.09945027486257	2.25327e-10\\
1.10144927536232	2.17597e-10\\
1.10344827586207	2.15095e-10\\
1.10544727636182	2.09184e-10\\
1.10744627686157	2.05318e-10\\
1.10944527736132	1.98725e-10\\
1.11144427786107	1.92586e-10\\
1.11344327836082	1.89175e-10\\
1.11544227886057	1.88038e-10\\
1.11744127936032	1.86901e-10\\
1.11944027986007	1.9304e-10\\
1.12143928035982	1.97588e-10\\
1.12343828085957	1.96678e-10\\
1.12543728135932	1.99634e-10\\
1.12743628185907	1.93722e-10\\
1.12943528235882	1.87356e-10\\
1.13143428285857	1.91221e-10\\
1.13343328335832	1.8531e-10\\
1.13543228385807	1.90312e-10\\
1.13743128435782	1.92813e-10\\
1.13943028485757	1.96223e-10\\
1.14142928535732	2.0259e-10\\
1.14342828585707	2.0259e-10\\
1.14542728635682	1.98611e-10\\
1.14742628685657	2.04523e-10\\
1.14942528735632	2.01453e-10\\
1.15142428785607	1.95769e-10\\
1.15342328835582	1.98725e-10\\
1.15542228885557	1.98497e-10\\
1.15742128935532	1.99975e-10\\
1.15942028985507	2.01567e-10\\
1.16141929035482	1.97815e-10\\
1.16341829085457	1.97588e-10\\
1.16541729135432	2.01794e-10\\
1.16741629185407	2.0168e-10\\
1.16941529235382	1.95769e-10\\
1.17141429285357	1.9611e-10\\
1.17341329335332	1.94291e-10\\
1.17541229385307	1.9395e-10\\
1.17741129435282	2.0168e-10\\
1.17941029485257	2.0782e-10\\
1.18140929535232	2.06796e-10\\
1.18340829585207	2.06228e-10\\
1.18540729635182	2.12481e-10\\
1.18740629685157	2.08843e-10\\
1.18940529735132	2.03613e-10\\
1.19140429785107	2.0907e-10\\
1.19340329835082	2.13731e-10\\
1.19540229885057	2.11799e-10\\
1.19740129935032	2.13163e-10\\
1.19940029985008	2.15437e-10\\
1.20139930034983	2.21007e-10\\
1.20339830084958	2.22258e-10\\
1.20539730134933	2.28169e-10\\
1.20739630184908	2.30443e-10\\
1.20939530234883	2.30898e-10\\
1.21139430284858	2.26805e-10\\
1.21339330334833	2.23395e-10\\
1.21539230384808	2.26237e-10\\
1.21739130434783	2.32376e-10\\
1.21939030484758	2.29306e-10\\
1.22138930534733	2.34877e-10\\
1.22338830584708	2.3249e-10\\
1.22538730634683	2.27033e-10\\
1.22738630684658	2.34081e-10\\
1.22938530734633	2.35787e-10\\
1.23138430784608	2.41812e-10\\
1.23338330834583	2.43517e-10\\
1.23538230884558	2.467e-10\\
1.23738130934533	2.45677e-10\\
1.23938030984508	2.48747e-10\\
1.24137931034483	2.46132e-10\\
1.24337831084458	2.51703e-10\\
1.24537731134433	2.5841e-10\\
1.24737631184408	2.54317e-10\\
1.24937531234383	2.50225e-10\\
1.25137431284358	2.48406e-10\\
1.25337331334333	2.55227e-10\\
1.25537231384308	2.54772e-10\\
1.25737131434283	2.50679e-10\\
1.25937031484258	2.51475e-10\\
1.26136931534233	2.51589e-10\\
1.26336831584208	2.56705e-10\\
1.26536731634183	2.59661e-10\\
1.26736631684158	2.67391e-10\\
1.26936531734133	2.62276e-10\\
1.27136431784108	2.56819e-10\\
1.27336331834083	2.54204e-10\\
1.27536231884058	2.54431e-10\\
1.27736131934033	2.58296e-10\\
1.27936031984008	2.52498e-10\\
1.28135932033983	2.47837e-10\\
1.28335832083958	2.55e-10\\
1.28535732133933	2.56023e-10\\
1.28735632183908	2.60684e-10\\
1.28935532233883	2.53635e-10\\
1.29135432283858	2.46359e-10\\
1.29335332333833	2.40334e-10\\
1.29535232383808	2.42267e-10\\
1.29735132433783	2.3681e-10\\
1.29935032483758	2.34763e-10\\
1.30134932533733	2.42267e-10\\
1.30334832583708	2.39766e-10\\
1.30534732633683	2.34081e-10\\
1.30734632683658	2.31694e-10\\
1.30934532733633	2.27033e-10\\
1.31134432783608	2.27828e-10\\
1.31334332833583	2.20666e-10\\
1.31534232883558	2.16801e-10\\
1.31734132933533	2.2203e-10\\
1.31934032983508	2.16005e-10\\
1.32133933033483	2.23508e-10\\
1.32333833083458	2.19416e-10\\
1.32533733133433	2.1862e-10\\
1.32733633183408	2.17824e-10\\
1.32933533233383	2.14072e-10\\
1.33133433283358	2.18847e-10\\
1.33333333333333	2.1555e-10\\
1.33533233383308	2.21803e-10\\
1.33733133433283	2.21803e-10\\
1.33933033483258	2.19075e-10\\
1.34132933533233	2.23054e-10\\
1.34332833583208	2.28965e-10\\
1.34532733633183	2.29306e-10\\
1.34732633683158	2.35673e-10\\
1.34932533733133	2.35559e-10\\
1.35132433783108	2.29079e-10\\
1.35332333833083	2.26919e-10\\
1.35532233883058	2.19302e-10\\
1.35732133933033	2.26464e-10\\
1.35932033983009	2.28965e-10\\
1.36131934032984	2.3465e-10\\
1.36331834082959	2.31012e-10\\
1.36531734132934	2.32603e-10\\
1.36731634182909	2.36923e-10\\
1.36931534232884	2.32376e-10\\
1.37131434282859	2.32262e-10\\
1.37331334332834	2.39652e-10\\
1.37531234382809	2.45905e-10\\
1.37731134432784	2.42608e-10\\
1.37931034482759	2.43404e-10\\
1.38130934532734	2.47269e-10\\
1.38330834582709	2.42835e-10\\
1.38530734632684	2.44995e-10\\
1.38730634682659	2.37378e-10\\
1.38930534732634	2.39652e-10\\
1.39130434782609	2.43858e-10\\
1.39330334832584	2.36241e-10\\
1.39530234882559	2.31239e-10\\
1.39730134932534	2.23281e-10\\
1.39930034982509	2.20211e-10\\
1.40129935032484	2.16914e-10\\
1.40329835082459	2.18961e-10\\
1.40529735132434	2.16801e-10\\
1.40729635182409	2.15209e-10\\
1.40929535232384	2.13277e-10\\
1.41129435282359	2.14641e-10\\
1.41329335332334	2.14015e-10\\
1.41529235382309	2.11116e-10\\
1.41729135432284	2.07706e-10\\
1.41929035482259	2.08161e-10\\
1.42128935532234	2.04068e-10\\
1.42328835582209	2.0043e-10\\
1.42528735632184	1.94063e-10\\
1.42728635682159	1.88322e-10\\
1.42928535732134	1.9071e-10\\
1.43128435782109	1.8747e-10\\
1.43328335832084	1.87924e-10\\
1.43528235882059	1.83149e-10\\
1.43728135932034	1.90369e-10\\
1.43928035982009	1.9196e-10\\
1.44127936031984	1.86446e-10\\
1.44327836081959	1.90141e-10\\
1.44527736131934	1.94461e-10\\
1.44727636181909	1.99691e-10\\
1.44927536231884	1.9736e-10\\
1.45127436281859	1.96565e-10\\
1.45327336331834	2.006e-10\\
1.45527236381809	1.99577e-10\\
1.45727136431784	1.97133e-10\\
1.45927036481759	2.03386e-10\\
1.46126936531734	2.02533e-10\\
1.46326836581709	1.99691e-10\\
1.46526736631684	1.97304e-10\\
1.46726636681659	2.03727e-10\\
1.46926536731634	2.02931e-10\\
1.47126436781609	2.09525e-10\\
1.47326336831584	2.04523e-10\\
1.47526236881559	2.0043e-10\\
1.47726136931534	1.9844e-10\\
1.47926036981509	2.03784e-10\\
1.48125937031484	1.97588e-10\\
1.48325837081459	2.04523e-10\\
1.48525737131434	2.00544e-10\\
1.48725637181409	2.00544e-10\\
1.48925537231384	1.99861e-10\\
1.49125437281359	1.9736e-10\\
1.49325337331334	2.04636e-10\\
1.49525237381309	2.08445e-10\\
1.49725137431284	2.08843e-10\\
1.49925037481259	2.01112e-10\\
1.50124937531234	1.96394e-10\\
1.50324837581209	2.02022e-10\\
1.50524737631184	2.07081e-10\\
1.50724637681159	2.07876e-10\\
1.50924537731134	2.05318e-10\\
1.51124437781109	2.06626e-10\\
1.51324337831084	1.99236e-10\\
1.51524237881059	2.06171e-10\\
1.51724137931034	2.00032e-10\\
1.51924037981009	1.93324e-10\\
1.52123938030985	1.86333e-10\\
1.5232383808096	1.8656e-10\\
1.52523738130935	1.87981e-10\\
1.5272363818091	1.82297e-10\\
1.52923538230885	1.74907e-10\\
1.5312343828086	1.72065e-10\\
1.53323338330835	1.74623e-10\\
1.5352323838081	1.66608e-10\\
1.53723138430785	1.69678e-10\\
1.5392303848076	1.68143e-10\\
1.54122938530735	1.72008e-10\\
1.5432283858071	1.72861e-10\\
1.54522738630685	1.70587e-10\\
1.5472263868066	1.69848e-10\\
1.54922538730635	1.64562e-10\\
1.5512243878061	1.66779e-10\\
1.55322338830585	1.70985e-10\\
1.5552223888056	1.76101e-10\\
1.55722138930535	1.69962e-10\\
1.5592203898051	1.68939e-10\\
1.56121939030485	1.62345e-10\\
1.5632183908046	1.59616e-10\\
1.56521739130435	1.56774e-10\\
1.5672163918041	1.59503e-10\\
1.56921539230385	1.60412e-10\\
1.5712143928036	1.67802e-10\\
1.57321339330335	1.60412e-10\\
1.5752123938031	1.61094e-10\\
1.57721139430285	1.64277e-10\\
1.5792103948026	1.58138e-10\\
1.58120939530235	1.58479e-10\\
1.5832083958021	1.56206e-10\\
1.58520739630185	1.48816e-10\\
1.5872063968016	1.43359e-10\\
1.58920539730135	1.46997e-10\\
1.5912043978011	1.49839e-10\\
1.59320339830085	1.55183e-10\\
1.5952023988006	1.49271e-10\\
1.59720139930035	1.44155e-10\\
1.5992003998001	1.37902e-10\\
1.60119940029985	1.41767e-10\\
1.6031984007996	1.35969e-10\\
1.60519740129935	1.32673e-10\\
1.6071964017991	1.31195e-10\\
1.60919540229885	1.24373e-10\\
1.6111944027986	1.21531e-10\\
1.61319340329835	1.17439e-10\\
1.6151924037981	1.23919e-10\\
1.61719140429785	1.20281e-10\\
1.6191904047976	1.26875e-10\\
1.62118940529735	1.28352e-10\\
1.6231884057971	1.27898e-10\\
1.62518740629685	1.24032e-10\\
1.6271864067966	1.18462e-10\\
1.62918540729635	1.20622e-10\\
1.6311844077961	1.24373e-10\\
1.63318340829585	1.28239e-10\\
1.6351824087956	1.35401e-10\\
1.63718140929535	1.27557e-10\\
1.6391804097951	1.22668e-10\\
1.64117941029485	1.21304e-10\\
1.6431784107946	1.15051e-10\\
1.64517741129435	1.21759e-10\\
1.6471764117941	1.14596e-10\\
1.64917541229385	1.17097e-10\\
1.6511744127936	1.21872e-10\\
1.65317341329335	1.19371e-10\\
1.6551724137931	1.25283e-10\\
1.65717141429285	1.29148e-10\\
1.6591704147926	1.33468e-10\\
1.66116941529235	1.28011e-10\\
1.6631684157921	1.29148e-10\\
1.66516741629185	1.31877e-10\\
1.6671664167916	1.38357e-10\\
1.66916541729135	1.39835e-10\\
1.6711644177911	1.39607e-10\\
1.67316341829085	1.40062e-10\\
1.6751624187906	1.41767e-10\\
1.67716141929035	1.41085e-10\\
1.6791604197901	1.33582e-10\\
1.68115942028985	1.37561e-10\\
1.68315842078961	1.39721e-10\\
1.68515742128936	1.37447e-10\\
1.68715642178911	1.43928e-10\\
1.68915542228886	1.42109e-10\\
1.69115442278861	1.37675e-10\\
1.69315342328836	1.40631e-10\\
1.69515242378811	1.46429e-10\\
1.69715142428786	1.51203e-10\\
1.69915042478761	1.45747e-10\\
1.70114942528736	1.5109e-10\\
1.70314842578711	1.56774e-10\\
1.70514742628686	1.50976e-10\\
1.70714642678661	1.45633e-10\\
1.70914542728636	1.53477e-10\\
1.71114442778611	1.59844e-10\\
1.71314342828586	1.61094e-10\\
1.71514242878561	1.57911e-10\\
1.71714142928536	1.50521e-10\\
1.71914042978511	1.44269e-10\\
1.72113943028486	1.39039e-10\\
1.72313843078461	1.42336e-10\\
1.72513743128436	1.42336e-10\\
1.72713643178411	1.38698e-10\\
1.72913543228386	1.35174e-10\\
1.73113443278361	1.34719e-10\\
1.73313343328336	1.33696e-10\\
1.73513243378311	1.28125e-10\\
1.73713143428286	1.2426e-10\\
1.73913043478261	1.18007e-10\\
1.74112943528236	1.10276e-10\\
1.74312843578211	1.05274e-10\\
1.74512743628186	1.07093e-10\\
1.74712643678161	1.13687e-10\\
1.74912543728136	1.21304e-10\\
1.75112443778111	1.24714e-10\\
1.75312343828086	1.22554e-10\\
1.75512243878061	1.1903e-10\\
1.75712143928036	1.26192e-10\\
1.75912043978011	1.27898e-10\\
1.76111944027986	1.2335e-10\\
1.76311844077961	1.27784e-10\\
1.76511744127936	1.24032e-10\\
1.76711644177911	1.30171e-10\\
1.76911544227886	1.34492e-10\\
1.77111444277861	1.41426e-10\\
1.77311344327836	1.38016e-10\\
1.77511244377811	1.34264e-10\\
1.77711144427786	1.33355e-10\\
1.77911044477761	1.3415e-10\\
1.78110944527736	1.40062e-10\\
1.78310844577711	1.45292e-10\\
1.78510744627686	1.47338e-10\\
1.78710644677661	1.51545e-10\\
1.78910544727636	1.51658e-10\\
1.79110444777611	1.47907e-10\\
1.79310344827586	1.42563e-10\\
1.79510244877561	1.39266e-10\\
1.79710144927536	1.33696e-10\\
1.79910044977511	1.26192e-10\\
1.80109945027486	1.21531e-10\\
1.80309845077461	1.2119e-10\\
1.80509745127436	1.1778e-10\\
1.80709645177411	1.10276e-10\\
1.80909545227386	1.09821e-10\\
1.81109445277361	1.15733e-10\\
1.81309345327336	1.14369e-10\\
1.81509245377311	1.11413e-10\\
1.81709145427286	1.07207e-10\\
1.81909045477261	1.07207e-10\\
1.82108945527236	1.06638e-10\\
1.82308845577211	1.08685e-10\\
1.82508745627186	1.04819e-10\\
1.82708645677161	1.02318e-10\\
1.82908545727136	9.91349e-11\\
1.83108445777111	9.1859e-11\\
1.83308345827086	9.09495e-11\\
1.83508245877061	9.89075e-11\\
1.83708145927036	1.02773e-10\\
1.83908045977011	9.75433e-11\\
1.84107946026987	9.70886e-11\\
1.84307846076962	9.27685e-11\\
1.84507746126937	9.45874e-11\\
1.84707646176912	9.54969e-11\\
1.84907546226887	9.77707e-11\\
1.85107446276862	9.77707e-11\\
1.85307346326837	1.04592e-10\\
1.85507246376812	1.11186e-10\\
1.85707146426787	1.10504e-10\\
1.85907046476762	1.05501e-10\\
1.86106946526737	1.03682e-10\\
1.86306846576712	9.57243e-11\\
1.86506746626687	9.50422e-11\\
1.86706646676662	9.84528e-11\\
1.86906546726637	9.86802e-11\\
1.87106446776612	9.50422e-11\\
1.87306346826587	9.45874e-11\\
1.87506246876562	9.95897e-11\\
1.87706146926537	1.02546e-10\\
1.87906046976512	9.77707e-11\\
1.88105947026487	9.11768e-11\\
1.88305847076462	9.52696e-11\\
1.88505747126437	9.07221e-11\\
1.88705647176412	9.004e-11\\
1.88905547226387	8.6402e-11\\
1.89105447276362	9.29958e-11\\
1.89305347326337	8.77662e-11\\
1.89505247376312	9.1859e-11\\
1.89705147426287	8.39009e-11\\
1.89905047476262	7.95808e-11\\
1.90104947526237	7.41238e-11\\
1.90304847576212	6.9349e-11\\
1.90504747626187	6.63931e-11\\
1.90704647676162	7.32143e-11\\
1.90904547726137	7.59428e-11\\
1.91104447776112	7.48059e-11\\
1.91304347826087	7.45786e-11\\
1.91504247876062	7.16227e-11\\
1.91704147926037	6.86668e-11\\
1.91904047976012	6.9349e-11\\
1.92103948025987	6.98037e-11\\
1.92303848075962	6.36646e-11\\
1.92503748125937	6.82121e-11\\
1.92703648175912	6.61657e-11\\
1.92903548225887	6.48015e-11\\
1.93103448275862	6.23004e-11\\
1.93303348325837	6.61657e-11\\
1.93503248375812	7.07132e-11\\
1.93703148425787	7.38964e-11\\
1.93903048475762	6.98037e-11\\
1.94102948525737	6.79847e-11\\
1.94302848575712	7.36691e-11\\
1.94502748625687	6.86668e-11\\
1.94702648675662	6.13909e-11\\
1.94902548725637	6.50289e-11\\
1.95102448775612	6.84395e-11\\
1.95302348825587	7.41238e-11\\
1.95502248875562	7.59428e-11\\
1.95702148925537	8.13998e-11\\
1.95902048975512	8.23093e-11\\
1.96101949025487	7.75344e-11\\
1.96301849075462	7.70797e-11\\
1.96501749125437	7.66249e-11\\
1.96701649175412	8.34461e-11\\
1.96901549225387	8.86757e-11\\
1.97101449275362	8.29914e-11\\
1.97301349325337	8.04903e-11\\
1.97501249375312	7.95808e-11\\
1.97701149425287	8.41283e-11\\
1.97901049475262	8.91305e-11\\
1.98100949525237	8.52651e-11\\
1.98300849575212	9.14042e-11\\
1.98500749625187	9.64064e-11\\
1.98700649675162	9.77707e-11\\
1.98900549725137	9.45874e-11\\
1.99100449775112	9.95897e-11\\
1.99300349825087	9.50422e-11\\
1.99500249875062	1.00727e-10\\
1.99700149925037	9.41327e-11\\
1.99900049975012	9.27685e-11\\
2.00099950024988	8.48104e-11\\
2.00299850074963	8.11724e-11\\
2.00499750124938	7.9126e-11\\
2.00699650174913	8.43556e-11\\
2.00899550224888	7.75344e-11\\
2.01099450274863	7.29869e-11\\
2.01299350324838	7.18501e-11\\
2.01499250374813	6.54836e-11\\
2.01699150424788	6.11635e-11\\
2.01899050474763	5.6616e-11\\
2.02098950524738	6.27551e-11\\
2.02298850574713	6.9349e-11\\
2.02498750624688	7.66249e-11\\
2.02698650674663	7.43512e-11\\
2.02898550724638	7.98082e-11\\
2.03098450774613	8.43556e-11\\
2.03298350824588	8.54925e-11\\
2.03498250874563	9.02673e-11\\
2.03698150924538	8.36735e-11\\
2.03898050974513	8.68567e-11\\
2.04097951024488	9.34506e-11\\
2.04297851074463	9.07221e-11\\
2.04497751124438	8.48104e-11\\
2.04697651174413	8.59472e-11\\
2.04897551224388	8.18545e-11\\
2.05097451274363	8.13998e-11\\
2.05297351324338	7.88987e-11\\
2.05497251374313	7.68523e-11\\
2.05697151424288	7.48059e-11\\
2.05897051474263	6.79847e-11\\
2.06096951524238	7.13953e-11\\
2.06296851574213	6.43468e-11\\
2.06496751624188	6.48015e-11\\
2.06696651674163	6.04814e-11\\
2.06896551724138	5.63887e-11\\
2.07096451774113	5.16138e-11\\
2.07296351824088	5.29781e-11\\
2.07496251874063	5.41149e-11\\
2.07696151924038	5.86624e-11\\
2.07896051974013	5.07043e-11\\
2.08095952023988	4.72937e-11\\
2.08295852073963	5.50244e-11\\
2.08495752123938	5.00222e-11\\
2.08695652173913	4.91127e-11\\
2.08895552223888	5.36602e-11\\
2.09095452273863	5.93445e-11\\
2.09295352323838	5.86624e-11\\
2.09495252373813	5.41149e-11\\
2.09695152423788	5.77529e-11\\
2.09895052473763	5.70708e-11\\
2.10094952523738	6.29825e-11\\
2.10294852573713	6.00267e-11\\
2.10494752623688	6.07088e-11\\
2.10694652673663	6.66205e-11\\
2.10894552723638	6.43468e-11\\
2.11094452773613	6.66205e-11\\
2.11294352823588	7.43512e-11\\
2.11494252873563	8.07177e-11\\
2.11694152923538	8.79936e-11\\
2.11894052973513	8.48104e-11\\
2.12093953023488	7.75344e-11\\
2.12293853073463	7.75344e-11\\
2.12493753123438	8.13998e-11\\
2.12693653173413	7.9126e-11\\
2.12893553223388	7.63976e-11\\
2.13093453273363	7.02585e-11\\
2.13293353323338	6.73026e-11\\
2.13493253373313	7.23048e-11\\
2.13693153423288	7.95808e-11\\
2.13893053473263	7.25322e-11\\
2.14092953523238	7.9126e-11\\
2.14292853573213	8.4583e-11\\
2.14492753623188	8.29914e-11\\
2.14692653673163	7.59428e-11\\
2.14892553723138	8.11724e-11\\
2.15092453773113	8.54925e-11\\
2.15292353823088	9.20863e-11\\
2.15492253873063	9.57243e-11\\
2.15692153923038	9.95897e-11\\
2.15892053973013	1.00727e-10\\
2.16091954022989	9.84528e-11\\
2.16291854072964	9.41327e-11\\
2.16491754122939	9.66338e-11\\
2.16691654172914	9.9817e-11\\
2.16891554222889	1.03e-10\\
2.17091454272864	9.70886e-11\\
2.17291354322839	9.59517e-11\\
2.17491254372814	9.07221e-11\\
2.17691154422789	9.29958e-11\\
2.17891054472764	9.39053e-11\\
2.18090954522739	9.29958e-11\\
2.18290854572714	8.4583e-11\\
2.18490754622689	9.20863e-11\\
2.18690654672664	9.16316e-11\\
2.18890554722639	8.50378e-11\\
2.19090454772614	7.82165e-11\\
2.19290354822589	7.50333e-11\\
2.19490254872564	8.00355e-11\\
2.19690154922539	8.02629e-11\\
2.19890054972514	7.52607e-11\\
2.20089955022489	6.98037e-11\\
2.20289855072464	7.68523e-11\\
2.20489755122439	7.25322e-11\\
2.20689655172414	6.98037e-11\\
2.20889555222389	6.23004e-11\\
2.21089455272364	6.09361e-11\\
2.21289355322339	5.91172e-11\\
2.21489255372314	5.59339e-11\\
2.21689155422289	4.95675e-11\\
2.21889055472264	4.97948e-11\\
2.22088955522239	5.34328e-11\\
2.22288855572214	5.6616e-11\\
2.22488755622189	4.88853e-11\\
2.22688655672164	5.61613e-11\\
2.22888555722139	5.43423e-11\\
2.23088455772114	4.70664e-11\\
2.23288355822089	4.18368e-11\\
2.23488255872064	4.61569e-11\\
2.23688155922039	4.95675e-11\\
2.23888055972014	5.11591e-11\\
2.24087956021989	5.22959e-11\\
2.24287856071964	4.72937e-11\\
2.24487756121939	4.61569e-11\\
2.24687656171914	4.6839e-11\\
2.24887556221889	5.45697e-11\\
2.25087456271864	5.43423e-11\\
2.25287356321839	5.91172e-11\\
2.25487256371814	6.66205e-11\\
2.25687156421789	6.48015e-11\\
2.25887056471764	6.5711e-11\\
2.26086956521739	6.59384e-11\\
2.26286856571714	7.13953e-11\\
2.26486756621689	6.43468e-11\\
2.26686656671664	6.2073e-11\\
2.26886556721639	6.3892e-11\\
2.27086456771614	5.79803e-11\\
2.27286356821589	5.41149e-11\\
2.27486256871564	5.77529e-11\\
2.27686156921539	5.97993e-11\\
2.27886056971514	5.36602e-11\\
2.28085957021489	6.04814e-11\\
2.28285857071464	6.16183e-11\\
2.28485757121439	5.75255e-11\\
2.28685657171414	6.2073e-11\\
2.28885557221389	6.54836e-11\\
2.29085457271364	6.98037e-11\\
2.29285357321339	7.41238e-11\\
2.29485257371314	7.18501e-11\\
2.29685157421289	7.02585e-11\\
2.29885057471264	7.02585e-11\\
2.30084957521239	7.43512e-11\\
2.30284857571214	7.70797e-11\\
2.30484757621189	8.43556e-11\\
2.30684657671164	8.66294e-11\\
2.30884557721139	8.39009e-11\\
2.31084457771114	8.20819e-11\\
2.31284357821089	7.98082e-11\\
2.31484257871064	8.48104e-11\\
2.31684157921039	9.02673e-11\\
2.31884057971015	9.45874e-11\\
2.3208395802099	9.91349e-11\\
2.32283858070965	9.75433e-11\\
2.3248375812094	9.16316e-11\\
2.32683658170915	8.68567e-11\\
2.3288355822089	9.1859e-11\\
2.33083458270865	8.75389e-11\\
2.3328335832084	9.25411e-11\\
2.33483258370815	8.50378e-11\\
2.3368315842079	7.75344e-11\\
2.33883058470765	7.43512e-11\\
2.3408295852074	6.98037e-11\\
2.34282858570715	6.2073e-11\\
2.3448275862069	6.00267e-11\\
2.34682658670665	5.8435e-11\\
2.3488255872064	5.29781e-11\\
2.35082458770615	6.04814e-11\\
2.3528235882059	5.50244e-11\\
2.35482258870565	4.82032e-11\\
2.3568215892054	4.27463e-11\\
2.35882058970515	3.97904e-11\\
2.3608195902049	4.25189e-11\\
2.36281859070465	4.29736e-11\\
2.3648175912044	3.52429e-11\\
2.36681659170415	2.95586e-11\\
2.3688155922039	3.50155e-11\\
2.37081459270365	3.06954e-11\\
2.3728135932034	3.22871e-11\\
2.37481259370315	3.86535e-11\\
2.3768115942029	3.36513e-11\\
2.37881059470265	2.56932e-11\\
2.3808095952024	2.16005e-11\\
2.38280859570215	2.4329e-11\\
2.3848075962019	1.72804e-11\\
2.38680659670165	2.41016e-11\\
2.3888055972014	2.86491e-11\\
2.39080459770115	2.77396e-11\\
2.3928035982009	2.13731e-11\\
2.39480259870065	1.7053e-11\\
2.3968015992004	2.22826e-11\\
2.39880059970015	2.41016e-11\\
2.4007996001999	2.29647e-11\\
2.40279860069965	2.6148e-11\\
2.4047976011994	2.50111e-11\\
2.40679660169915	2.72848e-11\\
2.4087956021989	3.22871e-11\\
2.41079460269865	3.34239e-11\\
2.4127936031984	3.11502e-11\\
2.41479260369815	3.18323e-11\\
2.4167916041979	3.20597e-11\\
2.41879060469765	3.70619e-11\\
2.4207896051974	4.16094e-11\\
2.42278860569715	4.84306e-11\\
2.4247876061969	5.57066e-11\\
2.42678660669665	5.34328e-11\\
2.4287856071964	5.93445e-11\\
2.43078460769615	5.75255e-11\\
2.4327836081959	6.2073e-11\\
2.43478260869565	5.54792e-11\\
2.4367816091954	5.91172e-11\\
2.43878060969515	6.66205e-11\\
2.4407796101949	6.82121e-11\\
2.44277861069465	7.18501e-11\\
2.4447776111944	7.20775e-11\\
2.44677661169415	7.84439e-11\\
2.4487756121939	7.41238e-11\\
2.45077461269365	7.04858e-11\\
2.4527736131934	7.38964e-11\\
2.45477261369315	6.66205e-11\\
2.4567716141929	6.54836e-11\\
2.45877061469265	6.29825e-11\\
2.4607696151924	6.09361e-11\\
2.46276861569215	6.0254e-11\\
2.4647676161919	6.52562e-11\\
2.46676661669165	7.02585e-11\\
2.4687656171914	6.54836e-11\\
2.47076461769115	7.16227e-11\\
2.4727636181909	6.70752e-11\\
2.47476261869065	6.77574e-11\\
2.4767616191904	6.09361e-11\\
2.47876061969016	6.82121e-11\\
2.48075962018991	6.09361e-11\\
2.48275862068966	6.23004e-11\\
2.48475762118941	6.0254e-11\\
2.48675662168916	6.36646e-11\\
2.48875562218891	6.86668e-11\\
2.49075462268866	6.2073e-11\\
2.49275362318841	7.00311e-11\\
2.49475262368816	7.20775e-11\\
2.49675162418791	7.66249e-11\\
2.49875062468766	8.39009e-11\\
2.50074962518741	8.77662e-11\\
2.50274862568716	9.52696e-11\\
2.50474762618691	9.61791e-11\\
2.50674662668666	8.98126e-11\\
2.50874562718641	9.004e-11\\
2.51074462768616	8.86757e-11\\
2.51274362818591	8.25366e-11\\
2.51474262868566	8.43556e-11\\
2.51674162918541	8.43556e-11\\
2.51874062968516	7.95808e-11\\
2.52073963018491	8.16271e-11\\
2.52273863068466	7.93534e-11\\
2.52473763118441	7.38964e-11\\
2.52673663168416	6.91216e-11\\
2.52873563218391	7.07132e-11\\
2.53073463268366	6.50289e-11\\
2.53273363318341	6.07088e-11\\
2.53473263368316	5.52518e-11\\
2.53673163418291	6.16183e-11\\
2.53873063468266	6.91216e-11\\
2.54072963518241	6.29825e-11\\
2.54272863568216	6.16183e-11\\
2.54472763618191	5.8435e-11\\
2.54672663668166	5.29781e-11\\
2.54872563718141	5.95719e-11\\
2.55072463768116	5.34328e-11\\
2.55272363818091	5.95719e-11\\
2.55472263868066	6.61657e-11\\
2.55672163918041	7.09406e-11\\
2.55872063968016	6.36646e-11\\
2.56071964017991	6.3892e-11\\
2.56271864067966	6.88942e-11\\
2.56471764117941	7.34417e-11\\
2.56671664167916	6.59384e-11\\
2.56871564217891	6.48015e-11\\
2.57071464267866	7.07132e-11\\
2.57271364317841	6.61657e-11\\
2.57471264367816	6.59384e-11\\
2.57671164417791	6.753e-11\\
2.57871064467766	6.34373e-11\\
2.58070964517741	6.95763e-11\\
2.58270864567716	7.50333e-11\\
2.58470764617691	8.20819e-11\\
2.58670664667666	7.84439e-11\\
2.58870564717641	8.00355e-11\\
2.59070464767616	7.45786e-11\\
2.59270364817591	7.00311e-11\\
2.59470264867566	7.70797e-11\\
2.59670164917541	7.02585e-11\\
2.59870064967516	7.32143e-11\\
2.60069965017491	6.5711e-11\\
2.60269865067466	6.11635e-11\\
2.60469765117441	5.79803e-11\\
2.60669665167416	6.36646e-11\\
2.60869565217391	7.07132e-11\\
2.61069465267366	6.3892e-11\\
2.61269365317341	6.98037e-11\\
2.61469265367316	6.23004e-11\\
2.61669165417291	5.77529e-11\\
2.61869065467266	6.43468e-11\\
2.62068965517241	6.09361e-11\\
2.62268865567216	6.86668e-11\\
2.62468765617191	6.18456e-11\\
2.62668665667166	6.54836e-11\\
2.62868565717141	6.13909e-11\\
2.63068465767116	6.32099e-11\\
2.63268365817091	6.32099e-11\\
2.63468265867066	6.86668e-11\\
2.63668165917041	7.18501e-11\\
2.63868065967017	7.79892e-11\\
2.64067966016992	8.00355e-11\\
2.64267866066967	7.84439e-11\\
2.64467766116942	7.45786e-11\\
2.64667666166917	7.36691e-11\\
2.64867566216892	7.13953e-11\\
2.65067466266867	7.07132e-11\\
2.65267366316842	6.5711e-11\\
2.65467266366817	5.86624e-11\\
2.65667166416792	5.50244e-11\\
2.65867066466767	4.77485e-11\\
2.66066966516742	4.91127e-11\\
2.66266866566717	5.43423e-11\\
2.66466766616692	4.93401e-11\\
2.66666666666667	5.34328e-11\\
2.66866566716642	5.61613e-11\\
2.67066466766617	6.2073e-11\\
2.67266366816592	5.50244e-11\\
2.67466266866567	5.72982e-11\\
2.67666166916542	6.2073e-11\\
2.67866066966517	6.88942e-11\\
2.68065967016492	6.88942e-11\\
2.68265867066467	7.43512e-11\\
2.68465767116442	7.50333e-11\\
2.68665667166417	6.9349e-11\\
2.68865567216392	6.84395e-11\\
2.69065467266367	6.18456e-11\\
2.69265367316342	6.48015e-11\\
2.69465267366317	6.09361e-11\\
2.69665167416292	5.75255e-11\\
2.69865067466267	6.36646e-11\\
2.70064967516242	7.1168e-11\\
2.70264867566217	7.7307e-11\\
2.70464767616192	8.16271e-11\\
2.70664667666167	8.86757e-11\\
2.70864567716142	8.18545e-11\\
2.71064467766117	8.13998e-11\\
2.71264367816092	8.8221e-11\\
2.71464267866067	8.36735e-11\\
2.71664167916042	8.4583e-11\\
2.71864067966017	8.95852e-11\\
2.72063968015992	8.32188e-11\\
2.72263868065967	8.73115e-11\\
2.72463768115942	9.004e-11\\
2.72663668165917	8.48104e-11\\
2.72863568215892	8.04903e-11\\
2.73063468265867	7.61702e-11\\
2.73263368315842	8.25366e-11\\
2.73463268365817	7.54881e-11\\
2.73663168415792	7.18501e-11\\
2.73863068465767	7.32143e-11\\
2.74062968515742	7.95808e-11\\
2.74262868565717	8.20819e-11\\
2.74462768615692	7.45786e-11\\
2.74662668665667	7.1168e-11\\
2.74862568715642	6.84395e-11\\
2.75062468765617	6.5711e-11\\
2.75262368815592	6.41194e-11\\
2.75462268865567	6.00267e-11\\
2.75662168915542	5.52518e-11\\
2.75862068965517	5.25233e-11\\
2.76061969015492	4.57021e-11\\
2.76261869065467	5.22959e-11\\
2.76461769115442	4.93401e-11\\
2.76661669165417	4.43379e-11\\
2.76861569215392	4.41105e-11\\
2.77061469265367	4.82032e-11\\
2.77261369315342	4.72937e-11\\
2.77461269365317	4.57021e-11\\
2.77661169415292	3.97904e-11\\
2.77861069465267	3.20597e-11\\
2.78060969515242	3.45608e-11\\
2.78260869565217	3.93356e-11\\
2.78460769615192	4.04725e-11\\
2.78660669665167	4.02451e-11\\
2.78860569715142	3.75167e-11\\
2.79060469765117	3.36513e-11\\
2.79260369815092	3.31966e-11\\
2.79460269865067	3.00133e-11\\
2.79660169915043	3.29692e-11\\
2.79860069965018	2.77396e-11\\
2.80059970014993	2.66027e-11\\
2.80259870064968	2.0691e-11\\
2.80459770114943	1.75078e-11\\
2.80659670164918	1.06866e-11\\
2.80859570214893	6.36646e-12\\
2.81059470264868	8.6402e-12\\
2.81259370314843	1.3415e-11\\
2.81459270364818	1.11413e-11\\
2.81659170414793	7.50333e-12\\
2.81859070464768	7.95808e-12\\
2.82058970514743	6.82121e-12\\
2.82258870564718	1.18234e-11\\
2.82458770614693	1.3074e-11\\
2.82658670664668	1.86446e-11\\
2.82858570714643	1.73941e-11\\
2.83058470764618	1.06866e-11\\
2.83258370814593	1.21645e-11\\
2.83458270864568	1.18234e-11\\
2.83658170914543	1.69393e-11\\
2.83858070964518	2.31921e-11\\
2.84057971014493	1.98952e-11\\
2.84257871064468	1.26192e-11\\
2.84457771114443	9.20863e-12\\
2.84657671164418	3.97904e-12\\
2.84857571214393	4.09273e-12\\
2.85057471264368	8.29914e-12\\
2.85257371314343	9.54969e-12\\
2.85457271364318	5.34328e-12\\
2.85657171414293	8.41283e-12\\
2.85857071464268	3.41061e-12\\
2.86056971514243	7.95808e-12\\
2.86256871564218	1.00044e-11\\
2.86456771614193	8.29914e-12\\
2.86656671664168	1.59162e-11\\
2.86856571714143	8.52651e-12\\
2.87056471764118	6.25278e-12\\
2.87256371814093	1.11413e-11\\
2.87456271864068	1.60298e-11\\
2.87656171914043	1.02318e-11\\
2.87856071964018	1.46656e-11\\
2.88055972013993	1.53477e-11\\
2.88255872063968	1.18234e-11\\
2.88455772113943	1.23919e-11\\
2.88655672163918	1.72804e-11\\
2.88855572213893	1.69393e-11\\
2.89055472263868	9.77707e-12\\
2.89255372313843	6.82121e-12\\
2.89455272363818	9.66338e-12\\
2.89655172413793	1.01181e-11\\
2.89855072463768	1.46656e-11\\
2.90054972513743	1.08002e-11\\
2.90254872563718	8.41283e-12\\
2.90454772613693	1.63709e-11\\
2.90654672663668	8.86757e-12\\
2.90854572713643	8.07177e-12\\
2.91054472763618	5.11591e-12\\
2.91254372813593	7.84439e-12\\
2.91454272863568	2.50111e-12\\
2.91654172913543	6.82121e-12\\
2.91854072963518	1.15961e-11\\
2.92053973013493	1.89857e-11\\
2.92253873063468	1.64846e-11\\
2.92453773113443	2.28511e-11\\
2.92653673163418	2.36469e-11\\
2.92853573213393	1.59162e-11\\
2.93053473263368	2.37605e-11\\
2.93253373313343	2.4329e-11\\
2.93453273363318	1.86446e-11\\
2.93653173413293	2.21689e-11\\
2.93853073463268	2.37605e-11\\
2.94052973513243	1.78488e-11\\
2.94252873563218	2.22826e-11\\
2.94452773613193	2.73985e-11\\
2.94652673663168	2.37605e-11\\
2.94852573713143	2.10321e-11\\
2.95052473763118	2.60343e-11\\
2.95252373813093	3.36513e-11\\
2.95452273863068	2.81943e-11\\
2.95652173913043	3.47882e-11\\
2.95852073963018	2.76259e-11\\
2.96051974012994	3.3765e-11\\
2.96251874062969	4.03588e-11\\
2.96451774112944	3.66072e-11\\
2.96651674162919	3.43334e-11\\
2.96851574212894	3.20597e-11\\
2.97051474262869	3.49019e-11\\
2.97251374312844	3.93356e-11\\
2.97451274362819	3.89946e-11\\
2.97651174412794	3.85398e-11\\
2.97851074462769	3.63798e-11\\
2.98050974512744	3.88809e-11\\
2.98250874562719	4.52474e-11\\
2.98450774612694	5.27507e-11\\
2.98650674662669	5.94582e-11\\
2.98850574712644	5.65024e-11\\
2.99050474762619	5.77529e-11\\
2.99250374812594	6.42331e-11\\
2.99450274862569	6.45741e-11\\
2.99650174912544	6.25278e-11\\
2.99850074962519	6.3892e-11\\
3.00049975012494	6.5711e-11\\
3.00249875062469	7.3328e-11\\
3.00449775112444	8.10587e-11\\
3.00649675162419	8.53788e-11\\
3.00849575212394	7.92397e-11\\
3.01049475262369	7.76481e-11\\
3.01249375312344	7.93534e-11\\
3.01449275362319	7.60565e-11\\
3.01649175412294	7.63976e-11\\
3.01849075462269	7.01448e-11\\
3.02048975512244	7.27596e-11\\
3.02248875562219	7.29869e-11\\
3.02448775612194	7.98082e-11\\
3.02648675662169	8.2764e-11\\
3.02848575712144	8.33325e-11\\
3.03048475762119	7.74207e-11\\
3.03248375812094	8.48104e-11\\
3.03448275862069	8.50378e-11\\
3.03648175912044	9.25411e-11\\
3.03848075962019	9.07221e-11\\
3.04047976011994	9.64064e-11\\
3.04247876061969	9.11768e-11\\
3.04447776111944	8.77662e-11\\
3.04647676161919	8.89031e-11\\
3.04847576211894	8.41283e-11\\
3.05047476261869	8.0945e-11\\
3.05247376311844	7.93534e-11\\
3.05447276361819	7.12816e-11\\
3.05647176411794	7.20775e-11\\
3.05847076461769	6.41194e-11\\
3.06046976511744	6.69615e-11\\
3.06246876561719	7.3328e-11\\
3.06446776611694	6.5711e-11\\
3.06646676661669	7.1509e-11\\
3.06846576711644	7.53744e-11\\
3.07046476761619	7.23048e-11\\
3.07246376811594	6.66205e-11\\
3.07446276861569	6.94627e-11\\
3.07646176911544	6.37783e-11\\
3.07846076961519	5.63887e-11\\
3.08045977011494	5.11591e-11\\
3.08245877061469	5.09317e-11\\
3.08445777111444	5.17275e-11\\
3.08645677161419	5.28644e-11\\
3.08845577211394	4.96811e-11\\
3.09045477261369	4.28599e-11\\
3.09245377311344	4.03588e-11\\
3.09445277361319	4.6839e-11\\
3.09645177411294	4.72369e-11\\
3.09845077461269	5.08749e-11\\
3.10044977511244	4.88853e-11\\
3.10244877561219	4.23483e-11\\
3.10444777611194	4.80895e-11\\
3.10644677661169	4.75211e-11\\
3.10844577711144	5.47971e-11\\
3.11044477761119	5.93445e-11\\
3.11244377811094	6.16751e-11\\
3.11444277861069	6.94627e-11\\
3.11644177911044	7.44649e-11\\
3.11844077961019	7.69091e-11\\
3.12043978010994	7.4067e-11\\
3.12243878060969	6.90079e-11\\
3.12443778110945	6.60521e-11\\
3.1264367816092	6.55973e-11\\
3.12843578210895	7.06564e-11\\
3.1304347826087	7.70228e-11\\
3.13243378310845	7.71934e-11\\
3.1344327836082	8.36735e-11\\
3.13643178410795	8.54925e-11\\
3.1384307846077	7.72502e-11\\
3.14042978510745	7.03153e-11\\
3.1424287856072	6.47447e-11\\
3.14442778610695	6.94058e-11\\
3.1464267866067	7.09406e-11\\
3.14842578710645	6.50289e-11\\
3.1504247876062	7.03153e-11\\
3.15242378810595	6.37783e-11\\
3.1544227886057	5.8435e-11\\
3.15642178910545	5.16707e-11\\
3.1584207896052	4.43947e-11\\
3.16041979010495	3.68914e-11\\
3.1624187906047	3.28555e-11\\
3.16441779110445	3.80851e-11\\
3.1664167916042	3.46745e-11\\
3.16841579210395	3.06386e-11\\
3.1704147926037	3.38218e-11\\
3.17241379310345	4.0302e-11\\
3.1744127936032	4.07567e-11\\
3.17641179410295	4.3201e-11\\
3.1784107946027	4.35989e-11\\
3.18040979510245	4.83737e-11\\
3.1824087956022	4.38831e-11\\
3.18440779610195	4.6839e-11\\
3.1864067966017	5.22391e-11\\
3.18840579710145	6.00835e-11\\
3.1904047976012	6.58247e-11\\
3.19240379810095	5.93445e-11\\
3.1944027986007	5.39444e-11\\
3.19640179910045	5.3376e-11\\
3.1984007996002	4.91127e-11\\
3.20039980009995	5.04201e-11\\
3.2023988005997	4.34852e-11\\
3.20439780109945	4.95675e-11\\
3.2063968015992	5.54223e-11\\
3.20839580209895	5.81508e-11\\
3.2103948025987	6.24709e-11\\
3.21239380309845	5.96287e-11\\
3.2143928035982	6.40057e-11\\
3.21639180409795	6.92353e-11\\
3.2183908045977	7.17364e-11\\
3.22038980509745	7.48628e-11\\
3.2223888055972	7.70228e-11\\
3.22438780609695	8.44125e-11\\
3.2263868065967	8.78799e-11\\
3.22838580709645	9.24842e-11\\
3.2303848075962	9.19727e-11\\
3.23238380809595	8.91305e-11\\
3.2343828085957	9.31095e-11\\
3.23638180909545	8.79936e-11\\
3.2383808095952	9.33369e-11\\
3.24037981009495	8.98126e-11\\
3.2423788105947	9.72022e-11\\
3.24437781109445	9.9476e-11\\
3.2463768115942	9.72022e-11\\
3.24837581209395	1.00272e-10\\
3.2503748125937	1.00499e-10\\
3.25237381309345	1.06411e-10\\
3.2543728135932	9.92486e-11\\
3.25637181409295	1.04137e-10\\
3.2583708145927	1.06297e-10\\
3.26036981509245	1.08571e-10\\
3.2623688155922	1.11868e-10\\
3.26436781609195	1.17552e-10\\
3.2663668165917	1.11072e-10\\
3.26836581709145	1.11982e-10\\
3.2703648175912	1.16529e-10\\
3.27236381809095	1.15961e-10\\
3.2743628185907	1.19144e-10\\
3.27636181909045	1.13459e-10\\
3.2783608195902	1.17552e-10\\
3.28035982008995	1.21531e-10\\
3.2823588205897	1.15165e-10\\
3.28435782108946	1.20849e-10\\
3.28635682158921	1.17097e-10\\
3.28835582208896	1.24714e-10\\
3.29035482258871	1.25056e-10\\
3.29235382308846	1.21076e-10\\
3.29435282358821	1.18348e-10\\
3.29635182408796	1.17893e-10\\
3.29835082458771	1.15278e-10\\
3.30034982508746	1.21531e-10\\
3.30234882558721	1.24601e-10\\
3.30434782608696	1.23464e-10\\
3.30634682658671	1.18803e-10\\
3.30834582708646	1.1687e-10\\
3.31034482758621	1.20622e-10\\
3.31234382808596	1.21645e-10\\
3.31434282858571	1.26306e-10\\
3.31634182908546	1.28921e-10\\
3.31834082958521	1.22327e-10\\
3.32033983008496	1.19371e-10\\
3.32233883058471	1.24373e-10\\
3.32433783108446	1.24032e-10\\
3.32633683158421	1.17325e-10\\
3.32833583208396	1.12664e-10\\
3.33033483258371	1.10163e-10\\
3.33233383308346	1.06297e-10\\
3.33433283358321	1.12436e-10\\
3.33633183408296	1.06525e-10\\
3.33833083458271	1.08571e-10\\
3.34032983508246	1.01522e-10\\
3.34232883558221	9.83391e-11\\
3.34432783608196	1.02659e-10\\
3.34632683658171	1.0823e-10\\
3.34832583708146	1.06184e-10\\
3.35032483758121	1.13346e-10\\
3.35232383808096	1.15733e-10\\
3.35432283858071	1.21759e-10\\
3.35632183908046	1.21418e-10\\
3.35832083958021	1.23009e-10\\
3.36031984007996	1.15961e-10\\
3.36231884057971	1.11754e-10\\
3.36431784107946	1.1778e-10\\
3.36631684157921	1.19599e-10\\
3.36831584207896	1.22782e-10\\
3.37031484257871	1.29603e-10\\
3.37231384307846	1.23805e-10\\
3.37431284357821	1.28694e-10\\
3.37631184407796	1.25397e-10\\
3.37831084457771	1.20281e-10\\
3.38030984507746	1.25965e-10\\
3.38230884557721	1.33468e-10\\
3.38430784607696	1.35742e-10\\
3.38630684657671	1.329e-10\\
3.38830584707646	1.25283e-10\\
3.39030484757621	1.3199e-10\\
3.39230384807596	1.39494e-10\\
3.39430284857571	1.41199e-10\\
3.39630184907546	1.48361e-10\\
3.39830084957521	1.42109e-10\\
3.40029985007496	1.38584e-10\\
3.40229885057471	1.45405e-10\\
3.40429785107446	1.41654e-10\\
3.40629685157421	1.42109e-10\\
3.40829585207396	1.36083e-10\\
3.41029485257371	1.35515e-10\\
3.41229385307346	1.3199e-10\\
3.41429285357321	1.39835e-10\\
3.41629185407296	1.44041e-10\\
3.41829085457271	1.47793e-10\\
3.42028985507246	1.40062e-10\\
3.42228885557221	1.39607e-10\\
3.42428785607196	1.41313e-10\\
3.42628685657171	1.39039e-10\\
3.42828585707146	1.39266e-10\\
3.43028485757121	1.40744e-10\\
3.43228385807096	1.44723e-10\\
3.43428285857071	1.48589e-10\\
3.43628185907046	1.55296e-10\\
3.43828085957021	1.55751e-10\\
3.44027986006996	1.57456e-10\\
3.44227886056971	1.60753e-10\\
3.44427786106947	1.53818e-10\\
3.44627686156922	1.51431e-10\\
3.44827586206897	1.437e-10\\
3.45027486256872	1.42222e-10\\
3.45227386306847	1.40858e-10\\
3.45427286356822	1.37106e-10\\
3.45627186406797	1.33127e-10\\
3.45827086456772	1.31081e-10\\
3.46026986506747	1.33582e-10\\
3.46226886556722	1.29376e-10\\
3.46426786606697	1.25965e-10\\
3.46626686656672	1.33468e-10\\
3.46826586706647	1.36993e-10\\
3.47026486756622	1.36311e-10\\
3.47226386806597	1.36197e-10\\
3.47426286856572	1.33127e-10\\
3.47626186906547	1.29489e-10\\
3.47826086956522	1.37106e-10\\
3.48025987006497	1.36197e-10\\
3.48225887056472	1.329e-10\\
3.48425787106447	1.3199e-10\\
3.48625687156422	1.38584e-10\\
3.48825587206397	1.39835e-10\\
3.49025487256372	1.32331e-10\\
3.49225387306347	1.25624e-10\\
3.49425287356322	1.18234e-10\\
3.49625187406297	1.24828e-10\\
3.49825087456272	1.29148e-10\\
3.50024987506247	1.36424e-10\\
3.50224887556222	1.37334e-10\\
3.50424787606197	1.37106e-10\\
3.50624687656172	1.29603e-10\\
3.50824587706147	1.24601e-10\\
3.51024487756122	1.3074e-10\\
3.51224387806097	1.34605e-10\\
3.51424287856072	1.40972e-10\\
3.51624187906047	1.35969e-10\\
3.51824087956022	1.43018e-10\\
3.52023988005997	1.37334e-10\\
3.52223888055972	1.35287e-10\\
3.52423788105947	1.40062e-10\\
3.52623688155922	1.44382e-10\\
3.52823588205897	1.3938e-10\\
3.53023488255872	1.39607e-10\\
3.53223388305847	1.37561e-10\\
3.53423288355822	1.35969e-10\\
3.53623188405797	1.42563e-10\\
3.53823088455772	1.35742e-10\\
3.54022988505747	1.35515e-10\\
3.54222888555722	1.35287e-10\\
3.54422788605697	1.42109e-10\\
3.54622688655672	1.41426e-10\\
3.54822588705647	1.41881e-10\\
3.55022488755622	1.42791e-10\\
3.55222388805597	1.40972e-10\\
3.55422288855572	1.39153e-10\\
3.55622188905547	1.46201e-10\\
3.55822088955522	1.51203e-10\\
3.56021989005497	1.55524e-10\\
3.56221889055472	1.61663e-10\\
3.56421789105447	1.63709e-10\\
3.56621689155422	1.70076e-10\\
3.56821589205397	1.69848e-10\\
3.57021489255372	1.66892e-10\\
3.57221389305347	1.72122e-10\\
3.57421289355322	1.72122e-10\\
3.57621189405297	1.68257e-10\\
3.57821089455272	1.65073e-10\\
3.58020989505247	1.64391e-10\\
3.58220889555222	1.63936e-10\\
3.58420789605197	1.55978e-10\\
3.58620689655172	1.5757e-10\\
3.58820589705147	1.5757e-10\\
3.59020489755122	1.61208e-10\\
3.59220389805097	1.60526e-10\\
3.59420289855072	1.64164e-10\\
3.59620189905047	1.60298e-10\\
3.59820089955022	1.55069e-10\\
3.60019990004997	1.52568e-10\\
3.60219890054973	1.55751e-10\\
3.60419790104948	1.62117e-10\\
3.60619690154923	1.59162e-10\\
3.60819590204898	1.52795e-10\\
3.61019490254873	1.53477e-10\\
3.61219390304848	1.5757e-10\\
3.61419290354823	1.59389e-10\\
3.61619190404798	1.64619e-10\\
3.61819090454773	1.57797e-10\\
3.62018990504748	1.64164e-10\\
3.62218890554723	1.66665e-10\\
3.62418790604698	1.72577e-10\\
3.62618690654673	1.7485e-10\\
3.62818590704648	1.7053e-10\\
3.63018490754623	1.68029e-10\\
3.63218390804598	1.62572e-10\\
3.63418290854573	1.62572e-10\\
3.63618190904548	1.57343e-10\\
3.63818090954523	1.53022e-10\\
3.64017991004498	1.50067e-10\\
3.64217891054473	1.47793e-10\\
3.64417791104448	1.5234e-10\\
3.64617691154423	1.50976e-10\\
3.64817591204398	1.55978e-10\\
3.65017491254373	1.54387e-10\\
3.65217391304348	1.60071e-10\\
3.65417291354323	1.56888e-10\\
3.65617191404298	1.60298e-10\\
3.65817091454273	1.628e-10\\
3.66016991504248	1.63936e-10\\
3.66216891554223	1.60298e-10\\
3.66416791604198	1.65073e-10\\
3.66616691654173	1.58025e-10\\
3.66816591704148	1.55978e-10\\
3.67016491754123	1.55751e-10\\
3.67216391804098	1.53477e-10\\
3.67416291854073	1.58025e-10\\
3.67616191904048	1.64391e-10\\
3.67816091954023	1.60981e-10\\
3.68015992003998	1.68029e-10\\
3.68215892053973	1.66892e-10\\
3.68415792103948	1.71212e-10\\
3.68615692153923	1.73713e-10\\
3.68815592203898	1.77124e-10\\
3.69015492253873	1.76442e-10\\
3.69215392303848	1.78488e-10\\
3.69415292353823	1.78034e-10\\
3.69615192403798	1.82126e-10\\
3.69815092453773	1.81899e-10\\
3.70014992503748	1.84855e-10\\
3.70214892553723	1.77806e-10\\
3.70414792603698	1.73259e-10\\
3.70614692653673	1.66665e-10\\
3.70814592703648	1.59844e-10\\
3.71014492753623	1.62572e-10\\
3.71214392803598	1.64391e-10\\
3.71414292853573	1.63709e-10\\
3.71614192903548	1.6621e-10\\
3.71814092953523	1.62345e-10\\
3.72013993003498	1.69621e-10\\
3.72213893053473	1.74396e-10\\
3.72413793103448	1.68257e-10\\
3.72613693153423	1.62345e-10\\
3.72813593203398	1.56888e-10\\
3.73013493253373	1.50294e-10\\
3.73213393303348	1.45519e-10\\
3.73413293353323	1.4029e-10\\
3.73613193403298	1.37788e-10\\
3.73813093453273	1.42791e-10\\
3.74012993503248	1.49157e-10\\
3.74212893553223	1.55524e-10\\
3.74412793603198	1.53932e-10\\
3.74612693653173	1.4893e-10\\
3.74812593703148	1.48475e-10\\
3.75012493753123	1.46201e-10\\
3.75212393803098	1.46429e-10\\
3.75412293853073	1.41426e-10\\
3.75612193903048	1.43928e-10\\
3.75812093953023	1.43245e-10\\
3.76011994002998	1.44155e-10\\
3.76211894052974	1.42563e-10\\
3.76411794102949	1.34605e-10\\
3.76611694152924	1.41654e-10\\
3.76811594202899	1.37106e-10\\
3.77011494252874	1.4461e-10\\
3.77211394302849	1.44382e-10\\
3.77411294352824	1.45519e-10\\
3.77611194402799	1.41199e-10\\
3.77811094452774	1.3506e-10\\
3.78010994502749	1.29148e-10\\
3.78210894552724	1.26647e-10\\
3.78410794602699	1.26875e-10\\
3.78610694652674	1.19826e-10\\
3.78810594702649	1.2642e-10\\
3.79010494752624	1.31422e-10\\
3.79210394802599	1.33923e-10\\
3.79410294852574	1.31877e-10\\
3.79610194902549	1.31422e-10\\
3.79810094952524	1.31877e-10\\
3.80009995002499	1.29603e-10\\
3.80209895052474	1.26875e-10\\
3.80409795102449	1.27102e-10\\
3.80609695152424	1.25056e-10\\
3.80809595202399	1.31422e-10\\
3.81009495252374	1.24601e-10\\
3.81209395302349	1.24146e-10\\
3.81409295352324	1.29603e-10\\
3.81609195402299	1.23464e-10\\
3.81809095452274	1.2642e-10\\
3.82008995502249	1.28466e-10\\
3.82208895552224	1.35969e-10\\
3.82408795602199	1.28466e-10\\
3.82608695652174	1.28921e-10\\
3.82808595702149	1.25056e-10\\
3.83008495752124	1.19144e-10\\
3.83208395802099	1.18234e-10\\
3.83408295852074	1.20281e-10\\
3.83608195902049	1.17552e-10\\
3.83808095952024	1.19144e-10\\
3.84007996001999	1.22554e-10\\
3.84207896051974	1.19371e-10\\
3.84407796101949	1.19371e-10\\
3.84607696151924	1.2551e-10\\
3.84807596201899	1.22782e-10\\
3.85007496251874	1.24373e-10\\
3.85207396301849	1.28466e-10\\
3.85407296351824	1.27329e-10\\
3.85607196401799	1.21645e-10\\
3.85807096451774	1.27784e-10\\
3.86006996501749	1.24146e-10\\
3.86206896551724	1.28239e-10\\
3.86406796601699	1.25965e-10\\
3.86606696651674	1.27784e-10\\
3.86806596701649	1.20281e-10\\
3.87006496751624	1.20281e-10\\
3.87206396801599	1.18689e-10\\
3.87406296851574	1.23919e-10\\
3.87606196901549	1.20963e-10\\
3.87806096951524	1.21645e-10\\
3.88005997001499	1.24828e-10\\
3.88205897051474	1.22782e-10\\
3.88405797101449	1.26192e-10\\
3.88605697151424	1.23919e-10\\
3.88805597201399	1.27329e-10\\
3.89005497251374	1.31195e-10\\
3.89205397301349	1.28239e-10\\
3.89405297351324	1.31649e-10\\
3.89605197401299	1.28694e-10\\
3.89805097451274	1.21418e-10\\
3.90004997501249	1.23691e-10\\
3.90204897551224	1.27557e-10\\
3.90404797601199	1.32786e-10\\
3.90604697651174	1.38471e-10\\
3.90804597701149	1.31649e-10\\
3.91004497751124	1.28239e-10\\
3.91204397801099	1.22327e-10\\
3.91404297851074	1.19144e-10\\
3.91604197901049	1.17325e-10\\
3.91804097951024	1.21418e-10\\
3.92003998000999	1.15506e-10\\
3.92203898050975	1.07548e-10\\
3.9240379810095	1.04365e-10\\
3.92603698150925	9.82254e-11\\
3.928035982009	9.25411e-11\\
3.93003498250875	8.89031e-11\\
3.9320339830085	9.41327e-11\\
3.93403298350825	9.45874e-11\\
3.936031984008	9.61791e-11\\
3.93803098450775	1.00044e-10\\
3.9400299850075	1.01181e-10\\
3.94202898550725	1.0391e-10\\
3.944027986007	1.05729e-10\\
3.94602698650675	1.08912e-10\\
3.9480259870065	1.14596e-10\\
3.95002498750625	1.1255e-10\\
3.952023988006	1.04365e-10\\
3.95402298850575	1.12323e-10\\
3.9560219890055	1.18462e-10\\
3.95802098950525	1.13232e-10\\
3.960019990005	1.20735e-10\\
3.96201899050475	1.2119e-10\\
3.9640179910045	1.28011e-10\\
3.96601699150425	1.34605e-10\\
3.968015992004	1.40062e-10\\
3.97001499250375	1.38243e-10\\
3.9720139930035	1.30967e-10\\
3.97401299350325	1.35287e-10\\
3.976011994003	1.39607e-10\\
3.97801099450275	1.38471e-10\\
3.9800099950025	1.33241e-10\\
3.98200899550225	1.34378e-10\\
3.984007996002	1.40744e-10\\
3.98600699650175	1.33468e-10\\
3.9880059970015	1.40972e-10\\
3.99000499750125	1.33696e-10\\
3.992003998001	1.37788e-10\\
3.99400299850075	1.35515e-10\\
3.9960019990005	1.38016e-10\\
3.99800099950025	1.34378e-10\\
4	1.37334e-10\\
};
\addlegendentry{c1};

\addplot [color=mycolor5,solid]
  table[row sep=crcr]{%
0	5.00222e-11\\
0.00199900049975012	5.00222e-11\\
0.00399800099950025	5.00222e-11\\
0.00599700149925037	5.00222e-11\\
0.0079960019990005	5.00222e-11\\
0.00999500249875063	4.91127e-11\\
0.0119940029985007	4.82032e-11\\
0.0139930034982509	4.82032e-11\\
0.015992003998001	4.82032e-11\\
0.0179910044977511	4.82032e-11\\
0.0199900049975013	4.91127e-11\\
0.0219890054972514	4.91127e-11\\
0.0239880059970015	4.91127e-11\\
0.0259870064967516	5.00222e-11\\
0.0279860069965017	5.00222e-11\\
0.0299850074962519	5.00222e-11\\
0.031984007996002	5.09317e-11\\
0.0339830084957521	5.00222e-11\\
0.0359820089955022	5.09317e-11\\
0.0379810094952524	5.09317e-11\\
0.0399800099950025	5.09317e-11\\
0.0419790104947526	5.09317e-11\\
0.0439780109945027	5.18412e-11\\
0.0459770114942529	5.09317e-11\\
0.047976011994003	5.09317e-11\\
0.0499750124937531	5.00222e-11\\
0.0519740129935032	5.00222e-11\\
0.0539730134932534	5.00222e-11\\
0.0559720139930035	5.00222e-11\\
0.0579710144927536	5.00222e-11\\
0.0599700149925037	5.00222e-11\\
0.0619690154922539	5.00222e-11\\
0.063968015992004	5.00222e-11\\
0.0659670164917541	4.91127e-11\\
0.0679660169915042	5.00222e-11\\
0.0699650174912544	5.00222e-11\\
0.0719640179910045	5.00222e-11\\
0.0739630184907546	4.91127e-11\\
0.0759620189905048	4.91127e-11\\
0.0779610194902549	4.82032e-11\\
0.079960019990005	4.91127e-11\\
0.0819590204897551	4.91127e-11\\
0.0839580209895052	5.00222e-11\\
0.0859570214892554	5.00222e-11\\
0.0879560219890055	5.00222e-11\\
0.0899550224887556	5.00222e-11\\
0.0919540229885057	5.00222e-11\\
0.0939530234882559	5.09317e-11\\
0.095952023988006	5.09317e-11\\
0.0979510244877561	5.00222e-11\\
0.0999500249875062	5.09317e-11\\
0.101949025487256	5.00222e-11\\
0.103948025987006	5.00222e-11\\
0.105947026486757	5.00222e-11\\
0.107946026986507	5.00222e-11\\
0.109945027486257	5.00222e-11\\
0.111944027986007	5.00222e-11\\
0.113943028485757	5.00222e-11\\
0.115942028985507	5.00222e-11\\
0.117941029485257	5.00222e-11\\
0.119940029985007	5.00222e-11\\
0.121939030484758	5.00222e-11\\
0.123938030984508	5.00222e-11\\
0.125937031484258	5.09317e-11\\
0.127936031984008	5.00222e-11\\
0.129935032483758	5.00222e-11\\
0.131934032983508	5.00222e-11\\
0.133933033483258	5.00222e-11\\
0.135932033983008	4.91127e-11\\
0.137931034482759	4.91127e-11\\
0.139930034982509	4.91127e-11\\
0.141929035482259	4.91127e-11\\
0.143928035982009	4.91127e-11\\
0.145927036481759	5.00222e-11\\
0.147926036981509	5.00222e-11\\
0.149925037481259	5.00222e-11\\
0.15192403798101	5.00222e-11\\
0.15392303848076	5.00222e-11\\
0.15592203898051	5.00222e-11\\
0.15792103948026	5.00222e-11\\
0.15992003998001	5.00222e-11\\
0.16191904047976	5.09317e-11\\
0.16391804097951	5.00222e-11\\
0.16591704147926	5.09317e-11\\
0.16791604197901	5.09317e-11\\
0.169915042478761	5.09317e-11\\
0.171914042978511	5.09317e-11\\
0.173913043478261	5.09317e-11\\
0.175912043978011	5.09317e-11\\
0.177911044477761	5.09317e-11\\
0.179910044977511	5.18412e-11\\
0.181909045477261	5.18412e-11\\
0.183908045977011	5.18412e-11\\
0.185907046476762	5.09317e-11\\
0.187906046976512	5.00222e-11\\
0.189905047476262	5.00222e-11\\
0.191904047976012	5.00222e-11\\
0.193903048475762	5.09317e-11\\
0.195902048975512	5.09317e-11\\
0.197901049475262	5.18412e-11\\
0.199900049975012	5.18412e-11\\
0.201899050474763	5.27507e-11\\
0.203898050974513	5.27507e-11\\
0.205897051474263	5.27507e-11\\
0.207896051974013	5.27507e-11\\
0.209895052473763	5.27507e-11\\
0.211894052973513	5.27507e-11\\
0.213893053473263	5.27507e-11\\
0.215892053973014	5.27507e-11\\
0.217891054472764	5.27507e-11\\
0.219890054972514	5.27507e-11\\
0.221889055472264	5.27507e-11\\
0.223888055972014	5.18412e-11\\
0.225887056471764	5.09317e-11\\
0.227886056971514	5.09317e-11\\
0.229885057471264	5.00222e-11\\
0.231884057971014	5.00222e-11\\
0.233883058470765	5.00222e-11\\
0.235882058970515	5.00222e-11\\
0.237881059470265	5.00222e-11\\
0.239880059970015	5.00222e-11\\
0.241879060469765	5.00222e-11\\
0.243878060969515	5.00222e-11\\
0.245877061469265	5.00222e-11\\
0.247876061969015	5.00222e-11\\
0.249875062468766	4.91127e-11\\
0.251874062968516	4.91127e-11\\
0.253873063468266	4.91127e-11\\
0.255872063968016	4.82032e-11\\
0.257871064467766	4.82032e-11\\
0.259870064967516	4.91127e-11\\
0.261869065467266	4.91127e-11\\
0.263868065967017	4.91127e-11\\
0.265867066466767	4.82032e-11\\
0.267866066966517	4.91127e-11\\
0.269865067466267	4.91127e-11\\
0.271864067966017	4.91127e-11\\
0.273863068465767	4.91127e-11\\
0.275862068965517	4.91127e-11\\
0.277861069465267	4.91127e-11\\
0.279860069965017	4.82032e-11\\
0.281859070464768	4.91127e-11\\
0.283858070964518	4.91127e-11\\
0.285857071464268	4.91127e-11\\
0.287856071964018	4.91127e-11\\
0.289855072463768	5.00222e-11\\
0.291854072963518	5.00222e-11\\
0.293853073463268	4.91127e-11\\
0.295852073963018	4.91127e-11\\
0.297851074462769	4.91127e-11\\
0.299850074962519	4.82032e-11\\
0.301849075462269	4.82032e-11\\
0.303848075962019	4.72937e-11\\
0.305847076461769	4.82032e-11\\
0.307846076961519	4.82032e-11\\
0.309845077461269	4.82032e-11\\
0.311844077961019	4.82032e-11\\
0.31384307846077	4.82032e-11\\
0.31584207896052	4.82032e-11\\
0.31784107946027	4.82032e-11\\
0.31984007996002	4.82032e-11\\
0.32183908045977	4.82032e-11\\
0.32383808095952	4.82032e-11\\
0.32583708145927	4.82032e-11\\
0.32783608195902	4.91127e-11\\
0.329835082458771	4.91127e-11\\
0.331834082958521	4.91127e-11\\
0.333833083458271	4.82032e-11\\
0.335832083958021	4.82032e-11\\
0.337831084457771	4.82032e-11\\
0.339830084957521	4.82032e-11\\
0.341829085457271	4.82032e-11\\
0.343828085957021	4.72937e-11\\
0.345827086456772	4.82032e-11\\
0.347826086956522	4.82032e-11\\
0.349825087456272	4.82032e-11\\
0.351824087956022	4.82032e-11\\
0.353823088455772	4.82032e-11\\
0.355822088955522	4.82032e-11\\
0.357821089455272	4.82032e-11\\
0.359820089955023	4.82032e-11\\
0.361819090454773	4.72937e-11\\
0.363818090954523	4.72937e-11\\
0.365817091454273	4.72937e-11\\
0.367816091954023	4.72937e-11\\
0.369815092453773	4.72937e-11\\
0.371814092953523	4.82032e-11\\
0.373813093453273	4.82032e-11\\
0.375812093953024	4.82032e-11\\
0.377811094452774	4.91127e-11\\
0.379810094952524	4.91127e-11\\
0.381809095452274	4.82032e-11\\
0.383808095952024	4.82032e-11\\
0.385807096451774	4.82032e-11\\
0.387806096951524	4.82032e-11\\
0.389805097451274	4.82032e-11\\
0.391804097951024	4.82032e-11\\
0.393803098450775	4.82032e-11\\
0.395802098950525	4.91127e-11\\
0.397801099450275	4.91127e-11\\
0.399800099950025	4.91127e-11\\
0.401799100449775	4.91127e-11\\
0.403798100949525	4.91127e-11\\
0.405797101449275	4.82032e-11\\
0.407796101949025	4.82032e-11\\
0.409795102448776	4.82032e-11\\
0.411794102948526	4.82032e-11\\
0.413793103448276	4.82032e-11\\
0.415792103948026	4.82032e-11\\
0.417791104447776	4.82032e-11\\
0.419790104947526	4.82032e-11\\
0.421789105447276	4.91127e-11\\
0.423788105947026	4.82032e-11\\
0.425787106446777	4.82032e-11\\
0.427786106946527	4.91127e-11\\
0.429785107446277	4.91127e-11\\
0.431784107946027	5.00222e-11\\
0.433783108445777	5.00222e-11\\
0.435782108945527	5.00222e-11\\
0.437781109445277	5.00222e-11\\
0.439780109945027	5.00222e-11\\
0.441779110444778	5.00222e-11\\
0.443778110944528	5.00222e-11\\
0.445777111444278	5.00222e-11\\
0.447776111944028	5.00222e-11\\
0.449775112443778	4.91127e-11\\
0.451774112943528	4.91127e-11\\
0.453773113443278	4.91127e-11\\
0.455772113943028	4.82032e-11\\
0.457771114442779	4.82032e-11\\
0.459770114942529	4.82032e-11\\
0.461769115442279	4.82032e-11\\
0.463768115942029	4.82032e-11\\
0.465767116441779	4.82032e-11\\
0.467766116941529	4.72937e-11\\
0.469765117441279	4.72937e-11\\
0.471764117941029	4.72937e-11\\
0.47376311844078	4.82032e-11\\
0.47576211894053	4.82032e-11\\
0.47776111944028	4.82032e-11\\
0.47976011994003	4.82032e-11\\
0.48175912043978	4.82032e-11\\
0.48375812093953	4.82032e-11\\
0.48575712143928	4.82032e-11\\
0.487756121939031	4.82032e-11\\
0.489755122438781	4.82032e-11\\
0.491754122938531	4.82032e-11\\
0.493753123438281	4.82032e-11\\
0.495752123938031	4.72937e-11\\
0.497751124437781	4.72937e-11\\
0.499750124937531	4.82032e-11\\
0.501749125437281	4.72937e-11\\
0.503748125937031	4.82032e-11\\
0.505747126436782	4.72937e-11\\
0.507746126936532	4.82032e-11\\
0.509745127436282	4.72937e-11\\
0.511744127936032	4.72937e-11\\
0.513743128435782	4.72937e-11\\
0.515742128935532	4.72937e-11\\
0.517741129435282	4.72937e-11\\
0.519740129935032	4.82032e-11\\
0.521739130434783	4.72937e-11\\
0.523738130934533	4.72937e-11\\
0.525737131434283	4.72937e-11\\
0.527736131934033	4.72937e-11\\
0.529735132433783	4.63842e-11\\
0.531734132933533	4.72937e-11\\
0.533733133433283	4.72937e-11\\
0.535732133933034	4.72937e-11\\
0.537731134432784	4.72937e-11\\
0.539730134932534	4.72937e-11\\
0.541729135432284	4.72937e-11\\
0.543728135932034	4.72937e-11\\
0.545727136431784	4.72937e-11\\
0.547726136931534	4.72937e-11\\
0.549725137431284	4.82032e-11\\
0.551724137931034	4.72937e-11\\
0.553723138430785	4.72937e-11\\
0.555722138930535	4.72937e-11\\
0.557721139430285	4.72937e-11\\
0.559720139930035	4.72937e-11\\
0.561719140429785	4.72937e-11\\
0.563718140929535	4.72937e-11\\
0.565717141429285	4.72937e-11\\
0.567716141929036	4.72937e-11\\
0.569715142428786	4.72937e-11\\
0.571714142928536	4.72937e-11\\
0.573713143428286	4.72937e-11\\
0.575712143928036	4.72937e-11\\
0.577711144427786	4.72937e-11\\
0.579710144927536	4.72937e-11\\
0.581709145427286	4.63842e-11\\
0.583708145927036	4.63842e-11\\
0.585707146426787	4.63842e-11\\
0.587706146926537	4.63842e-11\\
0.589705147426287	4.63842e-11\\
0.591704147926037	4.54747e-11\\
0.593703148425787	4.54747e-11\\
0.595702148925537	4.54747e-11\\
0.597701149425287	4.45652e-11\\
0.599700149925038	4.45652e-11\\
0.601699150424788	4.54747e-11\\
0.603698150924538	4.45652e-11\\
0.605697151424288	4.45652e-11\\
0.607696151924038	4.45652e-11\\
0.609695152423788	4.45652e-11\\
0.611694152923538	4.45652e-11\\
0.613693153423288	4.45652e-11\\
0.615692153923038	4.45652e-11\\
0.617691154422789	4.45652e-11\\
0.619690154922539	4.45652e-11\\
0.621689155422289	4.45652e-11\\
0.623688155922039	4.36557e-11\\
0.625687156421789	4.36557e-11\\
0.627686156921539	4.36557e-11\\
0.629685157421289	4.36557e-11\\
0.631684157921039	4.36557e-11\\
0.63368315842079	4.36557e-11\\
0.63568215892054	4.45652e-11\\
0.63768115942029	4.45652e-11\\
0.63968015992004	4.45652e-11\\
0.64167916041979	4.45652e-11\\
0.64367816091954	4.45652e-11\\
0.64567716141929	4.36557e-11\\
0.64767616191904	4.36557e-11\\
0.649675162418791	4.36557e-11\\
0.651674162918541	4.36557e-11\\
0.653673163418291	4.36557e-11\\
0.655672163918041	4.36557e-11\\
0.657671164417791	4.36557e-11\\
0.659670164917541	4.36557e-11\\
0.661669165417291	4.36557e-11\\
0.663668165917041	4.36557e-11\\
0.665667166416792	4.36557e-11\\
0.667666166916542	4.36557e-11\\
0.669665167416292	4.36557e-11\\
0.671664167916042	4.36557e-11\\
0.673663168415792	4.36557e-11\\
0.675662168915542	4.36557e-11\\
0.677661169415292	4.36557e-11\\
0.679660169915043	4.36557e-11\\
0.681659170414793	4.36557e-11\\
0.683658170914543	4.27463e-11\\
0.685657171414293	4.36557e-11\\
0.687656171914043	4.36557e-11\\
0.689655172413793	4.36557e-11\\
0.691654172913543	4.36557e-11\\
0.693653173413293	4.45652e-11\\
0.695652173913043	4.45652e-11\\
0.697651174412794	4.54747e-11\\
0.699650174912544	4.54747e-11\\
0.701649175412294	4.54747e-11\\
0.703648175912044	4.45652e-11\\
0.705647176411794	4.45652e-11\\
0.707646176911544	4.45652e-11\\
0.709645177411294	4.45652e-11\\
0.711644177911045	4.45652e-11\\
0.713643178410795	4.54747e-11\\
0.715642178910545	4.54747e-11\\
0.717641179410295	4.54747e-11\\
0.719640179910045	4.45652e-11\\
0.721639180409795	4.45652e-11\\
0.723638180909545	4.45652e-11\\
0.725637181409295	4.45652e-11\\
0.727636181909045	4.45652e-11\\
0.729635182408796	4.45652e-11\\
0.731634182908546	4.45652e-11\\
0.733633183408296	4.45652e-11\\
0.735632183908046	4.45652e-11\\
0.737631184407796	4.45652e-11\\
0.739630184907546	4.45652e-11\\
0.741629185407296	4.45652e-11\\
0.743628185907046	4.45652e-11\\
0.745627186406797	4.45652e-11\\
0.747626186906547	4.45652e-11\\
0.749625187406297	4.45652e-11\\
0.751624187906047	4.54747e-11\\
0.753623188405797	4.45652e-11\\
0.755622188905547	4.54747e-11\\
0.757621189405297	4.54747e-11\\
0.759620189905047	4.54747e-11\\
0.761619190404798	4.54747e-11\\
0.763618190904548	4.54747e-11\\
0.765617191404298	4.54747e-11\\
0.767616191904048	4.54747e-11\\
0.769615192403798	4.54747e-11\\
0.771614192903548	4.54747e-11\\
0.773613193403298	4.54747e-11\\
0.775612193903048	4.54747e-11\\
0.777611194402799	4.54747e-11\\
0.779610194902549	4.45652e-11\\
0.781609195402299	4.45652e-11\\
0.783608195902049	4.45652e-11\\
0.785607196401799	4.45652e-11\\
0.787606196901549	4.45652e-11\\
0.789605197401299	4.45652e-11\\
0.79160419790105	4.45652e-11\\
0.7936031984008	4.45652e-11\\
0.79560219890055	4.45652e-11\\
0.7976011994003	4.45652e-11\\
0.79960019990005	4.54747e-11\\
0.8015992003998	4.45652e-11\\
0.80359820089955	4.45652e-11\\
0.8055972013993	4.45652e-11\\
0.80759620189905	4.45652e-11\\
0.809595202398801	4.45652e-11\\
0.811594202898551	4.45652e-11\\
0.813593203398301	4.45652e-11\\
0.815592203898051	4.45652e-11\\
0.817591204397801	4.45652e-11\\
0.819590204897551	4.45652e-11\\
0.821589205397301	4.45652e-11\\
0.823588205897052	4.45652e-11\\
0.825587206396802	4.45652e-11\\
0.827586206896552	4.45652e-11\\
0.829585207396302	4.45652e-11\\
0.831584207896052	4.45652e-11\\
0.833583208395802	4.45652e-11\\
0.835582208895552	4.45652e-11\\
0.837581209395302	4.45652e-11\\
0.839580209895052	4.45652e-11\\
0.841579210394803	4.36557e-11\\
0.843578210894553	4.36557e-11\\
0.845577211394303	4.36557e-11\\
0.847576211894053	4.36557e-11\\
0.849575212393803	4.36557e-11\\
0.851574212893553	4.45652e-11\\
0.853573213393303	4.36557e-11\\
0.855572213893053	4.45652e-11\\
0.857571214392804	4.45652e-11\\
0.859570214892554	4.36557e-11\\
0.861569215392304	4.36557e-11\\
0.863568215892054	4.36557e-11\\
0.865567216391804	4.36557e-11\\
0.867566216891554	4.36557e-11\\
0.869565217391304	4.27463e-11\\
0.871564217891054	4.27463e-11\\
0.873563218390805	4.27463e-11\\
0.875562218890555	4.27463e-11\\
0.877561219390305	4.27463e-11\\
0.879560219890055	4.27463e-11\\
0.881559220389805	4.27463e-11\\
0.883558220889555	4.27463e-11\\
0.885557221389305	4.27463e-11\\
0.887556221889055	4.27463e-11\\
0.889555222388806	4.27463e-11\\
0.891554222888556	4.27463e-11\\
0.893553223388306	4.27463e-11\\
0.895552223888056	4.27463e-11\\
0.897551224387806	4.27463e-11\\
0.899550224887556	4.18368e-11\\
0.901549225387306	4.18368e-11\\
0.903548225887057	4.18368e-11\\
0.905547226386807	4.18368e-11\\
0.907546226886557	4.27463e-11\\
0.909545227386307	4.27463e-11\\
0.911544227886057	4.27463e-11\\
0.913543228385807	4.27463e-11\\
0.915542228885557	4.27463e-11\\
0.917541229385307	4.27463e-11\\
0.919540229885057	4.18368e-11\\
0.921539230384808	4.18368e-11\\
0.923538230884558	4.27463e-11\\
0.925537231384308	4.18368e-11\\
0.927536231884058	4.18368e-11\\
0.929535232383808	4.18368e-11\\
0.931534232883558	4.18368e-11\\
0.933533233383308	4.27463e-11\\
0.935532233883059	4.18368e-11\\
0.937531234382809	4.18368e-11\\
0.939530234882559	4.18368e-11\\
0.941529235382309	4.18368e-11\\
0.943528235882059	4.18368e-11\\
0.945527236381809	4.18368e-11\\
0.947526236881559	4.18368e-11\\
0.949525237381309	4.18368e-11\\
0.951524237881059	4.27463e-11\\
0.95352323838081	4.27463e-11\\
0.95552223888056	4.27463e-11\\
0.95752123938031	4.27463e-11\\
0.95952023988006	4.27463e-11\\
0.96151924037981	4.36557e-11\\
0.96351824087956	4.36557e-11\\
0.96551724137931	4.36557e-11\\
0.96751624187906	4.36557e-11\\
0.969515242378811	4.36557e-11\\
0.971514242878561	4.36557e-11\\
0.973513243378311	4.36557e-11\\
0.975512243878061	4.45652e-11\\
0.977511244377811	4.36557e-11\\
0.979510244877561	4.45652e-11\\
0.981509245377311	4.36557e-11\\
0.983508245877061	4.45652e-11\\
0.985507246376812	4.36557e-11\\
0.987506246876562	4.45652e-11\\
0.989505247376312	4.45652e-11\\
0.991504247876062	4.45652e-11\\
0.993503248375812	4.45652e-11\\
0.995502248875562	4.45652e-11\\
0.997501249375312	4.54747e-11\\
0.999500249875062	4.54747e-11\\
1.00149925037481	4.54747e-11\\
1.00349825087456	4.45652e-11\\
1.00549725137431	4.54747e-11\\
1.00749625187406	4.54747e-11\\
1.00949525237381	4.54747e-11\\
1.01149425287356	4.54747e-11\\
1.01349325337331	4.45652e-11\\
1.01549225387306	4.45652e-11\\
1.01749125437281	4.45652e-11\\
1.01949025487256	4.45652e-11\\
1.02148925537231	4.45652e-11\\
1.02348825587206	4.54747e-11\\
1.02548725637181	4.54747e-11\\
1.02748625687156	4.54747e-11\\
1.02948525737131	4.54747e-11\\
1.03148425787106	4.63842e-11\\
1.03348325837081	4.63842e-11\\
1.03548225887056	4.63842e-11\\
1.03748125937031	4.54747e-11\\
1.03948025987006	4.45652e-11\\
1.04147926036982	4.45652e-11\\
1.04347826086957	4.45652e-11\\
1.04547726136932	4.54747e-11\\
1.04747626186907	4.45652e-11\\
1.04947526236882	4.54747e-11\\
1.05147426286857	4.54747e-11\\
1.05347326336832	4.54747e-11\\
1.05547226386807	4.45652e-11\\
1.05747126436782	4.45652e-11\\
1.05947026486757	4.54747e-11\\
1.06146926536732	4.54747e-11\\
1.06346826586707	4.63842e-11\\
1.06546726636682	4.63842e-11\\
1.06746626686657	4.54747e-11\\
1.06946526736632	4.54747e-11\\
1.07146426786607	4.54747e-11\\
1.07346326836582	4.54747e-11\\
1.07546226886557	4.54747e-11\\
1.07746126936532	4.45652e-11\\
1.07946026986507	4.54747e-11\\
1.08145927036482	4.54747e-11\\
1.08345827086457	4.54747e-11\\
1.08545727136432	4.45652e-11\\
1.08745627186407	4.45652e-11\\
1.08945527236382	4.36557e-11\\
1.09145427286357	4.36557e-11\\
1.09345327336332	4.36557e-11\\
1.09545227386307	4.36557e-11\\
1.09745127436282	4.27463e-11\\
1.09945027486257	4.18368e-11\\
1.10144927536232	4.18368e-11\\
1.10344827586207	4.09273e-11\\
1.10544727636182	4.09273e-11\\
1.10744627686157	4.00178e-11\\
1.10944527736132	4.00178e-11\\
1.11144427786107	3.91083e-11\\
1.11344327836082	3.81988e-11\\
1.11544227886057	3.91083e-11\\
1.11744127936032	3.91083e-11\\
1.11944027986007	3.91083e-11\\
1.12143928035982	3.91083e-11\\
1.12343828085957	3.91083e-11\\
1.12543728135932	3.91083e-11\\
1.12743628185907	3.91083e-11\\
1.12943528235882	3.81988e-11\\
1.13143428285857	3.81988e-11\\
1.13343328335832	3.72893e-11\\
1.13543228385807	3.81988e-11\\
1.13743128435782	3.91083e-11\\
1.13943028485757	3.91083e-11\\
1.14142928535732	4.00178e-11\\
1.14342828585707	3.91083e-11\\
1.14542728635682	3.91083e-11\\
1.14742628685657	4.00178e-11\\
1.14942528735632	4.00178e-11\\
1.15142428785607	3.91083e-11\\
1.15342328835582	4.00178e-11\\
1.15542228885557	4.00178e-11\\
1.15742128935532	4.00178e-11\\
1.15942028985507	4.00178e-11\\
1.16141929035482	4.00178e-11\\
1.16341829085457	4.00178e-11\\
1.16541729135432	4.09273e-11\\
1.16741629185407	4.00178e-11\\
1.16941529235382	4.00178e-11\\
1.17141429285357	4.00178e-11\\
1.17341329335332	4.00178e-11\\
1.17541229385307	3.91083e-11\\
1.17741129435282	4.00178e-11\\
1.17941029485257	4.09273e-11\\
1.18140929535232	4.09273e-11\\
1.18340829585207	4.09273e-11\\
1.18540729635182	4.18368e-11\\
1.18740629685157	4.18368e-11\\
1.18940529735132	4.09273e-11\\
1.19140429785107	4.18368e-11\\
1.19340329835082	4.27463e-11\\
1.19540229885057	4.27463e-11\\
1.19740129935032	4.27463e-11\\
1.19940029985008	4.36557e-11\\
1.20139930034983	4.36557e-11\\
1.20339830084958	4.36557e-11\\
1.20539730134933	4.45652e-11\\
1.20739630184908	4.54747e-11\\
1.20939530234883	4.54747e-11\\
1.21139430284858	4.45652e-11\\
1.21339330334833	4.45652e-11\\
1.21539230384808	4.45652e-11\\
1.21739130434783	4.54747e-11\\
1.21939030484758	4.54747e-11\\
1.22138930534733	4.54747e-11\\
1.22338830584708	4.54747e-11\\
1.22538730634683	4.45652e-11\\
1.22738630684658	4.54747e-11\\
1.22938530734633	4.63842e-11\\
1.23138430784608	4.72937e-11\\
1.23338330834583	4.72937e-11\\
1.23538230884558	4.82032e-11\\
1.23738130934533	4.72937e-11\\
1.23938030984508	4.82032e-11\\
1.24137931034483	4.82032e-11\\
1.24337831084458	4.91127e-11\\
1.24537731134433	5.00222e-11\\
1.24737631184408	5.00222e-11\\
1.24937531234383	4.91127e-11\\
1.25137431284358	4.91127e-11\\
1.25337331334333	5.00222e-11\\
1.25537231384308	5.00222e-11\\
1.25737131434283	4.91127e-11\\
1.25937031484258	5.00222e-11\\
1.26136931534233	5.00222e-11\\
1.26336831584208	5.00222e-11\\
1.26536731634183	5.00222e-11\\
1.26736631684158	5.09317e-11\\
1.26936531734133	5.00222e-11\\
1.27136431784108	5.00222e-11\\
1.27336331834083	5.00222e-11\\
1.27536231884058	5.00222e-11\\
1.27736131934033	5.09317e-11\\
1.27936031984008	5.00222e-11\\
1.28135932033983	5.00222e-11\\
1.28335832083958	5.09317e-11\\
1.28535732133933	5.09317e-11\\
1.28735632183908	5.09317e-11\\
1.28935532233883	5.00222e-11\\
1.29135432283858	5.00222e-11\\
1.29335332333833	4.82032e-11\\
1.29535232383808	4.82032e-11\\
1.29735132433783	4.82032e-11\\
1.29935032483758	4.72937e-11\\
1.30134932533733	4.82032e-11\\
1.30334832583708	4.82032e-11\\
1.30534732633683	4.72937e-11\\
1.30734632683658	4.72937e-11\\
1.30934532733633	4.63842e-11\\
1.31134432783608	4.63842e-11\\
1.31334332833583	4.63842e-11\\
1.31534232883558	4.54747e-11\\
1.31734132933533	4.63842e-11\\
1.31934032983508	4.54747e-11\\
1.32133933033483	4.63842e-11\\
1.32333833083458	4.54747e-11\\
1.32533733133433	4.54747e-11\\
1.32733633183408	4.54747e-11\\
1.32933533233383	4.45652e-11\\
1.33133433283358	4.54747e-11\\
1.33333333333333	4.54747e-11\\
1.33533233383308	4.54747e-11\\
1.33733133433283	4.54747e-11\\
1.33933033483258	4.54747e-11\\
1.34132933533233	4.63842e-11\\
1.34332833583208	4.63842e-11\\
1.34532733633183	4.63842e-11\\
1.34732633683158	4.82032e-11\\
1.34932533733133	4.82032e-11\\
1.35132433783108	4.63842e-11\\
1.35332333833083	4.72937e-11\\
1.35532233883058	4.63842e-11\\
1.35732133933033	4.63842e-11\\
1.35932033983009	4.63842e-11\\
1.36131934032984	4.72937e-11\\
1.36331834082959	4.72937e-11\\
1.36531734132934	4.72937e-11\\
1.36731634182909	4.82032e-11\\
1.36931534232884	4.72937e-11\\
1.37131434282859	4.72937e-11\\
1.37331334332834	4.82032e-11\\
1.37531234382809	4.91127e-11\\
1.37731134432784	4.82032e-11\\
1.37931034482759	4.82032e-11\\
1.38130934532734	4.91127e-11\\
1.38330834582709	4.82032e-11\\
1.38530734632684	4.91127e-11\\
1.38730634682659	4.72937e-11\\
1.38930534732634	4.82032e-11\\
1.39130434782609	4.82032e-11\\
1.39330334832584	4.72937e-11\\
1.39530234882559	4.63842e-11\\
1.39730134932534	4.45652e-11\\
1.39930034982509	4.45652e-11\\
1.40129935032484	4.45652e-11\\
1.40329835082459	4.45652e-11\\
1.40529735132434	4.45652e-11\\
1.40729635182409	4.45652e-11\\
1.40929535232384	4.36557e-11\\
1.41129435282359	4.45652e-11\\
1.41329335332334	4.36557e-11\\
1.41529235382309	4.36557e-11\\
1.41729135432284	4.27463e-11\\
1.41929035482259	4.27463e-11\\
1.42128935532234	4.27463e-11\\
1.42328835582209	4.27463e-11\\
1.42528735632184	4.18368e-11\\
1.42728635682159	4.09273e-11\\
1.42928535732134	4.09273e-11\\
1.43128435782109	4.09273e-11\\
1.43328335832084	4.00178e-11\\
1.43528235882059	3.91083e-11\\
1.43728135932034	4.00178e-11\\
1.43928035982009	4.00178e-11\\
1.44127936031984	4.00178e-11\\
1.44327836081959	4.00178e-11\\
1.44527736131934	4.09273e-11\\
1.44727636181909	4.18368e-11\\
1.44927536231884	4.18368e-11\\
1.45127436281859	4.09273e-11\\
1.45327336331834	4.18368e-11\\
1.45527236381809	4.18368e-11\\
1.45727136431784	4.09273e-11\\
1.45927036481759	4.18368e-11\\
1.46126936531734	4.09273e-11\\
1.46326836581709	4.09273e-11\\
1.46526736631684	4.00178e-11\\
1.46726636681659	4.09273e-11\\
1.46926536731634	4.09273e-11\\
1.47126436781609	4.18368e-11\\
1.47326336831584	4.18368e-11\\
1.47526236881559	4.09273e-11\\
1.47726136931534	4.09273e-11\\
1.47926036981509	4.18368e-11\\
1.48125937031484	4.09273e-11\\
1.48325837081459	4.18368e-11\\
1.48525737131434	4.09273e-11\\
1.48725637181409	4.09273e-11\\
1.48925537231384	4.09273e-11\\
1.49125437281359	4.09273e-11\\
1.49325337331334	4.18368e-11\\
1.49525237381309	4.27463e-11\\
1.49725137431284	4.27463e-11\\
1.49925037481259	4.18368e-11\\
1.50124937531234	4.18368e-11\\
1.50324837581209	4.27463e-11\\
1.50524737631184	4.27463e-11\\
1.50724637681159	4.36557e-11\\
1.50924537731134	4.36557e-11\\
1.51124437781109	4.36557e-11\\
1.51324337831084	4.27463e-11\\
1.51524237881059	4.45652e-11\\
1.51724137931034	4.36557e-11\\
1.51924037981009	4.27463e-11\\
1.52123938030985	4.09273e-11\\
1.5232383808096	4.09273e-11\\
1.52523738130935	4.09273e-11\\
1.5272363818091	4.00178e-11\\
1.52923538230885	3.91083e-11\\
1.5312343828086	3.91083e-11\\
1.53323338330835	3.91083e-11\\
1.5352323838081	3.81988e-11\\
1.53723138430785	3.81988e-11\\
1.5392303848076	3.81988e-11\\
1.54122938530735	3.91083e-11\\
1.5432283858071	3.91083e-11\\
1.54522738630685	3.91083e-11\\
1.5472263868066	3.91083e-11\\
1.54922538730635	3.81988e-11\\
1.5512243878061	3.81988e-11\\
1.55322338830585	3.91083e-11\\
1.5552223888056	3.91083e-11\\
1.55722138930535	3.81988e-11\\
1.5592203898051	3.91083e-11\\
1.56121939030485	3.81988e-11\\
1.5632183908046	3.72893e-11\\
1.56521739130435	3.72893e-11\\
1.5672163918041	3.72893e-11\\
1.56921539230385	3.72893e-11\\
1.5712143928036	3.81988e-11\\
1.57321339330335	3.72893e-11\\
1.5752123938031	3.72893e-11\\
1.57721139430285	3.81988e-11\\
1.5792103948026	3.72893e-11\\
1.58120939530235	3.72893e-11\\
1.5832083958021	3.72893e-11\\
1.58520739630185	3.63798e-11\\
1.5872063968016	3.54703e-11\\
1.58920539730135	3.54703e-11\\
1.5912043978011	3.54703e-11\\
1.59320339830085	3.63798e-11\\
1.5952023988006	3.54703e-11\\
1.59720139930035	3.54703e-11\\
1.5992003998001	3.45608e-11\\
1.60119940029985	3.54703e-11\\
1.6031984007996	3.45608e-11\\
1.60519740129935	3.45608e-11\\
1.6071964017991	3.45608e-11\\
1.60919540229885	3.36513e-11\\
1.6111944027986	3.36513e-11\\
1.61319340329835	3.36513e-11\\
1.6151924037981	3.45608e-11\\
1.61719140429785	3.36513e-11\\
1.6191904047976	3.45608e-11\\
1.62118940529735	3.45608e-11\\
1.6231884057971	3.45608e-11\\
1.62518740629685	3.45608e-11\\
1.6271864067966	3.36513e-11\\
1.62918540729635	3.36513e-11\\
1.6311844077961	3.45608e-11\\
1.63318340829585	3.54703e-11\\
1.6351824087956	3.54703e-11\\
1.63718140929535	3.45608e-11\\
1.6391804097951	3.36513e-11\\
1.64117941029485	3.36513e-11\\
1.6431784107946	3.27418e-11\\
1.64517741129435	3.36513e-11\\
1.6471764117941	3.27418e-11\\
1.64917541229385	3.27418e-11\\
1.6511744127936	3.45608e-11\\
1.65317341329335	3.36513e-11\\
1.6551724137931	3.45608e-11\\
1.65717141429285	3.45608e-11\\
1.6591704147926	3.54703e-11\\
1.66116941529235	3.54703e-11\\
1.6631684157921	3.54703e-11\\
1.66516741629185	3.54703e-11\\
1.6671664167916	3.54703e-11\\
1.66916541729135	3.54703e-11\\
1.6711644177911	3.54703e-11\\
1.67316341829085	3.54703e-11\\
1.6751624187906	3.54703e-11\\
1.67716141929035	3.54703e-11\\
1.6791604197901	3.54703e-11\\
1.68115942028985	3.54703e-11\\
1.68315842078961	3.54703e-11\\
1.68515742128936	3.54703e-11\\
1.68715642178911	3.54703e-11\\
1.68915542228886	3.54703e-11\\
1.69115442278861	3.54703e-11\\
1.69315342328836	3.54703e-11\\
1.69515242378811	3.54703e-11\\
1.69715142428786	3.63798e-11\\
1.69915042478761	3.54703e-11\\
1.70114942528736	3.54703e-11\\
1.70314842578711	3.63798e-11\\
1.70514742628686	3.54703e-11\\
1.70714642678661	3.54703e-11\\
1.70914542728636	3.63798e-11\\
1.71114442778611	3.72893e-11\\
1.71314342828586	3.72893e-11\\
1.71514242878561	3.63798e-11\\
1.71714142928536	3.63798e-11\\
1.71914042978511	3.54703e-11\\
1.72113943028486	3.45608e-11\\
1.72313843078461	3.45608e-11\\
1.72513743128436	3.45608e-11\\
1.72713643178411	3.45608e-11\\
1.72913543228386	3.36513e-11\\
1.73113443278361	3.27418e-11\\
1.73313343328336	3.27418e-11\\
1.73513243378311	3.27418e-11\\
1.73713143428286	3.18323e-11\\
1.73913043478261	3.18323e-11\\
1.74112943528236	3.09228e-11\\
1.74312843578211	3.09228e-11\\
1.74512743628186	3.00133e-11\\
1.74712643678161	3.09228e-11\\
1.74912543728136	3.09228e-11\\
1.75112443778111	3.09228e-11\\
1.75312343828086	3.09228e-11\\
1.75512243878061	3.09228e-11\\
1.75712143928036	3.09228e-11\\
1.75912043978011	3.09228e-11\\
1.76111944027986	3.00133e-11\\
1.76311844077961	3.00133e-11\\
1.76511744127936	2.91038e-11\\
1.76711644177911	3.00133e-11\\
1.76911544227886	3.09228e-11\\
1.77111444277861	3.18323e-11\\
1.77311344327836	3.18323e-11\\
1.77511244377811	3.18323e-11\\
1.77711144427786	3.18323e-11\\
1.77911044477761	3.18323e-11\\
1.78110944527736	3.27418e-11\\
1.78310844577711	3.36513e-11\\
1.78510744627686	3.36513e-11\\
1.78710644677661	3.36513e-11\\
1.78910544727636	3.36513e-11\\
1.79110444777611	3.36513e-11\\
1.79310344827586	3.27418e-11\\
1.79510244877561	3.27418e-11\\
1.79710144927536	3.18323e-11\\
1.79910044977511	3.09228e-11\\
1.80109945027486	3.00133e-11\\
1.80309845077461	3.00133e-11\\
1.80509745127436	3.00133e-11\\
1.80709645177411	2.91038e-11\\
1.80909545227386	2.91038e-11\\
1.81109445277361	3.00133e-11\\
1.81309345327336	3.00133e-11\\
1.81509245377311	2.91038e-11\\
1.81709145427286	2.91038e-11\\
1.81909045477261	2.81943e-11\\
1.82108945527236	2.91038e-11\\
1.82308845577211	2.91038e-11\\
1.82508745627186	2.91038e-11\\
1.82708645677161	3.00133e-11\\
1.82908545727136	2.91038e-11\\
1.83108445777111	2.81943e-11\\
1.83308345827086	2.72848e-11\\
1.83508245877061	2.81943e-11\\
1.83708145927036	2.91038e-11\\
1.83908045977011	2.81943e-11\\
1.84107946026987	2.81943e-11\\
1.84307846076962	2.72848e-11\\
1.84507746126937	2.72848e-11\\
1.84707646176912	2.72848e-11\\
1.84907546226887	2.72848e-11\\
1.85107446276862	2.63753e-11\\
1.85307346326837	2.72848e-11\\
1.85507246376812	2.81943e-11\\
1.85707146426787	2.81943e-11\\
1.85907046476762	2.72848e-11\\
1.86106946526737	2.72848e-11\\
1.86306846576712	2.63753e-11\\
1.86506746626687	2.63753e-11\\
1.86706646676662	2.63753e-11\\
1.86906546726637	2.63753e-11\\
1.87106446776612	2.63753e-11\\
1.87306346826587	2.63753e-11\\
1.87506246876562	2.63753e-11\\
1.87706146926537	2.63753e-11\\
1.87906046976512	2.63753e-11\\
1.88105947026487	2.63753e-11\\
1.88305847076462	2.63753e-11\\
1.88505747126437	2.54659e-11\\
1.88705647176412	2.54659e-11\\
1.88905547226387	2.54659e-11\\
1.89105447276362	2.63753e-11\\
1.89305347326337	2.63753e-11\\
1.89505247376312	2.63753e-11\\
1.89705147426287	2.54659e-11\\
1.89905047476262	2.54659e-11\\
1.90104947526237	2.45564e-11\\
1.90304847576212	2.45564e-11\\
1.90504747626187	2.36469e-11\\
1.90704647676162	2.45564e-11\\
1.90904547726137	2.45564e-11\\
1.91104447776112	2.45564e-11\\
1.91304347826087	2.45564e-11\\
1.91504247876062	2.45564e-11\\
1.91704147926037	2.36469e-11\\
1.91904047976012	2.36469e-11\\
1.92103948025987	2.45564e-11\\
1.92303848075962	2.36469e-11\\
1.92503748125937	2.45564e-11\\
1.92703648175912	2.36469e-11\\
1.92903548225887	2.36469e-11\\
1.93103448275862	2.36469e-11\\
1.93303348325837	2.36469e-11\\
1.93503248375812	2.45564e-11\\
1.93703148425787	2.45564e-11\\
1.93903048475762	2.45564e-11\\
1.94102948525737	2.45564e-11\\
1.94302848575712	2.45564e-11\\
1.94502748625687	2.45564e-11\\
1.94702648675662	2.36469e-11\\
1.94902548725637	2.36469e-11\\
1.95102448775612	2.36469e-11\\
1.95302348825587	2.36469e-11\\
1.95502248875562	2.36469e-11\\
1.95702148925537	2.45564e-11\\
1.95902048975512	2.36469e-11\\
1.96101949025487	2.36469e-11\\
1.96301849075462	2.36469e-11\\
1.96501749125437	2.36469e-11\\
1.96701649175412	2.45564e-11\\
1.96901549225387	2.45564e-11\\
1.97101449275362	2.36469e-11\\
1.97301349325337	2.45564e-11\\
1.97501249375312	2.45564e-11\\
1.97701149425287	2.54659e-11\\
1.97901049475262	2.54659e-11\\
1.98100949525237	2.54659e-11\\
1.98300849575212	2.63753e-11\\
1.98500749625187	2.63753e-11\\
1.98700649675162	2.63753e-11\\
1.98900549725137	2.63753e-11\\
1.99100449775112	2.63753e-11\\
1.99300349825087	2.63753e-11\\
1.99500249875062	2.63753e-11\\
1.99700149925037	2.54659e-11\\
1.99900049975012	2.54659e-11\\
2.00099950024988	2.54659e-11\\
2.00299850074963	2.45564e-11\\
2.00499750124938	2.54659e-11\\
2.00699650174913	2.54659e-11\\
2.00899550224888	2.45564e-11\\
2.01099450274863	2.45564e-11\\
2.01299350324838	2.54659e-11\\
2.01499250374813	2.45564e-11\\
2.01699150424788	2.45564e-11\\
2.01899050474763	2.36469e-11\\
2.02098950524738	2.45564e-11\\
2.02298850574713	2.45564e-11\\
2.02498750624688	2.45564e-11\\
2.02698650674663	2.45564e-11\\
2.02898550724638	2.45564e-11\\
2.03098450774613	2.45564e-11\\
2.03298350824588	2.45564e-11\\
2.03498250874563	2.45564e-11\\
2.03698150924538	2.36469e-11\\
2.03898050974513	2.36469e-11\\
2.04097951024488	2.45564e-11\\
2.04297851074463	2.36469e-11\\
2.04497751124438	2.36469e-11\\
2.04697651174413	2.36469e-11\\
2.04897551224388	2.36469e-11\\
2.05097451274363	2.36469e-11\\
2.05297351324338	2.36469e-11\\
2.05497251374313	2.36469e-11\\
2.05697151424288	2.27374e-11\\
2.05897051474263	2.27374e-11\\
2.06096951524238	2.27374e-11\\
2.06296851574213	2.27374e-11\\
2.06496751624188	2.27374e-11\\
2.06696651674163	2.18279e-11\\
2.06896551724138	2.18279e-11\\
2.07096451774113	2.18279e-11\\
2.07296351824088	2.09184e-11\\
2.07496251874063	2.09184e-11\\
2.07696151924038	2.09184e-11\\
2.07896051974013	2.09184e-11\\
2.08095952023988	2.09184e-11\\
2.08295852073963	2.18279e-11\\
2.08495752123938	2.18279e-11\\
2.08695652173913	2.18279e-11\\
2.08895552223888	2.09184e-11\\
2.09095452273863	2.09184e-11\\
2.09295352323838	2.18279e-11\\
2.09495252373813	2.09184e-11\\
2.09695152423788	2.18279e-11\\
2.09895052473763	2.09184e-11\\
2.10094952523738	2.09184e-11\\
2.10294852573713	2.09184e-11\\
2.10494752623688	2.09184e-11\\
2.10694652673663	2.18279e-11\\
2.10894552723638	2.09184e-11\\
2.11094452773613	2.09184e-11\\
2.11294352823588	2.09184e-11\\
2.11494252873563	2.09184e-11\\
2.11694152923538	2.09184e-11\\
2.11894052973513	2.09184e-11\\
2.12093953023488	2.09184e-11\\
2.12293853073463	2.09184e-11\\
2.12493753123438	2.00089e-11\\
2.12693653173413	2.00089e-11\\
2.12893553223388	2.00089e-11\\
2.13093453273363	2.00089e-11\\
2.13293353323338	2.00089e-11\\
2.13493253373313	2.00089e-11\\
2.13693153423288	2.00089e-11\\
2.13893053473263	2.00089e-11\\
2.14092953523238	2.00089e-11\\
2.14292853573213	2.00089e-11\\
2.14492753623188	2.00089e-11\\
2.14692653673163	2.00089e-11\\
2.14892553723138	2.00089e-11\\
2.15092453773113	2.09184e-11\\
2.15292353823088	2.09184e-11\\
2.15492253873063	2.09184e-11\\
2.15692153923038	2.09184e-11\\
2.15892053973013	2.09184e-11\\
2.16091954022989	2.09184e-11\\
2.16291854072964	2.18279e-11\\
2.16491754122939	2.09184e-11\\
2.16691654172914	2.09184e-11\\
2.16891554222889	2.09184e-11\\
2.17091454272864	2.09184e-11\\
2.17291354322839	2.09184e-11\\
2.17491254372814	2.09184e-11\\
2.17691154422789	2.09184e-11\\
2.17891054472764	2.09184e-11\\
2.18090954522739	2.09184e-11\\
2.18290854572714	2.00089e-11\\
2.18490754622689	2.00089e-11\\
2.18690654672664	2.00089e-11\\
2.18890554722639	2.09184e-11\\
2.19090454772614	2.09184e-11\\
2.19290354822589	2.00089e-11\\
2.19490254872564	2.00089e-11\\
2.19690154922539	2.00089e-11\\
2.19890054972514	2.09184e-11\\
2.20089955022489	2.00089e-11\\
2.20289855072464	2.00089e-11\\
2.20489755122439	2.00089e-11\\
2.20689655172414	2.00089e-11\\
2.20889555222389	2.00089e-11\\
2.21089455272364	2.00089e-11\\
2.21289355322339	2.09184e-11\\
2.21489255372314	2.00089e-11\\
2.21689155422289	2.00089e-11\\
2.21889055472264	2.00089e-11\\
2.22088955522239	2.00089e-11\\
2.22288855572214	2.00089e-11\\
2.22488755622189	2.00089e-11\\
2.22688655672164	2.00089e-11\\
2.22888555722139	2.00089e-11\\
2.23088455772114	2.00089e-11\\
2.23288355822089	2.00089e-11\\
2.23488255872064	2.00089e-11\\
2.23688155922039	2.00089e-11\\
2.23888055972014	2.09184e-11\\
2.24087956021989	2.00089e-11\\
2.24287856071964	2.00089e-11\\
2.24487756121939	2.00089e-11\\
2.24687656171914	2.00089e-11\\
2.24887556221889	2.00089e-11\\
2.25087456271864	2.09184e-11\\
2.25287356321839	2.09184e-11\\
2.25487256371814	2.00089e-11\\
2.25687156421789	2.00089e-11\\
2.25887056471764	2.00089e-11\\
2.26086956521739	2.00089e-11\\
2.26286856571714	2.09184e-11\\
2.26486756621689	2.09184e-11\\
2.26686656671664	2.09184e-11\\
2.26886556721639	2.09184e-11\\
2.27086456771614	2.09184e-11\\
2.27286356821589	2.18279e-11\\
2.27486256871564	2.18279e-11\\
2.27686156921539	2.18279e-11\\
2.27886056971514	2.18279e-11\\
2.28085957021489	2.18279e-11\\
2.28285857071464	2.18279e-11\\
2.28485757121439	2.18279e-11\\
2.28685657171414	2.09184e-11\\
2.28885557221389	2.09184e-11\\
2.29085457271364	2.18279e-11\\
2.29285357321339	2.18279e-11\\
2.29485257371314	2.18279e-11\\
2.29685157421289	2.27374e-11\\
2.29885057471264	2.18279e-11\\
2.30084957521239	2.27374e-11\\
2.30284857571214	2.27374e-11\\
2.30484757621189	2.27374e-11\\
2.30684657671164	2.27374e-11\\
2.30884557721139	2.27374e-11\\
2.31084457771114	2.27374e-11\\
2.31284357821089	2.27374e-11\\
2.31484257871064	2.27374e-11\\
2.31684157921039	2.36469e-11\\
2.31884057971015	2.27374e-11\\
2.3208395802099	2.36469e-11\\
2.32283858070965	2.36469e-11\\
2.3248375812094	2.36469e-11\\
2.32683658170915	2.36469e-11\\
2.3288355822089	2.27374e-11\\
2.33083458270865	2.27374e-11\\
2.3328335832084	2.36469e-11\\
2.33483258370815	2.36469e-11\\
2.3368315842079	2.36469e-11\\
2.33883058470765	2.36469e-11\\
2.3408295852074	2.36469e-11\\
2.34282858570715	2.36469e-11\\
2.3448275862069	2.27374e-11\\
2.34682658670665	2.27374e-11\\
2.3488255872064	2.27374e-11\\
2.35082458770615	2.27374e-11\\
2.3528235882059	2.27374e-11\\
2.35482258870565	2.27374e-11\\
2.3568215892054	2.27374e-11\\
2.35882058970515	2.27374e-11\\
2.3608195902049	2.27374e-11\\
2.36281859070465	2.18279e-11\\
2.3648175912044	2.18279e-11\\
2.36681659170415	2.27374e-11\\
2.3688155922039	2.27374e-11\\
2.37081459270365	2.27374e-11\\
2.3728135932034	2.27374e-11\\
2.37481259370315	2.27374e-11\\
2.3768115942029	2.27374e-11\\
2.37881059470265	2.27374e-11\\
2.3808095952024	2.27374e-11\\
2.38280859570215	2.27374e-11\\
2.3848075962019	2.18279e-11\\
2.38680659670165	2.18279e-11\\
2.3888055972014	2.18279e-11\\
2.39080459770115	2.18279e-11\\
2.3928035982009	2.18279e-11\\
2.39480259870065	2.18279e-11\\
2.3968015992004	2.09184e-11\\
2.39880059970015	2.09184e-11\\
2.4007996001999	2.09184e-11\\
2.40279860069965	2.18279e-11\\
2.4047976011994	2.18279e-11\\
2.40679660169915	2.18279e-11\\
2.4087956021989	2.09184e-11\\
2.41079460269865	2.09184e-11\\
2.4127936031984	2.09184e-11\\
2.41479260369815	2.09184e-11\\
2.4167916041979	2.09184e-11\\
2.41879060469765	2.09184e-11\\
2.4207896051974	2.09184e-11\\
2.42278860569715	2.09184e-11\\
2.4247876061969	2.09184e-11\\
2.42678660669665	2.09184e-11\\
2.4287856071964	2.18279e-11\\
2.43078460769615	2.18279e-11\\
2.4327836081959	2.18279e-11\\
2.43478260869565	2.18279e-11\\
2.4367816091954	2.18279e-11\\
2.43878060969515	2.18279e-11\\
2.4407796101949	2.18279e-11\\
2.44277861069465	2.27374e-11\\
2.4447776111944	2.27374e-11\\
2.44677661169415	2.27374e-11\\
2.4487756121939	2.36469e-11\\
2.45077461269365	2.27374e-11\\
2.4527736131934	2.27374e-11\\
2.45477261369315	2.27374e-11\\
2.4567716141929	2.27374e-11\\
2.45877061469265	2.27374e-11\\
2.4607696151924	2.27374e-11\\
2.46276861569215	2.36469e-11\\
2.4647676161919	2.36469e-11\\
2.46676661669165	2.36469e-11\\
2.4687656171914	2.27374e-11\\
2.47076461769115	2.27374e-11\\
2.4727636181909	2.27374e-11\\
2.47476261869065	2.27374e-11\\
2.4767616191904	2.27374e-11\\
2.47876061969016	2.27374e-11\\
2.48075962018991	2.36469e-11\\
2.48275862068966	2.36469e-11\\
2.48475762118941	2.36469e-11\\
2.48675662168916	2.36469e-11\\
2.48875562218891	2.36469e-11\\
2.49075462268866	2.36469e-11\\
2.49275362318841	2.45564e-11\\
2.49475262368816	2.45564e-11\\
2.49675162418791	2.45564e-11\\
2.49875062468766	2.45564e-11\\
2.50074962518741	2.45564e-11\\
2.50274862568716	2.45564e-11\\
2.50474762618691	2.45564e-11\\
2.50674662668666	2.45564e-11\\
2.50874562718641	2.54659e-11\\
2.51074462768616	2.45564e-11\\
2.51274362818591	2.45564e-11\\
2.51474262868566	2.45564e-11\\
2.51674162918541	2.45564e-11\\
2.51874062968516	2.45564e-11\\
2.52073963018491	2.45564e-11\\
2.52273863068466	2.45564e-11\\
2.52473763118441	2.45564e-11\\
2.52673663168416	2.45564e-11\\
2.52873563218391	2.54659e-11\\
2.53073463268366	2.54659e-11\\
2.53273363318341	2.54659e-11\\
2.53473263368316	2.54659e-11\\
2.53673163418291	2.54659e-11\\
2.53873063468266	2.54659e-11\\
2.54072963518241	2.45564e-11\\
2.54272863568216	2.54659e-11\\
2.54472763618191	2.45564e-11\\
2.54672663668166	2.45564e-11\\
2.54872563718141	2.45564e-11\\
2.55072463768116	2.45564e-11\\
2.55272363818091	2.45564e-11\\
2.55472263868066	2.45564e-11\\
2.55672163918041	2.45564e-11\\
2.55872063968016	2.45564e-11\\
2.56071964017991	2.36469e-11\\
2.56271864067966	2.45564e-11\\
2.56471764117941	2.36469e-11\\
2.56671664167916	2.36469e-11\\
2.56871564217891	2.27374e-11\\
2.57071464267866	2.36469e-11\\
2.57271364317841	2.36469e-11\\
2.57471264367816	2.36469e-11\\
2.57671164417791	2.36469e-11\\
2.57871064467766	2.27374e-11\\
2.58070964517741	2.27374e-11\\
2.58270864567716	2.27374e-11\\
2.58470764617691	2.36469e-11\\
2.58670664667666	2.36469e-11\\
2.58870564717641	2.36469e-11\\
2.59070464767616	2.36469e-11\\
2.59270364817591	2.36469e-11\\
2.59470264867566	2.36469e-11\\
2.59670164917541	2.27374e-11\\
2.59870064967516	2.27374e-11\\
2.60069965017491	2.27374e-11\\
2.60269865067466	2.27374e-11\\
2.60469765117441	2.27374e-11\\
2.60669665167416	2.36469e-11\\
2.60869565217391	2.36469e-11\\
2.61069465267366	2.27374e-11\\
2.61269365317341	2.27374e-11\\
2.61469265367316	2.27374e-11\\
2.61669165417291	2.27374e-11\\
2.61869065467266	2.36469e-11\\
2.62068965517241	2.36469e-11\\
2.62268865567216	2.36469e-11\\
2.62468765617191	2.36469e-11\\
2.62668665667166	2.36469e-11\\
2.62868565717141	2.36469e-11\\
2.63068465767116	2.36469e-11\\
2.63268365817091	2.36469e-11\\
2.63468265867066	2.45564e-11\\
2.63668165917041	2.45564e-11\\
2.63868065967017	2.45564e-11\\
2.64067966016992	2.36469e-11\\
2.64267866066967	2.36469e-11\\
2.64467766116942	2.45564e-11\\
2.64667666166917	2.36469e-11\\
2.64867566216892	2.36469e-11\\
2.65067466266867	2.36469e-11\\
2.65267366316842	2.36469e-11\\
2.65467266366817	2.36469e-11\\
2.65667166416792	2.36469e-11\\
2.65867066466767	2.36469e-11\\
2.66066966516742	2.36469e-11\\
2.66266866566717	2.36469e-11\\
2.66466766616692	2.45564e-11\\
2.66666666666667	2.45564e-11\\
2.66866566716642	2.45564e-11\\
2.67066466766617	2.45564e-11\\
2.67266366816592	2.45564e-11\\
2.67466266866567	2.45564e-11\\
2.67666166916542	2.36469e-11\\
2.67866066966517	2.45564e-11\\
2.68065967016492	2.45564e-11\\
2.68265867066467	2.54659e-11\\
2.68465767116442	2.54659e-11\\
2.68665667166417	2.54659e-11\\
2.68865567216392	2.45564e-11\\
2.69065467266367	2.45564e-11\\
2.69265367316342	2.45564e-11\\
2.69465267366317	2.45564e-11\\
2.69665167416292	2.36469e-11\\
2.69865067466267	2.45564e-11\\
2.70064967516242	2.54659e-11\\
2.70264867566217	2.54659e-11\\
2.70464767616192	2.63753e-11\\
2.70664667666167	2.63753e-11\\
2.70864567716142	2.63753e-11\\
2.71064467766117	2.63753e-11\\
2.71264367816092	2.72848e-11\\
2.71464267866067	2.63753e-11\\
2.71664167916042	2.63753e-11\\
2.71864067966017	2.63753e-11\\
2.72063968015992	2.54659e-11\\
2.72263868065967	2.63753e-11\\
2.72463768115942	2.63753e-11\\
2.72663668165917	2.63753e-11\\
2.72863568215892	2.54659e-11\\
2.73063468265867	2.54659e-11\\
2.73263368315842	2.54659e-11\\
2.73463268365817	2.54659e-11\\
2.73663168415792	2.45564e-11\\
2.73863068465767	2.45564e-11\\
2.74062968515742	2.54659e-11\\
2.74262868565717	2.63753e-11\\
2.74462768615692	2.54659e-11\\
2.74662668665667	2.45564e-11\\
2.74862568715642	2.45564e-11\\
2.75062468765617	2.45564e-11\\
2.75262368815592	2.45564e-11\\
2.75462268865567	2.45564e-11\\
2.75662168915542	2.36469e-11\\
2.75862068965517	2.27374e-11\\
2.76061969015492	2.18279e-11\\
2.76261869065467	2.27374e-11\\
2.76461769115442	2.27374e-11\\
2.76661669165417	2.27374e-11\\
2.76861569215392	2.18279e-11\\
2.77061469265367	2.18279e-11\\
2.77261369315342	2.18279e-11\\
2.77461269365317	2.18279e-11\\
2.77661169415292	2.18279e-11\\
2.77861069465267	2.09184e-11\\
2.78060969515242	2.09184e-11\\
2.78260869565217	2.18279e-11\\
2.78460769615192	2.18279e-11\\
2.78660669665167	2.18279e-11\\
2.78860569715142	2.18279e-11\\
2.79060469765117	2.18279e-11\\
2.79260369815092	2.09184e-11\\
2.79460269865067	2.09184e-11\\
2.79660169915043	2.09184e-11\\
2.79860069965018	2.09184e-11\\
2.80059970014993	2.09184e-11\\
2.80259870064968	2.00089e-11\\
2.80459770114943	2.00089e-11\\
2.80659670164918	1.90994e-11\\
2.80859570214893	1.90994e-11\\
2.81059470264868	2.00089e-11\\
2.81259370314843	2.00089e-11\\
2.81459270364818	2.00089e-11\\
2.81659170414793	2.00089e-11\\
2.81859070464768	2.00089e-11\\
2.82058970514743	1.90994e-11\\
2.82258870564718	2.00089e-11\\
2.82458770614693	2.00089e-11\\
2.82658670664668	2.00089e-11\\
2.82858570714643	2.00089e-11\\
2.83058470764618	2.00089e-11\\
2.83258370814593	2.00089e-11\\
2.83458270864568	2.00089e-11\\
2.83658170914543	2.09184e-11\\
2.83858070964518	2.09184e-11\\
2.84057971014493	2.09184e-11\\
2.84257871064468	2.00089e-11\\
2.84457771114443	2.00089e-11\\
2.84657671164418	2.00089e-11\\
2.84857571214393	2.00089e-11\\
2.85057471264368	2.09184e-11\\
2.85257371314343	2.00089e-11\\
2.85457271364318	2.00089e-11\\
2.85657171414293	2.00089e-11\\
2.85857071464268	2.00089e-11\\
2.86056971514243	2.00089e-11\\
2.86256871564218	2.00089e-11\\
2.86456771614193	2.00089e-11\\
2.86656671664168	2.00089e-11\\
2.86856571714143	2.00089e-11\\
2.87056471764118	2.00089e-11\\
2.87256371814093	2.00089e-11\\
2.87456271864068	2.00089e-11\\
2.87656171914043	1.90994e-11\\
2.87856071964018	1.90994e-11\\
2.88055972013993	2.00089e-11\\
2.88255872063968	1.90994e-11\\
2.88455772113943	2.00089e-11\\
2.88655672163918	2.00089e-11\\
2.88855572213893	2.00089e-11\\
2.89055472263868	1.90994e-11\\
2.89255372313843	1.81899e-11\\
2.89455272363818	1.81899e-11\\
2.89655172413793	1.81899e-11\\
2.89855072463768	1.90994e-11\\
2.90054972513743	1.81899e-11\\
2.90254872563718	1.72804e-11\\
2.90454772613693	1.81899e-11\\
2.90654672663668	1.72804e-11\\
2.90854572713643	1.72804e-11\\
2.91054472763618	1.72804e-11\\
2.91254372813593	1.72804e-11\\
2.91454272863568	1.72804e-11\\
2.91654172913543	1.81899e-11\\
2.91854072963518	1.90994e-11\\
2.92053973013493	2.00089e-11\\
2.92253873063468	1.90994e-11\\
2.92453773113443	2.00089e-11\\
2.92653673163418	2.00089e-11\\
2.92853573213393	1.90994e-11\\
2.93053473263368	2.00089e-11\\
2.93253373313343	2.00089e-11\\
2.93453273363318	1.90994e-11\\
2.93653173413293	2.00089e-11\\
2.93853073463268	2.00089e-11\\
2.94052973513243	1.90994e-11\\
2.94252873563218	2.00089e-11\\
2.94452773613193	2.00089e-11\\
2.94652673663168	2.00089e-11\\
2.94852573713143	2.00089e-11\\
2.95052473763118	2.09184e-11\\
2.95252373813093	2.18279e-11\\
2.95452273863068	2.00089e-11\\
2.95652173913043	2.09184e-11\\
2.95852073963018	2.09184e-11\\
2.96051974012994	2.09184e-11\\
2.96251874062969	2.18279e-11\\
2.96451774112944	2.09184e-11\\
2.96651674162919	2.09184e-11\\
2.96851574212894	2.09184e-11\\
2.97051474262869	2.09184e-11\\
2.97251374312844	2.09184e-11\\
2.97451274362819	2.09184e-11\\
2.97651174412794	2.09184e-11\\
2.97851074462769	2.09184e-11\\
2.98050974512744	2.09184e-11\\
2.98250874562719	2.18279e-11\\
2.98450774612694	2.27374e-11\\
2.98650674662669	2.36469e-11\\
2.98850574712644	2.36469e-11\\
2.99050474762619	2.36469e-11\\
2.99250374812594	2.45564e-11\\
2.99450274862569	2.36469e-11\\
2.99650174912544	2.36469e-11\\
2.99850074962519	2.36469e-11\\
3.00049975012494	2.36469e-11\\
3.00249875062469	2.54659e-11\\
3.00449775112444	2.54659e-11\\
3.00649675162419	2.63753e-11\\
3.00849575212394	2.54659e-11\\
3.01049475262369	2.54659e-11\\
3.01249375312344	2.63753e-11\\
3.01449275362319	2.54659e-11\\
3.01649175412294	2.63753e-11\\
3.01849075462269	2.54659e-11\\
3.02048975512244	2.54659e-11\\
3.02248875562219	2.54659e-11\\
3.02448775612194	2.54659e-11\\
3.02648675662169	2.63753e-11\\
3.02848575712144	2.63753e-11\\
3.03048475762119	2.54659e-11\\
3.03248375812094	2.63753e-11\\
3.03448275862069	2.63753e-11\\
3.03648175912044	2.72848e-11\\
3.03848075962019	2.63753e-11\\
3.04047976011994	2.72848e-11\\
3.04247876061969	2.72848e-11\\
3.04447776111944	2.63753e-11\\
3.04647676161919	2.72848e-11\\
3.04847576211894	2.63753e-11\\
3.05047476261869	2.63753e-11\\
3.05247376311844	2.63753e-11\\
3.05447276361819	2.54659e-11\\
3.05647176411794	2.54659e-11\\
3.05847076461769	2.36469e-11\\
3.06046976511744	2.36469e-11\\
3.06246876561719	2.54659e-11\\
3.06446776611694	2.36469e-11\\
3.06646676661669	2.54659e-11\\
3.06846576711644	2.54659e-11\\
3.07046476761619	2.54659e-11\\
3.07246376811594	2.45564e-11\\
3.07446276861569	2.54659e-11\\
3.07646176911544	2.45564e-11\\
3.07846076961519	2.36469e-11\\
3.08045977011494	2.27374e-11\\
3.08245877061469	2.36469e-11\\
3.08445777111444	2.36469e-11\\
3.08645677161419	2.45564e-11\\
3.08845577211394	2.45564e-11\\
3.09045477261369	2.36469e-11\\
3.09245377311344	2.36469e-11\\
3.09445277361319	2.45564e-11\\
3.09645177411294	2.36469e-11\\
3.09845077461269	2.45564e-11\\
3.10044977511244	2.45564e-11\\
3.10244877561219	2.27374e-11\\
3.10444777611194	2.36469e-11\\
3.10644677661169	2.36469e-11\\
3.10844577711144	2.54659e-11\\
3.11044477761119	2.63753e-11\\
3.11244377811094	2.63753e-11\\
3.11444277861069	2.72848e-11\\
3.11644177911044	2.81943e-11\\
3.11844077961019	2.91038e-11\\
3.12043978010994	2.81943e-11\\
3.12243878060969	2.81943e-11\\
3.12443778110945	2.81943e-11\\
3.1264367816092	2.81943e-11\\
3.12843578210895	2.91038e-11\\
3.1304347826087	2.91038e-11\\
3.13243378310845	3.00133e-11\\
3.1344327836082	3.00133e-11\\
3.13643178410795	3.00133e-11\\
3.1384307846077	2.91038e-11\\
3.14042978510745	2.72848e-11\\
3.1424287856072	2.72848e-11\\
3.14442778610695	2.81943e-11\\
3.1464267866067	2.72848e-11\\
3.14842578710645	2.63753e-11\\
3.1504247876062	2.72848e-11\\
3.15242378810595	2.63753e-11\\
3.1544227886057	2.63753e-11\\
3.15642178910545	2.45564e-11\\
3.1584207896052	2.36469e-11\\
3.16041979010495	2.27374e-11\\
3.1624187906047	2.18279e-11\\
3.16441779110445	2.27374e-11\\
3.1664167916042	2.27374e-11\\
3.16841579210395	2.27374e-11\\
3.1704147926037	2.27374e-11\\
3.17241379310345	2.36469e-11\\
3.1744127936032	2.36469e-11\\
3.17641179410295	2.36469e-11\\
3.1784107946027	2.45564e-11\\
3.18040979510245	2.45564e-11\\
3.1824087956022	2.36469e-11\\
3.18440779610195	2.36469e-11\\
3.1864067966017	2.45564e-11\\
3.18840579710145	2.63753e-11\\
3.1904047976012	2.72848e-11\\
3.19240379810095	2.54659e-11\\
3.1944027986007	2.54659e-11\\
3.19640179910045	2.54659e-11\\
3.1984007996002	2.45564e-11\\
3.20039980009995	2.45564e-11\\
3.2023988005997	2.36469e-11\\
3.20439780109945	2.36469e-11\\
3.2063968015992	2.45564e-11\\
3.20839580209895	2.54659e-11\\
3.2103948025987	2.63753e-11\\
3.21239380309845	2.63753e-11\\
3.2143928035982	2.63753e-11\\
3.21639180409795	2.72848e-11\\
3.2183908045977	2.72848e-11\\
3.22038980509745	2.81943e-11\\
3.2223888055972	2.81943e-11\\
3.22438780609695	3.00133e-11\\
3.2263868065967	3.00133e-11\\
3.22838580709645	3.09228e-11\\
3.2303848075962	3.09228e-11\\
3.23238380809595	3.09228e-11\\
3.2343828085957	3.18323e-11\\
3.23638180909545	3.09228e-11\\
3.2383808095952	3.09228e-11\\
3.24037981009495	3.00133e-11\\
3.2423788105947	3.09228e-11\\
3.24437781109445	3.09228e-11\\
3.2463768115942	3.09228e-11\\
3.24837581209395	3.18323e-11\\
3.2503748125937	3.18323e-11\\
3.25237381309345	3.27418e-11\\
3.2543728135932	3.18323e-11\\
3.25637181409295	3.27418e-11\\
3.2583708145927	3.27418e-11\\
3.26036981509245	3.27418e-11\\
3.2623688155922	3.27418e-11\\
3.26436781609195	3.45608e-11\\
3.2663668165917	3.36513e-11\\
3.26836581709145	3.36513e-11\\
3.2703648175912	3.45608e-11\\
3.27236381809095	3.45608e-11\\
3.2743628185907	3.54703e-11\\
3.27636181909045	3.45608e-11\\
3.2783608195902	3.54703e-11\\
3.28035982008995	3.54703e-11\\
3.2823588205897	3.45608e-11\\
3.28435782108946	3.54703e-11\\
3.28635682158921	3.45608e-11\\
3.28835582208896	3.54703e-11\\
3.29035482258871	3.54703e-11\\
3.29235382308846	3.45608e-11\\
3.29435282358821	3.45608e-11\\
3.29635182408796	3.45608e-11\\
3.29835082458771	3.45608e-11\\
3.30034982508746	3.54703e-11\\
3.30234882558721	3.54703e-11\\
3.30434782608696	3.54703e-11\\
3.30634682658671	3.45608e-11\\
3.30834582708646	3.45608e-11\\
3.31034482758621	3.45608e-11\\
3.31234382808596	3.45608e-11\\
3.31434282858571	3.54703e-11\\
3.31634182908546	3.54703e-11\\
3.31834082958521	3.54703e-11\\
3.32033983008496	3.54703e-11\\
3.32233883058471	3.54703e-11\\
3.32433783108446	3.54703e-11\\
3.32633683158421	3.45608e-11\\
3.32833583208396	3.45608e-11\\
3.33033483258371	3.36513e-11\\
3.33233383308346	3.36513e-11\\
3.33433283358321	3.45608e-11\\
3.33633183408296	3.36513e-11\\
3.33833083458271	3.36513e-11\\
3.34032983508246	3.27418e-11\\
3.34232883558221	3.27418e-11\\
3.34432783608196	3.36513e-11\\
3.34632683658171	3.36513e-11\\
3.34832583708146	3.36513e-11\\
3.35032483758121	3.54703e-11\\
3.35232383808096	3.54703e-11\\
3.35432283858071	3.54703e-11\\
3.35632183908046	3.54703e-11\\
3.35832083958021	3.54703e-11\\
3.36031984007996	3.54703e-11\\
3.36231884057971	3.45608e-11\\
3.36431784107946	3.45608e-11\\
3.36631684157921	3.45608e-11\\
3.36831584207896	3.54703e-11\\
3.37031484257871	3.54703e-11\\
3.37231384307846	3.54703e-11\\
3.37431284357821	3.54703e-11\\
3.37631184407796	3.54703e-11\\
3.37831084457771	3.45608e-11\\
3.38030984507746	3.54703e-11\\
3.38230884557721	3.54703e-11\\
3.38430784607696	3.54703e-11\\
3.38630684657671	3.54703e-11\\
3.38830584707646	3.54703e-11\\
3.39030484757621	3.54703e-11\\
3.39230384807596	3.63798e-11\\
3.39430284857571	3.63798e-11\\
3.39630184907546	3.72893e-11\\
3.39830084957521	3.63798e-11\\
3.40029985007496	3.54703e-11\\
3.40229885057471	3.63798e-11\\
3.40429785107446	3.63798e-11\\
3.40629685157421	3.63798e-11\\
3.40829585207396	3.54703e-11\\
3.41029485257371	3.54703e-11\\
3.41229385307346	3.54703e-11\\
3.41429285357321	3.54703e-11\\
3.41629185407296	3.63798e-11\\
3.41829085457271	3.63798e-11\\
3.42028985507246	3.54703e-11\\
3.42228885557221	3.63798e-11\\
3.42428785607196	3.63798e-11\\
3.42628685657171	3.72893e-11\\
3.42828585707146	3.72893e-11\\
3.43028485757121	3.72893e-11\\
3.43228385807096	3.81988e-11\\
3.43428285857071	3.81988e-11\\
3.43628185907046	3.81988e-11\\
3.43828085957021	3.91083e-11\\
3.44027986006996	3.91083e-11\\
3.44227886056971	4.00178e-11\\
3.44427786106947	3.91083e-11\\
3.44627686156922	3.91083e-11\\
3.44827586206897	3.72893e-11\\
3.45027486256872	3.72893e-11\\
3.45227386306847	3.72893e-11\\
3.45427286356822	3.63798e-11\\
3.45627186406797	3.63798e-11\\
3.45827086456772	3.54703e-11\\
3.46026986506747	3.63798e-11\\
3.46226886556722	3.63798e-11\\
3.46426786606697	3.63798e-11\\
3.46626686656672	3.72893e-11\\
3.46826586706647	3.81988e-11\\
3.47026486756622	3.81988e-11\\
3.47226386806597	3.81988e-11\\
3.47426286856572	3.81988e-11\\
3.47626186906547	3.72893e-11\\
3.47826086956522	3.81988e-11\\
3.48025987006497	3.91083e-11\\
3.48225887056472	3.81988e-11\\
3.48425787106447	3.91083e-11\\
3.48625687156422	3.91083e-11\\
3.48825587206397	4.00178e-11\\
3.49025487256372	3.91083e-11\\
3.49225387306347	3.81988e-11\\
3.49425287356322	3.72893e-11\\
3.49625187406297	3.72893e-11\\
3.49825087456272	3.81988e-11\\
3.50024987506247	3.81988e-11\\
3.50224887556222	3.81988e-11\\
3.50424787606197	3.91083e-11\\
3.50624687656172	3.81988e-11\\
3.50824587706147	3.81988e-11\\
3.51024487756122	3.81988e-11\\
3.51224387806097	3.81988e-11\\
3.51424287856072	3.91083e-11\\
3.51624187906047	3.91083e-11\\
3.51824087956022	3.91083e-11\\
3.52023988005997	3.91083e-11\\
3.52223888055972	3.91083e-11\\
3.52423788105947	3.91083e-11\\
3.52623688155922	4.00178e-11\\
3.52823588205897	3.91083e-11\\
3.53023488255872	3.91083e-11\\
3.53223388305847	3.81988e-11\\
3.53423288355822	3.81988e-11\\
3.53623188405797	3.81988e-11\\
3.53823088455772	3.81988e-11\\
3.54022988505747	3.81988e-11\\
3.54222888555722	3.91083e-11\\
3.54422788605697	3.91083e-11\\
3.54622688655672	3.91083e-11\\
3.54822588705647	3.91083e-11\\
3.55022488755622	3.81988e-11\\
3.55222388805597	3.81988e-11\\
3.55422288855572	3.72893e-11\\
3.55622188905547	3.81988e-11\\
3.55822088955522	3.81988e-11\\
3.56021989005497	3.81988e-11\\
3.56221889055472	3.91083e-11\\
3.56421789105447	3.91083e-11\\
3.56621689155422	3.91083e-11\\
3.56821589205397	3.91083e-11\\
3.57021489255372	3.91083e-11\\
3.57221389305347	3.91083e-11\\
3.57421289355322	3.91083e-11\\
3.57621189405297	3.91083e-11\\
3.57821089455272	3.91083e-11\\
3.58020989505247	3.91083e-11\\
3.58220889555222	4.00178e-11\\
3.58420789605197	3.81988e-11\\
3.58620689655172	3.91083e-11\\
3.58820589705147	3.91083e-11\\
3.59020489755122	3.91083e-11\\
3.59220389805097	3.91083e-11\\
3.59420289855072	4.00178e-11\\
3.59620189905047	4.00178e-11\\
3.59820089955022	3.91083e-11\\
3.60019990004997	3.91083e-11\\
3.60219890054973	4.00178e-11\\
3.60419790104948	4.09273e-11\\
3.60619690154923	4.00178e-11\\
3.60819590204898	4.00178e-11\\
3.61019490254873	4.00178e-11\\
3.61219390304848	4.09273e-11\\
3.61419290354823	4.00178e-11\\
3.61619190404798	4.00178e-11\\
3.61819090454773	4.00178e-11\\
3.62018990504748	4.00178e-11\\
3.62218890554723	4.09273e-11\\
3.62418790604698	4.09273e-11\\
3.62618690654673	4.09273e-11\\
3.62818590704648	4.09273e-11\\
3.63018490754623	4.09273e-11\\
3.63218390804598	4.09273e-11\\
3.63418290854573	4.09273e-11\\
3.63618190904548	4.09273e-11\\
3.63818090954523	4.09273e-11\\
3.64017991004498	4.09273e-11\\
3.64217891054473	4.09273e-11\\
3.64417791104448	4.09273e-11\\
3.64617691154423	4.09273e-11\\
3.64817591204398	4.09273e-11\\
3.65017491254373	4.09273e-11\\
3.65217391304348	4.09273e-11\\
3.65417291354323	4.18368e-11\\
3.65617191404298	4.18368e-11\\
3.65817091454273	4.18368e-11\\
3.66016991504248	4.18368e-11\\
3.66216891554223	4.18368e-11\\
3.66416791604198	4.18368e-11\\
3.66616691654173	4.18368e-11\\
3.66816591704148	4.09273e-11\\
3.67016491754123	4.09273e-11\\
3.67216391804098	4.09273e-11\\
3.67416291854073	4.09273e-11\\
3.67616191904048	4.18368e-11\\
3.67816091954023	4.09273e-11\\
3.68015992003998	4.18368e-11\\
3.68215892053973	4.18368e-11\\
3.68415792103948	4.27463e-11\\
3.68615692153923	4.27463e-11\\
3.68815592203898	4.27463e-11\\
3.69015492253873	4.27463e-11\\
3.69215392303848	4.36557e-11\\
3.69415292353823	4.27463e-11\\
3.69615192403798	4.27463e-11\\
3.69815092453773	4.27463e-11\\
3.70014992503748	4.36557e-11\\
3.70214892553723	4.27463e-11\\
3.70414792603698	4.27463e-11\\
3.70614692653673	4.18368e-11\\
3.70814592703648	4.18368e-11\\
3.71014492753623	4.27463e-11\\
3.71214392803598	4.27463e-11\\
3.71414292853573	4.27463e-11\\
3.71614192903548	4.27463e-11\\
3.71814092953523	4.27463e-11\\
3.72013993003498	4.27463e-11\\
3.72213893053473	4.27463e-11\\
3.72413793103448	4.27463e-11\\
3.72613693153423	4.27463e-11\\
3.72813593203398	4.27463e-11\\
3.73013493253373	4.27463e-11\\
3.73213393303348	4.18368e-11\\
3.73413293353323	4.18368e-11\\
3.73613193403298	4.18368e-11\\
3.73813093453273	4.18368e-11\\
3.74012993503248	4.18368e-11\\
3.74212893553223	4.27463e-11\\
3.74412793603198	4.27463e-11\\
3.74612693653173	4.18368e-11\\
3.74812593703148	4.27463e-11\\
3.75012493753123	4.27463e-11\\
3.75212393803098	4.27463e-11\\
3.75412293853073	4.27463e-11\\
3.75612193903048	4.36557e-11\\
3.75812093953023	4.36557e-11\\
3.76011994002998	4.36557e-11\\
3.76211894052974	4.36557e-11\\
3.76411794102949	4.36557e-11\\
3.76611694152924	4.36557e-11\\
3.76811594202899	4.36557e-11\\
3.77011494252874	4.36557e-11\\
3.77211394302849	4.27463e-11\\
3.77411294352824	4.36557e-11\\
3.77611194402799	4.36557e-11\\
3.77811094452774	4.27463e-11\\
3.78010994502749	4.27463e-11\\
3.78210894552724	4.27463e-11\\
3.78410794602699	4.27463e-11\\
3.78610694652674	4.27463e-11\\
3.78810594702649	4.27463e-11\\
3.79010494752624	4.36557e-11\\
3.79210394802599	4.36557e-11\\
3.79410294852574	4.36557e-11\\
3.79610194902549	4.36557e-11\\
3.79810094952524	4.36557e-11\\
3.80009995002499	4.36557e-11\\
3.80209895052474	4.27463e-11\\
3.80409795102449	4.27463e-11\\
3.80609695152424	4.18368e-11\\
3.80809595202399	4.27463e-11\\
3.81009495252374	4.27463e-11\\
3.81209395302349	4.27463e-11\\
3.81409295352324	4.36557e-11\\
3.81609195402299	4.36557e-11\\
3.81809095452274	4.27463e-11\\
3.82008995502249	4.27463e-11\\
3.82208895552224	4.27463e-11\\
3.82408795602199	4.27463e-11\\
3.82608695652174	4.27463e-11\\
3.82808595702149	4.27463e-11\\
3.83008495752124	4.27463e-11\\
3.83208395802099	4.27463e-11\\
3.83408295852074	4.27463e-11\\
3.83608195902049	4.27463e-11\\
3.83808095952024	4.27463e-11\\
3.84007996001999	4.27463e-11\\
3.84207896051974	4.27463e-11\\
3.84407796101949	4.27463e-11\\
3.84607696151924	4.27463e-11\\
3.84807596201899	4.18368e-11\\
3.85007496251874	4.18368e-11\\
3.85207396301849	4.27463e-11\\
3.85407296351824	4.27463e-11\\
3.85607196401799	4.27463e-11\\
3.85807096451774	4.27463e-11\\
3.86006996501749	4.27463e-11\\
3.86206896551724	4.27463e-11\\
3.86406796601699	4.36557e-11\\
3.86606696651674	4.36557e-11\\
3.86806596701649	4.36557e-11\\
3.87006496751624	4.36557e-11\\
3.87206396801599	4.36557e-11\\
3.87406296851574	4.36557e-11\\
3.87606196901549	4.36557e-11\\
3.87806096951524	4.36557e-11\\
3.88005997001499	4.27463e-11\\
3.88205897051474	4.27463e-11\\
3.88405797101449	4.36557e-11\\
3.88605697151424	4.36557e-11\\
3.88805597201399	4.36557e-11\\
3.89005497251374	4.36557e-11\\
3.89205397301349	4.36557e-11\\
3.89405297351324	4.36557e-11\\
3.89605197401299	4.36557e-11\\
3.89805097451274	4.36557e-11\\
3.90004997501249	4.27463e-11\\
3.90204897551224	4.27463e-11\\
3.90404797601199	4.27463e-11\\
3.90604697651174	4.18368e-11\\
3.90804597701149	4.18368e-11\\
3.91004497751124	4.18368e-11\\
3.91204397801099	4.18368e-11\\
3.91404297851074	4.18368e-11\\
3.91604197901049	4.18368e-11\\
3.91804097951024	4.18368e-11\\
3.92003998000999	4.27463e-11\\
3.92203898050975	4.18368e-11\\
3.9240379810095	4.18368e-11\\
3.92603698150925	4.18368e-11\\
3.928035982009	4.18368e-11\\
3.93003498250875	4.09273e-11\\
3.9320339830085	4.09273e-11\\
3.93403298350825	4.09273e-11\\
3.936031984008	4.09273e-11\\
3.93803098450775	4.09273e-11\\
3.9400299850075	4.09273e-11\\
3.94202898550725	4.09273e-11\\
3.944027986007	4.09273e-11\\
3.94602698650675	4.09273e-11\\
3.9480259870065	4.09273e-11\\
3.95002498750625	4.18368e-11\\
3.952023988006	4.09273e-11\\
3.95402298850575	4.18368e-11\\
3.9560219890055	4.09273e-11\\
3.95802098950525	4.09273e-11\\
3.960019990005	4.09273e-11\\
3.96201899050475	4.00178e-11\\
3.9640179910045	4.00178e-11\\
3.96601699150425	4.09273e-11\\
3.968015992004	4.00178e-11\\
3.97001499250375	4.00178e-11\\
3.9720139930035	4.00178e-11\\
3.97401299350325	3.91083e-11\\
3.976011994003	3.91083e-11\\
3.97801099450275	4.00178e-11\\
3.9800099950025	4.00178e-11\\
3.98200899550225	3.91083e-11\\
3.984007996002	3.91083e-11\\
3.98600699650175	4.00178e-11\\
3.9880059970015	4.00178e-11\\
3.99000499750125	4.00178e-11\\
3.992003998001	4.00178e-11\\
3.99400299850075	4.00178e-11\\
3.9960019990005	4.00178e-11\\
3.99800099950025	4.00178e-11\\
4	4.00178e-11\\
};
\addlegendentry{c2};

\addplot [color=mycolor6,solid]
  table[row sep=crcr]{%
0	6.00267e-10\\
0.00199900049975012	6.00267e-10\\
0.00399800099950025	6.00267e-10\\
0.00599700149925037	6.00267e-10\\
0.0079960019990005	6.00267e-10\\
0.00999500249875063	5.82077e-10\\
0.0119940029985007	5.74801e-10\\
0.0139930034982509	5.74801e-10\\
0.015992003998001	5.74801e-10\\
0.0179910044977511	5.78439e-10\\
0.0199900049975013	5.82077e-10\\
0.0219890054972514	5.85715e-10\\
0.0239880059970015	5.82077e-10\\
0.0259870064967516	5.96629e-10\\
0.0279860069965017	5.96629e-10\\
0.0299850074962519	6.07542e-10\\
0.031984007996002	6.22094e-10\\
0.0339830084957521	6.03904e-10\\
0.0359820089955022	6.22094e-10\\
0.0379810094952524	6.18456e-10\\
0.0399800099950025	6.18456e-10\\
0.0419790104947526	6.18456e-10\\
0.0439780109945027	6.2937e-10\\
0.0459770114942529	6.18456e-10\\
0.047976011994003	6.14818e-10\\
0.0499750124937531	6.00267e-10\\
0.0519740129935032	5.89353e-10\\
0.0539730134932534	5.89353e-10\\
0.0559720139930035	5.96629e-10\\
0.0579710144927536	5.89353e-10\\
0.0599700149925037	5.96629e-10\\
0.0619690154922539	5.85715e-10\\
0.063968015992004	5.82077e-10\\
0.0659670164917541	5.71163e-10\\
0.0679660169915042	5.82077e-10\\
0.0699650174912544	5.82077e-10\\
0.0719640179910045	5.82077e-10\\
0.0739630184907546	5.63887e-10\\
0.0759620189905048	5.63887e-10\\
0.0779610194902549	5.56611e-10\\
0.079960019990005	5.67525e-10\\
0.0819590204897551	5.67525e-10\\
0.0839580209895052	5.82077e-10\\
0.0859570214892554	5.92991e-10\\
0.0879560219890055	5.89353e-10\\
0.0899550224887556	5.85715e-10\\
0.0919540229885057	5.85715e-10\\
0.0939530234882559	6.00267e-10\\
0.095952023988006	6.00267e-10\\
0.0979510244877561	5.89353e-10\\
0.0999500249875062	6.00267e-10\\
0.101949025487256	5.89353e-10\\
0.103948025987006	5.85715e-10\\
0.105947026486757	5.89353e-10\\
0.107946026986507	5.78439e-10\\
0.109945027486257	5.85715e-10\\
0.111944027986007	5.85715e-10\\
0.113943028485757	5.78439e-10\\
0.115942028985507	5.74801e-10\\
0.117941029485257	5.85715e-10\\
0.119940029985007	5.89353e-10\\
0.121939030484758	5.82077e-10\\
0.123938030984508	5.82077e-10\\
0.125937031484258	6.00267e-10\\
0.127936031984008	5.82077e-10\\
0.129935032483758	5.74801e-10\\
0.131934032983508	5.71163e-10\\
0.133933033483258	5.67525e-10\\
0.135932033983008	5.56611e-10\\
0.137931034482759	5.56611e-10\\
0.139930034982509	5.60249e-10\\
0.141929035482259	5.56611e-10\\
0.143928035982009	5.52973e-10\\
0.145927036481759	5.63887e-10\\
0.147926036981509	5.67525e-10\\
0.149925037481259	5.67525e-10\\
0.15192403798101	5.74801e-10\\
0.15392303848076	5.78439e-10\\
0.15592203898051	5.74801e-10\\
0.15792103948026	5.74801e-10\\
0.15992003998001	5.78439e-10\\
0.16191904047976	5.92991e-10\\
0.16391804097951	5.78439e-10\\
0.16591704147926	5.96629e-10\\
0.16791604197901	6.00267e-10\\
0.169915042478761	6.00267e-10\\
0.171914042978511	6.00267e-10\\
0.173913043478261	6.00267e-10\\
0.175912043978011	6.00267e-10\\
0.177911044477761	6.00267e-10\\
0.179910044977511	6.07542e-10\\
0.181909045477261	6.07542e-10\\
0.183908045977011	6.03904e-10\\
0.185907046476762	5.92991e-10\\
0.187906046976512	5.78439e-10\\
0.189905047476262	5.74801e-10\\
0.191904047976012	5.78439e-10\\
0.193903048475762	5.96629e-10\\
0.195902048975512	5.89353e-10\\
0.197901049475262	6.00267e-10\\
0.199900049975012	6.07542e-10\\
0.201899050474763	6.14818e-10\\
0.203898050974513	6.22094e-10\\
0.205897051474263	6.14818e-10\\
0.207896051974013	6.22094e-10\\
0.209895052473763	6.18456e-10\\
0.211894052973513	6.22094e-10\\
0.213893053473263	6.18456e-10\\
0.215892053973014	6.18456e-10\\
0.217891054472764	6.18456e-10\\
0.219890054972514	6.18456e-10\\
0.221889055472264	6.14818e-10\\
0.223888055972014	5.96629e-10\\
0.225887056471764	5.89353e-10\\
0.227886056971514	5.85715e-10\\
0.229885057471264	5.78439e-10\\
0.231884057971014	5.74801e-10\\
0.233883058470765	5.74801e-10\\
0.235882058970515	5.63887e-10\\
0.237881059470265	5.60249e-10\\
0.239880059970015	5.63887e-10\\
0.241879060469765	5.63887e-10\\
0.243878060969515	5.60249e-10\\
0.245877061469265	5.60249e-10\\
0.247876061969015	5.60249e-10\\
0.249875062468766	5.45697e-10\\
0.251874062968516	5.45697e-10\\
0.253873063468266	5.38421e-10\\
0.255872063968016	5.34783e-10\\
0.257871064467766	5.31145e-10\\
0.259870064967516	5.31145e-10\\
0.261869065467266	5.34783e-10\\
0.263868065967017	5.31145e-10\\
0.265867066466767	5.23869e-10\\
0.267866066966517	5.31145e-10\\
0.269865067466267	5.31145e-10\\
0.271864067966017	5.27507e-10\\
0.273863068465767	5.20231e-10\\
0.275862068965517	5.23869e-10\\
0.277861069465267	5.20231e-10\\
0.279860069965017	5.12955e-10\\
0.281859070464768	5.20231e-10\\
0.283858070964518	5.20231e-10\\
0.285857071464268	5.16593e-10\\
0.287856071964018	5.12955e-10\\
0.289855072463768	5.27507e-10\\
0.291854072963518	5.31145e-10\\
0.293853073463268	5.16593e-10\\
0.295852073963018	5.09317e-10\\
0.297851074462769	5.09317e-10\\
0.299850074962519	4.98403e-10\\
0.301849075462269	4.98403e-10\\
0.303848075962019	4.83851e-10\\
0.305847076461769	4.94765e-10\\
0.307846076961519	4.98403e-10\\
0.309845077461269	4.94765e-10\\
0.311844077961019	4.98403e-10\\
0.31384307846077	4.98403e-10\\
0.31584207896052	5.02041e-10\\
0.31784107946027	4.98403e-10\\
0.31984007996002	5.02041e-10\\
0.32183908045977	5.02041e-10\\
0.32383808095952	5.02041e-10\\
0.32583708145927	5.02041e-10\\
0.32783608195902	5.09317e-10\\
0.329835082458771	5.09317e-10\\
0.331834082958521	5.09317e-10\\
0.333833083458271	5.02041e-10\\
0.335832083958021	4.98403e-10\\
0.337831084457771	4.98403e-10\\
0.339830084957521	4.98403e-10\\
0.341829085457271	4.98403e-10\\
0.343828085957021	4.80213e-10\\
0.345827086456772	4.94765e-10\\
0.347826086956522	4.98403e-10\\
0.349825087456272	4.98403e-10\\
0.351824087956022	4.94765e-10\\
0.353823088455772	4.94765e-10\\
0.355822088955522	4.94765e-10\\
0.357821089455272	5.02041e-10\\
0.359820089955023	4.94765e-10\\
0.361819090454773	4.80213e-10\\
0.363818090954523	4.80213e-10\\
0.365817091454273	4.80213e-10\\
0.367816091954023	4.80213e-10\\
0.369815092453773	4.76575e-10\\
0.371814092953523	4.87489e-10\\
0.373813093453273	4.91127e-10\\
0.375812093953024	4.94765e-10\\
0.377811094452774	5.05679e-10\\
0.379810094952524	5.02041e-10\\
0.381809095452274	4.98403e-10\\
0.383808095952024	4.91127e-10\\
0.385807096451774	4.91127e-10\\
0.387806096951524	4.94765e-10\\
0.389805097451274	5.02041e-10\\
0.391804097951024	5.02041e-10\\
0.393803098450775	4.98403e-10\\
0.395802098950525	4.98403e-10\\
0.397801099450275	5.02041e-10\\
0.399800099950025	5.05679e-10\\
0.401799100449775	5.02041e-10\\
0.403798100949525	5.05679e-10\\
0.405797101449275	5.05679e-10\\
0.407796101949025	5.05679e-10\\
0.409795102448776	5.05679e-10\\
0.411794102948526	5.09317e-10\\
0.413793103448276	5.09317e-10\\
0.415792103948026	5.05679e-10\\
0.417791104447776	5.02041e-10\\
0.419790104947526	5.02041e-10\\
0.421789105447276	5.05679e-10\\
0.423788105947026	5.02041e-10\\
0.425787106446777	4.98403e-10\\
0.427786106946527	5.05679e-10\\
0.429785107446277	5.02041e-10\\
0.431784107946027	5.16593e-10\\
0.433783108445777	5.16593e-10\\
0.435782108945527	5.16593e-10\\
0.437781109445277	5.12955e-10\\
0.439780109945027	5.16593e-10\\
0.441779110444778	5.20231e-10\\
0.443778110944528	5.12955e-10\\
0.445777111444278	5.16593e-10\\
0.447776111944028	5.12955e-10\\
0.449775112443778	5.02041e-10\\
0.451774112943528	5.02041e-10\\
0.453773113443278	5.02041e-10\\
0.455772113943028	4.98403e-10\\
0.457771114442779	4.98403e-10\\
0.459770114942529	5.02041e-10\\
0.461769115442279	4.98403e-10\\
0.463768115942029	5.02041e-10\\
0.465767116441779	4.98403e-10\\
0.467766116941529	4.87489e-10\\
0.469765117441279	4.87489e-10\\
0.471764117941029	4.83851e-10\\
0.47376311844078	4.98403e-10\\
0.47576211894053	4.98403e-10\\
0.47776111944028	4.94765e-10\\
0.47976011994003	4.94765e-10\\
0.48175912043978	4.98403e-10\\
0.48375812093953	5.02041e-10\\
0.48575712143928	5.05679e-10\\
0.487756121939031	5.02041e-10\\
0.489755122438781	5.05679e-10\\
0.491754122938531	4.98403e-10\\
0.493753123438281	4.98403e-10\\
0.495752123938031	4.91127e-10\\
0.497751124437781	4.87489e-10\\
0.499750124937531	5.02041e-10\\
0.501749125437281	4.94765e-10\\
0.503748125937031	5.02041e-10\\
0.505747126436782	4.83851e-10\\
0.507746126936532	5.02041e-10\\
0.509745127436282	4.87489e-10\\
0.511744127936032	4.80213e-10\\
0.513743128435782	4.83851e-10\\
0.515742128935532	4.80213e-10\\
0.517741129435282	4.83851e-10\\
0.519740129935032	4.87489e-10\\
0.521739130434783	4.76575e-10\\
0.523738130934533	4.76575e-10\\
0.525737131434283	4.69299e-10\\
0.527736131934033	4.72937e-10\\
0.529735132433783	4.69299e-10\\
0.531734132933533	4.69299e-10\\
0.533733133433283	4.69299e-10\\
0.535732133933034	4.69299e-10\\
0.537731134432784	4.69299e-10\\
0.539730134932534	4.65661e-10\\
0.541729135432284	4.69299e-10\\
0.543728135932034	4.65661e-10\\
0.545727136431784	4.65661e-10\\
0.547726136931534	4.62023e-10\\
0.549725137431284	4.72937e-10\\
0.551724137931034	4.54747e-10\\
0.553723138430785	4.58385e-10\\
0.555722138930535	4.58385e-10\\
0.557721139430285	4.54747e-10\\
0.559720139930035	4.58385e-10\\
0.561719140429785	4.58385e-10\\
0.563718140929535	4.54747e-10\\
0.565717141429285	4.58385e-10\\
0.567716141929036	4.65661e-10\\
0.569715142428786	4.58385e-10\\
0.571714142928536	4.62023e-10\\
0.573713143428286	4.58385e-10\\
0.575712143928036	4.58385e-10\\
0.577711144427786	4.58385e-10\\
0.579710144927536	4.62023e-10\\
0.581709145427286	4.51109e-10\\
0.583708145927036	4.54747e-10\\
0.585707146426787	4.58385e-10\\
0.587706146926537	4.58385e-10\\
0.589705147426287	4.51109e-10\\
0.591704147926037	4.40195e-10\\
0.593703148425787	4.40195e-10\\
0.595702148925537	4.36557e-10\\
0.597701149425287	4.32919e-10\\
0.599700149925038	4.29281e-10\\
0.601699150424788	4.36557e-10\\
0.603698150924538	4.25644e-10\\
0.605697151424288	4.25644e-10\\
0.607696151924038	4.22006e-10\\
0.609695152423788	4.22006e-10\\
0.611694152923538	4.25644e-10\\
0.613693153423288	4.25644e-10\\
0.615692153923038	4.25644e-10\\
0.617691154422789	4.22006e-10\\
0.619690154922539	4.22006e-10\\
0.621689155422289	4.18368e-10\\
0.623688155922039	4.03816e-10\\
0.625687156421789	4.03816e-10\\
0.627686156921539	4.00178e-10\\
0.629685157421289	4.00178e-10\\
0.631684157921039	4.00178e-10\\
0.63368315842079	3.9654e-10\\
0.63568215892054	4.1473e-10\\
0.63768115942029	4.1473e-10\\
0.63968015992004	4.1473e-10\\
0.64167916041979	4.18368e-10\\
0.64367816091954	4.1473e-10\\
0.64567716141929	4.03816e-10\\
0.64767616191904	4.07454e-10\\
0.649675162418791	4.03816e-10\\
0.651674162918541	4.03816e-10\\
0.653673163418291	4.03816e-10\\
0.655672163918041	4.03816e-10\\
0.657671164417791	4.00178e-10\\
0.659670164917541	4.00178e-10\\
0.661669165417291	3.92902e-10\\
0.663668165917041	3.9654e-10\\
0.665667166416792	4.03816e-10\\
0.667666166916542	4.03816e-10\\
0.669665167416292	4.03816e-10\\
0.671664167916042	4.00178e-10\\
0.673663168415792	4.03816e-10\\
0.675662168915542	4.03816e-10\\
0.677661169415292	4.07454e-10\\
0.679660169915043	4.07454e-10\\
0.681659170414793	4.11092e-10\\
0.683658170914543	4.07454e-10\\
0.685657171414293	4.1473e-10\\
0.687656171914043	4.07454e-10\\
0.689655172413793	4.07454e-10\\
0.691654172913543	4.03816e-10\\
0.693653173413293	4.1473e-10\\
0.695652173913043	4.18368e-10\\
0.697651174412794	4.25644e-10\\
0.699650174912544	4.25644e-10\\
0.701649175412294	4.25644e-10\\
0.703648175912044	4.18368e-10\\
0.705647176411794	4.1473e-10\\
0.707646176911544	4.18368e-10\\
0.709645177411294	4.18368e-10\\
0.711644177911045	4.1473e-10\\
0.713643178410795	4.22006e-10\\
0.715642178910545	4.22006e-10\\
0.717641179410295	4.25644e-10\\
0.719640179910045	4.1473e-10\\
0.721639180409795	4.18368e-10\\
0.723638180909545	4.18368e-10\\
0.725637181409295	4.1473e-10\\
0.727636181909045	4.11092e-10\\
0.729635182408796	4.11092e-10\\
0.731634182908546	4.11092e-10\\
0.733633183408296	4.11092e-10\\
0.735632183908046	4.1473e-10\\
0.737631184407796	4.11092e-10\\
0.739630184907546	4.11092e-10\\
0.741629185407296	4.03816e-10\\
0.743628185907046	4.11092e-10\\
0.745627186406797	4.11092e-10\\
0.747626186906547	4.11092e-10\\
0.749625187406297	4.11092e-10\\
0.751624187906047	4.22006e-10\\
0.753623188405797	4.11092e-10\\
0.755622188905547	4.1473e-10\\
0.757621189405297	4.18368e-10\\
0.759620189905047	4.1473e-10\\
0.761619190404798	4.11092e-10\\
0.763618190904548	4.11092e-10\\
0.765617191404298	4.1473e-10\\
0.767616191904048	4.1473e-10\\
0.769615192403798	4.11092e-10\\
0.771614192903548	4.07454e-10\\
0.773613193403298	4.07454e-10\\
0.775612193903048	4.07454e-10\\
0.777611194402799	4.03816e-10\\
0.779610194902549	3.92902e-10\\
0.781609195402299	3.92902e-10\\
0.783608195902049	3.92902e-10\\
0.785607196401799	3.92902e-10\\
0.787606196901549	3.9654e-10\\
0.789605197401299	3.9654e-10\\
0.79160419790105	4.00178e-10\\
0.7936031984008	4.00178e-10\\
0.79560219890055	4.00178e-10\\
0.7976011994003	4.03816e-10\\
0.79960019990005	4.11092e-10\\
0.8015992003998	4.00178e-10\\
0.80359820089955	4.00178e-10\\
0.8055972013993	3.9654e-10\\
0.80759620189905	4.00178e-10\\
0.809595202398801	4.00178e-10\\
0.811594202898551	4.00178e-10\\
0.813593203398301	4.03816e-10\\
0.815592203898051	4.03816e-10\\
0.817591204397801	4.07454e-10\\
0.819590204897551	4.07454e-10\\
0.821589205397301	4.07454e-10\\
0.823588205897052	4.07454e-10\\
0.825587206396802	4.11092e-10\\
0.827586206896552	4.03816e-10\\
0.829585207396302	4.11092e-10\\
0.831584207896052	4.11092e-10\\
0.833583208395802	4.11092e-10\\
0.835582208895552	4.03816e-10\\
0.837581209395302	4.07454e-10\\
0.839580209895052	4.1473e-10\\
0.841579210394803	3.9654e-10\\
0.843578210894553	4.00178e-10\\
0.845577211394303	4.03816e-10\\
0.847576211894053	4.03816e-10\\
0.849575212393803	4.11092e-10\\
0.851574212893553	4.25644e-10\\
0.853573213393303	4.11092e-10\\
0.855572213893053	4.25644e-10\\
0.857571214392804	4.25644e-10\\
0.859570214892554	4.1473e-10\\
0.861569215392304	4.18368e-10\\
0.863568215892054	4.18368e-10\\
0.865567216391804	4.22006e-10\\
0.867566216891554	4.22006e-10\\
0.869565217391304	4.1473e-10\\
0.871564217891054	4.11092e-10\\
0.873563218390805	4.07454e-10\\
0.875562218890555	4.11092e-10\\
0.877561219390305	4.03816e-10\\
0.879560219890055	4.00178e-10\\
0.881559220389805	4.00178e-10\\
0.883558220889555	4.00178e-10\\
0.885557221389305	4.03816e-10\\
0.887556221889055	4.00178e-10\\
0.889555222388806	4.07454e-10\\
0.891554222888556	4.07454e-10\\
0.893553223388306	4.03816e-10\\
0.895552223888056	4.11092e-10\\
0.897551224387806	4.00178e-10\\
0.899550224887556	4.00178e-10\\
0.901549225387306	3.9654e-10\\
0.903548225887057	3.9654e-10\\
0.905547226386807	3.9654e-10\\
0.907546226886557	4.07454e-10\\
0.909545227386307	4.07454e-10\\
0.911544227886057	4.00178e-10\\
0.913543228385807	4.03816e-10\\
0.915542228885557	4.00178e-10\\
0.917541229385307	3.9654e-10\\
0.919540229885057	3.89264e-10\\
0.921539230384808	3.89264e-10\\
0.923538230884558	4.00178e-10\\
0.925537231384308	3.9654e-10\\
0.927536231884058	3.89264e-10\\
0.929535232383808	3.9654e-10\\
0.931534232883558	3.92902e-10\\
0.933533233383308	4.03816e-10\\
0.935532233883059	3.89264e-10\\
0.937531234382809	3.89264e-10\\
0.939530234882559	3.85626e-10\\
0.941529235382309	3.81988e-10\\
0.943528235882059	3.7835e-10\\
0.945527236381809	3.81988e-10\\
0.947526236881559	3.7835e-10\\
0.949525237381309	3.7835e-10\\
0.951524237881059	3.89264e-10\\
0.95352323838081	3.89264e-10\\
0.95552223888056	3.92902e-10\\
0.95752123938031	3.92902e-10\\
0.95952023988006	3.9654e-10\\
0.96151924037981	4.07454e-10\\
0.96351824087956	4.07454e-10\\
0.96551724137931	4.03816e-10\\
0.96751624187906	4.11092e-10\\
0.969515242378811	4.11092e-10\\
0.971514242878561	4.07454e-10\\
0.973513243378311	4.11092e-10\\
0.975512243878061	4.25644e-10\\
0.977511244377811	4.22006e-10\\
0.979510244877561	4.32919e-10\\
0.981509245377311	4.1473e-10\\
0.983508245877061	4.29281e-10\\
0.985507246376812	4.18368e-10\\
0.987506246876562	4.32919e-10\\
0.989505247376312	4.32919e-10\\
0.991504247876062	4.25644e-10\\
0.993503248375812	4.29281e-10\\
0.995502248875562	4.29281e-10\\
0.997501249375312	4.36557e-10\\
0.999500249875062	4.40195e-10\\
1.00149925037481	4.40195e-10\\
1.00349825087456	4.29281e-10\\
1.00549725137431	4.36557e-10\\
1.00749625187406	4.40195e-10\\
1.00949525237381	4.32919e-10\\
1.01149425287356	4.32919e-10\\
1.01349325337331	4.25644e-10\\
1.01549225387306	4.18368e-10\\
1.01749125437281	4.22006e-10\\
1.01949025487256	4.22006e-10\\
1.02148925537231	4.25644e-10\\
1.02348825587206	4.36557e-10\\
1.02548725637181	4.32919e-10\\
1.02748625687156	4.32919e-10\\
1.02948525737131	4.29281e-10\\
1.03148425787106	4.43833e-10\\
1.03348325837081	4.40195e-10\\
1.03548225887056	4.43833e-10\\
1.03748125937031	4.29281e-10\\
1.03948025987006	4.1473e-10\\
1.04147926036982	4.18368e-10\\
1.04347826086957	4.1473e-10\\
1.04547726136932	4.22006e-10\\
1.04747626186907	4.11092e-10\\
1.04947526236882	4.1473e-10\\
1.05147426286857	4.1473e-10\\
1.05347326336832	4.1473e-10\\
1.05547226386807	4.11092e-10\\
1.05747126436782	4.07454e-10\\
1.05947026486757	4.18368e-10\\
1.06146926536732	4.22006e-10\\
1.06346826586707	4.36557e-10\\
1.06546726636682	4.36557e-10\\
1.06746626686657	4.29281e-10\\
1.06946526736632	4.25644e-10\\
1.07146426786607	4.29281e-10\\
1.07346326836582	4.32919e-10\\
1.07546226886557	4.32919e-10\\
1.07746126936532	4.22006e-10\\
1.07946026986507	4.32919e-10\\
1.08145927036482	4.32919e-10\\
1.08345827086457	4.32919e-10\\
1.08545727136432	4.22006e-10\\
1.08745627186407	4.22006e-10\\
1.08945527236382	4.07454e-10\\
1.09145427286357	4.07454e-10\\
1.09345327336332	4.11092e-10\\
1.09545227386307	4.07454e-10\\
1.09745127436282	4.00178e-10\\
1.09945027486257	3.89264e-10\\
1.10144927536232	3.85626e-10\\
1.10344827586207	3.7835e-10\\
1.10544727636182	3.71074e-10\\
1.10744627686157	3.56522e-10\\
1.10944527736132	3.52884e-10\\
1.11144427786107	3.4197e-10\\
1.11344327836082	3.27418e-10\\
1.11544227886057	3.38332e-10\\
1.11744127936032	3.38332e-10\\
1.11944027986007	3.4197e-10\\
1.12143928035982	3.45608e-10\\
1.12343828085957	3.45608e-10\\
1.12543728135932	3.45608e-10\\
1.12743628185907	3.4197e-10\\
1.12943528235882	3.31056e-10\\
1.13143428285857	3.31056e-10\\
1.13343328335832	3.16504e-10\\
1.13543228385807	3.27418e-10\\
1.13743128435782	3.4197e-10\\
1.13943028485757	3.45608e-10\\
1.14142928535732	3.56522e-10\\
1.14342828585707	3.49246e-10\\
1.14542728635682	3.4197e-10\\
1.14742628685657	3.56522e-10\\
1.14942528735632	3.52884e-10\\
1.15142428785607	3.45608e-10\\
1.15342328835582	3.52884e-10\\
1.15542228885557	3.49246e-10\\
1.15742128935532	3.52884e-10\\
1.15942028985507	3.56522e-10\\
1.16141929035482	3.52884e-10\\
1.16341829085457	3.49246e-10\\
1.16541729135432	3.63798e-10\\
1.16741629185407	3.52884e-10\\
1.16941529235382	3.52884e-10\\
1.17141429285357	3.52884e-10\\
1.17341329335332	3.49246e-10\\
1.17541229385307	3.4197e-10\\
1.17741129435282	3.52884e-10\\
1.17941029485257	3.71074e-10\\
1.18140929535232	3.67436e-10\\
1.18340829585207	3.67436e-10\\
1.18540729635182	3.7835e-10\\
1.18740629685157	3.74712e-10\\
1.18940529735132	3.63798e-10\\
1.19140429785107	3.74712e-10\\
1.19340329835082	3.85626e-10\\
1.19540229885057	3.81988e-10\\
1.19740129935032	3.85626e-10\\
1.19940029985008	3.9654e-10\\
1.20139930034983	3.9654e-10\\
1.20339830084958	3.9654e-10\\
1.20539730134933	4.1473e-10\\
1.20739630184908	4.22006e-10\\
1.20939530234883	4.18368e-10\\
1.21139430284858	4.11092e-10\\
1.21339330334833	4.11092e-10\\
1.21539230384808	4.11092e-10\\
1.21739130434783	4.22006e-10\\
1.21939030484758	4.22006e-10\\
1.22138930534733	4.22006e-10\\
1.22338830584708	4.22006e-10\\
1.22538730634683	4.11092e-10\\
1.22738630684658	4.22006e-10\\
1.22938530734633	4.32919e-10\\
1.23138430784608	4.43833e-10\\
1.23338330834583	4.47471e-10\\
1.23538230884558	4.58385e-10\\
1.23738130934533	4.47471e-10\\
1.23938030984508	4.58385e-10\\
1.24137931034483	4.58385e-10\\
1.24337831084458	4.65661e-10\\
1.24537731134433	4.80213e-10\\
1.24737631184408	4.76575e-10\\
1.24937531234383	4.62023e-10\\
1.25137431284358	4.62023e-10\\
1.25337331334333	4.76575e-10\\
1.25537231384308	4.76575e-10\\
1.25737131434283	4.65661e-10\\
1.25937031484258	4.76575e-10\\
1.26136931534233	4.76575e-10\\
1.26336831584208	4.83851e-10\\
1.26536731634183	4.87489e-10\\
1.26736631684158	5.02041e-10\\
1.26936531734133	4.87489e-10\\
1.27136431784108	4.83851e-10\\
1.27336331834083	4.83851e-10\\
1.27536231884058	4.83851e-10\\
1.27736131934033	4.94765e-10\\
1.27936031984008	4.83851e-10\\
1.28135932033983	4.80213e-10\\
1.28335832083958	4.94765e-10\\
1.28535732133933	4.98403e-10\\
1.28735632183908	4.98403e-10\\
1.28935532233883	4.83851e-10\\
1.29135432283858	4.72937e-10\\
1.29335332333833	4.51109e-10\\
1.29535232383808	4.54747e-10\\
1.29735132433783	4.51109e-10\\
1.29935032483758	4.40195e-10\\
1.30134932533733	4.51109e-10\\
1.30334832583708	4.51109e-10\\
1.30534732633683	4.40195e-10\\
1.30734632683658	4.36557e-10\\
1.30934532733633	4.25644e-10\\
1.31134432783608	4.25644e-10\\
1.31334332833583	4.22006e-10\\
1.31534232883558	4.07454e-10\\
1.31734132933533	4.25644e-10\\
1.31934032983508	4.11092e-10\\
1.32133933033483	4.25644e-10\\
1.32333833083458	4.11092e-10\\
1.32533733133433	4.11092e-10\\
1.32733633183408	4.07454e-10\\
1.32933533233383	4.00178e-10\\
1.33133433283358	4.11092e-10\\
1.33333333333333	4.07454e-10\\
1.33533233383308	4.11092e-10\\
1.33733133433283	4.11092e-10\\
1.33933033483258	4.11092e-10\\
1.34132933533233	4.22006e-10\\
1.34332833583208	4.25644e-10\\
1.34532733633183	4.25644e-10\\
1.34732633683158	4.51109e-10\\
1.34932533733133	4.47471e-10\\
1.35132433783108	4.25644e-10\\
1.35332333833083	4.29281e-10\\
1.35532233883058	4.18368e-10\\
1.35732133933033	4.25644e-10\\
1.35932033983009	4.29281e-10\\
1.36131934032984	4.36557e-10\\
1.36331834082959	4.32919e-10\\
1.36531734132934	4.36557e-10\\
1.36731634182909	4.51109e-10\\
1.36931534232884	4.36557e-10\\
1.37131434282859	4.32919e-10\\
1.37331334332834	4.51109e-10\\
1.37531234382809	4.58385e-10\\
1.37731134432784	4.51109e-10\\
1.37931034482759	4.51109e-10\\
1.38130934532734	4.58385e-10\\
1.38330834582709	4.51109e-10\\
1.38530734632684	4.54747e-10\\
1.38730634682659	4.36557e-10\\
1.38930534732634	4.51109e-10\\
1.39130434782609	4.51109e-10\\
1.39330334832584	4.36557e-10\\
1.39530234882559	4.25644e-10\\
1.39730134932534	4.03816e-10\\
1.39930034982509	4.00178e-10\\
1.40129935032484	3.9654e-10\\
1.40329835082459	4.00178e-10\\
1.40529735132434	4.00178e-10\\
1.40729635182409	3.9654e-10\\
1.40929535232384	3.85626e-10\\
1.41129435282359	4.00178e-10\\
1.41329335332334	3.85626e-10\\
1.41529235382309	3.81988e-10\\
1.41729135432284	3.74712e-10\\
1.41929035482259	3.74712e-10\\
1.42128935532234	3.71074e-10\\
1.42328835582209	3.71074e-10\\
1.42528735632184	3.56522e-10\\
1.42728635682159	3.45608e-10\\
1.42928535732134	3.49246e-10\\
1.43128435782109	3.45608e-10\\
1.43328335832084	3.38332e-10\\
1.43528235882059	3.27418e-10\\
1.43728135932034	3.38332e-10\\
1.43928035982009	3.38332e-10\\
1.44127936031984	3.34694e-10\\
1.44327836081959	3.34694e-10\\
1.44527736131934	3.49246e-10\\
1.44727636181909	3.63798e-10\\
1.44927536231884	3.6016e-10\\
1.45127436281859	3.52884e-10\\
1.45327336331834	3.63798e-10\\
1.45527236381809	3.63798e-10\\
1.45727136431784	3.52884e-10\\
1.45927036481759	3.63798e-10\\
1.46126936531734	3.56522e-10\\
1.46326836581709	3.52884e-10\\
1.46526736631684	3.4197e-10\\
1.46726636681659	3.56522e-10\\
1.46926536731634	3.56522e-10\\
1.47126436781609	3.63798e-10\\
1.47326336831584	3.63798e-10\\
1.47526236881559	3.52884e-10\\
1.47726136931534	3.49246e-10\\
1.47926036981509	3.63798e-10\\
1.48125937031484	3.52884e-10\\
1.48325837081459	3.63798e-10\\
1.48525737131434	3.52884e-10\\
1.48725637181409	3.52884e-10\\
1.48925537231384	3.52884e-10\\
1.49125437281359	3.49246e-10\\
1.49325337331334	3.63798e-10\\
1.49525237381309	3.74712e-10\\
1.49725137431284	3.71074e-10\\
1.49925037481259	3.6016e-10\\
1.50124937531234	3.56522e-10\\
1.50324837581209	3.67436e-10\\
1.50524737631184	3.71074e-10\\
1.50724637681159	3.7835e-10\\
1.50924537731134	3.7835e-10\\
1.51124437781109	3.74712e-10\\
1.51324337831084	3.67436e-10\\
1.51524237881059	3.85626e-10\\
1.51724137931034	3.71074e-10\\
1.51924037981009	3.6016e-10\\
1.52123938030985	3.4197e-10\\
1.5232383808096	3.4197e-10\\
1.52523738130935	3.45608e-10\\
1.5272363818091	3.31056e-10\\
1.52923538230885	3.16504e-10\\
1.5312343828086	3.16504e-10\\
1.53323338330835	3.16504e-10\\
1.5352323838081	2.98314e-10\\
1.53723138430785	3.01952e-10\\
1.5392303848076	3.01952e-10\\
1.54122938530735	3.16504e-10\\
1.5432283858071	3.16504e-10\\
1.54522738630685	3.12866e-10\\
1.5472263868066	3.12866e-10\\
1.54922538730635	2.98314e-10\\
1.5512243878061	2.98314e-10\\
1.55322338830585	3.12866e-10\\
1.5552223888056	3.16504e-10\\
1.55722138930535	3.01952e-10\\
1.5592203898051	3.12866e-10\\
1.56121939030485	2.94676e-10\\
1.5632183908046	2.91038e-10\\
1.56521739130435	2.874e-10\\
1.5672163918041	2.91038e-10\\
1.56921539230385	2.91038e-10\\
1.5712143928036	2.98314e-10\\
1.57321339330335	2.874e-10\\
1.5752123938031	2.874e-10\\
1.57721139430285	2.94676e-10\\
1.5792103948026	2.874e-10\\
1.58120939530235	2.874e-10\\
1.5832083958021	2.874e-10\\
1.58520739630185	2.6921e-10\\
1.5872063968016	2.65572e-10\\
1.58920539730135	2.65572e-10\\
1.5912043978011	2.65572e-10\\
1.59320339830085	2.72848e-10\\
1.5952023988006	2.65572e-10\\
1.59720139930035	2.51021e-10\\
1.5992003998001	2.43745e-10\\
1.60119940029985	2.51021e-10\\
1.6031984007996	2.40107e-10\\
1.60519740129935	2.36469e-10\\
1.6071964017991	2.36469e-10\\
1.60919540229885	2.21917e-10\\
1.6111944027986	2.21917e-10\\
1.61319340329835	2.18279e-10\\
1.6151924037981	2.32831e-10\\
1.61719140429785	2.18279e-10\\
1.6191904047976	2.32831e-10\\
1.62118940529735	2.36469e-10\\
1.6231884057971	2.32831e-10\\
1.62518740629685	2.29193e-10\\
1.6271864067966	2.18279e-10\\
1.62918540729635	2.18279e-10\\
1.6311844077961	2.29193e-10\\
1.63318340829585	2.40107e-10\\
1.6351824087956	2.43745e-10\\
1.63718140929535	2.36469e-10\\
1.6391804097951	2.21917e-10\\
1.64117941029485	2.18279e-10\\
1.6431784107946	2.07365e-10\\
1.64517741129435	2.18279e-10\\
1.6471764117941	2.07365e-10\\
1.64917541229385	2.07365e-10\\
1.6511744127936	2.29193e-10\\
1.65317341329335	2.18279e-10\\
1.6551724137931	2.32831e-10\\
1.65717141429285	2.32831e-10\\
1.6591704147926	2.43745e-10\\
1.66116941529235	2.40107e-10\\
1.6631684157921	2.43745e-10\\
1.66516741629185	2.43745e-10\\
1.6671664167916	2.54659e-10\\
1.66916541729135	2.58296e-10\\
1.6711644177911	2.58296e-10\\
1.67316341829085	2.58296e-10\\
1.6751624187906	2.58296e-10\\
1.67716141929035	2.58296e-10\\
1.6791604197901	2.43745e-10\\
1.68115942028985	2.58296e-10\\
1.68315842078961	2.58296e-10\\
1.68515742128936	2.43745e-10\\
1.68715642178911	2.58296e-10\\
1.68915542228886	2.58296e-10\\
1.69115442278861	2.43745e-10\\
1.69315342328836	2.58296e-10\\
1.69515242378811	2.61934e-10\\
1.69715142428786	2.65572e-10\\
1.69915042478761	2.61934e-10\\
1.70114942528736	2.61934e-10\\
1.70314842578711	2.6921e-10\\
1.70514742628686	2.65572e-10\\
1.70714642678661	2.61934e-10\\
1.70914542728636	2.65572e-10\\
1.71114442778611	2.83762e-10\\
1.71314342828586	2.83762e-10\\
1.71514242878561	2.6921e-10\\
1.71714142928536	2.65572e-10\\
1.71914042978511	2.47383e-10\\
1.72113943028486	2.36469e-10\\
1.72313843078461	2.40107e-10\\
1.72513743128436	2.40107e-10\\
1.72713643178411	2.36469e-10\\
1.72913543228386	2.21917e-10\\
1.73113443278361	2.14641e-10\\
1.73313343328336	2.14641e-10\\
1.73513243378311	2.11003e-10\\
1.73713143428286	1.96451e-10\\
1.73913043478261	1.96451e-10\\
1.74112943528236	1.81899e-10\\
1.74312843578211	1.81899e-10\\
1.74512743628186	1.74623e-10\\
1.74712643678161	1.81899e-10\\
1.74912543728136	1.89175e-10\\
1.75112443778111	1.92813e-10\\
1.75312343828086	1.89175e-10\\
1.75512243878061	1.89175e-10\\
1.75712143928036	1.89175e-10\\
1.75912043978011	1.92813e-10\\
1.76111944027986	1.81899e-10\\
1.76311844077961	1.85537e-10\\
1.76511744127936	1.74623e-10\\
1.76711644177911	1.89175e-10\\
1.76911544227886	1.96451e-10\\
1.77111444277861	2.07365e-10\\
1.77311344327836	2.07365e-10\\
1.77511244377811	2.03727e-10\\
1.77711144427786	2.03727e-10\\
1.77911044477761	2.03727e-10\\
1.78110944527736	2.18279e-10\\
1.78310844577711	2.25555e-10\\
1.78510744627686	2.29193e-10\\
1.78710644677661	2.29193e-10\\
1.78910544727636	2.32831e-10\\
1.79110444777611	2.29193e-10\\
1.79310344827586	2.18279e-10\\
1.79510244877561	2.18279e-10\\
1.79710144927536	2.03727e-10\\
1.79910044977511	1.89175e-10\\
1.80109945027486	1.81899e-10\\
1.80309845077461	1.81899e-10\\
1.80509745127436	1.78261e-10\\
1.80709645177411	1.63709e-10\\
1.80909545227386	1.63709e-10\\
1.81109445277361	1.78261e-10\\
1.81309345327336	1.74623e-10\\
1.81509245377311	1.67347e-10\\
1.81709145427286	1.63709e-10\\
1.81909045477261	1.52795e-10\\
1.82108945527236	1.60071e-10\\
1.82308845577211	1.63709e-10\\
1.82508745627186	1.60071e-10\\
1.82708645677161	1.67347e-10\\
1.82908545727136	1.60071e-10\\
1.83108445777111	1.41881e-10\\
1.83308345827086	1.34605e-10\\
1.83508245877061	1.45519e-10\\
1.83708145927036	1.60071e-10\\
1.83908045977011	1.45519e-10\\
1.84107946026987	1.45519e-10\\
1.84307846076962	1.34605e-10\\
1.84507746126937	1.38243e-10\\
1.84707646176912	1.34605e-10\\
1.84907546226887	1.38243e-10\\
1.85107446276862	1.27329e-10\\
1.85307346326837	1.41881e-10\\
1.85507246376812	1.52795e-10\\
1.85707146426787	1.52795e-10\\
1.85907046476762	1.41881e-10\\
1.86106946526737	1.41881e-10\\
1.86306846576712	1.27329e-10\\
1.86506746626687	1.23691e-10\\
1.86706646676662	1.27329e-10\\
1.86906546726637	1.27329e-10\\
1.87106446776612	1.23691e-10\\
1.87306346826587	1.23691e-10\\
1.87506246876562	1.27329e-10\\
1.87706146926537	1.27329e-10\\
1.87906046976512	1.27329e-10\\
1.88105947026487	1.23691e-10\\
1.88305847076462	1.23691e-10\\
1.88505747126437	1.12777e-10\\
1.88705647176412	1.12777e-10\\
1.88905547226387	1.09139e-10\\
1.89105447276362	1.23691e-10\\
1.89305347326337	1.16415e-10\\
1.89505247376312	1.23691e-10\\
1.89705147426287	1.09139e-10\\
1.89905047476262	1.09139e-10\\
1.90104947526237	9.09495e-11\\
1.90304847576212	9.09495e-11\\
1.90504747626187	8.36735e-11\\
1.90704647676162	9.45874e-11\\
1.90904547726137	9.09495e-11\\
1.91104447776112	9.09495e-11\\
1.91304347826087	9.09495e-11\\
1.91504247876062	9.09495e-11\\
1.91704147926037	8.36735e-11\\
1.91904047976012	8.36735e-11\\
1.92103948025987	9.09495e-11\\
1.92303848075962	8.00355e-11\\
1.92503748125937	8.73115e-11\\
1.92703648175912	8.36735e-11\\
1.92903548225887	8.00355e-11\\
1.93103448275862	8.00355e-11\\
1.93303348325837	8.36735e-11\\
1.93503248375812	8.73115e-11\\
1.93703148425787	9.09495e-11\\
1.93903048475762	8.73115e-11\\
1.94102948525737	8.73115e-11\\
1.94302848575712	8.73115e-11\\
1.94502748625687	8.73115e-11\\
1.94702648675662	7.63976e-11\\
1.94902548725637	8.00355e-11\\
1.95102448775612	8.36735e-11\\
1.95302348825587	8.73115e-11\\
1.95502248875562	9.09495e-11\\
1.95702148925537	9.45874e-11\\
1.95902048975512	9.09495e-11\\
1.96101949025487	8.73115e-11\\
1.96301849075462	8.73115e-11\\
1.96501749125437	8.73115e-11\\
1.96701649175412	9.82254e-11\\
1.96901549225387	9.82254e-11\\
1.97101449275362	9.09495e-11\\
1.97301349325337	9.45874e-11\\
1.97501249375312	9.45874e-11\\
1.97701149425287	1.09139e-10\\
1.97901049475262	1.09139e-10\\
1.98100949525237	1.05501e-10\\
1.98300849575212	1.20053e-10\\
1.98500749625187	1.23691e-10\\
1.98700649675162	1.23691e-10\\
1.98900549725137	1.20053e-10\\
1.99100449775112	1.23691e-10\\
1.99300349825087	1.20053e-10\\
1.99500249875062	1.23691e-10\\
1.99700149925037	1.12777e-10\\
1.99900049975012	1.12777e-10\\
2.00099950024988	1.09139e-10\\
2.00299850074963	9.45874e-11\\
2.00499750124938	1.01863e-10\\
2.00699650174913	1.05501e-10\\
2.00899550224888	9.09495e-11\\
2.01099450274863	8.73115e-11\\
2.01299350324838	9.45874e-11\\
2.01499250374813	8.36735e-11\\
2.01699150424788	8.00355e-11\\
2.01899050474763	7.27596e-11\\
2.02098950524738	8.00355e-11\\
2.02298850574713	8.73115e-11\\
2.02498750624688	8.73115e-11\\
2.02698650674663	8.73115e-11\\
2.02898550724638	9.09495e-11\\
2.03098450774613	9.45874e-11\\
2.03298350824588	9.45874e-11\\
2.03498250874563	1.01863e-10\\
2.03698150924538	9.09495e-11\\
2.03898050974513	9.09495e-11\\
2.04097951024488	1.01863e-10\\
2.04297851074463	9.45874e-11\\
2.04497751124438	9.09495e-11\\
2.04697651174413	9.45874e-11\\
2.04897551224388	8.73115e-11\\
2.05097451274363	9.09495e-11\\
2.05297351324338	8.73115e-11\\
2.05497251374313	8.36735e-11\\
2.05697151424288	7.63976e-11\\
2.05897051474263	7.27596e-11\\
2.06096951524238	7.27596e-11\\
2.06296851574213	6.91216e-11\\
2.06496751624188	6.54836e-11\\
2.06696651674163	5.82077e-11\\
2.06896551724138	5.45697e-11\\
2.07096451774113	5.09317e-11\\
2.07296351824088	4.00178e-11\\
2.07496251874063	4.36557e-11\\
2.07696151924038	4.72937e-11\\
2.07896051974013	4.00178e-11\\
2.08095952023988	3.63798e-11\\
2.08295852073963	5.45697e-11\\
2.08495752123938	5.09317e-11\\
2.08695652173913	5.09317e-11\\
2.08895552223888	4.36557e-11\\
2.09095452273863	4.72937e-11\\
2.09295352323838	5.45697e-11\\
2.09495252373813	4.00178e-11\\
2.09695152423788	5.45697e-11\\
2.09895052473763	4.36557e-11\\
2.10094952523738	4.72937e-11\\
2.10294852573713	4.72937e-11\\
2.10494752623688	4.72937e-11\\
2.10694652673663	5.82077e-11\\
2.10894552723638	4.72937e-11\\
2.11094452773613	5.09317e-11\\
2.11294352823588	5.45697e-11\\
2.11494252873563	5.82077e-11\\
2.11694152923538	6.54836e-11\\
2.11894052973513	6.18456e-11\\
2.12093953023488	5.45697e-11\\
2.12293853073463	5.82077e-11\\
2.12493753123438	5.09317e-11\\
2.12693653173413	5.09317e-11\\
2.12893553223388	4.72937e-11\\
2.13093453273363	4.72937e-11\\
2.13293353323338	4.36557e-11\\
2.13493253373313	4.72937e-11\\
2.13693153423288	5.09317e-11\\
2.13893053473263	3.63798e-11\\
2.14092953523238	3.63798e-11\\
2.14292853573213	5.09317e-11\\
2.14492753623188	5.45697e-11\\
2.14692653673163	5.09317e-11\\
2.14892553723138	5.09317e-11\\
2.15092453773113	6.18456e-11\\
2.15292353823088	6.54836e-11\\
2.15492253873063	6.54836e-11\\
2.15692153923038	6.91216e-11\\
2.15892053973013	6.91216e-11\\
2.16091954022989	6.91216e-11\\
2.16291854072964	7.63976e-11\\
2.16491754122939	6.91216e-11\\
2.16691654172914	6.91216e-11\\
2.16891554222889	7.27596e-11\\
2.17091454272864	6.54836e-11\\
2.17291354322839	6.54836e-11\\
2.17491254372814	6.18456e-11\\
2.17691154422789	6.54836e-11\\
2.17891054472764	6.54836e-11\\
2.18090954522739	6.54836e-11\\
2.18290854572714	5.09317e-11\\
2.18490754622689	5.82077e-11\\
2.18690654672664	5.45697e-11\\
2.18890554722639	6.18456e-11\\
2.19090454772614	5.82077e-11\\
2.19290354822589	4.72937e-11\\
2.19490254872564	5.09317e-11\\
2.19690154922539	4.72937e-11\\
2.19890054972514	5.09317e-11\\
2.20089955022489	4.36557e-11\\
2.20289855072464	4.72937e-11\\
2.20489755122439	4.36557e-11\\
2.20689655172414	4.36557e-11\\
2.20889555222389	3.63798e-11\\
2.21089455272364	3.63798e-11\\
2.21289355322339	4.00178e-11\\
2.21489255372314	3.63798e-11\\
2.21689155422289	2.91038e-11\\
2.21889055472264	3.27418e-11\\
2.22088955522239	2.91038e-11\\
2.22288855572214	3.63798e-11\\
2.22488755622189	2.91038e-11\\
2.22688655672164	3.27418e-11\\
2.22888555722139	3.27418e-11\\
2.23088455772114	2.91038e-11\\
2.23288355822089	2.18279e-11\\
2.23488255872064	2.54659e-11\\
2.23688155922039	2.91038e-11\\
2.23888055972014	4.00178e-11\\
2.24087956021989	3.27418e-11\\
2.24287856071964	2.54659e-11\\
2.24487756121939	2.91038e-11\\
2.24687656171914	2.54659e-11\\
2.24887556221889	3.27418e-11\\
2.25087456271864	3.63798e-11\\
2.25287356321839	4.00178e-11\\
2.25487256371814	4.00178e-11\\
2.25687156421789	4.00178e-11\\
2.25887056471764	4.00178e-11\\
2.26086956521739	3.63798e-11\\
2.26286856571714	5.09317e-11\\
2.26486756621689	4.72937e-11\\
2.26686656671664	4.36557e-11\\
2.26886556721639	4.36557e-11\\
2.27086456771614	4.00178e-11\\
2.27286356821589	4.72937e-11\\
2.27486256871564	5.09317e-11\\
2.27686156921539	5.45697e-11\\
2.27886056971514	5.09317e-11\\
2.28085957021489	5.45697e-11\\
2.28285857071464	5.09317e-11\\
2.28485757121439	5.09317e-11\\
2.28685657171414	4.00178e-11\\
2.28885557221389	4.36557e-11\\
2.29085457271364	5.82077e-11\\
2.29285357321339	6.18456e-11\\
2.29485257371314	5.82077e-11\\
2.29685157421289	6.54836e-11\\
2.29885057471264	5.82077e-11\\
2.30084957521239	6.54836e-11\\
2.30284857571214	6.54836e-11\\
2.30484757621189	7.27596e-11\\
2.30684657671164	7.27596e-11\\
2.30884557721139	7.27596e-11\\
2.31084457771114	6.91216e-11\\
2.31284357821089	6.91216e-11\\
2.31484257871064	7.27596e-11\\
2.31684157921039	8.73115e-11\\
2.31884057971015	7.63976e-11\\
2.3208395802099	9.09495e-11\\
2.32283858070965	9.09495e-11\\
2.3248375812094	8.73115e-11\\
2.32683658170915	8.73115e-11\\
2.3288355822089	7.63976e-11\\
2.33083458270865	7.27596e-11\\
2.3328335832084	8.73115e-11\\
2.33483258370815	8.00355e-11\\
2.3368315842079	7.63976e-11\\
2.33883058470765	7.63976e-11\\
2.3408295852074	7.63976e-11\\
2.34282858570715	6.54836e-11\\
2.3448275862069	5.82077e-11\\
2.34682658670665	5.45697e-11\\
2.3488255872064	5.09317e-11\\
2.35082458770615	5.45697e-11\\
2.3528235882059	5.09317e-11\\
2.35482258870565	5.09317e-11\\
2.3568215892054	4.36557e-11\\
2.35882058970515	4.36557e-11\\
2.3608195902049	4.36557e-11\\
2.36281859070465	4.00178e-11\\
2.3648175912044	3.63798e-11\\
2.36681659170415	3.63798e-11\\
2.3688155922039	4.00178e-11\\
2.37081459270365	3.63798e-11\\
2.3728135932034	4.00178e-11\\
2.37481259370315	4.36557e-11\\
2.3768115942029	3.63798e-11\\
2.37881059470265	3.63798e-11\\
2.3808095952024	3.27418e-11\\
2.38280859570215	3.27418e-11\\
2.3848075962019	2.18279e-11\\
2.38680659670165	2.54659e-11\\
2.3888055972014	2.91038e-11\\
2.39080459770115	2.54659e-11\\
2.3928035982009	2.18279e-11\\
2.39480259870065	2.18279e-11\\
2.3968015992004	1.45519e-11\\
2.39880059970015	1.45519e-11\\
2.4007996001999	1.45519e-11\\
2.40279860069965	2.54659e-11\\
2.4047976011994	2.18279e-11\\
2.40679660169915	2.54659e-11\\
2.4087956021989	2.18279e-11\\
2.41079460269865	1.81899e-11\\
2.4127936031984	1.81899e-11\\
2.41479260369815	1.81899e-11\\
2.4167916041979	1.81899e-11\\
2.41879060469765	2.18279e-11\\
2.4207896051974	2.54659e-11\\
2.42278860569715	2.91038e-11\\
2.4247876061969	3.27418e-11\\
2.42678660669665	2.91038e-11\\
2.4287856071964	4.36557e-11\\
2.43078460769615	4.72937e-11\\
2.4327836081959	4.72937e-11\\
2.43478260869565	4.36557e-11\\
2.4367816091954	4.36557e-11\\
2.43878060969515	5.09317e-11\\
2.4407796101949	5.09317e-11\\
2.44277861069465	5.82077e-11\\
2.4447776111944	5.82077e-11\\
2.44677661169415	6.54836e-11\\
2.4487756121939	7.27596e-11\\
2.45077461269365	6.18456e-11\\
2.4527736131934	6.18456e-11\\
2.45477261369315	5.82077e-11\\
2.4567716141929	5.82077e-11\\
2.45877061469265	5.45697e-11\\
2.4607696151924	5.45697e-11\\
2.46276861569215	6.18456e-11\\
2.4647676161919	6.54836e-11\\
2.46676661669165	6.91216e-11\\
2.4687656171914	5.45697e-11\\
2.47076461769115	6.18456e-11\\
2.4727636181909	5.82077e-11\\
2.47476261869065	5.82077e-11\\
2.4767616191904	5.09317e-11\\
2.47876061969016	5.82077e-11\\
2.48075962018991	6.18456e-11\\
2.48275862068966	6.54836e-11\\
2.48475762118941	6.54836e-11\\
2.48675662168916	6.54836e-11\\
2.48875562218891	6.54836e-11\\
2.49075462268866	6.54836e-11\\
2.49275362318841	7.27596e-11\\
2.49475262368816	7.27596e-11\\
2.49675162418791	7.63976e-11\\
2.49875062468766	8.00355e-11\\
2.50074962518741	8.36735e-11\\
2.50274862568716	8.73115e-11\\
2.50474762618691	9.09495e-11\\
2.50674662668666	8.36735e-11\\
2.50874562718641	9.45874e-11\\
2.51074462768616	8.36735e-11\\
2.51274362818591	8.00355e-11\\
2.51474262868566	8.36735e-11\\
2.51674162918541	8.36735e-11\\
2.51874062968516	8.00355e-11\\
2.52073963018491	8.00355e-11\\
2.52273863068466	7.63976e-11\\
2.52473763118441	7.27596e-11\\
2.52673663168416	7.27596e-11\\
2.52873563218391	8.36735e-11\\
2.53073463268366	8.00355e-11\\
2.53273363318341	7.63976e-11\\
2.53473263368316	7.27596e-11\\
2.53673163418291	7.63976e-11\\
2.53873063468266	8.00355e-11\\
2.54072963518241	6.91216e-11\\
2.54272863568216	7.63976e-11\\
2.54472763618191	6.18456e-11\\
2.54672663668166	5.82077e-11\\
2.54872563718141	6.54836e-11\\
2.55072463768116	6.18456e-11\\
2.55272363818091	6.54836e-11\\
2.55472263868066	6.91216e-11\\
2.55672163918041	7.27596e-11\\
2.55872063968016	6.91216e-11\\
2.56071964017991	6.54836e-11\\
2.56271864067966	7.27596e-11\\
2.56471764117941	6.91216e-11\\
2.56671664167916	6.54836e-11\\
2.56871564217891	5.45697e-11\\
2.57071464267866	6.91216e-11\\
2.57271364317841	6.54836e-11\\
2.57471264367816	6.54836e-11\\
2.57671164417791	6.54836e-11\\
2.57871064467766	5.45697e-11\\
2.58070964517741	5.45697e-11\\
2.58270864567716	6.18456e-11\\
2.58470764617691	7.27596e-11\\
2.58670664667666	7.27596e-11\\
2.58870564717641	7.27596e-11\\
2.59070464767616	6.91216e-11\\
2.59270364817591	6.91216e-11\\
2.59470264867566	7.27596e-11\\
2.59670164917541	5.82077e-11\\
2.59870064967516	5.82077e-11\\
2.60069965017491	5.45697e-11\\
2.60269865067466	5.09317e-11\\
2.60469765117441	4.72937e-11\\
2.60669665167416	6.54836e-11\\
2.60869565217391	6.54836e-11\\
2.61069465267366	5.09317e-11\\
2.61269365317341	5.45697e-11\\
2.61469265367316	5.09317e-11\\
2.61669165417291	4.72937e-11\\
2.61869065467266	6.18456e-11\\
2.62068965517241	5.82077e-11\\
2.62268865567216	6.54836e-11\\
2.62468765617191	6.18456e-11\\
2.62668665667166	6.54836e-11\\
2.62868565717141	6.18456e-11\\
2.63068465767116	6.18456e-11\\
2.63268365817091	5.82077e-11\\
2.63468265867066	6.91216e-11\\
2.63668165917041	6.91216e-11\\
2.63868065967017	7.63976e-11\\
2.64067966016992	7.27596e-11\\
2.64267866066967	7.27596e-11\\
2.64467766116942	7.27596e-11\\
2.64667666166917	6.91216e-11\\
2.64867566216892	6.54836e-11\\
2.65067466266867	6.54836e-11\\
2.65267366316842	6.18456e-11\\
2.65467266366817	5.82077e-11\\
2.65667166416792	5.82077e-11\\
2.65867066466767	5.09317e-11\\
2.66066966516742	5.09317e-11\\
2.66266866566717	5.45697e-11\\
2.66466766616692	5.45697e-11\\
2.66666666666667	5.82077e-11\\
2.66866566716642	5.82077e-11\\
2.67066466766617	6.18456e-11\\
2.67266366816592	5.82077e-11\\
2.67466266866567	6.18456e-11\\
2.67666166916542	6.18456e-11\\
2.67866066966517	6.91216e-11\\
2.68065967016492	6.54836e-11\\
2.68265867066467	8.00355e-11\\
2.68465767116442	8.36735e-11\\
2.68665667166417	7.63976e-11\\
2.68865567216392	6.54836e-11\\
2.69065467266367	6.18456e-11\\
2.69265367316342	6.18456e-11\\
2.69465267366317	6.18456e-11\\
2.69665167416292	5.45697e-11\\
2.69865067466267	6.18456e-11\\
2.70064967516242	8.00355e-11\\
2.70264867566217	8.36735e-11\\
2.70464767616192	9.45874e-11\\
2.70664667666167	9.82254e-11\\
2.70864567716142	9.09495e-11\\
2.71064467766117	9.09495e-11\\
2.71264367816092	1.09139e-10\\
2.71464267866067	9.45874e-11\\
2.71664167916042	9.45874e-11\\
2.71864067966017	9.82254e-11\\
2.72063968015992	8.73115e-11\\
2.72263868065967	9.82254e-11\\
2.72463768115942	9.82254e-11\\
2.72663668165917	9.45874e-11\\
2.72863568215892	8.36735e-11\\
2.73063468265867	8.00355e-11\\
2.73263368315842	8.73115e-11\\
2.73463268365817	8.00355e-11\\
2.73663168415792	6.91216e-11\\
2.73863068465767	6.54836e-11\\
2.74062968515742	8.36735e-11\\
2.74262868565717	9.09495e-11\\
2.74462768615692	8.00355e-11\\
2.74662668665667	6.91216e-11\\
2.74862568715642	6.54836e-11\\
2.75062468765617	6.54836e-11\\
2.75262368815592	6.18456e-11\\
2.75462268865567	5.82077e-11\\
2.75662168915542	5.45697e-11\\
2.75862068965517	4.00178e-11\\
2.76061969015492	2.91038e-11\\
2.76261869065467	4.00178e-11\\
2.76461769115442	4.00178e-11\\
2.76661669165417	3.63798e-11\\
2.76861569215392	2.54659e-11\\
2.77061469265367	2.91038e-11\\
2.77261369315342	2.91038e-11\\
2.77461269365317	2.91038e-11\\
2.77661169415292	2.18279e-11\\
2.77861069465267	7.27596e-12\\
2.78060969515242	1.09139e-11\\
2.78260869565217	2.54659e-11\\
2.78460769615192	2.18279e-11\\
2.78660669665167	2.18279e-11\\
2.78860569715142	2.18279e-11\\
2.79060469765117	2.18279e-11\\
2.79260369815092	7.27596e-12\\
2.79460269865067	7.27596e-12\\
2.79660169915043	7.27596e-12\\
2.79860069965018	3.63798e-12\\
2.80059970014993	3.63798e-12\\
2.80259870064968	-7.27596e-12\\
2.80459770114943	-2.18279e-11\\
2.80659670164918	-2.91038e-11\\
2.80859570214893	-3.63798e-11\\
2.81059470264868	-2.54659e-11\\
2.81259370314843	-1.09139e-11\\
2.81459270364818	-1.09139e-11\\
2.81659170414793	-2.91038e-11\\
2.81859070464768	-2.54659e-11\\
2.82058970514743	-3.63798e-11\\
2.82258870564718	-2.54659e-11\\
2.82458770614693	-2.54659e-11\\
2.82658670664668	-7.27596e-12\\
2.82858570714643	-1.09139e-11\\
2.83058470764618	-2.54659e-11\\
2.83258370814593	-1.45519e-11\\
2.83458270864568	-1.45519e-11\\
2.83658170914543	0\\
2.83858070964518	0\\
2.84057971014493	0\\
2.84257871064468	-1.45519e-11\\
2.84457771114443	-1.45519e-11\\
2.84657671164418	-3.27418e-11\\
2.84857571214393	-2.18279e-11\\
2.85057471264368	-7.27596e-12\\
2.85257371314343	-1.81899e-11\\
2.85457271364318	-1.81899e-11\\
2.85657171414293	-1.81899e-11\\
2.85857071464268	-3.27418e-11\\
2.86056971514243	-1.45519e-11\\
2.86256871564218	-1.81899e-11\\
2.86456771614193	-1.45519e-11\\
2.86656671664168	-1.45519e-11\\
2.86856571714143	-2.91038e-11\\
2.87056471764118	-3.27418e-11\\
2.87256371814093	-2.54659e-11\\
2.87456271864068	-2.18279e-11\\
2.87656171914043	-3.27418e-11\\
2.87856071964018	-3.27418e-11\\
2.88055972013993	-2.54659e-11\\
2.88255872063968	-3.63798e-11\\
2.88455772113943	-2.54659e-11\\
2.88655672163918	-2.18279e-11\\
2.88855572213893	-2.54659e-11\\
2.89055472263868	-3.27418e-11\\
2.89255372313843	-4.72937e-11\\
2.89455272363818	-4.36557e-11\\
2.89655172413793	-4.00178e-11\\
2.89855072463768	-3.27418e-11\\
2.90054972513743	-4.36557e-11\\
2.90254872563718	-5.09317e-11\\
2.90454772613693	-4.00178e-11\\
2.90654672663668	-5.09317e-11\\
2.90854572713643	-5.45697e-11\\
2.91054472763618	-5.45697e-11\\
2.91254372813593	-5.09317e-11\\
2.91454272863568	-5.45697e-11\\
2.91654172913543	-4.36557e-11\\
2.91854072963518	-3.63798e-11\\
2.92053973013493	-2.54659e-11\\
2.92253873063468	-3.27418e-11\\
2.92453773113443	-2.18279e-11\\
2.92653673163418	-1.81899e-11\\
2.92853573213393	-3.63798e-11\\
2.93053473263368	-1.81899e-11\\
2.93253373313343	-2.18279e-11\\
2.93453273363318	-3.27418e-11\\
2.93653173413293	-2.18279e-11\\
2.93853073463268	-2.18279e-11\\
2.94052973513243	-2.91038e-11\\
2.94252873563218	-2.18279e-11\\
2.94452773613193	-1.09139e-11\\
2.94652673663168	-1.09139e-11\\
2.94852573713143	-1.09139e-11\\
2.95052473763118	-3.63798e-12\\
2.95252373813093	1.45519e-11\\
2.95452273863068	-1.09139e-11\\
2.95652173913043	3.63798e-12\\
2.95852073963018	0\\
2.96051974012994	0\\
2.96251874062969	1.45519e-11\\
2.96451774112944	7.27596e-12\\
2.96651674162919	3.63798e-12\\
2.96851574212894	0\\
2.97051474262869	0\\
2.97251374312844	7.27596e-12\\
2.97451274362819	7.27596e-12\\
2.97651174412794	7.27596e-12\\
2.97851074462769	3.63798e-12\\
2.98050974512744	3.63798e-12\\
2.98250874562719	1.81899e-11\\
2.98450774612694	3.27418e-11\\
2.98650674662669	4.72937e-11\\
2.98850574712644	4.36557e-11\\
2.99050474762619	4.72937e-11\\
2.99250374812594	5.09317e-11\\
2.99450274862569	5.09317e-11\\
2.99650174912544	5.09317e-11\\
2.99850074962519	5.09317e-11\\
3.00049975012494	5.09317e-11\\
3.00249875062469	6.91216e-11\\
3.00449775112444	7.63976e-11\\
3.00649675162419	8.36735e-11\\
3.00849575212394	7.27596e-11\\
3.01049475262369	6.91216e-11\\
3.01249375312344	8.00355e-11\\
3.01449275362319	6.91216e-11\\
3.01649175412294	7.63976e-11\\
3.01849075462269	6.54836e-11\\
3.02048975512244	6.91216e-11\\
3.02248875562219	6.91216e-11\\
3.02448775612194	7.27596e-11\\
3.02648675662169	8.00355e-11\\
3.02848575712144	8.00355e-11\\
3.03048475762119	7.27596e-11\\
3.03248375812094	8.36735e-11\\
3.03448275862069	8.36735e-11\\
3.03648175912044	9.82254e-11\\
3.03848075962019	8.73115e-11\\
3.04047976011994	1.01863e-10\\
3.04247876061969	9.45874e-11\\
3.04447776111944	8.36735e-11\\
3.04647676161919	9.45874e-11\\
3.04847576211894	8.00355e-11\\
3.05047476261869	8.00355e-11\\
3.05247376311844	7.63976e-11\\
3.05447276361819	6.91216e-11\\
3.05647176411794	6.54836e-11\\
3.05847076461769	4.72937e-11\\
3.06046976511744	4.72937e-11\\
3.06246876561719	6.54836e-11\\
3.06446776611694	4.72937e-11\\
3.06646676661669	6.54836e-11\\
3.06846576711644	6.91216e-11\\
3.07046476761619	6.54836e-11\\
3.07246376811594	5.09317e-11\\
3.07446276861569	6.54836e-11\\
3.07646176911544	4.72937e-11\\
3.07846076961519	4.00178e-11\\
3.08045977011494	2.91038e-11\\
3.08245877061469	4.00178e-11\\
3.08445777111444	3.63798e-11\\
3.08645677161419	4.36557e-11\\
3.08845577211394	4.36557e-11\\
3.09045477261369	3.27418e-11\\
3.09245377311344	2.91038e-11\\
3.09445277361319	4.00178e-11\\
3.09645177411294	3.27418e-11\\
3.09845077461269	4.00178e-11\\
3.10044977511244	4.00178e-11\\
3.10244877561219	2.18279e-11\\
3.10444777611194	3.63798e-11\\
3.10644677661169	3.27418e-11\\
3.10844577711144	5.45697e-11\\
3.11044477761119	6.54836e-11\\
3.11244377811094	6.54836e-11\\
3.11444277861069	8.36735e-11\\
3.11644177911044	9.09495e-11\\
3.11844077961019	1.05501e-10\\
3.12043978010994	9.09495e-11\\
3.12243878060969	8.73115e-11\\
3.12443778110945	8.36735e-11\\
3.1264367816092	8.36735e-11\\
3.12843578210895	1.01863e-10\\
3.1304347826087	1.01863e-10\\
3.13243378310845	1.09139e-10\\
3.1344327836082	1.12777e-10\\
3.13643178410795	1.12777e-10\\
3.1384307846077	1.01863e-10\\
3.14042978510745	8.00355e-11\\
3.1424287856072	7.63976e-11\\
3.14442778610695	8.73115e-11\\
3.1464267866067	8.00355e-11\\
3.14842578710645	6.18456e-11\\
3.1504247876062	8.00355e-11\\
3.15242378810595	6.54836e-11\\
3.1544227886057	6.18456e-11\\
3.15642178910545	3.63798e-11\\
3.1584207896052	3.27418e-11\\
3.16041979010495	1.45519e-11\\
3.1624187906047	7.27596e-12\\
3.16441779110445	1.45519e-11\\
3.1664167916042	1.09139e-11\\
3.16841579210395	1.09139e-11\\
3.1704147926037	1.45519e-11\\
3.17241379310345	2.91038e-11\\
3.1744127936032	2.91038e-11\\
3.17641179410295	2.91038e-11\\
3.1784107946027	3.27418e-11\\
3.18040979510245	3.63798e-11\\
3.1824087956022	2.91038e-11\\
3.18440779610195	3.27418e-11\\
3.1864067966017	4.00178e-11\\
3.18840579710145	5.82077e-11\\
3.1904047976012	7.27596e-11\\
3.19240379810095	5.09317e-11\\
3.1944027986007	5.09317e-11\\
3.19640179910045	4.72937e-11\\
3.1984007996002	3.27418e-11\\
3.20039980009995	3.63798e-11\\
3.2023988005997	2.54659e-11\\
3.20439780109945	2.91038e-11\\
3.2063968015992	3.63798e-11\\
3.20839580209895	5.09317e-11\\
3.2103948025987	6.18456e-11\\
3.21239380309845	6.18456e-11\\
3.2143928035982	6.18456e-11\\
3.21639180409795	7.27596e-11\\
3.2183908045977	7.63976e-11\\
3.22038980509745	8.36735e-11\\
3.2223888055972	8.73115e-11\\
3.22438780609695	1.09139e-10\\
3.2263868065967	1.09139e-10\\
3.22838580709645	1.20053e-10\\
3.2303848075962	1.23691e-10\\
3.23238380809595	1.20053e-10\\
3.2343828085957	1.30967e-10\\
3.23638180909545	1.20053e-10\\
3.2383808095952	1.23691e-10\\
3.24037981009495	1.12777e-10\\
3.2423788105947	1.23691e-10\\
3.24437781109445	1.23691e-10\\
3.2463768115942	1.27329e-10\\
3.24837581209395	1.34605e-10\\
3.2503748125937	1.30967e-10\\
3.25237381309345	1.49157e-10\\
3.2543728135932	1.30967e-10\\
3.25637181409295	1.45519e-10\\
3.2583708145927	1.49157e-10\\
3.26036981509245	1.45519e-10\\
3.2623688155922	1.52795e-10\\
3.26436781609195	1.74623e-10\\
3.2663668165917	1.56433e-10\\
3.26836581709145	1.60071e-10\\
3.2703648175912	1.70985e-10\\
3.27236381809095	1.67347e-10\\
3.2743628185907	1.78261e-10\\
3.27636181909045	1.70985e-10\\
3.2783608195902	1.78261e-10\\
3.28035982008995	1.81899e-10\\
3.2823588205897	1.67347e-10\\
3.28435782108946	1.78261e-10\\
3.28635682158921	1.70985e-10\\
3.28835582208896	1.81899e-10\\
3.29035482258871	1.81899e-10\\
3.29235382308846	1.70985e-10\\
3.29435282358821	1.70985e-10\\
3.29635182408796	1.70985e-10\\
3.29835082458771	1.70985e-10\\
3.30034982508746	1.81899e-10\\
3.30234882558721	1.81899e-10\\
3.30434782608696	1.81899e-10\\
3.30634682658671	1.70985e-10\\
3.30834582708646	1.67347e-10\\
3.31034482758621	1.74623e-10\\
3.31234382808596	1.74623e-10\\
3.31434282858571	1.85537e-10\\
3.31634182908546	1.81899e-10\\
3.31834082958521	1.81899e-10\\
3.32033983008496	1.78261e-10\\
3.32233883058471	1.78261e-10\\
3.32433783108446	1.81899e-10\\
3.32633683158421	1.70985e-10\\
3.32833583208396	1.67347e-10\\
3.33033483258371	1.56433e-10\\
3.33233383308346	1.52795e-10\\
3.33433283358321	1.67347e-10\\
3.33633183408296	1.49157e-10\\
3.33833083458271	1.52795e-10\\
3.34032983508246	1.41881e-10\\
3.34232883558221	1.38243e-10\\
3.34432783608196	1.49157e-10\\
3.34632683658171	1.52795e-10\\
3.34832583708146	1.49157e-10\\
3.35032483758121	1.74623e-10\\
3.35232383808096	1.74623e-10\\
3.35432283858071	1.89175e-10\\
3.35632183908046	1.78261e-10\\
3.35832083958021	1.89175e-10\\
3.36031984007996	1.74623e-10\\
3.36231884057971	1.63709e-10\\
3.36431784107946	1.70985e-10\\
3.36631684157921	1.70985e-10\\
3.36831584207896	1.78261e-10\\
3.37031484257871	1.96451e-10\\
3.37231384307846	1.78261e-10\\
3.37431284357821	1.81899e-10\\
3.37631184407796	1.78261e-10\\
3.37831084457771	1.67347e-10\\
3.38030984507746	1.78261e-10\\
3.38230884557721	1.92813e-10\\
3.38430784607696	1.96451e-10\\
3.38630684657671	1.92813e-10\\
3.38830584707646	1.78261e-10\\
3.39030484757621	1.92813e-10\\
3.39230384807596	2.03727e-10\\
3.39430284857571	2.03727e-10\\
3.39630184907546	2.18279e-10\\
3.39830084957521	2.07365e-10\\
3.40029985007496	1.96451e-10\\
3.40229885057471	2.07365e-10\\
3.40429785107446	2.03727e-10\\
3.40629685157421	2.03727e-10\\
3.40829585207396	1.96451e-10\\
3.41029485257371	1.96451e-10\\
3.41229385307346	1.92813e-10\\
3.41429285357321	2.00089e-10\\
3.41629185407296	2.03727e-10\\
3.41829085457271	2.07365e-10\\
3.42028985507246	2.00089e-10\\
3.42228885557221	2.03727e-10\\
3.42428785607196	2.03727e-10\\
3.42628685657171	2.11003e-10\\
3.42828585707146	2.11003e-10\\
3.43028485757121	2.11003e-10\\
3.43228385807096	2.21917e-10\\
3.43428285857071	2.25555e-10\\
3.43628185907046	2.29193e-10\\
3.43828085957021	2.40107e-10\\
3.44027986006996	2.40107e-10\\
3.44227886056971	2.51021e-10\\
3.44427786106947	2.40107e-10\\
3.44627686156922	2.36469e-10\\
3.44827586206897	2.14641e-10\\
3.45027486256872	2.14641e-10\\
3.45227386306847	2.11003e-10\\
3.45427286356822	2.00089e-10\\
3.45627186406797	1.96451e-10\\
3.45827086456772	1.89175e-10\\
3.46026986506747	1.96451e-10\\
3.46226886556722	1.96451e-10\\
3.46426786606697	1.89175e-10\\
3.46626686656672	2.07365e-10\\
3.46826586706647	2.18279e-10\\
3.47026486756622	2.18279e-10\\
3.47226386806597	2.18279e-10\\
3.47426286856572	2.14641e-10\\
3.47626186906547	2.03727e-10\\
3.47826086956522	2.14641e-10\\
3.48025987006497	2.25555e-10\\
3.48225887056472	2.14641e-10\\
3.48425787106447	2.21917e-10\\
3.48625687156422	2.29193e-10\\
3.48825587206397	2.36469e-10\\
3.49025487256372	2.21917e-10\\
3.49225387306347	2.07365e-10\\
3.49425287356322	1.96451e-10\\
3.49625187406297	2.00089e-10\\
3.49825087456272	2.07365e-10\\
3.50024987506247	2.14641e-10\\
3.50224887556222	2.14641e-10\\
3.50424787606197	2.25555e-10\\
3.50624687656172	2.11003e-10\\
3.50824587706147	2.03727e-10\\
3.51024487756122	2.11003e-10\\
3.51224387806097	2.14641e-10\\
3.51424287856072	2.29193e-10\\
3.51624187906047	2.25555e-10\\
3.51824087956022	2.25555e-10\\
3.52023988005997	2.25555e-10\\
3.52223888055972	2.21917e-10\\
3.52423788105947	2.25555e-10\\
3.52623688155922	2.36469e-10\\
3.52823588205897	2.29193e-10\\
3.53023488255872	2.29193e-10\\
3.53223388305847	2.11003e-10\\
3.53423288355822	2.14641e-10\\
3.53623188405797	2.14641e-10\\
3.53823088455772	2.11003e-10\\
3.54022988505747	2.11003e-10\\
3.54222888555722	2.21917e-10\\
3.54422788605697	2.25555e-10\\
3.54622688655672	2.25555e-10\\
3.54822588705647	2.25555e-10\\
3.55022488755622	2.18279e-10\\
3.55222388805597	2.14641e-10\\
3.55422288855572	2.07365e-10\\
3.55622188905547	2.18279e-10\\
3.55822088955522	2.21917e-10\\
3.56021989005497	2.25555e-10\\
3.56221889055472	2.40107e-10\\
3.56421789105447	2.43745e-10\\
3.56621689155422	2.43745e-10\\
3.56821589205397	2.43745e-10\\
3.57021489255372	2.43745e-10\\
3.57221389305347	2.47383e-10\\
3.57421289355322	2.43745e-10\\
3.57621189405297	2.43745e-10\\
3.57821089455272	2.40107e-10\\
3.58020989505247	2.40107e-10\\
3.58220889555222	2.47383e-10\\
3.58420789605197	2.25555e-10\\
3.58620689655172	2.36469e-10\\
3.58820589705147	2.32831e-10\\
3.59020489755122	2.40107e-10\\
3.59220389805097	2.36469e-10\\
3.59420289855072	2.47383e-10\\
3.59620189905047	2.47383e-10\\
3.59820089955022	2.36469e-10\\
3.60019990004997	2.32831e-10\\
3.60219890054973	2.40107e-10\\
3.60419790104948	2.58296e-10\\
3.60619690154923	2.43745e-10\\
3.60819590204898	2.40107e-10\\
3.61019490254873	2.40107e-10\\
3.61219390304848	2.51021e-10\\
3.61419290354823	2.43745e-10\\
3.61619190404798	2.47383e-10\\
3.61819090454773	2.43745e-10\\
3.62018990504748	2.43745e-10\\
3.62218890554723	2.58296e-10\\
3.62418790604698	2.61934e-10\\
3.62618690654673	2.61934e-10\\
3.62818590704648	2.61934e-10\\
3.63018490754623	2.58296e-10\\
3.63218390804598	2.54659e-10\\
3.63418290854573	2.54659e-10\\
3.63618190904548	2.51021e-10\\
3.63818090954523	2.51021e-10\\
3.64017991004498	2.47383e-10\\
3.64217891054473	2.47383e-10\\
3.64417791104448	2.51021e-10\\
3.64617691154423	2.47383e-10\\
3.64817591204398	2.51021e-10\\
3.65017491254373	2.47383e-10\\
3.65217391304348	2.54659e-10\\
3.65417291354323	2.58296e-10\\
3.65617191404298	2.61934e-10\\
3.65817091454273	2.61934e-10\\
3.66016991504248	2.61934e-10\\
3.66216891554223	2.58296e-10\\
3.66416791604198	2.61934e-10\\
3.66616691654173	2.58296e-10\\
3.66816591704148	2.51021e-10\\
3.67016491754123	2.51021e-10\\
3.67216391804098	2.47383e-10\\
3.67416291854073	2.51021e-10\\
3.67616191904048	2.61934e-10\\
3.67816091954023	2.54659e-10\\
3.68015992003998	2.61934e-10\\
3.68215892053973	2.61934e-10\\
3.68415792103948	2.72848e-10\\
3.68615692153923	2.76486e-10\\
3.68815592203898	2.76486e-10\\
3.69015492253873	2.76486e-10\\
3.69215392303848	2.83762e-10\\
3.69415292353823	2.76486e-10\\
3.69615192403798	2.80124e-10\\
3.69815092453773	2.76486e-10\\
3.70014992503748	2.91038e-10\\
3.70214892553723	2.80124e-10\\
3.70414792603698	2.72848e-10\\
3.70614692653673	2.61934e-10\\
3.70814592703648	2.58296e-10\\
3.71014492753623	2.6921e-10\\
3.71214392803598	2.65572e-10\\
3.71414292853573	2.6921e-10\\
3.71614192903548	2.6921e-10\\
3.71814092953523	2.65572e-10\\
3.72013993003498	2.72848e-10\\
3.72213893053473	2.72848e-10\\
3.72413793103448	2.6921e-10\\
3.72613693153423	2.65572e-10\\
3.72813593203398	2.61934e-10\\
3.73013493253373	2.58296e-10\\
3.73213393303348	2.51021e-10\\
3.73413293353323	2.43745e-10\\
3.73613193403298	2.43745e-10\\
3.73813093453273	2.43745e-10\\
3.74012993503248	2.47383e-10\\
3.74212893553223	2.61934e-10\\
3.74412793603198	2.61934e-10\\
3.74612693653173	2.51021e-10\\
3.74812593703148	2.58296e-10\\
3.75012493753123	2.54659e-10\\
3.75212393803098	2.54659e-10\\
3.75412293853073	2.54659e-10\\
3.75612193903048	2.61934e-10\\
3.75812093953023	2.61934e-10\\
3.76011994002998	2.61934e-10\\
3.76211894052974	2.61934e-10\\
3.76411794102949	2.54659e-10\\
3.76611694152924	2.58296e-10\\
3.76811594202899	2.58296e-10\\
3.77011494252874	2.61934e-10\\
3.77211394302849	2.54659e-10\\
3.77411294352824	2.61934e-10\\
3.77611194402799	2.58296e-10\\
3.77811094452774	2.47383e-10\\
3.78010994502749	2.43745e-10\\
3.78210894552724	2.40107e-10\\
3.78410794602699	2.43745e-10\\
3.78610694652674	2.36469e-10\\
3.78810594702649	2.43745e-10\\
3.79010494752624	2.51021e-10\\
3.79210394802599	2.51021e-10\\
3.79410294852574	2.54659e-10\\
3.79610194902549	2.51021e-10\\
3.79810094952524	2.54659e-10\\
3.80009995002499	2.51021e-10\\
3.80209895052474	2.40107e-10\\
3.80409795102449	2.40107e-10\\
3.80609695152424	2.32831e-10\\
3.80809595202399	2.43745e-10\\
3.81009495252374	2.40107e-10\\
3.81209395302349	2.40107e-10\\
3.81409295352324	2.51021e-10\\
3.81609195402299	2.47383e-10\\
3.81809095452274	2.40107e-10\\
3.82008995502249	2.43745e-10\\
3.82208895552224	2.47383e-10\\
3.82408795602199	2.40107e-10\\
3.82608695652174	2.43745e-10\\
3.82808595702149	2.43745e-10\\
3.83008495752124	2.36469e-10\\
3.83208395802099	2.36469e-10\\
3.83408295852074	2.40107e-10\\
3.83608195902049	2.32831e-10\\
3.83808095952024	2.36469e-10\\
3.84007996001999	2.40107e-10\\
3.84207896051974	2.36469e-10\\
3.84407796101949	2.36469e-10\\
3.84607696151924	2.40107e-10\\
3.84807596201899	2.32831e-10\\
3.85007496251874	2.32831e-10\\
3.85207396301849	2.43745e-10\\
3.85407296351824	2.40107e-10\\
3.85607196401799	2.40107e-10\\
3.85807096451774	2.40107e-10\\
3.86006996501749	2.36469e-10\\
3.86206896551724	2.43745e-10\\
3.86406796601699	2.47383e-10\\
3.86606696651674	2.47383e-10\\
3.86806596701649	2.47383e-10\\
3.87006496751624	2.43745e-10\\
3.87206396801599	2.43745e-10\\
3.87406296851574	2.43745e-10\\
3.87606196901549	2.43745e-10\\
3.87806096951524	2.43745e-10\\
3.88005997001499	2.40107e-10\\
3.88205897051474	2.36469e-10\\
3.88405797101449	2.47383e-10\\
3.88605697151424	2.47383e-10\\
3.88805597201399	2.47383e-10\\
3.89005497251374	2.47383e-10\\
3.89205397301349	2.47383e-10\\
3.89405297351324	2.51021e-10\\
3.89605197401299	2.47383e-10\\
3.89805097451274	2.47383e-10\\
3.90004997501249	2.40107e-10\\
3.90204897551224	2.40107e-10\\
3.90404797601199	2.43745e-10\\
3.90604697651174	2.40107e-10\\
3.90804597701149	2.32831e-10\\
3.91004497751124	2.32831e-10\\
3.91204397801099	2.29193e-10\\
3.91404297851074	2.25555e-10\\
3.91604197901049	2.25555e-10\\
3.91804097951024	2.29193e-10\\
3.92003998000999	2.32831e-10\\
3.92203898050975	2.18279e-10\\
3.9240379810095	2.18279e-10\\
3.92603698150925	2.14641e-10\\
3.928035982009	2.11003e-10\\
3.93003498250875	2.00089e-10\\
3.9320339830085	2.03727e-10\\
3.93403298350825	2.03727e-10\\
3.936031984008	2.07365e-10\\
3.93803098450775	2.07365e-10\\
3.9400299850075	2.07365e-10\\
3.94202898550725	2.11003e-10\\
3.944027986007	2.11003e-10\\
3.94602698650675	2.14641e-10\\
3.9480259870065	2.14641e-10\\
3.95002498750625	2.21917e-10\\
3.952023988006	2.11003e-10\\
3.95402298850575	2.21917e-10\\
3.9560219890055	2.18279e-10\\
3.95802098950525	2.18279e-10\\
3.960019990005	2.18279e-10\\
3.96201899050475	2.07365e-10\\
3.9640179910045	2.14641e-10\\
3.96601699150425	2.25555e-10\\
3.968015992004	2.21917e-10\\
3.97001499250375	2.21917e-10\\
3.9720139930035	2.14641e-10\\
3.97401299350325	2.11003e-10\\
3.976011994003	2.14641e-10\\
3.97801099450275	2.18279e-10\\
3.9800099950025	2.18279e-10\\
3.98200899550225	2.11003e-10\\
3.984007996002	2.14641e-10\\
3.98600699650175	2.18279e-10\\
3.9880059970015	2.18279e-10\\
3.99000499750125	2.18279e-10\\
3.992003998001	2.18279e-10\\
3.99400299850075	2.18279e-10\\
3.9960019990005	2.18279e-10\\
3.99800099950025	2.14641e-10\\
4	2.18279e-10\\
};
\addlegendentry{Energy Diff};

\end{axis}
\end{tikzpicture}%
	\caption{Showing the angle evolution for the spinning top problem when going backwards in time.}
	\label{fig:backwardData}
\end{figure}
\fi
\iftikz
\begin{figure}[H]
	\centering
	\setlength\figureheight{7cm} 
	\setlength\figurewidth{14cm}
	% This file was created by matlab2tikz.
% Minimal pgfplots version: 1.3
%
%The latest updates can be retrieved from
%  http://www.mathworks.com/matlabcentral/fileexchange/22022-matlab2tikz
%where you can also make suggestions and rate matlab2tikz.
%
\definecolor{mycolor1}{rgb}{0.00000,0.44700,0.74100}%
\definecolor{mycolor2}{rgb}{0.85000,0.32500,0.09800}%
\definecolor{mycolor3}{rgb}{0.92900,0.69400,0.12500}%
%
\begin{tikzpicture}

\begin{axis}[%
width=0.95092\figurewidth,
height=\figureheight,
at={(0\figurewidth,0\figureheight)},
scale only axis,
xmin=0,
xmax=4,
xlabel={Time (s)},
ymin=-6e-07,
ymax=3e-07,
ylabel={Degrees},
title style={font=\bfseries},
title={Top Spin [0,4] (s) Errors},
legend style={at={(0.03,0.97)},anchor=north west,legend cell align=left,align=left,draw=white!15!black},
title style={font=\small},ticklabel style={font=\tiny}
]
\addplot [color=mycolor1,solid]
  table[row sep=crcr]{%
0	0\\
0.00200100050025012	4.16881799906007e-10\\
0.00400200100050025	6.25320408027829e-10\\
0.00600300150075038	4.16881799906007e-10\\
0.0080040020010005	2.08440899953003e-10\\
0.0100050025012506	6.25320408027829e-10\\
0.0120060030015008	4.16881799906007e-10\\
0.0140070035017509	1.04220449976502e-09\\
0.016008004002001	1.66752490779285e-09\\
0.0180090045022511	1.4590885915022e-09\\
0.0200100050025013	1.4590885915022e-09\\
0.0220110055027514	8.33762453896423e-10\\
0.0240120060030015	8.33762453896423e-10\\
0.0260130065032516	6.25320408027829e-10\\
0.0280140070035018	2.08440899953003e-10\\
0.0300150075037519	0\\
0.032016008004002	-2.08440899953003e-10\\
0.0340170085042521	0\\
0.0360180090045022	-2.08440899953003e-10\\
0.0380190095047524	0\\
0.0400200100050025	-2.08440899953003e-10\\
0.0420210105052526	-2.08440899953003e-10\\
0.0440220110055028	-2.08440899953003e-10\\
0.0460230115057529	0\\
0.048024012006003	-2.08440899953003e-10\\
0.0500250125062531	-2.08440899953003e-10\\
0.0520260130065033	-2.08440899953003e-10\\
0.0540270135067534	-4.16881799906007e-10\\
0.0560280140070035	-2.08440899953003e-10\\
0.0580290145072536	-2.08440899953003e-10\\
0.0600300150075038	-4.16881799906007e-10\\
0.0620310155077539	-4.16881799906007e-10\\
0.064032016008004	-8.33762453896423e-10\\
0.0660330165082541	-8.33762453896423e-10\\
0.0680340170085043	-4.16881799906007e-10\\
0.0700350175087544	-2.08440899953003e-10\\
0.0720360180090045	0\\
0.0740370185092546	-2.08440899953003e-10\\
0.0760380190095048	0\\
0.0780390195097549	4.16881799906007e-10\\
0.080040020010005	-2.08440899953003e-10\\
0.0820410205102551	-6.25320408027829e-10\\
0.0840420210105053	-4.16881799906007e-10\\
0.0860430215107554	-8.33762453896423e-10\\
0.0880440220110055	-1.04220449976502e-09\\
0.0900450225112556	-1.25064654563361e-09\\
0.0920460230115058	-1.04220449976502e-09\\
0.0940470235117559	-1.04220449976502e-09\\
0.096048024012006	-8.33762453896423e-10\\
0.0980490245122561	-6.25320408027829e-10\\
0.100050025012506	-6.25320408027829e-10\\
0.102051025512756	-1.04220449976502e-09\\
0.104052026013007	-1.4590885915022e-09\\
0.106053026513257	-1.25064654563361e-09\\
0.108054027013507	-1.25064654563361e-09\\
0.110055027513757	-1.25064654563361e-09\\
0.112056028014007	-1.04220449976502e-09\\
0.114057028514257	-1.25064654563361e-09\\
0.116058029014507	-8.33762453896423e-10\\
0.118059029514757	-1.25064654563361e-09\\
0.120060030015008	-1.4590885915022e-09\\
0.122061030515258	-1.25064654563361e-09\\
0.124062031015508	-1.66752490779285e-09\\
0.126063031515758	-1.87596695366144e-09\\
0.128064032016008	-1.66752490779285e-09\\
0.130065032516258	-2.08440899953003e-09\\
0.132066033016508	-2.08440899953003e-09\\
0.134067033516758	-2.29285104539863e-09\\
0.136068034017009	-2.70972940755786e-09\\
0.138069034517259	-2.50128736168927e-09\\
0.140070035017509	-2.29285104539863e-09\\
0.142071035517759	-2.70972940755786e-09\\
0.144072036018009	-2.08440899953003e-09\\
0.146073036518259	-2.70972940755786e-09\\
0.148074037018509	-2.29285104539863e-09\\
0.150075037518759	-2.70972940755786e-09\\
0.15207603801901	-2.70972940755786e-09\\
0.15407703851926	-2.08440899953003e-09\\
0.15607803901951	-2.29285104539863e-09\\
0.15807903951976	-2.70972940755786e-09\\
0.16008004002001	-2.50128736168927e-09\\
0.16208104052026	-2.29285104539863e-09\\
0.16408204102051	-2.50128736168927e-09\\
0.16608304152076	-2.50128736168927e-09\\
0.168084042021011	-2.50128736168927e-09\\
0.170085042521261	-1.66752490779285e-09\\
0.172086043021511	-2.08440899953003e-09\\
0.174087043521761	-2.08440899953003e-09\\
0.176088044022011	-2.08440899953003e-09\\
0.178089044522261	-2.29285104539863e-09\\
0.180090045022511	-2.70972940755786e-09\\
0.182091045522761	-2.91817145342646e-09\\
0.184092046023012	-2.70972940755786e-09\\
0.186093046523262	-2.29285104539863e-09\\
0.188094047023512	-3.12661349929505e-09\\
0.190095047523762	-2.91817145342646e-09\\
0.192096048024012	-2.91817145342646e-09\\
0.194097048524262	-3.12661349929505e-09\\
0.196098049024512	-2.70972940755786e-09\\
0.198099049524762	-2.91817145342646e-09\\
0.200100050025012	-3.33505554516364e-09\\
0.202101050525263	-2.91817145342646e-09\\
0.204102051025513	-2.91817145342646e-09\\
0.206103051525763	-2.91817145342646e-09\\
0.208104052026013	-2.70972940755786e-09\\
0.210105052526263	-2.29285104539863e-09\\
0.212106053026513	-2.70972940755786e-09\\
0.214107053526763	-2.70972940755786e-09\\
0.216108054027013	-2.91817145342646e-09\\
0.218109054527264	-3.12661349929505e-09\\
0.220110055027514	-2.50128736168927e-09\\
0.222111055527764	-2.70972940755786e-09\\
0.224112056028014	-2.29285104539863e-09\\
0.226113056528264	-2.70972940755786e-09\\
0.228114057028514	-3.12661349929505e-09\\
0.230115057528764	-2.91817145342646e-09\\
0.232116058029014	-2.91817145342646e-09\\
0.234117058529265	-3.12661349929505e-09\\
0.236118059029515	-3.12661349929505e-09\\
0.238119059529765	-3.12661349929505e-09\\
0.240120060030015	-3.33505554516364e-09\\
0.242121060530265	-2.91817145342646e-09\\
0.244122061030515	-3.54349186145428e-09\\
0.246123061530765	-3.12661349929505e-09\\
0.248124062031016	-3.12661349929505e-09\\
0.250125062531266	-3.12661349929505e-09\\
0.252126063031516	-2.91817145342646e-09\\
0.254127063531766	-3.33505554516364e-09\\
0.256128064032016	-4.16881799906006e-09\\
0.258129064532266	-3.75193390732288e-09\\
0.260130065032516	-3.96037595319147e-09\\
0.262131065532766	-3.54349186145428e-09\\
0.264132066033017	-3.75193390732288e-09\\
0.266133066533267	-3.33505554516364e-09\\
0.268134067033517	-3.33505554516364e-09\\
0.270135067533767	-3.12661349929505e-09\\
0.272136068034017	-3.33505554516364e-09\\
0.274137068534267	-3.12661349929505e-09\\
0.276138069034517	-2.91817145342646e-09\\
0.278139069534767	-3.12661349929505e-09\\
0.280140070035018	-3.12661349929505e-09\\
0.282141070535268	-3.33505554516364e-09\\
0.284142071035518	-3.75193390732288e-09\\
0.286143071535768	-3.54349186145428e-09\\
0.288144072036018	-3.54349186145428e-09\\
0.290145072536268	-3.75193390732288e-09\\
0.292146073036518	-3.12661349929505e-09\\
0.294147073536768	-3.33505554516364e-09\\
0.296148074037018	-3.33505554516364e-09\\
0.298149074537269	-3.33505554516364e-09\\
0.300150075037519	-3.54349186145428e-09\\
0.302151075537769	-3.33505554516364e-09\\
0.304152076038019	-3.54349186145428e-09\\
0.306153076538269	-3.54349186145428e-09\\
0.308154077038519	-3.75193390732288e-09\\
0.310155077538769	-3.96037595319147e-09\\
0.31215607803902	-3.75193390732288e-09\\
0.31415707853927	-3.96037595319147e-09\\
0.31615807903952	-3.96037595319147e-09\\
0.31815907953977	-3.54349186145428e-09\\
0.32016008004002	-3.75193390732288e-09\\
0.32216108054027	-3.75193390732288e-09\\
0.32416208104052	-3.75193390732288e-09\\
0.32616308154077	-4.16881799906006e-09\\
0.32816408204102	-3.75193390732288e-09\\
0.330165082541271	-3.75193390732288e-09\\
0.332166083041521	-4.16881799906006e-09\\
0.334167083541771	-3.96037595319147e-09\\
0.336168084042021	-4.16881799906006e-09\\
0.338169084542271	-4.37726004492866e-09\\
0.340170085042521	-3.96037595319147e-09\\
0.342171085542771	-3.54349186145428e-09\\
0.344172086043022	-3.96037595319147e-09\\
0.346173086543272	-3.96037595319147e-09\\
0.348174087043522	-4.16881799906006e-09\\
0.350175087543772	-3.75193390732288e-09\\
0.352176088044022	-3.96037595319147e-09\\
0.354177088544272	-3.75193390732288e-09\\
0.356178089044522	-3.75193390732288e-09\\
0.358179089544772	-3.96037595319147e-09\\
0.360180090045022	-3.96037595319147e-09\\
0.362181090545273	-3.75193390732288e-09\\
0.364182091045523	-3.12661349929505e-09\\
0.366183091545773	-2.91817145342646e-09\\
0.368184092046023	-3.33505554516364e-09\\
0.370185092546273	-3.33505554516364e-09\\
0.372186093046523	-3.96037595319147e-09\\
0.374187093546773	-3.54349186145428e-09\\
0.376188094047024	-3.75193390732288e-09\\
0.378189094547274	-3.54349186145428e-09\\
0.380190095047524	-3.54349186145428e-09\\
0.382191095547774	-3.12661349929505e-09\\
0.384192096048024	-2.70972940755786e-09\\
0.386193096548274	-2.29285104539863e-09\\
0.388194097048524	-1.87596695366144e-09\\
0.390195097548774	-1.87596695366144e-09\\
0.392196098049024	-1.66752490779285e-09\\
0.394197098549275	-1.4590885915022e-09\\
0.396198099049525	-1.66752490779285e-09\\
0.398199099549775	-1.66752490779285e-09\\
0.400200100050025	-2.08440899953003e-09\\
0.402201100550275	-1.87596695366144e-09\\
0.404202101050525	-2.08440899953003e-09\\
0.406203101550775	-1.66752490779285e-09\\
0.408204102051026	-1.66752490779285e-09\\
0.410205102551276	-1.87596695366144e-09\\
0.412206103051526	-2.08440899953003e-09\\
0.414207103551776	-1.66752490779285e-09\\
0.416208104052026	-1.87596695366144e-09\\
0.418209104552276	-1.87596695366144e-09\\
0.420210105052526	-1.66752490779285e-09\\
0.422211105552776	-1.66752490779285e-09\\
0.424212106053027	-1.04220449976502e-09\\
0.426213106553277	-1.04220449976502e-09\\
0.428214107053527	-6.25320408027829e-10\\
0.430215107553777	-8.33762453896423e-10\\
0.432216108054027	-8.33762453896423e-10\\
0.434217108554277	-8.33762453896423e-10\\
0.436218109054527	-8.33762453896423e-10\\
0.438219109554777	-4.16881799906007e-10\\
0.440220110055028	-2.08440899953003e-10\\
0.442221110555278	-4.16881799906007e-10\\
0.444222111055528	-6.25320408027829e-10\\
0.446223111555778	-6.25320408027829e-10\\
0.448224112056028	-1.25064654563361e-09\\
0.450225112556278	-6.25320408027829e-10\\
0.452226113056528	-6.25320408027829e-10\\
0.454227113556778	-8.33762453896423e-10\\
0.456228114057029	-2.08440899953003e-10\\
0.458229114557279	0\\
0.460230115057529	-2.08440899953003e-10\\
0.462231115557779	-6.25320408027829e-10\\
0.464232116058029	-1.25064654563361e-09\\
0.466233116558279	-1.87596695366144e-09\\
0.468234117058529	-1.66752490779285e-09\\
0.470235117558779	-1.87596695366144e-09\\
0.47223611805903	-1.87596695366144e-09\\
0.47423711855928	-1.87596695366144e-09\\
0.47623811905953	-1.66752490779285e-09\\
0.47823911955978	-2.08440899953003e-09\\
0.48024012006003	-2.29285104539863e-09\\
0.48224112056028	-2.08440899953003e-09\\
0.48424212106053	-1.87596695366144e-09\\
0.48624312156078	-1.66752490779285e-09\\
0.488244122061031	-2.50128736168927e-09\\
0.490245122561281	-2.29285104539863e-09\\
0.492246123061531	-2.29285104539863e-09\\
0.494247123561781	-2.50128736168927e-09\\
0.496248124062031	-2.70972940755786e-09\\
0.498249124562281	-2.08440899953003e-09\\
0.500250125062531	-2.08440899953003e-09\\
0.502251125562781	-2.50128736168927e-09\\
0.504252126063031	-2.29285104539863e-09\\
0.506253126563282	-1.87596695366144e-09\\
0.508254127063532	-1.87596695366144e-09\\
0.510255127563782	-1.87596695366144e-09\\
0.512256128064032	-1.87596695366144e-09\\
0.514257128564282	-1.04220449976502e-09\\
0.516258129064532	-1.4590885915022e-09\\
0.518259129564782	-1.04220449976502e-09\\
0.520260130065032	-8.33762453896423e-10\\
0.522261130565283	-8.33762453896423e-10\\
0.524262131065533	-8.33762453896423e-10\\
0.526263131565783	-6.25320408027829e-10\\
0.528264132066033	-4.16881799906007e-10\\
0.530265132566283	-2.08440899953003e-10\\
0.532266133066533	-4.16881799906007e-10\\
0.534267133566783	-2.08440899953003e-10\\
0.536268134067034	-4.16881799906007e-10\\
0.538269134567284	-4.16881799906007e-10\\
0.540270135067534	-4.16881799906007e-10\\
0.542271135567784	-4.16881799906007e-10\\
0.544272136068034	-8.33762453896423e-10\\
0.546273136568284	-4.16881799906007e-10\\
0.548274137068534	-2.08440899953003e-10\\
0.550275137568784	-4.16881799906007e-10\\
0.552276138069035	-4.16881799906007e-10\\
0.554277138569285	2.08440899953003e-10\\
0.556278139069535	2.08440899953003e-10\\
0.558279139569785	6.25320408027829e-10\\
0.560280140070035	8.33762453896423e-10\\
0.562281140570285	4.16881799906007e-10\\
0.564282141070535	2.08440899953003e-10\\
0.566283141570785	2.08440899953003e-10\\
0.568284142071036	0\\
0.570285142571286	-2.08440899953003e-10\\
0.572286143071536	-2.08440899953003e-10\\
0.574287143571786	0\\
0.576288144072036	0\\
0.578289144572286	0\\
0.580290145072536	-4.16881799906007e-10\\
0.582291145572786	2.08440899953003e-10\\
0.584292146073036	-2.08440899953003e-10\\
0.586293146573287	-2.08440899953003e-10\\
0.588294147073537	2.08440899953003e-10\\
0.590295147573787	2.08440899953003e-10\\
0.592296148074037	6.25320408027829e-10\\
0.594297148574287	6.25320408027829e-10\\
0.596298149074537	2.08440899953003e-10\\
0.598299149574787	6.25320408027829e-10\\
0.600300150075038	4.16881799906007e-10\\
0.602301150575288	6.25320408027829e-10\\
0.604302151075538	2.08440899953003e-10\\
0.606303151575788	0\\
0.608304152076038	2.08440899953003e-10\\
0.610305152576288	6.25320408027829e-10\\
0.612306153076538	8.33762453896423e-10\\
0.614307153576788	1.25064654563361e-09\\
0.616308154077039	1.25064654563361e-09\\
0.618309154577289	1.4590885915022e-09\\
0.620310155077539	1.04220449976502e-09\\
0.622311155577789	4.16881799906007e-10\\
0.624312156078039	6.25320408027829e-10\\
0.626313156578289	8.33762453896423e-10\\
0.628314157078539	6.25320408027829e-10\\
0.630315157578789	6.25320408027829e-10\\
0.63231615807904	4.16881799906007e-10\\
0.63431715857929	0\\
0.63631815907954	2.08440899953003e-10\\
0.63831915957979	0\\
0.64032016008004	-8.33762453896423e-10\\
0.64232116058029	-4.16881799906007e-10\\
0.64432216108054	-8.33762453896423e-10\\
0.64632316158079	-6.25320408027829e-10\\
0.64832416208104	-8.33762453896423e-10\\
0.650325162581291	-4.16881799906007e-10\\
0.652326163081541	-6.25320408027829e-10\\
0.654327163581791	-8.33762453896423e-10\\
0.656328164082041	-4.16881799906007e-10\\
0.658329164582291	0\\
0.660330165082541	-2.08440899953003e-10\\
0.662331165582791	-4.16881799906007e-10\\
0.664332166083042	-4.16881799906007e-10\\
0.666333166583292	-4.16881799906007e-10\\
0.668334167083542	-4.16881799906007e-10\\
0.670335167583792	-4.16881799906007e-10\\
0.672336168084042	-2.08440899953003e-10\\
0.674337168584292	-6.25320408027829e-10\\
0.676338169084542	-6.25320408027829e-10\\
0.678339169584792	-6.25320408027829e-10\\
0.680340170085043	-4.16881799906007e-10\\
0.682341170585293	-2.08440899953003e-10\\
0.684342171085543	-4.16881799906007e-10\\
0.686343171585793	-4.16881799906007e-10\\
0.688344172086043	-2.08440899953003e-10\\
0.690345172586293	0\\
0.692346173086543	-2.08440899953003e-10\\
0.694347173586793	-2.08440899953003e-10\\
0.696348174087044	0\\
0.698349174587294	2.08440899953003e-10\\
0.700350175087544	2.08440899953003e-10\\
0.702351175587794	0\\
0.704352176088044	-2.08440899953003e-10\\
0.706353176588294	0\\
0.708354177088544	-2.08440899953003e-10\\
0.710355177588794	-2.08440899953003e-10\\
0.712356178089045	2.08440899953003e-10\\
0.714357178589295	0\\
0.716358179089545	-2.08440899953003e-10\\
0.718359179589795	-2.08440899953003e-10\\
0.720360180090045	-2.08440899953003e-10\\
0.722361180590295	-2.08440899953003e-10\\
0.724362181090545	2.08440899953003e-10\\
0.726363181590795	-4.16881799906007e-10\\
0.728364182091045	-6.25320408027829e-10\\
0.730365182591296	-4.16881799906007e-10\\
0.732366183091546	-8.33762453896423e-10\\
0.734367183591796	-4.16881799906007e-10\\
0.736368184092046	-4.16881799906007e-10\\
0.738369184592296	-2.08440899953003e-10\\
0.740370185092546	0\\
0.742371185592796	-4.16881799906007e-10\\
0.744372186093047	-4.16881799906007e-10\\
0.746373186593297	-6.25320408027829e-10\\
0.748374187093547	-2.08440899953003e-10\\
0.750375187593797	2.08440899953003e-10\\
0.752376188094047	-2.08440899953003e-10\\
0.754377188594297	-2.08440899953003e-10\\
0.756378189094547	-4.16881799906007e-10\\
0.758379189594797	-4.16881799906007e-10\\
0.760380190095048	-8.33762453896423e-10\\
0.762381190595298	-4.16881799906007e-10\\
0.764382191095548	-6.25320408027829e-10\\
0.766383191595798	-2.08440899953003e-10\\
0.768384192096048	-4.16881799906007e-10\\
0.770385192596298	0\\
0.772386193096548	-4.16881799906007e-10\\
0.774387193596798	-4.16881799906007e-10\\
0.776388194097049	-8.33762453896423e-10\\
0.778389194597299	-1.04220449976502e-09\\
0.780390195097549	-1.25064654563361e-09\\
0.782391195597799	-1.25064654563361e-09\\
0.784392196098049	-6.25320408027829e-10\\
0.786393196598299	-1.04220449976502e-09\\
0.788394197098549	-4.16881799906007e-10\\
0.790395197598799	-6.25320408027829e-10\\
0.792396198099049	-2.08440899953003e-10\\
0.7943971985993	-8.33762453896423e-10\\
0.79639819909955	-6.25320408027829e-10\\
0.7983991995998	-4.16881799906007e-10\\
0.80040020010005	-4.16881799906007e-10\\
0.8024012006003	-6.25320408027829e-10\\
0.80440220110055	-8.33762453896423e-10\\
0.8064032016008	-1.04220449976502e-09\\
0.808404202101051	-8.33762453896423e-10\\
0.810405202601301	-1.04220449976502e-09\\
0.812406203101551	-8.33762453896423e-10\\
0.814407203601801	-6.25320408027829e-10\\
0.816408204102051	-6.25320408027829e-10\\
0.818409204602301	-4.16881799906007e-10\\
0.820410205102551	-6.25320408027829e-10\\
0.822411205602801	-8.33762453896423e-10\\
0.824412206103052	-6.25320408027829e-10\\
0.826413206603302	-4.16881799906007e-10\\
0.828414207103552	-6.25320408027829e-10\\
0.830415207603802	-6.25320408027829e-10\\
0.832416208104052	-8.33762453896423e-10\\
0.834417208604302	-1.04220449976502e-09\\
0.836418209104552	-8.33762453896423e-10\\
0.838419209604802	-1.04220449976502e-09\\
0.840420210105053	-8.33762453896423e-10\\
0.842421210605303	-6.25320408027829e-10\\
0.844422211105553	-6.25320408027829e-10\\
0.846423211605803	-6.25320408027829e-10\\
0.848424212106053	-2.08440899953003e-10\\
0.850425212606303	0\\
0.852426213106553	2.08440899953003e-10\\
0.854427213606803	4.16881799906007e-10\\
0.856428214107053	8.33762453896423e-10\\
0.858429214607304	8.33762453896423e-10\\
0.860430215107554	1.25064654563361e-09\\
0.862431215607804	1.25064654563361e-09\\
0.864432216108054	1.4590885915022e-09\\
0.866433216608304	1.4590885915022e-09\\
0.868434217108554	1.04220449976502e-09\\
0.870435217608804	1.4590885915022e-09\\
0.872436218109054	1.25064654563361e-09\\
0.874437218609305	1.66752490779285e-09\\
0.876438219109555	1.87596695366144e-09\\
0.878439219609805	1.66752490779285e-09\\
0.880440220110055	1.04220449976502e-09\\
0.882441220610305	1.4590885915022e-09\\
0.884442221110555	1.25064654563361e-09\\
0.886443221610805	4.16881799906007e-10\\
0.888444222111056	6.25320408027829e-10\\
0.890445222611306	4.16881799906007e-10\\
0.892446223111556	2.08440899953003e-10\\
0.894447223611806	0\\
0.896448224112056	4.16881799906007e-10\\
0.898449224612306	4.16881799906007e-10\\
0.900450225112556	2.08440899953003e-10\\
0.902451225612806	4.16881799906007e-10\\
0.904452226113057	2.08440899953003e-10\\
0.906453226613307	2.08440899953003e-10\\
0.908454227113557	0\\
0.910455227613807	-2.08440899953003e-10\\
0.912456228114057	-6.25320408027829e-10\\
0.914457228614307	-6.25320408027829e-10\\
0.916458229114557	-6.25320408027829e-10\\
0.918459229614807	-6.25320408027829e-10\\
0.920460230115058	-4.16881799906007e-10\\
0.922461230615308	-2.08440899953003e-10\\
0.924462231115558	-4.16881799906007e-10\\
0.926463231615808	2.08440899953003e-10\\
0.928464232116058	0\\
0.930465232616308	0\\
0.932466233116558	2.08440899953003e-10\\
0.934467233616808	4.16881799906007e-10\\
0.936468234117058	4.16881799906007e-10\\
0.938469234617309	4.16881799906007e-10\\
0.940470235117559	2.08440899953003e-10\\
0.942471235617809	0\\
0.944472236118059	4.16881799906007e-10\\
0.946473236618309	8.33762453896423e-10\\
0.948474237118559	8.33762453896423e-10\\
0.950475237618809	8.33762453896423e-10\\
0.95247623811906	1.04220449976502e-09\\
0.95447723861931	8.33762453896423e-10\\
0.95647823911956	8.33762453896423e-10\\
0.95847923961981	1.04220449976502e-09\\
0.96048024012006	4.16881799906007e-10\\
0.96248124062031	2.08440899953003e-10\\
0.96448224112056	0\\
0.96648324162081	-2.08440899953003e-10\\
0.968484242121061	-4.16881799906007e-10\\
0.970485242621311	-4.16881799906007e-10\\
0.972486243121561	-4.16881799906007e-10\\
0.974487243621811	-6.25320408027829e-10\\
0.976488244122061	-2.08440899953003e-10\\
0.978489244622311	-2.08440899953003e-10\\
0.980490245122561	4.16881799906007e-10\\
0.982491245622811	6.25320408027829e-10\\
0.984492246123062	2.08440899953003e-10\\
0.986493246623312	2.08440899953003e-10\\
0.988494247123562	4.16881799906007e-10\\
0.990495247623812	6.25320408027829e-10\\
0.992496248124062	4.16881799906007e-10\\
0.994497248624312	2.08440899953003e-10\\
0.996498249124562	2.08440899953003e-10\\
0.998499249624812	4.16881799906007e-10\\
1.00050025012506	8.33762453896423e-10\\
1.00250125062531	8.33762453896423e-10\\
1.00450225112556	8.33762453896423e-10\\
1.00650325162581	8.33762453896423e-10\\
1.00850425212606	1.04220449976502e-09\\
1.01050525262631	8.33762453896423e-10\\
1.01250625312656	1.25064654563361e-09\\
1.01450725362681	1.4590885915022e-09\\
1.01650825412706	1.25064654563361e-09\\
1.01850925462731	1.4590885915022e-09\\
1.02051025512756	1.25064654563361e-09\\
1.02251125562781	1.66752490779285e-09\\
1.02451225612806	2.08440899953003e-09\\
1.02651325662831	1.25064654563361e-09\\
1.02851425712856	1.25064654563361e-09\\
1.03051525762881	8.33762453896423e-10\\
1.03251625812906	1.25064654563361e-09\\
1.03451725862931	1.04220449976502e-09\\
1.03651825912956	1.04220449976502e-09\\
1.03851925962981	1.04220449976502e-09\\
1.04052026013006	1.4590885915022e-09\\
1.04252126063032	1.4590885915022e-09\\
1.04452226113057	1.66752490779285e-09\\
1.04652326163082	1.4590885915022e-09\\
1.04852426213107	1.87596695366144e-09\\
1.05052526263132	1.25064654563361e-09\\
1.05252626313157	1.87596695366144e-09\\
1.05452726363182	1.4590885915022e-09\\
1.05652826413207	1.66752490779285e-09\\
1.05852926463232	1.4590885915022e-09\\
1.06053026513257	1.66752490779285e-09\\
1.06253126563282	2.29285104539863e-09\\
1.06453226613307	2.50128736168927e-09\\
1.06653326663332	2.91817145342646e-09\\
1.06853426713357	2.91817145342646e-09\\
1.07053526763382	2.70972940755786e-09\\
1.07253626813407	3.12661349929505e-09\\
1.07453726863432	3.12661349929505e-09\\
1.07653826913457	3.75193390732288e-09\\
1.07853926963482	3.33505554516364e-09\\
1.08054027013507	3.12661349929505e-09\\
1.08254127063532	3.33505554516364e-09\\
1.08454227113557	2.91817145342646e-09\\
1.08654327163582	2.50128736168927e-09\\
1.08854427213607	2.50128736168927e-09\\
1.09054527263632	2.50128736168927e-09\\
1.09254627313657	1.87596695366144e-09\\
1.09454727363682	2.08440899953003e-09\\
1.09654827413707	2.50128736168927e-09\\
1.09854927463732	2.70972940755786e-09\\
1.10055027513757	2.29285104539863e-09\\
1.10255127563782	2.29285104539863e-09\\
1.10455227613807	2.08440899953003e-09\\
1.10655327663832	2.08440899953003e-09\\
1.10855427713857	2.08440899953003e-09\\
1.11055527763882	1.87596695366144e-09\\
1.11255627813907	1.87596695366144e-09\\
1.11455727863932	1.87596695366144e-09\\
1.11655827913957	1.87596695366144e-09\\
1.11855927963982	1.87596695366144e-09\\
1.12056028014007	1.66752490779285e-09\\
1.12256128064032	2.08440899953003e-09\\
1.12456228114057	1.4590885915022e-09\\
1.12656328164082	1.25064654563361e-09\\
1.12856428214107	1.87596695366144e-09\\
1.13056528264132	1.4590885915022e-09\\
1.13256628314157	1.66752490779285e-09\\
1.13456728364182	1.4590885915022e-09\\
1.13656828414207	1.25064654563361e-09\\
1.13856928464232	1.87596695366144e-09\\
1.14057028514257	1.66752490779285e-09\\
1.14257128564282	1.66752490779285e-09\\
1.14457228614307	1.4590885915022e-09\\
1.14657328664332	1.25064654563361e-09\\
1.14857428714357	1.04220449976502e-09\\
1.15057528764382	1.04220449976502e-09\\
1.15257628814407	1.25064654563361e-09\\
1.15457728864432	1.25064654563361e-09\\
1.15657828914457	1.4590885915022e-09\\
1.15857928964482	1.66752490779285e-09\\
1.16058029014507	1.87596695366144e-09\\
1.16258129064532	2.50128736168927e-09\\
1.16458229114557	2.91817145342646e-09\\
1.16658329164582	2.50128736168927e-09\\
1.16858429214607	2.29285104539863e-09\\
1.17058529264632	2.08440899953003e-09\\
1.17258629314657	2.08440899953003e-09\\
1.17458729364682	2.50128736168927e-09\\
1.17658829414707	2.91817145342646e-09\\
1.17858929464732	3.12661349929505e-09\\
1.18059029514757	3.54349186145428e-09\\
1.18259129564782	3.54349186145428e-09\\
1.18459229614807	3.33505554516364e-09\\
1.18659329664832	3.54349186145428e-09\\
1.18859429714857	3.33505554516364e-09\\
1.19059529764882	3.54349186145428e-09\\
1.19259629814907	3.33505554516364e-09\\
1.19459729864932	3.12661349929505e-09\\
1.19659829914957	2.70972940755786e-09\\
1.19859929964982	3.12661349929505e-09\\
1.20060030015008	2.50128736168927e-09\\
1.20260130065033	2.91817145342646e-09\\
1.20460230115058	2.50128736168927e-09\\
1.20660330165083	2.70972940755786e-09\\
1.20860430215108	2.91817145342646e-09\\
1.21060530265133	2.70972940755786e-09\\
1.21260630315158	2.08440899953003e-09\\
1.21460730365183	2.50128736168927e-09\\
1.21660830415208	2.50128736168927e-09\\
1.21860930465233	2.29285104539863e-09\\
1.22061030515258	2.08440899953003e-09\\
1.22261130565283	2.50128736168927e-09\\
1.22461230615308	2.50128736168927e-09\\
1.22661330665333	2.50128736168927e-09\\
1.22861430715358	2.70972940755786e-09\\
1.23061530765383	3.12661349929505e-09\\
1.23261630815408	2.70972940755786e-09\\
1.23461730865433	2.50128736168927e-09\\
1.23661830915458	3.12661349929505e-09\\
1.23861930965483	3.33505554516364e-09\\
1.24062031015508	3.33505554516364e-09\\
1.24262131065533	3.33505554516364e-09\\
1.24462231115558	3.33505554516364e-09\\
1.24662331165583	3.75193390732288e-09\\
1.24862431215608	3.54349186145428e-09\\
1.25062531265633	3.54349186145428e-09\\
1.25262631315658	3.75193390732288e-09\\
1.25462731365683	3.96037595319147e-09\\
1.25662831415708	3.96037595319147e-09\\
1.25862931465733	4.16881799906006e-09\\
1.26063031515758	4.37726004492866e-09\\
1.26263131565783	3.96037595319147e-09\\
1.26463231615808	3.75193390732288e-09\\
1.26663331665833	3.96037595319147e-09\\
1.26863431715858	4.37726004492866e-09\\
1.27063531765883	3.96037595319147e-09\\
1.27263631815908	4.37726004492866e-09\\
1.27463731865933	4.37726004492866e-09\\
1.27663831915958	4.79413840708789e-09\\
1.27863931965983	5.00258045295649e-09\\
1.28064032016008	4.5856963612193e-09\\
1.28264132066033	4.79413840708789e-09\\
1.28464232116058	4.5856963612193e-09\\
1.28664332166083	4.5856963612193e-09\\
1.28864432216108	4.37726004492866e-09\\
1.29064532266133	4.16881799906006e-09\\
1.29264632316158	4.5856963612193e-09\\
1.29464732366183	4.5856963612193e-09\\
1.29664832416208	4.79413840708789e-09\\
1.29864932466233	4.37726004492866e-09\\
1.30065032516258	4.16881799906006e-09\\
1.30265132566283	4.37726004492866e-09\\
1.30465232616308	4.5856963612193e-09\\
1.30665332666333	4.16881799906006e-09\\
1.30865432716358	4.5856963612193e-09\\
1.31065532766383	4.37726004492866e-09\\
1.31265632816408	4.5856963612193e-09\\
1.31465732866433	4.5856963612193e-09\\
1.31665832916458	4.5856963612193e-09\\
1.31865932966483	4.37726004492866e-09\\
1.32066033016508	4.79413840708789e-09\\
1.32266133066533	4.5856963612193e-09\\
1.32466233116558	4.16881799906006e-09\\
1.32666333166583	4.37726004492866e-09\\
1.32866433216608	4.37726004492866e-09\\
1.33066533266633	3.96037595319147e-09\\
1.33266633316658	3.75193390732288e-09\\
1.33466733366683	3.75193390732288e-09\\
1.33666833416708	3.54349186145428e-09\\
1.33866933466733	3.96037595319147e-09\\
1.34067033516758	3.75193390732288e-09\\
1.34267133566783	3.96037595319147e-09\\
1.34467233616808	4.37726004492866e-09\\
1.34667333666833	4.16881799906006e-09\\
1.34867433716858	3.96037595319147e-09\\
1.35067533766883	3.96037595319147e-09\\
1.35267633816908	4.16881799906006e-09\\
1.35467733866933	3.96037595319147e-09\\
1.35667833916958	3.75193390732288e-09\\
1.35867933966983	3.54349186145428e-09\\
1.36068034017009	3.33505554516364e-09\\
1.36268134067034	3.54349186145428e-09\\
1.36468234117059	3.54349186145428e-09\\
1.36668334167084	3.12661349929505e-09\\
1.36868434217109	2.91817145342646e-09\\
1.37068534267134	3.33505554516364e-09\\
1.37268634317159	3.75193390732288e-09\\
1.37468734367184	3.54349186145428e-09\\
1.37668834417209	3.96037595319147e-09\\
1.37868934467234	4.16881799906006e-09\\
1.38069034517259	3.96037595319147e-09\\
1.38269134567284	4.16881799906006e-09\\
1.38469234617309	4.16881799906006e-09\\
1.38669334667334	3.54349186145428e-09\\
1.38869434717359	3.75193390732288e-09\\
1.39069534767384	4.16881799906006e-09\\
1.39269634817409	4.37726004492866e-09\\
1.39469734867434	4.5856963612193e-09\\
1.39669834917459	4.16881799906006e-09\\
1.39869934967484	4.16881799906006e-09\\
1.40070035017509	3.96037595319147e-09\\
1.40270135067534	3.96037595319147e-09\\
1.40470235117559	3.96037595319147e-09\\
1.40670335167584	4.16881799906006e-09\\
1.40870435217609	3.96037595319147e-09\\
1.41070535267634	3.54349186145428e-09\\
1.41270635317659	3.33505554516364e-09\\
1.41470735367684	3.33505554516364e-09\\
1.41670835417709	3.54349186145428e-09\\
1.41870935467734	3.33505554516364e-09\\
1.42071035517759	3.33505554516364e-09\\
1.42271135567784	3.75193390732288e-09\\
1.42471235617809	3.33505554516364e-09\\
1.42671335667834	3.54349186145428e-09\\
1.42871435717859	2.91817145342646e-09\\
1.43071535767884	2.91817145342646e-09\\
1.43271635817909	2.91817145342646e-09\\
1.43471735867934	2.70972940755786e-09\\
1.43671835917959	2.29285104539863e-09\\
1.43871935967984	2.50128736168927e-09\\
1.44072036018009	2.29285104539863e-09\\
1.44272136068034	2.50128736168927e-09\\
1.44472236118059	2.70972940755786e-09\\
1.44672336168084	3.12661349929505e-09\\
1.44872436218109	2.91817145342646e-09\\
1.45072536268134	2.91817145342646e-09\\
1.45272636318159	3.12661349929505e-09\\
1.45472736368184	3.12661349929505e-09\\
1.45672836418209	2.70972940755786e-09\\
1.45872936468234	2.91817145342646e-09\\
1.46073036518259	2.50128736168927e-09\\
1.46273136568284	2.08440899953003e-09\\
1.46473236618309	2.08440899953003e-09\\
1.46673336668334	1.87596695366144e-09\\
1.46873436718359	1.87596695366144e-09\\
1.47073536768384	1.4590885915022e-09\\
1.47273636818409	1.87596695366144e-09\\
1.47473736868434	1.87596695366144e-09\\
1.47673836918459	1.66752490779285e-09\\
1.47873936968484	1.4590885915022e-09\\
1.48074037018509	1.04220449976502e-09\\
1.48274137068534	8.33762453896423e-10\\
1.48474237118559	1.04220449976502e-09\\
1.48674337168584	1.04220449976502e-09\\
1.48874437218609	6.25320408027829e-10\\
1.49074537268634	6.25320408027829e-10\\
1.49274637318659	6.25320408027829e-10\\
1.49474737368684	8.33762453896423e-10\\
1.49674837418709	1.04220449976502e-09\\
1.49874937468734	1.04220449976502e-09\\
1.50075037518759	8.33762453896423e-10\\
1.50275137568784	1.04220449976502e-09\\
1.50475237618809	8.33762453896423e-10\\
1.50675337668834	8.33762453896423e-10\\
1.50875437718859	6.25320408027829e-10\\
1.51075537768884	6.25320408027829e-10\\
1.51275637818909	4.16881799906007e-10\\
1.51475737868934	-2.08440899953003e-10\\
1.51675837918959	2.08440899953003e-10\\
1.51875937968984	2.08440899953003e-10\\
1.5207603801901	4.16881799906007e-10\\
1.52276138069035	4.16881799906007e-10\\
1.5247623811906	2.08440899953003e-10\\
1.52676338169085	6.25320408027829e-10\\
1.5287643821911	4.16881799906007e-10\\
1.53076538269135	2.08440899953003e-10\\
1.5327663831916	6.25320408027829e-10\\
1.53476738369185	2.08440899953003e-10\\
1.5367683841921	4.16881799906007e-10\\
1.53876938469235	2.08440899953003e-10\\
1.5407703851926	-2.08440899953003e-10\\
1.54277138569285	-2.08440899953003e-10\\
1.5447723861931	-4.16881799906007e-10\\
1.54677338669335	-8.33762453896423e-10\\
1.5487743871936	-8.33762453896423e-10\\
1.55077538769385	-1.25064654563361e-09\\
1.5527763881941	-1.4590885915022e-09\\
1.55477738869435	-1.4590885915022e-09\\
1.5567783891946	-1.25064654563361e-09\\
1.55877938969485	-1.66752490779285e-09\\
1.5607803901951	-1.87596695366144e-09\\
1.56278139069535	-2.08440899953003e-09\\
1.5647823911956	-2.08440899953003e-09\\
1.56678339169585	-2.29285104539863e-09\\
1.5687843921961	-2.29285104539863e-09\\
1.57078539269635	-2.08440899953003e-09\\
1.5727863931966	-2.50128736168927e-09\\
1.57478739369685	-2.70972940755786e-09\\
1.5767883941971	-2.91817145342646e-09\\
1.57878939469735	-3.12661349929505e-09\\
1.5807903951976	-3.12661349929505e-09\\
1.58279139569785	-2.70972940755786e-09\\
1.5847923961981	-3.12661349929505e-09\\
1.58679339669835	-3.12661349929505e-09\\
1.5887943971986	-2.91817145342646e-09\\
1.59079539769885	-2.91817145342646e-09\\
1.5927963981991	-2.91817145342646e-09\\
1.59479739869935	-3.12661349929505e-09\\
1.5967983991996	-2.70972940755786e-09\\
1.59879939969985	-2.29285104539863e-09\\
1.6008004002001	-2.08440899953003e-09\\
1.60280140070035	-1.87596695366144e-09\\
1.6048024012006	-1.25064654563361e-09\\
1.60680340170085	-1.4590885915022e-09\\
1.6088044022011	-1.4590885915022e-09\\
1.61080540270135	-1.25064654563361e-09\\
1.6128064032016	-1.4590885915022e-09\\
1.61480740370185	-1.4590885915022e-09\\
1.6168084042021	-8.33762453896423e-10\\
1.61880940470235	-8.33762453896423e-10\\
1.6208104052026	-6.25320408027829e-10\\
1.62281140570285	-2.08440899953003e-10\\
1.6248124062031	-4.16881799906007e-10\\
1.62681340670335	-4.16881799906007e-10\\
1.6288144072036	-6.25320408027829e-10\\
1.63081540770385	-8.33762453896423e-10\\
1.6328164082041	-6.25320408027829e-10\\
1.63481740870435	-6.25320408027829e-10\\
1.6368184092046	-8.33762453896423e-10\\
1.63881940970485	-8.33762453896423e-10\\
1.6408204102051	-1.04220449976502e-09\\
1.64282141070535	-8.33762453896423e-10\\
1.6448224112056	-8.33762453896423e-10\\
1.64682341170585	-2.08440899953003e-10\\
1.6488244122061	0\\
1.65082541270635	0\\
1.6528264132066	-2.08440899953003e-10\\
1.65482741370685	-4.16881799906007e-10\\
1.6568284142071	-2.08440899953003e-10\\
1.65882941470735	0\\
1.6608304152076	0\\
1.66283141570785	2.08440899953003e-10\\
1.6648324162081	4.16881799906007e-10\\
1.66683341670835	4.16881799906007e-10\\
1.6688344172086	2.08440899953003e-10\\
1.67083541770885	2.08440899953003e-10\\
1.6728364182091	2.08440899953003e-10\\
1.67483741870935	4.16881799906007e-10\\
1.6768384192096	2.08440899953003e-10\\
1.67883941970985	4.16881799906007e-10\\
1.68084042021011	4.16881799906007e-10\\
1.68284142071036	4.16881799906007e-10\\
1.68484242121061	-2.08440899953003e-10\\
1.68684342171086	0\\
1.68884442221111	0\\
1.69084542271136	0\\
1.69284642321161	0\\
1.69484742371186	4.16881799906007e-10\\
1.69684842421211	6.25320408027829e-10\\
1.69884942471236	1.04220449976502e-09\\
1.70085042521261	1.25064654563361e-09\\
1.70285142571286	1.4590885915022e-09\\
1.70485242621311	1.4590885915022e-09\\
1.70685342671336	1.66752490779285e-09\\
1.70885442721361	1.87596695366144e-09\\
1.71085542771386	1.66752490779285e-09\\
1.71285642821411	1.87596695366144e-09\\
1.71485742871436	2.50128736168927e-09\\
1.71685842921461	2.29285104539863e-09\\
1.71885942971486	1.87596695366144e-09\\
1.72086043021511	1.87596695366144e-09\\
1.72286143071536	2.29285104539863e-09\\
1.72486243121561	1.66752490779285e-09\\
1.72686343171586	2.29285104539863e-09\\
1.72886443221611	2.08440899953003e-09\\
1.73086543271636	1.87596695366144e-09\\
1.73286643321661	1.87596695366144e-09\\
1.73486743371686	1.4590885915022e-09\\
1.73686843421711	1.66752490779285e-09\\
1.73886943471736	1.4590885915022e-09\\
1.74087043521761	1.25064654563361e-09\\
1.74287143571786	1.04220449976502e-09\\
1.74487243621811	1.4590885915022e-09\\
1.74687343671836	1.25064654563361e-09\\
1.74887443721861	1.04220449976502e-09\\
1.75087543771886	1.4590885915022e-09\\
1.75287643821911	1.25064654563361e-09\\
1.75487743871936	8.33762453896423e-10\\
1.75687843921961	4.16881799906007e-10\\
1.75887943971986	6.25320408027829e-10\\
1.76088044022011	6.25320408027829e-10\\
1.76288144072036	6.25320408027829e-10\\
1.76488244122061	6.25320408027829e-10\\
1.76688344172086	8.33762453896423e-10\\
1.76888444222111	6.25320408027829e-10\\
1.77088544272136	1.04220449976502e-09\\
1.77288644322161	1.04220449976502e-09\\
1.77488744372186	8.33762453896423e-10\\
1.77688844422211	8.33762453896423e-10\\
1.77888944472236	8.33762453896423e-10\\
1.78089044522261	8.33762453896423e-10\\
1.78289144572286	4.16881799906007e-10\\
1.78489244622311	4.16881799906007e-10\\
1.78689344672336	4.16881799906007e-10\\
1.78889444722361	6.25320408027829e-10\\
1.79089544772386	2.08440899953003e-10\\
1.79289644822411	4.16881799906007e-10\\
1.79489744872436	4.16881799906007e-10\\
1.79689844922461	6.25320408027829e-10\\
1.79889944972486	2.08440899953003e-10\\
1.80090045022511	2.08440899953003e-10\\
1.80290145072536	6.25320408027829e-10\\
1.80490245122561	6.25320408027829e-10\\
1.80690345172586	6.25320408027829e-10\\
1.80890445222611	6.25320408027829e-10\\
1.81090545272636	4.16881799906007e-10\\
1.81290645322661	2.08440899953003e-10\\
1.81490745372686	2.08440899953003e-10\\
1.81690845422711	2.08440899953003e-10\\
1.81890945472736	2.08440899953003e-10\\
1.82091045522761	4.16881799906007e-10\\
1.82291145572786	6.25320408027829e-10\\
1.82491245622811	4.16881799906007e-10\\
1.82691345672836	4.16881799906007e-10\\
1.82891445722861	4.16881799906007e-10\\
1.83091545772886	4.16881799906007e-10\\
1.83291645822911	8.33762453896423e-10\\
1.83491745872936	6.25320408027829e-10\\
1.83691845922961	8.33762453896423e-10\\
1.83891945972987	8.33762453896423e-10\\
1.84092046023012	6.25320408027829e-10\\
1.84292146073037	6.25320408027829e-10\\
1.84492246123062	1.04220449976502e-09\\
1.84692346173087	8.33762453896423e-10\\
1.84892446223112	6.25320408027829e-10\\
1.85092546273137	8.33762453896423e-10\\
1.85292646323162	8.33762453896423e-10\\
1.85492746373187	1.04220449976502e-09\\
1.85692846423212	4.16881799906007e-10\\
1.85892946473237	6.25320408027829e-10\\
1.86093046523262	6.25320408027829e-10\\
1.86293146573287	6.25320408027829e-10\\
1.86493246623312	6.25320408027829e-10\\
1.86693346673337	4.16881799906007e-10\\
1.86893446723362	8.33762453896423e-10\\
1.87093546773387	8.33762453896423e-10\\
1.87293646823412	4.16881799906007e-10\\
1.87493746873437	4.16881799906007e-10\\
1.87693846923462	6.25320408027829e-10\\
1.87893946973487	8.33762453896423e-10\\
1.88094047023512	8.33762453896423e-10\\
1.88294147073537	1.04220449976502e-09\\
1.88494247123562	1.25064654563361e-09\\
1.88694347173587	1.04220449976502e-09\\
1.88894447223612	6.25320408027829e-10\\
1.89094547273637	4.16881799906007e-10\\
1.89294647323662	2.08440899953003e-10\\
1.89494747373687	6.25320408027829e-10\\
1.89694847423712	6.25320408027829e-10\\
1.89894947473737	6.25320408027829e-10\\
1.90095047523762	8.33762453896423e-10\\
1.90295147573787	1.4590885915022e-09\\
1.90495247623812	1.04220449976502e-09\\
1.90695347673837	8.33762453896423e-10\\
1.90895447723862	8.33762453896423e-10\\
1.91095547773887	8.33762453896423e-10\\
1.91295647823912	6.25320408027829e-10\\
1.91495747873937	2.08440899953003e-10\\
1.91695847923962	0\\
1.91895947973987	2.08440899953003e-10\\
1.92096048024012	0\\
1.92296148074037	-2.08440899953003e-10\\
1.92496248124062	-2.08440899953003e-10\\
1.92696348174087	0\\
1.92896448224112	-2.08440899953003e-10\\
1.93096548274137	0\\
1.93296648324162	0\\
1.93496748374187	2.08440899953003e-10\\
1.93696848424212	4.16881799906007e-10\\
1.93896948474237	4.16881799906007e-10\\
1.94097048524262	0\\
1.94297148574287	-2.08440899953003e-10\\
1.94497248624312	-2.08440899953003e-10\\
1.94697348674337	-2.08440899953003e-10\\
1.94897448724362	2.08440899953003e-10\\
1.95097548774387	2.08440899953003e-10\\
1.95297648824412	4.16881799906007e-10\\
1.95497748874437	4.16881799906007e-10\\
1.95697848924462	4.16881799906007e-10\\
1.95897948974487	4.16881799906007e-10\\
1.96098049024512	4.16881799906007e-10\\
1.96298149074537	4.16881799906007e-10\\
1.96498249124562	4.16881799906007e-10\\
1.96698349174587	2.08440899953003e-10\\
1.96898449224612	0\\
1.97098549274637	-2.08440899953003e-10\\
1.97298649324662	-2.08440899953003e-10\\
1.97498749374687	0\\
1.97698849424712	-4.16881799906007e-10\\
1.97898949474737	-4.16881799906007e-10\\
1.98099049524762	-2.08440899953003e-10\\
1.98299149574787	-6.25320408027829e-10\\
1.98499249624812	0\\
1.98699349674837	2.08440899953003e-10\\
1.98899449724862	0\\
1.99099549774887	-4.16881799906007e-10\\
1.99299649824912	-4.16881799906007e-10\\
1.99499749874937	-4.16881799906007e-10\\
1.99699849924962	-4.16881799906007e-10\\
1.99899949974988	-2.08440899953003e-10\\
2.00100050025013	4.16881799906007e-10\\
2.00300150075038	4.16881799906007e-10\\
2.00500250125063	6.25320408027829e-10\\
2.00700350175088	4.16881799906007e-10\\
2.00900450225113	6.25320408027829e-10\\
2.01100550275138	6.25320408027829e-10\\
2.01300650325163	6.25320408027829e-10\\
2.01500750375188	6.25320408027829e-10\\
2.01700850425213	4.16881799906007e-10\\
2.01900950475238	2.08440899953003e-10\\
2.02101050525263	4.16881799906007e-10\\
2.02301150575288	4.16881799906007e-10\\
2.02501250625313	4.16881799906007e-10\\
2.02701350675338	4.16881799906007e-10\\
2.02901450725363	6.25320408027829e-10\\
2.03101550775388	4.16881799906007e-10\\
2.03301650825413	8.33762453896423e-10\\
2.03501750875438	6.25320408027829e-10\\
2.03701850925463	4.16881799906007e-10\\
2.03901950975488	2.08440899953003e-10\\
2.04102051025513	-4.16881799906007e-10\\
2.04302151075538	-2.08440899953003e-10\\
2.04502251125563	0\\
2.04702351175588	0\\
2.04902451225613	0\\
2.05102551275638	4.16881799906007e-10\\
2.05302651325663	6.25320408027829e-10\\
2.05502751375688	6.25320408027829e-10\\
2.05702851425713	2.08440899953003e-10\\
2.05902951475738	4.16881799906007e-10\\
2.06103051525763	6.25320408027829e-10\\
2.06303151575788	6.25320408027829e-10\\
2.06503251625813	2.08440899953003e-10\\
2.06703351675838	4.16881799906007e-10\\
2.06903451725863	8.33762453896423e-10\\
2.07103551775888	8.33762453896423e-10\\
2.07303651825913	8.33762453896423e-10\\
2.07503751875938	6.25320408027829e-10\\
2.07703851925963	8.33762453896423e-10\\
2.07903951975988	6.25320408027829e-10\\
2.08104052026013	6.25320408027829e-10\\
2.08304152076038	4.16881799906007e-10\\
2.08504252126063	4.16881799906007e-10\\
2.08704352176088	0\\
2.08904452226113	4.16881799906007e-10\\
2.09104552276138	6.25320408027829e-10\\
2.09304652326163	6.25320408027829e-10\\
2.09504752376188	1.04220449976502e-09\\
2.09704852426213	1.04220449976502e-09\\
2.09904952476238	1.04220449976502e-09\\
2.10105052526263	1.25064654563361e-09\\
2.10305152576288	1.25064654563361e-09\\
2.10505252626313	1.25064654563361e-09\\
2.10705352676338	1.25064654563361e-09\\
2.10905452726363	1.04220449976502e-09\\
2.11105552776388	1.25064654563361e-09\\
2.11305652826413	8.33762453896423e-10\\
2.11505752876438	8.33762453896423e-10\\
2.11705852926463	6.25320408027829e-10\\
2.11905952976488	4.16881799906007e-10\\
2.12106053026513	6.25320408027829e-10\\
2.12306153076538	2.08440899953003e-10\\
2.12506253126563	0\\
2.12706353176588	6.25320408027829e-10\\
2.12906453226613	6.25320408027829e-10\\
2.13106553276638	6.25320408027829e-10\\
2.13306653326663	1.04220449976502e-09\\
2.13506753376688	1.25064654563361e-09\\
2.13706853426713	1.66752490779285e-09\\
2.13906953476738	1.66752490779285e-09\\
2.14107053526763	1.4590885915022e-09\\
2.14307153576788	1.25064654563361e-09\\
2.14507253626813	8.33762453896423e-10\\
2.14707353676838	4.16881799906007e-10\\
2.14907453726863	4.16881799906007e-10\\
2.15107553776888	1.04220449976502e-09\\
2.15307653826913	1.04220449976502e-09\\
2.15507753876938	1.04220449976502e-09\\
2.15707853926963	1.25064654563361e-09\\
2.15907953976988	8.33762453896423e-10\\
2.16108054027013	6.25320408027829e-10\\
2.16308154077039	4.16881799906007e-10\\
2.16508254127064	4.16881799906007e-10\\
2.16708354177089	4.16881799906007e-10\\
2.16908454227114	2.08440899953003e-10\\
2.17108554277139	2.08440899953003e-10\\
2.17308654327164	0\\
2.17508754377189	2.08440899953003e-10\\
2.17708854427214	0\\
2.17908954477239	-2.08440899953003e-10\\
2.18109054527264	-2.08440899953003e-10\\
2.18309154577289	0\\
2.18509254627314	0\\
2.18709354677339	0\\
2.18909454727364	-2.08440899953003e-10\\
2.19109554777389	-4.16881799906007e-10\\
2.19309654827414	4.16881799906007e-10\\
2.19509754877439	6.25320408027829e-10\\
2.19709854927464	8.33762453896423e-10\\
2.19909954977489	6.25320408027829e-10\\
2.20110055027514	8.33762453896423e-10\\
2.20310155077539	1.4590885915022e-09\\
2.20510255127564	1.4590885915022e-09\\
2.20710355177589	1.66752490779285e-09\\
2.20910455227614	1.87596695366144e-09\\
2.21110555277639	1.4590885915022e-09\\
2.21310655327664	2.08440899953003e-09\\
2.21510755377689	2.08440899953003e-09\\
2.21710855427714	2.29285104539863e-09\\
2.21910955477739	1.87596695366144e-09\\
2.22111055527764	1.87596695366144e-09\\
2.22311155577789	1.87596695366144e-09\\
2.22511255627814	2.08440899953003e-09\\
2.22711355677839	1.4590885915022e-09\\
2.22911455727864	1.4590885915022e-09\\
2.23111555777889	1.66752490779285e-09\\
2.23311655827914	1.4590885915022e-09\\
2.23511755877939	1.04220449976502e-09\\
2.23711855927964	1.04220449976502e-09\\
2.23911955977989	1.4590885915022e-09\\
2.24112056028014	1.04220449976502e-09\\
2.24312156078039	4.16881799906007e-10\\
2.24512256128064	1.25064654563361e-09\\
2.24712356178089	1.25064654563361e-09\\
2.24912456228114	1.25064654563361e-09\\
2.25112556278139	1.4590885915022e-09\\
2.25312656328164	1.25064654563361e-09\\
2.25512756378189	1.66752490779285e-09\\
2.25712856428214	1.87596695366144e-09\\
2.25912956478239	1.87596695366144e-09\\
2.26113056528264	1.87596695366144e-09\\
2.26313156578289	2.08440899953003e-09\\
2.26513256628314	1.87596695366144e-09\\
2.26713356678339	1.87596695366144e-09\\
2.26913456728364	1.87596695366144e-09\\
2.27113556778389	1.66752490779285e-09\\
2.27313656828414	1.4590885915022e-09\\
2.27513756878439	1.87596695366144e-09\\
2.27713856928464	2.08440899953003e-09\\
2.27913956978489	1.66752490779285e-09\\
2.28114057028514	1.87596695366144e-09\\
2.28314157078539	1.87596695366144e-09\\
2.28514257128564	1.66752490779285e-09\\
2.28714357178589	1.04220449976502e-09\\
2.28914457228614	6.25320408027829e-10\\
2.29114557278639	8.33762453896423e-10\\
2.29314657328664	8.33762453896423e-10\\
2.29514757378689	6.25320408027829e-10\\
2.29714857428714	8.33762453896423e-10\\
2.29914957478739	8.33762453896423e-10\\
2.30115057528764	1.04220449976502e-09\\
2.30315157578789	1.66752490779285e-09\\
2.30515257628814	1.4590885915022e-09\\
2.30715357678839	1.04220449976502e-09\\
2.30915457728864	8.33762453896423e-10\\
2.31115557778889	1.04220449976502e-09\\
2.31315657828914	1.4590885915022e-09\\
2.31515757878939	2.08440899953003e-09\\
2.31715857928964	2.50128736168927e-09\\
2.31915957978989	2.50128736168927e-09\\
2.32116058029015	2.29285104539863e-09\\
2.3231615807904	2.29285104539863e-09\\
2.32516258129065	1.4590885915022e-09\\
2.3271635817909	1.66752490779285e-09\\
2.32916458229115	1.87596695366144e-09\\
2.3311655827914	1.66752490779285e-09\\
2.33316658329165	1.87596695366144e-09\\
2.3351675837919	1.87596695366144e-09\\
2.33716858429215	2.29285104539863e-09\\
2.3391695847924	1.87596695366144e-09\\
2.34117058529265	1.87596695366144e-09\\
2.3431715857929	1.4590885915022e-09\\
2.34517258629315	1.87596695366144e-09\\
2.3471735867934	2.08440899953003e-09\\
2.34917458729365	2.50128736168927e-09\\
2.3511755877939	2.70972940755786e-09\\
2.35317658829415	2.91817145342646e-09\\
2.3551775887944	2.50128736168927e-09\\
2.35717858929465	2.70972940755786e-09\\
2.3591795897949	3.12661349929505e-09\\
2.36118059029515	3.54349186145428e-09\\
2.3631815907954	3.54349186145428e-09\\
2.36518259129565	3.54349186145428e-09\\
2.3671835917959	4.16881799906006e-09\\
2.36918459229615	3.75193390732288e-09\\
2.3711855927964	4.16881799906006e-09\\
2.37318659329665	3.75193390732288e-09\\
2.3751875937969	3.54349186145428e-09\\
2.37718859429715	3.75193390732288e-09\\
2.3791895947974	3.75193390732288e-09\\
2.38119059529765	3.54349186145428e-09\\
2.3831915957979	3.75193390732288e-09\\
2.38519259629815	4.16881799906006e-09\\
2.3871935967984	4.16881799906006e-09\\
2.38919459729865	4.16881799906006e-09\\
2.3911955977989	4.5856963612193e-09\\
2.39319659829915	4.37726004492866e-09\\
2.3951975987994	4.5856963612193e-09\\
2.39719859929965	4.79413840708789e-09\\
2.3991995997999	4.5856963612193e-09\\
2.40120060030015	4.37726004492866e-09\\
2.4032016008004	3.96037595319147e-09\\
2.40520260130065	3.75193390732288e-09\\
2.4072036018009	3.75193390732288e-09\\
2.40920460230115	3.54349186145428e-09\\
2.4112056028014	2.91817145342646e-09\\
2.41320660330165	3.33505554516364e-09\\
2.4152076038019	3.54349186145428e-09\\
2.41720860430215	3.33505554516364e-09\\
2.4192096048024	2.91817145342646e-09\\
2.42121060530265	2.91817145342646e-09\\
2.4232116058029	2.91817145342646e-09\\
2.42521260630315	2.70972940755786e-09\\
2.4272136068034	2.70972940755786e-09\\
2.42921460730365	2.70972940755786e-09\\
2.4312156078039	3.12661349929505e-09\\
2.43321660830415	2.70972940755786e-09\\
2.4352176088044	2.70972940755786e-09\\
2.43721860930465	2.70972940755786e-09\\
2.4392196098049	2.70972940755786e-09\\
2.44122061030515	2.70972940755786e-09\\
2.4432216108054	3.12661349929505e-09\\
2.44522261130565	3.12661349929505e-09\\
2.4472236118059	3.54349186145428e-09\\
2.44922461230615	3.33505554516364e-09\\
2.4512256128064	3.12661349929505e-09\\
2.45322661330665	3.12661349929505e-09\\
2.4552276138069	2.70972940755786e-09\\
2.45722861430715	2.70972940755786e-09\\
2.4592296148074	2.70972940755786e-09\\
2.46123061530765	2.91817145342646e-09\\
2.4632316158079	2.91817145342646e-09\\
2.46523261630815	3.12661349929505e-09\\
2.4672336168084	2.50128736168927e-09\\
2.46923461730865	2.70972940755786e-09\\
2.4712356178089	2.91817145342646e-09\\
2.47323661830915	2.91817145342646e-09\\
2.4752376188094	2.70972940755786e-09\\
2.47723861930965	2.08440899953003e-09\\
2.4792396198099	2.70972940755786e-09\\
2.48124062031015	2.50128736168927e-09\\
2.48324162081041	2.08440899953003e-09\\
2.48524262131066	2.08440899953003e-09\\
2.48724362181091	1.87596695366144e-09\\
2.48924462231116	1.66752490779285e-09\\
2.49124562281141	1.87596695366144e-09\\
2.49324662331166	1.4590885915022e-09\\
2.49524762381191	1.4590885915022e-09\\
2.49724862431216	1.25064654563361e-09\\
2.49924962481241	1.04220449976502e-09\\
2.50125062531266	1.25064654563361e-09\\
2.50325162581291	1.04220449976502e-09\\
2.50525262631316	1.04220449976502e-09\\
2.50725362681341	1.04220449976502e-09\\
2.50925462731366	1.4590885915022e-09\\
2.51125562781391	1.25064654563361e-09\\
2.51325662831416	1.04220449976502e-09\\
2.51525762881441	1.04220449976502e-09\\
2.51725862931466	1.25064654563361e-09\\
2.51925962981491	8.33762453896423e-10\\
2.52126063031516	1.04220449976502e-09\\
2.52326163081541	1.04220449976502e-09\\
2.52526263131566	1.04220449976502e-09\\
2.52726363181591	1.04220449976502e-09\\
2.52926463231616	1.04220449976502e-09\\
2.53126563281641	8.33762453896423e-10\\
2.53326663331666	8.33762453896423e-10\\
2.53526763381691	8.33762453896423e-10\\
2.53726863431716	6.25320408027829e-10\\
2.53926963481741	1.04220449976502e-09\\
2.54127063531766	6.25320408027829e-10\\
2.54327163581791	6.25320408027829e-10\\
2.54527263631816	4.16881799906007e-10\\
2.54727363681841	4.16881799906007e-10\\
2.54927463731866	2.08440899953003e-10\\
2.55127563781891	2.08440899953003e-10\\
2.55327663831916	4.16881799906007e-10\\
2.55527763881941	-2.08440899953003e-10\\
2.55727863931966	-2.08440899953003e-10\\
2.55927963981991	-4.16881799906007e-10\\
2.56128064032016	-4.16881799906007e-10\\
2.56328164082041	-2.08440899953003e-10\\
2.56528264132066	-6.25320408027829e-10\\
2.56728364182091	-4.16881799906007e-10\\
2.56928464232116	-2.08440899953003e-10\\
2.57128564282141	2.08440899953003e-10\\
2.57328664332166	6.25320408027829e-10\\
2.57528764382191	0\\
2.57728864432216	4.16881799906007e-10\\
2.57928964482241	4.16881799906007e-10\\
2.58129064532266	6.25320408027829e-10\\
2.58329164582291	4.16881799906007e-10\\
2.58529264632316	2.08440899953003e-10\\
2.58729364682341	6.25320408027829e-10\\
2.58929464732366	4.16881799906007e-10\\
2.59129564782391	2.08440899953003e-10\\
2.59329664832416	4.16881799906007e-10\\
2.59529764882441	0\\
2.59729864932466	2.08440899953003e-10\\
2.59929964982491	6.25320408027829e-10\\
2.60130065032516	4.16881799906007e-10\\
2.60330165082541	6.25320408027829e-10\\
2.60530265132566	6.25320408027829e-10\\
2.60730365182591	6.25320408027829e-10\\
2.60930465232616	8.33762453896423e-10\\
2.61130565282641	4.16881799906007e-10\\
2.61330665332666	6.25320408027829e-10\\
2.61530765382691	8.33762453896423e-10\\
2.61730865432716	8.33762453896423e-10\\
2.61930965482741	1.25064654563361e-09\\
2.62131065532766	6.25320408027829e-10\\
2.62331165582791	8.33762453896423e-10\\
2.62531265632816	8.33762453896423e-10\\
2.62731365682841	8.33762453896423e-10\\
2.62931465732866	1.04220449976502e-09\\
2.63131565782891	1.04220449976502e-09\\
2.63331665832916	8.33762453896423e-10\\
2.63531765882941	6.25320408027829e-10\\
2.63731865932966	8.33762453896423e-10\\
2.63931965982992	6.25320408027829e-10\\
2.64132066033017	8.33762453896423e-10\\
2.64332166083042	1.04220449976502e-09\\
2.64532266133067	2.08440899953003e-10\\
2.64732366183092	6.25320408027829e-10\\
2.64932466233117	6.25320408027829e-10\\
2.65132566283142	8.33762453896423e-10\\
2.65332666333167	8.33762453896423e-10\\
2.65532766383192	8.33762453896423e-10\\
2.65732866433217	6.25320408027829e-10\\
2.65932966483242	4.16881799906007e-10\\
2.66133066533267	2.08440899953003e-10\\
2.66333166583292	-2.08440899953003e-10\\
2.66533266633317	0\\
2.66733366683342	0\\
2.66933466733367	-2.08440899953003e-10\\
2.67133566783392	-6.25320408027829e-10\\
2.67333666833417	-6.25320408027829e-10\\
2.67533766883442	-6.25320408027829e-10\\
2.67733866933467	-6.25320408027829e-10\\
2.67933966983492	-6.25320408027829e-10\\
2.68134067033517	-4.16881799906007e-10\\
2.68334167083542	-2.08440899953003e-10\\
2.68534267133567	-4.16881799906007e-10\\
2.68734367183592	-6.25320408027829e-10\\
2.68934467233617	-6.25320408027829e-10\\
2.69134567283642	-8.33762453896423e-10\\
2.69334667333667	-6.25320408027829e-10\\
2.69534767383692	-8.33762453896423e-10\\
2.69734867433717	0\\
2.69934967483742	0\\
2.70135067533767	-2.08440899953003e-10\\
2.70335167583792	-2.08440899953003e-10\\
2.70535267633817	-6.25320408027829e-10\\
2.70735367683842	-6.25320408027829e-10\\
2.70935467733867	-6.25320408027829e-10\\
2.71135567783892	-6.25320408027829e-10\\
2.71335667833917	-2.08440899953003e-10\\
2.71535767883942	2.08440899953003e-10\\
2.71735867933967	-4.16881799906007e-10\\
2.71935967983992	-2.08440899953003e-10\\
2.72136068034017	2.08440899953003e-10\\
2.72336168084042	-2.08440899953003e-10\\
2.72536268134067	4.16881799906007e-10\\
2.72736368184092	0\\
2.72936468234117	-2.08440899953003e-10\\
2.73136568284142	-8.33762453896423e-10\\
2.73336668334167	-6.25320408027829e-10\\
2.73536768384192	-4.16881799906007e-10\\
2.73736868434217	-6.25320408027829e-10\\
2.73936968484242	-8.33762453896423e-10\\
2.74137068534267	-8.33762453896423e-10\\
2.74337168584292	-1.04220449976502e-09\\
2.74537268634317	-1.25064654563361e-09\\
2.74737368684342	-1.25064654563361e-09\\
2.74937468734367	-1.25064654563361e-09\\
2.75137568784392	-1.25064654563361e-09\\
2.75337668834417	-1.4590885915022e-09\\
2.75537768884442	-1.4590885915022e-09\\
2.75737868934467	-1.4590885915022e-09\\
2.75937968984492	-1.66752490779285e-09\\
2.76138069034517	-1.4590885915022e-09\\
2.76338169084542	-1.25064654563361e-09\\
2.76538269134567	-6.25320408027829e-10\\
2.76738369184592	-1.04220449976502e-09\\
2.76938469234617	-1.04220449976502e-09\\
2.77138569284642	-1.25064654563361e-09\\
2.77338669334667	-1.04220449976502e-09\\
2.77538769384692	-1.04220449976502e-09\\
2.77738869434717	-1.4590885915022e-09\\
2.77938969484742	-6.25320408027829e-10\\
2.78139069534767	-6.25320408027829e-10\\
2.78339169584792	-4.16881799906007e-10\\
2.78539269634817	0\\
2.78739369684842	-2.08440899953003e-10\\
2.78939469734867	0\\
2.79139569784892	4.16881799906007e-10\\
2.79339669834917	4.16881799906007e-10\\
2.79539769884942	4.16881799906007e-10\\
2.79739869934967	4.16881799906007e-10\\
2.79939969984992	4.16881799906007e-10\\
2.80140070035018	4.16881799906007e-10\\
2.80340170085043	8.33762453896423e-10\\
2.80540270135068	1.04220449976502e-09\\
2.80740370185093	1.25064654563361e-09\\
2.80940470235118	1.4590885915022e-09\\
2.81140570285143	1.25064654563361e-09\\
2.81340670335168	1.4590885915022e-09\\
2.81540770385193	1.66752490779285e-09\\
2.81740870435218	1.4590885915022e-09\\
2.81940970485243	1.4590885915022e-09\\
2.82141070535268	1.66752490779285e-09\\
2.82341170585293	1.4590885915022e-09\\
2.82541270635318	8.33762453896423e-10\\
2.82741370685343	8.33762453896423e-10\\
2.82941470735368	8.33762453896423e-10\\
2.83141570785393	8.33762453896423e-10\\
2.83341670835418	1.25064654563361e-09\\
2.83541770885443	1.04220449976502e-09\\
2.83741870935468	1.04220449976502e-09\\
2.83941970985493	1.25064654563361e-09\\
2.84142071035518	1.4590885915022e-09\\
2.84342171085543	1.4590885915022e-09\\
2.84542271135568	1.25064654563361e-09\\
2.84742371185593	1.4590885915022e-09\\
2.84942471235618	1.66752490779285e-09\\
2.85142571285643	1.66752490779285e-09\\
2.85342671335668	1.25064654563361e-09\\
2.85542771385693	1.4590885915022e-09\\
2.85742871435718	1.4590885915022e-09\\
2.85942971485743	1.4590885915022e-09\\
2.86143071535768	2.08440899953003e-09\\
2.86343171585793	1.66752490779285e-09\\
2.86543271635818	1.25064654563361e-09\\
2.86743371685843	1.66752490779285e-09\\
2.86943471735868	1.66752490779285e-09\\
2.87143571785893	1.66752490779285e-09\\
2.87343671835918	2.29285104539863e-09\\
2.87543771885943	1.87596695366144e-09\\
2.87743871935968	1.25064654563361e-09\\
2.87943971985993	1.25064654563361e-09\\
2.88144072036018	1.4590885915022e-09\\
2.88344172086043	2.08440899953003e-09\\
2.88544272136068	2.08440899953003e-09\\
2.88744372186093	2.08440899953003e-09\\
2.88944472236118	2.50128736168927e-09\\
2.89144572286143	2.29285104539863e-09\\
2.89344672336168	2.29285104539863e-09\\
2.89544772386193	1.87596695366144e-09\\
2.89744872436218	1.87596695366144e-09\\
2.89944972486243	1.87596695366144e-09\\
2.90145072536268	2.50128736168927e-09\\
2.90345172586293	1.87596695366144e-09\\
2.90545272636318	2.08440899953003e-09\\
2.90745372686343	1.87596695366144e-09\\
2.90945472736368	2.08440899953003e-09\\
2.91145572786393	2.08440899953003e-09\\
2.91345672836418	2.08440899953003e-09\\
2.91545772886443	2.29285104539863e-09\\
2.91745872936468	1.87596695366144e-09\\
2.91945972986493	2.08440899953003e-09\\
2.92146073036518	2.50128736168927e-09\\
2.92346173086543	2.70972940755786e-09\\
2.92546273136568	3.12661349929505e-09\\
2.92746373186593	2.70972940755786e-09\\
2.92946473236618	2.70972940755786e-09\\
2.93146573286643	2.70972940755786e-09\\
2.93346673336668	2.70972940755786e-09\\
2.93546773386693	2.50128736168927e-09\\
2.93746873436718	2.08440899953003e-09\\
2.93946973486743	1.87596695366144e-09\\
2.94147073536768	1.25064654563361e-09\\
2.94347173586793	1.4590885915022e-09\\
2.94547273636818	1.4590885915022e-09\\
2.94747373686843	1.87596695366144e-09\\
2.94947473736868	1.66752490779285e-09\\
2.95147573786893	1.25064654563361e-09\\
2.95347673836918	1.4590885915022e-09\\
2.95547773886943	1.4590885915022e-09\\
2.95747873936968	1.04220449976502e-09\\
2.95947973986994	1.66752490779285e-09\\
2.96148074037019	1.66752490779285e-09\\
2.96348174087044	1.87596695366144e-09\\
2.96548274137069	2.08440899953003e-09\\
2.96748374187094	1.87596695366144e-09\\
2.96948474237119	1.66752490779285e-09\\
2.97148574287144	1.4590885915022e-09\\
2.97348674337169	1.25064654563361e-09\\
2.97548774387194	1.25064654563361e-09\\
2.97748874437219	1.04220449976502e-09\\
2.97948974487244	6.25320408027829e-10\\
2.98149074537269	2.08440899953003e-10\\
2.98349174587294	8.33762453896423e-10\\
2.98549274637319	4.16881799906007e-10\\
2.98749374687344	6.25320408027829e-10\\
2.98949474737369	4.16881799906007e-10\\
2.99149574787394	6.25320408027829e-10\\
2.99349674837419	4.16881799906007e-10\\
2.99549774887444	6.25320408027829e-10\\
2.99749874937469	1.04220449976502e-09\\
2.99949974987494	1.25064654563361e-09\\
3.00150075037519	1.66752490779285e-09\\
3.00350175087544	1.87596695366144e-09\\
3.00550275137569	1.4590885915022e-09\\
3.00750375187594	1.4590885915022e-09\\
3.00950475237619	1.25064654563361e-09\\
3.01150575287644	8.33762453896423e-10\\
3.01350675337669	6.25320408027829e-10\\
3.01550775387694	4.16881799906007e-10\\
3.01750875437719	2.08440899953003e-10\\
3.01950975487744	4.16881799906007e-10\\
3.02151075537769	2.08440899953003e-10\\
3.02351175587794	-2.08440899953003e-10\\
3.02551275637819	0\\
3.02751375687844	2.08440899953003e-10\\
3.02951475737869	-2.08440899953003e-10\\
3.03151575787894	-4.16881799906007e-10\\
3.03351675837919	0\\
3.03551775887944	4.16881799906007e-10\\
3.03751875937969	2.08440899953003e-10\\
3.03951975987994	2.08440899953003e-10\\
3.04152076038019	-2.08440899953003e-10\\
3.04352176088044	2.08440899953003e-10\\
3.04552276138069	6.25320408027829e-10\\
3.04752376188094	6.25320408027829e-10\\
3.04952476238119	4.16881799906007e-10\\
3.05152576288144	4.16881799906007e-10\\
3.05352676338169	2.08440899953003e-10\\
3.05552776388194	4.16881799906007e-10\\
3.05752876438219	-2.08440899953003e-10\\
3.05952976488244	-4.16881799906007e-10\\
3.06153076538269	-4.16881799906007e-10\\
3.06353176588294	-4.16881799906007e-10\\
3.06553276638319	-8.33762453896423e-10\\
3.06753376688344	-8.33762453896423e-10\\
3.06953476738369	-2.08440899953003e-10\\
3.07153576788394	-2.08440899953003e-10\\
3.07353676838419	-2.08440899953003e-10\\
3.07553776888444	-2.08440899953003e-10\\
3.07753876938469	-2.08440899953003e-10\\
3.07953976988494	-6.25320408027829e-10\\
3.08154077038519	-6.25320408027829e-10\\
3.08354177088544	-1.25064654563361e-09\\
3.08554277138569	-1.04220449976502e-09\\
3.08754377188594	-1.4590885915022e-09\\
3.08954477238619	-1.66752490779285e-09\\
3.09154577288644	-1.87596695366144e-09\\
3.09354677338669	-1.87596695366144e-09\\
3.09554777388694	-1.66752490779285e-09\\
3.09754877438719	-1.4590885915022e-09\\
3.09954977488744	-2.08440899953003e-09\\
3.10155077538769	-1.87596695366144e-09\\
3.10355177588794	-2.08440899953003e-09\\
3.10555277638819	-1.4590885915022e-09\\
3.10755377688844	-1.4590885915022e-09\\
3.10955477738869	-1.04220449976502e-09\\
3.11155577788894	-1.04220449976502e-09\\
3.11355677838919	-1.04220449976502e-09\\
3.11555777888944	-1.04220449976502e-09\\
3.11755877938969	-1.4590885915022e-09\\
3.11955977988994	-1.4590885915022e-09\\
3.1215607803902	-1.4590885915022e-09\\
3.12356178089045	-1.4590885915022e-09\\
3.1255627813907	-1.66752490779285e-09\\
3.12756378189095	-1.25064654563361e-09\\
3.1295647823912	-1.25064654563361e-09\\
3.13156578289145	-1.4590885915022e-09\\
3.1335667833917	-1.66752490779285e-09\\
3.13556778389195	-1.87596695366144e-09\\
3.1375687843922	-1.87596695366144e-09\\
3.13956978489245	-1.87596695366144e-09\\
3.1415707853927	-2.50128736168927e-09\\
3.14357178589295	-2.08440899953003e-09\\
3.1455727863932	-2.50128736168927e-09\\
3.14757378689345	-2.91817145342646e-09\\
3.1495747873937	-2.91817145342646e-09\\
3.15157578789395	-3.12661349929505e-09\\
3.1535767883942	-2.91817145342646e-09\\
3.15557778889445	-2.70972940755786e-09\\
3.1575787893947	-2.50128736168927e-09\\
3.15957978989495	-2.70972940755786e-09\\
3.1615807903952	-3.12661349929505e-09\\
3.16358179089545	-2.70972940755786e-09\\
3.1655827913957	-3.12661349929505e-09\\
3.16758379189595	-3.12661349929505e-09\\
3.1695847923962	-3.33505554516364e-09\\
3.17158579289645	-3.54349186145428e-09\\
3.1735867933967	-3.33505554516364e-09\\
3.17558779389695	-3.12661349929505e-09\\
3.1775887943972	-2.70972940755786e-09\\
3.17958979489745	-3.33505554516364e-09\\
3.1815907953977	-3.33505554516364e-09\\
3.18359179589795	-3.33505554516364e-09\\
3.1855927963982	-3.12661349929505e-09\\
3.18759379689845	-3.33505554516364e-09\\
3.1895947973987	-3.54349186145428e-09\\
3.19159579789895	-4.16881799906006e-09\\
3.1935967983992	-3.75193390732288e-09\\
3.19559779889945	-4.16881799906006e-09\\
3.1975987993997	-3.96037595319147e-09\\
3.19959979989995	-4.16881799906006e-09\\
3.2016008004002	-3.96037595319147e-09\\
3.20360180090045	-3.54349186145428e-09\\
3.2056028014007	-3.75193390732288e-09\\
3.20760380190095	-3.75193390732288e-09\\
3.2096048024012	-3.54349186145428e-09\\
3.21160580290145	-3.75193390732288e-09\\
3.2136068034017	-3.75193390732288e-09\\
3.21560780390195	-3.96037595319147e-09\\
3.2176088044022	-3.96037595319147e-09\\
3.21960980490245	-3.75193390732288e-09\\
3.2216108054027	-4.37726004492866e-09\\
3.22361180590295	-4.5856963612193e-09\\
3.2256128064032	-4.79413840708789e-09\\
3.22761380690345	-5.41945881511572e-09\\
3.2296148074037	-5.00258045295649e-09\\
3.23161580790395	-5.21102249882508e-09\\
3.2336168084042	-5.21102249882508e-09\\
3.23561780890445	-5.00258045295649e-09\\
3.2376188094047	-4.5856963612193e-09\\
3.23961980990495	-4.79413840708789e-09\\
3.2416208104052	-4.79413840708789e-09\\
3.24362181090545	-5.00258045295649e-09\\
3.2456228114057	-4.5856963612193e-09\\
3.24762381190595	-5.00258045295649e-09\\
3.2496248124062	-5.00258045295649e-09\\
3.25162581290645	-4.5856963612193e-09\\
3.2536268134067	-3.96037595319147e-09\\
3.25562781390695	-4.16881799906006e-09\\
3.2576288144072	-3.96037595319147e-09\\
3.25962981490745	-4.5856963612193e-09\\
3.2616308154077	-5.00258045295649e-09\\
3.26363181590795	-4.5856963612193e-09\\
3.2656328164082	-4.79413840708789e-09\\
3.26763381690845	-4.5856963612193e-09\\
3.2696348174087	-4.79413840708789e-09\\
3.27163581790895	-5.21102249882508e-09\\
3.2736368184092	-5.00258045295649e-09\\
3.27563781890945	-5.00258045295649e-09\\
3.27763881940971	-4.79413840708789e-09\\
3.27963981990996	-5.00258045295649e-09\\
3.2816408204102	-5.41945881511572e-09\\
3.28364182091046	-5.21102249882508e-09\\
3.28564282141071	-5.00258045295649e-09\\
3.28764382191096	-5.21102249882508e-09\\
3.28964482241121	-5.41945881511572e-09\\
3.29164582291146	-4.5856963612193e-09\\
3.29364682341171	-4.79413840708789e-09\\
3.29564782391196	-5.21102249882508e-09\\
3.29764882441221	-5.00258045295649e-09\\
3.29964982491246	-4.79413840708789e-09\\
3.30165082541271	-4.5856963612193e-09\\
3.30365182591296	-4.5856963612193e-09\\
3.30565282641321	-4.5856963612193e-09\\
3.30765382691346	-5.00258045295649e-09\\
3.30965482741371	-4.16881799906006e-09\\
3.31165582791396	-4.37726004492866e-09\\
3.31365682841421	-5.00258045295649e-09\\
3.31565782891446	-4.79413840708789e-09\\
3.31765882941471	-4.5856963612193e-09\\
3.31965982991496	-4.79413840708789e-09\\
3.32166083041521	-4.79413840708789e-09\\
3.32366183091546	-4.5856963612193e-09\\
3.32566283141571	-4.16881799906006e-09\\
3.32766383191596	-4.16881799906006e-09\\
3.32966483241621	-4.79413840708789e-09\\
3.33166583291646	-4.79413840708789e-09\\
3.33366683341671	-4.79413840708789e-09\\
3.33566783391696	-4.79413840708789e-09\\
3.33766883441721	-4.37726004492866e-09\\
3.33966983491746	-4.5856963612193e-09\\
3.34167083541771	-4.16881799906006e-09\\
3.34367183591796	-4.37726004492866e-09\\
3.34567283641821	-4.37726004492866e-09\\
3.34767383691846	-4.5856963612193e-09\\
3.34967483741871	-4.37726004492866e-09\\
3.35167583791896	-4.5856963612193e-09\\
3.35367683841921	-5.00258045295649e-09\\
3.35567783891946	-5.41945881511572e-09\\
3.35767883941971	-5.62790086098432e-09\\
3.35967983991996	-4.79413840708789e-09\\
3.36168084042021	-5.00258045295649e-09\\
3.36368184092046	-4.79413840708789e-09\\
3.36568284142071	-4.37726004492866e-09\\
3.36768384192096	-4.5856963612193e-09\\
3.36968484242121	-4.5856963612193e-09\\
3.37168584292146	-5.00258045295649e-09\\
3.37368684342171	-5.00258045295649e-09\\
3.37568784392196	-5.00258045295649e-09\\
3.37768884442221	-5.41945881511572e-09\\
3.37968984492246	-5.21102249882508e-09\\
3.38169084542271	-5.00258045295649e-09\\
3.38369184592296	-4.79413840708789e-09\\
3.38569284642321	-4.5856963612193e-09\\
3.38769384692346	-5.00258045295649e-09\\
3.38969484742371	-5.21102249882508e-09\\
3.39169584792396	-5.00258045295649e-09\\
3.39369684842421	-5.00258045295649e-09\\
3.39569784892446	-5.21102249882508e-09\\
3.39769884942471	-4.79413840708789e-09\\
3.39969984992496	-4.79413840708789e-09\\
3.40170085042521	-4.79413840708789e-09\\
3.40370185092546	-4.37726004492866e-09\\
3.40570285142571	-4.16881799906006e-09\\
3.40770385192596	-4.16881799906006e-09\\
3.40970485242621	-3.75193390732288e-09\\
3.41170585292646	-3.75193390732288e-09\\
3.41370685342671	-3.96037595319147e-09\\
3.41570785392696	-4.37726004492866e-09\\
3.41770885442721	-4.5856963612193e-09\\
3.41970985492746	-5.21102249882508e-09\\
3.42171085542771	-5.21102249882508e-09\\
3.42371185592796	-5.41945881511572e-09\\
3.42571285642821	-5.00258045295649e-09\\
3.42771385692846	-5.21102249882508e-09\\
3.42971485742871	-5.21102249882508e-09\\
3.43171585792896	-5.62790086098432e-09\\
3.43371685842921	-5.21102249882508e-09\\
3.43571785892946	-5.41945881511572e-09\\
3.43771885942971	-5.41945881511572e-09\\
3.43971985992996	-5.41945881511572e-09\\
3.44172086043022	-5.8363199885411e-09\\
3.44372186093047	-5.8363199885411e-09\\
3.44572286143072	-5.8363199885411e-09\\
3.44772386193097	-6.0447620344097e-09\\
3.44972486243122	-5.8363199885411e-09\\
3.45172586293147	-6.0447620344097e-09\\
3.45372686343172	-5.62790086098432e-09\\
3.45572786393197	-5.62790086098432e-09\\
3.45772886443222	-5.62790086098432e-09\\
3.45972986493247	-5.62790086098432e-09\\
3.46173086543272	-5.62790086098432e-09\\
3.46373186593297	-5.41945881511572e-09\\
3.46573286643322	-5.8363199885411e-09\\
3.46773386693347	-5.8363199885411e-09\\
3.46973486743372	-5.8363199885411e-09\\
3.47173586793397	-5.8363199885411e-09\\
3.47373686843422	-6.0447620344097e-09\\
3.47573786893447	-6.0447620344097e-09\\
3.47773886943472	-6.0447620344097e-09\\
3.47973986993497	-6.25320408027829e-09\\
3.48174087043522	-6.25320408027829e-09\\
3.48374187093547	-6.25320408027829e-09\\
3.48574287143572	-6.46164612614688e-09\\
3.48774387193597	-6.46164612614688e-09\\
3.48974487243622	-6.67008817201548e-09\\
3.49174587293647	-6.67008817201548e-09\\
3.49374687343672	-6.87853021788407e-09\\
3.49574787393697	-6.87853021788407e-09\\
3.49774887443722	-6.67008817201548e-09\\
3.49974987493747	-7.08697226375267e-09\\
3.50175087543772	-6.87853021788407e-09\\
3.50375187593797	-6.87853021788407e-09\\
3.50575287643822	-6.87853021788407e-09\\
3.50775387693847	-6.87853021788407e-09\\
3.50975487743872	-7.08697226375267e-09\\
3.51175587793897	-6.87853021788407e-09\\
3.51375687843922	-6.46164612614688e-09\\
3.51575787893947	-6.67008817201548e-09\\
3.51775887943972	-6.87853021788407e-09\\
3.51975987993997	-6.87853021788407e-09\\
3.52176088044022	-6.67008817201548e-09\\
3.52376188094047	-7.08697226375267e-09\\
3.52576288144072	-7.08697226375267e-09\\
3.52776388194097	-7.08697226375267e-09\\
3.52976488244122	-6.46164612614688e-09\\
3.53176588294147	-6.87853021788407e-09\\
3.53376688344172	-6.87853021788407e-09\\
3.53576788394197	-6.87853021788407e-09\\
3.53776888444222	-6.87853021788407e-09\\
3.53976988494247	-6.87853021788407e-09\\
3.54177088544272	-6.46164612614688e-09\\
3.54377188594297	-6.25320408027829e-09\\
3.54577288644322	-6.87853021788407e-09\\
3.54777388694347	-6.67008817201548e-09\\
3.54977488744372	-7.29541430962126e-09\\
3.55177588794397	-7.29541430962126e-09\\
3.55377688844422	-7.71229840135845e-09\\
3.55577788894447	-7.71229840135845e-09\\
3.55777888944472	-7.92074044722704e-09\\
3.55977988994497	-7.71229840135845e-09\\
3.56178089044522	-7.71229840135845e-09\\
3.56378189094547	-7.92074044722704e-09\\
3.56578289144572	-8.54606658483282e-09\\
3.56778389194597	-8.54606658483282e-09\\
3.56978489244622	-8.54606658483282e-09\\
3.57178589294647	-8.75450863070141e-09\\
3.57378689344672	-8.54606658483282e-09\\
3.57578789394697	-8.75450863070141e-09\\
3.57778889444722	-8.33762453896423e-09\\
3.57978989494747	-8.54606658483282e-09\\
3.58179089544772	-8.96295067657001e-09\\
3.58379189594797	-9.1713927224386e-09\\
3.58579289644822	-8.96295067657001e-09\\
3.58779389694847	-8.75450863070141e-09\\
3.58979489744872	-8.75450863070141e-09\\
3.59179589794897	-8.75450863070141e-09\\
3.59379689844922	-8.54606658483282e-09\\
3.59579789894947	-8.33762453896423e-09\\
3.59779889944972	-8.54606658483282e-09\\
3.59979989994997	-8.12918249309563e-09\\
3.60180090045022	-8.75450863070141e-09\\
3.60380190095048	-8.54606658483282e-09\\
3.60580290145073	-8.33762453896423e-09\\
3.60780390195098	-8.33762453896423e-09\\
3.60980490245123	-8.12918249309563e-09\\
3.61180590295148	-7.92074044722704e-09\\
3.61380690345173	-8.12918249309563e-09\\
3.61580790395198	-7.71229840135845e-09\\
3.61780890445223	-7.92074044722704e-09\\
3.61980990495248	-7.71229840135845e-09\\
3.62181090545273	-7.71229840135845e-09\\
3.62381190595298	-7.29541430962126e-09\\
3.62581290645323	-7.29541430962126e-09\\
3.62781390695348	-7.50385635548985e-09\\
3.62981490745373	-7.08697226375267e-09\\
3.63181590795398	-7.08697226375267e-09\\
3.63381690845423	-6.87853021788407e-09\\
3.63581790895448	-6.46164612614688e-09\\
3.63781890945473	-7.08697226375267e-09\\
3.63981990995498	-7.71229840135845e-09\\
3.64182091045523	-7.92074044722704e-09\\
3.64382191095548	-7.92074044722704e-09\\
3.64582291145573	-7.29541430962126e-09\\
3.64782391195598	-7.50385635548985e-09\\
3.64982491245623	-7.29541430962126e-09\\
3.65182591295648	-7.08697226375267e-09\\
3.65382691345673	-7.08697226375267e-09\\
3.65582791395698	-7.50385635548985e-09\\
3.65782891445723	-7.29541430962126e-09\\
3.65982991495748	-7.08697226375267e-09\\
3.66183091545773	-7.29541430962126e-09\\
3.66383191595798	-7.29541430962126e-09\\
3.66583291645823	-7.29541430962126e-09\\
3.66783391695848	-7.50385635548985e-09\\
3.66983491745873	-7.71229840135845e-09\\
3.67183591795898	-7.29541430962126e-09\\
3.67383691845923	-7.71229840135845e-09\\
3.67583791895948	-7.50385635548985e-09\\
3.67783891945973	-7.50385635548985e-09\\
3.67983991995998	-6.87853021788407e-09\\
3.68184092046023	-6.87853021788407e-09\\
3.68384192096048	-6.87853021788407e-09\\
3.68584292146073	-7.29541430962126e-09\\
3.68784392196098	-7.29541430962126e-09\\
3.68984492246123	-7.29541430962126e-09\\
3.69184592296148	-7.29541430962126e-09\\
3.69384692346173	-7.08697226375267e-09\\
3.69584792396198	-7.08697226375267e-09\\
3.69784892446223	-6.87853021788407e-09\\
3.69984992496248	-6.67008817201548e-09\\
3.70185092546273	-6.25320408027829e-09\\
3.70385192596298	-6.46164612614688e-09\\
3.70585292646323	-6.25320408027829e-09\\
3.70785392696348	-6.46164612614688e-09\\
3.70985492746373	-6.0447620344097e-09\\
3.71185592796398	-6.25320408027829e-09\\
3.71385692846423	-6.0447620344097e-09\\
3.71585792896448	-5.21102249882508e-09\\
3.71785892946473	-5.8363199885411e-09\\
3.71985992996498	-6.25320408027829e-09\\
3.72186093046523	-6.46164612614688e-09\\
3.72386193096548	-6.0447620344097e-09\\
3.72586293146573	-6.0447620344097e-09\\
3.72786393196598	-6.0447620344097e-09\\
3.72986493246623	-5.62790086098432e-09\\
3.73186593296648	-5.41945881511572e-09\\
3.73386693346673	-5.8363199885411e-09\\
3.73586793396698	-5.8363199885411e-09\\
3.73786893446723	-5.8363199885411e-09\\
3.73986993496748	-6.0447620344097e-09\\
3.74187093546773	-6.0447620344097e-09\\
3.74387193596798	-5.8363199885411e-09\\
3.74587293646823	-5.41945881511572e-09\\
3.74787393696848	-5.41945881511572e-09\\
3.74987493746873	-5.62790086098432e-09\\
3.75187593796898	-6.0447620344097e-09\\
3.75387693846923	-5.8363199885411e-09\\
3.75587793896948	-5.62790086098432e-09\\
3.75787893946973	-6.25320408027829e-09\\
3.75987993996999	-6.25320408027829e-09\\
3.76188094047024	-6.46164612614688e-09\\
3.76388194097049	-6.67008817201548e-09\\
3.76588294147074	-6.25320408027829e-09\\
3.76788394197099	-5.62790086098432e-09\\
3.76988494247124	-5.62790086098432e-09\\
3.77188594297149	-6.0447620344097e-09\\
3.77388694347174	-6.0447620344097e-09\\
3.77588794397199	-5.8363199885411e-09\\
3.77788894447224	-6.25320408027829e-09\\
3.77988994497249	-6.25320408027829e-09\\
3.78189094547274	-6.25320408027829e-09\\
3.78389194597299	-6.67008817201548e-09\\
3.78589294647324	-6.25320408027829e-09\\
3.78789394697349	-5.8363199885411e-09\\
3.78989494747374	-5.62790086098432e-09\\
3.79189594797399	-5.41945881511572e-09\\
3.79389694847424	-4.79413840708789e-09\\
3.79589794897449	-4.5856963612193e-09\\
3.79789894947474	-5.00258045295649e-09\\
3.79989994997499	-5.62790086098432e-09\\
3.80190095047524	-5.62790086098432e-09\\
3.80390195097549	-5.62790086098432e-09\\
3.80590295147574	-6.25320408027829e-09\\
3.80790395197599	-5.62790086098432e-09\\
3.80990495247624	-6.0447620344097e-09\\
3.81190595297649	-6.0447620344097e-09\\
3.81390695347674	-6.0447620344097e-09\\
3.81590795397699	-6.25320408027829e-09\\
3.81790895447724	-6.87853021788407e-09\\
3.81990995497749	-6.67008817201548e-09\\
3.82191095547774	-6.46164612614688e-09\\
3.82391195597799	-6.67008817201548e-09\\
3.82591295647824	-7.08697226375267e-09\\
3.82791395697849	-7.29541430962126e-09\\
3.82991495747874	-7.29541430962126e-09\\
3.83191595797899	-7.29541430962126e-09\\
3.83391695847924	-6.46164612614688e-09\\
3.83591795897949	-6.87853021788407e-09\\
3.83791895947974	-7.29541430962126e-09\\
3.83991995997999	-6.67008817201548e-09\\
3.84192096048024	-7.08697226375267e-09\\
3.84392196098049	-6.67008817201548e-09\\
3.84592296148074	-7.08697226375267e-09\\
3.84792396198099	-6.87853021788407e-09\\
3.84992496248124	-6.25320408027829e-09\\
3.85192596298149	-7.08697226375267e-09\\
3.85392696348174	-7.08697226375267e-09\\
3.85592796398199	-7.29541430962126e-09\\
3.85792896448224	-7.08697226375267e-09\\
3.85992996498249	-7.08697226375267e-09\\
3.86193096548274	-7.29541430962126e-09\\
3.86393196598299	-7.29541430962126e-09\\
3.86593296648324	-7.92074044722704e-09\\
3.86793396698349	-7.71229840135845e-09\\
3.86993496748374	-7.71229840135845e-09\\
3.87193596798399	-7.50385635548985e-09\\
3.87393696848424	-7.29541430962126e-09\\
3.87593796898449	-7.29541430962126e-09\\
3.87793896948474	-7.50385635548985e-09\\
3.87993996998499	-7.29541430962126e-09\\
3.88194097048524	-6.67008817201548e-09\\
3.88394197098549	-6.46164612614688e-09\\
3.88594297148574	-6.25320408027829e-09\\
3.88794397198599	-6.46164612614688e-09\\
3.88994497248624	-6.87853021788407e-09\\
3.89194597298649	-6.46164612614688e-09\\
3.89394697348674	-7.08697226375267e-09\\
3.89594797398699	-6.67008817201548e-09\\
3.89794897448724	-6.46164612614688e-09\\
3.89994997498749	-6.87853021788407e-09\\
3.90195097548774	-6.67008817201548e-09\\
3.90395197598799	-6.46164612614688e-09\\
3.90595297648824	-6.25320408027829e-09\\
3.90795397698849	-6.46164612614688e-09\\
3.90995497748874	-6.67008817201548e-09\\
3.91195597798899	-6.46164612614688e-09\\
3.91395697848924	-6.0447620344097e-09\\
3.91595797898949	-5.62790086098432e-09\\
3.91795897948974	-5.41945881511572e-09\\
3.91995997998999	-5.41945881511572e-09\\
3.92196098049025	-5.00258045295649e-09\\
3.9239619809905	-5.21102249882508e-09\\
3.92596298149075	-4.5856963612193e-09\\
3.927963981991	-4.5856963612193e-09\\
3.92996498249125	-4.79413840708789e-09\\
3.9319659829915	-5.00258045295649e-09\\
3.93396698349175	-5.41945881511572e-09\\
3.935967983992	-5.21102249882508e-09\\
3.93796898449225	-5.21102249882508e-09\\
3.9399699849925	-5.21102249882508e-09\\
3.94197098549275	-4.37726004492866e-09\\
3.943971985993	-3.96037595319147e-09\\
3.94597298649325	-3.96037595319147e-09\\
3.9479739869935	-3.54349186145428e-09\\
3.94997498749375	-3.75193390732288e-09\\
3.951975987994	-3.96037595319147e-09\\
3.95397698849425	-4.37726004492866e-09\\
3.9559779889945	-3.96037595319147e-09\\
3.95797898949475	-4.16881799906006e-09\\
3.959979989995	-3.75193390732288e-09\\
3.96198099049525	-3.12661349929505e-09\\
3.9639819909955	-3.12661349929505e-09\\
3.96598299149575	-3.33505554516364e-09\\
3.967983991996	-3.12661349929505e-09\\
3.96998499249625	-3.12661349929505e-09\\
3.9719859929965	-2.91817145342646e-09\\
3.97398699349675	-3.33505554516364e-09\\
3.975987993997	-3.54349186145428e-09\\
3.97798899449725	-3.96037595319147e-09\\
3.9799899949975	-4.16881799906006e-09\\
3.98199099549775	-4.16881799906006e-09\\
3.983991995998	-3.96037595319147e-09\\
3.98599299649825	-3.75193390732288e-09\\
3.9879939969985	-3.54349186145428e-09\\
3.98999499749875	-3.33505554516364e-09\\
3.991995997999	-3.54349186145428e-09\\
3.99399699849925	-3.33505554516364e-09\\
3.9959979989995	-3.33505554516364e-09\\
3.99799899949975	-3.12661349929505e-09\\
4	-2.70972940755786e-09\\
};
\addlegendentry{c1};

\addplot [color=mycolor2,solid]
  table[row sep=crcr]{%
0	0\\
0.00200100050025012	4.16881799906007e-10\\
0.00400200100050025	6.25320408027829e-10\\
0.00600300150075038	4.16881799906007e-10\\
0.0080040020010005	4.16881799906007e-10\\
0.0100050025012506	6.25320408027829e-10\\
0.0120060030015008	4.16881799906007e-10\\
0.0140070035017509	1.04220449976502e-09\\
0.016008004002001	1.66752490779285e-09\\
0.0180090045022511	1.4590885915022e-09\\
0.0200100050025013	1.4590885915022e-09\\
0.0220110055027514	1.04220449976502e-09\\
0.0240120060030015	6.25320408027829e-10\\
0.0260130065032516	6.25320408027829e-10\\
0.0280140070035018	0\\
0.0300150075037519	-2.08440899953003e-10\\
0.032016008004002	-2.08440899953003e-10\\
0.0340170085042521	-2.08440899953003e-10\\
0.0360180090045022	-2.08440899953003e-10\\
0.0380190095047524	-2.08440899953003e-10\\
0.0400200100050025	-2.08440899953003e-10\\
0.0420210105052526	-2.08440899953003e-10\\
0.0440220110055028	-2.08440899953003e-10\\
0.0460230115057529	-2.08440899953003e-10\\
0.048024012006003	-4.16881799906007e-10\\
0.0500250125062531	-4.16881799906007e-10\\
0.0520260130065033	-2.08440899953003e-10\\
0.0540270135067534	-6.25320408027829e-10\\
0.0560280140070035	-4.16881799906007e-10\\
0.0580290145072536	-2.08440899953003e-10\\
0.0600300150075038	-4.16881799906007e-10\\
0.0620310155077539	-6.25320408027829e-10\\
0.064032016008004	-8.33762453896423e-10\\
0.0660330165082541	-8.33762453896423e-10\\
0.0680340170085043	-6.25320408027829e-10\\
0.0700350175087544	-4.16881799906007e-10\\
0.0720360180090045	-2.08440899953003e-10\\
0.0740370185092546	-2.08440899953003e-10\\
0.0760380190095048	-2.08440899953003e-10\\
0.0780390195097549	0\\
0.080040020010005	-4.16881799906007e-10\\
0.0820410205102551	-8.33762453896423e-10\\
0.0840420210105053	-8.33762453896423e-10\\
0.0860430215107554	-1.04220449976502e-09\\
0.0880440220110055	-1.4590885915022e-09\\
0.0900450225112556	-1.4590885915022e-09\\
0.0920460230115058	-1.25064654563361e-09\\
0.0940470235117559	-1.25064654563361e-09\\
0.096048024012006	-1.04220449976502e-09\\
0.0980490245122561	-8.33762453896423e-10\\
0.100050025012506	-8.33762453896423e-10\\
0.102051025512756	-1.4590885915022e-09\\
0.104052026013007	-1.66752490779285e-09\\
0.106053026513257	-1.66752490779285e-09\\
0.108054027013507	-1.66752490779285e-09\\
0.110055027513757	-1.66752490779285e-09\\
0.112056028014007	-1.4590885915022e-09\\
0.114057028514257	-1.4590885915022e-09\\
0.116058029014507	-1.04220449976502e-09\\
0.118059029514757	-1.4590885915022e-09\\
0.120060030015008	-1.4590885915022e-09\\
0.122061030515258	-1.4590885915022e-09\\
0.124062031015508	-1.66752490779285e-09\\
0.126063031515758	-2.08440899953003e-09\\
0.128064032016008	-1.87596695366144e-09\\
0.130065032516258	-2.29285104539863e-09\\
0.132066033016508	-2.08440899953003e-09\\
0.134067033516758	-2.50128736168927e-09\\
0.136068034017009	-2.91817145342646e-09\\
0.138069034517259	-2.70972940755786e-09\\
0.140070035017509	-2.70972940755786e-09\\
0.142071035517759	-2.91817145342646e-09\\
0.144072036018009	-2.29285104539863e-09\\
0.146073036518259	-2.91817145342646e-09\\
0.148074037018509	-2.50128736168927e-09\\
0.150075037518759	-2.91817145342646e-09\\
0.15207603801901	-2.91817145342646e-09\\
0.15407703851926	-2.29285104539863e-09\\
0.15607803901951	-2.29285104539863e-09\\
0.15807903951976	-2.70972940755786e-09\\
0.16008004002001	-2.50128736168927e-09\\
0.16208104052026	-2.50128736168927e-09\\
0.16408204102051	-2.50128736168927e-09\\
0.16608304152076	-2.70972940755786e-09\\
0.168084042021011	-2.50128736168927e-09\\
0.170085042521261	-1.87596695366144e-09\\
0.172086043021511	-2.08440899953003e-09\\
0.174087043521761	-2.08440899953003e-09\\
0.176088044022011	-2.08440899953003e-09\\
0.178089044522261	-2.29285104539863e-09\\
0.180090045022511	-2.50128736168927e-09\\
0.182091045522761	-2.70972940755786e-09\\
0.184092046023012	-2.50128736168927e-09\\
0.186093046523262	-2.29285104539863e-09\\
0.188094047023512	-2.91817145342646e-09\\
0.190095047523762	-2.70972940755786e-09\\
0.192096048024012	-2.91817145342646e-09\\
0.194097048524262	-2.91817145342646e-09\\
0.196098049024512	-2.70972940755786e-09\\
0.198099049524762	-2.91817145342646e-09\\
0.200100050025012	-2.91817145342646e-09\\
0.202101050525263	-2.91817145342646e-09\\
0.204102051025513	-2.70972940755786e-09\\
0.206103051525763	-2.70972940755786e-09\\
0.208104052026013	-2.50128736168927e-09\\
0.210105052526263	-1.87596695366144e-09\\
0.212106053026513	-2.50128736168927e-09\\
0.214107053526763	-2.50128736168927e-09\\
0.216108054027013	-2.91817145342646e-09\\
0.218109054527264	-2.91817145342646e-09\\
0.220110055027514	-2.50128736168927e-09\\
0.222111055527764	-2.50128736168927e-09\\
0.224112056028014	-2.29285104539863e-09\\
0.226113056528264	-2.70972940755786e-09\\
0.228114057028514	-3.12661349929505e-09\\
0.230115057528764	-3.12661349929505e-09\\
0.232116058029014	-2.91817145342646e-09\\
0.234117058529265	-3.12661349929505e-09\\
0.236118059029515	-3.12661349929505e-09\\
0.238119059529765	-3.12661349929505e-09\\
0.240120060030015	-3.33505554516364e-09\\
0.242121060530265	-2.91817145342646e-09\\
0.244122061030515	-3.54349186145428e-09\\
0.246123061530765	-3.12661349929505e-09\\
0.248124062031016	-3.12661349929505e-09\\
0.250125062531266	-3.33505554516364e-09\\
0.252126063031516	-2.91817145342646e-09\\
0.254127063531766	-3.33505554516364e-09\\
0.256128064032016	-3.96037595319147e-09\\
0.258129064532266	-3.96037595319147e-09\\
0.260130065032516	-3.96037595319147e-09\\
0.262131065532766	-3.75193390732288e-09\\
0.264132066033017	-3.75193390732288e-09\\
0.266133066533267	-3.54349186145428e-09\\
0.268134067033517	-3.54349186145428e-09\\
0.270135067533767	-3.33505554516364e-09\\
0.272136068034017	-3.54349186145428e-09\\
0.274137068534267	-3.12661349929505e-09\\
0.276138069034517	-3.12661349929505e-09\\
0.278139069534767	-3.33505554516364e-09\\
0.280140070035018	-3.33505554516364e-09\\
0.282141070535268	-3.54349186145428e-09\\
0.284142071035518	-3.75193390732288e-09\\
0.286143071535768	-3.54349186145428e-09\\
0.288144072036018	-3.54349186145428e-09\\
0.290145072536268	-3.75193390732288e-09\\
0.292146073036518	-3.33505554516364e-09\\
0.294147073536768	-3.33505554516364e-09\\
0.296148074037018	-3.33505554516364e-09\\
0.298149074537269	-3.12661349929505e-09\\
0.300150075037519	-3.33505554516364e-09\\
0.302151075537769	-3.33505554516364e-09\\
0.304152076038019	-3.33505554516364e-09\\
0.306153076538269	-3.54349186145428e-09\\
0.308154077038519	-3.54349186145428e-09\\
0.310155077538769	-3.75193390732288e-09\\
0.31215607803902	-3.75193390732288e-09\\
0.31415707853927	-3.75193390732288e-09\\
0.31615807903952	-3.75193390732288e-09\\
0.31815907953977	-3.54349186145428e-09\\
0.32016008004002	-3.75193390732288e-09\\
0.32216108054027	-3.75193390732288e-09\\
0.32416208104052	-3.75193390732288e-09\\
0.32616308154077	-3.96037595319147e-09\\
0.32816408204102	-3.75193390732288e-09\\
0.330165082541271	-3.75193390732288e-09\\
0.332166083041521	-4.16881799906006e-09\\
0.334167083541771	-3.96037595319147e-09\\
0.336168084042021	-4.37726004492866e-09\\
0.338169084542271	-4.37726004492866e-09\\
0.340170085042521	-3.96037595319147e-09\\
0.342171085542771	-3.75193390732288e-09\\
0.344172086043022	-4.16881799906006e-09\\
0.346173086543272	-4.16881799906006e-09\\
0.348174087043522	-4.16881799906006e-09\\
0.350175087543772	-3.75193390732288e-09\\
0.352176088044022	-3.75193390732288e-09\\
0.354177088544272	-3.75193390732288e-09\\
0.356178089044522	-3.75193390732288e-09\\
0.358179089544772	-3.75193390732288e-09\\
0.360180090045022	-3.75193390732288e-09\\
0.362181090545273	-3.54349186145428e-09\\
0.364182091045523	-3.12661349929505e-09\\
0.366183091545773	-2.91817145342646e-09\\
0.368184092046023	-3.33505554516364e-09\\
0.370185092546273	-3.54349186145428e-09\\
0.372186093046523	-3.75193390732288e-09\\
0.374187093546773	-3.75193390732288e-09\\
0.376188094047024	-3.54349186145428e-09\\
0.378189094547274	-3.54349186145428e-09\\
0.380190095047524	-3.54349186145428e-09\\
0.382191095547774	-3.12661349929505e-09\\
0.384192096048024	-2.70972940755786e-09\\
0.386193096548274	-2.29285104539863e-09\\
0.388194097048524	-1.66752490779285e-09\\
0.390195097548774	-1.66752490779285e-09\\
0.392196098049024	-1.4590885915022e-09\\
0.394197098549275	-1.04220449976502e-09\\
0.396198099049525	-1.4590885915022e-09\\
0.398199099549775	-1.4590885915022e-09\\
0.400200100050025	-1.87596695366144e-09\\
0.402201100550275	-1.66752490779285e-09\\
0.404202101050525	-1.87596695366144e-09\\
0.406203101550775	-1.66752490779285e-09\\
0.408204102051026	-1.4590885915022e-09\\
0.410205102551276	-1.66752490779285e-09\\
0.412206103051526	-1.87596695366144e-09\\
0.414207103551776	-1.4590885915022e-09\\
0.416208104052026	-1.4590885915022e-09\\
0.418209104552276	-1.4590885915022e-09\\
0.420210105052526	-1.25064654563361e-09\\
0.422211105552776	-1.25064654563361e-09\\
0.424212106053027	-6.25320408027829e-10\\
0.426213106553277	-6.25320408027829e-10\\
0.428214107053527	-2.08440899953003e-10\\
0.430215107553777	-2.08440899953003e-10\\
0.432216108054027	-2.08440899953003e-10\\
0.434217108554277	-4.16881799906007e-10\\
0.436218109054527	-2.08440899953003e-10\\
0.438219109554777	0\\
0.440220110055028	2.08440899953003e-10\\
0.442221110555278	0\\
0.444222111055528	-2.08440899953003e-10\\
0.446223111555778	-2.08440899953003e-10\\
0.448224112056028	-6.25320408027829e-10\\
0.450225112556278	-2.08440899953003e-10\\
0.452226113056528	-2.08440899953003e-10\\
0.454227113556778	-4.16881799906007e-10\\
0.456228114057029	0\\
0.458229114557279	4.16881799906007e-10\\
0.460230115057529	2.08440899953003e-10\\
0.462231115557779	-2.08440899953003e-10\\
0.464232116058029	-8.33762453896423e-10\\
0.466233116558279	-1.25064654563361e-09\\
0.468234117058529	-1.04220449976502e-09\\
0.470235117558779	-1.4590885915022e-09\\
0.47223611805903	-1.4590885915022e-09\\
0.47423711855928	-1.25064654563361e-09\\
0.47623811905953	-1.25064654563361e-09\\
0.47823911955978	-1.66752490779285e-09\\
0.48024012006003	-2.08440899953003e-09\\
0.48224112056028	-1.87596695366144e-09\\
0.48424212106053	-1.66752490779285e-09\\
0.48624312156078	-1.4590885915022e-09\\
0.488244122061031	-2.08440899953003e-09\\
0.490245122561281	-1.87596695366144e-09\\
0.492246123061531	-2.08440899953003e-09\\
0.494247123561781	-2.08440899953003e-09\\
0.496248124062031	-2.50128736168927e-09\\
0.498249124562281	-1.87596695366144e-09\\
0.500250125062531	-1.87596695366144e-09\\
0.502251125562781	-2.29285104539863e-09\\
0.504252126063031	-2.08440899953003e-09\\
0.506253126563282	-1.87596695366144e-09\\
0.508254127063532	-1.87596695366144e-09\\
0.510255127563782	-1.66752490779285e-09\\
0.512256128064032	-1.66752490779285e-09\\
0.514257128564282	-1.04220449976502e-09\\
0.516258129064532	-1.25064654563361e-09\\
0.518259129564782	-8.33762453896423e-10\\
0.520260130065032	-6.25320408027829e-10\\
0.522261130565283	-6.25320408027829e-10\\
0.524262131065533	-6.25320408027829e-10\\
0.526263131565783	-6.25320408027829e-10\\
0.528264132066033	-4.16881799906007e-10\\
0.530265132566283	-2.08440899953003e-10\\
0.532266133066533	-2.08440899953003e-10\\
0.534267133566783	-2.08440899953003e-10\\
0.536268134067034	-2.08440899953003e-10\\
0.538269134567284	-2.08440899953003e-10\\
0.540270135067534	-2.08440899953003e-10\\
0.542271135567784	-4.16881799906007e-10\\
0.544272136068034	-8.33762453896423e-10\\
0.546273136568284	-2.08440899953003e-10\\
0.548274137068534	-2.08440899953003e-10\\
0.550275137568784	-2.08440899953003e-10\\
0.552276138069035	-2.08440899953003e-10\\
0.554277138569285	2.08440899953003e-10\\
0.556278139069535	4.16881799906007e-10\\
0.558279139569785	6.25320408027829e-10\\
0.560280140070035	8.33762453896423e-10\\
0.562281140570285	6.25320408027829e-10\\
0.564282141070535	2.08440899953003e-10\\
0.566283141570785	2.08440899953003e-10\\
0.568284142071036	0\\
0.570285142571286	-2.08440899953003e-10\\
0.572286143071536	-2.08440899953003e-10\\
0.574287143571786	-2.08440899953003e-10\\
0.576288144072036	-2.08440899953003e-10\\
0.578289144572286	-2.08440899953003e-10\\
0.580290145072536	-6.25320408027829e-10\\
0.582291145572786	-2.08440899953003e-10\\
0.584292146073036	-4.16881799906007e-10\\
0.586293146573287	-4.16881799906007e-10\\
0.588294147073537	-2.08440899953003e-10\\
0.590295147573787	-2.08440899953003e-10\\
0.592296148074037	2.08440899953003e-10\\
0.594297148574287	2.08440899953003e-10\\
0.596298149074537	0\\
0.598299149574787	2.08440899953003e-10\\
0.600300150075038	2.08440899953003e-10\\
0.602301150575288	2.08440899953003e-10\\
0.604302151075538	0\\
0.606303151575788	-2.08440899953003e-10\\
0.608304152076038	0\\
0.610305152576288	2.08440899953003e-10\\
0.612306153076538	6.25320408027829e-10\\
0.614307153576788	1.04220449976502e-09\\
0.616308154077039	1.25064654563361e-09\\
0.618309154577289	1.4590885915022e-09\\
0.620310155077539	1.04220449976502e-09\\
0.622311155577789	4.16881799906007e-10\\
0.624312156078039	6.25320408027829e-10\\
0.626313156578289	6.25320408027829e-10\\
0.628314157078539	6.25320408027829e-10\\
0.630315157578789	4.16881799906007e-10\\
0.63231615807904	2.08440899953003e-10\\
0.63431715857929	0\\
0.63631815907954	0\\
0.63831915957979	-2.08440899953003e-10\\
0.64032016008004	-8.33762453896423e-10\\
0.64232116058029	-6.25320408027829e-10\\
0.64432216108054	-1.04220449976502e-09\\
0.64632316158079	-8.33762453896423e-10\\
0.64832416208104	-8.33762453896423e-10\\
0.650325162581291	-4.16881799906007e-10\\
0.652326163081541	-6.25320408027829e-10\\
0.654327163581791	-8.33762453896423e-10\\
0.656328164082041	-2.08440899953003e-10\\
0.658329164582291	0\\
0.660330165082541	-2.08440899953003e-10\\
0.662331165582791	-4.16881799906007e-10\\
0.664332166083042	-6.25320408027829e-10\\
0.666333166583292	-6.25320408027829e-10\\
0.668334167083542	-8.33762453896423e-10\\
0.670335167583792	-6.25320408027829e-10\\
0.672336168084042	-2.08440899953003e-10\\
0.674337168584292	-6.25320408027829e-10\\
0.676338169084542	-6.25320408027829e-10\\
0.678339169584792	-6.25320408027829e-10\\
0.680340170085043	-4.16881799906007e-10\\
0.682341170585293	-4.16881799906007e-10\\
0.684342171085543	-6.25320408027829e-10\\
0.686343171585793	-6.25320408027829e-10\\
0.688344172086043	-4.16881799906007e-10\\
0.690345172586293	-2.08440899953003e-10\\
0.692346173086543	-4.16881799906007e-10\\
0.694347173586793	-2.08440899953003e-10\\
0.696348174087044	0\\
0.698349174587294	0\\
0.700350175087544	2.08440899953003e-10\\
0.702351175587794	0\\
0.704352176088044	-2.08440899953003e-10\\
0.706353176588294	-2.08440899953003e-10\\
0.708354177088544	-2.08440899953003e-10\\
0.710355177588794	-2.08440899953003e-10\\
0.712356178089045	0\\
0.714357178589295	-2.08440899953003e-10\\
0.716358179089545	-2.08440899953003e-10\\
0.718359179589795	-2.08440899953003e-10\\
0.720360180090045	-4.16881799906007e-10\\
0.722361180590295	-4.16881799906007e-10\\
0.724362181090545	-2.08440899953003e-10\\
0.726363181590795	-6.25320408027829e-10\\
0.728364182091045	-1.04220449976502e-09\\
0.730365182591296	-8.33762453896423e-10\\
0.732366183091546	-1.04220449976502e-09\\
0.734367183591796	-6.25320408027829e-10\\
0.736368184092046	-6.25320408027829e-10\\
0.738369184592296	-6.25320408027829e-10\\
0.740370185092546	-2.08440899953003e-10\\
0.742371185592796	-6.25320408027829e-10\\
0.744372186093047	-6.25320408027829e-10\\
0.746373186593297	-1.04220449976502e-09\\
0.748374187093547	-6.25320408027829e-10\\
0.750375187593797	-2.08440899953003e-10\\
0.752376188094047	-4.16881799906007e-10\\
0.754377188594297	-4.16881799906007e-10\\
0.756378189094547	-6.25320408027829e-10\\
0.758379189594797	-8.33762453896423e-10\\
0.760380190095048	-1.04220449976502e-09\\
0.762381190595298	-8.33762453896423e-10\\
0.764382191095548	-8.33762453896423e-10\\
0.766383191595798	-6.25320408027829e-10\\
0.768384192096048	-8.33762453896423e-10\\
0.770385192596298	-2.08440899953003e-10\\
0.772386193096548	-6.25320408027829e-10\\
0.774387193596798	-8.33762453896423e-10\\
0.776388194097049	-1.04220449976502e-09\\
0.778389194597299	-1.4590885915022e-09\\
0.780390195097549	-1.4590885915022e-09\\
0.782391195597799	-1.4590885915022e-09\\
0.784392196098049	-1.04220449976502e-09\\
0.786393196598299	-1.25064654563361e-09\\
0.788394197098549	-8.33762453896423e-10\\
0.790395197598799	-1.04220449976502e-09\\
0.792396198099049	-6.25320408027829e-10\\
0.7943971985993	-1.04220449976502e-09\\
0.79639819909955	-1.04220449976502e-09\\
0.7983991995998	-8.33762453896423e-10\\
0.80040020010005	-8.33762453896423e-10\\
0.8024012006003	-1.04220449976502e-09\\
0.80440220110055	-1.25064654563361e-09\\
0.8064032016008	-1.4590885915022e-09\\
0.808404202101051	-1.25064654563361e-09\\
0.810405202601301	-1.25064654563361e-09\\
0.812406203101551	-1.25064654563361e-09\\
0.814407203601801	-1.25064654563361e-09\\
0.816408204102051	-1.04220449976502e-09\\
0.818409204602301	-1.04220449976502e-09\\
0.820410205102551	-1.04220449976502e-09\\
0.822411205602801	-1.25064654563361e-09\\
0.824412206103052	-1.25064654563361e-09\\
0.826413206603302	-1.04220449976502e-09\\
0.828414207103552	-1.25064654563361e-09\\
0.830415207603802	-1.04220449976502e-09\\
0.832416208104052	-1.25064654563361e-09\\
0.834417208604302	-1.4590885915022e-09\\
0.836418209104552	-1.4590885915022e-09\\
0.838419209604802	-1.4590885915022e-09\\
0.840420210105053	-1.25064654563361e-09\\
0.842421210605303	-1.04220449976502e-09\\
0.844422211105553	-1.04220449976502e-09\\
0.846423211605803	-8.33762453896423e-10\\
0.848424212106053	-6.25320408027829e-10\\
0.850425212606303	-4.16881799906007e-10\\
0.852426213106553	-2.08440899953003e-10\\
0.854427213606803	0\\
0.856428214107053	2.08440899953003e-10\\
0.858429214607304	2.08440899953003e-10\\
0.860430215107554	6.25320408027829e-10\\
0.862431215607804	8.33762453896423e-10\\
0.864432216108054	8.33762453896423e-10\\
0.866433216608304	1.04220449976502e-09\\
0.868434217108554	6.25320408027829e-10\\
0.870435217608804	8.33762453896423e-10\\
0.872436218109054	8.33762453896423e-10\\
0.874437218609305	1.04220449976502e-09\\
0.876438219109555	1.25064654563361e-09\\
0.878439219609805	1.04220449976502e-09\\
0.880440220110055	6.25320408027829e-10\\
0.882441220610305	1.04220449976502e-09\\
0.884442221110555	8.33762453896423e-10\\
0.886443221610805	0\\
0.888444222111056	0\\
0.890445222611306	0\\
0.892446223111556	-2.08440899953003e-10\\
0.894447223611806	-4.16881799906007e-10\\
0.896448224112056	-2.08440899953003e-10\\
0.898449224612306	-2.08440899953003e-10\\
0.900450225112556	-2.08440899953003e-10\\
0.902451225612806	-2.08440899953003e-10\\
0.904452226113057	-2.08440899953003e-10\\
0.906453226613307	-2.08440899953003e-10\\
0.908454227113557	-6.25320408027829e-10\\
0.910455227613807	-8.33762453896423e-10\\
0.912456228114057	-1.25064654563361e-09\\
0.914457228614307	-1.04220449976502e-09\\
0.916458229114557	-1.25064654563361e-09\\
0.918459229614807	-1.04220449976502e-09\\
0.920460230115058	-8.33762453896423e-10\\
0.922461230615308	-8.33762453896423e-10\\
0.924462231115558	-8.33762453896423e-10\\
0.926463231615808	-4.16881799906007e-10\\
0.928464232116058	-6.25320408027829e-10\\
0.930465232616308	-6.25320408027829e-10\\
0.932466233116558	-2.08440899953003e-10\\
0.934467233616808	-2.08440899953003e-10\\
0.936468234117058	-2.08440899953003e-10\\
0.938469234617309	-2.08440899953003e-10\\
0.940470235117559	-2.08440899953003e-10\\
0.942471235617809	-6.25320408027829e-10\\
0.944472236118059	-2.08440899953003e-10\\
0.946473236618309	0\\
0.948474237118559	2.08440899953003e-10\\
0.950475237618809	2.08440899953003e-10\\
0.95247623811906	2.08440899953003e-10\\
0.95447723861931	0\\
0.95647823911956	2.08440899953003e-10\\
0.95847923961981	2.08440899953003e-10\\
0.96048024012006	-4.16881799906007e-10\\
0.96248124062031	-6.25320408027829e-10\\
0.96448224112056	-8.33762453896423e-10\\
0.96648324162081	-1.04220449976502e-09\\
0.968484242121061	-1.25064654563361e-09\\
0.970485242621311	-1.25064654563361e-09\\
0.972486243121561	-1.25064654563361e-09\\
0.974487243621811	-1.4590885915022e-09\\
0.976488244122061	-1.25064654563361e-09\\
0.978489244622311	-1.04220449976502e-09\\
0.980490245122561	-6.25320408027829e-10\\
0.982491245622811	-6.25320408027829e-10\\
0.984492246123062	-1.04220449976502e-09\\
0.986493246623312	-1.04220449976502e-09\\
0.988494247123562	-8.33762453896423e-10\\
0.990495247623812	-4.16881799906007e-10\\
0.992496248124062	-6.25320408027829e-10\\
0.994497248624312	-8.33762453896423e-10\\
0.996498249124562	-1.04220449976502e-09\\
0.998499249624812	-6.25320408027829e-10\\
1.00050025012506	-2.08440899953003e-10\\
1.00250125062531	-2.08440899953003e-10\\
1.00450225112556	-2.08440899953003e-10\\
1.00650325162581	-2.08440899953003e-10\\
1.00850425212606	0\\
1.01050525262631	-2.08440899953003e-10\\
1.01250625312656	2.08440899953003e-10\\
1.01450725362681	4.16881799906007e-10\\
1.01650825412706	2.08440899953003e-10\\
1.01850925462731	4.16881799906007e-10\\
1.02051025512756	4.16881799906007e-10\\
1.02251125562781	6.25320408027829e-10\\
1.02451225612806	1.04220449976502e-09\\
1.02651325662831	4.16881799906007e-10\\
1.02851425712856	2.08440899953003e-10\\
1.03051525762881	-2.08440899953003e-10\\
1.03251625812906	2.08440899953003e-10\\
1.03451725862931	-2.08440899953003e-10\\
1.03651825912956	0\\
1.03851925962981	0\\
1.04052026013006	4.16881799906007e-10\\
1.04252126063032	2.08440899953003e-10\\
1.04452226113057	6.25320408027829e-10\\
1.04652326163082	6.25320408027829e-10\\
1.04852426213107	8.33762453896423e-10\\
1.05052526263132	4.16881799906007e-10\\
1.05252626313157	1.04220449976502e-09\\
1.05452726363182	4.16881799906007e-10\\
1.05652826413207	6.25320408027829e-10\\
1.05852926463232	6.25320408027829e-10\\
1.06053026513257	8.33762453896423e-10\\
1.06253126563282	1.25064654563361e-09\\
1.06453226613307	1.66752490779285e-09\\
1.06653326663332	2.08440899953003e-09\\
1.06853426713357	2.08440899953003e-09\\
1.07053526763382	1.87596695366144e-09\\
1.07253626813407	2.29285104539863e-09\\
1.07453726863432	2.29285104539863e-09\\
1.07653826913457	2.91817145342646e-09\\
1.07853926963482	2.50128736168927e-09\\
1.08054027013507	2.29285104539863e-09\\
1.08254127063532	2.29285104539863e-09\\
1.08454227113557	1.87596695366144e-09\\
1.08654327163582	1.66752490779285e-09\\
1.08854427213607	1.66752490779285e-09\\
1.09054527263632	1.4590885915022e-09\\
1.09254627313657	8.33762453896423e-10\\
1.09454727363682	1.25064654563361e-09\\
1.09654827413707	1.66752490779285e-09\\
1.09854927463732	1.87596695366144e-09\\
1.10055027513757	1.4590885915022e-09\\
1.10255127563782	1.4590885915022e-09\\
1.10455227613807	1.25064654563361e-09\\
1.10655327663832	1.25064654563361e-09\\
1.10855427713857	1.25064654563361e-09\\
1.11055527763882	1.04220449976502e-09\\
1.11255627813907	1.04220449976502e-09\\
1.11455727863932	1.04220449976502e-09\\
1.11655827913957	1.25064654563361e-09\\
1.11855927963982	1.25064654563361e-09\\
1.12056028014007	1.04220449976502e-09\\
1.12256128064032	1.25064654563361e-09\\
1.12456228114057	8.33762453896423e-10\\
1.12656328164082	6.25320408027829e-10\\
1.12856428214107	1.04220449976502e-09\\
1.13056528264132	8.33762453896423e-10\\
1.13256628314157	8.33762453896423e-10\\
1.13456728364182	6.25320408027829e-10\\
1.13656828414207	4.16881799906007e-10\\
1.13856928464232	1.04220449976502e-09\\
1.14057028514257	1.04220449976502e-09\\
1.14257128564282	8.33762453896423e-10\\
1.14457228614307	6.25320408027829e-10\\
1.14657328664332	4.16881799906007e-10\\
1.14857428714357	2.08440899953003e-10\\
1.15057528764382	2.08440899953003e-10\\
1.15257628814407	4.16881799906007e-10\\
1.15457728864432	6.25320408027829e-10\\
1.15657828914457	8.33762453896423e-10\\
1.15857928964482	8.33762453896423e-10\\
1.16058029014507	1.04220449976502e-09\\
1.16258129064532	1.87596695366144e-09\\
1.16458229114557	2.08440899953003e-09\\
1.16658329164582	1.66752490779285e-09\\
1.16858429214607	1.66752490779285e-09\\
1.17058529264632	1.25064654563361e-09\\
1.17258629314657	1.25064654563361e-09\\
1.17458729364682	1.4590885915022e-09\\
1.17658829414707	2.08440899953003e-09\\
1.17858929464732	2.29285104539863e-09\\
1.18059029514757	2.70972940755786e-09\\
1.18259129564782	2.91817145342646e-09\\
1.18459229614807	2.50128736168927e-09\\
1.18659329664832	2.70972940755786e-09\\
1.18859429714857	2.70972940755786e-09\\
1.19059529764882	2.91817145342646e-09\\
1.19259629814907	2.50128736168927e-09\\
1.19459729864932	2.29285104539863e-09\\
1.19659829914957	2.29285104539863e-09\\
1.19859929964982	2.29285104539863e-09\\
1.20060030015008	1.87596695366144e-09\\
1.20260130065033	2.08440899953003e-09\\
1.20460230115058	1.87596695366144e-09\\
1.20660330165083	2.08440899953003e-09\\
1.20860430215108	2.08440899953003e-09\\
1.21060530265133	1.87596695366144e-09\\
1.21260630315158	1.25064654563361e-09\\
1.21460730365183	1.66752490779285e-09\\
1.21660830415208	1.66752490779285e-09\\
1.21860930465233	1.4590885915022e-09\\
1.22061030515258	1.4590885915022e-09\\
1.22261130565283	1.66752490779285e-09\\
1.22461230615308	1.87596695366144e-09\\
1.22661330665333	1.66752490779285e-09\\
1.22861430715358	1.87596695366144e-09\\
1.23061530765383	2.08440899953003e-09\\
1.23261630815408	1.87596695366144e-09\\
1.23461730865433	1.66752490779285e-09\\
1.23661830915458	2.08440899953003e-09\\
1.23861930965483	2.50128736168927e-09\\
1.24062031015508	2.50128736168927e-09\\
1.24262131065533	2.29285104539863e-09\\
1.24462231115558	2.29285104539863e-09\\
1.24662331165583	2.70972940755786e-09\\
1.24862431215608	2.50128736168927e-09\\
1.25062531265633	2.50128736168927e-09\\
1.25262631315658	2.50128736168927e-09\\
1.25462731365683	2.70972940755786e-09\\
1.25662831415708	2.91817145342646e-09\\
1.25862931465733	3.12661349929505e-09\\
1.26063031515758	3.12661349929505e-09\\
1.26263131565783	2.91817145342646e-09\\
1.26463231615808	2.91817145342646e-09\\
1.26663331665833	2.91817145342646e-09\\
1.26863431715858	3.33505554516364e-09\\
1.27063531765883	3.12661349929505e-09\\
1.27263631815908	3.33505554516364e-09\\
1.27463731865933	3.33505554516364e-09\\
1.27663831915958	3.54349186145428e-09\\
1.27863931965983	3.75193390732288e-09\\
1.28064032016008	3.54349186145428e-09\\
1.28264132066033	3.75193390732288e-09\\
1.28464232116058	3.54349186145428e-09\\
1.28664332166083	3.33505554516364e-09\\
1.28864432216108	3.33505554516364e-09\\
1.29064532266133	3.12661349929505e-09\\
1.29264632316158	3.54349186145428e-09\\
1.29464732366183	3.54349186145428e-09\\
1.29664832416208	3.75193390732288e-09\\
1.29864932466233	3.33505554516364e-09\\
1.30065032516258	3.12661349929505e-09\\
1.30265132566283	3.33505554516364e-09\\
1.30465232616308	3.75193390732288e-09\\
1.30665332666333	3.33505554516364e-09\\
1.30865432716358	3.75193390732288e-09\\
1.31065532766383	3.54349186145428e-09\\
1.31265632816408	3.75193390732288e-09\\
1.31465732866433	3.75193390732288e-09\\
1.31665832916458	3.75193390732288e-09\\
1.31865932966483	3.54349186145428e-09\\
1.32066033016508	3.75193390732288e-09\\
1.32266133066533	3.75193390732288e-09\\
1.32466233116558	3.33505554516364e-09\\
1.32666333166583	3.54349186145428e-09\\
1.32866433216608	3.54349186145428e-09\\
1.33066533266633	3.12661349929505e-09\\
1.33266633316658	3.12661349929505e-09\\
1.33466733366683	2.91817145342646e-09\\
1.33666833416708	2.91817145342646e-09\\
1.33866933466733	3.12661349929505e-09\\
1.34067033516758	3.12661349929505e-09\\
1.34267133566783	3.12661349929505e-09\\
1.34467233616808	3.54349186145428e-09\\
1.34667333666833	3.54349186145428e-09\\
1.34867433716858	3.12661349929505e-09\\
1.35067533766883	3.12661349929505e-09\\
1.35267633816908	3.12661349929505e-09\\
1.35467733866933	3.12661349929505e-09\\
1.35667833916958	2.70972940755786e-09\\
1.35867933966983	2.70972940755786e-09\\
1.36068034017009	2.29285104539863e-09\\
1.36268134067034	2.50128736168927e-09\\
1.36468234117059	2.50128736168927e-09\\
1.36668334167084	2.29285104539863e-09\\
1.36868434217109	2.08440899953003e-09\\
1.37068534267134	2.29285104539863e-09\\
1.37268634317159	2.70972940755786e-09\\
1.37468734367184	2.70972940755786e-09\\
1.37668834417209	2.91817145342646e-09\\
1.37868934467234	3.12661349929505e-09\\
1.38069034517259	3.12661349929505e-09\\
1.38269134567284	3.12661349929505e-09\\
1.38469234617309	3.33505554516364e-09\\
1.38669334667334	2.70972940755786e-09\\
1.38869434717359	2.91817145342646e-09\\
1.39069534767384	3.12661349929505e-09\\
1.39269634817409	3.54349186145428e-09\\
1.39469734867434	3.54349186145428e-09\\
1.39669834917459	3.33505554516364e-09\\
1.39869934967484	3.12661349929505e-09\\
1.40070035017509	3.12661349929505e-09\\
1.40270135067534	3.12661349929505e-09\\
1.40470235117559	3.12661349929505e-09\\
1.40670335167584	3.33505554516364e-09\\
1.40870435217609	3.12661349929505e-09\\
1.41070535267634	2.70972940755786e-09\\
1.41270635317659	2.70972940755786e-09\\
1.41470735367684	2.50128736168927e-09\\
1.41670835417709	2.70972940755786e-09\\
1.41870935467734	2.50128736168927e-09\\
1.42071035517759	2.70972940755786e-09\\
1.42271135567784	2.91817145342646e-09\\
1.42471235617809	2.70972940755786e-09\\
1.42671335667834	2.70972940755786e-09\\
1.42871435717859	2.29285104539863e-09\\
1.43071535767884	2.08440899953003e-09\\
1.43271635817909	2.29285104539863e-09\\
1.43471735867934	2.08440899953003e-09\\
1.43671835917959	1.66752490779285e-09\\
1.43871935967984	1.87596695366144e-09\\
1.44072036018009	1.87596695366144e-09\\
1.44272136068034	1.87596695366144e-09\\
1.44472236118059	2.08440899953003e-09\\
1.44672336168084	2.29285104539863e-09\\
1.44872436218109	2.29285104539863e-09\\
1.45072536268134	2.08440899953003e-09\\
1.45272636318159	2.29285104539863e-09\\
1.45472736368184	2.08440899953003e-09\\
1.45672836418209	1.87596695366144e-09\\
1.45872936468234	2.08440899953003e-09\\
1.46073036518259	1.66752490779285e-09\\
1.46273136568284	1.4590885915022e-09\\
1.46473236618309	1.25064654563361e-09\\
1.46673336668334	8.33762453896423e-10\\
1.46873436718359	1.04220449976502e-09\\
1.47073536768384	8.33762453896423e-10\\
1.47273636818409	1.04220449976502e-09\\
1.47473736868434	8.33762453896423e-10\\
1.47673836918459	6.25320408027829e-10\\
1.47873936968484	4.16881799906007e-10\\
1.48074037018509	0\\
1.48274137068534	-2.08440899953003e-10\\
1.48474237118559	0\\
1.48674337168584	0\\
1.48874437218609	-2.08440899953003e-10\\
1.49074537268634	-2.08440899953003e-10\\
1.49274637318659	-4.16881799906007e-10\\
1.49474737368684	-2.08440899953003e-10\\
1.49674837418709	0\\
1.49874937468734	2.08440899953003e-10\\
1.50075037518759	0\\
1.50275137568784	2.08440899953003e-10\\
1.50475237618809	0\\
1.50675337668834	-2.08440899953003e-10\\
1.50875437718859	-4.16881799906007e-10\\
1.51075537768884	-4.16881799906007e-10\\
1.51275637818909	-6.25320408027829e-10\\
1.51475737868934	-1.04220449976502e-09\\
1.51675837918959	-8.33762453896423e-10\\
1.51875937968984	-6.25320408027829e-10\\
1.5207603801901	-4.16881799906007e-10\\
1.52276138069035	-4.16881799906007e-10\\
1.5247623811906	-6.25320408027829e-10\\
1.52676338169085	-4.16881799906007e-10\\
1.5287643821911	-4.16881799906007e-10\\
1.53076538269135	-4.16881799906007e-10\\
1.5327663831916	-2.08440899953003e-10\\
1.53476738369185	-6.25320408027829e-10\\
1.5367683841921	-4.16881799906007e-10\\
1.53876938469235	-6.25320408027829e-10\\
1.5407703851926	-1.04220449976502e-09\\
1.54277138569285	-1.25064654563361e-09\\
1.5447723861931	-1.25064654563361e-09\\
1.54677338669335	-1.4590885915022e-09\\
1.5487743871936	-1.66752490779285e-09\\
1.55077538769385	-2.08440899953003e-09\\
1.5527763881941	-2.29285104539863e-09\\
1.55477738869435	-2.29285104539863e-09\\
1.5567783891946	-2.29285104539863e-09\\
1.55877938969485	-2.50128736168927e-09\\
1.5607803901951	-2.70972940755786e-09\\
1.56278139069535	-2.91817145342646e-09\\
1.5647823911956	-2.91817145342646e-09\\
1.56678339169585	-3.12661349929505e-09\\
1.5687843921961	-3.12661349929505e-09\\
1.57078539269635	-2.91817145342646e-09\\
1.5727863931966	-3.33505554516364e-09\\
1.57478739369685	-3.54349186145428e-09\\
1.5767883941971	-3.75193390732288e-09\\
1.57878939469735	-3.75193390732288e-09\\
1.5807903951976	-3.96037595319147e-09\\
1.58279139569785	-3.75193390732288e-09\\
1.5847923961981	-3.96037595319147e-09\\
1.58679339669835	-3.96037595319147e-09\\
1.5887943971986	-3.75193390732288e-09\\
1.59079539769885	-3.75193390732288e-09\\
1.5927963981991	-3.75193390732288e-09\\
1.59479739869935	-3.75193390732288e-09\\
1.5967983991996	-3.33505554516364e-09\\
1.59879939969985	-2.91817145342646e-09\\
1.6008004002001	-2.70972940755786e-09\\
1.60280140070035	-2.50128736168927e-09\\
1.6048024012006	-2.08440899953003e-09\\
1.60680340170085	-2.29285104539863e-09\\
1.6088044022011	-2.08440899953003e-09\\
1.61080540270135	-2.08440899953003e-09\\
1.6128064032016	-2.08440899953003e-09\\
1.61480740370185	-2.08440899953003e-09\\
1.6168084042021	-1.4590885915022e-09\\
1.61880940470235	-1.25064654563361e-09\\
1.6208104052026	-1.04220449976502e-09\\
1.62281140570285	-8.33762453896423e-10\\
1.6248124062031	-1.04220449976502e-09\\
1.62681340670335	-8.33762453896423e-10\\
1.6288144072036	-1.25064654563361e-09\\
1.63081540770385	-1.25064654563361e-09\\
1.6328164082041	-1.04220449976502e-09\\
1.63481740870435	-1.04220449976502e-09\\
1.6368184092046	-1.25064654563361e-09\\
1.63881940970485	-1.4590885915022e-09\\
1.6408204102051	-1.4590885915022e-09\\
1.64282141070535	-1.25064654563361e-09\\
1.6448224112056	-1.4590885915022e-09\\
1.64682341170585	-8.33762453896423e-10\\
1.6488244122061	-6.25320408027829e-10\\
1.65082541270635	-6.25320408027829e-10\\
1.6528264132066	-6.25320408027829e-10\\
1.65482741370685	-8.33762453896423e-10\\
1.6568284142071	-6.25320408027829e-10\\
1.65882941470735	-6.25320408027829e-10\\
1.6608304152076	-4.16881799906007e-10\\
1.66283141570785	-2.08440899953003e-10\\
1.6648324162081	0\\
1.66683341670835	0\\
1.6688344172086	-2.08440899953003e-10\\
1.67083541770885	-2.08440899953003e-10\\
1.6728364182091	-2.08440899953003e-10\\
1.67483741870935	-2.08440899953003e-10\\
1.6768384192096	-2.08440899953003e-10\\
1.67883941970985	-2.08440899953003e-10\\
1.68084042021011	-2.08440899953003e-10\\
1.68284142071036	-2.08440899953003e-10\\
1.68484242121061	-6.25320408027829e-10\\
1.68684342171086	-4.16881799906007e-10\\
1.68884442221111	-4.16881799906007e-10\\
1.69084542271136	-6.25320408027829e-10\\
1.69284642321161	-6.25320408027829e-10\\
1.69484742371186	-2.08440899953003e-10\\
1.69684842421211	0\\
1.69884942471236	4.16881799906007e-10\\
1.70085042521261	8.33762453896423e-10\\
1.70285142571286	6.25320408027829e-10\\
1.70485242621311	8.33762453896423e-10\\
1.70685342671336	8.33762453896423e-10\\
1.70885442721361	1.04220449976502e-09\\
1.71085542771386	1.04220449976502e-09\\
1.71285642821411	1.25064654563361e-09\\
1.71485742871436	1.66752490779285e-09\\
1.71685842921461	1.4590885915022e-09\\
1.71885942971486	1.25064654563361e-09\\
1.72086043021511	1.25064654563361e-09\\
1.72286143071536	1.4590885915022e-09\\
1.72486243121561	1.04220449976502e-09\\
1.72686343171586	1.4590885915022e-09\\
1.72886443221611	1.4590885915022e-09\\
1.73086543271636	1.04220449976502e-09\\
1.73286643321661	1.25064654563361e-09\\
1.73486743371686	8.33762453896423e-10\\
1.73686843421711	8.33762453896423e-10\\
1.73886943471736	8.33762453896423e-10\\
1.74087043521761	6.25320408027829e-10\\
1.74287143571786	4.16881799906007e-10\\
1.74487243621811	6.25320408027829e-10\\
1.74687343671836	4.16881799906007e-10\\
1.74887443721861	2.08440899953003e-10\\
1.75087543771886	6.25320408027829e-10\\
1.75287643821911	4.16881799906007e-10\\
1.75487743871936	-2.08440899953003e-10\\
1.75687843921961	-4.16881799906007e-10\\
1.75887943971986	-2.08440899953003e-10\\
1.76088044022011	-4.16881799906007e-10\\
1.76288144072036	-2.08440899953003e-10\\
1.76488244122061	-2.08440899953003e-10\\
1.76688344172086	-2.08440899953003e-10\\
1.76888444222111	-2.08440899953003e-10\\
1.77088544272136	0\\
1.77288644322161	0\\
1.77488744372186	-2.08440899953003e-10\\
1.77688844422211	-2.08440899953003e-10\\
1.77888944472236	-2.08440899953003e-10\\
1.78089044522261	-4.16881799906007e-10\\
1.78289144572286	-6.25320408027829e-10\\
1.78489244622311	-4.16881799906007e-10\\
1.78689344672336	-6.25320408027829e-10\\
1.78889444722361	-4.16881799906007e-10\\
1.79089544772386	-6.25320408027829e-10\\
1.79289644822411	-6.25320408027829e-10\\
1.79489744872436	-8.33762453896423e-10\\
1.79689844922461	-6.25320408027829e-10\\
1.79889944972486	-8.33762453896423e-10\\
1.80090045022511	-8.33762453896423e-10\\
1.80290145072536	-4.16881799906007e-10\\
1.80490245122561	-2.08440899953003e-10\\
1.80690345172586	-2.08440899953003e-10\\
1.80890445222611	-2.08440899953003e-10\\
1.81090545272636	-6.25320408027829e-10\\
1.81290645322661	-6.25320408027829e-10\\
1.81490745372686	-6.25320408027829e-10\\
1.81690845422711	-6.25320408027829e-10\\
1.81890945472736	-6.25320408027829e-10\\
1.82091045522761	-2.08440899953003e-10\\
1.82291145572786	-2.08440899953003e-10\\
1.82491245622811	-4.16881799906007e-10\\
1.82691345672836	-4.16881799906007e-10\\
1.82891445722861	-4.16881799906007e-10\\
1.83091545772886	-4.16881799906007e-10\\
1.83291645822911	0\\
1.83491745872936	-2.08440899953003e-10\\
1.83691845922961	0\\
1.83891945972987	-2.08440899953003e-10\\
1.84092046023012	-4.16881799906007e-10\\
1.84292146073037	-2.08440899953003e-10\\
1.84492246123062	0\\
1.84692346173087	0\\
1.84892446223112	-2.08440899953003e-10\\
1.85092546273137	0\\
1.85292646323162	-2.08440899953003e-10\\
1.85492746373187	2.08440899953003e-10\\
1.85692846423212	-2.08440899953003e-10\\
1.85892946473237	-2.08440899953003e-10\\
1.86093046523262	-2.08440899953003e-10\\
1.86293146573287	-2.08440899953003e-10\\
1.86493246623312	-2.08440899953003e-10\\
1.86693346673337	-4.16881799906007e-10\\
1.86893446723362	-2.08440899953003e-10\\
1.87093546773387	-2.08440899953003e-10\\
1.87293646823412	-6.25320408027829e-10\\
1.87493746873437	-8.33762453896423e-10\\
1.87693846923462	-4.16881799906007e-10\\
1.87893946973487	-2.08440899953003e-10\\
1.88094047023512	-2.08440899953003e-10\\
1.88294147073537	0\\
1.88494247123562	0\\
1.88694347173587	-2.08440899953003e-10\\
1.88894447223612	-4.16881799906007e-10\\
1.89094547273637	-6.25320408027829e-10\\
1.89294647323662	-8.33762453896423e-10\\
1.89494747373687	-4.16881799906007e-10\\
1.89694847423712	-4.16881799906007e-10\\
1.89894947473737	-4.16881799906007e-10\\
1.90095047523762	-2.08440899953003e-10\\
1.90295147573787	4.16881799906007e-10\\
1.90495247623812	0\\
1.90695347673837	-2.08440899953003e-10\\
1.90895447723862	-2.08440899953003e-10\\
1.91095547773887	-2.08440899953003e-10\\
1.91295647823912	-4.16881799906007e-10\\
1.91495747873937	-8.33762453896423e-10\\
1.91695847923962	-1.04220449976502e-09\\
1.91895947973987	-8.33762453896423e-10\\
1.92096048024012	-1.25064654563361e-09\\
1.92296148074037	-1.25064654563361e-09\\
1.92496248124062	-1.25064654563361e-09\\
1.92696348174087	-1.04220449976502e-09\\
1.92896448224112	-1.04220449976502e-09\\
1.93096548274137	-1.04220449976502e-09\\
1.93296648324162	-1.04220449976502e-09\\
1.93496748374187	-8.33762453896423e-10\\
1.93696848424212	-4.16881799906007e-10\\
1.93896948474237	-6.25320408027829e-10\\
1.94097048524262	-1.04220449976502e-09\\
1.94297148574287	-1.25064654563361e-09\\
1.94497248624312	-1.25064654563361e-09\\
1.94697348674337	-1.25064654563361e-09\\
1.94897448724362	-1.04220449976502e-09\\
1.95097548774387	-8.33762453896423e-10\\
1.95297648824412	-8.33762453896423e-10\\
1.95497748874437	-6.25320408027829e-10\\
1.95697848924462	-6.25320408027829e-10\\
1.95897948974487	-8.33762453896423e-10\\
1.96098049024512	-6.25320408027829e-10\\
1.96298149074537	-4.16881799906007e-10\\
1.96498249124562	-4.16881799906007e-10\\
1.96698349174587	-6.25320408027829e-10\\
1.96898449224612	-8.33762453896423e-10\\
1.97098549274637	-1.25064654563361e-09\\
1.97298649324662	-1.04220449976502e-09\\
1.97498749374687	-8.33762453896423e-10\\
1.97698849424712	-1.25064654563361e-09\\
1.97898949474737	-1.25064654563361e-09\\
1.98099049524762	-1.04220449976502e-09\\
1.98299149574787	-1.4590885915022e-09\\
1.98499249624812	-8.33762453896423e-10\\
1.98699349674837	-6.25320408027829e-10\\
1.98899449724862	-6.25320408027829e-10\\
1.99099549774887	-1.04220449976502e-09\\
1.99299649824912	-1.04220449976502e-09\\
1.99499749874937	-1.04220449976502e-09\\
1.99699849924962	-8.33762453896423e-10\\
1.99899949974988	-8.33762453896423e-10\\
2.00100050025013	-4.16881799906007e-10\\
2.00300150075038	-2.08440899953003e-10\\
2.00500250125063	0\\
2.00700350175088	-2.08440899953003e-10\\
2.00900450225113	0\\
2.01100550275138	0\\
2.01300650325163	0\\
2.01500750375188	0\\
2.01700850425213	-2.08440899953003e-10\\
2.01900950475238	-4.16881799906007e-10\\
2.02101050525263	-2.08440899953003e-10\\
2.02301150575288	-2.08440899953003e-10\\
2.02501250625313	-2.08440899953003e-10\\
2.02701350675338	-2.08440899953003e-10\\
2.02901450725363	0\\
2.03101550775388	-2.08440899953003e-10\\
2.03301650825413	0\\
2.03501750875438	-2.08440899953003e-10\\
2.03701850925463	-4.16881799906007e-10\\
2.03901950975488	-4.16881799906007e-10\\
2.04102051025513	-1.04220449976502e-09\\
2.04302151075538	-8.33762453896423e-10\\
2.04502251125563	-8.33762453896423e-10\\
2.04702351175588	-8.33762453896423e-10\\
2.04902451225613	-6.25320408027829e-10\\
2.05102551275638	-2.08440899953003e-10\\
2.05302651325663	-2.08440899953003e-10\\
2.05502751375688	-2.08440899953003e-10\\
2.05702851425713	-4.16881799906007e-10\\
2.05902951475738	-4.16881799906007e-10\\
2.06103051525763	-2.08440899953003e-10\\
2.06303151575788	-2.08440899953003e-10\\
2.06503251625813	-6.25320408027829e-10\\
2.06703351675838	-2.08440899953003e-10\\
2.06903451725863	2.08440899953003e-10\\
2.07103551775888	0\\
2.07303651825913	2.08440899953003e-10\\
2.07503751875938	0\\
2.07703851925963	0\\
2.07903951975988	-2.08440899953003e-10\\
2.08104052026013	-2.08440899953003e-10\\
2.08304152076038	-2.08440899953003e-10\\
2.08504252126063	-2.08440899953003e-10\\
2.08704352176088	-6.25320408027829e-10\\
2.08904452226113	-2.08440899953003e-10\\
2.09104552276138	-2.08440899953003e-10\\
2.09304652326163	-2.08440899953003e-10\\
2.09504752376188	2.08440899953003e-10\\
2.09704852426213	2.08440899953003e-10\\
2.09904952476238	2.08440899953003e-10\\
2.10105052526263	6.25320408027829e-10\\
2.10305152576288	6.25320408027829e-10\\
2.10505252626313	6.25320408027829e-10\\
2.10705352676338	6.25320408027829e-10\\
2.10905452726363	4.16881799906007e-10\\
2.11105552776388	4.16881799906007e-10\\
2.11305652826413	2.08440899953003e-10\\
2.11505752876438	0\\
2.11705852926463	-2.08440899953003e-10\\
2.11905952976488	-4.16881799906007e-10\\
2.12106053026513	-2.08440899953003e-10\\
2.12306153076538	-6.25320408027829e-10\\
2.12506253126563	-6.25320408027829e-10\\
2.12706353176588	-4.16881799906007e-10\\
2.12906453226613	-2.08440899953003e-10\\
2.13106553276638	0\\
2.13306653326663	4.16881799906007e-10\\
2.13506753376688	6.25320408027829e-10\\
2.13706853426713	1.04220449976502e-09\\
2.13906953476738	1.04220449976502e-09\\
2.14107053526763	8.33762453896423e-10\\
2.14307153576788	8.33762453896423e-10\\
2.14507253626813	4.16881799906007e-10\\
2.14707353676838	0\\
2.14907453726863	0\\
2.15107553776888	6.25320408027829e-10\\
2.15307653826913	6.25320408027829e-10\\
2.15507753876938	6.25320408027829e-10\\
2.15707853926963	6.25320408027829e-10\\
2.15907953976988	4.16881799906007e-10\\
2.16108054027013	2.08440899953003e-10\\
2.16308154077039	-2.08440899953003e-10\\
2.16508254127064	-2.08440899953003e-10\\
2.16708354177089	-2.08440899953003e-10\\
2.16908454227114	-2.08440899953003e-10\\
2.17108554277139	-2.08440899953003e-10\\
2.17308654327164	-6.25320408027829e-10\\
2.17508754377189	-4.16881799906007e-10\\
2.17708854427214	-4.16881799906007e-10\\
2.17908954477239	-8.33762453896423e-10\\
2.18109054527264	-6.25320408027829e-10\\
2.18309154577289	-6.25320408027829e-10\\
2.18509254627314	-8.33762453896423e-10\\
2.18709354677339	-6.25320408027829e-10\\
2.18909454727364	-1.04220449976502e-09\\
2.19109554777389	-1.04220449976502e-09\\
2.19309654827414	-2.08440899953003e-10\\
2.19509754877439	2.08440899953003e-10\\
2.19709854927464	4.16881799906007e-10\\
2.19909954977489	2.08440899953003e-10\\
2.20110055027514	4.16881799906007e-10\\
2.20310155077539	1.04220449976502e-09\\
2.20510255127564	1.04220449976502e-09\\
2.20710355177589	1.25064654563361e-09\\
2.20910455227614	1.25064654563361e-09\\
2.21110555277639	1.04220449976502e-09\\
2.21310655327664	1.66752490779285e-09\\
2.21510755377689	1.66752490779285e-09\\
2.21710855427714	1.66752490779285e-09\\
2.21910955477739	1.4590885915022e-09\\
2.22111055527764	1.66752490779285e-09\\
2.22311155577789	1.4590885915022e-09\\
2.22511255627814	1.66752490779285e-09\\
2.22711355677839	1.04220449976502e-09\\
2.22911455727864	1.25064654563361e-09\\
2.23111555777889	1.4590885915022e-09\\
2.23311655827914	1.25064654563361e-09\\
2.23511755877939	8.33762453896423e-10\\
2.23711855927964	1.04220449976502e-09\\
2.23911955977989	1.25064654563361e-09\\
2.24112056028014	1.04220449976502e-09\\
2.24312156078039	4.16881799906007e-10\\
2.24512256128064	1.04220449976502e-09\\
2.24712356178089	8.33762453896423e-10\\
2.24912456228114	8.33762453896423e-10\\
2.25112556278139	1.04220449976502e-09\\
2.25312656328164	1.04220449976502e-09\\
2.25512756378189	1.25064654563361e-09\\
2.25712856428214	1.4590885915022e-09\\
2.25912956478239	1.4590885915022e-09\\
2.26113056528264	1.4590885915022e-09\\
2.26313156578289	1.4590885915022e-09\\
2.26513256628314	1.4590885915022e-09\\
2.26713356678339	1.4590885915022e-09\\
2.26913456728364	1.4590885915022e-09\\
2.27113556778389	1.25064654563361e-09\\
2.27313656828414	1.04220449976502e-09\\
2.27513756878439	1.25064654563361e-09\\
2.27713856928464	1.25064654563361e-09\\
2.27913956978489	1.25064654563361e-09\\
2.28114057028514	1.4590885915022e-09\\
2.28314157078539	1.4590885915022e-09\\
2.28514257128564	1.04220449976502e-09\\
2.28714357178589	4.16881799906007e-10\\
2.28914457228614	0\\
2.29114557278639	2.08440899953003e-10\\
2.29314657328664	0\\
2.29514757378689	-2.08440899953003e-10\\
2.29714857428714	0\\
2.29914957478739	4.16881799906007e-10\\
2.30115057528764	4.16881799906007e-10\\
2.30315157578789	8.33762453896423e-10\\
2.30515257628814	6.25320408027829e-10\\
2.30715357678839	4.16881799906007e-10\\
2.30915457728864	0\\
2.31115557778889	4.16881799906007e-10\\
2.31315657828914	8.33762453896423e-10\\
2.31515757878939	1.25064654563361e-09\\
2.31715857928964	1.66752490779285e-09\\
2.31915957978989	1.66752490779285e-09\\
2.32116058029015	1.4590885915022e-09\\
2.3231615807904	1.25064654563361e-09\\
2.32516258129065	6.25320408027829e-10\\
2.3271635817909	6.25320408027829e-10\\
2.32916458229115	8.33762453896423e-10\\
2.3311655827914	6.25320408027829e-10\\
2.33316658329165	1.04220449976502e-09\\
2.3351675837919	1.04220449976502e-09\\
2.33716858429215	1.25064654563361e-09\\
2.3391695847924	1.04220449976502e-09\\
2.34117058529265	8.33762453896423e-10\\
2.3431715857929	4.16881799906007e-10\\
2.34517258629315	8.33762453896423e-10\\
2.3471735867934	1.04220449976502e-09\\
2.34917458729365	1.4590885915022e-09\\
2.3511755877939	1.66752490779285e-09\\
2.35317658829415	1.87596695366144e-09\\
2.3551775887944	1.4590885915022e-09\\
2.35717858929465	1.66752490779285e-09\\
2.3591795897949	2.08440899953003e-09\\
2.36118059029515	2.50128736168927e-09\\
2.3631815907954	2.50128736168927e-09\\
2.36518259129565	2.50128736168927e-09\\
2.3671835917959	2.91817145342646e-09\\
2.36918459229615	2.70972940755786e-09\\
2.3711855927964	2.91817145342646e-09\\
2.37318659329665	2.29285104539863e-09\\
2.3751875937969	2.29285104539863e-09\\
2.37718859429715	2.29285104539863e-09\\
2.3791895947974	2.50128736168927e-09\\
2.38119059529765	2.29285104539863e-09\\
2.3831915957979	2.70972940755786e-09\\
2.38519259629815	3.12661349929505e-09\\
2.3871935967984	2.70972940755786e-09\\
2.38919459729865	2.91817145342646e-09\\
2.3911955977989	3.12661349929505e-09\\
2.39319659829915	3.12661349929505e-09\\
2.3951975987994	3.33505554516364e-09\\
2.39719859929965	3.33505554516364e-09\\
2.3991995997999	3.12661349929505e-09\\
2.40120060030015	2.91817145342646e-09\\
2.4032016008004	2.70972940755786e-09\\
2.40520260130065	2.50128736168927e-09\\
2.4072036018009	2.29285104539863e-09\\
2.40920460230115	2.08440899953003e-09\\
2.4112056028014	1.66752490779285e-09\\
2.41320660330165	2.08440899953003e-09\\
2.4152076038019	2.29285104539863e-09\\
2.41720860430215	1.87596695366144e-09\\
2.4192096048024	1.66752490779285e-09\\
2.42121060530265	1.66752490779285e-09\\
2.4232116058029	1.66752490779285e-09\\
2.42521260630315	1.66752490779285e-09\\
2.4272136068034	1.66752490779285e-09\\
2.42921460730365	1.66752490779285e-09\\
2.4312156078039	1.87596695366144e-09\\
2.43321660830415	1.66752490779285e-09\\
2.4352176088044	1.66752490779285e-09\\
2.43721860930465	1.66752490779285e-09\\
2.4392196098049	1.66752490779285e-09\\
2.44122061030515	1.66752490779285e-09\\
2.4432216108054	2.08440899953003e-09\\
2.44522261130565	2.08440899953003e-09\\
2.4472236118059	2.50128736168927e-09\\
2.44922461230615	2.29285104539863e-09\\
2.4512256128064	2.29285104539863e-09\\
2.45322661330665	2.08440899953003e-09\\
2.4552276138069	1.87596695366144e-09\\
2.45722861430715	1.66752490779285e-09\\
2.4592296148074	1.66752490779285e-09\\
2.46123061530765	1.87596695366144e-09\\
2.4632316158079	2.08440899953003e-09\\
2.46523261630815	2.08440899953003e-09\\
2.4672336168084	1.66752490779285e-09\\
2.46923461730865	1.66752490779285e-09\\
2.4712356178089	1.87596695366144e-09\\
2.47323661830915	1.87596695366144e-09\\
2.4752376188094	1.4590885915022e-09\\
2.47723861930965	8.33762453896423e-10\\
2.4792396198099	1.25064654563361e-09\\
2.48124062031015	1.4590885915022e-09\\
2.48324162081041	1.04220449976502e-09\\
2.48524262131066	8.33762453896423e-10\\
2.48724362181091	6.25320408027829e-10\\
2.48924462231116	4.16881799906007e-10\\
2.49124562281141	6.25320408027829e-10\\
2.49324662331166	2.08440899953003e-10\\
2.49524762381191	2.08440899953003e-10\\
2.49724862431216	0\\
2.49924962481241	-2.08440899953003e-10\\
2.50125062531266	-2.08440899953003e-10\\
2.50325162581291	-2.08440899953003e-10\\
2.50525262631316	0\\
2.50725362681341	0\\
2.50925462731366	4.16881799906007e-10\\
2.51125562781391	2.08440899953003e-10\\
2.51325662831416	0\\
2.51525762881441	0\\
2.51725862931466	2.08440899953003e-10\\
2.51925962981491	-2.08440899953003e-10\\
2.52126063031516	-2.08440899953003e-10\\
2.52326163081541	-2.08440899953003e-10\\
2.52526263131566	-2.08440899953003e-10\\
2.52726363181591	-2.08440899953003e-10\\
2.52926463231616	-2.08440899953003e-10\\
2.53126563281641	-2.08440899953003e-10\\
2.53326663331666	0\\
2.53526763381691	0\\
2.53726863431716	-2.08440899953003e-10\\
2.53926963481741	-2.08440899953003e-10\\
2.54127063531766	-4.16881799906007e-10\\
2.54327163581791	-4.16881799906007e-10\\
2.54527263631816	-8.33762453896423e-10\\
2.54727363681841	-6.25320408027829e-10\\
2.54927463731866	-8.33762453896423e-10\\
2.55127563781891	-8.33762453896423e-10\\
2.55327663831916	-1.04220449976502e-09\\
2.55527763881941	-1.4590885915022e-09\\
2.55727863931966	-1.4590885915022e-09\\
2.55927963981991	-1.4590885915022e-09\\
2.56128064032016	-1.66752490779285e-09\\
2.56328164082041	-1.4590885915022e-09\\
2.56528264132066	-1.66752490779285e-09\\
2.56728364182091	-1.4590885915022e-09\\
2.56928464232116	-1.25064654563361e-09\\
2.57128564282141	-8.33762453896423e-10\\
2.57328664332166	-4.16881799906007e-10\\
2.57528764382191	-1.04220449976502e-09\\
2.57728864432216	-6.25320408027829e-10\\
2.57928964482241	-6.25320408027829e-10\\
2.58129064532266	-4.16881799906007e-10\\
2.58329164582291	-6.25320408027829e-10\\
2.58529264632316	-8.33762453896423e-10\\
2.58729364682341	-4.16881799906007e-10\\
2.58929464732366	-6.25320408027829e-10\\
2.59129564782391	-8.33762453896423e-10\\
2.59329664832416	-6.25320408027829e-10\\
2.59529764882441	-1.04220449976502e-09\\
2.59729864932466	-6.25320408027829e-10\\
2.59929964982491	-2.08440899953003e-10\\
2.60130065032516	-4.16881799906007e-10\\
2.60330165082541	-2.08440899953003e-10\\
2.60530265132566	-2.08440899953003e-10\\
2.60730365182591	-4.16881799906007e-10\\
2.60930465232616	-2.08440899953003e-10\\
2.61130565282641	-4.16881799906007e-10\\
2.61330665332666	-2.08440899953003e-10\\
2.61530765382691	0\\
2.61730865432716	0\\
2.61930965482741	4.16881799906007e-10\\
2.62131065532766	-2.08440899953003e-10\\
2.62331165582791	-2.08440899953003e-10\\
2.62531265632816	-2.08440899953003e-10\\
2.62731365682841	-2.08440899953003e-10\\
2.62931465732866	0\\
2.63131565782891	0\\
2.63331665832916	-2.08440899953003e-10\\
2.63531765882941	-4.16881799906007e-10\\
2.63731865932966	-2.08440899953003e-10\\
2.63931965982992	-4.16881799906007e-10\\
2.64132066033017	-2.08440899953003e-10\\
2.64332166083042	-2.08440899953003e-10\\
2.64532266133067	-8.33762453896423e-10\\
2.64732366183092	-4.16881799906007e-10\\
2.64932466233117	-4.16881799906007e-10\\
2.65132566283142	-2.08440899953003e-10\\
2.65332666333167	-2.08440899953003e-10\\
2.65532766383192	-2.08440899953003e-10\\
2.65732866433217	-4.16881799906007e-10\\
2.65932966483242	-6.25320408027829e-10\\
2.66133066533267	-8.33762453896423e-10\\
2.66333166583292	-1.4590885915022e-09\\
2.66533266633317	-1.25064654563361e-09\\
2.66733366683342	-1.25064654563361e-09\\
2.66933466733367	-1.4590885915022e-09\\
2.67133566783392	-1.66752490779285e-09\\
2.67333666833417	-1.66752490779285e-09\\
2.67533766883442	-1.66752490779285e-09\\
2.67733866933467	-1.87596695366144e-09\\
2.67933966983492	-1.66752490779285e-09\\
2.68134067033517	-1.25064654563361e-09\\
2.68334167083542	-1.25064654563361e-09\\
2.68534267133567	-1.4590885915022e-09\\
2.68734367183592	-1.66752490779285e-09\\
2.68934467233617	-1.87596695366144e-09\\
2.69134567283642	-1.66752490779285e-09\\
2.69334667333667	-1.4590885915022e-09\\
2.69534767383692	-1.66752490779285e-09\\
2.69734867433717	-8.33762453896423e-10\\
2.69934967483742	-8.33762453896423e-10\\
2.70135067533767	-1.04220449976502e-09\\
2.70335167583792	-1.25064654563361e-09\\
2.70535267633817	-1.4590885915022e-09\\
2.70735367683842	-1.4590885915022e-09\\
2.70935467733867	-1.4590885915022e-09\\
2.71135567783892	-1.4590885915022e-09\\
2.71335667833917	-8.33762453896423e-10\\
2.71535767883942	-6.25320408027829e-10\\
2.71735867933967	-1.25064654563361e-09\\
2.71935967983992	-1.04220449976502e-09\\
2.72136068034017	-8.33762453896423e-10\\
2.72336168084042	-1.04220449976502e-09\\
2.72536268134067	-6.25320408027829e-10\\
2.72736368184092	-8.33762453896423e-10\\
2.72936468234117	-1.04220449976502e-09\\
2.73136568284142	-1.66752490779285e-09\\
2.73336668334167	-1.4590885915022e-09\\
2.73536768384192	-1.4590885915022e-09\\
2.73736868434217	-1.4590885915022e-09\\
2.73936968484242	-1.87596695366144e-09\\
2.74137068534267	-1.66752490779285e-09\\
2.74337168584292	-1.87596695366144e-09\\
2.74537268634317	-2.08440899953003e-09\\
2.74737368684342	-2.08440899953003e-09\\
2.74937468734367	-2.08440899953003e-09\\
2.75137568784392	-2.08440899953003e-09\\
2.75337668834417	-2.29285104539863e-09\\
2.75537768884442	-2.29285104539863e-09\\
2.75737868934467	-2.29285104539863e-09\\
2.75937968984492	-2.50128736168927e-09\\
2.76138069034517	-2.29285104539863e-09\\
2.76338169084542	-2.08440899953003e-09\\
2.76538269134567	-1.4590885915022e-09\\
2.76738369184592	-1.87596695366144e-09\\
2.76938469234617	-1.66752490779285e-09\\
2.77138569284642	-1.87596695366144e-09\\
2.77338669334667	-1.87596695366144e-09\\
2.77538769384692	-1.66752490779285e-09\\
2.77738869434717	-2.08440899953003e-09\\
2.77938969484742	-1.4590885915022e-09\\
2.78139069534767	-1.25064654563361e-09\\
2.78339169584792	-1.04220449976502e-09\\
2.78539269634817	-6.25320408027829e-10\\
2.78739369684842	-6.25320408027829e-10\\
2.78939469734867	-4.16881799906007e-10\\
2.79139569784892	-2.08440899953003e-10\\
2.79339669834917	-2.08440899953003e-10\\
2.79539769884942	0\\
2.79739869934967	-2.08440899953003e-10\\
2.79939969984992	-2.08440899953003e-10\\
2.80140070035018	-2.08440899953003e-10\\
2.80340170085043	2.08440899953003e-10\\
2.80540270135068	4.16881799906007e-10\\
2.80740370185093	6.25320408027829e-10\\
2.80940470235118	8.33762453896423e-10\\
2.81140570285143	6.25320408027829e-10\\
2.81340670335168	8.33762453896423e-10\\
2.81540770385193	8.33762453896423e-10\\
2.81740870435218	6.25320408027829e-10\\
2.81940970485243	6.25320408027829e-10\\
2.82141070535268	8.33762453896423e-10\\
2.82341170585293	6.25320408027829e-10\\
2.82541270635318	0\\
2.82741370685343	2.08440899953003e-10\\
2.82941470735368	2.08440899953003e-10\\
2.83141570785393	2.08440899953003e-10\\
2.83341670835418	6.25320408027829e-10\\
2.83541770885443	6.25320408027829e-10\\
2.83741870935468	6.25320408027829e-10\\
2.83941970985493	6.25320408027829e-10\\
2.84142071035518	1.04220449976502e-09\\
2.84342171085543	1.25064654563361e-09\\
2.84542271135568	8.33762453896423e-10\\
2.84742371185593	1.04220449976502e-09\\
2.84942471235618	1.25064654563361e-09\\
2.85142571285643	1.25064654563361e-09\\
2.85342671335668	8.33762453896423e-10\\
2.85542771385693	1.04220449976502e-09\\
2.85742871435718	1.04220449976502e-09\\
2.85942971485743	1.04220449976502e-09\\
2.86143071535768	1.4590885915022e-09\\
2.86343171585793	1.25064654563361e-09\\
2.86543271635818	1.04220449976502e-09\\
2.86743371685843	1.4590885915022e-09\\
2.86943471735868	1.4590885915022e-09\\
2.87143571785893	1.25064654563361e-09\\
2.87343671835918	1.87596695366144e-09\\
2.87543771885943	1.66752490779285e-09\\
2.87743871935968	1.04220449976502e-09\\
2.87943971985993	1.04220449976502e-09\\
2.88144072036018	1.04220449976502e-09\\
2.88344172086043	1.87596695366144e-09\\
2.88544272136068	1.87596695366144e-09\\
2.88744372186093	1.87596695366144e-09\\
2.88944472236118	2.08440899953003e-09\\
2.89144572286143	1.87596695366144e-09\\
2.89344672336168	1.87596695366144e-09\\
2.89544772386193	1.66752490779285e-09\\
2.89744872436218	1.66752490779285e-09\\
2.89944972486243	1.66752490779285e-09\\
2.90145072536268	2.08440899953003e-09\\
2.90345172586293	1.66752490779285e-09\\
2.90545272636318	1.87596695366144e-09\\
2.90745372686343	1.66752490779285e-09\\
2.90945472736368	1.87596695366144e-09\\
2.91145572786393	1.66752490779285e-09\\
2.91345672836418	1.66752490779285e-09\\
2.91545772886443	2.08440899953003e-09\\
2.91745872936468	1.66752490779285e-09\\
2.91945972986493	1.87596695366144e-09\\
2.92146073036518	2.08440899953003e-09\\
2.92346173086543	2.50128736168927e-09\\
2.92546273136568	2.70972940755786e-09\\
2.92746373186593	2.29285104539863e-09\\
2.92946473236618	2.29285104539863e-09\\
2.93146573286643	2.29285104539863e-09\\
2.93346673336668	2.08440899953003e-09\\
2.93546773386693	1.87596695366144e-09\\
2.93746873436718	1.4590885915022e-09\\
2.93946973486743	1.25064654563361e-09\\
2.94147073536768	8.33762453896423e-10\\
2.94347173586793	8.33762453896423e-10\\
2.94547273636818	1.04220449976502e-09\\
2.94747373686843	1.25064654563361e-09\\
2.94947473736868	1.04220449976502e-09\\
2.95147573786893	8.33762453896423e-10\\
2.95347673836918	8.33762453896423e-10\\
2.95547773886943	8.33762453896423e-10\\
2.95747873936968	6.25320408027829e-10\\
2.95947973986994	1.25064654563361e-09\\
2.96148074037019	1.25064654563361e-09\\
2.96348174087044	1.25064654563361e-09\\
2.96548274137069	1.66752490779285e-09\\
2.96748374187094	1.25064654563361e-09\\
2.96948474237119	1.25064654563361e-09\\
2.97148574287144	1.25064654563361e-09\\
2.97348674337169	8.33762453896423e-10\\
2.97548774387194	8.33762453896423e-10\\
2.97748874437219	6.25320408027829e-10\\
2.97948974487244	2.08440899953003e-10\\
2.98149074537269	-2.08440899953003e-10\\
2.98349174587294	2.08440899953003e-10\\
2.98549274637319	-2.08440899953003e-10\\
2.98749374687344	-2.08440899953003e-10\\
2.98949474737369	-2.08440899953003e-10\\
2.99149574787394	-2.08440899953003e-10\\
2.99349674837419	-2.08440899953003e-10\\
2.99549774887444	0\\
2.99749874937469	4.16881799906007e-10\\
2.99949974987494	6.25320408027829e-10\\
3.00150075037519	1.04220449976502e-09\\
3.00350175087544	1.25064654563361e-09\\
3.00550275137569	1.04220449976502e-09\\
3.00750375187594	8.33762453896423e-10\\
3.00950475237619	6.25320408027829e-10\\
3.01150575287644	4.16881799906007e-10\\
3.01350675337669	2.08440899953003e-10\\
3.01550775387694	0\\
3.01750875437719	-4.16881799906007e-10\\
3.01950975487744	-2.08440899953003e-10\\
3.02151075537769	-4.16881799906007e-10\\
3.02351175587794	-8.33762453896423e-10\\
3.02551275637819	-6.25320408027829e-10\\
3.02751375687844	-6.25320408027829e-10\\
3.02951475737869	-8.33762453896423e-10\\
3.03151575787894	-1.04220449976502e-09\\
3.03351675837919	-8.33762453896423e-10\\
3.03551775887944	-6.25320408027829e-10\\
3.03751875937969	-6.25320408027829e-10\\
3.03951975987994	-6.25320408027829e-10\\
3.04152076038019	-8.33762453896423e-10\\
3.04352176088044	-6.25320408027829e-10\\
3.04552276138069	-2.08440899953003e-10\\
3.04752376188094	-2.08440899953003e-10\\
3.04952476238119	-4.16881799906007e-10\\
3.05152576288144	-6.25320408027829e-10\\
3.05352676338169	-6.25320408027829e-10\\
3.05552776388194	-6.25320408027829e-10\\
3.05752876438219	-1.25064654563361e-09\\
3.05952976488244	-1.4590885915022e-09\\
3.06153076538269	-1.25064654563361e-09\\
3.06353176588294	-1.25064654563361e-09\\
3.06553276638319	-1.66752490779285e-09\\
3.06753376688344	-1.66752490779285e-09\\
3.06953476738369	-1.25064654563361e-09\\
3.07153576788394	-1.25064654563361e-09\\
3.07353676838419	-1.25064654563361e-09\\
3.07553776888444	-1.25064654563361e-09\\
3.07753876938469	-1.04220449976502e-09\\
3.07953976988494	-1.4590885915022e-09\\
3.08154077038519	-1.4590885915022e-09\\
3.08354177088544	-2.08440899953003e-09\\
3.08554277138569	-1.87596695366144e-09\\
3.08754377188594	-2.29285104539863e-09\\
3.08954477238619	-2.50128736168927e-09\\
3.09154577288644	-2.70972940755786e-09\\
3.09354677338669	-2.50128736168927e-09\\
3.09554777388694	-2.29285104539863e-09\\
3.09754877438719	-2.29285104539863e-09\\
3.09954977488744	-2.70972940755786e-09\\
3.10155077538769	-2.70972940755786e-09\\
3.10355177588794	-2.50128736168927e-09\\
3.10555277638819	-2.08440899953003e-09\\
3.10755377688844	-2.08440899953003e-09\\
3.10955477738869	-1.66752490779285e-09\\
3.11155577788894	-1.66752490779285e-09\\
3.11355677838919	-1.66752490779285e-09\\
3.11555777888944	-1.66752490779285e-09\\
3.11755877938969	-1.87596695366144e-09\\
3.11955977988994	-1.87596695366144e-09\\
3.1215607803902	-1.87596695366144e-09\\
3.12356178089045	-2.08440899953003e-09\\
3.1255627813907	-2.29285104539863e-09\\
3.12756378189095	-1.66752490779285e-09\\
3.1295647823912	-1.4590885915022e-09\\
3.13156578289145	-1.66752490779285e-09\\
3.1335667833917	-1.87596695366144e-09\\
3.13556778389195	-2.29285104539863e-09\\
3.1375687843922	-2.29285104539863e-09\\
3.13956978489245	-2.29285104539863e-09\\
3.1415707853927	-2.91817145342646e-09\\
3.14357178589295	-2.29285104539863e-09\\
3.1455727863932	-2.70972940755786e-09\\
3.14757378689345	-3.12661349929505e-09\\
3.1495747873937	-3.33505554516364e-09\\
3.15157578789395	-3.33505554516364e-09\\
3.1535767883942	-3.12661349929505e-09\\
3.15557778889445	-2.91817145342646e-09\\
3.1575787893947	-2.70972940755786e-09\\
3.15957978989495	-2.91817145342646e-09\\
3.1615807903952	-3.33505554516364e-09\\
3.16358179089545	-3.12661349929505e-09\\
3.1655827913957	-3.33505554516364e-09\\
3.16758379189595	-3.54349186145428e-09\\
3.1695847923962	-3.75193390732288e-09\\
3.17158579289645	-3.75193390732288e-09\\
3.1735867933967	-3.75193390732288e-09\\
3.17558779389695	-3.54349186145428e-09\\
3.1775887943972	-3.12661349929505e-09\\
3.17958979489745	-3.54349186145428e-09\\
3.1815907953977	-3.75193390732288e-09\\
3.18359179589795	-3.75193390732288e-09\\
3.1855927963982	-3.54349186145428e-09\\
3.18759379689845	-3.75193390732288e-09\\
3.1895947973987	-3.96037595319147e-09\\
3.19159579789895	-4.37726004492866e-09\\
3.1935967983992	-4.16881799906006e-09\\
3.19559779889945	-4.37726004492866e-09\\
3.1975987993997	-4.37726004492866e-09\\
3.19959979989995	-4.16881799906006e-09\\
3.2016008004002	-4.16881799906006e-09\\
3.20360180090045	-3.75193390732288e-09\\
3.2056028014007	-3.75193390732288e-09\\
3.20760380190095	-3.75193390732288e-09\\
3.2096048024012	-3.75193390732288e-09\\
3.21160580290145	-3.75193390732288e-09\\
3.2136068034017	-3.75193390732288e-09\\
3.21560780390195	-3.96037595319147e-09\\
3.2176088044022	-3.96037595319147e-09\\
3.21960980490245	-3.75193390732288e-09\\
3.2216108054027	-4.5856963612193e-09\\
3.22361180590295	-4.79413840708789e-09\\
3.2256128064032	-5.21102249882508e-09\\
3.22761380690345	-5.41945881511572e-09\\
3.2296148074037	-5.00258045295649e-09\\
3.23161580790395	-5.21102249882508e-09\\
3.2336168084042	-5.21102249882508e-09\\
3.23561780890445	-5.00258045295649e-09\\
3.2376188094047	-4.5856963612193e-09\\
3.23961980990495	-4.79413840708789e-09\\
3.2416208104052	-4.79413840708789e-09\\
3.24362181090545	-5.00258045295649e-09\\
3.2456228114057	-4.5856963612193e-09\\
3.24762381190595	-4.79413840708789e-09\\
3.2496248124062	-5.00258045295649e-09\\
3.25162581290645	-4.5856963612193e-09\\
3.2536268134067	-4.16881799906006e-09\\
3.25562781390695	-4.16881799906006e-09\\
3.2576288144072	-4.16881799906006e-09\\
3.25962981490745	-4.5856963612193e-09\\
3.2616308154077	-5.00258045295649e-09\\
3.26363181590795	-4.79413840708789e-09\\
3.2656328164082	-4.79413840708789e-09\\
3.26763381690845	-4.5856963612193e-09\\
3.2696348174087	-5.00258045295649e-09\\
3.27163581790895	-5.21102249882508e-09\\
3.2736368184092	-5.21102249882508e-09\\
3.27563781890945	-5.21102249882508e-09\\
3.27763881940971	-5.00258045295649e-09\\
3.27963981990996	-5.21102249882508e-09\\
3.2816408204102	-5.41945881511572e-09\\
3.28364182091046	-5.21102249882508e-09\\
3.28564282141071	-5.00258045295649e-09\\
3.28764382191096	-5.21102249882508e-09\\
3.28964482241121	-5.21102249882508e-09\\
3.29164582291146	-4.5856963612193e-09\\
3.29364682341171	-4.79413840708789e-09\\
3.29564782391196	-5.21102249882508e-09\\
3.29764882441221	-5.00258045295649e-09\\
3.29964982491246	-4.79413840708789e-09\\
3.30165082541271	-4.5856963612193e-09\\
3.30365182591296	-4.37726004492866e-09\\
3.30565282641321	-4.5856963612193e-09\\
3.30765382691346	-4.79413840708789e-09\\
3.30965482741371	-3.96037595319147e-09\\
3.31165582791396	-4.16881799906006e-09\\
3.31365682841421	-4.79413840708789e-09\\
3.31565782891446	-4.79413840708789e-09\\
3.31765882941471	-4.5856963612193e-09\\
3.31965982991496	-4.79413840708789e-09\\
3.32166083041521	-4.79413840708789e-09\\
3.32366183091546	-4.5856963612193e-09\\
3.32566283141571	-4.37726004492866e-09\\
3.32766383191596	-4.37726004492866e-09\\
3.32966483241621	-4.79413840708789e-09\\
3.33166583291646	-4.79413840708789e-09\\
3.33366683341671	-4.79413840708789e-09\\
3.33566783391696	-4.5856963612193e-09\\
3.33766883441721	-4.37726004492866e-09\\
3.33966983491746	-4.5856963612193e-09\\
3.34167083541771	-4.16881799906006e-09\\
3.34367183591796	-4.37726004492866e-09\\
3.34567283641821	-4.37726004492866e-09\\
3.34767383691846	-4.5856963612193e-09\\
3.34967483741871	-4.37726004492866e-09\\
3.35167583791896	-4.5856963612193e-09\\
3.35367683841921	-5.00258045295649e-09\\
3.35567783891946	-5.21102249882508e-09\\
3.35767883941971	-5.62790086098432e-09\\
3.35967983991996	-5.00258045295649e-09\\
3.36168084042021	-5.00258045295649e-09\\
3.36368184092046	-4.79413840708789e-09\\
3.36568284142071	-4.37726004492866e-09\\
3.36768384192096	-4.5856963612193e-09\\
3.36968484242121	-4.5856963612193e-09\\
3.37168584292146	-4.79413840708789e-09\\
3.37368684342171	-5.00258045295649e-09\\
3.37568784392196	-4.79413840708789e-09\\
3.37768884442221	-5.21102249882508e-09\\
3.37968984492246	-5.21102249882508e-09\\
3.38169084542271	-4.79413840708789e-09\\
3.38369184592296	-4.79413840708789e-09\\
3.38569284642321	-4.5856963612193e-09\\
3.38769384692346	-5.00258045295649e-09\\
3.38969484742371	-5.00258045295649e-09\\
3.39169584792396	-4.79413840708789e-09\\
3.39369684842421	-4.79413840708789e-09\\
3.39569784892446	-5.00258045295649e-09\\
3.39769884942471	-4.79413840708789e-09\\
3.39969984992496	-4.79413840708789e-09\\
3.40170085042521	-4.79413840708789e-09\\
3.40370185092546	-4.37726004492866e-09\\
3.40570285142571	-4.37726004492866e-09\\
3.40770385192596	-4.16881799906006e-09\\
3.40970485242621	-3.75193390732288e-09\\
3.41170585292646	-3.75193390732288e-09\\
3.41370685342671	-4.16881799906006e-09\\
3.41570785392696	-4.37726004492866e-09\\
3.41770885442721	-4.79413840708789e-09\\
3.41970985492746	-5.41945881511572e-09\\
3.42171085542771	-5.41945881511572e-09\\
3.42371185592796	-5.62790086098432e-09\\
3.42571285642821	-5.21102249882508e-09\\
3.42771385692846	-5.62790086098432e-09\\
3.42971485742871	-5.62790086098432e-09\\
3.43171585792896	-5.8363199885411e-09\\
3.43371685842921	-5.62790086098432e-09\\
3.43571785892946	-5.62790086098432e-09\\
3.43771885942971	-5.8363199885411e-09\\
3.43971985992996	-5.8363199885411e-09\\
3.44172086043022	-6.46164612614688e-09\\
3.44372186093047	-6.25320408027829e-09\\
3.44572286143072	-6.25320408027829e-09\\
3.44772386193097	-6.25320408027829e-09\\
3.44972486243122	-6.25320408027829e-09\\
3.45172586293147	-6.25320408027829e-09\\
3.45372686343172	-6.0447620344097e-09\\
3.45572786393197	-6.0447620344097e-09\\
3.45772886443222	-6.0447620344097e-09\\
3.45972986493247	-5.8363199885411e-09\\
3.46173086543272	-5.8363199885411e-09\\
3.46373186593297	-5.8363199885411e-09\\
3.46573286643322	-6.25320408027829e-09\\
3.46773386693347	-6.0447620344097e-09\\
3.46973486743372	-6.25320408027829e-09\\
3.47173586793397	-6.25320408027829e-09\\
3.47373686843422	-6.25320408027829e-09\\
3.47573786893447	-6.46164612614688e-09\\
3.47773886943472	-6.25320408027829e-09\\
3.47973986993497	-6.46164612614688e-09\\
3.48174087043522	-6.46164612614688e-09\\
3.48374187093547	-6.46164612614688e-09\\
3.48574287143572	-6.87853021788407e-09\\
3.48774387193597	-6.87853021788407e-09\\
3.48974487243622	-6.87853021788407e-09\\
3.49174587293647	-6.87853021788407e-09\\
3.49374687343672	-7.08697226375267e-09\\
3.49574787393697	-7.08697226375267e-09\\
3.49774887443722	-7.08697226375267e-09\\
3.49974987493747	-7.08697226375267e-09\\
3.50175087543772	-7.08697226375267e-09\\
3.50375187593797	-7.08697226375267e-09\\
3.50575287643822	-7.08697226375267e-09\\
3.50775387693847	-7.08697226375267e-09\\
3.50975487743872	-7.08697226375267e-09\\
3.51175587793897	-7.08697226375267e-09\\
3.51375687843922	-6.46164612614688e-09\\
3.51575787893947	-6.87853021788407e-09\\
3.51775887943972	-7.08697226375267e-09\\
3.51975987993997	-7.08697226375267e-09\\
3.52176088044022	-6.87853021788407e-09\\
3.52376188094047	-7.08697226375267e-09\\
3.52576288144072	-7.08697226375267e-09\\
3.52776388194097	-6.87853021788407e-09\\
3.52976488244122	-6.46164612614688e-09\\
3.53176588294147	-6.87853021788407e-09\\
3.53376688344172	-6.67008817201548e-09\\
3.53576788394197	-6.87853021788407e-09\\
3.53776888444222	-6.87853021788407e-09\\
3.53976988494247	-6.87853021788407e-09\\
3.54177088544272	-6.46164612614688e-09\\
3.54377188594297	-6.25320408027829e-09\\
3.54577288644322	-6.67008817201548e-09\\
3.54777388694347	-6.67008817201548e-09\\
3.54977488744372	-7.08697226375267e-09\\
3.55177588794397	-7.29541430962126e-09\\
3.55377688844422	-7.50385635548985e-09\\
3.55577788894447	-7.50385635548985e-09\\
3.55777888944472	-7.71229840135845e-09\\
3.55977988994497	-7.71229840135845e-09\\
3.56178089044522	-7.71229840135845e-09\\
3.56378189094547	-7.92074044722704e-09\\
3.56578289144572	-8.33762453896423e-09\\
3.56778389194597	-8.54606658483282e-09\\
3.56978489244622	-8.54606658483282e-09\\
3.57178589294647	-8.54606658483282e-09\\
3.57378689344672	-8.54606658483282e-09\\
3.57578789394697	-8.75450863070141e-09\\
3.57778889444722	-8.54606658483282e-09\\
3.57978989494747	-8.75450863070141e-09\\
3.58179089544772	-8.96295067657001e-09\\
3.58379189594797	-9.1713927224386e-09\\
3.58579289644822	-8.96295067657001e-09\\
3.58779389694847	-8.75450863070141e-09\\
3.58979489744872	-8.96295067657001e-09\\
3.59179589794897	-8.96295067657001e-09\\
3.59379689844922	-8.75450863070141e-09\\
3.59579789894947	-8.75450863070141e-09\\
3.59779889944972	-8.75450863070141e-09\\
3.59979989994997	-8.54606658483282e-09\\
3.60180090045022	-8.96295067657001e-09\\
3.60380190095048	-8.75450863070141e-09\\
3.60580290145073	-8.54606658483282e-09\\
3.60780390195098	-8.54606658483282e-09\\
3.60980490245123	-8.54606658483282e-09\\
3.61180590295148	-8.54606658483282e-09\\
3.61380690345173	-8.54606658483282e-09\\
3.61580790395198	-8.12918249309563e-09\\
3.61780890445223	-8.33762453896423e-09\\
3.61980990495248	-8.33762453896423e-09\\
3.62181090545273	-8.33762453896423e-09\\
3.62381190595298	-7.71229840135845e-09\\
3.62581290645323	-7.71229840135845e-09\\
3.62781390695348	-7.92074044722704e-09\\
3.62981490745373	-7.92074044722704e-09\\
3.63181590795398	-7.71229840135845e-09\\
3.63381690845423	-7.29541430962126e-09\\
3.63581790895448	-7.08697226375267e-09\\
3.63781890945473	-7.50385635548985e-09\\
3.63981990995498	-8.12918249309563e-09\\
3.64182091045523	-8.33762453896423e-09\\
3.64382191095548	-8.33762453896423e-09\\
3.64582291145573	-7.92074044722704e-09\\
3.64782391195598	-7.92074044722704e-09\\
3.64982491245623	-7.92074044722704e-09\\
3.65182591295648	-7.71229840135845e-09\\
3.65382691345673	-7.71229840135845e-09\\
3.65582791395698	-7.92074044722704e-09\\
3.65782891445723	-7.92074044722704e-09\\
3.65982991495748	-7.71229840135845e-09\\
3.66183091545773	-7.92074044722704e-09\\
3.66383191595798	-7.71229840135845e-09\\
3.66583291645823	-7.71229840135845e-09\\
3.66783391695848	-7.92074044722704e-09\\
3.66983491745873	-8.12918249309563e-09\\
3.67183591795898	-7.92074044722704e-09\\
3.67383691845923	-8.12918249309563e-09\\
3.67583791895948	-7.92074044722704e-09\\
3.67783891945973	-7.92074044722704e-09\\
3.67983991995998	-7.50385635548985e-09\\
3.68184092046023	-7.50385635548985e-09\\
3.68384192096048	-7.71229840135845e-09\\
3.68584292146073	-7.92074044722704e-09\\
3.68784392196098	-8.12918249309563e-09\\
3.68984492246123	-7.92074044722704e-09\\
3.69184592296148	-8.12918249309563e-09\\
3.69384692346173	-7.92074044722704e-09\\
3.69584792396198	-7.71229840135845e-09\\
3.69784892446223	-7.50385635548985e-09\\
3.69984992496248	-7.29541430962126e-09\\
3.70185092546273	-7.08697226375267e-09\\
3.70385192596298	-7.08697226375267e-09\\
3.70585292646323	-7.08697226375267e-09\\
3.70785392696348	-7.08697226375267e-09\\
3.70985492746373	-6.87853021788407e-09\\
3.71185592796398	-6.67008817201548e-09\\
3.71385692846423	-6.67008817201548e-09\\
3.71585792896448	-5.8363199885411e-09\\
3.71785892946473	-6.46164612614688e-09\\
3.71985992996498	-6.87853021788407e-09\\
3.72186093046523	-7.08697226375267e-09\\
3.72386193096548	-6.67008817201548e-09\\
3.72586293146573	-6.87853021788407e-09\\
3.72786393196598	-6.67008817201548e-09\\
3.72986493246623	-6.0447620344097e-09\\
3.73186593296648	-6.0447620344097e-09\\
3.73386693346673	-6.25320408027829e-09\\
3.73586793396698	-6.25320408027829e-09\\
3.73786893446723	-6.25320408027829e-09\\
3.73986993496748	-6.67008817201548e-09\\
3.74187093546773	-6.67008817201548e-09\\
3.74387193596798	-6.25320408027829e-09\\
3.74587293646823	-6.0447620344097e-09\\
3.74787393696848	-6.0447620344097e-09\\
3.74987493746873	-6.25320408027829e-09\\
3.75187593796898	-6.46164612614688e-09\\
3.75387693846923	-6.46164612614688e-09\\
3.75587793896948	-6.25320408027829e-09\\
3.75787893946973	-6.67008817201548e-09\\
3.75987993996999	-6.67008817201548e-09\\
3.76188094047024	-7.08697226375267e-09\\
3.76388194097049	-7.08697226375267e-09\\
3.76588294147074	-6.87853021788407e-09\\
3.76788394197099	-6.25320408027829e-09\\
3.76988494247124	-6.0447620344097e-09\\
3.77188594297149	-6.46164612614688e-09\\
3.77388694347174	-6.46164612614688e-09\\
3.77588794397199	-6.25320408027829e-09\\
3.77788894447224	-6.67008817201548e-09\\
3.77988994497249	-6.46164612614688e-09\\
3.78189094547274	-6.67008817201548e-09\\
3.78389194597299	-6.67008817201548e-09\\
3.78589294647324	-6.46164612614688e-09\\
3.78789394697349	-6.0447620344097e-09\\
3.78989494747374	-5.8363199885411e-09\\
3.79189594797399	-5.62790086098432e-09\\
3.79389694847424	-4.79413840708789e-09\\
3.79589794897449	-4.5856963612193e-09\\
3.79789894947474	-5.00258045295649e-09\\
3.79989994997499	-5.41945881511572e-09\\
3.80190095047524	-5.62790086098432e-09\\
3.80390195097549	-5.62790086098432e-09\\
3.80590295147574	-6.0447620344097e-09\\
3.80790395197599	-5.62790086098432e-09\\
3.80990495247624	-5.8363199885411e-09\\
3.81190595297649	-6.0447620344097e-09\\
3.81390695347674	-6.0447620344097e-09\\
3.81590795397699	-6.25320408027829e-09\\
3.81790895447724	-6.87853021788407e-09\\
3.81990995497749	-6.46164612614688e-09\\
3.82191095547774	-6.25320408027829e-09\\
3.82391195597799	-6.46164612614688e-09\\
3.82591295647824	-6.67008817201548e-09\\
3.82791395697849	-7.08697226375267e-09\\
3.82991495747874	-7.08697226375267e-09\\
3.83191595797899	-6.87853021788407e-09\\
3.83391695847924	-6.25320408027829e-09\\
3.83591795897949	-6.67008817201548e-09\\
3.83791895947974	-7.08697226375267e-09\\
3.83991995997999	-6.46164612614688e-09\\
3.84192096048024	-7.08697226375267e-09\\
3.84392196098049	-6.46164612614688e-09\\
3.84592296148074	-6.87853021788407e-09\\
3.84792396198099	-6.67008817201548e-09\\
3.84992496248124	-6.25320408027829e-09\\
3.85192596298149	-6.87853021788407e-09\\
3.85392696348174	-6.87853021788407e-09\\
3.85592796398199	-7.08697226375267e-09\\
3.85792896448224	-7.08697226375267e-09\\
3.85992996498249	-7.08697226375267e-09\\
3.86193096548274	-7.08697226375267e-09\\
3.86393196598299	-7.08697226375267e-09\\
3.86593296648324	-7.71229840135845e-09\\
3.86793396698349	-7.50385635548985e-09\\
3.86993496748374	-7.50385635548985e-09\\
3.87193596798399	-7.29541430962126e-09\\
3.87393696848424	-7.08697226375267e-09\\
3.87593796898449	-7.08697226375267e-09\\
3.87793896948474	-7.08697226375267e-09\\
3.87993996998499	-7.08697226375267e-09\\
3.88194097048524	-6.46164612614688e-09\\
3.88394197098549	-6.25320408027829e-09\\
3.88594297148574	-5.8363199885411e-09\\
3.88794397198599	-6.0447620344097e-09\\
3.88994497248624	-6.25320408027829e-09\\
3.89194597298649	-6.0447620344097e-09\\
3.89394697348674	-6.46164612614688e-09\\
3.89594797398699	-6.25320408027829e-09\\
3.89794897448724	-6.0447620344097e-09\\
3.89994997498749	-6.46164612614688e-09\\
3.90195097548774	-6.25320408027829e-09\\
3.90395197598799	-6.0447620344097e-09\\
3.90595297648824	-5.8363199885411e-09\\
3.90795397698849	-6.0447620344097e-09\\
3.90995497748874	-6.25320408027829e-09\\
3.91195597798899	-5.8363199885411e-09\\
3.91395697848924	-5.41945881511572e-09\\
3.91595797898949	-5.00258045295649e-09\\
3.91795897948974	-4.79413840708789e-09\\
3.91995997998999	-4.79413840708789e-09\\
3.92196098049025	-4.37726004492866e-09\\
3.9239619809905	-4.37726004492866e-09\\
3.92596298149075	-3.75193390732288e-09\\
3.927963981991	-3.75193390732288e-09\\
3.92996498249125	-4.16881799906006e-09\\
3.9319659829915	-4.37726004492866e-09\\
3.93396698349175	-4.5856963612193e-09\\
3.935967983992	-4.37726004492866e-09\\
3.93796898449225	-4.37726004492866e-09\\
3.9399699849925	-4.16881799906006e-09\\
3.94197098549275	-3.75193390732288e-09\\
3.943971985993	-3.12661349929505e-09\\
3.94597298649325	-3.12661349929505e-09\\
3.9479739869935	-2.91817145342646e-09\\
3.94997498749375	-2.91817145342646e-09\\
3.951975987994	-3.33505554516364e-09\\
3.95397698849425	-3.54349186145428e-09\\
3.9559779889945	-3.33505554516364e-09\\
3.95797898949475	-3.54349186145428e-09\\
3.959979989995	-3.33505554516364e-09\\
3.96198099049525	-2.70972940755786e-09\\
3.9639819909955	-2.70972940755786e-09\\
3.96598299149575	-2.70972940755786e-09\\
3.967983991996	-2.50128736168927e-09\\
3.96998499249625	-2.50128736168927e-09\\
3.9719859929965	-2.50128736168927e-09\\
3.97398699349675	-2.70972940755786e-09\\
3.975987993997	-2.91817145342646e-09\\
3.97798899449725	-3.54349186145428e-09\\
3.9799899949975	-3.33505554516364e-09\\
3.98199099549775	-3.33505554516364e-09\\
3.983991995998	-3.33505554516364e-09\\
3.98599299649825	-3.12661349929505e-09\\
3.9879939969985	-2.91817145342646e-09\\
3.98999499749875	-2.70972940755786e-09\\
3.991995997999	-2.91817145342646e-09\\
3.99399699849925	-2.91817145342646e-09\\
3.9959979989995	-2.91817145342646e-09\\
3.99799899949975	-2.50128736168927e-09\\
4	-2.50128736168927e-09\\
};
\addlegendentry{c2};

\addplot [color=mycolor3,solid]
  table[row sep=crcr]{%
0	-6.67008817201548e-09\\
0.00200100050025012	2.66804099838414e-08\\
0.00400200100050025	3.33505554516364e-08\\
0.00600300150075038	2.66804099838414e-08\\
0.0080040020010005	2.66804099838414e-08\\
0.0100050025012506	4.00206436236519e-08\\
0.0120060030015008	2.66804099838414e-08\\
0.0140070035017509	6.67008817201548e-08\\
0.016008004002001	1.0005160905913e-07\\
0.0180090045022511	8.67114327150988e-08\\
0.0200100050025013	8.67114327150988e-08\\
0.0220110055027514	6.67008817201548e-08\\
0.0240120060030015	4.00206436236519e-08\\
0.0260130065032516	4.00206436236519e-08\\
0.0280140070035018	0\\
0.0300150075037519	-1.33402336398105e-08\\
0.032016008004002	-2.66804099838414e-08\\
0.0340170085042521	-2.00103218118259e-08\\
0.0360180090045022	-1.33402336398105e-08\\
0.0380190095047524	-2.00103218118259e-08\\
0.0400200100050025	-1.33402336398105e-08\\
0.0420210105052526	-1.33402336398105e-08\\
0.0440220110055028	-2.66804099838414e-08\\
0.0460230115057529	-2.00103218118259e-08\\
0.048024012006003	-4.00206436236519e-08\\
0.0500250125062531	-4.00206436236519e-08\\
0.0520260130065033	-2.00103218118259e-08\\
0.0540270135067534	-5.33608772634624e-08\\
0.0560280140070035	-3.33505554516364e-08\\
0.0580290145072536	-2.66804099838414e-08\\
0.0600300150075038	-3.33505554516364e-08\\
0.0620310155077539	-4.66907317956674e-08\\
0.064032016008004	-6.00310800270369e-08\\
0.0660330165082541	-6.67008817201548e-08\\
0.0680340170085043	-4.66907317956674e-08\\
0.0700350175087544	-4.00206436236519e-08\\
0.0720360180090045	-2.00103218118259e-08\\
0.0740370185092546	-2.00103218118259e-08\\
0.0760380190095048	-2.00103218118259e-08\\
0.0780390195097549	0\\
0.080040020010005	-4.00206436236519e-08\\
0.0820410205102551	-6.00310800270369e-08\\
0.0840420210105053	-6.00310800270369e-08\\
0.0860430215107554	-7.33712563710678e-08\\
0.0880440220110055	-9.33812344082167e-08\\
0.0900450225112556	-9.33812344082167e-08\\
0.0920460230115058	-8.00410580641858e-08\\
0.0940470235117559	-8.67114327150988e-08\\
0.096048024012006	-7.33712563710678e-08\\
0.0980490245122561	-6.67008817201548e-08\\
0.100050025012506	-6.67008817201548e-08\\
0.102051025512756	-9.33812344082167e-08\\
0.104052026013007	-1.13391785403161e-07\\
0.106053026513257	-1.13391785403161e-07\\
0.108054027013507	-1.06721983710043e-07\\
0.110055027513757	-1.13391785403161e-07\\
0.112056028014007	-1.0005160905913e-07\\
0.114057028514257	-1.0005160905913e-07\\
0.116058029014507	-7.33712563710678e-08\\
0.118059029514757	-1.0005160905913e-07\\
0.120060030015008	-9.33812344082167e-08\\
0.122061030515258	-1.0005160905913e-07\\
0.124062031015508	-1.13391785403161e-07\\
0.126063031515758	-1.26731961747192e-07\\
0.128064032016008	-1.20062160054074e-07\\
0.130065032516258	-1.46742512742136e-07\\
0.132066033016508	-1.33402336398105e-07\\
0.134067033516758	-1.60082689086167e-07\\
0.136068034017009	-1.80092667123315e-07\\
0.138069034517259	-1.66752490779285e-07\\
0.140070035017509	-1.73422865430198e-07\\
0.142071035517759	-1.86763041774229e-07\\
0.144072036018009	-1.40072138091223e-07\\
0.146073036518259	-1.80092667123315e-07\\
0.148074037018509	-1.60082689086167e-07\\
0.150075037518759	-1.86763041774229e-07\\
0.15207603801901	-1.80092667123315e-07\\
0.15407703851926	-1.46742512742136e-07\\
0.15607803901951	-1.40072138091223e-07\\
0.15807903951976	-1.66752490779285e-07\\
0.16008004002001	-1.53412314435254e-07\\
0.16208104052026	-1.53412314435254e-07\\
0.16408204102051	-1.53412314435254e-07\\
0.16608304152076	-1.66752490779285e-07\\
0.168084042021011	-1.53412314435254e-07\\
0.170085042521261	-1.26731961747192e-07\\
0.172086043021511	-1.33402336398105e-07\\
0.174087043521761	-1.33402336398105e-07\\
0.176088044022011	-1.33402336398105e-07\\
0.178089044522261	-1.46742512742136e-07\\
0.180090045022511	-1.53412314435254e-07\\
0.182091045522761	-1.73422865430198e-07\\
0.184092046023012	-1.53412314435254e-07\\
0.186093046523262	-1.46742512742136e-07\\
0.188094047023512	-1.80092667123315e-07\\
0.190095047523762	-1.73422865430198e-07\\
0.192096048024012	-1.80092667123315e-07\\
0.194097048524262	-1.86763041774229e-07\\
0.196098049024512	-1.73422865430198e-07\\
0.198099049524762	-1.86763041774229e-07\\
0.200100050025012	-1.80092667123315e-07\\
0.202101050525263	-1.86763041774229e-07\\
0.204102051025513	-1.73422865430198e-07\\
0.206103051525763	-1.73422865430198e-07\\
0.208104052026013	-1.60082689086167e-07\\
0.210105052526263	-1.20062160054074e-07\\
0.212106053026513	-1.60082689086167e-07\\
0.214107053526763	-1.60082689086167e-07\\
0.216108054027013	-1.86763041774229e-07\\
0.218109054527264	-1.86763041774229e-07\\
0.220110055027514	-1.53412314435254e-07\\
0.222111055527764	-1.60082689086167e-07\\
0.224112056028014	-1.40072138091223e-07\\
0.226113056528264	-1.66752490779285e-07\\
0.228114057028514	-1.93432843467346e-07\\
0.230115057528764	-1.93432843467346e-07\\
0.232116058029014	-1.86763041774229e-07\\
0.234117058529265	-1.93432843467346e-07\\
0.236118059029515	-1.93432843467346e-07\\
0.238119059529765	-1.93432843467346e-07\\
0.240120060030015	-2.06773019811377e-07\\
0.242121060530265	-1.86763041774229e-07\\
0.244122061030515	-2.26783570806321e-07\\
0.246123061530765	-2.0010321811826e-07\\
0.248124062031016	-1.93432843467346e-07\\
0.250125062531266	-2.06773019811377e-07\\
0.252126063031516	-1.86763041774229e-07\\
0.254127063531766	-2.1344339446229e-07\\
0.256128064032016	-2.53463923494383e-07\\
0.258129064532266	-2.53463923494383e-07\\
0.260130065032516	-2.60134298145296e-07\\
0.262131065532766	-2.46794121801265e-07\\
0.264132066033017	-2.40123747150352e-07\\
0.266133066533267	-2.26783570806321e-07\\
0.268134067033517	-2.20113769113203e-07\\
0.270135067533767	-2.06773019811377e-07\\
0.272136068034017	-2.26783570806321e-07\\
0.274137068534267	-1.93432843467346e-07\\
0.276138069034517	-1.93432843467346e-07\\
0.278139069534767	-2.06773019811377e-07\\
0.280140070035018	-2.06773019811377e-07\\
0.282141070535268	-2.20113769113203e-07\\
0.284142071035518	-2.40123747150352e-07\\
0.286143071535768	-2.20113769113203e-07\\
0.288144072036018	-2.26783570806321e-07\\
0.290145072536268	-2.26783570806321e-07\\
0.292146073036518	-2.1344339446229e-07\\
0.294147073536768	-2.06773019811377e-07\\
0.296148074037018	-2.06773019811377e-07\\
0.298149074537269	-2.0010321811826e-07\\
0.300150075037519	-2.06773019811377e-07\\
0.302151075537769	-2.06773019811377e-07\\
0.304152076038019	-2.06773019811377e-07\\
0.306153076538269	-2.26783570806321e-07\\
0.308154077038519	-2.20113769113203e-07\\
0.310155077538769	-2.26783570806321e-07\\
0.31215607803902	-2.33453945457234e-07\\
0.31415707853927	-2.33453945457234e-07\\
0.31615807903952	-2.33453945457234e-07\\
0.31815907953977	-2.20113769113203e-07\\
0.32016008004002	-2.40123747150352e-07\\
0.32216108054027	-2.40123747150352e-07\\
0.32416208104052	-2.46794121801265e-07\\
0.32616308154077	-2.60134298145296e-07\\
0.32816408204102	-2.40123747150352e-07\\
0.330165082541271	-2.33453945457234e-07\\
0.332166083041521	-2.73474474489327e-07\\
0.334167083541771	-2.53463923494383e-07\\
0.336168084042021	-2.80144276182445e-07\\
0.338169084542271	-2.86814650833358e-07\\
0.340170085042521	-2.53463923494383e-07\\
0.342171085542771	-2.40123747150352e-07\\
0.344172086043022	-2.73474474489327e-07\\
0.346173086543272	-2.73474474489327e-07\\
0.348174087043522	-2.66804099838414e-07\\
0.350175087543772	-2.40123747150352e-07\\
0.352176088044022	-2.33453945457234e-07\\
0.354177088544272	-2.26783570806321e-07\\
0.356178089044522	-2.40123747150352e-07\\
0.358179089544772	-2.40123747150352e-07\\
0.360180090045022	-2.33453945457234e-07\\
0.362181090545273	-2.26783570806321e-07\\
0.364182091045523	-1.93432843467346e-07\\
0.366183091545773	-1.80092667123315e-07\\
0.368184092046023	-2.1344339446229e-07\\
0.370185092546273	-2.26783570806321e-07\\
0.372186093046523	-2.33453945457234e-07\\
0.374187093546773	-2.26783570806321e-07\\
0.376188094047024	-2.20113769113203e-07\\
0.378189094547274	-2.26783570806321e-07\\
0.380190095047524	-2.20113769113203e-07\\
0.382191095547774	-1.93432843467346e-07\\
0.384192096048024	-1.66752490779285e-07\\
0.386193096548274	-1.46742512742136e-07\\
0.388194097048524	-1.13391785403161e-07\\
0.390195097548774	-1.06721983710043e-07\\
0.392196098049024	-1.0005160905913e-07\\
0.394197098549275	-8.00410580641858e-08\\
0.396198099049525	-9.33812344082167e-08\\
0.398199099549775	-9.33812344082167e-08\\
0.400200100050025	-1.20062160054074e-07\\
0.402201100550275	-1.13391785403161e-07\\
0.404202101050525	-1.26731961747192e-07\\
0.406203101550775	-1.06721983710043e-07\\
0.408204102051026	-9.33812344082167e-08\\
0.410205102551276	-1.13391785403161e-07\\
0.412206103051526	-1.26731961747192e-07\\
0.414207103551776	-1.0005160905913e-07\\
0.416208104052026	-1.0005160905913e-07\\
0.418209104552276	-9.33812344082167e-08\\
0.420210105052526	-8.00410580641858e-08\\
0.422211105552776	-8.00410580641858e-08\\
0.424212106053027	-4.66907317956674e-08\\
0.426213106553277	-4.66907317956674e-08\\
0.428214107053527	-2.00103218118259e-08\\
0.430215107553777	-2.00103218118259e-08\\
0.432216108054027	-2.00103218118259e-08\\
0.434217108554277	-4.00206436236519e-08\\
0.436218109054527	-2.00103218118259e-08\\
0.438219109554777	0\\
0.440220110055028	6.67008817201548e-09\\
0.442221110555278	-6.67008817201548e-09\\
0.444222111055528	-1.33402336398105e-08\\
0.446223111555778	-2.00103218118259e-08\\
0.448224112056028	-4.66907317956674e-08\\
0.450225112556278	-2.00103218118259e-08\\
0.452226113056528	-2.00103218118259e-08\\
0.454227113556778	-3.33505554516364e-08\\
0.456228114057029	0\\
0.458229114557279	2.00103218118259e-08\\
0.460230115057529	1.33402336398105e-08\\
0.462231115557779	-2.66804099838414e-08\\
0.464232116058029	-6.67008817201548e-08\\
0.466233116558279	-8.67114327150988e-08\\
0.468234117058529	-7.33712563710678e-08\\
0.470235117558779	-9.33812344082167e-08\\
0.47223611805903	-9.33812344082167e-08\\
0.47423711855928	-8.00410580641858e-08\\
0.47623811905953	-8.00410580641858e-08\\
0.47823911955978	-1.13391785403161e-07\\
0.48024012006003	-1.26731961747192e-07\\
0.48224112056028	-1.20062160054074e-07\\
0.48424212106053	-1.13391785403161e-07\\
0.48624312156078	-1.0005160905913e-07\\
0.488244122061031	-1.26731961747192e-07\\
0.490245122561281	-1.20062160054074e-07\\
0.492246123061531	-1.26731961747192e-07\\
0.494247123561781	-1.26731961747192e-07\\
0.496248124062031	-1.60082689086167e-07\\
0.498249124562281	-1.20062160054074e-07\\
0.500250125062531	-1.26731961747192e-07\\
0.502251125562781	-1.40072138091223e-07\\
0.504252126063031	-1.33402336398105e-07\\
0.506253126563282	-1.20062160054074e-07\\
0.508254127063532	-1.20062160054074e-07\\
0.510255127563782	-1.13391785403161e-07\\
0.512256128064032	-1.13391785403161e-07\\
0.514257128564282	-7.33712563710678e-08\\
0.516258129064532	-8.00410580641858e-08\\
0.518259129564782	-6.00310800270369e-08\\
0.520260130065032	-4.66907317956674e-08\\
0.522261130565283	-4.66907317956674e-08\\
0.524262131065533	-4.66907317956674e-08\\
0.526263131565783	-4.66907317956674e-08\\
0.528264132066033	-3.33505554516364e-08\\
0.530265132566283	-2.00103218118259e-08\\
0.532266133066533	-2.66804099838414e-08\\
0.534267133566783	-2.00103218118259e-08\\
0.536268134067034	-2.66804099838414e-08\\
0.538269134567284	-2.00103218118259e-08\\
0.540270135067534	-2.66804099838414e-08\\
0.542271135567784	-4.00206436236519e-08\\
0.544272136068034	-6.00310800270369e-08\\
0.546273136568284	-2.66804099838414e-08\\
0.548274137068534	-1.33402336398105e-08\\
0.550275137568784	-2.00103218118259e-08\\
0.552276138069035	-2.00103218118259e-08\\
0.554277138569285	6.67008817201548e-09\\
0.556278139069535	2.66804099838414e-08\\
0.558279139569785	3.33505554516364e-08\\
0.560280140070035	5.33608772634624e-08\\
0.562281140570285	4.00206436236519e-08\\
0.564282141070535	6.67008817201548e-09\\
0.566283141570785	1.33402336398105e-08\\
0.568284142071036	-6.67008817201548e-09\\
0.570285142571286	-1.33402336398105e-08\\
0.572286143071536	-2.00103218118259e-08\\
0.574287143571786	-1.33402336398105e-08\\
0.576288144072036	-2.66804099838414e-08\\
0.578289144572286	-2.00103218118259e-08\\
0.580290145072536	-4.66907317956674e-08\\
0.582291145572786	-2.66804099838414e-08\\
0.584292146073036	-3.33505554516364e-08\\
0.586293146573287	-3.33505554516364e-08\\
0.588294147073537	-2.66804099838414e-08\\
0.590295147573787	-2.00103218118259e-08\\
0.592296148074037	6.67008817201548e-09\\
0.594297148574287	1.33402336398105e-08\\
0.596298149074537	0\\
0.598299149574787	1.33402336398105e-08\\
0.600300150075038	6.67008817201548e-09\\
0.602301150575288	1.33402336398105e-08\\
0.604302151075538	-6.67008817201548e-09\\
0.606303151575788	-2.00103218118259e-08\\
0.608304152076038	-6.67008817201548e-09\\
0.610305152576288	1.33402336398105e-08\\
0.612306153076538	3.33505554516364e-08\\
0.614307153576788	6.00310800270369e-08\\
0.616308154077039	7.33712563710678e-08\\
0.618309154577289	9.33812344082167e-08\\
0.620310155077539	6.00310800270369e-08\\
0.622311155577789	2.66804099838414e-08\\
0.624312156078039	3.33505554516364e-08\\
0.626313156578289	3.33505554516364e-08\\
0.628314157078539	3.33505554516364e-08\\
0.630315157578789	2.00103218118259e-08\\
0.63231615807904	1.33402336398105e-08\\
0.63431715857929	-6.67008817201548e-09\\
0.63631815907954	0\\
0.63831915957979	-2.66804099838414e-08\\
0.64032016008004	-6.67008817201548e-08\\
0.64232116058029	-4.66907317956674e-08\\
0.64432216108054	-8.00410580641858e-08\\
0.64632316158079	-6.67008817201548e-08\\
0.64832416208104	-6.67008817201548e-08\\
0.650325162581291	-4.00206436236519e-08\\
0.652326163081541	-5.33608772634624e-08\\
0.654327163581791	-6.67008817201548e-08\\
0.656328164082041	-2.00103218118259e-08\\
0.658329164582291	0\\
0.660330165082541	-2.00103218118259e-08\\
0.662331165582791	-4.00206436236519e-08\\
0.664332166083042	-4.66907317956674e-08\\
0.666333166583292	-4.66907317956674e-08\\
0.668334167083542	-6.00310800270369e-08\\
0.670335167583792	-4.66907317956674e-08\\
0.672336168084042	-2.00103218118259e-08\\
0.674337168584292	-4.66907317956674e-08\\
0.676338169084542	-4.66907317956674e-08\\
0.678339169584792	-4.66907317956674e-08\\
0.680340170085043	-4.00206436236519e-08\\
0.682341170585293	-4.00206436236519e-08\\
0.684342171085543	-4.66907317956674e-08\\
0.686343171585793	-4.66907317956674e-08\\
0.688344172086043	-4.00206436236519e-08\\
0.690345172586293	-2.66804099838414e-08\\
0.692346173086543	-3.33505554516364e-08\\
0.694347173586793	-2.00103218118259e-08\\
0.696348174087044	0\\
0.698349174587294	0\\
0.700350175087544	6.67008817201548e-09\\
0.702351175587794	0\\
0.704352176088044	-2.66804099838414e-08\\
0.706353176588294	-2.00103218118259e-08\\
0.708354177088544	-2.66804099838414e-08\\
0.710355177588794	-2.00103218118259e-08\\
0.712356178089045	0\\
0.714357178589295	-2.00103218118259e-08\\
0.716358179089545	-2.00103218118259e-08\\
0.718359179589795	-1.33402336398105e-08\\
0.720360180090045	-3.33505554516364e-08\\
0.722361180590295	-3.33505554516364e-08\\
0.724362181090545	-2.00103218118259e-08\\
0.726363181590795	-5.33608772634624e-08\\
0.728364182091045	-8.00410580641858e-08\\
0.730365182591296	-6.67008817201548e-08\\
0.732366183091546	-7.33712563710678e-08\\
0.734367183591796	-5.33608772634624e-08\\
0.736368184092046	-4.66907317956674e-08\\
0.738369184592296	-4.66907317956674e-08\\
0.740370185092546	-2.00103218118259e-08\\
0.742371185592796	-4.66907317956674e-08\\
0.744372186093047	-5.33608772634624e-08\\
0.746373186593297	-8.00410580641858e-08\\
0.748374187093547	-4.66907317956674e-08\\
0.750375187593797	-1.33402336398105e-08\\
0.752376188094047	-4.00206436236519e-08\\
0.754377188594297	-4.00206436236519e-08\\
0.756378189094547	-4.66907317956674e-08\\
0.758379189594797	-6.00310800270369e-08\\
0.760380190095048	-7.33712563710678e-08\\
0.762381190595298	-6.00310800270369e-08\\
0.764382191095548	-6.00310800270369e-08\\
0.766383191595798	-4.66907317956674e-08\\
0.768384192096048	-6.00310800270369e-08\\
0.770385192596298	-2.66804099838414e-08\\
0.772386193096548	-4.66907317956674e-08\\
0.774387193596798	-6.67008817201548e-08\\
0.776388194097049	-7.33712563710678e-08\\
0.778389194597299	-1.0005160905913e-07\\
0.780390195097549	-9.33812344082167e-08\\
0.782391195597799	-1.0005160905913e-07\\
0.784392196098049	-8.00410580641858e-08\\
0.786393196598299	-8.67114327150988e-08\\
0.788394197098549	-6.00310800270369e-08\\
0.790395197598799	-7.33712563710678e-08\\
0.792396198099049	-5.33608772634624e-08\\
0.7943971985993	-7.33712563710678e-08\\
0.79639819909955	-8.00410580641858e-08\\
0.7983991995998	-6.00310800270369e-08\\
0.80040020010005	-6.67008817201548e-08\\
0.8024012006003	-8.00410580641858e-08\\
0.80440220110055	-8.00410580641858e-08\\
0.8064032016008	-1.0005160905913e-07\\
0.808404202101051	-8.00410580641858e-08\\
0.810405202601301	-8.67114327150988e-08\\
0.812406203101551	-8.67114327150988e-08\\
0.814407203601801	-8.00410580641858e-08\\
0.816408204102051	-7.33712563710678e-08\\
0.818409204602301	-7.33712563710678e-08\\
0.820410205102551	-7.33712563710678e-08\\
0.822411205602801	-8.00410580641858e-08\\
0.824412206103052	-8.00410580641858e-08\\
0.826413206603302	-8.00410580641858e-08\\
0.828414207103552	-8.00410580641858e-08\\
0.830415207603802	-7.33712563710678e-08\\
0.832416208104052	-8.67114327150988e-08\\
0.834417208604302	-1.0005160905913e-07\\
0.836418209104552	-9.33812344082167e-08\\
0.838419209604802	-9.33812344082167e-08\\
0.840420210105053	-8.67114327150988e-08\\
0.842421210605303	-8.00410580641858e-08\\
0.844422211105553	-7.33712563710678e-08\\
0.846423211605803	-6.67008817201548e-08\\
0.848424212106053	-5.33608772634624e-08\\
0.850425212606303	-3.33505554516364e-08\\
0.852426213106553	-2.00103218118259e-08\\
0.854427213606803	0\\
0.856428214107053	1.33402336398105e-08\\
0.858429214607304	6.67008817201548e-09\\
0.860430215107554	3.33505554516364e-08\\
0.862431215607804	5.33608772634624e-08\\
0.864432216108054	4.66907317956674e-08\\
0.866433216608304	6.67008817201548e-08\\
0.868434217108554	3.33505554516364e-08\\
0.870435217608804	5.33608772634624e-08\\
0.872436218109054	4.66907317956674e-08\\
0.874437218609305	6.00310800270369e-08\\
0.876438219109555	8.00410580641858e-08\\
0.878439219609805	6.00310800270369e-08\\
0.880440220110055	4.00206436236519e-08\\
0.882441220610305	6.00310800270369e-08\\
0.884442221110555	4.66907317956674e-08\\
0.886443221610805	0\\
0.888444222111056	0\\
0.890445222611306	0\\
0.892446223111556	-2.66804099838414e-08\\
0.894447223611806	-3.33505554516364e-08\\
0.896448224112056	-2.66804099838414e-08\\
0.898449224612306	-2.00103218118259e-08\\
0.900450225112556	-2.00103218118259e-08\\
0.902451225612806	-2.66804099838414e-08\\
0.904452226113057	-2.00103218118259e-08\\
0.906453226613307	-2.66804099838414e-08\\
0.908454227113557	-5.33608772634624e-08\\
0.910455227613807	-6.00310800270369e-08\\
0.912456228114057	-8.67114327150988e-08\\
0.914457228614307	-8.00410580641858e-08\\
0.916458229114557	-8.00410580641858e-08\\
0.918459229614807	-8.00410580641858e-08\\
0.920460230115058	-6.00310800270369e-08\\
0.922461230615308	-6.00310800270369e-08\\
0.924462231115558	-6.67008817201548e-08\\
0.926463231615808	-4.00206436236519e-08\\
0.928464232116058	-4.66907317956674e-08\\
0.930465232616308	-4.66907317956674e-08\\
0.932466233116558	-2.00103218118259e-08\\
0.934467233616808	-2.00103218118259e-08\\
0.936468234117058	-1.33402336398105e-08\\
0.938469234617309	-1.33402336398105e-08\\
0.940470235117559	-2.00103218118259e-08\\
0.942471235617809	-4.66907317956674e-08\\
0.944472236118059	-2.66804099838414e-08\\
0.946473236618309	-6.67008817201548e-09\\
0.948474237118559	6.67008817201548e-09\\
0.950475237618809	1.33402336398105e-08\\
0.95247623811906	1.33402336398105e-08\\
0.95447723861931	0\\
0.95647823911956	1.33402336398105e-08\\
0.95847923961981	6.67008817201548e-09\\
0.96048024012006	-4.00206436236519e-08\\
0.96248124062031	-4.66907317956674e-08\\
0.96448224112056	-6.00310800270369e-08\\
0.96648324162081	-7.33712563710678e-08\\
0.968484242121061	-8.00410580641858e-08\\
0.970485242621311	-8.00410580641858e-08\\
0.972486243121561	-8.00410580641858e-08\\
0.974487243621811	-1.0005160905913e-07\\
0.976488244122061	-8.67114327150988e-08\\
0.978489244622311	-7.33712563710678e-08\\
0.980490245122561	-5.33608772634624e-08\\
0.982491245622811	-5.33608772634624e-08\\
0.984492246123062	-7.33712563710678e-08\\
0.986493246623312	-7.33712563710678e-08\\
0.988494247123562	-6.00310800270369e-08\\
0.990495247623812	-3.33505554516364e-08\\
0.992496248124062	-4.66907317956674e-08\\
0.994497248624312	-6.67008817201548e-08\\
0.996498249124562	-7.33712563710678e-08\\
0.998499249624812	-5.33608772634624e-08\\
1.00050025012506	-2.66804099838414e-08\\
1.00250125062531	-2.00103218118259e-08\\
1.00450225112556	-1.33402336398105e-08\\
1.00650325162581	-2.00103218118259e-08\\
1.00850425212606	0\\
1.01050525262631	-2.00103218118259e-08\\
1.01250625312656	1.33402336398105e-08\\
1.01450725362681	2.66804099838414e-08\\
1.01650825412706	1.33402336398105e-08\\
1.01850925462731	2.66804099838414e-08\\
1.02051025512756	2.66804099838414e-08\\
1.02251125562781	3.33505554516364e-08\\
1.02451225612806	6.67008817201548e-08\\
1.02651325662831	2.66804099838414e-08\\
1.02851425712856	6.67008817201548e-09\\
1.03051525762881	-2.00103218118259e-08\\
1.03251625812906	1.33402336398105e-08\\
1.03451725862931	-2.00103218118259e-08\\
1.03651825912956	-6.67008817201548e-09\\
1.03851925962981	-6.67008817201548e-09\\
1.04052026013006	2.66804099838414e-08\\
1.04252126063032	1.33402336398105e-08\\
1.04452226113057	4.00206436236519e-08\\
1.04652326163082	4.00206436236519e-08\\
1.04852426213107	4.66907317956674e-08\\
1.05052526263132	2.66804099838414e-08\\
1.05252626313157	6.00310800270369e-08\\
1.05452726363182	2.66804099838414e-08\\
1.05652826413207	4.00206436236519e-08\\
1.05852926463232	4.00206436236519e-08\\
1.06053026513257	4.66907317956674e-08\\
1.06253126563282	8.00410580641858e-08\\
1.06453226613307	9.33812344082167e-08\\
1.06653326663332	1.26731961747192e-07\\
1.06853426713357	1.26731961747192e-07\\
1.07053526763382	1.13391785403161e-07\\
1.07253626813407	1.40072138091223e-07\\
1.07453726863432	1.33402336398105e-07\\
1.07653826913457	1.80092667123315e-07\\
1.07853926963482	1.53412314435254e-07\\
1.08054027013507	1.40072138091223e-07\\
1.08254127063532	1.40072138091223e-07\\
1.08454227113557	1.13391785403161e-07\\
1.08654327163582	9.33812344082167e-08\\
1.08854427213607	1.0005160905913e-07\\
1.09054527263632	8.67114327150988e-08\\
1.09254627313657	4.66907317956674e-08\\
1.09454727363682	7.33712563710678e-08\\
1.09654827413707	1.0005160905913e-07\\
1.09854927463732	1.13391785403161e-07\\
1.10055027513757	8.67114327150988e-08\\
1.10255127563782	9.33812344082167e-08\\
1.10455227613807	7.33712563710678e-08\\
1.10655327663832	7.33712563710678e-08\\
1.10855427713857	8.00410580641858e-08\\
1.11055527763882	6.00310800270369e-08\\
1.11255627813907	6.00310800270369e-08\\
1.11455727863932	6.67008817201548e-08\\
1.11655827913957	8.00410580641858e-08\\
1.11855927963982	7.33712563710678e-08\\
1.12056028014007	6.00310800270369e-08\\
1.12256128064032	8.00410580641858e-08\\
1.12456228114057	4.66907317956674e-08\\
1.12656328164082	4.00206436236519e-08\\
1.12856428214107	6.67008817201548e-08\\
1.13056528264132	5.33608772634624e-08\\
1.13256628314157	4.66907317956674e-08\\
1.13456728364182	4.00206436236519e-08\\
1.13656828414207	2.00103218118259e-08\\
1.13856928464232	6.00310800270369e-08\\
1.14057028514257	6.67008817201548e-08\\
1.14257128564282	5.33608772634624e-08\\
1.14457228614307	4.00206436236519e-08\\
1.14657328664332	2.66804099838414e-08\\
1.14857428714357	6.67008817201548e-09\\
1.15057528764382	6.67008817201548e-09\\
1.15257628814407	2.00103218118259e-08\\
1.15457728864432	4.00206436236519e-08\\
1.15657828914457	5.33608772634624e-08\\
1.15857928964482	4.66907317956674e-08\\
1.16058029014507	6.67008817201548e-08\\
1.16258129064532	1.13391785403161e-07\\
1.16458229114557	1.20062160054074e-07\\
1.16658329164582	9.33812344082167e-08\\
1.16858429214607	1.0005160905913e-07\\
1.17058529264632	8.00410580641858e-08\\
1.17258629314657	7.33712563710678e-08\\
1.17458729364682	8.67114327150988e-08\\
1.17658829414707	1.20062160054074e-07\\
1.17858929464732	1.40072138091223e-07\\
1.18059029514757	1.66752490779285e-07\\
1.18259129564782	1.80092667123315e-07\\
1.18459229614807	1.46742512742136e-07\\
1.18659329664832	1.60082689086167e-07\\
1.18859429714857	1.60082689086167e-07\\
1.19059529764882	1.80092667123315e-07\\
1.19259629814907	1.53412314435254e-07\\
1.19459729864932	1.40072138091223e-07\\
1.19659829914957	1.40072138091223e-07\\
1.19859929964982	1.40072138091223e-07\\
1.20060030015008	1.13391785403161e-07\\
1.20260130065033	1.20062160054074e-07\\
1.20460230115058	1.06721983710043e-07\\
1.20660330165083	1.20062160054074e-07\\
1.20860430215108	1.20062160054074e-07\\
1.21060530265133	1.13391785403161e-07\\
1.21260630315158	8.00410580641858e-08\\
1.21460730365183	9.33812344082167e-08\\
1.21660830415208	1.0005160905913e-07\\
1.21860930465233	8.67114327150988e-08\\
1.22061030515258	9.33812344082167e-08\\
1.22261130565283	1.0005160905913e-07\\
1.22461230615308	1.13391785403161e-07\\
1.22661330665333	9.33812344082167e-08\\
1.22861430715358	1.13391785403161e-07\\
1.23061530765383	1.20062160054074e-07\\
1.23261630815408	1.13391785403161e-07\\
1.23461730865433	9.33812344082167e-08\\
1.23661830915458	1.20062160054074e-07\\
1.23861930965483	1.53412314435254e-07\\
1.24062031015508	1.53412314435254e-07\\
1.24262131065533	1.33402336398105e-07\\
1.24462231115558	1.40072138091223e-07\\
1.24662331165583	1.60082689086167e-07\\
1.24862431215608	1.46742512742136e-07\\
1.25062531265633	1.53412314435254e-07\\
1.25262631315658	1.53412314435254e-07\\
1.25462731365683	1.60082689086167e-07\\
1.25662831415708	1.73422865430198e-07\\
1.25862931465733	1.86763041774229e-07\\
1.26063031515758	1.86763041774229e-07\\
1.26263131565783	1.80092667123315e-07\\
1.26463231615808	1.80092667123315e-07\\
1.26663331665833	1.80092667123315e-07\\
1.26863431715858	2.06773019811377e-07\\
1.27063531765883	2.0010321811826e-07\\
1.27263631815908	2.06773019811377e-07\\
1.27463731865933	2.1344339446229e-07\\
1.27663831915958	2.26783570806321e-07\\
1.27863931965983	2.33453945457234e-07\\
1.28064032016008	2.26783570806321e-07\\
1.28264132066033	2.33453945457234e-07\\
1.28464232116058	2.20113769113203e-07\\
1.28664332166083	2.1344339446229e-07\\
1.28864432216108	2.1344339446229e-07\\
1.29064532266133	2.06773019811377e-07\\
1.29264632316158	2.20113769113203e-07\\
1.29464732366183	2.26783570806321e-07\\
1.29664832416208	2.33453945457234e-07\\
1.29864932466233	2.06773019811377e-07\\
1.30065032516258	2.06773019811377e-07\\
1.30265132566283	2.06773019811377e-07\\
1.30465232616308	2.40123747150352e-07\\
1.30665332666333	2.1344339446229e-07\\
1.30865432716358	2.40123747150352e-07\\
1.31065532766383	2.20113769113203e-07\\
1.31265632816408	2.40123747150352e-07\\
1.31465732866433	2.40123747150352e-07\\
1.31665832916458	2.33453945457234e-07\\
1.31865932966483	2.20113769113203e-07\\
1.32066033016508	2.40123747150352e-07\\
1.32266133066533	2.40123747150352e-07\\
1.32466233116558	2.06773019811377e-07\\
1.32666333166583	2.26783570806321e-07\\
1.32866433216608	2.20113769113203e-07\\
1.33066533266633	2.0010321811826e-07\\
1.33266633316658	1.86763041774229e-07\\
1.33466733366683	1.73422865430198e-07\\
1.33666833416708	1.73422865430198e-07\\
1.33866933466733	2.06773019811377e-07\\
1.34067033516758	1.86763041774229e-07\\
1.34267133566783	2.06773019811377e-07\\
1.34467233616808	2.26783570806321e-07\\
1.34667333666833	2.26783570806321e-07\\
1.34867433716858	2.06773019811377e-07\\
1.35067533766883	1.93432843467346e-07\\
1.35267633816908	1.86763041774229e-07\\
1.35467733866933	1.86763041774229e-07\\
1.35667833916958	1.66752490779285e-07\\
1.35867933966983	1.66752490779285e-07\\
1.36068034017009	1.40072138091223e-07\\
1.36268134067034	1.53412314435254e-07\\
1.36468234117059	1.46742512742136e-07\\
1.36668334167084	1.40072138091223e-07\\
1.36868434217109	1.26731961747192e-07\\
1.37068534267134	1.33402336398105e-07\\
1.37268634317159	1.60082689086167e-07\\
1.37468734367184	1.66752490779285e-07\\
1.37668834417209	1.80092667123315e-07\\
1.37868934467234	1.86763041774229e-07\\
1.38069034517259	1.86763041774229e-07\\
1.38269134567284	1.86763041774229e-07\\
1.38469234617309	2.1344339446229e-07\\
1.38669334667334	1.66752490779285e-07\\
1.38869434717359	1.80092667123315e-07\\
1.39069534767384	1.86763041774229e-07\\
1.39269634817409	2.26783570806321e-07\\
1.39469734867434	2.26783570806321e-07\\
1.39669834917459	2.1344339446229e-07\\
1.39869934967484	2.06773019811377e-07\\
1.40070035017509	1.86763041774229e-07\\
1.40270135067534	2.0010321811826e-07\\
1.40470235117559	2.06773019811377e-07\\
1.40670335167584	2.06773019811377e-07\\
1.40870435217609	1.93432843467346e-07\\
1.41070535267634	1.66752490779285e-07\\
1.41270635317659	1.66752490779285e-07\\
1.41470735367684	1.46742512742136e-07\\
1.41670835417709	1.60082689086167e-07\\
1.41870935467734	1.46742512742136e-07\\
1.42071035517759	1.60082689086167e-07\\
1.42271135567784	1.80092667123315e-07\\
1.42471235617809	1.60082689086167e-07\\
1.42671335667834	1.66752490779285e-07\\
1.42871435717859	1.40072138091223e-07\\
1.43071535767884	1.20062160054074e-07\\
1.43271635817909	1.33402336398105e-07\\
1.43471735867934	1.20062160054074e-07\\
1.43671835917959	1.0005160905913e-07\\
1.43871935967984	1.06721983710043e-07\\
1.44072036018009	1.13391785403161e-07\\
1.44272136068034	1.06721983710043e-07\\
1.44472236118059	1.20062160054074e-07\\
1.44672336168084	1.40072138091223e-07\\
1.44872436218109	1.40072138091223e-07\\
1.45072536268134	1.26731961747192e-07\\
1.45272636318159	1.40072138091223e-07\\
1.45472736368184	1.20062160054074e-07\\
1.45672836418209	1.13391785403161e-07\\
1.45872936468234	1.26731961747192e-07\\
1.46073036518259	1.0005160905913e-07\\
1.46273136568284	9.33812344082167e-08\\
1.46473236618309	7.33712563710678e-08\\
1.46673336668334	5.33608772634624e-08\\
1.46873436718359	6.67008817201548e-08\\
1.47073536768384	4.66907317956674e-08\\
1.47273636818409	6.00310800270369e-08\\
1.47473736868434	5.33608772634624e-08\\
1.47673836918459	4.00206436236519e-08\\
1.47873936968484	2.66804099838414e-08\\
1.48074037018509	-6.67008817201548e-09\\
1.48274137068534	-1.33402336398105e-08\\
1.48474237118559	-6.67008817201548e-09\\
1.48674337168584	0\\
1.48874437218609	-2.00103218118259e-08\\
1.49074537268634	-2.00103218118259e-08\\
1.49274637318659	-4.00206436236519e-08\\
1.49474737368684	-2.00103218118259e-08\\
1.49674837418709	0\\
1.49874937468734	6.67008817201548e-09\\
1.50075037518759	-6.67008817201548e-09\\
1.50275137568784	1.33402336398105e-08\\
1.50475237618809	-6.67008817201548e-09\\
1.50675337668834	-2.00103218118259e-08\\
1.50875437718859	-4.00206436236519e-08\\
1.51075537768884	-3.33505554516364e-08\\
1.51275637818909	-5.33608772634624e-08\\
1.51475737868934	-8.00410580641858e-08\\
1.51675837918959	-6.67008817201548e-08\\
1.51875937968984	-5.33608772634624e-08\\
1.5207603801901	-3.33505554516364e-08\\
1.52276138069035	-3.33505554516364e-08\\
1.5247623811906	-4.66907317956674e-08\\
1.52676338169085	-3.33505554516364e-08\\
1.5287643821911	-3.33505554516364e-08\\
1.53076538269135	-3.33505554516364e-08\\
1.5327663831916	-2.00103218118259e-08\\
1.53476738369185	-4.66907317956674e-08\\
1.5367683841921	-3.33505554516364e-08\\
1.53876938469235	-5.33608772634624e-08\\
1.5407703851926	-8.00410580641858e-08\\
1.54277138569285	-8.00410580641858e-08\\
1.5447723861931	-8.00410580641858e-08\\
1.54677338669335	-9.33812344082167e-08\\
1.5487743871936	-1.06721983710043e-07\\
1.55077538769385	-1.33402336398105e-07\\
1.5527763881941	-1.46742512742136e-07\\
1.55477738869435	-1.40072138091223e-07\\
1.5567783891946	-1.40072138091223e-07\\
1.55877938969485	-1.60082689086167e-07\\
1.5607803901951	-1.66752490779285e-07\\
1.56278139069535	-1.80092667123315e-07\\
1.5647823911956	-1.86763041774229e-07\\
1.56678339169585	-1.93432843467346e-07\\
1.5687843921961	-2.0010321811826e-07\\
1.57078539269635	-1.86763041774229e-07\\
1.5727863931966	-2.1344339446229e-07\\
1.57478739369685	-2.26783570806321e-07\\
1.5767883941971	-2.26783570806321e-07\\
1.57878939469735	-2.40123747150352e-07\\
1.5807903951976	-2.60134298145296e-07\\
1.58279139569785	-2.46794121801265e-07\\
1.5847923961981	-2.60134298145296e-07\\
1.58679339669835	-2.60134298145296e-07\\
1.5887943971986	-2.26783570806321e-07\\
1.59079539769885	-2.26783570806321e-07\\
1.5927963981991	-2.46794121801265e-07\\
1.59479739869935	-2.26783570806321e-07\\
1.5967983991996	-2.1344339446229e-07\\
1.59879939969985	-1.86763041774229e-07\\
1.6008004002001	-1.73422865430198e-07\\
1.60280140070035	-1.53412314435254e-07\\
1.6048024012006	-1.26731961747192e-07\\
1.60680340170085	-1.46742512742136e-07\\
1.6088044022011	-1.33402336398105e-07\\
1.61080540270135	-1.33402336398105e-07\\
1.6128064032016	-1.26731961747192e-07\\
1.61480740370185	-1.26731961747192e-07\\
1.6168084042021	-9.33812344082167e-08\\
1.61880940470235	-8.67114327150988e-08\\
1.6208104052026	-7.33712563710678e-08\\
1.62281140570285	-6.67008817201548e-08\\
1.6248124062031	-8.00410580641858e-08\\
1.62681340670335	-6.67008817201548e-08\\
1.6288144072036	-8.67114327150988e-08\\
1.63081540770385	-8.67114327150988e-08\\
1.6328164082041	-7.33712563710678e-08\\
1.63481740870435	-8.00410580641858e-08\\
1.6368184092046	-8.00410580641858e-08\\
1.63881940970485	-9.33812344082167e-08\\
1.6408204102051	-1.0005160905913e-07\\
1.64282141070535	-8.67114327150988e-08\\
1.6448224112056	-9.33812344082167e-08\\
1.64682341170585	-6.00310800270369e-08\\
1.6488244122061	-5.33608772634624e-08\\
1.65082541270635	-5.33608772634624e-08\\
1.6528264132066	-4.66907317956674e-08\\
1.65482741370685	-6.67008817201548e-08\\
1.6568284142071	-4.66907317956674e-08\\
1.65882941470735	-4.66907317956674e-08\\
1.6608304152076	-3.33505554516364e-08\\
1.66283141570785	-1.33402336398105e-08\\
1.6648324162081	0\\
1.66683341670835	-6.67008817201548e-09\\
1.6688344172086	-2.00103218118259e-08\\
1.67083541770885	-2.66804099838414e-08\\
1.6728364182091	-2.00103218118259e-08\\
1.67483741870935	-2.00103218118259e-08\\
1.6768384192096	-2.00103218118259e-08\\
1.67883941970985	-2.00103218118259e-08\\
1.68084042021011	-2.00103218118259e-08\\
1.68284142071036	-2.66804099838414e-08\\
1.68484242121061	-5.33608772634624e-08\\
1.68684342171086	-4.00206436236519e-08\\
1.68884442221111	-4.00206436236519e-08\\
1.69084542271136	-5.33608772634624e-08\\
1.69284642321161	-4.66907317956674e-08\\
1.69484742371186	-2.00103218118259e-08\\
1.69684842421211	0\\
1.69884942471236	2.66804099838414e-08\\
1.70085042521261	5.33608772634624e-08\\
1.70285142571286	3.33505554516364e-08\\
1.70485242621311	5.33608772634624e-08\\
1.70685342671336	5.33608772634624e-08\\
1.70885442721361	6.67008817201548e-08\\
1.71085542771386	6.00310800270369e-08\\
1.71285642821411	7.33712563710678e-08\\
1.71485742871436	1.0005160905913e-07\\
1.71685842921461	9.33812344082167e-08\\
1.71885942971486	7.33712563710678e-08\\
1.72086043021511	7.33712563710678e-08\\
1.72286143071536	9.33812344082167e-08\\
1.72486243121561	6.67008817201548e-08\\
1.72686343171586	8.67114327150988e-08\\
1.72886443221611	8.67114327150988e-08\\
1.73086543271636	6.00310800270369e-08\\
1.73286643321661	8.00410580641858e-08\\
1.73486743371686	4.66907317956674e-08\\
1.73686843421711	5.33608772634624e-08\\
1.73886943471736	5.33608772634624e-08\\
1.74087043521761	4.00206436236519e-08\\
1.74287143571786	2.00103218118259e-08\\
1.74487243621811	4.00206436236519e-08\\
1.74687343671836	2.00103218118259e-08\\
1.74887443721861	6.67008817201548e-09\\
1.75087543771886	3.33505554516364e-08\\
1.75287643821911	2.66804099838414e-08\\
1.75487743871936	-1.33402336398105e-08\\
1.75687843921961	-3.33505554516364e-08\\
1.75887943971986	-2.00103218118259e-08\\
1.76088044022011	-4.00206436236519e-08\\
1.76288144072036	-2.00103218118259e-08\\
1.76488244122061	-2.66804099838414e-08\\
1.76688344172086	-2.00103218118259e-08\\
1.76888444222111	-2.00103218118259e-08\\
1.77088544272136	0\\
1.77288644322161	0\\
1.77488744372186	-2.00103218118259e-08\\
1.77688844422211	-2.00103218118259e-08\\
1.77888944472236	-2.66804099838414e-08\\
1.78089044522261	-3.33505554516364e-08\\
1.78289144572286	-4.66907317956674e-08\\
1.78489244622311	-3.33505554516364e-08\\
1.78689344672336	-5.33608772634624e-08\\
1.78889444722361	-3.33505554516364e-08\\
1.79089544772386	-5.33608772634624e-08\\
1.79289644822411	-4.66907317956674e-08\\
1.79489744872436	-6.67008817201548e-08\\
1.79689844922461	-5.33608772634624e-08\\
1.79889944972486	-6.00310800270369e-08\\
1.80090045022511	-6.00310800270369e-08\\
1.80290145072536	-4.00206436236519e-08\\
1.80490245122561	-2.00103218118259e-08\\
1.80690345172586	-2.00103218118259e-08\\
1.80890445222611	-2.66804099838414e-08\\
1.81090545272636	-5.33608772634624e-08\\
1.81290645322661	-4.66907317956674e-08\\
1.81490745372686	-4.66907317956674e-08\\
1.81690845422711	-4.66907317956674e-08\\
1.81890945472736	-5.33608772634624e-08\\
1.82091045522761	-2.00103218118259e-08\\
1.82291145572786	-1.33402336398105e-08\\
1.82491245622811	-4.00206436236519e-08\\
1.82691345672836	-3.33505554516364e-08\\
1.82891445722861	-4.00206436236519e-08\\
1.83091545772886	-3.33505554516364e-08\\
1.83291645822911	0\\
1.83491745872936	-1.33402336398105e-08\\
1.83691845922961	0\\
1.83891945972987	-2.66804099838414e-08\\
1.84092046023012	-4.00206436236519e-08\\
1.84292146073037	-2.00103218118259e-08\\
1.84492246123062	-6.67008817201548e-09\\
1.84692346173087	0\\
1.84892446223112	-2.00103218118259e-08\\
1.85092546273137	-6.67008817201548e-09\\
1.85292646323162	-1.33402336398105e-08\\
1.85492746373187	1.33402336398105e-08\\
1.85692846423212	-2.00103218118259e-08\\
1.85892946473237	-2.00103218118259e-08\\
1.86093046523262	-2.66804099838414e-08\\
1.86293146573287	-2.00103218118259e-08\\
1.86493246623312	-2.66804099838414e-08\\
1.86693346673337	-3.33505554516364e-08\\
1.86893446723362	-2.00103218118259e-08\\
1.87093546773387	-2.66804099838414e-08\\
1.87293646823412	-4.66907317956674e-08\\
1.87493746873437	-6.00310800270369e-08\\
1.87693846923462	-4.00206436236519e-08\\
1.87893946973487	-2.00103218118259e-08\\
1.88094047023512	-2.66804099838414e-08\\
1.88294147073537	0\\
1.88494247123562	-6.67008817201548e-09\\
1.88694347173587	-1.33402336398105e-08\\
1.88894447223612	-4.00206436236519e-08\\
1.89094547273637	-5.33608772634624e-08\\
1.89294647323662	-6.67008817201548e-08\\
1.89494747373687	-4.00206436236519e-08\\
1.89694847423712	-4.00206436236519e-08\\
1.89894947473737	-4.00206436236519e-08\\
1.90095047523762	-1.33402336398105e-08\\
1.90295147573787	2.00103218118259e-08\\
1.90495247623812	-6.67008817201548e-09\\
1.90695347673837	-1.33402336398105e-08\\
1.90895447723862	-2.00103218118259e-08\\
1.91095547773887	-2.00103218118259e-08\\
1.91295647823912	-3.33505554516364e-08\\
1.91495747873937	-6.67008817201548e-08\\
1.91695847923962	-7.33712563710678e-08\\
1.91895947973987	-6.00310800270369e-08\\
1.92096048024012	-8.67114327150988e-08\\
1.92296148074037	-8.00410580641858e-08\\
1.92496248124062	-8.67114327150988e-08\\
1.92696348174087	-8.00410580641858e-08\\
1.92896448224112	-8.00410580641858e-08\\
1.93096548274137	-7.33712563710678e-08\\
1.93296648324162	-8.00410580641858e-08\\
1.93496748374187	-6.67008817201548e-08\\
1.93696848424212	-3.33505554516364e-08\\
1.93896948474237	-5.33608772634624e-08\\
1.94097048524262	-8.00410580641858e-08\\
1.94297148574287	-8.00410580641858e-08\\
1.94497248624312	-8.00410580641858e-08\\
1.94697348674337	-8.67114327150988e-08\\
1.94897448724362	-8.00410580641858e-08\\
1.95097548774387	-6.67008817201548e-08\\
1.95297648824412	-6.67008817201548e-08\\
1.95497748874437	-4.66907317956674e-08\\
1.95697848924462	-4.66907317956674e-08\\
1.95897948974487	-6.67008817201548e-08\\
1.96098049024512	-4.66907317956674e-08\\
1.96298149074537	-4.00206436236519e-08\\
1.96498249124562	-3.33505554516364e-08\\
1.96698349174587	-4.66907317956674e-08\\
1.96898449224612	-6.67008817201548e-08\\
1.97098549274637	-8.00410580641858e-08\\
1.97298649324662	-8.00410580641858e-08\\
1.97498749374687	-6.00310800270369e-08\\
1.97698849424712	-8.00410580641858e-08\\
1.97898949474737	-8.00410580641858e-08\\
1.98099049524762	-7.33712563710678e-08\\
1.98299149574787	-9.33812344082167e-08\\
1.98499249624812	-6.00310800270369e-08\\
1.98699349674837	-5.33608772634624e-08\\
1.98899449724862	-4.66907317956674e-08\\
1.99099549774887	-7.33712563710678e-08\\
1.99299649824912	-7.33712563710678e-08\\
1.99499749874937	-7.33712563710678e-08\\
1.99699849924962	-6.00310800270369e-08\\
1.99899949974988	-6.00310800270369e-08\\
2.00100050025013	-4.00206436236519e-08\\
2.00300150075038	-2.00103218118259e-08\\
2.00500250125063	-6.67008817201548e-09\\
2.00700350175088	-1.33402336398105e-08\\
2.00900450225113	-6.67008817201548e-09\\
2.01100550275138	0\\
2.01300650325163	0\\
2.01500750375188	0\\
2.01700850425213	-2.00103218118259e-08\\
2.01900950475238	-3.33505554516364e-08\\
2.02101050525263	-2.00103218118259e-08\\
2.02301150575288	-1.33402336398105e-08\\
2.02501250625313	-1.33402336398105e-08\\
2.02701350675338	-1.33402336398105e-08\\
2.02901450725363	-6.67008817201548e-09\\
2.03101550775388	-1.33402336398105e-08\\
2.03301650825413	0\\
2.03501750875438	-1.33402336398105e-08\\
2.03701850925463	-3.33505554516364e-08\\
2.03901950975488	-3.33505554516364e-08\\
2.04102051025513	-7.33712563710678e-08\\
2.04302151075538	-6.00310800270369e-08\\
2.04502251125563	-6.00310800270369e-08\\
2.04702351175588	-6.67008817201548e-08\\
2.04902451225613	-4.66907317956674e-08\\
2.05102551275638	-2.00103218118259e-08\\
2.05302651325663	-1.33402336398105e-08\\
2.05502751375688	-1.33402336398105e-08\\
2.05702851425713	-4.00206436236519e-08\\
2.05902951475738	-3.33505554516364e-08\\
2.06103051525763	-2.00103218118259e-08\\
2.06303151575788	-2.00103218118259e-08\\
2.06503251625813	-5.33608772634624e-08\\
2.06703351675838	-2.00103218118259e-08\\
2.06903451725863	1.33402336398105e-08\\
2.07103551775888	-6.67008817201548e-09\\
2.07303651825913	1.33402336398105e-08\\
2.07503751875938	0\\
2.07703851925963	0\\
2.07903951975988	-2.00103218118259e-08\\
2.08104052026013	-1.33402336398105e-08\\
2.08304152076038	-1.33402336398105e-08\\
2.08504252126063	-2.66804099838414e-08\\
2.08704352176088	-4.66907317956674e-08\\
2.08904452226113	-2.66804099838414e-08\\
2.09104552276138	-2.00103218118259e-08\\
2.09304652326163	-1.33402336398105e-08\\
2.09504752376188	1.33402336398105e-08\\
2.09704852426213	1.33402336398105e-08\\
2.09904952476238	1.33402336398105e-08\\
2.10105052526263	4.00206436236519e-08\\
2.10305152576288	3.33505554516364e-08\\
2.10505252626313	4.00206436236519e-08\\
2.10705352676338	3.33505554516364e-08\\
2.10905452726363	2.66804099838414e-08\\
2.11105552776388	2.66804099838414e-08\\
2.11305652826413	1.33402336398105e-08\\
2.11505752876438	0\\
2.11705852926463	-2.00103218118259e-08\\
2.11905952976488	-3.33505554516364e-08\\
2.12106053026513	-2.00103218118259e-08\\
2.12306153076538	-5.33608772634624e-08\\
2.12506253126563	-4.66907317956674e-08\\
2.12706353176588	-4.00206436236519e-08\\
2.12906453226613	-1.33402336398105e-08\\
2.13106553276638	0\\
2.13306653326663	2.66804099838414e-08\\
2.13506753376688	4.00206436236519e-08\\
2.13706853426713	6.67008817201548e-08\\
2.13906953476738	6.67008817201548e-08\\
2.14107053526763	4.66907317956674e-08\\
2.14307153576788	4.66907317956674e-08\\
2.14507253626813	2.00103218118259e-08\\
2.14707353676838	0\\
2.14907453726863	-6.67008817201548e-09\\
2.15107553776888	4.00206436236519e-08\\
2.15307653826913	4.00206436236519e-08\\
2.15507753876938	3.33505554516364e-08\\
2.15707853926963	3.33505554516364e-08\\
2.15907953976988	2.66804099838414e-08\\
2.16108054027013	1.33402336398105e-08\\
2.16308154077039	-1.33402336398105e-08\\
2.16508254127064	-2.66804099838414e-08\\
2.16708354177089	-1.33402336398105e-08\\
2.16908454227114	-2.66804099838414e-08\\
2.17108554277139	-1.33402336398105e-08\\
2.17308654327164	-4.66907317956674e-08\\
2.17508754377189	-3.33505554516364e-08\\
2.17708854427214	-4.00206436236519e-08\\
2.17908954477239	-6.00310800270369e-08\\
2.18109054527264	-4.66907317956674e-08\\
2.18309154577289	-4.66907317956674e-08\\
2.18509254627314	-6.00310800270369e-08\\
2.18709354677339	-5.33608772634624e-08\\
2.18909454727364	-8.00410580641858e-08\\
2.19109554777389	-8.00410580641858e-08\\
2.19309654827414	-2.66804099838414e-08\\
2.19509754877439	6.67008817201548e-09\\
2.19709854927464	2.00103218118259e-08\\
2.19909954977489	1.33402336398105e-08\\
2.20110055027514	2.66804099838414e-08\\
2.20310155077539	6.67008817201548e-08\\
2.20510255127564	6.67008817201548e-08\\
2.20710355177589	7.33712563710678e-08\\
2.20910455227614	7.33712563710678e-08\\
2.21110555277639	6.67008817201548e-08\\
2.21310655327664	9.33812344082167e-08\\
2.21510755377689	9.33812344082167e-08\\
2.21710855427714	9.33812344082167e-08\\
2.21910955477739	8.67114327150988e-08\\
2.22111055527764	1.0005160905913e-07\\
2.22311155577789	8.67114327150988e-08\\
2.22511255627814	1.0005160905913e-07\\
2.22711355677839	6.00310800270369e-08\\
2.22911455727864	7.33712563710678e-08\\
2.23111555777889	8.67114327150988e-08\\
2.23311655827914	8.00410580641858e-08\\
2.23511755877939	4.66907317956674e-08\\
2.23711855927964	6.67008817201548e-08\\
2.23911955977989	7.33712563710678e-08\\
2.24112056028014	6.00310800270369e-08\\
2.24312156078039	2.00103218118259e-08\\
2.24512256128064	6.67008817201548e-08\\
2.24712356178089	4.66907317956674e-08\\
2.24912456228114	4.66907317956674e-08\\
2.25112556278139	6.00310800270369e-08\\
2.25312656328164	6.67008817201548e-08\\
2.25512756378189	8.00410580641858e-08\\
2.25712856428214	8.67114327150988e-08\\
2.25912956478239	8.67114327150988e-08\\
2.26113056528264	9.33812344082167e-08\\
2.26313156578289	9.33812344082167e-08\\
2.26513256628314	9.33812344082167e-08\\
2.26713356678339	9.33812344082167e-08\\
2.26913456728364	9.33812344082167e-08\\
2.27113556778389	7.33712563710678e-08\\
2.27313656828414	6.67008817201548e-08\\
2.27513756878439	8.00410580641858e-08\\
2.27713856928464	8.00410580641858e-08\\
2.27913956978489	7.33712563710678e-08\\
2.28114057028514	9.33812344082167e-08\\
2.28314157078539	8.67114327150988e-08\\
2.28514257128564	6.67008817201548e-08\\
2.28714357178589	2.00103218118259e-08\\
2.28914457228614	-6.67008817201548e-09\\
2.29114557278639	6.67008817201548e-09\\
2.29314657328664	0\\
2.29514757378689	-1.33402336398105e-08\\
2.29714857428714	-6.67008817201548e-09\\
2.29914957478739	2.66804099838414e-08\\
2.30115057528764	2.00103218118259e-08\\
2.30315157578789	5.33608772634624e-08\\
2.30515257628814	3.33505554516364e-08\\
2.30715357678839	2.00103218118259e-08\\
2.30915457728864	-6.67008817201548e-09\\
2.31115557778889	2.00103218118259e-08\\
2.31315657828914	5.33608772634624e-08\\
2.31515757878939	8.00410580641858e-08\\
2.31715857928964	9.33812344082167e-08\\
2.31915957978989	9.33812344082167e-08\\
2.32116058029015	8.67114327150988e-08\\
2.3231615807904	7.33712563710678e-08\\
2.32516258129065	3.33505554516364e-08\\
2.3271635817909	4.00206436236519e-08\\
2.32916458229115	5.33608772634624e-08\\
2.3311655827914	3.33505554516364e-08\\
2.33316658329165	6.67008817201548e-08\\
2.3351675837919	6.67008817201548e-08\\
2.33716858429215	7.33712563710678e-08\\
2.3391695847924	6.67008817201548e-08\\
2.34117058529265	4.66907317956674e-08\\
2.3431715857929	2.00103218118259e-08\\
2.34517258629315	4.66907317956674e-08\\
2.3471735867934	6.67008817201548e-08\\
2.34917458729365	8.67114327150988e-08\\
2.3511755877939	1.0005160905913e-07\\
2.35317658829415	1.13391785403161e-07\\
2.3551775887944	8.67114327150988e-08\\
2.35717858929465	9.33812344082167e-08\\
2.3591795897949	1.20062160054074e-07\\
2.36118059029515	1.46742512742136e-07\\
2.3631815907954	1.46742512742136e-07\\
2.36518259129565	1.53412314435254e-07\\
2.3671835917959	1.80092667123315e-07\\
2.36918459229615	1.60082689086167e-07\\
2.3711855927964	1.73422865430198e-07\\
2.37318659329665	1.40072138091223e-07\\
2.3751875937969	1.33402336398105e-07\\
2.37718859429715	1.40072138091223e-07\\
2.3791895947974	1.53412314435254e-07\\
2.38119059529765	1.40072138091223e-07\\
2.3831915957979	1.60082689086167e-07\\
2.38519259629815	2.0010321811826e-07\\
2.3871935967984	1.60082689086167e-07\\
2.38919459729865	1.80092667123315e-07\\
2.3911955977989	2.06773019811377e-07\\
2.39319659829915	2.0010321811826e-07\\
2.3951975987994	2.06773019811377e-07\\
2.39719859929965	2.06773019811377e-07\\
2.3991995997999	1.86763041774229e-07\\
2.40120060030015	1.73422865430198e-07\\
2.4032016008004	1.66752490779285e-07\\
2.40520260130065	1.46742512742136e-07\\
2.4072036018009	1.40072138091223e-07\\
2.40920460230115	1.26731961747192e-07\\
2.4112056028014	1.0005160905913e-07\\
2.41320660330165	1.26731961747192e-07\\
2.4152076038019	1.40072138091223e-07\\
2.41720860430215	1.13391785403161e-07\\
2.4192096048024	1.0005160905913e-07\\
2.42121060530265	9.33812344082167e-08\\
2.4232116058029	1.0005160905913e-07\\
2.42521260630315	9.33812344082167e-08\\
2.4272136068034	1.0005160905913e-07\\
2.42921460730365	9.33812344082167e-08\\
2.4312156078039	1.06721983710043e-07\\
2.43321660830415	1.0005160905913e-07\\
2.4352176088044	1.0005160905913e-07\\
2.43721860930465	9.33812344082167e-08\\
2.4392196098049	9.33812344082167e-08\\
2.44122061030515	1.0005160905913e-07\\
2.4432216108054	1.20062160054074e-07\\
2.44522261130565	1.26731961747192e-07\\
2.4472236118059	1.53412314435254e-07\\
2.44922461230615	1.33402336398105e-07\\
2.4512256128064	1.40072138091223e-07\\
2.45322661330665	1.26731961747192e-07\\
2.4552276138069	1.13391785403161e-07\\
2.45722861430715	1.0005160905913e-07\\
2.4592296148074	1.0005160905913e-07\\
2.46123061530765	1.06721983710043e-07\\
2.4632316158079	1.26731961747192e-07\\
2.46523261630815	1.26731961747192e-07\\
2.4672336168084	1.0005160905913e-07\\
2.46923461730865	9.33812344082167e-08\\
2.4712356178089	1.06721983710043e-07\\
2.47323661830915	1.13391785403161e-07\\
2.4752376188094	8.67114327150988e-08\\
2.47723861930965	5.33608772634624e-08\\
2.4792396198099	8.00410580641858e-08\\
2.48124062031015	9.33812344082167e-08\\
2.48324162081041	6.00310800270369e-08\\
2.48524262131066	4.66907317956674e-08\\
2.48724362181091	4.00206436236519e-08\\
2.48924462231116	2.00103218118259e-08\\
2.49124562281141	4.00206436236519e-08\\
2.49324662331166	1.33402336398105e-08\\
2.49524762381191	1.33402336398105e-08\\
2.49724862431216	0\\
2.49924962481241	-2.00103218118259e-08\\
2.50125062531266	-2.00103218118259e-08\\
2.50325162581291	-2.00103218118259e-08\\
2.50525262631316	0\\
2.50725362681341	0\\
2.50925462731366	2.00103218118259e-08\\
2.51125562781391	6.67008817201548e-09\\
2.51325662831416	0\\
2.51525762881441	-6.67008817201548e-09\\
2.51725862931466	1.33402336398105e-08\\
2.51925962981491	-1.33402336398105e-08\\
2.52126063031516	-2.00103218118259e-08\\
2.52326163081541	-1.33402336398105e-08\\
2.52526263131566	-2.00103218118259e-08\\
2.52726363181591	-1.33402336398105e-08\\
2.52926463231616	-2.00103218118259e-08\\
2.53126563281641	-2.00103218118259e-08\\
2.53326663331666	0\\
2.53526763381691	-6.67008817201548e-09\\
2.53726863431716	-2.00103218118259e-08\\
2.53926963481741	-1.33402336398105e-08\\
2.54127063531766	-3.33505554516364e-08\\
2.54327163581791	-3.33505554516364e-08\\
2.54527263631816	-6.67008817201548e-08\\
2.54727363681841	-5.33608772634624e-08\\
2.54927463731866	-6.67008817201548e-08\\
2.55127563781891	-6.00310800270369e-08\\
2.55327663831916	-8.00410580641858e-08\\
2.55527763881941	-9.33812344082167e-08\\
2.55727863931966	-1.0005160905913e-07\\
2.55927963981991	-1.0005160905913e-07\\
2.56128064032016	-1.13391785403161e-07\\
2.56328164082041	-1.0005160905913e-07\\
2.56528264132066	-1.06721983710043e-07\\
2.56728364182091	-9.33812344082167e-08\\
2.56928464232116	-8.00410580641858e-08\\
2.57128564282141	-6.00310800270369e-08\\
2.57328664332166	-4.00206436236519e-08\\
2.57528764382191	-8.00410580641858e-08\\
2.57728864432216	-4.66907317956674e-08\\
2.57928964482241	-4.66907317956674e-08\\
2.58129064532266	-4.00206436236519e-08\\
2.58329164582291	-4.66907317956674e-08\\
2.58529264632316	-6.67008817201548e-08\\
2.58729364682341	-4.00206436236519e-08\\
2.58929464732366	-4.66907317956674e-08\\
2.59129564782391	-6.00310800270369e-08\\
2.59329664832416	-4.66907317956674e-08\\
2.59529764882441	-7.33712563710678e-08\\
2.59729864932466	-5.33608772634624e-08\\
2.59929964982491	-2.00103218118259e-08\\
2.60130065032516	-3.33505554516364e-08\\
2.60330165082541	-2.00103218118259e-08\\
2.60530265132566	-1.33402336398105e-08\\
2.60730365182591	-4.00206436236519e-08\\
2.60930465232616	-2.66804099838414e-08\\
2.61130565282641	-3.33505554516364e-08\\
2.61330665332666	-1.33402336398105e-08\\
2.61530765382691	0\\
2.61730865432716	0\\
2.61930965482741	2.00103218118259e-08\\
2.62131065532766	-2.00103218118259e-08\\
2.62331165582791	-1.33402336398105e-08\\
2.62531265632816	-2.00103218118259e-08\\
2.62731365682841	-1.33402336398105e-08\\
2.62931465732866	-6.67008817201548e-09\\
2.63131565782891	-6.67008817201548e-09\\
2.63331665832916	-1.33402336398105e-08\\
2.63531765882941	-3.33505554516364e-08\\
2.63731865932966	-2.00103218118259e-08\\
2.63931965982992	-4.00206436236519e-08\\
2.64132066033017	-1.33402336398105e-08\\
2.64332166083042	-1.33402336398105e-08\\
2.64532266133067	-6.00310800270369e-08\\
2.64732366183092	-3.33505554516364e-08\\
2.64932466233117	-3.33505554516364e-08\\
2.65132566283142	-2.00103218118259e-08\\
2.65332666333167	-1.33402336398105e-08\\
2.65532766383192	-2.00103218118259e-08\\
2.65732866433217	-3.33505554516364e-08\\
2.65932966483242	-5.33608772634624e-08\\
2.66133066533267	-6.67008817201548e-08\\
2.66333166583292	-1.0005160905913e-07\\
2.66533266633317	-8.67114327150988e-08\\
2.66733366683342	-8.00410580641858e-08\\
2.66933466733367	-1.0005160905913e-07\\
2.67133566783392	-1.06721983710043e-07\\
2.67333666833417	-1.13391785403161e-07\\
2.67533766883442	-1.06721983710043e-07\\
2.67733866933467	-1.26731961747192e-07\\
2.67933966983492	-1.13391785403161e-07\\
2.68134067033517	-8.67114327150988e-08\\
2.68334167083542	-8.00410580641858e-08\\
2.68534267133567	-1.0005160905913e-07\\
2.68734367183592	-1.13391785403161e-07\\
2.68934467233617	-1.26731961747192e-07\\
2.69134567283642	-1.13391785403161e-07\\
2.69334667333667	-9.33812344082167e-08\\
2.69534767383692	-1.06721983710043e-07\\
2.69734867433717	-6.00310800270369e-08\\
2.69934967483742	-6.67008817201548e-08\\
2.70135067533767	-7.33712563710678e-08\\
2.70335167583792	-8.67114327150988e-08\\
2.70535267633817	-9.33812344082167e-08\\
2.70735367683842	-9.33812344082167e-08\\
2.70935467733867	-9.33812344082167e-08\\
2.71135567783892	-1.0005160905913e-07\\
2.71335667833917	-6.67008817201548e-08\\
2.71535767883942	-5.33608772634624e-08\\
2.71735867933967	-8.67114327150988e-08\\
2.71935967983992	-7.33712563710678e-08\\
2.72136068034017	-6.67008817201548e-08\\
2.72336168084042	-7.33712563710678e-08\\
2.72536268134067	-4.66907317956674e-08\\
2.72736368184092	-6.00310800270369e-08\\
2.72936468234117	-7.33712563710678e-08\\
2.73136568284142	-1.13391785403161e-07\\
2.73336668334167	-9.33812344082167e-08\\
2.73536768384192	-9.33812344082167e-08\\
2.73736868434217	-1.0005160905913e-07\\
2.73936968484242	-1.20062160054074e-07\\
2.74137068534267	-1.06721983710043e-07\\
2.74337168584292	-1.20062160054074e-07\\
2.74537268634317	-1.26731961747192e-07\\
2.74737368684342	-1.26731961747192e-07\\
2.74937468734367	-1.33402336398105e-07\\
2.75137568784392	-1.33402336398105e-07\\
2.75337668834417	-1.46742512742136e-07\\
2.75537768884442	-1.40072138091223e-07\\
2.75737868934467	-1.46742512742136e-07\\
2.75937968984492	-1.60082689086167e-07\\
2.76138069034517	-1.40072138091223e-07\\
2.76338169084542	-1.26731961747192e-07\\
2.76538269134567	-9.33812344082167e-08\\
2.76738369184592	-1.26731961747192e-07\\
2.76938469234617	-1.13391785403161e-07\\
2.77138569284642	-1.20062160054074e-07\\
2.77338669334667	-1.26731961747192e-07\\
2.77538769384692	-1.06721983710043e-07\\
2.77738869434717	-1.26731961747192e-07\\
2.77938969484742	-9.33812344082167e-08\\
2.78139069534767	-8.67114327150988e-08\\
2.78339169584792	-7.33712563710678e-08\\
2.78539269634817	-5.33608772634624e-08\\
2.78739369684842	-4.66907317956674e-08\\
2.78939469734867	-3.33505554516364e-08\\
2.79139569784892	-2.00103218118259e-08\\
2.79339669834917	-2.00103218118259e-08\\
2.79539769884942	-6.67008817201548e-09\\
2.79739869934967	-2.00103218118259e-08\\
2.79939969984992	-1.33402336398105e-08\\
2.80140070035018	-2.00103218118259e-08\\
2.80340170085043	1.33402336398105e-08\\
2.80540270135068	2.66804099838414e-08\\
2.80740370185093	4.00206436236519e-08\\
2.80940470235118	5.33608772634624e-08\\
2.81140570285143	4.00206436236519e-08\\
2.81340670335168	5.33608772634624e-08\\
2.81540770385193	4.66907317956674e-08\\
2.81740870435218	4.00206436236519e-08\\
2.81940970485243	4.00206436236519e-08\\
2.82141070535268	5.33608772634624e-08\\
2.82341170585293	4.00206436236519e-08\\
2.82541270635318	0\\
2.82741370685343	6.67008817201548e-09\\
2.82941470735368	1.33402336398105e-08\\
2.83141570785393	1.33402336398105e-08\\
2.83341670835418	3.33505554516364e-08\\
2.83541770885443	4.00206436236519e-08\\
2.83741870935468	3.33505554516364e-08\\
2.83941970985493	4.00206436236519e-08\\
2.84142071035518	6.00310800270369e-08\\
2.84342171085543	8.00410580641858e-08\\
2.84542271135568	5.33608772634624e-08\\
2.84742371185593	6.67008817201548e-08\\
2.84942471235618	7.33712563710678e-08\\
2.85142571285643	7.33712563710678e-08\\
2.85342671335668	4.66907317956674e-08\\
2.85542771385693	6.00310800270369e-08\\
2.85742871435718	6.67008817201548e-08\\
2.85942971485743	6.00310800270369e-08\\
2.86143071535768	9.33812344082167e-08\\
2.86343171585793	8.00410580641858e-08\\
2.86543271635818	6.67008817201548e-08\\
2.86743371685843	9.33812344082167e-08\\
2.86943471735868	8.67114327150988e-08\\
2.87143571785893	8.00410580641858e-08\\
2.87343671835918	1.13391785403161e-07\\
2.87543771885943	9.33812344082167e-08\\
2.87743871935968	6.67008817201548e-08\\
2.87943971985993	6.00310800270369e-08\\
2.88144072036018	6.00310800270369e-08\\
2.88344172086043	1.06721983710043e-07\\
2.88544272136068	1.06721983710043e-07\\
2.88744372186093	1.06721983710043e-07\\
2.88944472236118	1.26731961747192e-07\\
2.89144572286143	1.13391785403161e-07\\
2.89344672336168	1.06721983710043e-07\\
2.89544772386193	9.33812344082167e-08\\
2.89744872436218	9.33812344082167e-08\\
2.89944972486243	1.0005160905913e-07\\
2.90145072536268	1.20062160054074e-07\\
2.90345172586293	9.33812344082167e-08\\
2.90545272636318	1.06721983710043e-07\\
2.90745372686343	1.0005160905913e-07\\
2.90945472736368	1.13391785403161e-07\\
2.91145572786393	9.33812344082167e-08\\
2.91345672836418	1.0005160905913e-07\\
2.91545772886443	1.26731961747192e-07\\
2.91745872936468	1.0005160905913e-07\\
2.91945972986493	1.06721983710043e-07\\
2.92146073036518	1.20062160054074e-07\\
2.92346173086543	1.53412314435254e-07\\
2.92546273136568	1.66752490779285e-07\\
2.92746373186593	1.40072138091223e-07\\
2.92946473236618	1.33402336398105e-07\\
2.93146573286643	1.40072138091223e-07\\
2.93346673336668	1.20062160054074e-07\\
2.93546773386693	1.13391785403161e-07\\
2.93746873436718	9.33812344082167e-08\\
2.93946973486743	7.33712563710678e-08\\
2.94147073536768	4.66907317956674e-08\\
2.94347173586793	5.33608772634624e-08\\
2.94547273636818	6.00310800270369e-08\\
2.94747373686843	8.00410580641858e-08\\
2.94947473736868	6.67008817201548e-08\\
2.95147573786893	4.66907317956674e-08\\
2.95347673836918	5.33608772634624e-08\\
2.95547773886943	5.33608772634624e-08\\
2.95747873936968	3.33505554516364e-08\\
2.95947973986994	7.33712563710678e-08\\
2.96148074037019	7.33712563710678e-08\\
2.96348174087044	8.00410580641858e-08\\
2.96548274137069	1.0005160905913e-07\\
2.96748374187094	7.33712563710678e-08\\
2.96948474237119	7.33712563710678e-08\\
2.97148574287144	7.33712563710678e-08\\
2.97348674337169	4.66907317956674e-08\\
2.97548774387194	4.66907317956674e-08\\
2.97748874437219	3.33505554516364e-08\\
2.97948974487244	1.33402336398105e-08\\
2.98149074537269	-1.33402336398105e-08\\
2.98349174587294	1.33402336398105e-08\\
2.98549274637319	-2.00103218118259e-08\\
2.98749374687344	-1.33402336398105e-08\\
2.98949474737369	-1.33402336398105e-08\\
2.99149574787394	-1.33402336398105e-08\\
2.99349674837419	-1.33402336398105e-08\\
2.99549774887444	0\\
2.99749874937469	2.66804099838414e-08\\
2.99949974987494	4.00206436236519e-08\\
3.00150075037519	6.67008817201548e-08\\
3.00350175087544	8.00410580641858e-08\\
3.00550275137569	6.00310800270369e-08\\
3.00750375187594	4.66907317956674e-08\\
3.00950475237619	4.00206436236519e-08\\
3.01150575287644	2.00103218118259e-08\\
3.01350675337669	6.67008817201548e-09\\
3.01550775387694	0\\
3.01750875437719	-3.33505554516364e-08\\
3.01950975487744	-2.00103218118259e-08\\
3.02151075537769	-3.33505554516364e-08\\
3.02351175587794	-6.00310800270369e-08\\
3.02551275637819	-4.66907317956674e-08\\
3.02751375687844	-4.66907317956674e-08\\
3.02951475737869	-6.67008817201548e-08\\
3.03151575787894	-8.00410580641858e-08\\
3.03351675837919	-6.67008817201548e-08\\
3.03551775887944	-4.66907317956674e-08\\
3.03751875937969	-4.66907317956674e-08\\
3.03951975987994	-4.66907317956674e-08\\
3.04152076038019	-6.00310800270369e-08\\
3.04352176088044	-4.66907317956674e-08\\
3.04552276138069	-2.66804099838414e-08\\
3.04752376188094	-2.66804099838414e-08\\
3.04952476238119	-3.33505554516364e-08\\
3.05152576288144	-5.33608772634624e-08\\
3.05352676338169	-4.66907317956674e-08\\
3.05552776388194	-5.33608772634624e-08\\
3.05752876438219	-8.00410580641858e-08\\
3.05952976488244	-9.33812344082167e-08\\
3.06153076538269	-8.67114327150988e-08\\
3.06353176588294	-8.67114327150988e-08\\
3.06553276638319	-1.13391785403161e-07\\
3.06753376688344	-1.13391785403161e-07\\
3.06953476738369	-8.00410580641858e-08\\
3.07153576788394	-8.67114327150988e-08\\
3.07353676838419	-8.00410580641858e-08\\
3.07553776888444	-8.67114327150988e-08\\
3.07753876938469	-8.00410580641858e-08\\
3.07953976988494	-1.0005160905913e-07\\
3.08154077038519	-1.0005160905913e-07\\
3.08354177088544	-1.33402336398105e-07\\
3.08554277138569	-1.20062160054074e-07\\
3.08754377188594	-1.40072138091223e-07\\
3.08954477238619	-1.53412314435254e-07\\
3.09154577288644	-1.66752490779285e-07\\
3.09354677338669	-1.53412314435254e-07\\
3.09554777388694	-1.46742512742136e-07\\
3.09754877438719	-1.40072138091223e-07\\
3.09954977488744	-1.66752490779285e-07\\
3.10155077538769	-1.66752490779285e-07\\
3.10355177588794	-1.60082689086167e-07\\
3.10555277638819	-1.26731961747192e-07\\
3.10755377688844	-1.26731961747192e-07\\
3.10955477738869	-1.13391785403161e-07\\
3.11155577788894	-1.13391785403161e-07\\
3.11355677838919	-1.06721983710043e-07\\
3.11555777888944	-1.13391785403161e-07\\
3.11755877938969	-1.26731961747192e-07\\
3.11955977988994	-1.26731961747192e-07\\
3.1215607803902	-1.20062160054074e-07\\
3.12356178089045	-1.33402336398105e-07\\
3.1255627813907	-1.40072138091223e-07\\
3.12756378189095	-1.06721983710043e-07\\
3.1295647823912	-1.0005160905913e-07\\
3.13156578289145	-1.13391785403161e-07\\
3.1335667833917	-1.20062160054074e-07\\
3.13556778389195	-1.40072138091223e-07\\
3.1375687843922	-1.46742512742136e-07\\
3.13956978489245	-1.46742512742136e-07\\
3.1415707853927	-1.86763041774229e-07\\
3.14357178589295	-1.46742512742136e-07\\
3.1455727863932	-1.73422865430198e-07\\
3.14757378689345	-2.0010321811826e-07\\
3.1495747873937	-2.1344339446229e-07\\
3.15157578789395	-2.1344339446229e-07\\
3.1535767883942	-2.0010321811826e-07\\
3.15557778889445	-1.80092667123315e-07\\
3.1575787893947	-1.73422865430198e-07\\
3.15957978989495	-1.86763041774229e-07\\
3.1615807903952	-2.1344339446229e-07\\
3.16358179089545	-2.0010321811826e-07\\
3.1655827913957	-2.06773019811377e-07\\
3.16758379189595	-2.26783570806321e-07\\
3.1695847923962	-2.26783570806321e-07\\
3.17158579289645	-2.33453945457234e-07\\
3.1735867933967	-2.40123747150352e-07\\
3.17558779389695	-2.20113769113203e-07\\
3.1775887943972	-2.0010321811826e-07\\
3.17958979489745	-2.26783570806321e-07\\
3.1815907953977	-2.40123747150352e-07\\
3.18359179589795	-2.46794121801265e-07\\
3.1855927963982	-2.20113769113203e-07\\
3.18759379689845	-2.26783570806321e-07\\
3.1895947973987	-2.53463923494383e-07\\
3.19159579789895	-2.86814650833358e-07\\
3.1935967983992	-2.66804099838414e-07\\
3.19559779889945	-2.80144276182445e-07\\
3.1975987993997	-2.80144276182445e-07\\
3.19959979989995	-2.66804099838414e-07\\
3.2016008004002	-2.66804099838414e-07\\
3.20360180090045	-2.33453945457234e-07\\
3.2056028014007	-2.33453945457234e-07\\
3.20760380190095	-2.46794121801265e-07\\
3.2096048024012	-2.33453945457234e-07\\
3.21160580290145	-2.33453945457234e-07\\
3.2136068034017	-2.46794121801265e-07\\
3.21560780390195	-2.53463923494383e-07\\
3.2176088044022	-2.53463923494383e-07\\
3.21960980490245	-2.46794121801265e-07\\
3.2216108054027	-3.00154827177389e-07\\
3.22361180590295	-3.06824628870507e-07\\
3.2256128064032	-3.33505554516364e-07\\
3.22761380690345	-3.40175356209482e-07\\
3.2296148074037	-3.20165378172333e-07\\
3.23161580790395	-3.40175356209482e-07\\
3.2336168084042	-3.40175356209482e-07\\
3.23561780890445	-3.20165378172333e-07\\
3.2376188094047	-2.93484452526476e-07\\
3.23961980990495	-3.1349500352142e-07\\
3.2416208104052	-3.06824628870507e-07\\
3.24362181090545	-3.20165378172333e-07\\
3.2456228114057	-3.00154827177389e-07\\
3.24762381190595	-3.1349500352142e-07\\
3.2496248124062	-3.26835179865451e-07\\
3.25162581290645	-3.00154827177389e-07\\
3.2536268134067	-2.73474474489327e-07\\
3.25562781390695	-2.73474474489327e-07\\
3.2576288144072	-2.73474474489327e-07\\
3.25962981490745	-3.00154827177389e-07\\
3.2616308154077	-3.20165378172333e-07\\
3.26363181590795	-3.06824628870507e-07\\
3.2656328164082	-3.06824628870507e-07\\
3.26763381690845	-3.00154827177389e-07\\
3.2696348174087	-3.20165378172333e-07\\
3.27163581790895	-3.40175356209482e-07\\
3.2736368184092	-3.33505554516364e-07\\
3.27563781890945	-3.40175356209482e-07\\
3.27763881940971	-3.20165378172333e-07\\
3.27963981990996	-3.33505554516364e-07\\
3.2816408204102	-3.46845730860395e-07\\
3.28364182091046	-3.40175356209482e-07\\
3.28564282141071	-3.26835179865451e-07\\
3.28764382191096	-3.33505554516364e-07\\
3.28964482241121	-3.33505554516364e-07\\
3.29164582291146	-2.93484452526476e-07\\
3.29364682341171	-3.1349500352142e-07\\
3.29564782391196	-3.40175356209482e-07\\
3.29764882441221	-3.20165378172333e-07\\
3.29964982491246	-3.06824628870507e-07\\
3.30165082541271	-3.00154827177389e-07\\
3.30365182591296	-2.80144276182445e-07\\
3.30565282641321	-3.00154827177389e-07\\
3.30765382691346	-3.06824628870507e-07\\
3.30965482741371	-2.60134298145296e-07\\
3.31165582791396	-2.73474474489327e-07\\
3.31365682841421	-3.1349500352142e-07\\
3.31565782891446	-3.1349500352142e-07\\
3.31765882941471	-2.93484452526476e-07\\
3.31965982991496	-3.06824628870507e-07\\
3.32166083041521	-3.06824628870507e-07\\
3.32366183091546	-3.00154827177389e-07\\
3.32566283141571	-2.86814650833358e-07\\
3.32766383191596	-2.86814650833358e-07\\
3.32966483241621	-3.06824628870507e-07\\
3.33166583291646	-3.1349500352142e-07\\
3.33366683341671	-3.1349500352142e-07\\
3.33566783391696	-2.93484452526476e-07\\
3.33766883441721	-2.80144276182445e-07\\
3.33966983491746	-3.00154827177389e-07\\
3.34167083541771	-2.73474474489327e-07\\
3.34367183591796	-2.86814650833358e-07\\
3.34567283641821	-2.86814650833358e-07\\
3.34767383691846	-2.93484452526476e-07\\
3.34967483741871	-2.86814650833358e-07\\
3.35167583791896	-2.93484452526476e-07\\
3.35367683841921	-3.20165378172333e-07\\
3.35567783891946	-3.40175356209482e-07\\
3.35767883941971	-3.60185907204426e-07\\
3.35967983991996	-3.26835179865451e-07\\
3.36168084042021	-3.20165378172333e-07\\
3.36368184092046	-3.1349500352142e-07\\
3.36568284142071	-2.80144276182445e-07\\
3.36768384192096	-2.93484452526476e-07\\
3.36968484242121	-2.93484452526476e-07\\
3.37168584292146	-3.06824628870507e-07\\
3.37368684342171	-3.20165378172333e-07\\
3.37568784392196	-3.06824628870507e-07\\
3.37768884442221	-3.33505554516364e-07\\
3.37968984492246	-3.40175356209482e-07\\
3.38169084542271	-3.06824628870507e-07\\
3.38369184592296	-3.06824628870507e-07\\
3.38569284642321	-3.00154827177389e-07\\
3.38769384692346	-3.20165378172333e-07\\
3.38969484742371	-3.20165378172333e-07\\
3.39169584792396	-3.1349500352142e-07\\
3.39369684842421	-3.1349500352142e-07\\
3.39569784892446	-3.20165378172333e-07\\
3.39769884942471	-3.1349500352142e-07\\
3.39969984992496	-3.06824628870507e-07\\
3.40170085042521	-3.06824628870507e-07\\
3.40370185092546	-2.80144276182445e-07\\
3.40570285142571	-2.80144276182445e-07\\
3.40770385192596	-2.73474474489327e-07\\
3.40970485242621	-2.46794121801265e-07\\
3.41170585292646	-2.40123747150352e-07\\
3.41370685342671	-2.73474474489327e-07\\
3.41570785392696	-2.86814650833358e-07\\
3.41770885442721	-3.06824628870507e-07\\
3.41970985492746	-3.46845730860395e-07\\
3.42171085542771	-3.40175356209482e-07\\
3.42371185592796	-3.53515532553513e-07\\
3.42571285642821	-3.33505554516364e-07\\
3.42771385692846	-3.60185907204426e-07\\
3.42971485742871	-3.53515532553513e-07\\
3.43171585792896	-3.73526083548457e-07\\
3.43371685842921	-3.53515532553513e-07\\
3.43571785892946	-3.60185907204426e-07\\
3.43771885942971	-3.66855708897544e-07\\
3.43971985992996	-3.66855708897544e-07\\
3.44172086043022	-4.06876237929637e-07\\
3.44372186093047	-4.00206436236519e-07\\
3.44572286143072	-4.00206436236519e-07\\
3.44772386193097	-4.00206436236519e-07\\
3.44972486243122	-3.93536061585606e-07\\
3.45172586293147	-3.93536061585606e-07\\
3.45372686343172	-3.86866259892488e-07\\
3.45572786393197	-3.80195885241575e-07\\
3.45772886443222	-3.80195885241575e-07\\
3.45972986493247	-3.73526083548457e-07\\
3.46173086543272	-3.73526083548457e-07\\
3.46373186593297	-3.66855708897544e-07\\
3.46573286643322	-4.00206436236519e-07\\
3.46773386693347	-3.86866259892488e-07\\
3.46973486743372	-3.93536061585606e-07\\
3.47173586793397	-4.00206436236519e-07\\
3.47373686843422	-4.00206436236519e-07\\
3.47573786893447	-4.06876237929637e-07\\
3.47773886943472	-3.93536061585606e-07\\
3.47973986993497	-4.1354661258055e-07\\
3.48174087043522	-4.06876237929637e-07\\
3.48374187093547	-4.06876237929637e-07\\
3.48574287143572	-4.26886788924581e-07\\
3.48774387193597	-4.26886788924581e-07\\
3.48974487243622	-4.26886788924581e-07\\
3.49174587293647	-4.33557163575494e-07\\
3.49374687343672	-4.53567141612643e-07\\
3.49574787393697	-4.40226965268612e-07\\
3.49774887443722	-4.46897339919525e-07\\
3.49974987493747	-4.40226965268612e-07\\
3.50175087543772	-4.53567141612643e-07\\
3.50375187593797	-4.46897339919525e-07\\
3.50575287643822	-4.46897339919525e-07\\
3.50775387693847	-4.53567141612643e-07\\
3.50975487743872	-4.40226965268612e-07\\
3.51175587793897	-4.40226965268612e-07\\
3.51375687843922	-4.1354661258055e-07\\
3.51575787893947	-4.26886788924581e-07\\
3.51775887943972	-4.40226965268612e-07\\
3.51975987993997	-4.46897339919525e-07\\
3.52176088044022	-4.33557163575494e-07\\
3.52376188094047	-4.40226965268612e-07\\
3.52576288144072	-4.40226965268612e-07\\
3.52776388194097	-4.26886788924581e-07\\
3.52976488244122	-4.06876237929637e-07\\
3.53176588294147	-4.33557163575494e-07\\
3.53376688344172	-4.20216987231463e-07\\
3.53576788394197	-4.26886788924581e-07\\
3.53776888444222	-4.33557163575494e-07\\
3.53976988494247	-4.26886788924581e-07\\
3.54177088544272	-4.1354661258055e-07\\
3.54377188594297	-3.93536061585606e-07\\
3.54577288644322	-4.26886788924581e-07\\
3.54777388694347	-4.26886788924581e-07\\
3.54977488744372	-4.40226965268612e-07\\
3.55177588794397	-4.66907317956674e-07\\
3.55377688844422	-4.80247494300705e-07\\
3.55577788894447	-4.80247494300705e-07\\
3.55777888944472	-4.86917868951618e-07\\
3.55977988994497	-4.86917868951618e-07\\
3.56178089044522	-4.93587670644736e-07\\
3.56378189094547	-5.00258045295649e-07\\
3.56578289144572	-5.26938397983711e-07\\
3.56778389194597	-5.40278574327742e-07\\
3.56978489244622	-5.40278574327742e-07\\
3.57178589294647	-5.40278574327742e-07\\
3.57378689344672	-5.46948948978655e-07\\
3.57578789394697	-5.60289125322686e-07\\
3.57778889444722	-5.46948948978655e-07\\
3.57978989494747	-5.53618750671773e-07\\
3.58179089544772	-5.66958927015804e-07\\
3.58379189594797	-5.80297384486449e-07\\
3.58579289644822	-5.60289125322686e-07\\
3.58779389694847	-5.53618750671773e-07\\
3.58979489744872	-5.60289125322686e-07\\
3.59179589794897	-5.60289125322686e-07\\
3.59379689844922	-5.53618750671773e-07\\
3.59579789894947	-5.60289125322686e-07\\
3.59779889944972	-5.60289125322686e-07\\
3.59979989994997	-5.46948948978655e-07\\
3.60180090045022	-5.60289125322686e-07\\
3.60380190095048	-5.60289125322686e-07\\
3.60580290145073	-5.40278574327742e-07\\
3.60780390195098	-5.46948948978655e-07\\
3.60980490245123	-5.40278574327742e-07\\
3.61180590295148	-5.46948948978655e-07\\
3.61380690345173	-5.40278574327742e-07\\
3.61580790395198	-5.20268596290593e-07\\
3.61780890445223	-5.33608772634624e-07\\
3.61980990495248	-5.26938397983711e-07\\
3.62181090545273	-5.26938397983711e-07\\
3.62381190595298	-4.86917868951618e-07\\
3.62581290645323	-4.86917868951618e-07\\
3.62781390695348	-5.06927846988767e-07\\
3.62981490745373	-5.00258045295649e-07\\
3.63181590795398	-4.93587670644736e-07\\
3.63381690845423	-4.60237516263556e-07\\
3.63581790895448	-4.53567141612643e-07\\
3.63781890945473	-4.73577692607587e-07\\
3.63981990995498	-5.1359822163968e-07\\
3.64182091045523	-5.26938397983711e-07\\
3.64382191095548	-5.26938397983711e-07\\
3.64582291145573	-5.00258045295649e-07\\
3.64782391195598	-5.00258045295649e-07\\
3.64982491245623	-5.06927846988767e-07\\
3.65182591295648	-4.86917868951618e-07\\
3.65382691345673	-4.93587670644736e-07\\
3.65582791395698	-5.06927846988767e-07\\
3.65782891445723	-5.06927846988767e-07\\
3.65982991495748	-4.86917868951618e-07\\
3.66183091545773	-5.00258045295649e-07\\
3.66383191595798	-4.93587670644736e-07\\
3.66583291645823	-4.93587670644736e-07\\
3.66783391695848	-5.06927846988767e-07\\
3.66983491745873	-5.1359822163968e-07\\
3.67183591795898	-5.00258045295649e-07\\
3.67383691845923	-5.1359822163968e-07\\
3.67583791895948	-5.00258045295649e-07\\
3.67783891945973	-5.06927846988767e-07\\
3.67983991995998	-4.80247494300705e-07\\
3.68184092046023	-4.73577692607587e-07\\
3.68384192096048	-4.93587670644736e-07\\
3.68584292146073	-5.00258045295649e-07\\
3.68784392196098	-5.20268596290593e-07\\
3.68984492246123	-5.00258045295649e-07\\
3.69184592296148	-5.20268596290593e-07\\
3.69384692346173	-5.00258045295649e-07\\
3.69584792396198	-4.93587670644736e-07\\
3.69784892446223	-4.73577692607587e-07\\
3.69984992496248	-4.60237516263556e-07\\
3.70185092546273	-4.53567141612643e-07\\
3.70385192596298	-4.46897339919525e-07\\
3.70585292646323	-4.40226965268612e-07\\
3.70785392696348	-4.46897339919525e-07\\
3.70985492746373	-4.33557163575494e-07\\
3.71185592796398	-4.20216987231463e-07\\
3.71385692846423	-4.26886788924581e-07\\
3.71585792896448	-3.66855708897544e-07\\
3.71785892946473	-4.1354661258055e-07\\
3.71985992996498	-4.33557163575494e-07\\
3.72186093046523	-4.46897339919525e-07\\
3.72386193096548	-4.26886788924581e-07\\
3.72586293146573	-4.33557163575494e-07\\
3.72786393196598	-4.26886788924581e-07\\
3.72986493246623	-3.80195885241575e-07\\
3.73186593296648	-3.86866259892488e-07\\
3.73386693346673	-3.93536061585606e-07\\
3.73586793396698	-4.00206436236519e-07\\
3.73786893446723	-3.93536061585606e-07\\
3.73986993496748	-4.26886788924581e-07\\
3.74187093546773	-4.20216987231463e-07\\
3.74387193596798	-3.93536061585606e-07\\
3.74587293646823	-3.80195885241575e-07\\
3.74787393696848	-3.80195885241575e-07\\
3.74987493746873	-4.00206436236519e-07\\
3.75187593796898	-4.06876237929637e-07\\
3.75387693846923	-4.1354661258055e-07\\
3.75587793896948	-3.93536061585606e-07\\
3.75787893946973	-4.26886788924581e-07\\
3.75987993996999	-4.26886788924581e-07\\
3.76188094047024	-4.40226965268612e-07\\
3.76388194097049	-4.40226965268612e-07\\
3.76588294147074	-4.26886788924581e-07\\
3.76788394197099	-4.00206436236519e-07\\
3.76988494247124	-3.86866259892488e-07\\
3.77188594297149	-4.06876237929637e-07\\
3.77388694347174	-4.1354661258055e-07\\
3.77588794397199	-3.93536061585606e-07\\
3.77788894447224	-4.26886788924581e-07\\
3.77988994497249	-4.06876237929637e-07\\
3.78189094547274	-4.20216987231463e-07\\
3.78389194597299	-4.20216987231463e-07\\
3.78589294647324	-4.1354661258055e-07\\
3.78789394697349	-3.86866259892488e-07\\
3.78989494747374	-3.73526083548457e-07\\
3.79189594797399	-3.60185907204426e-07\\
3.79389694847424	-3.06824628870507e-07\\
3.79589794897449	-3.00154827177389e-07\\
3.79789894947474	-3.20165378172333e-07\\
3.79989994997499	-3.46845730860395e-07\\
3.80190095047524	-3.60185907204426e-07\\
3.80390195097549	-3.53515532553513e-07\\
3.80590295147574	-3.80195885241575e-07\\
3.80790395197599	-3.60185907204426e-07\\
3.80990495247624	-3.73526083548457e-07\\
3.81190595297649	-3.80195885241575e-07\\
3.81390695347674	-3.86866259892488e-07\\
3.81590795397699	-3.93536061585606e-07\\
3.81790895447724	-4.33557163575494e-07\\
3.81990995497749	-4.06876237929637e-07\\
3.82191095547774	-3.93536061585606e-07\\
3.82391195597799	-4.1354661258055e-07\\
3.82591295647824	-4.26886788924581e-07\\
3.82791395697849	-4.40226965268612e-07\\
3.82991495747874	-4.40226965268612e-07\\
3.83191595797899	-4.33557163575494e-07\\
3.83391695847924	-3.93536061585606e-07\\
3.83591795897949	-4.26886788924581e-07\\
3.83791895947974	-4.40226965268612e-07\\
3.83991995997999	-4.1354661258055e-07\\
3.84192096048024	-4.46897339919525e-07\\
3.84392196098049	-4.06876237929637e-07\\
3.84592296148074	-4.26886788924581e-07\\
3.84792396198099	-4.20216987231463e-07\\
3.84992496248124	-4.00206436236519e-07\\
3.85192596298149	-4.26886788924581e-07\\
3.85392696348174	-4.33557163575494e-07\\
3.85592796398199	-4.40226965268612e-07\\
3.85792896448224	-4.46897339919525e-07\\
3.85992996498249	-4.53567141612643e-07\\
3.86193096548274	-4.53567141612643e-07\\
3.86393196598299	-4.53567141612643e-07\\
3.86593296648324	-4.93587670644736e-07\\
3.86793396698349	-4.80247494300705e-07\\
3.86993496748374	-4.80247494300705e-07\\
3.87193596798399	-4.60237516263556e-07\\
3.87393696848424	-4.53567141612643e-07\\
3.87593796898449	-4.46897339919525e-07\\
3.87793896948474	-4.46897339919525e-07\\
3.87993996998499	-4.40226965268612e-07\\
3.88194097048524	-4.06876237929637e-07\\
3.88394197098549	-4.00206436236519e-07\\
3.88594297148574	-3.66855708897544e-07\\
3.88794397198599	-3.80195885241575e-07\\
3.88994497248624	-4.00206436236519e-07\\
3.89194597298649	-3.80195885241575e-07\\
3.89394697348674	-4.06876237929637e-07\\
3.89594797398699	-4.00206436236519e-07\\
3.89794897448724	-3.80195885241575e-07\\
3.89994997498749	-4.1354661258055e-07\\
3.90195097548774	-4.00206436236519e-07\\
3.90395197598799	-3.86866259892488e-07\\
3.90595297648824	-3.66855708897544e-07\\
3.90795397698849	-3.80195885241575e-07\\
3.90995497748874	-3.93536061585606e-07\\
3.91195597798899	-3.73526083548457e-07\\
3.91395697848924	-3.40175356209482e-07\\
3.91595797898949	-3.20165378172333e-07\\
3.91795897948974	-3.06824628870507e-07\\
3.91995997998999	-3.1349500352142e-07\\
3.92196098049025	-2.80144276182445e-07\\
3.9239619809905	-2.86814650833358e-07\\
3.92596298149075	-2.40123747150352e-07\\
3.927963981991	-2.46794121801265e-07\\
3.92996498249125	-2.66804099838414e-07\\
3.9319659829915	-2.86814650833358e-07\\
3.93396698349175	-2.93484452526476e-07\\
3.935967983992	-2.80144276182445e-07\\
3.93796898449225	-2.80144276182445e-07\\
3.9399699849925	-2.66804099838414e-07\\
3.94197098549275	-2.33453945457234e-07\\
3.943971985993	-2.0010321811826e-07\\
3.94597298649325	-1.93432843467346e-07\\
3.9479739869935	-1.80092667123315e-07\\
3.94997498749375	-1.80092667123315e-07\\
3.951975987994	-2.1344339446229e-07\\
3.95397698849425	-2.20113769113203e-07\\
3.9559779889945	-2.1344339446229e-07\\
3.95797898949475	-2.26783570806321e-07\\
3.959979989995	-2.06773019811377e-07\\
3.96198099049525	-1.73422865430198e-07\\
3.9639819909955	-1.73422865430198e-07\\
3.96598299149575	-1.66752490779285e-07\\
3.967983991996	-1.53412314435254e-07\\
3.96998499249625	-1.60082689086167e-07\\
3.9719859929965	-1.60082689086167e-07\\
3.97398699349675	-1.66752490779285e-07\\
3.975987993997	-1.80092667123315e-07\\
3.97798899449725	-2.26783570806321e-07\\
3.9799899949975	-2.06773019811377e-07\\
3.98199099549775	-2.06773019811377e-07\\
3.983991995998	-2.1344339446229e-07\\
3.98599299649825	-2.0010321811826e-07\\
3.9879939969985	-1.80092667123315e-07\\
3.98999499749875	-1.73422865430198e-07\\
3.991995997999	-1.86763041774229e-07\\
3.99399699849925	-1.86763041774229e-07\\
3.9959979989995	-1.86763041774229e-07\\
3.99799899949975	-1.53412314435254e-07\\
4	-1.60082689086167e-07\\
};
\addlegendentry{Energy Diff};

\end{axis}
\end{tikzpicture}%
	\caption{A zoom in on the error when stepping forward in time.}
	\label{fig:forwardDataError}
\end{figure}
\fi
oe
\iftikz
\begin{figure}[H]
	\centering
	\setlength\figureheight{7cm} 
	\setlength\figurewidth{14cm}
	% This file was created by matlab2tikz.
% Minimal pgfplots version: 1.3
%
%The latest updates can be retrieved from
%  http://www.mathworks.com/matlabcentral/fileexchange/22022-matlab2tikz
%where you can also make suggestions and rate matlab2tikz.
%
\definecolor{mycolor1}{rgb}{0.00000,0.44700,0.74100}%
\definecolor{mycolor2}{rgb}{0.85000,0.32500,0.09800}%
\definecolor{mycolor3}{rgb}{0.92900,0.69400,0.12500}%
%
\begin{tikzpicture}

\begin{axis}[%
width=0.95092\figurewidth,
height=\figureheight,
at={(0\figurewidth,0\figureheight)},
scale only axis,
xmin=0,
xmax=4,
xtick={0,0.5,1,1.5,2,2.5,3,3.5,4},
xticklabels={{4},{3.5},{3},{2.5},{2},{1.5},{1},{0.5},{0}},
xlabel={Time (s)},
ymin=-400000000000,
ymax=50000000000,
ylabel={Degrees},
title style={font=\bfseries},
title={Top Spin [4,0] (s) Errors},
legend style={legend cell align=left,align=left,draw=white!15!black},
title style={font=\small},ticklabel style={font=\tiny}
]
\addplot [color=mycolor1,solid,forget plot]
  table[row sep=crcr]{%
0	389202.781781032\\
0.000100002500062502	389205.073612212\\
0.000200005000125003	389206.792485597\\
0.000300007500187505	389209.084316778\\
0.000400010000250006	389211.376147959\\
0.000500012500312508	389213.095021344\\
0.000600015000375009	389215.386852524\\
0.000700017500437511	389217.10572591\\
0.000800020000500012	389219.39755709\\
0.000900022500562514	389221.116430476\\
0.00100002500062502	389222.835303861\\
0.00110002750068752	389225.127135042\\
0.00120003000075002	389226.846008427\\
0.00130003250081252	389228.564881812\\
0.00140003500087502	389230.856712993\\
0.00150003750093752	389232.575586378\\
0.00160004000100002	389234.294459764\\
0.00170004250106253	389236.013333149\\
0.00180004500112503	389237.732206534\\
0.00190004750118753	389239.45107992\\
0.00200005000125003	389241.169953305\\
0.00210005250131253	389242.888826691\\
0.00220005500137503	389244.607700076\\
0.00230005750143754	389246.326573461\\
0.00240006000150004	389248.045446847\\
0.00250006250156254	389249.764320232\\
0.00260006500162504	389251.483193618\\
0.00270006750168754	389253.202067003\\
0.00280007000175004	389254.920940388\\
0.00290007250181255	389256.066855979\\
0.00300007500187505	389257.785729364\\
0.00310007750193755	389259.504602749\\
0.00320008000200005	389260.65051834\\
0.00330008250206255	389262.369391725\\
0.00340008500212505	389264.088265111\\
0.00350008750218755	389265.234180701\\
0.00360009000225006	389266.953054086\\
0.00370009250231256	389268.098969676\\
0.00380009500237506	389269.817843062\\
0.00390009750243756	389270.963758652\\
0.00400010000250006	389272.109674242\\
0.00410010250256256	389273.828547628\\
0.00420010500262507	389274.974463218\\
0.00430010750268757	389276.120378808\\
0.00440011000275007	389277.839252194\\
0.00450011250281257	389278.985167784\\
0.00460011500287507	389280.131083374\\
0.00470011750293757	389281.276998964\\
0.00480012000300007	389282.422914555\\
0.00490012250306258	389283.568830145\\
0.00500012500312508	389284.714745735\\
0.00510012750318758	389285.860661325\\
0.00520013000325008	389287.006576916\\
0.00530013250331258	389288.152492506\\
0.00540013500337508	389289.298408096\\
0.00550013750343759	389290.444323687\\
0.00560014000350009	389291.590239277\\
0.00570014250356259	389292.736154867\\
0.00580014500362509	389293.882070457\\
0.00590014750368759	389294.455028253\\
0.00600015000375009	389295.600943843\\
0.0061001525038126	389296.746859433\\
0.0062001550038751	389297.319817228\\
0.0063001575039376	389298.465732818\\
0.0064001600040001	389299.611648409\\
0.0065001625040626	389300.184606204\\
0.0066001650041251	389301.330521794\\
0.0067001675041876	389301.903479589\\
0.00680017000425011	389303.049395179\\
0.00690017250431261	389303.622352974\\
0.00700017500437511	389304.19531077\\
0.00710017750443761	389305.34122636\\
0.00720018000450011	389305.914184155\\
0.00730018250456261	389306.48714195\\
0.00740018500462512	389307.63305754\\
0.00750018750468762	389308.206015336\\
0.00760019000475012	389308.778973131\\
0.00770019250481262	389309.351930926\\
0.00780019500487512	389309.924888721\\
0.00790019750493762	389310.497846516\\
0.00800020000500012	389311.070804311\\
0.00810020250506263	389311.643762106\\
0.00820020500512513	389312.216719901\\
0.00830020750518763	389312.789677697\\
0.00840021000525013	389313.362635492\\
0.00850021250531263	389313.935593287\\
0.00860021500537513	389314.508551082\\
0.00870021750543764	389315.081508877\\
0.00880022000550014	389315.654466672\\
0.00890022250556264	389315.654466672\\
0.00900022500562514	389316.227424467\\
0.00910022750568764	389316.800382263\\
0.00920023000575014	389316.800382263\\
0.00930023250581265	389317.373340058\\
0.00940023500587515	389317.946297853\\
0.00950023750593765	389317.946297853\\
0.00960024000600015	389318.519255648\\
0.00970024250606265	389318.519255648\\
0.00980024500612515	389319.092213443\\
0.00990024750618766	389319.092213443\\
0.0100002500062502	389319.665171238\\
0.0101002525063127	389319.665171238\\
0.0102002550063752	389319.665171238\\
0.0103002575064377	389319.665171238\\
0.0104002600065002	389320.238129033\\
0.0105002625065627	389320.238129033\\
0.0106002650066252	389320.238129033\\
0.0107002675066877	389320.238129033\\
0.0108002700067502	389320.238129033\\
0.0109002725068127	389320.811086829\\
0.0110002750068752	389320.811086829\\
0.0111002775069377	389320.811086829\\
0.0112002800070002	389320.811086829\\
0.0113002825070627	389320.811086829\\
0.0114002850071252	389320.238129033\\
0.0115002875071877	389320.238129033\\
0.0116002900072502	389320.238129033\\
0.0117002925073127	389320.238129033\\
0.0118002950073752	389320.238129033\\
0.0119002975074377	389320.238129033\\
0.0120003000075002	389319.665171238\\
0.0121003025075627	389319.665171238\\
0.0122003050076252	389319.665171238\\
0.0123003075076877	389319.092213443\\
0.0124003100077502	389319.092213443\\
0.0125003125078127	389318.519255648\\
0.0126003150078752	389318.519255648\\
0.0127003175079377	389317.946297853\\
0.0128003200080002	389317.946297853\\
0.0129003225080627	389317.373340058\\
0.0130003250081252	389317.373340058\\
0.0131003275081877	389316.800382263\\
0.0132003300082502	389316.227424467\\
0.0133003325083127	389316.227424467\\
0.0134003350083752	389315.654466672\\
0.0135003375084377	389315.081508877\\
0.0136003400085002	389314.508551082\\
0.0137003425085627	389313.935593287\\
0.0138003450086252	389313.935593287\\
0.0139003475086877	389313.362635492\\
0.0140003500087502	389312.789677697\\
0.0141003525088127	389312.216719901\\
0.0142003550088752	389311.643762106\\
0.0143003575089377	389311.070804311\\
0.0144003600090002	389310.497846516\\
0.0145003625090627	389309.351930926\\
0.0146003650091252	389308.778973131\\
0.0147003675091877	389308.206015336\\
0.0148003700092502	389307.63305754\\
0.0149003725093127	389307.060099745\\
0.0150003750093752	389305.914184155\\
0.0151003775094377	389305.34122636\\
0.0152003800095002	389304.768268565\\
0.0153003825095627	389303.622352974\\
0.0154003850096252	389303.049395179\\
0.0155003875096877	389302.476437384\\
0.0156003900097502	389301.330521794\\
0.0157003925098127	389300.757563999\\
0.0158003950098752	389299.611648409\\
0.0159003975099377	389298.465732818\\
0.0160004000100002	389297.892775023\\
0.0161004025100628	389296.746859433\\
0.0162004050101253	389295.600943843\\
0.0163004075101878	389295.027986048\\
0.0164004100102503	389293.882070457\\
0.0165004125103128	389292.736154867\\
0.0166004150103753	389291.590239277\\
0.0167004175104378	389291.017281482\\
0.0168004200105003	389289.871365891\\
0.0169004225105628	389288.725450301\\
0.0170004250106253	389287.579534711\\
0.0171004275106878	389286.433619121\\
0.0172004300107503	389285.28770353\\
0.0173004325108128	389284.14178794\\
0.0174004350108753	389282.99587235\\
0.0175004375109378	389281.84995676\\
0.0176004400110003	389280.131083374\\
0.0177004425110628	389278.985167784\\
0.0178004450111253	389277.839252194\\
0.0179004475111878	389276.693336603\\
0.0180004500112503	389275.547421013\\
0.0181004525113128	389273.828547628\\
0.0182004550113753	389272.682632037\\
0.0183004575114378	389271.536716447\\
0.0184004600115003	389269.817843062\\
0.0185004625115628	389268.671927472\\
0.0186004650116253	389266.953054086\\
0.0187004675116878	389265.807138496\\
0.0188004700117503	389264.088265111\\
0.0189004725118128	389262.94234952\\
0.0190004750118753	389261.223476135\\
0.0191004775119378	389259.504602749\\
0.0192004800120003	389258.358687159\\
0.0193004825120628	389256.639813774\\
0.0194004850121253	389254.920940388\\
0.0195004875121878	389253.202067003\\
0.0196004900122503	389252.056151413\\
0.0197004925123128	389250.337278027\\
0.0198004950123753	389248.618404642\\
0.0199004975124378	389246.899531257\\
0.0200005000125003	389245.180657871\\
0.0201005025125628	389243.461784486\\
0.0202005050126253	389241.7429111\\
0.0203005075126878	389240.024037715\\
0.0204005100127503	389238.30516433\\
0.0205005125128128	389236.586290944\\
0.0206005150128753	389234.867417559\\
0.0207005175129378	389233.148544173\\
0.0208005200130003	389230.856712993\\
0.0209005225130628	389229.137839607\\
0.0210005250131253	389227.418966222\\
0.0211005275131878	389225.127135042\\
0.0212005300132503	389223.408261656\\
0.0213005325133128	389221.689388271\\
0.0214005350133753	389219.39755709\\
0.0215005375134378	389217.678683705\\
0.0216005400135003	389215.386852524\\
0.0217005425135628	389213.667979139\\
0.0218005450136253	389211.376147959\\
0.0219005475136878	389209.657274573\\
0.0220005500137503	389207.365443393\\
0.0221005525138128	389205.646570007\\
0.0222005550138753	389203.354738827\\
0.0223005575139378	389201.062907646\\
0.0224005600140003	389198.771076466\\
0.0225005625140629	389197.05220308\\
0.0226005650141254	389194.7603719\\
0.0227005675141879	389192.468540719\\
0.0228005700142504	389190.176709539\\
0.0229005725143129	389187.884878358\\
0.0230005750143754	389185.593047178\\
0.0231005775144379	389183.301215997\\
0.0232005800145004	389181.009384817\\
0.0233005825145629	389178.717553636\\
0.0234005850146254	389176.425722455\\
0.0235005875146879	389174.133891275\\
0.0236005900147504	389171.842060094\\
0.0237005925148129	389169.550228914\\
0.0238005950148754	389167.258397733\\
0.0239005975149379	389164.393608758\\
0.0240006000150004	389162.101777577\\
0.0241006025150629	389159.809946397\\
0.0242006050151254	389156.945157421\\
0.0243006075151879	389154.65332624\\
0.0244006100152504	389152.36149506\\
0.0245006125153129	389149.496706084\\
0.0246006150153754	389147.204874904\\
0.0247006175154379	389144.340085928\\
0.0248006200155004	389142.048254748\\
0.0249006225155629	389139.183465772\\
0.0250006250156254	389136.891634592\\
0.0251006275156879	389134.026845616\\
0.0252006300157504	389131.16205664\\
0.0253006325158129	389128.297267664\\
0.0254006350158754	389126.005436484\\
0.0255006375159379	389123.140647508\\
0.0256006400160004	389120.275858533\\
0.0257006425160629	389117.411069557\\
0.0258006450161254	389114.546280581\\
0.0259006475161879	389112.254449401\\
0.0260006500162504	389109.389660425\\
0.0261006525163129	389106.52487145\\
0.0262006550163754	389103.660082474\\
0.0263006575164379	389100.795293498\\
0.0264006600165004	389097.930504523\\
0.0265006625165629	389094.492757752\\
0.0266006650166254	389091.627968776\\
0.0267006675166879	389088.7631798\\
0.0268006700167504	389085.898390825\\
0.0269006725168129	389083.033601849\\
0.0270006750168754	389079.595855078\\
0.0271006775169379	389076.731066103\\
0.0272006800170004	389073.866277127\\
0.0273006825170629	389070.428530356\\
0.0274006850171254	389067.563741381\\
0.0275006875171879	389064.698952405\\
0.0276006900172504	389061.261205634\\
0.0277006925173129	389058.396416659\\
0.0278006950173754	389054.958669888\\
0.0279006975174379	389051.520923117\\
0.0280007000175004	389048.656134141\\
0.0281007025175629	389045.218387371\\
0.0282007050176254	389042.353598395\\
0.0283007075176879	389038.915851624\\
0.0284007100177504	389035.478104853\\
0.0285007125178129	389032.040358083\\
0.0286007150178754	389029.175569107\\
0.028700717517938	389025.737822336\\
0.0288007200180005	389022.300075565\\
0.028900722518063	389018.862328795\\
0.0290007250181255	389015.424582024\\
0.029100727518188	389011.986835253\\
0.0292007300182505	389008.549088482\\
0.029300732518313	389005.111341711\\
0.0294007350183755	389001.673594941\\
0.029500737518438	388998.23584817\\
0.0296007400185005	388994.798101399\\
0.029700742518563	388991.360354628\\
0.0298007450186255	388987.349650062\\
0.029900747518688	388983.911903292\\
0.0300007500187505	388980.474156521\\
0.030100752518813	388977.03640975\\
0.0302007550188755	388973.025705184\\
0.030300757518938	388969.587958413\\
0.0304007600190005	388965.577253847\\
0.030500762519063	388962.139507077\\
0.0306007650191255	388958.701760306\\
0.030700767519188	388954.69105574\\
0.0308007700192505	388950.680351174\\
0.030900772519313	388947.242604403\\
0.0310007750193755	388943.231899837\\
0.031100777519438	388939.794153066\\
0.0312007800195005	388935.783448501\\
0.031300782519563	388931.772743935\\
0.0314007850196255	388928.334997164\\
0.031500787519688	388924.324292598\\
0.0316007900197505	388920.313588032\\
0.031700792519813	388916.302883466\\
0.0318007950198755	388912.2921789\\
0.031900797519938	388908.281474334\\
0.0320008000200005	388904.843727564\\
0.032100802520063	388900.833022998\\
0.0322008050201255	388896.822318432\\
0.032300807520188	388892.811613866\\
0.0324008100202505	388888.227951505\\
0.032500812520313	388884.217246939\\
0.0326008150203755	388880.206542373\\
0.032700817520438	388876.195837807\\
0.0328008200205005	388872.185133241\\
0.032900822520563	388868.174428675\\
0.0330008250206255	388863.590766314\\
0.033100827520688	388859.580061748\\
0.0332008300207505	388855.569357182\\
0.033300832520813	388850.985694821\\
0.0334008350208755	388846.974990255\\
0.033500837520938	388842.964285689\\
0.0336008400210005	388838.380623328\\
0.033700842521063	388834.369918762\\
0.0338008450211255	388829.786256401\\
0.033900847521188	388825.775551835\\
0.0340008500212505	388821.191889474\\
0.034100852521313	388816.608227113\\
0.0342008550213755	388812.597522547\\
0.034300857521438	388808.013860186\\
0.0344008600215005	388803.430197825\\
0.034500862521563	388798.846535464\\
0.0346008650216255	388794.835830898\\
0.034700867521688	388790.252168537\\
0.0348008700217505	388785.668506176\\
0.034900872521813	388781.084843815\\
0.0350008750218755	388776.501181454\\
0.0351008775219381	388771.917519093\\
0.0352008800220006	388767.333856732\\
0.0353008825220631	388762.750194371\\
0.0354008850221256	388758.16653201\\
0.0355008875221881	388753.582869649\\
0.0356008900222506	388748.999207288\\
0.0357008925223131	388744.415544927\\
0.0358008950223756	388739.831882566\\
0.0359008975224381	388734.67526241\\
0.0360009000225006	388730.091600049\\
0.0361009025225631	388725.507937688\\
0.0362009050226256	388720.351317531\\
0.0363009075226881	388715.76765517\\
0.0364009100227506	388711.183992809\\
0.0365009125228131	388706.027372653\\
0.0366009150228756	388701.443710292\\
0.0367009175229381	388696.287090136\\
0.0368009200230006	388691.703427775\\
0.0369009225230631	388686.546807619\\
0.0370009250231256	388681.963145258\\
0.0371009275231881	388676.806525101\\
0.0372009300232506	388671.649904945\\
0.0373009325233131	388667.066242584\\
0.0374009350233756	388661.909622428\\
0.0375009375234381	388656.753002272\\
0.0376009400235006	388651.596382116\\
0.0377009425235631	388647.012719755\\
0.0378009450236256	388641.856099598\\
0.0379009475236881	388636.699479442\\
0.0380009500237506	388631.542859286\\
0.0381009525238131	388626.38623913\\
0.0382009550238756	388621.229618974\\
0.0383009575239381	388616.072998818\\
0.0384009600240006	388610.916378661\\
0.0385009625240631	388605.759758505\\
0.0386009650241256	388600.603138349\\
0.0387009675241881	388595.446518193\\
0.0388009700242506	388589.716940242\\
0.0389009725243131	388584.560320085\\
0.0390009750243756	388579.403699929\\
0.0391009775244381	388574.247079773\\
0.0392009800245006	388568.517501822\\
0.0393009825245631	388563.360881666\\
0.0394009850246256	388558.204261509\\
0.0395009875246881	388552.474683558\\
0.0396009900247506	388547.318063402\\
0.0397009925248131	388541.588485451\\
0.0398009950248756	388536.431865294\\
0.0399009975249381	388530.702287343\\
0.0400010000250006	388525.545667187\\
0.0401010025250631	388519.816089236\\
0.0402010050251256	388514.086511284\\
0.0403010075251881	388508.929891128\\
0.0404010100252506	388503.200313177\\
0.0405010125253131	388497.470735225\\
0.0406010150253756	388491.741157274\\
0.0407010175254381	388486.584537118\\
0.0408010200255006	388480.854959167\\
0.0409010225255631	388475.125381215\\
0.0410010250256256	388469.395803264\\
0.0411010275256881	388463.666225313\\
0.0412010300257506	388457.936647361\\
0.0413010325258131	388452.20706941\\
0.0414010350258756	388446.477491459\\
0.0415010375259382	388440.747913507\\
0.0416010400260007	388435.018335556\\
0.0417010425260632	388429.288757605\\
0.0418010450261257	388423.559179654\\
0.0419010475261882	388417.256643907\\
0.0420010500262507	388411.527065956\\
0.0421010525263132	388405.797488005\\
0.0422010550263757	388399.494952258\\
0.0423010575264382	388393.765374307\\
0.0424010600265007	388388.035796356\\
0.0425010625265632	388381.733260609\\
0.0426010650266257	388376.003682658\\
0.0427010675266882	388369.701146911\\
0.0428010700267507	388363.97156896\\
0.0429010725268132	388357.669033214\\
0.0430010750268757	388351.939455262\\
0.0431010775269382	388345.636919516\\
0.0432010800270007	388339.907341565\\
0.0433010825270632	388333.604805818\\
0.0434010850271257	388327.302270072\\
0.0435010875271882	388320.999734325\\
0.0436010900272507	388315.270156374\\
0.0437010925273132	388308.967620627\\
0.0438010950273757	388302.665084881\\
0.0439010975274382	388296.362549135\\
0.0440011000275007	388290.060013388\\
0.0441011025275632	388283.757477642\\
0.0442011050276257	388277.454941895\\
0.0443011075276882	388271.152406149\\
0.0444011100277507	388264.849870402\\
0.0445011125278132	388258.547334656\\
0.0446011150278757	388252.244798909\\
0.0447011175279382	388245.942263163\\
0.0448011200280007	388239.639727417\\
0.0449011225280632	388233.33719167\\
0.0450011250281257	388226.461698129\\
0.0451011275281882	388220.159162382\\
0.0452011300282507	388213.856626636\\
0.0453011325283132	388207.554090889\\
0.0454011350283757	388200.678597348\\
0.0455011375284382	388194.376061601\\
0.0456011400285007	388187.50056806\\
0.0457011425285632	388181.198032313\\
0.0458011450286257	388174.322538772\\
0.0459011475286882	388168.020003025\\
0.0460011500287507	388161.144509484\\
0.0461011525288132	388154.841973737\\
0.0462011550288757	388147.966480196\\
0.0463011575289382	388141.090986654\\
0.0464011600290007	388134.788450908\\
0.0465011625290632	388127.912957366\\
0.0466011650291257	388121.037463824\\
0.0467011675291882	388114.161970283\\
0.0468011700292507	388107.859434537\\
0.0469011725293132	388100.983940995\\
0.0470011750293757	388094.108447453\\
0.0471011775294382	388087.232953912\\
0.0472011800295007	388080.35746037\\
0.0473011825295632	388073.481966829\\
0.0474011850296257	388066.606473287\\
0.0475011875296882	388059.730979746\\
0.0476011900297507	388052.855486204\\
0.0477011925298132	388045.979992662\\
0.0478011950298757	388039.104499121\\
0.0479011975299382	388031.656047784\\
0.0480012000300008	388024.780554243\\
0.0481012025300633	388017.905060701\\
0.0482012050301258	388011.029567159\\
0.0483012075301883	388003.581115823\\
0.0484012100302508	387996.705622281\\
0.0485012125303133	387989.83012874\\
0.0486012150303758	387982.381677403\\
0.0487012175304383	387975.506183861\\
0.0488012200305008	387968.057732525\\
0.0489012225305633	387961.182238983\\
0.0490012250306258	387953.733787646\\
0.0491012275306883	387946.858294105\\
0.0492012300307508	387939.409842768\\
0.0493012325308133	387931.961391431\\
0.0494012350308758	387925.08589789\\
0.0495012375309383	387917.637446553\\
0.0496012400310008	387910.188995216\\
0.0497012425310633	387902.74054388\\
0.0498012450311258	387895.865050338\\
0.0499012475311883	387888.416599001\\
0.0500012500312508	387880.968147665\\
0.0501012525313133	387873.519696328\\
0.0502012550313758	387866.071244991\\
0.0503012575314383	387858.622793655\\
0.0504012600315008	387851.174342318\\
0.0505012625315633	387843.725890981\\
0.0506012650316258	387836.277439645\\
0.0507012675316883	387828.828988308\\
0.0508012700317508	387821.380536971\\
0.0509012725318133	387813.359127839\\
0.0510012750318758	387805.910676503\\
0.0511012775319383	387798.462225166\\
0.0512012800320008	387791.013773829\\
0.0513012825320633	387782.992364697\\
0.0514012850321258	387775.543913361\\
0.0515012875321883	387768.095462024\\
0.0516012900322508	387760.074052892\\
0.0517012925323133	387752.625601555\\
0.0518012950323758	387744.604192424\\
0.0519012975324383	387737.155741087\\
0.0520013000325008	387729.134331955\\
0.0521013025325633	387721.685880618\\
0.0522013050326258	387713.664471486\\
0.0523013075326883	387705.643062355\\
0.0524013100327508	387698.194611018\\
0.0525013125328133	387690.173201886\\
0.0526013150328758	387682.151792754\\
0.0527013175329383	387674.703341418\\
0.0528013200330008	387666.681932286\\
0.0529013225330633	387658.660523154\\
0.0530013250331258	387650.639114022\\
0.0531013275331883	387642.61770489\\
0.0532013300332508	387634.596295758\\
0.0533013325333133	387626.574886627\\
0.0534013350333758	387618.553477495\\
0.0535013375334383	387610.532068363\\
0.0536013400335008	387602.510659231\\
0.0537013425335633	387594.489250099\\
0.0538013450336258	387586.467840967\\
0.0539013475336883	387578.446431836\\
0.0540013500337508	387570.425022704\\
0.0541013525338133	387561.830655777\\
0.0542013550338758	387553.809246645\\
0.0543013575339383	387545.787837513\\
0.0544013600340008	387537.766428381\\
0.0545013625340634	387529.172061454\\
0.0546013650341259	387521.150652323\\
0.0547013675341884	387512.556285396\\
0.0548013700342509	387504.534876264\\
0.0549013725343134	387495.940509337\\
0.0550013750343759	387487.919100205\\
0.0551013775344384	387479.324733278\\
0.0552013800345009	387471.303324146\\
0.0553013825345634	387462.708957219\\
0.0554013850346259	387454.687548087\\
0.0555013875346884	387446.09318116\\
0.0556013900347509	387437.498814233\\
0.0557013925348134	387428.904447307\\
0.0558013950348759	387420.883038175\\
0.0559013975349384	387412.288671248\\
0.0560014000350009	387403.694304321\\
0.0561014025350634	387395.099937394\\
0.0562014050351259	387386.505570467\\
0.0563014075351884	387377.91120354\\
0.0564014100352509	387369.316836613\\
0.0565014125353134	387360.722469686\\
0.0566014150353759	387352.128102759\\
0.0567014175354384	387343.533735832\\
0.0568014200355009	387334.939368905\\
0.0569014225355634	387326.345001978\\
0.0570014250356259	387317.750635051\\
0.0571014275356884	387308.583310329\\
0.0572014300357509	387299.988943402\\
0.0573014325358134	387291.394576475\\
0.0574014350358759	387282.800209548\\
0.0575014375359384	387273.632884826\\
0.0576014400360009	387265.038517899\\
0.0577014425360634	387255.871193177\\
0.0578014450361259	387247.27682625\\
0.0579014475361884	387238.682459323\\
0.0580014500362509	387229.515134601\\
0.0581014525363134	387220.920767674\\
0.0582014550363759	387211.753442952\\
0.0583014575364384	387202.58611823\\
0.0584014600365009	387193.991751303\\
0.0585014625365634	387184.824426581\\
0.0586014650366259	387175.657101859\\
0.0587014675366884	387167.062734932\\
0.0588014700367509	387157.89541021\\
0.0589014725368134	387148.728085488\\
0.0590014750368759	387139.560760765\\
0.0591014775369384	387130.393436043\\
0.0592014800370009	387121.226111321\\
0.0593014825370634	387112.631744394\\
0.0594014850371259	387103.464419672\\
0.0595014875371884	387094.29709495\\
0.0596014900372509	387085.129770228\\
0.0597014925373134	387075.962445506\\
0.0598014950373759	387066.222162989\\
0.0599014975374384	387057.054838267\\
0.0600015000375009	387047.887513545\\
0.0601015025375634	387038.720188822\\
0.0602015050376259	387029.5528641\\
0.0603015075376884	387019.812581583\\
0.0604015100377509	387010.645256861\\
0.0605015125378134	387001.477932139\\
0.0606015150378759	386992.310607417\\
0.0607015175379384	386982.5703249\\
0.0608015200380009	386973.403000178\\
0.0609015225380635	386963.66271766\\
0.061001525038126	386954.495392938\\
0.0611015275381885	386944.755110421\\
0.061201530038251	386935.587785699\\
0.0613015325383135	386925.847503182\\
0.061401535038376	386916.68017846\\
0.0615015375384385	386906.939895942\\
0.061601540038501	386897.199613425\\
0.0617015425385635	386887.459330908\\
0.061801545038626	386878.292006186\\
0.0619015475386885	386868.551723669\\
0.062001550038751	386858.811441151\\
0.0621015525388135	386849.071158634\\
0.062201555038876	386839.330876117\\
0.0623015575389385	386830.163551395\\
0.062401560039001	386820.423268878\\
0.0625015625390635	386810.68298636\\
0.062601565039126	386800.942703843\\
0.0627015675391885	386791.202421326\\
0.062801570039251	386781.462138809\\
0.0629015725393135	386771.148898496\\
0.063001575039376	386761.408615979\\
0.0631015775394385	386751.668333462\\
0.063201580039501	386741.928050945\\
0.0633015825395635	386732.187768427\\
0.063401585039626	386721.874528115\\
0.0635015875396885	386712.134245598\\
0.063601590039751	386702.393963081\\
0.0637015925398135	386692.080722768\\
0.063801595039876	386682.340440251\\
0.0639015975399385	386672.600157734\\
0.064001600040001	386662.286917421\\
0.0641016025400635	386652.546634904\\
0.064201605040126	386642.233394592\\
0.0643016075401885	386632.493112075\\
0.064401610040251	386622.179871762\\
0.0645016125403135	386611.86663145\\
0.064601615040376	386602.126348933\\
0.0647016175404385	386591.81310862\\
0.064801620040501	386581.499868308\\
0.0649016225405635	386571.759585791\\
0.065001625040626	386561.446345478\\
0.0651016275406885	386551.133105166\\
0.065201630040751	386540.819864854\\
0.0653016325408135	386530.506624541\\
0.065401635040876	386520.193384229\\
0.0655016375409385	386509.880143917\\
0.065601640041001	386499.566903604\\
0.0657016425410635	386489.253663292\\
0.065801645041126	386478.94042298\\
0.0659016475411885	386468.627182667\\
0.066001650041251	386458.313942355\\
0.0661016525413135	386448.000702042\\
0.066201655041376	386437.68746173\\
0.0663016575414385	386427.374221418\\
0.066401660041501	386416.48802331\\
0.0665016625415635	386406.174782998\\
0.066601665041626	386395.861542686\\
0.0667016675416885	386384.975344578\\
0.066801670041751	386374.662104266\\
0.0669016725418135	386364.348863953\\
0.067001675041876	386353.462665846\\
0.0671016775419385	386343.149425534\\
0.0672016800420011	386332.263227426\\
0.0673016825420635	386321.949987114\\
0.0674016850421261	386311.063789006\\
0.0675016875421885	386300.750548694\\
0.0676016900422511	386289.864350586\\
0.0677016925423136	386278.978152479\\
0.0678016950423761	386268.664912167\\
0.0679016975424386	386257.778714059\\
0.0680017000425011	386246.892515952\\
0.0681017025425636	386236.006317844\\
0.0682017050426261	386225.120119737\\
0.0683017075426886	386214.806879424\\
0.0684017100427511	386203.920681317\\
0.0685017125428136	386193.034483209\\
0.0686017150428761	386182.148285102\\
0.0687017175429386	386171.262086994\\
0.0688017200430011	386160.375888887\\
0.0689017225430636	386149.489690779\\
0.0690017250431261	386138.603492672\\
0.0691017275431886	386127.717294564\\
0.0692017300432511	386116.258138662\\
0.0693017325433136	386105.371940554\\
0.0694017350433761	386094.485742447\\
0.0695017375434386	386083.599544339\\
0.0696017400435011	386072.713346232\\
0.0697017425435636	386061.254190329\\
0.0698017450436261	386050.367992222\\
0.0699017475436886	386039.481794114\\
0.0700017500437511	386028.022638212\\
0.0701017525438136	386017.136440104\\
0.0702017550438761	386005.677284201\\
0.0703017575439386	385994.791086094\\
0.0704017600440011	385983.331930191\\
0.0705017625440636	385972.445732084\\
0.0706017650441261	385960.986576181\\
0.0707017675441886	385949.527420279\\
0.0708017700442511	385938.641222171\\
0.0709017725443136	385927.182066269\\
0.0710017750443761	385915.722910366\\
0.0711017775444386	385904.836712259\\
0.0712017800445011	385893.377556356\\
0.0713017825445636	385881.918400453\\
0.0714017850446261	385870.459244551\\
0.0715017875446886	385859.000088648\\
0.0716017900447511	385847.540932745\\
0.0717017925448136	385836.081776843\\
0.0718017950448761	385824.62262094\\
0.0719017975449386	385813.163465038\\
0.0720018000450011	385801.704309135\\
0.0721018025450636	385790.245153232\\
0.0722018050451261	385778.78599733\\
0.0723018075451886	385767.326841427\\
0.0724018100452511	385755.867685525\\
0.0725018125453136	385743.835571827\\
0.0726018150453761	385732.376415924\\
0.0727018175454386	385720.917260021\\
0.0728018200455011	385709.458104119\\
0.0729018225455636	385697.425990421\\
0.0730018250456261	385685.966834519\\
0.0731018275456886	385673.934720821\\
0.0732018300457511	385662.475564918\\
0.0733018325458136	385651.016409016\\
0.0734018350458761	385638.984295318\\
0.0735018375459386	385627.525139415\\
0.0736018400460011	385615.493025717\\
0.0737018425460637	385603.46091202\\
0.0738018450461261	385592.001756117\\
0.0739018475461887	385579.969642419\\
0.0740018500462511	385567.937528722\\
0.0741018525463137	385556.478372819\\
0.0742018550463762	385544.446259121\\
0.0743018575464387	385532.414145424\\
0.0744018600465012	385520.382031726\\
0.0745018625465637	385508.349918028\\
0.0746018650466262	385496.890762125\\
0.0747018675466887	385484.858648428\\
0.0748018700467512	385472.82653473\\
0.0749018725468137	385460.794421032\\
0.0750018750468762	385448.762307334\\
0.0751018775469387	385436.730193637\\
0.0752018800470012	385424.125122144\\
0.0753018825470637	385412.093008446\\
0.0754018850471262	385400.060894748\\
0.0755018875471887	385388.02878105\\
0.0756018900472512	385375.996667353\\
0.0757018925473137	385363.964553655\\
0.0758018950473762	385351.359482162\\
0.0759018975474387	385339.327368464\\
0.0760019000475012	385327.295254767\\
0.0761019025475637	385314.690183274\\
0.0762019050476262	385302.658069576\\
0.0763019075476887	385290.052998083\\
0.0764019100477512	385278.020884385\\
0.0765019125478137	385265.415812892\\
0.0766019150478762	385253.383699195\\
0.0767019175479387	385240.778627702\\
0.0768019200480012	385228.746514004\\
0.0769019225480637	385216.141442511\\
0.0770019250481262	385203.536371018\\
0.0771019275481887	385191.504257321\\
0.0772019300482512	385178.899185828\\
0.0773019325483137	385166.294114335\\
0.0774019350483762	385153.689042842\\
0.0775019375484387	385141.656929144\\
0.0776019400485012	385129.051857651\\
0.0777019425485637	385116.446786158\\
0.0778019450486262	385103.841714666\\
0.0779019475486887	385091.236643173\\
0.0780019500487512	385078.63157168\\
0.0781019525488137	385066.026500187\\
0.0782019550488762	385053.421428694\\
0.0783019575489387	385040.816357201\\
0.0784019600490012	385028.211285708\\
0.0785019625490637	385015.606214215\\
0.0786019650491262	385003.001142723\\
0.0787019675491887	384989.823113435\\
0.0788019700492512	384977.218041942\\
0.0789019725493137	384964.612970449\\
0.0790019750493762	384951.434941161\\
0.0791019775494387	384938.829869668\\
0.0792019800495012	384926.224798175\\
0.0793019825495637	384913.046768887\\
0.0794019850496262	384900.441697394\\
0.0795019875496887	384887.263668106\\
0.0796019900497512	384874.658596613\\
0.0797019925498137	384861.480567325\\
0.0798019950498762	384848.875495832\\
0.0799019975499387	384835.697466544\\
0.0800020000500013	384823.092395052\\
0.0801020025500638	384809.914365763\\
0.0802020050501263	384796.736336476\\
0.0803020075501888	384784.131264983\\
0.0804020100502513	384770.953235695\\
0.0805020125503138	384757.775206407\\
0.0806020150503763	384744.597177119\\
0.0807020175504388	384731.419147831\\
0.0808020200505013	384718.814076338\\
0.0809020225505638	384705.63604705\\
0.0810020250506263	384692.458017762\\
0.0811020275506888	384679.279988474\\
0.0812020300507513	384666.101959186\\
0.0813020325508138	384652.923929898\\
0.0814020350508763	384639.74590061\\
0.0815020375509388	384625.994913526\\
0.0816020400510013	384612.816884239\\
0.0817020425510638	384599.63885495\\
0.0818020450511263	384586.460825662\\
0.0819020475511888	384573.282796374\\
0.0820020500512513	384559.531809291\\
0.0821020525513138	384546.353780003\\
0.0822020550513763	384533.175750715\\
0.0823020575514388	384519.424763632\\
0.0824020600515013	384506.246734344\\
0.0825020625515638	384493.068705056\\
0.0826020650516263	384479.317717973\\
0.0827020675516888	384466.139688685\\
0.0828020700517513	384452.388701602\\
0.0829020725518138	384439.210672314\\
0.0830020750518763	384425.459685231\\
0.0831020775519388	384411.708698148\\
0.0832020800520013	384398.53066886\\
0.0833020825520638	384384.779681776\\
0.0834020850521263	384371.028694693\\
0.0835020875521888	384357.850665405\\
0.0836020900522513	384344.099678322\\
0.0837020925523138	384330.348691239\\
0.0838020950523763	384316.597704156\\
0.0839020975524388	384302.846717073\\
0.0840021000525013	384289.09572999\\
0.0841021025525638	384275.917700702\\
0.0842021050526263	384262.166713618\\
0.0843021075526888	384248.415726535\\
0.0844021100527513	384234.664739452\\
0.0845021125528138	384220.913752369\\
0.0846021150528763	384206.589807491\\
0.0847021175529388	384192.838820408\\
0.0848021200530013	384179.087833324\\
0.0849021225530638	384165.336846241\\
0.0850021250531263	384151.585859158\\
0.0851021275531888	384137.26191428\\
0.0852021300532513	384123.510927197\\
0.0853021325533138	384109.759940114\\
0.0854021350533763	384096.00895303\\
0.0855021375534388	384081.685008152\\
0.0856021400535013	384067.934021069\\
0.0857021425535638	384053.610076191\\
0.0858021450536263	384039.859089108\\
0.0859021475536888	384025.535144229\\
0.0860021500537514	384011.784157146\\
0.0861021525538138	383997.460212268\\
0.0862021550538764	383983.709225185\\
0.0863021575539388	383969.385280307\\
0.0864021600540014	383955.061335428\\
0.0865021625540639	383941.310348345\\
0.0866021650541264	383926.986403467\\
0.0867021675541889	383912.662458589\\
0.0868021700542514	383898.33851371\\
0.0869021725543139	383884.587526627\\
0.0870021750543764	383870.263581749\\
0.0871021775544389	383855.939636871\\
0.0872021800545014	383841.615691992\\
0.0873021825545639	383827.291747114\\
0.0874021850546264	383812.967802236\\
0.0875021875546889	383798.643857358\\
0.0876021900547514	383784.319912479\\
0.0877021925548139	383769.995967601\\
0.0878021950548764	383755.672022723\\
0.0879021975549389	383741.348077845\\
0.0880022000550014	383726.451175171\\
0.0881022025550639	383712.127230293\\
0.0882022050551264	383697.803285415\\
0.0883022075551889	383683.479340536\\
0.0884022100552514	383668.582437863\\
0.0885022125553139	383654.258492985\\
0.0886022150553764	383639.934548106\\
0.0887022175554389	383625.037645433\\
0.0888022200555014	383610.713700555\\
0.0889022225555639	383595.816797881\\
0.0890022250556264	383581.492853003\\
0.0891022275556889	383566.59595033\\
0.0892022300557514	383552.272005451\\
0.0893022325558139	383537.375102778\\
0.0894022350558764	383523.0511579\\
0.0895022375559389	383508.154255226\\
0.0896022400560014	383493.257352553\\
0.0897022425560639	383478.933407675\\
0.0898022450561264	383464.036505001\\
0.0899022475561889	383449.139602328\\
0.0900022500562514	383434.242699654\\
0.0901022525563139	383419.345796981\\
0.0902022550563764	383405.021852103\\
0.0903022575564389	383390.124949429\\
0.0904022600565014	383375.228046756\\
0.0905022625565639	383360.331144083\\
0.0906022650566264	383345.434241409\\
0.0907022675566889	383330.537338736\\
0.0908022700567514	383315.640436062\\
0.0909022725568139	383300.743533389\\
0.0910022750568764	383285.27367292\\
0.0911022775569389	383270.376770247\\
0.0912022800570014	383255.479867574\\
0.0913022825570639	383240.5829649\\
0.0914022850571264	383225.686062227\\
0.0915022875571889	383210.216201758\\
0.0916022900572514	383195.319299085\\
0.0917022925573139	383180.422396411\\
0.0918022950573764	383164.952535943\\
0.0919022975574389	383150.055633269\\
0.0920023000575014	383135.158730596\\
0.0921023025575639	383119.688870128\\
0.0922023050576264	383104.791967454\\
0.0923023075576889	383089.322106986\\
0.0924023100577514	383073.852246517\\
0.092502312557814	383058.955343844\\
0.0926023150578764	383043.485483375\\
0.092702317557939	383028.588580702\\
0.0928023200580014	383013.118720233\\
0.092902322558064	382997.648859765\\
0.0930023250581265	382982.178999296\\
0.093102327558189	382967.282096623\\
0.0932023300582515	382951.812236154\\
0.093302332558314	382936.342375686\\
0.0934023350583765	382920.872515217\\
0.093502337558439	382905.402654749\\
0.0936023400585015	382889.93279428\\
0.093702342558564	382874.462933812\\
0.0938023450586265	382858.993073343\\
0.093902347558689	382843.523212874\\
0.0940023500587515	382828.053352406\\
0.094102352558814	382812.583491937\\
0.0942023550588765	382797.113631469\\
0.094302357558939	382781.643771\\
0.0944023600590015	382766.173910532\\
0.094502362559064	382750.131092268\\
0.0946023650591265	382734.6612318\\
0.094702367559189	382719.191371331\\
0.0948023700592515	382703.721510863\\
0.094902372559314	382687.678692599\\
0.0950023750593765	382672.20883213\\
0.095102377559439	382656.166013867\\
0.0952023800595015	382640.696153398\\
0.095302382559564	382624.653335135\\
0.0954023850596265	382609.183474666\\
0.095502387559689	382593.140656402\\
0.0956023900597515	382577.670795934\\
0.095702392559814	382561.62797767\\
0.0958023950598765	382546.158117202\\
0.095902397559939	382530.115298938\\
0.0960024000600015	382514.072480674\\
0.096102402560064	382498.602620206\\
0.0962024050601265	382482.559801942\\
0.096302407560189	382466.516983678\\
0.0964024100602515	382450.474165415\\
0.096502412560314	382434.431347151\\
0.0966024150603765	382418.388528887\\
0.096702417560439	382402.918668419\\
0.0968024200605015	382386.875850155\\
0.096902422560564	382370.833031892\\
0.0970024250606265	382354.790213628\\
0.097102427560689	382338.747395364\\
0.0972024300607515	382322.704577101\\
0.097302432560814	382306.088801042\\
0.0974024350608765	382290.045982778\\
0.097502437560939	382274.003164514\\
0.0976024400610015	382257.960346251\\
0.097702442561064	382241.917527987\\
0.0978024450611265	382225.301751928\\
0.097902447561189	382209.258933665\\
0.0980024500612515	382193.216115401\\
0.098102452561314	382176.600339342\\
0.0982024550613765	382160.557521079\\
0.098302457561439	382144.514702815\\
0.0984024600615015	382127.898926756\\
0.098502462561564	382111.856108492\\
0.0986024650616265	382095.240332434\\
0.098702467561689	382079.19751417\\
0.0988024700617515	382062.581738111\\
0.098902472561814	382046.538919848\\
0.0990024750618766	382029.923143789\\
0.099102477561939	382013.30736773\\
0.0992024800620016	381997.264549466\\
0.099302482562064	381980.648773407\\
0.0994024850621266	381964.032997349\\
0.0995024875621891	381947.41722129\\
0.0996024900622516	381930.801445231\\
0.0997024925623141	381914.758626967\\
0.0998024950623766	381898.142850909\\
0.0999024975624391	381881.52707485\\
0.100002500062502	381864.911298791\\
0.100102502562564	381848.295522732\\
0.100202505062627	381831.679746674\\
0.100302507562689	381815.063970615\\
0.100402510062752	381798.448194556\\
0.100502512562814	381781.832418497\\
0.100602515062877	381765.216642438\\
0.100702517562939	381748.027908584\\
0.100802520063002	381731.412132526\\
0.100902522563064	381714.796356467\\
0.101002525063127	381698.180580408\\
0.101102527563189	381680.991846554\\
0.101202530063252	381664.376070495\\
0.101302532563314	381647.760294436\\
0.101402535063377	381630.571560583\\
0.101502537563439	381613.955784524\\
0.101602540063502	381597.340008465\\
0.101702542563564	381580.151274611\\
0.101802545063627	381563.535498552\\
0.101902547563689	381546.346764698\\
0.102002550063752	381529.158030844\\
0.102102552563814	381512.542254786\\
0.102202555063877	381495.353520932\\
0.102302557563939	381478.737744873\\
0.102402560064002	381461.549011019\\
0.102502562564064	381444.360277165\\
0.102602565064127	381427.171543311\\
0.102702567564189	381410.555767252\\
0.102802570064252	381393.367033398\\
0.102902572564314	381376.178299544\\
0.103002575064377	381358.989565691\\
0.103102577564439	381341.800831837\\
0.103202580064502	381324.612097983\\
0.103302582564564	381307.423364129\\
0.103402585064627	381290.234630275\\
0.103502587564689	381273.045896421\\
0.103602590064752	381255.857162567\\
0.103702592564814	381238.668428713\\
0.103802595064877	381221.479694859\\
0.103902597564939	381204.290961005\\
0.104002600065002	381187.102227151\\
0.104102602565064	381169.340535502\\
0.104202605065127	381152.151801648\\
0.104302607565189	381134.963067794\\
0.104402610065252	381117.77433394\\
0.104502612565314	381100.012642291\\
0.104602615065377	381082.823908438\\
0.104702617565439	381065.062216788\\
0.104802620065502	381047.873482934\\
0.104902622565564	381030.684749081\\
0.105002625065627	381012.923057432\\
0.105102627565689	380995.734323578\\
0.105202630065752	380977.972631929\\
0.105302632565814	380960.21094028\\
0.105402635065877	380943.022206426\\
0.105502637565939	380925.260514777\\
0.105602640066002	380907.498823127\\
0.105702642566064	380890.310089273\\
0.105802645066127	380872.548397624\\
0.105902647566189	380854.786705975\\
0.106002650066252	380837.025014326\\
0.106102652566314	380819.836280472\\
0.106202655066377	380802.074588823\\
0.106302657566439	380784.312897174\\
0.106402660066502	380766.551205525\\
0.106502662566564	380748.789513876\\
0.106602665066627	380731.027822227\\
0.106702667566689	380713.266130578\\
0.106802670066752	380695.504438929\\
0.106902672566814	380677.74274728\\
0.107002675066877	380659.981055631\\
0.107102677566939	380642.219363982\\
0.107202680067002	380623.884714538\\
0.107302682567064	380606.123022889\\
0.107402685067127	380588.36133124\\
0.107502687567189	380570.599639591\\
0.107602690067252	380552.264990146\\
0.107702692567314	380534.503298497\\
0.107802695067377	380516.741606848\\
0.107902697567439	380498.406957404\\
0.108002700067502	380480.645265755\\
0.108102702567564	380462.310616311\\
0.108202705067627	380444.548924662\\
0.108302707567689	380426.214275218\\
0.108402710067752	380408.452583568\\
0.108502712567814	380390.117934124\\
0.108602715067877	380372.356242475\\
0.108702717567939	380354.021593031\\
0.108802720068002	380335.686943587\\
0.108902722568064	380317.925251938\\
0.109002725068127	380299.590602494\\
0.109102727568189	380281.255953049\\
0.109202730068252	380262.921303605\\
0.109302732568314	380245.159611956\\
0.109402735068377	380226.824962512\\
0.109502737568439	380208.490313068\\
0.109602740068502	380190.155663624\\
0.109702742568564	380171.821014179\\
0.109802745068627	380153.486364735\\
0.109902747568689	380135.151715291\\
0.110002750068752	380116.817065847\\
0.110102752568814	380098.482416403\\
0.110202755068877	380080.147766959\\
0.110302757568939	380061.813117514\\
0.110402760069002	380043.47846807\\
0.110502762569064	380024.570860831\\
0.110602765069127	380006.236211387\\
0.110702767569189	379987.901561942\\
0.110802770069252	379969.566912498\\
0.110902772569314	379950.659305259\\
0.111002775069377	379932.324655815\\
0.111102777569439	379913.990006371\\
0.111202780069502	379895.082399131\\
0.111302782569564	379876.747749687\\
0.111402785069627	379857.840142448\\
0.111502787569689	379839.505493004\\
0.111602790069752	379820.597885764\\
0.111702792569814	379802.26323632\\
0.111802795069877	379783.355629081\\
0.111902797569939	379765.020979637\\
0.112002800070002	379746.113372397\\
0.112102802570064	379727.205765158\\
0.112202805070127	379708.871115714\\
0.112302807570189	379689.963508474\\
0.112402810070252	379671.055901235\\
0.112502812570314	379652.148293996\\
0.112602815070377	379633.240686757\\
0.112702817570439	379614.906037312\\
0.112802820070502	379595.998430073\\
0.112902822570564	379577.090822834\\
0.113002825070627	379558.183215594\\
0.113102827570689	379539.275608355\\
0.113202830070752	379520.368001116\\
0.113302832570814	379501.460393876\\
0.113402835070877	379482.552786637\\
0.113502837570939	379463.645179398\\
0.113602840071002	379444.164614363\\
0.113702842571064	379425.257007124\\
0.113802845071127	379406.349399885\\
0.113902847571189	379387.441792645\\
0.114002850071252	379368.534185406\\
0.114102852571314	379349.053620372\\
0.114202855071377	379330.146013132\\
0.114302857571439	379311.238405893\\
0.114402860071502	379291.757840859\\
0.114502862571564	379272.850233619\\
0.114602865071627	379253.369668585\\
0.114702867571689	379234.462061345\\
0.114802870071752	379214.981496311\\
0.114902872571814	379196.073889072\\
0.115002875071877	379176.593324037\\
0.115102877571939	379157.685716798\\
0.115202880072002	379138.205151764\\
0.115302882572064	379118.724586729\\
0.115402885072127	379099.81697949\\
0.115502887572189	379080.336414455\\
0.115602890072252	379060.855849421\\
0.115702892572314	379041.948242181\\
0.115802895072377	379022.467677147\\
0.115902897572439	379002.987112113\\
0.116002900072502	378983.506547078\\
0.116102902572564	378964.025982044\\
0.116202905072627	378944.545417009\\
0.116302907572689	378925.064851975\\
0.116402910072752	378905.58428694\\
0.116502912572814	378886.103721906\\
0.116602915072877	378866.623156871\\
0.116702917572939	378847.142591837\\
0.116802920073002	378827.662026803\\
0.116902922573064	378808.181461768\\
0.117002925073127	378788.700896734\\
0.117102927573189	378768.647373904\\
0.117202930073252	378749.16680887\\
0.117302932573314	378729.686243835\\
0.117402935073377	378710.205678801\\
0.117502937573439	378690.152155971\\
0.117602940073502	378670.671590937\\
0.117702942573564	378651.191025902\\
0.117802945073627	378631.137503073\\
0.117902947573689	378611.656938038\\
0.118002950073752	378591.603415209\\
0.118102952573814	378572.122850174\\
0.118202955073877	378552.069327345\\
0.118302957573939	378532.58876231\\
0.118402960074002	378512.535239481\\
0.118502962574064	378493.054674446\\
0.118602965074127	378473.001151617\\
0.118702967574189	378452.947628787\\
0.118802970074252	378432.894105957\\
0.118902972574314	378413.413540923\\
0.119002975074377	378393.360018093\\
0.119102977574439	378373.306495264\\
0.119202980074502	378353.252972434\\
0.119302982574564	378333.199449605\\
0.119402985074627	378313.71888457\\
0.119502987574689	378293.665361741\\
0.119602990074752	378273.611838911\\
0.119702992574814	378253.558316082\\
0.119802995074877	378233.504793252\\
0.119902997574939	378213.451270422\\
0.120003000075002	378193.397747593\\
0.120103002575064	378172.771266968\\
0.120203005075127	378152.717744138\\
0.120303007575189	378132.664221309\\
0.120403010075252	378112.610698479\\
0.120503012575314	378092.55717565\\
0.120603015075377	378072.50365282\\
0.120703017575439	378051.877172195\\
0.120803020075502	378031.823649366\\
0.120903022575564	378011.770126536\\
0.121003025075627	377991.143645912\\
0.121103027575689	377971.090123082\\
0.121203030075752	377950.463642457\\
0.121303032575814	377930.410119628\\
0.121403035075877	377909.783639003\\
0.121503037575939	377889.730116173\\
0.121603040076002	377869.103635549\\
0.121703042576064	377849.050112719\\
0.121803045076127	377828.423632094\\
0.121903047576189	377807.79715147\\
0.122003050076252	377787.74362864\\
0.122103052576314	377767.117148015\\
0.122203055076377	377746.490667391\\
0.122303057576439	377726.437144561\\
0.122403060076502	377705.810663936\\
0.122503062576564	377685.184183312\\
0.122603065076627	377664.557702687\\
0.122703067576689	377643.931222062\\
0.122803070076752	377623.304741438\\
0.122903072576814	377602.678260813\\
0.123003075076877	377582.051780188\\
0.123103077576939	377561.425299563\\
0.123203080077002	377540.798818939\\
0.123303082577064	377520.172338314\\
0.123403085077127	377499.545857689\\
0.123503087577189	377478.919377065\\
0.123603090077252	377458.29289644\\
0.123703092577314	377437.666415815\\
0.123803095077377	377417.039935191\\
0.123903097577439	377395.840496771\\
0.124003100077502	377375.214016146\\
0.124103102577564	377354.587535521\\
0.124203105077627	377333.388097101\\
0.124303107577689	377312.761616477\\
0.124403110077752	377292.135135852\\
0.124503112577814	377270.935697432\\
0.124603115077877	377250.309216807\\
0.124703117577939	377229.109778388\\
0.124803120078002	377208.483297763\\
0.124903122578064	377187.283859343\\
0.125003125078127	377166.657378718\\
0.125103127578189	377145.457940299\\
0.125203130078252	377124.258501879\\
0.125303132578314	377103.632021254\\
0.125403135078377	377082.432582834\\
0.125503137578439	377061.233144414\\
0.125603140078502	377040.60666379\\
0.125703142578564	377019.40722537\\
0.125803145078627	376998.20778695\\
0.125903147578689	376977.00834853\\
0.126003150078752	376955.80891011\\
0.126103152578814	376934.60947169\\
0.126203155078877	376913.41003327\\
0.126303157578939	376892.210594851\\
0.126403160079002	376871.011156431\\
0.126503162579064	376849.811718011\\
0.126603165079127	376828.612279591\\
0.126703167579189	376807.412841171\\
0.126803170079252	376786.213402751\\
0.126903172579314	376765.013964332\\
0.127003175079377	376743.814525912\\
0.127103177579439	376722.615087492\\
0.127203180079502	376700.842691277\\
0.127303182579564	376679.643252857\\
0.127403185079627	376658.443814437\\
0.127503187579689	376637.244376017\\
0.127603190079752	376615.471979803\\
0.127703192579814	376594.272541383\\
0.127803195079877	376572.500145168\\
0.12790319757994	376551.300706748\\
0.128003200080002	376530.101268328\\
0.128103202580064	376508.328872113\\
0.128203205080127	376487.129433693\\
0.12830320758019	376465.357037478\\
0.128403210080252	376443.584641263\\
0.128503212580314	376422.385202843\\
0.128603215080377	376400.612806628\\
0.12870321758044	376378.840410413\\
0.128803220080502	376357.640971994\\
0.128903222580565	376335.868575779\\
0.129003225080627	376314.096179564\\
0.12910322758069	376292.896741144\\
0.129203230080752	376271.124344929\\
0.129303232580815	376249.351948714\\
0.129403235080877	376227.579552499\\
0.12950323758094	376205.807156284\\
0.129603240081002	376184.034760069\\
0.129703242581065	376162.262363854\\
0.129803245081127	376140.489967639\\
0.12990324758119	376118.717571424\\
0.130003250081252	376096.945175209\\
0.130103252581315	376075.172778994\\
0.130203255081377	376053.400382779\\
0.13030325758144	376031.627986564\\
0.130403260081502	376009.855590349\\
0.130503262581565	375987.510236339\\
0.130603265081627	375965.737840124\\
0.13070326758169	375943.965443909\\
0.130803270081752	375922.193047694\\
0.130903272581815	375899.847693684\\
0.131003275081877	375878.075297469\\
0.13110327758194	375855.729943459\\
0.131203280082002	375833.957547244\\
0.131303282582065	375812.185151029\\
0.131403285082127	375789.839797019\\
0.13150328758219	375768.067400804\\
0.131603290082252	375745.722046794\\
0.131703292582315	375723.949650579\\
0.131803295082377	375701.604296569\\
0.13190329758244	375679.258942559\\
0.132003300082502	375657.486546344\\
0.132103302582565	375635.141192334\\
0.132203305082627	375612.795838324\\
0.13230330758269	375591.023442109\\
0.132403310082752	375568.678088098\\
0.132503312582815	375546.332734088\\
0.132603315082877	375523.987380078\\
0.13270331758294	375501.642026068\\
0.132803320083002	375479.296672058\\
0.132903322583065	375457.524275843\\
0.133003325083127	375435.178921833\\
0.13310332758319	375412.833567823\\
0.133203330083252	375390.488213813\\
0.133303332583315	375368.142859803\\
0.133403335083377	375345.797505793\\
0.13350333758344	375323.452151782\\
0.133603340083502	375300.533839977\\
0.133703342583565	375278.188485967\\
0.133803345083627	375255.843131957\\
0.13390334758369	375233.497777947\\
0.134003350083752	375211.152423937\\
0.134103352583815	375188.234112132\\
0.134203355083877	375165.888758121\\
0.13430335758394	375143.543404111\\
0.134403360084002	375120.625092306\\
0.134503362584065	375098.279738296\\
0.134603365084127	375075.934384286\\
0.13470336758419	375053.016072481\\
0.134803370084252	375030.670718471\\
0.134903372584315	375007.752406665\\
0.135003375084377	374985.407052655\\
0.13510337758444	374962.48874085\\
0.135203380084502	374940.14338684\\
0.135303382584565	374917.225075035\\
0.135403385084627	374894.306763229\\
0.13550338758469	374871.961409219\\
0.135603390084752	374849.043097414\\
0.135703392584815	374826.124785609\\
0.135803395084877	374803.206473804\\
0.13590339758494	374780.861119794\\
0.136003400085002	374757.942807988\\
0.136103402585065	374735.024496183\\
0.136203405085127	374712.106184378\\
0.13630340758519	374689.187872573\\
0.136403410085252	374666.269560767\\
0.136503412585315	374643.351248962\\
0.136603415085377	374620.432937157\\
0.13670341758544	374597.514625352\\
0.136803420085502	374574.596313546\\
0.136903422585565	374551.678001741\\
0.137003425085627	374528.759689936\\
0.13710342758569	374505.841378131\\
0.137203430085752	374482.923066326\\
0.137303432585815	374460.00475452\\
0.137403435085877	374436.51348492\\
0.13750343758594	374413.595173115\\
0.137603440086002	374390.67686131\\
0.137703442586065	374367.185591709\\
0.137803445086127	374344.267279904\\
0.13790344758619	374321.348968099\\
0.138003450086252	374297.857698498\\
0.138103452586315	374274.939386693\\
0.138203455086377	374251.448117093\\
0.13830345758644	374228.529805287\\
0.138403460086502	374205.038535687\\
0.138503462586565	374182.120223882\\
0.138603465086627	374158.628954282\\
0.13870346758669	374135.710642476\\
0.138803470086752	374112.219372876\\
0.138903472586815	374088.728103276\\
0.139003475086877	374065.80979147\\
0.13910347758694	374042.31852187\\
0.139203480087002	374018.82725227\\
0.139303482587065	373995.908940464\\
0.139403485087127	373972.417670864\\
0.13950348758719	373948.926401264\\
0.139603490087252	373925.435131663\\
0.139703492587315	373901.943862063\\
0.139803495087377	373878.452592463\\
0.13990349758744	373854.961322862\\
0.140003500087502	373831.470053262\\
0.140103502587565	373807.978783661\\
0.140203505087627	373784.487514061\\
0.14030350758769	373760.996244461\\
0.140403510087752	373737.50497486\\
0.140503512587815	373714.01370526\\
0.140603515087877	373690.52243566\\
0.14070351758794	373667.031166059\\
0.140803520088002	373642.966938664\\
0.140903522588065	373619.475669063\\
0.141003525088127	373595.984399463\\
0.14110352758819	373572.493129863\\
0.141203530088252	373548.428902467\\
0.141303532588315	373524.937632867\\
0.141403535088377	373501.446363266\\
0.14150353758844	373477.382135871\\
0.141603540088502	373453.890866271\\
0.141703542588565	373429.826638875\\
0.141803545088627	373406.335369275\\
0.14190354758869	373382.271141879\\
0.142003550088752	373358.779872279\\
0.142103552588815	373334.715644883\\
0.142203555088877	373310.651417488\\
0.14230355758894	373287.160147887\\
0.142403560089002	373263.095920492\\
0.142503562589065	373239.031693097\\
0.142603565089127	373215.540423496\\
0.14270356758919	373191.476196101\\
0.142803570089252	373167.411968705\\
0.142903572589315	373143.34774131\\
0.143003575089377	373119.283513914\\
0.14310357758944	373095.792244314\\
0.143203580089502	373071.728016918\\
0.143303582589565	373047.663789523\\
0.143403585089627	373023.599562127\\
0.14350358758969	372999.535334732\\
0.143603590089752	372975.471107336\\
0.143703592589815	372951.406879941\\
0.143803595089877	372927.342652545\\
0.14390359758994	372903.27842515\\
0.144003600090002	372878.641239959\\
0.144103602590065	372854.577012564\\
0.144203605090127	372830.512785168\\
0.14430360759019	372806.448557773\\
0.144403610090252	372782.384330377\\
0.144503612590315	372757.747145187\\
0.144603615090377	372733.682917791\\
0.14470361759044	372709.618690396\\
0.144803620090502	372684.981505205\\
0.144903622590565	372660.91727781\\
0.145003625090627	372636.853050414\\
0.14510362759069	372612.215865223\\
0.145203630090752	372588.151637828\\
0.145303632590815	372563.514452637\\
0.145403635090877	372539.450225242\\
0.14550363759094	372514.813040051\\
0.145603640091002	372490.748812656\\
0.145703642591065	372466.111627465\\
0.145803645091127	372441.474442274\\
0.14590364759119	372417.410214879\\
0.146003650091252	372392.773029688\\
0.146103652591315	372368.135844498\\
0.146203655091377	372343.498659307\\
0.14630365759144	372319.434431912\\
0.146403660091502	372294.797246721\\
0.146503662591565	372270.16006153\\
0.146603665091627	372245.52287634\\
0.14670366759169	372220.885691149\\
0.146803670091752	372196.248505958\\
0.146903672591815	372171.611320768\\
0.147003675091877	372146.974135577\\
0.14710367759194	372122.336950387\\
0.147203680092002	372097.699765196\\
0.147303682592065	372073.062580005\\
0.147403685092127	372048.425394815\\
0.14750368759219	372023.788209624\\
0.147603690092252	371999.151024433\\
0.147703692592315	371974.513839243\\
0.147803695092377	371949.303696257\\
0.14790369759244	371924.666511066\\
0.148003700092502	371900.029325876\\
0.148103702592565	371875.392140685\\
0.148203705092627	371850.181997699\\
0.14830370759269	371825.544812509\\
0.148403710092752	371800.907627318\\
0.148503712592815	371775.697484332\\
0.148603715092877	371751.060299142\\
0.14870371759294	371725.850156156\\
0.148803720093002	371701.212970965\\
0.148903722593065	371676.00282798\\
0.149003725093127	371651.365642789\\
0.14910372759319	371626.155499803\\
0.149203730093252	371600.945356818\\
0.149303732593315	371576.308171627\\
0.149403735093377	371551.098028641\\
0.14950373759344	371525.887885655\\
0.149603740093502	371501.250700465\\
0.149703742593565	371476.040557479\\
0.149803745093627	371450.830414493\\
0.14990374759369	371425.620271507\\
0.150003750093752	371400.983086317\\
0.150103752593815	371375.772943331\\
0.150203755093877	371350.562800345\\
0.15030375759394	371325.35265736\\
0.150403760094002	371300.142514374\\
0.150503762594065	371274.932371388\\
0.150603765094127	371249.722228402\\
0.15070376759419	371224.512085417\\
0.150803770094252	371199.301942431\\
0.150903772594315	371174.091799445\\
0.151003775094377	371148.881656459\\
0.15110377759444	371123.098555678\\
0.151203780094502	371097.888412693\\
0.151303782594565	371072.678269707\\
0.151403785094627	371047.468126721\\
0.15150378759469	371022.257983735\\
0.151603790094752	370996.474882954\\
0.151703792594815	370971.264739969\\
0.151803795094877	370946.054596983\\
0.15190379759494	370920.271496202\\
0.152003800095002	370895.061353216\\
0.152103802595065	370869.278252435\\
0.152203805095127	370844.06810945\\
0.15230380759519	370818.857966464\\
0.152403810095252	370793.074865683\\
0.152503812595315	370767.291764902\\
0.152603815095377	370742.081621916\\
0.15270381759544	370716.298521136\\
0.152803820095502	370691.08837815\\
0.152903822595565	370665.305277369\\
0.153003825095627	370639.522176588\\
0.15310382759569	370614.312033602\\
0.153203830095752	370588.528932821\\
0.153303832595815	370562.74583204\\
0.153403835095877	370536.96273126\\
0.15350383759594	370511.752588274\\
0.153603840096002	370485.969487493\\
0.153703842596065	370460.186386712\\
0.153803845096127	370434.403285931\\
0.15390384759619	370408.62018515\\
0.154003850096252	370382.837084369\\
0.154103852596315	370357.053983588\\
0.154203855096377	370331.270882808\\
0.15430385759644	370305.487782027\\
0.154403860096502	370279.704681246\\
0.154503862596565	370253.921580465\\
0.154603865096627	370228.138479684\\
0.15470386759669	370202.355378903\\
0.154803870096752	370175.999320327\\
0.154903872596815	370150.216219546\\
0.155003875096877	370124.433118765\\
0.15510387759694	370098.650017984\\
0.155203880097002	370072.293959408\\
0.155303882597065	370046.510858628\\
0.155403885097127	370020.727757847\\
0.15550388759719	369994.371699271\\
0.155603890097252	369968.58859849\\
0.155703892597315	369942.232539914\\
0.155803895097377	369916.449439133\\
0.15590389759744	369890.666338352\\
0.156003900097502	369864.310279776\\
0.156103902597565	369837.9542212\\
0.156203905097627	369812.171120419\\
0.15630390759769	369785.815061843\\
0.156403910097752	369760.031961062\\
0.156503912597815	369733.675902486\\
0.156603915097877	369707.31984391\\
0.15670391759794	369681.536743129\\
0.156803920098002	369655.180684553\\
0.156903922598065	369628.824625977\\
0.157003925098127	369602.468567401\\
0.15710392759819	369576.112508825\\
0.157203930098252	369550.329408044\\
0.157303932598315	369523.973349468\\
0.157403935098377	369497.617290892\\
0.15750393759844	369471.261232316\\
0.157603940098502	369444.90517374\\
0.157703942598565	369418.549115164\\
0.157803945098627	369392.193056588\\
0.15790394759869	369365.836998012\\
0.158003950098752	369339.480939436\\
0.158103952598815	369313.12488086\\
0.158203955098877	369286.768822284\\
0.15830395759894	369259.839805913\\
0.158403960099002	369233.483747337\\
0.158503962599065	369207.127688761\\
0.158603965099127	369180.771630185\\
0.15870396759919	369154.415571609\\
0.158803970099252	369127.486555238\\
0.158903972599315	369101.130496662\\
0.159003975099377	369074.774438086\\
0.15910397759944	369047.845421715\\
0.159203980099502	369021.489363139\\
0.159303982599565	368994.560346767\\
0.159403985099627	368968.204288191\\
0.15950398759969	368941.27527182\\
0.159603990099752	368914.919213244\\
0.159703992599815	368887.990196873\\
0.159803995099877	368861.634138297\\
0.15990399759994	368834.705121926\\
0.160004000100003	368808.34906335\\
0.160104002600065	368781.420046979\\
0.160204005100128	368754.491030608\\
0.16030400760019	368728.134972032\\
0.160404010100253	368701.20595566\\
0.160504012600315	368674.276939289\\
0.160604015100378	368647.347922918\\
0.16070401760044	368620.418906547\\
0.160804020100503	368594.062847971\\
0.160904022600565	368567.1338316\\
0.161004025100628	368540.204815229\\
0.16110402760069	368513.275798858\\
0.161204030100753	368486.346782486\\
0.161304032600815	368459.417766115\\
0.161404035100878	368432.488749744\\
0.16150403760094	368405.559733373\\
0.161604040101003	368378.630717002\\
0.161704042601065	368351.701700631\\
0.161804045101128	368324.772684259\\
0.16190404760119	368297.843667888\\
0.162004050101253	368270.341693722\\
0.162104052601315	368243.412677351\\
0.162204055101378	368216.48366098\\
0.16230405760144	368189.554644609\\
0.162404060101503	368162.052670442\\
0.162504062601565	368135.123654071\\
0.162604065101628	368108.1946377\\
0.16270406760169	368080.692663534\\
0.162804070101753	368053.763647163\\
0.162904072601815	368026.834630791\\
0.163004075101878	367999.332656625\\
0.16310407760194	367972.403640254\\
0.163204080102003	367944.901666088\\
0.163304082602065	367917.972649717\\
0.163404085102128	367890.47067555\\
0.16350408760219	367862.968701384\\
0.163604090102253	367836.039685013\\
0.163704092602315	367808.537710847\\
0.163804095102378	367781.608694475\\
0.16390409760244	367754.106720309\\
0.164004100102503	367726.604746143\\
0.164104102602565	367699.102771977\\
0.164204105102628	367672.173755605\\
0.16430410760269	367644.671781439\\
0.164404110102753	367617.169807273\\
0.164504112602815	367589.667833107\\
0.164604115102878	367562.16585894\\
0.16470411760294	367534.663884774\\
0.164804120103003	367507.161910608\\
0.164904122603065	367479.659936441\\
0.165004125103128	367452.157962275\\
0.16510412760319	367424.655988109\\
0.165204130103253	367397.154013943\\
0.165304132603315	367369.652039776\\
0.165404135103378	367342.15006561\\
0.16550413760344	367314.648091444\\
0.165604140103503	367287.146117278\\
0.165704142603565	367259.644143111\\
0.165804145103628	367232.142168945\\
0.16590414760369	367204.067236984\\
0.166004150103753	367176.565262817\\
0.166104152603815	367149.063288651\\
0.166204155103878	367120.98835669\\
0.16630415760394	367093.486382523\\
0.166404160104003	367065.984408357\\
0.166504162604065	367037.909476396\\
0.166604165104128	367010.407502229\\
0.16670416760419	366982.332570268\\
0.166804170104253	366954.830596102\\
0.166904172604315	366926.75566414\\
0.167004175104378	366899.253689974\\
0.16710417760444	366871.178758013\\
0.167204180104503	366843.676783846\\
0.167304182604565	366815.601851885\\
0.167404185104628	366788.099877719\\
0.16750418760469	366760.024945757\\
0.167604190104753	366731.950013796\\
0.167704192604815	366703.875081834\\
0.167804195104878	366676.373107668\\
0.16790419760494	366648.298175707\\
0.168004200105003	366620.223243745\\
0.168104202605065	366592.148311784\\
0.168204205105128	366564.073379822\\
0.16830420760519	366536.571405656\\
0.168404210105253	366508.496473695\\
0.168504212605315	366480.421541733\\
0.168604215105378	366452.346609772\\
0.16870421760544	366424.271677811\\
0.168804220105503	366396.196745849\\
0.168904222605565	366368.121813888\\
0.169004225105628	366340.046881926\\
0.16910422760569	366311.971949965\\
0.169204230105753	366283.324060208\\
0.169304232605815	366255.249128247\\
0.169404235105878	366227.174196286\\
0.16950423760594	366199.099264324\\
0.169604240106003	366171.024332363\\
0.169704242606065	366142.376442606\\
0.169804245106128	366114.301510645\\
0.16990424760619	366086.226578683\\
0.170004250106253	366057.578688927\\
0.170104252606315	366029.503756965\\
0.170204255106378	366001.428825004\\
0.17030425760644	365972.780935247\\
0.170404260106503	365944.706003286\\
0.170504262606565	365916.058113529\\
0.170604265106628	365887.983181568\\
0.17070426760669	365859.335291811\\
0.170804270106753	365831.26035985\\
0.170904272606815	365802.612470094\\
0.171004275106878	365774.537538132\\
0.17110427760694	365745.889648376\\
0.171204280107003	365717.241758619\\
0.171304282607065	365689.166826658\\
0.171404285107128	365660.518936901\\
0.17150428760719	365631.871047145\\
0.171604290107253	365603.223157388\\
0.171704292607315	365575.148225427\\
0.171804295107378	365546.50033567\\
0.17190429760744	365517.852445914\\
0.172004300107503	365489.204556157\\
0.172104302607565	365460.5566664\\
0.172204305107628	365431.908776644\\
0.17230430760769	365403.260886887\\
0.172404310107753	365374.612997131\\
0.172504312607815	365345.965107374\\
0.172604315107878	365317.317217618\\
0.17270431760794	365288.669327861\\
0.172804320108003	365260.021438105\\
0.172904322608065	365231.373548348\\
0.173004325108128	365202.725658592\\
0.17310432760819	365174.077768835\\
0.173204330108253	365145.429879079\\
0.173304332608315	365116.209031527\\
0.173404335108378	365087.56114177\\
0.17350433760844	365058.913252014\\
0.173604340108503	365030.265362257\\
0.173704342608565	365001.044514706\\
0.173804345108628	364972.396624949\\
0.17390434760869	364943.175777397\\
0.174004350108753	364914.527887641\\
0.174104352608815	364885.879997884\\
0.174204355108878	364856.659150333\\
0.17430435760894	364828.011260576\\
0.174404360109003	364798.790413024\\
0.174504362609065	364770.142523268\\
0.174604365109128	364740.921675716\\
0.17470436760919	364711.700828165\\
0.174804370109253	364683.052938408\\
0.174904372609315	364653.832090856\\
0.175004375109378	364625.1842011\\
0.17510437760944	364595.963353548\\
0.175204380109503	364566.742505996\\
0.175304382609565	364537.521658445\\
0.175404385109628	364508.873768688\\
0.17550438760969	364479.652921137\\
0.175604390109753	364450.432073585\\
0.175704392609815	364421.211226033\\
0.175804395109878	364391.990378481\\
0.17590439760994	364362.76953093\\
0.176004400110003	364333.548683378\\
0.176104402610065	364304.327835826\\
0.176204405110128	364275.106988275\\
0.17630440761019	364245.886140723\\
0.176404410110253	364216.665293171\\
0.176504412610315	364187.44444562\\
0.176604415110378	364158.223598068\\
0.17670441761044	364129.002750516\\
0.176804420110503	364099.781902965\\
0.176904422610565	364070.561055413\\
0.177004425110628	364041.340207861\\
0.17710442761069	364011.546402515\\
0.177204430110753	363982.325554963\\
0.177304432610815	363953.104707411\\
0.177404435110878	363923.88385986\\
0.17750443761094	363894.090054513\\
0.177604440111003	363864.869206961\\
0.177704442611065	363835.648359409\\
0.177804445111128	363805.854554063\\
0.17790444761119	363776.633706511\\
0.178004450111253	363746.839901164\\
0.178104452611315	363717.619053612\\
0.178204455111378	363687.825248266\\
0.17830445761144	363658.604400714\\
0.178404460111503	363628.810595367\\
0.178504462611565	363599.589747816\\
0.178604465111628	363569.795942469\\
0.17870446761169	363540.002137122\\
0.178804470111753	363510.78128957\\
0.178904472611815	363480.987484223\\
0.179004475111878	363451.193678877\\
0.17910447761194	363421.972831325\\
0.179204480112003	363392.179025978\\
0.179304482612065	363362.385220631\\
0.179404485112128	363332.591415285\\
0.17950448761219	363302.797609938\\
0.179604490112253	363273.003804591\\
0.179704492612315	363243.782957039\\
0.179804495112378	363213.989151692\\
0.17990449761244	363184.195346346\\
0.180004500112503	363154.401540999\\
0.180104502612565	363124.607735652\\
0.180204505112628	363094.813930305\\
0.18030450761269	363065.020124959\\
0.180404510112753	363035.226319612\\
0.180504512612815	363004.85955647\\
0.180604515112878	362975.065751123\\
0.18070451761294	362945.271945776\\
0.180804520113003	362915.478140429\\
0.180904522613065	362885.684335083\\
0.181004525113128	362855.317571941\\
0.18110452761319	362825.523766594\\
0.181204530113253	362795.729961247\\
0.181304532613315	362765.9361559\\
0.181404535113378	362735.569392758\\
0.18150453761344	362705.775587411\\
0.181604540113503	362675.981782065\\
0.181704542613565	362645.615018923\\
0.181804545113628	362615.821213576\\
0.18190454761369	362585.454450434\\
0.182004550113753	362555.660645087\\
0.182104552613815	362525.293881945\\
0.182204555113878	362495.500076598\\
0.18230455761394	362465.133313457\\
0.182404560114003	362434.766550315\\
0.182504562614065	362404.972744968\\
0.182604565114128	362374.605981826\\
0.18270456761419	362344.239218684\\
0.182804570114253	362314.445413337\\
0.182904572614315	362284.078650195\\
0.183004575114378	362253.711887053\\
0.18310457761444	362223.918081706\\
0.183204580114503	362193.551318565\\
0.183304582614565	362163.184555423\\
0.183404585114628	362132.817792281\\
0.18350458761469	362102.451029139\\
0.183604590114753	362072.084265997\\
0.183704592614815	362041.717502855\\
0.183804595114878	362011.350739713\\
0.18390459761494	361980.983976571\\
0.184004600115003	361950.617213429\\
0.184104602615065	361920.250450287\\
0.184204605115128	361889.883687145\\
0.18430460761519	361859.516924003\\
0.184404610115253	361829.150160861\\
0.184504612615315	361798.783397719\\
0.184604615115378	361768.416634577\\
0.18470461761544	361738.049871436\\
0.184804620115503	361707.110150498\\
0.184904622615565	361676.743387357\\
0.185004625115628	361646.376624215\\
0.18510462761569	361616.009861073\\
0.185204630115753	361585.070140136\\
0.185304632615815	361554.703376994\\
0.185404635115878	361524.336613852\\
0.18550463761594	361493.396892915\\
0.185604640116003	361463.030129773\\
0.185704642616065	361432.090408836\\
0.185804645116128	361401.723645694\\
0.18590464761619	361370.783924757\\
0.186004650116253	361340.417161615\\
0.186104652616315	361309.477440678\\
0.186204655116378	361279.110677536\\
0.18630465761644	361248.170956599\\
0.186404660116503	361217.231235662\\
0.186504662616565	361186.86447252\\
0.186604665116628	361155.924751583\\
0.18670466761669	361124.985030646\\
0.186804670116753	361094.618267504\\
0.186904672616815	361063.678546567\\
0.187004675116878	361032.738825629\\
0.18710467761694	361001.799104692\\
0.187204680117003	360970.859383755\\
0.187304682617065	360940.492620613\\
0.187404685117128	360909.552899676\\
0.18750468761719	360878.613178739\\
0.187604690117253	360847.673457802\\
0.187704692617315	360816.733736865\\
0.187804695117378	360785.794015928\\
0.18790469761744	360754.854294991\\
0.188004700117503	360723.914574054\\
0.188104702617565	360692.974853117\\
0.188204705117628	360662.03513218\\
0.18830470761769	360631.095411243\\
0.188404710117753	360600.155690306\\
0.188504712617815	360568.643011574\\
0.188604715117878	360537.703290636\\
0.18870471761794	360506.763569699\\
0.188804720118003	360475.823848762\\
0.188904722618065	360444.884127825\\
0.189004725118128	360413.371449093\\
0.18910472761819	360382.431728156\\
0.189204730118253	360351.492007219\\
0.189304732618315	360319.979328487\\
0.189404735118378	360289.03960755\\
0.18950473761844	360257.526928818\\
0.189604740118503	360226.58720788\\
0.189704742618565	360195.647486943\\
0.189804745118628	360164.134808211\\
0.18990474761869	360133.195087274\\
0.190004750118753	360101.682408542\\
0.190104752618815	360070.16972981\\
0.190204755118878	360039.230008873\\
0.19030475761894	360007.71733014\\
0.190404760119003	359976.777609203\\
0.190504762619065	359945.264930471\\
0.190604765119128	359913.752251739\\
0.19070476761919	359882.812530802\\
0.190804770119253	359851.29985207\\
0.190904772619315	359819.787173337\\
0.191004775119378	359788.274494605\\
0.19110477761944	359756.761815873\\
0.191204780119503	359725.822094936\\
0.191304782619565	359694.309416204\\
0.191404785119628	359662.796737472\\
0.191504787619691	359631.284058739\\
0.191604790119753	359599.771380007\\
0.191704792619815	359568.258701275\\
0.191804795119878	359536.746022543\\
0.191904797619941	359505.233343811\\
0.192004800120003	359473.720665078\\
0.192104802620065	359442.207986346\\
0.192204805120128	359410.695307614\\
0.192304807620191	359379.182628882\\
0.192404810120253	359347.66995015\\
0.192504812620316	359315.584313622\\
0.192604815120378	359284.07163489\\
0.192704817620441	359252.558956158\\
0.192804820120503	359221.046277426\\
0.192904822620566	359189.533598694\\
0.193004825120628	359157.447962166\\
0.193104827620691	359125.935283434\\
0.193204830120753	359094.422604702\\
0.193304832620816	359062.336968174\\
0.193404835120878	359030.824289442\\
0.193504837620941	358998.738652915\\
0.193604840121003	358967.225974183\\
0.193704842621066	358935.713295451\\
0.193804845121128	358903.627658923\\
0.193904847621191	358872.114980191\\
0.194004850121253	358840.029343664\\
0.194104852621316	358807.943707136\\
0.194204855121378	358776.431028404\\
0.194304857621441	358744.345391877\\
0.194404860121503	358712.832713145\\
0.194504862621566	358680.747076617\\
0.194604865121628	358648.66144009\\
0.194704867621691	358617.148761358\\
0.194804870121753	358585.063124831\\
0.194904872621816	358552.977488303\\
0.195004875121878	358520.891851776\\
0.195104877621941	358488.806215249\\
0.195204880122003	358457.293536516\\
0.195304882622066	358425.207899989\\
0.195404885122128	358393.122263462\\
0.195504887622191	358361.036626934\\
0.195604890122253	358328.950990407\\
0.195704892622316	358296.86535388\\
0.195804895122378	358264.779717352\\
0.195904897622441	358232.694080825\\
0.196004900122503	358200.608444298\\
0.196104902622566	358168.522807771\\
0.196204905122628	358136.437171243\\
0.196304907622691	358104.351534716\\
0.196404910122753	358072.265898188\\
0.196504912622816	358039.607303866\\
0.196604915122878	358007.521667339\\
0.196704917622941	357975.436030811\\
0.196804920123003	357943.350394284\\
0.196904922623066	357911.264757757\\
0.197004925123128	357878.606163434\\
0.197104927623191	357846.520526907\\
0.197204930123253	357814.43489038\\
0.197304932623316	357781.776296057\\
0.197404935123378	357749.69065953\\
0.197504937623441	357717.605023003\\
0.197604940123503	357684.94642868\\
0.197704942623566	357652.860792153\\
0.197804945123628	357620.20219783\\
0.197904947623691	357588.116561303\\
0.198004950123753	357555.457966981\\
0.198104952623816	357523.372330453\\
0.198204955123878	357490.713736131\\
0.198304957623941	357458.055141808\\
0.198404960124003	357425.969505281\\
0.198504962624066	357393.310910958\\
0.198604965124128	357361.225274431\\
0.198704967624191	357328.566680109\\
0.198804970124253	357295.908085786\\
0.198904972624316	357263.249491464\\
0.199004975124378	357231.163854936\\
0.199104977624441	357198.505260614\\
0.199204980124503	357165.846666292\\
0.199304982624566	357133.188071969\\
0.199404985124628	357100.529477647\\
0.199504987624691	357067.870883324\\
0.199604990124753	357035.212289002\\
0.199704992624816	357003.126652474\\
0.199804995124878	356970.468058152\\
0.199904997624941	356937.809463829\\
0.200005000125003	356905.150869507\\
0.200105002625066	356872.492275185\\
0.200205005125128	356839.260723067\\
0.200305007625191	356806.602128745\\
0.200405010125253	356773.943534422\\
0.200505012625316	356741.2849401\\
0.200605015125378	356708.626345777\\
0.200705017625441	356675.967751455\\
0.200805020125503	356643.309157132\\
0.200905022625566	356610.077605015\\
0.201005025125628	356577.419010692\\
0.201105027625691	356544.76041637\\
0.201205030125753	356511.528864252\\
0.201305032625816	356478.87026993\\
0.201405035125878	356446.211675607\\
0.201505037625941	356412.98012349\\
0.201605040126003	356380.321529167\\
0.201705042626066	356347.662934845\\
0.201805045126128	356314.431382727\\
0.201905047626191	356281.772788405\\
0.202005050126253	356248.541236287\\
0.202105052626316	356215.882641965\\
0.202205055126378	356182.651089847\\
0.202305057626441	356149.992495525\\
0.202405060126503	356116.760943407\\
0.202505062626566	356083.529391289\\
0.202605065126628	356050.870796967\\
0.202705067626691	356017.639244849\\
0.202805070126753	355984.407692732\\
0.202905072626816	355951.749098409\\
0.203005075126878	355918.517546292\\
0.203105077626941	355885.285994174\\
0.203205080127003	355852.054442057\\
0.203305082627066	355819.395847734\\
0.203405085127128	355786.164295617\\
0.203505087627191	355752.932743499\\
0.203605090127253	355719.701191381\\
0.203705092627316	355686.469639264\\
0.203805095127378	355653.238087146\\
0.203905097627441	355620.006535029\\
0.204005100127503	355586.774982911\\
0.204105102627566	355553.543430793\\
0.204205105127628	355520.311878676\\
0.204305107627691	355487.080326558\\
0.204405110127753	355453.848774441\\
0.204505112627816	355420.617222323\\
0.204605115127878	355387.385670206\\
0.204705117627941	355354.154118088\\
0.204805120128003	355320.349608175\\
0.204905122628066	355287.118056058\\
0.205005125128128	355253.88650394\\
0.205105127628191	355220.654951822\\
0.205205130128253	355187.423399705\\
0.205305132628316	355153.618889792\\
0.205405135128378	355120.387337675\\
0.205505137628441	355087.155785557\\
0.205605140128503	355053.351275644\\
0.205705142628566	355020.119723527\\
0.205805145128628	354986.315213614\\
0.205905147628691	354953.083661496\\
0.206005150128753	354919.852109379\\
0.206105152628816	354886.047599466\\
0.206205155128878	354852.816047348\\
0.206305157628941	354819.011537436\\
0.206405160129003	354785.207027523\\
0.206505162629066	354751.975475405\\
0.206605165129128	354718.170965493\\
0.206705167629191	354684.939413375\\
0.206805170129253	354651.134903462\\
0.206905172629316	354617.33039355\\
0.207005175129378	354584.098841432\\
0.207105177629441	354550.294331519\\
0.207205180129503	354516.489821607\\
0.207305182629566	354482.685311694\\
0.207405185129628	354449.453759576\\
0.207505187629691	354415.649249664\\
0.207605190129753	354381.844739751\\
0.207705192629816	354348.040229838\\
0.207805195129878	354314.235719925\\
0.207905197629941	354280.431210013\\
0.208005200130003	354246.6267001\\
0.208105202630066	354212.822190187\\
0.208205205130128	354179.017680275\\
0.208305207630191	354145.213170362\\
0.208405210130253	354111.408660449\\
0.208505212630316	354077.604150536\\
0.208605215130378	354043.799640624\\
0.208705217630441	354009.995130711\\
0.208805220130503	353976.190620798\\
0.208905222630566	353942.386110886\\
0.209005225130628	353908.581600973\\
0.209105227630691	353874.77709106\\
0.209205230130753	353840.399623352\\
0.209305232630816	353806.595113439\\
0.209405235130878	353772.790603527\\
0.209505237630941	353738.986093614\\
0.209605240131003	353704.608625906\\
0.209705242631066	353670.804115994\\
0.209805245131128	353636.999606081\\
0.209905247631191	353602.622138373\\
0.210005250131253	353568.81762846\\
0.210105252631316	353534.440160752\\
0.210205255131378	353500.63565084\\
0.210305257631441	353466.258183132\\
0.210405260131503	353432.453673219\\
0.210505262631566	353398.076205511\\
0.210605265131628	353364.271695599\\
0.210705267631691	353329.894227891\\
0.210805270131753	353296.089717978\\
0.210905272631816	353261.71225027\\
0.211005275131878	353227.334782562\\
0.211105277631941	353193.53027265\\
0.211205280132003	353159.152804942\\
0.211305282632066	353124.775337234\\
0.211405285132128	353090.970827321\\
0.211505287632191	353056.593359613\\
0.211605290132253	353022.215891905\\
0.211705292632316	352987.838424198\\
0.211805295132378	352953.46095649\\
0.211905297632441	352919.656446577\\
0.212005300132503	352885.278978869\\
0.212105302632566	352850.901511161\\
0.212205305132628	352816.524043453\\
0.212305307632691	352782.146575746\\
0.212405310132753	352747.769108038\\
0.212505312632816	352713.39164033\\
0.212605315132878	352679.014172622\\
0.212705317632941	352644.636704914\\
0.212805320133003	352610.259237206\\
0.212905322633066	352575.881769499\\
0.213005325133128	352541.504301791\\
0.213105327633191	352507.126834083\\
0.213205330133253	352472.17640858\\
0.213305332633316	352437.798940872\\
0.213405335133378	352403.421473164\\
0.213505337633441	352369.044005456\\
0.213605340133503	352334.666537748\\
0.213705342633566	352299.716112245\\
0.213805345133628	352265.338644538\\
0.213905347633691	352230.96117683\\
0.214005350133753	352196.010751327\\
0.214105352633816	352161.633283619\\
0.214205355133878	352127.255815911\\
0.214305357633941	352092.305390408\\
0.214405360134003	352057.9279227\\
0.214505362634066	352022.977497197\\
0.214605365134128	351988.600029489\\
0.214705367634191	351953.649603986\\
0.214805370134253	351919.272136279\\
0.214905372634316	351884.321710776\\
0.215005375134378	351849.944243068\\
0.215105377634441	351814.993817565\\
0.215205380134503	351780.616349857\\
0.215305382634566	351745.665924354\\
0.215405385134628	351710.715498851\\
0.215505387634691	351676.338031143\\
0.215605390134753	351641.38760564\\
0.215705392634816	351606.437180137\\
0.215805395134878	351571.486754634\\
0.215905397634941	351537.109286926\\
0.216005400135003	351502.158861423\\
0.216105402635066	351467.20843592\\
0.216205405135128	351432.258010417\\
0.216305407635191	351397.307584914\\
0.216405410135253	351362.357159411\\
0.216505412635316	351327.406733908\\
0.216605415135378	351293.029266201\\
0.216705417635441	351258.078840698\\
0.216805420135503	351223.128415195\\
0.216905422635566	351188.177989692\\
0.217005425135628	351153.227564189\\
0.217105427635691	351118.277138686\\
0.217205430135753	351082.753755388\\
0.217305432635816	351047.803329885\\
0.217405435135878	351012.852904382\\
0.217505437635941	350977.902478879\\
0.217605440136003	350942.952053376\\
0.217705442636066	350908.001627873\\
0.217805445136128	350873.05120237\\
0.217905447636191	350837.527819072\\
0.218005450136253	350802.577393569\\
0.218105452636316	350767.626968066\\
0.218205455136378	350732.103584768\\
0.218305457636441	350697.153159264\\
0.218405460136503	350662.202733762\\
0.218505462636566	350626.679350463\\
0.218605465136628	350591.72892496\\
0.218705467636691	350556.778499458\\
0.218805470136753	350521.255116159\\
0.218905472636816	350486.304690656\\
0.219005475136878	350450.781307358\\
0.219105477636941	350415.830881855\\
0.219205480137003	350380.307498557\\
0.219305482637066	350345.357073054\\
0.219405485137128	350309.833689756\\
0.219505487637191	350274.883264253\\
0.219605490137253	350239.359880955\\
0.219705492637316	350203.836497657\\
0.219805495137378	350168.886072154\\
0.219905497637441	350133.362688856\\
0.220005500137503	350097.839305558\\
0.220105502637566	350062.31592226\\
0.220205505137628	350027.365496757\\
0.220305507637691	349991.842113458\\
0.220405510137753	349956.31873016\\
0.220505512637816	349920.795346862\\
0.220605515137878	349885.271963564\\
0.220705517637941	349850.321538061\\
0.220805520138003	349814.798154763\\
0.220905522638066	349779.274771465\\
0.221005525138128	349743.751388167\\
0.221105527638191	349708.228004869\\
0.221205530138253	349672.704621571\\
0.221305532638316	349637.181238272\\
0.221405535138378	349601.657854974\\
0.221505537638441	349566.134471676\\
0.221605540138503	349530.611088378\\
0.221705542638566	349495.08770508\\
0.221805545138628	349459.564321782\\
0.221905547638691	349423.467980689\\
0.222005550138753	349387.944597391\\
0.222105552638816	349352.421214093\\
0.222205555138878	349316.897830794\\
0.222305557638941	349281.374447496\\
0.222405560139003	349245.278106403\\
0.222505562639066	349209.754723105\\
0.222605565139128	349174.231339807\\
0.222705567639191	349138.707956509\\
0.222805570139253	349102.611615415\\
0.222905572639316	349067.088232117\\
0.223005575139378	349030.991891024\\
0.223105577639441	348995.468507726\\
0.223205580139503	348959.945124428\\
0.223305582639566	348923.848783335\\
0.223405585139628	348888.325400037\\
0.223505587639691	348852.229058943\\
0.223605590139754	348816.705675645\\
0.223705592639816	348780.609334552\\
0.223805595139878	348745.085951254\\
0.223905597639941	348708.989610161\\
0.224005600140004	348672.893269067\\
0.224105602640066	348637.369885769\\
0.224205605140128	348601.273544676\\
0.224305607640191	348565.177203583\\
0.224405610140254	348529.653820285\\
0.224505612640316	348493.557479191\\
0.224605615140379	348457.461138098\\
0.224705617640441	348421.9377548\\
0.224805620140504	348385.841413707\\
0.224905622640566	348349.745072614\\
0.225005625140629	348313.64873152\\
0.225105627640691	348277.552390427\\
0.225205630140754	348241.456049334\\
0.225305632640816	348205.932666036\\
0.225405635140879	348169.836324942\\
0.225505637640941	348133.739983849\\
0.225605640141004	348097.643642756\\
0.225705642641066	348061.547301663\\
0.225805645141129	348025.450960569\\
0.225905647641191	347989.354619476\\
0.226005650141254	347953.258278383\\
0.226105652641316	347917.16193729\\
0.226205655141379	347881.065596197\\
0.226305657641441	347844.396297308\\
0.226405660141504	347808.299956215\\
0.226505662641566	347772.203615122\\
0.226605665141629	347736.107274028\\
0.226705667641691	347700.010932935\\
0.226805670141754	347663.914591842\\
0.226905672641816	347627.245292954\\
0.227005675141879	347591.14895186\\
0.227105677641941	347555.052610767\\
0.227205680142004	347518.383311879\\
0.227305682642066	347482.286970786\\
0.227405685142129	347446.190629692\\
0.227505687642191	347409.521330804\\
0.227605690142254	347373.424989711\\
0.227705692642316	347337.328648617\\
0.227805695142379	347300.659349729\\
0.227905697642441	347264.563008636\\
0.228005700142504	347227.893709747\\
0.228105702642566	347191.797368654\\
0.228205705142629	347155.128069766\\
0.228305707642691	347119.031728673\\
0.228405710142754	347082.362429784\\
0.228505712642816	347045.693130896\\
0.228605715142879	347009.596789803\\
0.228705717642941	346972.927490914\\
0.228805720143004	346936.258192026\\
0.228905722643066	346900.161850933\\
0.229005725143129	346863.492552044\\
0.229105727643191	346826.823253156\\
0.229205730143254	346790.726912063\\
0.229305732643316	346754.057613174\\
0.229405735143379	346717.388314286\\
0.229505737643441	346680.719015397\\
0.229605740143504	346644.049716509\\
0.229705742643566	346607.953375416\\
0.229805745143629	346571.284076528\\
0.229905747643691	346534.614777639\\
0.230005750143754	346497.945478751\\
0.230105752643816	346461.276179862\\
0.230205755143879	346424.606880974\\
0.230305757643941	346387.937582086\\
0.230405760144004	346351.268283197\\
0.230505762644066	346314.598984309\\
0.230605765144129	346277.92968542\\
0.230705767644191	346241.260386532\\
0.230805770144254	346204.591087644\\
0.230905772644316	346167.921788755\\
0.231005775144379	346130.679532072\\
0.231105777644441	346094.010233184\\
0.231205780144504	346057.340934295\\
0.231305782644566	346020.671635407\\
0.231405785144629	345984.002336518\\
0.231505787644691	345946.760079835\\
0.231605790144754	345910.090780947\\
0.231705792644816	345873.421482058\\
0.231805795144879	345836.75218317\\
0.231905797644941	345799.509926486\\
0.232005800145004	345762.840627598\\
0.232105802645066	345726.17132871\\
0.232205805145129	345688.929072026\\
0.232305807645191	345652.259773138\\
0.232405810145254	345615.017516454\\
0.232505812645316	345578.348217566\\
0.232605815145379	345541.105960882\\
0.232705817645441	345504.436661994\\
0.232805820145504	345467.19440531\\
0.232905822645566	345430.525106422\\
0.233005825145629	345393.282849738\\
0.233105827645691	345356.61355085\\
0.233205830145754	345319.371294167\\
0.233305832645816	345282.129037483\\
0.233405835145879	345245.459738595\\
0.233505837645941	345208.217481911\\
0.233605840146004	345170.975225228\\
0.233705842646066	345134.305926339\\
0.233805845146129	345097.063669656\\
0.233905847646191	345059.821412972\\
0.234005850146254	345022.579156289\\
0.234105852646316	344985.9098574\\
0.234205855146379	344948.667600717\\
0.234305857646441	344911.425344033\\
0.234405860146504	344874.18308735\\
0.234505862646566	344836.940830666\\
0.234605865146629	344799.698573983\\
0.234705867646691	344762.456317299\\
0.234805870146754	344725.214060616\\
0.234905872646816	344687.971803932\\
0.235005875146879	344650.729547249\\
0.235105877646941	344613.487290565\\
0.235205880147004	344576.245033882\\
0.235305882647066	344539.002777198\\
0.235405885147129	344501.760520515\\
0.235505887647191	344464.518263831\\
0.235605890147254	344427.276007148\\
0.235705892647316	344390.033750464\\
0.235805895147379	344352.791493781\\
0.235905897647441	344314.976279302\\
0.236005900147504	344277.734022619\\
0.236105902647566	344240.491765935\\
0.236205905147629	344203.249509252\\
0.236305907647691	344166.007252568\\
0.236405910147754	344128.19203809\\
0.236505912647816	344090.949781406\\
0.236605915147879	344053.707524723\\
0.236705917647941	344015.892310244\\
0.236805920148004	343978.650053561\\
0.236905922648066	343940.834839082\\
0.237005925148129	343903.592582398\\
0.237105927648191	343866.350325715\\
0.237205930148254	343828.535111236\\
0.237305932648316	343791.292854553\\
0.237405935148379	343753.477640074\\
0.237505937648441	343716.235383391\\
0.237605940148504	343678.420168912\\
0.237705942648566	343641.177912229\\
0.237805945148629	343603.36269775\\
0.237905947648691	343565.547483271\\
0.238005950148754	343528.305226588\\
0.238105952648816	343490.490012109\\
0.238205955148879	343452.67479763\\
0.238305957648941	343415.432540947\\
0.238405960149004	343377.617326468\\
0.238505962649066	343339.80211199\\
0.238605965149129	343302.559855306\\
0.238705967649191	343264.744640828\\
0.238805970149254	343226.929426349\\
0.238905972649316	343189.11421187\\
0.239005975149379	343151.298997392\\
0.239105977649441	343113.483782913\\
0.239205980149504	343076.241526229\\
0.239305982649566	343038.426311751\\
0.239405985149629	343000.611097272\\
0.239505987649691	342962.795882794\\
0.239605990149754	342924.980668315\\
0.239705992649816	342887.165453836\\
0.239805995149879	342849.350239358\\
0.239905997649941	342811.535024879\\
0.240006000150004	342773.7198104\\
0.240106002650066	342735.904595922\\
0.240206005150129	342698.089381443\\
0.240306007650191	342660.274166964\\
0.240406010150254	342621.885994691\\
0.240506012650316	342584.070780212\\
0.240606015150379	342546.255565733\\
0.240706017650441	342508.440351255\\
0.240806020150504	342470.625136776\\
0.240906022650566	342432.809922298\\
0.241006025150629	342394.421750024\\
0.241106027650691	342356.606535545\\
0.241206030150754	342318.791321066\\
0.241306032650816	342280.403148793\\
0.241406035150879	342242.587934314\\
0.241506037650941	342204.772719835\\
0.241606040151004	342166.384547562\\
0.241706042651066	342128.569333083\\
0.241806045151129	342090.754118604\\
0.241906047651191	342052.365946331\\
0.242006050151254	342014.550731852\\
0.242106052651316	341976.162559578\\
0.242206055151379	341938.3473451\\
0.242306057651441	341899.959172826\\
0.242406060151504	341862.143958347\\
0.242506062651566	341823.755786074\\
0.242606065151629	341785.3676138\\
0.242706067651691	341747.552399321\\
0.242806070151754	341709.164227047\\
0.242906072651816	341671.349012569\\
0.243006075151879	341632.960840295\\
0.243106077651941	341594.572668021\\
0.243206080152004	341556.757453542\\
0.243306082652066	341518.369281269\\
0.243406085152129	341479.981108995\\
0.243506087652191	341441.592936721\\
0.243606090152254	341403.777722243\\
0.243706092652316	341365.389549969\\
0.243806095152379	341327.001377695\\
0.243906097652441	341288.613205421\\
0.244006100152504	341250.225033147\\
0.244106102652566	341211.836860874\\
0.244206105152629	341173.4486886\\
0.244306107652691	341135.633474121\\
0.244406110152754	341097.245301848\\
0.244506112652816	341058.857129574\\
0.244606115152879	341020.4689573\\
0.244706117652941	340982.080785026\\
0.244806120153004	340943.692612753\\
0.244906122653066	340905.304440479\\
0.245006125153129	340866.34331041\\
0.245106127653191	340827.955138136\\
0.245206130153254	340789.566965862\\
0.245306132653316	340751.178793589\\
0.245406135153379	340712.790621315\\
0.245506137653441	340674.402449041\\
0.245606140153504	340636.014276767\\
0.245706142653566	340597.053146698\\
0.245806145153629	340558.664974425\\
0.245906147653691	340520.276802151\\
0.246006150153754	340481.888629877\\
0.246106152653816	340442.927499808\\
0.246206155153879	340404.539327534\\
0.246306157653941	340366.151155261\\
0.246406160154004	340327.190025192\\
0.246506162654066	340288.801852918\\
0.246606165154129	340250.413680644\\
0.246706167654191	340211.452550575\\
0.246806170154254	340173.064378302\\
0.246906172654316	340134.103248233\\
0.247006175154379	340095.715075959\\
0.247106177654441	340056.75394589\\
0.247206180154504	340018.365773616\\
0.247306182654566	339979.404643547\\
0.247406185154629	339941.016471274\\
0.247506187654691	339902.055341205\\
0.247606190154754	339863.667168931\\
0.247706192654816	339824.706038862\\
0.247806195154879	339785.744908793\\
0.247906197654941	339747.356736519\\
0.248006200155004	339708.39560645\\
0.248106202655066	339669.434476382\\
0.248206205155129	339631.046304108\\
0.248306207655191	339592.085174039\\
0.248406210155254	339553.12404397\\
0.248506212655316	339514.162913901\\
0.248606215155379	339475.774741627\\
0.248706217655441	339436.813611558\\
0.248806220155504	339397.85248149\\
0.248906222655566	339358.891351421\\
0.249006225155629	339319.930221352\\
0.249106227655691	339280.969091283\\
0.249206230155754	339242.007961214\\
0.249306232655816	339203.61978894\\
0.249406235155879	339164.658658871\\
0.249506237655941	339125.697528802\\
0.249606240156004	339086.736398734\\
0.249706242656066	339047.775268665\\
0.249806245156129	339008.814138596\\
0.249906247656191	338969.853008527\\
0.250006250156254	338930.318920663\\
0.250106252656316	338891.357790594\\
0.250206255156379	338852.396660525\\
0.250306257656441	338813.435530456\\
0.250406260156504	338774.474400387\\
0.250506262656566	338735.513270318\\
0.250606265156629	338696.552140249\\
0.250706267656691	338657.018052385\\
0.250806270156754	338618.056922317\\
0.250906272656816	338579.095792248\\
0.251006275156879	338540.134662179\\
0.251106277656941	338500.600574315\\
0.251206280157004	338461.639444246\\
0.251306282657066	338422.678314177\\
0.251406285157129	338383.144226313\\
0.251506287657191	338344.183096244\\
0.251606290157254	338305.221966175\\
0.251706292657316	338265.687878311\\
0.251806295157379	338226.726748242\\
0.251906297657441	338187.192660378\\
0.252006300157504	338148.231530309\\
0.252106302657566	338108.697442445\\
0.252206305157629	338069.736312376\\
0.252306307657691	338030.202224512\\
0.252406310157754	337991.241094443\\
0.252506312657816	337951.707006579\\
0.252606315157879	337912.74587651\\
0.252706317657941	337873.211788646\\
0.252806320158004	337834.250658578\\
0.252906322658066	337794.716570713\\
0.253006325158129	337755.18248285\\
0.253106327658191	337716.221352781\\
0.253206330158254	337676.687264917\\
0.253306332658316	337637.153177053\\
0.253406335158379	337597.619089189\\
0.253506337658441	337558.65795912\\
0.253606340158504	337519.123871256\\
0.253706342658566	337479.589783392\\
0.253806345158629	337440.055695528\\
0.253906347658691	337400.521607664\\
0.254006350158754	337361.560477595\\
0.254106352658816	337322.026389731\\
0.254206355158879	337282.492301867\\
0.254306357658941	337242.958214003\\
0.254406360159004	337203.424126139\\
0.254506362659066	337163.890038275\\
0.254606365159129	337124.355950411\\
0.254706367659191	337084.821862546\\
0.254806370159254	337045.287774682\\
0.254906372659316	337005.753686818\\
0.255006375159379	336966.219598954\\
0.255106377659441	336926.68551109\\
0.255206380159504	336887.151423226\\
0.255306382659567	336847.617335362\\
0.255406385159629	336808.083247498\\
0.255506387659691	336767.976201839\\
0.255606390159754	336728.442113975\\
0.255706392659816	336688.908026111\\
0.255806395159879	336649.373938247\\
0.255906397659941	336609.839850383\\
0.256006400160004	336569.732804724\\
0.256106402660067	336530.19871686\\
0.256206405160129	336490.664628996\\
0.256306407660192	336451.130541132\\
0.256406410160254	336411.023495473\\
0.256506412660316	336371.489407609\\
0.256606415160379	336331.955319745\\
0.256706417660441	336291.848274085\\
0.256806420160504	336252.314186221\\
0.256906422660567	336212.207140562\\
0.257006425160629	336172.673052698\\
0.257106427660692	336133.138964834\\
0.257206430160754	336093.031919175\\
0.257306432660817	336053.497831311\\
0.257406435160879	336013.390785652\\
0.257506437660942	335973.856697788\\
0.257606440161004	335933.749652129\\
0.257706442661067	335893.642606469\\
0.257806445161129	335854.108518605\\
0.257906447661192	335814.001472946\\
0.258006450161254	335774.467385082\\
0.258106452661317	335734.360339423\\
0.258206455161379	335694.253293764\\
0.258306457661442	335654.7192059\\
0.258406460161504	335614.612160241\\
0.258506462661567	335574.505114582\\
0.258606465161629	335534.398068922\\
0.258706467661692	335494.863981058\\
0.258806470161754	335454.756935399\\
0.258906472661817	335414.64988974\\
0.259006475161879	335374.542844081\\
0.259106477661942	335335.008756217\\
0.259206480162004	335294.901710558\\
0.259306482662067	335254.794664899\\
0.259406485162129	335214.687619239\\
0.259506487662192	335174.58057358\\
0.259606490162254	335134.473527921\\
0.259706492662317	335094.366482262\\
0.259806495162379	335054.259436603\\
0.259906497662442	335014.152390944\\
0.260006500162504	334974.045345285\\
0.260106502662567	334933.938299625\\
0.260206505162629	334893.831253966\\
0.260306507662692	334853.724208307\\
0.260406510162754	334813.617162648\\
0.260506512662817	334773.510116989\\
0.260606515162879	334733.40307133\\
0.260706517662942	334693.29602567\\
0.260806520163004	334653.188980011\\
0.260906522663067	334612.508976557\\
0.261006525163129	334572.401930898\\
0.261106527663192	334532.294885239\\
0.261206530163254	334492.187839579\\
0.261306532663317	334451.507836125\\
0.261406535163379	334411.400790466\\
0.261506537663442	334371.293744807\\
0.261606540163504	334331.186699148\\
0.261706542663567	334290.506695693\\
0.261806545163629	334250.399650034\\
0.261906547663692	334210.292604375\\
0.262006550163754	334169.612600921\\
0.262106552663817	334129.505555262\\
0.262206555163879	334088.825551807\\
0.262306557663942	334048.718506148\\
0.262406560164004	334008.611460489\\
0.262506562664067	333967.931457035\\
0.262606565164129	333927.824411376\\
0.262706567664192	333887.144407921\\
0.262806570164254	333847.037362262\\
0.262906572664317	333806.357358808\\
0.263006575164379	333765.677355354\\
0.263106577664442	333725.570309694\\
0.263206580164504	333684.89030624\\
0.263306582664567	333644.783260581\\
0.263406585164629	333604.103257127\\
0.263506587664692	333563.423253672\\
0.263606590164754	333523.316208013\\
0.263706592664817	333482.636204559\\
0.263806595164879	333441.956201105\\
0.263906597664942	333401.849155446\\
0.264006600165004	333361.169151991\\
0.264106602665067	333320.489148537\\
0.264206605165129	333279.809145083\\
0.264306607665192	333239.129141628\\
0.264406610165254	333199.022095969\\
0.264506612665317	333158.342092515\\
0.264606615165379	333117.662089061\\
0.264706617665442	333076.982085606\\
0.264806620165504	333036.302082152\\
0.264906622665567	332995.622078698\\
0.265006625165629	332954.942075243\\
0.265106627665692	332914.262071789\\
0.265206630165754	332873.582068335\\
0.265306632665817	332832.902064881\\
0.265406635165879	332792.222061426\\
0.265506637665942	332751.542057972\\
0.265606640166004	332710.862054518\\
0.265706642666067	332670.182051063\\
0.265806645166129	332629.502047609\\
0.265906647666192	332588.822044155\\
0.266006650166254	332548.142040701\\
0.266106652666317	332507.462037246\\
0.266206655166379	332466.782033792\\
0.266306657666442	332426.102030338\\
0.266406660166504	332384.849069088\\
0.266506662666567	332344.169065634\\
0.266606665166629	332303.48906218\\
0.266706667666692	332262.809058725\\
0.266806670166754	332221.556097476\\
0.266906672666817	332180.876094022\\
0.267006675166879	332140.196090567\\
0.267106677666942	332099.516087113\\
0.267206680167004	332058.263125864\\
0.267306682667067	332017.583122409\\
0.267406685167129	331976.903118955\\
0.267506687667192	331935.650157706\\
0.267606690167254	331894.970154251\\
0.267706692667317	331853.717193002\\
0.267806695167379	331813.037189548\\
0.267906697667442	331771.784228298\\
0.268006700167504	331731.104224844\\
0.268106702667567	331689.851263595\\
0.268206705167629	331649.17126014\\
0.268306707667692	331607.918298891\\
0.268406710167754	331567.238295437\\
0.268506712667817	331525.985334187\\
0.268606715167879	331485.305330733\\
0.268706717667942	331444.052369483\\
0.268806720168004	331403.372366029\\
0.268906722668067	331362.11940478\\
0.269006725168129	331320.86644353\\
0.269106727668192	331280.186440076\\
0.269206730168254	331238.933478827\\
0.269306732668317	331197.680517577\\
0.269406735168379	331156.427556328\\
0.269506737668442	331115.747552874\\
0.269606740168504	331074.494591624\\
0.269706742668567	331033.241630375\\
0.269806745168629	330991.988669125\\
0.269906747668692	330950.735707876\\
0.270006750168754	330910.055704422\\
0.270106752668817	330868.802743172\\
0.270206755168879	330827.549781923\\
0.270306757668942	330786.296820673\\
0.270406760169004	330745.043859424\\
0.270506762669067	330703.790898174\\
0.270606765169129	330662.537936925\\
0.270706767669192	330621.284975676\\
0.270806770169254	330580.032014426\\
0.270906772669317	330538.779053177\\
0.271006775169379	330497.526091927\\
0.271106777669442	330456.273130678\\
0.271206780169504	330415.020169429\\
0.271306782669567	330373.767208179\\
0.271406785169629	330332.51424693\\
0.271506787669692	330291.26128568\\
0.271606790169754	330250.008324431\\
0.271706792669817	330208.755363181\\
0.271806795169879	330167.502401932\\
0.271906797669942	330125.676482887\\
0.272006800170004	330084.423521638\\
0.272106802670067	330043.170560389\\
0.272206805170129	330001.917599139\\
0.272306807670192	329960.66463789\\
0.272406810170254	329918.838718845\\
0.272506812670317	329877.585757596\\
0.272606815170379	329836.332796346\\
0.272706817670442	329794.506877302\\
0.272806820170504	329753.253916052\\
0.272906822670567	329712.000954803\\
0.273006825170629	329670.175035758\\
0.273106827670692	329628.922074509\\
0.273206830170754	329587.66911326\\
0.273306832670817	329545.843194215\\
0.273406835170879	329504.590232966\\
0.273506837670942	329462.764313921\\
0.273606840171004	329421.511352672\\
0.273706842671067	329379.685433627\\
0.273806845171129	329338.432472378\\
0.273906847671192	329296.606553333\\
0.274006850171254	329255.353592084\\
0.274106852671317	329213.527673039\\
0.274206855171379	329172.27471179\\
0.274306857671442	329130.448792745\\
0.274406860171504	329088.622873701\\
0.274506862671567	329047.369912451\\
0.274606865171629	329005.543993407\\
0.274706867671692	328963.718074362\\
0.274806870171754	328922.465113113\\
0.274906872671817	328880.639194068\\
0.275006875171879	328838.813275024\\
0.275106877671942	328797.560313774\\
0.275206880172004	328755.73439473\\
0.275306882672067	328713.908475685\\
0.275406885172129	328672.082556641\\
0.275506887672192	328630.256637596\\
0.275606890172254	328589.003676347\\
0.275706892672317	328547.177757302\\
0.275806895172379	328505.351838258\\
0.275906897672442	328463.525919213\\
0.276006900172504	328421.700000168\\
0.276106902672567	328379.874081124\\
0.276206905172629	328338.048162079\\
0.276306907672692	328296.222243035\\
0.276406910172754	328254.39632399\\
0.276506912672817	328212.570404946\\
0.276606915172879	328170.744485901\\
0.276706917672942	328128.918566857\\
0.276806920173004	328087.092647812\\
0.276906922673067	328045.266728767\\
0.277006925173129	328003.440809723\\
0.277106927673192	327961.614890678\\
0.277206930173254	327919.788971634\\
0.277306932673317	327877.963052589\\
0.277406935173379	327836.137133545\\
0.277506937673442	327794.3112145\\
0.277606940173504	327752.485295456\\
0.277706942673567	327710.086418616\\
0.277806945173629	327668.260499571\\
0.277906947673692	327626.434580527\\
0.278006950173754	327584.608661482\\
0.278106952673817	327542.782742438\\
0.278206955173879	327500.383865598\\
0.278306957673942	327458.557946553\\
0.278406960174004	327416.732027509\\
0.278506962674067	327374.333150669\\
0.278606965174129	327332.507231625\\
0.278706967674192	327290.68131258\\
0.278806970174254	327248.28243574\\
0.278906972674317	327206.456516696\\
0.279006975174379	327164.630597651\\
0.279106977674442	327122.231720812\\
0.279206980174504	327080.405801767\\
0.279306982674567	327038.006924927\\
0.279406985174629	326996.181005883\\
0.279506987674692	326953.782129043\\
0.279606990174754	326911.956209999\\
0.279706992674817	326869.557333159\\
0.279806995174879	326827.731414114\\
0.279906997674942	326785.332537275\\
0.280007000175004	326743.50661823\\
0.280107002675067	326701.107741391\\
0.280207005175129	326658.708864551\\
0.280307007675192	326616.882945506\\
0.280407010175254	326574.484068667\\
0.280507012675317	326532.085191827\\
0.280607015175379	326490.259272782\\
0.280707017675442	326447.860395943\\
0.280807020175504	326405.461519103\\
0.280907022675567	326363.635600058\\
0.281007025175629	326321.236723219\\
0.281107027675692	326278.837846379\\
0.281207030175754	326236.438969539\\
0.281307032675817	326194.613050495\\
0.281407035175879	326152.214173655\\
0.281507037675942	326109.815296816\\
0.281607040176004	326067.416419976\\
0.281707042676067	326025.017543136\\
0.281807045176129	325982.618666297\\
0.281907047676192	325940.219789457\\
0.282007050176254	325898.393870412\\
0.282107052676317	325855.994993573\\
0.282207055176379	325813.596116733\\
0.282307057676442	325771.197239893\\
0.282407060176504	325728.798363054\\
0.282507062676567	325686.399486214\\
0.282607065176629	325644.000609374\\
0.282707067676692	325601.601732534\\
0.282807070176754	325559.202855695\\
0.282907072676817	325516.803978855\\
0.283007075176879	325474.405102015\\
0.283107077676942	325431.433267381\\
0.283207080177004	325389.034390541\\
0.283307082677067	325346.635513701\\
0.283407085177129	325304.236636862\\
0.283507087677192	325261.837760022\\
0.283607090177254	325219.438883182\\
0.283707092677317	325177.040006343\\
0.283807095177379	325134.068171708\\
0.283907097677442	325091.669294868\\
0.284007100177504	325049.270418028\\
0.284107102677567	325006.871541189\\
0.284207105177629	324963.899706554\\
0.284307107677692	324921.500829714\\
0.284407110177754	324879.101952875\\
0.284507112677817	324836.13011824\\
0.284607115177879	324793.7312414\\
0.284707117677942	324751.33236456\\
0.284807120178004	324708.360529925\\
0.284907122678067	324665.961653086\\
0.285007125178129	324623.562776246\\
0.285107127678192	324580.590941611\\
0.285207130178254	324538.192064772\\
0.285307132678317	324495.220230137\\
0.285407135178379	324452.821353297\\
0.285507137678442	324409.849518662\\
0.285607140178504	324367.450641823\\
0.285707142678567	324324.478807188\\
0.285807145178629	324282.079930348\\
0.285907147678692	324239.108095713\\
0.286007150178754	324196.709218874\\
0.286107152678817	324153.737384239\\
0.286207155178879	324110.765549604\\
0.286307157678942	324068.366672764\\
0.286407160179004	324025.39483813\\
0.286507162679067	323982.99596129\\
0.286607165179129	323940.024126655\\
0.286707167679192	323897.05229202\\
0.286807170179255	323854.653415181\\
0.286907172679317	323811.681580546\\
0.287007175179379	323768.709745911\\
0.287107177679442	323725.737911276\\
0.287207180179504	323683.339034437\\
0.287307182679567	323640.367199802\\
0.287407185179629	323597.395365167\\
0.287507187679692	323554.423530532\\
0.287607190179755	323511.451695897\\
0.287707192679817	323468.479861262\\
0.28780719517988	323426.080984423\\
0.287907197679942	323383.109149788\\
0.288007200180004	323340.137315153\\
0.288107202680067	323297.165480518\\
0.288207205180129	323254.193645884\\
0.288307207680192	323211.221811249\\
0.288407210180255	323168.249976614\\
0.288507212680317	323125.278141979\\
0.28860721518038	323082.306307344\\
0.288707217680442	323039.33447271\\
0.288807220180505	322996.362638075\\
0.288907222680567	322953.39080344\\
0.289007225180629	322910.418968805\\
0.289107227680692	322867.44713417\\
0.289207230180755	322824.475299535\\
0.289307232680817	322781.503464901\\
0.28940723518088	322738.531630266\\
0.289507237680942	322695.559795631\\
0.289607240181005	322652.015003201\\
0.289707242681067	322609.043168566\\
0.28980724518113	322566.071333931\\
0.289907247681192	322523.099499297\\
0.290007250181255	322480.127664662\\
0.290107252681317	322436.582872232\\
0.29020725518138	322393.611037597\\
0.290307257681442	322350.639202962\\
0.290407260181505	322307.667368327\\
0.290507262681567	322264.122575897\\
0.29060726518163	322221.150741263\\
0.290707267681692	322178.178906628\\
0.290807270181755	322134.634114198\\
0.290907272681817	322091.662279563\\
0.29100727518188	322048.690444928\\
0.291107277681942	322005.145652498\\
0.291207280182005	321962.173817864\\
0.291307282682067	321918.629025434\\
0.29140728518213	321875.657190799\\
0.291507287682192	321832.685356164\\
0.291607290182255	321789.140563734\\
0.291707292682317	321746.168729099\\
0.29180729518238	321702.623936669\\
0.291907297682442	321659.652102034\\
0.292007300182505	321616.107309604\\
0.292107302682567	321573.13547497\\
0.29220730518263	321529.59068254\\
0.292307307682692	321486.04589011\\
0.292407310182755	321443.074055475\\
0.292507312682817	321399.529263045\\
0.29260731518288	321356.55742841\\
0.292707317682942	321313.01263598\\
0.292807320183005	321269.46784355\\
0.292907322683067	321226.496008916\\
0.29300732518313	321182.951216486\\
0.293107327683192	321139.406424056\\
0.293207330183255	321096.434589421\\
0.293307332683317	321052.889796991\\
0.29340733518338	321009.345004561\\
0.293507337683442	320965.800212131\\
0.293607340183505	320922.828377496\\
0.293707342683567	320879.283585066\\
0.29380734518363	320835.738792636\\
0.293907347683692	320792.194000206\\
0.294007350183755	320748.649207776\\
0.294107352683817	320705.104415346\\
0.29420735518388	320662.132580712\\
0.294307357683942	320618.587788282\\
0.294407360184005	320575.042995852\\
0.294507362684067	320531.498203422\\
0.29460736518413	320487.953410992\\
0.294707367684192	320444.408618562\\
0.294807370184255	320400.863826132\\
0.294907372684317	320357.319033702\\
0.29500737518438	320313.774241272\\
0.295107377684442	320270.229448842\\
0.295207380184505	320226.684656412\\
0.295307382684567	320183.139863982\\
0.29540738518463	320139.595071552\\
0.295507387684692	320096.050279122\\
0.295607390184755	320052.505486692\\
0.295707392684817	320008.960694263\\
0.29580739518488	319965.415901833\\
0.295907397684942	319921.298151608\\
0.296007400185005	319877.753359178\\
0.296107402685067	319834.208566748\\
0.29620740518513	319790.663774318\\
0.296307407685192	319747.118981888\\
0.296407410185255	319703.001231663\\
0.296507412685317	319659.456439233\\
0.29660741518538	319615.911646803\\
0.296707417685442	319572.366854373\\
0.296807420185505	319528.822061943\\
0.296907422685567	319484.704311718\\
0.29700742518563	319441.159519288\\
0.297107427685692	319397.614726858\\
0.297207430185755	319353.496976633\\
0.297307432685817	319309.952184203\\
0.29740743518588	319266.407391773\\
0.297507437685942	319222.289641548\\
0.297607440186005	319178.744849118\\
0.297707442686067	319134.627098893\\
0.29780744518613	319091.082306463\\
0.297907447686192	319047.537514033\\
0.298007450186255	319003.419763808\\
0.298107452686317	318959.874971378\\
0.29820745518638	318915.757221153\\
0.298307457686442	318872.212428723\\
0.298407460186505	318828.094678498\\
0.298507462686567	318784.549886068\\
0.29860746518663	318740.432135843\\
0.298707467686692	318696.887343413\\
0.298807470186755	318652.769593188\\
0.298907472686817	318608.651842963\\
0.29900747518688	318565.107050533\\
0.299107477686942	318520.989300308\\
0.299207480187005	318476.871550083\\
0.299307482687067	318433.326757653\\
0.29940748518713	318389.209007428\\
0.299507487687192	318345.664214998\\
0.299607490187255	318301.546464773\\
0.299707492687317	318257.428714548\\
0.29980749518738	318213.310964323\\
0.299907497687442	318169.766171893\\
0.300007500187505	318125.648421668\\
0.300107502687567	318081.530671442\\
0.30020750518763	318037.412921217\\
0.300307507687692	317993.868128787\\
0.300407510187755	317949.750378562\\
0.300507512687817	317905.632628337\\
0.30060751518788	317861.514878112\\
0.300707517687942	317817.397127887\\
0.300807520188005	317773.279377662\\
0.300907522688067	317729.161627437\\
0.30100752518813	317685.616835007\\
0.301107527688192	317641.499084782\\
0.301207530188255	317597.381334557\\
0.301307532688317	317553.263584332\\
0.30140753518838	317509.145834107\\
0.301507537688442	317465.028083882\\
0.301607540188505	317420.910333657\\
0.301707542688567	317376.792583432\\
0.30180754518863	317332.674833206\\
0.301907547688692	317288.557082981\\
0.302007550188755	317244.439332756\\
0.302107552688817	317200.321582531\\
0.30220755518888	317156.203832306\\
0.302307557688942	317112.086082081\\
0.302407560189005	317067.395374061\\
0.302507562689067	317023.277623836\\
0.30260756518913	316979.159873611\\
0.302707567689192	316935.042123386\\
0.302807570189255	316890.924373161\\
0.302907572689317	316846.806622936\\
0.30300757518938	316802.115914915\\
0.303107577689442	316757.99816469\\
0.303207580189505	316713.880414465\\
0.303307582689567	316669.76266424\\
0.30340758518963	316625.644914015\\
0.303507587689692	316580.954205995\\
0.303607590189755	316536.83645577\\
0.303707592689817	316492.718705545\\
0.30380759518988	316448.027997524\\
0.303907597689942	316403.910247299\\
0.304007600190005	316359.792497074\\
0.304107602690067	316315.101789054\\
0.30420760519013	316270.984038829\\
0.304307607690192	316226.866288604\\
0.304407610190255	316182.175580584\\
0.304507612690317	316138.057830359\\
0.30460761519038	316093.367122338\\
0.304707617690442	316049.249372113\\
0.304807620190505	316004.558664093\\
0.304907622690567	315960.440913868\\
0.30500762519063	315916.323163643\\
0.305107627690692	315871.632455623\\
0.305207630190755	315827.514705398\\
0.305307632690817	315782.823997378\\
0.30540763519088	315738.133289357\\
0.305507637690942	315694.015539132\\
0.305607640191005	315649.324831112\\
0.305707642691067	315605.207080887\\
0.30580764519113	315560.516372867\\
0.305907647691192	315516.398622642\\
0.306007650191255	315471.707914622\\
0.306107652691317	315427.017206601\\
0.30620765519138	315382.899456376\\
0.306307657691442	315338.208748356\\
0.306407660191505	315293.518040336\\
0.306507662691567	315249.400290111\\
0.30660766519163	315204.709582091\\
0.306707667691692	315160.01887407\\
0.306807670191755	315115.32816605\\
0.306907672691817	315071.210415825\\
0.30700767519188	315026.519707805\\
0.307107677691942	314981.828999785\\
0.307207680192005	314937.138291764\\
0.307307682692067	314892.447583744\\
0.30740768519213	314848.329833519\\
0.307507687692192	314803.639125499\\
0.307607690192255	314758.948417479\\
0.307707692692317	314714.257709459\\
0.30780769519238	314669.567001438\\
0.307907697692442	314624.876293418\\
0.308007700192505	314580.185585398\\
0.308107702692567	314535.494877378\\
0.30820770519263	314490.804169358\\
0.308307707692692	314446.686419133\\
0.308407710192755	314401.995711112\\
0.308507712692817	314357.305003092\\
0.30860771519288	314312.614295072\\
0.308707717692942	314267.923587052\\
0.308807720193005	314223.232879031\\
0.308907722693067	314177.969213216\\
0.30900772519313	314133.278505196\\
0.309107727693192	314088.587797176\\
0.309207730193255	314043.897089156\\
0.309307732693317	313999.206381135\\
0.30940773519338	313954.515673115\\
0.309507737693442	313909.824965095\\
0.309607740193505	313865.134257075\\
0.309707742693567	313820.443549055\\
0.30980774519363	313775.752841034\\
0.309907747693692	313730.489175219\\
0.310007750193755	313685.798467199\\
0.310107752693817	313641.107759179\\
0.31020775519388	313596.417051158\\
0.310307757693942	313551.153385343\\
0.310407760194005	313506.462677323\\
0.310507762694067	313461.771969303\\
0.31060776519413	313417.081261282\\
0.310707767694192	313371.817595467\\
0.310807770194255	313327.126887447\\
0.310907772694317	313282.436179427\\
0.31100777519438	313237.172513611\\
0.311107777694442	313192.481805591\\
0.311207780194505	313147.791097571\\
0.311307782694567	313102.527431756\\
0.31140778519463	313057.836723735\\
0.311507787694692	313013.146015715\\
0.311607790194755	312967.8823499\\
0.311707792694817	312923.19164188\\
0.31180779519488	312877.927976064\\
0.311907797694942	312833.237268044\\
0.312007800195005	312787.973602229\\
0.312107802695067	312743.282894209\\
0.31220780519513	312698.019228393\\
0.312307807695192	312653.328520373\\
0.312407810195255	312608.064854558\\
0.312507812695317	312563.374146538\\
0.31260781519538	312518.110480722\\
0.312707817695442	312473.419772702\\
0.312807820195505	312428.156106887\\
0.312907822695567	312383.465398866\\
0.31300782519563	312338.201733051\\
0.313107827695692	312292.938067236\\
0.313207830195755	312248.247359216\\
0.313307832695817	312202.9836934\\
0.31340783519588	312157.720027585\\
0.313507837695942	312113.029319565\\
0.313607840196005	312067.765653749\\
0.313707842696067	312022.501987934\\
0.31380784519613	311977.811279914\\
0.313907847696192	311932.547614098\\
0.314007850196255	311887.283948283\\
0.314107852696317	311842.020282468\\
0.31420785519638	311797.329574448\\
0.314307857696442	311752.065908632\\
0.314407860196505	311706.802242817\\
0.314507862696567	311661.538577002\\
0.31460786519663	311616.274911186\\
0.314707867696692	311571.584203166\\
0.314807870196755	311526.320537351\\
0.314907872696817	311481.056871535\\
0.31500787519688	311435.79320572\\
0.315107877696942	311390.529539905\\
0.315207880197005	311345.265874089\\
0.315307882697067	311300.002208274\\
0.31540788519713	311254.738542459\\
0.315507887697192	311209.474876643\\
0.315607890197255	311164.211210828\\
0.315707892697317	311118.947545013\\
0.31580789519738	311074.256836993\\
0.315907897697442	311028.993171177\\
0.316007900197505	310983.156547567\\
0.316107902697567	310937.892881751\\
0.31620790519763	310892.629215936\\
0.316307907697692	310847.365550121\\
0.316407910197755	310802.101884305\\
0.316507912697817	310756.83821849\\
0.31660791519788	310711.574552675\\
0.316707917697942	310666.310886859\\
0.316807920198005	310621.047221044\\
0.316907922698067	310575.783555229\\
0.31700792519813	310530.519889413\\
0.317107927698192	310485.256223598\\
0.317207930198255	310439.419599988\\
0.317307932698317	310394.155934172\\
0.31740793519838	310348.892268357\\
0.317507937698442	310303.628602542\\
0.317607940198505	310258.364936726\\
0.317707942698567	310212.528313116\\
0.31780794519863	310167.2646473\\
0.317907947698692	310122.000981485\\
0.318007950198755	310076.73731567\\
0.318107952698817	310030.900692059\\
0.31820795519888	309985.637026244\\
0.318307957698942	309940.373360429\\
0.318407960199005	309894.536736818\\
0.318507962699068	309849.273071003\\
0.31860796519913	309804.009405187\\
0.318707967699192	309758.172781577\\
0.318807970199255	309712.909115762\\
0.318907972699317	309667.072492151\\
0.31900797519938	309621.808826336\\
0.319107977699442	309576.54516052\\
0.319207980199505	309530.70853691\\
0.319307982699568	309485.444871095\\
0.31940798519963	309439.608247484\\
0.319507987699693	309394.344581669\\
0.319607990199755	309348.507958058\\
0.319707992699817	309303.244292243\\
0.31980799519988	309257.407668633\\
0.319907997699942	309212.144002817\\
0.320008000200005	309166.307379207\\
0.320108002700068	309121.043713391\\
0.32020800520013	309075.207089781\\
0.320308007700193	309029.943423966\\
0.320408010200255	308984.106800355\\
0.320508012700318	308938.270176745\\
0.32060801520038	308893.006510929\\
0.320708017700442	308847.169887319\\
0.320808020200505	308801.906221504\\
0.320908022700568	308756.069597893\\
0.32100802520063	308710.232974283\\
0.321108027700693	308664.969308467\\
0.321208030200755	308619.132684857\\
0.321308032700818	308573.296061246\\
0.32140803520088	308527.459437636\\
0.321508037700943	308482.195771821\\
0.321608040201005	308436.35914821\\
0.321708042701068	308390.5225246\\
0.32180804520113	308344.685900989\\
0.321908047701193	308299.422235174\\
0.322008050201255	308253.585611563\\
0.322108052701318	308207.748987953\\
0.32220805520138	308161.912364342\\
0.322308057701443	308116.075740732\\
0.322408060201505	308070.812074917\\
0.322508062701568	308024.975451306\\
0.32260806520163	307979.138827696\\
0.322708067701693	307933.302204085\\
0.322808070201755	307887.465580475\\
0.322908072701818	307841.628956864\\
0.32300807520188	307795.792333254\\
0.323108077701943	307749.955709643\\
0.323208080202005	307704.119086033\\
0.323308082702068	307658.282462422\\
0.32340808520213	307613.018796607\\
0.323508087702193	307567.182172997\\
0.323608090202255	307521.345549386\\
0.323708092702318	307475.508925776\\
0.32380809520238	307429.672302165\\
0.323908097702443	307383.26272076\\
0.324008100202505	307337.426097149\\
0.324108102702568	307291.589473539\\
0.32420810520263	307245.752849928\\
0.324308107702693	307199.916226318\\
0.324408110202755	307154.079602707\\
0.324508112702818	307108.242979097\\
0.32460811520288	307062.406355486\\
0.324708117702943	307016.569731876\\
0.324808120203005	306970.733108266\\
0.324908122703068	306924.896484655\\
0.32500812520313	306878.486903249\\
0.325108127703193	306832.650279639\\
0.325208130203255	306786.813656028\\
0.325308132703318	306740.977032418\\
0.32540813520338	306695.140408808\\
0.325508137703443	306648.730827402\\
0.325608140203505	306602.894203792\\
0.325708142703568	306557.057580181\\
0.32580814520363	306511.220956571\\
0.325908147703693	306464.811375165\\
0.326008150203755	306418.974751555\\
0.326108152703818	306373.138127944\\
0.32620815520388	306326.728546538\\
0.326308157703943	306280.891922928\\
0.326408160204005	306235.055299318\\
0.326508162704068	306188.645717912\\
0.32660816520413	306142.809094301\\
0.326708167704193	306096.972470691\\
0.326808170204255	306050.562889285\\
0.326908172704318	306004.726265675\\
0.32700817520438	305958.316684269\\
0.327108177704443	305912.480060659\\
0.327208180204505	305866.643437048\\
0.327308182704568	305820.233855643\\
0.32740818520463	305774.397232032\\
0.327508187704693	305727.987650627\\
0.327608190204755	305682.151027016\\
0.327708192704818	305635.741445611\\
0.32780819520488	305589.904822\\
0.327908197704943	305543.495240595\\
0.328008200205005	305497.658616984\\
0.328108202705068	305451.249035579\\
0.32820820520513	305404.839454173\\
0.328308207705193	305359.002830563\\
0.328408210205255	305312.593249157\\
0.328508212705318	305266.756625546\\
0.32860821520538	305220.347044141\\
0.328708217705443	305173.937462735\\
0.328808220205505	305128.100839125\\
0.328908222705568	305081.691257719\\
0.32900822520563	305035.854634109\\
0.329108227705693	304989.445052703\\
0.329208230205755	304943.035471298\\
0.329308232705818	304896.625889892\\
0.32940823520588	304850.789266281\\
0.329508237705943	304804.379684876\\
0.329608240206005	304757.97010347\\
0.329708242706068	304712.13347986\\
0.32980824520613	304665.723898454\\
0.329908247706193	304619.314317049\\
0.330008250206255	304572.904735643\\
0.330108252706318	304527.068112033\\
0.33020825520638	304480.658530627\\
0.330308257706443	304434.248949221\\
0.330408260206505	304387.839367816\\
0.330508262706568	304341.42978641\\
0.33060826520663	304295.020205005\\
0.330708267706693	304249.183581394\\
0.330808270206755	304202.773999989\\
0.330908272706818	304156.364418583\\
0.33100827520688	304109.954837177\\
0.331108277706943	304063.545255772\\
0.331208280207005	304017.135674366\\
0.331308282707068	303970.726092961\\
0.33140828520713	303924.316511555\\
0.331508287707193	303877.906930149\\
0.331608290207255	303831.497348744\\
0.331708292707318	303785.087767338\\
0.33180829520738	303738.678185933\\
0.331908297707443	303692.268604527\\
0.332008300207505	303645.859023121\\
0.332108302707568	303599.449441716\\
0.33220830520763	303553.03986031\\
0.332308307707693	303506.630278905\\
0.332408310207755	303460.220697499\\
0.332508312707818	303413.811116093\\
0.33260831520788	303367.401534688\\
0.332708317707943	303320.991953282\\
0.332808320208005	303274.582371877\\
0.332908322708068	303228.172790471\\
0.33300832520813	303181.19025127\\
0.333108327708193	303134.780669865\\
0.333208330208255	303088.371088459\\
0.333308332708318	303041.961507053\\
0.33340833520838	302995.551925648\\
0.333508337708443	302949.142344242\\
0.333608340208505	302902.159805042\\
0.333708342708568	302855.750223636\\
0.33380834520863	302809.34064223\\
0.333908347708693	302762.931060825\\
0.334008350208755	302716.521479419\\
0.334108352708818	302669.538940218\\
0.33420835520888	302623.129358813\\
0.334308357708943	302576.719777407\\
0.334408360209005	302529.737238206\\
0.334508362709068	302483.327656801\\
0.33460836520913	302436.918075395\\
0.334708367709193	302390.50849399\\
0.334808370209255	302343.525954789\\
0.334908372709318	302297.116373383\\
0.33500837520938	302250.133834183\\
0.335108377709443	302203.724252777\\
0.335208380209505	302157.314671371\\
0.335308382709568	302110.332132171\\
0.33540838520963	302063.922550765\\
0.335508387709693	302016.940011564\\
0.335608390209755	301970.530430159\\
0.335708392709818	301924.120848753\\
0.33580839520988	301877.138309553\\
0.335908397709943	301830.728728147\\
0.336008400210005	301783.746188946\\
0.336108402710068	301737.336607541\\
0.33620840521013	301690.35406834\\
0.336308407710193	301643.944486934\\
0.336408410210255	301596.961947733\\
0.336508412710318	301550.552366328\\
0.33660841521038	301503.569827127\\
0.336708417710443	301457.160245722\\
0.336808420210505	301410.177706521\\
0.336908422710568	301363.19516732\\
0.33700842521063	301316.785585915\\
0.337108427710693	301269.803046714\\
0.337208430210755	301223.393465308\\
0.337308432710818	301176.410926108\\
0.33740843521088	301129.428386907\\
0.337508437710943	301083.018805501\\
0.337608440211005	301036.0362663\\
0.337708442711068	300989.0537271\\
0.33780844521113	300942.644145694\\
0.337908447711193	300895.661606493\\
0.338008450211255	300848.679067293\\
0.338108452711318	300802.269485887\\
0.33820845521138	300755.286946686\\
0.338308457711443	300708.304407486\\
0.338408460211505	300661.321868285\\
0.338508462711568	300614.912286879\\
0.33860846521163	300567.929747679\\
0.338708467711693	300520.947208478\\
0.338808470211755	300473.964669277\\
0.338908472711818	300427.555087872\\
0.33900847521188	300380.572548671\\
0.339108477711943	300333.59000947\\
0.339208480212005	300286.607470269\\
0.339308482712068	300239.624931069\\
0.33940848521213	300192.642391868\\
0.339508487712193	300146.232810462\\
0.339608490212255	300099.250271262\\
0.339708492712318	300052.267732061\\
0.33980849521238	300005.28519286\\
0.339908497712443	299958.302653659\\
0.340008500212505	299911.320114459\\
0.340108502712568	299864.337575258\\
0.34020850521263	299817.355036057\\
0.340308507712693	299770.372496856\\
0.340408510212755	299723.389957656\\
0.340508512712818	299676.407418455\\
0.34060851521288	299629.424879254\\
0.340708517712943	299582.442340054\\
0.340808520213005	299535.459800853\\
0.340908522713068	299488.477261652\\
0.34100852521313	299441.494722451\\
0.341108527713193	299394.512183251\\
0.341208530213255	299347.52964405\\
0.341308532713318	299300.547104849\\
0.34140853521338	299253.564565648\\
0.341508537713443	299206.582026448\\
0.341608540213505	299159.599487247\\
0.341708542713568	299112.616948046\\
0.34180854521363	299065.634408846\\
0.341908547713693	299018.07891185\\
0.342008550213755	298971.096372649\\
0.342108552713818	298924.113833448\\
0.34220855521388	298877.131294248\\
0.342308557713943	298830.148755047\\
0.342408560214005	298783.166215846\\
0.342508562714068	298735.61071885\\
0.34260856521413	298688.628179649\\
0.342708567714193	298641.645640449\\
0.342808570214255	298594.663101248\\
0.342908572714318	298547.680562047\\
0.34300857521438	298500.125065051\\
0.343108577714443	298453.142525851\\
0.343208580214505	298406.15998665\\
0.343308582714568	298359.177447449\\
0.34340858521463	298311.621950453\\
0.343508587714693	298264.639411253\\
0.343608590214755	298217.656872052\\
0.343708592714818	298170.101375056\\
0.34380859521488	298123.118835855\\
0.343908597714943	298076.136296655\\
0.344008600215005	298028.580799659\\
0.344108602715068	297981.598260458\\
0.34420860521513	297934.615721257\\
0.344308607715193	297887.060224261\\
0.344408610215255	297840.077685061\\
0.344508612715318	297792.522188065\\
0.34460861521538	297745.539648864\\
0.344708617715443	297698.557109663\\
0.344808620215505	297651.001612668\\
0.344908622715568	297604.019073467\\
0.34500862521563	297556.463576471\\
0.345108627715693	297509.48103727\\
0.345208630215755	297461.925540274\\
0.345308632715818	297414.943001074\\
0.34540863521588	297367.387504078\\
0.345508637715943	297320.404964877\\
0.345608640216005	297272.849467881\\
0.345708642716068	297225.86692868\\
0.34580864521613	297178.311431685\\
0.345908647716193	297131.328892484\\
0.346008650216255	297083.773395488\\
0.346108652716318	297036.217898492\\
0.34620865521638	296989.235359291\\
0.346308657716443	296941.679862296\\
0.346408660216505	296894.697323095\\
0.346508662716568	296847.141826099\\
0.34660866521663	296799.586329103\\
0.346708667716693	296752.603789902\\
0.346808670216755	296705.048292907\\
0.346908672716818	296658.065753706\\
0.34700867521688	296610.51025671\\
0.347108677716943	296562.954759714\\
0.347208680217005	296515.972220513\\
0.347308682717068	296468.416723518\\
0.34740868521713	296420.861226522\\
0.347508687717193	296373.305729526\\
0.347608690217255	296326.323190325\\
0.347708692717318	296278.767693329\\
0.34780869521738	296231.212196333\\
0.347908697717443	296183.656699338\\
0.348008700217505	296136.674160137\\
0.348108702717568	296089.118663141\\
0.34820870521763	296041.563166145\\
0.348308707717693	295994.007669149\\
0.348408710217755	295947.025129948\\
0.348508712717818	295899.469632953\\
0.34860871521788	295851.914135957\\
0.348708717717943	295804.358638961\\
0.348808720218005	295756.803141965\\
0.348908722718068	295709.247644969\\
0.34900872521813	295661.692147973\\
0.349108727718193	295614.709608773\\
0.349208730218255	295567.154111777\\
0.349308732718318	295519.598614781\\
0.34940873521838	295472.043117785\\
0.349508737718443	295424.487620789\\
0.349608740218505	295376.932123793\\
0.349708742718568	295329.376626797\\
0.34980874521863	295281.821129802\\
0.349908747718693	295234.265632806\\
0.350008750218755	295186.71013581\\
0.350108752718818	295139.154638814\\
0.35020875521888	295091.599141818\\
0.350308757718943	295044.043644822\\
0.350408760219005	294996.488147826\\
0.350508762719068	294948.932650831\\
0.35060876521913	294901.377153835\\
0.350708767719193	294853.821656839\\
0.350808770219255	294806.266159843\\
0.350908772719318	294758.710662847\\
0.351008775219381	294711.155165851\\
0.351108777719443	294663.599668855\\
0.351208780219505	294616.04417186\\
0.351308782719568	294568.488674864\\
0.35140878521963	294520.933177868\\
0.351508787719693	294473.377680872\\
0.351608790219755	294425.249226081\\
0.351708792719818	294377.693729085\\
0.351808795219881	294330.138232089\\
0.351908797719943	294282.582735093\\
0.352008800220006	294235.027238098\\
0.352108802720068	294187.471741102\\
0.35220880522013	294139.343286311\\
0.352308807720193	294091.787789315\\
0.352408810220255	294044.232292319\\
0.352508812720318	293996.676795323\\
0.352608815220381	293949.121298327\\
0.352708817720443	293900.992843536\\
0.352808820220506	293853.43734654\\
0.352908822720568	293805.881849545\\
0.353008825220631	293758.326352549\\
0.353108827720693	293710.197897758\\
0.353208830220755	293662.642400762\\
0.353308832720818	293615.086903766\\
0.353408835220881	293566.958448975\\
0.353508837720943	293519.402951979\\
0.353608840221006	293471.847454983\\
0.353708842721068	293424.291957987\\
0.353808845221131	293376.163503196\\
0.353908847721193	293328.608006201\\
0.354008850221256	293280.47955141\\
0.354108852721318	293232.924054414\\
0.354208855221381	293185.368557418\\
0.354308857721443	293137.240102627\\
0.354408860221506	293089.684605631\\
0.354508862721568	293042.129108635\\
0.354608865221631	292994.000653844\\
0.354708867721693	292946.445156848\\
0.354808870221756	292898.316702057\\
0.354908872721818	292850.761205062\\
0.355008875221881	292802.63275027\\
0.355108877721943	292755.077253275\\
0.355208880222006	292706.948798484\\
0.355308882722068	292659.393301488\\
0.355408885222131	292611.264846697\\
0.355508887722193	292563.709349701\\
0.355608890222256	292515.58089491\\
0.355708892722318	292468.025397914\\
0.355808895222381	292419.896943123\\
0.355908897722443	292372.341446127\\
0.356008900222506	292324.212991336\\
0.356108902722568	292276.65749434\\
0.356208905222631	292228.529039549\\
0.356308907722693	292180.973542554\\
0.356408910222756	292132.845087763\\
0.356508912722818	292084.716632972\\
0.356608915222881	292037.161135976\\
0.356708917722943	291989.032681185\\
0.356808920223006	291941.477184189\\
0.356908922723068	291893.348729398\\
0.357008925223131	291845.220274607\\
0.357108927723193	291797.664777611\\
0.357208930223256	291749.53632282\\
0.357308932723318	291701.407868029\\
0.357408935223381	291653.852371033\\
0.357508937723443	291605.723916242\\
0.357608940223506	291557.595461451\\
0.357708942723568	291510.039964455\\
0.357808945223631	291461.911509664\\
0.357908947723693	291413.783054873\\
0.358008950223756	291365.654600082\\
0.358108952723818	291318.099103087\\
0.358208955223881	291269.970648296\\
0.358308957723943	291221.842193505\\
0.358408960224006	291173.713738714\\
0.358508962724068	291126.158241718\\
0.358608965224131	291078.029786927\\
0.358708967724193	291029.901332136\\
0.358808970224256	290981.772877345\\
0.358908972724318	290933.644422554\\
0.359008975224381	290886.088925558\\
0.359108977724443	290837.960470767\\
0.359208980224506	290789.832015976\\
0.359308982724568	290741.703561185\\
0.359408985224631	290693.575106394\\
0.359508987724693	290645.446651603\\
0.359608990224756	290597.318196812\\
0.359708992724818	290549.189742021\\
0.359808995224881	290501.634245025\\
0.359908997724943	290453.505790234\\
0.360009000225006	290405.377335443\\
0.360109002725068	290357.248880652\\
0.360209005225131	290309.120425861\\
0.360309007725193	290260.99197107\\
0.360409010225256	290212.863516279\\
0.360509012725318	290164.735061488\\
0.360609015225381	290116.606606697\\
0.360709017725443	290068.478151906\\
0.360809020225506	290020.349697115\\
0.360909022725568	289972.221242324\\
0.361009025225631	289924.092787533\\
0.361109027725693	289875.964332742\\
0.361209030225756	289827.835877951\\
0.361309032725818	289779.70742316\\
0.361409035225881	289731.578968369\\
0.361509037725943	289683.450513578\\
0.361609040226006	289635.322058787\\
0.361709042726068	289587.193603996\\
0.361809045226131	289539.065149205\\
0.361909047726193	289490.936694414\\
0.362009050226256	289442.808239623\\
0.362109052726318	289394.106827037\\
0.362209055226381	289345.978372246\\
0.362309057726443	289297.849917455\\
0.362409060226506	289249.721462664\\
0.362509062726568	289201.593007873\\
0.362609065226631	289153.464553082\\
0.362709067726693	289105.336098291\\
0.362809070226756	289056.634685705\\
0.362909072726818	289008.506230914\\
0.363009075226881	288960.377776123\\
0.363109077726943	288912.249321332\\
0.363209080227006	288864.120866541\\
0.363309082727068	288815.419453955\\
0.363409085227131	288767.290999164\\
0.363509087727193	288719.162544373\\
0.363609090227256	288671.034089582\\
0.363709092727318	288622.905634791\\
0.363809095227381	288574.204222205\\
0.363909097727443	288526.075767414\\
0.364009100227506	288477.947312623\\
0.364109102727568	288429.818857832\\
0.364209105227631	288381.117445246\\
0.364309107727693	288332.988990455\\
0.364409110227756	288284.860535664\\
0.364509112727818	288236.159123078\\
0.364609115227881	288188.030668287\\
0.364709117727943	288139.902213496\\
0.364809120228006	288091.20080091\\
0.364909122728068	288043.072346119\\
0.365009125228131	287994.943891328\\
0.365109127728193	287946.242478742\\
0.365209130228256	287898.114023951\\
0.365309132728318	287849.412611365\\
0.365409135228381	287801.284156574\\
0.365509137728443	287753.155701783\\
0.365609140228506	287704.454289196\\
0.365709142728568	287656.325834405\\
0.365809145228631	287607.624421819\\
0.365909147728693	287559.495967028\\
0.366009150228756	287511.367512237\\
0.366109152728818	287462.666099651\\
0.366209155228881	287414.53764486\\
0.366309157728943	287365.836232274\\
0.366409160229006	287317.707777483\\
0.366509162729068	287269.006364897\\
0.366609165229131	287220.877910106\\
0.366709167729193	287172.17649752\\
0.366809170229256	287124.048042729\\
0.366909172729318	287075.346630143\\
0.367009175229381	287027.218175352\\
0.367109177729443	286978.516762766\\
0.367209180229506	286930.388307975\\
0.367309182729568	286881.686895389\\
0.367409185229631	286833.558440598\\
0.367509187729693	286784.857028012\\
0.367609190229756	286736.155615425\\
0.367709192729818	286688.027160634\\
0.367809195229881	286639.325748048\\
0.367909197729943	286591.197293257\\
0.368009200230006	286542.495880671\\
0.368109202730068	286493.794468085\\
0.368209205230131	286445.666013294\\
0.368309207730193	286396.964600708\\
0.368409210230256	286348.836145917\\
0.368509212730318	286300.134733331\\
0.368609215230381	286251.433320745\\
0.368709217730443	286203.304865954\\
0.368809220230506	286154.603453368\\
0.368909222730568	286105.902040781\\
0.369009225230631	286057.77358599\\
0.369109227730693	286009.072173404\\
0.369209230230756	285960.370760818\\
0.369309232730818	285911.669348232\\
0.369409235230881	285863.540893441\\
0.369509237730943	285814.839480855\\
0.369609240231006	285766.138068269\\
0.369709242731068	285717.436655683\\
0.369809245231131	285669.308200892\\
0.369909247731193	285620.606788306\\
0.370009250231256	285571.90537572\\
0.370109252731318	285523.203963133\\
0.370209255231381	285475.075508342\\
0.370309257731443	285426.374095756\\
0.370409260231506	285377.67268317\\
0.370509262731568	285328.971270584\\
0.370609265231631	285280.269857998\\
0.370709267731693	285232.141403207\\
0.370809270231756	285183.439990621\\
0.370909272731818	285134.738578035\\
0.371009275231881	285086.037165449\\
0.371109277731943	285037.335752863\\
0.371209280232006	284988.634340276\\
0.371309282732068	284939.93292769\\
0.371409285232131	284891.804472899\\
0.371509287732193	284843.103060313\\
0.371609290232256	284794.401647727\\
0.371709292732318	284745.700235141\\
0.371809295232381	284696.998822555\\
0.371909297732443	284648.297409969\\
0.372009300232506	284599.595997383\\
0.372109302732568	284550.894584796\\
0.372209305232631	284502.19317221\\
0.372309307732693	284453.491759624\\
0.372409310232756	284404.790347038\\
0.372509312732818	284356.088934452\\
0.372609315232881	284307.387521866\\
0.372709317732943	284258.68610928\\
0.372809320233006	284209.984696694\\
0.372909322733068	284161.283284107\\
0.373009325233131	284112.581871521\\
0.373109327733193	284063.880458935\\
0.373209330233256	284015.179046349\\
0.373309332733318	283966.477633763\\
0.373409335233381	283917.776221177\\
0.373509337733443	283869.074808591\\
0.373609340233506	283820.373396005\\
0.373709342733568	283771.671983418\\
0.373809345233631	283722.970570832\\
0.373909347733693	283674.269158246\\
0.374009350233756	283625.56774566\\
0.374109352733818	283576.866333074\\
0.374209355233881	283528.164920488\\
0.374309357733943	283479.463507902\\
0.374409360234006	283430.189137521\\
0.374509362734068	283381.487724934\\
0.374609365234131	283332.786312348\\
0.374709367734193	283284.084899762\\
0.374809370234256	283235.383487176\\
0.374909372734318	283186.68207459\\
0.375009375234381	283137.980662004\\
0.375109377734443	283088.706291623\\
0.375209380234506	283040.004879036\\
0.375309382734568	282991.30346645\\
0.375409385234631	282942.602053864\\
0.375509387734693	282893.900641278\\
0.375609390234756	282844.626270897\\
0.375709392734818	282795.924858311\\
0.375809395234881	282747.223445725\\
0.375909397734943	282698.522033138\\
0.376009400235006	282649.820620552\\
0.376109402735068	282600.546250171\\
0.376209405235131	282551.844837585\\
0.376309407735193	282503.143424999\\
0.376409410235256	282454.442012413\\
0.376509412735318	282405.167642031\\
0.376609415235381	282356.466229445\\
0.376709417735443	282307.764816859\\
0.376809420235506	282258.490446478\\
0.376909422735568	282209.789033892\\
0.377009425235631	282161.087621306\\
0.377109427735693	282111.813250924\\
0.377209430235756	282063.111838338\\
0.377309432735818	282014.410425752\\
0.377409435235881	281965.136055371\\
0.377509437735943	281916.434642785\\
0.377609440236006	281867.733230199\\
0.377709442736068	281818.458859817\\
0.377809445236131	281769.757447231\\
0.377909447736193	281721.056034645\\
0.378009450236256	281671.781664264\\
0.378109452736318	281623.080251678\\
0.378209455236381	281573.805881297\\
0.378309457736443	281525.10446871\\
0.378409460236506	281476.403056124\\
0.378509462736568	281427.128685743\\
0.378609465236631	281378.427273157\\
0.378709467736693	281329.152902776\\
0.378809470236756	281280.45149019\\
0.378909472736818	281231.177119808\\
0.379009475236881	281182.475707222\\
0.379109477736943	281133.774294636\\
0.379209480237006	281084.499924255\\
0.379309482737068	281035.798511669\\
0.379409485237131	280986.524141288\\
0.379509487737193	280937.822728701\\
0.379609490237256	280888.54835832\\
0.379709492737318	280839.846945734\\
0.379809495237381	280790.572575353\\
0.379909497737443	280741.871162767\\
0.380009500237506	280692.596792385\\
0.380109502737568	280643.895379799\\
0.380209505237631	280594.621009418\\
0.380309507737693	280545.346639037\\
0.380409510237756	280496.645226451\\
0.380509512737818	280447.370856069\\
0.380609515237881	280398.669443483\\
0.380709517737943	280349.395073102\\
0.380809520238006	280300.693660516\\
0.380909522738068	280251.419290135\\
0.381009525238131	280202.144919753\\
0.381109527738193	280153.443507167\\
0.381209530238256	280104.169136786\\
0.381309532738318	280055.4677242\\
0.381409535238381	280006.193353819\\
0.381509537738443	279956.918983437\\
0.381609540238506	279908.217570851\\
0.381709542738568	279858.94320047\\
0.381809545238631	279809.668830089\\
0.381909547738693	279760.967417503\\
0.382009550238756	279711.693047121\\
0.382109552738818	279662.41867674\\
0.382209555238881	279613.717264154\\
0.382309557738943	279564.442893773\\
0.382409560239006	279515.168523392\\
0.382509562739068	279466.467110805\\
0.382609565239131	279417.192740424\\
0.382709567739193	279367.918370043\\
0.382809570239256	279318.643999662\\
0.382909572739318	279269.942587076\\
0.383009575239381	279220.668216694\\
0.383109577739443	279171.393846313\\
0.383209580239506	279122.119475932\\
0.383309582739568	279073.418063346\\
0.383409585239631	279024.143692965\\
0.383509587739694	278974.869322583\\
0.383609590239756	278925.594952202\\
0.383709592739818	278876.893539616\\
0.383809595239881	278827.619169235\\
0.383909597739943	278778.344798853\\
0.384009600240006	278729.070428472\\
0.384109602740069	278679.796058091\\
0.384209605240131	278631.094645505\\
0.384309607740194	278581.820275123\\
0.384409610240256	278532.545904742\\
0.384509612740319	278483.271534361\\
0.384609615240381	278433.99716398\\
0.384709617740443	278384.722793598\\
0.384809620240506	278336.021381012\\
0.384909622740569	278286.747010631\\
0.385009625240631	278237.47264025\\
0.385109627740694	278188.198269869\\
0.385209630240756	278138.923899487\\
0.385309632740819	278089.649529106\\
0.385409635240881	278040.375158725\\
0.385509637740944	277991.100788344\\
0.385609640241006	277941.826417962\\
0.385709642741069	277892.552047581\\
0.385809645241131	277843.850634995\\
0.385909647741194	277794.576264614\\
0.386009650241256	277745.301894232\\
0.386109652741319	277696.027523851\\
0.386209655241381	277646.75315347\\
0.386309657741444	277597.478783089\\
0.386409660241506	277548.204412707\\
0.386509662741569	277498.930042326\\
0.386609665241631	277449.655671945\\
0.386709667741694	277400.381301564\\
0.386809670241756	277351.106931182\\
0.386909672741819	277301.832560801\\
0.387009675241881	277252.55819042\\
0.387109677741944	277203.283820039\\
0.387209680242006	277154.009449657\\
0.387309682742069	277104.735079276\\
0.387409685242131	277055.460708895\\
0.387509687742194	277006.186338514\\
0.387609690242256	276956.911968132\\
0.387709692742319	276907.637597751\\
0.387809695242381	276858.36322737\\
0.387909697742444	276809.088856989\\
0.388009700242506	276759.814486607\\
0.388109702742569	276709.967158431\\
0.388209705242631	276660.69278805\\
0.388309707742694	276611.418417669\\
0.388409710242756	276562.144047287\\
0.388509712742819	276512.869676906\\
0.388609715242881	276463.595306525\\
0.388709717742944	276414.320936144\\
0.388809720243006	276365.046565762\\
0.388909722743069	276315.772195381\\
0.389009725243131	276266.497825\\
0.389109727743194	276216.650496823\\
0.389209730243256	276167.376126442\\
0.389309732743319	276118.101756061\\
0.389409735243381	276068.82738568\\
0.389509737743444	276019.553015298\\
0.389609740243506	275970.278644917\\
0.389709742743569	275920.431316741\\
0.389809745243631	275871.15694636\\
0.389909747743694	275821.882575978\\
0.390009750243756	275772.608205597\\
0.390109752743819	275723.333835216\\
0.390209755243881	275673.486507039\\
0.390309757743944	275624.212136658\\
0.390409760244006	275574.937766277\\
0.390509762744069	275525.663395896\\
0.390609765244131	275476.389025514\\
0.390709767744194	275426.541697338\\
0.390809770244256	275377.267326957\\
0.390909772744319	275327.992956576\\
0.391009775244381	275278.718586194\\
0.391109777744444	275228.871258018\\
0.391209780244506	275179.596887637\\
0.391309782744569	275130.322517255\\
0.391409785244631	275080.475189079\\
0.391509787744694	275031.200818698\\
0.391609790244756	274981.926448317\\
0.391709792744819	274932.652077935\\
0.391809795244881	274882.804749759\\
0.391909797744944	274833.530379378\\
0.392009800245006	274784.256008996\\
0.392109802745069	274734.40868082\\
0.392209805245131	274685.134310439\\
0.392309807745194	274635.859940058\\
0.392409810245256	274586.012611881\\
0.392509812745319	274536.7382415\\
0.392609815245381	274487.463871119\\
0.392709817745444	274437.616542942\\
0.392809820245506	274388.342172561\\
0.392909822745569	274339.06780218\\
0.393009825245631	274289.220474003\\
0.393109827745694	274239.946103622\\
0.393209830245756	274190.098775446\\
0.393309832745819	274140.824405064\\
0.393409835245881	274091.550034683\\
0.393509837745944	274041.702706507\\
0.393609840246006	273992.428336126\\
0.393709842746069	273942.581007949\\
0.393809845246131	273893.306637568\\
0.393909847746194	273844.032267187\\
0.394009850246256	273794.18493901\\
0.394109852746319	273744.910568629\\
0.394209855246381	273695.063240453\\
0.394309857746444	273645.788870071\\
0.394409860246506	273595.941541895\\
0.394509862746569	273546.667171514\\
0.394609865246631	273496.819843337\\
0.394709867746694	273447.545472956\\
0.394809870246756	273397.69814478\\
0.394909872746819	273348.423774399\\
0.395009875246881	273298.576446222\\
0.395109877746944	273249.302075841\\
0.395209880247006	273199.454747665\\
0.395309882747069	273150.180377283\\
0.395409885247131	273100.333049107\\
0.395509887747194	273051.058678726\\
0.395609890247256	273001.211350549\\
0.395709892747319	272951.936980168\\
0.395809895247381	272902.089651992\\
0.395909897747444	272852.81528161\\
0.396009900247506	272802.967953434\\
0.396109902747569	272753.693583053\\
0.396209905247631	272703.846254876\\
0.396309907747694	272654.571884495\\
0.396409910247756	272604.724556319\\
0.396509912747819	272554.877228142\\
0.396609915247881	272505.602857761\\
0.396709917747944	272455.755529585\\
0.396809920248006	272406.481159204\\
0.396909922748069	272356.633831027\\
0.397009925248131	272306.786502851\\
0.397109927748194	272257.512132469\\
0.397209930248256	272207.664804293\\
0.397309932748319	272158.390433912\\
0.397409935248381	272108.543105735\\
0.397509937748444	272058.695777559\\
0.397609940248506	272009.421407178\\
0.397709942748569	271959.574079001\\
0.397809945248631	271909.726750825\\
0.397909947748694	271860.452380444\\
0.398009950248756	271810.605052267\\
0.398109952748819	271760.757724091\\
0.398209955248881	271711.48335371\\
0.398309957748944	271661.636025533\\
0.398409960249006	271611.788697357\\
0.398509962749069	271562.514326976\\
0.398609965249131	271512.666998799\\
0.398709967749194	271462.819670623\\
0.398809970249256	271413.545300242\\
0.398909972749319	271363.697972065\\
0.399009975249381	271313.850643889\\
0.399109977749444	271264.003315713\\
0.399209980249506	271214.728945331\\
0.399309982749569	271164.881617155\\
0.399409985249631	271115.034288979\\
0.399509987749694	271065.759918597\\
0.399609990249756	271015.912590421\\
0.399709992749819	270966.065262245\\
0.399809995249881	270916.217934068\\
0.399909997749944	270866.943563687\\
0.400010000250006	270817.096235511\\
};
\addplot [color=mycolor1,solid,forget plot]
  table[row sep=crcr]{%
0.400010000250006	270817.096235511\\
0.400110002750069	270767.248907334\\
0.400210005250131	270717.401579158\\
0.400310007750194	270667.554250981\\
0.400410010250256	270618.2798806\\
0.400510012750319	270568.432552424\\
0.400610015250381	270518.585224247\\
0.400710017750444	270468.737896071\\
0.400810020250506	270418.890567895\\
0.400910022750569	270369.616197513\\
0.401010025250631	270319.768869337\\
0.401110027750694	270269.921541161\\
0.401210030250756	270220.074212984\\
0.401310032750819	270170.226884808\\
0.401410035250881	270120.379556631\\
0.401510037750944	270071.10518625\\
0.401610040251006	270021.257858074\\
0.401710042751069	269971.410529897\\
0.401810045251131	269921.563201721\\
0.401910047751194	269871.715873545\\
0.402010050251256	269821.868545368\\
0.402110052751319	269772.021217192\\
0.402210055251381	269722.746846811\\
0.402310057751444	269672.899518634\\
0.402410060251506	269623.052190458\\
0.402510062751569	269573.204862282\\
0.402610065251631	269523.357534105\\
0.402710067751694	269473.510205929\\
0.402810070251756	269423.662877752\\
0.402910072751819	269373.815549576\\
0.403010075251881	269323.9682214\\
0.403110077751944	269274.120893223\\
0.403210080252006	269224.846522842\\
0.403310082752069	269174.999194666\\
0.403410085252131	269125.151866489\\
0.403510087752194	269075.304538313\\
0.403610090252256	269025.457210136\\
0.403710092752319	268975.60988196\\
0.403810095252381	268925.762553784\\
0.403910097752444	268875.915225607\\
0.404010100252506	268826.067897431\\
0.404110102752569	268776.220569255\\
0.404210105252631	268726.373241078\\
0.404310107752694	268676.525912902\\
0.404410110252756	268626.678584725\\
0.404510112752819	268576.831256549\\
0.404610115252881	268526.983928373\\
0.404710117752944	268477.136600196\\
0.404810120253006	268427.28927202\\
0.404910122753069	268377.441943843\\
0.405010125253131	268327.594615667\\
0.405110127753194	268277.747287491\\
0.405210130253256	268227.899959314\\
0.405310132753319	268178.052631138\\
0.405410135253381	268128.205302962\\
0.405510137753444	268078.357974785\\
0.405610140253506	268028.510646609\\
0.405710142753569	267978.663318432\\
0.405810145253631	267928.815990256\\
0.405910147753694	267878.96866208\\
0.406010150253756	267828.548376108\\
0.406110152753819	267778.701047932\\
0.406210155253881	267728.853719755\\
0.406310157753944	267679.006391579\\
0.406410160254006	267629.159063403\\
0.406510162754069	267579.311735226\\
0.406610165254131	267529.46440705\\
0.406710167754194	267479.617078874\\
0.406810170254256	267429.769750697\\
0.406910172754319	267379.922422521\\
0.407010175254381	267330.075094344\\
0.407110177754444	267279.654808373\\
0.407210180254506	267229.807480196\\
0.407310182754569	267179.96015202\\
0.407410185254631	267130.112823844\\
0.407510187754694	267080.265495667\\
0.407610190254756	267030.418167491\\
0.407710192754819	266980.570839315\\
0.407810195254881	266930.150553343\\
0.407910197754944	266880.303225167\\
0.408010200255006	266830.45589699\\
0.408110202755069	266780.608568814\\
0.408210205255131	266730.761240638\\
0.408310207755194	266680.913912461\\
0.408410210255256	266630.49362649\\
0.408510212755319	266580.646298313\\
0.408610215255381	266530.798970137\\
0.408710217755444	266480.951641961\\
0.408810220255506	266431.104313784\\
0.408910222755569	266380.684027813\\
0.409010225255631	266330.836699636\\
0.409110227755694	266280.98937146\\
0.409210230255756	266231.142043283\\
0.409310232755819	266181.294715107\\
0.409410235255881	266130.874429136\\
0.409510237755944	266081.027100959\\
0.409610240256006	266031.179772783\\
0.409710242756069	265981.332444606\\
0.409810245256131	265930.912158635\\
0.409910247756194	265881.064830459\\
0.410010250256256	265831.217502282\\
0.410110252756319	265781.370174106\\
0.410210255256381	265730.949888134\\
0.410310257756444	265681.102559958\\
0.410410260256506	265631.255231782\\
0.410510262756569	265581.407903605\\
0.410610265256631	265530.987617634\\
0.410710267756694	265481.140289457\\
0.410810270256756	265431.292961281\\
0.410910272756819	265381.445633104\\
0.411010275256881	265331.025347133\\
0.411110277756944	265281.178018957\\
0.411210280257006	265231.33069078\\
0.411310282757069	265180.910404809\\
0.411410285257131	265131.063076632\\
0.411510287757194	265081.215748456\\
0.411610290257256	265030.795462484\\
0.411710292757319	264980.948134308\\
0.411810295257381	264931.100806132\\
0.411910297757444	264880.68052016\\
0.412010300257506	264830.833191984\\
0.412110302757569	264780.985863807\\
0.412210305257631	264730.565577836\\
0.412310307757694	264680.718249659\\
0.412410310257756	264630.870921483\\
0.412510312757819	264580.450635512\\
0.412610315257881	264530.603307335\\
0.412710317757944	264480.755979159\\
0.412810320258006	264430.335693187\\
0.412910322758069	264380.488365011\\
0.413010325258131	264330.068079039\\
0.413110327758194	264280.220750863\\
0.413210330258256	264230.373422687\\
0.413310332758319	264179.953136715\\
0.413410335258381	264130.105808539\\
0.413510337758444	264080.258480362\\
0.413610340258506	264029.838194391\\
0.413710342758569	263979.990866214\\
0.413810345258631	263929.570580243\\
0.413910347758694	263879.723252067\\
0.414010350258756	263829.302966095\\
0.414110352758819	263779.455637919\\
0.414210355258881	263729.608309742\\
0.414310357758944	263679.188023771\\
0.414410360259006	263629.340695594\\
0.414510362759069	263578.920409623\\
0.414610365259131	263529.073081446\\
0.414710367759194	263478.652795475\\
0.414810370259256	263428.805467299\\
0.414910372759319	263378.958139122\\
0.415010375259382	263328.537853151\\
0.415110377759444	263278.690524974\\
0.415210380259506	263228.270239003\\
0.415310382759569	263178.422910826\\
0.415410385259631	263128.002624855\\
0.415510387759694	263078.155296679\\
0.415610390259756	263027.735010707\\
0.415710392759819	262977.887682531\\
0.415810395259882	262927.467396559\\
0.415910397759944	262877.620068383\\
0.416010400260007	262827.199782411\\
0.416110402760069	262777.352454235\\
0.416210405260131	262726.932168263\\
0.416310407760194	262677.084840087\\
0.416410410260256	262626.664554115\\
0.416510412760319	262576.817225939\\
0.416610415260382	262526.396939968\\
0.416710417760444	262476.549611791\\
0.416810420260507	262426.12932582\\
0.416910422760569	262376.281997643\\
0.417010425260632	262325.861711672\\
0.417110427760694	262276.014383495\\
0.417210430260756	262225.594097524\\
0.417310432760819	262175.746769348\\
0.417410435260882	262125.326483376\\
0.417510437760944	262074.906197404\\
0.417610440261007	262025.058869228\\
0.417710442761069	261974.638583257\\
0.417810445261132	261924.79125508\\
0.417910447761194	261874.370969109\\
0.418010450261257	261824.523640932\\
0.418110452761319	261774.103354961\\
0.418210455261382	261723.683068989\\
0.418310457761444	261673.835740813\\
0.418410460261507	261623.415454841\\
0.418510462761569	261573.568126665\\
0.418610465261632	261523.147840693\\
0.418710467761694	261472.727554722\\
0.418810470261757	261422.880226546\\
0.418910472761819	261372.459940574\\
0.419010475261882	261322.612612398\\
0.419110477761944	261272.192326426\\
0.419210480262007	261221.772040455\\
0.419310482762069	261171.924712278\\
0.419410485262132	261121.504426307\\
0.419510487762194	261071.65709813\\
0.419610490262257	261021.236812159\\
0.419710492762319	260970.816526187\\
0.419810495262382	260920.969198011\\
0.419910497762444	260870.548912039\\
0.420010500262507	260820.128626068\\
0.420110502762569	260770.281297892\\
0.420210505262632	260719.86101192\\
0.420310507762694	260669.440725949\\
0.420410510262757	260619.593397772\\
0.420510512762819	260569.173111801\\
0.420610515262882	260518.752825829\\
0.420710517762944	260468.905497653\\
0.420810520263007	260418.485211681\\
0.420910522763069	260368.06492571\\
0.421010525263132	260318.217597533\\
0.421110527763194	260267.797311562\\
0.421210530263257	260217.37702559\\
0.421310532763319	260167.529697414\\
0.421410535263382	260117.109411442\\
0.421510537763444	260066.689125471\\
0.421610540263507	260016.841797295\\
0.421710542763569	259966.421511323\\
0.421810545263632	259916.001225352\\
0.421910547763694	259865.58093938\\
0.422010550263757	259815.733611204\\
0.422110552763819	259765.313325232\\
0.422210555263882	259714.893039261\\
0.422310557763944	259665.045711084\\
0.422410560264007	259614.625425113\\
0.422510562764069	259564.205139141\\
0.422610565264132	259513.78485317\\
0.422710567764194	259463.937524993\\
0.422810570264257	259413.517239022\\
0.422910572764319	259363.09695305\\
0.423010575264382	259312.676667079\\
0.423110577764444	259262.829338902\\
0.423210580264507	259212.409052931\\
0.423310582764569	259161.988766959\\
0.423410585264632	259111.568480988\\
0.423510587764694	259061.721152811\\
0.423610590264757	259011.30086684\\
0.423710592764819	258960.880580868\\
0.423810595264882	258910.460294897\\
0.423910597764944	258860.612966721\\
0.424010600265007	258810.192680749\\
0.424110602765069	258759.772394778\\
0.424210605265132	258709.352108806\\
0.424310607765194	258658.931822835\\
0.424410610265257	258609.084494658\\
0.424510612765319	258558.664208687\\
0.424610615265382	258508.243922715\\
0.424710617765444	258457.823636744\\
0.424810620265507	258407.403350772\\
0.424910622765569	258357.556022596\\
0.425010625265632	258307.135736624\\
0.425110627765694	258256.715450653\\
0.425210630265757	258206.295164681\\
0.425310632765819	258155.87487871\\
0.425410635265882	258106.027550533\\
0.425510637765944	258055.607264562\\
0.425610640266007	258005.18697859\\
0.425710642766069	257954.766692619\\
0.425810645266132	257904.346406647\\
0.425910647766194	257853.926120676\\
0.426010650266257	257804.078792499\\
0.426110652766319	257753.658506528\\
0.426210655266382	257703.238220556\\
0.426310657766444	257652.817934585\\
0.426410660266507	257602.397648613\\
0.426510662766569	257551.977362642\\
0.426610665266632	257502.130034465\\
0.426710667766694	257451.709748494\\
0.426810670266757	257401.289462522\\
0.426910672766819	257350.869176551\\
0.427010675266882	257300.448890579\\
0.427110677766944	257250.028604608\\
0.427210680267007	257199.608318636\\
0.427310682767069	257149.188032665\\
0.427410685267132	257099.340704488\\
0.427510687767194	257048.920418517\\
0.427610690267257	256998.500132545\\
0.427710692767319	256948.079846574\\
0.427810695267382	256897.659560602\\
0.427910697767444	256847.239274631\\
0.428010700267507	256796.818988659\\
0.428110702767569	256746.398702688\\
0.428210705267632	256695.978416716\\
0.428310707767694	256646.13108854\\
0.428410710267757	256595.710802568\\
0.428510712767819	256545.290516597\\
0.428610715267882	256494.870230625\\
0.428710717767944	256444.449944654\\
0.428810720268007	256394.029658682\\
0.428910722768069	256343.609372711\\
0.429010725268132	256293.189086739\\
0.429110727768194	256242.768800768\\
0.429210730268257	256192.348514796\\
0.429310732768319	256141.928228825\\
0.429410735268382	256091.507942853\\
0.429510737768444	256041.087656882\\
0.429610740268507	255991.240328705\\
0.429710742768569	255940.820042734\\
0.429810745268632	255890.399756762\\
0.429910747768694	255839.979470791\\
0.430010750268757	255789.559184819\\
0.430110752768819	255739.138898848\\
0.430210755268882	255688.718612876\\
0.430310757768944	255638.298326905\\
0.430410760269007	255587.878040933\\
0.430510762769069	255537.457754962\\
0.430610765269132	255487.03746899\\
0.430710767769194	255436.617183019\\
0.430810770269257	255386.196897047\\
0.430910772769319	255335.776611076\\
0.431010775269382	255285.356325104\\
0.431110777769444	255234.936039133\\
0.431210780269507	255184.515753161\\
0.431310782769569	255134.09546719\\
0.431410785269632	255083.675181218\\
0.431510787769694	255033.254895247\\
0.431610790269757	254982.834609275\\
0.431710792769819	254932.414323304\\
0.431810795269882	254881.994037332\\
0.431910797769944	254831.573751361\\
0.432010800270007	254781.153465389\\
0.432110802770069	254730.733179418\\
0.432210805270132	254680.312893446\\
0.432310807770194	254629.892607475\\
0.432410810270257	254579.472321503\\
0.432510812770319	254529.052035532\\
0.432610815270382	254478.63174956\\
0.432710817770444	254428.211463588\\
0.432810820270507	254377.791177617\\
0.432910822770569	254327.370891645\\
0.433010825270632	254276.950605674\\
0.433110827770694	254226.530319702\\
0.433210830270757	254176.110033731\\
0.433310832770819	254125.689747759\\
0.433410835270882	254075.269461788\\
0.433510837770944	254024.849175816\\
0.433610840271007	253974.428889845\\
0.433710842771069	253924.008603873\\
0.433810845271132	253873.588317902\\
0.433910847771194	253823.16803193\\
0.434010850271257	253772.747745959\\
0.434110852771319	253721.754502192\\
0.434210855271382	253671.334216221\\
0.434310857771444	253620.913930249\\
0.434410860271507	253570.493644278\\
0.434510862771569	253520.073358306\\
0.434610865271632	253469.653072335\\
0.434710867771694	253419.232786363\\
0.434810870271757	253368.812500392\\
0.434910872771819	253318.39221442\\
0.435010875271882	253267.971928449\\
0.435110877771944	253217.551642477\\
0.435210880272007	253167.131356506\\
0.435310882772069	253116.711070534\\
0.435410885272132	253066.290784563\\
0.435510887772194	253015.870498591\\
0.435610890272257	252964.877254824\\
0.435710892772319	252914.456968853\\
0.435810895272382	252864.036682881\\
0.435910897772444	252813.61639691\\
0.436010900272507	252763.196110938\\
0.436110902772569	252712.775824967\\
0.436210905272632	252662.355538995\\
0.436310907772694	252611.935253024\\
0.436410910272757	252561.514967052\\
0.436510912772819	252511.094681081\\
0.436610915272882	252460.101437314\\
0.436710917772944	252409.681151343\\
0.436810920273007	252359.260865371\\
0.436910922773069	252308.8405794\\
0.437010925273132	252258.420293428\\
0.437110927773194	252208.000007457\\
0.437210930273257	252157.579721485\\
0.437310932773319	252107.159435514\\
0.437410935273382	252056.166191747\\
0.437510937773444	252005.745905775\\
0.437610940273507	251955.325619804\\
0.437710942773569	251904.905333832\\
0.437810945273632	251854.485047861\\
0.437910947773694	251804.064761889\\
0.438010950273757	251753.644475918\\
0.438110952773819	251703.224189946\\
0.438210955273882	251652.23094618\\
0.438310957773944	251601.810660208\\
0.438410960274007	251551.390374237\\
0.438510962774069	251500.970088265\\
0.438610965274132	251450.549802294\\
0.438710967774194	251400.129516322\\
0.438810970274257	251349.709230351\\
0.438910972774319	251298.715986584\\
0.439010975274382	251248.295700612\\
0.439110977774444	251197.875414641\\
0.439210980274507	251147.455128669\\
0.439310982774569	251097.034842698\\
0.439410985274632	251046.614556726\\
0.439510987774694	250995.62131296\\
0.439610990274757	250945.201026988\\
0.439710992774819	250894.780741017\\
0.439810995274882	250844.360455045\\
0.439910997774944	250793.940169074\\
0.440011000275007	250743.519883102\\
0.440111002775069	250692.526639335\\
0.440211005275132	250642.106353364\\
0.440311007775194	250591.686067392\\
0.440411010275257	250541.265781421\\
0.440511012775319	250490.845495449\\
0.440611015275382	250439.852251683\\
0.440711017775444	250389.431965711\\
0.440811020275507	250339.01167974\\
0.440911022775569	250288.591393768\\
0.441011025275632	250238.171107797\\
0.441111027775694	250187.17786403\\
0.441211030275757	250136.757578059\\
0.441311032775819	250086.337292087\\
0.441411035275882	250035.917006116\\
0.441511037775944	249985.496720144\\
0.441611040276007	249934.503476377\\
0.441711042776069	249884.083190406\\
0.441811045276132	249833.662904434\\
0.441911047776194	249783.242618463\\
0.442011050276257	249732.822332491\\
0.442111052776319	249681.829088725\\
0.442211055276382	249631.408802753\\
0.442311057776444	249580.988516782\\
0.442411060276507	249530.56823081\\
0.442511062776569	249480.147944839\\
0.442611065276632	249429.154701072\\
0.442711067776694	249378.734415101\\
0.442811070276757	249328.314129129\\
0.442911072776819	249277.893843158\\
0.443011075276882	249226.900599391\\
0.443111077776944	249176.480313419\\
0.443211080277007	249126.060027448\\
0.443311082777069	249075.639741476\\
0.443411085277132	249025.219455505\\
0.443511087777194	248974.226211738\\
0.443611090277257	248923.805925767\\
0.443711092777319	248873.385639795\\
0.443811095277382	248822.965353824\\
0.443911097777444	248771.972110057\\
0.444011100277507	248721.551824086\\
0.444111102777569	248671.131538114\\
0.444211105277632	248620.711252142\\
0.444311107777694	248569.718008376\\
0.444411110277757	248519.297722404\\
0.444511112777819	248468.877436433\\
0.444611115277882	248418.457150461\\
0.444711117777944	248367.463906695\\
0.444811120278007	248317.043620723\\
0.444911122778069	248266.623334752\\
0.445011125278132	248216.20304878\\
0.445111127778194	248165.209805013\\
0.445211130278257	248114.789519042\\
0.445311132778319	248064.36923307\\
0.445411135278382	248013.948947099\\
0.445511137778444	247962.955703332\\
0.445611140278507	247912.535417361\\
0.445711142778569	247862.115131389\\
0.445811145278632	247811.694845418\\
0.445911147778694	247760.701601651\\
0.446011150278757	247710.28131568\\
0.446111152778819	247659.861029708\\
0.446211155278882	247608.867785941\\
0.446311157778944	247558.44749997\\
0.446411160279007	247508.027213998\\
0.446511162779069	247457.606928027\\
0.446611165279132	247406.61368426\\
0.446711167779195	247356.193398289\\
0.446811170279257	247305.773112317\\
0.446911172779319	247255.352826346\\
0.447011175279382	247204.359582579\\
0.447111177779444	247153.939296608\\
0.447211180279507	247103.519010636\\
0.447311182779569	247052.525766869\\
0.447411185279632	247002.105480898\\
0.447511187779695	246951.685194926\\
0.447611190279757	246900.69195116\\
0.44771119277982	246850.271665188\\
0.447811195279882	246799.851379217\\
0.447911197779944	246749.431093245\\
0.448011200280007	246698.437849479\\
0.448111202780069	246648.017563507\\
0.448211205280132	246597.597277536\\
0.448311207780195	246546.604033769\\
0.448411210280257	246496.183747797\\
0.44851121278032	246445.763461826\\
0.448611215280382	246395.343175854\\
0.448711217780445	246344.349932088\\
0.448811220280507	246293.929646116\\
0.448911222780569	246243.509360145\\
0.449011225280632	246192.516116378\\
0.449111227780695	246142.095830407\\
0.449211230280757	246091.675544435\\
0.44931123278082	246040.682300668\\
0.449411235280882	245990.262014697\\
0.449511237780945	245939.841728725\\
0.449611240281007	245888.848484959\\
0.44971124278107	245838.428198987\\
0.449811245281132	245788.007913016\\
0.449911247781195	245737.587627044\\
0.450011250281257	245686.594383278\\
0.45011125278132	245636.174097306\\
0.450211255281382	245585.753811334\\
0.450311257781445	245534.760567568\\
0.450411260281507	245484.340281596\\
0.45051126278157	245433.919995625\\
0.450611265281632	245382.926751858\\
0.450711267781695	245332.506465887\\
0.450811270281757	245282.086179915\\
0.45091127278182	245231.092936149\\
0.451011275281882	245180.672650177\\
0.451111277781945	245130.252364206\\
0.451211280282007	245079.259120439\\
0.45131128278207	245028.838834467\\
0.451411285282132	244978.418548496\\
0.451511287782195	244927.425304729\\
0.451611290282257	244877.005018758\\
0.45171129278232	244826.584732786\\
0.451811295282382	244775.59148902\\
0.451911297782445	244725.171203048\\
0.452011300282507	244674.750917077\\
0.45211130278257	244623.75767331\\
0.452211305282632	244573.337387338\\
0.452311307782695	244522.917101367\\
0.452411310282757	244471.9238576\\
0.45251131278282	244421.503571629\\
0.452611315282882	244371.083285657\\
0.452711317782945	244320.090041891\\
0.452811320283007	244269.669755919\\
0.45291132278307	244219.249469948\\
0.453011325283132	244168.256226181\\
0.453111327783195	244117.835940209\\
0.453211330283257	244067.415654238\\
0.45331133278332	244016.422410471\\
0.453411335283382	243966.0021245\\
0.453511337783445	243915.581838528\\
0.453611340283507	243864.588594762\\
0.45371134278357	243814.16830879\\
0.453811345283632	243763.175065023\\
0.453911347783695	243712.754779052\\
0.454011350283757	243662.33449308\\
0.45411135278382	243611.341249314\\
0.454211355283882	243560.920963342\\
0.454311357783945	243510.500677371\\
0.454411360284007	243459.507433604\\
0.45451136278407	243409.087147632\\
0.454611365284132	243358.666861661\\
0.454711367784195	243307.673617894\\
0.454811370284257	243257.253331923\\
0.45491137278432	243206.833045951\\
0.455011375284382	243155.839802185\\
0.455111377784445	243105.419516213\\
0.455211380284507	243054.999230242\\
0.45531138278457	243004.005986475\\
0.455411385284632	242953.585700504\\
0.455511387784695	242902.592456737\\
0.455611390284757	242852.172170765\\
0.45571139278482	242801.751884794\\
0.455811395284882	242750.758641027\\
0.455911397784945	242700.338355056\\
0.456011400285007	242649.918069084\\
0.45611140278507	242598.924825317\\
0.456211405285132	242548.504539346\\
0.456311407785195	242498.084253375\\
0.456411410285257	242447.091009608\\
0.45651141278532	242396.670723636\\
0.456611415285382	242345.67747987\\
0.456711417785445	242295.257193898\\
0.456811420285507	242244.836907927\\
0.45691142278557	242193.84366416\\
0.457011425285632	242143.423378189\\
0.457111427785695	242093.003092217\\
0.457211430285757	242042.00984845\\
0.45731143278582	241991.589562479\\
0.457411435285882	241941.169276507\\
0.457511437785945	241890.176032741\\
0.457611440286007	241839.755746769\\
0.45771144278607	241788.762503003\\
0.457811445286132	241738.342217031\\
0.457911447786195	241687.92193106\\
0.458011450286257	241636.928687293\\
0.45811145278632	241586.508401321\\
0.458211455286382	241536.08811535\\
0.458311457786445	241485.094871583\\
0.458411460286507	241434.674585612\\
0.45851146278657	241383.681341845\\
0.458611465286632	241333.261055874\\
0.458711467786695	241282.840769902\\
0.458811470286757	241231.847526135\\
0.45891147278682	241181.427240164\\
0.459011475286882	241131.006954192\\
0.459111477786945	241080.013710426\\
0.459211480287007	241029.593424454\\
0.45931148278707	240978.600180688\\
0.459411485287132	240928.179894716\\
0.459511487787195	240877.759608744\\
0.459611490287257	240826.766364978\\
0.45971149278732	240776.346079006\\
0.459811495287382	240725.925793035\\
0.459911497787445	240674.932549268\\
0.460011500287507	240624.512263297\\
0.46011150278757	240573.51901953\\
0.460211505287632	240523.098733559\\
0.460311507787695	240472.678447587\\
0.460411510287757	240421.68520382\\
0.46051151278782	240371.264917849\\
0.460611515287882	240320.844631877\\
0.460711517787945	240269.851388111\\
0.460811520288007	240219.431102139\\
0.46091152278807	240168.437858373\\
0.461011525288132	240118.017572401\\
0.461111527788195	240067.59728643\\
0.461211530288257	240016.604042663\\
0.46131153278832	239966.183756691\\
0.461411535288382	239915.190512925\\
0.461511537788445	239864.770226953\\
0.461611540288507	239814.349940982\\
0.46171154278857	239763.356697215\\
0.461811545288632	239712.936411244\\
0.461911547788695	239662.516125272\\
0.462011550288757	239611.522881505\\
0.46211155278882	239561.102595534\\
0.462211555288882	239510.109351767\\
0.462311557788945	239459.689065796\\
0.462411560289007	239409.268779824\\
0.46251156278907	239358.275536058\\
0.462611565289132	239307.855250086\\
0.462711567789195	239257.434964115\\
0.462811570289257	239206.441720348\\
0.46291157278932	239156.021434376\\
0.463011575289382	239105.02819061\\
0.463111577789445	239054.607904638\\
0.463211580289507	239004.187618667\\
0.46331158278957	238953.1943749\\
0.463411585289632	238902.774088929\\
0.463511587789695	238852.353802957\\
0.463611590289757	238801.36055919\\
0.46371159278982	238750.940273219\\
0.463811595289882	238699.947029452\\
0.463911597789945	238649.526743481\\
0.464011600290007	238599.106457509\\
0.46411160279007	238548.113213743\\
0.464211605290132	238497.692927771\\
0.464311607790195	238446.699684004\\
0.464411610290257	238396.279398033\\
0.46451161279032	238345.859112061\\
0.464611615290382	238294.865868295\\
0.464711617790445	238244.445582323\\
0.464811620290507	238194.025296352\\
0.46491162279057	238143.032052585\\
0.465011625290632	238092.611766614\\
0.465111627790695	238041.618522847\\
0.465211630290757	237991.198236875\\
0.46531163279082	237940.777950904\\
0.465411635290882	237889.784707137\\
0.465511637790945	237839.364421166\\
0.465611640291007	237788.944135194\\
0.46571164279107	237737.950891428\\
0.465811645291132	237687.530605456\\
0.465911647791195	237636.537361689\\
0.466011650291257	237586.117075718\\
0.46611165279132	237535.696789746\\
0.466211655291382	237484.70354598\\
0.466311657791445	237434.283260008\\
0.466411660291507	237383.862974037\\
0.46651166279157	237332.86973027\\
0.466611665291632	237282.449444299\\
0.466711667791695	237231.456200532\\
0.466811670291757	237181.03591456\\
0.46691167279182	237130.615628589\\
0.467011675291882	237079.622384822\\
0.467111677791945	237029.202098851\\
0.467211680292007	236978.781812879\\
0.46731168279207	236927.788569113\\
0.467411685292132	236877.368283141\\
0.467511687792195	236826.375039374\\
0.467611690292257	236775.954753403\\
0.46771169279232	236725.534467431\\
0.467811695292382	236674.541223665\\
0.467911697792445	236624.120937693\\
0.468011700292507	236573.700651722\\
0.46811170279257	236522.707407955\\
0.468211705292632	236472.287121984\\
0.468311707792695	236421.293878217\\
0.468411710292757	236370.873592245\\
0.46851171279282	236320.453306274\\
0.468611715292882	236269.460062507\\
0.468711717792945	236219.039776536\\
0.468811720293007	236168.619490564\\
0.46891172279307	236117.626246798\\
0.469011725293132	236067.205960826\\
0.469111727793195	236016.785674855\\
0.469211730293257	235965.792431088\\
0.46931173279332	235915.372145116\\
0.469411735293382	235864.37890135\\
0.469511737793445	235813.958615378\\
0.469611740293507	235763.538329407\\
0.46971174279357	235712.54508564\\
0.469811745293632	235662.124799669\\
0.469911747793695	235611.704513697\\
0.470011750293757	235560.71126993\\
0.47011175279382	235510.290983959\\
0.470211755293882	235459.870697987\\
0.470311757793945	235408.877454221\\
0.470411760294007	235358.457168249\\
0.47051176279407	235307.463924483\\
0.470611765294132	235257.043638511\\
0.470711767794195	235206.62335254\\
0.470811770294257	235155.630108773\\
0.47091177279432	235105.209822801\\
0.471011775294382	235054.78953683\\
0.471111777794445	235003.796293063\\
0.471211780294507	234953.376007092\\
0.47131178279457	234902.95572112\\
0.471411785294632	234851.962477354\\
0.471511787794695	234801.542191382\\
0.471611790294757	234751.121905411\\
0.47171179279482	234700.128661644\\
0.471811795294882	234649.708375672\\
0.471911797794945	234599.288089701\\
0.472011800295007	234548.294845934\\
0.47211180279507	234497.874559963\\
0.472211805295132	234446.881316196\\
0.472311807795195	234396.461030225\\
0.472411810295257	234346.040744253\\
0.47251181279532	234295.047500486\\
0.472611815295382	234244.627214515\\
0.472711817795445	234194.206928543\\
0.472811820295507	234143.213684777\\
0.47291182279557	234092.793398805\\
0.473011825295632	234042.373112834\\
0.473111827795695	233991.379869067\\
0.473211830295757	233940.959583096\\
0.47331183279582	233890.539297124\\
0.473411835295882	233839.546053357\\
0.473511837795945	233789.125767386\\
0.473611840296007	233738.705481414\\
0.47371184279607	233687.712237648\\
0.473811845296132	233637.291951676\\
0.473911847796195	233586.871665705\\
0.474011850296257	233535.878421938\\
0.47411185279632	233485.458135967\\
0.474211855296382	233435.037849995\\
0.474311857796445	233384.044606228\\
0.474411860296507	233333.624320257\\
0.47451186279657	233283.204034285\\
0.474611865296632	233232.210790519\\
0.474711867796695	233181.790504547\\
0.474811870296757	233131.370218576\\
0.47491187279682	233080.376974809\\
0.475011875296882	233029.956688838\\
0.475111877796945	232979.536402866\\
0.475211880297007	232928.543159099\\
0.47531188279707	232878.122873128\\
0.475411885297132	232827.702587156\\
0.475511887797195	232776.70934339\\
0.475611890297257	232726.289057418\\
0.47571189279732	232675.868771447\\
0.475811895297382	232624.87552768\\
0.475911897797445	232574.455241709\\
0.476011900297507	232524.034955737\\
0.47611190279757	232473.614669766\\
0.476211905297632	232422.621425999\\
0.476311907797695	232372.201140027\\
0.476411910297757	232321.780854056\\
0.47651191279782	232270.787610289\\
0.476611915297882	232220.367324318\\
0.476711917797945	232169.947038346\\
0.476811920298007	232118.95379458\\
0.47691192279807	232068.533508608\\
0.477011925298132	232018.113222637\\
0.477111927798195	231967.11997887\\
0.477211930298257	231916.699692898\\
0.47731193279832	231866.279406927\\
0.477411935298382	231815.859120955\\
0.477511937798445	231764.865877189\\
0.477611940298507	231714.445591217\\
0.47771194279857	231664.025305246\\
0.477811945298632	231613.032061479\\
0.477911947798695	231562.611775508\\
0.478011950298757	231512.191489536\\
0.47811195279882	231461.771203565\\
0.478211955298882	231410.777959798\\
0.478311957798945	231360.357673826\\
0.478411960299007	231309.937387855\\
0.47851196279907	231258.944144088\\
0.478611965299132	231208.523858117\\
0.478711967799195	231158.103572145\\
0.478811970299257	231107.110328379\\
0.47891197279932	231056.690042407\\
0.479011975299382	231006.269756436\\
0.479111977799445	230955.849470464\\
0.479211980299508	230904.856226697\\
0.47931198279957	230854.435940726\\
0.479411985299632	230804.015654754\\
0.479511987799695	230753.595368783\\
0.479611990299757	230702.602125016\\
0.47971199279982	230652.181839045\\
0.479811995299882	230601.761553073\\
0.479911997799945	230550.768309307\\
0.480012000300008	230500.348023335\\
0.48011200280007	230449.927737364\\
0.480212005300133	230399.507451392\\
0.480312007800195	230348.514207625\\
0.480412010300257	230298.093921654\\
0.48051201280032	230247.673635682\\
0.480612015300383	230197.253349711\\
0.480712017800445	230146.260105944\\
0.480812020300508	230095.839819973\\
0.48091202280057	230045.419534001\\
0.481012025300633	229994.99924803\\
0.481112027800695	229944.006004263\\
0.481212030300758	229893.585718292\\
0.48131203280082	229843.16543232\\
0.481412035300883	229792.745146348\\
0.481512037800945	229741.751902582\\
0.481612040301008	229691.33161661\\
0.48171204280107	229640.911330639\\
0.481812045301133	229590.491044667\\
0.481912047801195	229539.497800901\\
0.482012050301258	229489.077514929\\
0.48211205280132	229438.657228958\\
0.482212055301383	229388.236942986\\
0.482312057801445	229337.243699219\\
0.482412060301508	229286.823413248\\
0.48251206280157	229236.403127276\\
0.482612065301633	229185.982841305\\
0.482712067801695	229134.989597538\\
0.482812070301758	229084.569311567\\
0.48291207280182	229034.149025595\\
0.483012075301883	228983.728739624\\
0.483112077801945	228933.308453652\\
0.483212080302008	228882.315209886\\
0.48331208280207	228831.894923914\\
0.483412085302133	228781.474637943\\
0.483512087802195	228731.054351971\\
0.483612090302258	228680.061108204\\
0.48371209280232	228629.640822233\\
0.483812095302383	228579.220536261\\
0.483912097802445	228528.80025029\\
0.484012100302508	228478.379964318\\
0.48411210280257	228427.386720552\\
0.484212105302633	228376.96643458\\
0.484312107802695	228326.546148609\\
0.484412110302758	228276.125862637\\
0.48451211280282	228225.705576666\\
0.484612115302883	228174.712332899\\
0.484712117802945	228124.292046928\\
0.484812120303008	228073.871760956\\
0.48491212280307	228023.451474985\\
0.485012125303133	227973.031189013\\
0.485112127803195	227922.037945246\\
0.485212130303258	227871.617659275\\
0.48531213280332	227821.197373303\\
0.485412135303383	227770.777087332\\
0.485512137803445	227720.35680136\\
0.485612140303508	227669.363557594\\
0.48571214280357	227618.943271622\\
0.485812145303633	227568.522985651\\
0.485912147803695	227518.102699679\\
0.486012150303758	227467.682413708\\
0.48611215280382	227416.689169941\\
0.486212155303883	227366.268883969\\
0.486312157803945	227315.848597998\\
0.486412160304008	227265.428312026\\
0.48651216280407	227215.008026055\\
0.486612165304133	227164.587740083\\
0.486712167804195	227113.594496317\\
0.486812170304258	227063.174210345\\
0.48691217280432	227012.753924374\\
0.487012175304383	226962.333638402\\
0.487112177804445	226911.913352431\\
0.487212180304508	226861.493066459\\
0.48731218280457	226811.072780488\\
0.487412185304633	226760.079536721\\
0.487512187804695	226709.65925075\\
0.487612190304758	226659.238964778\\
0.48771219280482	226608.818678807\\
0.487812195304883	226558.398392835\\
0.487912197804945	226507.978106864\\
0.488012200305008	226456.984863097\\
0.48811220280507	226406.564577125\\
0.488212205305133	226356.144291154\\
0.488312207805195	226305.724005182\\
0.488412210305258	226255.303719211\\
0.48851221280532	226204.883433239\\
0.488612215305383	226154.463147268\\
0.488712217805445	226104.042861296\\
0.488812220305508	226053.04961753\\
0.48891222280557	226002.629331558\\
0.489012225305633	225952.209045587\\
0.489112227805695	225901.788759615\\
0.489212230305758	225851.368473644\\
0.48931223280582	225800.948187672\\
0.489412235305883	225750.527901701\\
0.489512237805945	225700.107615729\\
0.489612240306008	225649.114371962\\
0.48971224280607	225598.694085991\\
0.489812245306133	225548.273800019\\
0.489912247806195	225497.853514048\\
0.490012250306258	225447.433228076\\
0.49011225280632	225397.012942105\\
0.490212255306383	225346.592656133\\
0.490312257806445	225296.172370162\\
0.490412260306508	225245.75208419\\
0.49051226280657	225195.331798219\\
0.490612265306633	225144.338554452\\
0.490712267806695	225093.918268481\\
0.490812270306758	225043.497982509\\
0.49091227280682	224993.077696538\\
0.491012275306883	224942.657410566\\
0.491112277806945	224892.237124595\\
0.491212280307008	224841.816838623\\
0.49131228280707	224791.396552652\\
0.491412285307133	224740.97626668\\
0.491512287807195	224690.555980709\\
0.491612290307258	224640.135694737\\
0.49171229280732	224589.715408765\\
0.491812295307383	224538.722164999\\
0.491912297807445	224488.301879027\\
0.492012300307508	224437.881593056\\
0.49211230280757	224387.461307084\\
0.492212305307633	224337.041021113\\
0.492312307807695	224286.620735141\\
0.492412310307758	224236.20044917\\
0.49251231280782	224185.780163198\\
0.492612315307883	224135.359877227\\
0.492712317807945	224084.939591255\\
0.492812320308008	224034.519305284\\
0.49291232280807	223984.099019312\\
0.493012325308133	223933.678733341\\
0.493112327808195	223883.258447369\\
0.493212330308258	223832.838161398\\
0.49331233280832	223782.417875426\\
0.493412335308383	223731.997589455\\
0.493512337808445	223681.577303483\\
0.493612340308508	223631.157017512\\
0.49371234280857	223580.73673154\\
0.493812345308633	223530.316445569\\
0.493912347808695	223479.896159597\\
0.494012350308758	223429.475873626\\
0.49411235280882	223379.055587654\\
0.494212355308883	223328.635301683\\
0.494312357808945	223278.215015711\\
0.494412360309008	223227.79472974\\
0.49451236280907	223177.374443768\\
0.494612365309133	223126.954157796\\
0.494712367809195	223076.533871825\\
0.494812370309258	223026.113585853\\
0.49491237280932	222975.693299882\\
0.495012375309383	222925.27301391\\
0.495112377809445	222874.852727939\\
0.495212380309508	222824.432441967\\
0.49531238280957	222774.012155996\\
0.495412385309633	222723.591870024\\
0.495512387809695	222673.171584053\\
0.495612390309758	222622.751298081\\
0.49571239280982	222572.33101211\\
0.495812395309883	222521.910726138\\
0.495912397809945	222471.490440167\\
0.496012400310008	222421.070154195\\
0.49611240281007	222370.649868224\\
0.496212405310133	222320.229582252\\
0.496312407810195	222269.809296281\\
0.496412410310258	222219.389010309\\
0.49651241281032	222168.968724338\\
0.496612415310383	222118.548438366\\
0.496712417810445	222068.128152395\\
0.496812420310508	222017.707866423\\
0.49691242281057	221967.287580452\\
0.497012425310633	221916.86729448\\
0.497112427810695	221866.447008509\\
0.497212430310758	221816.026722537\\
0.49731243281082	221765.606436566\\
0.497412435310883	221715.186150594\\
0.497512437810945	221664.765864623\\
0.497612440311008	221614.345578651\\
0.49771244281107	221564.498250475\\
0.497812445311133	221514.077964503\\
0.497912447811195	221463.657678532\\
0.498012450311258	221413.23739256\\
0.49811245281132	221362.817106589\\
0.498212455311383	221312.396820617\\
0.498312457811445	221261.976534646\\
0.498412460311508	221211.556248674\\
0.49851246281157	221161.135962703\\
0.498612465311633	221110.715676731\\
0.498712467811695	221060.29539076\\
0.498812470311758	221010.448062583\\
0.49891247281182	220960.027776612\\
0.499012475311883	220909.60749064\\
0.499112477811945	220859.187204669\\
0.499212480312008	220808.766918697\\
0.49931248281207	220758.346632726\\
0.499412485312133	220707.926346754\\
0.499512487812195	220657.506060783\\
0.499612490312258	220607.085774811\\
0.49971249281232	220557.238446635\\
0.499812495312383	220506.818160663\\
0.499912497812445	220456.397874692\\
0.500012500312508	220405.97758872\\
0.50011250281257	220355.557302749\\
0.500212505312633	220305.137016777\\
0.500312507812695	220254.716730806\\
0.500412510312758	220204.296444834\\
0.50051251281282	220154.449116658\\
0.500612515312883	220104.028830686\\
0.500712517812945	220053.608544715\\
0.500812520313008	220003.188258743\\
0.50091252281307	219952.767972772\\
0.501012525313133	219902.3476868\\
0.501112527813195	219852.500358624\\
0.501212530313258	219802.080072652\\
0.50131253281332	219751.659786681\\
0.501412535313383	219701.239500709\\
0.501512537813445	219650.819214738\\
0.501612540313508	219600.398928766\\
0.50171254281357	219550.55160059\\
0.501812545313633	219500.131314618\\
0.501912547813695	219449.711028647\\
0.502012550313758	219399.290742675\\
0.50211255281382	219348.870456704\\
0.502212555313883	219298.450170732\\
0.502312557813945	219248.602842556\\
0.502412560314008	219198.182556584\\
0.50251256281407	219147.762270613\\
0.502612565314133	219097.341984641\\
0.502712567814195	219047.494656465\\
0.502812570314258	218997.074370494\\
0.50291257281432	218946.654084522\\
0.503012575314383	218896.233798551\\
0.503112577814445	218845.813512579\\
0.503212580314508	218795.966184403\\
0.50331258281457	218745.545898431\\
0.503412585314633	218695.12561246\\
0.503512587814695	218644.705326488\\
0.503612590314758	218594.285040517\\
0.50371259281482	218544.43771234\\
0.503812595314883	218494.017426369\\
0.503912597814945	218443.597140397\\
0.504012600315008	218393.176854426\\
0.50411260281507	218343.329526249\\
0.504212605315133	218292.909240278\\
0.504312607815195	218242.488954306\\
0.504412610315258	218192.068668335\\
0.50451261281532	218142.221340158\\
0.504612615315383	218091.801054187\\
0.504712617815445	218041.380768215\\
0.504812620315508	217991.533440039\\
0.50491262281557	217941.113154067\\
0.505012625315633	217890.692868096\\
0.505112627815695	217840.272582124\\
0.505212630315758	217790.425253948\\
0.50531263281582	217740.004967977\\
0.505412635315883	217689.584682005\\
0.505512637815945	217639.164396033\\
0.505612640316008	217589.317067857\\
0.50571264281607	217538.896781886\\
0.505812645316133	217488.476495914\\
0.505912647816195	217438.629167738\\
0.506012650316258	217388.208881766\\
0.50611265281632	217337.788595795\\
0.506212655316383	217287.941267618\\
0.506312657816445	217237.520981647\\
0.506412660316508	217187.100695675\\
0.50651266281657	217137.253367499\\
0.506612665316633	217086.833081527\\
0.506712667816695	217036.412795556\\
0.506812670316758	216986.565467379\\
0.50691267281682	216936.145181408\\
0.507012675316883	216885.724895436\\
0.507112677816945	216835.87756726\\
0.507212680317008	216785.457281289\\
0.50731268281707	216735.036995317\\
0.507412685317133	216685.189667141\\
0.507512687817195	216634.769381169\\
0.507612690317258	216584.349095198\\
0.50771269281732	216534.501767021\\
0.507812695317383	216484.08148105\\
0.507912697817445	216433.661195078\\
0.508012700317508	216383.813866902\\
0.50811270281757	216333.39358093\\
0.508212705317633	216282.973294959\\
0.508312707817695	216233.125966782\\
0.508412710317758	216182.705680811\\
0.50851271281782	216132.858352635\\
0.508612715317883	216082.438066663\\
0.508712717817945	216032.017780692\\
0.508812720318008	215982.170452515\\
0.50891272281807	215931.750166544\\
0.509012725318133	215881.902838367\\
0.509112727818195	215831.482552396\\
0.509212730318258	215781.062266424\\
0.50931273281832	215731.214938248\\
0.509412735318383	215680.794652276\\
0.509512737818445	215630.9473241\\
0.509612740318508	215580.527038128\\
0.50971274281857	215530.679709952\\
0.509812745318633	215480.259423981\\
0.509912747818695	215429.839138009\\
0.510012750318758	215379.991809833\\
0.510112752818821	215329.571523861\\
0.510212755318883	215279.724195685\\
0.510312757818945	215229.303909713\\
0.510412760319008	215179.456581537\\
0.51051276281907	215129.036295565\\
0.510612765319133	215078.616009594\\
0.510712767819196	215028.768681417\\
0.510812770319258	214978.348395446\\
0.51091277281932	214928.50106727\\
0.511012775319383	214878.080781298\\
0.511112777819446	214828.233453122\\
0.511212780319508	214777.81316715\\
0.51131278281957	214727.965838974\\
0.511412785319633	214677.545553002\\
0.511512787819695	214627.698224826\\
0.511612790319758	214577.277938854\\
0.511712792819821	214527.430610678\\
0.511812795319883	214477.010324706\\
0.511912797819945	214427.16299653\\
0.512012800320008	214376.742710559\\
0.512112802820071	214326.895382382\\
0.512212805320133	214276.475096411\\
0.512312807820196	214226.627768234\\
0.512412810320258	214176.207482263\\
0.51251281282032	214126.360154086\\
0.512612815320383	214075.939868115\\
0.512712817820446	214026.092539939\\
0.512812820320508	213975.672253967\\
0.51291282282057	213925.824925791\\
0.513012825320633	213875.404639819\\
0.513112827820696	213825.557311643\\
0.513212830320758	213775.709983466\\
0.513312832820821	213725.289697495\\
0.513412835320883	213675.442369318\\
0.513512837820945	213625.022083347\\
0.513612840321008	213575.174755171\\
0.513712842821071	213524.754469199\\
0.513812845321133	213474.907141023\\
0.513912847821196	213424.486855051\\
0.514012850321258	213374.639526875\\
0.514112852821321	213324.792198698\\
0.514212855321383	213274.371912727\\
0.514312857821446	213224.52458455\\
0.514412860321508	213174.104298579\\
0.51451286282157	213124.256970403\\
0.514612865321633	213074.409642226\\
0.514712867821696	213023.989356255\\
0.514812870321758	212974.142028078\\
0.514912872821821	212923.721742107\\
0.515012875321883	212873.87441393\\
0.515112877821946	212824.027085754\\
0.515212880322008	212773.606799783\\
0.515312882822071	212723.759471606\\
0.515412885322133	212673.339185635\\
0.515512887822196	212623.491857458\\
0.515612890322258	212573.644529282\\
0.515712892822321	212523.22424331\\
0.515812895322383	212473.376915134\\
0.515912897822446	212423.529586958\\
0.516012900322508	212373.109300986\\
0.516112902822571	212323.26197281\\
0.516212905322633	212273.414644633\\
0.516312907822696	212222.994358662\\
0.516412910322758	212173.147030485\\
0.516512912822821	212123.299702309\\
0.516612915322883	212072.879416337\\
0.516712917822946	212023.032088161\\
0.516812920323008	211973.184759985\\
0.516912922823071	211922.764474013\\
0.517012925323133	211872.917145837\\
0.517112927823196	211823.06981766\\
0.517212930323258	211772.649531689\\
0.517312932823321	211722.802203513\\
0.517412935323383	211672.954875336\\
0.517512937823446	211622.534589365\\
0.517612940323508	211572.687261188\\
0.517712942823571	211522.839933012\\
0.517812945323633	211472.992604836\\
0.517912947823696	211422.572318864\\
0.518012950323758	211372.724990688\\
0.518112952823821	211322.877662511\\
0.518212955323883	211272.45737654\\
0.518312957823946	211222.610048363\\
0.518412960324008	211172.762720187\\
0.518512962824071	211122.915392011\\
0.518612965324133	211072.495106039\\
0.518712967824196	211022.647777863\\
0.518812970324258	210972.800449686\\
0.518912972824321	210922.95312151\\
0.519012975324383	210872.532835538\\
0.519112977824446	210822.685507362\\
0.519212980324508	210772.838179186\\
0.519312982824571	210722.990851009\\
0.519412985324633	210673.143522833\\
0.519512987824696	210622.723236861\\
0.519612990324758	210572.875908685\\
0.519712992824821	210523.028580509\\
0.519812995324883	210473.181252332\\
0.519912997824946	210423.333924156\\
0.520013000325008	210372.913638184\\
0.520113002825071	210323.066310008\\
0.520213005325133	210273.218981832\\
0.520313007825196	210223.371653655\\
0.520413010325258	210173.524325479\\
0.520513012825321	210123.104039507\\
0.520613015325383	210073.256711331\\
0.520713017825446	210023.409383155\\
0.520813020325508	209973.562054978\\
0.520913022825571	209923.714726802\\
0.521013025325633	209873.867398625\\
0.521113027825696	209823.447112654\\
0.521213030325758	209773.599784477\\
0.521313032825821	209723.752456301\\
0.521413035325883	209673.905128125\\
0.521513037825946	209624.057799948\\
0.521613040326008	209574.210471772\\
0.521713042826071	209524.363143596\\
0.521813045326133	209473.942857624\\
0.521913047826196	209424.095529448\\
0.522013050326258	209374.248201271\\
0.522113052826321	209324.400873095\\
0.522213055326383	209274.553544919\\
0.522313057826446	209224.706216742\\
0.522413060326508	209174.858888566\\
0.522513062826571	209125.011560389\\
0.522613065326633	209075.164232213\\
0.522713067826696	209025.316904037\\
0.522813070326758	208974.896618065\\
0.522913072826821	208925.049289889\\
0.523013075326883	208875.201961712\\
0.523113077826946	208825.354633536\\
0.523213080327008	208775.50730536\\
0.523313082827071	208725.659977183\\
0.523413085327133	208675.812649007\\
0.523513087827196	208625.96532083\\
0.523613090327258	208576.117992654\\
0.523713092827321	208526.270664478\\
0.523813095327383	208476.423336301\\
0.523913097827446	208426.576008125\\
0.524013100327508	208376.728679949\\
0.524113102827571	208326.881351772\\
0.524213105327633	208277.034023596\\
0.524313107827696	208227.186695419\\
0.524413110327758	208177.339367243\\
0.524513112827821	208127.492039067\\
0.524613115327883	208077.64471089\\
0.524713117827946	208027.797382714\\
0.524813120328008	207977.950054538\\
0.524913122828071	207928.102726361\\
0.525013125328133	207878.255398185\\
0.525113127828196	207828.408070008\\
0.525213130328258	207778.560741832\\
0.525313132828321	207728.713413656\\
0.525413135328383	207678.866085479\\
0.525513137828446	207629.018757303\\
0.525613140328508	207579.171429126\\
0.525713142828571	207529.32410095\\
0.525813145328633	207479.476772774\\
0.525913147828696	207429.629444597\\
0.526013150328758	207379.782116421\\
0.526113152828821	207329.934788245\\
0.526213155328883	207280.087460068\\
0.526313157828946	207230.240131892\\
0.526413160329008	207180.392803715\\
0.526513162829071	207130.545475539\\
0.526613165329133	207080.698147363\\
0.526713167829196	207031.423776981\\
0.526813170329258	206981.576448805\\
0.526913172829321	206931.729120629\\
0.527013175329383	206881.881792452\\
0.527113177829446	206832.034464276\\
0.527213180329508	206782.187136099\\
0.527313182829571	206732.339807923\\
0.527413185329633	206682.492479747\\
0.527513187829696	206632.64515157\\
0.527613190329758	206582.797823394\\
0.527713192829821	206533.523453013\\
0.527813195329883	206483.676124836\\
0.527913197829946	206433.82879666\\
0.528013200330008	206383.981468484\\
0.528113202830071	206334.134140307\\
0.528213205330133	206284.286812131\\
0.528313207830196	206234.439483954\\
0.528413210330258	206185.165113573\\
0.528513212830321	206135.317785397\\
0.528613215330383	206085.47045722\\
0.528713217830446	206035.623129044\\
0.528813220330508	205985.775800868\\
0.528913222830571	205936.501430486\\
0.529013225330633	205886.65410231\\
0.529113227830696	205836.806774134\\
0.529213230330758	205786.959445957\\
0.529313232830821	205737.112117781\\
0.529413235330883	205687.8377474\\
0.529513237830946	205637.990419223\\
0.529613240331008	205588.143091047\\
0.529713242831071	205538.29576287\\
0.529813245331133	205488.448434694\\
0.529913247831196	205439.174064313\\
0.530013250331258	205389.326736136\\
0.530113252831321	205339.47940796\\
0.530213255331383	205289.632079784\\
0.530313257831446	205240.357709402\\
0.530413260331508	205190.510381226\\
0.530513262831571	205140.66305305\\
0.530613265331633	205090.815724873\\
0.530713267831696	205041.541354492\\
0.530813270331758	204991.694026316\\
0.530913272831821	204941.846698139\\
0.531013275331883	204891.999369963\\
0.531113277831946	204842.724999582\\
0.531213280332008	204792.877671405\\
0.531313282832071	204743.030343229\\
0.531413285332133	204693.755972848\\
0.531513287832196	204643.908644671\\
0.531613290332258	204594.061316495\\
0.531713292832321	204544.786946114\\
0.531813295332383	204494.939617937\\
0.531913297832446	204445.092289761\\
0.532013300332508	204395.244961584\\
0.532113302832571	204345.970591203\\
0.532213305332633	204296.123263027\\
0.532313307832696	204246.848892646\\
0.532413310332758	204197.001564469\\
0.532513312832821	204147.154236293\\
0.532613315332883	204097.879865912\\
0.532713317832946	204048.032537735\\
0.532813320333008	203998.185209559\\
0.532913322833071	203948.910839178\\
0.533013325333133	203899.063511001\\
0.533113327833196	203849.216182825\\
0.533213330333258	203799.941812444\\
0.533313332833321	203750.094484267\\
0.533413335333383	203700.820113886\\
0.533513337833446	203650.97278571\\
0.533613340333508	203601.125457533\\
0.533713342833571	203551.851087152\\
0.533813345333633	203502.003758976\\
0.533913347833696	203452.729388594\\
0.534013350333758	203402.882060418\\
0.534113352833821	203353.034732242\\
0.534213355333883	203303.76036186\\
0.534313357833946	203253.913033684\\
0.534413360334008	203204.638663303\\
0.534513362834071	203154.791335126\\
0.534613365334133	203105.516964745\\
0.534713367834196	203055.669636569\\
0.534813370334258	203006.395266187\\
0.534913372834321	202956.547938011\\
0.535013375334383	202907.27356763\\
0.535113377834446	202857.426239453\\
0.535213380334508	202808.151869072\\
0.535313382834571	202758.304540896\\
0.535413385334633	202708.457212719\\
0.535513387834696	202659.182842338\\
0.535613390334758	202609.335514162\\
0.535713392834821	202560.06114378\\
0.535813395334883	202510.786773399\\
0.535913397834946	202460.939445223\\
0.536013400335008	202411.665074842\\
0.536113402835071	202361.817746665\\
0.536213405335133	202312.543376284\\
0.536313407835196	202262.696048108\\
0.536413410335258	202213.421677726\\
0.536513412835321	202163.57434955\\
0.536613415335383	202114.299979169\\
0.536713417835446	202064.452650992\\
0.536813420335508	202015.178280611\\
0.536913422835571	201965.330952435\\
0.537013425335633	201916.056582053\\
0.537113427835696	201866.782211672\\
0.537213430335758	201816.934883496\\
0.537313432835821	201767.660513115\\
0.537413435335883	201717.813184938\\
0.537513437835946	201668.538814557\\
0.537613440336008	201619.264444176\\
0.537713442836071	201569.417115999\\
0.537813445336133	201520.142745618\\
0.537913447836196	201470.295417442\\
0.538013450336258	201421.02104706\\
0.538113452836321	201371.746676679\\
0.538213455336383	201321.899348503\\
0.538313457836446	201272.624978121\\
0.538413460336508	201223.35060774\\
0.538513462836571	201173.503279564\\
0.538613465336633	201124.228909183\\
0.538713467836696	201074.954538801\\
0.538813470336758	201025.107210625\\
0.538913472836821	200975.832840244\\
0.539013475336883	200926.558469863\\
0.539113477836946	200876.711141686\\
0.539213480337008	200827.436771305\\
0.539313482837071	200778.162400924\\
0.539413485337133	200728.315072747\\
0.539513487837196	200679.040702366\\
0.539613490337258	200629.766331985\\
0.539713492837321	200579.919003808\\
0.539813495337383	200530.644633427\\
0.539913497837446	200481.370263046\\
0.540013500337508	200432.095892665\\
0.540113502837571	200382.248564488\\
0.540213505337633	200332.974194107\\
0.540313507837696	200283.699823726\\
0.540413510337758	200233.852495549\\
0.540513512837821	200184.578125168\\
0.540613515337883	200135.303754787\\
0.540713517837946	200086.029384406\\
0.540813520338008	200036.755014024\\
0.540913522838071	199986.907685848\\
0.541013525338133	199937.633315467\\
0.541113527838196	199888.358945085\\
0.541213530338258	199839.084574704\\
0.541313532838321	199789.237246528\\
0.541413535338383	199739.962876147\\
0.541513537838446	199690.688505765\\
0.541613540338509	199641.414135384\\
0.541713542838571	199592.139765003\\
0.541813545338633	199542.292436826\\
0.541913547838696	199493.018066445\\
0.542013550338758	199443.743696064\\
0.542113552838821	199394.469325683\\
0.542213555338883	199345.194955301\\
0.542313557838946	199295.92058492\\
0.542413560339008	199246.646214539\\
0.542513562839071	199196.798886363\\
0.542613565339134	199147.524515981\\
0.542713567839196	199098.2501456\\
0.542813570339258	199048.975775219\\
0.542913572839321	198999.701404838\\
0.543013575339383	198950.427034456\\
0.543113577839446	198901.152664075\\
0.543213580339509	198851.878293694\\
0.543313582839571	198802.030965517\\
0.543413585339633	198752.756595136\\
0.543513587839696	198703.482224755\\
0.543613590339759	198654.207854374\\
0.543713592839821	198604.933483992\\
0.543813595339883	198555.659113611\\
0.543913597839946	198506.38474323\\
0.544013600340008	198457.110372849\\
0.544113602840071	198407.836002467\\
0.544213605340134	198358.561632086\\
0.544313607840196	198309.287261705\\
0.544413610340258	198260.012891324\\
0.544513612840321	198210.738520942\\
0.544613615340384	198161.464150561\\
0.544713617840446	198112.18978018\\
0.544813620340509	198062.915409799\\
0.544913622840571	198013.641039417\\
0.545013625340633	197964.366669036\\
0.545113627840696	197915.092298655\\
0.545213630340759	197865.817928274\\
0.545313632840821	197816.543557892\\
0.545413635340883	197767.269187511\\
0.545513637840946	197717.99481713\\
0.545613640341009	197668.720446749\\
0.545713642841071	197619.446076367\\
0.545813645341134	197570.171705986\\
0.545913647841196	197520.897335605\\
0.546013650341258	197471.622965224\\
0.546113652841321	197422.348594842\\
0.546213655341384	197373.074224461\\
0.546313657841446	197323.79985408\\
0.546413660341509	197274.525483699\\
0.546513662841571	197225.251113317\\
0.546613665341634	197175.976742936\\
0.546713667841696	197126.702372555\\
0.546813670341759	197077.428002174\\
0.546913672841821	197028.153631792\\
0.547013675341883	196979.452219206\\
0.547113677841946	196930.177848825\\
0.547213680342009	196880.903478444\\
0.547313682842071	196831.629108063\\
0.547413685342134	196782.354737681\\
0.547513687842196	196733.0803673\\
0.547613690342259	196683.805996919\\
0.547713692842321	196634.531626537\\
0.547813695342384	196585.830213951\\
0.547913697842446	196536.55584357\\
0.548013700342509	196487.281473189\\
0.548113702842571	196438.007102808\\
0.548213705342634	196388.732732426\\
0.548313707842696	196339.458362045\\
0.548413710342759	196290.756949459\\
0.548513712842821	196241.482579078\\
0.548613715342884	196192.208208696\\
0.548713717842946	196142.933838315\\
0.548813720343009	196093.659467934\\
0.548913722843071	196044.958055348\\
0.549013725343134	195995.683684967\\
0.549113727843196	195946.409314585\\
0.549213730343259	195897.134944204\\
0.549313732843321	195847.860573823\\
0.549413735343384	195799.159161237\\
0.549513737843446	195749.884790855\\
0.549613740343509	195700.610420474\\
0.549713742843571	195651.336050093\\
0.549813745343634	195602.634637507\\
0.549913747843696	195553.360267126\\
0.550013750343759	195504.085896744\\
0.550113752843821	195455.384484158\\
0.550213755343884	195406.110113777\\
0.550313757843946	195356.835743396\\
0.550413760344009	195307.561373015\\
0.550513762844071	195258.859960428\\
0.550613765344134	195209.585590047\\
0.550713767844196	195160.311219666\\
0.550813770344259	195111.60980708\\
0.550913772844321	195062.335436699\\
0.551013775344384	195013.061066317\\
0.551113777844446	194964.359653731\\
0.551213780344509	194915.08528335\\
0.551313782844571	194865.810912969\\
0.551413785344634	194817.109500383\\
0.551513787844696	194767.835130001\\
0.551613790344759	194718.56075962\\
0.551713792844821	194669.859347034\\
0.551813795344884	194620.584976653\\
0.551913797844946	194571.883564067\\
0.552013800345009	194522.609193685\\
0.552113802845071	194473.334823304\\
0.552213805345134	194424.633410718\\
0.552313807845196	194375.359040337\\
0.552413810345259	194326.657627751\\
0.552513812845321	194277.383257369\\
0.552613815345384	194228.108886988\\
0.552713817845446	194179.407474402\\
0.552813820345509	194130.133104021\\
0.552913822845571	194081.431691435\\
0.553013825345634	194032.157321053\\
0.553113827845696	193983.455908467\\
0.553213830345759	193934.181538086\\
0.553313832845821	193885.4801255\\
0.553413835345884	193836.205755119\\
0.553513837845946	193787.504342532\\
0.553613840346009	193738.229972151\\
0.553713842846071	193689.528559565\\
0.553813845346134	193640.254189184\\
0.553913847846196	193591.552776598\\
0.554013850346259	193542.278406216\\
0.554113852846321	193493.57699363\\
0.554213855346384	193444.302623249\\
0.554313857846446	193395.601210663\\
0.554413860346509	193346.326840282\\
0.554513862846571	193297.625427696\\
0.554613865346634	193248.351057314\\
0.554713867846696	193199.649644728\\
0.554813870346759	193150.375274347\\
0.554913872846821	193101.673861761\\
0.555013875346884	193052.39949138\\
0.555113877846946	193003.698078793\\
0.555213880347009	192954.996666207\\
0.555313882847071	192905.722295826\\
0.555413885347134	192857.02088324\\
0.555513887847196	192807.746512859\\
0.555613890347259	192759.045100273\\
0.555713892847321	192710.343687686\\
0.555813895347384	192661.069317305\\
0.555913897847446	192612.367904719\\
0.556013900347509	192563.093534338\\
0.556113902847571	192514.392121752\\
0.556213905347634	192465.690709166\\
0.556313907847696	192416.416338784\\
0.556413910347759	192367.714926198\\
0.556513912847821	192319.013513612\\
0.556613915347884	192269.739143231\\
0.556713917847946	192221.037730645\\
0.556813920348009	192172.336318059\\
0.556913922848071	192123.061947677\\
0.557013925348134	192074.360535091\\
0.557113927848196	192025.659122505\\
0.557213930348259	191976.384752124\\
0.557313932848321	191927.683339538\\
0.557413935348384	191878.981926952\\
0.557513937848446	191830.280514366\\
0.557613940348509	191781.006143984\\
0.557713942848571	191732.304731398\\
0.557813945348634	191683.603318812\\
0.557913947848696	191634.901906226\\
0.558013950348759	191585.627535845\\
0.558113952848821	191536.926123259\\
0.558213955348884	191488.224710672\\
0.558313957848946	191439.523298086\\
0.558413960349009	191390.248927705\\
0.558513962849071	191341.547515119\\
0.558613965349134	191292.846102533\\
0.558713967849196	191244.144689947\\
0.558813970349259	191195.443277361\\
0.558913972849321	191146.168906979\\
0.559013975349384	191097.467494393\\
0.559113977849446	191048.766081807\\
0.559213980349509	191000.064669221\\
0.559313982849571	190951.363256635\\
0.559413985349634	190902.661844049\\
0.559513987849696	190853.387473667\\
0.559613990349759	190804.686061081\\
0.559713992849821	190755.984648495\\
0.559813995349884	190707.283235909\\
0.559913997849946	190658.581823323\\
0.560014000350009	190609.880410737\\
0.560114002850071	190561.178998151\\
0.560214005350134	190511.90462777\\
0.560314007850196	190463.203215183\\
0.560414010350259	190414.501802597\\
0.560514012850321	190365.800390011\\
0.560614015350384	190317.098977425\\
0.560714017850446	190268.397564839\\
0.560814020350509	190219.696152253\\
0.560914022850571	190170.994739667\\
0.561014025350634	190122.293327081\\
0.561114027850696	190073.591914494\\
0.561214030350759	190024.890501908\\
0.561314032850821	189976.189089322\\
0.561414035350884	189927.487676736\\
0.561514037850946	189878.78626415\\
0.561614040351009	189830.084851564\\
0.561714042851071	189781.383438978\\
0.561814045351134	189732.682026392\\
0.561914047851196	189683.980613805\\
0.562014050351259	189635.279201219\\
0.562114052851321	189586.577788633\\
0.562214055351384	189537.876376047\\
0.562314057851446	189489.174963461\\
0.562414060351509	189440.473550875\\
0.562514062851571	189391.772138289\\
0.562614065351634	189343.070725703\\
0.562714067851696	189294.369313117\\
0.562814070351759	189245.66790053\\
0.562914072851821	189196.966487944\\
0.563014075351884	189148.265075358\\
0.563114077851946	189099.563662772\\
0.563214080352009	189050.862250186\\
0.563314082852071	189002.1608376\\
0.563414085352134	188953.459425014\\
0.563514087852196	188904.758012428\\
0.563614090352259	188856.629557637\\
0.563714092852321	188807.92814505\\
0.563814095352384	188759.226732464\\
0.563914097852446	188710.525319878\\
0.564014100352509	188661.823907292\\
0.564114102852571	188613.122494706\\
0.564214105352634	188564.42108212\\
0.564314107852696	188515.719669534\\
0.564414110352759	188467.591214743\\
0.564514112852821	188418.889802157\\
0.564614115352884	188370.188389571\\
0.564714117852946	188321.486976984\\
0.564814120353009	188272.785564398\\
0.564914122853071	188224.084151812\\
0.565014125353134	188175.955697021\\
0.565114127853196	188127.254284435\\
0.565214130353259	188078.552871849\\
0.565314132853321	188029.851459263\\
0.565414135353384	187981.723004472\\
0.565514137853446	187933.021591886\\
0.565614140353509	187884.3201793\\
0.565714142853571	187835.618766713\\
0.565814145353634	187787.490311922\\
0.565914147853696	187738.788899336\\
0.566014150353759	187690.08748675\\
0.566114152853821	187641.386074164\\
0.566214155353884	187593.257619373\\
0.566314157853946	187544.556206787\\
0.566414160354009	187495.854794201\\
0.566514162854071	187447.153381615\\
0.566614165354134	187399.024926824\\
0.566714167854196	187350.323514238\\
0.566814170354259	187301.622101652\\
0.566914172854321	187253.493646861\\
0.567014175354384	187204.792234274\\
0.567114177854446	187156.090821688\\
0.567214180354509	187107.962366897\\
0.567314182854571	187059.260954311\\
0.567414185354634	187010.559541725\\
0.567514187854696	186962.431086934\\
0.567614190354759	186913.729674348\\
0.567714192854821	186865.601219557\\
0.567814195354884	186816.899806971\\
0.567914197854946	186768.198394385\\
0.568014200355009	186720.069939594\\
0.568114202855071	186671.368527008\\
0.568214205355134	186623.240072217\\
0.568314207855196	186574.538659631\\
0.568414210355259	186525.837247044\\
0.568514212855321	186477.708792253\\
0.568614215355384	186429.007379667\\
0.568714217855446	186380.878924876\\
0.568814220355509	186332.17751229\\
0.568914222855571	186284.049057499\\
0.569014225355634	186235.347644913\\
0.569114227855696	186187.219190122\\
0.569214230355759	186138.517777536\\
0.569314232855821	186090.389322745\\
0.569414235355884	186041.687910159\\
0.569514237855946	185992.986497573\\
0.569614240356009	185944.858042782\\
0.569714242856071	185896.729587991\\
0.569814245356134	185848.028175405\\
0.569914247856196	185799.899720614\\
0.570014250356259	185751.198308027\\
0.570114252856321	185703.069853237\\
0.570214255356384	185654.36844065\\
0.570314257856446	185606.239985859\\
0.570414260356509	185557.538573273\\
0.570514262856571	185509.410118482\\
0.570614265356634	185460.708705896\\
0.570714267856696	185412.580251105\\
0.570814270356759	185364.451796314\\
0.570914272856821	185315.750383728\\
0.571014275356884	185267.621928937\\
0.571114277856946	185218.920516351\\
0.571214280357009	185170.79206156\\
0.571314282857071	185122.663606769\\
0.571414285357134	185073.962194183\\
0.571514287857196	185025.833739392\\
0.571614290357259	184977.705284601\\
0.571714292857321	184929.003872015\\
0.571814295357384	184880.875417224\\
0.571914297857446	184832.746962433\\
0.572014300357509	184784.045549847\\
0.572114302857571	184735.917095056\\
0.572214305357634	184687.788640265\\
0.572314307857696	184639.087227679\\
0.572414310357759	184590.958772888\\
0.572514312857821	184542.830318097\\
0.572614315357884	184494.12890551\\
0.572714317857946	184446.000450719\\
0.572814320358009	184397.871995928\\
0.572914322858071	184349.743541138\\
0.573014325358134	184301.042128551\\
0.573114327858196	184252.91367376\\
0.573214330358259	184204.785218969\\
0.573314332858321	184156.656764178\\
0.573414335358384	184107.955351592\\
0.573514337858446	184059.826896801\\
0.573614340358509	184011.69844201\\
0.573714342858571	183963.569987219\\
0.573814345358634	183914.868574633\\
0.573914347858696	183866.740119842\\
0.574014350358759	183818.611665051\\
0.574114352858822	183770.48321026\\
0.574214355358884	183722.354755469\\
0.574314357858946	183674.226300678\\
0.574414360359009	183625.524888092\\
0.574514362859071	183577.396433301\\
0.574614365359134	183529.26797851\\
0.574714367859196	183481.139523719\\
0.574814370359259	183433.011068928\\
0.574914372859321	183384.882614137\\
0.575014375359384	183336.754159346\\
0.575114377859447	183288.05274676\\
0.575214380359509	183239.924291969\\
0.575314382859571	183191.795837178\\
0.575414385359634	183143.667382387\\
0.575514387859696	183095.538927596\\
0.575614390359759	183047.410472805\\
0.575714392859822	182999.282018014\\
0.575814395359884	182951.153563223\\
0.575914397859946	182903.025108432\\
0.576014400360009	182854.896653641\\
0.576114402860072	182806.76819885\\
0.576214405360134	182758.639744059\\
0.576314407860196	182710.511289268\\
0.576414410360259	182662.382834477\\
0.576514412860321	182614.254379686\\
0.576614415360384	182566.125924895\\
0.576714417860447	182517.997470104\\
0.576814420360509	182469.869015313\\
0.576914422860571	182421.740560522\\
0.577014425360634	182373.612105731\\
0.577114427860697	182325.48365094\\
0.577214430360759	182277.355196149\\
0.577314432860822	182229.226741358\\
0.577414435360884	182181.098286567\\
0.577514437860946	182132.969831776\\
0.577614440361009	182084.841376985\\
0.577714442861072	182036.712922194\\
0.577814445361134	181988.584467403\\
0.577914447861197	181940.456012612\\
0.578014450361259	181892.327557821\\
0.578114452861322	181844.19910303\\
0.578214455361384	181796.070648239\\
0.578314457861447	181748.515151244\\
0.578414460361509	181700.386696453\\
0.578514462861571	181652.258241662\\
0.578614465361634	181604.129786871\\
0.578714467861697	181556.00133208\\
0.578814470361759	181507.872877289\\
0.578914472861822	181459.744422498\\
0.579014475361884	181412.188925502\\
0.579114477861947	181364.060470711\\
0.579214480362009	181315.93201592\\
0.579314482862072	181267.803561129\\
0.579414485362134	181219.675106338\\
0.579514487862197	181171.546651547\\
0.579614490362259	181123.991154551\\
0.579714492862322	181075.86269976\\
0.579814495362384	181027.734244969\\
0.579914497862447	180979.605790178\\
0.580014500362509	180932.050293182\\
0.580114502862572	180883.921838391\\
0.580214505362634	180835.7933836\\
0.580314507862697	180787.664928809\\
0.580414510362759	180740.109431813\\
0.580514512862822	180691.980977022\\
0.580614515362884	180643.852522231\\
0.580714517862947	180595.72406744\\
0.580814520363009	180548.168570444\\
0.580914522863072	180500.040115654\\
0.581014525363134	180451.911660863\\
0.581114527863197	180404.356163867\\
0.581214530363259	180356.227709076\\
0.581314532863322	180308.099254285\\
0.581414535363384	180260.543757289\\
0.581514537863447	180212.415302498\\
0.581614540363509	180164.286847707\\
0.581714542863572	180116.731350711\\
0.581814545363634	180068.60289592\\
0.581914547863697	180021.047398924\\
0.582014550363759	179972.918944133\\
0.582114552863822	179924.790489342\\
0.582214555363884	179877.234992346\\
0.582314557863947	179829.106537555\\
0.582414560364009	179781.551040559\\
0.582514562864072	179733.422585768\\
0.582614565364134	179685.294130977\\
0.582714567864197	179637.738633982\\
0.582814570364259	179589.610179191\\
0.582914572864322	179542.054682195\\
0.583014575364384	179493.926227404\\
0.583114577864447	179446.370730408\\
0.583214580364509	179398.242275617\\
0.583314582864572	179350.686778621\\
0.583414585364634	179302.55832383\\
0.583514587864697	179255.002826834\\
0.583614590364759	179206.874372043\\
0.583714592864822	179159.318875047\\
0.583814595364884	179111.190420256\\
0.583914597864947	179063.634923261\\
0.584014600365009	179015.50646847\\
0.584114602865072	178967.950971474\\
0.584214605365134	178919.822516683\\
0.584314607865197	178872.267019687\\
0.584414610365259	178824.138564896\\
0.584514612865322	178776.5830679\\
0.584614615365384	178729.027570904\\
0.584714617865447	178680.899116113\\
0.584814620365509	178633.343619117\\
0.584914622865572	178585.215164326\\
0.585014625365634	178537.65966733\\
0.585114627865697	178490.104170335\\
0.585214630365759	178441.975715544\\
0.585314632865822	178394.420218548\\
0.585414635365884	178346.291763757\\
0.585514637865947	178298.736266761\\
0.585614640366009	178251.180769765\\
0.585714642866072	178203.052314974\\
0.585814645366134	178155.496817978\\
0.585914647866197	178107.941320982\\
0.586014650366259	178059.812866191\\
0.586114652866322	178012.257369195\\
0.586214655366384	177964.7018722\\
0.586314657866447	177917.146375204\\
0.586414660366509	177869.017920413\\
0.586514662866572	177821.462423417\\
0.586614665366634	177773.906926421\\
0.586714667866697	177725.77847163\\
0.586814670366759	177678.222974634\\
0.586914672866822	177630.667477638\\
0.587014675366884	177583.111980642\\
0.587114677866947	177534.983525851\\
0.587214680367009	177487.428028856\\
0.587314682867072	177439.87253186\\
0.587414685367134	177392.317034864\\
0.587514687867197	177344.761537868\\
0.587614690367259	177296.633083077\\
0.587714692867322	177249.077586081\\
0.587814695367384	177201.522089085\\
0.587914697867447	177153.966592089\\
0.588014700367509	177106.411095094\\
0.588114702867572	177058.855598098\\
0.588214705367634	177010.727143307\\
0.588314707867697	176963.171646311\\
0.588414710367759	176915.616149315\\
0.588514712867822	176868.060652319\\
0.588614715367884	176820.505155323\\
0.588714717867947	176772.949658328\\
0.588814720368009	176725.394161332\\
0.588914722868072	176677.838664336\\
0.589014725368134	176629.710209545\\
0.589114727868197	176582.154712549\\
0.589214730368259	176534.599215553\\
0.589314732868322	176487.043718557\\
0.589414735368384	176439.488221561\\
0.589514737868447	176391.932724566\\
0.589614740368509	176344.37722757\\
0.589714742868572	176296.821730574\\
0.589814745368634	176249.266233578\\
0.589914747868697	176201.710736582\\
0.590014750368759	176154.155239586\\
0.590114752868822	176106.59974259\\
0.590214755368884	176059.044245595\\
0.590314757868947	176011.488748599\\
0.590414760369009	175963.933251603\\
0.590514762869072	175916.377754607\\
0.590614765369134	175868.822257611\\
0.590714767869197	175821.266760615\\
0.590814770369259	175773.711263619\\
0.590914772869322	175726.155766623\\
0.591014775369384	175678.600269628\\
0.591114777869447	175631.044772632\\
0.591214780369509	175584.062233431\\
0.591314782869572	175536.506736435\\
0.591414785369634	175488.951239439\\
0.591514787869697	175441.395742443\\
0.591614790369759	175393.840245448\\
0.591714792869822	175346.284748452\\
0.591814795369884	175298.729251456\\
0.591914797869947	175251.17375446\\
0.592014800370009	175203.618257464\\
0.592114802870072	175156.635718263\\
0.592214805370134	175109.080221268\\
0.592314807870197	175061.524724272\\
0.592414810370259	175013.969227276\\
0.592514812870322	174966.41373028\\
0.592614815370384	174919.431191079\\
0.592714817870447	174871.875694083\\
0.592814820370509	174824.320197088\\
0.592914822870572	174776.764700092\\
0.593014825370634	174729.209203096\\
0.593114827870697	174682.226663895\\
0.593214830370759	174634.671166899\\
0.593314832870822	174587.115669903\\
0.593414835370884	174539.560172908\\
0.593514837870947	174492.577633707\\
0.593614840371009	174445.022136711\\
0.593714842871072	174397.466639715\\
0.593814845371134	174349.911142719\\
0.593914847871197	174302.928603518\\
0.594014850371259	174255.373106523\\
0.594114852871322	174207.817609527\\
0.594214855371384	174160.835070326\\
0.594314857871447	174113.27957333\\
0.594414860371509	174065.724076334\\
0.594514862871572	174018.741537134\\
0.594614865371634	173971.186040138\\
0.594714867871697	173923.630543142\\
0.594814870371759	173876.648003941\\
0.594914872871822	173829.092506945\\
0.595014875371884	173782.109967745\\
0.595114877871947	173734.554470749\\
0.595214880372009	173686.998973753\\
0.595314882872072	173640.016434552\\
0.595414885372134	173592.460937556\\
0.595514887872197	173545.478398356\\
0.595614890372259	173497.92290136\\
0.595714892872322	173450.367404364\\
0.595814895372384	173403.384865163\\
0.595914897872447	173355.829368167\\
0.596014900372509	173308.846828967\\
0.596114902872572	173261.291331971\\
0.596214905372634	173214.30879277\\
0.596314907872697	173166.753295774\\
0.596414910372759	173119.770756573\\
0.596514912872822	173072.215259578\\
0.596614915372884	173025.232720377\\
0.596714917872947	172977.677223381\\
0.596814920373009	172930.69468418\\
0.596914922873072	172883.139187184\\
0.597014925373134	172836.156647984\\
0.597114927873197	172788.601150988\\
0.597214930373259	172741.618611787\\
0.597314932873322	172694.636072586\\
0.597414935373384	172647.08057559\\
0.597514937873447	172600.09803639\\
0.597614940373509	172552.542539394\\
0.597714942873572	172505.560000193\\
0.597814945373634	172458.577460992\\
0.597914947873697	172411.021963997\\
0.598014950373759	172364.039424796\\
0.598114952873822	172316.4839278\\
0.598214955373884	172269.501388599\\
0.598314957873947	172222.518849399\\
0.598414960374009	172174.963352403\\
0.598514962874072	172127.980813202\\
0.598614965374134	172080.998274001\\
0.598714967874197	172033.442777005\\
0.598814970374259	171986.460237805\\
0.598914972874322	171939.477698604\\
0.599014975374384	171891.922201608\\
0.599114977874447	171844.939662407\\
0.599214980374509	171797.957123207\\
0.599314982874572	171750.974584006\\
0.599414985374634	171703.41908701\\
0.599514987874697	171656.436547809\\
0.599614990374759	171609.454008609\\
0.599714992874822	171562.471469408\\
0.599814995374884	171514.915972412\\
0.599914997874947	171467.933433211\\
0.600015000375009	171420.95089401\\
0.600115002875072	171373.96835481\\
0.600215005375134	171326.985815609\\
0.600315007875197	171279.430318613\\
0.600415010375259	171232.447779412\\
0.600515012875322	171185.465240212\\
0.600615015375384	171138.482701011\\
0.600715017875447	171091.50016181\\
0.600815020375509	171044.51762261\\
0.600915022875572	170996.962125614\\
0.601015025375634	170949.979586413\\
0.601115027875697	170902.997047212\\
0.601215030375759	170856.014508011\\
0.601315032875822	170809.031968811\\
0.601415035375884	170762.04942961\\
0.601515037875947	170715.066890409\\
0.601615040376009	170668.084351209\\
0.601715042876072	170621.101812008\\
0.601815045376134	170574.119272807\\
0.601915047876197	170527.136733606\\
0.602015050376259	170479.581236611\\
0.602115052876322	170432.59869741\\
0.602215055376384	170385.616158209\\
0.602315057876447	170338.633619008\\
0.602415060376509	170291.651079808\\
0.602515062876572	170244.668540607\\
0.602615065376634	170197.686001406\\
0.602715067876697	170150.703462205\\
0.602815070376759	170103.720923005\\
0.602915072876822	170056.738383804\\
0.603015075376884	170009.755844603\\
0.603115077876947	169962.773305403\\
0.603215080377009	169916.363723997\\
0.603315082877072	169869.381184796\\
0.603415085377134	169822.398645596\\
0.603515087877197	169775.416106395\\
0.603615090377259	169728.433567194\\
0.603715092877322	169681.451027993\\
0.603815095377384	169634.468488793\\
0.603915097877447	169587.485949592\\
0.604015100377509	169540.503410391\\
0.604115102877572	169493.52087119\\
0.604215105377634	169447.111289785\\
0.604315107877697	169400.128750584\\
0.604415110377759	169353.146211383\\
0.604515112877822	169306.163672183\\
0.604615115377884	169259.181132982\\
0.604715117877947	169212.198593781\\
0.604815120378009	169165.789012376\\
0.604915122878072	169118.806473175\\
0.605015125378134	169071.823933974\\
0.605115127878197	169024.841394773\\
0.605215130378259	168977.858855573\\
0.605315132878322	168931.449274167\\
0.605415135378384	168884.466734966\\
0.605515137878447	168837.484195766\\
0.605615140378509	168790.501656565\\
0.605715142878572	168744.092075159\\
0.605815145378634	168697.109535959\\
0.605915147878697	168650.126996758\\
0.606015150378759	168603.144457557\\
0.606115152878822	168556.734876151\\
0.606215155378884	168509.752336951\\
0.606315157878947	168462.76979775\\
0.606415160379009	168416.360216344\\
0.606515162879072	168369.377677144\\
0.606615165379135	168322.395137943\\
0.606715167879197	168275.985556537\\
0.606815170379259	168229.003017337\\
0.606915172879322	168182.020478136\\
0.607015175379384	168135.61089673\\
0.607115177879447	168088.62835753\\
0.60721518037951	168041.645818329\\
0.607315182879572	167995.236236923\\
0.607415185379634	167948.253697723\\
0.607515187879697	167901.844116317\\
0.60761519037976	167854.861577116\\
0.607715192879822	167808.451995711\\
0.607815195379884	167761.46945651\\
0.607915197879947	167714.486917309\\
0.608015200380009	167668.077335904\\
0.608115202880072	167621.094796703\\
0.608215205380135	167574.685215297\\
0.608315207880197	167527.702676097\\
0.608415210380259	167481.293094691\\
0.608515212880322	167434.31055549\\
0.608615215380385	167387.900974085\\
0.608715217880447	167340.918434884\\
0.60881522038051	167294.508853478\\
0.608915222880572	167247.526314278\\
0.609015225380634	167201.116732872\\
0.609115227880697	167154.134193671\\
0.60921523038076	167107.724612266\\
0.609315232880822	167061.31503086\\
0.609415235380884	167014.332491659\\
0.609515237880947	166967.922910254\\
0.60961524038101	166920.940371053\\
0.609715242881072	166874.530789647\\
0.609815245381135	166828.121208242\\
0.609915247881197	166781.138669041\\
0.610015250381259	166734.729087635\\
0.610115252881322	166687.746548435\\
0.610215255381385	166641.336967029\\
0.610315257881447	166594.927385624\\
0.61041526038151	166547.944846423\\
0.610515262881572	166501.535265017\\
0.610615265381635	166455.125683612\\
0.610715267881697	166408.143144411\\
0.61081527038176	166361.733563005\\
0.610915272881822	166315.3239816\\
0.611015275381884	166268.914400194\\
0.611115277881947	166221.931860993\\
0.61121528038201	166175.522279588\\
0.611315282882072	166129.112698182\\
0.611415285382135	166082.130158981\\
0.611515287882197	166035.720577576\\
0.61161529038226	165989.31099617\\
0.611715292882322	165942.901414765\\
0.611815295382385	165896.491833359\\
0.611915297882447	165849.509294158\\
0.61201530038251	165803.099712753\\
0.612115302882572	165756.690131347\\
0.612215305382635	165710.280549942\\
0.612315307882697	165663.870968536\\
0.61241531038276	165616.888429335\\
0.612515312882822	165570.47884793\\
0.612615315382885	165524.069266524\\
0.612715317882947	165477.659685118\\
0.61281532038301	165431.250103713\\
0.612915322883072	165384.840522307\\
0.613015325383135	165338.430940902\\
0.613115327883197	165292.021359496\\
0.61321533038326	165245.038820295\\
0.613315332883322	165198.62923889\\
0.613415335383385	165152.219657484\\
0.613515337883447	165105.810076079\\
0.61361534038351	165059.400494673\\
0.613715342883572	165012.990913267\\
0.613815345383635	164966.581331862\\
0.613915347883697	164920.171750456\\
0.61401535038376	164873.762169051\\
0.614115352883822	164827.352587645\\
0.614215355383885	164780.943006239\\
0.614315357883947	164734.533424834\\
0.61441536038401	164688.123843428\\
0.614515362884072	164641.714262023\\
0.614615365384135	164595.304680617\\
0.614715367884197	164548.895099211\\
0.61481537038426	164502.485517806\\
0.614915372884322	164456.0759364\\
0.615015375384385	164409.666354995\\
0.615115377884447	164363.256773589\\
0.61521538038451	164316.847192183\\
0.615315382884572	164270.437610778\\
0.615415385384635	164224.600987167\\
0.615515387884697	164178.191405762\\
0.61561539038476	164131.781824356\\
0.615715392884822	164085.372242951\\
0.615815395384885	164038.962661545\\
0.615915397884947	163992.553080139\\
0.61601540038501	163946.143498734\\
0.616115402885072	163899.733917328\\
0.616215405385135	163853.897293718\\
0.616315407885197	163807.487712312\\
0.61641541038526	163761.078130907\\
0.616515412885322	163714.668549501\\
0.616615415385385	163668.258968095\\
0.616715417885447	163622.422344485\\
0.61681542038551	163576.012763079\\
0.616915422885572	163529.603181674\\
0.617015425385635	163483.193600268\\
0.617115427885697	163437.356976658\\
0.61721543038576	163390.947395252\\
0.617315432885822	163344.537813846\\
0.617415435385885	163298.128232441\\
0.617515437885947	163252.29160883\\
0.61761544038601	163205.882027425\\
0.617715442886072	163159.472446019\\
0.617815445386135	163113.062864614\\
0.617915447886197	163067.226241003\\
0.61801545038626	163020.816659597\\
0.618115452886322	162974.407078192\\
0.618215455386385	162928.570454581\\
0.618315457886447	162882.160873176\\
0.61841546038651	162836.324249565\\
0.618515462886572	162789.91466816\\
0.618615465386635	162743.505086754\\
0.618715467886697	162697.668463144\\
0.61881547038676	162651.258881738\\
0.618915472886822	162604.849300332\\
0.619015475386885	162559.012676722\\
0.619115477886947	162512.603095316\\
0.61921548038701	162466.766471706\\
0.619315482887072	162420.3568903\\
0.619415485387135	162374.52026669\\
0.619515487887197	162328.110685284\\
0.61961549038726	162282.274061674\\
0.619715492887322	162235.864480268\\
0.619815495387385	162190.027856658\\
0.619915497887447	162143.618275252\\
0.62001550038751	162097.781651642\\
0.620115502887572	162051.372070236\\
0.620215505387635	162005.535446626\\
0.620315507887697	161959.12586522\\
0.62041551038776	161913.28924161\\
0.620515512887822	161866.879660204\\
0.620615515387885	161821.043036594\\
0.620715517887947	161774.633455188\\
0.62081552038801	161728.796831577\\
0.620915522888072	161682.387250172\\
0.621015525388135	161636.550626561\\
0.621115527888197	161590.714002951\\
0.62121553038826	161544.304421545\\
0.621315532888322	161498.467797935\\
0.621415535388385	161452.631174324\\
0.621515537888447	161406.221592919\\
0.62161554038851	161360.384969308\\
0.621715542888572	161314.548345698\\
0.621815545388635	161268.138764292\\
0.621915547888697	161222.302140682\\
0.62201555038876	161176.465517071\\
0.622115552888822	161130.055935666\\
0.622215555388885	161084.219312055\\
0.622315557888947	161038.382688445\\
0.62241556038901	160991.973107039\\
0.622515562889072	160946.136483429\\
0.622615565389135	160900.299859818\\
0.622715567889197	160854.463236208\\
0.62281557038926	160808.053654802\\
0.622915572889322	160762.217031192\\
0.623015575389385	160716.380407581\\
0.623115577889447	160670.543783971\\
0.62321558038951	160624.134202565\\
0.623315582889572	160578.297578955\\
0.623415585389635	160532.460955344\\
0.623515587889697	160486.624331734\\
0.62361559038976	160440.787708123\\
0.623715592889822	160394.951084513\\
0.623815595389885	160348.541503107\\
0.623915597889947	160302.704879497\\
0.62401560039001	160256.868255886\\
0.624115602890072	160211.031632276\\
0.624215605390135	160165.195008665\\
0.624315607890197	160119.358385055\\
0.62441561039026	160073.521761445\\
0.624515612890322	160027.685137834\\
0.624615615390385	159981.848514224\\
0.624715617890447	159935.438932818\\
0.62481562039051	159889.602309208\\
0.624915622890572	159843.765685597\\
0.625015625390635	159797.929061987\\
0.625115627890697	159752.092438376\\
0.62521563039076	159706.255814766\\
0.625315632890822	159660.419191155\\
0.625415635390885	159614.582567545\\
0.625515637890947	159568.745943934\\
0.62561564039101	159522.909320324\\
0.625715642891072	159477.072696713\\
0.625815645391135	159431.236073103\\
0.625915647891197	159385.399449492\\
0.62601565039126	159339.562825882\\
0.626115652891322	159294.299160067\\
0.626215655391385	159248.462536456\\
0.626315657891447	159202.625912846\\
0.62641566039151	159156.789289235\\
0.626515662891572	159110.952665625\\
0.626615665391635	159065.116042014\\
0.626715667891697	159019.279418404\\
0.62681567039176	158973.442794793\\
0.626915672891822	158927.606171183\\
0.627015675391885	158881.769547572\\
0.627115677891947	158836.505881757\\
0.62721568039201	158790.669258147\\
0.627315682892072	158744.832634536\\
0.627415685392135	158698.996010926\\
0.627515687892197	158653.159387315\\
0.62761569039226	158607.8957215\\
0.627715692892322	158562.059097889\\
0.627815695392385	158516.222474279\\
0.627915697892447	158470.385850668\\
0.62801570039251	158424.549227058\\
0.628115702892572	158379.285561243\\
0.628215705392635	158333.448937632\\
0.628315707892697	158287.612314022\\
0.62841571039276	158241.775690411\\
0.628515712892822	158196.512024596\\
0.628615715392885	158150.675400985\\
0.628715717892947	158104.838777375\\
0.62881572039301	158059.57511156\\
0.628915722893072	158013.738487949\\
0.629015725393135	157967.901864339\\
0.629115727893197	157922.638198523\\
0.62921573039326	157876.801574913\\
0.629315732893322	157830.964951303\\
0.629415735393385	157785.701285487\\
0.629515737893447	157739.864661877\\
0.62961574039351	157694.028038266\\
0.629715742893572	157648.764372451\\
0.629815745393635	157602.92774884\\
0.629915747893697	157557.664083025\\
0.63001575039376	157511.827459415\\
0.630115752893822	157465.990835804\\
0.630215755393885	157420.727169989\\
0.630315757893947	157374.890546378\\
0.63041576039401	157329.626880563\\
0.630515762894072	157283.790256953\\
0.630615765394135	157238.526591137\\
0.630715767894197	157192.689967527\\
0.63081577039426	157147.426301711\\
0.630915772894322	157101.589678101\\
0.631015775394385	157056.326012286\\
0.631115777894447	157010.489388675\\
0.63121578039451	156965.22572286\\
0.631315782894572	156919.389099249\\
0.631415785394635	156874.125433434\\
0.631515787894697	156828.861767619\\
0.63161579039476	156783.025144008\\
0.631715792894822	156737.761478193\\
0.631815795394885	156691.924854582\\
0.631915797894947	156646.661188767\\
0.63201580039501	156600.824565157\\
0.632115802895072	156555.560899341\\
0.632215805395135	156510.297233526\\
0.632315807895197	156464.460609915\\
0.63241581039526	156419.1969441\\
0.632515812895322	156373.933278285\\
0.632615815395385	156328.096654674\\
0.632715817895447	156282.832988859\\
0.63281582039551	156237.569323044\\
0.632915822895572	156191.732699433\\
0.633015825395635	156146.469033618\\
0.633115827895697	156101.205367803\\
0.63321583039576	156055.368744192\\
0.633315832895822	156010.105078377\\
0.633415835395885	155964.841412561\\
0.633515837895947	155919.577746746\\
0.63361584039601	155873.741123136\\
0.633715842896072	155828.47745732\\
0.633815845396135	155783.213791505\\
0.633915847896197	155737.95012569\\
0.63401585039626	155692.686459874\\
0.634115852896322	155646.849836264\\
0.634215855396385	155601.586170448\\
0.634315857896447	155556.322504633\\
0.63441586039651	155511.058838818\\
0.634515862896572	155465.795173002\\
0.634615865396635	155420.531507187\\
0.634715867896697	155374.694883577\\
0.63481587039676	155329.431217761\\
0.634915872896822	155284.167551946\\
0.635015875396885	155238.903886131\\
0.635115877896947	155193.640220315\\
0.63521588039701	155148.3765545\\
0.635315882897072	155103.112888685\\
0.635415885397135	155057.849222869\\
0.635515887897197	155012.585557054\\
0.63561589039726	154967.321891239\\
0.635715892897322	154922.058225423\\
0.635815895397385	154876.794559608\\
0.635915897897447	154831.530893793\\
0.63601590039751	154786.267227977\\
0.636115902897572	154741.003562162\\
0.636215905397635	154695.739896347\\
0.636315907897697	154650.476230531\\
0.63641591039776	154605.212564716\\
0.636515912897823	154559.948898901\\
0.636615915397885	154514.685233085\\
0.636715917897947	154469.42156727\\
0.63681592039801	154424.157901455\\
0.636915922898072	154378.894235639\\
0.637015925398135	154333.630569824\\
0.637115927898197	154288.366904009\\
0.63721593039826	154243.103238193\\
0.637315932898322	154197.839572378\\
0.637415935398385	154152.575906563\\
0.637515937898448	154107.885198542\\
0.63761594039851	154062.621532727\\
0.637715942898572	154017.357866912\\
0.637815945398635	153972.094201096\\
0.637915947898697	153926.830535281\\
0.63801595039876	153881.566869466\\
0.638115952898823	153836.876161446\\
0.638215955398885	153791.61249563\\
0.638315957898947	153746.348829815\\
0.63841596039901	153701.085164\\
0.638515962899073	153655.821498184\\
0.638615965399135	153611.130790164\\
0.638715967899197	153565.867124349\\
0.63881597039926	153520.603458533\\
0.638915972899322	153475.339792718\\
0.639015975399385	153430.649084698\\
0.639115977899448	153385.385418882\\
0.63921598039951	153340.121753067\\
0.639315982899572	153295.431045047\\
0.639415985399635	153250.167379232\\
0.639515987899698	153204.903713416\\
0.63961599039976	153160.213005396\\
0.639715992899823	153114.949339581\\
0.639815995399885	153069.685673765\\
0.639915997899947	153024.994965745\\
0.64001600040001	152979.73129993\\
0.640116002900073	152934.467634114\\
0.640216005400135	152889.776926094\\
0.640316007900197	152844.513260279\\
0.64041601040026	152799.822552259\\
0.640516012900323	152754.558886443\\
0.640616015400385	152709.295220628\\
0.640716017900448	152664.604512608\\
0.64081602040051	152619.340846793\\
0.640916022900572	152574.650138772\\
0.641016025400635	152529.386472957\\
0.641116027900698	152484.695764937\\
0.64121603040076	152439.432099121\\
0.641316032900823	152394.741391101\\
0.641416035400885	152349.477725286\\
0.641516037900948	152304.787017266\\
0.64161604040101	152259.52335145\\
0.641716042901073	152214.83264343\\
0.641816045401135	152169.568977615\\
0.641916047901197	152124.878269595\\
0.64201605040126	152079.614603779\\
0.642116052901323	152034.923895759\\
0.642216055401385	151990.233187739\\
0.642316057901448	151944.969521924\\
0.64241606040151	151900.278813903\\
0.642516062901573	151855.015148088\\
0.642616065401635	151810.324440068\\
0.642716067901698	151765.633732048\\
0.64281607040176	151720.370066232\\
0.642916072901823	151675.679358212\\
0.643016075401885	151630.988650192\\
0.643116077901948	151585.724984377\\
0.64321608040201	151541.034276356\\
0.643316082902073	151496.343568336\\
0.643416085402135	151451.079902521\\
0.643516087902198	151406.389194501\\
0.64361609040226	151361.69848648\\
0.643716092902323	151317.00777846\\
0.643816095402385	151271.744112645\\
0.643916097902448	151227.053404625\\
0.64401610040251	151182.362696604\\
0.644116102902573	151137.671988584\\
0.644216105402635	151092.408322769\\
0.644316107902698	151047.717614749\\
0.64441611040276	151003.026906728\\
0.644516112902823	150958.336198708\\
0.644616115402885	150913.645490688\\
0.644716117902948	150868.381824873\\
0.64481612040301	150823.691116853\\
0.644916122903073	150779.000408832\\
0.645016125403135	150734.309700812\\
0.645116127903198	150689.618992792\\
0.64521613040326	150644.928284772\\
0.645316132903323	150600.237576751\\
0.645416135403385	150554.973910936\\
0.645516137903448	150510.283202916\\
0.64561614040351	150465.592494896\\
0.645716142903573	150420.901786876\\
0.645816145403635	150376.211078855\\
0.645916147903698	150331.520370835\\
0.64601615040376	150286.829662815\\
0.646116152903823	150242.138954795\\
0.646216155403885	150197.448246775\\
0.646316157903948	150152.757538754\\
0.64641616040401	150108.066830734\\
0.646516162904073	150063.376122714\\
0.646616165404135	150018.685414694\\
0.646716167904198	149973.994706674\\
0.64681617040426	149929.303998653\\
0.646916172904323	149884.613290633\\
0.647016175404385	149839.922582613\\
0.647116177904448	149795.231874593\\
0.64721618040451	149750.541166572\\
0.647316182904573	149705.850458552\\
0.647416185404635	149661.732708327\\
0.647516187904698	149617.042000307\\
0.64761619040476	149572.351292287\\
0.647716192904823	149527.660584267\\
0.647816195404885	149482.969876246\\
0.647916197904948	149438.279168226\\
0.64801620040501	149393.588460206\\
0.648116202905073	149349.470709981\\
0.648216205405135	149304.780001961\\
0.648316207905198	149260.08929394\\
0.64841621040526	149215.39858592\\
0.648516212905323	149170.7078779\\
0.648616215405385	149126.01716988\\
0.648716217905448	149081.899419655\\
0.64881622040551	149037.208711635\\
0.648916222905573	148992.518003614\\
0.649016225405635	148947.827295594\\
0.649116227905698	148903.709545369\\
0.64921623040576	148859.018837349\\
0.649316232905823	148814.328129329\\
0.649416235405885	148770.210379104\\
0.649516237905948	148725.519671083\\
0.64961624040601	148680.828963063\\
0.649716242906073	148636.711212838\\
0.649816245406135	148592.020504818\\
0.649916247906198	148547.329796798\\
0.65001625040626	148503.212046573\\
0.650116252906323	148458.521338552\\
0.650216255406385	148413.830630532\\
0.650316257906448	148369.712880307\\
0.65041626040651	148325.022172287\\
0.650516262906573	148280.331464267\\
0.650616265406635	148236.213714042\\
0.650716267906698	148191.523006022\\
0.65081627040676	148147.405255796\\
0.650916272906823	148102.714547776\\
0.651016275406885	148058.596797551\\
0.651116277906948	148013.906089531\\
0.65121628040701	147969.788339306\\
0.651316282907073	147925.097631286\\
0.651416285407135	147880.979881061\\
0.651516287907198	147836.28917304\\
0.65161629040726	147792.171422815\\
0.651716292907323	147747.480714795\\
0.651816295407385	147703.36296457\\
0.651916297907448	147658.67225655\\
0.65201630040751	147614.554506325\\
0.652116302907573	147569.863798305\\
0.652216305407635	147525.746048079\\
0.652316307907698	147481.628297854\\
0.65241631040776	147436.937589834\\
0.652516312907823	147392.819839609\\
0.652616315407885	147348.129131589\\
0.652716317907948	147304.011381364\\
0.65281632040801	147259.893631139\\
0.652916322908073	147215.202923119\\
0.653016325408135	147171.085172894\\
0.653116327908198	147126.967422668\\
0.65321633040826	147082.276714648\\
0.653316332908323	147038.158964423\\
0.653416335408385	146994.041214198\\
0.653516337908448	146949.350506178\\
0.65361634040851	146905.232755953\\
0.653716342908573	146861.115005728\\
0.653816345408635	146816.997255503\\
0.653916347908698	146772.306547482\\
0.65401635040876	146728.188797257\\
0.654116352908823	146684.071047032\\
0.654216355408885	146639.953296807\\
0.654316357908948	146595.262588787\\
0.65441636040901	146551.144838562\\
0.654516362909073	146507.027088337\\
0.654616365409135	146462.909338112\\
0.654716367909198	146418.791587887\\
0.65481637040926	146374.100879867\\
0.654916372909323	146329.983129641\\
0.655016375409385	146285.865379416\\
0.655116377909448	146241.747629191\\
0.65521638040951	146197.629878966\\
0.655316382909573	146153.512128741\\
0.655416385409635	146109.394378516\\
0.655516387909698	146065.276628291\\
0.65561639040976	146021.158878066\\
0.655716392909823	145976.468170046\\
0.655816395409885	145932.350419821\\
0.655916397909948	145888.232669596\\
0.65601640041001	145844.114919371\\
0.656116402910073	145799.997169145\\
0.656216405410135	145755.87941892\\
0.656316407910198	145711.761668695\\
0.65641641041026	145667.64391847\\
0.656516412910323	145623.526168245\\
0.656616415410385	145579.40841802\\
0.656716417910448	145535.290667795\\
0.65681642041051	145491.17291757\\
0.656916422910573	145447.055167345\\
0.657016425410635	145402.93741712\\
0.657116427910698	145359.39262469\\
0.65721643041076	145315.274874465\\
0.657316432910823	145271.15712424\\
0.657416435410885	145227.039374015\\
0.657516437910948	145182.92162379\\
0.65761644041101	145138.803873564\\
0.657716442911073	145094.686123339\\
0.657816445411135	145050.568373114\\
0.657916447911198	145006.450622889\\
0.65801645041126	144962.905830459\\
0.658116452911323	144918.788080234\\
0.658216455411385	144874.670330009\\
0.658316457911448	144830.552579784\\
0.65841646041151	144786.434829559\\
0.658516462911573	144742.890037129\\
0.658616465411635	144698.772286904\\
0.658716467911698	144654.654536679\\
0.65881647041176	144610.536786454\\
0.658916472911823	144566.991994024\\
0.659016475411885	144522.874243799\\
0.659116477911948	144478.756493574\\
0.65921648041201	144434.638743349\\
0.659316482912073	144391.093950919\\
0.659416485412135	144346.976200694\\
0.659516487912198	144302.858450469\\
0.65961649041226	144259.313658039\\
0.659716492912323	144215.195907814\\
0.659816495412385	144171.078157589\\
0.659916497912448	144127.533365159\\
0.66001650041251	144083.415614934\\
0.660116502912573	144039.297864708\\
0.660216505412635	143995.753072278\\
0.660316507912698	143951.635322053\\
0.66041651041276	143908.090529623\\
0.660516512912823	143863.972779398\\
0.660616515412885	143819.855029173\\
0.660716517912948	143776.310236743\\
0.66081652041301	143732.192486518\\
0.660916522913073	143688.647694088\\
0.661016525413135	143644.529943863\\
0.661116527913198	143600.985151433\\
0.66121653041326	143556.867401208\\
0.661316532913323	143513.322608778\\
0.661416535413385	143469.204858553\\
0.661516537913448	143425.660066123\\
0.66161654041351	143381.542315898\\
0.661716542913573	143337.997523468\\
0.661816545413635	143293.879773243\\
0.661916547913698	143250.334980813\\
0.66201655041376	143206.790188383\\
0.662116552913823	143162.672438158\\
0.662216555413885	143119.127645728\\
0.662316557913948	143075.009895503\\
0.66241656041401	143031.465103073\\
0.662516562914073	142987.920310643\\
0.662616565414135	142943.802560418\\
0.662716567914198	142900.257767988\\
0.66281657041426	142856.712975558\\
0.662916572914323	142812.595225333\\
0.663016575414385	142769.050432903\\
0.663116577914448	142725.505640473\\
0.66321658041451	142681.387890248\\
0.663316582914573	142637.843097818\\
0.663416585414635	142594.298305388\\
0.663516587914698	142550.180555163\\
0.66361659041476	142506.635762733\\
0.663716592914823	142463.090970304\\
0.663816595414885	142419.546177874\\
0.663916597914948	142375.428427649\\
0.66401660041501	142331.883635219\\
0.664116602915073	142288.338842789\\
0.664216605415135	142244.794050359\\
0.664316607915198	142201.249257929\\
0.66441661041526	142157.704465499\\
0.664516612915323	142113.586715274\\
0.664616615415385	142070.041922844\\
0.664716617915448	142026.497130414\\
0.66481662041551	141982.952337984\\
0.664916622915573	141939.407545554\\
0.665016625415635	141895.862753124\\
0.665116627915698	141852.317960694\\
0.66521663041576	141808.200210469\\
0.665316632915823	141764.655418039\\
0.665416635415885	141721.110625609\\
0.665516637915948	141677.565833179\\
0.66561664041601	141634.021040749\\
0.665716642916073	141590.476248319\\
0.665816645416135	141546.931455889\\
0.665916647916198	141503.386663459\\
0.66601665041626	141459.841871029\\
0.666116652916323	141416.2970786\\
0.666216655416385	141372.75228617\\
0.666316657916448	141329.20749374\\
0.66641666041651	141285.66270131\\
0.666516662916573	141242.11790888\\
0.666616665416635	141198.57311645\\
0.666716667916698	141155.02832402\\
0.66681667041676	141111.48353159\\
0.666916672916823	141068.511696955\\
0.667016675416885	141024.966904525\\
0.667116677916948	140981.422112095\\
0.66721668041701	140937.877319665\\
0.667316682917073	140894.332527235\\
0.667416685417135	140850.787734805\\
0.667516687917198	140807.242942375\\
0.66761669041726	140763.698149946\\
0.667716692917323	140720.726315311\\
0.667816695417385	140677.181522881\\
0.667916697917448	140633.636730451\\
0.66801670041751	140590.091938021\\
0.668116702917573	140546.547145591\\
0.668216705417635	140503.575310956\\
0.668316707917698	140460.030518526\\
0.66841671041776	140416.485726096\\
0.668516712917823	140372.940933666\\
0.668616715417885	140329.969099031\\
0.668716717917948	140286.424306602\\
0.66881672041801	140242.879514172\\
0.668916722918073	140199.334721742\\
0.669016725418136	140156.362887107\\
0.669116727918198	140112.818094677\\
0.66921673041826	140069.273302247\\
0.669316732918323	140026.301467612\\
0.669416735418385	139982.756675182\\
0.669516737918448	139939.211882752\\
0.66961674041851	139896.240048117\\
0.669716742918573	139852.695255687\\
0.669816745418635	139809.723421053\\
0.669916747918698	139766.178628623\\
0.670016750418761	139722.633836193\\
0.670116752918823	139679.662001558\\
0.670216755418885	139636.117209128\\
0.670316757918948	139593.145374493\\
0.67041676041901	139549.600582063\\
0.670516762919073	139506.628747429\\
0.670616765419136	139463.083954999\\
0.670716767919198	139420.112120364\\
0.67081677041926	139376.567327934\\
0.670916772919323	139333.595493299\\
0.671016775419386	139290.050700869\\
0.671116777919448	139247.078866234\\
0.67121678041951	139203.534073804\\
0.671316782919573	139160.562239169\\
0.671416785419635	139117.01744674\\
0.671516787919698	139074.045612105\\
0.671616790419761	139030.500819675\\
0.671716792919823	138987.52898504\\
0.671816795419885	138944.557150405\\
0.671916797919948	138901.012357975\\
0.672016800420011	138858.04052334\\
0.672116802920073	138814.49573091\\
0.672216805420136	138771.523896276\\
0.672316807920198	138728.552061641\\
0.67241681042026	138685.007269211\\
0.672516812920323	138642.035434576\\
0.672616815420386	138599.063599941\\
0.672716817920448	138555.518807511\\
0.67281682042051	138512.546972877\\
0.672916822920573	138469.575138242\\
0.673016825420636	138426.603303607\\
0.673116827920698	138383.058511177\\
0.673216830420761	138340.086676542\\
0.673316832920823	138297.114841907\\
0.673416835420885	138254.143007273\\
0.673516837920948	138210.598214843\\
0.673616840421011	138167.626380208\\
0.673716842921073	138124.654545573\\
0.673816845421136	138081.682710938\\
0.673916847921198	138038.710876303\\
0.674016850421261	137995.166083873\\
0.674116852921323	137952.194249239\\
0.674216855421386	137909.222414604\\
0.674316857921448	137866.250579969\\
0.674416860421511	137823.278745334\\
0.674516862921573	137780.306910699\\
0.674616865421636	137737.335076065\\
0.674716867921698	137694.36324143\\
0.674816870421761	137651.391406795\\
0.674916872921823	137607.846614365\\
0.675016875421886	137564.87477973\\
0.675116877921948	137521.902945095\\
0.675216880422011	137478.931110461\\
0.675316882922073	137435.959275826\\
0.675416885422136	137392.987441191\\
0.675516887922198	137350.015606556\\
0.675616890422261	137307.043771921\\
0.675716892922323	137264.071937286\\
0.675816895422386	137221.100102652\\
0.675916897922448	137178.128268017\\
0.676016900422511	137135.156433382\\
0.676116902922573	137092.184598747\\
0.676216905422636	137049.785721908\\
0.676316907922698	137006.813887273\\
0.676416910422761	136963.842052638\\
0.676516912922823	136920.870218003\\
0.676616915422886	136877.898383368\\
0.676716917922948	136834.926548733\\
0.676816920423011	136791.954714099\\
0.676916922923073	136748.982879464\\
0.677016925423136	136706.011044829\\
0.677116927923198	136663.612167989\\
0.677216930423261	136620.640333355\\
0.677316932923323	136577.66849872\\
0.677416935423386	136534.696664085\\
0.677516937923448	136491.72482945\\
0.677616940423511	136449.32595261\\
0.677716942923573	136406.354117976\\
0.677816945423636	136363.382283341\\
0.677916947923698	136320.410448706\\
0.678016950423761	136278.011571866\\
0.678116952923823	136235.039737232\\
0.678216955423886	136192.067902597\\
0.678316957923948	136149.096067962\\
0.678416960424011	136106.697191122\\
0.678516962924073	136063.725356487\\
0.678616965424136	136020.753521853\\
0.678716967924198	135978.354645013\\
0.678816970424261	135935.382810378\\
0.678916972924323	135892.410975743\\
0.679016975424386	135850.012098904\\
0.679116977924448	135807.040264269\\
0.679216980424511	135764.641387429\\
0.679316982924573	135721.669552794\\
0.679416985424636	135678.697718159\\
0.679516987924698	135636.29884132\\
0.679616990424761	135593.327006685\\
0.679716992924823	135550.928129845\\
0.679816995424886	135507.95629521\\
0.679916997924948	135465.557418371\\
0.680017000425011	135422.585583736\\
0.680117002925073	135380.186706896\\
0.680217005425136	135337.214872261\\
0.680317007925198	135294.815995422\\
0.680417010425261	135251.844160787\\
0.680517012925323	135209.445283947\\
0.680617015425386	135166.473449312\\
0.680717017925448	135124.074572473\\
0.680817020425511	135081.102737838\\
0.680917022925573	135038.703860998\\
0.681017025425636	134996.304984159\\
0.681117027925698	134953.333149524\\
0.681217030425761	134910.934272684\\
0.681317032925823	134867.962438049\\
0.681417035425886	134825.56356121\\
0.681517037925948	134783.16468437\\
0.681617040426011	134740.192849735\\
0.681717042926073	134697.793972896\\
0.681817045426136	134655.395096056\\
0.681917047926198	134612.423261421\\
0.682017050426261	134570.024384581\\
0.682117052926323	134527.625507742\\
0.682217055426386	134484.653673107\\
0.682317057926448	134442.254796267\\
0.682417060426511	134399.855919427\\
0.682517062926573	134357.457042588\\
0.682617065426636	134314.485207953\\
0.682717067926698	134272.086331113\\
0.682817070426761	134229.687454274\\
0.682917072926823	134187.288577434\\
0.683017075426886	134144.889700594\\
0.683117077926948	134101.917865959\\
0.683217080427011	134059.51898912\\
0.683317082927073	134017.12011228\\
0.683417085427136	133974.72123544\\
0.683517087927198	133932.322358601\\
0.683617090427261	133889.923481761\\
0.683717092927323	133846.951647126\\
0.683817095427386	133804.552770287\\
0.683917097927448	133762.153893447\\
0.684017100427511	133719.755016607\\
0.684117102927573	133677.356139768\\
0.684217105427636	133634.957262928\\
0.684317107927698	133592.558386088\\
0.684417110427761	133550.159509248\\
0.684517112927823	133507.760632409\\
0.684617115427886	133465.361755569\\
0.684717117927948	133422.962878729\\
0.684817120428011	133380.56400189\\
0.684917122928073	133338.16512505\\
0.685017125428136	133295.76624821\\
0.685117127928198	133253.367371371\\
0.685217130428261	133210.968494531\\
0.685317132928323	133168.569617691\\
0.685417135428386	133126.170740852\\
0.685517137928448	133083.771864012\\
0.685617140428511	133041.372987172\\
0.685717142928573	132998.974110333\\
0.685817145428636	132956.575233493\\
0.685917147928698	132914.749314448\\
0.686017150428761	132872.350437609\\
0.686117152928823	132829.951560769\\
0.686217155428886	132787.552683929\\
0.686317157928948	132745.15380709\\
0.686417160429011	132702.75493025\\
0.686517162929073	132660.35605341\\
0.686617165429136	132618.530134366\\
0.686717167929198	132576.131257526\\
0.686817170429261	132533.732380686\\
0.686917172929323	132491.333503847\\
0.687017175429386	132449.507584802\\
0.687117177929448	132407.108707962\\
0.687217180429511	132364.709831123\\
0.687317182929573	132322.310954283\\
0.687417185429636	132280.485035239\\
0.687517187929698	132238.086158399\\
0.687617190429761	132195.687281559\\
0.687717192929823	132153.861362515\\
0.687817195429886	132111.462485675\\
0.687917197929948	132069.063608835\\
0.688017200430011	132027.237689791\\
0.688117202930073	131984.838812951\\
0.688217205430136	131942.439936111\\
0.688317207930198	131900.614017067\\
0.688417210430261	131858.215140227\\
0.688517212930323	131816.389221183\\
0.688617215430386	131773.990344343\\
0.688717217930448	131731.591467503\\
0.688817220430511	131689.765548459\\
0.688917222930573	131647.366671619\\
0.689017225430636	131605.540752574\\
0.689117227930698	131563.141875735\\
0.689217230430761	131521.31595669\\
0.689317232930823	131478.917079851\\
0.689417235430886	131437.091160806\\
0.689517237930948	131394.692283966\\
0.689617240431011	131352.866364922\\
0.689717242931073	131310.467488082\\
0.689817245431136	131268.641569038\\
0.689917247931198	131226.242692198\\
0.690017250431261	131184.416773153\\
0.690117252931323	131142.017896314\\
0.690217255431386	131100.191977269\\
0.690317257931448	131058.366058225\\
0.690417260431511	131015.967181385\\
0.690517262931573	130974.14126234\\
0.690617265431636	130932.315343296\\
0.690717267931698	130889.916466456\\
0.690817270431761	130848.090547411\\
0.690917272931823	130806.264628367\\
0.691017275431886	130763.865751527\\
0.691117277931948	130722.039832483\\
0.691217280432011	130680.213913438\\
0.691317282932073	130637.815036598\\
0.691417285432136	130595.989117554\\
0.691517287932198	130554.163198509\\
0.691617290432261	130512.337279465\\
0.691717292932323	130469.938402625\\
0.691817295432386	130428.112483581\\
0.691917297932448	130386.286564536\\
0.692017300432511	130344.460645492\\
0.692117302932573	130302.061768652\\
0.692217305432636	130260.235849607\\
0.692317307932698	130218.409930563\\
0.692417310432761	130176.584011518\\
0.692517312932823	130134.758092474\\
0.692617315432886	130092.932173429\\
0.692717317932948	130050.533296589\\
0.692817320433011	130008.707377545\\
0.692917322933073	129966.8814585\\
0.693017325433136	129925.055539456\\
0.693117327933198	129883.229620411\\
0.693217330433261	129841.403701367\\
0.693317332933323	129799.577782322\\
0.693417335433386	129757.751863278\\
0.693517337933448	129715.925944233\\
0.693617340433511	129674.100025188\\
0.693717342933573	129632.274106144\\
0.693817345433636	129590.448187099\\
0.693917347933698	129548.622268055\\
0.694017350433761	129506.79634901\\
0.694117352933823	129464.970429966\\
0.694217355433886	129423.144510921\\
0.694317357933948	129381.318591877\\
0.694417360434011	129339.492672832\\
0.694517362934073	129297.666753788\\
0.694617365434136	129255.840834743\\
0.694717367934198	129214.014915698\\
0.694817370434261	129172.188996654\\
0.694917372934323	129130.363077609\\
0.695017375434386	129088.537158565\\
0.695117377934448	129047.284197315\\
0.695217380434511	129005.458278271\\
0.695317382934573	128963.632359226\\
0.695417385434636	128921.806440182\\
0.695517387934698	128879.980521137\\
0.695617390434761	128838.154602093\\
0.695717392934823	128796.901640843\\
0.695817395434886	128755.075721799\\
0.695917397934948	128713.249802754\\
0.696017400435011	128671.423883709\\
0.696117402935073	128630.17092246\\
0.696217405435136	128588.345003416\\
0.696317407935198	128546.519084371\\
0.696417410435261	128504.693165326\\
0.696517412935323	128463.440204077\\
0.696617415435386	128421.614285032\\
0.696717417935448	128379.788365988\\
0.696817420435511	128338.535404738\\
0.696917422935573	128296.709485694\\
0.697017425435636	128254.883566649\\
0.697117427935698	128213.6306054\\
0.697217430435761	128171.804686355\\
0.697317432935823	128129.978767311\\
0.697417435435886	128088.725806061\\
0.697517437935948	128046.899887017\\
0.697617440436011	128005.646925767\\
0.697717442936073	127963.821006723\\
0.697817445436136	127921.995087678\\
0.697917447936198	127880.742126429\\
0.698017450436261	127838.916207384\\
0.698117452936323	127797.663246135\\
0.698217455436386	127755.83732709\\
0.698317457936448	127714.584365841\\
0.698417460436511	127672.758446796\\
0.698517462936573	127631.505485547\\
0.698617465436636	127589.679566503\\
0.698717467936698	127548.426605253\\
0.698817470436761	127506.600686209\\
0.698917472936823	127465.347724959\\
0.699017475436886	127424.09476371\\
0.699117477936948	127382.268844665\\
0.699217480437011	127341.015883416\\
0.699317482937073	127299.189964371\\
0.699417485437136	127257.937003122\\
0.699517487937198	127216.684041872\\
0.699617490437261	127174.858122828\\
0.699717492937323	127133.605161578\\
0.699817495437386	127092.352200329\\
0.699917497937448	127050.526281284\\
0.700017500437511	127009.273320035\\
0.700117502937573	126968.020358786\\
0.700217505437636	126926.194439741\\
0.700317507937698	126884.941478492\\
0.700417510437761	126843.688517242\\
0.700517512937823	126802.435555993\\
0.700617515437886	126760.609636948\\
0.700717517937948	126719.356675699\\
0.700817520438011	126678.103714449\\
0.700917522938073	126636.8507532\\
0.701017525438136	126595.024834155\\
0.701117527938198	126553.771872906\\
0.701217530438261	126512.518911657\\
0.701317532938323	126471.265950407\\
0.701417535438386	126430.012989158\\
0.701517537938449	126388.760027908\\
0.701617540438511	126346.934108864\\
0.701717542938573	126305.681147614\\
0.701817545438636	126264.428186365\\
0.701917547938698	126223.175225115\\
0.702017550438761	126181.922263866\\
0.702117552938823	126140.669302617\\
0.702217555438886	126099.416341367\\
0.702317557938948	126058.163380118\\
0.702417560439011	126016.910418868\\
0.702517562939074	125975.657457619\\
0.702617565439136	125934.40449637\\
0.702717567939198	125893.15153512\\
0.702817570439261	125851.898573871\\
0.702917572939323	125810.645612621\\
0.703017575439386	125769.392651372\\
0.703117577939449	125728.139690122\\
0.703217580439511	125686.886728873\\
0.703317582939573	125645.633767624\\
0.703417585439636	125604.380806374\\
0.703517587939699	125563.127845125\\
0.703617590439761	125521.874883875\\
0.703717592939823	125480.621922626\\
0.703817595439886	125439.368961377\\
0.703917597939948	125398.116000127\\
0.704017600440011	125356.863038878\\
0.704117602940074	125316.183035423\\
0.704217605440136	125274.930074174\\
0.704317607940198	125233.677112925\\
0.704417610440261	125192.424151675\\
0.704517612940324	125151.171190426\\
0.704617615440386	125110.491186971\\
0.704717617940449	125069.238225722\\
0.704817620440511	125027.985264473\\
0.704917622940573	124986.732303223\\
0.705017625440636	124945.479341974\\
0.705117627940699	124904.799338519\\
0.705217630440761	124863.54637727\\
0.705317632940824	124822.293416021\\
0.705417635440886	124781.613412566\\
0.705517637940949	124740.360451317\\
0.705617640441011	124699.107490068\\
0.705717642941074	124657.854528818\\
0.705817645441136	124617.174525364\\
0.705917647941198	124575.921564114\\
0.706017650441261	124535.24156066\\
0.706117652941324	124493.988599411\\
0.706217655441386	124452.735638161\\
0.706317657941449	124412.055634707\\
0.706417660441511	124370.802673458\\
0.706517662941574	124330.122670003\\
0.706617665441636	124288.869708754\\
0.706717667941699	124247.616747504\\
0.706817670441761	124206.93674405\\
0.706917672941824	124165.683782801\\
0.707017675441886	124125.003779346\\
0.707117677941949	124083.750818097\\
0.707217680442011	124043.070814643\\
0.707317682942074	124001.817853393\\
0.707417685442136	123961.137849939\\
0.707517687942199	123919.88488869\\
0.707617690442261	123879.204885235\\
0.707717692942324	123837.951923986\\
0.707817695442386	123797.271920532\\
0.707917697942449	123756.591917077\\
0.708017700442511	123715.338955828\\
0.708117702942574	123674.658952374\\
0.708217705442636	123633.405991124\\
0.708317707942699	123592.72598767\\
0.708417710442761	123552.045984216\\
0.708517712942824	123510.793022966\\
0.708617715442886	123470.113019512\\
0.708717717942949	123429.433016058\\
0.708817720443011	123388.180054808\\
0.708917722943074	123347.500051354\\
0.709017725443136	123306.8200479\\
0.709117727943199	123265.56708665\\
0.709217730443261	123224.887083196\\
0.709317732943324	123184.207079742\\
0.709417735443386	123143.527076287\\
0.709517737943449	123102.274115038\\
0.709617740443511	123061.594111584\\
0.709717742943574	123020.914108129\\
0.709817745443636	122980.234104675\\
0.709917747943699	122938.981143426\\
0.710017750443761	122898.301139971\\
0.710117752943824	122857.621136517\\
0.710217755443886	122816.941133063\\
0.710317757943949	122776.261129608\\
0.710417760444011	122735.581126154\\
0.710517762944074	122694.9011227\\
0.710617765444136	122653.64816145\\
0.710717767944199	122612.968157996\\
0.710817770444261	122572.288154542\\
0.710917772944324	122531.608151088\\
0.711017775444386	122490.928147633\\
0.711117777944449	122450.248144179\\
0.711217780444511	122409.568140725\\
0.711317782944574	122368.88813727\\
0.711417785444636	122328.208133816\\
0.711517787944699	122287.528130362\\
0.711617790444761	122246.848126908\\
0.711717792944824	122206.168123453\\
0.711817795444886	122165.488119999\\
0.711917797944949	122124.808116545\\
0.712017800445011	122084.12811309\\
0.712117802945074	122043.448109636\\
0.712217805445136	122002.768106182\\
0.712317807945199	121962.088102728\\
0.712417810445261	121921.408099273\\
0.712517812945324	121880.728095819\\
0.712617815445386	121840.62105016\\
0.712717817945449	121799.941046706\\
0.712817820445511	121759.261043251\\
0.712917822945574	121718.581039797\\
0.713017825445636	121677.901036343\\
0.713117827945699	121637.221032888\\
0.713217830445761	121596.541029434\\
0.713317832945824	121556.433983775\\
0.713417835445886	121515.753980321\\
0.713517837945949	121475.073976866\\
0.713617840446011	121434.393973412\\
0.713717842946074	121394.286927753\\
0.713817845446136	121353.606924299\\
0.713917847946199	121312.926920844\\
0.714017850446261	121272.24691739\\
0.714117852946324	121232.139871731\\
0.714217855446386	121191.459868277\\
0.714317857946449	121150.779864822\\
0.714417860446511	121110.672819163\\
0.714517862946574	121069.992815709\\
0.714617865446636	121029.312812255\\
0.714717867946699	120989.205766595\\
0.714817870446761	120948.525763141\\
0.714917872946824	120907.845759687\\
0.715017875446886	120867.738714028\\
0.715117877946949	120827.058710573\\
0.715217880447011	120786.951664914\\
0.715317882947074	120746.27166146\\
0.715417885447136	120705.591658006\\
0.715517887947199	120665.484612347\\
0.715617890447261	120624.804608892\\
0.715717892947324	120584.697563233\\
0.715817895447386	120544.017559779\\
0.715917897947449	120503.91051412\\
0.716017900447511	120463.230510665\\
0.716117902947574	120423.123465006\\
0.716217905447636	120382.443461552\\
0.716317907947699	120342.336415893\\
0.716417910447761	120301.656412438\\
0.716517912947824	120261.549366779\\
0.716617915447886	120221.44232112\\
0.716717917947949	120180.762317666\\
0.716817920448011	120140.655272007\\
0.716917922948074	120099.975268552\\
0.717017925448136	120059.868222893\\
0.717117927948199	120019.761177234\\
0.717217930448261	119979.08117378\\
0.717317932948324	119938.974128121\\
0.717417935448386	119898.867082461\\
0.717517937948449	119858.187079007\\
0.717617940448511	119818.080033348\\
0.717717942948574	119777.972987689\\
0.717817945448636	119737.292984235\\
0.717917947948699	119697.185938575\\
0.718017950448761	119657.078892916\\
0.718117952948824	119616.971847257\\
0.718217955448886	119576.291843803\\
0.718317957948949	119536.184798144\\
0.718417960449011	119496.077752484\\
0.718517962949074	119455.970706825\\
0.718617965449136	119415.290703371\\
0.718717967949199	119375.183657712\\
0.718817970449261	119335.076612053\\
0.718917972949324	119294.969566394\\
0.719017975449386	119254.862520734\\
0.719117977949449	119214.755475075\\
0.719217980449511	119174.648429416\\
0.719317982949574	119133.968425962\\
0.719417985449636	119093.861380303\\
0.719517987949699	119053.754334643\\
0.719617990449761	119013.647288984\\
0.719717992949824	118973.540243325\\
0.719817995449886	118933.433197666\\
0.719917997949949	118893.326152007\\
0.720018000450011	118853.219106348\\
0.720118002950074	118813.112060689\\
0.720218005450136	118773.005015029\\
0.720318007950199	118732.89796937\\
0.720418010450261	118692.790923711\\
0.720518012950324	118652.683878052\\
0.720618015450386	118612.576832393\\
0.720718017950449	118572.469786734\\
0.720818020450511	118532.362741074\\
0.720918022950574	118492.255695415\\
0.721018025450636	118452.148649756\\
0.721118027950699	118412.041604097\\
0.721218030450761	118371.934558438\\
0.721318032950824	118332.400470574\\
0.721418035450886	118292.293424915\\
0.721518037950949	118252.186379255\\
0.721618040451011	118212.079333596\\
0.721718042951074	118171.972287937\\
0.721818045451136	118131.865242278\\
0.721918047951199	118092.331154414\\
0.722018050451261	118052.224108755\\
0.722118052951324	118012.117063096\\
0.722218055451386	117972.010017437\\
0.722318057951449	117931.902971777\\
0.722418060451511	117892.368883913\\
0.722518062951574	117852.261838254\\
0.722618065451636	117812.154792595\\
0.722718067951699	117772.047746936\\
0.722818070451761	117732.513659072\\
0.722918072951824	117692.406613413\\
0.723018075451886	117652.299567754\\
0.723118077951949	117612.765479889\\
0.723218080452011	117572.65843423\\
0.723318082952074	117532.551388571\\
0.723418085452136	117493.017300707\\
0.723518087952199	117452.910255048\\
0.723618090452261	117413.376167184\\
0.723718092952324	117373.269121525\\
0.723818095452386	117333.162075866\\
0.723918097952449	117293.627988002\\
0.724018100452511	117253.520942342\\
0.724118102952574	117213.986854478\\
0.724218105452636	117173.879808819\\
0.724318107952699	117134.345720955\\
0.724418110452761	117094.238675296\\
0.724518112952824	117054.704587432\\
0.724618115452886	117014.597541773\\
0.724718117952949	116975.063453909\\
0.724818120453011	116934.95640825\\
0.724918122953074	116895.422320386\\
0.725018125453136	116855.315274727\\
0.725118127953199	116815.781186862\\
0.725218130453261	116775.674141203\\
0.725318132953324	116736.140053339\\
0.725418135453386	116696.03300768\\
0.725518137953449	116656.498919816\\
0.725618140453511	116616.964831952\\
0.725718142953574	116576.857786293\\
0.725818145453636	116537.323698429\\
0.725918147953699	116497.789610565\\
0.726018150453761	116457.682564906\\
0.726118152953824	116418.148477042\\
0.726218155453886	116378.614389178\\
0.726318157953949	116338.507343519\\
0.726418160454011	116298.973255655\\
0.726518162954074	116259.43916779\\
0.726618165454136	116219.905079926\\
0.726718167954199	116179.798034267\\
0.726818170454261	116140.263946403\\
0.726918172954324	116100.729858539\\
0.727018175454386	116061.195770675\\
0.727118177954449	116021.088725016\\
0.727218180454511	115981.554637152\\
0.727318182954574	115942.020549288\\
0.727418185454636	115902.486461424\\
0.727518187954699	115862.95237356\\
0.727618190454761	115823.418285696\\
0.727718192954824	115783.311240037\\
0.727818195454886	115743.777152173\\
0.727918197954949	115704.243064309\\
0.728018200455011	115664.708976445\\
0.728118202955074	115625.174888581\\
0.728218205455136	115585.640800717\\
0.728318207955199	115546.106712853\\
0.728418210455261	115506.572624989\\
0.728518212955324	115467.038537125\\
0.728618215455386	115427.504449261\\
0.728718217955449	115387.970361396\\
0.728818220455511	115348.436273532\\
0.728918222955574	115308.902185668\\
0.729018225455636	115269.368097804\\
0.729118227955699	115229.83400994\\
0.729218230455761	115190.299922076\\
0.729318232955824	115150.765834212\\
0.729418235455886	115111.231746348\\
0.729518237955949	115071.697658484\\
0.729618240456011	115032.16357062\\
0.729718242956074	114992.629482756\\
0.729818245456136	114953.095394892\\
0.729918247956199	114913.561307028\\
0.730018250456261	114874.600176959\\
0.730118252956324	114835.066089095\\
0.730218255456386	114795.532001231\\
0.730318257956449	114755.997913367\\
0.730418260456511	114716.463825503\\
0.730518262956574	114676.929737639\\
0.730618265456636	114637.96860757\\
0.730718267956699	114598.434519706\\
0.730818270456761	114558.900431842\\
0.730918272956824	114519.366343978\\
0.731018275456886	114480.405213909\\
0.731118277956949	114440.871126045\\
0.731218280457011	114401.337038181\\
0.731318282957074	114361.802950317\\
0.731418285457136	114322.841820248\\
0.731518287957199	114283.307732384\\
0.731618290457261	114243.77364452\\
0.731718292957324	114204.812514451\\
0.731818295457386	114165.278426587\\
0.731918297957449	114125.744338723\\
0.732018300457511	114086.783208654\\
0.732118302957574	114047.24912079\\
0.732218305457636	114007.715032926\\
0.732318307957699	113968.753902857\\
0.732418310457761	113929.219814993\\
0.732518312957824	113890.258684925\\
0.732618315457886	113850.72459706\\
0.732718317957949	113811.763466992\\
0.732818320458011	113772.229379128\\
0.732918322958074	113733.268249059\\
0.733018325458136	113693.734161195\\
0.733118327958199	113654.773031126\\
0.733218330458261	113615.238943262\\
0.733318332958324	113576.277813193\\
0.733418335458386	113536.743725329\\
0.733518337958449	113497.78259526\\
0.733618340458511	113458.248507396\\
0.733718342958574	113419.287377327\\
0.733818345458636	113379.753289463\\
0.733918347958699	113340.792159394\\
0.734018350458762	113301.831029325\\
0.734118352958824	113262.296941461\\
0.734218355458886	113223.335811392\\
0.734318357958949	113183.801723528\\
0.734418360459011	113144.840593459\\
0.734518362959074	113105.87946339\\
0.734618365459137	113066.345375526\\
0.734718367959199	113027.384245458\\
0.734818370459261	112988.423115389\\
0.734918372959324	112949.46198532\\
0.735018375459387	112909.927897456\\
0.735118377959449	112870.966767387\\
0.735218380459511	112832.005637318\\
0.735318382959574	112793.044507249\\
0.735418385459636	112753.510419385\\
0.735518387959699	112714.549289316\\
0.735618390459762	112675.588159247\\
0.735718392959824	112636.627029178\\
0.735818395459886	112597.665899109\\
0.735918397959949	112558.131811245\\
0.736018400460012	112519.170681176\\
0.736118402960074	112480.209551108\\
0.736218405460137	112441.248421039\\
0.736318407960199	112402.28729097\\
0.736418410460261	112363.326160901\\
0.736518412960324	112324.365030832\\
0.736618415460387	112285.403900763\\
0.736718417960449	112245.869812899\\
0.736818420460511	112206.90868283\\
0.736918422960574	112167.947552761\\
0.737018425460637	112128.986422692\\
0.737118427960699	112090.025292623\\
0.737218430460762	112051.064162555\\
0.737318432960824	112012.103032486\\
0.737418435460886	111973.141902417\\
0.737518437960949	111934.180772348\\
0.737618440461012	111895.219642279\\
0.737718442961074	111856.25851221\\
0.737818445461137	111817.297382141\\
0.737918447961199	111778.336252072\\
0.738018450461262	111739.948079799\\
0.738118452961324	111700.98694973\\
0.738218455461387	111662.025819661\\
0.738318457961449	111623.064689592\\
0.738418460461511	111584.103559523\\
0.738518462961574	111545.142429454\\
0.738618465461637	111506.181299385\\
0.738718467961699	111467.220169316\\
0.738818470461762	111428.831997043\\
0.738918472961824	111389.870866974\\
0.739018475461887	111350.909736905\\
0.739118477961949	111311.948606836\\
0.739218480462012	111272.987476767\\
0.739318482962074	111234.599304493\\
0.739418485462137	111195.638174424\\
0.739518487962199	111156.677044355\\
0.739618490462262	111117.715914286\\
0.739718492962324	111079.327742013\\
0.739818495462387	111040.366611944\\
0.739918497962449	111001.405481875\\
0.740018500462512	110963.017309601\\
0.740118502962574	110924.056179532\\
0.740218505462637	110885.095049463\\
0.740318507962699	110846.70687719\\
0.740418510462762	110807.745747121\\
0.740518512962824	110768.784617052\\
0.740618515462887	110730.396444778\\
0.740718517962949	110691.435314709\\
0.740818520463012	110653.047142435\\
0.740918522963074	110614.086012366\\
0.741018525463137	110575.124882298\\
0.741118527963199	110536.736710024\\
0.741218530463262	110497.775579955\\
0.741318532963324	110459.387407681\\
0.741418535463387	110420.426277612\\
0.741518537963449	110382.038105338\\
0.741618540463512	110343.07697527\\
0.741718542963574	110304.688802996\\
0.741818545463637	110265.727672927\\
0.741918547963699	110227.339500653\\
0.742018550463762	110188.378370584\\
0.742118552963824	110149.990198311\\
0.742218555463887	110111.602026037\\
0.742318557963949	110072.640895968\\
0.742418560464012	110034.252723694\\
0.742518562964074	109995.291593625\\
0.742618565464137	109956.903421351\\
0.742718567964199	109918.515249078\\
0.742818570464262	109879.554119009\\
0.742918572964324	109841.165946735\\
0.743018575464387	109802.777774461\\
0.743118577964449	109763.816644392\\
0.743218580464512	109725.428472119\\
0.743318582964574	109687.040299845\\
0.743418585464637	109648.652127571\\
0.743518587964699	109609.690997502\\
0.743618590464762	109571.302825228\\
0.743718592964824	109532.914652955\\
0.743818595464887	109494.526480681\\
0.743918597964949	109455.565350612\\
0.744018600465012	109417.177178338\\
0.744118602965074	109378.789006064\\
0.744218605465137	109340.400833791\\
0.744318607965199	109302.012661517\\
0.744418610465262	109263.051531448\\
0.744518612965324	109224.663359174\\
0.744618615465387	109186.2751869\\
0.744718617965449	109147.887014627\\
0.744818620465512	109109.498842353\\
0.744918622965574	109071.110670079\\
0.745018625465637	109032.722497805\\
0.745118627965699	108994.334325532\\
0.745218630465762	108955.946153258\\
0.745318632965824	108917.557980984\\
0.745418635465887	108878.596850915\\
0.745518637965949	108840.208678641\\
0.745618640466012	108801.820506368\\
0.745718642966074	108763.432334094\\
0.745818645466137	108725.04416182\\
0.745918647966199	108686.655989546\\
0.746018650466262	108648.267817273\\
0.746118652966324	108610.452602794\\
0.746218655466387	108572.06443052\\
0.746318657966449	108533.676258246\\
0.746418660466512	108495.288085973\\
0.746518662966574	108456.899913699\\
0.746618665466637	108418.511741425\\
0.746718667966699	108380.123569151\\
0.746818670466762	108341.735396878\\
0.746918672966824	108303.347224604\\
0.747018675466887	108264.95905233\\
0.747118677966949	108227.143837851\\
0.747218680467012	108188.755665578\\
0.747318682967074	108150.367493304\\
0.747418685467137	108111.97932103\\
0.747518687967199	108073.591148756\\
0.747618690467262	108035.775934278\\
0.747718692967324	107997.387762004\\
0.747818695467387	107958.99958973\\
0.747918697967449	107920.611417456\\
0.748018700467512	107882.796202978\\
0.748118702967574	107844.408030704\\
0.748218705467637	107806.01985843\\
0.748318707967699	107768.204643952\\
0.748418710467762	107729.816471678\\
0.748518712967824	107691.428299404\\
0.748618715467887	107653.613084925\\
0.748718717967949	107615.224912652\\
0.748818720468012	107576.836740378\\
0.748918722968074	107539.021525899\\
0.749018725468137	107500.633353626\\
0.749118727968199	107462.245181352\\
0.749218730468262	107424.429966873\\
0.749318732968324	107386.041794599\\
0.749418735468387	107348.226580121\\
0.749518737968449	107309.838407847\\
0.749618740468512	107272.023193368\\
0.749718742968574	107233.635021095\\
0.749818745468637	107195.819806616\\
0.749918747968699	107157.431634342\\
0.750018750468762	107119.616419864\\
0.750118752968824	107081.22824759\\
0.750218755468887	107043.413033111\\
0.750318757968949	107005.024860837\\
0.750418760469012	106967.209646359\\
0.750518762969074	106928.821474085\\
0.750618765469137	106891.006259606\\
0.750718767969199	106853.191045128\\
0.750818770469262	106814.802872854\\
0.750918772969324	106776.987658375\\
0.751018775469387	106738.599486102\\
0.751118777969449	106700.784271623\\
0.751218780469512	106662.969057144\\
0.751318782969574	106624.580884871\\
0.751418785469637	106586.765670392\\
0.751518787969699	106548.950455913\\
0.751618790469762	106510.56228364\\
0.751718792969824	106472.747069161\\
0.751818795469887	106434.931854682\\
0.751918797969949	106397.116640204\\
0.752018800470012	106358.72846793\\
0.752118802970074	106320.913253451\\
0.752218805470137	106283.098038973\\
0.752318807970199	106245.282824494\\
0.752418810470262	106207.467610015\\
0.752518812970324	106169.079437742\\
0.752618815470387	106131.264223263\\
0.752718817970449	106093.449008784\\
0.752818820470512	106055.633794306\\
0.752918822970574	106017.818579827\\
0.753018825470637	105980.003365348\\
0.753118827970699	105941.615193075\\
0.753218830470762	105903.799978596\\
0.753318832970824	105865.984764117\\
0.753418835470887	105828.169549639\\
0.753518837970949	105790.35433516\\
0.753618840471012	105752.539120681\\
0.753718842971074	105714.723906203\\
0.753818845471137	105676.908691724\\
0.753918847971199	105639.093477246\\
0.754018850471262	105601.278262767\\
0.754118852971324	105563.463048288\\
0.754218855471387	105525.64783381\\
0.754318857971449	105487.832619331\\
0.754418860471512	105450.017404852\\
0.754518862971574	105412.202190374\\
0.754618865471637	105374.386975895\\
0.754718867971699	105336.571761416\\
0.754818870471762	105298.756546938\\
0.754918872971824	105260.941332459\\
0.755018875471887	105223.699075776\\
0.755118877971949	105185.883861297\\
0.755218880472012	105148.068646818\\
0.755318882972074	105110.25343234\\
0.755418885472137	105072.438217861\\
0.755518887972199	105034.623003383\\
0.755618890472262	104997.380746699\\
0.755718892972324	104959.56553222\\
0.755818895472387	104921.750317742\\
0.755918897972449	104883.935103263\\
0.756018900472512	104846.119888784\\
0.756118902972574	104808.877632101\\
0.756218905472637	104771.062417622\\
0.756318907972699	104733.247203144\\
0.756418910472762	104695.431988665\\
0.756518912972824	104658.189731982\\
0.756618915472887	104620.374517503\\
0.756718917972949	104582.559303024\\
0.756818920473012	104545.317046341\\
0.756918922973074	104507.501831862\\
0.757018925473137	104469.686617384\\
0.757118927973199	104432.4443607\\
0.757218930473262	104394.629146221\\
0.757318932973324	104357.386889538\\
0.757418935473387	104319.571675059\\
0.757518937973449	104281.756460581\\
0.757618940473512	104244.514203897\\
0.757718942973574	104206.698989418\\
0.757818945473637	104169.456732735\\
0.757918947973699	104131.641518256\\
0.758018950473762	104094.399261573\\
0.758118952973824	104056.584047094\\
0.758218955473887	104019.341790411\\
0.758318957973949	103981.526575932\\
0.758418960474012	103944.284319249\\
0.758518962974074	103906.46910477\\
0.758618965474137	103869.226848086\\
0.758718967974199	103831.411633608\\
0.758818970474262	103794.169376924\\
0.758918972974324	103756.354162446\\
0.759018975474387	103719.111905762\\
0.759118977974449	103681.869649079\\
0.759218980474512	103644.0544346\\
0.759318982974574	103606.812177917\\
0.759418985474637	103569.569921233\\
0.759518987974699	103531.754706754\\
0.759618990474762	103494.512450071\\
0.759718992974824	103457.270193387\\
0.759818995474887	103419.454978909\\
0.759918997974949	103382.212722225\\
0.760019000475012	103344.970465542\\
0.760119002975074	103307.155251063\\
0.760219005475137	103269.91299438\\
0.760319007975199	103232.670737696\\
0.760419010475262	103195.428481013\\
0.760519012975324	103157.613266534\\
0.760619015475387	103120.37100985\\
0.760719017975449	103083.128753167\\
0.760819020475512	103045.886496483\\
0.760919022975574	103008.6442398\\
0.761019025475637	102971.401983116\\
0.761119027975699	102933.586768638\\
0.761219030475762	102896.344511954\\
0.761319032975824	102859.102255271\\
0.761419035475887	102821.859998587\\
0.761519037975949	102784.617741904\\
0.761619040476012	102747.37548522\\
0.761719042976074	102710.133228537\\
0.761819045476137	102672.890971853\\
0.761919047976199	102635.64871517\\
0.762019050476262	102598.406458486\\
0.762119052976324	102561.164201803\\
0.762219055476387	102523.921945119\\
0.762319057976449	102486.679688436\\
0.762419060476512	102449.437431752\\
0.762519062976574	102412.195175069\\
0.762619065476637	102374.952918385\\
0.762719067976699	102337.710661702\\
0.762819070476762	102300.468405018\\
0.762919072976824	102263.226148335\\
0.763019075476887	102225.983891651\\
0.763119077976949	102188.741634968\\
0.763219080477012	102151.499378284\\
0.763319082977074	102114.257121601\\
0.763419085477137	102077.014864917\\
0.763519087977199	102039.772608234\\
0.763619090477262	102003.103309345\\
0.763719092977324	101965.861052662\\
0.763819095477387	101928.618795978\\
0.763919097977449	101891.376539295\\
0.764019100477512	101854.134282611\\
0.764119102977574	101817.464983723\\
0.764219105477637	101780.222727039\\
0.764319107977699	101742.980470356\\
0.764419110477762	101705.738213672\\
0.764519112977824	101669.068914784\\
0.764619115477887	101631.826658101\\
0.764719117977949	101594.584401417\\
0.764819120478012	101557.342144734\\
0.764919122978074	101520.672845845\\
0.765019125478137	101483.430589162\\
0.765119127978199	101446.188332478\\
0.765219130478262	101409.51903359\\
0.765319132978324	101372.276776906\\
0.765419135478387	101335.607478018\\
0.76551913797845	101298.365221334\\
0.765619140478512	101261.122964651\\
0.765719142978574	101224.453665763\\
0.765819145478637	101187.211409079\\
0.765919147978699	101150.542110191\\
0.766019150478762	101113.299853507\\
0.766119152978824	101076.057596824\\
0.766219155478887	101039.388297935\\
0.766319157978949	101002.146041252\\
0.766419160479012	100965.476742363\\
0.766519162979075	100928.23448568\\
0.766619165479137	100891.565186792\\
0.766719167979199	100854.322930108\\
0.766819170479262	100817.65363122\\
0.766919172979324	100780.984332331\\
0.767019175479387	100743.742075648\\
0.76711917797945	100707.072776759\\
0.767219180479512	100669.830520076\\
0.767319182979574	100633.161221188\\
0.767419185479637	100596.491922299\\
0.7675191879797	100559.249665616\\
0.767619190479762	100522.580366727\\
0.767719192979824	100485.338110044\\
0.767819195479887	100448.668811155\\
0.767919197979949	100411.999512267\\
0.768019200480012	100375.330213379\\
0.768119202980075	100338.087956695\\
0.768219205480137	100301.418657807\\
0.768319207980199	100264.749358918\\
0.768419210480262	100227.507102235\\
0.768519212980325	100190.837803347\\
0.768619215480387	100154.168504458\\
0.76871921798045	100117.49920557\\
0.768819220480512	100080.256948886\\
0.768919222980574	100043.587649998\\
0.769019225480637	100006.91835111\\
0.7691192279807	99970.2490522212\\
0.769219230480762	99933.5797533328\\
0.769319232980824	99896.9104544444\\
0.769419235480887	99859.6681977609\\
0.76951923798095	99822.9988988726\\
0.769619240481012	99786.3295999842\\
0.769719242981075	99749.6603010958\\
0.769819245481137	99712.9910022074\\
0.769919247981199	99676.3217033191\\
0.770019250481262	99639.6524044307\\
0.770119252981325	99602.9831055423\\
0.770219255481387	99566.3138066539\\
0.77031925798145	99529.6445077656\\
0.770419260481512	99492.9752088772\\
0.770519262981575	99456.3059099888\\
0.770619265481637	99419.6366111004\\
0.7707192679817	99382.9673122121\\
0.770819270481762	99346.2980133237\\
0.770919272981824	99309.6287144353\\
0.771019275481887	99272.959415547\\
0.77111927798195	99236.2901166586\\
0.771219280482012	99199.6208177702\\
0.771319282982075	99162.9515188818\\
0.771419285482137	99126.2822199935\\
0.7715192879822	99089.6129211051\\
0.771619290482262	99053.5165800118\\
0.771719292982325	99016.8472811235\\
0.771819295482387	98980.1779822351\\
0.77191929798245	98943.5086833467\\
0.772019300482512	98906.8393844584\\
0.772119302982575	98870.17008557\\
0.772219305482637	98834.0737444768\\
0.7723193079827	98797.4044455884\\
0.772419310482762	98760.7351467\\
0.772519312982825	98724.0658478116\\
0.772619315482887	98687.3965489233\\
0.77271931798295	98651.30020783\\
0.772819320483012	98614.6309089416\\
0.772919322983075	98577.9616100533\\
0.773019325483137	98541.86526896\\
0.7731193279832	98505.1959700717\\
0.773219330483262	98468.5266711833\\
0.773319332983325	98432.43033009\\
0.773419335483387	98395.7610312016\\
0.77351933798345	98359.0917323133\\
0.773619340483512	98322.99539122\\
0.773719342983575	98286.3260923317\\
0.773819345483637	98249.6567934433\\
0.7739193479837	98213.5604523501\\
0.774019350483762	98176.8911534617\\
0.774119352983825	98140.7948123685\\
0.774219355483887	98104.1255134801\\
0.77431935798395	98068.0291723868\\
0.774419360484012	98031.3598734985\\
0.774519362984075	97995.2635324052\\
0.774619365484137	97958.5942335169\\
0.7747193679842	97922.4978924236\\
0.774819370484262	97885.8285935352\\
0.774919372984325	97849.732252442\\
0.775019375484387	97813.0629535536\\
0.77511937798445	97776.9666124604\\
0.775219380484512	97740.297313572\\
0.775319382984575	97704.2009724788\\
0.775419385484637	97668.1046313855\\
0.7755193879847	97631.4353324971\\
0.775619390484762	97595.3389914039\\
0.775719392984825	97558.6696925155\\
0.775819395484887	97522.5733514223\\
0.77591939798495	97486.4770103291\\
0.776019400485012	97449.8077114407\\
0.776119402985075	97413.7113703474\\
0.776219405485137	97377.6150292542\\
0.7763194079852	97340.9457303658\\
0.776419410485262	97304.8493892726\\
0.776519412985325	97268.7530481793\\
0.776619415485387	97232.6567070861\\
0.77671941798545	97195.9874081977\\
0.776819420485512	97159.8910671045\\
0.776919422985575	97123.7947260112\\
0.777019425485637	97087.698384918\\
0.7771194279857	97051.6020438248\\
0.777219430485762	97014.9327449364\\
0.777319432985825	96978.8364038431\\
0.777419435485887	96942.7400627499\\
0.77751943798595	96906.6437216567\\
0.777619440486012	96870.5473805634\\
0.777719442986075	96834.4510394702\\
0.777819445486137	96797.7817405818\\
0.7779194479862	96761.6853994886\\
0.778019450486262	96725.5890583953\\
0.778119452986325	96689.4927173021\\
0.778219455486387	96653.3963762088\\
0.77831945798645	96617.3000351156\\
0.778419460486512	96581.2036940223\\
0.778519462986575	96545.1073529291\\
0.778619465486637	96509.0110118359\\
0.7787194679867	96472.9146707426\\
0.778819470486762	96436.8183296494\\
0.778919472986825	96400.7219885561\\
0.779019475486887	96364.6256474629\\
0.77911947798695	96328.5293063697\\
0.779219480487012	96292.4329652764\\
0.779319482987075	96256.3366241832\\
0.779419485487137	96220.2402830899\\
0.7795194879872	96184.1439419967\\
0.779619490487262	96148.6205586986\\
0.779719492987325	96112.5242176053\\
0.779819495487387	96076.4278765121\\
0.77991949798745	96040.3315354188\\
0.780019500487512	96004.2351943256\\
0.780119502987575	95968.1388532324\\
0.780219505487637	95932.0425121391\\
0.7803195079877	95896.519128841\\
0.780419510487762	95860.4227877478\\
0.780519512987825	95824.3264466545\\
0.780619515487887	95788.2301055613\\
0.78071951798795	95752.7067222632\\
0.780819520488012	95716.6103811699\\
0.780919522988075	95680.5140400767\\
0.781019525488137	95644.4176989835\\
0.7811195279882	95608.8943156853\\
0.781219530488262	95572.7979745921\\
0.781319532988325	95536.7016334989\\
0.781419535488387	95501.1782502007\\
0.78151953798845	95465.0819091075\\
0.781619540488512	95428.9855680143\\
0.781719542988575	95393.4621847162\\
0.781819545488637	95357.3658436229\\
0.7819195479887	95321.2695025297\\
0.782019550488762	95285.7461192316\\
0.782119552988825	95249.6497781383\\
0.782219555488887	95214.1263948402\\
0.78231955798895	95178.030053747\\
0.782419560489012	95142.5066704489\\
0.782519562989075	95106.4103293556\\
0.782619565489137	95070.8869460575\\
0.7827195679892	95034.7906049643\\
0.782819570489262	94999.2672216661\\
0.782919572989325	94963.1708805729\\
0.783019575489387	94927.6474972748\\
0.78311957798945	94891.5511561816\\
0.783219580489512	94856.0277728834\\
0.783319582989575	94819.9314317902\\
0.783419585489637	94784.4080484921\\
0.7835195879897	94748.3117073989\\
0.783619590489762	94712.7883241007\\
0.783719592989825	94677.2649408026\\
0.783819595489887	94641.1685997094\\
0.78391959798995	94605.6452164113\\
0.784019600490012	94570.1218331132\\
0.784119602990075	94534.0254920199\\
0.784219605490137	94498.5021087218\\
0.7843196079902	94462.9787254237\\
0.784419610490262	94426.8823843305\\
0.784519612990325	94391.3590010324\\
0.784619615490387	94355.8356177342\\
0.78471961799045	94319.739276641\\
0.784819620490512	94284.2158933429\\
0.784919622990575	94248.6925100448\\
0.785019625490637	94213.1691267467\\
0.7851196279907	94177.6457434485\\
0.785219630490762	94141.5494023553\\
0.785319632990825	94106.0260190572\\
0.785419635490887	94070.5026357591\\
0.78551963799095	94034.979252461\\
0.785619640491012	93999.4558691629\\
0.785719642991075	93963.9324858648\\
0.785819645491137	93927.8361447715\\
0.7859196479912	93892.3127614734\\
0.786019650491262	93856.7893781753\\
0.786119652991325	93821.2659948772\\
0.786219655491387	93785.7426115791\\
0.78631965799145	93750.219228281\\
0.786419660491512	93714.6958449829\\
0.786519662991575	93679.1724616847\\
0.786619665491637	93643.6490783866\\
0.7867196679917	93608.1256950885\\
0.786819670491762	93572.6023117904\\
0.786919672991825	93537.0789284923\\
0.787019675491887	93501.5555451942\\
0.78711967799195	93466.0321618961\\
0.787219680492012	93430.508778598\\
0.787319682992075	93394.9853952998\\
0.787419685492137	93359.4620120017\\
0.7875196879922	93323.9386287036\\
0.787619690492262	93288.4152454055\\
0.787719692992325	93252.8918621074\\
0.787819695492387	93217.3684788093\\
0.78791969799245	93182.4180533063\\
0.788019700492512	93146.8946700082\\
0.788119702992575	93111.3712867101\\
0.788219705492637	93075.847903412\\
0.7883197079927	93040.3245201139\\
0.788419710492762	93004.8011368158\\
0.788519712992825	92969.8507113128\\
0.788619715492887	92934.3273280147\\
0.78871971799295	92898.8039447166\\
0.788819720493012	92863.2805614184\\
0.788919722993075	92828.3301359155\\
0.789019725493137	92792.8067526174\\
0.7891197279932	92757.2833693193\\
0.789219730493262	92721.7599860211\\
0.789319732993325	92686.8095605181\\
0.789419735493387	92651.28617722\\
0.78951973799345	92615.7627939219\\
0.789619740493512	92580.8123684189\\
0.789719742993575	92545.2889851208\\
0.789819745493637	92509.7656018227\\
0.7899197479937	92474.8151763197\\
0.790019750493762	92439.2917930216\\
0.790119752993825	92404.3413675186\\
0.790219755493887	92368.8179842205\\
0.79031975799395	92333.2946009224\\
0.790419760494012	92298.3441754194\\
0.790519762994075	92262.8207921213\\
0.790619765494137	92227.8703666184\\
0.7907197679942	92192.3469833202\\
0.790819770494262	92157.3965578173\\
0.790919772994325	92121.8731745191\\
0.791019775494387	92086.9227490162\\
0.79111977799445	92051.3993657181\\
0.791219780494512	92016.4489402151\\
0.791319782994575	91980.925556917\\
0.791419785494637	91945.975131414\\
0.7915197879947	91911.024705911\\
0.791619790494762	91875.5013226129\\
0.791719792994825	91840.5508971099\\
0.791819795494887	91805.0275138118\\
0.79191979799495	91770.0770883088\\
0.792019800495012	91735.1266628058\\
0.792119802995075	91699.6032795077\\
0.792219805495137	91664.6528540048\\
0.7923198079952	91629.7024285018\\
0.792419810495262	91594.1790452037\\
0.792519812995325	91559.2286197007\\
0.792619815495387	91524.2781941977\\
0.79271981799545	91488.7548108996\\
0.792819820495512	91453.8043853966\\
0.792919822995575	91418.8539598936\\
0.793019825495637	91383.9035343907\\
0.7931198279957	91348.3801510925\\
0.793219830495762	91313.4297255896\\
0.793319832995825	91278.4793000866\\
0.793419835495887	91243.5288745836\\
0.79351983799595	91208.5784490806\\
0.793619840496012	91173.0550657825\\
0.793719842996075	91138.1046402795\\
0.793819845496137	91103.1542147765\\
0.7939198479962	91068.2037892736\\
0.794019850496262	91033.2533637706\\
0.794119852996325	90998.3029382676\\
0.794219855496387	90963.3525127646\\
0.79431985799645	90928.4020872617\\
0.794419860496512	90893.4516617587\\
0.794519862996575	90858.5012362557\\
0.794619865496637	90822.9778529576\\
0.7947198679967	90788.0274274546\\
0.794819870496762	90753.0770019516\\
0.794919872996825	90718.1265764486\\
0.795019875496887	90683.1761509456\\
0.79511987799695	90648.2257254427\\
0.795219880497012	90613.2752999397\\
0.795319882997075	90578.8978322318\\
0.795419885497137	90543.9474067289\\
0.7955198879972	90508.9969812259\\
0.795619890497262	90474.0465557229\\
0.795719892997325	90439.0961302199\\
0.795819895497387	90404.1457047169\\
0.79591989799745	90369.195279214\\
0.796019900497512	90334.244853711\\
0.796119902997575	90299.294428208\\
0.796219905497637	90264.9169605002\\
0.7963199079977	90229.9665349972\\
0.796419910497762	90195.0161094942\\
0.796519912997825	90160.0656839912\\
0.796619915497887	90125.1152584882\\
0.79671991799795	90090.1648329852\\
0.796819920498012	90055.7873652774\\
0.796919922998075	90020.8369397744\\
0.797019925498137	89985.8865142714\\
0.7971199279982	89950.9360887685\\
0.797219930498262	89916.5586210606\\
0.797319932998325	89881.6081955576\\
0.797419935498387	89846.6577700546\\
0.79751993799845	89812.2803023468\\
0.797619940498512	89777.3298768438\\
0.797719942998575	89742.3794513408\\
0.797819945498637	89708.001983633\\
0.7979199479987	89673.05155813\\
0.798019950498763	89638.101132627\\
0.798119952998825	89603.7236649192\\
0.798219955498887	89568.7732394162\\
0.79831995799895	89534.3957717084\\
0.798419960499012	89499.4453462054\\
0.798519962999075	89465.0678784975\\
0.798619965499137	89430.1174529945\\
0.7987199679992	89395.1670274916\\
0.798819970499262	89360.7895597837\\
0.798919972999325	89325.8391342807\\
0.799019975499388	89291.4616665729\\
0.79911997799945	89256.5112410699\\
0.799219980499512	89222.133773362\\
0.799319982999575	89187.7563056542\\
0.799419985499637	89152.8058801512\\
0.7995199879997	89118.4284124434\\
0.799619990499763	89083.4779869404\\
0.799719992999825	89049.1005192326\\
0.799819995499887	89014.1500937296\\
0.79991999799995	88979.7726260217\\
0.800020000500013	88945.3951583139\\
};
\addplot [color=mycolor1,solid,forget plot]
  table[row sep=crcr]{%
0.800020000500013	88945.3951583139\\
0.800120003000075	88910.4447328109\\
0.800220005500137	88876.067265103\\
0.8003200080002	88841.6897973952\\
0.800420010500262	88806.7393718922\\
0.800520013000325	88772.3619041843\\
0.800620015500388	88737.9844364765\\
0.80072001800045	88703.0340109735\\
0.800820020500512	88668.6565432657\\
0.800920023000575	88634.2790755578\\
0.801020025500638	88599.90160785\\
0.8011200280007	88564.951182347\\
0.801220030500763	88530.5737146391\\
0.801320033000825	88496.1962469313\\
0.801420035500887	88461.8187792235\\
0.80152003800095	88427.4413115156\\
0.801620040501013	88392.4908860126\\
0.801720043001075	88358.1134183048\\
0.801820045501138	88323.7359505969\\
0.8019200480012	88289.3584828891\\
0.802020050501263	88254.9810151812\\
0.802120053001325	88220.6035474734\\
0.802220055501388	88186.2260797655\\
0.80232005800145	88151.2756542626\\
0.802420060501512	88116.8981865547\\
0.802520063001575	88082.5207188468\\
0.802620065501638	88048.143251139\\
0.8027200680017	88013.7657834312\\
0.802820070501763	87979.3883157233\\
0.802920073001825	87945.0108480155\\
0.803020075501888	87910.6333803076\\
0.80312007800195	87876.2559125998\\
0.803220080502013	87841.8784448919\\
0.803320083002075	87807.5009771841\\
0.803420085502138	87773.1235094762\\
0.8035200880022	87738.7460417683\\
0.803620090502263	87704.3685740605\\
0.803720093002325	87670.5640641478\\
0.803820095502388	87636.1865964399\\
0.80392009800245	87601.8091287321\\
0.804020100502513	87567.4316610242\\
0.804120103002575	87533.0541933164\\
0.804220105502638	87498.6767256085\\
0.8043201080027	87464.2992579007\\
0.804420110502763	87430.494747988\\
0.804520113002825	87396.1172802801\\
0.804620115502888	87361.7398125723\\
0.80472011800295	87327.3623448644\\
0.804820120503013	87292.9848771566\\
0.804920123003075	87259.1803672439\\
0.805020125503138	87224.802899536\\
0.8051201280032	87190.4254318281\\
0.805220130503263	87156.0479641203\\
0.805320133003325	87122.2434542076\\
0.805420135503388	87087.8659864997\\
0.80552013800345	87053.4885187919\\
0.805620140503513	87019.6840088792\\
0.805720143003575	86985.3065411713\\
0.805820145503638	86950.9290734635\\
0.8059201480037	86917.1245635508\\
0.806020150503763	86882.7470958429\\
0.806120153003825	86848.3696281351\\
0.806220155503888	86814.5651182223\\
0.80632015800395	86780.1876505145\\
0.806420160504013	86746.3831406018\\
0.806520163004075	86712.0056728939\\
0.806620165504138	86677.6282051861\\
0.8067201680042	86643.8236952733\\
0.806820170504263	86609.4462275655\\
0.806920173004325	86575.6417176528\\
0.807020175504388	86541.2642499449\\
0.80712017800445	86507.4597400322\\
0.807220180504513	86473.0822723244\\
0.807320183004575	86439.2777624116\\
0.807420185504638	86404.9002947038\\
0.8075201880047	86371.0957847911\\
0.807620190504763	86337.2912748784\\
0.807720193004825	86302.9138071705\\
0.807820195504888	86269.1092972578\\
0.80792019800495	86234.7318295499\\
0.808020200505013	86200.9273196372\\
0.808120203005075	86167.1228097245\\
0.808220205505138	86132.7453420167\\
0.8083202080052	86098.9408321039\\
0.808420210505263	86065.1363221912\\
0.808520213005325	86030.7588544834\\
0.808620215505388	85996.9543445707\\
0.80872021800545	85963.1498346579\\
0.808820220505513	85928.7723669501\\
0.808920223005575	85894.9678570374\\
0.809020225505638	85861.1633471246\\
0.8091202280057	85827.3588372119\\
0.809220230505763	85792.9813695041\\
0.809320233005825	85759.1768595914\\
0.809420235505888	85725.3723496786\\
0.80952023800595	85691.5678397659\\
0.809620240506013	85657.7633298532\\
0.809720243006075	85623.3858621454\\
0.809820245506138	85589.5813522326\\
0.8099202480062	85555.7768423199\\
0.810020250506263	85521.9723324072\\
0.810120253006325	85488.1678224945\\
0.810220255506388	85454.3633125818\\
0.81032025800645	85420.558802669\\
0.810420260506513	85386.7542927563\\
0.810520263006575	85352.9497828436\\
0.810620265506638	85319.1452729309\\
0.8107202680067	85284.767805223\\
0.810820270506763	85250.9632953103\\
0.810920273006825	85217.1587853976\\
0.811020275506888	85183.3542754849\\
0.81112027800695	85149.5497655722\\
0.811220280507013	85115.7452556595\\
0.811320283007075	85081.9407457467\\
0.811420285507138	85048.7091936291\\
0.8115202880072	85014.9046837164\\
0.811620290507263	84981.1001738037\\
0.811720293007325	84947.295663891\\
0.811820295507388	84913.4911539783\\
0.81192029800745	84879.6866440656\\
0.812020300507513	84845.8821341528\\
0.812120303007575	84812.0776242401\\
0.812220305507638	84778.2731143274\\
0.8123203080077	84744.4686044147\\
0.812420310507763	84711.2370522971\\
0.812520313007825	84677.4325423844\\
0.812620315507888	84643.6280324716\\
0.81272031800795	84609.8235225589\\
0.812820320508013	84576.0190126462\\
0.812920323008075	84542.7874605286\\
0.813020325508138	84508.9829506159\\
0.8131203280082	84475.1784407032\\
0.813220330508263	84441.3739307905\\
0.813320333008325	84408.1423786729\\
0.813420335508388	84374.3378687602\\
0.81352033800845	84340.5333588474\\
0.813620340508513	84307.3018067299\\
0.813720343008575	84273.4972968171\\
0.813820345508638	84239.6927869044\\
0.8139203480087	84206.4612347868\\
0.814020350508763	84172.6567248741\\
0.814120353008825	84138.8522149614\\
0.814220355508888	84105.6206628438\\
0.81432035800895	84071.8161529311\\
0.814420360509013	84038.5846008135\\
0.814520363009075	84004.7800909008\\
0.814620365509138	83970.9755809881\\
0.8147203680092	83937.7440288705\\
0.814820370509263	83903.9395189577\\
0.814920373009325	83870.7079668402\\
0.815020375509388	83836.9034569275\\
0.81512037800945	83803.6719048099\\
0.815220380509513	83769.8673948971\\
0.815320383009575	83736.6358427796\\
0.815420385509638	83702.8313328668\\
0.8155203880097	83669.5997807492\\
0.815620390509763	83636.3682286317\\
0.815720393009825	83602.5637187189\\
0.815820395509888	83569.3321666014\\
0.81592039800995	83535.5276566886\\
0.816020400510013	83502.2961045711\\
0.816120403010075	83469.0645524534\\
0.816220405510138	83435.2600425407\\
0.8163204080102	83402.0284904232\\
0.816420410510263	83368.7969383056\\
0.816520413010325	83334.9924283928\\
0.816620415510388	83301.7608762753\\
0.81672041801045	83268.5293241577\\
0.816820420510513	83234.724814245\\
0.816920423010575	83201.4932621274\\
0.817020425510638	83168.2617100098\\
0.8171204280107	83135.0301578922\\
0.817220430510763	83101.2256479795\\
0.817320433010825	83067.9940958619\\
0.817420435510888	83034.7625437443\\
0.81752043801095	83001.5309916267\\
0.817620440511013	82967.726481714\\
0.817720443011075	82934.4949295964\\
0.817820445511138	82901.2633774788\\
0.8179204480112	82868.0318253612\\
0.818020450511263	82834.8002732436\\
0.818120453011325	82801.5687211261\\
0.818220455511388	82768.3371690085\\
0.81832045801145	82735.1056168909\\
0.818420460511513	82701.3011069782\\
0.818520463011575	82668.0695548606\\
0.818620465511638	82634.838002743\\
0.8187204680117	82601.6064506254\\
0.818820470511763	82568.3748985078\\
0.818920473011825	82535.1433463902\\
0.819020475511888	82501.9117942726\\
0.81912047801195	82468.680242155\\
0.819220480512013	82435.4486900375\\
0.819320483012075	82402.2171379199\\
0.819420485512138	82368.9855858023\\
0.8195204880122	82335.7540336847\\
0.819620490512263	82302.5224815671\\
0.819720493012325	82269.2909294495\\
0.819820495512388	82236.6323351271\\
0.81992049801245	82203.4007830095\\
0.820020500512513	82170.1692308919\\
0.820120503012575	82136.9376787743\\
0.820220505512638	82103.7061266567\\
0.8203205080127	82070.4745745391\\
0.820420510512763	82037.2430224215\\
0.820520513012825	82004.5844280991\\
0.820620515512888	81971.3528759815\\
0.82072051801295	81938.1213238639\\
0.820820520513013	81904.8897717463\\
0.820920523013075	81871.6582196287\\
0.821020525513138	81838.9996253063\\
0.8211205280132	81805.7680731887\\
0.821220530513263	81772.5365210711\\
0.821320533013325	81739.3049689535\\
0.821420535513388	81706.646374631\\
0.82152053801345	81673.4148225135\\
0.821620540513513	81640.1832703959\\
0.821720543013575	81607.5246760734\\
0.821820545513638	81574.2931239558\\
0.8219205480137	81541.0615718382\\
0.822020550513763	81508.4029775158\\
0.822120553013825	81475.1714253982\\
0.822220555513888	81441.9398732806\\
0.82232055801395	81409.2812789582\\
0.822420560514013	81376.0497268406\\
0.822520563014075	81343.3911325181\\
0.822620565514138	81310.1595804005\\
0.8227205680142	81277.5009860781\\
0.822820570514263	81244.2694339605\\
0.822920573014325	81211.0378818429\\
0.823020575514388	81178.3792875204\\
0.82312057801445	81145.1477354028\\
0.823220580514513	81112.4891410804\\
0.823320583014575	81079.2575889628\\
0.823420585514638	81046.5989946403\\
0.8235205880147	81013.9404003179\\
0.823620590514763	80980.7088482003\\
0.823720593014825	80948.0502538778\\
0.823820595514888	80914.8187017602\\
0.82392059801495	80882.1601074378\\
0.824020600515013	80849.5015131153\\
0.824120603015075	80816.2699609977\\
0.824220605515138	80783.6113666753\\
0.8243206080152	80750.3798145577\\
0.824420610515263	80717.7212202352\\
0.824520613015325	80685.0626259128\\
0.824620615515388	80651.8310737952\\
0.82472061801545	80619.1724794727\\
0.824820620515513	80586.5138851503\\
0.824920623015575	80553.8552908278\\
0.825020625515638	80520.6237387102\\
0.8251206280157	80487.9651443878\\
0.825220630515763	80455.3065500653\\
0.825320633015825	80422.6479557429\\
0.825420635515888	80389.4164036253\\
0.82552063801595	80356.7578093028\\
0.825620640516013	80324.0992149804\\
0.825720643016075	80291.4406206579\\
0.825820645516138	80258.7820263355\\
0.8259206480162	80226.123432013\\
0.826020650516263	80192.8918798954\\
0.826120653016325	80160.2332855729\\
0.826220655516388	80127.5746912505\\
0.82632065801645	80094.916096928\\
0.826420660516513	80062.2575026056\\
0.826520663016575	80029.5989082831\\
0.826620665516638	79996.9403139607\\
0.8267206680167	79964.2817196382\\
0.826820670516763	79931.6231253157\\
0.826920673016825	79898.9645309933\\
0.827020675516888	79866.3059366708\\
0.82712067801695	79833.6473423484\\
0.827220680517013	79800.9887480259\\
0.827320683017075	79768.3301537035\\
0.827420685517138	79735.671559381\\
0.8275206880172	79703.0129650586\\
0.827620690517263	79670.3543707361\\
0.827720693017325	79637.6957764136\\
0.827820695517388	79605.0371820912\\
0.82792069801745	79572.3785877687\\
0.828020700517513	79539.7199934463\\
0.828120703017575	79507.0613991238\\
0.828220705517638	79474.4028048014\\
0.8283207080177	79442.317168274\\
0.828420710517763	79409.6585739516\\
0.828520713017825	79376.9999796291\\
0.828620715517888	79344.3413853067\\
0.82872071801795	79311.6827909842\\
0.828820720518013	79279.5971544569\\
0.828920723018075	79246.9385601344\\
0.829020725518138	79214.279965812\\
0.8291207280182	79181.6213714895\\
0.829220730518263	79149.5357349622\\
0.829320733018325	79116.8771406397\\
0.829420735518388	79084.2185463173\\
0.82952073801845	79051.5599519948\\
0.829620740518513	79019.4743154675\\
0.829720743018575	78986.815721145\\
0.829820745518638	78954.1571268226\\
0.8299207480187	78922.0714902952\\
0.830020750518763	78889.4128959728\\
0.830120753018825	78856.7543016503\\
0.830220755518888	78824.668665123\\
0.83032075801895	78792.0100708006\\
0.830420760519013	78759.9244342732\\
0.830520763019076	78727.2658399508\\
0.830620765519138	78694.6072456283\\
0.8307207680192	78662.521609101\\
0.830820770519263	78629.8630147785\\
0.830920773019325	78597.7773782512\\
0.831020775519388	78565.1187839287\\
0.831120778019451	78533.0331474014\\
0.831220780519513	78500.374553079\\
0.831320783019575	78468.2889165516\\
0.831420785519638	78435.6303222292\\
0.831520788019701	78403.5446857019\\
0.831620790519763	78371.4590491745\\
0.831720793019825	78338.8004548521\\
0.831820795519888	78306.7148183247\\
0.83192079801995	78274.0562240023\\
0.832020800520013	78241.970587475\\
0.832120803020076	78209.8849509476\\
0.832220805520138	78177.2263566252\\
0.8323208080202	78145.1407200979\\
0.832420810520263	78112.4821257754\\
0.832520813020326	78080.3964892481\\
0.832620815520388	78048.3108527207\\
0.832720818020451	78015.6522583983\\
0.832820820520513	77983.566621871\\
0.832920823020575	77951.4809853436\\
0.833020825520638	77919.3953488163\\
0.833120828020701	77886.7367544939\\
0.833220830520763	77854.6511179665\\
0.833320833020825	77822.5654814392\\
0.833420835520888	77790.4798449119\\
0.833520838020951	77758.3942083845\\
0.833620840521013	77725.7356140621\\
0.833720843021076	77693.6499775348\\
0.833820845521138	77661.5643410074\\
0.8339208480212	77629.4787044801\\
0.834020850521263	77597.3930679528\\
0.834120853021326	77565.3074314255\\
0.834220855521388	77532.648837103\\
0.834320858021451	77500.5632005757\\
0.834420860521513	77468.4775640483\\
0.834520863021576	77436.391927521\\
0.834620865521638	77404.3062909937\\
0.834720868021701	77372.2206544664\\
0.834820870521763	77340.135017939\\
0.834920873021825	77308.0493814117\\
0.835020875521888	77275.9637448844\\
0.835120878021951	77243.8781083571\\
0.835220880522013	77211.7924718297\\
0.835320883022076	77179.7068353024\\
0.835420885522138	77147.6211987751\\
0.835520888022201	77115.5355622478\\
0.835620890522263	77083.4499257204\\
0.835720893022326	77051.3642891931\\
0.835820895522388	77019.2786526658\\
0.835920898022451	76987.1930161385\\
0.836020900522513	76955.6803374063\\
0.836120903022576	76923.5947008789\\
0.836220905522638	76891.5090643516\\
0.836320908022701	76859.4234278243\\
0.836420910522763	76827.337791297\\
0.836520913022826	76795.2521547696\\
0.836620915522888	76763.1665182423\\
0.836720918022951	76731.6538395101\\
0.836820920523013	76699.5682029828\\
0.836920923023076	76667.4825664555\\
0.837020925523138	76635.3969299281\\
0.837120928023201	76603.8842511959\\
0.837220930523263	76571.7986146686\\
0.837320933023326	76539.7129781413\\
0.837420935523388	76507.627341614\\
0.837520938023451	76476.1146628818\\
0.837620940523513	76444.0290263544\\
0.837720943023576	76411.9433898271\\
0.837820945523638	76380.4307110949\\
0.837920948023701	76348.3450745676\\
0.838020950523763	76316.2594380403\\
0.838120953023826	76284.7467593081\\
0.838220955523888	76252.6611227807\\
0.838320958023951	76220.5754862534\\
0.838420960524013	76189.0628075212\\
0.838520963024076	76156.9771709939\\
0.838620965524138	76125.4644922617\\
0.838720968024201	76093.3788557344\\
0.838820970524263	76061.8661770022\\
0.838920973024326	76029.7805404749\\
0.839020975524388	75998.2678617427\\
0.839120978024451	75966.1822252153\\
0.839220980524513	75934.6695464831\\
0.839320983024576	75902.5839099558\\
0.839420985524638	75871.0712312236\\
0.839520988024701	75838.9855946963\\
0.839620990524763	75807.4729159641\\
0.839720993024826	75775.3872794368\\
0.839820995524888	75743.8746007046\\
0.839920998024951	75711.7889641772\\
0.840021000525013	75680.276285445\\
0.840121003025076	75648.7636067128\\
0.840221005525138	75616.6779701855\\
0.840321008025201	75585.1652914533\\
0.840421010525263	75553.6526127211\\
0.840521013025326	75521.5669761938\\
0.840621015525388	75490.0542974616\\
0.840721018025451	75458.5416187294\\
0.840821020525513	75426.4559822021\\
0.840921023025576	75394.9433034699\\
0.841021025525638	75363.4306247377\\
0.841121028025701	75331.9179460055\\
0.841221030525763	75299.8323094782\\
0.841321033025826	75268.319630746\\
0.841421035525888	75236.8069520138\\
0.841521038025951	75205.2942732816\\
0.841621040526013	75173.7815945494\\
0.841721043026076	75141.6959580221\\
0.841821045526138	75110.1832792899\\
0.841921048026201	75078.6706005577\\
0.842021050526263	75047.1579218255\\
0.842121053026326	75015.6452430933\\
0.842221055526388	74984.1325643611\\
0.842321058026451	74952.6198856289\\
0.842421060526513	74920.5342491016\\
0.842521063026576	74889.0215703694\\
0.842621065526638	74857.5088916372\\
0.842721068026701	74825.996212905\\
0.842821070526763	74794.4835341728\\
0.842921073026826	74762.9708554406\\
0.843021075526888	74731.4581767084\\
0.843121078026951	74699.9454979762\\
0.843221080527013	74668.432819244\\
0.843321083027076	74636.9201405118\\
0.843421085527138	74605.4074617796\\
0.843521088027201	74573.8947830474\\
0.843621090527263	74542.3821043152\\
0.843721093027326	74510.869425583\\
0.843821095527388	74479.3567468509\\
0.843921098027451	74448.4170259138\\
0.844021100527513	74416.9043471816\\
0.844121103027576	74385.3916684494\\
0.844221105527638	74353.8789897172\\
0.844321108027701	74322.366310985\\
0.844421110527763	74290.8536322528\\
0.844521113027826	74259.3409535206\\
0.844621115527888	74228.4012325835\\
0.844721118027951	74196.8885538513\\
0.844821120528013	74165.3758751192\\
0.844921123028076	74133.863196387\\
0.845021125528138	74102.3505176548\\
0.845121128028201	74071.4107967177\\
0.845221130528263	74039.8981179855\\
0.845321133028326	74008.3854392533\\
0.845421135528388	73976.8727605211\\
0.845521138028451	73945.933039584\\
0.845621140528513	73914.4203608518\\
0.845721143028576	73882.9076821197\\
0.845821145528638	73851.9679611826\\
0.845921148028701	73820.4552824504\\
0.846021150528763	73788.9426037182\\
0.846121153028826	73758.0028827811\\
0.846221155528888	73726.4902040489\\
0.846321158028951	73695.5504831119\\
0.846421160529013	73664.0378043797\\
0.846521163029076	73632.5251256475\\
0.846621165529138	73601.5854047104\\
0.846721168029201	73570.0727259782\\
0.846821170529263	73539.1330050412\\
0.846921173029326	73507.620326309\\
0.847021175529388	73476.6806053719\\
0.847121178029451	73445.1679266397\\
0.847221180529513	73414.2282057026\\
0.847321183029576	73382.7155269705\\
0.847421185529638	73351.7758060334\\
0.847521188029701	73320.2631273012\\
0.847621190529763	73289.3234063641\\
0.847721193029826	73257.8107276319\\
0.847821195529888	73226.8710066949\\
0.847921198029951	73195.9312857578\\
0.848021200530013	73164.4186070256\\
0.848121203030076	73133.4788860885\\
0.848221205530138	73101.9662073563\\
0.848321208030201	73071.0264864193\\
0.848421210530263	73040.0867654822\\
0.848521213030326	73008.57408675\\
0.848621215530388	72977.6343658129\\
0.848721218030451	72946.6946448759\\
0.848821220530513	72915.7549239388\\
0.848921223030576	72884.2422452066\\
0.849021225530638	72853.3025242696\\
0.849121228030701	72822.3628033325\\
0.849221230530763	72790.8501246003\\
0.849321233030826	72759.9104036632\\
0.849421235530888	72728.9706827262\\
0.849521238030951	72698.0309617891\\
0.849621240531013	72667.0912408521\\
0.849721243031076	72635.5785621198\\
0.849821245531138	72604.6388411828\\
0.849921248031201	72573.6991202457\\
0.850021250531263	72542.7593993087\\
0.850121253031326	72511.8196783716\\
0.850221255531388	72480.8799574345\\
0.850321258031451	72449.9402364975\\
0.850421260531513	72419.0005155604\\
0.850521263031576	72387.4878368282\\
0.850621265531638	72356.5481158911\\
0.850721268031701	72325.6083949541\\
0.850821270531763	72294.668674017\\
0.850921273031826	72263.7289530799\\
0.851021275531888	72232.7892321429\\
0.851121278031951	72201.8495112058\\
0.851221280532013	72170.9097902687\\
0.851321283032076	72139.9700693317\\
0.851421285532138	72109.0303483946\\
0.851521288032201	72078.0906274576\\
0.851621290532263	72047.1509065205\\
0.851721293032326	72016.2111855834\\
0.851821295532388	71985.8444224415\\
0.851921298032451	71954.9047015044\\
0.852021300532513	71923.9649805674\\
0.852121303032576	71893.0252596303\\
0.852221305532638	71862.0855386932\\
0.852321308032701	71831.1458177562\\
0.852421310532763	71800.2060968191\\
0.852521313032826	71769.266375882\\
0.852621315532888	71738.8996127401\\
0.852721318032951	71707.959891803\\
0.852821320533013	71677.020170866\\
0.852921323033076	71646.0804499289\\
0.853021325533138	71615.1407289919\\
0.853121328033201	71584.7739658499\\
0.853221330533263	71553.8342449129\\
0.853321333033326	71522.8945239758\\
0.853421335533388	71491.9548030387\\
0.853521338033451	71461.5880398968\\
0.853621340533513	71430.6483189597\\
0.853721343033576	71399.7085980227\\
0.853821345533638	71369.3418348807\\
0.853921348033701	71338.4021139437\\
0.854021350533763	71307.4623930066\\
0.854121353033826	71277.0956298647\\
0.854221355533888	71246.1559089276\\
0.854321358033951	71215.7891457857\\
0.854421360534013	71184.8494248486\\
0.854521363034076	71153.9097039115\\
0.854621365534138	71123.5429407696\\
0.854721368034201	71092.6032198325\\
0.854821370534263	71062.2364566906\\
0.854921373034326	71031.2967357536\\
0.855021375534388	71000.9299726116\\
0.855121378034451	70969.9902516745\\
0.855221380534513	70939.6234885326\\
0.855321383034576	70908.6837675955\\
0.855421385534638	70878.3170044536\\
0.855521388034701	70847.3772835166\\
0.855621390534763	70817.0105203746\\
0.855721393034826	70786.0707994376\\
0.855821395534888	70755.7040362956\\
0.855921398034951	70725.3372731537\\
0.856021400535013	70694.3975522166\\
0.856121403035076	70664.0307890747\\
0.856221405535138	70633.0910681376\\
0.856321408035201	70602.7243049957\\
0.856421410535263	70572.3575418538\\
0.856521413035326	70541.4178209167\\
0.856621415535388	70511.0510577748\\
0.856721418035451	70480.6842946328\\
0.856821420535513	70449.7445736958\\
0.856921423035576	70419.3778105538\\
0.857021425535638	70389.0110474119\\
0.857121428035701	70358.64428427\\
0.857221430535763	70327.7045633329\\
0.857321433035826	70297.337800191\\
0.857421435535888	70266.971037049\\
0.857521438035951	70236.6042739071\\
0.857621440536013	70206.2375107652\\
0.857721443036076	70175.2977898281\\
0.857821445536138	70144.9310266862\\
0.857921448036201	70114.5642635442\\
0.858021450536263	70084.1975004023\\
0.858121453036326	70053.8307372604\\
0.858221455536388	70023.4639741184\\
0.858321458036451	69993.0972109765\\
0.858421460536513	69962.1574900394\\
0.858521463036576	69931.7907268975\\
0.858621465536638	69901.4239637556\\
0.858721468036701	69871.0572006136\\
0.858821470536763	69840.6904374717\\
0.858921473036826	69810.3236743298\\
0.859021475536888	69779.9569111878\\
0.859121478036951	69749.5901480459\\
0.859221480537013	69719.223384904\\
0.859321483037076	69688.856621762\\
0.859421485537138	69658.4898586201\\
0.859521488037201	69628.1230954782\\
0.859621490537263	69597.7563323362\\
0.859721493037326	69567.3895691943\\
0.859821495537388	69537.0228060524\\
0.859921498037451	69506.6560429104\\
0.860021500537513	69476.8622375636\\
0.860121503037576	69446.4954744217\\
0.860221505537638	69416.1287112798\\
0.860321508037701	69385.7619481378\\
0.860421510537763	69355.3951849959\\
0.860521513037826	69325.028421854\\
0.860621515537888	69294.661658712\\
0.860721518037951	69264.8678533652\\
0.860821520538013	69234.5010902233\\
0.860921523038076	69204.1343270814\\
0.861021525538138	69173.7675639394\\
0.861121528038201	69143.9737585926\\
0.861221530538263	69113.6069954507\\
0.861321533038326	69083.2402323087\\
0.861421535538388	69052.8734691668\\
0.861521538038451	69023.07966382\\
0.861621540538513	68992.7129006781\\
0.861721543038576	68962.3461375361\\
0.861821545538638	68932.5523321893\\
0.861921548038701	68902.1855690474\\
0.862021550538764	68871.8188059055\\
0.862121553038826	68842.0250005587\\
0.862221555538888	68811.6582374167\\
0.862321558038951	68781.2914742748\\
0.862421560539013	68751.497668928\\
0.862521563039076	68721.1309057861\\
0.862621565539138	68691.3371004393\\
0.862721568039201	68660.9703372973\\
0.862821570539263	68631.1765319505\\
0.862921573039326	68600.8097688086\\
0.863021575539389	68570.4430056667\\
0.863121578039451	68540.6492003199\\
0.863221580539513	68510.2824371779\\
0.863321583039576	68480.4886318311\\
0.863421585539638	68450.6948264843\\
0.863521588039701	68420.3280633424\\
0.863621590539764	68390.5342579956\\
0.863721593039826	68360.1674948537\\
0.863821595539888	68330.3736895068\\
0.863921598039951	68300.0069263649\\
0.864021600540014	68270.2131210181\\
0.864121603040076	68240.4193156713\\
0.864221605540138	68210.0525525294\\
0.864321608040201	68180.2587471826\\
0.864421610540263	68149.8919840406\\
0.864521613040326	68120.0981786938\\
0.864621615540389	68090.304373347\\
0.864721618040451	68060.5105680002\\
0.864821620540513	68030.1438048583\\
0.864921623040576	68000.3499995115\\
0.865021625540639	67970.5561941647\\
0.865121628040701	67940.1894310228\\
0.865221630540764	67910.395625676\\
0.865321633040826	67880.6018203292\\
0.865421635540888	67850.8080149824\\
0.865521638040951	67820.4412518404\\
0.865621640541014	67790.6474464936\\
0.865721643041076	67760.8536411468\\
0.865821645541138	67731.0598358\\
0.865921648041201	67701.2660304532\\
0.866021650541264	67671.4722251064\\
0.866121653041326	67641.1054619645\\
0.866221655541389	67611.3116566177\\
0.866321658041451	67581.5178512709\\
0.866421660541513	67551.7240459241\\
0.866521663041576	67521.9302405773\\
0.866621665541639	67492.1364352305\\
0.866721668041701	67462.3426298837\\
0.866821670541764	67432.5488245368\\
0.866921673041826	67402.7550191901\\
0.867021675541889	67372.9612138433\\
0.867121678041951	67343.1674084964\\
0.867221680542014	67313.3736031496\\
0.867321683042076	67283.5797978028\\
0.867421685542138	67253.785992456\\
0.867521688042201	67223.9921871092\\
0.867621690542264	67194.1983817624\\
0.867721693042326	67164.4045764156\\
0.867821695542389	67134.6107710688\\
0.867921698042451	67104.816965722\\
0.868021700542514	67075.0231603752\\
0.868121703042576	67045.2293550284\\
0.868221705542639	67015.4355496816\\
0.868321708042701	66986.2147021299\\
0.868421710542764	66956.4208967831\\
0.868521713042826	66926.6270914363\\
0.868621715542889	66896.8332860895\\
0.868721718042951	66867.0394807427\\
0.868821720543014	66837.8186331911\\
0.868921723043076	66808.0248278442\\
0.869021725543139	66778.2310224974\\
0.869121728043201	66748.4372171506\\
0.869221730543264	66718.6434118038\\
0.869321733043326	66689.4225642522\\
0.869421735543389	66659.6287589054\\
0.869521738043451	66629.8349535586\\
0.869621740543514	66600.6141060069\\
0.869721743043576	66570.8203006601\\
0.869821745543639	66541.0264953133\\
0.869921748043701	66511.8056477616\\
0.870021750543764	66482.0118424148\\
0.870121753043826	66452.218037068\\
0.870221755543889	66422.9971895163\\
0.870321758043951	66393.2033841695\\
0.870421760544014	66363.4095788227\\
0.870521763044076	66334.1887312711\\
0.870621765544139	66304.3949259243\\
0.870721768044201	66275.1740783726\\
0.870821770544264	66245.3802730258\\
0.870921773044326	66216.1594254741\\
0.871021775544389	66186.3656201273\\
0.871121778044451	66157.1447725756\\
0.871221780544514	66127.3509672288\\
0.871321783044576	66098.1301196772\\
0.871421785544639	66068.3363143304\\
0.871521788044701	66039.1154667787\\
0.871621790544764	66009.3216614319\\
0.871721793044826	65980.1008138802\\
0.871821795544889	65950.3070085334\\
0.871921798044951	65921.0861609817\\
0.872021800545014	65891.2923556349\\
0.872121803045076	65862.0715080833\\
0.872221805545139	65832.8506605316\\
0.872321808045201	65803.0568551848\\
0.872421810545264	65773.8360076331\\
0.872521813045326	65744.6151600814\\
0.872621815545389	65714.8213547346\\
0.872721818045451	65685.600507183\\
0.872821820545514	65656.3796596313\\
0.872921823045576	65626.5858542845\\
0.873021825545639	65597.3650067328\\
0.873121828045701	65568.1441591812\\
0.873221830545764	65538.9233116295\\
0.873321833045826	65509.1295062827\\
0.873421835545889	65479.908658731\\
0.873521838045951	65450.6878111793\\
0.873621840546014	65421.4669636277\\
0.873721843046076	65392.246116076\\
0.873821845546139	65362.4523107292\\
0.873921848046201	65333.2314631775\\
0.874021850546264	65304.0106156258\\
0.874121853046326	65274.7897680742\\
0.874221855546389	65245.5689205225\\
0.874321858046451	65216.3480729708\\
0.874421860546514	65187.1272254191\\
0.874521863046576	65157.3334200724\\
0.874621865546639	65128.1125725207\\
0.874721868046701	65098.891724969\\
0.874821870546764	65069.6708774173\\
0.874921873046826	65040.4500298657\\
0.875021875546889	65011.229182314\\
0.875121878046951	64982.0083347623\\
0.875221880547014	64952.7874872106\\
0.875321883047076	64923.566639659\\
0.875421885547139	64894.3457921073\\
0.875521888047201	64865.1249445556\\
0.875621890547264	64835.9040970039\\
0.875721893047326	64806.6832494523\\
0.875821895547389	64777.4624019006\\
0.875921898047451	64748.8145121441\\
0.876021900547514	64719.5936645924\\
0.876121903047576	64690.3728170407\\
0.876221905547639	64661.1519694891\\
0.876321908047701	64631.9311219374\\
0.876421910547764	64602.7102743857\\
0.876521913047826	64573.489426834\\
0.876621915547889	64544.2685792824\\
0.876721918047951	64515.6206895258\\
0.876821920548014	64486.3998419742\\
0.876921923048076	64457.1789944225\\
0.877021925548139	64427.9581468708\\
0.877121928048201	64398.7372993191\\
0.877221930548264	64370.0894095626\\
0.877321933048326	64340.8685620109\\
0.877421935548389	64311.6477144592\\
0.877521938048451	64282.9998247027\\
0.877621940548514	64253.778977151\\
0.877721943048576	64224.5581295994\\
0.877821945548639	64195.3372820477\\
0.877921948048701	64166.6893922912\\
0.878021950548764	64137.4685447395\\
0.878121953048826	64108.2476971878\\
0.878221955548889	64079.5998074313\\
0.878321958048951	64050.3789598796\\
0.878421960549014	64021.7310701231\\
0.878521963049076	63992.5102225714\\
0.878621965549139	63963.2893750197\\
0.878721968049201	63934.6414852632\\
0.878821970549264	63905.4206377115\\
0.878921973049326	63876.772747955\\
0.879021975549389	63847.5519004033\\
0.879121978049451	63818.9040106467\\
0.879221980549514	63789.6831630951\\
0.879321983049576	63761.0352733385\\
0.879421985549639	63731.8144257869\\
0.879521988049701	63703.1665360303\\
0.879621990549764	63673.9456884786\\
0.879721993049826	63645.2977987221\\
0.879821995549889	63616.0769511704\\
0.879921998049951	63587.4290614139\\
0.880022000550014	63558.7811716573\\
0.880122003050076	63529.5603241057\\
0.880222005550139	63500.9124343491\\
0.880322008050201	63471.6915867975\\
0.880422010550264	63443.0436970409\\
0.880522013050326	63414.3958072844\\
0.880622015550389	63385.1749597327\\
0.880722018050451	63356.5270699762\\
0.880822020550514	63327.8791802196\\
0.880922023050576	63299.2312904631\\
0.881022025550639	63270.0104429114\\
0.881122028050701	63241.3625531549\\
0.881222030550764	63212.7146633983\\
0.881322033050826	63183.4938158467\\
0.881422035550889	63154.8459260901\\
0.881522038050951	63126.1980363336\\
0.881622040551014	63097.550146577\\
0.881722043051076	63068.9022568205\\
0.881822045551139	63039.6814092688\\
0.881922048051201	63011.0335195123\\
0.882022050551264	62982.3856297557\\
0.882122053051326	62953.7377399992\\
0.882222055551389	62925.0898502427\\
0.882322058051451	62896.4419604861\\
0.882422060551514	62867.7940707296\\
0.882522063051576	62838.5732231779\\
0.882622065551639	62809.9253334214\\
0.882722068051701	62781.2774436648\\
0.882822070551764	62752.6295539083\\
0.882922073051826	62723.9816641517\\
0.883022075551889	62695.3337743952\\
0.883122078051951	62666.6858846387\\
0.883222080552014	62638.0379948821\\
0.883322083052076	62609.3901051256\\
0.883422085552139	62580.742215369\\
0.883522088052201	62552.0943256125\\
0.883622090552264	62523.446435856\\
0.883722093052326	62494.7985460994\\
0.883822095552389	62466.1506563429\\
0.883922098052451	62437.5027665863\\
0.884022100552514	62409.4278346249\\
0.884122103052576	62380.7799448684\\
0.884222105552639	62352.1320551118\\
0.884322108052701	62323.4841653553\\
0.884422110552764	62294.8362755988\\
0.884522113052826	62266.1883858422\\
0.884622115552889	62237.5404960857\\
0.884722118052951	62208.8926063291\\
0.884822120553014	62180.8176743677\\
0.884922123053076	62152.1697846112\\
0.885022125553139	62123.5218948546\\
0.885122128053201	62094.8740050981\\
0.885222130553264	62066.7990731367\\
0.885322133053326	62038.1511833801\\
0.885422135553389	62009.5032936236\\
0.885522138053451	61980.8554038671\\
0.885622140553514	61952.7804719057\\
0.885722143053576	61924.1325821491\\
0.885822145553639	61895.4846923926\\
0.885922148053701	61867.4097604312\\
0.886022150553764	61838.7618706746\\
0.886122153053826	61810.1139809181\\
0.886222155553889	61782.0390489567\\
0.886322158053951	61753.3911592001\\
0.886422160554014	61724.7432694436\\
0.886522163054076	61696.6683374822\\
0.886622165554139	61668.0204477256\\
0.886722168054201	61639.9455157642\\
0.886822170554264	61611.2976260077\\
0.886922173054326	61582.6497362511\\
0.887022175554389	61554.5748042897\\
0.887122178054451	61525.9269145332\\
0.887222180554514	61497.8519825718\\
0.887322183054576	61469.2040928152\\
0.887422185554639	61441.1291608538\\
0.887522188054701	61412.4812710973\\
0.887622190554764	61384.4063391359\\
0.887722193054826	61356.3314071745\\
0.887822195554889	61327.6835174179\\
0.887922198054951	61299.6085854565\\
0.888022200555014	61270.9606957\\
0.888122203055076	61242.8857637386\\
0.888222205555139	61214.237873982\\
0.888322208055201	61186.1629420206\\
0.888422210555264	61158.0880100592\\
0.888522213055326	61129.4401203027\\
0.888622215555389	61101.3651883413\\
0.888722218055451	61073.2902563798\\
0.888822220555514	61044.6423666233\\
0.888922223055576	61016.5674346619\\
0.889022225555639	60988.4925027005\\
0.889122228055701	60959.8446129439\\
0.889222230555764	60931.7696809825\\
0.889322233055826	60903.6947490211\\
0.889422235555889	60875.6198170597\\
0.889522238055951	60847.5448850983\\
0.889622240556014	60818.8969953418\\
0.889722243056076	60790.8220633803\\
0.889822245556139	60762.7471314189\\
0.889922248056201	60734.6721994575\\
0.890022250556264	60706.5972674961\\
0.890122253056326	60677.9493777396\\
0.890222255556389	60649.8744457782\\
0.890322258056451	60621.7995138168\\
0.890422260556514	60593.7245818553\\
0.890522263056576	60565.6496498939\\
0.890622265556639	60537.5747179325\\
0.890722268056701	60509.4997859711\\
0.890822270556764	60481.4248540097\\
0.890922273056826	60453.3499220483\\
0.891022275556889	60425.2749900869\\
0.891122278056951	60397.2000581255\\
0.891222280557014	60369.1251261641\\
0.891322283057076	60341.0501942027\\
0.891422285557139	60312.9752622412\\
0.891522288057201	60284.9003302798\\
0.891622290557264	60256.8253983184\\
0.891722293057326	60228.750466357\\
0.891822295557389	60200.6755343956\\
0.891922298057451	60172.6006024342\\
0.892022300557514	60144.5256704728\\
0.892122303057576	60116.4507385114\\
0.892222305557639	60088.37580655\\
0.892322308057701	60060.3008745885\\
0.892422310557764	60032.2259426271\\
0.892522313057826	60004.1510106657\\
0.892622315557889	59976.6490364994\\
0.892722318057951	59948.574104538\\
0.892822320558014	59920.4991725766\\
0.892922323058077	59892.4242406152\\
0.893022325558139	59864.3493086538\\
0.893122328058201	59836.8473344875\\
0.893222330558264	59808.7724025261\\
0.893322333058326	59780.6974705647\\
0.893422335558389	59752.6225386033\\
0.893522338058451	59724.5476066419\\
0.893622340558514	59697.0456324756\\
0.893722343058576	59668.9707005142\\
0.893822345558639	59640.8957685528\\
0.893922348058702	59613.3937943865\\
0.894022350558764	59585.3188624251\\
0.894122353058826	59557.2439304637\\
0.894222355558889	59529.7419562974\\
0.894322358058951	59501.667024336\\
0.894422360559014	59473.5920923746\\
0.894522363059077	59446.0901182083\\
0.894622365559139	59418.0151862469\\
0.894722368059201	59390.5132120806\\
0.894822370559264	59362.4382801192\\
0.894922373059327	59334.9363059529\\
0.895022375559389	59306.8613739915\\
0.895122378059451	59278.7864420301\\
0.895222380559514	59251.2844678638\\
0.895322383059576	59223.2095359024\\
0.895422385559639	59195.7075617361\\
0.895522388059702	59167.6326297747\\
0.895622390559764	59140.1306556084\\
0.895722393059826	59112.055723647\\
0.895822395559889	59084.5537494808\\
0.895922398059952	59057.0517753145\\
0.896022400560014	59028.9768433531\\
0.896122403060077	59001.4748691868\\
0.896222405560139	58973.3999372254\\
0.896322408060201	58945.8979630591\\
0.896422410560264	58918.3959888928\\
0.896522413060327	58890.3210569314\\
0.896622415560389	58862.8190827651\\
0.896722418060451	58835.3171085988\\
0.896822420560514	58807.2421766374\\
0.896922423060577	58779.7402024712\\
0.897022425560639	58752.2382283049\\
0.897122428060702	58724.1632963435\\
0.897222430560764	58696.6613221772\\
0.897322433060826	58669.1593480109\\
0.897422435560889	58641.6573738446\\
0.897522438060952	58613.5824418832\\
0.897622440561014	58586.0804677169\\
0.897722443061077	58558.5784935507\\
0.897822445561139	58531.0765193844\\
0.897922448061202	58503.001587423\\
0.898022450561264	58475.4996132567\\
0.898122453061327	58447.9976390904\\
0.898222455561389	58420.4956649241\\
0.898322458061452	58392.9936907579\\
0.898422460561514	58365.4917165916\\
0.898522463061577	58337.9897424253\\
0.898622465561639	58309.9148104639\\
0.898722468061702	58282.4128362976\\
0.898822470561764	58254.9108621313\\
0.898922473061827	58227.408887965\\
0.899022475561889	58199.9069137988\\
0.899122478061952	58172.4049396325\\
0.899222480562014	58144.9029654662\\
0.899322483062077	58117.4009912999\\
0.899422485562139	58089.8990171336\\
0.899522488062202	58062.3970429674\\
0.899622490562264	58034.8950688011\\
0.899722493062327	58007.3930946348\\
0.899822495562389	57979.8911204685\\
0.899922498062452	57952.3891463023\\
0.900022500562514	57924.887172136\\
0.900122503062577	57897.3851979697\\
0.900222505562639	57869.8832238034\\
0.900322508062702	57842.9542074323\\
0.900422510562764	57815.452233266\\
0.900522513062827	57787.9502590997\\
0.900622515562889	57760.4482849334\\
0.900722518062952	57732.9463107671\\
0.900822520563014	57705.4443366009\\
0.900922523063077	57677.9423624346\\
0.901022525563139	57651.0133460634\\
0.901122528063202	57623.5113718972\\
0.901222530563264	57596.0093977309\\
0.901322533063327	57568.5074235646\\
0.901422535563389	57541.5784071934\\
0.901522538063452	57514.0764330272\\
0.901622540563514	57486.5744588609\\
0.901722543063577	57459.0724846946\\
0.901822545563639	57432.1434683235\\
0.901922548063702	57404.6414941572\\
0.902022550563764	57377.1395199909\\
0.902122553063827	57350.2105036198\\
0.902222555563889	57322.7085294535\\
0.902322558063952	57295.3784426257\\
0.902422560564014	57268.1056515775\\
0.902522563064077	57240.8328605293\\
0.902622565564139	57213.5027737015\\
0.902722568064202	57186.2299826533\\
0.902822570564264	57159.0144873846\\
0.902922573064327	57131.7416963364\\
0.903022575564389	57104.4689052881\\
0.903122578064452	57077.2534100194\\
0.903222580564514	57049.9806189712\\
0.903322583064577	57022.7651237025\\
0.903422585564639	56995.5496284338\\
0.903522588064702	56968.3341331651\\
0.903622590564764	56941.1186378963\\
0.903722593064827	56913.9031426276\\
0.903822595564889	56886.7449431384\\
0.903922598064952	56859.5294478697\\
0.904022600565014	56832.3712483805\\
0.904122603065077	56805.2130488913\\
0.904222605565139	56778.0548494021\\
0.904322608065202	56750.8966499129\\
0.904422610565264	56723.7384504237\\
0.904522613065327	56696.5802509345\\
0.904622615565389	56669.4220514453\\
0.904722618065452	56642.3211477356\\
0.904822620565514	56615.1629482464\\
0.904922623065577	56588.0620445367\\
0.905022625565639	56560.961140827\\
0.905122628065702	56533.8602371173\\
0.905222630565764	56506.7593334077\\
0.905322633065827	56479.658429698\\
0.905422635565889	56452.6148217678\\
0.905522638065952	56425.5139180581\\
0.905622640566014	56398.4703101279\\
0.905722643066077	56371.3694064183\\
0.905822645566139	56344.3257984881\\
0.905922648066202	56317.2821905579\\
0.906022650566264	56290.2385826277\\
0.906122653066327	56263.2522704771\\
0.906222655566389	56236.2086625469\\
0.906322658066452	56209.1650546167\\
0.906422660566514	56182.1787424661\\
0.906522663066577	56155.1924303154\\
0.906622665566639	56128.2061181647\\
0.906722668066702	56101.1625102346\\
0.906822670566764	56074.2334938634\\
0.906922673066827	56047.2471817127\\
0.907022675566889	56020.2608695621\\
0.907122678066952	55993.3318531909\\
0.907222680567014	55966.3455410403\\
0.907322683067077	55939.4165246691\\
0.907422685567139	55912.487508298\\
0.907522688067202	55885.5584919268\\
0.907622690567264	55858.6294755557\\
0.907722693067327	55831.7004591845\\
0.907822695567389	55804.7714428134\\
0.907922698067452	55777.8997222217\\
0.908022700567514	55750.9707058506\\
0.908122703067577	55724.098985259\\
0.908222705567639	55697.2272646673\\
0.908322708067702	55670.2982482962\\
0.908422710567764	55643.4265277045\\
0.908522713067827	55616.6121028924\\
0.908622715567889	55589.7403823008\\
0.908722718067952	55562.8686617091\\
0.908822720568014	55536.054236897\\
0.908922723068077	55509.2398120849\\
0.909022725568139	55482.3680914933\\
0.909122728068202	55455.5536666811\\
0.909222730568264	55428.739241869\\
0.909322733068327	55401.9248170569\\
0.909422735568389	55375.1676880243\\
0.909522738068452	55348.3532632122\\
0.909622740568514	55321.5961341796\\
0.909722743068577	55294.7817093674\\
0.909822745568639	55268.0245803348\\
0.909922748068702	55241.2674513022\\
0.910022750568764	55214.5103222696\\
0.910122753068827	55187.753193237\\
0.910222755568889	55160.9960642044\\
0.910322758068952	55134.2962309513\\
0.910422760569014	55107.5391019187\\
0.910522763069077	55080.8392686656\\
0.910622765569139	55054.1394354125\\
0.910722768069202	55027.3823063799\\
0.910822770569264	55000.6824731268\\
0.910922773069327	54974.0399356532\\
0.911022775569389	54947.3401024001\\
0.911122778069452	54920.640269147\\
0.911222780569514	54893.9977316734\\
0.911322783069577	54867.2978984203\\
0.911422785569639	54840.6553609467\\
0.911522788069702	54814.0128234732\\
0.911622790569764	54787.3702859996\\
0.911722793069827	54760.727748526\\
0.911822795569889	54734.0852110524\\
0.911922798069952	54707.4999693583\\
0.912022800570014	54680.8574318848\\
0.912122803070077	54654.2721901907\\
0.912222805570139	54627.6296527171\\
0.912322808070202	54601.044411023\\
0.912422810570264	54574.459169329\\
0.912522813070327	54547.8739276349\\
0.912622815570389	54521.2886859408\\
0.912722818070452	54494.7607400263\\
0.912822820570514	54468.1754983322\\
0.912922823070577	54441.6475524176\\
0.913022825570639	54415.0623107236\\
0.913122828070702	54388.534364809\\
0.913222830570764	54362.0064188945\\
0.913322833070827	54335.4784729799\\
0.913422835570889	54309.0078228449\\
0.913522838070952	54282.4798769303\\
0.913622840571014	54255.9519310157\\
0.913722843071077	54229.4812808807\\
0.913822845571139	54203.0106307457\\
0.913922848071202	54176.4826848311\\
0.914022850571264	54150.0120346961\\
0.914122853071327	54123.541384561\\
0.914222855571389	54097.070734426\\
0.914322858071452	54070.6573800704\\
0.914422860571514	54044.1867299354\\
0.914522863071577	54017.7733755799\\
0.914622865571639	53991.3027254448\\
0.914722868071702	53964.8893710893\\
0.914822870571764	53938.4760167338\\
0.914922873071827	53912.0626623782\\
0.915022875571889	53885.6493080227\\
0.915122878071952	53859.2932494467\\
0.915222880572014	53832.8798950911\\
0.915322883072077	53806.4665407356\\
0.915422885572139	53780.1104821596\\
0.915522888072202	53753.7544235836\\
0.915622890572264	53727.3983650076\\
0.915722893072327	53701.0423064315\\
0.915822895572389	53674.6862478555\\
0.915922898072452	53648.3301892795\\
0.916022900572514	53622.031426483\\
0.916122903072577	53595.675367907\\
0.916222905572639	53569.3766051105\\
0.916322908072702	53543.0205465345\\
0.916422910572764	53516.721783738\\
0.916522913072827	53490.4230209414\\
0.916622915572889	53464.1242581449\\
0.916722918072952	53437.882791128\\
0.916822920573014	53411.5840283314\\
0.916922923073077	53385.3425613145\\
0.917022925573139	53359.043798518\\
0.917122928073202	53332.802331501\\
0.917222930573264	53306.560864484\\
0.917322933073327	53280.319397467\\
0.917422935573389	53254.07793045\\
0.917522938073452	53227.836463433\\
0.917622940573514	53201.594996416\\
0.917722943073577	53175.4108251785\\
0.917822945573639	53149.1693581615\\
0.917922948073702	53122.9851869241\\
0.918022950573764	53096.8010156866\\
0.918122953073827	53070.6168444491\\
0.918222955573889	53044.4326732116\\
0.918322958073952	53018.2485019741\\
0.918422960574014	52992.1216265162\\
0.918522963074077	52965.9374552787\\
0.918622965574139	52939.8105798207\\
0.918722968074202	52913.6264085832\\
0.918822970574264	52887.4995331253\\
0.918922973074327	52861.3726576673\\
0.919022975574389	52835.2457822094\\
0.919122978074452	52809.1189067514\\
0.919222980574514	52783.0493270729\\
0.919322983074577	52756.922451615\\
0.919422985574639	52730.795576157\\
0.919522988074702	52704.7259964785\\
0.919622990574764	52678.6564168001\\
0.919722993074827	52652.5868371216\\
0.919822995574889	52626.5172574432\\
0.919922998074952	52600.4476777647\\
0.920023000575014	52574.3780980863\\
0.920123003075077	52548.3658141873\\
0.920223005575139	52522.2962345089\\
0.920323008075202	52496.28395061\\
0.920423010575264	52470.271666711\\
0.920523013075327	52444.2593828121\\
0.920623015575389	52418.2470989131\\
0.920723018075452	52392.2348150142\\
0.920823020575514	52366.2225311153\\
0.920923023075577	52340.2102472163\\
0.921023025575639	52314.2552590969\\
0.921123028075702	52288.3002709775\\
0.921223030575764	52262.2879870785\\
0.921323033075827	52236.3329989591\\
0.921423035575889	52210.3780108397\\
0.921523038075952	52184.4230227202\\
0.921623040576014	52158.5253303803\\
0.921723043076077	52132.5703422609\\
0.921823045576139	52106.6153541415\\
0.921923048076202	52080.7176618016\\
0.922023050576264	52054.8199694617\\
0.922123053076327	52028.9222771217\\
0.922223055576389	52003.0245847818\\
0.922323058076452	51977.1268924419\\
0.922423060576514	51951.229200102\\
0.922523063076577	51925.3315077621\\
0.922623065576639	51899.4911112017\\
0.922723068076702	51873.5934188618\\
0.922823070576764	51847.7530223014\\
0.922923073076827	51821.912625741\\
0.923023075576889	51796.0722291806\\
0.923123078076952	51770.2318326202\\
0.923223080577014	51744.3914360598\\
0.923323083077077	51718.5510394994\\
0.923423085577139	51692.7679387185\\
0.923523088077202	51666.9275421581\\
0.923623090577264	51641.1444413772\\
0.923723093077327	51615.3613405963\\
0.923823095577389	51589.5782398154\\
0.923923098077452	51563.7951390345\\
0.924023100577514	51538.0120382537\\
0.924123103077577	51512.2289374728\\
0.924223105577639	51486.5031324714\\
0.924323108077702	51460.7200316905\\
0.924423110577764	51434.9942266891\\
0.924523113077827	51409.2684216878\\
0.924623115577889	51383.5426166864\\
0.924723118077952	51357.816811685\\
0.924823120578014	51332.0910066836\\
0.924923123078077	51306.3652016823\\
0.925023125578139	51280.6966924604\\
0.925123128078202	51254.970887459\\
0.925223130578264	51229.3023782372\\
0.925323133078327	51203.5765732358\\
0.92542313557839	51177.9080640139\\
0.925523138078452	51152.2395547921\\
0.925623140578514	51126.5710455702\\
0.925723143078577	51100.9598321279\\
0.925823145578639	51075.291322906\\
0.925923148078702	51049.6801094636\\
0.926023150578764	51024.0116002418\\
0.926123153078827	50998.4003867994\\
0.926223155578889	50972.7891733571\\
0.926323158078952	50947.1779599147\\
0.926423160579015	50921.5667464724\\
0.926523163079077	50895.9555330301\\
0.926623165579139	50870.4016153672\\
0.926723168079202	50844.7904019249\\
0.926823170579264	50819.236484262\\
0.926923173079327	50793.6252708197\\
0.92702317557939	50768.0713531569\\
0.927123178079452	50742.517435494\\
0.927223180579514	50716.9635178312\\
0.927323183079577	50691.4096001684\\
0.92742318557964	50665.912978285\\
0.927523188079702	50640.3590606222\\
0.927623190579765	50614.8624387389\\
0.927723193079827	50589.308521076\\
0.927823195579889	50563.8118991927\\
0.927923198079952	50538.3152773094\\
0.928023200580015	50512.8186554261\\
0.928123203080077	50487.3220335428\\
0.928223205580139	50461.8827074389\\
0.928323208080202	50436.3860855556\\
0.928423210580265	50410.9467594518\\
0.928523213080327	50385.4501375685\\
0.92862321558039	50360.0108114647\\
0.928723218080452	50334.5714853609\\
0.928823220580514	50309.1321592571\\
0.928923223080577	50283.6928331533\\
0.92902322558064	50258.310802829\\
0.929123228080702	50232.8714767252\\
0.929223230580765	50207.4894464009\\
0.929323233080827	50182.050120297\\
0.92942323558089	50156.6680899728\\
0.929523238080952	50131.2860596485\\
0.929623240581015	50105.9040293242\\
0.929723243081077	50080.5219989999\\
0.929823245581139	50055.1399686756\\
0.929923248081202	50029.8152341308\\
0.930023250581265	50004.4332038065\\
0.930123253081327	49979.1084692617\\
0.93022325558139	49953.7264389374\\
0.930323258081452	49928.4017043926\\
0.930423260581515	49903.0769698478\\
0.930523263081577	49877.7522353031\\
0.93062326558164	49852.4847965378\\
0.930723268081702	49827.160061993\\
0.930823270581765	49801.8353274482\\
0.930923273081827	49776.567888683\\
0.93102327558189	49751.3004499177\\
0.931123278081952	49726.0330111524\\
0.931223280582015	49700.7655723872\\
0.931323283082077	49675.4981336219\\
0.93142328558214	49650.2306948566\\
0.931523288082202	49624.9632560914\\
0.931623290582265	49599.7531131056\\
0.931723293082327	49574.4856743403\\
0.93182329558239	49549.2755313546\\
0.931923298082452	49524.0653883688\\
0.932023300582515	49498.855245383\\
0.932123303082577	49473.6451023973\\
0.93222330558264	49448.4349594115\\
0.932323308082702	49423.2248164258\\
0.932423310582765	49398.0719692195\\
0.932523313082827	49372.8618262338\\
0.93262331558289	49347.7089790275\\
0.932723318082952	49322.5561318213\\
0.932823320583015	49297.4032846151\\
0.932923323083077	49272.2504374088\\
0.93302332558314	49247.0975902026\\
0.933123328083202	49221.9447429963\\
0.933223330583265	49196.7918957901\\
0.933323333083327	49171.6963443634\\
0.93342333558339	49146.6007929366\\
0.933523338083452	49121.4479457304\\
0.933623340583515	49096.3523943036\\
0.933723343083577	49071.2568428769\\
0.93382334558364	49046.1612914502\\
0.933923348083702	49021.123035803\\
0.934023350583765	48996.0274843762\\
0.934123353083827	48970.9319329495\\
0.93422335558389	48945.8936773023\\
0.934323358083952	48920.8554216551\\
0.934423360584015	48895.8171660079\\
0.934523363084077	48870.7789103606\\
0.93462336558414	48845.7406547134\\
0.934723368084202	48820.7023990662\\
0.934823370584265	48795.664143419\\
0.934923373084327	48770.6831835513\\
0.93502337558439	48745.6449279041\\
0.935123378084452	48720.6639680364\\
0.935223380584515	48695.6830081687\\
0.935323383084577	48670.702048301\\
0.93542338558464	48645.7210884333\\
0.935523388084702	48620.7401285656\\
0.935623390584765	48595.7591686978\\
0.935723393084827	48570.8355046097\\
0.93582339558489	48545.854544742\\
0.935923398084952	48520.9308806538\\
0.936023400585015	48496.0072165656\\
0.936123403085077	48471.0835524774\\
0.93622340558514	48446.1598883892\\
0.936323408085202	48421.236224301\\
0.936423410585265	48396.3125602128\\
0.936523413085327	48371.4461919041\\
0.93662341558539	48346.5225278159\\
0.936723418085452	48321.6561595073\\
0.936823420585515	48296.7897911986\\
0.936923423085577	48271.9234228899\\
0.93702342558564	48247.0570545812\\
0.937123428085702	48222.1906862725\\
0.937223430585765	48197.3243179639\\
0.937323433085827	48172.4579496552\\
0.93742343558589	48147.648877126\\
0.937523438085952	48122.7825088174\\
0.937623440586015	48097.9734362882\\
0.937723443086077	48073.164363759\\
0.93782344558614	48048.3552912299\\
0.937923448086202	48023.5462187007\\
0.938023450586265	47998.7371461715\\
0.938123453086327	47973.9853694219\\
0.93822345558639	47949.1762968927\\
0.938323458086452	47924.4245201431\\
0.938423460586515	47899.6727433934\\
0.938523463086577	47874.8636708642\\
0.93862346558664	47850.1118941146\\
0.938723468086702	47825.360117365\\
0.938823470586765	47800.6656363948\\
0.938923473086827	47775.9138596452\\
0.93902347558689	47751.1620828955\\
0.939123478086952	47726.4676019254\\
0.939223480587015	47701.7731209552\\
0.939323483087077	47677.0213442056\\
0.93942348558714	47652.3268632354\\
0.939523488087202	47627.6323822653\\
0.939623490587265	47602.9951970747\\
0.939723493087327	47578.3007161045\\
0.93982349558739	47553.6062351344\\
0.939923498087452	47528.9690499438\\
0.940023500587515	47504.2745689736\\
0.940123503087577	47479.637383783\\
0.94022350558764	47455.0001985924\\
0.940323508087702	47430.3630134018\\
0.940423510587765	47405.7258282111\\
0.940523513087827	47381.1459388\\
0.94062351558789	47356.5087536094\\
0.940723518087952	47331.8715684188\\
0.940823520588015	47307.2916790077\\
0.940923523088077	47282.7117895965\\
0.94102352558814	47258.1319001854\\
0.941123528088202	47233.5520107743\\
0.941223530588265	47208.9721213632\\
0.941323533088327	47184.3922319521\\
0.94142353558839	47159.812342541\\
0.941523538088452	47135.2897489094\\
0.941623540588515	47110.7098594983\\
0.941723543088577	47086.1872658667\\
0.94182354558864	47061.6646722351\\
0.941923548088702	47037.1420786035\\
0.942023550588765	47012.6194849719\\
0.942123553088827	46988.0968913403\\
0.94222355558889	46963.6315934882\\
0.942323558088952	46939.1089998566\\
0.942423560589015	46914.6437020045\\
0.942523563089077	46890.1211083729\\
0.94262356558914	46865.6558105208\\
0.942723568089202	46841.1905126687\\
0.942823570589265	46816.7252148166\\
0.942923573089327	46792.2599169646\\
0.94302357558939	46767.851914892\\
0.943123578089452	46743.3866170399\\
0.943223580589515	46718.9213191878\\
0.943323583089577	46694.5133171152\\
0.94342358558964	46670.1053150427\\
0.943523588089702	46645.6973129701\\
0.943623590589765	46621.2893108975\\
0.943723593089827	46596.881308825\\
0.94382359558989	46572.4733067524\\
0.943923598089952	46548.1226004593\\
0.944023600590015	46523.7145983867\\
0.944123603090077	46499.3638920937\\
0.94422360559014	46474.9558900211\\
0.944323608090202	46450.605183728\\
0.944423610590265	46426.254477435\\
0.944523613090327	46401.9037711419\\
0.94462361559039	46377.6103606284\\
0.944723618090452	46353.2596543353\\
0.944823620590515	46328.9089480423\\
0.944923623090577	46304.6155375287\\
0.94502362559064	46280.3221270152\\
0.945123628090702	46256.0287165016\\
0.945223630590765	46231.7353059881\\
0.945323633090827	46207.4418954745\\
0.94542363559089	46183.148484961\\
0.945523638090952	46158.8550744474\\
0.945623640591015	46134.6189597134\\
0.945723643091077	46110.3255491999\\
0.94582364559114	46086.0894344658\\
0.945923648091202	46061.8533197318\\
0.946023650591265	46037.6172049978\\
0.946123653091327	46013.3810902637\\
0.94622365559139	45989.1449755297\\
0.946323658091452	45964.9088607957\\
0.946423660591515	45940.6727460616\\
0.946523663091577	45916.4939271071\\
0.94662366559164	45892.3151081526\\
0.946723668091702	45868.0789934185\\
0.946823670591765	45843.900174464\\
0.946923673091827	45819.7213555095\\
0.94702367559189	45795.542536555\\
0.947123678091952	45771.42101338\\
0.947223680592015	45747.2421944255\\
0.947323683092077	45723.0633754709\\
0.94742368559214	45698.9418522959\\
0.947523688092202	45674.8203291209\\
0.947623690592265	45650.6988059459\\
0.947723693092327	45626.5772827709\\
0.94782369559239	45602.4557595959\\
0.947923698092452	45578.3342364209\\
0.948023700592515	45554.2127132459\\
0.948123703092577	45530.1484858504\\
0.94822370559264	45506.0269626754\\
0.948323708092702	45481.9627352799\\
0.948423710592765	45457.8985078844\\
0.948523713092827	45433.8342804889\\
0.94862371559289	45409.7700530934\\
0.948723718092952	45385.7058256979\\
0.948823720593015	45361.6415983024\\
0.948923723093077	45337.6346666864\\
0.94902372559314	45313.5704392909\\
0.949123728093202	45289.5635076749\\
0.949223730593265	45265.556576059\\
0.949323733093327	45241.549644443\\
0.94942373559339	45217.542712827\\
0.949523738093452	45193.535781211\\
0.949623740593515	45169.528849595\\
0.949723743093577	45145.5219179791\\
0.94982374559364	45121.5722821426\\
0.949923748093702	45097.6226463061\\
0.950023750593765	45073.6157146901\\
0.950123753093827	45049.6660788537\\
0.95022375559389	45025.7164430172\\
0.950323758093952	45001.7668071807\\
0.950423760594015	44977.8744671238\\
0.950523763094077	44953.9248312873\\
0.95062376559414	44929.9751954509\\
0.950723768094202	44906.0828553939\\
0.950823770594265	44882.1905153369\\
0.950923773094327	44858.29817528\\
0.95102377559439	44834.3485394435\\
0.951123778094452	44810.5134951661\\
0.951223780594515	44786.6211551091\\
0.951323783094577	44762.7288150521\\
0.95142378559464	44738.8364749952\\
0.951523788094702	44715.0014307178\\
0.951623790594765	44691.1663864403\\
0.951723793094827	44667.3313421629\\
0.95182379559489	44643.4390021059\\
0.951923798094952	44619.6039578285\\
0.952023800595015	44595.8262093306\\
0.952123803095077	44571.9911650531\\
0.95222380559514	44548.1561207757\\
0.952323808095202	44524.3783722777\\
0.952423810595265	44500.5433280003\\
0.952523813095327	44476.7655795024\\
0.95262381559539	44452.9878310044\\
0.952723818095452	44429.2100825065\\
0.952823820595515	44405.4323340086\\
0.952923823095577	44381.7118812902\\
0.95302382559564	44357.9341327922\\
0.953123828095702	44334.1563842943\\
0.953223830595765	44310.4359315759\\
0.953323833095827	44286.7154788575\\
0.95342383559589	44262.995026139\\
0.953523838095952	44239.2172776411\\
0.953623840596015	44215.5541207022\\
0.953723843096077	44191.8336679838\\
0.95382384559614	44168.1132152654\\
0.953923848096202	44144.4500583265\\
0.954023850596265	44120.7296056081\\
0.954123853096327	44097.0664486692\\
0.95422385559639	44073.4032917303\\
0.954323858096452	44049.7401347914\\
0.954423860596515	44026.0769778525\\
0.954523863096577	44002.4138209136\\
0.95462386559664	43978.7506639746\\
0.954723868096702	43955.0875070357\\
0.954823870596765	43931.4816458764\\
0.954923873096827	43907.875784717\\
0.95502387559689	43884.2126277781\\
0.955123878096952	43860.6067666187\\
0.955223880597015	43837.0009054593\\
0.955323883097077	43813.3950442999\\
0.95542388559714	43789.84647892\\
0.955523888097202	43766.2406177606\\
0.955623890597265	43742.6920523807\\
0.955723893097327	43719.0861912214\\
0.95582389559739	43695.5376258415\\
0.955923898097452	43671.9890604616\\
0.956023900597515	43648.4404950817\\
0.956123903097577	43624.8919297019\\
0.95622390559764	43601.343364322\\
0.956323908097702	43577.7947989421\\
0.956423910597765	43554.3035293417\\
0.956523913097827	43530.7549639619\\
0.95662391559789	43507.2636943615\\
0.956723918097952	43483.7724247611\\
0.956823920598015	43460.2811551608\\
0.956923923098078	43436.7898855604\\
0.95702392559814	43413.29861596\\
0.957123928098202	43389.8073463597\\
0.957223930598265	43366.3733725388\\
0.957323933098327	43342.8821029385\\
0.95742393559839	43319.4481291176\\
0.957523938098452	43296.0141552968\\
0.957623940598515	43272.5801814759\\
0.957723943098577	43249.1462076551\\
0.95782394559864	43225.7122338342\\
0.957923948098703	43202.2782600134\\
0.958023950598765	43178.8442861925\\
0.958123953098827	43155.4676081512\\
0.95822395559889	43132.0909301098\\
0.958323958098952	43108.656956289\\
0.958423960599015	43085.2802782476\\
0.958523963099078	43061.9036002063\\
0.95862396559914	43038.526922165\\
0.958723968099202	43015.1502441236\\
0.958823970599265	42991.8308618618\\
0.958923973099328	42968.4541838205\\
0.95902397559939	42945.1348015586\\
0.959123978099452	42921.8154192968\\
0.959223980599515	42898.4387412555\\
0.959323983099577	42875.1193589937\\
0.95942398559964	42851.7999767318\\
0.959523988099703	42828.5378902495\\
0.959623990599765	42805.2185079877\\
0.959723993099827	42781.8991257259\\
0.95982399559989	42758.6370392436\\
0.959923998099953	42735.3176569817\\
0.960024000600015	42712.0555704994\\
0.960124003100078	42688.7934840171\\
0.96022400560014	42665.5313975348\\
0.960324008100202	42642.2693110525\\
0.960424010600265	42619.0072245702\\
0.960524013100328	42595.8024338674\\
0.96062401560039	42572.5403473851\\
0.960724018100452	42549.3355566823\\
0.960824020600515	42526.1307659795\\
0.960924023100578	42502.9259752767\\
0.96102402560064	42479.7211845739\\
0.961124028100703	42456.5163938711\\
0.961224030600765	42433.3116031683\\
0.961324033100827	42410.1068124655\\
0.96142403560089	42386.9593175422\\
0.961524038100953	42363.7545268394\\
0.961624040601015	42340.6070319161\\
0.961724043101078	42317.4595369928\\
0.96182404560114	42294.3120420695\\
0.961924048101203	42271.1645471463\\
0.962024050601265	42248.017052223\\
0.962124053101328	42224.8695572997\\
0.96222405560139	42201.7793581559\\
0.962324058101452	42178.6318632326\\
0.962424060601515	42155.5416640889\\
0.962524063101578	42132.4514649451\\
0.96262406560164	42109.3039700218\\
0.962724068101703	42086.213770878\\
0.962824070601765	42063.1808675138\\
0.962924073101828	42040.09066837\\
0.96302407560189	42017.0004692262\\
0.963124078101953	41993.967565862\\
0.963224080602015	41970.8773667182\\
0.963324083102078	41947.8444633539\\
0.96342408560214	41924.8115599897\\
0.963524088102203	41901.7786566254\\
0.963624090602265	41878.7457532612\\
0.963724093102328	41855.7128498969\\
0.96382409560239	41832.6799465326\\
0.963924098102453	41809.7043389479\\
0.964024100602515	41786.6714355836\\
0.964124103102578	41763.6958279989\\
0.96422410560264	41740.7202204141\\
0.964324108102703	41717.7446128294\\
0.964424110602765	41694.7690052446\\
0.964524113102828	41671.7933976599\\
0.96462411560289	41648.8177900752\\
0.964724118102953	41625.8994782699\\
0.964824120603015	41602.9238706852\\
0.964924123103078	41580.0055588799\\
0.96502412560314	41557.0872470747\\
0.965124128103203	41534.1689352695\\
0.965224130603265	41511.2506234642\\
0.965324133103328	41488.332311659\\
0.96542413560339	41465.4139998538\\
0.965524138103453	41442.4956880485\\
0.965624140603515	41419.6346720228\\
0.965724143103578	41396.7163602176\\
0.96582414560364	41373.8553441919\\
0.965924148103703	41350.9943281662\\
0.966024150603765	41328.1333121404\\
0.966124153103828	41305.2722961147\\
0.96622415560389	41282.411280089\\
0.966324158103953	41259.6075598428\\
0.966424160604015	41236.7465438171\\
0.966524163104078	41213.8855277913\\
0.96662416560414	41191.0818075451\\
0.966724168104203	41168.2780872989\\
0.966824170604265	41145.4743670527\\
0.966924173104328	41122.6706468065\\
0.96702417560439	41099.8669265603\\
0.967124178104453	41077.0632063141\\
0.967224180604515	41054.3167818474\\
0.967324183104578	41031.5130616012\\
0.96742418560464	41008.7666371345\\
0.967524188104703	40986.0202126678\\
0.967624190604765	40963.2164924216\\
0.967724193104828	40940.4700679549\\
0.96782419560489	40917.7809392677\\
0.967924198104953	40895.0345148011\\
0.968024200605015	40872.2880903344\\
0.968124203105078	40849.5989616472\\
0.96822420560514	40826.8525371805\\
0.968324208105203	40804.1634084933\\
0.968424210605265	40781.4742798061\\
0.968524213105328	40758.7851511189\\
0.96862421560539	40736.0960224318\\
0.968724218105453	40713.4068937446\\
0.968824220605515	40690.7177650574\\
0.968924223105578	40668.0859321497\\
0.96902422560564	40645.3968034626\\
0.969124228105703	40622.7649705549\\
0.969224230605765	40600.1331376472\\
0.969324233105828	40577.5013047395\\
0.96942423560589	40554.8694718319\\
0.969524238105953	40532.2376389242\\
0.969624240606015	40509.6058060165\\
0.969724243106078	40486.9739731089\\
0.96982424560614	40464.3994359807\\
0.969924248106203	40441.7676030731\\
0.970024250606265	40419.1930659449\\
0.970124253106328	40396.6185288167\\
0.97022425560639	40374.0439916886\\
0.970324258106453	40351.4694545604\\
0.970424260606515	40328.8949174323\\
0.970524263106578	40306.3776760836\\
0.97062426560664	40283.8031389555\\
0.970724268106703	40261.2858976068\\
0.970824270606765	40238.7113604787\\
0.970924273106828	40216.19411913\\
0.97102427560689	40193.6768777814\\
0.971124278106953	40171.1596364328\\
0.971224280607015	40148.6423950841\\
0.971324283107078	40126.1251537355\\
0.97142428560714	40103.6652081664\\
0.971524288107203	40081.1479668177\\
0.971624290607265	40058.6880212486\\
0.971724293107328	40036.2280756795\\
0.97182429560739	40013.7681301103\\
0.971924298107453	39991.3081845412\\
0.972024300607515	39968.8482389721\\
0.972124303107578	39946.3882934029\\
0.97222430560764	39923.9283478338\\
0.972324308107703	39901.5256980442\\
0.972424310607765	39879.0657524751\\
0.972524313107828	39856.6631026855\\
0.97262431560789	39834.2604528958\\
0.972724318107953	39811.8578031062\\
0.972824320608015	39789.4551533166\\
0.972924323108078	39767.052503527\\
0.97302432560814	39744.6498537374\\
0.973124328108203	39722.3044997273\\
0.973224330608265	39699.9018499377\\
0.973324333108328	39677.5564959276\\
0.97342433560839	39655.2111419175\\
0.973524338108453	39632.8657879074\\
0.973624340608515	39610.5204338973\\
0.973724343108578	39588.1750798872\\
0.97382434560864	39565.8297258771\\
0.973924348108703	39543.5416676465\\
0.974024350608765	39521.1963136364\\
0.974124353108828	39498.9082554058\\
0.97422435560889	39476.5629013957\\
0.974324358108953	39454.2748431651\\
0.974424360609015	39431.9867849345\\
0.974524363109078	39409.6987267039\\
0.97462436560914	39387.4679642528\\
0.974724368109203	39365.1799060222\\
0.974824370609265	39342.8918477916\\
0.974924373109328	39320.6610853406\\
0.97502437560939	39298.4303228895\\
0.975124378109453	39276.1422646589\\
0.975224380609515	39253.9115022078\\
0.975324383109578	39231.6807397568\\
0.97542438560964	39209.4499773057\\
0.975524388109703	39187.2765106341\\
0.975624390609765	39165.045748183\\
0.975724393109828	39142.8722815115\\
0.97582439560989	39120.6415190604\\
0.975924398109953	39098.4680523888\\
0.976024400610015	39076.2945857173\\
0.976124403110078	39054.1211190457\\
0.97622440561014	39031.9476523741\\
0.976324408110203	39009.7741857026\\
0.976424410610265	38987.6580148105\\
0.976524413110328	38965.484548139\\
0.97662441561039	38943.3683772469\\
0.976724418110453	38921.1949105754\\
0.976824420610515	38899.0787396833\\
0.976924423110578	38876.9625687913\\
0.97702442561064	38854.8463978992\\
0.977124428110703	38832.7302270072\\
0.977224430610765	38810.6713518946\\
0.977324433110828	38788.5551810026\\
0.97742443561089	38766.49630589\\
0.977524438110953	38744.380134998\\
0.977624440611015	38722.3212598855\\
0.977724443111078	38700.2623847729\\
0.97782444561114	38678.2035096604\\
0.977924448111203	38656.1446345478\\
0.978024450611265	38634.0857594353\\
0.978124453111328	38612.0841801023\\
0.97822445561139	38590.0253049897\\
0.978324458111453	38568.0237256567\\
0.978424460611515	38545.9648505442\\
0.978524463111578	38523.9632712112\\
0.97862446561164	38501.9616918781\\
0.978724468111703	38479.9601125451\\
0.978824470611765	38457.9585332121\\
0.978924473111828	38436.0142496586\\
0.97902447561189	38414.0126703256\\
0.979124478111953	38392.068386772\\
0.979224480612015	38370.066807439\\
0.979324483112078	38348.1225238855\\
0.97942448561214	38326.178240332\\
0.979524488112203	38304.2339567785\\
0.979624490612265	38282.289673225\\
0.979724493112328	38260.402685451\\
0.97982449561239	38238.4584018975\\
0.979924498112453	38216.514118344\\
0.980024500612515	38194.62713057\\
0.980124503112578	38172.740142796\\
0.98022450561264	38150.853155022\\
0.980324508112703	38128.966167248\\
0.980424510612765	38107.079179474\\
0.980524513112828	38085.1921917\\
0.98062451561289	38063.305203926\\
0.980724518112953	38041.4755119315\\
0.980824520613015	38019.5885241575\\
0.980924523113078	37997.758832163\\
0.98102452561314	37975.9291401685\\
0.981124528113203	37954.099448174\\
0.981224530613265	37932.2697561796\\
0.981324533113328	37910.4400641851\\
0.98142453561339	37888.6103721906\\
0.981524538113453	37866.7806801961\\
0.981624540613515	37845.0082839811\\
0.981724543113578	37823.2358877662\\
0.98182454561364	37801.4061957717\\
0.981924548113703	37779.6337995567\\
0.982024550613765	37757.8614033417\\
0.982124553113828	37736.0890071268\\
0.98222455561389	37714.3739066913\\
0.982324558113953	37692.6015104763\\
0.982424560614015	37670.8291142614\\
0.982524563114078	37649.1140138259\\
0.98262456561414	37627.3989133904\\
0.982724568114203	37605.6265171755\\
0.982824570614265	37583.91141674\\
0.982924573114328	37562.1963163046\\
0.98302457561439	37540.5385116486\\
0.983124578114453	37518.8234112132\\
0.983224580614515	37497.1083107777\\
0.983324583114578	37475.4505061218\\
0.98342458561464	37453.7354056863\\
0.983524588114703	37432.0776010304\\
0.983624590614765	37410.4197963744\\
0.983724593114828	37388.7619917185\\
0.98382459561489	37367.1041870625\\
0.983924598114953	37345.4463824066\\
0.984024600615015	37323.8458735301\\
0.984124603115078	37302.1880688742\\
0.98422460561514	37280.5875599978\\
0.984324608115203	37258.9297553418\\
0.984424610615265	37237.3292464654\\
0.984524613115328	37215.728737589\\
0.98462461561539	37194.1282287125\\
0.984724618115453	37172.5277198361\\
0.984824620615515	37150.9845067392\\
0.984924623115578	37129.3839978627\\
0.98502462561564	37107.8407847658\\
0.985124628115703	37086.2402758894\\
0.985224630615765	37064.6970627925\\
0.985324633115828	37043.1538496955\\
0.98542463561589	37021.6106365986\\
0.985524638115953	37000.0674235017\\
0.985624640616015	36978.5242104048\\
0.985724643116078	36957.0382930874\\
0.98582464561614	36935.4950799905\\
0.985924648116203	36914.0091626731\\
0.986024650616265	36892.4659495761\\
0.986124653116328	36870.9800322587\\
0.98622465561639	36849.4941149413\\
0.986324658116453	36828.0081976239\\
0.986424660616515	36806.5222803065\\
0.986524663116578	36785.0936587686\\
0.98662466561664	36763.6077414512\\
0.986724668116703	36742.1791199133\\
0.986824670616765	36720.6932025959\\
0.986924673116828	36699.264581058\\
0.98702467561689	36677.8359595201\\
0.987124678116953	36656.4073379822\\
0.987224680617015	36634.9787164443\\
0.987324683117078	36613.5500949065\\
0.98742468561714	36592.1787691481\\
0.987524688117203	36570.7501476102\\
0.987624690617265	36549.3788218518\\
0.987724693117328	36528.0074960934\\
0.98782469561739	36506.5788745555\\
0.987924698117453	36485.2075487972\\
0.988024700617515	36463.8362230388\\
0.988124703117578	36442.5221930599\\
0.98822470561764	36421.1508673015\\
0.988324708117703	36399.7795415432\\
0.988424710617765	36378.4655115643\\
0.988524713117828	36357.0941858059\\
0.98862471561789	36335.780155827\\
0.988724718117953	36314.4661258482\\
0.988824720618015	36293.1520958693\\
0.988924723118078	36271.8380658904\\
0.98902472561814	36250.5813316911\\
0.989124728118203	36229.2673017122\\
0.989224730618265	36207.9532717333\\
0.989324733118328	36186.696537534\\
0.989424735618391	36165.4398033346\\
0.989524738118453	36144.1830691353\\
0.989624740618515	36122.8690391564\\
0.989724743118578	36101.6696007366\\
0.98982474561864	36080.4128665372\\
0.989924748118703	36059.1561323379\\
0.990024750618765	36037.8993981385\\
0.990124753118828	36016.6999597187\\
0.99022475561889	35995.5005212988\\
0.990324758118953	35974.2437870995\\
0.990424760619016	35953.0443486797\\
0.990524763119078	35931.8449102598\\
0.99062476561914	35910.64547184\\
0.990724768119203	35889.5033291996\\
0.990824770619265	35868.3038907798\\
0.990924773119328	35847.1617481395\\
0.991024775619391	35825.9623097196\\
0.991124778119453	35804.8201670793\\
0.991224780619515	35783.678024439\\
0.991324783119578	35762.5358817986\\
0.991424785619641	35741.3937391583\\
0.991524788119703	35720.251596518\\
0.991624790619765	35699.1094538777\\
0.991724793119828	35678.0246070169\\
0.99182479561989	35656.8824643765\\
0.991924798119953	35635.7976175157\\
0.992024800620016	35614.7127706549\\
0.992124803120078	35593.6279237941\\
0.99222480562014	35572.5430769333\\
0.992324808120203	35551.4582300725\\
0.992424810620266	35530.3733832116\\
0.992524813120328	35509.2885363508\\
0.992624815620391	35488.2609852695\\
0.992724818120453	35467.1761384087\\
0.992824820620515	35446.1485873274\\
0.992924823120578	35425.1210362461\\
0.993024825620641	35404.0934851648\\
0.993124828120703	35383.0659340835\\
0.993224830620765	35362.0383830022\\
0.993324833120828	35341.0108319209\\
0.993424835620891	35320.0405766191\\
0.993524838120953	35299.0130255378\\
0.993624840621016	35278.042770236\\
0.993724843121078	35257.0725149342\\
0.99382484562114	35236.1022596324\\
0.993924848121203	35215.1320043307\\
0.994024850621266	35194.1617490289\\
0.994124853121328	35173.1914937271\\
0.994224855621391	35152.2212384253\\
0.994324858121453	35131.308278903\\
0.994424860621516	35110.3953193807\\
0.994524863121578	35089.425064079\\
0.994624865621641	35068.5121045567\\
0.994724868121703	35047.5991450344\\
0.994824870621766	35026.6861855121\\
0.994924873121828	35005.7732259899\\
0.995024875621891	34984.9175622471\\
0.995124878121953	34964.0046027248\\
0.995224880622016	34943.0916432025\\
0.995324883122078	34922.2359794598\\
0.995424885622141	34901.380315717\\
0.995524888122203	34880.5246519743\\
0.995624890622266	34859.6689882315\\
0.995724893122328	34838.8133244887\\
0.995824895622391	34817.957660746\\
0.995924898122453	34797.1019970032\\
0.996024900622516	34776.30362904\\
0.996124903122578	34755.5052610767\\
0.996224905622641	34734.649597334\\
0.996324908122703	34713.8512293707\\
0.996424910622766	34693.0528614075\\
0.996524913122828	34672.2544934442\\
0.996624915622891	34651.456125481\\
0.996724918122953	34630.7150532972\\
0.996824920623016	34609.916685334\\
0.996924923123078	34589.1183173707\\
0.997024925623141	34568.377245187\\
0.997124928123203	34547.6361730033\\
0.997224930623266	34526.8951008195\\
0.997324933123328	34506.1540286358\\
0.997424935623391	34485.412956452\\
0.997524938123453	34464.6718842683\\
0.997624940623516	34443.9308120846\\
0.997724943123578	34423.2470356803\\
0.997824945623641	34402.5059634966\\
0.997924948123703	34381.8221870924\\
0.998024950623766	34361.1384106882\\
0.998124953123828	34340.4546342839\\
0.998224955623891	34319.7708578797\\
0.998324958123953	34299.0870814755\\
0.998424960624016	34278.4033050713\\
0.998524963124078	34257.7768244466\\
0.998624965624141	34237.0930480423\\
0.998724968124203	34216.4665674176\\
0.998824970624266	34195.8400867929\\
0.998924973124328	34175.2136061682\\
0.999024975624391	34154.5871255435\\
0.999124978124453	34133.9606449188\\
0.999224980624516	34113.3341642941\\
0.999324983124578	34092.7076836694\\
0.999424985624641	34072.1384988242\\
0.999524988124703	34051.5120181995\\
0.999624990624766	34030.9428333543\\
0.999724993124828	34010.3736485091\\
0.999824995624891	33989.8044636639\\
0.999924998124953	33969.2352788187\\
1.00002500062502	33948.6660939735\\
1.00012500312508	33928.0969091283\\
1.00022500562514	33907.5850200626\\
1.0003250081252	33887.0158352174\\
1.00042501062527	33866.5039461517\\
1.00052501312533	33845.9347613065\\
1.00062501562539	33825.4228722408\\
1.00072501812545	33804.9109831752\\
1.00082502062552	33784.3990941095\\
1.00092502312558	33763.8872050438\\
1.00102502562564	33743.4326117576\\
1.0011250281257	33722.9207226919\\
1.00122503062577	33702.4661294058\\
1.00132503312583	33681.9542403401\\
1.00142503562589	33661.4996470539\\
1.00152503812595	33641.0450537677\\
1.00162504062602	33620.5904604816\\
1.00172504312608	33600.1358671954\\
1.00182504562614	33579.7385696888\\
1.0019250481262	33559.2839764026\\
1.00202505062627	33538.8293831164\\
1.00212505312633	33518.4320856097\\
1.00222505562639	33498.0347881031\\
1.00232505812645	33477.6374905964\\
1.00242506062652	33457.2401930898\\
1.00252506312658	33436.8428955831\\
1.00262506562664	33416.4455980765\\
1.0027250681267	33396.0483005698\\
1.00282507062677	33375.7082988427\\
1.00292507312683	33355.311001336\\
1.00302507562689	33334.9709996089\\
1.00312507812695	33314.6309978817\\
1.00322508062702	33294.2337003751\\
1.00332508312708	33273.8936986479\\
1.00342508562714	33253.6109927003\\
1.0035250881272	33233.2709909731\\
1.00362509062727	33212.930989246\\
1.00372509312733	33192.6482832984\\
1.00382509562739	33172.3082815712\\
1.00392509812745	33152.0255756236\\
1.00402510062752	33131.742869676\\
1.00412510312758	33111.4601637283\\
1.00422510562764	33091.1774577807\\
1.0043251081277	33070.8947518331\\
1.00442511062777	33050.6120458854\\
1.00452511312783	33030.3866357173\\
1.00462511562789	33010.1039297697\\
1.00472511812795	32989.8785196016\\
1.00482512062802	32969.6531094334\\
1.00492512312808	32949.3704034858\\
1.00502512562814	32929.1449933177\\
1.0051251281282	32908.9768789291\\
1.00522513062827	32888.751468761\\
1.00532513312833	32868.5260585929\\
1.00542513562839	32848.3579442043\\
1.00552513812845	32828.1325340361\\
1.00562514062852	32807.9644196475\\
1.00572514312858	32787.7963052589\\
1.00582514562864	32767.6281908703\\
1.0059251481287	32747.4600764817\\
1.00602515062877	32727.2919620931\\
1.00612515312883	32707.1238477045\\
1.00622515562889	32686.9557333159\\
1.00632515812895	32666.8449147068\\
1.00642516062902	32646.7340960977\\
1.00652516312908	32626.5659817091\\
1.00662516562914	32606.4551631\\
1.0067251681292	32586.3443444909\\
1.00682517062927	32566.2335258818\\
1.00692517312933	32546.1227072727\\
1.00702517562939	32526.0691844432\\
1.00712517812945	32505.9583658341\\
1.00722518062952	32485.9048430045\\
1.00732518312958	32465.7940243954\\
1.00742518562964	32445.7405015658\\
1.0075251881297	32425.6869787362\\
1.00762519062977	32405.6334559067\\
1.00772519312983	32385.5799330771\\
1.00782519562989	32365.5264102475\\
1.00792519812995	32345.5301831974\\
1.00802520063002	32325.4766603679\\
1.00812520313008	32305.4804333178\\
1.00822520563014	32285.4842062677\\
1.0083252081302	32265.4306834382\\
1.00842521063027	32245.4344563881\\
1.00852521313033	32225.438229338\\
1.00862521563039	32205.4992980675\\
1.00872521813045	32185.5030710174\\
1.00882522063052	32165.5068439673\\
1.00892522313058	32145.5679126968\\
1.00902522563064	32125.5716856467\\
1.0091252281307	32105.6327543762\\
1.00922523063077	32085.6938231056\\
1.00932523313083	32065.7548918351\\
1.00942523563089	32045.8159605645\\
1.00952523813095	32025.877029294\\
1.00962524063102	32005.9953938029\\
1.00972524313108	31986.0564625324\\
1.00982524563114	31966.1748270413\\
1.0099252481312	31946.2931915503\\
1.01002525063127	31926.3542602797\\
1.01012525313133	31906.4726247887\\
1.01022525563139	31886.5909892977\\
1.01032525813145	31866.7093538066\\
1.01042526063152	31846.8850140951\\
1.01052526313158	31827.003378604\\
1.01062526563164	31807.1790388925\\
1.0107252681317	31787.2974034015\\
1.01082527063177	31767.47306369\\
1.01092527313183	31747.6487239784\\
1.01102527563189	31727.8243842669\\
1.01112527813195	31708.0000445554\\
1.01122528063202	31688.1757048438\\
1.01132528313208	31668.3513651323\\
1.01142528563214	31648.5843212003\\
1.0115252881322	31628.7599814888\\
1.01162529063227	31608.9929375568\\
1.01172529313233	31589.2258936248\\
1.01182529563239	31569.4588496927\\
1.01192529813245	31549.6918057607\\
1.01202530063252	31529.9247618287\\
1.01212530313258	31510.1577178967\\
1.01222530563264	31490.3906739647\\
1.0123253081327	31470.6809258122\\
1.01242531063277	31450.9138818802\\
1.01252531313283	31431.2041337277\\
1.01262531563289	31411.4943855752\\
1.01272531813295	31391.7846374227\\
1.01282532063302	31372.0748892702\\
1.01292532313308	31352.3651411177\\
1.01302532563314	31332.6553929652\\
1.0131253281332	31313.0029405922\\
1.01322533063327	31293.2931924397\\
1.01332533313333	31273.6407400667\\
1.01342533563339	31253.9882876937\\
1.01352533813345	31234.3358353207\\
1.01362534063352	31214.6833829477\\
1.01372534313358	31195.0309305748\\
1.01382534563364	31175.3784782018\\
1.0139253481337	31155.7260258288\\
1.01402535063377	31136.1308692353\\
1.01412535313383	31116.4784168623\\
1.01422535563389	31096.8832602688\\
1.01432535813395	31077.2881036754\\
1.01442536063402	31057.6929470819\\
1.01452536313408	31038.0977904884\\
1.01462536563414	31018.5026338949\\
1.0147253681342	30998.9074773015\\
1.01482537063427	30979.312320708\\
1.01492537313433	30959.774459894\\
1.01502537563439	30940.2365990801\\
1.01512537813445	30920.6414424866\\
1.01522538063452	30901.1035816726\\
1.01532538313458	30881.5657208587\\
1.01542538563464	30862.0278600447\\
1.0155253881347	30842.4899992308\\
1.01562539063477	30823.0094341963\\
1.01572539313483	30803.4715733823\\
1.01582539563489	30783.9337125684\\
1.01592539813495	30764.4531475339\\
1.01602540063502	30744.9725824995\\
1.01612540313508	30725.492017465\\
1.01622540563514	30706.0114524306\\
1.0163254081352	30686.5308873961\\
1.01642541063527	30667.0503223617\\
1.01652541313533	30647.5697573272\\
1.01662541563539	30628.1464880723\\
1.01672541813545	30608.6659230379\\
1.01682542063552	30589.2426537829\\
1.01692542313558	30569.819384528\\
1.01702542563564	30550.3961152731\\
1.0171254281357	30530.9728460181\\
1.01722543063577	30511.5495767632\\
1.01732543313583	30492.1263075083\\
1.01742543563589	30472.7603340328\\
1.01752543813595	30453.3370647779\\
1.01762544063602	30433.9710913025\\
1.01772544313608	30414.5478220475\\
1.01782544563614	30395.1818485721\\
1.0179254481362	30375.8158750967\\
1.01802545063627	30356.4499016213\\
1.01812545313633	30337.1412239254\\
1.01822545563639	30317.7752504499\\
1.01832545813645	30298.4092769745\\
1.01842546063652	30279.1005992786\\
1.01852546313658	30259.7346258032\\
1.01862546563664	30240.4259481073\\
1.0187254681367	30221.1172704114\\
1.01882547063677	30201.8085927155\\
1.01892547313683	30182.4999150196\\
1.01902547563689	30163.1912373237\\
1.01912547813695	30143.9398554073\\
1.01922548063702	30124.6311777113\\
1.01932548313708	30105.379795795\\
1.01942548563714	30086.071118099\\
1.0195254881372	30066.8197361826\\
1.01962549063727	30047.5683542662\\
1.01972549313733	30028.3169723499\\
1.01982549563739	30009.0655904335\\
1.01992549813745	29989.8715042966\\
1.02002550063752	29970.6201223802\\
1.02012550313758	29951.3687404638\\
1.02022550563764	29932.1746543269\\
1.0203255081377	29912.98056819\\
1.02042551063777	29893.7864820531\\
1.02052551313783	29874.5923959163\\
1.02062551563789	29855.3983097794\\
1.02072551813795	29836.2042236425\\
1.02082552063802	29817.0101375056\\
1.02092552313808	29797.8733471482\\
1.02102552563814	29778.6792610114\\
1.0211255281382	29759.542470654\\
1.02122553063827	29740.4056802966\\
1.02132553313833	29721.2115941597\\
1.02142553563839	29702.0748038024\\
1.02152553813845	29682.9953092245\\
1.02162554063852	29663.8585188671\\
1.02172554313858	29644.7217285098\\
1.02182554563864	29625.6422339319\\
1.0219255481387	29606.5054435745\\
1.02202555063877	29587.4259489967\\
1.02212555313883	29568.3464544188\\
1.02222555563889	29549.2096640615\\
1.02232555813895	29530.1874652631\\
1.02242556063902	29511.1079706853\\
1.02252556313908	29492.0284761074\\
1.02262556563914	29472.9489815295\\
1.0227255681392	29453.9267827312\\
1.02282557063927	29434.8472881533\\
1.02292557313933	29415.825089355\\
1.02302557563939	29396.8028905567\\
1.02312557813945	29377.7806917583\\
1.02322558063952	29358.75849296\\
1.02332558313958	29339.7362941616\\
1.02342558563964	29320.7140953633\\
1.0235255881397	29301.7491923445\\
1.02362559063977	29282.7269935461\\
1.02372559313983	29263.7620905273\\
1.02382559563989	29244.7971875085\\
1.02392559813995	29225.8322844896\\
1.02402560064002	29206.8673814708\\
1.02412560314008	29187.902478452\\
1.02422560564014	29168.9375754331\\
1.0243256081402	29149.9726724143\\
1.02442561064027	29131.065065175\\
1.02452561314033	29112.1001621562\\
1.02462561564039	29093.1925549168\\
1.02472561814045	29074.2849476775\\
1.02482562064052	29055.3773404382\\
1.02492562314058	29036.4697331989\\
1.02502562564064	29017.5621259596\\
1.0251256281407	28998.6545187203\\
1.02522563064077	28979.7469114809\\
1.02532563314083	28960.8966000211\\
1.02542563564089	28941.9889927818\\
1.02552563814095	28923.138681322\\
1.02562564064102	28904.2883698622\\
1.02572564314108	28885.4380584024\\
1.02582564564114	28866.5877469426\\
1.0259256481412	28847.7374354828\\
1.02602565064127	28828.887124023\\
1.02612565314133	28810.0941083427\\
1.02622565564139	28791.2437968829\\
1.02632565814145	28772.4507812026\\
1.02642566064152	28753.6577655223\\
1.02652566314158	28734.864749842\\
1.02662566564164	28716.0717341617\\
1.0267256681417	28697.2787184814\\
1.02682567064177	28678.4857028011\\
1.02692567314183	28659.6926871209\\
1.02702567564189	28640.9569672201\\
1.02712567814195	28622.1639515398\\
1.02722568064202	28603.428231639\\
1.02732568314208	28584.6925117382\\
1.02742568564214	28565.8994960579\\
1.0275256881422	28547.1637761572\\
1.02762569064227	28528.4853520359\\
1.02772569314233	28509.7496321351\\
1.02782569564239	28491.0139122343\\
1.02792569814245	28472.3354881131\\
1.02802570064252	28453.5997682123\\
1.02812570314258	28434.921344091\\
1.02822570564264	28416.2429199698\\
1.0283257081427	28397.5644958485\\
1.02842571064277	28378.8860717272\\
1.02852571314283	28360.207647606\\
1.02862571564289	28341.5292234847\\
1.02872571814295	28322.8507993634\\
1.02882572064302	28304.2296710217\\
1.02892572314308	28285.6085426799\\
1.02902572564314	28266.9301185587\\
1.0291257281432	28248.3089902169\\
1.02922573064327	28229.6878618752\\
1.02932573314333	28211.0667335334\\
1.02942573564339	28192.4456051917\\
1.02952573814345	28173.8817726294\\
1.02962574064352	28155.2606442877\\
1.02972574314358	28136.6968117254\\
1.02982574564364	28118.0756833837\\
1.0299257481437	28099.5118508215\\
1.03002575064377	28080.9480182592\\
1.03012575314383	28062.384185697\\
1.03022575564389	28043.8203531347\\
1.03032575814395	28025.2565205725\\
1.03042576064402	28006.6926880103\\
1.03052576314408	27988.1861512275\\
1.03062576564414	27969.6223186653\\
1.0307257681442	27951.1157818826\\
1.03082577064427	27932.6092450998\\
1.03092577314433	27914.1027083171\\
1.03102577564439	27895.5961715344\\
1.03112577814445	27877.0896347517\\
1.03122578064452	27858.5830979689\\
1.03132578314458	27840.1338569657\\
1.03142578564464	27821.627320183\\
1.0315257881447	27803.1780791798\\
1.03162579064477	27784.6715423971\\
1.03172579314483	27766.2223013939\\
1.03182579564489	27747.7730603906\\
1.03192579814495	27729.3238193874\\
1.03202580064502	27710.8745783842\\
1.03212580314508	27692.4826331605\\
1.03222580564514	27674.0333921573\\
1.0323258081452	27655.5841511541\\
1.03242581064527	27637.1922059304\\
1.03252581314533	27618.8002607067\\
1.03262581564539	27600.408315483\\
1.03272581814545	27582.0163702593\\
1.03282582064552	27563.6244250356\\
1.03292582314558	27545.2324798119\\
1.03302582564564	27526.8405345882\\
1.0331258281457	27508.505885144\\
1.03322583064577	27490.1139399203\\
1.03332583314583	27471.7792904761\\
1.03342583564589	27453.3873452524\\
1.03352583814595	27435.0526958082\\
1.03362584064602	27416.7180463641\\
1.03372584314608	27398.3833969199\\
1.03382584564614	27380.1060432552\\
1.0339258481462	27361.771393811\\
1.03402585064627	27343.4367443668\\
1.03412585314633	27325.1593907021\\
1.03422585564639	27306.8820370375\\
1.03432585814645	27288.5473875933\\
1.03442586064652	27270.2700339286\\
1.03452586314658	27251.9926802639\\
1.03462586564664	27233.7153265993\\
1.0347258681467	27215.4952687141\\
1.03482587064677	27197.2179150494\\
1.03492587314683	27178.9978571643\\
1.03502587564689	27160.7205034996\\
1.03512587814695	27142.5004456144\\
1.03522588064702	27124.2803877293\\
1.03532588314708	27106.0603298441\\
1.03542588564714	27087.840271959\\
1.0355258881472	27069.6202140738\\
1.03562589064727	27051.4001561886\\
1.03572589314733	27033.1800983035\\
1.03582589564739	27015.0173361978\\
1.03592589814745	26996.8545740922\\
1.03602590064752	26978.634516207\\
1.03612590314758	26960.4717541014\\
1.03622590564764	26942.3089919957\\
1.0363259081477	26924.1462298901\\
1.03642591064777	26905.9834677844\\
1.03652591314783	26887.8780014583\\
1.03662591564789	26869.7152393527\\
1.03672591814795	26851.6097730265\\
1.03682592064802	26833.4470109209\\
1.03692592314808	26815.3415445947\\
1.03702592564814	26797.2360782686\\
1.0371259281482	26779.1306119425\\
1.03722593064827	26761.0251456163\\
1.03732593314833	26742.9196792902\\
1.03742593564839	26724.8142129641\\
1.03752593814845	26706.7660424174\\
1.03762594064852	26688.6605760913\\
1.03772594314858	26670.6124055447\\
1.03782594564864	26652.5642349981\\
1.0379259481487	26634.5160644515\\
1.03802595064877	26616.4678939048\\
1.03812595314883	26598.4197233582\\
1.03822595564889	26580.3715528116\\
1.03832595814895	26562.323382265\\
1.03842596064902	26544.3325074979\\
1.03852596314908	26526.2843369512\\
1.03862596564914	26508.2934621841\\
1.0387259681492	26490.302587417\\
1.03882597064927	26472.3117126499\\
1.03892597314933	26454.3208378828\\
1.03902597564939	26436.3299631157\\
1.03912597814945	26418.3390883486\\
1.03922598064952	26400.405509361\\
1.03932598314958	26382.4146345939\\
1.03942598564964	26364.4810556063\\
1.0395259881497	26346.4901808392\\
1.03962599064977	26328.5566018516\\
1.03972599314983	26310.623022864\\
1.03982599564989	26292.6894438764\\
1.03992599814995	26274.7558648888\\
1.04002600065002	26256.8795816807\\
1.04012600315008	26238.9460026931\\
1.04022600565014	26221.069719485\\
1.0403260081502	26203.1361404975\\
1.04042601065027	26185.2598572894\\
1.04052601315033	26167.3835740813\\
1.04062601565039	26149.5072908732\\
1.04072601815045	26131.6310076651\\
1.04082602065052	26113.754724457\\
1.04092602315058	26095.878441249\\
1.04102602565064	26078.0594538204\\
1.0411260281507	26060.1831706123\\
1.04122603065077	26042.3641831837\\
1.04132603315083	26024.5451957552\\
1.04142603565089	26006.7262083266\\
1.04152603815095	25988.907220898\\
1.04162604065102	25971.0882334695\\
1.04172604315108	25953.2692460409\\
1.04182604565114	25935.4502586123\\
1.0419260481512	25917.6885669633\\
1.04202605065127	25899.8695795347\\
1.04212605315133	25882.1078878857\\
1.04222605565139	25864.3461962366\\
1.04232605815145	25846.5845045875\\
1.04242606065152	25828.8228129385\\
1.04252606315158	25811.0611212894\\
1.04262606565164	25793.2994296404\\
1.0427260681517	25775.5377379913\\
1.04282607065177	25757.8333421218\\
1.04292607315183	25740.0716504727\\
1.04302607565189	25722.3672546032\\
1.04312607815195	25704.6628587336\\
1.04322608065202	25686.9584628641\\
1.04332608315208	25669.2540669945\\
1.04342608565214	25651.549671125\\
1.0435260881522	25633.8452752555\\
1.04362609065227	25616.1408793859\\
1.04372609315233	25598.4937792959\\
1.04382609565239	25580.8466792059\\
1.04392609815245	25563.1422833363\\
1.04402610065252	25545.4951832463\\
1.04412610315258	25527.8480831563\\
1.04422610565264	25510.2009830662\\
1.0443261081527	25492.5538829762\\
1.04442611065277	25474.9067828862\\
1.04452611315283	25457.3169785757\\
1.04462611565289	25439.6698784856\\
1.04472611815295	25422.0800741751\\
1.04482612065302	25404.4902698646\\
1.04492612315308	25386.9004655541\\
1.04502612565314	25369.2533654641\\
1.0451261281532	25351.720856933\\
1.04522613065327	25334.1310526225\\
1.04532613315333	25316.541248312\\
1.04542613565339	25298.9514440015\\
1.04552613815345	25281.4189354705\\
1.04562614065352	25263.8864269395\\
1.04572614315358	25246.296622629\\
1.04582614565364	25228.764114098\\
1.0459261481537	25211.231605567\\
1.04602615065377	25193.699097036\\
1.04612615315383	25176.166588505\\
1.04622615565389	25158.6913757535\\
1.04632615815395	25141.1588672225\\
1.04642616065402	25123.683654471\\
1.04652616315408	25106.15114594\\
1.04662616565414	25088.6759331885\\
1.0467261681542	25071.200720437\\
1.04682617065427	25053.7255076855\\
1.04692617315433	25036.250294934\\
1.04702617565439	25018.7750821825\\
1.04712617815445	25001.3571652105\\
1.04722618065452	24983.8819524591\\
1.04732618315458	24966.4640354871\\
1.04742618565464	24948.9888227356\\
1.0475261881547	24931.5709057636\\
1.04762619065477	24914.1529887916\\
1.04772619315483	24896.7350718197\\
1.04782619565489	24879.3171548477\\
1.04792619815495	24861.8992378757\\
1.04802620065502	24844.5386166832\\
1.04812620315508	24827.1206997113\\
1.04822620565514	24809.7600785188\\
1.0483262081552	24792.3994573263\\
1.04842621065527	24774.9815403544\\
1.04852621315533	24757.6209191619\\
1.04862621565539	24740.2602979694\\
1.04872621815545	24722.899676777\\
1.04882622065552	24705.596351364\\
1.04892622315558	24688.2357301716\\
1.04902622565564	24670.8751089791\\
1.0491262281557	24653.5717835661\\
1.04922623065577	24636.2684581532\\
1.04932623315583	24618.9651327402\\
1.04942623565589	24601.6618073273\\
1.04952623815595	24584.3584819143\\
1.04962624065602	24567.0551565014\\
1.04972624315608	24549.7518310884\\
1.04982624565614	24532.4485056755\\
1.0499262481562	24515.202476042\\
1.05002625065627	24497.9564464086\\
1.05012625315633	24480.6531209957\\
1.05022625565639	24463.4070913622\\
1.05032625815645	24446.1610617288\\
1.05042626065652	24428.9150320953\\
1.05052626315658	24411.6690024619\\
1.05062626565664	24394.480268608\\
1.0507262681567	24377.2342389745\\
1.05082627065677	24359.9882093411\\
1.05092627315683	24342.7994754872\\
1.05102627565689	24325.6107416333\\
1.05112627815695	24308.4220077793\\
1.05122628065702	24291.2332739254\\
1.05132628315708	24274.0445400715\\
1.05142628565714	24256.8558062176\\
1.0515262881572	24239.6670723636\\
1.05162629065727	24222.5356342892\\
1.05172629315733	24205.3469004353\\
1.05182629565739	24188.2154623609\\
1.05192629815745	24171.0840242865\\
1.05202630065752	24153.8952904325\\
1.05212630315758	24136.7638523581\\
1.05222630565764	24119.6324142837\\
1.0523263081577	24102.5582719888\\
1.05242631065777	24085.4268339144\\
1.05252631315783	24068.29539584\\
1.05262631565789	24051.2212535451\\
1.05272631815795	24034.1471112502\\
1.05282632065802	24017.0156731758\\
1.05292632315808	23999.9415308809\\
1.05302632565814	23982.867388586\\
1.0531263281582	23965.7932462911\\
1.05322633065827	23948.7763997757\\
1.05332633315833	23931.7022574808\\
1.05342633565839	23914.6281151859\\
1.05352633815845	23897.6112686705\\
1.05362634065852	23880.5944221551\\
1.05372634315858	23863.5202798603\\
1.05382634565864	23846.5034333449\\
1.0539263481587	23829.4865868295\\
1.05402635065877	23812.4697403141\\
1.05412635315883	23795.5101895782\\
1.05422635565889	23778.4933430628\\
1.05432635815895	23761.4764965474\\
1.05442636065902	23744.5169458116\\
1.05452636315908	23727.5573950757\\
1.05462636565914	23710.5405485603\\
1.0547263681592	23693.5809978244\\
1.05482637065927	23676.6214470886\\
1.05492637315933	23659.6618963527\\
1.05502637565939	23642.7596413963\\
1.05512637815945	23625.8000906605\\
1.05522638065952	23608.8978357041\\
1.05532638315958	23591.9382849682\\
1.05542638565964	23575.0360300119\\
1.0555263881597	23558.1337750555\\
1.05562639065977	23541.2315200992\\
1.05572639315983	23524.3292651428\\
1.05582639565989	23507.4270101864\\
1.05592639815995	23490.5247552301\\
1.05602640066002	23473.6225002737\\
1.05612640316008	23456.7775410969\\
1.05622640566014	23439.8752861405\\
1.0563264081602	23423.0303269637\\
1.05642641066027	23406.1853677868\\
1.05652641316033	23389.34040861\\
1.05662641566039	23372.4954494331\\
1.05672641816045	23355.6504902563\\
1.05682642066052	23338.8055310794\\
1.05692642316058	23322.0178676821\\
1.05702642566064	23305.1729085053\\
1.0571264281607	23288.3852451079\\
1.05722643066077	23271.5402859311\\
1.05732643316083	23254.7526225337\\
1.05742643566089	23237.9649591364\\
1.05752643816095	23221.1772957391\\
1.05762644066102	23204.3896323418\\
1.05772644316108	23187.6592647239\\
1.05782644566114	23170.8716013266\\
1.0579264481612	23154.1412337088\\
1.05802645066127	23137.3535703114\\
1.05812645316133	23120.6232026936\\
1.05822645566139	23103.8928350758\\
1.05832645816145	23087.162467458\\
1.05842646066152	23070.4320998402\\
1.05852646316158	23053.7017322223\\
1.05862646566164	23036.9713646045\\
1.0587264681617	23020.2982927662\\
1.05882647066177	23003.5679251484\\
1.05892647316183	22986.8948533101\\
1.05902647566189	22970.2217814718\\
1.05912647816195	22953.491413854\\
1.05922648066202	22936.8183420157\\
1.05932648316208	22920.1452701773\\
1.05942648566214	22903.5294941186\\
1.0595264881622	22886.8564222802\\
1.05962649066227	22870.1833504419\\
1.05972649316233	22853.5675743831\\
1.05982649566239	22836.8945025448\\
1.05992649816245	22820.278726486\\
1.06002650066252	22803.6629504273\\
1.06012650316258	22787.0471743685\\
1.06022650566264	22770.4313983097\\
1.0603265081627	22753.8156222509\\
1.06042651066277	22737.2571419716\\
1.06052651316283	22720.6413659128\\
1.06062651566289	22704.0828856335\\
1.06072651816295	22687.4671095747\\
1.06082652066302	22670.9086292954\\
1.06092652316308	22654.3501490162\\
1.06102652566314	22637.7916687369\\
1.0611265281632	22621.2331884576\\
1.06122653066327	22604.6747081783\\
1.06132653316333	22588.116227899\\
1.06142653566339	22571.6150433993\\
1.06152653816345	22555.05656312\\
1.06162654066352	22538.5553786202\\
1.06172654316358	22522.0541941205\\
1.06182654566364	22505.5530096207\\
1.0619265481637	22489.0518251209\\
1.06202655066377	22472.5506406211\\
1.06212655316383	22456.0494561214\\
1.06222655566389	22439.5482716216\\
1.06232655816395	22423.1043829014\\
1.06242656066402	22406.6031984016\\
1.06252656316408	22390.1593096813\\
1.06262656566414	22373.7154209611\\
1.0627265681642	22357.2715322408\\
1.06282657066427	22340.8276435206\\
1.06292657316433	22324.3837548003\\
1.06302657566439	22307.9398660801\\
1.06312657816445	22291.4959773598\\
1.06322658066452	22275.1093844191\\
1.06332658316458	22258.6654956988\\
1.06342658566464	22242.2789027581\\
1.0635265881647	22225.8923098173\\
1.06362659066477	22209.5057168766\\
1.06372659316483	22193.1191239358\\
1.06382659566489	22176.7325309951\\
1.06392659816495	22160.3459380544\\
1.06402660066502	22143.9593451136\\
1.06412660316508	22127.6300479524\\
1.06422660566514	22111.2434550117\\
1.0643266081652	22094.9141578504\\
1.06442661066527	22078.5848606892\\
1.06452661316533	22062.255563528\\
1.06462661566539	22045.9262663667\\
1.06472661816545	22029.5969692055\\
1.06482662066552	22013.2676720443\\
1.06492662316558	21996.9383748831\\
1.06502662566564	21980.6663735013\\
1.0651266281657	21964.3370763401\\
1.06522663066577	21948.0650749584\\
1.06532663316583	21931.7930735767\\
1.06542663566589	21915.521072195\\
1.06552663816595	21899.2490708133\\
1.06562664066602	21882.9770694315\\
1.06572664316608	21866.7050680498\\
1.06582664566614	21850.4330666681\\
1.0659266481662	21834.2183610659\\
1.06602665066627	21817.9463596842\\
1.06612665316633	21801.731654082\\
1.06622665566639	21785.5169484798\\
1.06632665816645	21769.3022428776\\
1.06642666066652	21753.0875372754\\
1.06652666316658	21736.8728316732\\
1.06662666566664	21720.658126071\\
1.0667266681667	21704.4434204688\\
1.06682667066677	21688.2860106461\\
1.06692667316683	21672.0713050439\\
1.06702667566689	21655.9138952212\\
1.06712667816695	21639.7564853985\\
1.06722668066702	21623.5990755758\\
1.06732668316708	21607.4416657531\\
1.06742668566714	21591.2842559304\\
1.0675266881672	21575.1268461077\\
1.06762669066727	21559.0267320646\\
1.06772669316733	21542.8693222419\\
1.06782669566739	21526.7692081987\\
1.06792669816745	21510.611798376\\
1.06802670066752	21494.5116843328\\
1.06812670316758	21478.4115702897\\
1.06822670566764	21462.3114562465\\
1.0683267081677	21446.2113422033\\
1.06842671066777	21430.1112281601\\
1.06852671316783	21414.0684098965\\
1.06862671566789	21397.9682958533\\
1.06872671816795	21381.9254775896\\
1.06882672066802	21365.8253635464\\
1.06892672316808	21349.7825452828\\
1.06902672566814	21333.7397270191\\
1.0691267281682	21317.6969087555\\
1.06922673066827	21301.6540904918\\
1.06932673316833	21285.6685680076\\
1.06942673566839	21269.625749744\\
1.06952673816845	21253.5829314803\\
1.06962674066852	21237.5974089962\\
1.06972674316858	21221.611886512\\
1.06982674566864	21205.5690682484\\
1.0699267481687	21189.5835457642\\
1.07002675066877	21173.5980232801\\
1.07012675316883	21157.6697965754\\
1.07022675566889	21141.6842740913\\
1.07032675816895	21125.6987516071\\
1.07042676066902	21109.7705249025\\
1.07052676316908	21093.7850024183\\
1.07062676566914	21077.8567757137\\
1.0707267681692	21061.9285490091\\
1.07082677066927	21046.0003223044\\
1.07092677316933	21030.0720955998\\
1.07102677566939	21014.1438688952\\
1.07112677816945	20998.2156421905\\
1.07122678066952	20982.2874154859\\
1.07132678316958	20966.4164845608\\
1.07142678566964	20950.4882578561\\
1.0715267881697	20934.617326931\\
1.07162679066977	20918.7463960059\\
1.07172679316983	20902.8754650807\\
1.07182679566989	20887.0045341556\\
1.07192679816995	20871.1336032305\\
1.07202680067002	20855.2626723054\\
1.07212680317008	20839.4490371598\\
1.07222680567014	20823.5781062346\\
1.0723268081702	20807.764471089\\
1.07242681067027	20791.8935401639\\
1.07252681317033	20776.0799050183\\
1.07262681567039	20760.2662698727\\
1.07272681817045	20744.4526347271\\
1.07282682067052	20728.6389995815\\
1.07292682317058	20712.8253644359\\
1.07302682567064	20697.0690250698\\
1.0731268281707	20681.2553899241\\
1.07322683067077	20665.499050558\\
1.07332683317083	20649.7427111919\\
1.07342683567089	20633.9290760463\\
1.07352683817095	20618.1727366802\\
1.07362684067102	20602.4163973141\\
1.07372684317108	20586.660057948\\
1.07382684567114	20570.9610143615\\
1.0739268481712	20555.2046749954\\
1.07402685067127	20539.4483356293\\
1.07412685317133	20523.7492920427\\
1.07422685567139	20508.0502484561\\
1.07432685817145	20492.3512048695\\
1.07442686067152	20476.5948655034\\
1.07452686317158	20460.8958219168\\
1.07462686567164	20445.2540741098\\
1.0747268681717	20429.5550305232\\
1.07482687067177	20413.8559869366\\
1.07492687317183	20398.2142391295\\
1.07502687567189	20382.5151955429\\
1.07512687817195	20366.8734477359\\
1.07522688067202	20351.2316999288\\
1.07532688317208	20335.5899521217\\
1.07542688567214	20319.9482043146\\
1.0755268881722	20304.3064565076\\
1.07562689067227	20288.6647087005\\
1.07572689317233	20273.0229608934\\
1.07582689567239	20257.4385088659\\
1.07592689817245	20241.7967610588\\
1.07602690067252	20226.2123090312\\
1.07612690317258	20210.6278570037\\
1.07622690567264	20195.0434049761\\
1.0763269081727	20179.4589529486\\
1.07642691067277	20163.874500921\\
1.07652691317283	20148.2900488935\\
1.07662691567289	20132.7628926454\\
1.07672691817295	20117.1784406178\\
1.07682692067302	20101.6512843698\\
1.07692692317308	20086.0668323422\\
1.07702692567314	20070.5396760942\\
1.0771269281732	20055.0125198462\\
1.07722693067327	20039.4853635981\\
1.07732693317333	20023.9582073501\\
1.07742693567339	20008.431051102\\
1.07752693817345	19992.9611906335\\
1.07762694067352	19977.4340343854\\
1.07772694317358	19961.9641739169\\
1.07782694567364	19946.4370176689\\
1.0779269481737	19930.9671572003\\
1.07802695067377	19915.4972967318\\
1.07812695317383	19900.0274362633\\
1.07822695567389	19884.5575757947\\
1.07832695817395	19869.0877153262\\
1.07842696067402	19853.6751506372\\
1.07852696317408	19838.2052901687\\
1.07862696567414	19822.7927254796\\
1.0787269681742	19807.3228650111\\
1.07882697067427	19791.9103003221\\
1.07892697317433	19776.4977356331\\
1.07902697567439	19761.085170944\\
1.07912697817445	19745.672606255\\
1.07922698067452	19730.260041566\\
1.07932698317458	19714.9047726565\\
1.07942698567464	19699.4922079675\\
1.0795269881747	19684.136939058\\
1.07962699067477	19668.7816701485\\
1.07972699317483	19653.3691054594\\
1.07982699567489	19638.0138365499\\
1.07992699817495	19622.6585676404\\
1.08002700067502	19607.3032987309\\
1.08012700317508	19592.0053256009\\
1.08022700567514	19576.6500566914\\
1.0803270081752	19561.2947877819\\
1.08042701067527	19545.9968146519\\
1.08052701317533	19530.6415457424\\
1.08062701567539	19515.3435726124\\
1.08072701817545	19500.0455994824\\
1.08082702067552	19484.7476263524\\
1.08092702317558	19469.4496532225\\
1.08102702567564	19454.1516800925\\
1.0811270281757	19438.911002742\\
1.08122703067577	19423.613029612\\
1.08132703317583	19408.3723522615\\
1.08142703567589	19393.0743791315\\
1.08152703817595	19377.833701781\\
1.08162704067602	19362.5930244306\\
1.08172704317608	19347.3523470801\\
1.08182704567614	19332.1116697296\\
1.0819270481762	19316.8709923791\\
1.08202705067627	19301.6876108081\\
1.08212705317633	19286.4469334577\\
1.08222705567639	19271.2635518867\\
1.08232705817645	19256.0228745362\\
1.08242706067652	19240.8394929653\\
1.08252706317658	19225.6561113943\\
1.08262706567664	19210.4727298233\\
1.0827270681767	19195.2893482524\\
1.08282707067677	19180.1059666814\\
1.08292707317683	19164.9798808899\\
1.08302707567689	19149.796499319\\
1.08312707817695	19134.613117748\\
1.08322708067702	19119.4870319565\\
1.08332708317708	19104.3609461651\\
1.08342708567714	19089.2348603736\\
1.0835270881772	19074.1087745822\\
1.08362709067727	19058.9826887907\\
1.08372709317733	19043.8566029993\\
1.08382709567739	19028.7305172078\\
1.08392709817745	19013.6617271959\\
1.08402710067752	18998.5356414044\\
1.08412710317758	18983.4668513925\\
1.08422710567764	18968.3980613805\\
1.0843271081777	18953.2719755891\\
1.08442711067777	18938.2031855772\\
1.08452711317783	18923.1343955652\\
1.08462711567789	18908.1229013328\\
1.08472711817795	18893.0541113208\\
1.08482712067802	18877.9853213089\\
1.08492712317808	18862.9738270765\\
1.08502712567814	18847.9050370645\\
1.0851271281782	18832.8935428321\\
1.08522713067827	18817.8820485997\\
1.08532713317833	18802.8705543673\\
1.08542713567839	18787.8590601348\\
1.08552713817845	18772.8475659024\\
1.08562714067852	18757.8933674495\\
1.08572714317858	18742.8818732171\\
1.08582714567864	18727.8703789846\\
1.0859271481787	18712.9161805317\\
1.08602715067877	18697.9619820788\\
1.08612715317883	18683.0077836259\\
1.08622715567889	18668.053585173\\
1.08632715817895	18653.0993867201\\
1.08642716067902	18638.1451882671\\
1.08652716317908	18623.1909898142\\
1.08662716567914	18608.2367913613\\
1.0867271681792	18593.3398886879\\
1.08682717067927	18578.4429860145\\
1.08692717317933	18563.4887875616\\
1.08702717567939	18548.5918848882\\
1.08712717817945	18533.6949822148\\
1.08722718067952	18518.7980795414\\
1.08732718317958	18503.901176868\\
1.08742718567964	18489.0042741946\\
1.0875271881797	18474.1646673007\\
1.08762719067977	18459.2677646273\\
1.08772719317983	18444.4281577334\\
1.08782719567989	18429.5885508395\\
1.08792719817995	18414.6916481661\\
1.08802720068002	18399.8520412722\\
1.08812720318008	18385.0124343783\\
1.08822720568014	18370.230123264\\
1.0883272081802	18355.3905163701\\
1.08842721068027	18340.5509094762\\
1.08852721318033	18325.7685983618\\
1.08862721568039	18310.9289914679\\
1.08872721818045	18296.1466803536\\
1.08882722068052	18281.3643692392\\
1.08892722318058	18266.5820581248\\
1.08902722568064	18251.7997470104\\
1.0891272281807	18237.0174358961\\
1.08922723068077	18222.2351247817\\
1.08932723318083	18207.4528136673\\
1.08942723568089	18192.7277983324\\
1.08952723818095	18177.9454872181\\
1.08962724068102	18163.2204718832\\
1.08972724318108	18148.4954565483\\
1.08982724568114	18133.7704412135\\
1.0899272481812	18119.0454258786\\
1.09002725068127	18104.3204105438\\
1.09012725318133	18089.5953952089\\
1.09022725568139	18074.870379874\\
1.09032725818145	18060.2026603187\\
1.09042726068152	18045.4776449838\\
1.09052726318158	18030.8099254285\\
1.09062726568164	18016.1422058731\\
1.0907272681817	18001.4744863178\\
1.09082727068177	17986.8067667624\\
1.09092727318183	17972.1390472071\\
1.09102727568189	17957.4713276517\\
1.09112727818195	17942.8036080964\\
1.09122728068202	17928.1931843205\\
1.09132728318208	17913.5254647652\\
1.09142728568214	17898.9150409894\\
1.0915272881822	17884.3046172135\\
1.09162729068227	17869.6941934377\\
1.09172729318233	17855.0837696618\\
1.09182729568239	17840.473345886\\
1.09192729818245	17825.8629221102\\
1.09202730068252	17811.2524983343\\
1.09212730318258	17796.699370338\\
1.09222730568264	17782.0889465622\\
1.0923273081827	17767.5358185659\\
1.09242731068277	17752.9826905695\\
1.09252731318283	17738.3722667937\\
1.09262731568289	17723.8191387974\\
1.09272731818295	17709.266010801\\
1.09282732068302	17694.7701785842\\
1.09292732318308	17680.2170505879\\
1.09302732568314	17665.6639225916\\
1.0931273281832	17651.1680903748\\
1.09322733068327	17636.6149623785\\
1.09332733318333	17622.1191301617\\
1.09342733568339	17607.6232979448\\
1.09352733818345	17593.127465728\\
1.09362734068352	17578.6316335112\\
1.09372734318358	17564.1358012944\\
1.09382734568364	17549.6972648571\\
1.0939273481837	17535.2014326403\\
1.09402735068377	17520.7056004235\\
1.09412735318383	17506.2670639862\\
1.09422735568389	17491.8285275489\\
1.09432735818395	17477.3899911116\\
1.09442736068402	17462.9514546743\\
1.09452736318408	17448.512918237\\
1.09462736568414	17434.0743817997\\
1.0947273681842	17419.6358453624\\
1.09482737068427	17405.1973089251\\
1.09492737318433	17390.8160682673\\
1.09502737568439	17376.37753183\\
1.09512737818445	17361.9962911723\\
1.09522738068452	17347.6150505145\\
1.09532738318458	17333.2338098567\\
1.09542738568464	17318.8525691989\\
1.0955273881847	17304.4713285411\\
1.09562739068477	17290.0900878833\\
1.09572739318483	17275.7661430051\\
1.09582739568489	17261.3849023473\\
1.09592739818495	17247.060957469\\
1.09602740068502	17232.6797168112\\
1.09612740318508	17218.355771933\\
1.09622740568514	17204.0318270547\\
1.0963274081852	17189.7078821764\\
1.09642741068527	17175.3839372981\\
1.09652741318533	17161.0599924199\\
1.09662741568539	17146.7933433211\\
1.09672741818545	17132.4693984428\\
1.09682742068552	17118.2027493441\\
1.09692742318558	17103.8788044658\\
1.09702742568564	17089.6121553671\\
1.0971274281857	17075.3455062683\\
1.09722743068577	17061.0788571695\\
1.09732743318583	17046.8122080708\\
1.09742743568589	17032.545558972\\
1.09752743818595	17018.3362056528\\
1.09762744068602	17004.069556554\\
1.09772744318608	16989.8602032348\\
1.09782744568614	16975.593554136\\
1.0979274481862	16961.3842008168\\
1.09802745068627	16947.1748474975\\
1.09812745318633	16932.9654941783\\
1.09822745568639	16918.7561408591\\
1.09832745818645	16904.5467875398\\
1.09842746068652	16890.3947300001\\
1.09852746318658	16876.1853766808\\
1.09862746568664	16861.9760233616\\
1.0987274681867	16847.8239658219\\
1.09882747068677	16833.6719082821\\
1.09892747318683	16819.5198507424\\
1.09902747568689	16805.3677932027\\
1.09912747818695	16791.2157356629\\
1.09922748068702	16777.0636781232\\
1.09932748318708	16762.9116205835\\
1.09942748568714	16748.8168588233\\
1.0995274881872	16734.6648012835\\
1.09962749068727	16720.5700395233\\
1.09972749318733	16706.4179819836\\
1.09982749568739	16692.3232202234\\
1.09992749818745	16678.2284584631\\
1.10002750068752	16664.1336967029\\
1.10012750318758	16650.0389349427\\
1.10022750568764	16636.001468962\\
1.1003275081877	16621.9067072018\\
1.10042751068777	16607.8692412211\\
1.10052751318783	16593.7744794608\\
1.10062751568789	16579.7370134801\\
1.10072751818795	16565.6995474994\\
1.10082752068802	16551.6620815187\\
1.10092752318808	16537.624615538\\
1.10102752568814	16523.5871495573\\
1.1011275281882	16509.5496835766\\
1.10122753068827	16495.5122175959\\
1.10132753318833	16481.5320473947\\
1.10142753568839	16467.494581414\\
1.10152753818845	16453.5144112128\\
1.10162754068852	16439.5342410116\\
1.10172754318858	16425.5540708104\\
1.10182754568864	16411.5739006092\\
1.1019275481887	16397.5937304081\\
1.10202755068877	16383.6135602069\\
1.10212755318883	16369.6333900057\\
1.10222755568889	16355.710515584\\
1.10232755818895	16341.7303453828\\
1.10242756068902	16327.8074709611\\
1.10252756318908	16313.8845965394\\
1.10262756568914	16299.9617221178\\
1.1027275681892	16286.0388476961\\
1.10282757068927	16272.1159732744\\
1.10292757318933	16258.1930988527\\
1.10302757568939	16244.270224431\\
1.10312757818945	16230.4046457889\\
1.10322758068952	16216.4817713672\\
1.10332758318958	16202.616192725\\
1.10342758568964	16188.7506140829\\
1.1035275881897	16174.8277396612\\
1.10362759068977	16160.962161019\\
1.10372759318983	16147.0965823769\\
1.10382759568989	16133.2882995142\\
1.10392759818995	16119.422720872\\
1.10402760069002	16105.5571422299\\
1.10412760319008	16091.7488593672\\
1.10422760569014	16077.8832807251\\
1.1043276081902	16064.0749978624\\
1.10442761069027	16050.2667149998\\
1.10452761319033	16036.4584321371\\
1.10462761569039	16022.6501492744\\
1.10472761819045	16008.8418664118\\
1.10482762069052	15995.0335835491\\
1.10492762319058	15981.282596466\\
1.10502762569064	15967.4743136033\\
1.1051276281907	15953.7233265202\\
1.10522763069077	15939.9723394371\\
1.10532763319083	15926.1640565744\\
1.10542763569089	15912.4130694913\\
1.10552763819095	15898.6620824081\\
1.10562764069102	15884.911095325\\
1.10572764319108	15871.2174040214\\
1.10582764569114	15857.4664169382\\
1.1059276481912	15843.7727256346\\
1.10602765069127	15830.0217385515\\
1.10612765319133	15816.3280472478\\
1.10622765569139	15802.6343559442\\
1.10632765819145	15788.8833688611\\
1.10642766069152	15775.1896775574\\
1.10652766319158	15761.5532820333\\
1.10662766569164	15747.8595907297\\
1.1067276681917	15734.1658994261\\
1.10682767069177	15720.529503902\\
1.10692767319183	15706.8358125983\\
1.10702767569189	15693.1994170742\\
1.10712767819195	15679.5630215501\\
1.10722768069202	15665.8693302465\\
1.10732768319208	15652.2329347224\\
1.10742768569214	15638.5965391983\\
1.1075276881922	15625.0174394537\\
1.10762769069227	15611.3810439295\\
1.10772769319233	15597.7446484054\\
1.10782769569239	15584.1655486608\\
1.10792769819245	15570.5291531367\\
1.10802770069252	15556.9500533921\\
1.10812770319258	15543.3709536475\\
1.10822770569264	15529.7918539029\\
1.1083277081927	15516.2127541583\\
1.10842771069277	15502.6336544137\\
1.10852771319283	15489.1118504486\\
1.10862771569289	15475.532750704\\
1.10872771819295	15461.9536509594\\
1.10882772069302	15448.4318469943\\
1.10892772319308	15434.9100430292\\
1.10902772569314	15421.3882390642\\
1.1091277281932	15407.8091393196\\
1.10922773069327	15394.344631134\\
1.10932773319333	15380.8228271689\\
1.10942773569339	15367.3010232038\\
1.10952773819345	15353.7792192387\\
1.10962774069352	15340.3147110531\\
1.10972774319358	15326.7929070881\\
1.10982774569364	15313.3283989025\\
1.1099277481937	15299.8638907169\\
1.11002775069377	15286.3993825313\\
1.11012775319383	15272.9348743458\\
1.11022775569389	15259.4703661602\\
1.11032775819395	15246.0058579746\\
1.11042776069402	15232.541349789\\
1.11052776319408	15219.134137383\\
1.11062776569414	15205.6696291974\\
1.1107277681942	15192.2624167913\\
1.11082777069427	15178.8552043853\\
1.11092777319433	15165.4479919792\\
1.11102777569439	15152.0407795732\\
1.11112777819445	15138.6335671671\\
1.11122778069452	15125.226354761\\
1.11132778319458	15111.819142355\\
1.11142778569464	15098.4692257284\\
1.1115277881947	15085.0620133224\\
1.11162779069477	15071.7120966958\\
1.11172779319483	15058.3621800693\\
1.11182779569489	15044.9549676632\\
1.11192779819495	15031.6050510367\\
1.11202780069502	15018.2551344101\\
1.11212780319508	15004.9625135631\\
1.11222780569514	14991.6125969365\\
1.1123278081952	14978.26268031\\
1.11242781069527	14964.9700594629\\
1.11252781319533	14951.6201428364\\
1.11262781569539	14938.3275219894\\
1.11272781819545	14925.0349011423\\
1.11282782069552	14911.7422802953\\
1.11292782319558	14898.4496594483\\
1.11302782569564	14885.1570386012\\
1.1131278281957	14871.8644177542\\
1.11322783069577	14858.6290926867\\
1.11332783319583	14845.3364718396\\
1.11342783569589	14832.1011467721\\
1.11352783819595	14818.8085259251\\
1.11362784069602	14805.5732008576\\
1.11372784319608	14792.33787579\\
1.11382784569614	14779.1025507225\\
1.1139278481962	14765.867225655\\
1.11402785069627	14752.6319005875\\
1.11412785319633	14739.4538712995\\
1.11422785569639	14726.2185462319\\
1.11432785819645	14713.0405169439\\
1.11442786069652	14699.8051918764\\
1.11452786319658	14686.6271625884\\
1.11462786569664	14673.4491333004\\
1.1147278681967	14660.2711040124\\
1.11482787069677	14647.0930747244\\
1.11492787319683	14633.9150454364\\
1.11502787569689	14620.7370161483\\
1.11512787819695	14607.6162826399\\
1.11522788069702	14594.4382533518\\
1.11532788319708	14581.3175198433\\
1.11542788569714	14568.1967863349\\
1.1155278881972	14555.0760528264\\
1.11562789069727	14541.8980235383\\
1.11572789319733	14528.8345858094\\
1.11582789569739	14515.7138523009\\
1.11592789819745	14502.5931187924\\
1.11602790069752	14489.4723852839\\
1.11612790319758	14476.4089475549\\
1.11622790569764	14463.2882140464\\
1.1163279081977	14450.2247763174\\
1.11642791069777	14437.1613385884\\
1.11652791319783	14424.0979008594\\
1.11662791569789	14411.0344631305\\
1.11672791819796	14397.9710254015\\
1.11682792069802	14384.9075876725\\
1.11692792319808	14371.901445723\\
1.11702792569814	14358.838007994\\
1.1171279281982	14345.8318660446\\
1.11722793069827	14332.7684283156\\
1.11732793319833	14319.7622863661\\
1.11742793569839	14306.7561444167\\
1.11752793819845	14293.7500024672\\
1.11762794069852	14280.7438605177\\
1.11772794319858	14267.7377185682\\
1.11782794569864	14254.7888723983\\
1.11792794819871	14241.7827304488\\
1.11802795069877	14228.7765884993\\
1.11812795319883	14215.8277423294\\
1.11822795569889	14202.8788961594\\
1.11832795819895	14189.9300499895\\
1.11842796069902	14176.9812038195\\
1.11852796319908	14164.0323576496\\
1.11862796569914	14151.0835114796\\
1.11872796819921	14138.1346653097\\
1.11882797069927	14125.2431149192\\
1.11892797319933	14112.2942687493\\
1.11902797569939	14099.4027183588\\
1.11912797819945	14086.4538721889\\
1.11922798069952	14073.5623217984\\
1.11932798319958	14060.670771408\\
1.11942798569964	14047.7792210175\\
1.1195279881997	14034.8876706271\\
1.11962799069977	14022.0534160162\\
1.11972799319983	14009.1618656257\\
1.11982799569989	13996.2703152353\\
1.11992799819996	13983.4360606243\\
1.12002800070002	13970.6018060134\\
1.12012800320008	13957.710255623\\
1.12022800570014	13944.876001012\\
1.1203280082002	13932.0417464011\\
1.12042801070027	13919.2074917902\\
1.12052801320033	13906.4305329587\\
1.12062801570039	13893.5962783478\\
1.12072801820046	13880.7620237369\\
1.12082802070052	13867.9850649055\\
1.12092802320058	13855.2081060741\\
1.12102802570064	13842.3738514631\\
1.12112802820071	13829.5968926317\\
1.12122803070077	13816.8199338003\\
1.12132803320083	13804.0429749689\\
1.12142803570089	13791.2660161375\\
1.12152803820095	13778.5463530856\\
1.12162804070102	13765.7693942541\\
1.12172804320108	13753.0497312022\\
1.12182804570114	13740.2727723708\\
1.12192804820121	13727.5531093189\\
1.12202805070127	13714.833446267\\
1.12212805320133	13702.1137832151\\
1.12222805570139	13689.3941201632\\
1.12232805820145	13676.6744571113\\
1.12242806070152	13663.9547940594\\
1.12252806320158	13651.2351310075\\
1.12262806570164	13638.5727637351\\
1.12272806820171	13625.9103964627\\
1.12282807070177	13613.1907334108\\
1.12292807320183	13600.5283661384\\
1.12302807570189	13587.865998866\\
1.12312807820196	13575.2036315936\\
1.12322808070202	13562.5412643212\\
1.12332808320208	13549.8788970488\\
1.12342808570214	13537.273825556\\
1.1235280882022	13524.6114582836\\
1.12362809070227	13512.0063867907\\
1.12372809320233	13499.3440195183\\
1.12382809570239	13486.7389480254\\
1.12392809820246	13474.1338765325\\
1.12402810070252	13461.5288050397\\
1.12412810320258	13448.9237335468\\
1.12422810570264	13436.3186620539\\
1.12432810820271	13423.713590561\\
1.12442811070277	13411.1658148477\\
1.12452811320283	13398.5607433548\\
1.12462811570289	13386.0129676414\\
1.12472811820296	13373.4651919281\\
1.12482812070302	13360.8601204352\\
1.12492812320308	13348.3123447218\\
1.12502812570314	13335.7645690085\\
1.12512812820321	13323.2167932951\\
1.12522813070327	13310.7263133612\\
1.12532813320333	13298.1785376479\\
1.12542813570339	13285.688057714\\
1.12552813820345	13273.1402820007\\
1.12562814070352	13260.6498020668\\
1.12572814320358	13248.1593221329\\
1.12582814570364	13235.6688421991\\
1.12592814820371	13223.1783622652\\
1.12602815070377	13210.6878823314\\
1.12612815320383	13198.1974023975\\
1.12622815570389	13185.7069224637\\
1.12632815820396	13173.2737383093\\
1.12642816070402	13160.7832583755\\
1.12652816320408	13148.3500742212\\
1.12662816570414	13135.9168900668\\
1.12672816820421	13123.426410133\\
1.12682817070427	13110.9932259786\\
1.12692817320433	13098.5600418243\\
1.12702817570439	13086.1841534495\\
1.12712817820446	13073.7509692951\\
1.12722818070452	13061.3177851408\\
1.12732818320458	13048.941896766\\
1.12742818570464	13036.5087126116\\
1.12752818820471	13024.1328242368\\
1.12762819070477	13011.756935862\\
1.12772819320483	12999.3810474871\\
1.12782819570489	12987.0051591123\\
1.12792819820496	12974.6292707375\\
1.12802820070502	12962.2533823627\\
1.12812820320508	12949.9347897674\\
1.12822820570514	12937.5589013925\\
1.12832820820521	12925.2403087972\\
1.12842821070527	12912.8644204224\\
1.12852821320533	12900.5458278271\\
1.12862821570539	12888.2272352318\\
1.12872821820546	12875.9086426365\\
1.12882822070552	12863.5900500411\\
1.12892822320558	12851.2714574458\\
1.12902822570564	12839.01016063\\
1.12912822820571	12826.6915680347\\
1.12922823070577	12814.4302712189\\
1.12932823320583	12802.1116786236\\
1.12942823570589	12789.8503818078\\
1.12952823820596	12777.589084992\\
1.12962824070602	12765.3277881762\\
1.12972824320608	12753.0664913604\\
1.12982824570614	12740.8051945446\\
1.12992824820621	12728.5438977288\\
1.13002825070627	12716.3398966925\\
1.13012825320633	12704.0785998767\\
1.13022825570639	12691.8745988404\\
1.13032825820646	12679.6133020246\\
1.13042826070652	12667.4093009883\\
1.13052826320658	12655.2052999521\\
1.13062826570664	12643.0012989158\\
1.13072826820671	12630.7972978795\\
1.13082827070677	12618.6505926227\\
1.13092827320683	12606.4465915864\\
1.13102827570689	12594.2425905501\\
1.13112827820696	12582.0958852934\\
1.13122828070702	12569.9491800366\\
1.13132828320708	12557.7451790003\\
1.13142828570714	12545.5984737435\\
1.13152828820721	12533.4517684868\\
1.13162829070727	12521.30506323\\
1.13172829320733	12509.2156537527\\
1.13182829570739	12497.068948496\\
1.13192829820746	12484.9222432392\\
1.13202830070752	12472.8328337619\\
1.13212830320758	12460.6861285051\\
1.13222830570764	12448.5967190279\\
1.13232830820771	12436.5073095506\\
1.13242831070777	12424.4179000734\\
1.13252831320783	12412.3284905961\\
1.13262831570789	12400.2390811188\\
1.13272831820796	12388.1496716416\\
1.13282832070802	12376.1175579438\\
1.13292832320808	12364.0281484666\\
1.13302832570814	12351.9960347688\\
1.13312832820821	12339.9066252916\\
1.13322833070827	12327.8745115938\\
1.13332833320833	12315.8423978961\\
1.13342833570839	12303.8102841983\\
1.13352833820846	12291.7781705006\\
1.13362834070852	12279.7460568028\\
1.13372834320858	12267.7712388846\\
1.13382834570864	12255.7391251868\\
1.13392834820871	12243.7643072686\\
1.13402835070877	12231.7321935709\\
1.13412835320883	12219.7573756526\\
1.13422835570889	12207.7825577344\\
1.13432835820896	12195.8077398162\\
1.13442836070902	12183.8329218979\\
1.13452836320908	12171.8581039797\\
1.13462836570914	12159.940581841\\
1.13472836820921	12147.9657639227\\
1.13482837070927	12135.9909460045\\
1.13492837320933	12124.0734238658\\
1.13502837570939	12112.1559017271\\
1.13512837820946	12100.2383795883\\
1.13522838070952	12088.2635616701\\
1.13532838320958	12076.4033353109\\
1.13542838570964	12064.4858131722\\
1.13552838820971	12052.5682910335\\
1.13562839070977	12040.6507688947\\
1.13572839320983	12028.7905425355\\
1.13582839570989	12016.8730203968\\
1.13592839820996	12005.0127940376\\
1.13602840071002	11993.1525676784\\
1.13612840321008	11981.2923413192\\
1.13622840571014	11969.43211496\\
1.13632840821021	11957.5718886008\\
1.13642841071027	11945.7116622416\\
1.13652841321033	11933.8514358824\\
1.13662841571039	11922.0485053027\\
1.13672841821046	11910.1882789434\\
1.13682842071052	11898.3853483638\\
1.13692842321058	11886.5251220045\\
1.13702842571064	11874.7221914249\\
1.13712842821071	11862.9192608452\\
1.13722843071077	11851.1163302655\\
1.13732843321083	11839.3133996858\\
1.13742843571089	11827.5677648856\\
1.13752843821096	11815.7648343059\\
1.13762844071102	11803.9619037262\\
1.13772844321108	11792.216268926\\
1.13782844571114	11780.4706341258\\
1.13792844821121	11768.7249993256\\
1.13802845071127	11756.922068746\\
1.13812845321133	11745.1764339458\\
1.13822845571139	11733.4880949251\\
1.13832845821146	11721.7424601249\\
1.13842846071152	11709.9968253247\\
1.13852846321158	11698.3084863041\\
1.13862846571164	11686.5628515039\\
1.13872846821171	11674.8745124832\\
1.13882847071177	11663.128877683\\
1.13892847321183	11651.4405386624\\
1.13902847571189	11639.7521996417\\
1.13912847821196	11628.063860621\\
1.13922848071202	11616.3755216004\\
1.13932848321208	11604.7444783592\\
1.13942848571214	11593.0561393385\\
1.13952848821221	11581.4250960974\\
1.13962849071227	11569.7367570767\\
1.13972849321233	11558.1057138356\\
1.13982849571239	11546.4746705944\\
1.13992849821246	11534.8436273532\\
1.14002850071252	11523.2125841121\\
1.14012850321258	11511.5815408709\\
1.14022850571264	11499.9504976298\\
1.14032850821271	11488.3194543886\\
1.14042851071277	11476.745706927\\
1.14052851321283	11465.1146636858\\
1.14062851571289	11453.5409162242\\
1.14072851821296	11441.9671687625\\
1.14082852071302	11430.3934213009\\
1.14092852321308	11418.8196738393\\
1.14102852571314	11407.2459263776\\
1.14112852821321	11395.672178916\\
1.14122853071327	11384.0984314543\\
1.14132853321333	11372.5819797722\\
1.14142853571339	11361.0082323106\\
1.14152853821346	11349.4917806284\\
1.14162854071352	11337.9180331668\\
1.14172854321358	11326.4015814847\\
1.14182854571364	11314.8851298025\\
1.14192854821371	11303.3686781204\\
1.14202855071377	11291.8522264383\\
1.14212855321383	11280.3357747561\\
1.14222855571389	11268.8766188535\\
1.14232855821396	11257.3601671714\\
1.14242856071402	11245.9010112688\\
1.14252856321408	11234.4418553662\\
1.14262856571414	11222.925403684\\
1.14272856821421	11211.4662477814\\
1.14282857071427	11200.0070918788\\
1.14292857321433	11188.5479359762\\
1.14302857571439	11177.0887800736\\
1.14312857821446	11165.6869199505\\
1.14322858071452	11154.2277640478\\
1.14332858321458	11142.8259039247\\
1.14342858571464	11131.3667480221\\
1.14352858821471	11119.964887899\\
1.14362859071477	11108.5630277759\\
1.14372859321483	11097.1611676528\\
1.14382859571489	11085.7593075297\\
1.14392859821496	11074.3574474066\\
1.14402860071502	11062.9555872835\\
1.14412860321508	11051.6110229399\\
1.14422860571514	11040.2091628168\\
1.14432860821521	11028.8645984732\\
1.14442861071527	11017.4627383501\\
1.14452861321533	11006.1181740065\\
1.14462861571539	10994.7736096629\\
1.14472861821546	10983.4290453193\\
1.14482862071552	10972.0844809758\\
1.14492862321558	10960.7399166322\\
1.14502862571564	10949.4526480681\\
1.14512862821571	10938.1080837245\\
1.14522863071577	10926.7635193809\\
1.14532863321583	10915.4762508168\\
1.14542863571589	10904.1889822527\\
1.14552863821596	10892.9017136887\\
1.14562864071602	10881.6144451246\\
1.14572864321608	10870.3271765605\\
1.14582864571614	10859.0399079964\\
1.14592864821621	10847.7526394324\\
1.14602865071627	10836.4653708683\\
1.14612865321633	10825.2353980837\\
1.14622865571639	10813.9481295196\\
1.14632865821646	10802.7181567351\\
1.14642866071652	10791.4881839505\\
1.14652866321658	10780.258211166\\
1.14662866571664	10769.0282383814\\
1.14672866821671	10757.7982655968\\
1.14682867071677	10746.5682928123\\
1.14692867321683	10735.3383200277\\
1.14702867571689	10724.1656430226\\
1.14712867821696	10712.9356702381\\
1.14722868071702	10701.762993233\\
1.14732868321708	10690.590316228\\
1.14742868571714	10679.4176392229\\
1.14752868821721	10668.2449622179\\
1.14762869071727	10657.0722852128\\
1.14772869321733	10645.8996082078\\
1.14782869571739	10634.7269312027\\
1.14792869821746	10623.5542541977\\
1.14802870071752	10612.4388729721\\
1.14812870321758	10601.2661959671\\
1.14822870571764	10590.1508147415\\
1.14832870821771	10579.035433516\\
1.14842871071777	10567.9200522905\\
1.14852871321783	10556.8046710649\\
1.14862871571789	10545.6892898394\\
1.14872871821796	10534.5739086139\\
1.14882872071802	10523.4585273883\\
1.14892872321808	10512.4004419423\\
1.14902872571814	10501.2850607168\\
1.14912872821821	10490.2269752707\\
1.14922873071827	10479.1688898247\\
1.14932873321833	10468.1108043787\\
1.14942873571839	10457.0527189327\\
1.14952873821846	10445.9946334866\\
1.14962874071852	10434.9365480406\\
1.14972874321858	10423.8784625946\\
1.14982874571864	10412.8203771486\\
1.14992874821871	10401.819587482\\
1.15002875071877	10390.8187978155\\
1.15012875321883	10379.7607123695\\
1.15022875571889	10368.759922703\\
1.15032875821896	10357.7591330365\\
1.15042876071902	10346.75834337\\
1.15052876321908	10335.7575537035\\
1.15062876571914	10324.7567640369\\
1.15072876821921	10313.8132701499\\
1.15082877071927	10302.8124804834\\
1.15092877321933	10291.8689865964\\
1.15102877571939	10280.8681969299\\
1.15112877821946	10269.9247030429\\
1.15122878071952	10258.9812091559\\
1.15132878321958	10248.0377152689\\
1.15142878571964	10237.0942213819\\
1.15152878821971	10226.1507274949\\
1.15162879071977	10215.2072336079\\
1.15172879321983	10204.3210355004\\
1.15182879571989	10193.3775416135\\
1.15192879821996	10182.491343506\\
1.15202880072002	10171.547849619\\
1.15212880322008	10160.6616515115\\
1.15222880572014	10149.775453404\\
1.15232880822021	10138.8892552965\\
1.15242881072027	10128.003057189\\
1.15252881322033	10117.1168590815\\
1.15262881572039	10106.2879567536\\
1.15272881822046	10095.4017586461\\
1.15282882072052	10084.5728563181\\
1.15292882322058	10073.6866582106\\
1.15302882572064	10062.8577558826\\
1.15312882822071	10052.0288535547\\
1.15322883072077	10041.1999512267\\
1.15332883322083	10030.3710488987\\
1.15342883572089	10019.5421465708\\
1.15352883822096	10008.7132442428\\
1.15362884072102	9997.94163769433\\
1.15372884322108	9987.11273536635\\
1.15382884572114	9976.34112881789\\
1.15392884822121	9965.51222648992\\
1.15402885072127	9954.74061994146\\
1.15412885322133	9943.969013393\\
1.15422885572139	9933.19740684454\\
1.15432885822146	9922.42580029608\\
1.15442886072152	9911.65419374762\\
1.15452886322158	9900.93988297868\\
1.15462886572164	9890.16827643022\\
1.15472886822171	9879.45396566127\\
1.15482887072177	9868.68235911281\\
1.15492887322183	9857.96804834387\\
1.15502887572189	9847.25373757492\\
1.15512887822196	9836.53942680597\\
1.15522888072202	9825.82511603703\\
1.15532888322208	9815.11080526808\\
1.15542888572214	9804.39649449913\\
1.15552888822221	9793.7394795097\\
1.15562889072227	9783.02516874075\\
1.15572889322233	9772.36815375132\\
1.15582889572239	9761.65384298237\\
1.15592889822246	9750.99682799294\\
1.15602890072252	9740.33981300351\\
1.15612890322258	9729.68279801408\\
1.15622890572264	9719.02578302464\\
1.15632890822271	9708.36876803521\\
1.15642891072277	9697.76904882529\\
1.15652891322283	9687.11203383585\\
1.15662891572289	9676.51231462593\\
1.15672891822296	9665.8552996365\\
1.15682892072302	9655.25558042658\\
1.15692892322308	9644.65586121666\\
1.15702892572314	9634.05614200674\\
1.15712892822321	9623.45642279682\\
1.15722893072327	9612.8567035869\\
1.15732893322333	9602.25698437698\\
1.15742893572339	9591.65726516706\\
1.15752893822346	9581.11484173665\\
1.15762894072352	9570.51512252673\\
1.15772894322358	9559.97269909632\\
1.15782894572364	9549.43027566592\\
1.15792894822371	9538.88785223551\\
1.15802895072377	9528.3454288051\\
1.15812895322383	9517.8030053747\\
1.15822895572389	9507.26058194429\\
1.15832895822396	9496.71815851388\\
1.15842896072402	9486.23303086299\\
1.15852896322408	9475.69060743258\\
1.15862896572414	9465.20547978169\\
1.15872896822421	9454.72035213079\\
1.15882897072427	9444.17792870038\\
1.15892897322433	9433.69280104949\\
1.15902897572439	9423.2076733986\\
1.15912897822446	9412.77984152722\\
1.15922898072452	9402.29471387632\\
1.15932898322458	9391.80958622543\\
1.15942898572464	9381.38175435405\\
1.15952898822471	9370.89662670315\\
1.15962899072477	9360.46879483177\\
1.15972899322483	9350.04096296039\\
1.15982899572489	9339.5558353095\\
1.15992899822496	9329.12800343812\\
1.16002900072502	9318.75746734625\\
1.16012900322508	9308.32963547487\\
1.16022900572514	9297.90180360349\\
1.16032900822521	9287.47397173211\\
1.16042901072527	9277.10343564024\\
1.16052901322533	9266.73289954837\\
1.16062901572539	9256.30506767699\\
1.16072901822546	9245.93453158512\\
1.16082902072552	9235.56399549325\\
1.16092902322558	9225.19345940138\\
1.16102902572564	9214.82292330952\\
1.16112902822571	9204.45238721765\\
1.16122903072577	9194.13914690529\\
1.16132903322583	9183.76861081343\\
1.16142903572589	9173.45537050107\\
1.16152903822596	9163.0848344092\\
1.16162904072602	9152.77159409685\\
1.16172904322608	9142.45835378449\\
1.16182904572614	9132.14511347214\\
1.16192904822621	9121.83187315978\\
1.16202905072627	9111.51863284743\\
1.16212905322633	9101.20539253507\\
1.16222905572639	9090.94944800223\\
1.16232905822646	9080.63620768988\\
1.16242906072652	9070.38026315704\\
1.16252906322658	9060.12431862419\\
1.16262906572664	9049.81107831184\\
1.16272906822671	9039.555133779\\
1.16282907072677	9029.29918924616\\
1.16292907322683	9019.04324471332\\
1.16302907572689	9008.84459595999\\
1.16312907822696	8998.58865142714\\
1.16322908072702	8988.3327068943\\
1.16332908322708	8978.13405814097\\
1.16342908572714	8967.93540938765\\
1.16352908822721	8957.6794648548\\
1.16362909072727	8947.48081610148\\
1.16372909322733	8937.28216734815\\
1.16382909572739	8927.08351859482\\
1.16392909822746	8916.88486984149\\
1.16402910072752	8906.74351686767\\
1.16412910322758	8896.54486811434\\
1.16422910572764	8886.34621936102\\
1.16432910822771	8876.2048663872\\
1.16442911072777	8866.06351341338\\
1.16452911322783	8855.86486466006\\
1.16462911572789	8845.72351168624\\
1.16472911822796	8835.58215871243\\
1.16482912072802	8825.44080573861\\
1.16492912322808	8815.35674854431\\
1.16502912572814	8805.21539557049\\
1.16512912822821	8795.07404259667\\
1.16522913072827	8784.98998540237\\
1.16532913322833	8774.90592820807\\
1.16542913572839	8764.76457523426\\
1.16552913822846	8754.68051803995\\
1.16562914072852	8744.59646084565\\
1.16572914322858	8734.51240365135\\
1.16582914572864	8724.42834645705\\
1.16592914822871	8714.34428926274\\
1.16602915072877	8704.31752784795\\
1.16612915322883	8694.23347065365\\
1.16622915572889	8684.20670923886\\
1.16632915822896	8674.12265204456\\
1.16642916072902	8664.09589062977\\
1.16652916322908	8654.06912921498\\
1.16662916572914	8644.04236780019\\
1.16672916822921	8634.0156063854\\
1.16682917072927	8623.98884497061\\
1.16692917322933	8614.01937933534\\
1.16702917572939	8603.99261792055\\
1.16712917822946	8594.02315228527\\
1.16722918072952	8583.99639087048\\
1.16732918322958	8574.0269252352\\
1.16742918572964	8564.05745959993\\
1.16752918822971	8554.08799396465\\
1.16762919072977	8544.11852832937\\
1.16772919322983	8534.1490626941\\
1.16782919572989	8524.17959705882\\
1.16792919822996	8514.21013142355\\
1.16802920073002	8504.29796156778\\
1.16812920323008	8494.32849593251\\
1.16822920573014	8484.41632607674\\
1.16832920823021	8474.50415622098\\
1.16842921073027	8464.5346905857\\
1.16852921323033	8454.62252072994\\
1.16862921573039	8444.71035087418\\
1.16872921823046	8434.85547679793\\
1.16882922073052	8424.94330694216\\
1.16892922323058	8415.0311370864\\
1.16902922573064	8405.17626301015\\
1.16912922823071	8395.26409315439\\
1.16922923073077	8385.40921907814\\
1.16932923323083	8375.55434500189\\
1.16942923573089	8365.69947092564\\
1.16952923823096	8355.84459684939\\
1.16962924073102	8345.98972277314\\
1.16972924323108	8336.13484869689\\
1.16982924573114	8326.27997462064\\
1.16992924823121	8316.4823963239\\
1.17002925073127	8306.62752224765\\
1.17012925323133	8296.82994395091\\
1.17022925573139	8287.03236565418\\
1.17032925823146	8277.17749157792\\
1.17042926073152	8267.37991328119\\
1.17052926323158	8257.58233498445\\
1.17062926573164	8247.84205246723\\
1.17072926823171	8238.04447417049\\
1.17082927073177	8228.24689587375\\
1.17092927323183	8218.50661335653\\
1.17102927573189	8208.70903505979\\
1.17112927823196	8198.96875254257\\
1.17122928073202	8189.22847002534\\
1.17132928323208	8179.43089172861\\
1.17142928573214	8169.69060921138\\
1.17152928823221	8159.95032669416\\
1.17162929073227	8150.26733995645\\
1.17172929323233	8140.52705743922\\
1.17182929573239	8130.786774922\\
1.17192929823246	8121.10378818429\\
1.17202930073252	8111.36350566706\\
1.17212930323258	8101.68051892935\\
1.17222930573264	8091.99753219164\\
1.17232930823271	8082.31454545393\\
1.17242931073277	8072.63155871622\\
1.17252931323283	8062.94857197851\\
1.17262931573289	8053.2655852408\\
1.17272931823296	8043.6398942826\\
1.17282932073302	8033.95690754489\\
1.17292932323308	8024.27392080718\\
1.17302932573314	8014.64822984898\\
1.17312932823321	8005.02253889078\\
1.17322933073327	7995.39684793259\\
1.17332933323333	7985.77115697439\\
1.17342933573339	7976.14546601619\\
1.17352933823346	7966.51977505799\\
1.17362934073352	7956.89408409979\\
1.17372934323358	7947.2683931416\\
1.17382934573364	7937.69999796291\\
1.17392934823371	7928.13160278423\\
1.17402935073377	7918.50591182603\\
1.17412935323383	7908.93751664734\\
1.17422935573389	7899.36912146866\\
1.17432935823396	7889.80072628998\\
1.17442936073402	7880.23233111129\\
1.17452936323408	7870.66393593261\\
1.17462936573414	7861.09554075392\\
1.17472936823421	7851.58444135475\\
1.17482937073427	7842.01604617606\\
1.17492937323433	7832.50494677689\\
1.17502937573439	7822.99384737772\\
1.17512937823446	7813.42545219904\\
1.17522938073452	7803.91435279986\\
1.17532938323458	7794.40325340069\\
1.17542938573464	7784.94944978103\\
1.17552938823471	7775.43835038186\\
1.17562939073477	7765.92725098269\\
1.17572939323483	7756.47344736303\\
1.17582939573489	7746.96234796386\\
1.17592939823496	7737.5085443442\\
1.17602940073502	7727.99744494503\\
1.17612940323508	7718.54364132537\\
1.17622940573514	7709.08983770571\\
1.17632940823521	7699.63603408605\\
1.17642941073527	7690.1822304664\\
1.17652941323533	7680.78572262625\\
1.17662941573539	7671.33191900659\\
1.17672941823546	7661.93541116645\\
1.17682942073552	7652.48160754679\\
1.17692942323558	7643.08509970664\\
1.17702942573564	7633.6885918665\\
1.17712942823571	7624.29208402635\\
1.17722943073577	7614.89557618621\\
1.17732943323583	7605.49906834606\\
1.17742943573589	7596.10256050592\\
1.17752943823596	7586.70605266577\\
1.17762944073602	7577.36684060514\\
1.17772944323608	7567.97033276499\\
1.17782944573614	7558.63112070436\\
1.17792944823621	7549.23461286421\\
1.17802945073627	7539.89540080358\\
1.17812945323633	7530.55618874295\\
1.17822945573639	7521.21697668232\\
1.17832945823646	7511.87776462168\\
1.17842946073652	7502.59584834056\\
1.17852946323658	7493.25663627993\\
1.17862946573664	7483.9174242193\\
1.17872946823671	7474.63550793818\\
1.17882947073677	7465.35359165706\\
1.17892947323683	7456.01437959643\\
1.17902947573689	7446.73246331531\\
1.17912947823696	7437.45054703419\\
1.17922948073702	7428.16863075307\\
1.17932948323708	7418.88671447195\\
1.17942948573714	7409.66209397035\\
1.17952948823721	7400.38017768923\\
1.17962949073727	7391.09826140811\\
1.17972949323733	7381.8736409065\\
1.17982949573739	7372.64902040489\\
1.17992949823746	7363.36710412377\\
1.18002950073752	7354.14248362217\\
1.18012950323758	7344.91786312056\\
1.18022950573764	7335.69324261896\\
1.18032950823771	7326.52591789686\\
1.18042951073777	7317.30129739526\\
1.18052951323783	7308.07667689365\\
1.18062951573789	7298.90935217156\\
1.18072951823796	7289.74202744946\\
1.18082952073802	7280.51740694786\\
1.18092952323808	7271.35008222577\\
1.18102952573814	7262.18275750367\\
1.18112952823821	7253.01543278158\\
1.18122953073827	7243.84810805949\\
1.18132953323833	7234.68078333739\\
1.18142953573839	7225.57075439481\\
1.18152953823846	7216.40342967272\\
1.18162954073852	7207.29340073014\\
1.18172954323858	7198.12607600805\\
1.18182954573864	7189.01604706546\\
1.18192954823871	7179.90601812289\\
1.18202955073877	7170.79598918031\\
1.18212955323883	7161.68596023773\\
1.18222955573889	7152.57593129514\\
1.18232955823896	7143.46590235256\\
1.18242956073902	7134.4131691895\\
1.18252956323908	7125.30314024692\\
1.18262956573914	7116.25040708385\\
1.18272956823921	7107.19767392078\\
1.18282957073927	7098.0876449782\\
1.18292957323933	7089.03491181514\\
1.18302957573939	7079.98217865207\\
1.18312957823946	7070.929445489\\
1.18322958073952	7061.87671232594\\
1.18332958323958	7052.88127494238\\
1.18342958573964	7043.82854177931\\
1.18352958823971	7034.83310439576\\
1.18362959073977	7025.78037123269\\
1.18372959323983	7016.78493384914\\
1.18382959573989	7007.78949646559\\
1.18392959823996	6998.79405908203\\
1.18402960074002	6989.79862169848\\
1.18412960324008	6980.80318431492\\
1.18422960574014	6971.80774693137\\
1.18432960824021	6962.81230954782\\
1.18442961074027	6953.87416794377\\
1.18452961324033	6944.87873056022\\
1.18462961574039	6935.94058895618\\
1.18472961824046	6927.00244735214\\
1.18482962074052	6918.0643057481\\
1.18492962324058	6909.12616414406\\
1.18502962574064	6900.18802254002\\
1.18512962824071	6891.24988093598\\
1.18522963074077	6882.31173933194\\
1.18532963324083	6873.43089350741\\
1.18542963574089	6864.49275190337\\
1.18552963824096	6855.61190607884\\
1.18562964074102	6846.6737644748\\
1.18572964324108	6837.79291865027\\
1.18582964574114	6828.91207282574\\
1.18592964824121	6820.03122700121\\
1.18602965074127	6811.15038117669\\
1.18612965324133	6802.26953535216\\
1.18622965574139	6793.38868952763\\
1.18632965824146	6784.56513948262\\
1.18642966074152	6775.68429365809\\
1.18652966324158	6766.86074361308\\
1.18662966574164	6758.03719356806\\
1.18672966824171	6749.15634774353\\
1.18682967074177	6740.33279769852\\
1.18692967324183	6731.5092476535\\
1.18702967574189	6722.742993388\\
1.18712967824196	6713.91944334299\\
1.18722968074202	6705.09589329797\\
1.18732968324208	6696.27234325296\\
1.18742968574214	6687.50608898745\\
1.18752968824221	6678.73983472195\\
1.18762969074227	6669.91628467694\\
1.18772969324233	6661.15003041144\\
1.18782969574239	6652.38377614594\\
1.18792969824246	6643.61752188043\\
1.18802970074252	6634.85126761493\\
1.18812970324258	6626.14230912894\\
1.18822970574264	6617.37605486344\\
1.18832970824271	6608.60980059794\\
1.18842971074277	6599.90084211195\\
1.18852971324283	6591.19188362596\\
1.18862971574289	6582.42562936046\\
1.18872971824296	6573.71667087448\\
1.18882972074302	6565.00771238849\\
1.18892972324308	6556.2987539025\\
1.18902972574314	6547.64709119602\\
1.18912972824321	6538.93813271003\\
1.18922973074327	6530.22917422404\\
1.18932973324333	6521.57751151757\\
1.18942973574339	6512.86855303158\\
1.18952973824346	6504.2168903251\\
1.18962974074352	6495.56522761863\\
1.18972974324358	6486.91356491216\\
1.18982974574364	6478.26190220568\\
1.18992974824371	6469.6102394992\\
1.19002975074377	6460.95857679273\\
1.19012975324383	6452.30691408625\\
1.19022975574389	6443.71254715929\\
1.19032975824396	6435.06088445282\\
1.19042976074402	6426.46651752585\\
1.19052976324408	6417.87215059889\\
1.19062976574414	6409.22048789241\\
1.19072976824421	6400.62612096545\\
1.19082977074427	6392.03175403849\\
1.19092977324433	6383.43738711153\\
1.19102977574439	6374.90031596408\\
1.19112977824446	6366.30594903712\\
1.19122978074452	6357.71158211015\\
1.19132978324458	6349.1745109627\\
1.19142978574464	6340.63743981526\\
1.19152978824471	6332.04307288829\\
1.19162979074477	6323.50600174084\\
1.19172979324483	6314.9689305934\\
1.19182979574489	6306.43185944594\\
1.19192979824496	6297.8947882985\\
1.19202980074502	6289.41501293056\\
1.19212980324508	6280.87794178311\\
1.19222980574514	6272.39816641517\\
1.19232980824521	6263.86109526772\\
1.19242981074527	6255.38131989979\\
1.19252981324533	6246.90154453185\\
1.19262981574539	6238.3644733844\\
1.19272981824546	6229.88469801647\\
1.19282982074552	6221.46221842804\\
1.19292982324558	6212.98244306011\\
1.19302982574564	6204.50266769217\\
1.19312982824571	6196.02289232424\\
1.19322983074577	6187.60041273581\\
1.19332983324583	6179.17793314739\\
1.19342983574589	6170.69815777945\\
1.19352983824596	6162.27567819103\\
1.19362984074602	6153.85319860261\\
1.19372984324608	6145.43071901418\\
1.19382984574614	6137.00823942576\\
1.19392984824621	6128.58575983734\\
1.19402985074627	6120.22057602843\\
1.19412985324633	6111.79809644\\
1.19422985574639	6103.43291263109\\
1.19432985824646	6095.01043304267\\
1.19442986074652	6086.64524923376\\
1.19452986324658	6078.28006542485\\
1.19462986574664	6069.91488161594\\
1.19472986824671	6061.54969780703\\
1.19482987074677	6053.18451399812\\
1.19492987324683	6044.81933018921\\
1.19502987574689	6036.4541463803\\
1.19512987824696	6028.1462583509\\
1.19522988074702	6019.78107454199\\
1.19532988324708	6011.4731865126\\
1.19542988574714	6003.1652984832\\
1.19552988824721	5994.8574104538\\
1.19562989074727	5986.54952242441\\
1.19572989324733	5978.24163439501\\
1.19582989574739	5969.93374636561\\
1.19592989824746	5961.62585833622\\
1.19602990074752	5953.31797030682\\
1.19612990324758	5945.06737805694\\
1.19622990574764	5936.81678580705\\
1.19632990824771	5928.50889777765\\
1.19642991074777	5920.25830552777\\
1.19652991324783	5912.00771327789\\
1.19662991574789	5903.757121028\\
1.19672991824796	5895.50652877812\\
1.19682992074802	5887.25593652823\\
1.19692992324808	5879.00534427835\\
1.19702992574814	5870.81204780798\\
1.19712992824821	5862.5614555581\\
1.19722993074827	5854.36815908773\\
1.19732993324833	5846.17486261735\\
1.19742993574839	5837.92427036747\\
1.19752993824846	5829.7309738971\\
1.19762994074852	5821.53767742673\\
1.19772994324858	5813.34438095636\\
1.19782994574864	5805.2083802655\\
1.19792994824871	5797.01508379513\\
1.19802995074877	5788.82178732476\\
1.19812995324883	5780.6857866339\\
1.19822995574889	5772.54978594304\\
1.19832995824896	5764.35648947267\\
1.19842996074902	5756.22048878182\\
1.19852996324908	5748.08448809096\\
1.19862996574914	5739.9484874001\\
1.19872996824921	5731.81248670924\\
1.19882997074927	5723.70513390814\\
1.19892997324933	5715.58632195114\\
1.19902997574939	5707.47896915004\\
1.19912997824946	5699.37161634894\\
1.19922998074952	5691.27572270374\\
1.19932998324958	5683.17982905854\\
1.19942998574964	5675.09539456924\\
1.19952998824971	5667.01096007995\\
1.19962999074977	5658.93798474655\\
1.19972999324983	5650.87073899111\\
1.19982999574989	5642.80349323567\\
1.19992999824996	5634.74770663613\\
1.20003000075002	5626.69764961454\\
};
\addplot [color=mycolor1,solid,forget plot]
  table[row sep=crcr]{%
1.20003000075002	5626.69764961454\\
1.20013000325008	5618.64759259295\\
1.20023000575014	5610.60899472727\\
1.20033000825021	5602.57612643953\\
1.20043001075027	5594.54898772975\\
1.20053001325033	5586.52184901997\\
1.20063001575039	5578.50616946609\\
1.20073001825046	5570.49621949016\\
1.20083002075052	5562.49199909218\\
1.20093002325058	5554.49350827215\\
1.20103002575064	5546.50074703008\\
1.20113002825071	5538.51371536596\\
1.20123003075077	5530.53241327978\\
1.20133003325083	5522.55684077156\\
1.20143003575089	5514.58699784129\\
1.20153003825096	5506.62288448897\\
1.20163004075102	5498.66450071461\\
1.20173004325108	5490.71184651819\\
1.20183004575114	5482.76492189973\\
1.20193004825121	5474.82372685921\\
1.20203005075127	5466.88826139665\\
1.20213005325133	5458.95852551204\\
1.20223005575139	5451.04024878333\\
1.20233005825146	5443.12197205463\\
1.20243006075152	5435.20942490387\\
1.20253006325158	5427.30260733106\\
1.20263006575164	5419.40724891416\\
1.20273006825171	5411.51189049726\\
1.20283007075177	5403.62226165831\\
1.20293007325183	5395.74409197526\\
1.20303007575189	5387.86592229221\\
1.20313007825196	5379.99348218711\\
1.20323008075202	5372.13250123792\\
1.20333008325208	5364.27152028872\\
1.20343008575214	5356.42199849543\\
1.20353008825221	5348.57247670214\\
1.20363009075227	5340.73441406475\\
1.20373009325233	5332.89635142736\\
1.20383009575239	5325.06974794587\\
1.20393009825246	5317.24314446438\\
1.20403010075252	5309.4280001388\\
1.20413010325258	5301.61285581322\\
1.20423010575264	5293.80917064353\\
1.20433010825271	5286.0112150518\\
1.20443011075277	5278.21325946007\\
1.20453011325283	5270.42676302424\\
1.20463011575289	5262.64599616637\\
1.20473011825296	5254.86522930849\\
1.20483012075302	5247.09592160652\\
1.20493012325308	5239.3323434825\\
1.20503012575314	5231.57449493642\\
1.20513012825321	5223.8223759683\\
1.20523013075327	5216.07025700018\\
1.20533013325333	5208.32959718797\\
1.20543013575339	5200.5946669537\\
1.20553013825346	5192.86546629738\\
1.20563014075352	5185.14199521902\\
1.20573014325358	5177.42425371861\\
1.20583014575364	5169.71224179615\\
1.20593014825371	5162.00595945164\\
1.20603015075377	5154.30540668508\\
1.20613015325383	5146.61058349647\\
1.20623015575389	5138.92148988582\\
1.20633015825396	5131.23812585311\\
1.20643016075402	5123.56049139836\\
1.20653016325408	5115.88858652156\\
1.20663016575414	5108.22241122271\\
1.20673016825421	5100.56196550181\\
1.20683017075427	5092.91297893681\\
1.20693017325433	5085.26399237182\\
1.20703017575439	5077.62073538477\\
1.20713017825446	5069.98320797568\\
1.20723018075452	5062.35141014453\\
1.20733018325458	5054.73107146929\\
1.20743018575464	5047.11073279406\\
1.20753018825471	5039.49612369677\\
1.20763019075477	5031.89297375538\\
1.20773019325483	5024.28982381399\\
1.20783019575489	5016.69240345056\\
1.20793019825496	5009.10644224303\\
1.20803020075502	5001.5204810355\\
1.20813020325508	4993.94597898387\\
1.20823020575514	4986.37147693224\\
1.20833020825521	4978.80270445856\\
1.20843021075527	4971.24539114078\\
1.20853021325533	4963.68807782301\\
1.20863021575539	4956.14222366113\\
1.20873021825546	4948.59636949926\\
1.20883022075552	4941.06197449329\\
1.20893022325558	4933.53330906527\\
1.20903022575564	4926.00464363725\\
1.20913022825571	4918.48743736514\\
1.20923023075577	4910.97023109302\\
1.20933023325583	4903.46448397681\\
1.20943023575589	4895.96446643854\\
1.20953023825596	4888.46444890028\\
1.20963024075602	4880.97589051792\\
1.20973024325608	4873.49306171351\\
1.20983024575614	4866.01596248705\\
1.20993024825621	4858.5388632606\\
1.21003025075627	4851.07322319004\\
1.21013025325633	4843.61331269744\\
1.21023025575639	4836.15913178279\\
1.21033025825646	4828.71068044609\\
1.21043026075652	4821.26222910939\\
1.21053026325658	4813.82523692859\\
1.21063026575664	4806.39397432574\\
1.21073026825671	4798.96844130085\\
1.21083027075677	4791.5486378539\\
1.21093027325683	4784.13456398491\\
1.21103027575689	4776.72621969387\\
1.21113027825696	4769.32360498078\\
1.21123028075702	4761.92671984564\\
1.21133028325708	4754.53556428845\\
1.21143028575714	4747.15013830921\\
1.21153028825721	4739.77044190793\\
1.21163029075727	4732.3964750846\\
1.21173029325733	4725.02823783921\\
1.21183029575739	4717.66573017178\\
1.21193029825746	4710.3089520823\\
1.21203030075752	4702.96363314873\\
1.21213030325758	4695.61831421515\\
1.21223030575764	4688.27872485952\\
1.21233030825771	4680.94486508185\\
1.21243031075777	4673.61673488212\\
1.21253031325783	4666.29433426035\\
1.21263031575789	4658.98339279448\\
1.21273031825796	4651.67245132861\\
1.21283032075802	4644.3672394407\\
1.21293032325808	4637.07348670868\\
1.21303032575814	4629.77973397667\\
1.21313032825821	4622.4917108226\\
1.21323033075827	4615.20941724649\\
1.21333033325833	4607.93858282628\\
1.21343033575839	4600.66774840607\\
1.21353033825846	4593.40837314176\\
1.21363034075852	4586.14899787745\\
1.21373034325858	4578.8953521911\\
1.21383034575864	4571.65316566064\\
1.21393034825871	4564.41097913019\\
1.21403035075877	4557.18025175564\\
1.21413035325883	4549.94952438109\\
1.21423035575889	4542.73025616244\\
1.21433035825896	4535.51098794379\\
1.21443036075902	4528.30317888105\\
1.21453036325908	4521.0953698183\\
1.21463036575914	4513.89901991146\\
1.21473036825921	4506.70839958256\\
1.21483037075927	4499.51777925367\\
1.21493037325933	4492.33861808068\\
1.21503037575939	4485.1594569077\\
1.21513037825946	4477.99175489061\\
1.21523038075952	4470.82978245147\\
1.21533038325958	4463.66781001234\\
1.21543038575964	4456.5172967291\\
1.21553038825971	4449.37251302382\\
1.21563039075977	4442.23345889649\\
1.21573039325983	4435.09440476916\\
1.21583039575989	4427.96680979774\\
1.21593039825996	4420.84494440426\\
1.21603040076002	4413.72880858873\\
1.21613040326008	4406.61840235116\\
1.21623040576014	4399.50799611359\\
1.21633040826021	4392.40904903192\\
1.21643041076027	4385.3158315282\\
1.21653041326033	4378.22834360243\\
1.21663041576039	4371.14658525461\\
1.21673041826046	4364.07055648475\\
1.21683042076052	4357.00025729283\\
1.21693042326058	4349.93568767887\\
1.21703042576064	4342.87684764286\\
1.21713042826071	4335.81800760685\\
1.21723043076077	4328.77062672674\\
1.21733043326083	4321.72897542458\\
1.21743043576089	4314.69305370037\\
1.21753043826096	4307.66859113207\\
1.21763044076102	4300.64412856376\\
1.21773044326108	4293.62539557341\\
1.21783044576114	4286.61239216101\\
1.21793044826121	4279.60511832656\\
1.21803045076127	4272.60357407006\\
1.21813045326133	4265.60775939151\\
1.21823045576139	4258.61767429092\\
1.21833045826146	4251.63331876827\\
1.21843046076152	4244.65469282358\\
1.21853046326158	4237.68752603479\\
1.21863046576164	4230.720359246\\
1.21873046826171	4223.75892203516\\
1.21883047076177	4216.80321440227\\
1.21893047326183	4209.85323634734\\
1.21903047576189	4202.9147174483\\
1.21913047826196	4195.97619854927\\
1.21923048076202	4189.04340922818\\
1.21933048326208	4182.11634948505\\
1.21943048576214	4175.20074889782\\
1.21953048826221	4168.28514831059\\
1.21963049076227	4161.37527730131\\
1.21973049326233	4154.47686544794\\
1.21983049576239	4147.57845359456\\
1.21993049826246	4140.68577131914\\
1.22003050076252	4133.80454819962\\
1.22013050326258	4126.9233250801\\
1.22023050576264	4120.05356111648\\
1.22033050826271	4113.18379715286\\
1.22043051076277	4106.31976276719\\
1.22053051326283	4099.46718753743\\
1.22063051576289	4092.61461230767\\
1.22073051826296	4085.7734962338\\
1.22083052076302	4078.93238015994\\
1.22093052326308	4072.10272324198\\
1.22103052576314	4065.27306632402\\
1.22113052826321	4058.45486856197\\
1.22123053076327	4051.63667079991\\
1.22133053326333	4044.82993219375\\
1.22143053576339	4038.02892316555\\
1.22153053826346	4031.22791413735\\
1.22163054076352	4024.43836426505\\
1.22173054326358	4017.64881439275\\
1.22183054576364	4010.87072367635\\
1.22193054826371	4004.0983625379\\
1.22203055076377	3997.32600139946\\
1.22213055326383	3990.56509941691\\
1.22223055576389	3983.80992701232\\
1.22233055826396	3977.05475460773\\
1.22243056076402	3970.31104135904\\
1.22253056326408	3963.5730576883\\
1.22263056576414	3956.83507401756\\
1.22273056826421	3950.10854950273\\
1.22283057076427	3943.38775456584\\
1.22293057326433	3936.67268920691\\
1.22303057576439	3929.95762384798\\
1.22313057826446	3923.25401764494\\
1.22323058076452	3916.55614101987\\
1.22333058326458	3909.86399397274\\
1.22343058576464	3903.17757650356\\
1.22353058826471	3896.49115903438\\
1.22363059076477	3889.81620072111\\
1.22373059326483	3883.14697198579\\
1.22383059576489	3876.48347282842\\
1.22393059826496	3869.825703249\\
1.22403060076502	3863.17366324753\\
1.22413060326508	3856.52735282401\\
1.22423060576514	3849.88677197844\\
1.22433060826521	3843.25192071083\\
1.22443061076527	3836.61706944321\\
1.22453061326533	3829.9936773315\\
1.22463061576539	3823.37601479774\\
1.22473061826546	3816.76408184193\\
1.22483062076552	3810.15787846407\\
1.22493062326558	3803.55740466417\\
1.22503062576564	3796.96266044221\\
1.22513062826571	3790.3736457982\\
1.22523063076577	3783.79036073215\\
1.22533063326583	3777.21280524405\\
1.22543063576589	3770.6409793339\\
1.22553063826596	3764.08061257965\\
1.22563064076602	3757.5202458254\\
1.22573064326608	3750.96560864911\\
1.22583064576614	3744.41670105076\\
1.22593064826621	3737.87352303037\\
1.22603065076627	3731.33607458792\\
1.22613065326633	3724.80435572343\\
1.22623065576639	3718.27836643689\\
1.22633065826646	3711.7581067283\\
1.22643066076652	3705.24930617562\\
1.22653066326658	3698.74050562293\\
1.22663066576664	3692.2374346482\\
1.22673066826671	3685.74009325141\\
1.22683067076677	3679.24848143258\\
1.22693067326683	3672.76832876965\\
1.22703067576689	3666.28817610672\\
1.22713067826696	3659.81375302174\\
1.22723068076702	3653.34505951472\\
1.22733068326708	3646.88782516359\\
1.22743068576714	3640.43059081247\\
1.22753068826721	3633.97908603929\\
1.22763069076727	3627.53331084407\\
1.22773069326733	3621.09899480475\\
1.22783069576739	3614.66467876543\\
1.22793069826746	3608.23609230407\\
1.22803070076752	3601.8189649986\\
1.22813070326758	3595.40183769314\\
1.22823070576764	3588.99043996562\\
1.22833070826771	3582.59050139401\\
1.22843071076777	3576.1905628224\\
1.22853071326783	3569.80208340669\\
1.22863071576789	3563.41360399098\\
1.22873071826796	3557.03085415323\\
1.22883072076802	3550.65956347137\\
1.22893072326808	3544.28827278952\\
1.22903072576814	3537.92844126356\\
1.22913072826821	3531.56860973761\\
1.22923073076827	3525.22023736756\\
1.22933073326833	3518.87186499751\\
1.22943073576839	3512.52922220541\\
1.22953073826846	3506.19803856922\\
1.22963074076852	3499.86685493302\\
1.22973074326858	3493.54713045273\\
1.22983074576864	3487.22740597244\\
1.22993074826871	3480.91914064805\\
1.23003075076877	3474.61660490161\\
1.23013075326883	3468.31406915517\\
1.23023075576889	3462.02299256463\\
1.23033075826896	3455.7319159741\\
1.23043076076902	3449.45229853946\\
1.23053076326908	3443.17268110483\\
1.23063076576914	3436.9045228261\\
1.23073076826921	3430.63636454737\\
1.23083077076927	3424.37966542454\\
1.23093077326933	3418.12869587966\\
1.23103077576939	3411.87772633478\\
1.23113077826946	3405.63821594581\\
1.23123078076952	3399.40443513478\\
1.23133078326958	3393.17065432376\\
1.23143078576964	3386.94833266864\\
1.23153078826971	3380.73174059147\\
1.23163079076977	3374.5151485143\\
1.23173079326983	3368.31001559304\\
1.23183079576989	3362.11061224972\\
1.23193079826996	3355.9112089064\\
1.23203080077002	3349.72326471899\\
1.23213080327008	3343.54105010953\\
1.23223080577014	3337.35883550007\\
1.23233080827021	3331.18808004651\\
1.23243081077027	3325.0230541709\\
1.23253081327033	3318.86375787325\\
1.23263081577039	3312.70446157559\\
1.23273081827046	3306.55662443383\\
1.23283082077052	3300.41451687003\\
1.23293082327058	3294.27813888418\\
1.23303082577064	3288.14749047628\\
1.23313082827071	3282.01684206838\\
1.23323083077077	3275.89765281638\\
1.23333083327083	3269.78419314234\\
1.23343083577089	3263.67646304624\\
1.23353083827096	3257.5744625281\\
1.23363084077102	3251.47246200996\\
1.23373084327108	3245.38192064772\\
1.23383084577114	3239.29710886343\\
1.23393084827121	3233.21802665709\\
1.23403085077127	3227.1446740287\\
1.23413085327133	3221.07705097827\\
1.23423085577139	3215.01515750578\\
1.23433085827146	3208.95899361125\\
1.23443086077152	3202.90282971672\\
1.23453086327158	3196.85812497809\\
1.23463086577164	3190.81914981741\\
1.23473086827171	3184.78590423468\\
1.23483087077177	3178.7583882299\\
1.23493087327183	3172.73660180308\\
1.23503087577189	3166.72054495421\\
1.23513087827196	3160.71021768328\\
1.23523088077202	3154.70561999031\\
1.23533088327208	3148.70675187529\\
1.23543088577214	3142.71361333822\\
1.23553088827221	3136.72620437911\\
1.23563089077227	3130.74452499794\\
1.23573089327233	3124.76857519473\\
1.23583089577239	3118.79835496946\\
1.23593089827246	3112.83386432215\\
1.23603090077252	3106.87510325279\\
1.23613090327258	3100.92207176138\\
1.23623090577264	3094.97476984792\\
1.23633090827271	3089.03319751242\\
1.23643091077277	3083.09735475486\\
1.23653091327283	3077.16724157526\\
1.23663091577289	3071.24285797361\\
1.23673091827296	3065.3242039499\\
1.23683092077302	3059.41127950415\\
1.23693092327308	3053.50408463636\\
1.23703092577314	3047.60261934651\\
1.23713092827321	3041.70688363461\\
1.23723093077327	3035.82260707862\\
1.23733093327333	3029.93833052262\\
1.23743093577339	3024.05978354458\\
1.23753093827346	3018.18696614449\\
1.23763094077352	3012.31987832235\\
1.23773094327358	3006.45852007816\\
1.23783094577364	3000.60289141193\\
1.23793094827371	2994.75299232364\\
1.23803095077377	2988.91455239126\\
1.23813095327383	2983.07611245887\\
1.23823095577389	2977.24340210444\\
1.23833095827396	2971.41642132796\\
1.23843096077402	2965.59517012943\\
1.23853096327408	2959.77964850886\\
1.23863096577414	2953.97558604418\\
1.23873096827421	2948.1715235795\\
1.23883097077427	2942.37319069278\\
1.23893097327433	2936.58058738401\\
1.23903097577439	2930.79371365319\\
1.23913097827446	2925.01829907827\\
1.23923098077452	2919.24288450335\\
1.23933098327458	2913.47319950638\\
1.23943098577464	2907.70924408737\\
1.23953098827471	2901.9510182463\\
1.23963099077477	2896.20425156114\\
1.23973099327483	2890.45748487598\\
1.23983099577489	2884.71644776877\\
1.23993099827496	2878.98114023951\\
1.24003100077502	2873.25729186615\\
1.24013100327508	2867.53344349279\\
1.24023100577514	2861.81532469739\\
1.24033100827521	2856.10293547993\\
1.24043101077527	2850.40200541838\\
1.24053101327533	2844.70107535683\\
1.24063101577539	2839.00587487323\\
1.24073101827546	2833.32213354553\\
1.24083102077552	2827.63839221783\\
1.24093102327558	2821.96038046809\\
1.24103102577564	2816.29382787424\\
1.24113102827571	2810.6272752804\\
1.24123103077577	2804.96645226451\\
1.24133103327583	2799.31135882657\\
1.24143103577589	2793.66772454453\\
1.24153103827596	2788.02409026249\\
1.24163104077602	2782.3861855584\\
1.24173104327608	2776.75974001022\\
1.24183104577614	2771.13329446203\\
1.24193104827621	2765.51830806975\\
1.24203105077627	2759.90332167747\\
1.24213105327633	2754.29406486314\\
1.24223105577639	2748.69626720471\\
1.24233105827646	2743.09846954628\\
1.24243106077652	2737.5064014658\\
1.24253106327658	2731.92579254123\\
1.24263106577664	2726.34518361666\\
1.24273106827671	2720.77603384798\\
1.24283107077677	2715.20688407931\\
1.24293107327683	2709.64346388859\\
1.24303107577689	2704.09150285377\\
1.24313107827696	2698.53954181896\\
1.24323108077702	2692.99903994004\\
1.24333108327708	2687.45853806113\\
1.24343108577714	2681.92376576016\\
1.24353108827721	2676.4004526151\\
1.24363109077727	2670.87713947004\\
1.24373109327733	2665.36528548088\\
1.24383109577739	2659.85343149172\\
1.24393109827746	2654.34730708052\\
1.24403110077752	2648.85264182521\\
1.24413110327758	2643.35797656991\\
1.24423110577764	2637.87477047051\\
1.24433110827771	2632.3915643711\\
1.24443111077777	2626.9198174276\\
1.24453111327783	2621.4480704841\\
1.24463111577789	2615.98778269651\\
1.24473111827796	2610.52749490891\\
1.24483112077802	2605.07866627722\\
1.24493112327808	2599.62983764552\\
1.24503112577814	2594.19246816973\\
1.24513112827821	2588.75509869394\\
1.24523113077827	2583.3234587961\\
1.24533113327833	2577.90327805416\\
1.24543113577839	2572.48309731222\\
1.24553113827846	2567.07437572619\\
1.24563114077852	2561.66565414015\\
1.24573114327858	2556.26839171002\\
1.24583114577864	2550.87112927989\\
1.24593114827871	2545.48532600566\\
1.24603115077877	2540.09952273143\\
1.24613115327883	2534.7251786131\\
1.24623115577889	2529.35656407273\\
1.24633115827896	2523.98794953235\\
1.24643116077902	2518.63079414788\\
1.24653116327908	2513.2736387634\\
1.24663116577914	2507.92794253483\\
1.24673116827921	2502.58224630626\\
1.24683117077927	2497.2480092336\\
1.24693117327933	2491.91377216093\\
1.24703117577939	2486.59099424416\\
1.24713117827946	2481.2682163274\\
1.24723118077952	2475.95689756653\\
1.24733118327958	2470.64557880567\\
1.24743118577964	2465.34571920071\\
1.24753118827971	2460.04585959575\\
1.24763119077977	2454.75745914669\\
1.24773119327983	2449.47478827559\\
1.24783119577989	2444.19211740448\\
1.24793119827996	2438.92090568928\\
1.24803120078002	2433.64969397407\\
1.24813120328008	2428.38994141477\\
1.24823120578014	2423.13018885547\\
1.24833120828021	2417.88189545207\\
1.24843121078027	2412.63933162663\\
1.24853121328033	2407.39676780118\\
1.24863121578039	2402.16566313164\\
1.24873121828046	2396.93455846209\\
1.24883122078052	2391.71491294845\\
1.24893122328058	2386.49526743481\\
1.24903122578064	2381.28708107707\\
1.24913122828071	2376.08462429728\\
1.24923123078077	2370.88216751749\\
1.24933123328083	2365.69116989361\\
1.24943123578089	2360.50017226972\\
1.24953123828096	2355.32063380174\\
1.24963124078102	2350.14682491171\\
1.24973124328108	2344.97301602168\\
1.24983124578114	2339.81066628755\\
1.24993124828121	2334.64831655342\\
1.25003125078127	2329.49742597519\\
1.25013125328133	2324.35226497492\\
1.25023125578139	2319.20710397464\\
1.25033125828146	2314.07340213027\\
1.25043126078152	2308.9397002859\\
1.25053126328158	2303.81745759743\\
1.25063126578164	2298.70094448691\\
1.25073126828171	2293.58443137639\\
1.25083127078177	2288.47937742178\\
1.25093127328183	2283.37432346716\\
1.25103127578189	2278.28072866845\\
1.25113127828196	2273.19286344769\\
1.25123128078202	2268.10499822693\\
1.25133128328208	2263.02859216207\\
1.25143128578214	2257.95791567516\\
1.25153128828221	2252.88723918825\\
1.25163129078227	2247.82802185725\\
1.25173129328233	2242.76880452624\\
1.25183129578239	2237.72104635114\\
1.25193129828246	2232.67901775399\\
1.25203130078252	2227.63698915684\\
1.25213130328258	2222.60641971559\\
1.25223130578264	2217.58157985229\\
1.25233130828271	2212.55673998899\\
1.25243131078277	2207.5433592816\\
1.25253131328283	2202.5299785742\\
1.25263131578289	2197.52805702271\\
1.25273131828296	2192.53186504917\\
1.25283132078302	2187.53567307563\\
1.25293132328308	2182.55094025799\\
1.25303132578314	2177.5719370183\\
1.25313132828321	2172.59293377862\\
1.25323133078327	2167.62538969483\\
1.25333133328333	2162.663575189\\
1.25343133578339	2157.70176068317\\
1.25353133828346	2152.75140533324\\
1.25363134078352	2147.80104998331\\
1.25373134328358	2142.86215378928\\
1.25383134578364	2137.9289871732\\
1.25393134828371	2132.99582055713\\
1.25403135078377	2128.07411309695\\
1.25413135328383	2123.15813521473\\
1.25423135578389	2118.24215733251\\
1.25433135828396	2113.33763860619\\
1.25443136078402	2108.43884945782\\
1.25453136328408	2103.54006030945\\
1.25463136578414	2098.65273031698\\
1.25473136828421	2093.77112990247\\
1.25483137078427	2088.88952948796\\
1.25493137328433	2084.01938822934\\
1.25503137578439	2079.14924697073\\
1.25513137828446	2074.29056486802\\
1.25523138078452	2069.43761234326\\
1.25533138328458	2064.58465981851\\
1.25543138578464	2059.74316644965\\
1.25553138828471	2054.90740265875\\
1.25563139078477	2050.07163886784\\
1.25573139328483	2045.24733423284\\
1.25583139578489	2040.42875917579\\
1.25593139828496	2035.61018411874\\
1.25603140078502	2030.80306821759\\
1.25613140328508	2026.0016818944\\
1.25623140578514	2021.2002955712\\
1.25633140828521	2016.41036840391\\
1.25643141078527	2011.62044123661\\
1.25653141328533	2006.84197322522\\
1.25663141578539	2002.06923479178\\
1.25673141828546	1997.29649635834\\
1.25683142078552	1992.5352170808\\
1.25693142328558	1987.77966738122\\
1.25703142578564	1983.02411768163\\
1.25713142828571	1978.28002713795\\
1.25723143078577	1973.54166617222\\
1.25733143328583	1968.80330520649\\
1.25743143578589	1964.07640339666\\
1.25753143828596	1959.34950158683\\
1.25763144078602	1954.6340589329\\
1.25773144328608	1949.92434585693\\
1.25783144578614	1945.21463278095\\
1.25793144828621	1940.51637886088\\
1.25803145078627	1935.82385451876\\
1.25813145328633	1931.13133017663\\
1.25823145578639	1926.45026499042\\
1.25833145828646	1921.7691998042\\
1.25843146078652	1917.09959377388\\
1.25853146328658	1912.43571732152\\
1.25863146578664	1907.77184086915\\
1.25873146828671	1903.11942357269\\
1.25883147078677	1898.47273585418\\
1.25893147328683	1893.82604813567\\
1.25903147578689	1889.19081957306\\
1.25913147828696	1884.55559101045\\
1.25923148078702	1879.93182160374\\
1.25933148328708	1875.31378177499\\
1.25943148578714	1870.69574194624\\
1.25953148828721	1866.08916127338\\
1.25963149078727	1861.48258060053\\
1.25973149328733	1856.88745908358\\
1.25983149578739	1852.29806714458\\
1.25993149828746	1847.70867520559\\
1.26003150078752	1843.13074242249\\
1.26013150328758	1838.5528096394\\
1.26023150578764	1833.9863360122\\
1.26033150828771	1829.42559196296\\
1.26043151078777	1824.86484791372\\
1.26053151328783	1820.31556302038\\
1.26063151578789	1815.76627812704\\
1.26073151828796	1811.22845238961\\
1.26083152078802	1806.69635623012\\
1.26093152328808	1802.16426007064\\
1.26103152578814	1797.64362306706\\
1.26113152828821	1793.12298606347\\
1.26123153078827	1788.61380821579\\
1.26133153328833	1784.10463036811\\
1.26143153578839	1779.60691167634\\
1.26153153828846	1775.11492256251\\
1.26163154078852	1770.62293344869\\
1.26173154328858	1766.14240349076\\
1.26183154578864	1761.66187353284\\
1.26193154828871	1757.19280273082\\
1.26203155078877	1752.7237319288\\
1.26213155328883	1748.26612028268\\
1.26223155578889	1743.80850863656\\
1.26233155828896	1739.36235614635\\
1.26243156078902	1734.92193323408\\
1.26253156328908	1730.48151032182\\
1.26263156578914	1726.05254656546\\
1.26273156828921	1721.6235828091\\
1.26283157078927	1717.20607820864\\
1.26293157328933	1712.78857360818\\
1.26303157578939	1708.38252816362\\
1.26313157828946	1703.97648271907\\
1.26323158078952	1699.58189643041\\
1.26333158328958	1695.18731014176\\
1.26343158578964	1690.80418300901\\
1.26353158828971	1686.42105587626\\
1.26363159078977	1682.04938789941\\
1.26373159328983	1677.68344950051\\
1.26383159578989	1673.31751110162\\
1.26393159828996	1668.96303185862\\
1.26403160079002	1664.60855261563\\
1.26413160329008	1660.26553252854\\
1.26423160579014	1655.92251244145\\
1.26433160829021	1651.59095151026\\
1.26443161079027	1647.25939057907\\
1.26453161329033	1642.93928880378\\
1.26463161579039	1638.6191870285\\
1.26473161829046	1634.31054440911\\
1.26483162079052	1630.00190178973\\
1.26493162329058	1625.6989887483\\
1.26503162579064	1621.40753486277\\
1.26513162829071	1617.11608097724\\
1.26523163079077	1612.83608624761\\
1.26533163329083	1608.55609151798\\
1.26543163579089	1604.28755594426\\
1.26553163829096	1600.01902037053\\
1.26563164079102	1595.76194395271\\
1.26573164329108	1591.50486753489\\
1.26583164579114	1587.25925027297\\
1.26593164829121	1583.01363301105\\
1.26603165079127	1578.77947490503\\
1.26613165329133	1574.54531679902\\
1.26623165579139	1570.31688827095\\
1.26633165829146	1566.09991889879\\
1.26643166079152	1561.88294952662\\
1.26653166329158	1557.67743931036\\
1.26663166579164	1553.4719290941\\
1.26673166829171	1549.27787803375\\
1.26683167079177	1545.08382697339\\
1.26693167329183	1540.89550549098\\
1.26703167579189	1536.71864316448\\
1.26713167829196	1532.54178083797\\
1.26723168079202	1528.37637766737\\
1.26733168329208	1524.21097449677\\
1.26743168579214	1520.05130090412\\
1.26753168829221	1515.90308646738\\
1.26763169079227	1511.75487203063\\
1.26773169329233	1507.61811674978\\
1.26783169579239	1503.48136146894\\
1.26793169829246	1499.35033576605\\
1.26803170079252	1495.23076921906\\
1.26813170329258	1491.11120267206\\
1.26823170579264	1486.99736570303\\
1.26833170829271	1482.89498788989\\
1.26843171079277	1478.79261007675\\
1.26853171329283	1474.69596184157\\
1.26863171579289	1470.61077276228\\
1.26873171829296	1466.525583683\\
1.26883172079302	1462.44612418167\\
1.26893172329308	1458.37812383624\\
1.26903172579314	1454.31012349081\\
1.26913172829321	1450.24785272333\\
1.26923173079327	1446.19704111176\\
1.26933173329333	1442.14622950018\\
1.26943173579339	1438.10114746656\\
1.26953173829346	1434.06752458884\\
1.26963174079352	1430.03390171112\\
1.26973174329358	1426.00600841135\\
1.26983174579364	1421.98957426748\\
1.26993174829371	1417.97314012362\\
1.27003175079377	1413.9624355577\\
1.27013175329383	1409.95746056973\\
1.27023175579389	1405.96394473767\\
1.27033175829396	1401.97042890561\\
1.27043176079402	1397.9826426515\\
1.27053176329408	1394.00631555329\\
1.27063176579414	1390.02998845509\\
1.27073176829421	1386.05939093483\\
1.27083177079427	1382.09452299252\\
1.27093177329433	1378.14111420612\\
1.27103177579439	1374.18770541972\\
1.27113177829446	1370.24002621127\\
1.27123178079452	1366.29807658077\\
1.27133178329458	1362.36185652822\\
1.27143178579464	1358.43709563157\\
1.27153178829471	1354.51233473493\\
1.27163179079477	1350.59330341623\\
1.27173179329483	1346.68000167549\\
1.27183179579489	1342.77242951269\\
1.27193179829496	1338.87631650581\\
1.27203180079502	1334.98020349892\\
1.27213180329508	1331.08982006998\\
1.27223180579514	1327.20516621899\\
1.27233180829521	1323.32624194595\\
1.27243181079527	1319.45877682882\\
1.27253181329533	1315.59131171169\\
1.27263181579539	1311.72957617251\\
1.27273181829546	1307.87357021128\\
1.27283182079552	1304.023293828\\
1.27293182329558	1300.17874702267\\
1.27303182579565	1296.33992979529\\
1.27313182829571	1292.51257172382\\
1.27323183079577	1288.68521365235\\
1.27333183329583	1284.86358515882\\
1.27343183579589	1281.04768624325\\
1.27353183829596	1277.23751690563\\
1.27363184079602	1273.43307714596\\
1.27373184329608	1269.63436696425\\
1.27383184579614	1265.84138636048\\
1.27393184829621	1262.05413533466\\
1.27403185079627	1258.2726138868\\
1.27413185329633	1254.50255159484\\
1.27423185579639	1250.73248930288\\
1.27433185829646	1246.96815658887\\
1.27443186079652	1243.20955345281\\
1.27453186329658	1239.4566798947\\
1.27463186579664	1235.70953591455\\
1.27473186829671	1231.96812151234\\
1.27483187079677	1228.23243668809\\
1.27493187329683	1224.50248144179\\
1.2750318757969	1220.77825577344\\
1.27513187829696	1217.05975968304\\
1.27523188079702	1213.34699317059\\
1.27533188329708	1209.6399562361\\
1.27543188579714	1205.93864887955\\
1.27553188829721	1202.24307110096\\
1.27563189079727	1198.55322290032\\
1.27573189329733	1194.86910427762\\
1.27583189579739	1191.19071523288\\
1.27593189829746	1187.5180557661\\
1.27603190079752	1183.85112587726\\
1.27613190329758	1180.18992556637\\
1.27623190579765	1176.53445483344\\
1.27633190829771	1172.88471367845\\
1.27643191079777	1169.24070210142\\
1.27653191329783	1165.59669052439\\
1.27663191579789	1161.96413810326\\
1.27673191829796	1158.33731526008\\
1.27683192079802	1154.71622199486\\
1.27693192329808	1151.10085830758\\
1.27703192579815	1147.49122419826\\
1.27713192829821	1143.88731966688\\
1.27723193079827	1140.28914471346\\
1.27733193329833	1136.69669933799\\
1.27743193579839	1133.10998354047\\
1.27753193829846	1129.52326774295\\
1.27763194079852	1125.94801110134\\
1.27773194329858	1122.37848403767\\
1.27783194579864	1118.81468655196\\
1.27793194829871	1115.2566186442\\
1.27803195079877	1111.70428031438\\
1.27813195329883	1108.15194198457\\
1.2782319557989	1104.61106281067\\
1.27833195829896	1101.07591321471\\
1.27843196079902	1097.5464931967\\
1.27853196329908	1094.02280275665\\
1.27863196579914	1090.49911231659\\
1.27873196829921	1086.98688103244\\
1.27883197079927	1083.48037932624\\
1.27893197329933	1079.97960719799\\
1.2790319757994	1076.48456464769\\
1.27913197829946	1072.9895220974\\
1.27923198079952	1069.505938703\\
1.27933198329958	1066.02808488656\\
1.27943198579965	1062.55596064806\\
1.27953198829971	1059.08383640957\\
1.27963199079977	1055.62317132698\\
1.27973199329983	1052.16823582234\\
1.27983199579989	1048.7133003177\\
1.27993199829996	1045.26982396897\\
1.28003200080002	1041.83207719818\\
1.28013200330008	1038.3943304274\\
1.28023200580015	1034.96804281251\\
1.28033200830021	1031.54748477558\\
1.28043201080027	1028.1326563166\\
1.28053201330033	1024.71782785762\\
1.28063201580039	1021.31445855455\\
1.28073201830046	1017.91108925147\\
1.28083202080052	1014.51917910429\\
1.28093202330058	1011.13299853507\\
1.28103202580065	1007.74681796585\\
1.28113202830071	1004.37209655253\\
1.28123203080077	1001.00310471716\\
1.28133203330083	997.634112881789\\
1.2814320358009	994.276580202323\\
1.28153203830096	990.919047522856\\
1.28163204080102	987.572973999292\\
1.28173204330108	984.226900475728\\
1.28183204580114	980.892286108067\\
1.28193204830121	977.563401318357\\
1.28203205080127	974.234516528647\\
1.28213205330133	970.917090894839\\
1.2822320558014	967.599665261032\\
1.28233205830146	964.293698783127\\
1.28243206080152	960.987732305222\\
1.28253206330158	957.69322498322\\
1.28263206580165	954.398717661217\\
1.28273206830171	951.115669495118\\
1.28283207080177	947.832621329018\\
1.28293207330183	944.561032318821\\
1.2830320758019	941.289443308624\\
1.28313207830196	938.02931345433\\
1.28323208080202	934.769183600036\\
1.28333208330208	931.520512901644\\
1.28343208580215	928.271842203252\\
1.28353208830221	925.028901082811\\
1.28363209080227	921.797419118274\\
1.28373209330233	918.565937153736\\
1.28383209580239	915.3459143451\\
1.28393209830246	912.125891536465\\
1.28403210080252	908.911598305781\\
1.28413210330258	905.708764231\\
1.28423210580265	902.505930156219\\
1.28433210830271	899.31455523734\\
1.28443211080277	896.123180318461\\
1.28453211330283	892.937534977534\\
1.2846321158029	889.763348792509\\
1.28473211830296	886.589162607485\\
1.28483212080302	883.420706000411\\
1.28493212330308	880.26370854924\\
1.28503212580315	877.106711098069\\
1.28513212830321	873.95544322485\\
1.28523213080327	870.809904929582\\
1.28533213330333	867.675825790216\\
1.2854321358034	864.541746650851\\
1.28553213830346	861.413397089436\\
1.28563214080352	858.290777105973\\
1.28573214330358	855.179616278413\\
1.28583214580365	852.068455450852\\
1.28593214830371	848.963024201243\\
1.28603215080377	845.863322529586\\
1.28613215330383	842.77508001383\\
1.2862321558039	839.686837498075\\
1.28633215830396	836.604324560271\\
1.28643216080402	833.527541200419\\
1.28653216330408	830.456487418518\\
1.28663216580415	827.396892792519\\
1.28673216830421	824.337298166521\\
1.28683217080427	821.283433118473\\
1.28693217330433	818.235297648377\\
1.2870321758044	815.192891756233\\
1.28713217830446	812.156215442039\\
1.28723218080452	809.125268705797\\
1.28733218330458	806.105781125458\\
1.28743218580465	803.086293545118\\
1.28753218830471	800.07253554273\\
1.28763219080477	797.064507118293\\
1.28773219330483	794.062208271808\\
1.2878321958049	791.065639003274\\
1.28793219830496	788.074799312691\\
1.28803220080502	785.089689200059\\
1.28813220330508	782.110308665379\\
1.28823220580515	779.13665770865\\
1.28833220830521	776.168736329872\\
1.28843221080527	773.206544529046\\
1.28853221330533	770.250082306171\\
1.2886322158054	767.299349661247\\
1.28873221830546	764.354346594275\\
1.28883222080552	761.415073105254\\
1.28893222330558	758.481529194184\\
1.28903222580565	755.553714861065\\
1.28913222830571	752.631630105898\\
1.28923223080577	749.715274928682\\
1.28933223330583	746.804649329418\\
1.2894322358059	743.899753308104\\
1.28953223830596	741.000586864742\\
1.28963224080602	738.107149999332\\
1.28973224330608	735.213713133921\\
1.28983224580615	732.331735424413\\
1.28993224830621	729.455487292856\\
1.29003225080627	726.584968739251\\
1.29013225330633	723.720179763597\\
1.2902322558064	720.861120365894\\
1.29033225830646	718.007790546142\\
1.29043226080652	715.154460726391\\
1.29053226330658	712.312590062542\\
1.29063226580665	709.476448976645\\
1.29073226830671	706.646037468698\\
1.29083227080677	703.821355538703\\
1.29093227330683	700.996673608708\\
1.2910322758069	698.183450834616\\
1.29113227830696	695.375957638475\\
1.29123228080702	692.574194020285\\
1.29133228330708	689.772430402095\\
1.29143228580715	686.982125939808\\
1.29153228830721	684.197551055472\\
1.29163229080727	681.418705749088\\
1.29173229330733	678.639860442704\\
1.2918322958074	675.872474292222\\
1.29193229830746	673.110817719691\\
1.29203230080752	670.349161147161\\
1.29213230330758	667.598963730533\\
1.29223230580765	664.854495891856\\
1.29233230830771	662.110028053179\\
1.29243231080777	659.377019370405\\
1.29253231330783	656.649740265583\\
1.2926323158079	653.92246116076\\
1.29273231830796	651.20664121184\\
1.29283232080802	648.49082126292\\
1.29293232330808	645.786460469902\\
1.29303232580815	643.087829254836\\
1.29313232830821	640.38919803977\\
1.29323233080827	637.702025980606\\
1.29333233330833	635.014853921443\\
1.2934323358084	632.339141018182\\
1.29353233830846	629.663428114921\\
1.29363234080852	626.999174367562\\
1.29373234330858	624.334920620204\\
1.29383234580865	621.682126028748\\
1.29393234830871	619.029331437293\\
1.29403235080877	616.38799600174\\
1.29413235330883	613.746660566187\\
1.2942323558089	611.116784286536\\
1.29433235830896	608.486908006886\\
1.29443236080902	605.868490883138\\
1.29453236330908	603.25007375939\\
1.29463236580915	600.643115791545\\
1.29473236830921	598.036157823699\\
1.29483237080927	595.434929433805\\
1.29493237330933	592.845160199814\\
1.2950323758094	590.255390965823\\
1.29513237830946	587.671351309783\\
1.29523238080952	585.098770809645\\
1.29533238330958	582.526190309508\\
1.29543238580965	579.959339387322\\
1.29553238830971	577.403947621038\\
1.29563239080977	574.848555854755\\
1.29573239330983	572.301185497603\\
1.2958323958099	569.758971760608\\
1.29593239830996	567.222487601564\\
1.29603240081002	564.691160062676\\
1.29613240331008	562.166135059534\\
1.29623240581015	559.646839634344\\
1.29633240831021	557.13270082931\\
1.29643241081027	554.624291602227\\
1.29653241331033	552.121611953096\\
1.2966324158104	549.625234839711\\
1.29673241831046	547.134014346482\\
1.29683242081052	544.647950473409\\
1.29693242331058	542.168189136083\\
1.29703242581065	539.694157376708\\
1.29713242831071	537.22528223749\\
1.29723243081077	534.762709634017\\
1.29733243331083	532.305293650701\\
1.2974324358109	529.853607245336\\
1.29753243831096	527.408223375718\\
1.29763244081102	524.967996126256\\
1.29773244331108	522.53292549695\\
1.29783244581115	520.10415740339\\
1.29793244831121	517.681118887782\\
1.29803245081127	515.263809950125\\
1.29813245331133	512.851657632624\\
1.2982324558114	510.445234893075\\
1.29833245831146	508.045114689272\\
1.29843246081152	505.650151105625\\
1.29853246331158	503.260917099929\\
1.29863246581165	500.877412672185\\
1.29873246831171	498.499637822392\\
1.29883247081177	496.127019592756\\
1.29893247331183	493.760703898865\\
1.2990324758119	491.399544825131\\
1.29913247831196	489.044688287143\\
1.29923248081202	486.694988369312\\
1.29933248331208	484.351018029432\\
1.29943248581215	482.012777267503\\
1.29953248831221	479.680266083525\\
1.29963249081227	477.353484477499\\
1.29973249331233	475.032432449424\\
1.2998324958124	472.716537041505\\
1.29993249831246	470.406944169333\\
1.30003250081252	468.102507917317\\
1.30013250331258	465.804374201047\\
1.30023250581265	463.511397104933\\
1.30033250831271	461.224149586771\\
1.30043251081277	458.94263164656\\
1.30053251331283	456.666843284301\\
1.3006325158129	454.396211542197\\
1.30073251831296	452.13188233584\\
1.30083252081302	449.873282707434\\
1.30093252331308	447.619839699185\\
1.30103252581315	445.372126268887\\
1.30113252831321	443.13014241654\\
1.30123253081327	440.893888142144\\
1.30133253331333	438.6633634457\\
1.3014325358134	436.438568327207\\
1.30153253831346	434.219502786665\\
1.30163254081352	432.00559386628\\
1.30173254331358	429.797987481641\\
1.30183254581365	427.595537717158\\
1.30193254831371	425.399390488421\\
1.30203255081377	423.208399879841\\
1.30213255331383	421.023138849212\\
1.3022325558139	418.843607396534\\
1.30233255831396	416.669232564013\\
1.30243256081402	414.501160267238\\
1.30253256331408	412.338817548414\\
1.30263256581415	410.181631449747\\
1.30273256831421	408.030747886826\\
1.30283257081427	405.885020944061\\
1.30293257331433	403.745023579247\\
1.3030325758144	401.610755792385\\
1.30313257831446	399.482217583474\\
1.30323258081452	397.358835994719\\
1.30333258331458	395.24175694171\\
1.30343258581465	393.130407466653\\
1.30353258831471	391.024214611752\\
1.30363259081477	388.923751334803\\
1.30373259331483	386.8295905936\\
1.3038325958149	384.740586472553\\
1.30393259831496	382.657311929457\\
1.30403260081502	380.579194006518\\
1.30413260331508	378.507378619324\\
1.30423260581515	376.441292810083\\
1.30433260831521	374.380363620997\\
1.30443261081527	372.325164009863\\
1.30453261331533	370.276266934475\\
1.3046326158154	368.232526479243\\
1.30473261831546	366.194515601963\\
1.30483262081552	364.162234302634\\
1.30493262331558	362.135109623461\\
1.30503262581565	360.114287480035\\
1.30513262831571	358.09919491456\\
1.30523263081577	356.089258969241\\
1.30533263331583	354.085052601873\\
1.3054326358159	352.086575812457\\
1.30553263831596	350.094401558787\\
1.30563264081602	348.106810967478\\
1.30573264331608	346.125522911916\\
1.30583264581615	344.149964434305\\
1.30593264831621	342.180135534645\\
1.30603265081627	340.215463255141\\
1.30613265331633	338.256520553589\\
1.3062326558164	336.303880387783\\
1.30633265831646	334.356396842133\\
1.30643266081652	332.414642874435\\
1.30653266331658	330.478045526893\\
1.30663266581665	328.547750715097\\
1.30673266831671	326.623185481253\\
1.30683267081677	324.703776867564\\
1.30693267331683	322.790670789623\\
1.3070326758169	320.882721331837\\
1.30713267831696	318.980501452003\\
1.30723268081702	317.08401115012\\
1.30733268331708	315.193250426188\\
1.30743268581715	313.308219280208\\
1.30753268831721	311.428344754383\\
1.30763269081727	309.554772764306\\
1.30773269331733	307.686357394384\\
1.3078326958174	305.823671602414\\
1.30793269831746	303.96728834619\\
1.30803270081752	302.116061710122\\
1.30813270331758	300.269991694211\\
1.30823270581765	298.430224214045\\
1.30833270831771	296.596186311832\\
1.30843271081777	294.767305029774\\
1.30853271331783	292.944726283463\\
1.3086327158179	291.127304157308\\
1.30873271831796	289.315611609104\\
1.30883272081802	287.509648638852\\
1.30893272331808	285.709415246551\\
1.30903272581815	283.914911432201\\
1.30913272831821	282.126137195803\\
1.30923273081827	280.34251957956\\
1.30933273331833	278.565204499065\\
1.3094327358184	276.793046038725\\
1.30953273831846	275.026617156337\\
1.30963274081852	273.2659178519\\
1.30973274331858	271.510948125414\\
1.30983274581865	269.76170797688\\
1.30993274831871	268.018197406297\\
1.31003275081877	266.27984345587\\
1.31013275331883	264.547219083394\\
1.3102327558189	262.820897246665\\
1.31033275831896	261.099732030092\\
1.31043276081902	259.38429639147\\
1.31053276331908	257.6745903308\\
1.31063276581915	255.970613848081\\
1.31073276831921	254.271793985518\\
1.31083277081927	252.579276658701\\
1.31093277331933	250.891915952041\\
1.3110327758194	249.210857781127\\
1.31113277831946	247.53495623037\\
1.31123278081952	245.864784257563\\
1.31133278331958	244.200341862708\\
1.31143278581965	242.541056088009\\
1.31153278831971	240.888072849057\\
1.31163279081977	239.240819188056\\
1.31173279331983	237.598722147211\\
1.3118327958199	235.962354684317\\
1.31193279831996	234.331716799375\\
1.31203280082002	232.706808492384\\
1.31213280332008	231.087629763344\\
1.31223280582015	229.474180612256\\
1.31233280832021	227.866461039119\\
1.31243281082027	226.263898086138\\
1.31253281332033	224.667637668903\\
1.3126328158204	223.076533871825\\
1.31273281832046	221.491159652698\\
1.31283282082052	219.911515011522\\
1.31293282332058	218.337599948298\\
1.31303282582065	216.76884150523\\
1.31313282832071	215.206385597908\\
1.31323283082077	213.649659268538\\
1.31333283332083	212.098089559323\\
1.3134328358209	210.55224942806\\
1.31353283832096	209.012138874749\\
1.31363284082102	207.477757899388\\
1.31373284332108	205.949106501979\\
1.31383284582115	204.426184682522\\
1.31393284832121	202.90841948322\\
1.31403285082127	201.396956819665\\
1.31413285332133	199.890650776266\\
1.3142328558214	198.390074310818\\
1.31433285832146	196.895227423322\\
1.31443286082152	195.406110113777\\
1.31453286332158	193.922722382183\\
1.31463286582165	192.445064228541\\
1.31473286832171	190.972562695055\\
1.31483287082177	189.506363697315\\
1.31493287332183	188.045321319731\\
1.3150328758219	186.590008520099\\
1.31513287832196	185.140425298418\\
1.31523288082202	183.696571654688\\
1.31533288332208	182.25844758891\\
1.31543288582215	180.825480143288\\
1.31553288832221	179.398815233412\\
1.31563289082227	177.977306943692\\
1.31573289332233	176.561528231924\\
1.3158328958224	175.151479098107\\
1.31593289832246	173.747159542242\\
1.31603290082252	172.348569564327\\
1.31613290332258	170.955709164364\\
1.31623290582265	169.568005384557\\
1.31633290832271	168.186604140497\\
1.31643291082277	166.810359516593\\
1.31653291332283	165.43984447064\\
1.3166329158229	164.075059002638\\
1.31673291832296	162.716003112588\\
1.31683292082302	161.362676800489\\
1.31693292332308	160.015080066341\\
1.31703292582315	158.67263995235\\
1.31713292832321	157.335929416309\\
1.31723293082327	156.005521416016\\
1.31733293332333	154.680270035878\\
1.3174329358234	153.360748233692\\
1.31753293832346	152.046956009457\\
1.31763294082352	150.738320405378\\
1.31773294332358	149.435987337046\\
1.31783294582365	148.13881088887\\
1.31793294832371	146.84793697644\\
1.31803295082377	145.562219684166\\
1.31813295332383	144.282231969844\\
1.3182329558239	143.007973833473\\
1.31833295832396	141.738872317258\\
1.31843296082402	140.47607333679\\
1.31853296332408	139.219003934273\\
1.31863296582415	137.967091151912\\
1.31873296832421	136.720907947502\\
1.31883297082427	135.480454321044\\
1.31893297332433	134.245730272537\\
1.3190329758244	133.016735801982\\
1.31913297832446	131.793470909377\\
1.31923298082452	130.575362636929\\
1.31933298332458	129.363556900228\\
1.31943298582465	128.156907783682\\
1.31953298832471	126.955988245088\\
1.31963299082477	125.760798284445\\
1.31973299332483	124.571337901753\\
1.3198329958249	123.387607097013\\
1.31993299832496	122.209032912429\\
1.32003300082502	121.036761263591\\
1.32013300332508	119.86964623491\\
1.32023300582515	118.70826078418\\
1.32033300832521	117.553177869196\\
1.32043301082527	116.402678616573\\
1.32053301332533	115.258481899697\\
1.3206330158254	114.120014760772\\
1.32073301832546	112.986704242003\\
1.32083302082552	111.859696258981\\
1.32093302332558	110.737844896115\\
1.32103302582565	109.6217231112\\
1.32113302832571	108.511330904236\\
1.32123303082577	107.406668275224\\
1.32133303332583	106.307735224163\\
1.3214330358259	105.213958793258\\
1.32153303832596	104.1264848981\\
1.32163304082602	103.044167623098\\
1.32173304332608	101.967579926047\\
1.32183304582615	100.896721806948\\
1.32193304832621	99.8315932657995\\
1.32203305082627	98.7721943026026\\
1.32213305332633	97.718524917357\\
1.3222330558264	96.6700121522676\\
1.32233305832646	95.6272289651295\\
1.32243306082652	94.5907483137379\\
1.32253306332658	93.5594242825024\\
1.32263306582665	92.5338298292182\\
1.32273306832671	91.5133919960902\\
1.32283307082677	90.4992566987087\\
1.32293307332683	89.4908509792784\\
1.3230330758269	88.4876018800043\\
1.32313307832696	87.4900823586816\\
1.32323308082702	86.4982924153101\\
1.32333308332708	85.51223204989\\
1.32343308582715	84.5319012624211\\
1.32353308832721	83.5573000529036\\
1.32363309082727	82.5878554635423\\
1.32373309332733	81.6247134099273\\
1.3238330958274	80.6667279764686\\
1.32393309832746	79.7144721209612\\
1.32403310082752	78.767945843405\\
1.32413310332758	77.8271491438002\\
1.32423310582765	76.8920820221467\\
1.32433310832771	75.9621715206494\\
1.32443311082777	75.0385635548985\\
1.32453311332783	74.1201122093038\\
1.3246331158279	73.2073904416604\\
1.32473311832796	72.3003982519683\\
1.32483312082802	71.3991356402275\\
1.32493312332808	70.5036026064381\\
1.32503312582815	69.6132261928048\\
1.32513312832821	68.7291523149179\\
1.32523313082827	67.8502350571872\\
1.32533313332833	66.9770473774078\\
1.3254331358284	66.1101622333749\\
1.32553313832846	65.247860751703\\
1.32563314082852	64.3918618057776\\
1.32573314332858	63.5415924378034\\
1.32583314582865	62.6964796899855\\
1.32593314832871	61.8576694779139\\
1.32603315082877	61.0240158859986\\
1.32613315332883	60.1960918720346\\
1.3262331558289	59.3738974360218\\
1.32633315832896	58.5574325779604\\
1.32643316082902	57.7461243400551\\
1.32653316332908	56.9411186378963\\
1.32663316582915	56.1414987390118\\
1.32673316832921	55.3474938265195\\
1.32683317082927	54.559161196199\\
1.32693317332933	53.7765008480503\\
1.3270331758294	52.9994554862938\\
1.32713317832946	52.2280824067092\\
1.32723318082952	51.4623243135169\\
1.32733318332958	50.7022385024963\\
1.32743318582965	49.9478249736476\\
1.32753318832971	49.1990264311911\\
1.32763319082977	48.4559001709064\\
1.32773319332983	47.718388897014\\
1.3278331958299	46.9865499052934\\
1.32793319832996	46.2603831957446\\
1.32803320083002	45.5398314725881\\
1.32813320333008	44.8249520316034\\
1.32823320583015	44.1156875770109\\
1.32833320833021	43.4120954045903\\
1.32843321083027	42.7141755143414\\
1.32853321333033	42.0218706104848\\
1.3286332158304	41.3352379888001\\
1.32873321833046	40.6542203535076\\
1.32883322083052	39.9788750003869\\
1.32893322333058	39.3091446336584\\
1.32903322583065	38.6451438448813\\
1.32913322833071	37.986700746717\\
1.32923323083077	37.3339299307244\\
1.32933323333083	36.6868313969037\\
1.3294332358309	36.0454051452547\\
1.32953323833096	35.4095938799981\\
1.32963324083102	34.7793976011337\\
1.32973324333108	34.1549309002206\\
1.32983324583115	33.5360218899203\\
1.32993324833121	32.9228424575713\\
1.33003325083127	32.3152780116145\\
1.33013325333133	31.7133285520501\\
1.3302332558314	31.1170513746574\\
1.33033325833146	30.5264464794366\\
1.33043326083152	29.941456570608\\
1.33053326333158	29.3621389439512\\
1.33063326583165	28.7884935994663\\
1.33073326833171	28.2204632413736\\
1.33083327083177	27.6580478696732\\
1.33093327333183	27.101362075924\\
1.3310332758319	26.5502339727877\\
1.33113327833196	26.0048354476027\\
1.33123328083202	25.4650519088099\\
1.33133328333208	24.9308833564095\\
1.33143328583215	24.4023870861808\\
1.33153328833221	23.8795630981239\\
1.33163329083227	23.3623540964593\\
1.33173329333233	22.8508173769665\\
1.3318332958324	22.344895643866\\
1.33193329833246	21.8446461929373\\
1.33203330083252	21.3500690241803\\
1.33213330333258	20.8611068418157\\
1.33223330583265	20.3778169416229\\
1.33233330833271	19.9001420278223\\
1.33243331083277	19.4281393961935\\
1.33253331333283	18.961751750957\\
1.3326333158329	18.5010363878923\\
1.33273331833296	18.0459933069994\\
1.33283332083302	17.5965652124988\\
1.33293332333308	17.15280940017\\
1.33303332583315	16.7146685742335\\
1.33313332833321	16.2822000304687\\
1.33323333083327	15.8554037688758\\
1.33333333333333	15.4342224936751\\
1.3334333358334	15.0186562048667\\
1.33353333833346	14.6087621982301\\
1.33363334083352	14.2045404737653\\
1.33373334333358	13.8059337356928\\
1.33383334583365	13.4129992797921\\
1.33393334833371	13.0257371060632\\
1.33403335083377	12.6440899187265\\
1.33413335333383	12.2680577177822\\
1.3342333558339	11.8977550947891\\
1.33433335833396	11.5330101624088\\
1.33443336083402	11.1739948079799\\
1.33453336333408	10.8205371441636\\
1.33463336583415	10.4728090582988\\
1.33473336833421	10.1306959588261\\
1.33483337083427	9.79425514152532\\
1.33493337333433	9.46342931061678\\
1.3350333758344	9.13821846610053\\
1.33513337833446	8.81873719953558\\
1.33523338083452	8.5048136235834\\
1.33533338333458	8.19661962558253\\
1.33543338583465	7.89404061397394\\
1.33553338833471	7.59707658875764\\
1.33563339083477	7.30584214149264\\
1.33573339333483	7.02016538484041\\
1.3358333958349	6.74021820613949\\
1.33593339833496	6.46582871805134\\
1.33603340083502	6.1971688079145\\
1.33613340333508	5.93412388416994\\
1.33623340583515	5.67670540597356\\
1.33633340833521	5.42495348037103\\
1.33643341083527	5.17884518905053\\
1.33653341333533	4.93838626159003\\
1.3366334158354	4.70357669798952\\
1.33673341833546	4.47441076867104\\
1.33683342083552	4.25089420321255\\
1.33693342333558	4.03302700161406\\
1.33703342583565	3.8208034342976\\
1.33713342833571	3.61422923084114\\
1.33723343083577	3.41330439124466\\
1.33733343333583	3.21802318593022\\
1.3374334358359	3.02839134447577\\
1.33753343833596	2.84440886688131\\
1.33763344083602	2.66607002356889\\
1.33773344333608	2.49337481453851\\
1.33783344583615	2.32633469894607\\
1.33793344833621	2.16493821763567\\
1.33803345083627	2.00919110018526\\
1.33813345333633	1.85908761701689\\
1.3382334558364	1.7146334977085\\
1.33833345833646	1.57582301268216\\
1.33843346083652	1.4426618915158\\
1.33853346333658	1.31515013420944\\
1.33863346583665	1.19328774076306\\
1.33873346833671	1.07706898159873\\
1.33883347083677	0.966493856716429\\
1.33893347333683	0.861573825272073\\
1.3390334758369	0.762297428109755\\
1.33913347833696	0.668664665229476\\
1.33923348083702	0.580681266209187\\
1.33933348333708	0.498347231048887\\
1.33943348583715	0.421658549044012\\
1.33953348833721	0.350617512025741\\
1.33963349083727	0.28522354703628\\
1.33973349333733	0.225476654075628\\
1.3398334958374	0.171377406101581\\
1.33993349833746	0.122924657198547\\
1.34003350083752	0.0801195532821187\\
1.34013350333758	0.0429611776198223\\
1.34023350583765	0.0114501604652329\\
1.34033350833771	-0.0144136700689881\\
1.34043351083777	-0.0346304285743997\\
1.34053351333783	-0.0492000004594428\\
1.3406335158379	-0.0581225576114561\\
1.34073351833796	-0.0613975843684239\\
1.34083352083802	-0.0590261120543774\\
1.34093352333808	-0.0510068801621674\\
1.34103352583815	-0.037340690832707\\
1.34113352833821	-0.0180274294744372\\
1.34123353083827	0.00693301850420101\\
1.34133353333833	0.0375406531032076\\
1.3414335358384	0.0737952451394646\\
1.34153353833846	0.115697367570767\\
1.34163354083852	0.163246562030879\\
1.34173354333858	0.2164428285198\\
1.34183354583865	0.275286167037531\\
1.34193354833871	0.339776577584071\\
1.34203355083877	0.409914060159421\\
1.34213355333883	0.485699187721374\\
1.3422335558389	0.567130814354343\\
1.34233355833896	0.654208940058325\\
1.34243356083902	0.746936429622298\\
1.34253356333908	0.845307553468309\\
1.34263356583915	0.94932804117431\\
1.34273356833921	1.0589978927403\\
1.34283357083927	1.17431137858833\\
1.34293357333933	1.29527422829635\\
1.3430335758394	1.42188644186436\\
1.34313357833946	1.55414228971441\\
1.34323358083952	1.69204750142444\\
1.34333358333958	1.83559634741652\\
1.34343358583965	1.98479455726859\\
1.34353358833971	2.13964213098064\\
1.34363359083977	2.30013333897474\\
1.34373359333983	2.46627391082882\\
1.3438335958399	2.6380638465429\\
1.34393359833996	2.81549741653901\\
1.34403360084002	2.99858035039511\\
1.34413360334008	3.18730691853326\\
1.34423360584015	3.38168285053139\\
1.34433360834021	3.58170814638951\\
1.34443361084027	3.78737707652967\\
1.34453361334033	3.99869537052982\\
1.3446336158404	4.21566302838996\\
1.34473361834046	4.43827432053214\\
1.34483362084052	4.66653497653431\\
1.34493362334058	4.90044499639647\\
1.34503362584065	5.13999865054066\\
1.34513362834071	5.38520166854485\\
1.34523363084077	5.63604832083108\\
1.34533363334083	5.89252714824344\\
1.3454336358409	6.1547126352953\\
1.34553363834096	6.42245581295994\\
1.34563364084102	6.69592856857588\\
1.34573364334108	6.9750163105841\\
1.34583364584115	7.25971903898461\\
1.34593364834121	7.55015134533642\\
1.34603365084127	7.84614134230101\\
1.34613365334133	8.14780362143738\\
1.3462336558414	8.45513818274556\\
1.34633365834146	8.76814502622553\\
1.34643366084152	9.08676685609778\\
1.34653366334158	9.41100367236231\\
1.34663366584165	9.74091277079864\\
1.34673366834171	10.0764941514068\\
1.34683367084177	10.4176905184072\\
1.34693367334183	10.7645591675794\\
1.3470336758419	11.1170428031439\\
1.34713367834196	11.4751987208801\\
1.34723368084202	11.8390269207882\\
1.34733368334208	12.2084701070886\\
1.34743368584215	12.5835282797812\\
1.34753368834221	12.9643160304251\\
1.34763369084227	13.3506614716819\\
1.34773369334233	13.7427364908899\\
1.3478336958424	14.1404264964902\\
1.34793369834246	14.5437314884828\\
1.34803370084252	14.9527087626471\\
1.34813370334258	15.3673583189833\\
1.34823370584265	15.7876228617118\\
1.34833370834271	16.213559686612\\
1.34843371084277	16.6451114979046\\
1.34853371334283	17.0823355913689\\
1.3486337158429	17.5251746712255\\
1.34873371834296	17.9737433290334\\
1.34883372084302	18.4278696774541\\
1.34893372334308	18.8876683080466\\
1.34903372584315	19.3531392208109\\
1.34913372834321	19.824282415747\\
1.34923373084327	20.3010405970753\\
1.34933373334333	20.783413764796\\
1.3494337358434	21.2714592146884\\
1.34953373834346	21.7651769467526\\
1.34963374084352	22.2645096652091\\
1.34973374334358	22.7695146658375\\
1.34983374584365	23.280134652858\\
1.34993374834371	23.7964269220504\\
1.35003375084377	24.3183914734146\\
1.35013375334383	24.8459710111711\\
1.3502337558439	25.3792228310993\\
1.35033375834396	25.9180896374199\\
1.35043376084402	26.4626287259122\\
1.35053376334408	27.0127828007968\\
1.35063376584415	27.5686091578532\\
1.35073376834421	28.1301077970814\\
1.35083377084427	28.6972214227019\\
1.35093377334433	29.2700073304942\\
1.3510337758444	29.8484082246788\\
1.35113377834446	30.4324814010351\\
1.35123378084452	31.0222268595633\\
1.35133378334458	31.6175873044837\\
1.35143378584465	32.2185627357965\\
1.35153378834471	32.8252677450605\\
1.35163379084477	33.4375877407168\\
1.35173379334483	34.0555227227654\\
1.3518337958449	34.6791299869858\\
1.35193379834496	35.308409533378\\
1.35203380084502	35.9433040661624\\
1.35213380334508	36.5838708811187\\
1.35223380584515	37.2300526824672\\
1.35233380834521	37.8819067659876\\
1.35243381084527	38.5394331316797\\
1.35253381334533	39.2025744837641\\
1.3526338158454	39.8713308222408\\
1.35273381834546	40.5458167386688\\
1.35283382084552	41.2259176414891\\
1.35293382334558	41.9116335307017\\
1.35303382584565	42.603021702086\\
1.35313382834571	43.3000821556422\\
1.35323383084577	44.0027575955906\\
1.35333383334583	44.7111053177109\\
1.3534338358459	45.4250680262234\\
1.35353383834596	46.1447030169077\\
1.35363384084602	46.8700102897638\\
1.35373384334608	47.6009325490122\\
1.35383384584615	48.3375270904324\\
1.35393384834621	49.0797939140244\\
1.35403385084627	49.8276757240086\\
1.35413385334633	50.5811725203852\\
1.3542338558464	51.3403415989335\\
1.35433385834646	52.1051829596537\\
1.35443386084652	52.8756966025456\\
1.35453386334658	53.6518252318298\\
1.35463386584665	54.4335688475063\\
1.35473386834671	55.2209847453546\\
1.35483387084677	56.0140729253747\\
1.35493387334683	56.8128333875665\\
1.3550338758469	57.6172088361507\\
1.35513387834696	58.4273711584657\\
1.35523388084702	59.2426901009368\\
1.35533388334708	60.0643115791545\\
1.35543388584715	60.8910896775282\\
1.35553388834721	61.7235973538533\\
1.35563389084727	62.5618346081297\\
1.35573389334733	63.4058014403574\\
1.3558338958474	64.2554978505364\\
1.35593389834746	65.1109238386668\\
1.35603390084752	65.9715064469533\\
1.35613390334758	66.8383915909862\\
1.35623390584765	67.7104333551753\\
1.35633390834771	68.5882046973157\\
1.35643391084777	69.4717056174075\\
1.35653391334783	70.3609361154505\\
1.3566339158479	71.2553232336497\\
1.35673391834796	72.1560128875954\\
1.35683392084802	73.0618591616972\\
1.35693392334808	73.9740079715455\\
1.35703392584815	74.8913134015499\\
1.35713392834821	75.8143484095057\\
1.35723393084827	76.7425400376176\\
1.35733393334833	77.677034201476\\
1.3574339358484	78.6172579432857\\
1.35753393834846	79.5626383052515\\
1.35763394084852	80.5137482451687\\
1.35773394334858	81.4711607208323\\
1.35783394584865	82.433729816652\\
1.35793394834871	83.401455532628\\
1.35803395084877	84.3754837843504\\
1.35813395334883	85.3552416140241\\
1.3582339558489	86.340156063854\\
1.35833395834896	87.3313730494303\\
1.35843396084902	88.3277466551628\\
1.35853396334908	89.3298498388466\\
1.35863396584915	90.3376826004818\\
1.35873396834921	91.3506719822731\\
1.35883397084927	92.3699638998108\\
1.35893397334933	93.3949853952998\\
1.3590339758494	94.4251635109451\\
1.35913397834946	95.4610712045416\\
1.35923398084952	96.5027084760894\\
1.35933398334958	97.5500753255886\\
1.35943398584965	98.603171753039\\
1.35953398834971	99.6619977584408\\
1.35963399084977	100.725980383999\\
1.35973399334983	101.796265545303\\
1.3598339958499	102.871707326764\\
1.35993399834996	103.952878686176\\
1.36003400085002	105.039779623539\\
1.36013400335008	106.132410138853\\
1.36023400585015	107.230197274324\\
1.36033400835021	108.334286945541\\
1.36043401085027	109.443533236914\\
1.36053401335033	110.559082064034\\
1.3606340158504	111.67978751131\\
1.36073401835046	112.806222536537\\
1.36083402085052	113.938387139716\\
1.36093402335058	115.076281320845\\
1.36103402585065	116.219332122131\\
1.36113402835071	117.368685459164\\
1.36123403085077	118.523195416352\\
1.36133403335083	119.683434951492\\
1.3614340358509	120.849404064583\\
1.36153403835096	122.021102755626\\
1.36163404085102	123.19853102462\\
1.36173404335108	124.381688871565\\
1.36183404585115	125.570003338666\\
1.36193404835121	126.764620341514\\
1.36203405085127	127.964393964518\\
1.36213405335133	129.169897165473\\
1.3622340558514	130.38112994438\\
1.36233405835146	131.598092301238\\
1.36243406085152	132.820784236047\\
1.36253406335158	134.048632791012\\
1.36263406585165	135.282783881724\\
1.36273406835171	136.522091592592\\
1.36283407085177	137.767128881411\\
1.36293407335183	139.017895748182\\
1.3630340758519	140.274392192904\\
1.36313407835196	141.536618215577\\
1.36323408085202	142.804573816202\\
1.36333408335208	144.077686036982\\
1.36343408585215	145.357100793509\\
1.36353408835221	146.641672170193\\
1.36363409085227	147.931973124827\\
1.36373409335233	149.228003657413\\
1.3638340958524	150.52976376795\\
1.36393409835246	151.837253456439\\
1.36403410085252	153.149899765084\\
1.36413410335258	154.468848609475\\
1.36423410585265	155.792954074022\\
1.36433410835271	157.122789116521\\
1.36443411085277	158.458353736971\\
1.36453411335283	159.799647935372\\
1.3646341158529	161.146671711725\\
1.36473411835296	162.499425066028\\
1.36483412085302	163.857335040488\\
1.36493412335308	165.2209745929\\
1.36503412585315	166.590916681058\\
1.36513412835321	167.966015389372\\
1.36523413085327	169.346843675637\\
1.36533413335333	170.733401539853\\
1.3654341358534	172.125116024226\\
1.36553413835346	173.523133044345\\
1.36563414085352	174.926879642416\\
1.36573414335358	176.335782860643\\
1.36583414585365	177.750415656821\\
1.36593414835371	179.17077803095\\
1.36603415085377	180.596869983031\\
1.36613415335383	182.028691513063\\
1.3662341558539	183.466242621046\\
1.36633415835396	184.908950349185\\
1.36643416085402	186.357960613071\\
1.36653416335408	187.812127497113\\
1.36663416585415	189.272023959106\\
1.36673416835421	190.737649999051\\
1.36683417085427	192.209005616947\\
1.36693417335433	193.686090812794\\
1.3670341758544	195.168332628798\\
1.36713417835446	196.656876980548\\
1.36723418085452	198.150577952454\\
1.36733418335458	199.650008502311\\
1.36743418585465	201.155741587915\\
1.36753418835471	202.66605833588\\
1.36763419085477	204.182677619591\\
1.36773419335483	205.705026481254\\
1.3678341958549	207.233104920867\\
1.36793419835496	208.766339980638\\
1.36803420085502	210.305304618359\\
1.36813420335508	211.850571791827\\
1.36823420585515	213.400995585451\\
1.36833420835521	214.957148957026\\
1.36843421085527	216.519031906553\\
1.36853421335533	218.086071476236\\
1.3686342158554	219.659413581665\\
1.36873421835546	221.23791230725\\
1.36883422085552	222.822140610787\\
1.36893422335558	224.41267145007\\
1.36903422585565	226.008358909509\\
1.36913422835571	227.6097759469\\
1.36923423085577	229.216349604447\\
1.36933423335583	230.82922579774\\
1.3694342358559	232.447831568985\\
1.36953423835596	234.071593960385\\
1.36963424085602	235.701085929738\\
1.36973424335608	237.336307477041\\
1.36983424585615	238.977258602296\\
1.36993424835621	240.623939305502\\
1.37003425085627	242.276349586659\\
1.37013425335633	243.934489445767\\
1.3702342558564	245.597785925032\\
1.37033425835646	247.267384940043\\
1.37043426085652	248.942140575211\\
1.37053426335658	250.62262578833\\
1.37063426585665	252.3088405794\\
1.37073426835671	254.000784948421\\
1.37083427085677	255.698458895394\\
1.37093427335683	257.401289462522\\
1.3710342758569	259.110422565398\\
1.37113427835696	260.824712288429\\
1.37123428085702	262.545304547207\\
1.37133428335708	264.271053426141\\
1.37143428585715	266.002531883026\\
1.37153428835721	267.739739917863\\
1.37163429085727	269.482104572856\\
1.37173429335733	271.230771763595\\
1.3718342958574	272.98459557449\\
1.37193429835746	274.744721921132\\
1.37203430085752	276.51000488793\\
1.37213430335758	278.28101743268\\
1.37223430585765	280.05775955538\\
1.37233430835771	281.840231256032\\
1.37243431085777	283.628432534636\\
1.37253431335783	285.421790433395\\
1.3726343158579	287.221450867901\\
1.37273431835796	289.026267922563\\
1.37283432085802	290.837387512972\\
1.37293432335808	292.653663723537\\
1.37303432585815	294.475669512053\\
1.37313432835821	296.30340487852\\
1.37323433085827	298.136296865143\\
1.37333433335833	299.975491387513\\
1.3734343358584	301.820415487835\\
1.37353433835846	303.670496208312\\
1.37363434085852	305.526306506741\\
1.37373434335858	307.387846383121\\
1.37383434585865	309.255115837452\\
1.37393434835871	311.128114869735\\
1.37403435085877	313.006843479969\\
1.37413435335883	314.891301668154\\
1.3742343558589	316.780916476496\\
1.37433435835896	318.676833820583\\
1.37443436085902	320.577907784827\\
1.37453436335908	322.484711327023\\
1.37463436585915	324.397244447169\\
1.37473436835921	326.315507145267\\
1.37483437085927	328.239499421317\\
1.37493437335933	330.169221275317\\
1.3750343758594	332.104099749474\\
1.37513437835946	334.045280759377\\
1.37523438085952	335.991618389437\\
1.37533438335958	337.943685597448\\
1.37543438585965	339.90148238341\\
1.37553438835971	341.865008747323\\
1.37563439085977	343.834264689187\\
1.37573439335983	345.809250209004\\
1.3758343958599	347.789965306771\\
1.37593439835996	349.775837024694\\
1.37603440086002	351.767438320569\\
1.37613440336008	353.76534215219\\
1.37623440586015	355.768402603967\\
1.37633440836021	357.777192633696\\
1.37643441086027	359.791712241376\\
1.37653441336033	361.811388469212\\
1.3766344158604	363.837367232795\\
1.37673441836046	365.869075574329\\
1.37683442086052	367.905940536019\\
1.37693442336058	369.94853507566\\
1.37703442586065	371.997432151048\\
1.37713442836071	374.051485846592\\
1.37723443086077	376.111269120087\\
1.37733443336083	378.176209013739\\
1.3774344358609	380.247451443137\\
1.37753443836096	382.324423450486\\
1.37763444086102	384.406552077991\\
1.37773444336108	386.494983241243\\
1.37783444586115	388.588571024651\\
1.37793444836121	390.687888386011\\
1.37803445086127	392.792935325321\\
1.37813445336133	394.903711842583\\
1.3782344558614	397.020217937797\\
1.37833445836146	399.141880653166\\
1.37843446086152	401.269845904282\\
1.37853446336158	403.402967775554\\
1.37863446586165	405.542392182572\\
1.37873446836171	407.686973209747\\
1.37883447086177	409.837283814873\\
1.37893447336183	411.99332399795\\
1.3790344758619	414.155093758979\\
1.37913447836196	416.322020140164\\
1.37923448086202	418.495249057095\\
1.37933448336208	420.674207551977\\
1.37943448586215	422.858322667016\\
1.37953448836221	425.048167360006\\
1.37963449086227	427.243741630947\\
1.37973449336233	429.44504547984\\
1.3798344958624	431.652078906684\\
1.37993449836246	433.864841911479\\
1.38003450086252	436.083334494226\\
1.38013450336258	438.306983697128\\
1.38023450586265	440.536935435778\\
1.38033450836271	442.772043794583\\
1.38043451086277	445.013454689135\\
1.38053451336283	447.260022203843\\
1.3806345158629	449.512319296502\\
1.38073451836296	451.770345967113\\
1.38083452086302	454.033529257879\\
1.38093452336308	456.303015084392\\
1.38103452586315	458.578230488857\\
1.38113452836321	460.858602513478\\
1.38123453086327	463.145277073845\\
1.38133453336333	465.437108254368\\
1.3814345358634	467.734669012843\\
1.38153453836346	470.037959349269\\
1.38163454086352	472.346979263646\\
1.38173454336358	474.661728755974\\
1.38183454586365	476.981634868459\\
1.38193454836371	479.30784351669\\
1.38203455086377	481.639208785077\\
1.38213455336383	483.976876589211\\
1.3822345558639	486.319701013501\\
1.38233455836396	488.668255015742\\
1.38243456086402	491.022538595935\\
1.38253456336408	493.382551754079\\
1.38263456586415	495.748294490174\\
1.38273456836421	498.11976680422\\
1.38283457086427	500.496395738423\\
1.38293457336433	502.879327208372\\
1.3830345758644	505.267415298478\\
1.38313457836446	507.661232966534\\
1.38323458086452	510.060780212542\\
1.38333458336458	512.466057036501\\
1.38343458586465	514.877063438412\\
1.38353458836471	517.293799418274\\
1.38363459086477	519.716264976087\\
1.38373459336483	522.144460111851\\
1.3838345958649	524.577811867772\\
1.38393459836496	527.016893201644\\
1.38403460086502	529.462277071262\\
1.38413460336508	531.912817561036\\
1.38423460586515	534.369087628762\\
1.38433460836521	536.831087274439\\
1.38443461086527	539.298816498068\\
1.38453461336533	541.772275299648\\
1.3846346158654	544.250890721384\\
1.38473461836546	546.735808678866\\
1.38483462086552	549.225883256505\\
1.38493462336558	551.722260369889\\
1.38503462586565	554.223794103431\\
1.38513462836571	556.731057414923\\
1.38523463086577	559.244050304367\\
1.38533463336583	561.762772771762\\
1.3854346358659	564.287224817108\\
1.38553463836596	566.816833482611\\
1.38563464086602	569.35274468386\\
1.38573464336608	571.893812505265\\
1.38583464586615	574.441755820212\\
1.38593464836621	576.991418008544\\
1.38603465086627	579.552539352779\\
1.38613465336633	582.113660697014\\
1.3862346558664	584.686241197151\\
1.38633465836646	587.258821697289\\
1.38643466086652	589.842861353329\\
1.38653466336658	592.426901009369\\
1.38663466586665	595.022399821311\\
1.38673466836671	597.617898633254\\
1.38683467086677	600.224856601099\\
1.38693467336683	602.831814568944\\
1.3870346758669	605.450231692692\\
1.38713467836696	608.06864881644\\
1.38723468086702	610.692795518139\\
1.38733468336708	613.328401375741\\
1.38743468586715	615.964007233343\\
1.38753468836721	618.605342668896\\
1.38763469086727	621.258137260352\\
1.38773469336733	623.910931851807\\
1.3878346958674	626.575185599166\\
1.38793469836746	629.239439346524\\
1.38803470086752	631.909422671834\\
1.38813470336758	634.585135575095\\
1.38823470586765	637.272307634258\\
1.38833470836771	639.959479693422\\
1.38843471086777	642.652381330537\\
1.38853471336783	645.356742123554\\
1.3886347158679	648.061102916571\\
1.38873471836796	650.77119328754\\
1.38883472086802	653.48701323646\\
1.38893472336808	656.214292341283\\
1.38903472586815	658.941571446106\\
1.38913472836821	661.67458012888\\
1.38923473086827	664.413318389605\\
1.38933473336833	667.157786228282\\
1.3894347358684	669.913713222861\\
1.38953473836846	672.66964021744\\
1.38963474086852	675.431296789971\\
1.38973474336858	678.198682940453\\
1.38983474586865	680.971798668886\\
1.38993474836871	683.75064397527\\
1.39003475086877	686.535218859606\\
1.39013475336883	689.325523321893\\
1.3902347558689	692.127286940083\\
1.39033475836896	694.929050558273\\
1.39043476086902	697.736543754414\\
1.39053476336908	700.549766528506\\
1.39063476586915	703.36871888055\\
1.39073476836921	706.193400810545\\
1.39083477086927	709.023812318491\\
1.39093477336933	711.859953404389\\
1.3910347758694	714.701824068238\\
1.39113477836946	717.549424310038\\
1.39123478086952	720.402754129789\\
1.39133478336958	723.261813527492\\
1.39143478586965	726.126602503146\\
1.39153478836971	728.997121056752\\
1.39163479086977	731.873369188308\\
1.39173479336983	734.755346897816\\
1.3918347958699	737.643054185276\\
1.39193479836996	740.536491050686\\
1.39203480087002	743.435657494048\\
1.39213480337008	746.340553515362\\
1.39223480587015	749.251179114626\\
1.39233480837021	752.161804713891\\
1.39243481087027	755.083889469058\\
1.39253481337033	758.011703802176\\
1.3926348158704	760.945247713246\\
1.39273481837046	763.884521202267\\
1.39283482087052	766.82952426924\\
1.39293482337058	769.780256914164\\
1.39303482587065	772.736719137039\\
1.39313482837071	775.693181359914\\
1.39323483087077	778.661102738691\\
1.39333483337083	781.63475369542\\
1.3934348358709	784.614134230101\\
1.39353483837096	787.599244342732\\
1.39363484087102	790.584354455364\\
1.39373484337108	793.580923723898\\
1.39383484587115	796.583222570384\\
1.39393484837121	799.59125099482\\
1.39403485087127	802.599279419257\\
1.39413485337133	805.618766999597\\
1.3942348558714	808.643984157887\\
1.39433485837146	811.674930894129\\
1.39443486087152	814.705877630372\\
1.39453486337158	817.748283522516\\
1.39463486587165	820.796418992612\\
1.39473486837171	823.844554462708\\
1.39483487087177	826.904149088707\\
1.39493487337183	829.969473292657\\
1.3950348758719	833.034797496607\\
1.39513487837196	836.111580856459\\
1.39523488087202	839.194093794263\\
1.39533488337208	842.276606732067\\
1.39543488587215	845.370578825773\\
1.39553488837221	848.470280497431\\
1.39563489087227	851.569982169089\\
1.39573489337233	854.681142996649\\
1.3958348958724	857.792303824209\\
1.39593489837246	860.914923807672\\
1.39603490087252	864.043273369087\\
1.39613490337258	867.171622930501\\
1.39623490587265	870.311431647818\\
1.39633490837271	873.451240365135\\
1.39643491087277	876.602508238354\\
1.39653491337283	879.753776111574\\
1.3966349158729	882.916503140696\\
1.39673491837296	886.079230169818\\
1.39683492087302	889.253416354843\\
1.39693492337308	892.427602539868\\
1.39703492587315	895.613247880795\\
1.39713492837321	898.798893221722\\
1.39723493087327	901.995997718552\\
1.39733493337333	905.193102215382\\
1.3974349358734	908.401665868115\\
1.39753493837346	911.610229520848\\
1.39763494087352	914.830252329483\\
1.39773494337358	918.050275138118\\
1.39783494587365	921.281757102656\\
1.39793494837371	924.513239067194\\
1.39803495087377	927.750450609683\\
1.39813495337383	930.999121308075\\
1.3982349558739	934.247792006467\\
1.39833495837396	937.507921860761\\
1.39843496087402	940.768051715055\\
1.39853496337408	944.033911147301\\
1.39863496587415	947.311229735449\\
1.39873496837421	950.588548323598\\
1.39883497087427	953.871596489697\\
1.39893497337433	957.166103811699\\
1.3990349758744	960.460611133702\\
1.39913497837446	963.760848033655\\
1.39923498087452	967.072544089511\\
1.39933498337458	970.384240145367\\
1.39943498587465	973.701665779175\\
1.39953498837471	977.030550568885\\
1.39963499087477	980.359435358595\\
1.39973499337483	983.694049726256\\
1.3998349958749	987.034393671869\\
1.39993499837496	990.386196773384\\
1.40003500087502	993.7379998749\\
1.40013500337508	997.095532554366\\
1.40023500587515	1000.46452438974\\
1.40033500837521	1003.8335162251\\
1.40043501087527	1007.20823763843\\
1.40053501337533	1010.5886886297\\
1.4006350158754	1013.97486919892\\
1.40073501837546	1017.37250892405\\
1.40083502087552	1020.77014864917\\
1.40093502337558	1024.17351795225\\
1.40103502587565	1027.58261683328\\
1.40113502837571	1030.99744529226\\
1.40123503087577	1034.42373290714\\
1.40133503337583	1037.85002052202\\
1.4014350358759	1041.28203771486\\
1.40153503837596	1044.71978448564\\
1.40163504087602	1048.16326083438\\
1.40173504337608	1051.61246676106\\
1.40183504587615	1055.0674022657\\
1.40193504837621	1058.53379692624\\
1.40203505087627	1062.00019158679\\
1.40213505337633	1065.47231582528\\
1.4022350558764	1068.95016964172\\
1.40233505837646	1072.43375303612\\
1.40243506087652	1075.92306600847\\
1.40253506337658	1079.41810855876\\
1.40263506587665	1082.91888068701\\
1.40273506837671	1086.42538239321\\
1.40283507087677	1089.93761367736\\
1.40293507337683	1093.45557453947\\
1.4030350758769	1096.97926497952\\
1.40313507837696	1100.50868499753\\
1.40323508087702	1104.04383459349\\
1.40333508337708	1107.58471376739\\
1.40343508587715	1111.13705209721\\
1.40353508837721	1114.68939042702\\
1.40363509087727	1118.24745833478\\
1.40373509337733	1121.81125582049\\
1.4038350958774	1125.38078288416\\
1.40393509837746	1128.95030994782\\
1.40403510087752	1132.53129616739\\
1.40413510337758	1136.11801196491\\
1.40423510587765	1139.71045734038\\
1.40433510837771	1143.3086322938\\
1.40443511087777	1146.91253682517\\
1.40453511337783	1150.5221709345\\
1.4046351158779	1154.13753462177\\
1.40473511837796	1157.758627887\\
1.40483512087802	1161.38545073018\\
1.40493512337808	1165.01800315131\\
1.40503512587815	1168.65628515039\\
1.40513512837821	1172.30029672742\\
1.40523513087827	1175.9500378824\\
1.40533513337833	1179.60550861534\\
1.4054351358784	1183.26097934827\\
1.40553513837846	1186.92790923711\\
1.40563514087852	1190.6005687039\\
1.40573514337858	1194.27895774864\\
1.40583514587865	1197.96307637133\\
1.40593514837871	1201.65292457197\\
1.40603515087877	1205.34850235057\\
1.40613515337883	1209.04980970711\\
1.4062351558789	1212.75111706366\\
1.40633515837896	1216.4638835761\\
1.40643516087902	1220.1823796665\\
1.40653516337908	1223.90660533485\\
1.40663516587915	1227.63656058116\\
1.40673516837921	1231.36651582746\\
1.40683517087927	1235.10793022966\\
1.40693517337933	1238.85507420982\\
1.4070351758794	1242.60794776792\\
1.40713517837946	1246.36655090398\\
1.40723518087952	1250.12515404004\\
1.40733518337958	1253.895216332\\
1.40743518587965	1257.67100820191\\
1.40753518837971	1261.45252964978\\
1.40763519087977	1265.23405109764\\
1.40773519337983	1269.02703170141\\
1.4078351958799	1272.82574188312\\
1.40793519837996	1276.63018164279\\
1.40803520088002	1280.43462140246\\
1.40813520338008	1284.25052031803\\
1.40823520588015	1288.07214881156\\
1.40833520838021	1291.89950688303\\
1.40843521088027	1295.7268649545\\
1.40853521338033	1299.56568218188\\
1.4086352158804	1303.41022898721\\
1.40873521838046	1307.25477579253\\
1.40883522088052	1311.11078175377\\
1.40893522338058	1314.97251729295\\
1.40903522588065	1318.83998241008\\
1.40913522838071	1322.70744752721\\
1.40923523088077	1326.58637180025\\
1.40933523338083	1330.47102565124\\
1.4094352358809	1334.35567950222\\
1.40953523838096	1338.25179250911\\
1.40963524088102	1342.147905516\\
1.40973524338108	1346.05547767879\\
1.40983524588115	1349.96877941954\\
1.40993524838121	1353.88208116028\\
1.41003525088127	1357.80684205693\\
1.41013525338133	1361.73733253152\\
1.4102352558814	1365.66782300612\\
1.41033525838146	1369.60977263662\\
1.41043526088152	1373.55745184507\\
1.41053526338158	1377.50513105353\\
1.41063526588165	1381.46426941788\\
1.41073526838171	1385.42340778223\\
1.41083527088177	1389.39400530249\\
1.41093527338183	1393.36460282275\\
1.4110352758819	1397.34665949891\\
1.41113527838196	1401.33444575302\\
1.41123528088202	1405.32223200713\\
1.41133528338208	1409.32147741714\\
1.41143528588215	1413.32072282715\\
1.41153528838221	1417.33142739307\\
1.41163529088227	1421.34213195898\\
1.41173529338233	1425.3642956808\\
1.4118352958824	1429.39218898057\\
1.41193529838246	1433.42008228034\\
1.41203530088252	1437.45943473601\\
1.41213530338258	1441.49878719169\\
1.41223530588265	1445.54959880326\\
1.41233530838271	1449.60041041484\\
1.41243531088277	1453.66268118231\\
1.41253531338283	1457.72495194979\\
1.4126353158829	1461.79868187317\\
1.41273531838296	1465.87241179655\\
1.41283532088302	1469.95760087583\\
1.41293532338308	1474.04278995512\\
1.41303532588315	1478.1394381903\\
1.41313532838321	1482.23608642549\\
1.41323533088327	1486.34419381658\\
1.41333533338333	1490.45230120766\\
1.4134353358834	1494.57186775466\\
1.41353533838346	1498.69143430165\\
1.41363534088352	1502.82246000454\\
1.41373534338358	1506.95348570743\\
1.41383534588365	1511.09597056623\\
1.41393534838371	1515.23845542502\\
1.41403535088377	1519.38666986177\\
1.41413535338383	1523.54634345442\\
1.4142353558839	1527.70601704707\\
1.41433535838396	1531.87714979562\\
1.41443536088402	1536.04828254418\\
1.41453536338408	1540.23087444863\\
1.41463536588415	1544.41346635309\\
1.41473536838421	1548.60178783549\\
1.41483537088427	1552.8015684738\\
1.41493537338433	1557.00134911211\\
1.4150353758844	1561.21258890632\\
1.41513537838446	1565.42382870053\\
1.41523538088452	1569.64652765065\\
1.41533538338458	1573.86922660076\\
1.41543538588465	1578.09765512883\\
1.41553538838471	1582.33754281279\\
1.41563539088477	1586.57743049676\\
1.41573539338483	1590.82877733663\\
1.4158353958849	1595.0801241765\\
1.41593539838496	1599.33720059433\\
1.41603540088502	1603.60573616805\\
1.41613540338508	1607.87427174178\\
1.41623540588515	1612.14853689345\\
1.41633540838521	1616.43426120103\\
1.41643541088527	1620.71998550861\\
1.41653541338533	1625.01716897209\\
1.4166354158854	1629.31435243557\\
1.41673541838546	1633.617265477\\
1.41683542088552	1637.93163767434\\
1.41693542338558	1642.24600987167\\
1.41703542588565	1646.56611164696\\
1.41713542838571	1650.89767257815\\
1.41723543088577	1655.22923350934\\
1.41733543338583	1659.56652401848\\
1.4174354358859	1663.91527368352\\
1.41753543838596	1668.26402334856\\
1.41763544088602	1672.61850259156\\
1.41773544338608	1676.98444099046\\
1.41783544588615	1681.35037938935\\
1.41793544838621	1685.7220473662\\
1.41803545088627	1690.10517449895\\
1.41813545338633	1694.4883016317\\
1.4182354558864	1698.8771583424\\
1.41833545838646	1703.27747420901\\
1.41843546088652	1707.67779007561\\
1.41853546338658	1712.08383552017\\
1.41863546588665	1716.49561054268\\
1.41873546838671	1720.91884472109\\
1.41883547088677	1725.3420788995\\
1.41893547338683	1729.77104265586\\
1.4190354758869	1734.21146556812\\
1.41913547838696	1738.65188848039\\
1.41923548088702	1743.0980409706\\
1.41933548338708	1747.55565261672\\
1.41943548588715	1752.01326426284\\
1.41953548838721	1756.47660548691\\
1.41963549088727	1760.94567628893\\
1.41973549338733	1765.42620624685\\
1.4198354958874	1769.90673620477\\
1.41993549838746	1774.39299574065\\
1.42003550088752	1778.88498485447\\
1.42013550338758	1783.3884331242\\
1.42023550588765	1787.89188139393\\
1.42033550838771	1792.40105924161\\
1.42043551088777	1796.91596666724\\
1.42053551338783	1801.44233324877\\
1.4206355158879	1805.96869983031\\
1.42073551838796	1810.50079598979\\
1.42083552088802	1815.04435130518\\
1.42093552338808	1819.58790662057\\
1.42103552588815	1824.1371915139\\
1.42113552838821	1828.69220598519\\
1.42123553088827	1833.25867961239\\
1.42133553338833	1837.82515323958\\
1.4214355358884	1842.39735644472\\
1.42153553838846	1846.97528922782\\
1.42163554088852	1851.55895158887\\
1.42173554338858	1856.15407310581\\
1.42183554588865	1860.74919462276\\
1.42193554838871	1865.35004571766\\
1.42203555088877	1869.95662639052\\
1.42213555338883	1874.57466621927\\
1.4222355558889	1879.19270604803\\
1.42233555838896	1883.81647545473\\
1.42243556088902	1888.44597443939\\
1.42253556338908	1893.08693257995\\
1.42263556588915	1897.72789072051\\
1.42273556838921	1902.37457843902\\
1.42283557088927	1907.02699573548\\
1.42293557338933	1911.68514260989\\
1.4230355758894	1916.35474864021\\
1.42313557838946	1921.02435467053\\
1.42323558088952	1925.69969027879\\
1.42333558338958	1930.38075546501\\
1.42343558588965	1935.07327980713\\
1.42353558838971	1939.76580414926\\
1.42363559088977	1944.46405806933\\
1.42373559338983	1949.16804156735\\
1.4238355958899	1953.87775464333\\
1.42393559838996	1958.59892687521\\
1.42403560089002	1963.32009910708\\
1.42413560339008	1968.04700091691\\
1.42423560589015	1972.77963230469\\
1.42433560839021	1977.51799327043\\
1.42443561089027	1982.26781339206\\
1.42453561339033	1987.01763351369\\
1.4246356158904	1991.77318321328\\
1.42473561839046	1996.53446249082\\
1.42483562089052	2001.30147134631\\
1.42493562339058	2006.0799393577\\
1.42503562589065	2010.85840736909\\
1.42513562839071	2015.64260495843\\
1.42523563089077	2020.43253212572\\
1.42533563339083	2025.22818887097\\
1.4254356358909	2030.03530477212\\
1.42553563839096	2034.84242067326\\
1.42563564089102	2039.65526615236\\
1.42573564339108	2044.47384120941\\
1.42583564589115	2049.29814584442\\
1.42593564839121	2054.13390963532\\
1.42603565089127	2058.96967342622\\
1.42613565339133	2063.81116679508\\
1.4262356558914	2068.65838974189\\
1.42633565839146	2073.51134226664\\
1.42643566089152	2078.3757539473\\
1.42653566339158	2083.24016562797\\
1.42663566589165	2088.11030688658\\
1.42673566839171	2092.98617772314\\
1.42683567089177	2097.86777813766\\
1.42693567339183	2102.76083770807\\
1.4270356758919	2107.65389727849\\
1.42713567839196	2112.55268642686\\
1.42723568089202	2117.45720515318\\
1.42733568339208	2122.36745345745\\
1.42743568589215	2127.28916091762\\
1.42753568839221	2132.2108683778\\
1.42763569089227	2137.13830541592\\
1.42773569339233	2142.071472032\\
1.4278356958924	2147.01036822603\\
1.42793569839246	2151.96072357596\\
1.42803570089252	2156.91107892589\\
1.42813570339258	2161.86716385377\\
1.42823570589265	2166.8289783596\\
1.42833570839271	2171.79652244339\\
1.42843571089277	2176.77552568307\\
1.42853571339283	2181.75452892276\\
1.4286357158929	2186.7392617404\\
1.42873571839296	2191.72972413599\\
1.42883572089302	2196.72591610953\\
1.42893572339308	2201.73356723897\\
1.42903572589315	2206.74121836841\\
1.42913572839321	2211.75459907581\\
1.42923573089327	2216.77370936115\\
1.42933573339333	2221.8042788024\\
1.4294357358934	2226.83484824365\\
1.42953573839346	2231.87114726285\\
1.42963574089352	2236.91317586\\
1.42973574339358	2241.96093403511\\
1.42983574589365	2247.02015136611\\
1.42993574839371	2252.07936869712\\
1.43003575089377	2257.14431560607\\
1.43013575339383	2262.21499209298\\
1.4302357558939	2267.29139815784\\
1.43033575839396	2272.3792633786\\
1.43043576089402	2277.46712859936\\
1.43053576339408	2282.56072339808\\
1.43063576589415	2287.66004777474\\
1.43073576839421	2292.77083130731\\
1.43083577089427	2297.88161483987\\
1.43093577339433	2302.99812795039\\
1.4310357758944	2308.12037063886\\
1.43113577839446	2313.24834290528\\
1.43123578089452	2318.38777432761\\
1.43133578339458	2323.52720574993\\
1.43143578589465	2328.6723667502\\
1.43153578839471	2333.82325732843\\
1.43163579089477	2338.98560706256\\
1.43173579339483	2344.14795679669\\
1.4318357958949	2349.31603610877\\
1.43193579839496	2354.4898449988\\
1.43203580089502	2359.67511304473\\
1.43213580339508	2364.86038109067\\
1.43223580589515	2370.05137871455\\
1.43233580839521	2375.24810591639\\
1.43243581089527	2380.45629227413\\
1.43253581339533	2385.66447863187\\
1.4326358158954	2390.87839456756\\
1.43273581839546	2396.0980400812\\
1.43283582089552	2401.32914475074\\
1.43293582339558	2406.56024942029\\
1.43303582589565	2411.79708366778\\
1.43313582839571	2417.03964749323\\
1.43323583089577	2422.29367047458\\
1.43333583339583	2427.54769345593\\
1.4334358358959	2432.80744601523\\
1.43353583839596	2438.07865773044\\
1.43363584089602	2443.34986944564\\
1.43373584339608	2448.62681073879\\
1.43383584589615	2453.9094816099\\
1.43393584839621	2459.20361163691\\
1.43403585089627	2464.49774166392\\
1.43413585339633	2469.79760126888\\
1.4342358558964	2475.10892002974\\
1.43433585839646	2480.4202387906\\
1.43443586089652	2485.73728712942\\
1.43453586339659	2491.06006504618\\
1.43463586589665	2496.39430211885\\
1.43473586839671	2501.72853919152\\
1.43483587089677	2507.06850584214\\
1.43493587339683	2512.41993164866\\
1.4350358758969	2517.77135745518\\
1.43513587839696	2523.12851283966\\
1.43523588089702	2528.49712738003\\
1.43533588339708	2533.86574192041\\
1.43543588589715	2539.24008603873\\
1.43553588839721	2544.62588931296\\
1.43563589089727	2550.01169258719\\
1.43573589339733	2555.40322543937\\
1.4358358958974	2560.80621744746\\
1.43593589839746	2566.20920945554\\
1.43603590089752	2571.61793104158\\
1.43613590339758	2577.03811178351\\
1.43623590589765	2582.45829252545\\
1.43633590839771	2587.88420284534\\
1.43643591089777	2593.32157232113\\
1.43653591339784	2598.75894179692\\
1.4366359158979	2604.20204085067\\
1.43673591839796	2609.65659906031\\
1.43683592089802	2615.11115726996\\
1.43693592339808	2620.57717463551\\
1.43703592589815	2626.04319200105\\
1.43713592839821	2631.51493894455\\
1.43723593089827	2636.99814504395\\
1.43733593339833	2642.48135114336\\
1.4374359358984	2647.97601639866\\
1.43753593839846	2653.47068165397\\
1.43763594089852	2658.97107648722\\
1.43773594339859	2664.48293047638\\
1.43783594589865	2669.99478446554\\
1.43793594839871	2675.5180976106\\
1.43803595089877	2681.04141075566\\
1.43813595339883	2686.57045347867\\
1.4382359558989	2692.11095535759\\
1.43833595839896	2697.6514572365\\
1.43843596089902	2703.20341827132\\
1.43853596339909	2708.75537930614\\
1.43863596589915	2714.31879949686\\
1.43873596839921	2719.88221968758\\
1.43883597089927	2725.45136945625\\
1.43893597339933	2731.03197838083\\
1.4390359758994	2736.6125873054\\
1.43913597839946	2742.20465538588\\
1.43923598089952	2747.79672346635\\
1.43933598339958	2753.40025070273\\
1.43943598589965	2759.00377793911\\
1.43953598839971	2764.61876433139\\
1.43963599089977	2770.23375072368\\
1.43973599339984	2775.86019627186\\
1.4398359958999	2781.48664182005\\
1.43993599839996	2787.12454652413\\
1.44003600090002	2792.76245122822\\
1.44013600340008	2798.41181508821\\
1.44023600590015	2804.0611789482\\
1.44033600840021	2809.72200196409\\
1.44043601090027	2815.38282497999\\
1.44053601340034	2821.05510715178\\
1.4406360159004	2826.72738932358\\
1.44073601840046	2832.41113065127\\
1.44083602090052	2838.09487197897\\
1.44093602340059	2843.79007246257\\
1.44103602590065	2849.49100252412\\
1.44113602840071	2855.19193258568\\
1.44123603090077	2860.90432180313\\
1.44133603340083	2866.61671102058\\
1.4414360359009	2872.34055939394\\
1.44153603840096	2878.0644077673\\
1.44163604090102	2883.79971529656\\
1.44173604340109	2889.54075240377\\
1.44183604590115	2895.28178951098\\
1.44193604840121	2901.03428577409\\
1.44203605090127	2906.78678203721\\
1.44213605340133	2912.55073745622\\
1.4422360559014	2918.32042245319\\
1.44233605840146	2924.09010745016\\
1.44243606090152	2929.87125160303\\
1.44253606340159	2935.6523957559\\
1.44263606590165	2941.44499906467\\
1.44273606840171	2947.24333195139\\
1.44283607090177	2953.04166483812\\
1.44293607340184	2958.85145688074\\
1.4430360759019	2964.66697850132\\
1.44313607840196	2970.4825001219\\
1.44323608090202	2976.30948089838\\
1.44333608340208	2982.14219125281\\
1.44343608590215	2987.97490160724\\
1.44353608840221	2993.81907111758\\
1.44363609090227	2999.66897020586\\
1.44373609340234	3005.5245988721\\
1.4438360959024	3011.38022753834\\
1.44393609840246	3017.24731536048\\
1.44403610090252	3023.12013276057\\
1.44413610340259	3028.99295016066\\
1.44423610590265	3034.87722671665\\
1.44433610840271	3040.7672328506\\
1.44443611090277	3046.66296856249\\
1.44453611340284	3052.55870427439\\
1.4446361159029	3058.46589914219\\
1.44473611840296	3064.37882358794\\
1.44483612090302	3070.29747761164\\
1.44493612340309	3076.22186121329\\
1.44503612590315	3082.14624481495\\
1.44513612840321	3088.0820875725\\
1.44523613090327	3094.02365990801\\
1.44533613340333	3099.97096182147\\
1.4454361359034	3105.92399331287\\
1.44553613840346	3111.87702480428\\
1.44563614090352	3117.8415154516\\
1.44573614340359	3123.81173567686\\
1.44583614590365	3129.78768548007\\
1.44593614840371	3135.76936486124\\
1.44603615090377	3141.75677382036\\
1.44613615340384	3147.74418277947\\
1.4462361559039	3153.74305089449\\
1.44633615840396	3159.74764858746\\
1.44643616090402	3165.75797585839\\
1.44653616340409	3171.77403270726\\
1.44663616590415	3177.79581913408\\
1.44673616840421	3183.82333513886\\
1.44683617090427	3189.85658072159\\
1.44693617340434	3195.89555588227\\
1.4470361759044	3201.9402606209\\
1.44713617840446	3207.99069493748\\
1.44723618090452	3214.04112925406\\
1.44733618340459	3220.10302272655\\
1.44743618590465	3226.17064577698\\
1.44753618840471	3232.24399840537\\
1.44763619090477	3238.32308061171\\
1.44773619340484	3244.40789239599\\
1.4478361959049	3250.49843375824\\
1.44793619840496	3256.59470469843\\
1.44803620090502	3262.69670521657\\
1.44813620340509	3268.80443531267\\
1.44823620590515	3274.91789498671\\
1.44833620840521	3281.03708423871\\
1.44843621090527	3287.16200306866\\
1.44853621340534	3293.29265147656\\
1.4486362159054	3299.43475904036\\
1.44873621840546	3305.57686660416\\
1.44883622090552	3311.72470374591\\
1.44893622340559	3317.87827046562\\
1.44903622590565	3324.03756676328\\
1.44913622840571	3330.20259263888\\
1.44923623090577	3336.37334809244\\
1.44933623340584	3342.54983312395\\
1.4494362359059	3348.73204773341\\
1.44953623840596	3354.91999192083\\
1.44963624090602	3361.11939526414\\
1.44973624340609	3367.31879860746\\
1.44983624590615	3373.52393152873\\
1.44993624840621	3379.73479402794\\
1.45003625090627	3385.95138610511\\
1.45013625340634	3392.17370776023\\
1.4502362559064	3398.40748857126\\
1.45033625840646	3404.64126938228\\
1.45043626090652	3410.88077977126\\
1.45053626340659	3417.12601973818\\
1.45063626590665	3423.37698928306\\
1.45073626840671	3429.63941798384\\
1.45083627090677	3435.90184668462\\
1.45093627340684	3442.17000496335\\
1.4510362759069	3448.44389282003\\
1.45113627840696	3454.72923983262\\
1.45123628090702	3461.0145868452\\
1.45133628340709	3467.30566343574\\
1.45143628590715	3473.60819918218\\
1.45153628840721	3479.91073492862\\
1.45163629090727	3486.21900025301\\
1.45173629340734	3492.5387247333\\
1.4518362959074	3498.85844921359\\
1.45193629840746	3505.18390327184\\
1.45203630090752	3511.52081648598\\
1.45213630340759	3517.85772970013\\
1.45223630590765	3524.20037249223\\
1.45233630840771	3530.55447444023\\
1.45243631090777	3536.90857638823\\
1.45253631340784	3543.26840791418\\
1.4526363159079	3549.63969859604\\
1.45273631840796	3556.01098927789\\
1.45283632090802	3562.39373911565\\
1.45293632340809	3568.77648895341\\
1.45303632590815	3575.17069794707\\
1.45313632840821	3581.56490694073\\
1.45323633090827	3587.97057509029\\
1.45333633340834	3594.37624323985\\
1.4534363359084	3600.79337054532\\
1.45353633840846	3607.21049785078\\
1.45363634090852	3613.63908431215\\
1.45373634340859	3620.06767077352\\
1.45383634590865	3626.50771639079\\
1.45393634840871	3632.94776200806\\
1.45403635090877	3639.39926678123\\
1.45413635340884	3645.85077155441\\
1.4542363559089	3652.31373548348\\
1.45433635840896	3658.77669941256\\
1.45443636090902	3665.25112249754\\
1.45453636340909	3671.73127516046\\
1.45463636590915	3678.21142782339\\
1.45473636840921	3684.70303964223\\
1.45483637090927	3691.20038103901\\
1.45493637340934	3697.69772243579\\
1.4550363759094	3704.20652298848\\
1.45513637840946	3710.72105311912\\
1.45523638090952	3717.23558324976\\
1.45533638340959	3723.76157253629\\
1.45543638590965	3730.29329140079\\
1.45553638840971	3736.82501026528\\
1.45563639090977	3743.36818828567\\
1.45573639340984	3749.91709588402\\
1.4558363959099	3756.47173306031\\
1.45593639840996	3763.02637023661\\
1.45603640091002	3769.59246656881\\
1.45613640341009	3776.16429247896\\
1.45623640591015	3782.74184796706\\
1.45633640841021	3789.31940345516\\
1.45643641091027	3795.90841809917\\
1.45653641341034	3802.50316232112\\
1.4566364159104	3809.10363612103\\
1.45673641841046	3815.70983949889\\
1.45683642091052	3822.3217724547\\
1.45693642341059	3828.93943498846\\
1.45703642591065	3835.55709752222\\
1.45713642841071	3842.18621921189\\
1.45723643091077	3848.8210704795\\
1.45733643341084	3855.46165132507\\
1.4574364359109	3862.10796174858\\
1.45753643841096	3868.76000175005\\
1.45763644091102	3875.41777132947\\
1.45773644341109	3882.08127048684\\
1.45783644591115	3888.75049922217\\
1.45793644841121	3895.42545753544\\
1.45803645091127	3902.10614542667\\
1.45813645341134	3908.79256289584\\
1.4582364559114	3915.48470994297\\
1.45833645841146	3922.18258656805\\
1.45843646091152	3928.88619277108\\
1.45853646341159	3935.59552855206\\
1.45863646591165	3942.310593911\\
1.45873646841171	3949.03138884788\\
1.45883647091177	3955.75791336272\\
1.45893647341184	3962.49589703346\\
1.4590364759119	3969.23388070419\\
1.45913647841196	3975.97759395288\\
1.45923648091202	3982.72703677952\\
1.45933648341209	3989.48220918412\\
1.45943648591215	3996.24311116666\\
1.45953648841221	4003.01547230511\\
1.45963649091227	4009.78783344355\\
1.45973649341234	4016.56592415995\\
1.4598364959124	4023.3497444543\\
1.45993649841246	4030.1392943266\\
1.46003650091252	4036.9403033548\\
1.46013650341259	4043.74131238301\\
1.46023650591265	4050.54805098916\\
1.46033650841271	4057.36624875122\\
1.46043651091277	4064.18444651327\\
1.46053651341284	4071.00837385328\\
1.4606365159129	4077.83803077124\\
1.46073651841296	4084.6791468451\\
1.46083652091302	4091.52026291897\\
1.46093652341309	4098.37283814873\\
1.46103652591315	4105.22541337849\\
1.46113652841321	4112.08371818621\\
1.46123653091327	4118.95348214983\\
1.46133653341334	4125.82324611345\\
1.4614365359134	4132.70446923297\\
1.46153653841346	4139.58569235249\\
1.46163654091352	4146.47837462791\\
1.46173654341359	4153.37105690334\\
1.46183654591365	4160.27519833466\\
1.46193654841371	4167.17933976599\\
1.46203655091377	4174.09494035322\\
1.46213655341384	4181.01054094045\\
1.4622365559139	4187.93760068358\\
1.46233655841396	4194.86466042671\\
1.46243656091402	4201.80317932575\\
1.46253656341409	4208.74169822478\\
1.46263656591415	4215.69167627972\\
1.46273656841421	4222.64738391261\\
1.46283657091427	4229.60309154549\\
1.46293657341434	4236.57025833428\\
1.4630365759144	4243.54315470103\\
1.46313657841446	4250.51605106777\\
1.46323658091452	4257.50040659041\\
1.46333658341459	4264.49049169101\\
1.46343658591465	4271.48057679161\\
1.46353658841471	4278.4821210481\\
1.46363659091477	4285.48939488255\\
1.46373659341484	4292.50239829496\\
1.4638365959149	4299.52113128531\\
1.46393659841496	4306.53986427566\\
1.46403660091502	4313.57005642192\\
1.46413660341509	4320.60597814612\\
1.46423660591515	4327.64762944828\\
1.46433660841521	4334.69501032839\\
1.46443661091527	4341.74812078645\\
1.46453661341534	4348.80123124451\\
1.4646366159154	4355.86580085847\\
1.46473661841546	4362.93610005039\\
1.46483662091552	4370.01212882025\\
1.46493662341559	4377.09388716807\\
1.46503662591565	4384.18137509384\\
1.46513662841571	4391.27459259756\\
1.46523663091577	4398.37353967923\\
1.46533663341584	4405.47821633885\\
1.4654366359159	4412.58862257643\\
1.46553663841596	4419.70475839195\\
1.46563664091602	4426.82662378543\\
1.46573664341609	4433.95421875685\\
1.46583664591615	4441.08754330623\\
1.46593664841621	4448.22659743356\\
1.46603665091627	4455.37711071679\\
1.46613665341634	4462.52762400003\\
1.4662366559164	4469.68386686121\\
1.46633665841646	4476.84583930035\\
1.46643666091652	4484.01354131743\\
1.46653666341659	4491.18697291247\\
1.46663666591665	4498.37186366341\\
1.46673666841671	4505.55675441435\\
1.46683667091677	4512.74737474324\\
1.46693667341684	4519.94372465009\\
1.4670366759169	4527.15153371283\\
1.46713667841696	4534.35934277558\\
1.46723668091702	4541.57288141628\\
1.46733668341709	4548.79787921288\\
1.46743668591715	4556.02287700947\\
1.46753668841721	4563.25360438403\\
1.46763669091727	4570.49579091448\\
1.46773669341734	4577.73797744493\\
1.4678366959174	4584.98589355334\\
1.46793669841746	4592.24526881765\\
1.46803670091752	4599.50464408195\\
1.46813670341759	4606.77547850216\\
1.46823670591765	4614.04631292237\\
1.46833670841771	4621.32860649849\\
1.46843671091777	4628.6109000746\\
1.46853671341784	4635.90465280661\\
1.4686367159179	4643.19840553863\\
1.46873671841796	4650.50361742655\\
1.46883672091802	4657.81455889242\\
1.46893672341809	4665.12550035829\\
1.46903672591815	4672.44790098006\\
1.46913672841821	4679.77030160183\\
1.46923673091827	4687.10416137951\\
1.46933673341834	4694.44375073513\\
1.4694367359184	4701.78334009076\\
1.46953673841846	4709.13438860228\\
1.46963674091852	4716.49116669176\\
1.46973674341859	4723.8536743592\\
1.46983674591865	4731.21618202663\\
1.46993674841871	4738.59014884996\\
1.47003675091877	4745.96984525125\\
1.47013675341884	4753.35527123048\\
1.4702367559189	4760.74069720972\\
1.47033675841896	4768.13758234486\\
1.47043676091902	4775.54019705795\\
1.47053676341909	4782.94854134899\\
1.47063676591915	4790.36261521798\\
1.47073676841921	4797.78241866493\\
1.47083677091927	4805.20795168982\\
1.47093677341934	4812.63921429267\\
1.4710367759194	4820.07620647347\\
1.47113677841946	4827.51892823222\\
1.47123678091952	4834.96737956892\\
1.47133678341959	4842.42156048357\\
1.47143678591965	4849.88147097617\\
1.47153678841971	4857.34711104673\\
1.47163679091977	4864.81848069523\\
1.47173679341984	4872.29557992169\\
1.4718367959199	4879.7784087261\\
1.47193679841996	4887.26696710846\\
1.47203680092002	4894.76125506877\\
1.47213680342009	4902.26700218498\\
1.47223680592015	4909.7727493012\\
1.47233680842021	4917.28422599536\\
1.47243681092027	4924.80143226748\\
1.47253681342034	4932.32436811755\\
1.4726368159204	4939.85876312352\\
1.47273681842046	4947.39315812949\\
1.47283682092052	4954.93328271341\\
1.47293682342059	4962.48486645323\\
1.47303682592065	4970.03645019306\\
1.47313682842071	4977.59376351083\\
1.47323683092077	4985.16253598451\\
1.47333683342084	4992.73130845819\\
1.4734368359209	5000.30581050982\\
1.47353683842096	5007.89177171735\\
1.47363684092102	5015.47773292488\\
1.47373684342109	5023.07515328832\\
1.47383684592115	5030.67257365175\\
1.47393684842121	5038.28145317109\\
1.47403685092127	5045.89033269043\\
1.47413685342134	5053.51067136567\\
1.4742368559214	5061.13673961886\\
1.47433685842146	5068.76280787205\\
1.47443686092152	5076.40033528114\\
1.47453686342159	5084.03786269024\\
1.47463686592165	5091.68684925523\\
1.47473686842171	5099.34156539818\\
1.47483687092177	5106.99628154113\\
1.47493687342184	5114.66245683998\\
1.4750368759219	5122.33436171678\\
1.47513687842196	5130.01199617153\\
1.47523688092202	5137.68963062629\\
1.47533688342209	5145.37872423694\\
1.47543688592215	5153.07354742555\\
1.47553688842221	5160.77410019211\\
1.47563689092227	5168.48038253662\\
1.47573689342234	5176.19239445908\\
1.4758368959224	5183.90440638154\\
1.47593689842246	5191.6278774599\\
1.47603690092252	5199.35707811622\\
1.47613690342259	5207.09200835048\\
1.47623690592265	5214.8326681627\\
1.47633690842271	5222.57905755287\\
1.47643691092277	5230.33117652099\\
1.47653691342284	5238.08902506706\\
1.4766369159229	5245.85260319108\\
1.47673691842296	5253.62191089306\\
1.47683692092302	5261.40267775093\\
1.47693692342309	5269.18344460881\\
1.47703692592315	5276.96994104464\\
1.47713692842321	5284.76216705842\\
1.47723693092327	5292.56012265015\\
1.47733693342334	5300.36380781983\\
1.4774369359234	5308.17895214541\\
1.47753693842346	5315.994096471\\
1.47763694092352	5323.81497037453\\
1.47773694342359	5331.64157385602\\
1.47783694592365	5339.47963649341\\
1.47793694842371	5347.3176991308\\
1.47803695092377	5355.16149134614\\
1.47813695342384	5363.01674271739\\
1.4782369559239	5370.87199408863\\
1.47833695842396	5378.73870461578\\
1.47843696092402	5386.60541514292\\
1.47853696342409	5394.48358482597\\
1.47863696592415	5402.36175450902\\
1.47873696842421	5410.25138334797\\
1.47883697092427	5418.14101218692\\
1.47893697342434	5426.04210018178\\
1.4790369759244	5433.94318817663\\
1.47913697842446	5441.85573532739\\
1.47923698092452	5449.76828247814\\
1.47933698342459	5457.6922887848\\
1.47943698592465	5465.62202466941\\
1.47953698842471	5473.55176055402\\
1.47963699092477	5481.49295559454\\
1.47973699342484	5489.439880213\\
1.4798369959249	5497.39253440942\\
1.47993699842496	5505.34518860583\\
1.48003700092502	5513.30930195815\\
1.48013700342509	5521.27914488842\\
1.48023700592515	5529.25471739664\\
1.48033700842521	5537.23601948282\\
1.48043701092527	5545.22305114694\\
1.48053701342534	5553.21581238901\\
1.4806370159254	5561.21430320904\\
1.48073701842546	5569.21852360702\\
1.48083702092552	5577.22847358295\\
1.48093702342559	5585.24415313683\\
1.48103702592565	5593.26556226866\\
1.48113702842571	5601.29270097844\\
1.48123703092577	5609.32556926618\\
1.48133703342584	5617.36416713186\\
1.4814370359259	5625.4084945755\\
1.48153703842596	5633.45855159708\\
1.48163704092602	5641.51433819662\\
1.48173704342609	5649.57585437412\\
1.48183704592615	5657.64882970751\\
1.48193704842621	5665.7218050409\\
1.48203705092627	5673.80050995225\\
1.48213705342634	5681.88494444154\\
1.4822370559264	5689.98083808674\\
1.48233705842646	5698.07673173194\\
1.48243706092652	5706.17835495509\\
1.48253706342659	5714.29143733414\\
1.48263706592665	5722.40451971319\\
1.48273706842671	5730.55197955995\\
1.48283707092677	5738.6306844713\\
1.48293707342684	5746.76668516216\\
1.4830370759269	5754.90268585301\\
1.48313707842696	5763.09598232339\\
1.48323708092702	5771.23198301424\\
1.48333708342709	5779.3679837051\\
1.48343708592715	5787.56128017547\\
1.48353708842721	5795.69728086633\\
1.48363709092727	5803.8905773367\\
1.48373709342734	5812.08387380707\\
1.4838370959274	5820.27717027744\\
1.48393709842746	5828.4131709683\\
1.48403710092752	5836.66376321818\\
1.48413710342759	5844.85705968855\\
1.48423710592765	5853.05035615892\\
1.48433710842771	5861.24365262929\\
1.48443711092777	5869.49424487918\\
1.48453711342784	5877.68754134955\\
1.4846371159279	5885.93813359943\\
1.48473711842796	5894.18872584932\\
1.48483712092802	5902.4393180992\\
1.48493712342809	5910.68991034909\\
1.48503712592815	5918.94050259897\\
1.48513712842821	5927.19109484885\\
1.48523713092827	5935.44168709874\\
1.48533713342834	5943.74957512813\\
1.4854371359284	5952.00016737802\\
1.48553713842846	5960.30805540741\\
1.48563714092852	5968.61594343681\\
1.48573714342859	5976.92383146621\\
1.48583714592865	5985.17442371609\\
1.48593714842871	5993.48231174549\\
1.48603715092877	6001.8474955544\\
1.48613715342884	6010.1553835838\\
1.4862371559289	6018.46327161319\\
1.48633715842896	6026.8284554221\\
1.48643716092902	6035.1363434515\\
1.48653716342909	6043.50152726041\\
1.48663716592915	6051.80941528981\\
1.48673716842921	6060.17459909872\\
1.48683717092927	6068.53978290763\\
1.48693717342934	6076.90496671654\\
1.4870371759294	6085.27015052545\\
1.48713717842946	6093.69263011387\\
1.48723718092952	6102.05781392278\\
1.48733718342959	6110.4802935112\\
1.48743718592965	6118.84547732011\\
1.48753718842971	6127.26795690854\\
1.48763719092977	6135.69043649696\\
1.48773719342984	6144.05562030587\\
1.4878371959299	6152.47809989429\\
1.48793719842996	6160.90057948272\\
1.48803720093002	6169.38035485065\\
1.48813720343009	6177.80283443908\\
1.48823720593015	6186.2253140275\\
1.48833720843021	6194.70508939543\\
1.48843721093027	6203.12756898386\\
1.48853721343034	6211.60734435179\\
1.4886372159304	6220.08711971973\\
1.48873721843046	6228.56689508767\\
1.48883722093052	6237.0466704556\\
1.48893722343059	6245.52644582354\\
1.48903722593065	6254.00622119147\\
1.48913722843071	6262.48599655941\\
1.48923723093077	6271.02306770686\\
1.48933723343084	6279.5028430748\\
1.4894372359309	6288.03991422224\\
1.48953723843096	6296.57698536969\\
1.48963724093102	6305.05676073763\\
1.48973724343109	6313.59383188508\\
1.48983724593115	6322.13090303253\\
1.48993724843121	6330.72526995949\\
1.49003725093127	6339.26234110694\\
1.49013725343134	6347.79941225439\\
1.4902372559314	6356.39377918135\\
1.49033725843146	6364.9308503288\\
1.49043726093152	6373.52521725576\\
1.49053726343159	6382.06228840321\\
1.49063726593165	6390.65665533018\\
1.49073726843171	6399.25102225714\\
1.49083727093177	6407.8453891841\\
1.49093727343184	6416.49705189058\\
1.4910372759319	6425.09141881754\\
1.49113727843196	6433.6857857445\\
1.49123728093202	6442.33744845098\\
1.49133728343209	6450.93181537794\\
1.49143728593215	6459.58347808441\\
1.49153728843221	6468.23514079089\\
1.49163729093227	6476.88680349737\\
1.49173729343234	6485.48117042433\\
1.4918372959324	6494.19012891032\\
1.49193729843246	6502.84179161679\\
1.49203730093252	6511.49345432327\\
1.49213730343259	6520.14511702974\\
1.49223730593265	6528.85407551573\\
1.49233730843271	6537.56303400172\\
1.49243731093277	6546.2146967082\\
1.49253731343284	6554.92365519418\\
1.4926373159329	6563.63261368017\\
1.49273731843296	6572.34157216616\\
1.49283732093302	6581.05053065215\\
1.49293732343309	6589.75948913814\\
1.49303732593315	6598.52574340364\\
1.49313732843321	6607.23470188963\\
1.49323733093327	6616.00095615513\\
1.49333733343334	6624.70991464112\\
1.4934373359334	6633.47616890662\\
1.49353733843346	6642.24242317212\\
1.49363734093352	6651.00867743762\\
1.49373734343359	6659.77493170312\\
1.49383734593365	6668.54118596863\\
1.49393734843371	6677.30744023413\\
1.49403735093377	6686.07369449963\\
1.49413735343384	6694.89724454464\\
1.4942373559339	6703.66349881015\\
1.49433735843396	6712.48704885516\\
1.49443736093402	6721.31059890017\\
1.49453736343409	6730.13414894519\\
1.49463736593415	6738.9576989902\\
1.49473736843421	6747.78124903522\\
1.49483737093427	6756.60479908023\\
1.49493737343434	6765.42834912525\\
1.4950373759344	6774.30919494977\\
1.49513737843446	6783.13274499479\\
1.49523738093452	6792.01359081932\\
1.49533738343459	6800.83714086433\\
1.49543738593465	6809.71798668886\\
1.49553738843471	6818.59883251339\\
1.49563739093477	6827.47967833792\\
1.49573739343484	6836.36052416244\\
1.4958373959349	6845.24136998697\\
1.49593739843496	6854.17951159101\\
1.49603740093502	6863.06035741554\\
1.49613740343509	6871.99849901958\\
1.49623740593515	6880.87934484411\\
1.49633740843521	6889.81748644815\\
1.49643741093527	6898.75562805219\\
1.49653741343534	6907.69376965623\\
1.4966374159354	6916.63191126027\\
1.49673741843546	6925.57005286431\\
1.49683742093552	6934.50819446835\\
1.49693742343559	6943.44633607239\\
1.49703742593565	6952.44177345595\\
1.49713742843571	6961.37991505999\\
1.49723743093577	6970.37535244354\\
1.49733743343584	6979.3707898271\\
1.4974374359359	6988.36622721065\\
1.49753743843596	6997.3616645942\\
1.49763744093602	7006.35710197776\\
1.49773744343609	7015.35253936131\\
1.49783744593615	7024.34797674487\\
1.49793744843621	7033.40070990793\\
1.49803745093627	7042.39614729149\\
1.49813745343634	7051.44888045455\\
1.4982374559364	7060.44431783811\\
1.49833745843646	7069.49705100118\\
1.49843746093652	7078.54978416424\\
1.49853746343659	7087.60251732731\\
1.49863746593665	7096.65525049038\\
1.49873746843671	7105.70798365344\\
1.49883747093677	7114.81801259602\\
1.49893747343684	7123.87074575909\\
1.4990374759369	7132.92347892216\\
1.49913747843696	7142.03350786474\\
1.49923748093702	7151.14353680732\\
1.49933748343709	7160.2535657499\\
1.49943748593715	7169.30629891296\\
1.49953748843721	7178.47362363506\\
1.49963749093727	7187.58365257764\\
1.49973749343734	7196.69368152022\\
1.4998374959374	7205.8037104628\\
1.49993749843746	7214.97103518489\\
1.50003750093752	7224.08106412747\\
1.50013750343759	7233.24838884956\\
1.50023750593765	7242.41571357166\\
1.50033750843771	7251.52574251424\\
1.50043751093777	7260.69306723633\\
1.50053751343784	7269.86039195842\\
1.5006375159379	7279.08501246003\\
1.50073751843796	7288.25233718212\\
1.50083752093802	7297.41966190422\\
1.50093752343809	7306.64428240582\\
1.50103752593815	7315.81160712792\\
1.50113752843821	7325.03622762952\\
1.50123753093827	7334.26084813113\\
1.50133753343834	7343.48546863274\\
1.5014375359384	7352.71008913434\\
1.50153753843846	7361.93470963595\\
1.50163754093852	7371.15933013755\\
1.50173754343859	7380.38395063916\\
1.50183754593865	7389.66586692028\\
1.50193754843871	7398.89048742189\\
1.50203755093877	7408.172403703\\
1.50213755343884	7417.39702420461\\
1.5022375559389	7426.67894048573\\
1.50233755843896	7435.96085676685\\
1.50243756093902	7445.24277304797\\
1.50253756343909	7454.52468932909\\
1.50263756593915	7463.86390138972\\
1.50273756843921	7473.14581767084\\
1.50283757093927	7482.42773395196\\
1.50293757343934	7491.76694601259\\
1.5030375759394	7501.04886229371\\
1.50313757843946	7510.38807435434\\
1.50323758093952	7519.72728641498\\
1.50333758343959	7529.06649847561\\
1.50343758593965	7538.40571053624\\
1.50353758843971	7547.74492259687\\
1.50363759093977	7557.14143043702\\
1.50373759343984	7566.48064249765\\
1.5038375959399	7575.81985455828\\
1.50393759843996	7585.21636239843\\
1.50403760094002	7594.61287023857\\
1.50413760344009	7603.95208229921\\
1.50423760594015	7613.34859013935\\
1.50433760844021	7622.7450979795\\
1.50443761094027	7632.14160581964\\
1.50453761344034	7641.5954094393\\
1.5046376159404	7650.99191727945\\
1.50473761844046	7660.3884251196\\
1.50483762094052	7669.84222873925\\
1.50493762344059	7679.2387365794\\
1.50503762594065	7688.69254019906\\
1.50513762844071	7698.14634381872\\
1.50523763094077	7707.60014743837\\
1.50533763344084	7717.05395105803\\
1.5054376359409	7726.50775467769\\
1.50553763844096	7735.96155829735\\
1.50563764094102	7745.47265769652\\
1.50573764344109	7754.92646131618\\
1.50583764594115	7764.43756071535\\
1.50593764844121	7773.89136433501\\
1.50603765094127	7783.40246373418\\
1.50613765344134	7792.91356313335\\
1.5062376559414	7802.42466253252\\
1.50633765844146	7811.9357619317\\
1.50643766094152	7821.44686133087\\
1.50653766344159	7830.95796073004\\
1.50663766594165	7840.52635590872\\
1.50673766844171	7850.0374553079\\
1.50683767094177	7859.60585048658\\
1.50693767344184	7869.11694988575\\
1.5070376759419	7878.68534506444\\
1.50713767844196	7888.25374024312\\
1.50723768094202	7897.82213542181\\
1.50733768344209	7907.39053060049\\
1.50743768594215	7916.95892577917\\
1.50753768844221	7926.58461673737\\
1.50763769094227	7936.15301191606\\
1.50773769344234	7945.77870287426\\
1.5078376959424	7955.34709805294\\
1.50793769844246	7964.97278901114\\
1.50803770094252	7974.59847996934\\
1.50813770344259	7984.22417092753\\
1.50823770594265	7993.84986188573\\
1.50833770844271	8003.47555284393\\
1.50843771094277	8013.10124380213\\
1.50853771344284	8022.78423053984\\
1.5086377159429	8032.40992149804\\
1.50873771844296	8042.09290823575\\
1.50883772094302	8051.71859919395\\
1.50893772344309	8061.40158593166\\
1.50903772594315	8071.08457266937\\
1.50913772844321	8080.76755940708\\
1.50923773094327	8090.45054614479\\
1.50933773344334	8100.1335328825\\
1.5094377359434	8109.81651962021\\
1.50953773844346	8119.55680213744\\
1.50963774094352	8129.23978887515\\
1.50973774344359	8138.98007139237\\
1.50983774594365	8148.7203539096\\
1.50993774844371	8158.4033406473\\
1.51003775094377	8168.14362316453\\
1.51013775344384	8177.88390568175\\
1.5102377559439	8187.62418819898\\
1.51033775844396	8197.42176649571\\
1.51043776094402	8207.16204901294\\
1.51053776344409	8216.90233153016\\
1.51063776594415	8226.6999098269\\
1.51073776844421	8236.44019234412\\
1.51083777094427	8246.23777064086\\
1.51093777344434	8256.0353489376\\
1.5110377759444	8265.83292723433\\
1.51113777844446	8275.63050553107\\
1.51123778094452	8285.42808382781\\
1.51133778344459	8295.22566212455\\
1.51143778594465	8305.0805362008\\
1.51153778844471	8314.87811449753\\
1.51163779094477	8324.73298857378\\
1.51173779344484	8334.58786265003\\
1.5118377959449	8344.38544094677\\
1.51193779844496	8354.24031502302\\
1.51203780094502	8364.09518909927\\
1.51213780344509	8373.95006317552\\
1.51223780594515	8383.80493725177\\
1.51233780844521	8393.71710710753\\
1.51243781094527	8403.57198118378\\
1.51253781344534	8413.48415103955\\
1.5126378159454	8423.3390251158\\
1.51273781844546	8433.25119497156\\
1.51283782094552	8443.16336482732\\
1.51293782344559	8453.07553468309\\
1.51303782594565	8462.98770453885\\
1.51313782844571	8472.89987439461\\
1.51323783094577	8482.81204425038\\
1.51333783344584	8492.72421410614\\
1.5134378359459	8502.69367974142\\
1.51353783844596	8512.60584959718\\
1.51363784094602	8522.57531523246\\
1.51373784344609	8532.54478086773\\
1.51383784594615	8542.51424650301\\
1.51393784844621	8552.48371213829\\
1.51403785094627	8562.45317777356\\
1.51413785344634	8572.42264340884\\
1.5142378559464	8582.39210904411\\
1.51433785844646	8592.4188704589\\
1.51443786094652	8602.38833609418\\
1.51453786344659	8612.41509750897\\
1.51463786594665	8622.38456314425\\
1.51473786844671	8632.41132455903\\
1.51483787094677	8642.43808597382\\
1.51493787344684	8652.46484738861\\
1.5150378759469	8662.4916088034\\
1.51513787844696	8672.51837021819\\
1.51523788094702	8682.60242741249\\
1.51533788344709	8692.62918882728\\
1.51543788594715	8702.71324602159\\
1.51553788844721	8712.74000743638\\
1.51563789094727	8722.82406463068\\
1.51573789344734	8732.90812182498\\
1.5158378959474	8742.99217901928\\
1.51593789844746	8753.07623621359\\
1.51603790094752	8763.16029340789\\
1.51613790344759	8773.24435060219\\
1.51623790594765	8783.38570357601\\
1.51633790844771	8793.46976077031\\
1.51643791094777	8803.61111374412\\
1.51653791344784	8813.69517093843\\
1.5166379159479	8823.83652391224\\
1.51673791844796	8833.97787688606\\
1.51683792094802	8844.11922985987\\
1.51693792344809	8854.26058283369\\
1.51703792594815	8864.40193580751\\
1.51713792844821	8874.60058456083\\
1.51723793094827	8884.74193753465\\
1.51733793344834	8894.94058628798\\
1.5174379359484	8905.08193926179\\
1.51753793844846	8915.28058801512\\
1.51763794094852	8925.47923676845\\
1.51773794344859	8935.67788552178\\
1.51783794594865	8945.87653427511\\
1.51793794844871	8956.07518302844\\
1.51803795094877	8966.27383178177\\
1.51813795344884	8976.47248053509\\
1.5182379559489	8986.72842506794\\
1.51833795844896	8996.92707382127\\
1.51843796094902	9007.18301835411\\
1.51853796344909	9017.43896288695\\
1.51863796594915	9027.69490741979\\
1.51873796844921	9037.95085195263\\
1.51883797094927	9048.20679648547\\
1.51893797344934	9058.46274101831\\
1.5190379759494	9068.71868555116\\
1.51913797844946	9079.03192586351\\
1.51923798094952	9089.28787039635\\
1.51933798344959	9099.60111070871\\
1.51943798594965	9109.85705524155\\
1.51953798844971	9120.1702955539\\
1.51963799094977	9130.48353586626\\
1.51973799344984	9140.79677617861\\
1.5198379959499	9151.11001649097\\
1.51993799844996	9161.42325680332\\
1.52003800095002	9171.79379289519\\
1.52013800345009	9182.10703320755\\
1.52023800595015	9192.47756929942\\
1.52033800845021	9202.79080961177\\
1.52043801095027	9213.16134570364\\
1.52053801345034	9223.5318817955\\
1.5206380159504	9233.90241788737\\
1.52073801845046	9244.27295397924\\
1.52083802095052	9254.64349007111\\
1.52093802345059	9265.07132194249\\
1.52103802595065	9275.44185803436\\
1.52113802845071	9285.81239412623\\
1.52123803095077	9296.24022599761\\
1.52133803345084	9306.66805786899\\
1.5214380359509	9317.03859396085\\
1.52153803845096	9327.46642583224\\
1.52163804095102	9337.89425770362\\
1.52173804345109	9348.322089575\\
1.52183804595115	9358.80721722589\\
1.52193804845121	9369.23504909727\\
1.52203805095127	9379.66288096865\\
1.52213805345134	9390.14800861955\\
1.5222380559514	9400.63313627044\\
1.52233805845146	9411.06096814182\\
1.52243806095152	9421.54609579272\\
1.52253806345159	9432.03122344361\\
1.52263806595165	9442.51635109451\\
1.52273806845171	9453.0014787454\\
1.52283807095177	9463.54390217581\\
1.52293807345184	9474.0290298267\\
1.5230380759519	9484.5141574776\\
1.52313807845196	9495.056580908\\
1.52323808095202	9505.59900433841\\
1.52333808345209	9516.0841319893\\
1.52343808595215	9526.62655541971\\
1.52353808845221	9537.16897885012\\
1.52363809095227	9547.71140228053\\
1.52373809345234	9558.31112149045\\
1.5238380959524	9568.85354492085\\
1.52393809845246	9579.39596835126\\
1.52403810095252	9589.99568756118\\
1.52413810345259	9600.5954067711\\
1.52423810595265	9611.13783020151\\
1.52433810845271	9621.73754941143\\
1.52443811095277	9632.33726862135\\
1.52453811345284	9642.93698783127\\
1.5246381159529	9653.53670704119\\
1.52473811845296	9664.19372203062\\
1.52483812095302	9674.79344124054\\
1.52493812345309	9685.39316045046\\
1.52503812595315	9696.0501754399\\
1.52513812845321	9706.70719042933\\
1.52523813095327	9717.30690963925\\
1.52533813345334	9727.96392462868\\
1.5254381359534	9738.62093961811\\
1.52553813845346	9749.27795460755\\
1.52563814095352	9759.9922653765\\
1.52573814345359	9770.64928036593\\
1.52583814595365	9781.30629535536\\
1.52593814845371	9792.02060612431\\
1.52603815095377	9802.67762111374\\
1.52613815345384	9813.39193188269\\
1.5262381559539	9824.10624265163\\
1.52633815845396	9834.82055342058\\
1.52643816095402	9845.53486418953\\
1.52653816345409	9856.24917495847\\
1.52663816595415	9866.96348572742\\
1.52673816845421	9877.73509227588\\
1.52683817095427	9888.44940304483\\
1.52693817345434	9899.22100959329\\
1.5270381759544	9909.93532036223\\
1.52713817845446	9920.70692691069\\
1.52723818095452	9931.47853345915\\
1.52733818345459	9942.25014000761\\
1.52743818595465	9953.02174655607\\
1.52753818845471	9963.79335310453\\
1.52763819095477	9974.6222554325\\
1.52773819345484	9985.39386198096\\
1.5278381959549	9996.16546852942\\
1.52793819845496	10006.9943708574\\
1.52803820095502	10017.8232731854\\
1.52813820345509	10028.6521755133\\
1.52823820595515	10039.4810778413\\
1.52833820845521	10050.3099801693\\
1.52843821095527	10061.1388824973\\
1.52853821345534	10071.9677848252\\
1.5286382159554	10082.7966871532\\
1.52873821845546	10093.6828852607\\
1.52883822095552	10104.5117875887\\
1.52893822345559	10115.3979856961\\
1.52903822595565	10126.2841838036\\
1.52913822845571	10137.1703819111\\
1.52923823095577	10148.0565800186\\
1.52933823345584	10158.9427781261\\
1.5294382359559	10169.8289762336\\
1.52953823845596	10180.7151743411\\
1.52963824095602	10191.6586682281\\
1.52973824345609	10202.5448663355\\
1.52983824595615	10213.4883602225\\
1.52993824845621	10224.37455833\\
1.53003825095627	10235.318052217\\
1.53013825345634	10246.261546104\\
1.5302382559564	10257.205039991\\
1.53033825845646	10268.148533878\\
1.53043826095652	10279.1493235445\\
1.53053826345659	10290.0928174315\\
1.53063826595665	10301.0363113185\\
1.53073826845671	10312.037100985\\
1.53083827095677	10323.0378906516\\
1.53093827345684	10333.9813845386\\
1.5310382759569	10344.9821742051\\
1.53113827845696	10355.9829638716\\
1.53123828095702	10366.9837535381\\
1.53133828345709	10377.9845432046\\
1.53143828595715	10389.0426286506\\
1.53153828845721	10400.0434183171\\
1.53163829095727	10411.1015037632\\
1.53173829345734	10422.1022934297\\
1.5318382959574	10433.1603788757\\
1.53193829845746	10444.2184643217\\
1.53203830095752	10455.2765497678\\
1.53213830345759	10466.3346352138\\
1.53223830595765	10477.3927206598\\
1.53233830845771	10488.4508061058\\
1.53243831095777	10499.5088915518\\
1.53253831345784	10510.6242727774\\
1.5326383159579	10521.6823582234\\
1.53273831845796	10532.7977394489\\
1.53283832095802	10543.9131206745\\
1.53293832345809	10555.0285019\\
1.53303832595815	10566.1438831256\\
1.53313832845821	10577.2592643511\\
1.53323833095827	10588.3746455766\\
1.53333833345834	10599.4900268022\\
1.5334383359584	10610.6627038072\\
1.53353833845846	10621.7780850328\\
1.53363834095852	10632.9507620378\\
1.53373834345859	10644.1234390429\\
1.53383834595865	10655.2388202684\\
1.53393834845871	10666.4114972735\\
1.53403835095877	10677.5841742785\\
1.53413835345884	10688.8141470631\\
1.5342383559589	10699.9868240681\\
1.53433835845896	10711.1595010732\\
1.53443836095902	10722.3894738577\\
1.53453836345909	10733.5621508628\\
1.53463836595915	10744.7921236474\\
1.53473836845921	10756.0220964319\\
1.53483837095927	10767.194773437\\
1.53493837345934	10778.4247462215\\
1.5350383759594	10789.7120147856\\
1.53513837845946	10800.9419875702\\
1.53523838095952	10812.1719603547\\
1.53533838345959	10823.4019331393\\
1.53543838595965	10834.6892017034\\
1.53553838845971	10845.9764702675\\
1.53563839095977	10857.206443052\\
1.53573839345984	10868.4937116161\\
1.5358383959599	10879.7809801802\\
1.53593839845996	10891.0682487443\\
1.53603840096002	10902.3555173083\\
1.53613840346009	10913.7000816519\\
1.53623840596015	10924.987350216\\
1.53633840846021	10936.2746187801\\
1.53643841096027	10947.6191831237\\
1.53653841346034	10958.9637474673\\
1.5366384159604	10970.2510160313\\
1.53673841846046	10981.5955803749\\
1.53683842096052	10992.9401447185\\
1.53693842346059	11004.2847090621\\
1.53703842596065	11015.6865691852\\
1.53713842846071	11027.0311335288\\
1.53723843096077	11038.3756978724\\
1.53733843346084	11049.7775579955\\
1.5374384359609	11061.1221223391\\
1.53753843846096	11072.5239824622\\
1.53763844096102	11083.9258425853\\
1.53773844346109	11095.3277027084\\
1.53783844596115	11106.7295628315\\
1.53793844846121	11118.1314229546\\
1.53803845096127	11129.5332830777\\
1.53813845346134	11140.9924389803\\
1.5382384559614	11152.3942991034\\
1.53833845846146	11163.853455006\\
1.53843846096152	11175.2553151291\\
1.53853846346159	11186.7144710318\\
1.53863846596165	11198.1736269344\\
1.53873846846171	11209.632782837\\
1.53883847096177	11221.0919387396\\
1.53893847346184	11232.5510946422\\
1.5390384759619	11244.0675463244\\
1.53913847846196	11255.526702227\\
1.53923848096202	11267.0431539091\\
1.53933848346209	11278.5023098117\\
1.53943848596215	11290.0187614938\\
1.53953848846221	11301.535213176\\
1.53963849096227	11313.0516648581\\
1.53973849346234	11324.5681165402\\
1.5398384959624	11336.0845682224\\
1.53993849846246	11347.6010199045\\
1.54003850096252	11359.1747673661\\
1.54013850346259	11370.6912190483\\
1.54023850596265	11382.2649665099\\
1.54033850846271	11393.8387139716\\
1.54043851096277	11405.3551656537\\
1.54053851346284	11416.9289131153\\
1.5406385159629	11428.502660577\\
1.54073851846296	11440.0764080386\\
1.54083852096302	11451.7074512798\\
1.54093852346309	11463.2811987414\\
1.54103852596315	11474.854946203\\
1.54113852846321	11486.4859894442\\
1.54123853096327	11498.1170326854\\
1.54133853346334	11509.690780147\\
1.5414385359634	11521.3218233882\\
1.54153853846346	11532.9528666293\\
1.54163854096352	11544.5839098705\\
1.54173854346359	11556.2149531116\\
1.54183854596365	11567.9032921323\\
1.54193854846371	11579.5343353735\\
1.54203855096377	11591.2226743941\\
1.54213855346384	11602.8537176353\\
1.5422385559639	11614.5420566559\\
1.54233855846396	11626.2303956766\\
1.54243856096402	11637.9187346973\\
1.54253856346409	11649.607073718\\
1.54263856596415	11661.2954127386\\
1.54273856846421	11672.9837517593\\
1.54283857096427	11684.67209078\\
1.54293857346434	11696.4177255801\\
1.5430385759644	11708.1060646008\\
1.54313857846446	11719.851699401\\
1.54323858096452	11731.5973342012\\
1.54333858346459	11743.3429690014\\
1.54343858596465	11755.0886038015\\
1.54353858846471	11766.8342386017\\
1.54363859096477	11778.5798734019\\
1.54373859346484	11790.3255082021\\
1.5438385959649	11802.1284387818\\
1.54393859846496	11813.874073582\\
1.54403860096502	11825.6770041617\\
1.54413860346509	11837.4226389618\\
1.54423860596515	11849.2255695415\\
1.54433860846521	11861.0285001212\\
1.54443861096527	11872.8314307009\\
1.54453861346534	11884.6343612806\\
1.5446386159654	11896.4945876398\\
1.54473861846546	11908.2975182195\\
1.54483862096552	11920.1577445787\\
1.54493862346559	11931.9606751584\\
1.54503862596565	11943.8209015176\\
1.54513862846571	11955.6811278768\\
1.54523863096577	11967.4840584565\\
1.54533863346584	11979.3442848157\\
1.5454386359659	11991.2618069545\\
1.54553863846596	12003.1220333137\\
1.54563864096602	12014.9822596729\\
1.54573864346609	12026.8424860321\\
1.54583864596615	12038.7600081708\\
1.54593864846621	12050.6775303095\\
1.54603865096627	12062.5377566687\\
1.54613865346634	12074.4552788075\\
1.5462386559664	12086.3728009462\\
1.54633865846646	12098.2903230849\\
1.54643866096652	12110.2078452236\\
1.54653866346659	12122.1826631419\\
1.54663866596665	12134.1001852806\\
1.54673866846671	12146.0750031988\\
1.54683867096677	12157.9925253375\\
1.54693867346684	12169.9673432558\\
1.5470386759669	12181.942161174\\
1.54713867846696	12193.9169790922\\
1.54723868096702	12205.8917970105\\
1.54733868346709	12217.8666149287\\
1.54743868596715	12229.8414328469\\
1.54753868846721	12241.8162507652\\
1.54763869096727	12253.8483644629\\
1.54773869346734	12265.8231823812\\
1.5478386959674	12277.8552960789\\
1.54793869846746	12289.8874097766\\
1.54803870096752	12301.9195234744\\
1.54813870346759	12313.9516371721\\
1.54823870596765	12325.9837508699\\
1.54833870846771	12338.0158645676\\
1.54843871096777	12350.0479782654\\
1.54853871346784	12362.1373877426\\
1.5486387159679	12374.1695014404\\
1.54873871846796	12386.2589109176\\
1.54883872096802	12398.2910246154\\
1.54893872346809	12410.3804340927\\
1.54903872596815	12422.4698435699\\
1.54913872846821	12434.5592530472\\
1.54923873096827	12446.6486625244\\
1.54933873346834	12458.7953677812\\
1.5494387359684	12470.8847772585\\
1.54953873846846	12482.9741867357\\
1.54963874096852	12495.1208919925\\
1.54973874346859	12507.2675972493\\
1.54983874596865	12519.3570067265\\
1.54993874846871	12531.5037119833\\
1.55003875096877	12543.6504172401\\
1.55013875346884	12555.7971224969\\
1.5502387559689	12568.0011235331\\
1.55033875846896	12580.1478287899\\
1.55043876096902	12592.2945340467\\
1.55053876346909	12604.498535083\\
1.55063876596915	12616.7025361193\\
1.55073876846921	12628.849241376\\
1.55083877096927	12641.0532424123\\
1.55093877346934	12653.2572434486\\
1.5510387759694	12665.4612444849\\
1.55113877846946	12677.6652455212\\
1.55123878096952	12689.926542337\\
1.55133878346959	12702.1305433733\\
1.55143878596965	12714.3918401891\\
1.55153878846971	12726.5958412254\\
1.55163879096977	12738.8571380412\\
1.55173879346984	12751.118434857\\
1.5518387959699	12763.3797316728\\
1.55193879846996	12775.6410284886\\
1.55203880097002	12787.9023253044\\
1.55213880347009	12800.1636221202\\
1.55223880597015	12812.424918936\\
1.55233880847021	12824.7435115313\\
1.55243881097027	12837.0048083471\\
1.55253881347034	12849.3234009424\\
1.5526388159704	12861.6419935377\\
1.55273881847046	12873.960586133\\
1.55283882097052	12886.2791787283\\
1.55293882347059	12898.5977713236\\
1.55303882597065	12910.9163639189\\
1.55313882847071	12923.2349565143\\
1.55323883097077	12935.6108448891\\
1.55333883347084	12947.9294374844\\
1.5534388359709	12960.3053258592\\
1.55353883847096	12972.681214234\\
1.55363884097102	12984.9998068294\\
1.55373884347109	12997.3756952042\\
1.55383884597115	13009.751583579\\
1.55393884847121	13022.1847677334\\
1.55403885097127	13034.5606561082\\
1.55413885347134	13046.936544483\\
1.5542388559714	13059.3697286373\\
1.55433885847146	13071.7456170122\\
1.55443886097152	13084.1788011665\\
1.55453886347159	13096.6119853208\\
1.55463886597165	13109.0451694752\\
1.55473886847171	13121.4783536295\\
1.55483887097177	13133.9115377839\\
1.55493887347184	13146.3447219382\\
1.5550388759719	13158.7779060925\\
1.55513887847196	13171.2683860264\\
1.55523888097202	13183.7015701807\\
1.55533888347209	13196.1920501146\\
1.55543888597215	13208.6825300484\\
1.55553888847221	13221.1730099823\\
1.55563889097227	13233.6634899161\\
1.55573889347234	13246.15396985\\
1.5558388959724	13258.6444497838\\
1.55593889847246	13271.1349297177\\
1.55603890097252	13283.6827054311\\
1.55613890347259	13296.1731853649\\
1.55623890597265	13308.7209610783\\
1.55633890847271	13321.2114410121\\
1.55643891097277	13333.7592167255\\
1.55653891347284	13346.3069924389\\
1.5566389159729	13358.8547681522\\
1.55673891847296	13371.4025438656\\
1.55683892097302	13384.0076153585\\
1.55693892347309	13396.5553910718\\
1.55703892597315	13409.1604625647\\
1.55713892847321	13421.7082382781\\
1.55723893097327	13434.313309771\\
1.55733893347334	13446.9183812638\\
1.5574389359734	13459.5234527567\\
1.55753893847346	13472.1285242496\\
1.55763894097352	13484.7335957425\\
1.55773894347359	13497.3386672353\\
1.55783894597365	13509.9437387282\\
1.55793894847371	13522.6061060006\\
1.55803895097377	13535.2111774935\\
1.55813895347384	13547.8735447659\\
1.5582389559739	13560.5359120383\\
1.55833895847396	13573.1982793107\\
1.55843896097402	13585.8606465831\\
1.55853896347409	13598.5230138554\\
1.55863896597415	13611.1853811278\\
1.55873896847421	13623.8477484002\\
1.55883897097427	13636.5674114521\\
1.55893897347434	13649.2297787245\\
1.5590389759744	13661.9494417764\\
1.55913897847446	13674.6118090488\\
1.55923898097452	13687.3314721007\\
1.55933898347459	13700.0511351526\\
1.55943898597465	13712.7707982045\\
1.55953898847471	13725.4904612564\\
1.55963899097477	13738.2674200879\\
1.55973899347484	13750.9870831398\\
1.5598389959749	13763.7067461917\\
1.55993899847496	13776.4837050231\\
1.56003900097502	13789.2606638545\\
1.56013900347509	13802.0376226859\\
1.56023900597515	13814.7572857378\\
1.56033900847521	13827.5342445692\\
1.56043901097527	13840.3684991802\\
1.56053901347534	13853.1454580116\\
1.5606390159754	13865.922416843\\
1.56073901847546	13878.7566714539\\
1.56083902097552	13891.5336302853\\
1.56093902347559	13904.3678848963\\
1.56103902597565	13917.2021395072\\
1.56113902847571	13929.9790983386\\
1.56123903097577	13942.8133529496\\
1.56133903347584	13955.70490334\\
1.5614390359759	13968.5391579509\\
1.56153903847596	13981.3734125619\\
1.56163904097602	13994.2076671728\\
1.56173904347609	14007.0992175632\\
1.56183904597615	14019.9907679537\\
1.56193904847621	14032.8250225646\\
1.56203905097627	14045.7165729551\\
1.56213905347634	14058.6081233455\\
1.5622390559764	14071.4996737359\\
1.56233905847646	14084.3912241264\\
1.56243906097652	14097.3400702963\\
1.56253906347659	14110.2316206868\\
1.56263906597665	14123.1231710772\\
1.56273906847671	14136.0720172472\\
1.56283907097677	14149.0208634171\\
1.56293907347684	14161.9697095871\\
1.5630390759769	14174.9185557571\\
1.56313907847696	14187.867401927\\
1.56323908097702	14200.816248097\\
1.56333908347709	14213.7650942669\\
1.56343908597715	14226.7139404369\\
1.56353908847721	14239.7200823863\\
1.56363909097727	14252.6689285563\\
1.56373909347734	14265.6750705058\\
1.5638390959774	14278.6812124552\\
1.56393909847746	14291.6873544047\\
1.56403910097752	14304.6934963542\\
1.56413910347759	14317.6996383037\\
1.56423910597765	14330.7057802531\\
1.56433910847771	14343.7119222026\\
1.56443911097777	14356.7753599316\\
1.56453911347784	14369.781501881\\
1.5646391159779	14382.84493961\\
1.56473911847796	14395.908377339\\
1.56483912097802	14408.9145192885\\
1.56493912347809	14421.9779570175\\
1.56503912597815	14435.0413947464\\
1.56513912847821	14448.1621282549\\
1.56523913097827	14461.2255659839\\
1.56533913347834	14474.2890037129\\
1.5654391359784	14487.4097372214\\
1.56553913847846	14500.4731749504\\
1.56563914097852	14513.5939084589\\
1.56573914347859	14526.7146419674\\
1.56583914597865	14539.8353754759\\
1.56593914847871	14552.9561089844\\
1.56603915097877	14566.0768424929\\
1.56613915347884	14579.1975760014\\
1.5662391559789	14592.3756052894\\
1.56633915847896	14605.4963387979\\
1.56643916097902	14618.6743680859\\
1.56653916347909	14631.7951015944\\
1.56663916597915	14644.9731308824\\
1.56673916847921	14658.1511601704\\
1.56683917097927	14671.3291894584\\
1.56693917347934	14684.5072187464\\
1.5670391759794	14697.6852480344\\
1.56713917847946	14710.9205731019\\
1.56723918097952	14724.0986023899\\
1.56733918347959	14737.3339274575\\
1.56743918597965	14750.5119567455\\
1.56753918847971	14763.747281813\\
1.56763919097977	14776.9826068805\\
1.56773919347984	14790.217931948\\
1.5678391959799	14803.4532570156\\
1.56793919847996	14816.6885820831\\
1.56803920098002	14829.9812029301\\
1.56813920348009	14843.2165279976\\
1.56823920598015	14856.5091488447\\
1.56833920848021	14869.7444739122\\
1.56843921098027	14883.0370947592\\
1.56853921348034	14896.3297156063\\
1.5686392159804	14909.6223364533\\
1.56873921848046	14922.9149573003\\
1.56883922098052	14936.2075781474\\
1.56893922348059	14949.5001989944\\
1.56903922598065	14962.850115621\\
1.56913922848071	14976.142736468\\
1.56923923098077	14989.4926530945\\
1.56933923348084	15002.7852739416\\
1.5694392359809	15016.1351905681\\
1.56953923848096	15029.4851071947\\
1.56963924098102	15042.8350238212\\
1.56973924348109	15056.1849404478\\
1.56983924598115	15069.5348570743\\
1.56993924848121	15082.9420694804\\
1.57003925098127	15096.2919861069\\
1.57013925348134	15109.699198513\\
1.5702392559814	15123.1064109191\\
1.57033925848146	15136.4563275456\\
1.57043926098152	15149.8635399517\\
1.57053926348159	15163.2707523577\\
1.57063926598165	15176.6779647638\\
1.57073926848171	15190.1424729494\\
1.57083927098177	15203.5496853554\\
1.57093927348184	15216.9568977615\\
1.5710392759819	15230.4214059471\\
1.57113927848196	15243.8286183531\\
1.57123928098202	15257.2931265387\\
1.57133928348209	15270.7576347243\\
1.57143928598215	15284.2221429098\\
1.57153928848221	15297.6866510954\\
1.57163929098227	15311.151159281\\
1.57173929348234	15324.6729632461\\
1.5718392959824	15338.1374714317\\
1.57193929848246	15351.6592753967\\
1.57203930098252	15365.1237835823\\
1.57213930348259	15378.6455875474\\
1.57223930598265	15392.1673915125\\
1.57233930848271	15405.6891954776\\
1.57243931098277	15419.2109994427\\
1.57253931348284	15432.7328034077\\
1.5726393159829	15446.2546073728\\
1.57273931848296	15459.8337071174\\
1.57283932098302	15473.3555110825\\
1.57293932348309	15486.9346108271\\
1.57303932598315	15500.4564147922\\
1.57313932848321	15514.0355145368\\
1.57323933098327	15527.6146142814\\
1.57333933348334	15541.193714026\\
1.5734393359834	15554.7728137706\\
1.57353933848346	15568.4092092947\\
1.57363934098352	15581.9883090393\\
1.57373934348359	15595.5674087839\\
1.57383934598365	15609.203804308\\
1.57393934848371	15622.8401998322\\
1.57403935098377	15636.4192995768\\
1.57413935348384	15650.0556951009\\
1.5742393559839	15663.692090625\\
1.57433935848396	15677.3284861491\\
1.57443936098402	15691.0221774527\\
1.57453936348409	15704.6585729768\\
1.57463936598415	15718.294968501\\
1.57473936848421	15731.9886598046\\
1.57483937098427	15745.6823511082\\
1.57493937348434	15759.3187466323\\
1.5750393759844	15773.0124379359\\
1.57513937848446	15786.7061292396\\
1.57523938098452	15800.3998205432\\
1.57533938348459	15814.0935118468\\
1.57543938598465	15827.84449893\\
1.57553938848471	15841.5381902336\\
1.57563939098477	15855.2891773167\\
1.57573939348484	15868.9828686204\\
1.5758393959849	15882.7338557035\\
1.57593939848496	15896.4848427866\\
1.57603940098502	15910.2358298698\\
1.57613940348509	15923.9868169529\\
1.57623940598515	15937.7378040361\\
1.57633940848521	15951.4887911192\\
1.57643941098527	15965.2970739819\\
1.57653941348534	15979.048061065\\
1.5766394159854	15992.8563439276\\
1.57673941848546	16006.6646267903\\
1.57683942098552	16020.4156138734\\
1.57693942348559	16034.2238967361\\
1.57703942598565	16048.0321795987\\
1.57713942848571	16061.8977582409\\
1.57723943098577	16075.7060411036\\
1.57733943348584	16089.5143239662\\
1.5774394359859	16103.3799026084\\
1.57753943848596	16117.188185471\\
1.57763944098602	16131.0537641132\\
1.57773944348609	16144.9193427554\\
1.57783944598615	16158.7849213975\\
1.57793944848621	16172.6505000397\\
1.57803945098627	16186.5160786819\\
1.57813945348634	16200.381657324\\
1.5782394559864	16214.2472359662\\
1.57833945848646	16228.1701103879\\
1.57843946098652	16242.03568903\\
1.57853946348659	16255.9585634517\\
1.57863946598665	16269.8814378734\\
1.57873946848671	16283.8043122951\\
1.57883947098677	16297.7271867168\\
1.57893947348684	16311.6500611384\\
1.5790394759869	16325.5729355601\\
1.57913947848696	16339.4958099818\\
1.57923948098702	16353.475980183\\
1.57933948348709	16367.3988546047\\
1.57943948598715	16381.3790248059\\
1.57953948848721	16395.359195007\\
1.57963949098727	16409.3393652082\\
1.57973949348734	16423.3195354094\\
1.5798394959874	16437.2997056106\\
1.57993949848746	16451.2798758118\\
1.58003950098752	16465.260046013\\
1.58013950348759	16479.2975119937\\
1.58023950598765	16493.2776821949\\
1.58033950848771	16507.3151481756\\
1.58043951098777	16521.3526141563\\
1.58053951348784	16535.3327843575\\
1.5806395159879	16549.3702503382\\
1.58073951848796	16563.4077163189\\
1.58083952098802	16577.5024780791\\
1.58093952348809	16591.5399440598\\
1.58103952598815	16605.5774100405\\
1.58113952848821	16619.6721718008\\
1.58123953098827	16633.7096377815\\
1.58133953348834	16647.8043995417\\
1.5814395359884	16661.8991613019\\
1.58153953848846	16675.9939230621\\
1.58163954098852	16690.0886848223\\
1.58173954348859	16704.1834465826\\
1.58183954598865	16718.2782083428\\
1.58193954848871	16732.4302658825\\
1.58203955098877	16746.5250276427\\
1.58213955348884	16760.6770851825\\
1.5822395559889	16774.7718469427\\
1.58233955848896	16788.9239044824\\
1.58243956098902	16803.0759620221\\
1.58253956348909	16817.2280195619\\
1.58263956598915	16831.3800771016\\
1.58273956848921	16845.5894304208\\
1.58283957098927	16859.7414879606\\
1.58293957348934	16873.8935455003\\
1.5830395759894	16888.1028988196\\
1.58313957848946	16902.3122521388\\
1.58323958098952	16916.4643096785\\
1.58333958348959	16930.6736629978\\
1.58343958598965	16944.883016317\\
1.58353958848971	16959.0923696363\\
1.58363959098977	16973.359018735\\
1.58373959348984	16987.5683720543\\
1.5838395959899	17001.7777253735\\
1.58393959848996	17016.0443744723\\
1.58403960099002	17030.311023571\\
1.58413960349009	17044.5203768903\\
1.58423960599015	17058.787025989\\
1.58433960849021	17073.0536750878\\
1.58443961099027	17087.3203241865\\
1.58453961349034	17101.6442690648\\
1.5846396159904	17115.9109181636\\
1.58473961849046	17130.1775672623\\
1.58483962099052	17144.5015121406\\
1.58493962349059	17158.7681612394\\
1.58503962599065	17173.0921061176\\
1.58513962849071	17187.4160509959\\
1.58523963099077	17201.7399958742\\
1.58533963349084	17216.0639407524\\
1.5854396359909	17230.3878856307\\
1.58553963849096	17244.7691262885\\
1.58563964099102	17259.0930711668\\
1.58573964349109	17273.4743118245\\
1.58583964599115	17287.7982567028\\
1.58593964849121	17302.1794973606\\
1.58603965099127	17316.5607380184\\
1.58613965349134	17330.9419786762\\
1.5862396559914	17345.3232193339\\
1.58633965849146	17359.7044599917\\
1.58643966099152	17374.0857006495\\
1.58653966349159	17388.5242370868\\
1.58663966599165	17402.9054777446\\
1.58673966849171	17417.3440141819\\
1.58683967099177	17431.7252548397\\
1.58693967349184	17446.163791277\\
1.5870396759919	17460.6023277143\\
1.58713967849196	17475.0408641516\\
1.58723968099202	17489.4794005889\\
1.58733968349209	17503.9752328057\\
1.58743968599215	17518.413769243\\
1.58753968849221	17532.9096014598\\
1.58763969099227	17547.3481378971\\
1.58773969349234	17561.8439701139\\
1.5878396959924	17576.3398023307\\
1.58793969849246	17590.8356345475\\
1.58803970099252	17605.3314667643\\
1.58813970349259	17619.8272989811\\
1.58823970599265	17634.3231311979\\
1.58833970849271	17648.8189634147\\
1.58843971099277	17663.3720914111\\
1.58853971349284	17677.8679236279\\
1.5886397159929	17692.4210516242\\
1.58873971849296	17706.9741796205\\
1.58883972099302	17721.5273076168\\
1.58893972349309	17736.0804356132\\
1.58903972599315	17750.6335636095\\
1.58913972849321	17765.1866916058\\
1.58923973099327	17779.7398196021\\
1.58933973349334	17794.350243378\\
1.5894397359934	17808.9033713743\\
1.58953973849346	17823.5137951501\\
1.58963974099352	17838.124218926\\
1.58973974349359	17852.7346427018\\
1.58983974599365	17867.3450664776\\
1.58993974849371	17881.9554902535\\
1.59003975099377	17896.5659140293\\
1.59013975349384	17911.2336335847\\
1.5902397559939	17925.8440573605\\
1.59033975849396	17940.4544811363\\
1.59043976099402	17955.1222006917\\
1.59053976349409	17969.789920247\\
1.59063976599415	17984.4576398024\\
1.59073976849421	17999.1253593577\\
1.59083977099427	18013.7930789131\\
1.59093977349434	18028.4607984684\\
1.5910397759944	18043.1285180238\\
1.59113977849446	18057.8535333586\\
1.59123978099452	18072.521252914\\
1.59133978349459	18087.2462682489\\
1.59143978599465	18101.9712835837\\
1.59153978849471	18116.6962989186\\
1.59163979099477	18131.4213142534\\
1.59173979349484	18146.1463295883\\
1.5918397959949	18160.8713449232\\
1.59193979849496	18175.596360258\\
1.59203980099502	18190.3786713724\\
1.59213980349509	18205.1036867073\\
1.59223980599515	18219.8859978216\\
1.59233980849521	18234.6110131565\\
1.59243981099527	18249.3933242709\\
1.59253981349534	18264.1756353853\\
1.5926398159954	18278.9579464996\\
1.59273981849546	18293.7975533935\\
1.59283982099552	18308.5798645079\\
1.59293982349559	18323.3621756223\\
1.59303982599565	18338.2017825162\\
1.59313982849571	18352.9840936305\\
1.59323983099577	18367.8237005244\\
1.59333983349584	18382.6633074183\\
1.5934398359959	18397.5029143122\\
1.59353983849596	18412.3425212061\\
1.59363984099602	18427.1821281\\
1.59373984349609	18442.0217349939\\
1.59383984599615	18456.9186376673\\
1.59393984849621	18471.7582445611\\
1.59403985099627	18486.6551472346\\
1.59413985349634	18501.552049908\\
1.5942398559964	18516.3916568018\\
1.59433985849646	18531.2885594752\\
1.59443986099652	18546.1854621486\\
1.59453986349659	18561.1396606016\\
1.59463986599665	18576.036563275\\
1.59473986849671	18590.9334659484\\
1.59483987099677	18605.8876644013\\
1.59493987349684	18620.7845670747\\
1.5950398759969	18635.7387655276\\
1.59513987849696	18650.6929639805\\
1.59523988099702	18665.6471624334\\
1.59533988349709	18680.6013608863\\
1.59543988599715	18695.5555593392\\
1.59553988849721	18710.5097577922\\
1.59563989099727	18725.5212520246\\
1.59573989349734	18740.4754504775\\
1.5958398959974	18755.4869447099\\
1.59593989849746	18770.4411431628\\
1.59603990099753	18785.4526373953\\
1.59613990349759	18800.4641316277\\
1.59623990599765	18815.4756258601\\
1.59633990849771	18830.4871200926\\
1.59643991099777	18845.5559101045\\
1.59653991349784	18860.5674043369\\
1.5966399159979	18875.5788985694\\
1.59673991849796	18890.6476885813\\
1.59683992099802	18905.7164785932\\
1.59693992349809	18920.7279728257\\
1.59703992599815	18935.7967628376\\
1.59713992849821	18950.8655528495\\
1.59723993099827	18965.9343428615\\
1.59733993349834	18981.0604286529\\
1.5974399359984	18996.1292186649\\
1.59753993849846	19011.2553044563\\
1.59763994099852	19026.3240944683\\
1.59773994349859	19041.4501802597\\
1.59783994599865	19056.5762660512\\
1.59793994849871	19071.6450560631\\
1.59803995099878	19086.7711418546\\
1.59813995349884	19101.9545234255\\
1.5982399559989	19117.080609217\\
1.59833995849896	19132.2066950085\\
1.59843996099902	19147.3900765794\\
1.59853996349909	19162.5161623709\\
1.59863996599915	19177.6995439418\\
1.59873996849921	19192.8829255128\\
1.59883997099927	19208.0663070838\\
1.59893997349934	19223.2496886547\\
1.5990399759994	19238.4330702257\\
1.59913997849946	19253.6164517967\\
1.59923998099953	19268.7998333676\\
1.59933998349959	19284.0405107181\\
1.59943998599965	19299.2238922891\\
1.59953998849971	19314.4645696396\\
1.59963999099977	19329.70524699\\
1.59973999349984	19344.9459243405\\
1.5998399959999	19360.186601691\\
1.59993999849996	19375.4272790415\\
1.60004000100003	19390.667956392\\
};
\addplot [color=mycolor1,solid,forget plot]
  table[row sep=crcr]{%
1.60004000100003	19390.667956392\\
1.60014000350009	19405.9086337424\\
1.60024000600015	19421.2066068724\\
1.60034000850021	19436.4472842229\\
1.60044001100027	19451.7452573529\\
1.60054001350034	19467.0432304829\\
1.6006400160004	19482.2839078334\\
1.60074001850046	19497.5818809634\\
1.60084002100052	19512.9371498729\\
1.60094002350059	19528.2351230029\\
1.60104002600065	19543.5330961329\\
1.60114002850071	19558.8310692629\\
1.60124003100078	19574.1863381724\\
1.60134003350084	19589.5416070819\\
1.6014400360009	19604.8395802119\\
1.60154003850096	19620.1948491214\\
1.60164004100102	19635.5501180309\\
1.60174004350109	19650.9053869404\\
1.60184004600115	19666.3179516294\\
1.60194004850121	19681.6732205389\\
1.60204005100128	19697.0284894484\\
1.60214005350134	19712.4410541374\\
1.6022400560014	19727.8536188265\\
1.60234005850146	19743.208887736\\
1.60244006100153	19758.621452425\\
1.60254006350159	19774.034017114\\
1.60264006600165	19789.446581803\\
1.60274006850171	19804.859146492\\
1.60284007100177	19820.3290069606\\
1.60294007350184	19835.7415716496\\
1.6030400760019	19851.2114321181\\
1.60314007850196	19866.6239968071\\
1.60324008100203	19882.0938572757\\
1.60334008350209	19897.5637177442\\
1.60344008600215	19913.0335782127\\
1.60354008850221	19928.5034386813\\
1.60364009100228	19943.9732991498\\
1.60374009350234	19959.4431596183\\
1.6038400960024	19974.9703158664\\
1.60394009850246	19990.4401763349\\
1.60404010100253	20005.967332583\\
1.60414010350259	20021.494488831\\
1.60424010600265	20037.021645079\\
1.60434010850271	20052.5488013271\\
1.60444011100278	20068.0759575751\\
1.60454011350284	20083.6031138232\\
1.6046401160029	20099.1302700712\\
1.60474011850296	20114.7147220988\\
1.60484012100302	20130.2418783468\\
1.60494012350309	20145.8263303744\\
1.60504012600315	20161.3534866224\\
1.60514012850321	20176.93793865\\
1.60524013100328	20192.5223906775\\
1.60534013350334	20208.1068427051\\
1.6054401360034	20223.7485905122\\
1.60554013850346	20239.3330425397\\
1.60564014100353	20254.9174945673\\
1.60574014350359	20270.5592423744\\
1.60584014600365	20286.1436944019\\
1.60594014850371	20301.785442209\\
1.60604015100378	20317.4271900161\\
1.60614015350384	20333.0689378231\\
1.6062401560039	20348.7106856302\\
1.60634015850396	20364.3524334373\\
1.60644016100403	20380.0514770239\\
1.60654016350409	20395.6932248309\\
1.60664016600415	20411.334972638\\
1.60674016850421	20427.0340162246\\
1.60684017100428	20442.7330598112\\
1.60694017350434	20458.4321033978\\
1.6070401760044	20474.1311469844\\
1.60714017850446	20489.8301905709\\
1.60724018100453	20505.5292341575\\
1.60734018350459	20521.2282777441\\
1.60744018600465	20536.9846171102\\
1.60754018850471	20552.6836606968\\
1.60764019100478	20568.4400000629\\
1.60774019350484	20584.1390436495\\
1.6078401960049	20599.8953830156\\
1.60794019850496	20615.6517223817\\
1.60804020100503	20631.4080617478\\
1.60814020350509	20647.1644011139\\
1.60824020600515	20662.9780362595\\
1.60834020850521	20678.7343756256\\
1.60844021100528	20694.5480107712\\
1.60854021350534	20710.3043501373\\
1.6086402160054	20726.1179852829\\
1.60874021850546	20741.9316204285\\
1.60884022100553	20757.7452555741\\
1.60894022350559	20773.5588907197\\
1.60904022600565	20789.3725258653\\
1.60914022850571	20805.1861610109\\
1.60924023100578	20821.0570919361\\
1.60934023350584	20836.8707270817\\
1.6094402360059	20852.7416580068\\
1.60954023850596	20868.6125889319\\
1.60964024100603	20884.483519857\\
1.60974024350609	20900.2971550027\\
1.60984024600615	20916.2253817073\\
1.60994024850621	20932.0963126324\\
1.61004025100628	20947.9672435575\\
1.61014025350634	20963.8381744827\\
1.6102402560064	20979.7664011873\\
1.61034025850646	20995.6946278919\\
1.61044026100653	21011.5655588171\\
1.61054026350659	21027.4937855217\\
1.61064026600665	21043.4220122263\\
1.61074026850671	21059.350238931\\
1.61084027100678	21075.2784656356\\
1.61094027350684	21091.2639881198\\
1.6110402760069	21107.1922148244\\
1.61114027850696	21123.1777373085\\
1.61124028100703	21139.1059640132\\
1.61134028350709	21155.0914864973\\
1.61144028600715	21171.0770089815\\
1.61154028850721	21187.0625314656\\
1.61164029100728	21203.0480539498\\
1.61174029350734	21219.0335764339\\
1.6118402960074	21235.0190989181\\
1.61194029850746	21251.0619171817\\
1.61204030100753	21267.0474396659\\
1.61214030350759	21283.0902579296\\
1.61224030600765	21299.1330761932\\
1.61234030850771	21315.1185986774\\
1.61244031100778	21331.161416941\\
1.61254031350784	21347.2042352047\\
1.6126403160079	21363.3043492479\\
1.61274031850796	21379.3471675115\\
1.61284032100803	21395.3899857752\\
1.61294032350809	21411.4900998184\\
1.61304032600815	21427.532918082\\
1.61314032850821	21443.6330321252\\
1.61324033100828	21459.7331461684\\
1.61334033350834	21475.8332602116\\
1.6134403360084	21491.9333742547\\
1.61354033850846	21508.0334882979\\
1.61364034100853	21524.1908981206\\
1.61374034350859	21540.2910121638\\
1.61384034600865	21556.391126207\\
1.61394034850871	21572.5485360297\\
1.61404035100878	21588.7059458523\\
1.61414035350884	21604.863355675\\
1.6142403560089	21621.0207654977\\
1.61434035850896	21637.1781753204\\
1.61444036100903	21653.3355851431\\
1.61454036350909	21669.4929949658\\
1.61464036600915	21685.707700568\\
1.61474036850921	21701.8651103907\\
1.61484037100928	21718.0798159929\\
1.61494037350934	21734.2945215951\\
1.6150403760094	21750.4519314178\\
1.61514037850946	21766.66663702\\
1.61524038100953	21782.8813426222\\
1.61534038350959	21799.1533440039\\
1.61544038600965	21815.3680496061\\
1.61554038850971	21831.5827552083\\
1.61564039100978	21847.85475659\\
1.61574039350984	21864.1267579717\\
1.6158403960099	21880.3414635739\\
1.61594039850996	21896.6134649556\\
1.61604040101003	21912.8854663374\\
1.61614040351009	21929.1574677191\\
1.61624040601015	21945.4294691008\\
1.61634040851021	21961.758766262\\
1.61644041101028	21978.0307676437\\
1.61654041351034	21994.360064805\\
1.6166404160104	22010.6320661867\\
1.61674041851046	22026.9613633479\\
1.61684042101053	22043.2906605091\\
1.61694042351059	22059.6199576704\\
1.61704042601065	22075.9492548316\\
1.61714042851071	22092.2785519928\\
1.61724043101078	22108.6651449336\\
1.61734043351084	22124.9944420948\\
1.6174404360109	22141.323739256\\
1.61754043851096	22157.7103321968\\
1.61764044101103	22174.0969251375\\
1.61774044351109	22190.4835180782\\
1.61784044601115	22206.870111019\\
1.61794044851121	22223.2567039597\\
1.61804045101128	22239.6432969005\\
1.61814045351134	22256.0298898412\\
1.6182404560114	22272.4737785615\\
1.61834045851146	22288.8603715022\\
1.61844046101153	22305.3042602225\\
1.61854046351159	22321.7481489427\\
1.61864046601165	22338.192037663\\
1.61874046851171	22354.6359263832\\
1.61884047101178	22371.0798151035\\
1.61894047351184	22387.5237038237\\
1.6190404760119	22403.967592544\\
1.61914047851196	22420.4687770438\\
1.61924048101203	22436.912665764\\
1.61934048351209	22453.4138502638\\
1.61944048601215	22469.9150347635\\
1.61954048851221	22486.4162192633\\
1.61964049101228	22502.9174037631\\
1.61974049351234	22519.4185882628\\
1.6198404960124	22535.9197727626\\
1.61994049851246	22552.4209572624\\
1.62004050101253	22568.9794375417\\
1.62014050351259	22585.4806220414\\
1.62024050601265	22602.0391023207\\
1.62034050851271	22618.5975826\\
1.62044051101278	22635.1560628793\\
1.62054051351284	22651.7145431586\\
1.6206405160129	22668.2730234378\\
1.62074051851296	22684.8315037171\\
1.62084052101303	22701.3899839964\\
1.62094052351309	22718.0057600552\\
1.62104052601315	22734.5642403345\\
1.62114052851321	22751.1800163933\\
1.62124053101328	22767.7957924521\\
1.62134053351334	22784.4115685109\\
1.6214405360134	22801.0273445696\\
1.62154053851346	22817.6431206284\\
1.62164054101353	22834.2588966872\\
1.62174054351359	22850.874672746\\
1.62184054601365	22867.5477445843\\
1.62194054851371	22884.1635206431\\
1.62204055101378	22900.8365924814\\
1.62214055351384	22917.5096643197\\
1.6222405560139	22934.1827361581\\
1.62234055851396	22950.8558079964\\
1.62244056101403	22967.5288798347\\
1.62254056351409	22984.201951673\\
1.62264056601415	23000.9323192908\\
1.62274056851421	23017.6053911291\\
1.62284057101428	23034.3357587469\\
1.62294057351434	23051.0088305852\\
1.6230405760144	23067.739198203\\
1.62314057851446	23084.4695658209\\
1.62324058101453	23101.1999334387\\
1.62334058351459	23117.9303010565\\
1.62344058601465	23134.6606686743\\
1.62354058851471	23151.4483320717\\
1.62364059101478	23168.1786996895\\
1.62374059351484	23184.9663630868\\
1.6238405960149	23201.7540264841\\
1.62394059851496	23218.484394102\\
1.62404060101503	23235.2720574993\\
1.62414060351509	23252.0597208966\\
1.62424060601515	23268.9046800735\\
1.62434060851521	23285.6923434708\\
1.62444061101528	23302.4800068681\\
1.62454061351534	23319.324966045\\
1.6246406160154	23336.1126294423\\
1.62474061851546	23352.9575886192\\
1.62484062101553	23369.802547796\\
1.62494062351559	23386.6475069729\\
1.62504062601565	23403.4924661497\\
1.62514062851571	23420.3374253266\\
1.62524063101578	23437.1823845034\\
1.62534063351584	23454.0846394598\\
1.6254406360159	23470.9295986366\\
1.62554063851596	23487.831853593\\
1.62564064101603	23504.7341085493\\
1.62574064351609	23521.5790677262\\
1.62584064601615	23538.4813226825\\
1.62594064851621	23555.3835776389\\
1.62604065101628	23572.3431283748\\
1.62614065351634	23589.2453833311\\
1.6262406560164	23606.1476382875\\
1.62634065851646	23623.1071890234\\
1.62644066101653	23640.0094439797\\
1.62654066351659	23656.9689947156\\
1.62664066601665	23673.9285454515\\
1.62674066851671	23690.8880961873\\
1.62684067101678	23707.8476469232\\
1.62694067351684	23724.8071976591\\
1.6270406760169	23741.8240441745\\
1.62714067851696	23758.7835949103\\
1.62724068101703	23775.7431456462\\
1.62734068351709	23792.7599921616\\
1.62744068601715	23809.776838677\\
1.62754068851721	23826.7936851924\\
1.62764069101728	23843.8105317077\\
1.62774069351734	23860.8273782231\\
1.6278406960174	23877.8442247385\\
1.62794069851746	23894.8610712539\\
1.62804070101753	23911.9352135488\\
1.62814070351759	23928.9520600642\\
1.62824070601765	23946.0262023591\\
1.62834070851771	23963.100344654\\
1.62844071101778	23980.1744869489\\
1.62854071351784	23997.2486292438\\
1.6286407160179	24014.3227715387\\
1.62874071851796	24031.3969138336\\
1.62884072101803	24048.4710561285\\
1.62894072351809	24065.6024942029\\
1.62904072601815	24082.6766364978\\
1.62914072851821	24099.8080745722\\
1.62924073101828	24116.9395126466\\
1.62934073351834	24134.070950721\\
1.6294407360184	24151.2023887954\\
1.62954073851846	24168.3338268698\\
1.62964074101853	24185.4652649443\\
1.62974074351859	24202.5967030187\\
1.62984074601865	24219.7854368726\\
1.62994074851871	24236.916874947\\
1.63004075101878	24254.1056088009\\
1.63014075351884	24271.2943426549\\
1.6302407560189	24288.4830765088\\
1.63034075851896	24305.6718103627\\
1.63044076101903	24322.8605442166\\
1.63054076351909	24340.0492780706\\
1.63064076601915	24357.2380119245\\
1.63074076851921	24374.4840415579\\
1.63084077101928	24391.6727754118\\
1.63094077351934	24408.9188050453\\
1.6310407760194	24426.1648346787\\
1.63114077851946	24443.4108643122\\
1.63124078101953	24460.6568939456\\
1.63134078351959	24477.902923579\\
1.63144078601965	24495.1489532125\\
1.63154078851971	24512.4522786254\\
1.63164079101978	24529.6983082589\\
1.63174079351984	24547.0016336718\\
1.6318407960199	24564.2476633052\\
1.63194079851996	24581.5509887182\\
1.63204080102003	24598.8543141311\\
1.63214080352009	24616.1576395441\\
1.63224080602015	24633.460964957\\
1.63234080852021	24650.8215861495\\
1.63244081102028	24668.1249115625\\
1.63254081352034	24685.4855327549\\
1.6326408160204	24702.7888581679\\
1.63274081852046	24720.1494793603\\
1.63284082102053	24737.5101005528\\
1.63294082352059	24754.8707217453\\
1.63304082602065	24772.2313429377\\
1.63314082852071	24789.5919641302\\
1.63324083102078	24806.9525853227\\
1.63334083352084	24824.3705022946\\
1.6334408360209	24841.7311234871\\
1.63354083852096	24859.1490404591\\
1.63364084102103	24876.5669574311\\
1.63374084352109	24893.9275786235\\
1.63384084602115	24911.3454955955\\
1.63394084852121	24928.820708347\\
1.63404085102128	24946.238625319\\
1.63414085352134	24963.6565422909\\
1.6342408560214	24981.0744592629\\
1.63434085852146	24998.5496720144\\
1.63444086102153	25016.0248847659\\
1.63454086352159	25033.4428017379\\
1.63464086602165	25050.9180144894\\
1.63474086852171	25068.3932272409\\
1.63484087102178	25085.8684399923\\
1.63494087352184	25103.4009485234\\
1.6350408760219	25120.8761612748\\
1.63514087852196	25138.3513740263\\
1.63524088102203	25155.8838825573\\
1.63534088352209	25173.4163910883\\
1.63544088602215	25190.8916038398\\
1.63554088852221	25208.4241123708\\
1.63564089102228	25225.9566209018\\
1.63574089352234	25243.4891294328\\
1.6358408960224	25261.0789337434\\
1.63594089852246	25278.6114422744\\
1.63604090102253	25296.1439508054\\
1.63614090352259	25313.7337551159\\
1.63624090602265	25331.3235594264\\
1.63634090852271	25348.8560679574\\
1.63644091102278	25366.4458722679\\
1.63654091352284	25384.0356765784\\
1.6366409160229	25401.6827766685\\
1.63674091852296	25419.272580979\\
1.63684092102303	25436.8623852895\\
1.63694092352309	25454.5094853795\\
1.63704092602315	25472.09928969\\
1.63714092852321	25489.7463897801\\
1.63724093102328	25507.3934898701\\
1.63734093352334	25525.0405899601\\
1.6374409360234	25542.6876900502\\
1.63754093852346	25560.3347901402\\
1.63764094102353	25577.9818902302\\
1.63774094352359	25595.6862860998\\
1.63784094602365	25613.3333861898\\
1.63794094852371	25631.0377820593\\
1.63804095102378	25648.6848821494\\
1.63814095352384	25666.3892780189\\
1.6382409560239	25684.0936738884\\
1.63834095852396	25701.798069758\\
1.63844096102403	25719.5024656275\\
1.63854096352409	25737.2641572766\\
1.63864096602415	25754.9685531461\\
1.63874096852421	25772.7302447952\\
1.63884097102428	25790.4346406647\\
1.63894097352434	25808.1963323138\\
1.6390409760244	25825.9580239628\\
1.63914097852446	25843.7197156119\\
1.63924098102453	25861.4814072609\\
1.63934098352459	25879.24309891\\
1.63944098602465	25897.0620863386\\
1.63954098852471	25914.8237779876\\
1.63964099102478	25932.5854696367\\
1.63974099352484	25950.4044570652\\
1.6398409960249	25968.2234444938\\
1.63994099852496	25986.0424319224\\
1.64004100102503	26003.861419351\\
1.64014100352509	26021.6804067795\\
1.64024100602515	26039.4993942081\\
1.64034100852521	26057.3183816367\\
1.64044101102528	26075.1946648447\\
1.64054101352534	26093.0136522733\\
1.6406410160254	26110.8899354814\\
1.64074101852546	26128.7662186895\\
1.64084102102553	26146.6425018976\\
1.64094102352559	26164.5187851056\\
1.64104102602565	26182.3950683137\\
1.64114102852571	26200.2713515218\\
1.64124103102578	26218.2049305094\\
1.64134103352584	26236.0812137175\\
1.6414410360259	26254.0147927051\\
1.64154103852596	26271.8910759132\\
1.64164104102603	26289.8246549007\\
1.64174104352609	26307.7582338883\\
1.64184104602615	26325.6918128759\\
1.64194104852621	26343.6253918635\\
1.64204105102628	26361.6162666306\\
1.64214105352634	26379.5498456182\\
1.6422410560264	26397.4834246058\\
1.64234105852646	26415.4742993729\\
1.64244106102653	26433.46517414\\
1.64254106352659	26451.4560489071\\
1.64264106602665	26469.4469236743\\
1.64274106852671	26487.4377984414\\
1.64284107102678	26505.4286732085\\
1.64294107352684	26523.4195479756\\
1.6430410760269	26541.4677185222\\
1.64314107852696	26559.4585932893\\
1.64324108102703	26577.5067638359\\
1.64334108352709	26595.497638603\\
1.64344108602715	26613.5458091497\\
1.64354108852721	26631.5939796963\\
1.64364109102728	26649.6421502429\\
1.64374109352734	26667.747616569\\
1.6438410960274	26685.7957871157\\
1.64394109852746	26703.8439576623\\
1.64404110102753	26721.9494239884\\
1.64414110352759	26740.0548903145\\
1.64424110602765	26758.1030608612\\
1.64434110852771	26776.2085271873\\
1.64444111102778	26794.3139935134\\
1.64454111352784	26812.4194598396\\
1.6446411160279	26830.5822219452\\
1.64474111852796	26848.6876882714\\
1.64484112102803	26866.7931545975\\
1.64494112352809	26884.9559167031\\
1.64504112602815	26903.1186788088\\
1.64514112852821	26921.2241451349\\
1.64524113102828	26939.3869072406\\
1.64534113352834	26957.5496693462\\
1.6454411360284	26975.7697272314\\
1.64554113852846	26993.932489337\\
1.64564114102853	27012.0952514427\\
1.64574114352859	27030.3153093278\\
1.64584114602865	27048.4780714335\\
1.64594114852871	27066.6981293186\\
1.64604115102878	27084.9181872038\\
1.64614115352884	27103.138245089\\
1.6462411560289	27121.3583029741\\
1.64634115852896	27139.5783608593\\
1.64644116102903	27157.7984187444\\
1.64654116352909	27176.0757724091\\
1.64664116602915	27194.2958302943\\
1.64674116852921	27212.5731839589\\
1.64684117102928	27230.7932418441\\
1.64694117352934	27249.0705955088\\
1.6470411760294	27267.3479491734\\
1.64714117852946	27285.6253028381\\
1.64724118102953	27303.9599522823\\
1.64734118352959	27322.237305947\\
1.64744118602965	27340.5146596117\\
1.64754118852971	27358.8493090558\\
1.64764119102978	27377.1266627205\\
1.64774119352984	27395.4613121647\\
1.6478411960299	27413.7959616089\\
1.64794119852996	27432.1306110531\\
1.64804120103003	27450.4652604973\\
1.64814120353009	27468.7999099414\\
1.64824120603015	27487.1918551651\\
1.64834120853021	27505.5265046093\\
1.64844121103028	27523.918449833\\
1.64854121353034	27542.3103950567\\
1.6486412160304	27560.6450445009\\
1.64874121853046	27579.0369897246\\
1.64884122103053	27597.4289349483\\
1.64894122353059	27615.820880172\\
1.64904122603065	27634.2701211752\\
1.64914122853071	27652.6620663989\\
1.64924123103078	27671.0540116226\\
1.64934123353084	27689.5032526258\\
1.6494412360309	27707.952493629\\
1.64954123853096	27726.4017346323\\
1.64964124103103	27744.8509756355\\
1.64974124353109	27763.3002166387\\
1.64984124603115	27781.7494576419\\
1.64994124853121	27800.1986986451\\
1.65004125103128	27818.6479396483\\
1.65014125353134	27837.154476431\\
1.6502412560314	27855.6610132138\\
1.65034125853146	27874.110254217\\
1.65044126103153	27892.6167909997\\
1.65054126353159	27911.1233277824\\
1.65064126603165	27929.6298645652\\
1.65074126853171	27948.1364013479\\
1.65084127103178	27966.7002339101\\
1.65094127353184	27985.2067706929\\
1.6510412760319	28003.7706032551\\
1.65114127853196	28022.2771400378\\
1.65124128103203	28040.8409726001\\
1.65134128353209	28059.4048051623\\
1.65144128603215	28077.9686377245\\
1.65154128853221	28096.5324702868\\
1.65164129103228	28115.096302849\\
1.65174129353234	28133.7174311908\\
1.6518412960324	28152.281263753\\
1.65194129853246	28170.9023920948\\
1.65204130103253	28189.466224657\\
1.65214130353259	28208.0873529987\\
1.65224130603265	28226.7084813405\\
1.65234130853271	28245.3296096822\\
1.65244131103278	28263.950738024\\
1.65254131353284	28282.6291621453\\
1.6526413160329	28301.250290487\\
1.65274131853296	28319.8714188288\\
1.65284132103303	28338.54984295\\
1.65294132353309	28357.2282670713\\
1.65304132603315	28375.9066911926\\
1.65314132853321	28394.5278195343\\
1.65324133103328	28413.2635394351\\
1.65334133353334	28431.9419635564\\
1.6534413360334	28450.6203876776\\
1.65354133853346	28469.2988117989\\
1.65364134103353	28488.0345316997\\
1.65374134353359	28506.7702516004\\
1.65384134603365	28525.4486757217\\
1.65394134853371	28544.1843956225\\
1.65404135103378	28562.9201155233\\
1.65414135353384	28581.655835424\\
1.6542413560339	28600.3915553248\\
1.65434135853396	28619.1845710051\\
1.65444136103403	28637.9202909059\\
1.65454136353409	28656.7133065862\\
1.65464136603415	28675.449026487\\
1.65474136853421	28694.2420421672\\
1.65484137103428	28713.0350578475\\
1.65494137353434	28731.8280735278\\
1.6550413760344	28750.6210892081\\
1.65514137853446	28769.4141048884\\
1.65524138103453	28788.2644163482\\
1.65534138353459	28807.0574320285\\
1.65544138603465	28825.9077434883\\
1.65554138853471	28844.7580549481\\
1.65564139103478	28863.5510706284\\
1.65574139353484	28882.4013820882\\
1.6558413960349	28901.251693548\\
1.65594139853496	28920.1020050078\\
1.65604140103503	28939.0096122471\\
1.65614140353509	28957.8599237069\\
1.65624140603515	28976.7675309463\\
1.65634140853521	28995.6178424061\\
1.65644141103528	29014.5254496454\\
1.65654141353534	29033.4330568847\\
1.6566414160354	29052.340664124\\
1.65674141853546	29071.2482713633\\
1.65684142103553	29090.1558786026\\
1.65694142353559	29109.063485842\\
1.65704142603565	29128.0283888608\\
1.65714142853571	29146.9359961001\\
1.65724143103578	29165.9008991189\\
1.65734143353584	29184.8658021378\\
1.6574414360359	29203.8307051566\\
1.65754143853596	29222.7956081754\\
1.65764144103603	29241.7605111943\\
1.65774144353609	29260.7254142131\\
1.65784144603615	29279.6903172319\\
1.65794144853621	29298.7125160303\\
1.65804145103628	29317.6774190491\\
1.65814145353634	29336.6996178474\\
1.6582414560364	29355.7218166458\\
1.65834145853646	29374.7440154441\\
1.65844146103653	29393.7662142425\\
1.65854146353659	29412.7884130408\\
1.65864146603665	29431.8106118392\\
1.65874146853671	29450.890106417\\
1.65884147103678	29469.9123052154\\
1.65894147353684	29488.9917997932\\
1.6590414760369	29508.0139985915\\
1.65914147853696	29527.0934931694\\
1.65924148103703	29546.1729877473\\
1.65934148353709	29565.2524823251\\
1.65944148603715	29584.331976903\\
1.65954148853721	29603.4687672603\\
1.65964149103728	29622.5482618382\\
1.65974149353734	29641.6850521956\\
1.6598414960374	29660.7645467734\\
1.65994149853746	29679.9013371308\\
1.66004150103753	29699.0381274882\\
1.66014150353759	29718.1749178455\\
1.66024150603765	29737.3117082029\\
1.66034150853771	29756.4484985603\\
1.66044151103778	29775.6425846972\\
1.66054151353784	29794.7793750545\\
1.6606415160379	29813.9734611914\\
1.66074151853796	29833.1102515488\\
1.66084152103803	29852.3043376857\\
1.66094152353809	29871.4984238225\\
1.66104152603815	29890.6925099594\\
1.66114152853821	29909.8865960963\\
1.66124153103828	29929.0806822332\\
1.66134153353834	29948.3320641496\\
1.6614415360384	29967.5261502865\\
1.66154153853846	29986.7775322029\\
1.66164154103853	30006.0289141193\\
1.66174154353859	30025.2230002561\\
1.66184154603865	30044.4743821725\\
1.66194154853871	30063.7257640889\\
1.66204155103878	30083.0344417848\\
1.66214155353884	30102.2858237012\\
1.6622415560389	30121.5372056176\\
1.66234155853896	30140.8458833136\\
1.66244156103903	30160.1545610095\\
1.66254156353909	30179.4059429259\\
1.66264156603915	30198.7146206218\\
1.66274156853921	30218.0232983177\\
1.66284157103928	30237.3319760136\\
1.66294157353934	30256.6406537095\\
1.6630415760394	30276.0066271849\\
1.66314157853946	30295.3153048808\\
1.66324158103953	30314.6812783562\\
1.66334158353959	30334.0472518317\\
1.66344158603965	30353.3559295276\\
1.66354158853971	30372.721903003\\
1.66364159103978	30392.0878764784\\
1.66374159353984	30411.4538499538\\
1.6638415960399	30430.8771192088\\
1.66394159853996	30450.2430926842\\
1.66404160104003	30469.6090661596\\
1.66414160354009	30489.0323354145\\
1.66424160604015	30508.4556046695\\
1.66434160854021	30527.8788739244\\
1.66444161104028	30547.3021431794\\
1.66454161354034	30566.7254124343\\
1.6646416160404	30586.1486816892\\
1.66474161854046	30605.5719509442\\
1.66484162104053	30625.0525159786\\
1.66494162354059	30644.4757852335\\
1.66504162604065	30663.956350268\\
1.66514162854071	30683.4369153024\\
1.66524163104078	30702.8601845574\\
1.66534163354084	30722.3407495918\\
1.6654416360409	30741.8213146263\\
1.66554163854096	30761.3591754402\\
1.66564164104103	30780.8397404747\\
1.66574164354109	30800.3203055091\\
1.66584164604115	30819.8581663231\\
1.66594164854121	30839.396027137\\
1.66604165104128	30858.933887951\\
1.66614165354134	30878.4144529855\\
1.6662416560414	30897.9523137994\\
1.66634165854146	30917.5474703929\\
1.66644166104153	30937.0853312069\\
1.66654166354159	30956.6231920208\\
1.66664166604165	30976.2183486143\\
1.66674166854171	30995.7562094282\\
1.66684167104178	31015.3513660217\\
1.66694167354184	31034.9465226152\\
1.6670416760419	31054.5416792087\\
1.66714167854196	31074.1368358021\\
1.66724168104203	31093.7319923956\\
1.66734168354209	31113.3271489891\\
1.66744168604215	31132.9796013621\\
1.66754168854221	31152.5747579556\\
1.66764169104228	31172.2272103285\\
1.66774169354234	31191.8796627015\\
1.6678416960424	31211.5321150745\\
1.66794169854246	31231.1845674475\\
1.66804170104253	31250.8370198205\\
1.66814170354259	31270.4894721935\\
1.66824170604265	31290.1419245665\\
1.66834170854271	31309.851672719\\
1.66844171104278	31329.504125092\\
1.66854171354284	31349.2138732445\\
1.6686417160429	31368.923621397\\
1.66874171854296	31388.6333695495\\
1.66884172104303	31408.343117702\\
1.66894172354309	31428.0528658545\\
1.66904172604315	31447.762614007\\
1.66914172854321	31467.529657939\\
1.66924173104328	31487.2394060915\\
1.66934173354334	31507.0064500235\\
1.6694417360434	31526.7734939555\\
1.66954173854346	31546.5405378875\\
1.66964174104353	31566.25028604\\
1.66974174354359	31586.0746257515\\
1.66984174604365	31605.8416696835\\
1.66994174854371	31625.6087136156\\
1.67004175104378	31645.4330533271\\
1.67014175354384	31665.2000972591\\
1.6702417560439	31685.0244369706\\
1.67034175854396	31704.8487766822\\
1.67044176104403	31724.6158206142\\
1.67054176354409	31744.4401603257\\
1.67064176604415	31764.3217958167\\
1.67074176854421	31784.1461355283\\
1.67084177104428	31803.9704752398\\
1.67094177354434	31823.8521107308\\
1.6710417760444	31843.6764504424\\
1.67114177854446	31863.5580859334\\
1.67124178104453	31883.4397214244\\
1.67134178354459	31903.3213569155\\
1.67144178604465	31923.2029924065\\
1.67154178854471	31943.0846278975\\
1.67164179104478	31962.9662633886\\
1.67174179354484	31982.9051946591\\
1.6718417960449	32002.7868301502\\
1.67194179854496	32022.7257614207\\
1.67204180104503	32042.6646926913\\
1.67214180354509	32062.5463281823\\
1.67224180604515	32082.4852594529\\
1.67234180854521	32102.4814865029\\
1.67244181104528	32122.4204177735\\
1.67254181354534	32142.3593490441\\
1.6726418160454	32162.3555760941\\
1.67274181854546	32182.2945073647\\
1.67284182104553	32202.2907344147\\
1.67294182354559	32222.2869614648\\
1.67304182604565	32242.2831885149\\
1.67314182854571	32262.2794155649\\
1.67324183104578	32282.275642615\\
1.67334183354584	32302.2718696651\\
1.6734418360459	32322.2680967151\\
1.67354183854596	32342.3216195447\\
1.67364184104603	32362.3178465948\\
1.67374184354609	32382.3713694244\\
1.67384184604615	32402.4248922539\\
1.67394184854621	32422.4784150835\\
1.67404185104628	32442.5319379131\\
1.67414185354634	32462.5854607427\\
1.6742418560464	32482.6962793518\\
1.67434185854646	32502.7498021813\\
1.67444186104653	32522.8606207904\\
1.67454186354659	32542.91414362\\
1.67464186604665	32563.0249622291\\
1.67474186854671	32583.1357808382\\
1.67484187104678	32603.2465994473\\
1.67494187354684	32623.3574180564\\
1.6750418760469	32643.4682366655\\
1.67514187854696	32663.6363510541\\
1.67524188104703	32683.7471696632\\
1.67534188354709	32703.9152840518\\
1.67544188604715	32724.0261026609\\
1.67554188854721	32744.1942170495\\
1.67564189104728	32764.3623314381\\
1.67574189354734	32784.5304458267\\
1.6758418960474	32804.6985602153\\
1.67594189854746	32824.9239703834\\
1.67604190104753	32845.092084772\\
1.67614190354759	32865.3174949401\\
1.67624190604765	32885.4856093287\\
1.67634190854771	32905.7110194968\\
1.67644191104778	32925.936429665\\
1.67654191354784	32946.1618398331\\
1.6766419160479	32966.3872500012\\
1.67674191854796	32986.6126601693\\
1.67684192104803	33006.895366117\\
1.67694192354809	33027.1207762851\\
1.67704192604815	33047.4034822327\\
1.67714192854821	33067.6288924008\\
1.67724193104828	33087.9115983485\\
1.67734193354834	33108.1943042961\\
1.6774419360484	33128.4770102437\\
1.67754193854846	33148.7597161913\\
1.67764194104853	33169.0997179185\\
1.67774194354859	33189.3824238661\\
1.67784194604865	33209.6651298138\\
1.67794194854871	33230.0051315409\\
1.67804195104878	33250.345133268\\
1.67814195354884	33270.6851349952\\
1.6782419560489	33291.0251367223\\
1.67834195854896	33311.3651384495\\
1.67844196104903	33331.7051401766\\
1.67854196354909	33352.0451419038\\
1.67864196604915	33372.4424394104\\
1.67874196854921	33392.7824411376\\
1.67884197104928	33413.1797386442\\
1.67894197354934	33433.5770361509\\
1.6790419760494	33453.9743336575\\
1.67914197854946	33474.3716311642\\
1.67924198104953	33494.7689286708\\
1.67934198354959	33515.1662261775\\
1.67944198604965	33535.5635236842\\
1.67954198854971	33556.0181169703\\
1.67964199104978	33576.4727102565\\
1.67974199354984	33596.8700077632\\
1.6798419960499	33617.3246010493\\
1.67994199854996	33637.7791943355\\
1.68004200105003	33658.2337876217\\
1.68014200355009	33678.6883809078\\
1.68024200605015	33699.2002699735\\
1.68034200855021	33719.6548632597\\
1.68044201105028	33740.1667523254\\
1.68054201355034	33760.6213456116\\
1.6806420160504	33781.1332346772\\
1.68074201855046	33801.6451237429\\
1.68084202105053	33822.1570128086\\
1.68094202355059	33842.6689018743\\
1.68104202605065	33863.18079094\\
1.68114202855071	33883.7499757852\\
1.68124203105078	33904.2618648509\\
1.68134203355084	33924.831049696\\
1.6814420360509	33945.3429387617\\
1.68154203855096	33965.9121236069\\
1.68164204105103	33986.4813084521\\
1.68174204355109	34007.0504932973\\
1.68184204605115	34027.6196781425\\
1.68194204855121	34048.2461587672\\
1.68204205105128	34068.8153436124\\
1.68214205355134	34089.4418242371\\
1.6822420560514	34110.0110090823\\
1.68234205855146	34130.637489707\\
1.68244206105153	34151.2639703317\\
1.68254206355159	34171.8904509565\\
1.68264206605165	34192.5169315812\\
1.68274206855171	34213.1434122059\\
1.68284207105178	34233.8271886101\\
1.68294207355184	34254.4536692348\\
1.6830420760519	34275.137445639\\
1.68314207855196	34295.7639262637\\
1.68324208105203	34316.447702668\\
1.68334208355209	34337.1314790722\\
1.68344208605215	34357.8152554764\\
1.68354208855221	34378.4990318806\\
1.68364209105228	34399.2401040644\\
1.68374209355234	34419.9238804686\\
1.6838420960524	34440.6649526523\\
1.68394209855246	34461.3487290565\\
1.68404210105253	34482.0898012403\\
1.68414210355259	34502.830873424\\
1.68424210605265	34523.5719456077\\
1.68434210855271	34544.3130177915\\
1.68444211105278	34565.0540899752\\
1.68454211355284	34585.795162159\\
1.6846421160529	34606.5935301222\\
1.68474211855296	34627.3918980855\\
1.68484212105303	34648.1329702692\\
1.68494212355309	34668.9313382324\\
1.68504212605315	34689.7297061957\\
1.68514212855321	34710.5280741589\\
1.68524213105328	34731.3264421222\\
1.68534213355334	34752.1248100854\\
1.6854421360534	34772.9804738282\\
1.68554213855346	34793.7788417914\\
1.68564214105353	34814.6345055342\\
1.68574214355359	34835.490169277\\
1.68584214605365	34856.3458330197\\
1.68594214855371	34877.2014967625\\
1.68604215105378	34898.0571605053\\
1.68614215355384	34918.912824248\\
1.6862421560539	34939.7684879908\\
1.68634215855396	34960.6814475131\\
1.68644216105403	34981.5371112558\\
1.68654216355409	35002.4500707781\\
1.68664216605415	35023.3630303004\\
1.68674216855421	35044.2759898227\\
1.68684217105428	35065.1889493449\\
1.68694217355434	35086.1019088672\\
1.6870421760544	35107.0148683895\\
1.68714217855446	35127.9278279117\\
1.68724218105453	35148.8980832135\\
1.68734218355459	35169.8683385153\\
1.68744218605465	35190.7812980376\\
1.68754218855471	35211.7515533394\\
1.68764219105478	35232.7218086412\\
1.68774219355484	35253.692063943\\
1.6878421960549	35274.6623192448\\
1.68794219855496	35295.6898703261\\
1.68804220105503	35316.6601256278\\
1.68814220355509	35337.6876767091\\
1.68824220605515	35358.6579320109\\
1.68834220855521	35379.6854830922\\
1.68844221105528	35400.7130341735\\
1.68854221355534	35421.7405852548\\
1.6886422160554	35442.7681363361\\
1.68874221855546	35463.8529831969\\
1.68884222105553	35484.8805342782\\
1.68894222355559	35505.9080853596\\
1.68904222605565	35526.9929322204\\
1.68914222855571	35548.0777790812\\
1.68924223105578	35569.162625942\\
1.68934223355584	35590.2474728028\\
1.6894422360559	35611.3323196636\\
1.68954223855596	35632.4171665244\\
1.68964224105603	35653.5020133853\\
1.68974224355609	35674.6441560256\\
1.68984224605615	35695.7290028864\\
1.68994224855621	35716.8711455267\\
1.69004225105628	35738.013288167\\
1.69014225355634	35759.1554308074\\
1.6902422560564	35780.2975734477\\
1.69034225855646	35801.439716088\\
1.69044226105653	35822.5818587284\\
1.69054226355659	35843.7240013687\\
1.69064226605665	35864.9234397885\\
1.69074226855671	35886.1228782084\\
1.69084227105678	35907.2650208487\\
1.69094227355684	35928.4644592685\\
1.6910422760569	35949.6638976884\\
1.69114227855696	35970.8633361082\\
1.69124228105703	35992.0627745281\\
1.69134228355709	36013.3195087274\\
1.69144228605715	36034.5189471473\\
1.69154228855721	36055.7756813466\\
1.69164229105728	36076.9751197664\\
1.69174229355734	36098.2318539658\\
1.6918422960574	36119.4885881651\\
1.69194229855746	36140.7453223645\\
1.69204230105753	36162.0020565639\\
1.69214230355759	36183.3160865427\\
1.69224230605765	36204.5728207421\\
1.69234230855771	36225.8295549414\\
1.69244231105778	36247.1435849203\\
1.69254231355784	36268.4576148992\\
1.6926423160579	36289.771644878\\
1.69274231855796	36311.0856748569\\
1.69284232105803	36332.3997048358\\
1.69294232355809	36353.7137348146\\
1.69304232605815	36375.0277647935\\
1.69314232855821	36396.3990905519\\
1.69324233105828	36417.7131205307\\
1.69334233355834	36439.0844462891\\
1.6934423360584	36460.4557720475\\
1.69354233855846	36481.8270978059\\
1.69364234105853	36503.1984235643\\
1.69374234355859	36524.5697493226\\
1.69384234605865	36545.941075081\\
1.69394234855871	36567.3124008394\\
1.69404235105878	36588.7410223773\\
1.69414235355884	36610.1696439152\\
1.6942423560589	36631.5409696736\\
1.69434235855896	36652.9695912115\\
1.69444236105903	36674.3982127494\\
1.69454236355909	36695.8268342872\\
1.69464236605915	36717.2554558251\\
1.69474236855921	36738.7413731425\\
1.69484237105928	36760.1699946804\\
1.69494237355934	36781.6559119978\\
1.6950423760594	36803.1418293152\\
1.69514237855946	36824.5704508531\\
1.69524238105953	36846.0563681705\\
1.69534238355959	36867.542285488\\
1.69544238605965	36889.0282028054\\
1.69554238855971	36910.5714159023\\
1.69564239105978	36932.0573332197\\
1.69574239355984	36953.6005463166\\
1.6958423960599	36975.086463634\\
1.69594239855996	36996.6296767309\\
1.69604240106003	37018.1728898278\\
1.69614240356009	37039.7161029248\\
1.69624240606015	37061.2593160217\\
1.69634240856021	37082.8025291186\\
1.69644241106028	37104.3457422155\\
1.69654241356034	37125.946251092\\
1.6966424160604	37147.4894641889\\
1.69674241856046	37169.0899730653\\
1.69684242106053	37190.6904819417\\
1.69694242356059	37212.2909908182\\
1.69704242606065	37233.8914996946\\
1.69714242856071	37255.492008571\\
1.69724243106078	37277.0925174475\\
1.69734243356084	37298.7503221034\\
1.6974424360609	37320.3508309798\\
1.69754243856096	37342.0086356358\\
1.69764244106103	37363.6664402917\\
1.69774244356109	37385.3242449477\\
1.69784244606115	37406.9820496036\\
1.69794244856121	37428.6398542596\\
1.69804245106128	37450.2976589155\\
1.69814245356134	37471.9554635715\\
1.6982424560614	37493.6705640069\\
1.69834245856146	37515.3283686629\\
1.69844246106153	37537.0434690983\\
1.69854246356159	37558.7585695338\\
1.69864246606165	37580.4736699692\\
1.69874246856171	37602.1887704047\\
1.69884247106178	37623.9038708401\\
1.69894247356184	37645.6189712756\\
1.6990424760619	37667.3913674906\\
1.69914247856196	37689.106467926\\
1.69924248106203	37710.878864141\\
1.69934248356209	37732.651260356\\
1.69944248606215	37754.3663607914\\
1.69954248856221	37776.1387570064\\
1.69964249106228	37797.9684490009\\
1.69974249356234	37819.7408452159\\
1.6998424960624	37841.5132414308\\
1.69994249856246	37863.3429334253\\
1.70004250106253	37885.1153296403\\
1.70014250356259	37906.9450216348\\
1.70024250606265	37928.7747136293\\
1.70034250856271	37950.6044056238\\
1.70044251106278	37972.4340976182\\
1.70054251356284	37994.2637896127\\
1.7006425160629	38016.0934816072\\
1.70074251856296	38037.9804693812\\
1.70084252106303	38059.8101613757\\
1.70094252356309	38081.6971491497\\
1.70104252606315	38103.5841369237\\
1.70114252856321	38125.4138289182\\
1.70124253106328	38147.3008166922\\
1.70134253356334	38169.2451002457\\
1.7014425360634	38191.1320880197\\
1.70154253856346	38213.0190757937\\
1.70164254106353	38234.9633593472\\
1.70174254356359	38256.8503471212\\
1.70184254606365	38278.7946306747\\
1.70194254856371	38300.7389142282\\
1.70204255106378	38322.6831977817\\
1.70214255356384	38344.6274813352\\
1.7022425560639	38366.5717648887\\
1.70234255856396	38388.5160484422\\
1.70244256106403	38410.5176277753\\
1.70254256356409	38432.4619113288\\
1.70264256606415	38454.4634906618\\
1.70274256856421	38476.4650699948\\
1.70284257106428	38498.4666493278\\
1.70294257356434	38520.4682286609\\
1.7030425760644	38542.4698079939\\
1.70314257856446	38564.4713873269\\
1.70324258106453	38586.5302624394\\
1.70334258356459	38608.5318417725\\
1.70344258606465	38630.590716885\\
1.70354258856471	38652.592296218\\
1.70364259106478	38674.6511713306\\
1.70374259356484	38696.7100464431\\
1.7038425960649	38718.7689215556\\
1.70394259856496	38740.8850924477\\
1.70404260106503	38762.9439675602\\
1.70414260356509	38785.0028426728\\
1.70424260606515	38807.1190135648\\
1.70434260856521	38829.2351844569\\
1.70444261106528	38851.2940595694\\
1.70454261356534	38873.4102304615\\
1.7046426160654	38895.5264013535\\
1.70474261856546	38917.6998680251\\
1.70484262106553	38939.8160389171\\
1.70494262356559	38961.9322098092\\
1.70504262606565	38984.1056764807\\
1.70514262856571	39006.2218473728\\
1.70524263106578	39028.3953140443\\
1.70534263356584	39050.5687807159\\
1.7054426360659	39072.7422473875\\
1.70554263856596	39094.915714059\\
1.70564264106603	39117.0891807306\\
1.70574264356609	39139.3199431817\\
1.70584264606615	39161.4934098532\\
1.70594264856621	39183.7241723043\\
1.70604265106628	39205.8976389759\\
1.70614265356634	39228.1284014269\\
1.7062426560664	39250.359163878\\
1.70634265856646	39272.5899263291\\
1.70644266106653	39294.8206887802\\
1.70654266356659	39317.1087470108\\
1.70664266606665	39339.3395094618\\
1.70674266856671	39361.6275676924\\
1.70684267106678	39383.8583301435\\
1.70694267356684	39406.1463883741\\
1.7070426760669	39428.4344466047\\
1.70714267856696	39450.7225048353\\
1.70724268106703	39473.0105630659\\
1.70734268356709	39495.2986212964\\
1.70744268606715	39517.6439753065\\
1.70754268856721	39539.9320335371\\
1.70764269106728	39562.2773875472\\
1.70774269356734	39584.6227415573\\
1.7078426960674	39606.9107997879\\
1.70794269856746	39629.256153798\\
1.70804270106753	39651.6015078081\\
1.70814270356759	39674.0041575978\\
1.70824270606765	39696.3495116078\\
1.70834270856771	39718.694865618\\
1.70844271106778	39741.0975154076\\
1.70854271356784	39763.5001651972\\
1.7086427160679	39785.8455192073\\
1.70874271856796	39808.2481689969\\
1.70884272106803	39830.6508187865\\
1.70894272356809	39853.1107643556\\
1.70904272606815	39875.5134141453\\
1.70914272856821	39897.9160639349\\
1.70924273106828	39920.376009504\\
1.70934273356834	39942.7786592936\\
1.7094427360684	39965.2386048627\\
1.70954273856846	39987.6985504319\\
1.70964274106853	40010.158496001\\
1.70974274356859	40032.6184415701\\
1.70984274606865	40055.0783871393\\
1.70994274856871	40077.5956284879\\
1.71004275106878	40100.055574057\\
1.71014275356884	40122.5728154057\\
1.7102427560689	40145.0327609748\\
1.71034275856896	40167.5500023234\\
1.71044276106903	40190.0672436721\\
1.71054276356909	40212.5844850207\\
1.71064276606915	40235.1017263694\\
1.71074276856921	40257.6762634975\\
1.71084277106928	40280.1935048462\\
1.71094277356934	40302.7680419743\\
1.7110427760694	40325.285283323\\
1.71114277856946	40347.8598204511\\
1.71124278106953	40370.4343575793\\
1.71134278356959	40393.0088947074\\
1.71144278606965	40415.5834318356\\
1.71154278856971	40438.1579689637\\
1.71164279106978	40460.7898018714\\
1.71174279356984	40483.3643389995\\
1.7118427960699	40505.9961719072\\
1.71194279856996	40528.6280048149\\
1.71204280107003	40551.202541943\\
1.71214280357009	40573.8343748507\\
1.71224280607015	40596.4662077584\\
1.71234280857021	40619.1553364456\\
1.71244281107028	40641.7871693532\\
1.71254281357034	40664.4190022609\\
1.7126428160704	40687.1081309481\\
1.71274281857046	40709.7972596352\\
1.71284282107053	40732.4290925429\\
1.71294282357059	40755.1182212301\\
1.71304282607065	40777.8073499173\\
1.71314282857071	40800.553774384\\
1.71324283107078	40823.2429030712\\
1.71334283357084	40845.9320317583\\
1.7134428360709	40868.678456225\\
1.71354283857096	40891.3675849122\\
1.71364284107103	40914.1140093789\\
1.71374284357109	40936.8604338456\\
1.71384284607115	40959.6068583123\\
1.71394284857121	40982.353282779\\
1.71404285107128	41005.0997072457\\
1.71414285357134	41027.9034274919\\
1.7142428560714	41050.6498519586\\
1.71434285857146	41073.4535722048\\
1.71444286107153	41096.1999966715\\
1.71454286357159	41119.0037169177\\
1.71464286607165	41141.8074371639\\
1.71474286857171	41164.6111574101\\
1.71484287107178	41187.4148776563\\
1.71494287357184	41210.275893682\\
1.7150428760719	41233.0796139282\\
1.71514287857196	41255.940629954\\
1.71524288107203	41278.7443502002\\
1.71534288357209	41301.6053662259\\
1.71544288607215	41324.4663822516\\
1.71554288857221	41347.3273982773\\
1.71564289107228	41370.188414303\\
1.71574289357234	41393.1067261083\\
1.7158428960724	41415.967742134\\
1.71594289857246	41438.8287581597\\
1.71604290107253	41461.7470699649\\
1.71614290357259	41484.6653817702\\
1.71624290607265	41507.5836935754\\
1.71634290857271	41530.5020053807\\
1.71644291107278	41553.4203171859\\
1.71654291357284	41576.3386289911\\
1.7166429160729	41599.2569407963\\
1.71674291857296	41622.2325483811\\
1.71684292107303	41645.1508601863\\
1.71694292357309	41668.1264677711\\
1.71704292607315	41691.1020753558\\
1.71714292857321	41714.0776829406\\
1.71724293107328	41737.0532905253\\
1.71734293357334	41760.0288981101\\
1.7174429360734	41783.0045056948\\
1.71754293857346	41806.0374090591\\
1.71764294107353	41829.0130166438\\
1.71774294357359	41852.0459200081\\
1.71784294607365	41875.0788233723\\
1.71794294857371	41898.1117267366\\
1.71804295107378	41921.1446301008\\
1.71814295357384	41944.1775334651\\
1.7182429560739	41967.2104368294\\
1.71834295857396	41990.2433401936\\
1.71844296107403	42013.3335393374\\
1.71854296357409	42036.3664427016\\
1.71864296607415	42059.4566418454\\
1.71874296857421	42082.5468409892\\
1.71884297107428	42105.637040133\\
1.71894297357434	42128.7272392767\\
1.7190429760744	42151.8174384205\\
1.71914297857446	42174.9649333438\\
1.71924298107453	42198.0551324876\\
1.71934298357459	42221.2026274109\\
1.71944298607465	42244.2928265546\\
1.71954298857471	42267.4403214779\\
1.71964299107478	42290.5878164012\\
1.71974299357484	42313.7353113245\\
1.7198429960749	42336.8828062478\\
1.71994299857496	42360.0875969506\\
1.72004300107503	42383.2350918738\\
1.72014300357509	42406.3825867971\\
1.72024300607515	42429.5873774999\\
1.72034300857521	42452.7921682027\\
1.72044301107528	42475.9969589055\\
1.72054301357534	42499.2017496083\\
1.7206430160754	42522.4065403111\\
1.72074301857546	42545.6113310139\\
1.72084302107553	42568.8161217167\\
1.72094302357559	42592.078208199\\
1.72104302607565	42615.3402946813\\
1.72114302857571	42638.5450853841\\
1.72124303107578	42661.8071718665\\
1.72134303357584	42685.0692583488\\
1.7214430360759	42708.3313448311\\
1.72154303857596	42731.5934313134\\
1.72164304107603	42754.9128135752\\
1.72174304357609	42778.1749000575\\
1.72184304607615	42801.4942823194\\
1.72194304857621	42824.7563688017\\
1.72204305107628	42848.0757510635\\
1.72214305357634	42871.3951333253\\
1.7222430560764	42894.7145155871\\
1.72234305857646	42918.033897849\\
1.72244306107653	42941.4105758903\\
1.72254306357659	42964.7299581521\\
1.72264306607665	42988.1066361935\\
1.72274306857671	43011.4260184553\\
1.72284307107678	43034.8026964966\\
1.72294307357684	43058.179374538\\
1.7230430760769	43081.5560525793\\
1.72314307857696	43104.9327306206\\
1.72324308107703	43128.309408662\\
1.72334308357709	43151.7433824828\\
1.72344308607715	43175.1200605242\\
1.72354308857721	43198.554034345\\
1.72364309107728	43221.9307123863\\
1.72374309357734	43245.3646862072\\
1.7238430960774	43268.798660028\\
1.72394309857746	43292.2326338489\\
1.72404310107753	43315.7239034493\\
1.72414310357759	43339.1578772701\\
1.72424310607765	43362.591851091\\
1.72434310857771	43386.0831206913\\
1.72444311107778	43409.5743902917\\
1.72454311357784	43433.0083641125\\
1.7246431160779	43456.4996337129\\
1.72474311857796	43479.9909033133\\
1.72484312107803	43503.4821729136\\
1.72494312357809	43527.0307382935\\
1.72504312607815	43550.5220078939\\
1.72514312857821	43574.0705732737\\
1.72524313107828	43597.5618428741\\
1.72534313357834	43621.110408254\\
1.7254431360784	43644.6589736339\\
1.72554313857846	43668.2075390137\\
1.72564314107853	43691.7561043936\\
1.72574314357859	43715.3046697735\\
1.72584314607865	43738.9105309329\\
1.72594314857871	43762.4590963128\\
1.72604315107878	43786.0649574722\\
1.72614315357884	43809.6708186316\\
1.7262431560789	43833.2193840114\\
1.72634315857896	43856.8252451708\\
1.72644316107903	43880.4311063302\\
1.72654316357909	43904.0942632691\\
1.72664316607915	43927.7001244285\\
1.72674316857921	43951.3059855879\\
1.72684317107928	43974.9691425268\\
1.72694317357934	43998.6322994657\\
1.7270431760794	44022.2954564046\\
1.72714317857946	44045.901317564\\
1.72724318107953	44069.6217702824\\
1.72734318357959	44093.2849272213\\
1.72744318607965	44116.9480841602\\
1.72754318857971	44140.6112410991\\
1.72764319107978	44164.3316938175\\
1.72774319357984	44188.0521465359\\
1.7278431960799	44211.7153034748\\
1.72794319857996	44235.4357561933\\
1.72804320108003	44259.1562089117\\
1.72814320358009	44282.8766616301\\
1.72824320608015	44306.654410128\\
1.72834320858021	44330.3748628464\\
1.72844321108028	44354.1526113444\\
1.72854321358034	44377.8730640628\\
1.7286432160804	44401.6508125607\\
1.72874321858046	44425.4285610586\\
1.72884322108053	44449.2063095566\\
1.72894322358059	44472.9840580545\\
1.72904322608065	44496.7618065524\\
1.72914322858071	44520.5395550504\\
1.72924323108078	44544.3745993278\\
1.72934323358084	44568.1523478257\\
1.7294432360809	44591.9873921032\\
1.72954323858096	44615.8224363806\\
1.72964324108103	44639.6574806581\\
1.72974324358109	44663.4925249355\\
1.72984324608115	44687.3275692129\\
1.72994324858121	44711.1626134904\\
1.73004325108128	44735.0549535473\\
1.73014325358134	44758.8899978248\\
1.7302432560814	44782.7823378817\\
1.73034325858146	44806.6746779387\\
1.73044326108153	44830.5670179957\\
1.73054326358159	44854.4593580526\\
1.73064326608165	44878.3516981096\\
1.73074326858171	44902.2440381665\\
1.73084327108178	44926.193674003\\
1.73094327358184	44950.0860140599\\
1.7310432760819	44974.0356498964\\
1.73114327858196	44997.9279899534\\
1.73124328108203	45021.8776257898\\
1.73134328358209	45045.8272616263\\
1.73144328608215	45069.7768974628\\
1.73154328858221	45093.7838290787\\
1.73164329108228	45117.7334649152\\
1.73174329358234	45141.6831007517\\
1.7318432960824	45165.6900323677\\
1.73194329858246	45189.6969639836\\
1.73204330108253	45213.7038955996\\
1.73214330358259	45237.7108272156\\
1.73224330608265	45261.7177588316\\
1.73234330858271	45285.7246904476\\
1.73244331108278	45309.7316220636\\
1.73254331358284	45333.795849459\\
1.7326433160829	45357.802781075\\
1.73274331858296	45381.8670084705\\
1.73284332108303	45405.931235866\\
1.73294332358309	45429.9954632615\\
1.73304332608315	45454.059690657\\
1.73314332858321	45478.1239180525\\
1.73324333108328	45502.188145448\\
1.73334333358334	45526.2523728435\\
1.7334433360834	45550.3738960185\\
1.73354333858346	45574.4954191935\\
1.73364334108353	45598.559646589\\
1.73374334358359	45622.681169764\\
1.73384334608365	45646.802692939\\
1.73394334858371	45670.924216114\\
1.73404335108378	45695.1030350685\\
1.73414335358384	45719.2245582435\\
1.7342433560839	45743.4033771981\\
1.73434335858396	45767.5249003731\\
1.73444336108403	45791.7037193276\\
1.73454336358409	45815.8825382821\\
1.73464336608415	45840.0613572366\\
1.73474336858421	45864.2401761912\\
1.73484337108428	45888.4189951457\\
1.73494337358434	45912.5978141002\\
1.7350433760844	45936.8339288342\\
1.73514337858446	45961.0127477888\\
1.73524338108453	45985.2488625228\\
1.73534338358459	46009.4849772568\\
1.73544338608465	46033.7210919909\\
1.73554338858471	46057.9572067249\\
1.73564339108478	46082.1933214589\\
1.73574339358484	46106.429436193\\
1.7358433960849	46130.7228467065\\
1.73594339858496	46154.9589614405\\
1.73604340108503	46179.2523719541\\
1.73614340358509	46203.5457824676\\
1.73624340608515	46227.8391929812\\
1.73634340858521	46252.1326034947\\
1.73644341108528	46276.4260140083\\
1.73654341358534	46300.7194245218\\
1.7366434160854	46325.0701308149\\
1.73674341858546	46349.3635413284\\
1.73684342108553	46373.7142476215\\
1.73694342358559	46398.007658135\\
1.73704342608565	46422.3583644281\\
1.73714342858571	46446.7090707212\\
1.73724343108578	46471.0597770142\\
1.73734343358584	46495.4677790868\\
1.7374434360859	46519.8184853799\\
1.73754343858596	46544.1691916729\\
1.73764344108603	46568.5771937455\\
1.73774344358609	46592.9851958181\\
1.73784344608615	46617.3931978906\\
1.73794344858621	46641.8011999632\\
1.73804345108628	46666.2092020358\\
1.73814345358634	46690.6172041083\\
1.7382434560864	46715.0252061809\\
1.73834345858646	46739.490504033\\
1.73844346108653	46763.8985061056\\
1.73854346358659	46788.3638039577\\
1.73864346608665	46812.8291018098\\
1.73874346858671	46837.2943996618\\
1.73884347108678	46861.7596975139\\
1.73894347358684	46886.224995366\\
1.7390434760869	46910.6902932181\\
1.73914347858696	46935.2128868497\\
1.73924348108703	46959.6781847018\\
1.73934348358709	46984.2007783334\\
1.73944348608715	47008.723371965\\
1.73954348858721	47033.2459655966\\
1.73964349108728	47057.7685592282\\
1.73974349358734	47082.2911528598\\
1.7398434960874	47106.8137464914\\
1.73994349858746	47131.336340123\\
1.74004350108753	47155.9162295341\\
1.74014350358759	47180.4388231657\\
1.74024350608765	47205.0187125768\\
1.74034350858771	47229.5986019879\\
1.74044351108778	47254.178491399\\
1.74054351358784	47278.7583808101\\
1.7406435160879	47303.3382702213\\
1.74074351858796	47327.9754554119\\
1.74084352108803	47352.555344823\\
1.74094352358809	47377.1925300136\\
1.74104352608815	47401.8297152042\\
1.74114352858821	47426.4096046153\\
1.74124353108828	47451.046789806\\
1.74134353358834	47475.6839749966\\
1.7414435360884	47500.3784559667\\
1.74154353858846	47525.0156411574\\
1.74164354108853	47549.652826348\\
1.74174354358859	47574.3473073181\\
1.74184354608865	47599.0417882883\\
1.74194354858871	47623.6789734789\\
1.74204355108878	47648.373454449\\
1.74214355358884	47673.0679354192\\
1.7422435560889	47697.8197121688\\
1.74234355858896	47722.514193139\\
1.74244356108903	47747.2086741091\\
1.74254356358909	47771.9604508588\\
1.74264356608915	47796.7122276084\\
1.74274356858921	47821.4067085785\\
1.74284357108928	47846.1584853282\\
1.74294357358934	47870.9102620778\\
1.7430435760894	47895.6620388275\\
1.74314357858946	47920.4711113567\\
1.74324358108953	47945.2228881063\\
1.74334358358959	47970.0319606355\\
1.74344358608965	47994.7837373851\\
1.74354358858971	48019.5928099143\\
1.74364359108978	48044.4018824435\\
1.74374359358984	48069.2109549726\\
1.7438435960899	48094.0200275018\\
1.74394359858996	48118.829100031\\
1.74404360109003	48143.6381725601\\
1.74414360359009	48168.5045408688\\
1.74424360609015	48193.3709091775\\
1.74434360859021	48218.1799817066\\
1.74444361109028	48243.0463500153\\
1.74454361359034	48267.912718324\\
1.7446436160904	48292.7790866327\\
1.74474361859046	48317.6454549414\\
1.74484362109053	48342.5691190295\\
1.74494362359059	48367.4354873382\\
1.74504362609065	48392.3591514264\\
1.74514362859071	48417.2255197351\\
1.74524363109078	48442.1491838233\\
1.74534363359084	48467.0728479115\\
1.7454436360909	48491.9965119997\\
1.74554363859096	48516.9201760878\\
1.74564364109103	48541.9011359556\\
1.74574364359109	48566.8248000437\\
1.74584364609115	48591.8057599114\\
1.74594364859121	48616.7294239996\\
1.74604365109128	48641.7103838673\\
1.74614365359134	48666.6913437351\\
1.7462436560914	48691.6723036027\\
1.74634365859146	48716.6532634705\\
1.74644366109153	48741.6915191177\\
1.74654366359159	48766.6724789854\\
1.74664366609165	48791.6534388531\\
1.74674366859171	48816.6916945003\\
1.74684367109178	48841.7299501475\\
1.74694367359184	48866.7682057947\\
1.7470436760919	48891.8064614419\\
1.74714367859196	48916.8447170892\\
1.74724368109203	48941.8829727364\\
1.74734368359209	48966.9212283836\\
1.74744368609215	48992.0167798103\\
1.74754368859221	49017.1123312371\\
1.74764369109228	49042.1505868843\\
1.74774369359234	49067.246138311\\
1.7478436960924	49092.3416897377\\
1.74794369859246	49117.4372411645\\
1.74804370109253	49142.5327925912\\
1.74814370359259	49167.6856397974\\
1.74824370609265	49192.7811912242\\
1.74834370859271	49217.9340384304\\
1.74844371109278	49243.0868856367\\
1.74854371359284	49268.1824370634\\
1.7486437160929	49293.3352842696\\
1.74874371859296	49318.4881314759\\
1.74884372109303	49343.6982744616\\
1.74894372359309	49368.8511216679\\
1.74904372609315	49394.0039688741\\
1.74914372859321	49419.2141118599\\
1.74924373109328	49444.4242548456\\
1.74934373359334	49469.5771020519\\
1.7494437360934	49494.7872450376\\
1.74954373859346	49519.9973880234\\
1.74964374109353	49545.2075310091\\
1.74974374359359	49570.4749697744\\
1.74984374609365	49595.6851127602\\
1.74994374859371	49620.9525515254\\
1.75004375109378	49646.1626945112\\
1.75014375359384	49671.4301332765\\
1.7502437560939	49696.6975720417\\
1.75034375859396	49721.965010807\\
1.75044376109403	49747.2324495723\\
1.75054376359409	49772.4998883375\\
1.75064376609415	49797.8246228823\\
1.75074376859421	49823.0920616476\\
1.75084377109428	49848.4167961924\\
1.75094377359434	49873.7415307372\\
1.7510437760944	49899.0662652819\\
1.75114377859446	49924.3909998267\\
1.75124378109453	49949.7157343715\\
1.75134378359459	49975.0404689163\\
1.75144378609465	50000.3652034611\\
1.75154378859471	50025.7472337854\\
1.75164379109478	50051.0719683302\\
1.75174379359484	50076.4539986544\\
1.7518437960949	50101.8360289787\\
1.75194379859496	50127.218059303\\
1.75204380109503	50152.6000896273\\
1.75214380359509	50177.9821199516\\
1.75224380609515	50203.4214460554\\
1.75234380859521	50228.8034763797\\
1.75244381109528	50254.2428024835\\
1.75254381359534	50279.6248328078\\
1.7526438160954	50305.0641589116\\
1.75274381859546	50330.5034850154\\
1.75284382109553	50355.9428111192\\
1.75294382359559	50381.3821372231\\
1.75304382609565	50406.8787591064\\
1.75314382859571	50432.3180852102\\
1.75324383109578	50457.8147070935\\
1.75334383359584	50483.2540331973\\
1.7534438360959	50508.7506550806\\
1.75354383859596	50534.247276964\\
1.75364384109603	50559.7438988473\\
1.75374384359609	50585.2405207306\\
1.75384384609615	50610.7944383934\\
1.75394384859621	50636.2910602768\\
1.75404385109628	50661.7876821601\\
1.75414385359634	50687.3415998229\\
1.7542438560964	50712.8955174858\\
1.75434385859647	50738.4494351486\\
1.75444386109653	50764.0033528114\\
1.75454386359659	50789.5572704743\\
1.75464386609665	50815.1111881371\\
1.75474386859671	50840.7224015794\\
1.75484387109678	50866.2763192423\\
1.75494387359684	50891.8875326846\\
1.7550438760969	50917.4414503475\\
1.75514387859696	50943.0526637898\\
1.75524388109703	50968.6638772322\\
1.75534388359709	50994.2750906745\\
1.75544388609715	51019.9435998964\\
1.75554388859721	51045.5548133387\\
1.75564389109728	51071.2233225606\\
1.75574389359734	51096.8345360029\\
1.7558438960974	51122.5030452248\\
1.75594389859746	51148.1715544466\\
1.75604390109753	51173.8400636685\\
1.75614390359759	51199.5085728904\\
1.75624390609765	51225.1770821122\\
1.75634390859772	51250.8455913341\\
1.75644391109778	51276.5713963355\\
1.75654391359784	51302.2399055573\\
1.7566439160979	51327.9657105587\\
1.75674391859796	51353.6915155601\\
1.75684392109803	51379.4173205614\\
1.75694392359809	51405.1431255628\\
1.75704392609815	51430.8689305642\\
1.75714392859821	51456.5947355656\\
1.75724393109828	51482.3778363465\\
1.75734393359834	51508.1036413478\\
1.7574439360984	51533.8867421287\\
1.75754393859847	51559.6698429096\\
1.75764394109853	51585.4529436905\\
1.75774394359859	51611.2360444714\\
1.75784394609865	51637.0191452523\\
1.75794394859871	51662.8022460331\\
1.75804395109878	51688.6426425935\\
1.75814395359884	51714.4257433744\\
1.7582439560989	51740.2661399348\\
1.75834395859897	51766.1065364952\\
1.75844396109903	51791.9469330556\\
1.75854396359909	51817.787329616\\
1.75864396609915	51843.6277261764\\
1.75874396859922	51869.4681227368\\
1.75884397109928	51895.3085192972\\
1.75894397359934	51921.2062116371\\
1.7590439760994	51947.1039039771\\
1.75914397859946	51972.9443005375\\
1.75924398109953	51998.8419928774\\
1.75934398359959	52024.7396852173\\
1.75944398609965	52050.6373775572\\
1.75954398859972	52076.5923656766\\
1.75964399109978	52102.4900580165\\
1.75974399359984	52128.445046136\\
1.7598439960999	52154.3427384759\\
1.75994399859996	52180.2977265953\\
1.76004400110003	52206.2527147147\\
1.76014400360009	52232.2077028342\\
1.76024400610015	52258.1626909536\\
1.76034400860022	52284.117679073\\
1.76044401110028	52310.0726671924\\
1.76054401360034	52336.0849510914\\
1.7606440161004	52362.0399392108\\
1.76074401860047	52388.0522231097\\
1.76084402110053	52414.0645070087\\
1.76094402360059	52440.0767909076\\
1.76104402610065	52466.0890748066\\
1.76114402860071	52492.1013587055\\
1.76124403110078	52518.170938384\\
1.76134403360084	52544.1832222829\\
1.7614440361009	52570.2528019613\\
1.76154403860097	52596.2650858603\\
1.76164404110103	52622.3346655387\\
1.76174404360109	52648.4042452172\\
1.76184404610115	52674.4738248956\\
1.76194404860122	52700.5434045741\\
1.76204405110128	52726.6702800321\\
1.76214405360134	52752.7398597105\\
1.7622440561014	52778.8667351685\\
1.76234405860147	52804.9363148469\\
1.76244406110153	52831.0631903049\\
1.76254406360159	52857.1900657629\\
1.76264406610165	52883.3169412208\\
1.76274406860172	52909.4438166788\\
1.76284407110178	52935.6279879163\\
1.76294407360184	52961.7548633742\\
1.7630440761019	52987.8817388322\\
1.76314407860196	53014.0659100697\\
1.76324408110203	53040.2500813072\\
1.76334408360209	53066.4342525446\\
1.76344408610215	53092.6184237821\\
1.76354408860222	53118.8025950196\\
1.76364409110228	53144.9867662571\\
1.76374409360234	53171.2282332741\\
1.7638440961024	53197.4124045115\\
1.76394409860247	53223.6538715285\\
1.76404410110253	53249.8953385455\\
1.76414410360259	53276.079509783\\
1.76424410610265	53302.3209768\\
1.76434410860272	53328.6197395965\\
1.76444411110278	53354.8612066135\\
1.76454411360284	53381.1026736305\\
1.7646441161029	53407.401436427\\
1.76474411860297	53433.642903444\\
1.76484412110303	53459.9416662405\\
1.76494412360309	53486.240429037\\
1.76504412610315	53512.5391918335\\
1.76514412860322	53538.83795463\\
1.76524413110328	53565.1367174265\\
1.76534413360334	53591.4927760025\\
1.7654441361034	53617.791538799\\
1.76554413860347	53644.147597375\\
1.76564414110353	53670.4463601715\\
1.76574414360359	53696.8024187476\\
1.76584414610365	53723.1584773236\\
1.76594414860372	53749.5145358996\\
1.76604415110378	53775.8705944756\\
1.76614415360384	53802.2839488312\\
1.7662441561039	53828.6400074072\\
1.76634415860397	53855.0533617627\\
1.76644416110403	53881.4667161182\\
1.76654416360409	53907.8227746943\\
1.76664416610415	53934.2361290498\\
1.76674416860422	53960.6494834053\\
1.76684417110428	53987.1201335404\\
1.76694417360434	54013.5334878959\\
1.7670441761044	54039.9468422514\\
1.76714417860447	54066.4174923865\\
1.76724418110453	54092.8881425215\\
1.76734418360459	54119.301496877\\
1.76744418610465	54145.7721470121\\
1.76754418860472	54172.2427971471\\
1.76764419110478	54198.7707430617\\
1.76774419360484	54225.2413931967\\
1.7678441961049	54251.7120433318\\
1.76794419860497	54278.2399892463\\
1.76804420110503	54304.7106393814\\
1.76814420360509	54331.2385852959\\
1.76824420610515	54357.7665312105\\
1.76834420860522	54384.294477125\\
1.76844421110528	54410.8224230396\\
1.76854421360534	54437.4076647337\\
1.7686442161054	54463.9356106482\\
1.76874421860547	54490.5208523423\\
1.76884422110553	54517.0487982569\\
1.76894422360559	54543.6340399509\\
1.76904422610565	54570.219281645\\
1.76914422860572	54596.8045233391\\
1.76924423110578	54623.3897650331\\
1.76934423360584	54649.9750067272\\
1.7694442361059	54676.6175442008\\
1.76954423860597	54703.2027858949\\
1.76964424110603	54729.8453233684\\
1.76974424360609	54756.487860842\\
1.76984424610615	54783.0731025361\\
1.76994424860622	54809.7156400097\\
1.77004425110628	54836.4154732628\\
1.77014425360634	54863.0580107364\\
1.7702442561064	54889.7005482099\\
1.77034425860647	54916.400381463\\
1.77044426110653	54943.0429189366\\
1.77054426360659	54969.7427521897\\
1.77064426610665	54996.4425854428\\
1.77074426860672	55023.1424186959\\
1.77084427110678	55049.842251949\\
1.77094427360684	55076.5420852021\\
1.7710442761069	55103.2992142347\\
1.77114427860697	55129.9990474878\\
1.77124428110703	55156.7561765204\\
1.77134428360709	55183.4560097735\\
1.77144428610715	55210.2131388061\\
1.77154428860722	55236.9702678387\\
1.77164429110728	55263.7273968713\\
1.77174429360734	55290.5418216835\\
1.7718442961074	55317.2989507161\\
1.77194429860747	55344.0560797487\\
1.77204430110753	55370.8705045608\\
1.77214430360759	55397.6849293729\\
1.77224430610765	55424.4420584055\\
1.77234430860772	55451.2564832177\\
1.77244431110778	55478.0709080298\\
1.77254431360784	55504.9426286214\\
1.7726443161079	55531.7570534335\\
1.77274431860797	55558.5714782457\\
1.77284432110803	55585.4431988373\\
1.77294432360809	55612.3149194289\\
1.77304432610815	55639.1293442411\\
1.77314432860822	55666.0010648327\\
1.77324433110828	55692.8727854243\\
1.77334433360834	55719.8018017955\\
1.7734443361084	55746.6735223871\\
1.77354433860847	55773.5452429788\\
1.77364434110853	55800.4742593499\\
1.77374434360859	55827.403275721\\
1.77384434610865	55854.2749963127\\
1.77394434860872	55881.2040126838\\
1.77404435110878	55908.133029055\\
1.77414435360884	55935.1193412056\\
1.7742443561089	55962.0483575768\\
1.77434435860897	55988.9773739479\\
1.77444436110903	56015.9636860986\\
1.77454436360909	56042.8927024698\\
1.77464436610915	56069.8790146204\\
1.77474436860922	56096.8653267711\\
1.77484437110928	56123.8516389217\\
1.77494437360934	56150.8379510724\\
1.7750443761094	56177.8815590026\\
1.77514437860947	56204.8678711532\\
1.77524438110953	56231.8541833039\\
1.77534438360959	56258.8977912341\\
1.77544438610965	56285.9413991643\\
1.77554438860972	56312.9850070944\\
1.77564439110978	56340.0286150246\\
1.77574439360984	56367.0722229548\\
1.7758443961099	56394.1158308849\\
1.77594439860997	56421.2167345946\\
1.77604440111003	56448.2603425248\\
1.77614440361009	56475.3612462345\\
1.77624440611015	56502.4048541647\\
1.77634440861022	56529.5057578744\\
1.77644441111028	56556.6066615841\\
1.77654441361034	56583.7075652937\\
1.7766444161104	56610.8657647829\\
1.77674441861047	56637.9666684926\\
1.77684442111053	56665.1248679818\\
1.77694442361059	56692.2257716915\\
1.77704442611065	56719.3839711807\\
1.77714442861072	56746.5421706699\\
1.77724443111078	56773.7003701591\\
1.77734443361084	56800.8585696483\\
1.7774444361109	56828.0167691375\\
1.77754443861097	56855.1749686267\\
1.77764444111103	56882.3904638954\\
1.77774444361109	56909.6059591641\\
1.77784444611115	56936.7641586534\\
1.77794444861122	56963.9796539221\\
1.77804445111128	56991.1951491908\\
1.77814445361134	57018.4106444595\\
1.7782444561114	57045.6261397282\\
1.77834445861147	57072.8989307764\\
1.77844446111153	57100.1144260452\\
1.77854446361159	57127.3872170934\\
1.77864446611165	57154.6027123621\\
1.77874446861172	57181.8755034103\\
1.77884447111178	57209.1482944585\\
1.77894447361184	57236.4210855068\\
1.7790444761119	57263.7511723345\\
1.77914447861197	57291.0239633827\\
1.77924448111203	57318.1248670924\\
1.77934448361209	57345.6268412587\\
1.77944448611215	57373.128815425\\
1.77954448861222	57400.0578317961\\
1.77964449111228	57427.5598059624\\
1.77974449361234	57455.0617801287\\
1.7798444961124	57481.9907964998\\
1.77994449861247	57509.4927706661\\
1.78004450111253	57536.9947448324\\
1.78014450361259	57564.4967189987\\
1.78024450611265	57591.4257353698\\
1.78034450861272	57618.9277095361\\
1.78044451111278	57646.4296837024\\
1.78054451361284	57673.9316578687\\
1.7806445161129	57701.4336320349\\
1.78074451861297	57728.3626484061\\
1.78084452111303	57755.8646225724\\
1.78094452361309	57783.3665967386\\
1.78104452611315	57810.8685709049\\
1.78114452861322	57838.3705450712\\
1.78124453111328	57865.8725192375\\
1.78134453361334	57893.3744934038\\
1.7814445361134	57920.87646757\\
1.78154453861347	57948.3784417363\\
1.78164454111353	57975.8804159026\\
1.78174454361359	58002.8094322738\\
1.78184454611365	58030.31140644\\
1.78194454861372	58057.8133806063\\
1.78204455111378	58085.3153547726\\
1.78214455361384	58112.8173289389\\
1.7822445561139	58140.3193031052\\
1.78234455861397	58167.8212772714\\
1.78244456111403	58195.3232514377\\
1.78254456361409	58223.3981833991\\
1.78264456611415	58250.9001575654\\
1.78274456861422	58278.4021317317\\
1.78284457111428	58305.904105898\\
1.78294457361434	58333.4060800642\\
1.7830445761144	58360.9080542305\\
1.78314457861447	58388.4100283968\\
1.78324458111453	58415.9120025631\\
1.78334458361459	58443.4139767294\\
1.78344458611465	58471.4889086908\\
1.78354458861472	58498.9908828571\\
1.78364459111478	58526.4928570233\\
1.78374459361484	58553.9948311896\\
1.7838445961149	58581.4968053559\\
1.78394459861497	58609.5717373173\\
1.78404460111503	58637.0737114836\\
1.78414460361509	58664.5756856499\\
1.78424460611515	58692.0776598161\\
1.78434460861522	58720.1525917775\\
1.78444461111528	58747.6545659438\\
1.78454461361534	58775.1565401101\\
1.7846446161154	58802.6585142764\\
1.78474461861547	58830.7334462378\\
1.78484462111553	58858.2354204041\\
1.78494462361559	58885.7373945704\\
1.78504462611565	58913.8123265318\\
1.78514462861572	58941.314300698\\
1.78524463111578	58969.3892326595\\
1.78534463361584	58996.8912068257\\
1.7854446361159	59024.393180992\\
1.78554463861597	59052.4681129534\\
1.78564464111603	59079.9700871197\\
1.78574464361609	59108.0450190811\\
1.78584464611615	59135.5469932474\\
1.78594464861622	59163.6219252088\\
1.78604465111628	59191.1238993751\\
1.78614465361634	59219.1988313365\\
1.7862446561164	59246.7008055028\\
1.78634465861647	59274.7757374642\\
1.78644466111653	59302.2777116305\\
1.78654466361659	59330.3526435919\\
1.78664466611665	59357.8546177582\\
1.78674466861672	59385.9295497196\\
1.78684467111678	59413.4315238859\\
1.78694467361684	59441.5064558473\\
1.7870446761169	59469.5813878087\\
1.78714467861697	59497.0833619749\\
1.78724468111703	59525.1582939364\\
1.78734468361709	59553.2332258978\\
1.78744468611715	59580.7352000641\\
1.78754468861722	59608.8101320255\\
1.78764469111728	59636.8850639869\\
1.78774469361734	59664.3870381531\\
1.7878446961174	59692.4619701146\\
1.78794469861747	59720.536902076\\
1.78804470111753	59748.0388762422\\
1.78814470361759	59776.1138082036\\
1.78824470611765	59804.1887401651\\
1.78834470861772	59832.2636721265\\
1.78844471111778	59859.7656462928\\
1.78854471361784	59887.8405782542\\
1.7886447161179	59915.9155102156\\
1.78874471861797	59943.990442177\\
1.78884472111803	59972.0653741384\\
1.78894472361809	60000.1403060998\\
1.78904472611815	60027.6422802661\\
1.78914472861822	60055.7172122275\\
1.78924473111828	60083.7921441889\\
1.78934473361834	60111.8670761503\\
1.7894447361184	60139.9420081117\\
1.78954473861847	60168.0169400731\\
1.78964474111853	60196.0918720345\\
1.78974474361859	60224.166803996\\
1.78984474611865	60252.2417359574\\
1.78994474861872	60280.3166679188\\
1.79004475111878	60308.3915998802\\
1.79014475361884	60336.4665318416\\
1.7902447561189	60364.541463803\\
1.79034475861897	60392.6163957644\\
1.79044476111903	60420.6913277258\\
1.79054476361909	60448.7662596872\\
1.79064476611915	60476.8411916487\\
1.79074476861922	60504.9161236101\\
1.79084477111928	60532.9910555715\\
1.79094477361934	60561.0659875329\\
1.7910447761194	60589.1409194943\\
1.79114477861947	60617.2158514557\\
1.79124478111953	60645.2907834171\\
1.79134478361959	60673.9386731737\\
1.79144478611965	60702.0136051351\\
1.79154478861972	60730.0885370965\\
1.79164479111978	60758.1634690579\\
1.79174479361984	60786.2384010193\\
1.7918447961199	60814.3133329807\\
1.79194479861997	60842.9612227373\\
1.79204480112003	60871.0361546987\\
1.79214480362009	60899.1110866601\\
1.79224480612015	60927.1860186215\\
1.79234480862022	60955.833908378\\
1.79244481112028	60983.9088403394\\
1.79254481362034	61011.9837723008\\
1.7926448161204	61040.0587042622\\
1.79274481862047	61068.7065940188\\
1.79284482112053	61096.7815259802\\
1.79294482362059	61124.8564579416\\
1.79304482612065	61153.5043476982\\
1.79314482862072	61181.5792796596\\
1.79324483112078	61210.2271694161\\
1.79334483362084	61238.3021013775\\
1.7934448361209	61266.3770333389\\
1.79354483862097	61295.0249230955\\
1.79364484112103	61323.0998550569\\
1.79374484362109	61351.7477448134\\
1.79384484612115	61379.8226767748\\
1.79394484862122	61408.4705665314\\
1.79404485112128	61436.5454984928\\
1.79414485362134	61464.6204304542\\
1.7942448561214	61493.2683202107\\
1.79434485862147	61521.9162099673\\
1.79444486112153	61549.9911419287\\
1.79454486362159	61578.6390316852\\
1.79464486612165	61606.7139636466\\
1.79474486862172	61635.3618534032\\
1.79484487112178	61663.4367853646\\
1.79494487362184	61692.0846751211\\
1.7950448761219	61720.1596070825\\
1.79514487862197	61748.8074968391\\
1.79524488112203	61777.4553865956\\
1.79534488362209	61805.530318557\\
1.79544488612215	61834.1782083136\\
1.79554488862222	61862.8260980701\\
1.79564489112228	61890.9010300315\\
1.79574489362234	61919.5489197881\\
1.7958448961224	61948.1968095446\\
1.79594489862247	61976.271741506\\
1.79604490112253	62004.9196312626\\
1.79614490362259	62033.5675210191\\
1.79624490612265	62062.2154107756\\
1.79634490862272	62090.2903427371\\
1.79644491112278	62118.9382324936\\
1.79654491362284	62147.5861222501\\
1.7966449161229	62176.2340120067\\
1.79674491862297	62204.8819017632\\
1.79684492112303	62232.9568337246\\
1.79694492362309	62261.6047234812\\
1.79704492612315	62290.2526132377\\
1.79714492862322	62318.9005029942\\
1.79724493112328	62347.5483927508\\
1.79734493362334	62376.1962825073\\
1.7974449361234	62404.8441722639\\
1.79754493862347	62432.9191042253\\
1.79764494112353	62461.5669939818\\
1.79774494362359	62490.2148837384\\
1.79784494612365	62518.8627734949\\
1.79794494862372	62547.5106632515\\
1.79804495112378	62576.158553008\\
1.79814495362384	62604.8064427645\\
1.7982449561239	62633.4543325211\\
1.79834495862397	62662.1022222776\\
1.79844496112403	62690.7501120342\\
1.79854496362409	62719.3980017907\\
1.79864496612415	62748.0458915472\\
1.79874496862422	62776.6937813038\\
1.79884497112428	62805.3416710603\\
1.79894497362434	62833.9895608169\\
1.7990449761244	62863.2104083685\\
1.79914497862447	62891.8582981251\\
1.79924498112453	62920.5061878816\\
1.79934498362459	62949.1540776382\\
1.79944498612465	62977.8019673947\\
1.79954498862472	63006.4498571512\\
1.79964499112478	63035.0977469078\\
1.79974499362484	63064.3185944595\\
1.7998449961249	63092.966484216\\
1.79994499862497	63121.6143739725\\
1.80004500112503	63150.2622637291\\
1.80014500362509	63178.9101534856\\
1.80024500612515	63208.1310010373\\
1.80034500862522	63236.7788907938\\
1.80044501112528	63265.4267805504\\
1.80054501362534	63294.0746703069\\
1.8006450161254	63323.2955178586\\
1.80074501862547	63351.9434076151\\
1.80084502112553	63380.5912973717\\
1.80094502362559	63409.8121449233\\
1.80104502612565	63438.4600346799\\
1.80114502862572	63467.1079244364\\
1.80124503112578	63496.3287719881\\
1.80134503362584	63524.9766617446\\
1.8014450361259	63554.1975092963\\
1.80154503862597	63582.8453990528\\
1.80164504112603	63611.4932888094\\
1.80174504362609	63640.7141363611\\
1.80184504612615	63669.3620261176\\
1.80194504862622	63698.5828736693\\
1.80204505112628	63727.2307634258\\
1.80214505362634	63756.4516109775\\
1.8022450561264	63785.099500734\\
1.80234505862647	63814.3203482857\\
1.80244506112653	63842.9682380422\\
1.80254506362659	63872.1890855939\\
1.80264506612665	63900.8369753505\\
1.80274506862672	63930.0578229021\\
1.80284507112678	63958.7057126587\\
1.80294507362684	63987.9265602103\\
1.8030450761269	64017.147407762\\
1.80314507862697	64045.7952975185\\
1.80324508112703	64075.0161450702\\
1.80334508362709	64103.6640348268\\
1.80344508612715	64132.8848823784\\
1.80354508862722	64162.1057299301\\
1.80364509112728	64190.7536196866\\
1.80374509362734	64219.9744672383\\
1.8038450961274	64249.19531479\\
1.80394509862747	64277.8432045465\\
1.80404510112753	64307.0640520982\\
1.80414510362759	64336.2848996499\\
1.80424510612765	64365.5057472016\\
1.80434510862772	64394.1536369581\\
1.80444511112778	64423.3744845098\\
1.80454511362784	64452.5953320614\\
1.8046451161279	64481.8161796131\\
1.80474511862797	64510.4640693696\\
1.80484512112803	64539.6849169213\\
1.80494512362809	64568.905764473\\
1.80504512612815	64598.1266120247\\
1.80514512862822	64627.3474595763\\
1.80524513112828	64656.568307128\\
1.80534513362834	64685.7891546797\\
1.8054451361284	64714.4370444362\\
1.80554513862847	64743.6578919879\\
1.80564514112853	64772.8787395396\\
1.80574514362859	64802.0995870912\\
1.80584514612865	64831.3204346429\\
1.80594514862872	64860.5412821946\\
1.80604515112878	64889.7621297462\\
1.80614515362884	64918.9829772979\\
1.8062451561289	64948.2038248496\\
1.80634515862897	64977.4246724013\\
1.80644516112903	65006.6455199529\\
1.80654516362909	65035.8663675046\\
1.80664516612915	65065.0872150563\\
1.80674516862922	65094.308062608\\
1.80684517112928	65123.5289101596\\
1.80694517362934	65152.7497577113\\
1.8070451761294	65181.970605263\\
1.80714517862947	65211.1914528147\\
1.80724518112953	65240.9852581615\\
1.80734518362959	65270.2061057131\\
1.80744518612965	65299.4269532648\\
1.80754518862972	65328.6478008165\\
1.80764519112978	65357.8686483681\\
1.80774519362984	65387.0894959198\\
1.8078451961299	65416.8833012666\\
1.80794519862997	65446.1041488183\\
1.80804520113003	65475.3249963699\\
1.80814520363009	65504.5458439216\\
1.80824520613015	65533.7666914733\\
1.80834520863022	65563.5604968201\\
1.80844521113028	65592.7813443718\\
1.80854521363034	65622.0021919234\\
1.8086452161304	65651.7959972702\\
1.80874521863047	65681.0168448219\\
1.80884522113053	65710.2376923736\\
1.80894522363059	65739.4585399253\\
1.80904522613065	65769.2523452721\\
1.80914522863072	65798.4731928237\\
1.80924523113078	65827.6940403754\\
1.80934523363084	65857.4878457222\\
1.8094452361309	65886.7086932739\\
1.80954523863097	65916.5024986207\\
1.80964524113103	65945.7233461724\\
1.80974524363109	65974.944193724\\
1.80984524613115	66004.7379990708\\
1.80994524863122	66033.9588466225\\
1.81004525113128	66063.7526519693\\
1.81014525363134	66092.973499521\\
1.8102452561314	66122.7673048678\\
1.81034525863147	66151.9881524194\\
1.81044526113153	66181.7819577663\\
1.81054526363159	66211.0028053179\\
1.81064526613165	66240.7966106647\\
1.81074526863172	66270.0174582164\\
1.81084527113178	66299.8112635632\\
1.81094527363184	66329.60506891\\
1.8110452761319	66358.8259164617\\
1.81114527863197	66388.6197218085\\
1.81124528113203	66417.8405693602\\
1.81134528363209	66447.634374707\\
1.81144528613215	66477.4281800538\\
1.81154528863222	66506.6490276054\\
1.81164529113228	66536.4428329522\\
1.81174529363234	66566.236638299\\
1.8118452961324	66595.4574858507\\
1.81194529863247	66625.2512911975\\
1.81204530113253	66655.0450965443\\
1.81214530363259	66684.265944096\\
1.81224530613265	66714.0597494428\\
1.81234530863272	66743.8535547896\\
1.81244531113278	66773.6473601364\\
1.81254531363284	66802.8682076881\\
1.8126453161329	66832.6620130349\\
1.81274531863297	66862.4558183817\\
1.81284532113303	66892.2496237285\\
1.81294532363309	66922.0434290753\\
1.81304532613315	66951.264276627\\
1.81314532863322	66981.0580819738\\
1.81324533113328	67010.8518873206\\
1.81334533363334	67040.6456926674\\
1.8134453361334	67070.4394980142\\
1.81354533863347	67100.233303361\\
1.81364534113353	67130.0271087078\\
1.81374534363359	67159.8209140546\\
1.81384534613365	67189.6147194014\\
1.81394534863372	67219.4085247482\\
1.81404535113378	67249.202330095\\
1.81414535363384	67278.4231776466\\
1.8142453561339	67308.2169829935\\
1.81434535863397	67338.0107883403\\
1.81444536113403	67367.8045936871\\
1.81454536363409	67398.171356829\\
1.81464536613415	67427.9651621758\\
1.81474536863422	67457.7589675226\\
1.81484537113428	67487.5527728694\\
1.81494537363434	67517.3465782162\\
1.8150453761344	67547.140383563\\
1.81514537863447	67576.9341889098\\
1.81524538113453	67606.7279942566\\
1.81534538363459	67636.5217996034\\
1.81544538613465	67666.3156049502\\
1.81554538863472	67696.109410297\\
1.81564539113478	67726.476173439\\
1.81574539363484	67756.2699787858\\
1.8158453961349	67786.0637841326\\
1.81594539863497	67815.8575894794\\
1.81604540113503	67845.6513948262\\
1.81614540363509	67876.0181579681\\
1.81624540613515	67905.8119633149\\
1.81634540863522	67935.6057686617\\
1.81644541113528	67965.3995740085\\
1.81654541363534	67995.7663371504\\
1.8166454161354	68025.5601424973\\
1.81674541863547	68055.3539478441\\
1.81684542113553	68085.1477531908\\
1.81694542363559	68115.5145163328\\
1.81704542613565	68145.3083216796\\
1.81714542863572	68175.1021270264\\
1.81724543113578	68205.4688901683\\
1.81734543363584	68235.2626955151\\
1.8174454361359	68265.6294586571\\
1.81754543863597	68295.4232640039\\
1.81764544113603	68325.2170693507\\
1.81774544363609	68355.5838324926\\
1.81784544613615	68385.3776378394\\
1.81794544863622	68415.7444009813\\
1.81804545113628	68445.5382063281\\
1.81814545363634	68475.9049694701\\
1.8182454561364	68505.6987748169\\
1.81834545863647	68536.0655379588\\
1.81844546113653	68565.8593433056\\
1.81854546363659	68596.2261064475\\
1.81864546613665	68626.0199117944\\
1.81874546863672	68656.3866749363\\
1.81884547113678	68686.1804802831\\
1.81894547363684	68716.547243425\\
1.8190454761369	68746.3410487718\\
1.81914547863697	68776.7078119138\\
1.81924548113703	68807.0745750557\\
1.81934548363709	68836.8683804025\\
1.81944548613715	68867.2351435444\\
1.81954548863722	68897.6019066864\\
1.81964549113728	68927.3957120332\\
1.81974549363734	68957.7624751751\\
1.8198454961374	68988.129238317\\
1.81994549863747	69017.9230436638\\
1.82004550113753	69048.2898068058\\
1.82014550363759	69078.6565699477\\
1.82024550613765	69108.4503752945\\
1.82034550863772	69138.8171384364\\
1.82044551113778	69169.1839015784\\
1.82054551363784	69199.5506647203\\
1.8206455161379	69229.3444700671\\
1.82074551863797	69259.711233209\\
1.82084552113803	69290.077996351\\
1.82094552363809	69320.4447594929\\
1.82104552613815	69350.8115226349\\
1.82114552863822	69381.1782857768\\
1.82124553113828	69410.9720911236\\
1.82134553363834	69441.3388542655\\
1.8214455361384	69471.7056174074\\
1.82154553863847	69502.0723805494\\
1.82164554113853	69532.4391436913\\
1.82174554363859	69562.8059068332\\
1.82184554613865	69593.1726699752\\
1.82194554863872	69623.5394331171\\
1.82204555113878	69653.906196259\\
1.82214555363884	69684.272959401\\
1.8222455561389	69714.6397225429\\
1.82234555863897	69745.0064856848\\
1.82244556113903	69775.3732488268\\
1.82254556363909	69805.7400119687\\
1.82264556613915	69836.1067751106\\
1.82274556863922	69866.4735382526\\
1.82284557113928	69896.8403013945\\
1.82294557363934	69927.2070645365\\
1.8230455761394	69957.5738276784\\
1.82314557863947	69987.9405908203\\
1.82324558113953	70018.3073539623\\
1.82334558363959	70048.6741171042\\
1.82344558613965	70079.0408802461\\
1.82354558863972	70109.9806011832\\
1.82364559113978	70140.3473643251\\
1.82374559363984	70170.7141274671\\
1.8238455961399	70201.080890609\\
1.82394559863997	70231.4476537509\\
1.82404560114003	70261.8144168929\\
1.82414560364009	70292.7541378299\\
1.82424560614015	70323.1209009718\\
1.82434560864022	70353.4876641138\\
1.82444561114028	70383.8544272557\\
1.82454561364034	70414.7941481928\\
1.8246456161404	70445.1609113347\\
1.82474561864047	70475.5276744766\\
1.82484562114053	70506.4673954137\\
1.82494562364059	70536.8341585556\\
1.82504562614065	70567.2009216976\\
1.82514562864072	70598.1406426347\\
1.82524563114078	70628.5074057766\\
1.82534563364084	70658.8741689185\\
1.8254456361409	70689.8138898556\\
1.82554563864097	70720.1806529975\\
1.82564564114103	70750.5474161394\\
1.82574564364109	70781.4871370765\\
1.82584564614115	70811.8539002185\\
1.82594564864122	70842.7936211555\\
1.82604565114128	70873.1603842974\\
1.82614565364134	70904.1001052345\\
1.8262456561414	70934.4668683764\\
1.82634565864147	70965.4065893135\\
1.82644566114153	70995.7733524554\\
1.82654566364159	71026.7130733925\\
1.82664566614165	71057.0798365344\\
1.82674566864172	71088.0195574715\\
1.82684567114178	71118.3863206134\\
1.82694567364184	71149.3260415505\\
1.8270456761419	71179.6928046924\\
1.82714567864197	71210.6325256295\\
1.82724568114203	71241.5722465666\\
1.82734568364209	71271.9390097085\\
1.82744568614215	71302.8787306456\\
1.82754568864222	71333.8184515826\\
1.82764569114228	71364.1852147246\\
1.82774569364234	71395.1249356616\\
1.8278456961424	71426.0646565987\\
1.82794569864247	71456.4314197406\\
1.82804570114253	71487.3711406777\\
1.82814570364259	71518.3108616148\\
1.82824570614265	71548.6776247567\\
1.82834570864272	71579.6173456937\\
1.82844571114278	71610.5570666308\\
1.82854571364284	71641.4967875679\\
1.8286457161429	71671.8635507098\\
1.82874571864297	71702.8032716469\\
1.82884572114303	71733.7429925839\\
1.82894572364309	71764.682713521\\
1.82904572614315	71795.6224344581\\
1.82914572864322	71825.9891976\\
1.82924573114328	71856.9289185371\\
1.82934573364334	71887.8686394741\\
1.8294457361434	71918.8083604112\\
1.82954573864347	71949.7480813483\\
1.82964574114353	71980.6878022853\\
1.82974574364359	72011.6275232224\\
1.82984574614365	72042.5672441595\\
1.82994574864372	72073.5069650965\\
1.83004575114378	72104.4466860336\\
1.83014575364384	72135.3864069706\\
1.8302457561439	72166.3261279077\\
1.83034575864397	72197.2658488448\\
1.83044576114403	72228.2055697818\\
1.83054576364409	72259.1452907189\\
1.83064576614415	72290.085011656\\
1.83074576864422	72321.024732593\\
1.83084577114428	72351.9644535301\\
1.83094577364434	72382.9041744672\\
1.8310457761444	72413.8438954042\\
1.83114577864447	72444.7836163413\\
1.83124578114453	72475.7233372784\\
1.83134578364459	72506.6630582154\\
1.83144578614465	72537.6027791525\\
1.83154578864472	72568.5425000895\\
1.83164579114478	72600.0551788217\\
1.83174579364484	72630.9948997588\\
1.8318457961449	72661.9346206959\\
1.83194579864497	72692.8743416329\\
1.83204580114503	72723.81406257\\
1.83214580364509	72755.3267413022\\
1.83224580614515	72786.2664622393\\
1.83234580864522	72817.2061831763\\
1.83244581114528	72848.1459041134\\
1.83254581364534	72879.6585828456\\
1.8326458161454	72910.5983037827\\
1.83274581864547	72941.5380247197\\
1.83284582114553	72972.4777456568\\
1.83294582364559	73003.990424389\\
1.83304582614565	73034.930145326\\
1.83314582864572	73065.8698662631\\
1.83324583114578	73097.3825449953\\
1.83334583364584	73128.3222659324\\
1.8334458361459	73159.8349446646\\
1.83354583864597	73190.7746656016\\
1.83364584114603	73221.7143865387\\
1.83374584364609	73253.2270652709\\
1.83384584614615	73284.1667862079\\
1.83394584864622	73315.6794649401\\
1.83404585114628	73346.6191858772\\
1.83414585364634	73378.1318646094\\
1.8342458561464	73409.0715855465\\
1.83434585864647	73440.0113064835\\
1.83444586114653	73471.5239852157\\
1.83454586364659	73502.4637061528\\
1.83464586614665	73533.976384885\\
1.83474586864672	73565.4890636172\\
1.83484587114678	73596.4287845542\\
1.83494587364684	73627.9414632864\\
1.8350458761469	73658.8811842235\\
1.83514587864697	73690.3938629557\\
1.83524588114703	73721.3335838928\\
1.83534588364709	73752.846262625\\
1.83544588614715	73784.3589413571\\
1.83554588864722	73815.2986622942\\
1.83564589114728	73846.8113410264\\
1.83574589364734	73878.3240197586\\
1.8358458961474	73909.2637406957\\
1.83594589864747	73940.7764194279\\
1.83604590114753	73972.2890981601\\
1.83614590364759	74003.2288190971\\
1.83624590614765	74034.7414978293\\
1.83634590864772	74066.2541765615\\
1.83644591114778	74097.7668552937\\
1.83654591364784	74128.7065762308\\
1.8366459161479	74160.219254963\\
1.83674591864797	74191.7319336952\\
1.83684592114803	74223.2446124274\\
1.83694592364809	74254.7572911596\\
1.83704592614815	74285.6970120966\\
1.83714592864822	74317.2096908288\\
1.83724593114828	74348.722369561\\
1.83734593364834	74380.2350482932\\
1.8374459361484	74411.7477270254\\
1.83754593864847	74443.2604057576\\
1.83764594114853	74474.7730844898\\
1.83774594364859	74506.285763222\\
1.83784594614865	74537.7984419542\\
1.83794594864872	74568.7381628913\\
1.83804595114878	74600.2508416234\\
1.83814595364884	74631.7635203556\\
1.8382459561489	74663.2761990878\\
1.83834595864897	74694.78887782\\
1.83844596114903	74726.3015565522\\
1.83854596364909	74757.8142352844\\
1.83864596614915	74789.3269140166\\
1.83874596864922	74820.8395927488\\
1.83884597114928	74852.352271481\\
1.83894597364934	74884.4379080083\\
1.8390459761494	74915.9505867405\\
1.83914597864947	74947.4632654727\\
1.83924598114953	74978.9759442049\\
1.83934598364959	75010.4886229371\\
1.83944598614965	75042.0013016693\\
1.83954598864972	75073.5139804015\\
1.83964599114978	75105.0266591337\\
1.83974599364984	75137.112295661\\
1.8398459961499	75168.6249743932\\
1.83994599864997	75200.1376531254\\
1.84004600115003	75231.6503318576\\
1.84014600365009	75263.1630105898\\
1.84024600615015	75295.2486471171\\
1.84034600865022	75326.7613258493\\
1.84044601115028	75358.2740045815\\
1.84054601365034	75389.7866833137\\
1.8406460161504	75421.872319841\\
1.84074601865047	75453.3849985733\\
1.84084602115053	75484.8976773054\\
1.84094602365059	75516.4103560376\\
1.84104602615065	75548.4959925649\\
1.84114602865072	75580.0086712972\\
1.84124603115078	75611.5213500293\\
1.84134603365084	75643.6069865567\\
1.8414460361509	75675.1196652889\\
1.84154603865097	75707.2053018162\\
1.84164604115103	75738.7179805484\\
1.84174604365109	75770.2306592806\\
1.84184604615115	75802.3162958079\\
1.84194604865122	75833.8289745401\\
1.84204605115128	75865.9146110674\\
1.84214605365134	75897.4272897996\\
1.8422460561514	75929.512926327\\
1.84234605865147	75961.0256050592\\
1.84244606115153	75993.1112415865\\
1.84254606365159	76024.6239203187\\
1.84264606615165	76056.709556846\\
1.84274606865172	76088.2222355782\\
1.84284607115178	76120.3078721055\\
1.84294607365184	76151.8205508377\\
1.8430460761519	76183.906187365\\
1.84314607865197	76215.9918238924\\
1.84324608115203	76247.5045026246\\
1.84334608365209	76279.5901391519\\
1.84344608615215	76311.1028178841\\
1.84354608865222	76343.1884544114\\
1.84364609115228	76375.2740909387\\
1.84374609365234	76406.7867696709\\
1.8438460961524	76438.8724061983\\
1.84394609865247	76470.9580427256\\
1.84404610115253	76502.4707214578\\
1.84414610365259	76534.5563579851\\
1.84424610615265	76566.6419945124\\
1.84434610865272	76598.7276310398\\
1.84444611115278	76630.240309772\\
1.84454611365284	76662.3259462993\\
1.8446461161529	76694.4115828266\\
1.84474611865297	76726.4972193539\\
1.84484612115303	76758.0098980861\\
1.84494612365309	76790.0955346135\\
1.84504612615315	76822.1811711408\\
1.84514612865322	76854.2668076681\\
1.84524613115328	76886.3524441954\\
1.84534613365334	76918.4380807228\\
1.8454461361534	76950.5237172501\\
1.84554613865347	76982.0363959823\\
1.84564614115353	77014.1220325096\\
1.84574614365359	77046.2076690369\\
1.84584614615365	77078.2933055643\\
1.84594614865372	77110.3789420916\\
1.84604615115378	77142.4645786189\\
1.84614615365384	77174.5502151462\\
1.8462461561539	77206.6358516736\\
1.84634615865397	77238.7214882009\\
1.84644616115403	77270.8071247282\\
1.84654616365409	77302.8927612555\\
1.84664616615415	77334.9783977829\\
1.84674616865422	77367.0640343102\\
1.84684617115428	77399.1496708375\\
1.84694617365434	77431.2353073649\\
1.8470461761544	77463.3209438922\\
1.84714617865447	77495.4065804195\\
1.84724618115453	77527.4922169468\\
1.84734618365459	77560.1508112693\\
1.84744618615465	77592.2364477966\\
1.84754618865472	77624.3220843239\\
1.84764619115478	77656.4077208513\\
1.84774619365484	77688.4933573786\\
1.8478461961549	77720.5789939059\\
1.84794619865497	77753.2375882284\\
1.84804620115503	77785.3232247557\\
1.84814620365509	77817.408861283\\
1.84824620615515	77849.4944978103\\
1.84834620865522	77881.5801343377\\
1.84844621115528	77914.2387286601\\
1.84854621365534	77946.3243651875\\
1.8486462161554	77978.4100017148\\
1.84874621865547	78010.4956382421\\
1.84884622115553	78043.1542325646\\
1.84894622365559	78075.2398690919\\
1.84904622615565	78107.3255056192\\
1.84914622865572	78139.9840999417\\
1.84924623115578	78172.069736469\\
1.84934623365584	78204.1553729963\\
1.8494462361559	78236.8139673188\\
1.84954623865597	78268.8996038461\\
1.84964624115603	78301.5581981686\\
1.84974624365609	78333.6438346959\\
1.84984624615615	78365.7294712232\\
1.84994624865622	78398.3880655457\\
1.85004625115628	78430.473702073\\
1.85014625365634	78463.1322963955\\
1.8502462561564	78495.2179329228\\
1.85034625865647	78527.8765272452\\
1.85044626115653	78559.9621637726\\
1.85054626365659	78592.620758095\\
1.85064626615665	78624.7063946223\\
1.85074626865672	78657.3649889448\\
1.85084627115678	78689.4506254721\\
1.85094627365684	78722.1092197946\\
1.8510462761569	78754.767814117\\
1.85114627865697	78786.8534506444\\
1.85124628115703	78819.5120449668\\
1.85134628365709	78851.5976814942\\
1.85144628615715	78884.2562758166\\
1.85154628865722	78916.9148701391\\
1.85164629115728	78949.0005066664\\
1.85174629365734	78981.6591009889\\
1.8518462961574	79014.3176953113\\
1.85194629865747	79046.4033318386\\
1.85204630115753	79079.0619261611\\
1.85214630365759	79111.7205204835\\
1.85224630615765	79143.8061570109\\
1.85234630865772	79176.4647513333\\
1.85244631115778	79209.1233456558\\
1.85254631365784	79241.7819399782\\
1.8526463161579	79273.8675765056\\
1.85274631865797	79306.526170828\\
1.85284632115803	79339.1847651505\\
1.85294632365809	79371.8433594729\\
1.85304632615815	79404.5019537954\\
1.85314632865822	79437.1605481179\\
1.85324633115828	79469.2461846452\\
1.85334633365834	79501.9047789676\\
1.8534463361584	79534.5633732901\\
1.85354633865847	79567.2219676126\\
1.85364634115853	79599.880561935\\
1.85374634365859	79632.5391562575\\
1.85384634615865	79665.1977505799\\
1.85394634865872	79697.8563449024\\
1.85404635115878	79730.5149392248\\
1.85414635365884	79763.1735335473\\
1.8542463561589	79795.8321278698\\
1.85434635865897	79828.4907221922\\
1.85444636115903	79861.1493165147\\
1.85454636365909	79893.8079108371\\
1.85464636615915	79926.4665051596\\
1.85474636865922	79959.125099482\\
1.85484637115928	79991.7836938045\\
1.85494637365934	80024.442288127\\
1.8550463761594	80057.1008824494\\
1.85514637865947	80089.7594767719\\
1.85524638115953	80122.4180710943\\
1.85534638365959	80155.0766654168\\
1.85544638615965	80187.7352597392\\
1.85554638865972	80220.3938540617\\
1.85564639115978	80253.6254061793\\
1.85574639365984	80286.2840005017\\
1.8558463961599	80318.9425948242\\
1.85594639865997	80351.6011891467\\
1.85604640116003	80384.2597834691\\
1.85614640366009	80416.9183777916\\
1.85624640616015	80450.1499299091\\
1.85634640866022	80482.8085242316\\
1.85644641116028	80515.4671185541\\
1.85654641366034	80548.1257128765\\
1.8566464161604	80581.3572649941\\
1.85674641866047	80614.0158593166\\
1.85684642116053	80646.674453639\\
1.85694642366059	80679.9060057566\\
1.85704642616065	80712.5646000791\\
1.85714642866072	80745.2231944015\\
1.85724643116078	80778.4547465191\\
1.85734643366084	80811.1133408416\\
1.8574464361609	80843.771935164\\
1.85754643866097	80877.0034872816\\
1.85764644116103	80909.6620816041\\
1.85774644366109	80942.8936337217\\
1.85784644616115	80975.5522280441\\
1.85794644866122	81008.2108223666\\
1.85804645116128	81041.4423744842\\
1.85814645366134	81074.1009688066\\
1.8582464561614	81107.3325209242\\
1.85834645866147	81139.9911152467\\
1.85844646116153	81173.2226673643\\
1.85854646366159	81205.8812616867\\
1.85864646616165	81239.1128138043\\
1.85874646866172	81271.7714081268\\
1.85884647116178	81305.0029602443\\
1.85894647366184	81337.6615545668\\
1.8590464761619	81370.8931066844\\
1.85914647866197	81404.124658802\\
1.85924648116203	81436.7832531244\\
1.85934648366209	81470.014805242\\
1.85944648616215	81502.6733995645\\
1.85954648866222	81535.9049516821\\
1.85964649116228	81569.1365037997\\
1.85974649366234	81601.7950981221\\
1.8598464961624	81635.0266502397\\
1.85994649866247	81668.2582023573\\
1.86004650116253	81700.9167966797\\
1.86014650366259	81734.1483487973\\
1.86024650616265	81767.3799009149\\
1.86034650866272	81800.6114530325\\
1.86044651116278	81833.2700473549\\
1.86054651366284	81866.5015994725\\
1.8606465161629	81899.7331515901\\
1.86074651866297	81932.9647037077\\
1.86084652116303	81965.6232980302\\
1.86094652366309	81998.8548501478\\
1.86104652616315	82032.0864022654\\
1.86114652866322	82065.3179543829\\
1.86124653116328	82098.5495065005\\
1.86134653366334	82131.7810586181\\
1.8614465361634	82164.4396529406\\
1.86154653866347	82197.6712050582\\
1.86164654116353	82230.9027571757\\
1.86174654366359	82264.1343092933\\
1.86184654616365	82297.3658614109\\
1.86194654866372	82330.5974135285\\
1.86204655116378	82363.8289656461\\
1.86214655366384	82397.0605177637\\
1.8622465561639	82430.2920698813\\
1.86234655866397	82463.5236219989\\
1.86244656116403	82496.7551741164\\
1.86254656366409	82529.986726234\\
1.86264656616415	82563.2182783516\\
1.86274656866422	82596.4498304692\\
1.86284657116428	82629.6813825868\\
1.86294657366434	82662.9129347044\\
1.8630465761644	82696.144486822\\
1.86314657866447	82729.3760389396\\
1.86324658116453	82762.6075910572\\
1.86334658366459	82795.8391431747\\
1.86344658616465	82829.6436530875\\
1.86354658866472	82862.875205205\\
1.86364659116478	82896.1067573226\\
1.86374659366484	82929.3383094402\\
1.8638465961649	82962.5698615578\\
1.86394659866497	82995.8014136754\\
1.86404660116503	83029.6059235881\\
1.86414660366509	83062.8374757057\\
1.86424660616515	83096.0690278233\\
1.86434660866522	83129.3005799409\\
1.86444661116528	83162.5321320585\\
1.86454661366534	83196.3366419712\\
1.8646466161654	83229.5681940888\\
1.86474661866547	83262.7997462064\\
1.86484662116553	83296.6042561191\\
1.86494662366559	83329.8358082367\\
1.86504662616565	83363.0673603543\\
1.86514662866572	83396.871870267\\
1.86524663116578	83430.1034223846\\
1.86534663366584	83463.3349745021\\
1.8654466361659	83497.1394844149\\
1.86554663866597	83530.3710365325\\
1.86564664116603	83563.60258865\\
1.86574664366609	83597.4070985628\\
1.86584664616615	83630.6386506804\\
1.86594664866622	83664.4431605931\\
1.86604665116628	83697.6747127107\\
1.86614665366634	83731.4792226234\\
1.8662466561664	83764.710774741\\
1.86634665866647	83797.9423268586\\
1.86644666116653	83831.7468367713\\
1.86654666366659	83864.9783888888\\
1.86664666616665	83898.7828988016\\
1.86674666866672	83932.5874087143\\
1.86684667116678	83965.8189608319\\
1.86694667366684	83999.6234707446\\
1.8670466761669	84032.8550228622\\
1.86714667866697	84066.6595327749\\
1.86724668116703	84099.8910848925\\
1.86734668366709	84133.6955948052\\
1.86744668616715	84167.5001047179\\
1.86754668866722	84200.7316568355\\
1.86764669116728	84234.5361667483\\
1.86774669366734	84268.340676661\\
1.8678466961674	84301.5722287785\\
1.86794669866747	84335.3767386913\\
1.86804670116753	84369.181248604\\
1.86814670366759	84402.4128007216\\
1.86824670616765	84436.2173106343\\
1.86834670866772	84470.021820547\\
1.86844671116778	84503.2533726646\\
1.86854671366784	84537.0578825773\\
1.8686467161679	84570.86239249\\
1.86874671866797	84604.6669024028\\
1.86884672116803	84638.4714123155\\
1.86894672366809	84671.7029644331\\
1.86904672616815	84705.5074743458\\
1.86914672866822	84739.3119842585\\
1.86924673116828	84773.1164941712\\
1.86934673366834	84806.9210040839\\
1.8694467361684	84840.7255139966\\
1.86954673866847	84873.9570661142\\
1.86964674116853	84907.761576027\\
1.86974674366859	84941.5660859397\\
1.86984674616865	84975.3705958524\\
1.86994674866872	85009.1751057651\\
1.87004675116878	85042.9796156778\\
1.87014675366884	85076.7841255905\\
1.8702467561689	85110.5886355033\\
1.87034675866897	85144.393145416\\
1.87044676116903	85178.1976553287\\
1.87054676366909	85212.0021652414\\
1.87064676616915	85245.8066751541\\
1.87074676866922	85279.6111850669\\
1.87084677116928	85313.4156949796\\
1.87094677366934	85347.2202048923\\
1.8710467761694	85381.024714805\\
1.87114677866947	85414.8292247177\\
1.87124678116953	85448.6337346304\\
1.87134678366959	85482.4382445432\\
1.87144678616965	85516.815712251\\
1.87154678866972	85550.6202221638\\
1.87164679116978	85584.4247320765\\
1.87174679366984	85618.2292419892\\
1.8718467961699	85652.0337519019\\
1.87194679866997	85685.8382618146\\
1.87204680117003	85720.2157295225\\
1.87214680367009	85754.0202394352\\
1.87224680617015	85787.8247493479\\
1.87234680867022	85821.6292592606\\
1.87244681117028	85855.4337691733\\
1.87254681367034	85889.8112368812\\
1.8726468161704	85923.6157467939\\
1.87274681867047	85957.4202567066\\
1.87284682117053	85991.7977244145\\
1.87294682367059	86025.6022343272\\
1.87304682617065	86059.4067442399\\
1.87314682867072	86093.2112541526\\
1.87324683117078	86127.5887218605\\
1.87334683367084	86161.3932317732\\
1.8734468361709	86195.770699481\\
1.87354683867097	86229.5752093938\\
1.87364684117103	86263.3797193065\\
1.87374684367109	86297.7571870143\\
1.87384684617115	86331.561696927\\
1.87394684867122	86365.9391646349\\
1.87404685117128	86399.7436745476\\
1.87414685367134	86433.5481844603\\
1.8742468561714	86467.9256521682\\
1.87434685867147	86501.7301620809\\
1.87444686117153	86536.1076297888\\
1.87454686367159	86569.9121397015\\
1.87464686617165	86604.2896074093\\
1.87474686867172	86638.094117322\\
1.87484687117178	86672.4715850299\\
1.87494687367184	86706.2760949426\\
1.8750468761719	86740.6535626505\\
1.87514687867197	86775.0310303583\\
1.87524688117203	86808.835540271\\
1.87534688367209	86843.2130079789\\
1.87544688617215	86877.0175178916\\
1.87554688867222	86911.3949855994\\
1.87564689117228	86945.7724533073\\
1.87574689367234	86979.57696322\\
1.8758468961724	87013.9544309279\\
1.87594689867247	87048.3318986357\\
1.87604690117253	87082.1364085484\\
1.87614690367259	87116.5138762563\\
1.87624690617265	87150.8913439641\\
1.87634690867272	87185.268811672\\
1.87644691117278	87219.0733215847\\
1.87654691367284	87253.4507892925\\
1.8766469161729	87287.8282570004\\
1.87674691867297	87322.2057247082\\
1.87684692117303	87356.010234621\\
1.87694692367309	87390.3877023288\\
1.87704692617315	87424.7651700367\\
1.87714692867322	87459.1426377445\\
1.87724693117328	87493.5201054524\\
1.87734693367334	87527.3246153651\\
1.8774469361734	87561.7020830729\\
1.87754693867347	87596.0795507808\\
1.87764694117353	87630.4570184886\\
1.87774694367359	87664.8344861965\\
1.87784694617365	87699.2119539043\\
1.87794694867372	87733.5894216122\\
1.87804695117378	87767.96688932\\
1.87814695367384	87802.3443570279\\
1.8782469561739	87836.7218247357\\
1.87834695867397	87871.0992924436\\
1.87844696117403	87905.4767601514\\
1.87854696367409	87939.8542278593\\
1.87864696617415	87974.2316955671\\
1.87874696867422	88008.609163275\\
1.87884697117428	88042.9866309828\\
1.87894697367434	88077.3640986907\\
1.8790469761744	88111.7415663985\\
1.87914697867447	88146.1190341064\\
1.87924698117453	88180.4965018142\\
1.87934698367459	88214.8739695221\\
1.87944698617465	88249.2514372299\\
1.87954698867472	88283.6289049378\\
1.87964699117478	88318.0063726456\\
1.87974699367484	88352.9567981486\\
1.8798469961749	88387.3342658565\\
1.87994699867497	88421.7117335643\\
1.88004700117503	88456.0892012721\\
1.88014700367509	88490.46666898\\
1.88024700617515	88524.8441366878\\
1.88034700867522	88559.7945621908\\
1.88044701117528	88594.1720298987\\
1.88054701367534	88628.5494976065\\
1.8806470161754	88662.9269653144\\
1.88074701867547	88697.8773908173\\
1.88084702117553	88732.2548585252\\
1.88094702367559	88766.632326233\\
1.88104702617565	88801.582751736\\
1.88114702867572	88835.9602194439\\
1.88124703117578	88870.3376871517\\
1.88134703367584	88905.2881126547\\
1.8814470361759	88939.6655803626\\
1.88154703867597	88974.0430480704\\
1.88164704117603	89008.9934735734\\
1.88174704367609	89043.3709412812\\
1.88184704617615	89078.3213667842\\
1.88194704867622	89112.6988344921\\
1.88204705117628	89147.0763021999\\
1.88214705367634	89182.0267277029\\
1.8822470561764	89216.4041954107\\
1.88234705867647	89251.3546209137\\
1.88244706117653	89285.7320886216\\
1.88254706367659	89320.6825141246\\
1.88264706617665	89355.0599818324\\
1.88274706867672	89390.0104073354\\
1.88284707117678	89424.3878750432\\
1.88294707367684	89459.3383005462\\
1.8830470761769	89493.7157682541\\
1.88314707867697	89528.666193757\\
1.88324708117703	89563.61661926\\
1.88334708367709	89597.9940869679\\
1.88344708617715	89632.9445124709\\
1.88354708867722	89667.3219801787\\
1.88364709117728	89702.2724056817\\
1.88374709367734	89737.2228311847\\
1.8838470961774	89771.6002988925\\
1.88394709867747	89806.5507243955\\
1.88404710117753	89841.5011498985\\
1.88414710367759	89875.8786176063\\
1.88424710617765	89910.8290431093\\
1.88434710867772	89945.7794686123\\
1.88444711117778	89980.1569363201\\
1.88454711367784	90015.1073618231\\
1.8846471161779	90050.0577873261\\
1.88474711867797	90085.0082128291\\
1.88484712117803	90119.9586383321\\
1.88494712367809	90154.3361060399\\
1.88504712617815	90189.2865315429\\
1.88514712867822	90224.2369570459\\
1.88524713117828	90259.1873825488\\
1.88534713367834	90294.1378080518\\
1.8854471361784	90328.5152757597\\
1.88554713867847	90363.4657012627\\
1.88564714117853	90398.4161267656\\
1.88574714367859	90433.3665522686\\
1.88584714617865	90468.3169777716\\
1.88594714867872	90503.2674032746\\
1.88604715117878	90538.2178287776\\
1.88614715367884	90573.1682542805\\
1.8862471561789	90608.1186797835\\
1.88634715867897	90643.0691052865\\
1.88644716117903	90678.0195307895\\
1.88654716367909	90712.9699562925\\
1.88664716617915	90747.9203817954\\
1.88674716867922	90782.8708072984\\
1.88684717117928	90817.8212328014\\
1.88694717367934	90852.7716583044\\
1.8870471761794	90887.7220838074\\
1.88714717867947	90922.6725093103\\
1.88724718117953	90957.6229348133\\
1.88734718367959	90992.5733603163\\
1.88744718617965	91027.5237858193\\
1.88754718867972	91062.4742113223\\
1.88764719117978	91097.4246368252\\
1.88774719367984	91132.9480201233\\
1.8878471961799	91167.8984456263\\
1.88794719867997	91202.8488711293\\
1.88804720118003	91237.7992966323\\
1.88814720368009	91272.7497221353\\
1.88824720618015	91307.7001476382\\
1.88834720868022	91343.2235309364\\
1.88844721118028	91378.1739564394\\
1.88854721368034	91413.1243819423\\
1.8886472161804	91448.0748074453\\
1.88874721868047	91483.5981907434\\
1.88884722118053	91518.5486162464\\
1.88894722368059	91553.4990417494\\
1.88904722618065	91588.4494672523\\
1.88914722868072	91623.9728505505\\
1.88924723118078	91658.9232760535\\
1.88934723368084	91693.8737015564\\
1.8894472361809	91729.3970848545\\
1.88954723868097	91764.3475103575\\
1.88964724118103	91799.2979358605\\
1.88974724368109	91834.8213191586\\
1.88984724618115	91869.7717446616\\
1.88994724868122	91905.2951279597\\
1.89004725118128	91940.2455534627\\
1.89014725368134	91975.7689367608\\
1.8902472561814	92010.7193622638\\
1.89034725868147	92045.6697877667\\
1.89044726118153	92081.1931710648\\
1.89054726368159	92116.1435965679\\
1.89064726618165	92151.666979866\\
1.89074726868172	92186.6174053689\\
1.89084727118178	92222.140788667\\
1.89094727368184	92257.09121417\\
1.8910472761819	92292.6145974681\\
1.89114727868197	92328.1379807663\\
1.89124728118203	92363.0884062692\\
1.89134728368209	92398.6117895673\\
1.89144728618215	92433.5622150703\\
1.89154728868222	92469.0855983684\\
1.89164729118228	92504.6089816665\\
1.89174729368234	92539.5594071695\\
1.8918472961824	92575.0827904676\\
1.89194729868247	92610.0332159706\\
1.89204730118253	92645.5565992687\\
1.89214730368259	92681.0799825668\\
1.89224730618265	92716.6033658649\\
1.89234730868272	92751.5537913679\\
1.89244731118278	92787.077174666\\
1.89254731368284	92822.6005579641\\
1.8926473161829	92857.5509834671\\
1.89274731868297	92893.0743667652\\
1.89284732118303	92928.5977500633\\
1.89294732368309	92964.1211333615\\
1.89304732618315	92999.6445166596\\
1.89314732868322	93034.5949421626\\
1.89324733118328	93070.1183254607\\
1.89334733368334	93105.6417087588\\
1.8934473361834	93141.1650920569\\
1.89354733868347	93176.688475355\\
1.89364734118353	93212.2118586531\\
1.89374734368359	93247.1622841561\\
1.89384734618365	93282.6856674542\\
1.89394734868372	93318.2090507523\\
1.89404735118378	93353.7324340504\\
1.89414735368384	93389.2558173485\\
1.8942473561839	93424.7792006466\\
1.89434735868397	93460.3025839448\\
1.89444736118403	93495.8259672429\\
1.89454736368409	93531.349350541\\
1.89464736618415	93566.8727338391\\
1.89474736868422	93602.3961171372\\
1.89484737118428	93637.9195004353\\
1.89494737368434	93673.4428837334\\
1.8950473761844	93708.9662670315\\
1.89514737868447	93744.4896503296\\
1.89524738118453	93780.0130336278\\
1.89534738368459	93815.5364169259\\
1.89544738618465	93851.059800224\\
1.89554738868472	93886.5831835221\\
1.89564739118478	93922.1065668202\\
1.89574739368484	93958.2029079135\\
1.8958473961849	93993.7262912116\\
1.89594739868497	94029.2496745097\\
1.89604740118503	94064.7730578078\\
1.89614740368509	94100.2964411059\\
1.89624740618515	94135.819824404\\
1.89634740868522	94171.9161654972\\
1.89644741118528	94207.4395487954\\
1.89654741368534	94242.9629320935\\
1.8966474161854	94278.4863153916\\
1.89674741868547	94314.5826564848\\
1.89684742118553	94350.1060397829\\
1.89694742368559	94385.629423081\\
1.89704742618565	94421.1528063791\\
1.89714742868572	94457.2491474724\\
1.89724743118578	94492.7725307705\\
1.89734743368584	94528.2959140686\\
1.8974474361859	94564.3922551618\\
1.89754743868597	94599.9156384599\\
1.89764744118603	94635.4390217581\\
1.89774744368609	94671.5353628513\\
1.89784744618615	94707.0587461494\\
1.89794744868622	94742.5821294475\\
1.89804745118628	94778.6784705408\\
1.89814745368634	94814.2018538389\\
1.8982474561864	94850.2981949321\\
1.89834745868647	94885.8215782302\\
1.89844746118653	94921.9179193235\\
1.89854746368659	94957.4413026216\\
1.89864746618665	94993.5376437148\\
1.89874746868672	95029.0610270129\\
1.89884747118678	95065.1573681062\\
1.89894747368684	95100.6807514043\\
1.8990474761869	95136.7770924975\\
1.89914747868697	95172.3004757957\\
1.89924748118703	95208.3968168889\\
1.89934748368709	95243.920200187\\
1.89944748618715	95280.0165412802\\
1.89954748868722	95315.5399245783\\
1.89964749118728	95351.6362656716\\
1.89974749368734	95387.7326067648\\
1.8998474961874	95423.255990063\\
1.89994749868747	95459.3523311562\\
1.90004750118753	95495.4486722494\\
1.90014750368759	95530.9720555475\\
1.90024750618765	95567.0683966408\\
1.90034750868772	95603.164737734\\
1.90044751118778	95638.6881210321\\
1.90054751368784	95674.7844621254\\
1.9006475161879	95710.8808032186\\
1.90074751868797	95746.9771443119\\
1.90084752118803	95782.50052761\\
1.90094752368809	95818.5968687032\\
1.90104752618815	95854.6932097965\\
1.90114752868822	95890.7895508897\\
1.90124753118828	95926.3129341878\\
1.90134753368834	95962.4092752811\\
1.9014475361884	95998.5056163743\\
1.90154753868847	96034.6019574675\\
1.90164754118853	96070.6982985608\\
1.90174754368859	96106.794639654\\
1.90184754618865	96142.8909807473\\
1.90194754868872	96178.4143640454\\
1.90204755118878	96214.5107051386\\
1.90214755368884	96250.6070462319\\
1.9022475561889	96286.7033873251\\
1.90234755868897	96322.7997284184\\
1.90244756118903	96358.8960695116\\
1.90254756368909	96394.9924106048\\
1.90264756618915	96431.0887516981\\
1.90274756868922	96467.1850927913\\
1.90284757118928	96503.2814338846\\
1.90294757368934	96539.3777749778\\
1.9030475761894	96575.474116071\\
1.90314757868947	96611.5704571643\\
1.90324758118953	96647.6667982575\\
1.90334758368959	96683.7631393508\\
1.90344758618965	96719.859480444\\
1.90354758868972	96755.9558215373\\
1.90364759118978	96792.0521626305\\
1.90374759368984	96828.1485037237\\
1.9038475961899	96864.8178026121\\
1.90394759868997	96900.9141437053\\
1.90404760119003	96937.0104847986\\
1.90414760369009	96973.1068258918\\
1.90424760619015	97009.2031669851\\
1.90434760869022	97045.2995080783\\
1.90444761119028	97081.9688069667\\
1.90454761369034	97118.0651480599\\
1.9046476161904	97154.1614891532\\
1.90474761869047	97190.2578302464\\
1.90484762119053	97226.3541713397\\
1.90494762369059	97263.023470228\\
1.90504762619065	97299.1198113213\\
1.90514762869072	97335.2161524145\\
1.90524763119078	97371.8854513029\\
1.90534763369084	97407.9817923961\\
1.9054476361909	97444.0781334894\\
1.90554763869097	97480.7474323777\\
1.90564764119103	97516.843773471\\
1.90574764369109	97552.9401145642\\
1.90584764619115	97589.6094134526\\
1.90594764869122	97625.7057545458\\
1.90604765119128	97661.8020956391\\
1.90614765369134	97698.4713945275\\
1.9062476561914	97734.5677356207\\
1.90634765869147	97771.2370345091\\
1.90644766119153	97807.3333756023\\
1.90654766369159	97844.0026744907\\
1.90664766619165	97880.0990155839\\
1.90674766869172	97916.1953566772\\
1.90684767119178	97952.8646555655\\
1.90694767369184	97988.9609966588\\
1.9070476761919	98025.6302955471\\
1.90714767869197	98061.7266366404\\
1.90724768119203	98098.3959355288\\
1.90734768369209	98135.0652344172\\
1.90744768619215	98171.1615755104\\
1.90754768869222	98207.8308743988\\
1.90764769119228	98243.927215492\\
1.90774769369234	98280.5965143804\\
1.9078476961924	98316.6928554736\\
1.90794769869247	98353.362154362\\
1.90804770119253	98390.0314532504\\
1.90814770369259	98426.1277943436\\
1.90824770619265	98462.797093232\\
1.90834770869272	98499.4663921203\\
1.90844771119278	98535.5627332136\\
1.90854771369284	98572.232032102\\
1.9086477161929	98608.9013309903\\
1.90874771869297	98645.5706298787\\
1.90884772119303	98681.6669709719\\
1.90894772369309	98718.3362698603\\
1.90904772619315	98755.0055687487\\
1.90914772869322	98791.6748676371\\
1.90924773119328	98827.7712087303\\
1.90934773369334	98864.4405076187\\
1.9094477361934	98901.1098065071\\
1.90954773869347	98937.7791053954\\
1.90964774119353	98974.4484042838\\
1.90974774369359	99010.544745377\\
1.90984774619365	99047.2140442654\\
1.90994774869372	99083.8833431538\\
1.91004775119378	99120.5526420422\\
1.91014775369384	99157.2219409305\\
1.9102477561939	99193.8912398189\\
1.91034775869397	99230.5605387073\\
1.91044776119403	99267.2298375956\\
1.91054776369409	99303.899136484\\
1.91064776619415	99340.5684353724\\
1.91074776869422	99377.2377342608\\
1.91084777119428	99413.9070331491\\
1.91094777369434	99450.5763320375\\
1.9110477761944	99487.2456309259\\
1.91114777869447	99523.9149298142\\
1.91124778119453	99560.5842287026\\
1.91134778369459	99597.253527591\\
1.91144778619465	99633.9228264794\\
1.91154778869472	99670.5921253677\\
1.91164779119478	99707.2614242561\\
1.91174779369484	99743.9307231445\\
1.9118477961949	99780.6000220329\\
1.91194779869497	99817.2693209212\\
1.91204780119503	99853.9386198096\\
1.91214780369509	99890.607918698\\
1.91224780619515	99927.2772175864\\
1.91234780869522	99964.5194742699\\
1.91244781119528	100001.188773158\\
1.91254781369534	100037.858072047\\
1.9126478161954	100074.527370935\\
1.91274781869547	100111.196669823\\
1.91284782119553	100148.438926507\\
1.91294782369559	100185.108225395\\
1.91304782619565	100221.777524284\\
1.91314782869572	100258.446823172\\
1.91324783119578	100295.689079855\\
1.91334783369584	100332.358378744\\
1.9134478361959	100369.027677632\\
1.91354783869597	100406.269934316\\
1.91364784119603	100442.939233204\\
1.91374784369609	100479.608532092\\
1.91384784619616	100516.850788776\\
1.91394784869622	100553.520087664\\
1.91404785119628	100590.189386553\\
1.91414785369634	100627.431643236\\
1.9142478561964	100664.100942125\\
1.91434785869647	100700.770241013\\
1.91444786119653	100738.012497696\\
1.91454786369659	100774.681796585\\
1.91464786619665	100811.924053268\\
1.91474786869672	100848.593352157\\
1.91484787119678	100885.83560884\\
1.91494787369684	100922.504907729\\
1.9150478761969	100959.747164412\\
1.91514787869697	100996.4164633\\
1.91524788119703	101033.658719984\\
1.91534788369709	101070.328018872\\
1.91544788619715	101107.570275556\\
1.91554788869722	101144.239574444\\
1.91564789119728	101181.481831128\\
1.91574789369734	101218.151130016\\
1.91584789619741	101255.3933867\\
1.91594789869747	101292.635643383\\
1.91604790119753	101329.304942271\\
1.91614790369759	101366.547198955\\
1.91624790619765	101403.216497843\\
1.91634790869772	101440.458754527\\
1.91644791119778	101477.70101121\\
1.91654791369784	101514.370310099\\
1.9166479161979	101551.612566782\\
1.91674791869797	101588.854823466\\
1.91684792119803	101626.097080149\\
1.91694792369809	101662.766379038\\
1.91704792619816	101700.008635721\\
1.91714792869822	101737.250892405\\
1.91724793119828	101773.920191293\\
1.91734793369834	101811.162447977\\
1.9174479361984	101848.40470466\\
1.91754793869847	101885.646961344\\
1.91764794119853	101922.889218027\\
1.91774794369859	101959.558516915\\
1.91784794619866	101996.800773599\\
1.91794794869872	102034.043030282\\
1.91804795119878	102071.285286966\\
1.91814795369884	102108.527543649\\
1.9182479561989	102145.769800333\\
1.91834795869897	102183.012057016\\
1.91844796119903	102220.2543137\\
1.91854796369909	102256.923612588\\
1.91864796619915	102294.165869272\\
1.91874796869922	102331.408125955\\
1.91884797119928	102368.650382639\\
1.91894797369934	102405.892639322\\
1.91904797619941	102443.134896006\\
1.91914797869947	102480.377152689\\
1.91924798119953	102517.619409373\\
1.91934798369959	102554.861666056\\
1.91944798619965	102592.10392274\\
1.91954798869972	102629.346179423\\
1.91964799119978	102666.588436107\\
1.91974799369984	102703.83069279\\
1.91984799619991	102741.072949474\\
1.91994799869997	102778.888163952\\
1.92004800120003	102816.130420636\\
1.92014800370009	102853.372677319\\
1.92024800620016	102890.614934003\\
1.92034800870022	102927.857190686\\
1.92044801120028	102965.09944737\\
1.92054801370034	103002.341704053\\
1.9206480162004	103039.583960737\\
1.92074801870047	103077.399175216\\
1.92084802120053	103114.641431899\\
1.92094802370059	103151.883688583\\
1.92104802620066	103189.125945266\\
1.92114802870072	103226.36820195\\
1.92124803120078	103264.183416428\\
1.92134803370084	103301.425673112\\
1.9214480362009	103338.667929795\\
1.92154803870097	103376.483144274\\
1.92164804120103	103413.725400957\\
1.92174804370109	103450.967657641\\
1.92184804620116	103488.209914324\\
1.92194804870122	103526.025128803\\
1.92204805120128	103563.267385487\\
1.92214805370134	103600.50964217\\
1.92224805620141	103638.324856649\\
1.92234805870147	103675.567113332\\
1.92244806120153	103713.382327811\\
1.92254806370159	103750.624584494\\
1.92264806620165	103787.866841178\\
1.92274806870172	103825.682055656\\
1.92284807120178	103862.92431234\\
1.92294807370184	103900.739526819\\
1.92304807620191	103937.981783502\\
1.92314807870197	103975.796997981\\
1.92324808120203	104013.039254664\\
1.92334808370209	104050.854469143\\
1.92344808620216	104088.096725826\\
1.92354808870222	104125.911940305\\
1.92364809120228	104163.154196989\\
1.92374809370234	104200.969411467\\
1.92384809620241	104238.211668151\\
1.92394809870247	104276.026882629\\
1.92404810120253	104313.269139313\\
1.92414810370259	104351.084353791\\
1.92424810620266	104388.89956827\\
1.92434810870272	104426.141824954\\
1.92444811120278	104463.957039432\\
1.92454811370284	104501.772253911\\
1.9246481162029	104539.014510594\\
1.92474811870297	104576.829725073\\
1.92484812120303	104614.644939552\\
1.92494812370309	104651.887196235\\
1.92504812620316	104689.702410714\\
1.92514812870322	104727.517625192\\
1.92524813120328	104764.759881876\\
1.92534813370334	104802.575096355\\
1.92544813620341	104840.390310833\\
1.92554813870347	104878.205525312\\
1.92564814120353	104915.447781995\\
1.92574814370359	104953.262996474\\
1.92584814620366	104991.078210953\\
1.92594814870372	105028.893425431\\
1.92604815120378	105066.70863991\\
1.92614815370384	105103.950896593\\
1.92624815620391	105141.766111072\\
1.92634815870397	105179.581325551\\
1.92644816120403	105217.396540029\\
1.92654816370409	105255.211754508\\
1.92664816620416	105293.026968987\\
1.92674816870422	105330.842183465\\
1.92684817120428	105368.657397944\\
1.92694817370434	105406.472612422\\
1.92704817620441	105444.287826901\\
1.92714817870447	105481.530083585\\
1.92724818120453	105519.345298063\\
1.92734818370459	105557.160512542\\
1.92744818620466	105594.97572702\\
1.92754818870472	105632.790941499\\
1.92764819120478	105670.606155978\\
1.92774819370484	105708.421370456\\
1.92784819620491	105746.80954273\\
1.92794819870497	105784.624757209\\
1.92804820120503	105822.439971687\\
1.92814820370509	105860.255186166\\
1.92824820620516	105898.070400645\\
1.92834820870522	105935.885615123\\
1.92844821120528	105973.700829602\\
1.92854821370534	106011.516044081\\
1.92864821620541	106049.331258559\\
1.92874821870547	106087.146473038\\
1.92884822120553	106125.534645312\\
1.92894822370559	106163.34985979\\
1.92904822620566	106201.165074269\\
1.92914822870572	106238.980288747\\
1.92924823120578	106276.795503226\\
1.92934823370584	106315.1836755\\
1.92944823620591	106352.998889979\\
1.92954823870597	106390.814104457\\
1.92964824120603	106428.629318936\\
1.92974824370609	106467.01749121\\
1.92984824620616	106504.832705688\\
1.92994824870622	106542.647920167\\
1.93004825120628	106580.463134645\\
1.93014825370634	106618.851306919\\
1.93024825620641	106656.666521398\\
1.93034825870647	106694.481735877\\
1.93044826120653	106732.86990815\\
1.93054826370659	106770.685122629\\
1.93064826620666	106809.073294903\\
1.93074826870672	106846.888509381\\
1.93084827120678	106884.70372386\\
1.93094827370684	106923.091896134\\
1.93104827620691	106960.907110612\\
1.93114827870697	106999.295282886\\
1.93124828120703	107037.110497365\\
1.93134828370709	107075.498669639\\
1.93144828620716	107113.313884117\\
1.93154828870722	107151.129098596\\
1.93164829120728	107189.51727087\\
1.93174829370734	107227.332485348\\
1.93184829620741	107265.720657622\\
1.93194829870747	107304.108829896\\
1.93204830120753	107341.924044374\\
1.93214830370759	107380.312216648\\
1.93224830620766	107418.127431127\\
1.93234830870772	107456.5156034\\
1.93244831120778	107494.330817879\\
1.93254831370784	107532.718990153\\
1.93264831620791	107571.107162427\\
1.93274831870797	107608.922376905\\
1.93284832120803	107647.310549179\\
1.93294832370809	107685.698721453\\
1.93304832620816	107723.513935931\\
1.93314832870822	107761.902108205\\
1.93324833120828	107800.290280479\\
1.93334833370834	107838.105494958\\
1.93344833620841	107876.493667231\\
1.93354833870847	107914.881839505\\
1.93364834120853	107952.697053984\\
1.93374834370859	107991.085226258\\
1.93384834620866	108029.473398531\\
1.93394834870872	108067.861570805\\
1.93404835120878	108106.249743079\\
1.93414835370884	108144.064957557\\
1.93424835620891	108182.453129831\\
1.93434835870897	108220.841302105\\
1.93444836120903	108259.229474379\\
1.93454836370909	108297.617646653\\
1.93464836620916	108335.432861131\\
1.93474836870922	108373.821033405\\
1.93484837120928	108412.209205679\\
1.93494837370934	108450.597377952\\
1.93504837620941	108488.985550226\\
1.93514837870947	108527.3737225\\
1.93524838120953	108565.761894774\\
1.93534838370959	108604.150067048\\
1.93544838620966	108642.538239321\\
1.93554838870972	108680.926411595\\
1.93564839120978	108719.314583869\\
1.93574839370984	108757.702756143\\
1.93584839620991	108796.090928416\\
1.93594839870997	108834.47910069\\
1.93604840121003	108872.867272964\\
1.93614840371009	108911.255445238\\
1.93624840621016	108949.643617511\\
1.93634840871022	108988.031789785\\
1.93644841121028	109026.419962059\\
1.93654841371034	109064.808134333\\
1.93664841621041	109103.196306606\\
1.93674841871047	109141.58447888\\
1.93684842121053	109179.972651154\\
1.93694842371059	109218.933781223\\
1.93704842621066	109257.321953497\\
1.93714842871072	109295.71012577\\
1.93724843121078	109334.098298044\\
1.93734843371084	109372.486470318\\
1.93744843621091	109410.874642592\\
1.93754843871097	109449.835772661\\
1.93764844121103	109488.223944934\\
1.93774844371109	109526.612117208\\
1.93784844621116	109565.000289482\\
1.93794844871122	109603.388461756\\
1.93804845121128	109642.349591825\\
1.93814845371134	109680.737764098\\
1.93824845621141	109719.125936372\\
1.93834845871147	109758.087066441\\
1.93844846121153	109796.475238715\\
1.93854846371159	109834.863410989\\
1.93864846621166	109873.824541057\\
1.93874846871172	109912.212713331\\
1.93884847121178	109950.600885605\\
1.93894847371184	109989.562015674\\
1.93904847621191	110027.950187948\\
1.93914847871197	110066.338360221\\
1.93924848121203	110105.29949029\\
1.93934848371209	110143.687662564\\
1.93944848621216	110182.648792633\\
1.93954848871222	110221.036964907\\
1.93964849121228	110259.998094976\\
1.93974849371234	110298.386267249\\
1.93984849621241	110337.347397318\\
1.93994849871247	110375.735569592\\
1.94004850121253	110414.696699661\\
1.94014850371259	110453.084871935\\
1.94024850621266	110492.046002004\\
1.94034850871272	110530.434174277\\
1.94044851121278	110569.395304346\\
1.94054851371284	110607.78347662\\
1.94064851621291	110646.744606689\\
1.94074851871297	110685.132778963\\
1.94084852121303	110724.093909032\\
1.94094852371309	110763.0550391\\
1.94104852621316	110801.443211374\\
1.94114852871322	110840.404341443\\
1.94124853121328	110878.792513717\\
1.94134853371334	110917.753643786\\
1.94144853621341	110956.714773855\\
1.94154853871347	110995.675903924\\
1.94164854121353	111034.064076197\\
1.94174854371359	111073.025206266\\
1.94184854621366	111111.986336335\\
1.94194854871372	111150.374508609\\
1.94204855121378	111189.335638678\\
1.94214855371384	111228.296768747\\
1.94224855621391	111267.257898816\\
1.94234855871397	111305.646071089\\
1.94244856121403	111344.607201158\\
1.94254856371409	111383.568331227\\
1.94264856621416	111422.529461296\\
1.94274856871422	111461.490591365\\
1.94284857121428	111499.878763639\\
1.94294857371434	111538.839893708\\
1.94304857621441	111577.801023777\\
1.94314857871447	111616.762153845\\
1.94324858121453	111655.723283914\\
1.94334858371459	111694.684413983\\
1.94344858621466	111733.645544052\\
1.94354858871472	111772.606674121\\
1.94364859121478	111811.56780419\\
1.94374859371484	111849.955976464\\
1.94384859621491	111888.917106533\\
1.94394859871497	111927.878236601\\
1.94404860121503	111966.83936667\\
1.94414860371509	112005.800496739\\
1.94424860621516	112044.761626808\\
1.94434860871522	112083.722756877\\
1.94444861121528	112122.683886946\\
1.94454861371534	112161.645017015\\
1.94464861621541	112200.606147084\\
1.94474861871547	112240.140234948\\
1.94484862121553	112279.101365017\\
1.94494862371559	112318.062495086\\
1.94504862621566	112357.023625154\\
1.94514862871572	112395.984755223\\
1.94524863121578	112434.945885292\\
1.94534863371584	112473.907015361\\
1.94544863621591	112512.86814543\\
1.94554863871597	112551.829275499\\
1.94564864121603	112591.363363363\\
1.94574864371609	112630.324493432\\
1.94584864621616	112669.285623501\\
1.94594864871622	112708.24675357\\
1.94604865121628	112747.207883639\\
1.94614865371634	112786.741971503\\
1.94624865621641	112825.703101571\\
1.94634865871647	112864.66423164\\
1.94644866121653	112903.625361709\\
1.94654866371659	112943.159449573\\
1.94664866621666	112982.120579642\\
1.94674866871672	113021.081709711\\
1.94684867121678	113060.615797575\\
1.94694867371684	113099.576927644\\
1.94704867621691	113138.538057713\\
1.94714867871697	113178.072145577\\
1.94724868121703	113217.033275646\\
1.94734868371709	113255.994405715\\
1.94744868621716	113295.528493579\\
1.94754868871722	113334.489623648\\
1.94764869121728	113373.450753717\\
1.94774869371734	113412.984841581\\
1.94784869621741	113451.945971649\\
1.94794869871747	113491.480059513\\
1.94804870121753	113530.441189582\\
1.94814870371759	113569.975277446\\
1.94824870621766	113608.936407515\\
1.94834870871772	113648.470495379\\
1.94844871121778	113687.431625448\\
1.94854871371784	113726.965713312\\
1.94864871621791	113765.926843381\\
1.94874871871797	113805.460931245\\
1.94884872121803	113844.422061314\\
1.94894872371809	113883.956149178\\
1.94904872621816	113922.917279247\\
1.94914872871822	113962.451367111\\
1.94924873121828	114001.985454975\\
1.94934873371834	114040.946585044\\
1.94944873621841	114080.480672908\\
1.94954873871847	114119.441802977\\
1.94964874121853	114158.975890841\\
1.94974874371859	114198.509978705\\
1.94984874621866	114237.471108774\\
1.94994874871872	114277.005196638\\
1.95004875121878	114316.539284502\\
1.95014875371884	114355.500414571\\
1.95024875621891	114395.034502435\\
1.95034875871897	114434.568590299\\
1.95044876121903	114474.102678163\\
1.95054876371909	114513.063808232\\
1.95064876621916	114552.597896096\\
1.95074876871922	114592.13198396\\
1.95084877121928	114631.666071824\\
1.95094877371934	114670.627201893\\
1.95104877621941	114710.161289757\\
1.95114877871947	114749.695377621\\
1.95124878121953	114789.229465485\\
1.95134878371959	114828.763553349\\
1.95144878621966	114868.297641213\\
1.95154878871972	114907.258771282\\
1.95164879121978	114946.792859146\\
1.95174879371984	114986.32694701\\
1.95184879621991	115025.861034874\\
1.95194879871997	115065.395122738\\
1.95204880122003	115104.929210602\\
1.95214880372009	115144.463298466\\
1.95224880622016	115183.99738633\\
1.95234880872022	115223.531474194\\
1.95244881122028	115263.065562058\\
1.95254881372034	115302.599649922\\
1.95264881622041	115342.133737786\\
1.95274881872047	115381.66782565\\
1.95284882122053	115421.201913514\\
1.95294882372059	115460.736001378\\
1.95304882622066	115500.270089242\\
1.95314882872072	115539.804177106\\
1.95324883122078	115579.33826497\\
1.95334883372084	115618.872352834\\
1.95344883622091	115658.406440698\\
1.95354883872097	115697.940528562\\
1.95364884122103	115737.474616426\\
1.95374884372109	115777.00870429\\
1.95384884622116	115816.542792154\\
1.95394884872122	115856.649837813\\
1.95404885122128	115896.183925678\\
1.95414885372134	115935.718013542\\
1.95424885622141	115975.252101406\\
1.95434885872147	116014.78618927\\
1.95444886122153	116054.893234929\\
1.95454886372159	116094.427322793\\
1.95464886622166	116133.961410657\\
1.95474886872172	116173.495498521\\
1.95484887122178	116213.029586385\\
1.95494887372184	116253.136632044\\
1.95504887622191	116292.670719908\\
1.95514887872197	116332.204807772\\
1.95524888122203	116372.311853431\\
1.95534888372209	116411.845941295\\
1.95544888622216	116451.380029159\\
1.95554888872222	116491.487074818\\
1.95564889122228	116531.021162682\\
1.95574889372234	116570.555250547\\
1.95584889622241	116610.662296206\\
1.95594889872247	116650.19638407\\
1.95604890122253	116689.730471934\\
1.95614890372259	116729.837517593\\
1.95624890622266	116769.371605457\\
1.95634890872272	116809.478651116\\
1.95644891122278	116849.01273898\\
1.95654891372284	116888.546826844\\
1.95664891622291	116928.653872503\\
1.95674891872297	116968.187960367\\
1.95684892122303	117008.295006026\\
1.95694892372309	117047.82909389\\
1.95704892622316	117087.93613955\\
1.95714892872322	117127.470227414\\
1.95724893122328	117167.577273073\\
1.95734893372334	117207.111360937\\
1.95744893622341	117247.218406596\\
1.95754893872347	117287.325452255\\
1.95764894122353	117326.859540119\\
1.95774894372359	117366.966585778\\
1.95784894622366	117406.500673642\\
1.95794894872372	117446.607719302\\
1.95804895122378	117486.714764961\\
1.95814895372384	117526.248852825\\
1.95824895622391	117566.355898484\\
1.95834895872397	117606.462944143\\
1.95844896122403	117645.997032007\\
1.95854896372409	117686.104077666\\
1.95864896622416	117726.211123325\\
1.95874896872422	117765.745211189\\
1.95884897122428	117805.852256849\\
1.95894897372434	117845.959302508\\
1.95904897622441	117885.493390372\\
1.95914897872447	117925.600436031\\
1.95924898122453	117965.70748169\\
1.95934898372459	118005.814527349\\
1.95944898622466	118045.921573008\\
1.95954898872472	118085.455660872\\
1.95964899122478	118125.562706532\\
1.95974899372484	118165.669752191\\
1.95984899622491	118205.77679785\\
1.95994899872497	118245.883843509\\
1.96004900122503	118285.990889168\\
1.96014900372509	118325.524977032\\
1.96024900622516	118365.632022691\\
1.96034900872522	118405.739068351\\
1.96044901122528	118445.84611401\\
1.96054901372534	118485.953159669\\
1.96064901622541	118526.060205328\\
1.96074901872547	118566.167250987\\
1.96084902122553	118606.274296646\\
1.96094902372559	118646.381342305\\
1.96104902622566	118686.488387965\\
1.96114902872572	118726.595433624\\
1.96124903122578	118766.702479283\\
1.96134903372584	118806.809524942\\
1.96144903622591	118846.916570601\\
1.96154903872597	118887.02361626\\
1.96164904122603	118927.13066192\\
1.96174904372609	118967.237707579\\
1.96184904622616	119007.344753238\\
1.96194904872622	119047.451798897\\
1.96204905122628	119087.558844556\\
1.96214905372634	119127.665890215\\
1.96224905622641	119167.772935875\\
1.96234905872647	119207.879981534\\
1.96244906122653	119248.559984988\\
1.96254906372659	119288.667030647\\
1.96264906622666	119328.774076306\\
1.96274906872672	119368.881121965\\
1.96284907122678	119408.988167625\\
1.96294907372684	119449.095213284\\
1.96304907622691	119489.775216738\\
1.96314907872697	119529.882262397\\
1.96324908122703	119569.989308056\\
1.96334908372709	119610.096353716\\
1.96344908622716	119650.77635717\\
1.96354908872722	119690.883402829\\
1.96364909122728	119730.990448488\\
1.96374909372734	119771.097494147\\
1.96384909622741	119811.777497602\\
1.96394909872747	119851.884543261\\
1.96404910122753	119891.99158892\\
1.96414910372759	119932.671592374\\
1.96424910622766	119972.778638033\\
1.96434910872772	120012.885683692\\
1.96444911122778	120053.565687147\\
1.96454911372784	120093.672732806\\
1.96464911622791	120133.779778465\\
1.96474911872797	120174.459781919\\
1.96484912122803	120214.566827579\\
1.96494912372809	120255.246831033\\
1.96504912622816	120295.353876692\\
1.96514912872822	120336.033880146\\
1.96524913122828	120376.140925805\\
1.96534913372834	120416.82092926\\
1.96544913622841	120456.927974919\\
1.96554913872847	120497.035020578\\
1.96564914122853	120537.715024032\\
1.96574914372859	120577.822069691\\
1.96584914622866	120618.502073146\\
1.96594914872872	120659.1820766\\
1.96604915122878	120699.289122259\\
1.96614915372884	120739.969125714\\
1.96624915622891	120780.076171373\\
1.96634915872897	120820.756174827\\
1.96644916122903	120860.863220486\\
1.96654916372909	120901.54322394\\
1.96664916622916	120942.223227395\\
1.96674916872922	120982.330273054\\
1.96684917122928	121023.010276508\\
1.96694917372934	121063.690279962\\
1.96704917622941	121103.797325622\\
1.96714917872947	121144.477329076\\
1.96724918122953	121185.15733253\\
1.96734918372959	121225.264378189\\
1.96744918622966	121265.944381644\\
1.96754918872972	121306.624385098\\
1.96764919122978	121346.731430757\\
1.96774919372984	121387.411434211\\
1.96784919622991	121428.091437666\\
1.96794919872997	121468.77144112\\
1.96804920123003	121509.451444574\\
1.96814920373009	121549.558490233\\
1.96824920623016	121590.238493688\\
1.96834920873022	121630.918497142\\
1.96844921123028	121671.598500596\\
1.96854921373034	121712.278504051\\
1.96864921623041	121752.38554971\\
1.96874921873047	121793.065553164\\
1.96884922123053	121833.745556618\\
1.96894922373059	121874.425560073\\
1.96904922623066	121915.105563527\\
1.96914922873072	121955.785566981\\
1.96924923123078	121996.465570435\\
1.96934923373084	122037.14557389\\
1.96944923623091	122077.825577344\\
1.96954923873097	122118.505580798\\
1.96964924123103	122159.185584253\\
1.96974924373109	122199.865587707\\
1.96984924623116	122240.545591161\\
1.96994924873122	122281.225594615\\
1.97004925123128	122321.90559807\\
1.97014925373134	122362.585601524\\
1.97024925623141	122403.265604978\\
1.97034925873147	122443.945608433\\
1.97044926123153	122484.625611887\\
1.97054926373159	122525.305615341\\
1.97064926623166	122565.985618795\\
1.97074926873172	122606.66562225\\
1.97084927123178	122647.345625704\\
1.97094927373184	122688.025629158\\
1.97104927623191	122728.705632613\\
1.97114927873197	122769.385636067\\
1.97124928123203	122810.638597316\\
1.97134928373209	122851.318600771\\
1.97144928623216	122891.998604225\\
1.97154928873222	122932.678607679\\
1.97164929123228	122973.358611133\\
1.97174929373234	123014.038614588\\
1.97184929623241	123055.291575837\\
1.97194929873247	123095.971579291\\
1.97204930123253	123136.651582746\\
1.97214930373259	123177.3315862\\
1.97224930623266	123218.584547449\\
1.97234930873272	123259.264550904\\
1.97244931123278	123299.944554358\\
1.97254931373284	123341.197515607\\
1.97264931623291	123381.877519062\\
1.97274931873297	123422.557522516\\
1.97284932123303	123463.23752597\\
1.97294932373309	123504.49048722\\
1.97304932623316	123545.170490674\\
1.97314932873322	123586.423451923\\
1.97324933123328	123627.103455378\\
1.97334933373334	123667.783458832\\
1.97344933623341	123709.036420081\\
1.97354933873347	123749.716423536\\
1.97364934123353	123790.969384785\\
1.97374934373359	123831.649388239\\
1.97384934623366	123872.329391694\\
1.97394934873372	123913.582352943\\
1.97404935123378	123954.262356397\\
1.97414935373384	123995.515317647\\
1.97424935623391	124036.195321101\\
1.97434935873397	124077.448282351\\
1.97444936123403	124118.128285805\\
1.97454936373409	124159.381247054\\
1.97464936623416	124200.061250509\\
1.97474936873422	124241.314211758\\
1.97484937123428	124281.994215212\\
1.97494937373434	124323.247176462\\
1.97504937623441	124364.500137711\\
1.97514937873447	124405.180141165\\
1.97524938123453	124446.433102415\\
1.97534938373459	124487.113105869\\
1.97544938623466	124528.366067119\\
1.97554938873472	124569.619028368\\
1.97564939123478	124610.299031822\\
1.97574939373484	124651.551993072\\
1.97584939623491	124692.804954321\\
1.97594939873497	124733.484957775\\
1.97604940123503	124774.737919025\\
1.97614940373509	124815.990880274\\
1.97624940623516	124856.670883728\\
1.97634940873522	124897.923844978\\
1.97644941123528	124939.176806227\\
1.97654941373534	124980.429767477\\
1.97664941623541	125021.109770931\\
1.97674941873547	125062.36273218\\
1.97684942123553	125103.61569343\\
1.97694942373559	125144.868654679\\
1.97704942623566	125186.121615929\\
1.97714942873572	125226.801619383\\
1.97724943123578	125268.054580632\\
1.97734943373584	125309.307541882\\
1.97744943623591	125350.560503131\\
1.97754943873597	125391.813464381\\
1.97764944123603	125433.06642563\\
1.97774944373609	125474.31938688\\
1.97784944623616	125515.572348129\\
1.97794944873622	125556.252351583\\
1.97804945123628	125597.505312833\\
1.97814945373634	125638.758274082\\
1.97824945623641	125680.011235331\\
1.97834945873647	125721.264196581\\
1.97844946123653	125762.51715783\\
1.97854946373659	125803.77011908\\
1.97864946623666	125845.023080329\\
1.97874946873672	125886.276041579\\
1.97884947123678	125927.529002828\\
1.97894947373684	125968.781964077\\
1.97904947623691	126010.034925327\\
1.97914947873697	126051.287886576\\
1.97924948123703	126092.540847826\\
1.97934948373709	126133.793809075\\
1.97944948623716	126175.046770325\\
1.97954948873722	126216.872689369\\
1.97964949123728	126258.125650618\\
1.97974949373734	126299.378611868\\
1.97984949623741	126340.631573117\\
1.97994949873747	126381.884534367\\
1.98004950123753	126423.137495616\\
1.98014950373759	126464.390456866\\
1.98024950623766	126505.643418115\\
1.98034950873772	126547.46933716\\
1.98044951123778	126588.722298409\\
1.98054951373784	126629.975259658\\
1.98064951623791	126671.228220908\\
1.98074951873797	126712.481182157\\
1.98084952123803	126754.307101202\\
1.98094952373809	126795.560062451\\
1.98104952623816	126836.813023701\\
1.98114952873822	126878.06598495\\
1.98124953123828	126919.891903995\\
1.98134953373834	126961.144865244\\
1.98144953623841	127002.397826493\\
1.98154953873847	127044.223745538\\
1.98164954123853	127085.476706787\\
1.98174954373859	127126.729668037\\
1.98184954623866	127168.555587081\\
1.98194954873872	127209.808548331\\
1.98204955123878	127251.06150958\\
1.98214955373884	127292.887428625\\
1.98224955623891	127334.140389874\\
1.98234955873897	127375.966308919\\
1.98244956123903	127417.219270168\\
1.98254956373909	127458.472231418\\
1.98264956623916	127500.298150462\\
1.98274956873922	127541.551111712\\
1.98284957123928	127583.377030756\\
1.98294957373934	127624.629992005\\
1.98304957623941	127666.45591105\\
1.98314957873947	127707.708872299\\
1.98324958123953	127749.534791344\\
1.98334958373959	127790.787752593\\
1.98344958623966	127832.613671638\\
1.98354958873972	127873.866632887\\
1.98364959123978	127915.692551932\\
1.98374959373984	127956.945513181\\
1.98384959623991	127998.771432226\\
1.98394959873997	128040.59735127\\
1.98404960124003	128081.85031252\\
1.98414960374009	128123.676231564\\
1.98424960624016	128164.929192814\\
1.98434960874022	128206.755111858\\
1.98444961124028	128248.581030903\\
1.98454961374034	128289.833992152\\
1.98464961624041	128331.659911197\\
1.98474961874047	128373.485830241\\
1.98484962124053	128414.738791491\\
1.98494962374059	128456.564710535\\
1.98504962624066	128498.39062958\\
1.98514962874072	128539.643590829\\
1.98524963124078	128581.469509874\\
1.98534963374084	128623.295428918\\
1.98544963624091	128665.121347963\\
1.98554963874097	128706.374309212\\
1.98564964124103	128748.200228257\\
1.98574964374109	128790.026147302\\
1.98584964624116	128831.852066346\\
1.98594964874122	128873.677985391\\
1.98604965124128	128914.93094664\\
1.98614965374134	128956.756865685\\
1.98624965624141	128998.582784729\\
1.98634965874147	129040.408703774\\
1.98644966124153	129082.234622818\\
1.98654966374159	129124.060541863\\
1.98664966624166	129165.886460907\\
1.98674966874172	129207.139422157\\
1.98684967124178	129248.965341201\\
1.98694967374184	129290.791260246\\
1.98704967624191	129332.61717929\\
1.98714967874197	129374.443098335\\
1.98724968124203	129416.26901738\\
1.98734968374209	129458.094936424\\
1.98744968624216	129499.920855469\\
1.98754968874222	129541.746774513\\
1.98764969124228	129583.572693558\\
1.98774969374234	129625.398612602\\
1.98784969624241	129667.224531647\\
1.98794969874247	129709.050450691\\
1.98804970124253	129750.876369736\\
1.98814970374259	129792.702288781\\
1.98824970624266	129834.528207825\\
1.98834970874272	129876.35412687\\
1.98844971124278	129918.180045914\\
1.98854971374284	129960.578922754\\
1.98864971624291	130002.404841798\\
1.98874971874297	130044.230760843\\
1.98884972124303	130086.056679887\\
1.98894972374309	130127.882598932\\
1.98904972624316	130169.708517977\\
1.98914972874322	130211.534437021\\
1.98924973124328	130253.933313861\\
1.98934973374334	130295.759232905\\
1.98944973624341	130337.58515195\\
1.98954973874347	130379.411070994\\
1.98964974124353	130421.236990039\\
1.98974974374359	130463.635866879\\
1.98984974624366	130505.461785923\\
1.98994974874372	130547.287704968\\
1.99004975124378	130589.113624012\\
1.99014975374384	130631.512500852\\
1.99024975624391	130673.338419897\\
1.99034975874397	130715.164338941\\
1.99044976124403	130756.990257986\\
1.99054976374409	130799.389134825\\
1.99064976624416	130841.21505387\\
1.99074976874422	130883.040972914\\
1.99084977124428	130925.439849754\\
1.99094977374434	130967.265768799\\
1.99104977624441	131009.664645638\\
1.99114977874447	131051.490564683\\
1.99124978124453	131093.316483728\\
1.99134978374459	131135.715360567\\
1.99144978624466	131177.541279612\\
1.99154978874472	131219.940156451\\
1.99164979124478	131261.766075496\\
1.99174979374484	131303.59199454\\
1.99184979624491	131345.99087138\\
1.99194979874497	131387.816790425\\
1.99204980124503	131430.215667264\\
1.99214980374509	131472.041586309\\
1.99224980624516	131514.440463149\\
1.99234980874522	131556.266382193\\
1.99244981124528	131598.665259033\\
1.99254981374534	131640.491178077\\
1.99264981624541	131682.890054917\\
1.99274981874547	131725.288931757\\
1.99284982124553	131767.114850801\\
1.99294982374559	131809.513727641\\
1.99304982624566	131851.339646686\\
1.99314982874572	131893.738523525\\
1.99324983124578	131936.137400365\\
1.99334983374584	131977.963319409\\
1.99344983624591	132020.362196249\\
1.99354983874597	132062.188115294\\
1.99364984124603	132104.586992133\\
1.99374984374609	132146.985868973\\
1.99384984624616	132188.811788018\\
1.99394984874622	132231.210664857\\
1.99404985124628	132273.609541697\\
1.99414985374634	132316.008418537\\
1.99424985624641	132357.834337581\\
1.99434985874647	132400.233214421\\
1.99444986124653	132442.632091261\\
1.99454986374659	132485.0309681\\
1.99464986624666	132526.856887145\\
1.99474986874672	132569.255763984\\
1.99484987124678	132611.654640824\\
1.99494987374684	132654.053517664\\
1.99504987624691	132695.879436708\\
1.99514987874697	132738.278313548\\
1.99524988124703	132780.677190388\\
1.99534988374709	132823.076067227\\
1.99544988624716	132865.474944067\\
1.99554988874722	132907.873820907\\
1.99564989124728	132950.272697746\\
1.99574989374734	132992.098616791\\
1.99584989624741	133034.497493631\\
1.99594989874747	133076.89637047\\
1.99604990124753	133119.29524731\\
1.99614990374759	133161.69412415\\
1.99624990624766	133204.093000989\\
1.99634990874772	133246.491877829\\
1.99644991124778	133288.890754669\\
1.99654991374784	133331.289631508\\
1.99664991624791	133373.688508348\\
1.99674991874797	133416.087385188\\
1.99684992124803	133458.486262028\\
1.99694992374809	133500.885138867\\
1.99704992624816	133543.284015707\\
1.99714992874822	133585.682892547\\
1.99724993124828	133628.081769386\\
1.99734993374834	133670.480646226\\
1.99744993624841	133712.879523066\\
1.99754993874847	133755.278399905\\
1.99764994124853	133797.677276745\\
1.99774994374859	133840.64911138\\
1.99784994624866	133883.047988219\\
1.99794994874872	133925.446865059\\
1.99804995124878	133967.845741899\\
1.99814995374884	134010.244618739\\
1.99824995624891	134052.643495578\\
1.99834995874897	134095.042372418\\
1.99844996124903	134138.014207053\\
1.99854996374909	134180.413083892\\
1.99864996624916	134222.811960732\\
1.99874996874922	134265.210837572\\
1.99884997124928	134307.609714411\\
1.99894997374934	134350.581549046\\
1.99904997624941	134392.980425886\\
1.99914997874947	134435.379302726\\
1.99924998124953	134478.35113736\\
1.99934998374959	134520.7500142\\
1.99944998624966	134563.14889104\\
1.99954998874972	134605.547767879\\
1.99964999124978	134648.519602514\\
1.99974999374984	134690.918479354\\
1.99984999624991	134733.317356194\\
1.99994999874997	134776.289190828\\
2.00005000125003	134818.688067668\\
};
\addplot [color=mycolor1,solid,forget plot]
  table[row sep=crcr]{%
2.00005000125003	134818.688067668\\
2.00015000375009	134861.086944508\\
2.00025000625016	134904.058779143\\
2.00035000875022	134946.457655982\\
2.00045001125028	134989.429490617\\
2.00055001375034	135031.828367457\\
2.00065001625041	135074.227244296\\
2.00075001875047	135117.199078931\\
2.00085002125053	135159.597955771\\
2.00095002375059	135202.569790406\\
2.00105002625066	135244.968667245\\
2.00115002875072	135287.94050188\\
2.00125003125078	135330.33937872\\
2.00135003375084	135373.311213355\\
2.00145003625091	135415.710090194\\
2.00155003875097	135458.681924829\\
2.00165004125103	135501.080801669\\
2.00175004375109	135544.052636304\\
2.00185004625116	135586.451513143\\
2.00195004875122	135629.423347778\\
2.00205005125128	135671.822224618\\
2.00215005375134	135714.794059253\\
2.00225005625141	135757.765893888\\
2.00235005875147	135800.164770727\\
2.00245006125153	135843.136605362\\
2.00255006375159	135886.108439997\\
2.00265006625166	135928.507316837\\
2.00275006875172	135971.479151471\\
2.00285007125178	136013.878028311\\
2.00295007375184	136056.849862946\\
2.00305007625191	136099.821697581\\
2.00315007875197	136142.793532215\\
2.00325008125203	136185.192409055\\
2.00335008375209	136228.16424369\\
2.00345008625216	136271.136078325\\
2.00355008875222	136313.534955164\\
2.00365009125228	136356.506789799\\
2.00375009375234	136399.478624434\\
2.00385009625241	136442.450459069\\
2.00395009875247	136484.849335909\\
2.00405010125253	136527.821170543\\
2.00415010375259	136570.793005178\\
2.00425010625266	136613.764839813\\
2.00435010875272	136656.736674448\\
2.00445011125278	136699.135551287\\
2.00455011375284	136742.107385922\\
2.00465011625291	136785.079220557\\
2.00475011875297	136828.051055192\\
2.00485012125303	136871.022889827\\
2.00495012375309	136913.994724462\\
2.00505012625316	136956.966559096\\
2.00515012875322	136999.938393731\\
2.00525013125328	137042.910228366\\
2.00535013375334	137085.309105206\\
2.00545013625341	137128.28093984\\
2.00555013875347	137171.252774475\\
2.00565014125353	137214.22460911\\
2.00575014375359	137257.196443745\\
2.00585014625366	137300.16827838\\
2.00595014875372	137343.140113015\\
2.00605015125378	137386.111947649\\
2.00615015375384	137429.083782284\\
2.00625015625391	137472.055616919\\
2.00635015875397	137515.027451554\\
2.00645016125403	137557.999286189\\
2.00655016375409	137600.971120823\\
2.00665016625416	137644.515913253\\
2.00675016875422	137687.487747888\\
2.00685017125428	137730.459582523\\
2.00695017375434	137773.431417158\\
2.00705017625441	137816.403251793\\
2.00715017875447	137859.375086427\\
2.00725018125453	137902.346921062\\
2.00735018375459	137945.318755697\\
2.00745018625466	137988.290590332\\
2.00755018875472	138031.835382762\\
2.00765019125478	138074.807217397\\
2.00775019375484	138117.779052031\\
2.00785019625491	138160.750886666\\
2.00795019875497	138203.722721301\\
2.00805020125503	138247.267513731\\
2.00815020375509	138290.239348366\\
2.00825020625516	138333.211183001\\
2.00835020875522	138376.183017635\\
2.00845021125528	138419.727810065\\
2.00855021375534	138462.6996447\\
2.00865021625541	138505.671479335\\
2.00875021875547	138548.64331397\\
2.00885022125553	138592.1881064\\
2.00895022375559	138635.159941035\\
2.00905022625566	138678.131775669\\
2.00915022875572	138721.676568099\\
2.00925023125578	138764.648402734\\
2.00935023375584	138807.620237369\\
2.00945023625591	138851.165029799\\
2.00955023875597	138894.136864434\\
2.00965024125603	138937.108699068\\
2.00975024375609	138980.653491498\\
2.00985024625616	139023.625326133\\
2.00995024875622	139067.170118563\\
2.01005025125628	139110.141953198\\
2.01015025375634	139153.686745628\\
2.01025025625641	139196.658580263\\
2.01035025875647	139239.630414898\\
2.01045026125653	139283.175207327\\
2.01055026375659	139326.147041962\\
2.01065026625666	139369.691834392\\
2.01075026875672	139412.663669027\\
2.01085027125678	139456.208461457\\
2.01095027375684	139499.180296092\\
2.01105027625691	139542.725088522\\
2.01115027875697	139586.269880952\\
2.01125028125703	139629.241715586\\
2.01135028375709	139672.786508016\\
2.01145028625716	139715.758342651\\
2.01155028875722	139759.303135081\\
2.01165029125728	139802.274969716\\
2.01175029375734	139845.819762146\\
2.01185029625741	139889.364554576\\
2.01195029875747	139932.336389211\\
2.01205030125753	139975.881181641\\
2.01215030375759	140019.425974071\\
2.01225030625766	140062.397808705\\
2.01235030875772	140105.942601135\\
2.01245031125778	140149.487393565\\
2.01255031375784	140192.4592282\\
2.01265031625791	140236.00402063\\
2.01275031875797	140279.54881306\\
2.01285032125803	140322.520647695\\
2.01295032375809	140366.065440125\\
2.01305032625816	140409.610232555\\
2.01315032875822	140453.155024985\\
2.01325033125828	140496.126859619\\
2.01335033375834	140539.671652049\\
2.01345033625841	140583.216444479\\
2.01355033875847	140626.761236909\\
2.01365034125853	140670.306029339\\
2.01375034375859	140713.277863974\\
2.01385034625866	140756.822656404\\
2.01395034875872	140800.367448834\\
2.01405035125878	140843.912241264\\
2.01415035375884	140887.457033694\\
2.01425035625891	140931.001826124\\
2.01435035875897	140974.546618554\\
2.01445036125903	141017.518453188\\
2.01455036375909	141061.063245618\\
2.01465036625916	141104.608038048\\
2.01475036875922	141148.152830478\\
2.01485037125928	141191.697622908\\
2.01495037375934	141235.242415338\\
2.01505037625941	141278.787207768\\
2.01515037875947	141322.332000198\\
2.01525038125953	141365.876792628\\
2.01535038375959	141409.421585058\\
2.01545038625966	141452.966377488\\
2.01555038875972	141496.511169918\\
2.01565039125978	141540.055962348\\
2.01575039375984	141583.600754778\\
2.01585039625991	141627.145547208\\
2.01595039875997	141670.690339638\\
2.01605040126003	141714.235132068\\
2.01615040376009	141757.779924497\\
2.01625040626016	141801.324716927\\
2.01635040876022	141844.869509357\\
2.01645041126028	141888.987259582\\
2.01655041376034	141932.532052012\\
2.01665041626041	141976.076844442\\
2.01675041876047	142019.621636872\\
2.01685042126053	142063.166429302\\
2.01695042376059	142106.711221732\\
2.01705042626066	142150.256014162\\
2.01715042876072	142194.373764387\\
2.01725043126078	142237.918556817\\
2.01735043376084	142281.463349247\\
2.01745043626091	142325.008141677\\
2.01755043876097	142368.552934107\\
2.01765044126103	142412.670684332\\
2.01775044376109	142456.215476762\\
2.01785044626116	142499.760269192\\
2.01795044876122	142543.305061622\\
2.01805045126128	142587.422811847\\
2.01815045376134	142630.967604277\\
2.01825045626141	142674.512396707\\
2.01835045876147	142718.057189137\\
2.01845046126153	142762.174939362\\
2.01855046376159	142805.719731792\\
2.01865046626166	142849.264524222\\
2.01875046876172	142893.382274447\\
2.01885047126178	142936.927066877\\
2.01895047376184	142980.471859307\\
2.01905047626191	143024.589609532\\
2.01915047876197	143068.134401962\\
2.01925048126203	143112.252152187\\
2.01935048376209	143155.796944617\\
2.01945048626216	143199.341737047\\
2.01955048876222	143243.459487272\\
2.01965049126228	143287.004279702\\
2.01975049376234	143331.122029927\\
2.01985049626241	143374.666822357\\
2.01995049876247	143418.784572582\\
2.02005050126253	143462.329365012\\
2.02015050376259	143506.447115237\\
2.02025050626266	143549.991907667\\
2.02035050876272	143594.109657892\\
2.02045051126278	143637.654450322\\
2.02055051376284	143681.772200547\\
2.02065051626291	143725.316992977\\
2.02075051876297	143769.434743202\\
2.02085052126303	143812.979535632\\
2.02095052376309	143857.097285857\\
2.02105052626316	143900.642078287\\
2.02115052876322	143944.759828512\\
2.02125053126328	143988.877578737\\
2.02135053376334	144032.422371167\\
2.02145053626341	144076.540121392\\
2.02155053876347	144120.084913822\\
2.02165054126353	144164.202664047\\
2.02175054376359	144208.320414272\\
2.02185054626366	144251.865206702\\
2.02195054876372	144295.982956927\\
2.02205055126378	144340.100707152\\
2.02215055376384	144383.645499582\\
2.02225055626391	144427.763249807\\
2.02235055876397	144471.881000032\\
2.02245056126403	144515.998750257\\
2.02255056376409	144559.543542687\\
2.02265056626416	144603.661292912\\
2.02275056876422	144647.779043137\\
2.02285057126428	144691.896793362\\
2.02295057376434	144735.441585792\\
2.02305057626441	144779.559336017\\
2.02315057876447	144823.677086243\\
2.02325058126453	144867.794836468\\
2.02335058376459	144911.912586693\\
2.02345058626466	144955.457379123\\
2.02355058876472	144999.575129348\\
2.02365059126478	145043.692879573\\
2.02375059376484	145087.810629798\\
2.02385059626491	145131.928380023\\
2.02395059876497	145176.046130248\\
2.02405060126503	145219.590922678\\
2.02415060376509	145263.708672903\\
2.02425060626516	145307.826423128\\
2.02435060876522	145351.944173353\\
2.02445061126528	145396.061923578\\
2.02455061376534	145440.179673803\\
2.02465061626541	145484.297424028\\
2.02475061876547	145528.415174253\\
2.02485062126553	145572.532924479\\
2.02495062376559	145616.650674704\\
2.02505062626566	145660.768424929\\
2.02515062876572	145704.886175154\\
2.02525063126578	145749.003925379\\
2.02535063376584	145793.121675604\\
2.02545063626591	145837.239425829\\
2.02555063876597	145881.357176054\\
2.02565064126603	145925.474926279\\
2.02575064376609	145969.592676504\\
2.02585064626616	146013.710426729\\
2.02595064876622	146057.828176954\\
2.02605065126628	146101.945927179\\
2.02615065376634	146146.063677404\\
2.02625065626641	146190.754385425\\
2.02635065876647	146234.87213565\\
2.02645066126653	146278.989885875\\
2.02655066376659	146323.1076361\\
2.02665066626666	146367.225386325\\
2.02675066876672	146411.34313655\\
2.02685067126678	146455.460886775\\
2.02695067376684	146500.151594795\\
2.02705067626691	146544.26934502\\
2.02715067876697	146588.387095245\\
2.02725068126703	146632.504845471\\
2.02735068376709	146676.622595696\\
2.02745068626716	146721.313303716\\
2.02755068876722	146765.431053941\\
2.02765069126728	146809.548804166\\
2.02775069376734	146853.666554391\\
2.02785069626741	146898.357262411\\
2.02795069876747	146942.475012636\\
2.02805070126753	146986.592762861\\
2.02815070376759	147031.283470882\\
2.02825070626766	147075.401221107\\
2.02835070876772	147119.518971332\\
2.02845071126778	147164.209679352\\
2.02855071376784	147208.327429577\\
2.02865071626791	147252.445179802\\
2.02875071876797	147297.135887822\\
2.02885072126803	147341.253638047\\
2.02895072376809	147385.371388272\\
2.02905072626816	147430.062096293\\
2.02915072876822	147474.179846518\\
2.02925073126828	147518.870554538\\
2.02935073376834	147562.988304763\\
2.02945073626841	147607.106054988\\
2.02955073876847	147651.796763008\\
2.02965074126853	147695.914513233\\
2.02975074376859	147740.605221254\\
2.02985074626866	147784.722971479\\
2.02995074876872	147829.413679499\\
2.03005075126878	147873.531429724\\
2.03015075376884	147918.222137744\\
2.03025075626891	147962.339887969\\
2.03035075876897	148007.030595989\\
2.03045076126903	148051.148346214\\
2.03055076376909	148095.839054235\\
2.03065076626916	148139.95680446\\
2.03075076876922	148184.64751248\\
2.03085077126928	148228.765262705\\
2.03095077376934	148273.455970725\\
2.03105077626941	148318.146678745\\
2.03115077876947	148362.26442897\\
2.03125078126953	148406.955136991\\
2.03135078376959	148451.645845011\\
2.03145078626966	148495.763595236\\
2.03155078876972	148540.454303256\\
2.03165079126978	148584.572053481\\
2.03175079376984	148629.262761501\\
2.03185079626991	148673.953469522\\
2.03195079876997	148718.071219747\\
2.03205080127003	148762.761927767\\
2.03215080377009	148807.452635787\\
2.03225080627016	148852.143343807\\
2.03235080877022	148896.261094032\\
2.03245081127028	148940.951802053\\
2.03255081377034	148985.642510073\\
2.03265081627041	149030.333218093\\
2.03275081877047	149074.450968318\\
2.03285082127053	149119.141676338\\
2.03295082377059	149163.832384359\\
2.03305082627066	149208.523092379\\
2.03315082877072	149252.640842604\\
2.03325083127078	149297.331550624\\
2.03335083377084	149342.022258644\\
2.03345083627091	149386.712966664\\
2.03355083877097	149431.403674685\\
2.03365084127103	149476.094382705\\
2.03375084377109	149520.21213293\\
2.03385084627116	149564.90284095\\
2.03395084877122	149609.59354897\\
2.03405085127128	149654.284256991\\
2.03415085377134	149698.974965011\\
2.03425085627141	149743.665673031\\
2.03435085877147	149788.356381051\\
2.03445086127153	149833.047089071\\
2.03455086377159	149877.737797092\\
2.03465086627166	149922.428505112\\
2.03475086877172	149967.119213132\\
2.03485087127178	150011.809921152\\
2.03495087377184	150056.500629172\\
2.03505087627191	150101.191337193\\
2.03515087877197	150145.882045213\\
2.03525088127203	150190.572753233\\
2.03535088377209	150235.263461253\\
2.03545088627216	150279.954169273\\
2.03555088877222	150324.644877294\\
2.03565089127228	150369.335585314\\
2.03575089377234	150414.026293334\\
2.03585089627241	150458.717001354\\
2.03595089877247	150503.407709374\\
2.03605090127253	150548.098417395\\
2.03615090377259	150592.789125415\\
2.03625090627266	150637.479833435\\
2.03635090877272	150682.170541455\\
2.03645091127278	150726.861249475\\
2.03655091377284	150772.124915291\\
2.03665091627291	150816.815623311\\
2.03675091877297	150861.506331331\\
2.03685092127303	150906.197039351\\
2.03695092377309	150950.887747372\\
2.03705092627316	150995.578455392\\
2.03715092877322	151040.842121207\\
2.03725093127328	151085.532829227\\
2.03735093377334	151130.223537248\\
2.03745093627341	151174.914245268\\
2.03755093877347	151220.177911083\\
2.03765094127353	151264.868619103\\
2.03775094377359	151309.559327123\\
2.03785094627366	151354.250035144\\
2.03795094877372	151399.513700959\\
2.03805095127378	151444.204408979\\
2.03815095377384	151488.895116999\\
2.03825095627391	151533.58582502\\
2.03835095877397	151578.849490835\\
2.03845096127403	151623.540198855\\
2.03855096377409	151668.230906875\\
2.03865096627416	151713.494572691\\
2.03875096877422	151758.185280711\\
2.03885097127428	151802.875988731\\
2.03895097377434	151848.139654546\\
2.03905097627441	151892.830362567\\
2.03915097877447	151938.094028382\\
2.03925098127453	151982.784736402\\
2.03935098377459	152027.475444422\\
2.03945098627466	152072.739110238\\
2.03955098877472	152117.429818258\\
2.03965099127478	152162.693484073\\
2.03975099377484	152207.384192093\\
2.03985099627491	152252.647857909\\
2.03995099877497	152297.338565929\\
2.04005100127503	152342.602231744\\
2.04015100377509	152387.292939765\\
2.04025100627516	152432.55660558\\
2.04035100877522	152477.2473136\\
2.04045101127528	152522.510979415\\
2.04055101377534	152567.201687436\\
2.04065101627541	152612.465353251\\
2.04075101877547	152657.156061271\\
2.04085102127553	152702.419727086\\
2.04095102377559	152747.110435107\\
2.04105102627566	152792.374100922\\
2.04115102877572	152837.637766737\\
2.04125103127578	152882.328474758\\
2.04135103377584	152927.592140573\\
2.04145103627591	152972.282848593\\
2.04155103877597	153017.546514408\\
2.04165104127603	153062.810180224\\
2.04175104377609	153107.500888244\\
2.04185104627616	153152.764554059\\
2.04195104877622	153198.028219875\\
2.04205105127628	153242.718927895\\
2.04215105377634	153287.98259371\\
2.04225105627641	153333.246259526\\
2.04235105877647	153377.936967546\\
2.04245106127653	153423.200633361\\
2.04255106377659	153468.464299176\\
2.04265106627666	153513.727964992\\
2.04275106877672	153558.418673012\\
2.04285107127678	153603.682338827\\
2.04295107377684	153648.946004643\\
2.04305107627691	153694.209670458\\
2.04315107877697	153738.900378478\\
2.04325108127703	153784.164044293\\
2.04335108377709	153829.427710109\\
2.04345108627716	153874.691375924\\
2.04355108877722	153919.955041739\\
2.04365109127728	153964.64574976\\
2.04375109377734	154009.909415575\\
2.04385109627741	154055.17308139\\
2.04395109877747	154100.436747206\\
2.04405110127753	154145.700413021\\
2.04415110377759	154190.964078836\\
2.04425110627766	154236.227744652\\
2.04435110877772	154280.918452672\\
2.04445111127778	154326.182118487\\
2.04455111377784	154371.445784303\\
2.04465111627791	154416.709450118\\
2.04475111877797	154461.973115933\\
2.04485112127803	154507.236781749\\
2.04495112377809	154552.500447564\\
2.04505112627816	154597.764113379\\
2.04515112877822	154643.027779195\\
2.04525113127828	154688.29144501\\
2.04535113377834	154733.555110825\\
2.04545113627841	154778.818776641\\
2.04555113877847	154824.082442456\\
2.04565114127853	154869.346108271\\
2.04575114377859	154914.609774087\\
2.04585114627866	154959.873439902\\
2.04595114877872	155005.137105717\\
2.04605115127878	155050.400771533\\
2.04615115377884	155095.664437348\\
2.04625115627891	155140.928103163\\
2.04635115877897	155186.191768979\\
2.04645116127903	155231.455434794\\
2.04655116377909	155277.292058404\\
2.04665116627916	155322.55572422\\
2.04675116877922	155367.819390035\\
2.04685117127928	155413.08305585\\
2.04695117377934	155458.346721666\\
2.04705117627941	155503.610387481\\
2.04715117877947	155548.874053296\\
2.04725118127953	155594.710676907\\
2.04735118377959	155639.974342722\\
2.04745118627966	155685.238008538\\
2.04755118877972	155730.501674353\\
2.04765119127978	155775.765340168\\
2.04775119377984	155821.601963779\\
2.04785119627991	155866.865629594\\
2.04795119877997	155912.129295409\\
2.04805120128003	155957.392961225\\
2.04815120378009	156003.229584835\\
2.04825120628016	156048.49325065\\
2.04835120878022	156093.756916466\\
2.04845121128028	156139.020582281\\
2.04855121378034	156184.857205892\\
2.04865121628041	156230.120871707\\
2.04875121878047	156275.384537522\\
2.04885122128053	156321.221161133\\
2.04895122378059	156366.484826948\\
2.04905122628066	156411.748492763\\
2.04915122878072	156457.585116374\\
2.04925123128078	156502.848782189\\
2.04935123378084	156548.112448005\\
2.04945123628091	156593.949071615\\
2.04955123878097	156639.21273743\\
2.04965124128103	156685.049361041\\
2.04975124378109	156730.313026856\\
2.04985124628116	156775.576692672\\
2.04995124878122	156821.413316282\\
2.05005125128128	156866.676982097\\
2.05015125378134	156912.513605708\\
2.05025125628141	156957.777271523\\
2.05035125878147	157003.613895134\\
2.05045126128153	157048.877560949\\
2.05055126378159	157094.714184559\\
2.05065126628166	157139.977850375\\
2.05075126878172	157185.814473985\\
2.05085127128178	157231.0781398\\
2.05095127378184	157276.914763411\\
2.05105127628191	157322.178429226\\
2.05115127878197	157368.015052837\\
2.05125128128203	157413.278718652\\
2.05135128378209	157459.115342263\\
2.05145128628216	157504.379008078\\
2.05155128878222	157550.215631688\\
2.05165129128228	157595.479297504\\
2.05175129378234	157641.315921114\\
2.05185129628241	157687.152544725\\
2.05195129878247	157732.41621054\\
2.05205130128253	157778.25283415\\
2.05215130378259	157824.089457761\\
2.05225130628266	157869.353123576\\
2.05235130878272	157915.189747187\\
2.05245131128278	157961.026370797\\
2.05255131378284	158006.290036613\\
2.05265131628291	158052.126660223\\
2.05275131878297	158097.963283833\\
2.05285132128303	158143.226949649\\
2.05295132378309	158189.063573259\\
2.05305132628316	158234.90019687\\
2.05315132878322	158280.163862685\\
2.05325133128328	158326.000486296\\
2.05335133378334	158371.837109906\\
2.05345133628341	158417.673733516\\
2.05355133878347	158462.937399332\\
2.05365134128353	158508.774022942\\
2.05375134378359	158554.610646553\\
2.05385134628366	158600.447270163\\
2.05395134878372	158646.283893774\\
2.05405135128378	158691.547559589\\
2.05415135378384	158737.384183199\\
2.05425135628391	158783.22080681\\
2.05435135878397	158829.05743042\\
2.05445136128403	158874.894054031\\
2.05455136378409	158920.730677641\\
2.05465136628416	158965.994343457\\
2.05475136878422	159011.830967067\\
2.05485137128428	159057.667590678\\
2.05495137378434	159103.504214288\\
2.05505137628441	159149.340837898\\
2.05515137878447	159195.177461509\\
2.05525138128453	159241.014085119\\
2.05535138378459	159286.85070873\\
2.05545138628466	159332.68733234\\
2.05555138878472	159378.523955951\\
2.05565139128478	159424.360579561\\
2.05575139378484	159470.197203172\\
2.05585139628491	159515.460868987\\
2.05595139878497	159561.297492598\\
2.05605140128503	159607.134116208\\
2.05615140378509	159652.970739818\\
2.05625140628516	159698.807363429\\
2.05635140878522	159745.216944835\\
2.05645141128528	159791.053568445\\
2.05655141378534	159836.890192055\\
2.05665141628541	159882.726815666\\
2.05675141878547	159928.563439276\\
2.05685142128553	159974.400062887\\
2.05695142378559	160020.236686497\\
2.05705142628566	160066.073310108\\
2.05715142878572	160111.909933718\\
2.05725143128578	160157.746557329\\
2.05735143378584	160203.583180939\\
2.05745143628591	160249.41980455\\
2.05755143878597	160295.25642816\\
2.05765144128603	160341.666009566\\
2.05775144378609	160387.502633176\\
2.05785144628616	160433.339256787\\
2.05795144878622	160479.175880397\\
2.05805145128628	160525.012504008\\
2.05815145378634	160570.849127618\\
2.05825145628641	160617.258709024\\
2.05835145878647	160663.095332634\\
2.05845146128653	160708.931956245\\
2.05855146378659	160754.768579855\\
2.05865146628666	160801.178161261\\
2.05875146878672	160847.014784871\\
2.05885147128678	160892.851408482\\
2.05895147378684	160938.688032092\\
2.05905147628691	160985.097613498\\
2.05915147878697	161030.934237108\\
2.05925148128703	161076.770860719\\
2.05935148378709	161122.607484329\\
2.05945148628716	161169.017065735\\
2.05955148878722	161214.853689345\\
2.05965149128728	161260.690312956\\
2.05975149378734	161307.099894361\\
2.05985149628741	161352.936517972\\
2.05995149878747	161398.773141582\\
2.06005150128753	161445.182722988\\
2.06015150378759	161491.019346598\\
2.06025150628766	161536.855970209\\
2.06035150878772	161583.265551614\\
2.06045151128778	161629.102175225\\
2.06055151378784	161675.51175663\\
2.06065151628791	161721.348380241\\
2.06075151878797	161767.185003851\\
2.06085152128803	161813.594585257\\
2.06095152378809	161859.431208867\\
2.06105152628816	161905.840790273\\
2.06115152878822	161951.677413883\\
2.06125153128828	161998.086995289\\
2.06135153378834	162043.923618899\\
2.06145153628841	162090.333200305\\
2.06155153878847	162136.169823915\\
2.06165154128853	162182.579405321\\
2.06175154378859	162228.416028932\\
2.06185154628866	162274.825610337\\
2.06195154878872	162320.662233948\\
2.06205155128878	162367.071815353\\
2.06215155378884	162412.908438964\\
2.06225155628891	162459.318020369\\
2.06235155878897	162505.15464398\\
2.06245156128903	162551.564225385\\
2.06255156378909	162597.973806791\\
2.06265156628916	162643.810430401\\
2.06275156878922	162690.220011807\\
2.06285157128928	162736.056635417\\
2.06295157378934	162782.466216823\\
2.06305157628941	162828.875798229\\
2.06315157878947	162874.712421839\\
2.06325158128953	162921.122003245\\
2.06335158378959	162967.53158465\\
2.06345158628966	163013.368208261\\
2.06355158878972	163059.777789666\\
2.06365159128978	163106.187371072\\
2.06375159378984	163152.023994682\\
2.06385159628991	163198.433576088\\
2.06395159878997	163244.843157494\\
2.06405160129003	163290.679781104\\
2.06415160379009	163337.08936251\\
2.06425160629016	163383.498943915\\
2.06435160879022	163429.908525321\\
2.06445161129028	163475.745148931\\
2.06455161379034	163522.154730337\\
2.06465161629041	163568.564311743\\
2.06475161879047	163614.973893148\\
2.06485162129053	163660.810516759\\
2.06495162379059	163707.220098164\\
2.06505162629066	163753.62967957\\
2.06515162879072	163800.039260975\\
2.06525163129078	163846.448842381\\
2.06535163379084	163892.858423787\\
2.06545163629091	163938.695047397\\
2.06555163879097	163985.104628803\\
2.06565164129103	164031.514210208\\
2.06575164379109	164077.923791614\\
2.06585164629116	164124.333373019\\
2.06595164879122	164170.742954425\\
2.06605165129128	164217.152535831\\
2.06615165379134	164262.989159441\\
2.06625165629141	164309.398740847\\
2.06635165879147	164355.808322252\\
2.06645166129153	164402.217903658\\
2.06655166379159	164448.627485064\\
2.06665166629166	164495.037066469\\
2.06675166879172	164541.446647875\\
2.06685167129178	164587.85622928\\
2.06695167379184	164634.265810686\\
2.06705167629191	164680.675392091\\
2.06715167879197	164727.084973497\\
2.06725168129203	164773.494554903\\
2.06735168379209	164819.904136308\\
2.06745168629216	164866.313717714\\
2.06755168879222	164912.723299119\\
2.06765169129228	164959.132880525\\
2.06775169379234	165005.542461931\\
2.06785169629241	165051.952043336\\
2.06795169879247	165098.361624742\\
2.06805170129253	165144.771206147\\
2.06815170379259	165191.180787553\\
2.06825170629266	165237.590368959\\
2.06835170879272	165284.572908159\\
2.06845171129278	165330.982489565\\
2.06855171379284	165377.392070971\\
2.06865171629291	165423.801652376\\
2.06875171879297	165470.211233782\\
2.06885172129303	165516.620815187\\
2.06895172379309	165563.030396593\\
2.06905172629316	165609.439977999\\
2.06915172879322	165656.422517199\\
2.06925173129328	165702.832098605\\
2.06935173379334	165749.24168001\\
2.06945173629341	165795.651261416\\
2.06955173879347	165842.060842822\\
2.06965174129353	165889.043382022\\
2.06975174379359	165935.452963428\\
2.06985174629366	165981.862544834\\
2.06995174879372	166028.272126239\\
2.07005175129378	166074.681707645\\
2.07015175379384	166121.664246845\\
2.07025175629391	166168.073828251\\
2.07035175879397	166214.483409657\\
2.07045176129403	166261.465948857\\
2.07055176379409	166307.875530263\\
2.07065176629416	166354.285111669\\
2.07075176879422	166400.694693074\\
2.07085177129428	166447.677232275\\
2.07095177379435	166494.086813681\\
2.07105177629441	166540.496395086\\
2.07115177879447	166587.478934287\\
2.07125178129453	166633.888515692\\
2.07135178379459	166680.298097098\\
2.07145178629466	166727.280636299\\
2.07155178879472	166773.690217704\\
2.07165179129478	166820.672756905\\
2.07175179379484	166867.082338311\\
2.07185179629491	166913.491919716\\
2.07195179879497	166960.474458917\\
2.07205180129503	167006.884040323\\
2.07215180379509	167053.866579523\\
2.07225180629516	167100.276160929\\
2.07235180879522	167146.685742335\\
2.07245181129528	167193.668281535\\
2.07255181379534	167240.077862941\\
2.07265181629541	167287.060402142\\
2.07275181879547	167333.469983547\\
2.07285182129553	167380.452522748\\
2.0729518237956	167426.862104154\\
2.07305182629566	167473.844643354\\
2.07315182879572	167520.25422476\\
2.07325183129578	167567.236763961\\
2.07335183379584	167613.646345366\\
2.07345183629591	167660.628884567\\
2.07355183879597	167707.038465972\\
2.07365184129603	167754.021005173\\
2.07375184379609	167801.003544374\\
2.07385184629616	167847.41312578\\
2.07395184879622	167894.39566498\\
2.07405185129628	167940.805246386\\
2.07415185379634	167987.787785587\\
2.07425185629641	168034.197366992\\
2.07435185879647	168081.179906193\\
2.07445186129653	168128.162445394\\
2.07455186379659	168174.572026799\\
2.07465186629666	168221.554566\\
2.07475186879672	168268.537105201\\
2.07485187129678	168314.946686606\\
2.07495187379685	168361.929225807\\
2.07505187629691	168408.911765008\\
2.07515187879697	168455.321346413\\
2.07525188129703	168502.303885614\\
2.0753518837971	168549.286424815\\
2.07545188629716	168595.69600622\\
2.07555188879722	168642.678545421\\
2.07565189129728	168689.661084622\\
2.07575189379734	168736.070666027\\
2.07585189629741	168783.053205228\\
2.07595189879747	168830.035744429\\
2.07605190129753	168877.01828363\\
2.07615190379759	168923.427865035\\
2.07625190629766	168970.410404236\\
2.07635190879772	169017.392943437\\
2.07645191129778	169064.375482637\\
2.07655191379784	169111.358021838\\
2.07665191629791	169157.767603244\\
2.07675191879797	169204.750142444\\
2.07685192129803	169251.732681645\\
2.0769519237981	169298.715220846\\
2.07705192629816	169345.697760047\\
2.07715192879822	169392.680299247\\
2.07725193129828	169439.089880653\\
2.07735193379835	169486.072419854\\
2.07745193629841	169533.054959054\\
2.07755193879847	169580.037498255\\
2.07765194129853	169627.020037456\\
2.07775194379859	169674.002576657\\
2.07785194629866	169720.985115857\\
2.07795194879872	169767.967655058\\
2.07805195129878	169814.950194259\\
2.07815195379884	169861.359775664\\
2.07825195629891	169908.342314865\\
2.07835195879897	169955.324854066\\
2.07845196129903	170002.307393267\\
2.07855196379909	170049.289932467\\
2.07865196629916	170096.272471668\\
2.07875196879922	170143.255010869\\
2.07885197129928	170190.237550069\\
2.07895197379935	170237.22008927\\
2.07905197629941	170284.202628471\\
2.07915197879947	170331.185167672\\
2.07925198129953	170378.167706872\\
2.0793519837996	170425.150246073\\
2.07945198629966	170472.132785274\\
2.07955198879972	170519.115324475\\
2.07965199129978	170566.097863675\\
2.07975199379984	170613.653360671\\
2.07985199629991	170660.635899872\\
2.07995199879997	170707.618439073\\
2.08005200130003	170754.600978273\\
2.08015200380009	170801.583517474\\
2.08025200630016	170848.566056675\\
2.08035200880022	170895.548595876\\
2.08045201130028	170942.531135076\\
2.08055201380034	170989.513674277\\
2.08065201630041	171036.496213478\\
2.08075201880047	171084.051710474\\
2.08085202130053	171131.034249674\\
2.0809520238006	171178.016788875\\
2.08105202630066	171224.999328076\\
2.08115202880072	171271.981867276\\
2.08125203130078	171318.964406477\\
2.08135203380085	171366.519903473\\
2.08145203630091	171413.502442674\\
2.08155203880097	171460.484981874\\
2.08165204130103	171507.467521075\\
2.0817520438011	171554.450060276\\
2.08185204630116	171602.005557272\\
2.08195204880122	171648.988096473\\
2.08205205130128	171695.970635673\\
2.08215205380134	171742.953174874\\
2.08225205630141	171790.50867187\\
2.08235205880147	171837.491211071\\
2.08245206130153	171884.473750271\\
2.08255206380159	171932.029247267\\
2.08265206630166	171979.011786468\\
2.08275206880172	172025.994325669\\
2.08285207130178	172073.549822664\\
2.08295207380185	172120.532361865\\
2.08305207630191	172167.514901066\\
2.08315207880197	172215.070398062\\
2.08325208130203	172262.052937263\\
2.0833520838021	172309.035476463\\
2.08345208630216	172356.590973459\\
2.08355208880222	172403.57351266\\
2.08365209130228	172450.556051861\\
2.08375209380235	172498.111548856\\
2.08385209630241	172545.094088057\\
2.08395209880247	172592.649585053\\
2.08405210130253	172639.632124254\\
2.08415210380259	172686.614663454\\
2.08425210630266	172734.17016045\\
2.08435210880272	172781.152699651\\
2.08445211130278	172828.708196647\\
2.08455211380284	172875.690735848\\
2.08465211630291	172923.246232843\\
2.08475211880297	172970.228772044\\
2.08485212130303	173017.78426904\\
2.0849521238031	173064.766808241\\
2.08505212630316	173112.322305237\\
2.08515212880322	173159.304844437\\
2.08525213130328	173206.860341433\\
2.08535213380335	173253.842880634\\
2.08545213630341	173301.39837763\\
2.08555213880347	173348.380916831\\
2.08565214130353	173395.936413826\\
2.0857521438036	173442.918953027\\
2.08585214630366	173490.474450023\\
2.08595214880372	173537.456989224\\
2.08605215130378	173585.01248622\\
2.08615215380384	173632.567983215\\
2.08625215630391	173679.550522416\\
2.08635215880397	173727.106019412\\
2.08645216130403	173774.088558613\\
2.08655216380409	173821.644055609\\
2.08665216630416	173869.199552605\\
2.08675216880422	173916.182091805\\
2.08685217130428	173963.737588801\\
2.08695217380435	174010.720128002\\
2.08705217630441	174058.275624998\\
2.08715217880447	174105.831121994\\
2.08725218130453	174152.813661194\\
2.0873521838046	174200.36915819\\
2.08745218630466	174247.924655186\\
2.08755218880472	174295.480152182\\
2.08765219130478	174342.462691383\\
2.08775219380485	174390.018188378\\
2.08785219630491	174437.573685374\\
2.08795219880497	174484.556224575\\
2.08805220130503	174532.111721571\\
2.0881522038051	174579.667218567\\
2.08825220630516	174627.222715563\\
2.08835220880522	174674.205254763\\
2.08845221130528	174721.760751759\\
2.08855221380534	174769.316248755\\
2.08865221630541	174816.871745751\\
2.08875221880547	174863.854284952\\
2.08885222130553	174911.409781947\\
2.0889522238056	174958.965278943\\
2.08905222630566	175006.520775939\\
2.08915222880572	175054.076272935\\
2.08925223130578	175101.631769931\\
2.08935223380585	175148.614309132\\
2.08945223630591	175196.169806127\\
2.08955223880597	175243.725303123\\
2.08965224130603	175291.280800119\\
2.0897522438061	175338.836297115\\
2.08985224630616	175386.391794111\\
2.08995224880622	175433.947291107\\
2.09005225130628	175480.929830307\\
2.09015225380635	175528.485327303\\
2.09025225630641	175576.040824299\\
2.09035225880647	175623.596321295\\
2.09045226130653	175671.151818291\\
2.09055226380659	175718.707315287\\
2.09065226630666	175766.262812283\\
2.09075226880672	175813.818309279\\
2.09085227130678	175861.373806274\\
2.09095227380685	175908.92930327\\
2.09105227630691	175956.484800266\\
2.09115227880697	176004.040297262\\
2.09125228130703	176051.595794258\\
2.0913522838071	176099.151291254\\
2.09145228630716	176146.70678825\\
2.09155228880722	176194.262285245\\
2.09165229130728	176241.817782241\\
2.09175229380735	176289.373279237\\
2.09185229630741	176336.928776233\\
2.09195229880747	176384.484273229\\
2.09205230130753	176432.039770225\\
2.0921523038076	176479.595267221\\
2.09225230630766	176527.150764216\\
2.09235230880772	176574.706261212\\
2.09245231130778	176622.261758208\\
2.09255231380785	176669.817255204\\
2.09265231630791	176717.3727522\\
2.09275231880797	176764.928249196\\
2.09285232130803	176813.056703987\\
2.0929523238081	176860.612200983\\
2.09305232630816	176908.167697978\\
2.09315232880822	176955.723194974\\
2.09325233130828	177003.27869197\\
2.09335233380835	177050.834188966\\
2.09345233630841	177098.389685962\\
2.09355233880847	177146.518140753\\
2.09365234130853	177194.073637749\\
2.0937523438086	177241.629134745\\
2.09385234630866	177289.18463174\\
2.09395234880872	177336.740128736\\
2.09405235130878	177384.295625732\\
2.09415235380885	177432.424080523\\
2.09425235630891	177479.979577519\\
2.09435235880897	177527.535074515\\
2.09445236130903	177575.090571511\\
2.0945523638091	177623.219026302\\
2.09465236630916	177670.774523297\\
2.09475236880922	177718.330020293\\
2.09485237130928	177765.885517289\\
2.09495237380935	177814.01397208\\
2.09505237630941	177861.569469076\\
2.09515237880947	177909.124966072\\
2.09525238130953	177956.680463068\\
2.0953523838096	178004.808917859\\
2.09545238630966	178052.364414855\\
2.09555238880972	178099.919911851\\
2.09565239130978	178148.048366641\\
2.09575239380985	178195.603863637\\
2.09585239630991	178243.159360633\\
2.09595239880997	178291.287815424\\
2.09605240131003	178338.84331242\\
2.0961524038101	178386.398809416\\
2.09625240631016	178434.527264207\\
2.09635240881022	178482.082761203\\
2.09645241131028	178530.211215994\\
2.09655241381035	178577.76671299\\
2.09665241631041	178625.322209985\\
2.09675241881047	178673.450664776\\
2.09685242131053	178721.006161772\\
2.0969524238106	178769.134616563\\
2.09705242631066	178816.690113559\\
2.09715242881072	178864.245610555\\
2.09725243131078	178912.374065346\\
2.09735243381085	178959.929562342\\
2.09745243631091	179008.058017133\\
2.09755243881097	179055.613514129\\
2.09765244131103	179103.74196892\\
2.0977524438111	179151.297465916\\
2.09785244631116	179199.425920707\\
2.09795244881122	179246.981417702\\
2.09805245131128	179295.109872493\\
2.09815245381135	179342.665369489\\
2.09825245631141	179390.79382428\\
2.09835245881147	179438.349321276\\
2.09845246131153	179486.477776067\\
2.0985524638116	179534.033273063\\
2.09865246631166	179582.161727854\\
2.09875246881172	179629.71722485\\
2.09885247131178	179677.845679641\\
2.09895247381185	179725.974134432\\
2.09905247631191	179773.529631428\\
2.09915247881197	179821.658086219\\
2.09925248131203	179869.213583214\\
2.0993524838121	179917.342038005\\
2.09945248631216	179965.470492796\\
2.09955248881222	180013.025989792\\
2.09965249131228	180061.154444583\\
2.09975249381235	180108.709941579\\
2.09985249631241	180156.83839637\\
2.09995249881247	180204.966851161\\
2.10005250131253	180252.522348157\\
2.1001525038126	180300.650802948\\
2.10025250631266	180348.779257739\\
2.10035250881272	180396.334754735\\
2.10045251131278	180444.463209526\\
2.10055251381285	180492.591664317\\
2.10065251631291	180540.147161313\\
2.10075251881297	180588.275616104\\
2.10085252131303	180636.404070895\\
2.1009525238131	180684.532525686\\
2.10105252631316	180732.088022681\\
2.10115252881322	180780.216477472\\
2.10125253131328	180828.344932263\\
2.10135253381335	180875.900429259\\
2.10145253631341	180924.02888405\\
2.10155253881347	180972.157338841\\
2.10165254131353	181020.285793632\\
2.1017525438136	181067.841290628\\
2.10185254631366	181115.969745419\\
2.10195254881372	181164.09820021\\
2.10205255131378	181212.226655001\\
2.10215255381385	181260.355109792\\
2.10225255631391	181307.910606788\\
2.10235255881397	181356.039061579\\
2.10245256131403	181404.16751637\\
2.1025525638141	181452.295971161\\
2.10265256631416	181500.424425952\\
2.10275256881422	181548.552880743\\
2.10285257131428	181596.681335534\\
2.10295257381435	181644.23683253\\
2.10305257631441	181692.365287321\\
2.10315257881447	181740.493742112\\
2.10325258131453	181788.622196903\\
2.1033525838146	181836.750651694\\
2.10345258631466	181884.879106485\\
2.10355258881472	181933.007561276\\
2.10365259131478	181981.136016067\\
2.10375259381485	182029.264470858\\
2.10385259631491	182076.819967854\\
2.10395259881497	182124.948422645\\
2.10405260131503	182173.076877435\\
2.1041526038151	182221.205332226\\
2.10425260631516	182269.333787017\\
2.10435260881522	182317.462241808\\
2.10445261131528	182365.590696599\\
2.10455261381535	182413.71915139\\
2.10465261631541	182461.847606181\\
2.10475261881547	182509.976060972\\
2.10485262131553	182558.104515763\\
2.1049526238156	182606.232970554\\
2.10505262631566	182654.361425345\\
2.10515262881572	182702.489880136\\
2.10525263131578	182750.618334927\\
2.10535263381585	182798.746789718\\
2.10545263631591	182846.875244509\\
2.10555263881597	182895.0036993\\
2.10565264131603	182943.132154091\\
2.1057526438161	182991.260608882\\
2.10585264631616	183039.389063673\\
2.10595264881622	183088.090476259\\
2.10605265131628	183136.21893105\\
2.10615265381635	183184.347385841\\
2.10625265631641	183232.475840632\\
2.10635265881647	183280.604295423\\
2.10645266131653	183328.732750214\\
2.1065526638166	183376.861205005\\
2.10665266631666	183424.989659796\\
2.10675266881672	183473.118114587\\
2.10685267131678	183521.819527173\\
2.10695267381685	183569.947981964\\
2.10705267631691	183618.076436755\\
2.10715267881697	183666.204891546\\
2.10725268131703	183714.333346337\\
2.1073526838171	183762.461801128\\
2.10745268631716	183811.163213715\\
2.10755268881722	183859.291668506\\
2.10765269131728	183907.420123296\\
2.10775269381735	183955.548578088\\
2.10785269631741	184003.677032878\\
2.10795269881747	184052.378445465\\
2.10805270131753	184100.506900256\\
2.1081527038176	184148.635355047\\
2.10825270631766	184196.763809838\\
2.10835270881772	184245.465222424\\
2.10845271131778	184293.593677215\\
2.10855271381785	184341.722132006\\
2.10865271631791	184389.850586797\\
2.10875271881797	184438.551999383\\
2.10885272131803	184486.680454174\\
2.1089527238181	184534.808908965\\
2.10905272631816	184582.937363756\\
2.10915272881822	184631.638776342\\
2.10925273131828	184679.767231133\\
2.10935273381835	184727.895685924\\
2.10945273631841	184776.59709851\\
2.10955273881847	184824.725553301\\
2.10965274131853	184872.854008092\\
2.1097527438186	184921.555420678\\
2.10985274631866	184969.683875469\\
2.10995274881872	185017.81233026\\
2.11005275131878	185066.513742846\\
2.11015275381885	185114.642197637\\
2.11025275631891	185163.343610223\\
2.11035275881897	185211.472065014\\
2.11045276131903	185259.600519805\\
2.1105527638191	185308.301932391\\
2.11065276631916	185356.430387182\\
2.11075276881922	185405.131799768\\
2.11085277131928	185453.260254559\\
2.11095277381935	185501.38870935\\
2.11105277631941	185550.090121937\\
2.11115277881947	185598.218576728\\
2.11125278131953	185646.919989314\\
2.1113527838196	185695.048444105\\
2.11145278631966	185743.749856691\\
2.11155278881972	185791.878311482\\
2.11165279131978	185840.579724068\\
2.11175279381985	185888.708178859\\
2.11185279631991	185936.83663365\\
2.11195279881997	185985.538046236\\
2.11205280132003	186033.666501027\\
2.1121528038201	186082.367913613\\
2.11225280632016	186130.496368404\\
2.11235280882022	186179.19778099\\
2.11245281132028	186227.899193576\\
2.11255281382035	186276.027648367\\
2.11265281632041	186324.729060953\\
2.11275281882047	186372.857515744\\
2.11285282132053	186421.558928331\\
2.1129528238206	186469.687383122\\
2.11305282632066	186518.388795708\\
2.11315282882072	186566.517250499\\
2.11325283132078	186615.218663085\\
2.11335283382085	186663.920075671\\
2.11345283632091	186712.048530462\\
2.11355283882097	186760.749943048\\
2.11365284132103	186808.878397839\\
2.1137528438211	186857.579810425\\
2.11385284632116	186906.281223011\\
2.11395284882122	186954.409677802\\
2.11405285132128	187003.111090388\\
2.11415285382135	187051.812502974\\
2.11425285632141	187099.940957765\\
2.11435285882147	187148.642370352\\
2.11445286132153	187197.343782938\\
2.1145528638216	187245.472237729\\
2.11465286632166	187294.173650315\\
2.11475286882172	187342.875062901\\
2.11485287132178	187391.003517692\\
2.11495287382185	187439.704930278\\
2.11505287632191	187488.406342864\\
2.11515287882197	187536.534797655\\
2.11525288132203	187585.236210241\\
2.1153528838221	187633.937622827\\
2.11545288632216	187682.066077618\\
2.11555288882222	187730.767490204\\
2.11565289132228	187779.468902791\\
2.11575289382235	187828.170315377\\
2.11585289632241	187876.298770168\\
2.11595289882247	187925.000182754\\
2.11605290132253	187973.70159534\\
2.1161529038226	188022.403007926\\
2.11625290632266	188071.104420512\\
2.11635290882272	188119.232875303\\
2.11645291132278	188167.934287889\\
2.11655291382285	188216.635700475\\
2.11665291632291	188265.337113062\\
2.11675291882297	188314.038525648\\
2.11685292132303	188362.166980439\\
2.1169529238231	188410.868393025\\
2.11705292632316	188459.569805611\\
2.11715292882322	188508.271218197\\
2.11725293132328	188556.972630783\\
2.11735293382335	188605.674043369\\
2.11745293632341	188653.80249816\\
2.11755293882347	188702.503910746\\
2.11765294132353	188751.205323332\\
2.1177529438236	188799.906735919\\
2.11785294632366	188848.608148505\\
2.11795294882372	188897.309561091\\
2.11805295132378	188946.010973677\\
2.11815295382385	188994.712386263\\
2.11825295632391	189043.413798849\\
2.11835295882397	189091.54225364\\
2.11845296132403	189140.243666226\\
2.1185529638241	189188.945078812\\
2.11865296632416	189237.646491399\\
2.11875296882422	189286.347903985\\
2.11885297132428	189335.049316571\\
2.11895297382435	189383.750729157\\
2.11905297632441	189432.452141743\\
2.11915297882447	189481.153554329\\
2.11925298132453	189529.854966915\\
2.1193529838246	189578.556379501\\
2.11945298632466	189627.257792088\\
2.11955298882472	189675.959204674\\
2.11965299132478	189724.66061726\\
2.11975299382485	189773.362029846\\
2.11985299632491	189822.063442432\\
2.11995299882497	189870.764855018\\
2.12005300132503	189919.466267604\\
2.1201530038251	189968.16768019\\
2.12025300632516	190016.869092776\\
2.12035300882522	190065.570505363\\
2.12045301132528	190114.271917949\\
2.12055301382535	190162.973330535\\
2.12065301632541	190211.674743121\\
2.12075301882547	190260.376155707\\
2.12085302132553	190309.077568293\\
2.1209530238256	190358.351938674\\
2.12105302632566	190407.053351261\\
2.12115302882572	190455.754763847\\
2.12125303132578	190504.456176433\\
2.12135303382585	190553.157589019\\
2.12145303632591	190601.859001605\\
2.12155303882597	190650.560414191\\
2.12165304132603	190699.261826777\\
2.1217530438261	190747.963239363\\
2.12185304632616	190797.237609745\\
2.12195304882622	190845.939022331\\
2.12205305132628	190894.640434917\\
2.12215305382635	190943.341847503\\
2.12225305632641	190992.043260089\\
2.12235305882647	191040.744672675\\
2.12245306132653	191090.019043056\\
2.1225530638266	191138.720455643\\
2.12265306632666	191187.421868229\\
2.12275306882672	191236.123280815\\
2.12285307132678	191284.824693401\\
2.12295307382685	191334.099063782\\
2.12305307632691	191382.800476368\\
2.12315307882697	191431.501888954\\
2.12325308132703	191480.203301541\\
2.1233530838271	191528.904714127\\
2.12345308632716	191578.179084508\\
2.12355308882722	191626.880497094\\
2.12365309132728	191675.58190968\\
2.12375309382735	191724.283322266\\
2.12385309632741	191773.557692648\\
2.12395309882747	191822.259105234\\
2.12405310132753	191870.96051782\\
2.1241531038276	191920.234888201\\
2.12425310632766	191968.936300787\\
2.12435310882772	192017.637713373\\
2.12445311132778	192066.339125959\\
2.12455311382785	192115.613496341\\
2.12465311632791	192164.314908927\\
2.12475311882797	192213.016321513\\
2.12485312132803	192262.290691894\\
2.1249531238281	192310.99210448\\
2.12505312632816	192359.693517066\\
2.12515312882822	192408.967887448\\
2.12525313132828	192457.669300034\\
2.12535313382835	192506.37071262\\
2.12545313632841	192555.645083001\\
2.12555313882847	192604.346495587\\
2.12565314132853	192653.620865969\\
2.1257531438286	192702.322278555\\
2.12585314632866	192751.023691141\\
2.12595314882872	192800.298061522\\
2.12605315132878	192848.999474108\\
2.12615315382885	192898.273844489\\
2.12625315632891	192946.975257076\\
2.12635315882897	192995.676669662\\
2.12645316132903	193044.951040043\\
2.1265531638291	193093.652452629\\
2.12665316632916	193142.92682301\\
2.12675316882922	193191.628235596\\
2.12685317132928	193240.902605978\\
2.12695317382935	193289.604018564\\
2.12705317632941	193338.30543115\\
2.12715317882947	193387.579801531\\
2.12725318132953	193436.281214117\\
2.1273531838296	193485.555584499\\
2.12745318632966	193534.256997085\\
2.12755318882972	193583.531367466\\
2.12765319132978	193632.232780052\\
2.12775319382985	193681.507150433\\
2.12785319632991	193730.208563019\\
2.12795319882997	193779.482933401\\
2.12805320133003	193828.184345987\\
2.1281532038301	193877.458716368\\
2.12825320633016	193926.160128954\\
2.12835320883022	193975.434499335\\
2.12845321133028	194024.708869717\\
2.12855321383035	194073.410282303\\
2.12865321633041	194122.684652684\\
2.12875321883047	194171.38606527\\
2.12885322133053	194220.660435651\\
2.1289532238306	194269.361848237\\
2.12905322633066	194318.636218619\\
2.12915322883072	194367.337631205\\
2.12925323133078	194416.612001586\\
2.12935323383085	194465.886371967\\
2.12945323633091	194514.587784553\\
2.12955323883097	194563.862154935\\
2.12965324133103	194613.136525316\\
2.1297532438311	194661.837937902\\
2.12985324633116	194711.112308283\\
2.12995324883122	194759.813720869\\
2.13005325133128	194809.088091251\\
2.13015325383135	194858.362461632\\
2.13025325633141	194907.063874218\\
2.13035325883147	194956.338244599\\
2.13045326133153	195005.612614981\\
2.1305532638316	195054.314027567\\
2.13065326633166	195103.588397948\\
2.13075326883172	195152.862768329\\
2.13085327133178	195201.564180915\\
2.13095327383185	195250.838551297\\
2.13105327633191	195300.112921678\\
2.13115327883197	195348.814334264\\
2.13125328133203	195398.088704645\\
2.1313532838321	195447.363075026\\
2.13145328633216	195496.637445408\\
2.13155328883222	195545.338857994\\
2.13165329133228	195594.613228375\\
2.13175329383235	195643.887598756\\
2.13185329633241	195693.161969138\\
2.13195329883247	195741.863381724\\
2.13205330133253	195791.137752105\\
2.1321533038326	195840.412122486\\
2.13225330633266	195889.686492867\\
2.13235330883272	195938.387905454\\
2.13245331133278	195987.662275835\\
2.13255331383285	196036.936646216\\
2.13265331633291	196086.211016597\\
2.13275331883297	196134.912429183\\
2.13285332133303	196184.186799565\\
2.1329533238331	196233.461169946\\
2.13305332633316	196282.735540327\\
2.13315332883322	196332.009910708\\
2.13325333133328	196381.28428109\\
2.13335333383335	196429.985693676\\
2.13345333633341	196479.260064057\\
2.13355333883347	196528.534434438\\
2.13365334133353	196577.80880482\\
2.1337533438336	196627.083175201\\
2.13385334633366	196676.357545582\\
2.13395334883372	196725.058958168\\
2.13405335133378	196774.333328549\\
2.13415335383385	196823.607698931\\
2.13425335633391	196872.882069312\\
2.13435335883397	196922.156439693\\
2.13445336133403	196971.430810074\\
2.1345533638341	197020.705180456\\
2.13465336633416	197069.979550837\\
2.13475336883422	197119.253921218\\
2.13485337133428	197167.955333804\\
2.13495337383435	197217.229704186\\
2.13505337633441	197266.504074567\\
2.13515337883447	197315.778444948\\
2.13525338133453	197365.052815329\\
2.1353533838346	197414.327185711\\
2.13545338633466	197463.601556092\\
2.13555338883472	197512.875926473\\
2.13565339133478	197562.150296854\\
2.13575339383485	197611.424667236\\
2.13585339633491	197660.699037617\\
2.13595339883497	197709.973407998\\
2.13605340133503	197759.247778379\\
2.1361534038351	197808.522148761\\
2.13625340633516	197857.796519142\\
2.13635340883522	197907.070889523\\
2.13645341133528	197956.345259904\\
2.13655341383535	198005.619630286\\
2.13665341633541	198054.894000667\\
2.13675341883547	198104.168371048\\
2.13685342133553	198153.442741429\\
2.1369534238356	198202.717111811\\
2.13705342633566	198251.991482192\\
2.13715342883572	198301.265852573\\
2.13725343133578	198350.540222954\\
2.13735343383585	198399.814593336\\
2.13745343633591	198449.088963717\\
2.13755343883597	198498.363334098\\
2.13765344133603	198547.637704479\\
2.1377534438361	198596.912074861\\
2.13785344633616	198646.186445242\\
2.13795344883622	198695.460815623\\
2.13805345133628	198745.308143799\\
2.13815345383635	198794.582514181\\
2.13825345633641	198843.856884562\\
2.13835345883647	198893.131254943\\
2.13845346133653	198942.405625324\\
2.1385534638366	198991.679995706\\
2.13865346633666	199040.954366087\\
2.13875346883672	199090.228736468\\
2.13885347133678	199139.503106849\\
2.13895347383685	199189.350435026\\
2.13905347633691	199238.624805407\\
2.13915347883697	199287.899175788\\
2.13925348133703	199337.17354617\\
2.1393534838371	199386.447916551\\
2.13945348633716	199435.722286932\\
2.13955348883722	199485.569615108\\
2.13965349133728	199534.84398549\\
2.13975349383735	199584.118355871\\
2.13985349633741	199633.392726252\\
2.13995349883747	199682.667096633\\
2.14005350133753	199731.941467015\\
2.1401535038376	199781.788795191\\
2.14025350633766	199831.063165572\\
2.14035350883772	199880.337535954\\
2.14045351133778	199929.611906335\\
2.14055351383785	199979.459234511\\
2.14065351633791	200028.733604892\\
2.14075351883797	200078.007975274\\
2.14085352133803	200127.282345655\\
2.1409535238381	200176.556716036\\
2.14105352633816	200226.404044213\\
2.14115352883822	200275.678414594\\
2.14125353133828	200324.952784975\\
2.14135353383835	200374.227155356\\
2.14145353633841	200424.074483533\\
2.14155353883847	200473.348853914\\
2.14165354133853	200522.623224295\\
2.1417535438386	200572.470552472\\
2.14185354633866	200621.744922853\\
2.14195354883872	200671.019293234\\
2.14205355133878	200720.293663615\\
2.14215355383885	200770.140991792\\
2.14225355633891	200819.415362173\\
2.14235355883897	200868.689732554\\
2.14245356133903	200918.537060731\\
2.1425535638391	200967.811431112\\
2.14265356633916	201017.085801493\\
2.14275356883922	201066.93312967\\
2.14285357133928	201116.207500051\\
2.14295357383935	201165.481870432\\
2.14305357633941	201215.329198608\\
2.14315357883947	201264.60356899\\
2.14325358133953	201313.877939371\\
2.1433535838396	201363.725267547\\
2.14345358633966	201412.999637929\\
2.14355358883972	201462.846966105\\
2.14365359133978	201512.121336486\\
2.14375359383985	201561.395706867\\
2.14385359633991	201611.243035044\\
2.14395359883997	201660.517405425\\
2.14405360134003	201710.364733601\\
2.1441536038401	201759.639103983\\
2.14425360634016	201808.913474364\\
2.14435360884022	201858.76080254\\
2.14445361134028	201908.035172922\\
2.14455361384035	201957.882501098\\
2.14465361634041	202007.156871479\\
2.14475361884047	202057.004199656\\
2.14485362134053	202106.278570037\\
2.1449536238406	202155.552940418\\
2.14505362634066	202205.400268594\\
2.14515362884072	202254.674638976\\
2.14525363134078	202304.521967152\\
2.14535363384085	202353.796337533\\
2.14545363634091	202403.64366571\\
2.14555363884097	202452.918036091\\
2.14565364134103	202502.765364267\\
2.1457536438411	202552.039734649\\
2.14585364634116	202601.887062825\\
2.14595364884122	202651.161433206\\
2.14605365134128	202701.008761383\\
2.14615365384135	202750.283131764\\
2.14625365634141	202800.13045994\\
2.14635365884147	202849.404830322\\
2.14645366134153	202899.252158498\\
2.1465536638416	202948.526528879\\
2.14665366634166	202998.373857056\\
2.14675366884172	203047.648227437\\
2.14685367134178	203097.495555613\\
2.14695367384185	203146.769925994\\
2.14705367634191	203196.617254171\\
2.14715367884197	203246.464582347\\
2.14725368134203	203295.738952728\\
2.1473536838421	203345.586280905\\
2.14745368634216	203394.860651286\\
2.14755368884222	203444.707979462\\
2.14765369134228	203493.982349844\\
2.14775369384235	203543.82967802\\
2.14785369634241	203593.677006196\\
2.14795369884247	203642.951376578\\
2.14805370134253	203692.798704754\\
2.1481537038426	203742.073075135\\
2.14825370634266	203791.920403312\\
2.14835370884272	203841.767731488\\
2.14845371134278	203891.042101869\\
2.14855371384285	203940.889430046\\
2.14865371634291	203990.163800427\\
2.14875371884297	204040.011128603\\
2.14885372134303	204089.85845678\\
2.1489537238431	204139.132827161\\
2.14905372634316	204188.980155337\\
2.14915372884322	204238.827483514\\
2.14925373134328	204288.101853895\\
2.14935373384335	204337.949182071\\
2.14945373634341	204387.796510248\\
2.14955373884347	204437.070880629\\
2.14965374134353	204486.918208805\\
2.1497537438436	204536.765536982\\
2.14985374634366	204586.039907363\\
2.14995374884372	204635.887235539\\
2.15005375134378	204685.734563716\\
2.15015375384385	204735.008934097\\
2.15025375634391	204784.856262273\\
2.15035375884397	204834.70359045\\
2.15045376134403	204884.550918626\\
2.1505537638441	204933.825289007\\
2.15065376634416	204983.672617184\\
2.15075376884422	205033.51994536\\
2.15085377134428	205082.794315741\\
2.15095377384435	205132.641643918\\
2.15105377634441	205182.488972094\\
2.15115377884447	205232.336300271\\
2.15125378134453	205281.610670652\\
2.1513537838446	205331.457998828\\
2.15145378634466	205381.305327005\\
2.15155378884472	205431.152655181\\
2.15165379134478	205480.427025562\\
2.15175379384485	205530.274353739\\
2.15185379634491	205580.121681915\\
2.15195379884497	205629.969010091\\
2.15205380134503	205679.816338268\\
2.1521538038451	205729.090708649\\
2.15225380634516	205778.938036825\\
2.15235380884522	205828.785365002\\
2.15245381134528	205878.632693178\\
2.15255381384535	205928.480021355\\
2.15265381634541	205977.754391736\\
2.15275381884547	206027.601719912\\
2.15285382134553	206077.449048089\\
2.1529538238456	206127.296376265\\
2.15305382634566	206177.143704441\\
2.15315382884572	206226.991032618\\
2.15325383134578	206276.265402999\\
2.15335383384585	206326.112731175\\
2.15345383634591	206375.960059352\\
2.15355383884597	206425.807387528\\
2.15365384134603	206475.654715705\\
2.1537538438461	206525.502043881\\
2.15385384634616	206575.349372057\\
2.15395384884622	206624.623742439\\
2.15405385134628	206674.471070615\\
2.15415385384635	206724.318398791\\
2.15425385634641	206774.165726968\\
2.15435385884647	206824.013055144\\
2.15445386134653	206873.86038332\\
2.1545538638466	206923.707711497\\
2.15465386634666	206973.555039673\\
2.15475386884672	207023.40236785\\
2.15485387134678	207073.249696026\\
2.15495387384685	207122.524066407\\
2.15505387634691	207172.371394584\\
2.15515387884697	207222.21872276\\
2.15525388134703	207272.066050936\\
2.1553538838471	207321.913379113\\
2.15545388634716	207371.760707289\\
2.15555388884722	207421.608035465\\
2.15565389134728	207471.455363642\\
2.15575389384735	207521.302691818\\
2.15585389634741	207571.150019995\\
2.15595389884747	207620.997348171\\
2.15605390134753	207670.844676347\\
2.1561539038476	207720.692004524\\
2.15625390634766	207770.5393327\\
2.15635390884772	207820.386660877\\
2.15645391134778	207870.233989053\\
2.15655391384785	207920.081317229\\
2.15665391634791	207969.928645406\\
2.15675391884797	208019.775973582\\
2.15685392134803	208069.623301758\\
2.1569539238481	208119.470629935\\
2.15705392634816	208169.317958111\\
2.15715392884822	208219.165286288\\
2.15725393134828	208269.012614464\\
2.15735393384835	208318.85994264\\
2.15745393634841	208368.707270817\\
2.15755393884847	208418.554598993\\
2.15765394134853	208468.40192717\\
2.1577539438486	208518.249255346\\
2.15785394634866	208568.096583522\\
2.15795394884872	208617.943911699\\
2.15805395134878	208667.791239875\\
2.15815395384885	208717.638568051\\
2.15825395634891	208767.485896228\\
2.15835395884897	208817.333224404\\
2.15845396134903	208867.180552581\\
2.1585539638491	208917.027880757\\
2.15865396634916	208967.448166728\\
2.15875396884922	209017.295494905\\
2.15885397134928	209067.142823081\\
2.15895397384935	209116.990151258\\
2.15905397634941	209166.837479434\\
2.15915397884947	209216.68480761\\
2.15925398134953	209266.532135787\\
2.1593539838496	209316.379463963\\
2.15945398634966	209366.22679214\\
2.15955398884972	209416.074120316\\
2.15965399134978	209466.494406287\\
2.15975399384985	209516.341734464\\
2.15985399634991	209566.18906264\\
2.15995399884997	209616.036390817\\
2.16005400135003	209665.883718993\\
2.1601540038501	209715.731047169\\
2.16025400635016	209765.578375346\\
2.16035400885022	209815.425703522\\
2.16045401135028	209865.845989494\\
2.16055401385035	209915.69331767\\
2.16065401635041	209965.540645846\\
2.16075401885047	210015.387974023\\
2.16085402135053	210065.235302199\\
2.1609540238506	210115.082630375\\
2.16105402635066	210165.502916347\\
2.16115402885072	210215.350244523\\
2.16125403135078	210265.1975727\\
2.16135403385085	210315.044900876\\
2.16145403635091	210364.892229053\\
2.16155403885097	210415.312515024\\
2.16165404135103	210465.1598432\\
2.1617540438511	210515.007171377\\
2.16185404635116	210564.854499553\\
2.16195404885122	210614.70182773\\
2.16205405135128	210665.122113701\\
2.16215405385135	210714.969441877\\
2.16225405635141	210764.816770054\\
2.16235405885147	210814.66409823\\
2.16245406135153	210865.084384202\\
2.1625540638516	210914.931712378\\
2.16265406635166	210964.779040555\\
2.16275406885172	211014.626368731\\
2.16285407135178	211064.473696907\\
2.16295407385185	211114.893982879\\
2.16305407635191	211164.741311055\\
2.16315407885197	211214.588639232\\
2.16325408135203	211265.008925203\\
2.1633540838521	211314.856253379\\
2.16345408635216	211364.703581556\\
2.16355408885222	211414.550909732\\
2.16365409135228	211464.971195704\\
2.16375409385235	211514.81852388\\
2.16385409635241	211564.665852056\\
2.16395409885247	211614.513180233\\
2.16405410135253	211664.933466204\\
2.1641541038526	211714.780794381\\
2.16425410635266	211764.628122557\\
2.16435410885272	211815.048408529\\
2.16445411135278	211864.895736705\\
2.16455411385285	211914.743064881\\
2.16465411635291	211965.163350853\\
2.16475411885297	212015.010679029\\
2.16485412135303	212064.858007206\\
2.1649541238531	212115.278293177\\
2.16505412635316	212165.125621354\\
2.16515412885322	212214.97294953\\
2.16525413135328	212265.393235501\\
2.16535413385335	212315.240563678\\
2.16545413635341	212365.087891854\\
2.16555413885347	212415.508177826\\
2.16565414135353	212465.355506002\\
2.1657541438536	212515.202834179\\
2.16585414635366	212565.62312015\\
2.16595414885372	212615.470448326\\
2.16605415135378	212665.317776503\\
2.16615415385385	212715.738062474\\
2.16625415635391	212765.585390651\\
2.16635415885397	212816.005676622\\
2.16645416135403	212865.853004799\\
2.1665541638541	212915.700332975\\
2.16665416635416	212966.120618947\\
2.16675416885422	213015.967947123\\
2.16685417135428	213066.388233094\\
2.16695417385435	213116.235561271\\
2.16705417635441	213166.082889447\\
2.16715417885447	213216.503175419\\
2.16725418135453	213266.350503595\\
2.1673541838546	213316.770789567\\
2.16745418635466	213366.618117743\\
2.16755418885472	213416.465445919\\
2.16765419135478	213466.885731891\\
2.16775419385485	213516.733060067\\
2.16785419635491	213567.153346039\\
2.16795419885497	213617.000674215\\
2.16805420135503	213667.420960187\\
2.1681542038551	213717.268288363\\
2.16825420635516	213767.688574334\\
2.16835420885522	213817.535902511\\
2.16845421135528	213867.383230687\\
2.16855421385535	213917.803516659\\
2.16865421635541	213967.650844835\\
2.16875421885547	214018.071130807\\
2.16885422135553	214067.918458983\\
2.1689542238556	214118.338744955\\
2.16905422635566	214168.186073131\\
2.16915422885572	214218.606359102\\
2.16925423135578	214268.453687279\\
2.16935423385585	214318.87397325\\
2.16945423635591	214368.721301427\\
2.16955423885597	214419.141587398\\
2.16965424135603	214468.988915575\\
2.1697542438561	214519.409201546\\
2.16985424635616	214569.256529723\\
2.16995424885622	214619.676815694\\
2.17005425135628	214669.52414387\\
2.17015425385635	214719.944429842\\
2.17025425635641	214769.791758018\\
2.17035425885647	214820.21204399\\
2.17045426135653	214870.059372166\\
2.1705542638566	214920.479658138\\
2.17065426635666	214970.326986314\\
2.17075426885672	215020.747272286\\
2.17085427135678	215070.594600462\\
2.17095427385685	215121.014886433\\
2.17105427635691	215171.435172405\\
2.17115427885697	215221.282500581\\
2.17125428135703	215271.702786553\\
2.1713542838571	215321.550114729\\
2.17145428635716	215371.970400701\\
2.17155428885722	215421.817728877\\
2.17165429135728	215472.238014849\\
2.17175429385735	215522.085343025\\
2.17185429635741	215572.505628997\\
2.17195429885747	215622.925914968\\
2.17205430135753	215672.773243144\\
2.1721543038576	215723.193529116\\
2.17225430635766	215773.040857292\\
2.17235430885772	215823.461143264\\
2.17245431135778	215873.881429235\\
2.17255431385785	215923.728757412\\
2.17265431635791	215974.149043383\\
2.17275431885797	216023.99637156\\
2.17285432135803	216074.416657531\\
2.1729543238581	216124.836943503\\
2.17305432635816	216174.684271679\\
2.17315432885822	216225.104557651\\
2.17325433135828	216274.951885827\\
2.17335433385835	216325.372171799\\
2.17345433635841	216375.79245777\\
2.17355433885847	216425.639785946\\
2.17365434135853	216476.060071918\\
2.1737543438586	216526.480357889\\
2.17385434635866	216576.327686066\\
2.17395434885872	216626.747972037\\
2.17405435135878	216677.168258009\\
2.17415435385885	216727.015586185\\
2.17425435635891	216777.435872157\\
2.17435435885897	216827.856158128\\
2.17445436135903	216877.703486305\\
2.1745543638591	216928.123772276\\
2.17465436635916	216978.544058248\\
2.17475436885922	217028.391386424\\
2.17485437135928	217078.811672396\\
2.17495437385935	217129.231958367\\
2.17505437635941	217179.079286543\\
2.17515437885947	217229.499572515\\
2.17525438135953	217279.919858486\\
2.1753543838596	217329.767186663\\
2.17545438635966	217380.187472634\\
2.17555438885972	217430.607758606\\
2.17565439135978	217480.455086782\\
2.17575439385985	217530.875372754\\
2.17585439635991	217581.295658725\\
2.17595439885997	217631.142986902\\
2.17605440136003	217681.563272873\\
2.1761544038601	217731.983558845\\
2.17625440636016	217782.403844816\\
2.17635440886022	217832.251172993\\
2.17645441136028	217882.671458964\\
2.17655441386035	217933.091744936\\
2.17665441636041	217982.939073112\\
2.17675441886047	218033.359359083\\
2.17685442136053	218083.779645055\\
2.1769544238606	218134.199931026\\
2.17705442636066	218184.047259203\\
2.17715442886072	218234.467545174\\
2.17725443136078	218284.887831146\\
2.17735443386085	218335.308117117\\
2.17745443636091	218385.155445294\\
2.17755443886097	218435.575731265\\
2.17765444136103	218485.996017237\\
2.1777544438611	218536.416303208\\
2.17785444636116	218586.263631385\\
2.17795444886122	218636.683917356\\
2.17805445136128	218687.104203328\\
2.17815445386135	218737.524489299\\
2.17825445636141	218787.944775271\\
2.17835445886147	218837.792103447\\
2.17845446136153	218888.212389419\\
2.1785544638616	218938.63267539\\
2.17865446636166	218989.052961362\\
2.17875446886172	219038.900289538\\
2.17885447136178	219089.32057551\\
2.17895447386185	219139.740861481\\
2.17905447636191	219190.161147453\\
2.17915447886197	219240.581433424\\
2.17925448136203	219290.428761601\\
2.1793544838621	219340.849047572\\
2.17945448636216	219391.269333544\\
2.17955448886222	219441.689619515\\
2.17965449136228	219492.109905487\\
2.17975449386235	219542.530191458\\
2.17985449636241	219592.377519634\\
2.17995449886247	219642.797805606\\
2.18005450136253	219693.218091577\\
2.1801545038626	219743.638377549\\
2.18025450636266	219794.05866352\\
2.18035450886272	219843.905991697\\
2.18045451136278	219894.326277668\\
2.18055451386285	219944.74656364\\
2.18065451636291	219995.166849611\\
2.18075451886297	220045.587135583\\
2.18085452136303	220096.007421554\\
2.1809545238631	220146.427707526\\
2.18105452636316	220196.275035702\\
2.18115452886322	220246.695321674\\
2.18125453136328	220297.115607645\\
2.18135453386335	220347.535893617\\
2.18145453636341	220397.956179588\\
2.18155453886347	220448.37646556\\
2.18165454136353	220498.796751531\\
2.1817545438636	220548.644079708\\
2.18185454636366	220599.064365679\\
2.18195454886372	220649.484651651\\
2.18205455136378	220699.904937622\\
2.18215455386385	220750.325223594\\
2.18225455636391	220800.745509565\\
2.18235455886397	220851.165795537\\
2.18245456136403	220901.586081508\\
2.1825545638641	220952.00636748\\
2.18265456636416	221001.853695656\\
2.18275456886422	221052.273981628\\
2.18285457136428	221102.694267599\\
2.18295457386435	221153.114553571\\
2.18305457636441	221203.534839542\\
2.18315457886447	221253.955125514\\
2.18325458136453	221304.375411485\\
2.1833545838646	221354.795697457\\
2.18345458636466	221405.215983428\\
2.18355458886472	221455.6362694\\
2.18365459136478	221506.056555371\\
2.18375459386485	221556.476841343\\
2.18385459636491	221606.324169519\\
2.18395459886497	221656.744455491\\
2.18405460136503	221707.164741462\\
2.1841546038651	221757.585027434\\
2.18425460636516	221808.005313405\\
2.18435460886522	221858.425599377\\
2.18445461136528	221908.845885348\\
2.18455461386535	221959.26617132\\
2.18465461636541	222009.686457291\\
2.18475461886547	222060.106743263\\
2.18485462136553	222110.527029234\\
2.1849546238656	222160.947315206\\
2.18505462636566	222211.367601177\\
2.18515462886572	222261.787887149\\
2.18525463136578	222312.20817312\\
2.18535463386585	222362.628459092\\
2.18545463636591	222413.048745064\\
2.18555463886597	222463.469031035\\
2.18565464136603	222513.889317007\\
2.1857546438661	222564.309602978\\
2.18585464636616	222614.72988895\\
2.18595464886622	222665.150174921\\
2.18605465136628	222714.997503097\\
2.18615465386635	222765.417789069\\
2.18625465636641	222815.83807504\\
2.18635465886647	222866.258361012\\
2.18645466136653	222916.678646983\\
2.1865546638666	222967.098932955\\
2.18665466636666	223017.519218926\\
2.18675466886672	223067.939504898\\
2.18685467136678	223118.35979087\\
2.18695467386685	223168.780076841\\
2.18705467636691	223219.200362813\\
2.18715467886697	223269.620648784\\
2.18725468136703	223320.040934756\\
2.1873546838671	223370.461220727\\
2.18745468636716	223420.881506699\\
2.18755468886722	223471.874750465\\
2.18765469136728	223522.295036437\\
2.18775469386735	223572.715322408\\
2.18785469636741	223623.13560838\\
2.18795469886747	223673.555894351\\
2.18805470136753	223723.976180323\\
2.1881547038676	223774.396466294\\
2.18825470636766	223824.816752266\\
2.18835470886772	223875.237038237\\
2.18845471136778	223925.657324209\\
2.18855471386785	223976.07761018\\
2.18865471636791	224026.497896152\\
2.18875471886797	224076.918182123\\
2.18885472136803	224127.338468095\\
2.1889547238681	224177.758754066\\
2.18905472636816	224228.179040038\\
2.18915472886822	224278.599326009\\
2.18925473136828	224329.019611981\\
2.18935473386835	224379.439897952\\
2.18945473636841	224429.860183924\\
2.18955473886847	224480.280469895\\
2.18965474136853	224530.700755867\\
2.1897547438686	224581.121041839\\
2.18985474636866	224632.114285605\\
2.18995474886872	224682.534571577\\
2.19005475136878	224732.954857548\\
2.19015475386885	224783.37514352\\
2.19025475636891	224833.795429491\\
2.19035475886897	224884.215715463\\
2.19045476136903	224934.636001434\\
2.1905547638691	224985.056287406\\
2.19065476636916	225035.476573377\\
2.19075476886922	225085.896859349\\
2.19085477136928	225136.31714532\\
2.19095477386935	225186.737431292\\
2.19105477636941	225237.730675058\\
2.19115477886947	225288.15096103\\
2.19125478136953	225338.571247001\\
2.1913547838696	225388.991532973\\
2.19145478636966	225439.411818944\\
2.19155478886972	225489.832104916\\
2.19165479136978	225540.252390888\\
2.19175479386985	225590.672676859\\
2.19185479636991	225641.092962831\\
2.19195479886997	225691.513248802\\
2.19205480137003	225742.506492569\\
2.1921548038701	225792.92677854\\
2.19225480637016	225843.347064512\\
2.19235480887022	225893.767350483\\
2.19245481137028	225944.187636455\\
2.19255481387035	225994.607922426\\
2.19265481637041	226045.028208398\\
2.19275481887047	226095.448494369\\
2.19285482137053	226146.441738136\\
2.1929548238706	226196.862024107\\
2.19305482637066	226247.282310079\\
2.19315482887072	226297.70259605\\
2.19325483137078	226348.122882022\\
2.19335483387085	226398.543167993\\
2.19345483637091	226448.963453965\\
2.19355483887097	226499.956697732\\
2.19365484137103	226550.376983703\\
2.1937548438711	226600.797269675\\
2.19385484637116	226651.217555646\\
2.19395484887122	226701.637841618\\
2.19405485137128	226752.058127589\\
2.19415485387135	226802.478413561\\
2.19425485637141	226853.471657327\\
2.19435485887147	226903.891943299\\
2.19445486137153	226954.31222927\\
2.1945548638716	227004.732515242\\
2.19465486637166	227055.152801213\\
2.19475486887172	227105.573087185\\
2.19485487137178	227156.566330952\\
2.19495487387185	227206.986616923\\
2.19505487637191	227257.406902895\\
2.19515487887197	227307.827188866\\
2.19525488137203	227358.247474838\\
2.1953548838721	227408.667760809\\
2.19545488637216	227459.661004576\\
2.19555488887222	227510.081290547\\
2.19565489137228	227560.501576519\\
2.19575489387235	227610.92186249\\
2.19585489637241	227661.342148462\\
2.19595489887247	227711.762434433\\
2.19605490137253	227762.7556782\\
2.1961549038726	227813.175964172\\
2.19625490637266	227863.596250143\\
2.19635490887272	227914.016536115\\
2.19645491137278	227964.436822086\\
2.19655491387285	228015.430065853\\
2.19665491637291	228065.850351824\\
2.19675491887297	228116.270637796\\
2.19685492137303	228166.690923767\\
2.1969549238731	228217.111209739\\
2.19705492637316	228268.104453505\\
2.19715492887322	228318.524739477\\
2.19725493137328	228368.945025448\\
2.19735493387335	228419.36531142\\
2.19745493637341	228469.785597391\\
2.19755493887347	228520.778841158\\
2.19765494137353	228571.19912713\\
2.1977549438736	228621.619413101\\
2.19785494637366	228672.039699073\\
2.19795494887372	228723.032942839\\
2.19805495137378	228773.453228811\\
2.19815495387385	228823.873514782\\
2.19825495637391	228874.293800754\\
2.19835495887397	228924.714086725\\
2.19845496137403	228975.707330492\\
2.1985549638741	229026.127616463\\
2.19865496637416	229076.547902435\\
2.19875496887422	229126.968188406\\
2.19885497137428	229177.961432173\\
2.19895497387435	229228.381718145\\
2.19905497637441	229278.802004116\\
2.19915497887447	229329.222290088\\
2.19925498137453	229379.642576059\\
2.1993549838746	229430.635819826\\
2.19945498637466	229481.056105797\\
2.19955498887472	229531.476391769\\
2.19965499137478	229581.89667774\\
2.19975499387485	229632.889921507\\
2.19985499637491	229683.310207478\\
2.19995499887497	229733.73049345\\
2.20005500137503	229784.150779421\\
2.2001550038751	229835.144023188\\
2.20025500637516	229885.56430916\\
2.20035500887522	229935.984595131\\
2.20045501137528	229986.404881103\\
2.20055501387535	230037.398124869\\
2.20065501637541	230087.818410841\\
2.20075501887547	230138.238696812\\
2.20085502137553	230188.658982784\\
2.2009550238756	230239.652226551\\
2.20105502637566	230290.072512522\\
2.20115502887572	230340.492798494\\
2.20125503137578	230391.48604226\\
2.20135503387585	230441.906328232\\
2.20145503637591	230492.326614203\\
2.20155503887597	230542.746900175\\
2.20165504137603	230593.740143941\\
2.2017550438761	230644.160429913\\
2.20185504637616	230694.580715884\\
2.20195504887622	230745.001001856\\
2.20205505137628	230795.994245623\\
2.20215505387635	230846.414531594\\
2.20225505637641	230896.834817566\\
2.20235505887647	230947.828061332\\
2.20245506137653	230998.248347304\\
2.2025550638766	231048.668633275\\
2.20265506637666	231099.088919247\\
2.20275506887672	231150.082163013\\
2.20285507137678	231200.502448985\\
2.20295507387685	231250.922734956\\
2.20305507637691	231301.915978723\\
2.20315507887697	231352.336264695\\
2.20325508137703	231402.756550666\\
2.2033550838771	231453.176836638\\
2.20345508637716	231504.170080404\\
2.20355508887722	231554.590366376\\
2.20365509137728	231605.010652347\\
2.20375509387735	231656.003896114\\
2.20385509637741	231706.424182085\\
2.20395509887747	231756.844468057\\
2.20405510137753	231807.264754028\\
2.2041551038776	231858.257997795\\
2.20425510637766	231908.678283767\\
2.20435510887772	231959.098569738\\
2.20445511137778	232010.091813505\\
2.20455511387785	232060.512099476\\
2.20465511637791	232110.932385448\\
2.20475511887797	232161.925629214\\
2.20485512137803	232212.345915186\\
2.2049551238781	232262.766201157\\
2.20505512637816	232313.186487129\\
2.20515512887822	232364.179730896\\
2.20525513137828	232414.600016867\\
2.20535513387835	232465.020302839\\
2.20545513637841	232516.013546605\\
2.20555513887847	232566.433832577\\
2.20565514137853	232616.854118548\\
2.2057551438786	232667.847362315\\
2.20585514637866	232718.267648286\\
2.20595514887872	232768.687934258\\
2.20605515137878	232819.681178025\\
2.20615515387885	232870.101463996\\
2.20625515637891	232920.521749968\\
2.20635515887897	232971.514993734\\
2.20645516137903	233021.935279706\\
2.2065551638791	233072.355565677\\
2.20665516637916	233123.348809444\\
2.20675516887922	233173.769095415\\
2.20685517137928	233224.189381387\\
2.20695517387935	233275.182625154\\
2.20705517637941	233325.602911125\\
2.20715517887947	233376.023197097\\
2.20725518137953	233427.016440863\\
2.2073551838796	233477.436726835\\
2.20745518637966	233527.857012806\\
2.20755518887972	233578.850256573\\
2.20765519137978	233629.270542544\\
2.20775519387985	233679.690828516\\
2.20785519637991	233730.684072283\\
2.20795519887997	233781.104358254\\
2.20805520138003	233831.524644226\\
2.2081552038801	233882.517887992\\
2.20825520638016	233932.938173964\\
2.20835520888022	233983.358459935\\
2.20845521138028	234034.351703702\\
2.20855521388035	234084.771989673\\
2.20865521638041	234135.192275645\\
2.20875521888047	234186.185519412\\
2.20885522138053	234236.605805383\\
2.2089552238806	234287.026091355\\
2.20905522638066	234338.019335121\\
2.20915522888072	234388.439621093\\
2.20925523138078	234438.859907064\\
2.20935523388085	234489.853150831\\
2.20945523638091	234540.273436802\\
2.20955523888097	234590.693722774\\
2.20965524138103	234641.686966541\\
2.2097552438811	234692.107252512\\
2.20985524638116	234742.527538484\\
2.20995524888122	234793.52078225\\
2.21005525138128	234843.941068222\\
2.21015525388135	234894.361354193\\
2.21025525638141	234945.35459796\\
2.21035525888147	234995.774883932\\
2.21045526138153	235046.768127698\\
2.2105552638816	235097.18841367\\
2.21065526638166	235147.608699641\\
2.21075526888172	235198.601943408\\
2.21085527138178	235249.022229379\\
2.21095527388185	235299.442515351\\
2.21105527638191	235350.435759118\\
2.21115527888197	235400.856045089\\
2.21125528138203	235451.27633106\\
2.2113552838821	235502.269574827\\
2.21145528638216	235552.689860799\\
2.21155528888222	235603.11014677\\
2.21165529138228	235654.103390537\\
2.21175529388235	235704.523676508\\
2.21185529638241	235755.516920275\\
2.21195529888247	235805.937206247\\
2.21205530138253	235856.357492218\\
2.2121553038826	235907.350735985\\
2.21225530638266	235957.771021956\\
2.21235530888272	236008.191307928\\
2.21245531138278	236059.184551694\\
2.21255531388285	236109.604837666\\
2.21265531638291	236160.025123637\\
2.21275531888297	236211.018367404\\
2.21285532138303	236261.438653375\\
2.2129553238831	236312.431897142\\
2.21305532638316	236362.852183114\\
2.21315532888322	236413.272469085\\
2.21325533138328	236464.265712852\\
2.21335533388335	236514.685998823\\
2.21345533638341	236565.106284795\\
2.21355533888347	236616.099528561\\
2.21365534138353	236666.519814533\\
2.2137553438836	236717.5130583\\
2.21385534638366	236767.933344271\\
2.21395534888372	236818.353630243\\
2.21405535138378	236869.346874009\\
2.21415535388385	236919.767159981\\
2.21425535638391	236970.187445952\\
2.21435535888397	237021.180689719\\
2.21445536138403	237071.60097569\\
2.2145553638841	237122.594219457\\
2.21465536638416	237173.014505429\\
2.21475536888422	237223.4347914\\
2.21485537138428	237274.428035167\\
2.21495537388435	237324.848321138\\
2.21505537638441	237375.26860711\\
2.21515537888447	237426.261850876\\
2.21525538138453	237476.682136848\\
2.2153553838846	237527.675380615\\
2.21545538638466	237578.095666586\\
2.21555538888472	237628.515952558\\
2.21565539138478	237679.509196324\\
2.21575539388485	237729.929482296\\
2.21585539638491	237780.349768267\\
2.21595539888497	237831.343012034\\
2.21605540138503	237881.763298005\\
2.2161554038851	237932.756541772\\
2.21625540638516	237983.176827744\\
2.21635540888522	238033.597113715\\
2.21645541138528	238084.590357482\\
2.21655541388535	238135.010643453\\
2.21665541638541	238185.430929425\\
2.21675541888547	238236.424173191\\
2.21685542138553	238286.844459163\\
2.2169554238856	238337.83770293\\
2.21705542638566	238388.257988901\\
2.21715542888572	238438.678274873\\
2.21725543138578	238489.671518639\\
2.21735543388585	238540.091804611\\
2.21745543638591	238590.512090582\\
2.21755543888597	238641.505334349\\
2.21765544138603	238691.92562032\\
2.2177554438861	238742.918864087\\
2.21785544638616	238793.339150059\\
2.21795544888622	238843.75943603\\
2.21805545138628	238894.752679797\\
2.21815545388635	238945.172965768\\
2.21825545638641	238996.166209535\\
2.21835545888647	239046.586495506\\
2.21845546138653	239097.006781478\\
2.2185554638866	239148.000025245\\
2.21865546638666	239198.420311216\\
2.21875546888672	239248.840597188\\
2.21885547138678	239299.833840954\\
2.21895547388685	239350.254126926\\
2.21905547638691	239401.247370692\\
2.21915547888697	239451.667656664\\
2.21925548138703	239502.087942635\\
2.2193554838871	239553.081186402\\
2.21945548638716	239603.501472374\\
2.21955548888722	239653.921758345\\
2.21965549138728	239704.915002112\\
2.21975549388735	239755.335288083\\
2.21985549638741	239806.32853185\\
2.21995549888747	239856.748817821\\
2.22005550138753	239907.169103793\\
2.2201555038876	239958.16234756\\
2.22025550638766	240008.582633531\\
2.22035550888772	240059.002919503\\
2.22045551138778	240109.996163269\\
2.22055551388785	240160.416449241\\
2.22065551638791	240211.409693007\\
2.22075551888797	240261.829978979\\
2.22085552138803	240312.25026495\\
2.2209555238881	240363.243508717\\
2.22105552638816	240413.663794689\\
2.22115552888822	240464.657038455\\
2.22125553138828	240515.077324427\\
2.22135553388835	240565.497610398\\
2.22145553638841	240616.490854165\\
2.22155553888847	240666.911140136\\
2.22165554138853	240717.331426108\\
2.2217555438886	240768.324669875\\
2.22185554638866	240818.744955846\\
2.22195554888872	240869.738199613\\
2.22205555138878	240920.158485584\\
2.22215555388885	240970.578771556\\
2.22225555638891	241021.572015322\\
2.22235555888897	241071.992301294\\
2.22245556138903	241122.412587265\\
2.2225555638891	241173.405831032\\
2.22265556638916	241223.826117004\\
2.22275556888922	241274.81936077\\
2.22285557138928	241325.239646742\\
2.22295557388935	241375.659932713\\
2.22305557638941	241426.65317648\\
2.22315557888947	241477.073462451\\
2.22325558138953	241527.493748423\\
2.2233555838896	241578.48699219\\
2.22345558638966	241628.907278161\\
2.22355558888972	241679.900521928\\
2.22365559138978	241730.320807899\\
2.22375559388985	241780.741093871\\
2.22385559638991	241831.734337637\\
2.22395559888997	241882.154623609\\
2.22405560139003	241932.57490958\\
2.2241556038901	241983.568153347\\
2.22425560639016	242033.988439319\\
2.22435560889022	242084.40872529\\
2.22445561139028	242135.401969057\\
2.22455561389035	242185.822255028\\
2.22465561639041	242236.815498795\\
2.22475561889047	242287.235784766\\
2.22485562139053	242337.656070738\\
2.2249556238906	242388.649314504\\
2.22505562639066	242439.069600476\\
2.22515562889072	242489.489886448\\
2.22525563139078	242540.483130214\\
2.22535563389085	242590.903416186\\
2.22545563639091	242641.896659952\\
2.22555563889097	242692.316945924\\
2.22565564139103	242742.737231895\\
2.2257556438911	242793.730475662\\
2.22585564639116	242844.150761634\\
2.22595564889122	242894.571047605\\
2.22605565139128	242945.564291372\\
2.22615565389135	242995.984577343\\
2.22625565639141	243046.404863315\\
2.22635565889147	243097.398107081\\
2.22645566139153	243147.818393053\\
2.2265556638916	243198.238679024\\
2.22665566639166	243249.231922791\\
2.22675566889172	243299.652208763\\
2.22685567139178	243350.645452529\\
2.22695567389185	243401.065738501\\
2.22705567639191	243451.486024472\\
2.22715567889197	243502.479268239\\
2.22725568139203	243552.89955421\\
2.2273556838921	243603.319840182\\
2.22745568639216	243654.313083948\\
2.22755568889222	243704.73336992\\
2.22765569139228	243755.153655892\\
2.22775569389235	243806.146899658\\
2.22785569639241	243856.56718563\\
2.22795569889247	243906.987471601\\
2.22805570139253	243957.980715368\\
2.2281557038926	244008.401001339\\
2.22825570639266	244058.821287311\\
2.22835570889272	244109.814531078\\
2.22845571139278	244160.234817049\\
2.22855571389285	244210.65510302\\
2.22865571639291	244261.648346787\\
2.22875571889297	244312.068632759\\
2.22885572139303	244363.061876525\\
2.2289557238931	244413.482162497\\
2.22905572639316	244463.902448468\\
2.22915572889322	244514.895692235\\
2.22925573139328	244565.315978207\\
2.22935573389335	244615.736264178\\
2.22945573639341	244666.729507945\\
2.22955573889347	244717.149793916\\
2.22965574139353	244767.570079888\\
2.2297557438936	244818.563323654\\
2.22985574639366	244868.983609626\\
2.22995574889372	244919.403895597\\
2.23005575139378	244970.397139364\\
2.23015575389385	245020.817425335\\
2.23025575639391	245071.237711307\\
2.23035575889397	245122.230955074\\
2.23045576139404	245172.651241045\\
2.2305557638941	245223.071527017\\
2.23065576639416	245273.491812988\\
2.23075576889422	245324.485056755\\
2.23085577139428	245374.905342726\\
2.23095577389435	245425.325628698\\
2.23105577639441	245476.318872465\\
2.23115577889447	245526.739158436\\
2.23125578139453	245577.159444408\\
2.2313557838946	245628.152688174\\
2.23145578639466	245678.572974146\\
2.23155578889472	245728.993260117\\
2.23165579139478	245779.986503884\\
2.23175579389485	245830.406789855\\
2.23185579639491	245880.827075827\\
2.23195579889497	245931.820319594\\
2.23205580139503	245982.240605565\\
2.2321558038951	246032.660891537\\
2.23225580639516	246083.654135303\\
2.23235580889522	246134.074421275\\
2.23245581139529	246184.494707246\\
2.23255581389535	246234.914993218\\
2.23265581639541	246285.908236984\\
2.23275581889547	246336.328522956\\
2.23285582139553	246386.748808927\\
2.2329558238956	246437.742052694\\
2.23305582639566	246488.162338666\\
2.23315582889572	246538.582624637\\
2.23325583139578	246589.575868404\\
2.23335583389585	246639.996154375\\
2.23345583639591	246690.416440347\\
2.23355583889597	246740.836726318\\
2.23365584139603	246791.829970085\\
2.2337558438961	246842.250256056\\
2.23385584639616	246892.670542028\\
2.23395584889622	246943.663785795\\
2.23405585139628	246994.084071766\\
2.23415585389635	247044.504357738\\
2.23425585639641	247095.497601504\\
2.23435585889647	247145.917887476\\
2.23445586139654	247196.338173447\\
2.2345558638966	247246.758459419\\
2.23465586639666	247297.751703185\\
2.23475586889672	247348.171989157\\
2.23485587139678	247398.592275128\\
2.23495587389685	247449.0125611\\
2.23505587639691	247500.005804867\\
2.23515587889697	247550.426090838\\
2.23525588139703	247600.84637681\\
2.2353558838971	247651.839620576\\
2.23545588639716	247702.259906548\\
2.23555588889722	247752.680192519\\
2.23565589139728	247803.100478491\\
2.23575589389735	247854.093722257\\
2.23585589639741	247904.514008229\\
2.23595589889747	247954.9342942\\
2.23605590139753	248005.354580172\\
2.2361559038976	248056.347823939\\
2.23625590639766	248106.76810991\\
2.23635590889772	248157.188395882\\
2.23645591139779	248208.181639648\\
2.23655591389785	248258.60192562\\
2.23665591639791	248309.022211591\\
2.23675591889797	248359.442497563\\
2.23685592139804	248410.435741329\\
2.2369559238981	248460.856027301\\
2.23705592639816	248511.276313273\\
2.23715592889822	248561.696599244\\
2.23725593139828	248612.689843011\\
2.23735593389835	248663.110128982\\
2.23745593639841	248713.530414954\\
2.23755593889847	248763.950700925\\
2.23765594139853	248814.370986897\\
2.2377559438986	248865.364230663\\
2.23785594639866	248915.784516635\\
2.23795594889872	248966.204802606\\
2.23805595139878	249016.625088578\\
2.23815595389885	249067.618332345\\
2.23825595639891	249118.038618316\\
2.23835595889897	249168.458904288\\
2.23845596139904	249218.879190259\\
2.2385559638991	249269.872434026\\
2.23865596639916	249320.292719997\\
2.23875596889922	249370.713005969\\
2.23885597139929	249421.13329194\\
2.23895597389935	249471.553577912\\
2.23905597639941	249522.546821678\\
2.23915597889947	249572.96710765\\
2.23925598139953	249623.387393621\\
2.2393559838996	249673.807679593\\
2.23945598639966	249724.80092336\\
2.23955598889972	249775.221209331\\
2.23965599139978	249825.641495303\\
2.23975599389985	249876.061781274\\
2.23985599639991	249926.482067246\\
2.23995599889997	249977.475311012\\
2.24005600140003	250027.895596984\\
2.2401560039001	250078.315882955\\
2.24025600640016	250128.736168927\\
2.24035600890022	250179.156454898\\
2.24045601140029	250230.149698665\\
2.24055601390035	250280.569984637\\
2.24065601640041	250330.990270608\\
2.24075601890047	250381.410556579\\
2.24085602140054	250431.830842551\\
2.2409560239006	250482.824086318\\
2.24105602640066	250533.244372289\\
2.24115602890072	250583.664658261\\
2.24125603140078	250634.084944232\\
2.24135603390085	250684.505230204\\
2.24145603640091	250734.925516175\\
2.24155603890097	250785.918759942\\
2.24165604140103	250836.339045913\\
2.2417560439011	250886.759331885\\
2.24185604640116	250937.179617856\\
2.24195604890122	250987.599903828\\
2.24205605140128	251038.020189799\\
2.24215605390135	251089.013433566\\
2.24225605640141	251139.433719538\\
2.24235605890147	251189.854005509\\
2.24245606140154	251240.274291481\\
2.2425560639016	251290.694577452\\
2.24265606640166	251341.114863424\\
2.24275606890172	251392.10810719\\
2.24285607140179	251442.528393162\\
2.24295607390185	251492.948679133\\
2.24305607640191	251543.368965105\\
2.24315607890197	251593.789251076\\
2.24325608140204	251644.209537048\\
2.2433560839021	251694.629823019\\
2.24345608640216	251745.623066786\\
2.24355608890222	251796.043352757\\
2.24365609140228	251846.463638729\\
2.24375609390235	251896.883924701\\
2.24385609640241	251947.304210672\\
2.24395609890247	251997.724496644\\
2.24405610140253	252048.144782615\\
2.2441561039026	252099.138026382\\
2.24425610640266	252149.558312353\\
2.24435610890272	252199.978598325\\
2.24445611140279	252250.398884296\\
2.24455611390285	252300.819170268\\
2.24465611640291	252351.239456239\\
2.24475611890297	252401.659742211\\
2.24485612140304	252452.080028182\\
2.2449561239031	252502.500314154\\
2.24505612640316	252553.49355792\\
2.24515612890322	252603.913843892\\
2.24525613140329	252654.334129863\\
2.24535613390335	252704.754415835\\
2.24545613640341	252755.174701806\\
2.24555613890347	252805.594987778\\
2.24565614140353	252856.01527375\\
2.2457561439036	252906.435559721\\
2.24585614640366	252956.855845693\\
2.24595614890372	253007.276131664\\
2.24605615140378	253058.269375431\\
2.24615615390385	253108.689661402\\
2.24625615640391	253159.109947374\\
2.24635615890397	253209.530233345\\
2.24645616140404	253259.950519317\\
2.2465561639041	253310.370805288\\
2.24665616640416	253360.79109126\\
2.24675616890422	253411.211377231\\
2.24685617140429	253461.631663203\\
2.24695617390435	253512.051949174\\
2.24705617640441	253562.472235146\\
2.24715617890447	253612.892521117\\
2.24725618140454	253663.312807089\\
2.2473561839046	253713.73309306\\
2.24745618640466	253764.153379032\\
2.24755618890472	253815.146622798\\
2.24765619140479	253865.56690877\\
2.24775619390485	253915.987194742\\
2.24785619640491	253966.407480713\\
2.24795619890497	254016.827766685\\
2.24805620140503	254067.248052656\\
2.2481562039051	254117.668338628\\
2.24825620640516	254168.088624599\\
2.24835620890522	254218.508910571\\
2.24845621140529	254268.929196542\\
2.24855621390535	254319.349482514\\
2.24865621640541	254369.769768485\\
2.24875621890547	254420.190054457\\
2.24885622140554	254470.610340428\\
2.2489562239056	254521.0306264\\
2.24905622640566	254571.450912371\\
2.24915622890572	254621.871198343\\
2.24925623140579	254672.291484314\\
2.24935623390585	254722.711770286\\
2.24945623640591	254773.132056257\\
2.24955623890597	254823.552342229\\
2.24965624140604	254873.9726282\\
2.2497562439061	254924.392914172\\
2.24985624640616	254974.813200143\\
2.24995624890622	255025.233486115\\
2.25005625140628	255075.653772086\\
2.25015625390635	255126.074058058\\
2.25025625640641	255176.494344029\\
2.25035625890647	255226.914630001\\
2.25045626140654	255277.334915972\\
2.2505562639066	255327.755201944\\
2.25065626640666	255378.175487915\\
2.25075626890672	255428.595773887\\
2.25085627140679	255479.016059858\\
2.25095627390685	255529.43634583\\
2.25105627640691	255579.856631801\\
2.25115627890697	255630.276917773\\
2.25125628140704	255680.697203744\\
2.2513562839071	255731.117489716\\
2.25145628640716	255781.537775687\\
2.25155628890722	255831.958061659\\
2.25165629140729	255882.37834763\\
2.25175629390735	255932.225675807\\
2.25185629640741	255982.645961778\\
2.25195629890747	256033.06624775\\
2.25205630140753	256083.486533721\\
2.2521563039076	256133.906819693\\
2.25225630640766	256184.327105664\\
2.25235630890772	256234.747391636\\
2.25245631140779	256285.167677608\\
2.25255631390785	256335.587963579\\
2.25265631640791	256386.00824955\\
2.25275631890797	256436.428535522\\
2.25285632140804	256486.848821494\\
2.2529563239081	256537.269107465\\
2.25305632640816	256587.689393437\\
2.25315632890822	256637.536721613\\
2.25325633140829	256687.957007584\\
2.25335633390835	256738.377293556\\
2.25345633640841	256788.797579528\\
2.25355633890847	256839.217865499\\
2.25365634140854	256889.638151471\\
2.2537563439086	256940.058437442\\
2.25385634640866	256990.478723414\\
2.25395634890872	257040.899009385\\
2.25405635140879	257091.319295357\\
2.25415635390885	257141.166623533\\
2.25425635640891	257191.586909504\\
2.25435635890897	257242.007195476\\
2.25445636140904	257292.427481448\\
2.2545563639091	257342.847767419\\
2.25465636640916	257393.26805339\\
2.25475636890922	257443.688339362\\
2.25485637140929	257493.535667538\\
2.25495637390935	257543.95595351\\
2.25505637640941	257594.376239481\\
2.25515637890947	257644.796525453\\
2.25525638140954	257695.216811424\\
2.2553563839096	257745.637097396\\
2.25545638640966	257796.057383368\\
2.25555638890972	257845.904711544\\
2.25565639140979	257896.324997515\\
2.25575639390985	257946.745283487\\
2.25585639640991	257997.165569458\\
2.25595639890997	258047.58585543\\
2.25605640141004	258098.006141401\\
2.2561564039101	258147.853469578\\
2.25625640641016	258198.273755549\\
2.25635640891022	258248.694041521\\
2.25645641141029	258299.114327492\\
2.25655641391035	258349.534613464\\
2.25665641641041	258399.38194164\\
2.25675641891047	258449.802227612\\
2.25685642141054	258500.222513583\\
2.2569564239106	258550.642799555\\
2.25705642641066	258601.063085526\\
2.25715642891072	258650.910413703\\
2.25725643141079	258701.330699674\\
2.25735643391085	258751.750985646\\
2.25745643641091	258802.171271617\\
2.25755643891097	258852.591557589\\
2.25765644141104	258902.438885765\\
2.2577564439111	258952.859171737\\
2.25785644641116	259003.279457708\\
2.25795644891122	259053.69974368\\
2.25805645141129	259103.547071856\\
2.25815645391135	259153.967357828\\
2.25825645641141	259204.387643799\\
2.25835645891147	259254.807929771\\
2.25845646141154	259304.655257947\\
2.2585564639116	259355.075543918\\
2.25865646641166	259405.49582989\\
2.25875646891172	259455.916115861\\
2.25885647141179	259505.763444038\\
2.25895647391185	259556.183730009\\
2.25905647641191	259606.604015981\\
2.25915647891197	259657.024301952\\
2.25925648141204	259706.871630129\\
2.2593564839121	259757.2919161\\
2.25945648641216	259807.712202072\\
2.25955648891222	259857.559530248\\
2.25965649141229	259907.97981622\\
2.25975649391235	259958.400102191\\
2.25985649641241	260008.820388163\\
2.25995649891247	260058.667716339\\
2.26005650141254	260109.088002311\\
2.2601565039126	260159.508288282\\
2.26025650641266	260209.355616459\\
2.26035650891272	260259.77590243\\
2.26045651141279	260310.196188402\\
2.26055651391285	260360.043516578\\
2.26065651641291	260410.463802549\\
2.26075651891297	260460.884088521\\
2.26085652141304	260510.731416697\\
2.2609565239131	260561.151702669\\
2.26105652641316	260611.57198864\\
2.26115652891322	260661.419316817\\
2.26125653141329	260711.839602788\\
2.26135653391335	260762.25988876\\
2.26145653641341	260812.107216936\\
2.26155653891347	260862.527502908\\
2.26165654141354	260912.947788879\\
2.2617565439136	260962.795117056\\
2.26185654641366	261013.215403027\\
2.26195654891372	261063.635688999\\
2.26205655141379	261113.483017175\\
2.26215655391385	261163.903303146\\
2.26225655641391	261213.750631323\\
2.26235655891397	261264.170917294\\
2.26245656141404	261314.591203266\\
2.2625565639141	261364.438531442\\
2.26265656641416	261414.858817414\\
2.26275656891422	261464.70614559\\
2.26285657141429	261515.126431562\\
2.26295657391435	261565.546717533\\
2.26305657641441	261615.39404571\\
2.26315657891447	261665.814331681\\
2.26325658141454	261715.661659857\\
2.2633565839146	261766.081945829\\
2.26345658641466	261816.5022318\\
2.26355658891472	261866.349559977\\
2.26365659141479	261916.769845948\\
2.26375659391485	261966.617174125\\
2.26385659641491	262017.037460096\\
2.26395659891497	262066.884788273\\
2.26405660141504	262117.305074244\\
2.2641566039151	262167.152402421\\
2.26425660641516	262217.572688392\\
2.26435660891522	262267.992974364\\
2.26445661141529	262317.84030254\\
2.26455661391535	262368.260588511\\
2.26465661641541	262418.107916688\\
2.26475661891547	262468.528202659\\
2.26485662141554	262518.375530836\\
2.2649566239156	262568.795816807\\
2.26505662641566	262618.643144984\\
2.26515662891572	262669.063430955\\
2.26525663141579	262718.910759132\\
2.26535663391585	262769.331045103\\
2.26545663641591	262819.178373279\\
2.26555663891597	262869.598659251\\
2.26565664141604	262919.445987427\\
2.2657566439161	262969.866273399\\
2.26585664641616	263019.713601575\\
2.26595664891622	263070.133887547\\
2.26605665141629	263119.981215723\\
2.26615665391635	263170.401501695\\
2.26625665641641	263220.248829871\\
2.26635665891647	263270.669115842\\
2.26645666141654	263320.516444019\\
2.2665566639166	263370.93672999\\
2.26665666641666	263420.784058167\\
2.26675666891672	263470.631386343\\
2.26685667141679	263521.051672315\\
2.26695667391685	263570.899000491\\
2.26705667641691	263621.319286463\\
2.26715667891697	263671.166614639\\
2.26725668141704	263721.58690061\\
2.2673566839171	263771.434228787\\
2.26745668641716	263821.854514758\\
2.26755668891722	263871.701842935\\
2.26765669141729	263921.549171111\\
2.26775669391735	263971.969457083\\
2.26785669641741	264021.816785259\\
2.26795669891747	264072.237071231\\
2.26805670141754	264122.084399407\\
2.2681567039176	264171.931727583\\
2.26825670641766	264222.352013555\\
2.26835670891772	264272.199341731\\
2.26845671141779	264322.619627703\\
2.26855671391785	264372.466955879\\
2.26865671641791	264422.314284055\\
2.26875671891797	264472.734570027\\
2.26885672141804	264522.581898203\\
2.2689567239181	264572.42922638\\
2.26905672641816	264622.849512351\\
2.26915672891822	264672.696840528\\
2.26925673141829	264722.544168704\\
2.26935673391835	264772.964454676\\
2.26945673641841	264822.811782852\\
2.26955673891847	264872.659111028\\
2.26965674141854	264923.079397\\
2.2697567439186	264972.926725176\\
2.26985674641866	265022.774053353\\
2.26995674891872	265073.194339324\\
2.27005675141879	265123.0416675\\
2.27015675391885	265172.888995677\\
2.27025675641891	265223.309281648\\
2.27035675891897	265273.156609825\\
2.27045676141904	265323.003938001\\
2.2705567639191	265373.424223973\\
2.27065676641916	265423.271552149\\
2.27075676891922	265473.118880325\\
2.27085677141929	265522.966208502\\
2.27095677391935	265573.386494473\\
2.27105677641941	265623.23382265\\
2.27115677891947	265673.081150826\\
2.27125678141954	265723.501436798\\
2.2713567839196	265773.348764974\\
2.27145678641966	265823.19609315\\
2.27155678891972	265873.043421327\\
2.27165679141979	265923.463707298\\
2.27175679391985	265973.311035475\\
2.27185679641991	266023.158363651\\
2.27195679891997	266073.005691827\\
2.27205680142004	266122.853020004\\
2.2721568039201	266173.273305975\\
2.27225680642016	266223.120634152\\
2.27235680892022	266272.967962328\\
2.27245681142029	266322.815290504\\
2.27255681392035	266373.235576476\\
2.27265681642041	266423.082904652\\
2.27275681892047	266472.930232829\\
2.27285682142054	266522.777561005\\
2.2729568239206	266572.624889181\\
2.27305682642066	266622.472217358\\
2.27315682892072	266672.892503329\\
2.27325683142079	266722.739831506\\
2.27335683392085	266772.587159682\\
2.27345683642091	266822.434487858\\
2.27355683892097	266872.281816035\\
2.27365684142104	266922.702102006\\
2.2737568439211	266972.549430183\\
2.27385684642116	267022.396758359\\
2.27395684892122	267072.244086535\\
2.27405685142129	267122.091414712\\
2.27415685392135	267171.938742888\\
2.27425685642141	267221.786071065\\
2.27435685892147	267271.633399241\\
2.27445686142154	267322.053685213\\
2.2745568639216	267371.901013389\\
2.27465686642166	267421.748341565\\
2.27475686892172	267471.595669742\\
2.27485687142179	267521.442997918\\
2.27495687392185	267571.290326094\\
2.27505687642191	267621.137654271\\
2.27515687892197	267670.984982447\\
2.27525688142204	267720.832310624\\
2.2753568839221	267770.6796388\\
2.27545688642216	267821.099924771\\
2.27555688892222	267870.947252948\\
2.27565689142229	267920.794581124\\
2.27575689392235	267970.641909301\\
2.27585689642241	268020.489237477\\
2.27595689892247	268070.336565653\\
2.27605690142254	268120.18389383\\
2.2761569039226	268170.031222006\\
2.27625690642266	268219.878550183\\
2.27635690892272	268269.725878359\\
2.27645691142279	268319.573206535\\
2.27655691392285	268369.420534712\\
2.27665691642291	268419.267862888\\
2.27675691892297	268469.115191064\\
2.27685692142304	268518.962519241\\
2.2769569239231	268568.809847417\\
2.27705692642316	268618.657175594\\
2.27715692892322	268668.50450377\\
2.27725693142329	268718.351831946\\
2.27735693392335	268768.199160123\\
2.27745693642341	268818.046488299\\
2.27755693892347	268867.893816476\\
2.27765694142354	268917.741144652\\
2.2777569439236	268967.588472828\\
2.27785694642366	269017.435801005\\
2.27795694892372	269067.283129181\\
2.27805695142379	269117.130457357\\
2.27815695392385	269166.977785534\\
2.27825695642391	269216.82511371\\
2.27835695892397	269266.672441887\\
2.27845696142404	269315.946812268\\
2.2785569639241	269365.794140444\\
2.27865696642416	269415.641468621\\
2.27875696892422	269465.488796797\\
2.27885697142429	269515.336124973\\
2.27895697392435	269565.18345315\\
2.27905697642441	269615.030781326\\
2.27915697892447	269664.878109502\\
2.27925698142454	269714.725437679\\
2.2793569839246	269764.572765855\\
2.27945698642466	269813.847136236\\
2.27955698892472	269863.694464413\\
2.27965699142479	269913.541792589\\
2.27975699392485	269963.389120766\\
2.27985699642491	270013.236448942\\
2.27995699892497	270063.083777118\\
2.28005700142504	270112.931105295\\
2.2801570039251	270162.205475676\\
2.28025700642516	270212.052803852\\
2.28035700892522	270261.900132029\\
2.28045701142529	270311.747460205\\
2.28055701392535	270361.594788382\\
2.28065701642541	270411.442116558\\
2.28075701892547	270460.716486939\\
2.28085702142554	270510.563815116\\
2.2809570239256	270560.411143292\\
2.28105702642566	270610.258471468\\
2.28115702892572	270660.105799645\\
2.28125703142579	270709.380170026\\
2.28135703392585	270759.227498202\\
2.28145703642591	270809.074826379\\
2.28155703892597	270858.922154555\\
2.28165704142604	270908.196524936\\
2.2817570439261	270958.043853113\\
2.28185704642616	271007.891181289\\
2.28195704892622	271057.738509466\\
2.28205705142629	271107.012879847\\
2.28215705392635	271156.860208023\\
2.28225705642641	271206.7075362\\
2.28235705892647	271256.554864376\\
2.28245706142654	271305.829234757\\
2.2825570639266	271355.676562934\\
2.28265706642666	271405.52389111\\
2.28275706892672	271454.798261491\\
2.28285707142679	271504.645589667\\
2.28295707392685	271554.492917844\\
2.28305707642691	271604.34024602\\
2.28315707892697	271653.614616402\\
2.28325708142704	271703.461944578\\
2.2833570839271	271753.309272754\\
2.28345708642716	271802.583643136\\
2.28355708892722	271852.430971312\\
2.28365709142729	271902.278299488\\
2.28375709392735	271951.55266987\\
2.28385709642741	272001.399998046\\
2.28395709892747	272050.674368427\\
2.28405710142754	272100.521696604\\
2.2841571039276	272150.36902478\\
2.28425710642766	272199.643395161\\
2.28435710892772	272249.490723338\\
2.28445711142779	272299.338051514\\
2.28455711392785	272348.612421895\\
2.28465711642791	272398.459750072\\
2.28475711892797	272447.734120453\\
2.28485712142804	272497.581448629\\
2.2849571239281	272547.428776806\\
2.28505712642816	272596.703147187\\
2.28515712892822	272646.550475363\\
2.28525713142829	272695.824845745\\
2.28535713392835	272745.672173921\\
2.28545713642841	272794.946544302\\
2.28555713892847	272844.793872479\\
2.28565714142854	272894.06824286\\
2.2857571439286	272943.915571036\\
2.28585714642866	272993.762899213\\
2.28595714892872	273043.037269594\\
2.28605715142879	273092.88459777\\
2.28615715392885	273142.158968151\\
2.28625715642891	273192.006296328\\
2.28635715892897	273241.280666709\\
2.28645716142904	273291.127994885\\
2.2865571639291	273340.402365267\\
2.28665716642916	273390.249693443\\
2.28675716892922	273439.524063824\\
2.28685717142929	273489.371392001\\
2.28695717392935	273538.645762382\\
2.28705717642941	273588.493090558\\
2.28715717892947	273637.76746094\\
2.28725718142954	273687.041831321\\
2.2873571839296	273736.889159497\\
2.28745718642966	273786.163529879\\
2.28755718892972	273836.010858055\\
2.28765719142979	273885.285228436\\
2.28775719392985	273935.132556612\\
2.28785719642991	273984.406926994\\
2.28795719892997	274033.681297375\\
2.28805720143004	274083.528625551\\
2.2881572039301	274132.802995933\\
2.28825720643016	274182.650324109\\
2.28835720893022	274231.92469449\\
2.28845721143029	274281.199064871\\
2.28855721393035	274331.046393048\\
2.28865721643041	274380.320763429\\
2.28875721893047	274429.59513381\\
2.28885722143054	274479.442461987\\
2.2889572239306	274528.716832368\\
2.28905722643066	274578.564160544\\
2.28915722893072	274627.838530926\\
2.28925723143079	274677.112901307\\
2.28935723393085	274726.960229483\\
2.28945723643091	274776.234599865\\
2.28955723893097	274825.508970246\\
2.28965724143104	274875.356298422\\
2.2897572439311	274924.630668803\\
2.28985724643116	274973.905039185\\
2.28995724893122	275023.179409566\\
2.29005725143129	275073.026737742\\
2.29015725393135	275122.301108124\\
2.29025725643141	275171.575478505\\
2.29035725893147	275221.422806681\\
2.29045726143154	275270.697177062\\
2.2905572639316	275319.971547444\\
2.29065726643166	275369.245917825\\
2.29075726893172	275419.093246001\\
2.29085727143179	275468.367616383\\
2.29095727393185	275517.641986764\\
2.29105727643191	275566.916357145\\
2.29115727893197	275616.190727526\\
2.29125728143204	275666.038055703\\
2.2913572839321	275715.312426084\\
2.29145728643216	275764.586796465\\
2.29155728893222	275813.861166846\\
2.29165729143229	275863.135537228\\
2.29175729393235	275912.982865404\\
2.29185729643241	275962.257235785\\
2.29195729893247	276011.531606167\\
2.29205730143254	276060.805976548\\
2.2921573039326	276110.080346929\\
2.29225730643266	276159.35471731\\
2.29235730893272	276209.202045487\\
2.29245731143279	276258.476415868\\
2.29255731393285	276307.750786249\\
2.29265731643291	276357.02515663\\
2.29275731893297	276406.299527012\\
2.29285732143304	276455.573897393\\
2.2929573239331	276504.848267774\\
2.29305732643316	276554.122638155\\
2.29315732893322	276603.969966332\\
2.29325733143329	276653.244336713\\
2.29335733393335	276702.518707094\\
2.29345733643341	276751.793077476\\
2.29355733893347	276801.067447857\\
2.29365734143354	276850.341818238\\
2.2937573439336	276899.616188619\\
2.29385734643366	276948.890559001\\
2.29395734893372	276998.164929382\\
2.29405735143379	277047.439299763\\
2.29415735393385	277096.713670144\\
2.29425735643391	277145.988040526\\
2.29435735893397	277195.262410907\\
2.29445736143404	277244.536781288\\
2.2945573639341	277293.811151669\\
2.29465736643416	277343.085522051\\
2.29475736893422	277392.359892432\\
2.29485737143429	277441.634262813\\
2.29495737393435	277490.908633194\\
2.29505737643441	277540.183003576\\
2.29515737893447	277589.457373957\\
2.29525738143454	277638.731744338\\
2.2953573839346	277688.006114719\\
2.29545738643466	277737.280485101\\
2.29555738893472	277786.554855482\\
2.29565739143479	277835.829225863\\
2.29575739393485	277885.103596244\\
2.29585739643491	277934.377966626\\
2.29595739893497	277983.652337007\\
2.29605740143504	278032.353749593\\
2.2961574039351	278081.628119974\\
2.29625740643516	278130.902490356\\
2.29635740893522	278180.176860737\\
2.29645741143529	278229.451231118\\
2.29655741393535	278278.725601499\\
2.29665741643541	278327.999971881\\
2.29675741893547	278377.274342262\\
2.29685742143554	278425.975754848\\
2.2969574239356	278475.250125229\\
2.29705742643566	278524.52449561\\
2.29715742893572	278573.798865992\\
2.29725743143579	278623.073236373\\
2.29735743393585	278672.347606754\\
2.29745743643591	278721.04901934\\
2.29755743893597	278770.323389722\\
2.29765744143604	278819.597760103\\
2.2977574439361	278868.872130484\\
2.29785744643616	278918.146500865\\
2.29795744893622	278966.847913451\\
2.29805745143629	279016.122283833\\
2.29815745393635	279065.396654214\\
2.29825745643641	279114.671024595\\
2.29835745893647	279163.372437181\\
2.29845746143654	279212.646807563\\
2.2985574639366	279261.921177944\\
2.29865746643666	279311.195548325\\
2.29875746893672	279359.896960911\\
2.29885747143679	279409.171331292\\
2.29895747393685	279458.445701674\\
2.29905747643691	279507.720072055\\
2.29915747893697	279556.421484641\\
2.29925748143704	279605.695855022\\
2.2993574839371	279654.970225403\\
2.29945748643716	279703.67163799\\
2.29955748893722	279752.946008371\\
2.29965749143729	279802.220378752\\
2.29975749393735	279850.921791338\\
2.29985749643741	279900.19616172\\
2.29995749893747	279949.470532101\\
2.30005750143754	279998.171944687\\
2.3001575039376	280047.446315068\\
2.30025750643766	280096.147727654\\
2.30035750893772	280145.422098036\\
2.30045751143779	280194.696468417\\
2.30055751393785	280243.397881003\\
2.30065751643791	280292.672251384\\
2.30075751893797	280341.37366397\\
2.30085752143804	280390.648034351\\
2.3009575239381	280439.922404733\\
2.30105752643816	280488.623817319\\
2.30115752893822	280537.8981877\\
2.30125753143829	280586.599600286\\
2.30135753393835	280635.873970667\\
2.30145753643841	280684.575383254\\
2.30155753893847	280733.849753635\\
2.30165754143854	280782.551166221\\
2.3017575439386	280831.825536602\\
2.30185754643866	280880.526949188\\
2.30195754893872	280929.80131957\\
2.30205755143879	280978.502732156\\
2.30215755393885	281027.777102537\\
2.30225755643891	281076.478515123\\
2.30235755893897	281125.752885504\\
2.30245756143904	281174.45429809\\
2.3025575639391	281223.728668472\\
2.30265756643916	281272.430081058\\
2.30275756893922	281321.704451439\\
2.30285757143929	281370.405864025\\
2.30295757393935	281419.680234406\\
2.30305757643941	281468.381646993\\
2.30315757893947	281517.083059579\\
2.30325758143954	281566.35742996\\
2.3033575839396	281615.058842546\\
2.30345758643966	281664.333212927\\
2.30355758893972	281713.034625513\\
2.30365759143979	281761.7360381\\
2.30375759393985	281811.010408481\\
2.30385759643991	281859.711821067\\
2.30395759893997	281908.413233653\\
2.30405760144004	281957.687604034\\
2.3041576039401	282006.38901662\\
2.30425760644016	282055.090429206\\
2.30435760894022	282104.364799588\\
2.30445761144029	282153.066212174\\
2.30455761394035	282201.76762476\\
2.30465761644041	282251.041995141\\
2.30475761894047	282299.743407727\\
2.30485762144054	282348.444820313\\
2.3049576239406	282397.719190695\\
2.30505762644066	282446.420603281\\
2.30515762894072	282495.122015867\\
2.30525763144079	282543.823428453\\
2.30535763394085	282593.097798834\\
2.30545763644091	282641.79921142\\
2.30555763894097	282690.500624007\\
2.30565764144104	282739.202036593\\
2.3057576439411	282788.476406974\\
2.30585764644116	282837.17781956\\
2.30595764894122	282885.879232146\\
2.30605765144129	282934.580644732\\
2.30615765394135	282983.282057318\\
2.30625765644141	283032.5564277\\
2.30635765894147	283081.257840286\\
2.30645766144154	283129.959252872\\
2.3065576639416	283178.660665458\\
2.30665766644166	283227.362078044\\
2.30675766894172	283276.06349063\\
2.30685767144179	283325.337861012\\
2.30695767394185	283374.039273598\\
2.30705767644191	283422.740686184\\
2.30715767894197	283471.44209877\\
2.30725768144204	283520.143511356\\
2.3073576839421	283568.844923942\\
2.30745768644216	283617.546336528\\
2.30755768894222	283666.247749114\\
2.30765769144229	283714.949161701\\
2.30775769394235	283763.650574287\\
2.30785769644241	283812.924944668\\
2.30795769894247	283861.626357254\\
2.30805770144254	283910.32776984\\
2.3081577039426	283959.029182426\\
2.30825770644266	284007.730595012\\
2.30835770894272	284056.432007599\\
2.30845771144279	284105.133420185\\
2.30855771394285	284153.834832771\\
2.30865771644291	284202.536245357\\
2.30875771894297	284251.237657943\\
2.30885772144304	284299.939070529\\
2.3089577239431	284348.640483115\\
2.30905772644316	284397.341895701\\
2.30915772894322	284446.043308288\\
2.30925773144329	284494.744720874\\
2.30935773394335	284543.44613346\\
2.30945773644341	284591.574588251\\
2.30955773894347	284640.276000837\\
2.30965774144354	284688.977413423\\
2.3097577439436	284737.678826009\\
2.30985774644366	284786.380238595\\
2.30995774894372	284835.081651181\\
2.31005775144379	284883.783063767\\
2.31015775394385	284932.484476354\\
2.31025775644391	284981.18588894\\
2.31035775894397	285029.887301526\\
2.31045776144404	285078.015756317\\
2.3105577639441	285126.717168903\\
2.31065776644416	285175.418581489\\
2.31075776894422	285224.119994075\\
2.31085777144429	285272.821406661\\
2.31095777394435	285321.522819247\\
2.31105777644441	285369.651274038\\
2.31115777894447	285418.352686624\\
2.31125778144454	285467.054099211\\
2.3113577839446	285515.755511797\\
2.31145778644466	285564.456924383\\
2.31155778894472	285612.585379174\\
2.31165779144479	285661.28679176\\
2.31175779394485	285709.988204346\\
2.31185779644491	285758.689616932\\
2.31195779894497	285806.818071723\\
2.31205780144504	285855.519484309\\
2.3121578039451	285904.220896895\\
2.31225780644516	285952.922309482\\
2.31235780894522	286001.050764273\\
2.31245781144529	286049.752176859\\
2.31255781394535	286098.453589445\\
2.31265781644541	286146.582044236\\
2.31275781894547	286195.283456822\\
2.31285782144554	286243.984869408\\
2.3129578239456	286292.113324199\\
2.31305782644566	286340.814736785\\
2.31315782894572	286389.516149371\\
2.31325783144579	286437.644604162\\
2.31335783394585	286486.346016748\\
2.31345783644591	286534.474471539\\
2.31355783894597	286583.175884125\\
2.31365784144604	286631.877296712\\
2.3137578439461	286680.005751503\\
2.31385784644616	286728.707164089\\
2.31395784894622	286776.83561888\\
2.31405785144629	286825.537031466\\
2.31415785394635	286874.238444052\\
2.31425785644641	286922.366898843\\
2.31435785894647	286971.068311429\\
2.31445786144654	287019.19676622\\
2.3145578639466	287067.898178806\\
2.31465786644666	287116.026633597\\
2.31475786894672	287164.728046183\\
2.31485787144679	287212.856500974\\
2.31495787394685	287261.55791356\\
2.31505787644691	287309.686368351\\
2.31515787894697	287358.387780937\\
2.31525788144704	287406.516235728\\
2.3153578839471	287455.217648315\\
2.31545788644716	287503.346103106\\
2.31555788894722	287552.047515692\\
2.31565789144729	287600.175970483\\
2.31575789394735	287648.304425274\\
2.31585789644741	287697.00583786\\
2.31595789894747	287745.134292651\\
2.31605790144754	287793.835705237\\
2.3161579039476	287841.964160028\\
2.31625790644766	287890.092614819\\
2.31635790894772	287938.794027405\\
2.31645791144779	287986.922482196\\
2.31655791394785	288035.050936987\\
2.31665791644791	288083.752349573\\
2.31675791894797	288131.880804364\\
2.31685792144804	288180.009259155\\
2.3169579239481	288228.710671741\\
2.31705792644816	288276.839126532\\
2.31715792894822	288324.967581323\\
2.31725793144829	288373.668993909\\
2.31735793394835	288421.7974487\\
2.31745793644841	288469.925903491\\
2.31755793894847	288518.627316077\\
2.31765794144854	288566.755770868\\
2.3177579439486	288614.884225659\\
2.31785794644866	288663.01268045\\
2.31795794894872	288711.714093036\\
2.31805795144879	288759.842547827\\
2.31815795394885	288807.971002618\\
2.31825795644891	288856.099457409\\
2.31835795894897	288904.2279122\\
2.31845796144904	288952.929324787\\
2.3185579639491	289001.057779578\\
2.31865796644916	289049.186234368\\
2.31875796894922	289097.314689159\\
2.31885797144929	289145.44314395\\
2.31895797394935	289193.571598741\\
2.31905797644941	289242.273011328\\
2.31915797894947	289290.401466119\\
2.31925798144954	289338.52992091\\
2.3193579839496	289386.658375701\\
2.31945798644966	289434.786830492\\
2.31955798894972	289482.915285283\\
2.31965799144979	289531.043740074\\
2.31975799394985	289579.172194864\\
2.31985799644991	289627.300649656\\
2.31995799894997	289675.429104446\\
2.32005800145004	289724.130517033\\
2.3201580039501	289772.258971824\\
2.32025800645016	289820.387426615\\
2.32035800895022	289868.515881406\\
2.32045801145029	289916.644336197\\
2.32055801395035	289964.772790988\\
2.32065801645041	290012.901245779\\
2.32075801895047	290061.02970057\\
2.32085802145054	290109.15815536\\
2.3209580239506	290157.286610152\\
2.32105802645066	290205.415064942\\
2.32115802895072	290253.543519733\\
2.32125803145079	290301.099016729\\
2.32135803395085	290349.22747152\\
2.32145803645091	290397.355926311\\
2.32155803895097	290445.484381102\\
2.32165804145104	290493.612835893\\
2.3217580439511	290541.741290684\\
2.32185804645116	290589.869745475\\
2.32195804895122	290637.998200266\\
2.32205805145129	290686.126655057\\
2.32215805395135	290734.255109848\\
2.32225805645141	290781.810606844\\
2.32235805895147	290829.939061635\\
2.32245806145154	290878.067516426\\
2.3225580639516	290926.195971217\\
2.32265806645166	290974.324426008\\
2.32275806895172	291022.452880799\\
2.32285807145179	291070.008377795\\
2.32295807395185	291118.136832586\\
2.32305807645191	291166.265287377\\
2.32315807895197	291214.393742168\\
2.32325808145204	291261.949239164\\
2.3233580839521	291310.077693955\\
2.32345808645216	291358.206148746\\
2.32355808895222	291406.334603537\\
2.32365809145229	291453.890100533\\
2.32375809395235	291502.018555324\\
2.32385809645241	291550.147010115\\
2.32395809895247	291598.275464906\\
2.32405810145254	291645.830961901\\
2.3241581039526	291693.959416692\\
2.32425810645266	291742.087871483\\
2.32435810895272	291789.643368479\\
2.32445811145279	291837.77182327\\
2.32455811395285	291885.900278061\\
2.32465811645291	291933.455775057\\
2.32475811895297	291981.584229848\\
2.32485812145304	292029.139726844\\
2.3249581239531	292077.268181635\\
2.32505812645316	292125.396636426\\
2.32515812895322	292172.952133422\\
2.32525813145329	292221.080588213\\
2.32535813395335	292268.636085209\\
2.32545813645341	292316.76454\\
2.32555813895347	292364.320036995\\
2.32565814145354	292412.448491786\\
2.3257581439536	292460.576946577\\
2.32585814645366	292508.132443573\\
2.32595814895372	292556.260898364\\
2.32605815145379	292603.81639536\\
2.32615815395385	292651.944850151\\
2.32625815645391	292699.500347147\\
2.32635815895397	292747.628801938\\
2.32645816145404	292795.184298934\\
2.3265581639541	292842.73979593\\
2.32665816645416	292890.868250721\\
2.32675816895422	292938.423747717\\
2.32685817145429	292986.552202508\\
2.32695817395435	293034.107699503\\
2.32705817645441	293082.236154294\\
2.32715817895447	293129.79165129\\
2.32725818145454	293177.347148286\\
2.3273581839546	293225.475603077\\
2.32745818645466	293273.031100073\\
2.32755818895472	293320.586597069\\
2.32765819145479	293368.71505186\\
2.32775819395485	293416.270548856\\
2.32785819645491	293464.399003647\\
2.32795819895497	293511.954500642\\
2.32805820145504	293559.509997638\\
2.3281582039551	293607.065494634\\
2.32825820645516	293655.193949425\\
2.32835820895522	293702.749446421\\
2.32845821145529	293750.304943417\\
2.32855821395535	293798.433398208\\
2.32865821645541	293845.988895204\\
2.32875821895547	293893.5443922\\
2.32885822145554	293941.099889195\\
2.3289582239556	293989.228343986\\
2.32905822645566	294036.783840982\\
2.32915822895572	294084.339337978\\
2.32925823145579	294131.894834974\\
2.32935823395585	294179.45033197\\
2.32945823645591	294227.578786761\\
2.32955823895597	294275.134283757\\
2.32965824145604	294322.689780753\\
2.3297582439561	294370.245277748\\
2.32985824645616	294417.800774744\\
2.32995824895622	294465.35627174\\
2.33005825145629	294512.911768736\\
2.33015825395635	294560.467265732\\
2.33025825645641	294608.595720523\\
2.33035825895647	294656.151217519\\
2.33045826145654	294703.706714515\\
2.3305582639566	294751.26221151\\
2.33065826645666	294798.817708506\\
2.33075826895672	294846.373205502\\
2.33085827145679	294893.928702498\\
2.33095827395685	294941.484199494\\
2.33105827645691	294989.03969649\\
2.33115827895697	295036.595193486\\
2.33125828145704	295084.150690481\\
2.3313582839571	295131.706187477\\
2.33145828645716	295179.261684473\\
2.33155828895722	295226.817181469\\
2.33165829145729	295274.372678465\\
2.33175829395735	295321.928175461\\
2.33185829645741	295369.483672457\\
2.33195829895747	295417.039169452\\
2.33205830145754	295464.594666448\\
2.3321583039576	295512.150163444\\
2.33225830645766	295559.132702645\\
2.33235830895772	295606.688199641\\
2.33245831145779	295654.243696637\\
2.33255831395785	295701.799193632\\
2.33265831645791	295749.354690628\\
2.33275831895797	295796.910187624\\
2.33285832145804	295844.46568462\\
2.3329583239581	295891.448223821\\
2.33305832645816	295939.003720817\\
2.33315832895822	295986.559217812\\
2.33325833145829	296034.114714808\\
2.33335833395835	296081.670211804\\
2.33345833645841	296128.652751005\\
2.33355833895847	296176.208248001\\
2.33365834145854	296223.763744997\\
2.3337583439586	296271.319241992\\
2.33385834645866	296318.301781193\\
2.33395834895872	296365.857278189\\
2.33405835145879	296413.412775185\\
2.33415835395885	296460.968272181\\
2.33425835645891	296507.950811382\\
2.33435835895897	296555.506308377\\
2.33445836145904	296603.061805373\\
2.3345583639591	296650.044344574\\
2.33465836645916	296697.59984157\\
2.33475836895922	296745.155338566\\
2.33485837145929	296792.137877766\\
2.33495837395935	296839.693374762\\
2.33505837645941	296886.675913963\\
2.33515837895947	296934.231410959\\
2.33525838145954	296981.786907955\\
2.3353583839596	297028.769447155\\
2.33545838645966	297076.324944151\\
2.33555838895972	297123.307483352\\
2.33565839145979	297170.862980348\\
2.33575839395985	297218.418477344\\
2.33585839645991	297265.401016544\\
2.33595839895997	297312.95651354\\
2.33605840146004	297359.939052741\\
2.3361584039601	297407.494549737\\
2.33625840646016	297454.477088938\\
2.33635840896022	297502.032585934\\
2.33645841146029	297549.015125134\\
2.33655841396035	297595.997664335\\
2.33665841646041	297643.553161331\\
2.33675841896047	297690.535700532\\
2.33685842146054	297738.091197527\\
2.3369584239606	297785.073736728\\
2.33705842646066	297832.629233724\\
2.33715842896072	297879.611772925\\
2.33725843146079	297926.594312125\\
2.33735843396085	297974.149809121\\
2.33745843646091	298021.132348322\\
2.33755843896097	298068.114887523\\
2.33765844146104	298115.670384519\\
2.3377584439611	298162.652923719\\
2.33785844646116	298209.63546292\\
2.33795844896122	298257.190959916\\
2.33805845146129	298304.173499117\\
2.33815845396135	298351.156038317\\
2.33825845646141	298398.711535313\\
2.33835845896147	298445.694074514\\
2.33845846146154	298492.676613715\\
2.3385584639616	298539.659152915\\
2.33865846646166	298587.214649911\\
2.33875846896172	298634.197189112\\
2.33885847146179	298681.179728313\\
2.33895847396185	298728.162267513\\
2.33905847646191	298775.144806714\\
2.33915847896197	298822.70030371\\
2.33925848146204	298869.682842911\\
2.3393584839621	298916.665382112\\
2.33945848646216	298963.647921312\\
2.33955848896222	299010.630460513\\
2.33965849146229	299057.612999714\\
2.33975849396235	299105.16849671\\
2.33985849646241	299152.15103591\\
2.33995849896247	299199.133575111\\
2.34005850146254	299246.116114312\\
2.3401585039626	299293.098653512\\
2.34025850646266	299340.081192713\\
2.34035850896272	299387.063731914\\
2.34045851146279	299434.046271115\\
2.34055851396285	299481.028810315\\
2.34065851646291	299528.011349516\\
2.34075851896297	299574.993888717\\
2.34085852146304	299621.976427918\\
2.3409585239631	299668.958967118\\
2.34105852646316	299715.941506319\\
2.34115852896322	299762.92404552\\
2.34125853146329	299809.90658472\\
2.34135853396335	299856.889123921\\
2.34145853646341	299903.871663122\\
2.34155853896347	299950.854202323\\
2.34165854146354	299997.836741523\\
2.3417585439636	300044.819280724\\
2.34185854646366	300091.801819925\\
2.34195854896372	300138.21140133\\
2.34205855146379	300185.193940531\\
2.34215855396385	300232.176479732\\
2.34225855646391	300279.159018933\\
2.34235855896397	300326.141558133\\
2.34245856146404	300373.124097334\\
2.3425585639641	300419.53367874\\
2.34265856646416	300466.51621794\\
2.34275856896422	300513.498757141\\
2.34285857146429	300560.481296342\\
2.34295857396435	300607.463835543\\
2.34305857646441	300653.873416948\\
2.34315857896447	300700.855956149\\
2.34325858146454	300747.83849535\\
2.3433585839646	300794.82103455\\
2.34345858646466	300841.230615956\\
2.34355858896472	300888.213155157\\
2.34365859146479	300935.195694357\\
2.34375859396485	300981.605275763\\
2.34385859646491	301028.587814964\\
2.34395859896497	301075.570354164\\
2.34405860146504	301121.97993557\\
2.3441586039651	301168.962474771\\
2.34425860646516	301215.945013972\\
2.34435860896522	301262.354595377\\
2.34445861146529	301309.337134578\\
2.34455861396535	301355.746715983\\
2.34465861646541	301402.729255184\\
2.34475861896547	301449.711794385\\
2.34485862146554	301496.121375791\\
2.3449586239656	301543.103914991\\
2.34505862646566	301589.513496397\\
2.34515862896572	301636.496035598\\
2.34525863146579	301682.905617003\\
2.34535863396585	301729.888156204\\
2.34545863646591	301776.297737609\\
2.34555863896597	301823.28027681\\
2.34565864146604	301869.689858216\\
2.3457586439661	301916.672397416\\
2.34585864646616	301963.081978822\\
2.34595864896622	302009.491560228\\
2.34605865146629	302056.474099428\\
2.34615865396635	302102.883680834\\
2.34625865646641	302149.866220035\\
2.34635865896647	302196.27580144\\
2.34645866146654	302242.685382846\\
2.3465586639666	302289.667922047\\
2.34665866646666	302336.077503452\\
2.34675866896672	302382.487084858\\
2.34685867146679	302429.469624059\\
2.34695867396685	302475.879205464\\
2.34705867646691	302522.28878687\\
2.34715867896697	302569.271326071\\
2.34725868146704	302615.680907476\\
2.3473586839671	302662.090488882\\
2.34745868646716	302709.073028082\\
2.34755868896722	302755.482609488\\
2.34765869146729	302801.892190894\\
2.34775869396735	302848.301772299\\
2.34785869646741	302894.711353705\\
2.34795869896747	302941.693892906\\
2.34805870146754	302988.103474311\\
2.3481587039676	303034.513055717\\
2.34825870646766	303080.922637122\\
2.34835870896772	303127.332218528\\
2.34845871146779	303174.314757729\\
2.34855871396785	303220.724339134\\
2.34865871646791	303267.13392054\\
2.34875871896797	303313.543501945\\
2.34885872146804	303359.953083351\\
2.3489587239681	303406.362664757\\
2.34905872646816	303452.772246162\\
2.34915872896822	303499.181827568\\
2.34925873146829	303545.591408973\\
2.34935873396835	303592.000990379\\
2.34945873646841	303638.410571785\\
2.34955873896847	303684.82015319\\
2.34965874146854	303731.229734596\\
2.3497587439686	303777.639316001\\
2.34985874646866	303824.048897407\\
2.34995874896872	303870.458478813\\
2.35005875146879	303916.868060218\\
2.35015875396885	303963.277641624\\
2.35025875646891	304009.687223029\\
2.35035875896897	304056.096804435\\
2.35045876146904	304102.506385841\\
2.3505587639691	304148.915967246\\
2.35065876646916	304195.325548652\\
2.35075876896922	304241.735130057\\
2.35085877146929	304287.571753668\\
2.35095877396935	304333.981335073\\
2.35105877646941	304380.390916479\\
2.35115877896947	304426.800497885\\
2.35125878146954	304473.21007929\\
2.3513587839696	304519.619660696\\
2.35145878646966	304565.456284306\\
2.35155878896972	304611.865865712\\
2.35165879146979	304658.275447118\\
2.35175879396985	304704.685028523\\
2.35185879646991	304750.521652134\\
2.35195879896997	304796.931233539\\
2.35205880147004	304843.340814945\\
2.3521588039701	304889.75039635\\
2.35225880647016	304935.587019961\\
2.35235880897022	304981.996601366\\
2.35245881147029	305028.406182772\\
2.35255881397035	305074.242806382\\
2.35265881647041	305120.652387788\\
2.35275881897047	305167.061969194\\
2.35285882147054	305212.898592804\\
2.3529588239706	305259.30817421\\
2.35305882647066	305305.14479782\\
2.35315882897072	305351.554379226\\
2.35325883147079	305397.963960631\\
2.35335883397085	305443.800584242\\
2.35345883647091	305490.210165647\\
2.35355883897097	305536.046789258\\
2.35365884147104	305582.456370664\\
2.3537588439711	305628.292994274\\
2.35385884647116	305674.70257568\\
2.35395884897122	305720.53919929\\
2.35405885147129	305766.948780696\\
2.35415885397135	305812.785404306\\
2.35425885647141	305859.194985712\\
2.35435885897147	305905.031609322\\
2.35445886147154	305951.441190728\\
2.3545588639716	305997.277814338\\
2.35465886647166	306043.114437949\\
2.35475886897172	306089.524019354\\
2.35485887147179	306135.360642965\\
2.35495887397185	306181.197266575\\
2.35505887647191	306227.606847981\\
2.35515887897197	306273.443471591\\
2.35525888147204	306319.853052997\\
2.3553588839721	306365.689676607\\
2.35545888647216	306411.526300218\\
2.35555888897222	306457.362923828\\
2.35565889147229	306503.772505234\\
2.35575889397235	306549.609128844\\
2.35585889647241	306595.445752455\\
2.35595889897247	306641.282376065\\
2.35605890147254	306687.691957471\\
2.3561589039726	306733.528581081\\
2.35625890647266	306779.365204692\\
2.35635890897272	306825.201828302\\
2.35645891147279	306871.611409708\\
2.35655891397285	306917.448033318\\
2.35665891647291	306963.284656929\\
2.35675891897297	307009.121280539\\
2.35685892147304	307054.95790415\\
2.3569589239731	307100.79452776\\
2.35705892647316	307146.631151371\\
2.35715892897322	307192.467774981\\
2.35725893147329	307238.877356387\\
2.35735893397335	307284.713979997\\
2.35745893647341	307330.550603608\\
2.35755893897347	307376.387227218\\
2.35765894147354	307422.223850829\\
2.3577589439736	307468.060474439\\
2.35785894647366	307513.89709805\\
2.35795894897372	307559.73372166\\
2.35805895147379	307605.57034527\\
2.35815895397385	307651.406968881\\
2.35825895647391	307697.243592491\\
2.35835895897397	307743.080216102\\
2.35845896147404	307788.916839712\\
2.3585589639741	307834.180505528\\
2.35865896647416	307880.017129138\\
2.35875896897422	307925.853752749\\
2.35885897147429	307971.690376359\\
2.35895897397435	308017.52699997\\
2.35905897647441	308063.36362358\\
2.35915897897447	308109.20024719\\
2.35925898147454	308155.036870801\\
2.3593589839746	308200.300536616\\
2.35945898647466	308246.137160227\\
2.35955898897472	308291.973783837\\
2.35965899147479	308337.810407448\\
2.35975899397485	308383.647031058\\
2.35985899647491	308428.910696873\\
2.35995899897497	308474.747320484\\
2.36005900147504	308520.583944094\\
2.3601590039751	308565.84760991\\
2.36025900647516	308611.68423352\\
2.36035900897522	308657.520857131\\
2.36045901147529	308703.357480741\\
2.36055901397535	308748.621146556\\
2.36065901647541	308794.457770167\\
2.36075901897547	308839.721435982\\
2.36085902147554	308885.558059593\\
2.3609590239756	308931.394683203\\
2.36105902647566	308976.658349018\\
2.36115902897572	309022.494972629\\
2.36125903147579	309067.758638444\\
2.36135903397585	309113.595262055\\
2.36145903647591	309159.431885665\\
2.36155903897597	309204.695551481\\
2.36165904147604	309250.532175091\\
2.3617590439761	309295.795840906\\
2.36185904647616	309341.632464517\\
2.36195904897622	309386.896130332\\
2.36205905147629	309432.732753943\\
2.36215905397635	309477.996419758\\
2.36225905647641	309523.833043368\\
2.36235905897647	309569.096709184\\
2.36245906147654	309614.360374999\\
2.3625590639766	309660.19699861\\
2.36265906647666	309705.460664425\\
2.36275906897672	309751.297288035\\
2.36285907147679	309796.560953851\\
2.36295907397685	309841.824619666\\
2.36305907647691	309887.661243277\\
2.36315907897697	309932.924909092\\
2.36325908147704	309978.188574907\\
2.3633590839771	310024.025198518\\
2.36345908647716	310069.288864333\\
2.36355908897722	310114.552530148\\
2.36365909147729	310159.816195964\\
2.36375909397735	310205.652819574\\
2.36385909647741	310250.916485389\\
2.36395909897747	310296.180151205\\
2.36405910147754	310341.44381702\\
2.3641591039776	310386.707482836\\
2.36425910647766	310432.544106446\\
2.36435910897772	310477.807772261\\
2.36445911147779	310523.071438077\\
2.36455911397785	310568.335103892\\
2.36465911647791	310613.598769707\\
2.36475911897797	310658.862435523\\
2.36485912147804	310704.699059133\\
2.3649591239781	310749.962724948\\
2.36505912647816	310795.226390764\\
2.36515912897822	310840.490056579\\
2.36525913147829	310885.753722394\\
2.36535913397835	310931.01738821\\
2.36545913647841	310976.281054025\\
2.36555913897847	311021.54471984\\
2.36565914147854	311066.808385656\\
2.3657591439786	311112.072051471\\
2.36585914647866	311157.335717286\\
2.36595914897872	311202.599383102\\
2.36605915147879	311247.863048917\\
2.36615915397885	311293.126714732\\
2.36625915647891	311338.390380548\\
2.36635915897897	311383.081088568\\
2.36645916147904	311428.344754383\\
2.3665591639791	311473.608420199\\
2.36665916647916	311518.872086014\\
2.36675916897922	311564.135751829\\
2.36685917147929	311609.399417645\\
2.36695917397935	311654.66308346\\
2.36705917647941	311699.35379148\\
2.36715917897947	311744.617457296\\
2.36725918147954	311789.881123111\\
2.3673591839796	311835.144788926\\
2.36745918647966	311880.408454742\\
2.36755918897972	311925.099162762\\
2.36765919147979	311970.362828577\\
2.36775919397985	312015.626494392\\
2.36785919647991	312060.317202413\\
2.36795919897997	312105.580868228\\
2.36805920148004	312150.844534043\\
2.3681592039801	312195.535242064\\
2.36825920648016	312240.798907879\\
2.36835920898022	312286.062573694\\
2.36845921148029	312330.753281714\\
2.36855921398035	312376.01694753\\
2.36865921648041	312421.280613345\\
2.36875921898047	312465.971321365\\
2.36885922148054	312511.234987181\\
2.3689592239806	312555.925695201\\
2.36905922648066	312601.189361016\\
2.36915922898072	312645.880069036\\
2.36925923148079	312691.143734852\\
2.36935923398085	312735.834442872\\
2.36945923648091	312781.098108687\\
2.36955923898097	312825.788816707\\
2.36965924148104	312871.052482523\\
2.3697592439811	312915.743190543\\
2.36985924648116	312961.006856358\\
2.36995924898122	313005.697564378\\
2.37005925148129	313050.388272399\\
2.37015925398135	313095.651938214\\
2.37025925648141	313140.342646234\\
2.37035925898147	313185.60631205\\
2.37045926148154	313230.29702007\\
2.3705592639816	313274.98772809\\
2.37065926648166	313320.251393905\\
2.37075926898172	313364.942101925\\
2.37085927148179	313409.632809946\\
2.37095927398185	313454.323517966\\
2.37105927648191	313499.587183781\\
2.37115927898197	313544.277891801\\
2.37125928148204	313588.968599822\\
2.3713592839821	313633.659307842\\
2.37145928648216	313678.922973657\\
2.37155928898222	313723.613681677\\
2.37165929148229	313768.304389698\\
2.37175929398235	313812.995097718\\
2.37185929648241	313857.685805738\\
2.37195929898247	313902.376513758\\
2.37205930148254	313947.640179574\\
2.3721593039826	313992.330887594\\
2.37225930648266	314037.021595614\\
2.37235930898272	314081.712303634\\
2.37245931148279	314126.403011654\\
2.37255931398285	314171.093719675\\
2.37265931648291	314215.784427695\\
2.37275931898297	314260.475135715\\
2.37285932148304	314305.165843735\\
2.3729593239831	314349.856551755\\
2.37305932648316	314394.547259776\\
2.37315932898322	314439.237967796\\
2.37325933148329	314483.928675816\\
2.37335933398335	314528.619383836\\
2.37345933648341	314573.310091856\\
2.37355933898347	314618.000799877\\
2.37365934148354	314662.691507897\\
2.3737593439836	314707.382215917\\
2.37385934648366	314751.499966142\\
2.37395934898372	314796.190674162\\
2.37405935148379	314840.881382183\\
2.37415935398385	314885.572090203\\
2.37425935648391	314930.262798223\\
2.37435935898397	314974.953506243\\
2.37445936148404	315019.071256468\\
2.3745593639841	315063.761964488\\
2.37465936648416	315108.452672509\\
2.37475936898422	315153.143380529\\
2.37485937148429	315197.261130754\\
2.37495937398435	315241.951838774\\
2.37505937648441	315286.642546794\\
2.37515937898447	315330.760297019\\
2.37525938148454	315375.45100504\\
2.3753593839846	315420.14171306\\
2.37545938648466	315464.259463285\\
2.37555938898472	315508.950171305\\
2.37565939148479	315553.640879325\\
2.37575939398485	315597.75862955\\
2.37585939648491	315642.449337571\\
2.37595939898497	315686.567087796\\
2.37605940148504	315731.257795816\\
2.3761594039851	315775.948503836\\
2.37625940648516	315820.066254061\\
2.37635940898522	315864.756962081\\
2.37645941148529	315908.874712306\\
2.37655941398535	315953.565420327\\
2.37665941648541	315997.683170552\\
2.37675941898547	316042.373878572\\
2.37685942148554	316086.491628797\\
2.3769594239856	316130.609379022\\
2.37705942648566	316175.300087042\\
2.37715942898572	316219.417837267\\
2.37725943148579	316264.108545288\\
2.37735943398585	316308.226295513\\
2.37745943648591	316352.344045738\\
2.37755943898597	316397.034753758\\
2.37765944148604	316441.152503983\\
2.3777594439861	316485.270254208\\
2.37785944648616	316529.960962228\\
2.37795944898622	316574.078712453\\
2.37805945148629	316618.196462678\\
2.37815945398635	316662.314212903\\
2.37825945648641	316707.004920924\\
2.37835945898647	316751.122671149\\
2.37845946148654	316795.240421374\\
2.3785594639866	316839.358171599\\
2.37865946648666	316883.475921824\\
2.37875946898672	316928.166629844\\
2.37885947148679	316972.284380069\\
2.37895947398685	317016.402130294\\
2.37905947648691	317060.519880519\\
2.37915947898697	317104.637630744\\
2.37925948148704	317148.755380969\\
2.3793594839871	317192.873131195\\
2.37945948648716	317236.99088142\\
2.37955948898722	317281.108631645\\
2.37965949148729	317325.799339665\\
2.37975949398735	317369.91708989\\
2.37985949648741	317414.034840115\\
2.37995949898747	317458.15259034\\
2.38005950148754	317502.270340565\\
2.3801595039876	317545.815132995\\
2.38025950648766	317589.93288322\\
2.38035950898772	317634.050633445\\
2.38045951148779	317678.16838367\\
2.38055951398785	317722.286133895\\
2.38065951648791	317766.40388412\\
2.38075951898797	317810.521634346\\
2.38085952148804	317854.639384571\\
2.3809595239881	317898.757134796\\
2.38105952648816	317942.301927226\\
2.38115952898822	317986.419677451\\
2.38125953148829	318030.537427676\\
2.38135953398835	318074.655177901\\
2.38145953648841	318118.772928126\\
2.38155953898847	318162.317720556\\
2.38165954148854	318206.435470781\\
2.3817595439886	318250.553221006\\
2.38185954648866	318294.670971231\\
2.38195954898872	318338.215763661\\
2.38205955148879	318382.333513886\\
2.38215955398885	318426.451264111\\
2.38225955648891	318469.996056541\\
2.38235955898897	318514.113806766\\
2.38245956148904	318557.658599196\\
2.3825595639891	318601.776349421\\
2.38265956648916	318645.894099646\\
2.38275956898922	318689.438892076\\
2.38285957148929	318733.556642301\\
2.38295957398935	318777.101434731\\
2.38305957648941	318821.219184956\\
2.38315957898947	318864.763977386\\
2.38325958148954	318908.881727611\\
2.3833595839896	318952.426520041\\
2.38345958648966	318996.544270266\\
2.38355958898972	319040.089062696\\
2.38365959148979	319084.206812921\\
2.38375959398985	319127.751605351\\
2.38385959648991	319171.869355576\\
2.38395959898997	319215.414148006\\
2.38405960149004	319258.958940436\\
2.3841596039901	319303.076690661\\
2.38425960649016	319346.621483091\\
2.38435960899022	319390.166275521\\
2.38445961149029	319434.284025746\\
2.38455961399035	319477.828818176\\
2.38465961649041	319521.373610606\\
2.38475961899047	319565.491360831\\
2.38485962149054	319609.036153261\\
2.3849596239906	319652.580945691\\
2.38505962649066	319696.125738121\\
2.38515962899072	319740.243488346\\
2.38525963149079	319783.788280776\\
2.38535963399085	319827.333073206\\
2.38545963649091	319870.877865636\\
2.38555963899097	319914.422658066\\
2.38565964149104	319957.967450496\\
2.3857596439911	320001.512242926\\
2.38585964649116	320045.629993151\\
2.38595964899122	320089.174785581\\
2.38605965149129	320132.719578011\\
2.38615965399135	320176.264370441\\
2.38625965649141	320219.809162871\\
2.38635965899147	320263.353955301\\
2.38645966149154	320306.898747731\\
2.3865596639916	320350.44354016\\
2.38665966649166	320393.98833259\\
2.38675966899172	320437.53312502\\
2.38685967149179	320481.07791745\\
2.38695967399185	320524.62270988\\
2.38705967649191	320568.16750231\\
2.38715967899197	320611.71229474\\
2.38725968149204	320654.684129375\\
2.3873596839921	320698.228921805\\
2.38745968649216	320741.773714235\\
2.38755968899222	320785.318506665\\
2.38765969149229	320828.863299095\\
2.38775969399235	320872.408091525\\
2.38785969649241	320915.379926159\\
2.38795969899247	320958.924718589\\
2.38805970149254	321002.469511019\\
2.3881597039926	321046.014303449\\
2.38825970649266	321088.986138084\\
2.38835970899272	321132.530930514\\
2.38845971149279	321176.075722944\\
2.38855971399285	321219.620515374\\
2.38865971649291	321262.592350009\\
2.38875971899297	321306.137142439\\
2.38885972149304	321349.681934869\\
2.3889597239931	321392.653769503\\
2.38905972649316	321436.198561933\\
2.38915972899322	321479.170396568\\
2.38925973149329	321522.715188998\\
2.38935973399335	321565.687023633\\
2.38945973649341	321609.231816063\\
2.38955973899347	321652.776608493\\
2.38965974149354	321695.748443128\\
2.3897597439936	321739.293235558\\
2.38985974649366	321782.265070192\\
2.38995974899372	321825.809862622\\
2.39005975149379	321868.781697257\\
2.39015975399385	321911.753531892\\
2.39025975649391	321955.298324322\\
2.39035975899397	321998.270158957\\
2.39045976149404	322041.814951387\\
2.3905597639941	322084.786786021\\
2.39065976649416	322127.758620656\\
2.39075976899422	322171.303413086\\
2.39085977149429	322214.275247721\\
2.39095977399435	322257.247082356\\
2.39105977649441	322300.791874786\\
2.39115977899447	322343.763709421\\
2.39125978149454	322386.735544055\\
2.3913597839946	322429.70737869\\
2.39145978649466	322473.25217112\\
2.39155978899472	322516.224005755\\
2.39165979149479	322559.19584039\\
2.39175979399485	322602.167675025\\
2.39185979649491	322645.139509659\\
2.39195979899498	322688.684302089\\
2.39205980149504	322731.656136724\\
2.3921598039951	322774.627971359\\
2.39225980649516	322817.599805994\\
2.39235980899522	322860.571640629\\
2.39245981149529	322903.543475263\\
2.39255981399535	322946.515309898\\
2.39265981649541	322989.487144533\\
2.39275981899547	323032.458979168\\
2.39285982149554	323075.430813803\\
2.3929598239956	323118.402648437\\
2.39305982649566	323161.374483072\\
2.39315982899572	323204.346317707\\
2.39325983149579	323247.318152342\\
2.39335983399585	323290.289986977\\
2.39345983649591	323333.261821612\\
2.39355983899597	323376.233656246\\
2.39365984149604	323419.205490881\\
2.3937598439961	323462.177325516\\
2.39385984649616	323504.576202356\\
2.39395984899623	323547.548036991\\
2.39405985149629	323590.519871625\\
2.39415985399635	323633.49170626\\
2.39425985649641	323676.463540895\\
2.39435985899647	323718.862417735\\
2.39445986149654	323761.834252369\\
2.3945598639966	323804.806087004\\
2.39465986649666	323847.777921639\\
2.39475986899672	323890.176798479\\
2.39485987149679	323933.148633114\\
2.39495987399685	323976.120467748\\
2.39505987649691	324018.519344588\\
2.39515987899697	324061.491179223\\
2.39525988149704	324104.463013858\\
2.3953598839971	324146.861890697\\
2.39545988649716	324189.833725332\\
2.39555988899722	324232.232602172\\
2.39565989149729	324275.204436807\\
2.39575989399735	324317.603313646\\
2.39585989649741	324360.575148281\\
2.39595989899748	324402.974025121\\
2.39605990149754	324445.945859756\\
2.3961599039976	324488.344736595\\
2.39625990649766	324531.31657123\\
2.39635990899772	324573.71544807\\
2.39645991149779	324616.687282705\\
2.39655991399785	324659.086159544\\
2.39665991649791	324701.485036384\\
2.39675991899797	324744.456871019\\
2.39685992149804	324786.855747858\\
2.3969599239981	324829.827582493\\
2.39705992649816	324872.226459333\\
2.39715992899822	324914.625336173\\
2.39725993149829	324957.024213012\\
2.39735993399835	324999.996047647\\
2.39745993649841	325042.394924487\\
2.39755993899847	325084.793801326\\
2.39765994149854	325127.192678166\\
2.3977599439986	325170.164512801\\
2.39785994649866	325212.563389641\\
2.39795994899873	325254.96226648\\
2.39805995149879	325297.36114332\\
2.39815995399885	325339.76002016\\
2.39825995649891	325382.158896999\\
2.39835995899898	325425.130731634\\
2.39845996149904	325467.529608474\\
2.3985599639991	325509.928485314\\
2.39865996649916	325552.327362153\\
2.39875996899922	325594.726238993\\
2.39885997149929	325637.125115833\\
2.39895997399935	325679.523992672\\
2.39905997649941	325721.922869512\\
2.39915997899947	325764.321746352\\
2.39925998149954	325806.720623191\\
2.3993599839996	325849.119500031\\
2.39945998649966	325891.518376871\\
2.39955998899972	325933.91725371\\
2.39965999149979	325976.31613055\\
2.39975999399985	326018.142049595\\
2.39985999649991	326060.540926434\\
2.39995999899998	326102.939803274\\
2.40006000150004	326145.338680114\\
};
\addplot [color=mycolor1,solid,forget plot]
  table[row sep=crcr]{%
2.40006000150004	326145.338680114\\
2.4001600040001	326187.737556953\\
2.40026000650016	326230.136433793\\
2.40036000900023	326271.962352838\\
2.40046001150029	326314.361229677\\
2.40056001400035	326356.760106517\\
2.40066001650041	326399.158983357\\
2.40076001900047	326440.984902401\\
2.40086002150054	326483.383779241\\
2.4009600240006	326525.78265608\\
2.40106002650066	326567.608575125\\
2.40116002900072	326610.007451965\\
2.40126003150079	326652.406328804\\
2.40136003400085	326694.232247849\\
2.40146003650091	326736.631124689\\
2.40156003900097	326778.457043733\\
2.40166004150104	326820.855920573\\
2.4017600440011	326862.681839617\\
2.40186004650116	326905.080716457\\
2.40196004900123	326946.906635502\\
2.40206005150129	326989.305512341\\
2.40216005400135	327031.131431386\\
2.40226005650141	327073.530308226\\
2.40236005900148	327115.35622727\\
2.40246006150154	327157.75510411\\
2.4025600640016	327199.581023154\\
2.40266006650166	327241.979899994\\
2.40276006900173	327283.805819039\\
2.40286007150179	327325.631738083\\
2.40296007400185	327368.030614923\\
2.40306007650191	327409.856533967\\
2.40316007900197	327451.682453012\\
2.40326008150204	327494.081329852\\
2.4033600840021	327535.907248896\\
2.40346008650216	327577.733167941\\
2.40356008900222	327619.559086985\\
2.40366009150229	327661.957963825\\
2.40376009400235	327703.783882869\\
2.40386009650241	327745.609801914\\
2.40396009900248	327787.435720959\\
2.40406010150254	327829.261640003\\
2.4041601040026	327871.087559048\\
2.40426010650266	327912.913478092\\
2.40436010900273	327955.312354932\\
2.40446011150279	327997.138273976\\
2.40456011400285	328038.964193021\\
2.40466011650291	328080.790112066\\
2.40476011900298	328122.61603111\\
2.40486012150304	328164.441950155\\
2.4049601240031	328206.267869199\\
2.40506012650316	328248.093788244\\
2.40516012900322	328289.919707288\\
2.40526013150329	328331.745626333\\
2.40536013400335	328373.571545377\\
2.40546013650341	328415.397464422\\
2.40556013900347	328456.650425671\\
2.40566014150354	328498.476344716\\
2.4057601440036	328540.30226376\\
2.40586014650366	328582.128182805\\
2.40596014900373	328623.95410185\\
2.40606015150379	328665.780020894\\
2.40616015400385	328707.032982144\\
2.40626015650391	328748.858901188\\
2.40636015900398	328790.684820233\\
2.40646016150404	328832.510739277\\
2.4065601640041	328873.763700527\\
2.40666016650416	328915.589619571\\
2.40676016900423	328957.415538616\\
2.40686017150429	328998.668499865\\
2.40696017400435	329040.49441891\\
2.40706017650441	329082.320337954\\
2.40716017900447	329123.573299204\\
2.40726018150454	329165.399218248\\
2.4073601840046	329207.225137293\\
2.40746018650466	329248.478098542\\
2.40756018900472	329290.304017587\\
2.40766019150479	329331.556978836\\
2.40776019400485	329373.382897881\\
2.40786019650491	329414.63585913\\
2.40796019900498	329456.461778175\\
2.40806020150504	329497.714739424\\
2.4081602040051	329539.540658469\\
2.40826020650516	329580.793619718\\
2.40836020900523	329622.046580967\\
2.40846021150529	329663.872500012\\
2.40856021400535	329705.125461261\\
2.40866021650541	329746.378422511\\
2.40876021900548	329788.204341555\\
2.40886022150554	329829.457302805\\
2.4089602240056	329870.710264054\\
2.40906022650566	329912.536183099\\
2.40916022900573	329953.789144348\\
2.40926023150579	329995.042105598\\
2.40936023400585	330036.295066847\\
2.40946023650591	330078.120985892\\
2.40956023900597	330119.373947141\\
2.40966024150604	330160.62690839\\
2.4097602440061	330201.87986964\\
2.40986024650616	330243.132830889\\
2.40996024900623	330284.385792139\\
2.41006025150629	330326.211711183\\
2.41016025400635	330367.464672433\\
2.41026025650641	330408.717633682\\
2.41036025900648	330449.970594932\\
2.41046026150654	330491.223556181\\
2.4105602640066	330532.47651743\\
2.41066026650666	330573.72947868\\
2.41076026900673	330614.982439929\\
2.41086027150679	330656.235401179\\
2.41096027400685	330697.488362428\\
2.41106027650691	330738.741323677\\
2.41116027900698	330779.994284927\\
2.41126028150704	330820.674288381\\
2.4113602840071	330861.927249631\\
2.41146028650716	330903.18021088\\
2.41156028900722	330944.433172129\\
2.41166029150729	330985.686133379\\
2.41176029400735	331026.939094628\\
2.41186029650741	331067.619098083\\
2.41196029900748	331108.872059332\\
2.41206030150754	331150.125020581\\
2.4121603040076	331191.377981831\\
2.41226030650766	331232.057985285\\
2.41236030900773	331273.310946535\\
2.41246031150779	331314.563907784\\
2.41256031400785	331355.243911238\\
2.41266031650791	331396.496872488\\
2.41276031900798	331437.749833737\\
2.41286032150804	331478.429837191\\
2.4129603240081	331519.682798441\\
2.41306032650816	331560.362801895\\
2.41316032900823	331601.615763144\\
2.41326033150829	331642.868724394\\
2.41336033400835	331683.548727848\\
2.41346033650841	331724.801689098\\
2.41356033900847	331765.481692552\\
2.41366034150854	331806.161696006\\
2.4137603440086	331847.414657256\\
2.41386034650866	331888.09466071\\
2.41396034900873	331929.347621959\\
2.41406035150879	331970.027625414\\
2.41416035400885	332011.280586663\\
2.41426035650891	332051.960590117\\
2.41436035900898	332092.640593572\\
2.41446036150904	332133.893554821\\
2.4145603640091	332174.573558275\\
2.41466036650916	332215.25356173\\
2.41476036900923	332255.933565184\\
2.41486037150929	332297.186526433\\
2.41496037400935	332337.866529888\\
2.41506037650941	332378.546533342\\
2.41516037900948	332419.226536796\\
2.41526038150954	332459.90654025\\
2.4153603840096	332501.1595015\\
2.41546038650966	332541.839504954\\
2.41556038900973	332582.519508408\\
2.41566039150979	332623.199511863\\
2.41576039400985	332663.879515317\\
2.41586039650991	332704.559518771\\
2.41596039900998	332745.239522226\\
2.41606040151004	332785.91952568\\
2.4161604040101	332826.599529134\\
2.41626040651016	332867.279532588\\
2.41636040901023	332907.959536043\\
2.41646041151029	332948.639539497\\
2.41656041401035	332989.319542951\\
2.41666041651041	333029.999546406\\
2.41676041901048	333070.67954986\\
2.41686042151054	333111.359553314\\
2.4169604240106	333151.466598973\\
2.41706042651066	333192.146602428\\
2.41716042901073	333232.826605882\\
2.41726043151079	333273.506609336\\
2.41736043401085	333314.186612791\\
2.41746043651091	333354.866616245\\
2.41756043901098	333394.973661904\\
2.41766044151104	333435.653665358\\
2.4177604440111	333476.333668813\\
2.41786044651116	333516.440714472\\
2.41796044901123	333557.120717926\\
2.41806045151129	333597.80072138\\
2.41816045401135	333637.907767039\\
2.41826045651141	333678.587770494\\
2.41836045901148	333719.267773948\\
2.41846046151154	333759.374819607\\
2.4185604640116	333800.054823061\\
2.41866046651166	333840.161868721\\
2.41876046901173	333880.841872175\\
2.41886047151179	333920.948917834\\
2.41896047401185	333961.628921288\\
2.41906047651191	334001.735966948\\
2.41916047901198	334042.415970402\\
2.41926048151204	334082.523016061\\
2.4193604840121	334123.203019515\\
2.41946048651216	334163.310065174\\
2.41956048901223	334203.417110834\\
2.41966049151229	334244.097114288\\
2.41976049401235	334284.204159947\\
2.41986049651241	334324.311205606\\
2.41996049901248	334364.99120906\\
2.42006050151254	334405.09825472\\
2.4201605040126	334445.205300379\\
2.42026050651266	334485.312346038\\
2.42036050901273	334525.992349492\\
2.42046051151279	334566.099395151\\
2.42056051401285	334606.20644081\\
2.42066051651291	334646.31348647\\
2.42076051901298	334686.420532129\\
2.42086052151304	334727.100535583\\
2.4209605240131	334767.207581242\\
2.42106052651316	334807.314626901\\
2.42116052901323	334847.421672561\\
2.42126053151329	334887.52871822\\
2.42136053401335	334927.635763879\\
2.42146053651341	334967.742809538\\
2.42156053901348	335007.849855197\\
2.42166054151354	335047.956900856\\
2.4217605440136	335088.063946515\\
2.42186054651366	335128.170992175\\
2.42196054901373	335168.278037834\\
2.42206055151379	335208.385083493\\
2.42216055401385	335248.492129152\\
2.42226055651391	335288.599174811\\
2.42236055901398	335328.133262675\\
2.42246056151404	335368.240308334\\
2.4225605640141	335408.347353994\\
2.42266056651416	335448.454399653\\
2.42276056901423	335488.561445312\\
2.42286057151429	335528.095533176\\
2.42296057401435	335568.202578835\\
2.42306057651441	335608.309624494\\
2.42316057901448	335648.416670153\\
2.42326058151454	335687.950758018\\
2.4233605840146	335728.057803677\\
2.42346058651466	335767.591891541\\
2.42356058901473	335807.6989372\\
2.42366059151479	335847.805982859\\
2.42376059401485	335887.340070723\\
2.42386059651491	335927.447116382\\
2.42396059901498	335966.981204246\\
2.42406060151504	336007.088249905\\
2.4241606040151	336046.622337769\\
2.42426060651516	336086.729383429\\
2.42436060901523	336126.263471293\\
2.42446061151529	336166.370516952\\
2.42456061401535	336205.904604816\\
2.42466061651541	336246.011650475\\
2.42476061901548	336285.545738339\\
2.42486062151554	336325.652783998\\
2.4249606240156	336365.186871862\\
2.42506062651566	336404.720959726\\
2.42516062901573	336444.828005385\\
2.42526063151579	336484.362093249\\
2.42536063401585	336523.896181113\\
2.42546063651591	336563.430268977\\
2.42556063901598	336603.537314637\\
2.42566064151604	336643.071402501\\
2.4257606440161	336682.605490365\\
2.42586064651616	336722.139578229\\
2.42596064901623	336761.673666093\\
2.42606065151629	336801.780711752\\
2.42616065401635	336841.314799616\\
2.42626065651641	336880.84888748\\
2.42636065901648	336920.382975344\\
2.42646066151654	336959.917063208\\
2.4265606640166	336999.451151072\\
2.42666066651666	337038.985238936\\
2.42676066901673	337078.5193268\\
2.42686067151679	337118.053414664\\
2.42696067401685	337157.587502528\\
2.42706067651691	337197.121590392\\
2.42716067901698	337236.655678256\\
2.42726068151704	337276.18976612\\
2.4273606840171	337315.723853984\\
2.42746068651716	337355.257941848\\
2.42756068901723	337394.219071917\\
2.42766069151729	337433.753159781\\
2.42776069401735	337473.287247645\\
2.42786069651741	337512.821335509\\
2.42796069901748	337552.355423373\\
2.42806070151754	337591.316553442\\
2.4281607040176	337630.850641306\\
2.42826070651766	337670.38472917\\
2.42836070901773	337709.918817034\\
2.42846071151779	337748.879947103\\
2.42856071401785	337788.414034967\\
2.42866071651791	337827.948122831\\
2.42876071901798	337866.9092529\\
2.42886072151804	337906.443340764\\
2.4289607240181	337945.404470833\\
2.42906072651816	337984.938558697\\
2.42916072901823	338023.899688766\\
2.42926073151829	338063.43377663\\
2.42936073401835	338102.967864494\\
2.42946073651841	338141.928994563\\
2.42956073901848	338180.890124632\\
2.42966074151854	338220.424212496\\
2.4297607440186	338259.385342565\\
2.42986074651866	338298.919430429\\
2.42996074901873	338337.880560498\\
2.43006075151879	338376.841690566\\
2.43016075401885	338416.37577843\\
2.43026075651891	338455.336908499\\
2.43036075901898	338494.298038568\\
2.43046076151904	338533.832126432\\
2.4305607640191	338572.793256501\\
2.43066076651916	338611.75438657\\
2.43076076901923	338651.288474434\\
2.43086077151929	338690.249604503\\
2.43096077401935	338729.210734572\\
2.43106077651941	338768.171864641\\
2.43116077901948	338807.13299471\\
2.43126078151954	338846.094124779\\
2.4313607840196	338885.055254848\\
2.43146078651966	338924.589342711\\
2.43156078901973	338963.55047278\\
2.43166079151979	339002.511602849\\
2.43176079401985	339041.472732918\\
2.43186079651991	339080.433862987\\
2.43196079901998	339119.394993056\\
2.43206080152004	339158.356123125\\
2.4321608040201	339197.317253194\\
2.43226080652016	339236.278383263\\
2.43236080902023	339275.239513332\\
2.43246081152029	339313.627685605\\
2.43256081402035	339352.588815674\\
2.43266081652041	339391.549945743\\
2.43276081902048	339430.511075812\\
2.43286082152054	339469.472205881\\
2.4329608240206	339508.43333595\\
2.43306082652066	339546.821508224\\
2.43316082902073	339585.782638292\\
2.43326083152079	339624.743768361\\
2.43336083402085	339663.70489843\\
2.43346083652091	339702.093070704\\
2.43356083902098	339741.054200773\\
2.43366084152104	339780.015330842\\
2.4337608440211	339818.403503116\\
2.43386084652116	339857.364633184\\
2.43396084902123	339895.752805458\\
2.43406085152129	339934.713935527\\
2.43416085402135	339973.675065596\\
2.43426085652141	340012.06323787\\
2.43436085902148	340051.024367939\\
2.43446086152154	340089.412540213\\
2.4345608640216	340128.373670281\\
2.43466086652166	340166.761842555\\
2.43476086902173	340205.150014829\\
2.43486087152179	340244.111144898\\
2.43496087402185	340282.499317172\\
2.43506087652191	340321.46044724\\
2.43516087902198	340359.848619514\\
2.43526088152204	340398.236791788\\
2.4353608840221	340437.197921857\\
2.43546088652216	340475.586094131\\
2.43556088902223	340513.974266404\\
2.43566089152229	340552.362438678\\
2.43576089402235	340591.323568747\\
2.43586089652241	340629.711741021\\
2.43596089902248	340668.099913295\\
2.43606090152254	340706.488085568\\
2.4361609040226	340744.876257842\\
2.43626090652266	340783.837387911\\
2.43636090902273	340822.225560185\\
2.43646091152279	340860.613732459\\
2.43656091402285	340899.001904732\\
2.43666091652291	340937.390077006\\
2.43676091902298	340975.77824928\\
2.43686092152304	341014.166421554\\
2.4369609240231	341052.554593827\\
2.43706092652316	341090.942766101\\
2.43716092902323	341129.330938375\\
2.43726093152329	341167.719110649\\
2.43736093402335	341206.107282922\\
2.43746093652341	341244.495455196\\
2.43756093902348	341282.310669675\\
2.43766094152354	341320.698841949\\
2.4377609440236	341359.087014222\\
2.43786094652366	341397.475186496\\
2.43796094902373	341435.86335877\\
2.43806095152379	341473.678573249\\
2.43816095402385	341512.066745522\\
2.43826095652391	341550.454917796\\
2.43836095902398	341588.84309007\\
2.43846096152404	341626.658304548\\
2.4385609640241	341665.046476822\\
2.43866096652416	341703.434649096\\
2.43876096902423	341741.249863575\\
2.43886097152429	341779.638035848\\
2.43896097402435	341817.453250327\\
2.43906097652441	341855.841422601\\
2.43916097902448	341894.229594875\\
2.43926098152454	341932.044809353\\
2.4393609840246	341970.432981627\\
2.43946098652466	342008.248196106\\
2.43956098902473	342046.063410584\\
2.43966099152479	342084.451582858\\
2.43976099402485	342122.266797337\\
2.43986099652491	342160.65496961\\
2.43996099902498	342198.470184089\\
2.44006100152504	342236.285398568\\
2.4401610040251	342274.673570841\\
2.44026100652516	342312.48878532\\
2.44036100902523	342350.303999799\\
2.44046101152529	342388.692172072\\
2.44056101402535	342426.507386551\\
2.44066101652541	342464.32260103\\
2.44076101902548	342502.137815508\\
2.44086102152554	342540.525987782\\
2.4409610240256	342578.341202261\\
2.44106102652566	342616.156416739\\
2.44116102902573	342653.971631218\\
2.44126103152579	342691.786845697\\
2.44136103402585	342729.602060175\\
2.44146103652591	342767.417274654\\
2.44156103902598	342805.232489133\\
2.44166104152604	342843.047703611\\
2.4417610440261	342880.86291809\\
2.44186104652616	342918.678132568\\
2.44196104902623	342956.493347047\\
2.44206105152629	342994.308561526\\
2.44216105402635	343032.123776004\\
2.44226105652641	343069.938990483\\
2.44236105902648	343107.754204962\\
2.44246106152654	343145.56941944\\
2.4425610640266	343183.384633919\\
2.44266106652666	343220.626890602\\
2.44276106902673	343258.442105081\\
2.44286107152679	343296.25731956\\
2.44296107402685	343334.072534038\\
2.44306107652691	343371.887748517\\
2.44316107902698	343409.130005201\\
2.44326108152704	343446.945219679\\
2.4433610840271	343484.760434158\\
2.44346108652716	343522.002690841\\
2.44356108902723	343559.81790532\\
2.44366109152729	343597.060162003\\
2.44376109402735	343634.875376482\\
2.44386109652741	343672.690590961\\
2.44396109902748	343709.932847644\\
2.44406110152754	343747.748062123\\
2.4441611040276	343784.990318806\\
2.44426110652766	343822.805533285\\
2.44436110902773	343860.047789968\\
2.44446111152779	343897.863004447\\
2.44456111402785	343935.105261131\\
2.44466111652791	343972.347517814\\
2.44476111902798	344010.162732293\\
2.44486112152804	344047.404988976\\
2.4449611240281	344084.64724566\\
2.44506112652816	344122.462460138\\
2.44516112902823	344159.704716822\\
2.44526113152829	344196.946973505\\
2.44536113402835	344234.762187984\\
2.44546113652841	344272.004444668\\
2.44556113902848	344309.246701351\\
2.44566114152854	344346.488958034\\
2.4457611440286	344383.731214718\\
2.44586114652866	344421.546429197\\
2.44596114902873	344458.78868588\\
2.44606115152879	344496.030942564\\
2.44616115402885	344533.273199247\\
2.44626115652891	344570.515455931\\
2.44636115902898	344607.757712614\\
2.44646116152904	344644.999969298\\
2.4465611640291	344682.242225981\\
2.44666116652916	344719.484482665\\
2.44676116902923	344756.726739348\\
2.44686117152929	344793.968996032\\
2.44696117402935	344831.211252715\\
2.44706117652941	344868.453509399\\
2.44716117902948	344905.122808287\\
2.44726118152954	344942.365064971\\
2.4473611840296	344979.607321654\\
2.44746118652966	345016.849578338\\
2.44756118902973	345054.091835021\\
2.44766119152979	345091.334091705\\
2.44776119402985	345128.003390593\\
2.44786119652991	345165.245647276\\
2.44796119902998	345202.48790396\\
2.44806120153004	345239.157202848\\
2.4481612040301	345276.399459532\\
2.44826120653016	345313.641716215\\
2.44836120903023	345350.311015104\\
2.44846121153029	345387.553271787\\
2.44856121403035	345424.222570676\\
2.44866121653041	345461.464827359\\
2.44876121903048	345498.134126247\\
2.44886122153054	345535.376382931\\
2.4489612240306	345572.045681819\\
2.44906122653066	345609.287938503\\
2.44916122903073	345645.957237391\\
2.44926123153079	345683.199494075\\
2.44936123403085	345719.868792963\\
2.44946123653091	345757.111049647\\
2.44956123903098	345793.780348535\\
2.44966124153104	345830.449647423\\
2.4497612440311	345867.691904107\\
2.44986124653116	345904.361202995\\
2.44996124903123	345941.030501884\\
2.45006125153129	345977.699800772\\
2.45016125403135	346014.942057455\\
2.45026125653141	346051.611356344\\
2.45036125903148	346088.280655232\\
2.45046126153154	346124.949954121\\
2.4505612640316	346161.619253009\\
2.45066126653166	346198.288551897\\
2.45076126903173	346235.530808581\\
2.45086127153179	346272.200107469\\
2.45096127403185	346308.869406358\\
2.45106127653191	346345.538705246\\
2.45116127903198	346382.208004134\\
2.45126128153204	346418.877303023\\
2.4513612840321	346455.546601911\\
2.45146128653216	346492.215900799\\
2.45156128903223	346528.885199688\\
2.45166129153229	346564.981540781\\
2.45176129403235	346601.650839669\\
2.45186129653241	346638.320138558\\
2.45196129903248	346674.989437446\\
2.45206130153254	346711.658736335\\
2.4521613040326	346748.328035223\\
2.45226130653266	346784.424376316\\
2.45236130903273	346821.093675205\\
2.45246131153279	346857.762974093\\
2.45256131403285	346894.432272981\\
2.45266131653291	346930.528614074\\
2.45276131903298	346967.197912963\\
2.45286132153304	347003.867211851\\
2.4529613240331	347039.963552944\\
2.45306132653316	347076.632851833\\
2.45316132903323	347112.729192926\\
2.45326133153329	347149.398491814\\
2.45336133403335	347185.494832908\\
2.45346133653341	347222.164131796\\
2.45356133903348	347258.260472889\\
2.45366134153354	347294.929771778\\
2.4537613440336	347331.026112871\\
2.45386134653366	347367.695411759\\
2.45396134903373	347403.791752853\\
2.45406135153379	347440.461051741\\
2.45416135403385	347476.557392834\\
2.45426135653391	347512.653733927\\
2.45436135903398	347549.323032816\\
2.45446136153404	347585.419373909\\
2.4545613640341	347621.515715002\\
2.45466136653416	347657.612056096\\
2.45476136903423	347694.281354984\\
2.45486137153429	347730.377696077\\
2.45496137403435	347766.47403717\\
2.45506137653441	347802.570378264\\
2.45516137903448	347838.666719357\\
2.45526138153454	347874.76306045\\
2.4553613840346	347911.432359339\\
2.45546138653466	347947.528700432\\
2.45556138903473	347983.625041525\\
2.45566139153479	348019.721382618\\
2.45576139403485	348055.817723711\\
2.45586139653491	348091.914064805\\
2.45596139903498	348128.010405898\\
2.45606140153504	348164.106746991\\
2.4561614040351	348200.203088084\\
2.45626140653516	348235.726471383\\
2.45636140903523	348271.822812476\\
2.45646141153529	348307.919153569\\
2.45656141403535	348344.015494662\\
2.45666141653541	348380.111835755\\
2.45676141903548	348416.208176849\\
2.45686142153554	348451.731560147\\
2.4569614240356	348487.82790124\\
2.45706142653566	348523.924242333\\
2.45716142903573	348559.447625631\\
2.45726143153579	348595.543966725\\
2.45736143403585	348631.640307818\\
2.45746143653591	348667.163691116\\
2.45756143903598	348703.260032209\\
2.45766144153604	348739.356373303\\
2.4577614440361	348774.879756601\\
2.45786144653616	348810.976097694\\
2.45796144903623	348846.499480992\\
2.45806145153629	348882.595822085\\
2.45816145403635	348918.119205383\\
2.45826145653641	348954.215546477\\
2.45836145903648	348989.738929775\\
2.45846146153654	349025.835270868\\
2.4585614640366	349061.358654166\\
2.45866146653666	349096.882037464\\
2.45876146903673	349132.978378557\\
2.45886147153679	349168.501761855\\
2.45896147403685	349204.025145154\\
2.45906147653691	349239.548528452\\
2.45916147903698	349275.644869545\\
2.45926148153704	349311.168252843\\
2.4593614840371	349346.691636141\\
2.45946148653716	349382.215019439\\
2.45956148903723	349418.311360533\\
2.45966149153729	349453.834743831\\
2.45976149403735	349489.358127129\\
2.45986149653741	349524.881510427\\
2.45996149903748	349560.404893725\\
2.46006150153754	349595.928277023\\
2.4601615040376	349631.451660321\\
2.46026150653766	349666.975043619\\
2.46036150903773	349702.498426917\\
2.46046151153779	349738.021810215\\
2.46056151403785	349773.545193514\\
2.46066151653791	349809.068576812\\
2.46076151903798	349844.59196011\\
2.46086152153804	349880.115343408\\
2.4609615240381	349915.065768911\\
2.46106152653816	349950.589152209\\
2.46116152903823	349986.112535507\\
2.46126153153829	350021.635918805\\
2.46136153403835	350057.159302103\\
2.46146153653841	350092.109727606\\
2.46156153903848	350127.633110904\\
2.46166154153854	350163.156494203\\
2.4617615440386	350198.106919706\\
2.46186154653866	350233.630303004\\
2.46196154903873	350269.153686302\\
2.46206155153879	350304.104111805\\
2.46216155403885	350339.627495103\\
2.46226155653891	350374.577920606\\
2.46236155903898	350410.101303904\\
2.46246156153904	350445.051729407\\
2.4625615640391	350480.575112705\\
2.46266156653916	350515.525538208\\
2.46276156903923	350551.048921506\\
2.46286157153929	350585.999347009\\
2.46296157403935	350621.522730307\\
2.46306157653941	350656.47315581\\
2.46316157903948	350691.423581313\\
2.46326158153954	350726.946964611\\
2.4633615840396	350761.897390114\\
2.46346158653966	350796.847815617\\
2.46356158903973	350832.371198915\\
2.46366159153979	350867.321624418\\
2.46376159403985	350902.272049921\\
2.46386159653991	350937.222475424\\
2.46396159903998	350972.172900927\\
2.46406160154004	351007.12332643\\
2.4641616040401	351042.646709728\\
2.46426160654016	351077.597135231\\
2.46436160904023	351112.547560734\\
2.46446161154029	351147.497986237\\
2.46456161404035	351182.44841174\\
2.46466161654041	351217.398837243\\
2.46476161904048	351252.349262746\\
2.46486162154054	351287.299688249\\
2.4649616240406	351322.250113752\\
2.46506162654066	351357.200539255\\
2.46516162904073	351392.150964758\\
2.46526163154079	351426.528432466\\
2.46536163404085	351461.478857969\\
2.46546163654091	351496.429283472\\
2.46556163904098	351531.379708975\\
2.46566164154104	351566.330134478\\
2.4657616440411	351600.707602186\\
2.46586164654116	351635.658027689\\
2.46596164904123	351670.608453192\\
2.46606165154129	351705.558878695\\
2.46616165404135	351739.936346403\\
2.46626165654141	351774.886771906\\
2.46636165904148	351809.264239613\\
2.46646166154154	351844.214665116\\
2.4665616640416	351879.165090619\\
2.46666166654166	351913.542558327\\
2.46676166904173	351948.49298383\\
2.46686167154179	351982.870451538\\
2.46696167404185	352017.820877041\\
2.46706167654191	352052.198344749\\
2.46716167904198	352087.148770252\\
2.46726168154204	352121.52623796\\
2.4673616840421	352155.903705668\\
2.46746168654216	352190.854131171\\
2.46756168904223	352225.231598878\\
2.46766169154229	352259.609066586\\
2.46776169404235	352294.559492089\\
2.46786169654241	352328.936959797\\
2.46796169904248	352363.314427505\\
2.46806170154254	352397.691895213\\
2.4681617040426	352432.642320716\\
2.46826170654266	352467.019788424\\
2.46836170904273	352501.397256132\\
2.46846171154279	352535.774723839\\
2.46856171404285	352570.152191547\\
2.46866171654291	352604.529659255\\
2.46876171904298	352638.907126963\\
2.46886172154304	352673.284594671\\
2.4689617240431	352707.662062379\\
2.46906172654316	352742.039530086\\
2.46916172904323	352776.416997794\\
2.46926173154329	352810.794465502\\
2.46936173404335	352845.17193321\\
2.46946173654341	352879.549400918\\
2.46956173904348	352913.926868626\\
2.46966174154354	352948.304336334\\
2.4697617440436	352982.681804041\\
2.46986174654366	353016.486313954\\
2.46996174904373	353050.863781662\\
2.47006175154379	353085.24124937\\
2.47016175404385	353119.618717078\\
2.47026175654391	353153.42322699\\
2.47036175904398	353187.800694698\\
2.47046176154404	353222.178162406\\
2.4705617640441	353255.982672319\\
2.47066176654416	353290.360140027\\
2.47076176904423	353324.737607735\\
2.47086177154429	353358.542117647\\
2.47096177404435	353392.919585355\\
2.47106177654441	353426.724095268\\
2.47116177904448	353461.101562976\\
2.47126178154454	353494.906072888\\
2.4713617840446	353529.283540596\\
2.47146178654466	353563.088050509\\
2.47156178904473	353597.465518217\\
2.47166179154479	353631.270028129\\
2.47176179404485	353665.074538042\\
2.47186179654491	353699.45200575\\
2.47196179904498	353733.256515663\\
2.47206180154504	353767.061025576\\
2.4721618040451	353801.438493283\\
2.47226180654516	353835.243003196\\
2.47236180904523	353869.047513109\\
2.47246181154529	353902.852023022\\
2.47256181404535	353936.656532934\\
2.47266181654541	353971.034000642\\
2.47276181904548	354004.838510555\\
2.47286182154554	354038.643020467\\
2.4729618240456	354072.44753038\\
2.47306182654566	354106.252040293\\
2.47316182904573	354140.056550206\\
2.47326183154579	354173.861060118\\
2.47336183404585	354207.665570031\\
2.47346183654591	354241.470079944\\
2.47356183904598	354275.274589857\\
2.47366184154604	354309.079099769\\
2.4737618440461	354342.883609682\\
2.47386184654616	354376.688119595\\
2.47396184904623	354409.919671712\\
2.47406185154629	354443.724181625\\
2.47416185404635	354477.528691538\\
2.47426185654641	354511.33320145\\
2.47436185904648	354544.564753568\\
2.47446186154654	354578.369263481\\
2.4745618640466	354612.173773393\\
2.47466186654666	354645.978283306\\
2.47476186904673	354679.209835424\\
2.47486187154679	354713.014345336\\
2.47496187404685	354746.245897454\\
2.47506187654691	354780.050407367\\
2.47516187904698	354813.85491728\\
2.47526188154704	354847.086469397\\
2.4753618840471	354880.89097931\\
2.47546188654716	354914.122531427\\
2.47556188904723	354947.92704134\\
2.47566189154729	354981.158593458\\
2.47576189404735	355014.390145575\\
2.47586189654741	355048.194655488\\
2.47596189904748	355081.426207606\\
2.47606190154754	355115.230717518\\
2.4761619040476	355148.462269636\\
2.47626190654766	355181.693821753\\
2.47636190904773	355214.925373871\\
2.47646191154779	355248.729883784\\
2.47656191404785	355281.961435901\\
2.47666191654791	355315.192988019\\
2.47676191904798	355348.424540137\\
2.47686192154804	355381.656092254\\
2.4769619240481	355415.460602167\\
2.47706192654816	355448.692154284\\
2.47716192904823	355481.923706402\\
2.47726193154829	355515.15525852\\
2.47736193404835	355548.386810637\\
2.47746193654841	355581.618362755\\
2.47756193904848	355614.849914872\\
2.47766194154854	355648.08146699\\
2.4777619440486	355681.313019108\\
2.47786194654866	355714.544571225\\
2.47796194904873	355747.776123343\\
2.47806195154879	355780.434717665\\
2.47816195404885	355813.666269783\\
2.47826195654891	355846.8978219\\
2.47836195904898	355880.129374018\\
2.47846196154904	355913.360926136\\
2.4785619640491	355946.019520458\\
2.47866196654916	355979.251072576\\
2.47876196904923	356012.482624693\\
2.47886197154929	356045.141219016\\
2.47896197404935	356078.372771133\\
2.47906197654941	356111.604323251\\
2.47916197904948	356144.262917573\\
2.47926198154954	356177.494469691\\
2.4793619840496	356210.153064013\\
2.47946198654966	356243.384616131\\
2.47956198904973	356276.043210453\\
2.47966199154979	356309.274762571\\
2.47976199404985	356341.933356893\\
2.47986199654991	356375.164909011\\
2.47996199904998	356407.823503333\\
2.48006200155004	356441.055055451\\
2.4801620040501	356473.713649774\\
2.48026200655016	356506.372244096\\
2.48036200905023	356539.603796214\\
2.48046201155029	356572.262390536\\
2.48056201405035	356604.920984859\\
2.48066201655041	356637.579579181\\
2.48076201905048	356670.811131298\\
2.48086202155054	356703.469725621\\
2.4809620240506	356736.128319943\\
2.48106202655066	356768.786914266\\
2.48116202905073	356801.445508588\\
2.48126203155079	356834.104102911\\
2.48136203405085	356866.762697233\\
2.48146203655091	356899.421291556\\
2.48156203905098	356932.079885878\\
2.48166204155104	356964.738480201\\
2.4817620440511	356997.397074523\\
2.48186204655116	357030.055668846\\
2.48196204905123	357062.714263168\\
2.48206205155129	357095.37285749\\
2.48216205405135	357128.031451813\\
2.48226205655141	357160.690046135\\
2.48236205905148	357193.348640458\\
2.48246206155154	357226.00723478\\
2.4825620640516	357258.092871308\\
2.48266206655166	357290.75146563\\
2.48276206905173	357323.410059953\\
2.48286207155179	357355.49569648\\
2.48296207405185	357388.154290802\\
2.48306207655191	357420.812885125\\
2.48316207905198	357452.898521652\\
2.48326208155204	357485.557115975\\
2.4833620840521	357518.215710297\\
2.48346208655216	357550.301346824\\
2.48356208905223	357582.959941147\\
2.48366209155229	357615.045577674\\
2.48376209405235	357647.704171997\\
2.48386209655241	357679.789808524\\
2.48396209905248	357712.448402846\\
2.48406210155254	357744.534039374\\
2.4841621040526	357776.619675901\\
2.48426210655266	357809.278270223\\
2.48436210905273	357841.363906751\\
2.48446211155279	357873.449543278\\
2.48456211405285	357906.108137601\\
2.48466211655291	357938.193774128\\
2.48476211905298	357970.279410655\\
2.48486212155304	358002.365047182\\
2.4849621240531	358034.45068371\\
2.48506212655316	358067.109278032\\
2.48516212905323	358099.19491456\\
2.48526213155329	358131.280551087\\
2.48536213405335	358163.366187614\\
2.48546213655341	358195.451824142\\
2.48556213905348	358227.537460669\\
2.48566214155354	358259.623097196\\
2.4857621440536	358291.708733724\\
2.48586214655366	358323.794370251\\
2.48596214905373	358355.880006778\\
2.48606215155379	358387.965643306\\
2.48616215405385	358420.051279833\\
2.48626215655391	358452.13691636\\
2.48636215905398	358483.649595092\\
2.48646216155404	358515.73523162\\
2.4865621640541	358547.820868147\\
2.48666216655416	358579.906504674\\
2.48676216905423	358611.992141202\\
2.48686217155429	358643.504819934\\
2.48696217405435	358675.590456461\\
2.48706217655441	358707.676092989\\
2.48716217905448	358739.188771721\\
2.48726218155454	358771.274408248\\
2.4873621840546	358803.360044775\\
2.48746218655466	358834.872723508\\
2.48756218905473	358866.958360035\\
2.48766219155479	358898.471038767\\
2.48776219405485	358930.556675295\\
2.48786219655491	358962.069354027\\
2.48796219905498	358994.154990554\\
2.48806220155504	359025.667669286\\
2.4881622040551	359057.180348018\\
2.48826220655516	359089.265984546\\
2.48836220905523	359120.778663278\\
2.48846221155529	359152.29134201\\
2.48856221405535	359184.376978537\\
2.48866221655541	359215.88965727\\
2.48876221905548	359247.402336002\\
2.48886222155554	359278.915014734\\
2.4889622240556	359311.000651261\\
2.48906222655566	359342.513329994\\
2.48916222905573	359374.026008726\\
2.48926223155579	359405.538687458\\
2.48936223405585	359437.05136619\\
2.48946223655591	359468.564044922\\
2.48956223905598	359500.076723655\\
2.48966224155604	359531.589402387\\
2.4897622440561	359563.102081119\\
2.48986224655616	359594.614759851\\
2.48996224905623	359626.127438583\\
2.49006225155629	359657.640117316\\
2.49016225405635	359689.152796048\\
2.49026225655641	359720.66547478\\
2.49036225905648	359752.178153512\\
2.49046226155654	359783.117874449\\
2.4905622640566	359814.630553181\\
2.49066226655666	359846.143231914\\
2.49076226905673	359877.655910646\\
2.49086227155679	359909.168589378\\
2.49096227405685	359940.108310315\\
2.49106227655691	359971.620989047\\
2.49116227905698	360003.133667779\\
2.49126228155704	360034.073388716\\
2.4913622840571	360065.586067449\\
2.49146228655716	360096.525788386\\
2.49156228905723	360128.038467118\\
2.49166229155729	360158.978188055\\
2.49176229405735	360190.490866787\\
2.49186229655741	360221.430587724\\
2.49196229905748	360252.943266456\\
2.49206230155754	360283.882987393\\
2.4921623040576	360315.395666126\\
2.49226230655766	360346.335387063\\
2.49236230905773	360377.275108\\
2.49246231155779	360408.787786732\\
2.49256231405785	360439.727507669\\
2.49266231655791	360470.667228606\\
2.49276231905798	360501.606949543\\
2.49286232155804	360533.119628275\\
2.4929623240581	360564.059349212\\
2.49306232655816	360594.999070149\\
2.49316232905823	360625.938791087\\
2.49326233155829	360656.878512024\\
2.49336233405835	360687.818232961\\
2.49346233655841	360718.757953898\\
2.49356233905848	360749.697674835\\
2.49366234155854	360780.637395772\\
2.4937623440586	360811.577116709\\
2.49386234655866	360842.516837646\\
2.49396234905873	360873.456558583\\
2.49406235155879	360904.39627952\\
2.49416235405885	360935.336000457\\
2.49426235655891	360966.275721394\\
2.49436235905898	360997.215442331\\
2.49446236155904	361028.155163268\\
2.4945623640591	361058.52192641\\
2.49466236655916	361089.461647347\\
2.49476236905923	361120.401368284\\
2.49486237155929	361150.768131426\\
2.49496237405935	361181.707852363\\
2.49506237655941	361212.647573301\\
2.49516237905948	361243.014336442\\
2.49526238155954	361273.95405738\\
2.4953623840596	361304.893778317\\
2.49546238655966	361335.260541458\\
2.49556238905973	361366.200262396\\
2.49566239155979	361396.567025538\\
2.49576239405985	361427.506746475\\
2.49586239655991	361457.873509617\\
2.49596239905998	361488.813230554\\
2.49606240156004	361519.179993696\\
2.4961624040601	361549.546756837\\
2.49626240656016	361580.486477775\\
2.49636240906023	361610.853240916\\
2.49646241156029	361641.220004058\\
2.49656241406035	361672.159724995\\
2.49666241656041	361702.526488137\\
2.49676241906048	361732.893251279\\
2.49686242156054	361763.260014421\\
2.4969624240606	361793.626777563\\
2.49706242656066	361824.5664985\\
2.49716242906073	361854.933261642\\
2.49726243156079	361885.300024784\\
2.49736243406085	361915.666787926\\
2.49746243656091	361946.033551068\\
2.49756243906098	361976.40031421\\
2.49766244156104	362006.767077352\\
2.4977624440611	362037.133840494\\
2.49786244656116	362067.500603636\\
2.49796244906123	362097.867366778\\
2.49806245156129	362128.23412992\\
2.49816245406135	362158.027935266\\
2.49826245656141	362188.394698408\\
2.49836245906148	362218.76146155\\
2.49846246156154	362249.128224692\\
2.4985624640616	362279.494987834\\
2.49866246656166	362309.288793181\\
2.49876246906173	362339.655556323\\
2.49886247156179	362370.022319465\\
2.49896247406185	362399.816124812\\
2.49906247656191	362430.182887954\\
2.49916247906198	362460.549651095\\
2.49926248156204	362490.343456442\\
2.4993624840621	362520.710219584\\
2.49946248656216	362550.504024931\\
2.49956248906223	362580.870788073\\
2.49966249156229	362610.66459342\\
2.49976249406235	362641.031356562\\
2.49986249656241	362670.825161908\\
2.49996249906248	362701.19192505\\
2.50006250156254	362730.985730397\\
2.5001625040626	362760.779535744\\
2.50026250656266	362791.146298886\\
2.50036250906273	362820.940104233\\
2.50046251156279	362850.73390958\\
2.50056251406285	362880.527714926\\
2.50066251656291	362910.894478068\\
2.50076251906298	362940.688283415\\
2.50086252156304	362970.482088762\\
2.5009625240631	363000.275894109\\
2.50106252656316	363030.069699455\\
2.50116252906323	363059.863504802\\
2.50126253156329	363089.657310149\\
2.50136253406335	363119.451115496\\
2.50146253656341	363149.244920843\\
2.50156253906348	363179.03872619\\
2.50166254156354	363208.832531536\\
2.5017625440636	363238.626336883\\
2.50186254656366	363268.42014223\\
2.50196254906373	363298.213947577\\
2.50206255156379	363328.007752924\\
2.50216255406385	363357.80155827\\
2.50226255656391	363387.022405822\\
2.50236255906398	363416.816211169\\
2.50246256156404	363446.610016516\\
2.5025625640641	363476.403821862\\
2.50266256656416	363505.624669414\\
2.50276256906423	363535.418474761\\
2.50286257156429	363565.212280108\\
2.50296257406435	363594.433127659\\
2.50306257656441	363624.226933006\\
2.50316257906448	363654.020738353\\
2.50326258156454	363683.241585905\\
2.5033625840646	363713.035391251\\
2.50346258656466	363742.256238803\\
2.50356258906473	363772.05004415\\
2.50366259156479	363801.270891702\\
2.50376259406485	363830.491739253\\
2.50386259656491	363860.2855446\\
2.50396259906498	363889.506392152\\
2.50406260156504	363918.727239703\\
2.5041626040651	363948.52104505\\
2.50426260656516	363977.741892602\\
2.50436260906523	364006.962740154\\
2.50446261156529	364036.183587705\\
2.50456261406535	364065.977393052\\
2.50466261656541	364095.198240604\\
2.50476261906548	364124.419088155\\
2.50486262156554	364153.639935707\\
2.5049626240656	364182.860783259\\
2.50506262656566	364212.08163081\\
2.50516262906573	364241.302478362\\
2.50526263156579	364270.523325914\\
2.50536263406585	364299.744173465\\
2.50546263656591	364328.965021017\\
2.50556263906598	364358.185868569\\
2.50566264156604	364387.40671612\\
2.5057626440661	364416.627563672\\
2.50586264656616	364445.848411224\\
2.50596264906623	364475.069258775\\
2.50606265156629	364504.290106327\\
2.50616265406635	364532.937996084\\
2.50626265656641	364562.158843635\\
2.50636265906648	364591.379691187\\
2.50646266156654	364620.027580943\\
2.5065626640666	364649.248428495\\
2.50666266656666	364678.469276047\\
2.50676266906673	364707.117165803\\
2.50686267156679	364736.338013355\\
2.50696267406685	364765.558860907\\
2.50706267656691	364794.206750663\\
2.50716267906698	364823.427598215\\
2.50726268156704	364852.075487972\\
2.5073626840671	364881.296335523\\
2.50746268656716	364909.94422528\\
2.50756268906723	364938.592115036\\
2.50766269156729	364967.812962588\\
2.50776269406735	364996.460852344\\
2.50786269656741	365025.108742101\\
2.50796269906748	365054.329589653\\
2.50806270156754	365082.977479409\\
2.5081627040676	365111.625369166\\
2.50826270656766	365140.846216718\\
2.50836270906773	365169.494106474\\
2.50846271156779	365198.141996231\\
2.50856271406785	365226.789885987\\
2.50866271656791	365255.437775744\\
2.50876271906798	365284.0856655\\
2.50886272156804	365312.733555257\\
2.5089627240681	365341.381445013\\
2.50906272656816	365370.02933477\\
2.50916272906823	365398.677224526\\
2.50926273156829	365427.325114283\\
2.50936273406835	365455.973004039\\
2.50946273656841	365484.620893796\\
2.50956273906848	365513.268783553\\
2.50966274156854	365541.916673309\\
2.5097627440686	365570.564563066\\
2.50986274656866	365598.639495027\\
2.50996274906873	365627.287384784\\
2.51006275156879	365655.93527454\\
2.51016275406885	365684.583164297\\
2.51026275656891	365712.658096258\\
2.51036275906898	365741.305986015\\
2.51046276156904	365769.953875771\\
2.5105627640691	365798.028807733\\
2.51066276656916	365826.676697489\\
2.51076276906923	365854.75162945\\
2.51086277156929	365883.399519207\\
2.51096277406935	365911.474451168\\
2.51106277656941	365940.122340925\\
2.51116277906948	365968.197272886\\
2.51126278156954	365996.845162643\\
2.5113627840696	366024.920094604\\
2.51146278656966	366053.567984361\\
2.51156278906973	366081.642916322\\
2.51166279156979	366109.717848284\\
2.51176279406985	366138.36573804\\
2.51186279656991	366166.440670002\\
2.51196279906998	366194.515601963\\
2.51206280157004	366222.590533924\\
2.5121628040701	366250.665465886\\
2.51226280657016	366279.313355642\\
2.51236280907023	366307.388287604\\
2.51246281157029	366335.463219565\\
2.51256281407035	366363.538151527\\
2.51266281657041	366391.613083488\\
2.51276281907048	366419.688015449\\
2.51286282157054	366447.762947411\\
2.5129628240706	366475.837879372\\
2.51306282657066	366503.912811334\\
2.51316282907073	366531.987743295\\
2.51326283157079	366560.062675257\\
2.51336283407085	366588.137607218\\
2.51346283657091	366615.639581384\\
2.51356283907098	366643.714513346\\
2.51366284157104	366671.789445307\\
2.5137628440711	366699.864377268\\
2.51386284657116	366727.366351435\\
2.51396284907123	366755.441283396\\
2.51406285157129	366783.516215358\\
2.51416285407135	366811.018189524\\
2.51426285657141	366839.093121485\\
2.51436285907148	366867.168053447\\
2.51446286157154	366894.670027613\\
2.5145628640716	366922.744959574\\
2.51466286657166	366950.246933741\\
2.51476286907173	366978.321865702\\
2.51486287157179	367005.823839868\\
2.51496287407185	367033.89877183\\
2.51506287657191	367061.400745996\\
2.51516287907198	367088.902720162\\
2.51526288157204	367116.977652124\\
2.5153628840721	367144.47962629\\
2.51546288657216	367171.981600456\\
2.51556288907223	367200.056532418\\
2.51566289157229	367227.558506584\\
2.51576289407235	367255.06048075\\
2.51586289657241	367282.562454916\\
2.51596289907248	367310.064429083\\
2.51606290157254	367338.139361044\\
2.5161629040726	367365.64133521\\
2.51626290657266	367393.143309377\\
2.51636290907273	367420.645283543\\
2.51646291157279	367448.147257709\\
2.51656291407285	367475.649231876\\
2.51666291657291	367503.151206042\\
2.51676291907298	367530.653180208\\
2.51686292157304	367558.155154374\\
2.5169629240731	367585.084170746\\
2.51706292657316	367612.586144912\\
2.51716292907323	367640.088119078\\
2.51726293157329	367667.590093244\\
2.51736293407335	367695.092067411\\
2.51746293657341	367722.594041577\\
2.51756293907348	367749.523057948\\
2.51766294157354	367777.025032114\\
2.5177629440736	367804.527006281\\
2.51786294657366	367831.456022652\\
2.51796294907373	367858.957996818\\
2.51806295157379	367885.887013189\\
2.51816295407385	367913.388987356\\
2.51826295657391	367940.890961522\\
2.51836295907398	367967.819977893\\
2.51846296157404	367995.321952059\\
2.5185629640741	368022.25096843\\
2.51866296657416	368049.179984802\\
2.51876296907423	368076.681958968\\
2.51886297157429	368103.610975339\\
2.51896297407435	368130.53999171\\
2.51906297657441	368158.041965876\\
2.51916297907448	368184.970982247\\
2.51926298157454	368211.899998619\\
2.5193629840746	368239.401972785\\
2.51946298657466	368266.330989156\\
2.51956298907473	368293.260005527\\
2.51966299157479	368320.189021898\\
2.51976299407485	368347.11803827\\
2.51986299657491	368374.047054641\\
2.51996299907498	368400.976071012\\
2.52006300157504	368427.905087383\\
2.5201630040751	368454.834103754\\
2.52026300657516	368481.763120125\\
2.52036300907523	368508.692136496\\
2.52046301157529	368535.621152868\\
2.52056301407535	368562.550169239\\
2.52066301657541	368589.47918561\\
2.52076301907548	368616.408201981\\
2.52086302157554	368643.337218352\\
2.5209630240756	368670.266234723\\
2.52106302657566	368696.622293299\\
2.52116302907573	368723.55130967\\
2.52126303157579	368750.480326042\\
2.52136303407585	368776.836384618\\
2.52146303657591	368803.765400989\\
2.52156303907598	368830.69441736\\
2.52166304157604	368857.050475936\\
2.5217630440761	368883.979492307\\
2.52186304657616	368910.335550883\\
2.52196304907623	368937.264567254\\
2.52206305157629	368963.62062583\\
2.52216305407635	368990.549642201\\
2.52226305657641	369016.905700777\\
2.52236305907648	369043.834717149\\
2.52246306157654	369070.190775725\\
2.5225630640766	369097.119792096\\
2.52266306657666	369123.475850672\\
2.52276306907673	369149.831909248\\
2.52286307157679	369176.187967824\\
2.52296307407685	369203.116984195\\
2.52306307657691	369229.473042771\\
2.52316307907698	369255.829101347\\
2.52326308157704	369282.185159923\\
2.5233630840771	369308.541218499\\
2.52346308657716	369334.897277075\\
2.52356308907723	369361.826293446\\
2.52366309157729	369388.182352022\\
2.52376309407735	369414.538410598\\
2.52386309657741	369440.894469174\\
2.52396309907748	369467.25052775\\
2.52406310157754	369493.033628531\\
2.5241631040776	369519.389687107\\
2.52426310657766	369545.745745683\\
2.52436310907773	369572.101804259\\
2.52446311157779	369598.457862835\\
2.52456311407785	369624.813921411\\
2.52466311657791	369651.169979987\\
2.52476311907798	369676.953080768\\
2.52486312157804	369703.309139344\\
2.5249631240781	369729.66519792\\
2.52506312657816	369755.448298701\\
2.52516312907823	369781.804357277\\
2.52526313157829	369808.160415853\\
2.52536313407835	369833.943516634\\
2.52546313657841	369860.29957521\\
2.52556313907848	369886.082675991\\
2.52566314157854	369912.438734567\\
2.5257631440786	369938.221835348\\
2.52586314657866	369964.577893924\\
2.52596314907873	369990.360994705\\
2.52606315157879	370016.144095486\\
2.52616315407885	370042.500154062\\
2.52626315657891	370068.283254843\\
2.52636315907898	370094.066355623\\
2.52646316157904	370120.422414199\\
2.5265631640791	370146.20551498\\
2.52666316657916	370171.988615761\\
2.52676316907923	370197.771716542\\
2.52686317157929	370223.554817323\\
2.52696317407935	370249.910875899\\
2.52706317657941	370275.69397668\\
2.52716317907948	370301.477077461\\
2.52726318157954	370327.260178242\\
2.5273631840796	370353.043279023\\
2.52746318657966	370378.826379803\\
2.52756318907973	370404.609480584\\
2.52766319157979	370430.392581365\\
2.52776319407985	370456.175682146\\
2.52786319657991	370481.958782927\\
2.52796319907998	370507.168925913\\
2.52806320158004	370532.952026694\\
2.5281632040801	370558.735127475\\
2.52826320658016	370584.518228255\\
2.52836320908023	370610.301329036\\
2.52846321158029	370635.511472022\\
2.52856321408035	370661.294572803\\
2.52866321658041	370687.077673584\\
2.52876321908048	370712.28781657\\
2.52886322158054	370738.07091735\\
2.5289632240806	370763.281060336\\
2.52906322658066	370789.064161117\\
2.52916322908073	370814.274304103\\
2.52926323158079	370840.057404884\\
2.52936323408085	370865.267547869\\
2.52946323658091	370891.05064865\\
2.52956323908098	370916.260791636\\
2.52966324158104	370942.043892417\\
2.5297632440811	370967.254035403\\
2.52986324658116	370992.464178389\\
2.52996324908123	371018.247279169\\
2.53006325158129	371043.457422155\\
2.53016325408135	371068.667565141\\
2.53026325658141	371093.877708127\\
2.53036325908148	371119.087851112\\
2.53046326158154	371144.870951893\\
2.5305632640816	371170.081094879\\
2.53066326658166	371195.291237865\\
2.53076326908173	371220.501380851\\
2.53086327158179	371245.711523836\\
2.53096327408185	371270.921666822\\
2.53106327658191	371296.131809808\\
2.53116327908198	371321.341952794\\
2.53126328158204	371346.552095779\\
2.5313632840821	371371.762238765\\
2.53146328658216	371396.972381751\\
2.53156328908223	371421.609566942\\
2.53166329158229	371446.819709927\\
2.53176329408235	371472.029852913\\
2.53186329658241	371497.239995899\\
2.53196329908248	371521.877181089\\
2.53206330158254	371547.087324075\\
2.5321633040826	371572.297467061\\
2.53226330658266	371596.934652252\\
2.53236330908273	371622.144795237\\
2.53246331158279	371647.354938223\\
2.53256331408285	371671.992123414\\
2.53266331658291	371697.202266399\\
2.53276331908298	371721.83945159\\
2.53286332158304	371747.049594576\\
2.5329633240831	371771.686779766\\
2.53306332658316	371796.896922752\\
2.53316332908323	371821.534107943\\
2.53326333158329	371846.171293133\\
2.53336333408335	371871.381436119\\
2.53346333658341	371896.01862131\\
2.53356333908348	371920.6558065\\
2.53366334158354	371945.292991691\\
2.5337633440836	371970.503134677\\
2.53386334658366	371995.140319867\\
2.53396334908373	372019.777505058\\
2.53406335158379	372044.414690249\\
2.53416335408385	372069.051875439\\
2.53426335658391	372093.68906063\\
2.53436335908398	372118.326245821\\
2.53446336158404	372142.963431011\\
2.5345633640841	372167.600616202\\
2.53466336658416	372192.237801393\\
2.53476336908423	372216.874986583\\
2.53486337158429	372241.512171774\\
2.53496337408435	372266.149356964\\
2.53506337658441	372290.786542155\\
2.53516337908448	372315.423727346\\
2.53526338158454	372340.060912536\\
2.5353633840846	372364.125139932\\
2.53546338658466	372388.762325122\\
2.53556338908473	372413.399510313\\
2.53566339158479	372437.463737709\\
2.53576339408485	372462.100922899\\
2.53586339658491	372486.73810809\\
2.53596339908498	372510.802335485\\
2.53606340158504	372535.439520676\\
2.5361634040851	372559.503748071\\
2.53626340658516	372584.140933262\\
2.53636340908523	372608.205160657\\
2.53646341158529	372632.842345848\\
2.53656341408535	372656.906573244\\
2.53666341658541	372681.543758434\\
2.53676341908548	372705.60798583\\
2.53686342158554	372729.672213225\\
2.5369634240856	372754.309398416\\
2.53706342658566	372778.373625811\\
2.53716342908573	372802.437853207\\
2.53726343158579	372826.502080602\\
2.53736343408585	372851.139265793\\
2.53746343658591	372875.203493188\\
2.53756343908598	372899.267720584\\
2.53766344158604	372923.331947979\\
2.5377634440861	372947.396175375\\
2.53786344658616	372971.46040277\\
2.53796344908623	372995.524630166\\
2.53806345158629	373019.588857561\\
2.53816345408635	373043.653084957\\
2.53826345658641	373067.717312352\\
2.53836345908648	373091.781539748\\
2.53846346158654	373115.845767143\\
2.5385634640866	373139.909994539\\
2.53866346658666	373163.401264139\\
2.53876346908673	373187.465491535\\
2.53886347158679	373211.52971893\\
2.53896347408685	373235.593946326\\
2.53906347658691	373259.085215926\\
2.53916347908698	373283.149443322\\
2.53926348158704	373307.213670717\\
2.5393634840871	373330.704940317\\
2.53946348658716	373354.769167713\\
2.53956348908723	373378.833395108\\
2.53966349158729	373402.324664709\\
2.53976349408735	373426.388892104\\
2.53986349658741	373449.880161705\\
2.53996349908748	373473.371431305\\
2.54006350158754	373497.435658701\\
2.5401635040876	373520.926928301\\
2.54026350658766	373544.991155696\\
2.54036350908773	373568.482425297\\
2.54046351158779	373591.973694897\\
2.54056351408785	373616.037922293\\
2.54066351658791	373639.529191893\\
2.54076351908798	373663.020461493\\
2.54086352158804	373686.511731094\\
2.5409635240881	373710.003000694\\
2.54106352658816	373733.494270294\\
2.54116352908823	373757.55849769\\
2.54126353158829	373781.04976729\\
2.54136353408835	373804.541036891\\
2.54146353658841	373828.032306491\\
2.54156353908848	373851.523576091\\
2.54166354158854	373875.014845692\\
2.5417635440886	373898.506115292\\
2.54186354658866	373921.424427097\\
2.54196354908873	373944.915696698\\
2.54206355158879	373968.406966298\\
2.54216355408885	373991.898235898\\
2.54226355658891	374015.389505499\\
2.54236355908898	374038.880775099\\
2.54246356158904	374061.799086904\\
2.5425635640891	374085.290356505\\
2.54266356658916	374108.781626105\\
2.54276356908923	374131.69993791\\
2.54286357158929	374155.191207511\\
2.54296357408935	374178.109519316\\
2.54306357658941	374201.600788916\\
2.54316357908948	374224.519100722\\
2.54326358158954	374248.010370322\\
2.5433635840896	374270.928682127\\
2.54346358658966	374294.419951728\\
2.54356358908973	374317.338263533\\
2.54366359158979	374340.829533133\\
2.54376359408985	374363.747844938\\
2.54386359658991	374386.666156744\\
2.54396359908998	374410.157426344\\
2.54406360159004	374433.075738149\\
2.5441636040901	374455.994049954\\
2.54426360659016	374478.91236176\\
2.54436360909023	374501.830673565\\
2.54446361159029	374525.321943165\\
2.54456361409035	374548.24025497\\
2.54466361659041	374571.158566776\\
2.54476361909048	374594.076878581\\
2.54486362159054	374616.995190386\\
2.5449636240906	374639.913502191\\
2.54506362659066	374662.831813997\\
2.54516362909073	374685.750125802\\
2.54526363159079	374708.668437607\\
2.54536363409085	374731.586749412\\
2.54546363659091	374753.932103422\\
2.54556363909098	374776.850415228\\
2.54566364159104	374799.768727033\\
2.5457636440911	374822.687038838\\
2.54586364659116	374845.605350643\\
2.54596364909123	374867.950704654\\
2.54606365159129	374890.869016459\\
2.54616365409135	374913.787328264\\
2.54626365659141	374936.132682274\\
2.54636365909148	374959.050994079\\
2.54646366159154	374981.396348089\\
2.5465636640916	375004.314659895\\
2.54666366659166	375026.660013905\\
2.54676366909173	375049.57832571\\
2.54686367159179	375071.92367972\\
2.54696367409185	375094.841991525\\
2.54706367659192	375117.187345535\\
2.54716367909198	375139.532699545\\
2.54726368159204	375162.451011351\\
2.5473636840921	375184.796365361\\
2.54746368659216	375207.141719371\\
2.54756368909223	375230.060031176\\
2.54766369159229	375252.405385186\\
2.54776369409235	375274.750739196\\
2.54786369659241	375297.096093206\\
2.54796369909248	375319.441447217\\
2.54806370159254	375341.786801227\\
2.5481637040926	375364.132155237\\
2.54826370659266	375386.477509247\\
2.54836370909273	375408.822863257\\
2.54846371159279	375431.168217267\\
2.54856371409285	375453.513571277\\
2.54866371659291	375475.858925287\\
2.54876371909298	375498.204279297\\
2.54886372159304	375520.549633307\\
2.5489637240931	375542.894987318\\
2.54906372659317	375565.240341328\\
2.54916372909323	375587.012737543\\
2.54926373159329	375609.358091553\\
2.54936373409335	375631.703445563\\
2.54946373659341	375654.048799573\\
2.54956373909348	375675.821195788\\
2.54966374159354	375698.166549798\\
2.5497637440936	375719.938946013\\
2.54986374659366	375742.284300023\\
2.54996374909373	375764.056696238\\
2.55006375159379	375786.402050248\\
2.55016375409385	375808.174446463\\
2.55026375659391	375830.519800473\\
2.55036375909398	375852.292196688\\
2.55046376159404	375874.637550698\\
2.5505637640941	375896.409946913\\
2.55066376659416	375918.182343128\\
2.55076376909423	375940.527697138\\
2.55086377159429	375962.300093353\\
2.55096377409435	375984.072489568\\
2.55106377659442	376005.844885783\\
2.55116377909448	376028.190239793\\
2.55126378159454	376049.962636008\\
2.5513637840946	376071.735032223\\
2.55146378659466	376093.507428438\\
2.55156378909473	376115.279824653\\
2.55166379159479	376137.052220868\\
2.55176379409485	376158.824617083\\
2.55186379659491	376180.597013298\\
2.55196379909498	376202.369409513\\
2.55206380159504	376224.141805728\\
2.5521638040951	376245.914201943\\
2.55226380659516	376267.686598158\\
2.55236380909523	376288.886036578\\
2.55246381159529	376310.658432793\\
2.55256381409535	376332.430829008\\
2.55266381659541	376354.203225223\\
2.55276381909548	376375.402663643\\
2.55286382159554	376397.175059858\\
2.5529638240956	376418.947456073\\
2.55306382659567	376440.146894492\\
2.55316382909573	376461.919290707\\
2.55326383159579	376483.691686922\\
2.55336383409585	376504.891125342\\
2.55346383659592	376526.663521557\\
2.55356383909598	376547.862959977\\
2.55366384159604	376569.062398397\\
2.5537638440961	376590.834794612\\
2.55386384659616	376612.034233032\\
2.55396384909623	376633.806629247\\
2.55406385159629	376655.006067667\\
2.55416385409635	376676.205506086\\
2.55426385659641	376697.977902301\\
2.55436385909648	376719.177340721\\
2.55446386159654	376740.376779141\\
2.5545638640966	376761.576217561\\
2.55466386659666	376782.775655981\\
2.55476386909673	376803.975094401\\
2.55486387159679	376825.17453282\\
2.55496387409685	376846.946929035\\
2.55506387659692	376868.146367455\\
2.55516387909698	376889.345805875\\
2.55526388159704	376909.9722865\\
2.5553638840971	376931.17172492\\
2.55546388659717	376952.371163339\\
2.55556388909723	376973.570601759\\
2.55566389159729	376994.770040179\\
2.55576389409735	377015.969478599\\
2.55586389659741	377037.168917019\\
2.55596389909748	377057.795397643\\
2.55606390159754	377078.994836063\\
2.5561639040976	377100.194274483\\
2.55626390659766	377121.393712903\\
2.55636390909773	377142.020193528\\
2.55646391159779	377163.219631948\\
2.55656391409785	377183.846112572\\
2.55666391659791	377205.045550992\\
2.55676391909798	377225.672031617\\
2.55686392159804	377246.871470037\\
2.5569639240981	377267.497950661\\
2.55706392659817	377288.697389081\\
2.55716392909823	377309.323869706\\
2.55726393159829	377330.523308126\\
2.55736393409835	377351.14978875\\
2.55746393659842	377371.776269375\\
2.55756393909848	377392.40275\\
2.55766394159854	377413.60218842\\
2.5577639440986	377434.228669044\\
2.55786394659867	377454.855149669\\
2.55796394909873	377475.481630294\\
2.55806395159879	377496.108110919\\
2.55816395409885	377517.307549338\\
2.55826395659891	377537.934029963\\
2.55836395909898	377558.560510588\\
2.55846396159904	377579.186991212\\
2.5585639640991	377599.813471837\\
2.55866396659916	377620.439952462\\
2.55876396909923	377640.493475291\\
2.55886397159929	377661.119955916\\
2.55896397409935	377681.746436541\\
2.55906397659942	377702.372917166\\
2.55916397909948	377722.99939779\\
2.55926398159954	377743.625878415\\
2.5593639840996	377763.679401245\\
2.55946398659967	377784.305881869\\
2.55956398909973	377804.932362494\\
2.55966399159979	377824.985885324\\
2.55976399409985	377845.612365948\\
2.55986399659992	377866.238846573\\
2.55996399909998	377886.292369403\\
2.56006400160004	377906.918850027\\
2.5601640041001	377926.972372857\\
2.56026400660016	377947.598853482\\
2.56036400910023	377967.652376311\\
2.56046401160029	377988.278856936\\
2.56056401410035	378008.332379765\\
2.56066401660041	378028.385902595\\
2.56076401910048	378049.01238322\\
2.56086402160054	378069.065906049\\
2.5609640241006	378089.119428879\\
2.56106402660067	378109.172951709\\
2.56116402910073	378129.799432333\\
2.56126403160079	378149.852955163\\
2.56136403410085	378169.906477992\\
2.56146403660092	378189.960000822\\
2.56156403910098	378210.013523652\\
2.56166404160104	378230.067046481\\
2.5617640441011	378250.120569311\\
2.56186404660117	378270.17409214\\
2.56196404910123	378290.22761497\\
2.56206405160129	378310.281137799\\
2.56216405410135	378330.334660629\\
2.56226405660141	378350.388183459\\
2.56236405910148	378370.441706288\\
2.56246406160154	378389.922271323\\
2.5625640641016	378409.975794152\\
2.56266406660166	378430.029316982\\
2.56276406910173	378450.082839811\\
2.56286407160179	378469.563404846\\
2.56296407410185	378489.616927675\\
2.56306407660192	378509.670450505\\
2.56316407910198	378529.151015539\\
2.56326408160204	378549.204538369\\
2.5633640841021	378568.685103403\\
2.56346408660217	378588.738626233\\
2.56356408910223	378608.219191268\\
2.56366409160229	378628.272714097\\
2.56376409410235	378647.753279131\\
2.56386409660242	378667.806801961\\
2.56396409910248	378687.287366996\\
2.56406410160254	378706.76793203\\
2.5641641041026	378726.82145486\\
2.56426410660267	378746.302019894\\
2.56436410910273	378765.782584928\\
2.56446411160279	378785.263149963\\
2.56456411410285	378804.743714997\\
2.56466411660291	378824.797237827\\
2.56476411910298	378844.277802861\\
2.56486412160304	378863.758367896\\
2.5649641241031	378883.23893293\\
2.56506412660317	378902.719497965\\
2.56516412910323	378922.200062999\\
2.56526413160329	378941.680628034\\
2.56536413410335	378961.161193068\\
2.56546413660342	378980.641758103\\
2.56556413910348	378999.549365342\\
2.56566414160354	379019.029930376\\
2.5657641441036	379038.510495411\\
2.56586414660367	379057.991060445\\
2.56596414910373	379077.47162548\\
2.56606415160379	379096.379232719\\
2.56616415410385	379115.859797753\\
2.56626415660392	379135.340362788\\
2.56636415910398	379154.247970027\\
2.56646416160404	379173.728535062\\
2.5665641641041	379192.636142301\\
2.56666416660416	379212.116707335\\
2.56676416910423	379231.59727237\\
2.56686417160429	379250.504879609\\
2.56696417410435	379269.412486848\\
2.56706417660442	379288.893051883\\
2.56716417910448	379307.800659122\\
2.56726418160454	379327.281224157\\
2.5673641841046	379346.188831396\\
2.56746418660467	379365.096438635\\
2.56756418910473	379384.004045875\\
2.56766419160479	379403.484610909\\
2.56776419410485	379422.392218148\\
2.56786419660492	379441.299825388\\
2.56796419910498	379460.207432627\\
2.56806420160504	379479.115039866\\
2.5681642041051	379498.022647106\\
2.56826420660517	379516.930254345\\
2.56836420910523	379535.837861584\\
2.56846421160529	379554.745468824\\
2.56856421410535	379573.653076063\\
2.56866421660541	379592.560683302\\
2.56876421910548	379611.468290542\\
2.56886422160554	379630.375897781\\
2.5689642241056	379649.28350502\\
2.56906422660567	379668.19111226\\
2.56916422910573	379687.098719499\\
2.56926423160579	379705.433368943\\
2.56936423410585	379724.340976182\\
2.56946423660592	379743.248583422\\
2.56956423910598	379761.583232866\\
2.56966424160604	379780.490840105\\
2.5697642441061	379799.398447344\\
2.56986424660617	379817.733096789\\
2.56996424910623	379836.640704028\\
2.57006425160629	379854.975353472\\
2.57016425410635	379873.882960711\\
2.57026425660642	379892.217610156\\
2.57036425910648	379910.5522596\\
2.57046426160654	379929.459866839\\
2.5705642641066	379947.794516283\\
2.57066426660667	379966.129165728\\
2.57076426910673	379985.036772967\\
2.57086427160679	380003.371422411\\
2.57096427410685	380021.706071855\\
2.57106427660692	380040.040721299\\
2.57116427910698	380058.948328539\\
2.57126428160704	380077.282977983\\
2.5713642841071	380095.617627427\\
2.57146428660717	380113.952276871\\
2.57156428910723	380132.286926315\\
2.57166429160729	380150.62157576\\
2.57176429410735	380168.956225204\\
2.57186429660742	380187.290874648\\
2.57196429910748	380205.625524092\\
2.57206430160754	380223.960173536\\
2.5721643041076	380242.294822981\\
2.57226430660767	380260.05651463\\
2.57236430910773	380278.391164074\\
2.57246431160779	380296.725813518\\
2.57256431410785	380315.060462962\\
2.57266431660792	380332.822154611\\
2.57276431910798	380351.156804055\\
2.57286432160804	380369.4914535\\
2.5729643241081	380387.253145149\\
2.57306432660817	380405.587794593\\
2.57316432910823	380423.349486242\\
2.57326433160829	380441.684135686\\
2.57336433410835	380459.445827335\\
2.57346433660842	380477.780476779\\
2.57356433910848	380495.542168428\\
2.57366434160854	380513.876817873\\
2.5737643441086	380531.638509522\\
2.57386434660867	380549.400201171\\
2.57396434910873	380567.734850615\\
2.57406435160879	380585.496542264\\
2.57416435410885	380603.258233913\\
2.57426435660892	380621.019925562\\
2.57436435910898	380639.354575006\\
2.57446436160904	380657.116266655\\
2.5745643641091	380674.877958304\\
2.57466436660917	380692.639649953\\
2.57476436910923	380710.401341602\\
2.57486437160929	380728.163033252\\
2.57496437410935	380745.924724901\\
2.57506437660942	380763.68641655\\
2.57516437910948	380781.448108199\\
2.57526438160954	380799.209799848\\
2.5753643841096	380816.971491497\\
2.57546438660967	380834.733183146\\
2.57556438910973	380851.921917\\
2.57566439160979	380869.683608649\\
2.57576439410985	380887.445300298\\
2.57586439660992	380905.206991947\\
2.57596439910998	380922.395725801\\
2.57606440161004	380940.15741745\\
2.5761644041101	380957.919109099\\
2.57626440661017	380975.107842953\\
2.57636440911023	380992.869534602\\
2.57646441161029	381010.058268456\\
2.57656441411035	381027.819960105\\
2.57666441661042	381045.008693959\\
2.57676441911048	381062.770385608\\
2.57686442161054	381079.959119462\\
2.5769644241106	381097.147853316\\
2.57706442661067	381114.909544965\\
2.57716442911073	381132.098278819\\
2.57726443161079	381149.287012673\\
2.57736443411085	381167.048704322\\
2.57746443661092	381184.237438176\\
2.57756443911098	381201.42617203\\
2.57766444161104	381218.614905883\\
2.5777644441111	381235.803639737\\
2.57786444661117	381252.992373591\\
2.57796444911123	381270.181107445\\
2.57806445161129	381287.942799094\\
2.57816445411135	381305.131532948\\
2.57826445661142	381321.747309007\\
2.57836445911148	381338.936042861\\
2.57846446161154	381356.124776715\\
2.5785644641116	381373.313510569\\
2.57866446661167	381390.502244423\\
2.57876446911173	381407.690978277\\
2.57886447161179	381424.879712131\\
2.57896447411185	381441.495488189\\
2.57906447661192	381458.684222043\\
2.57916447911198	381475.872955897\\
2.57926448161204	381493.061689751\\
2.5793644841121	381509.67746581\\
2.57946448661217	381526.866199664\\
2.57956448911223	381543.481975723\\
2.57966449161229	381560.670709577\\
2.57976449411235	381577.286485635\\
2.57986449661242	381594.475219489\\
2.57996449911248	381611.090995548\\
2.58006450161254	381628.279729402\\
2.5801645041126	381644.895505461\\
2.58026450661267	381661.51128152\\
2.58036450911273	381678.700015374\\
2.58046451161279	381695.315791432\\
2.58056451411285	381711.931567491\\
2.58066451661292	381729.120301345\\
2.58076451911298	381745.736077404\\
2.58086452161304	381762.351853463\\
2.5809645241131	381778.967629521\\
2.58106452661317	381795.58340558\\
2.58116452911323	381812.199181639\\
2.58126453161329	381828.814957698\\
2.58136453411335	381845.430733757\\
2.58146453661342	381862.046509815\\
2.58156453911348	381878.662285874\\
2.58166454161354	381895.278061933\\
2.5817645441136	381911.893837992\\
2.58186454661367	381928.509614051\\
2.58196454911373	381945.125390109\\
2.58206455161379	381961.168208373\\
2.58216455411385	381977.783984432\\
2.58226455661392	381994.399760491\\
2.58236455911398	382011.015536549\\
2.58246456161404	382027.058354813\\
2.5825645641141	382043.674130872\\
2.58266456661417	382060.289906931\\
2.58276456911423	382076.332725194\\
2.58286457161429	382092.948501253\\
2.58296457411435	382108.991319517\\
2.58306457661442	382125.607095576\\
2.58316457911448	382141.649913839\\
2.58326458161454	382158.265689898\\
2.5833645841146	382174.308508162\\
2.58346458661467	382190.351326425\\
2.58356458911473	382206.967102484\\
2.58366459161479	382223.009920748\\
2.58376459411485	382239.052739011\\
2.58386459661492	382255.095557275\\
2.58396459911498	382271.711333334\\
2.58406460161504	382287.754151598\\
2.5841646041151	382303.796969861\\
2.58426460661517	382319.839788125\\
2.58436460911523	382335.882606389\\
2.58446461161529	382351.925424652\\
2.58456461411535	382367.968242916\\
2.58466461661542	382384.01106118\\
2.58476461911548	382400.053879443\\
2.58486462161554	382416.096697707\\
2.5849646241156	382432.139515971\\
2.58506462661567	382448.182334234\\
2.58516462911573	382464.225152498\\
2.58526463161579	382479.695012966\\
2.58536463411585	382495.73783123\\
2.58546463661592	382511.780649494\\
2.58556463911598	382527.823467757\\
2.58566464161604	382543.293328226\\
2.5857646441161	382559.33614649\\
2.58586464661617	382574.806006958\\
2.58596464911623	382590.848825222\\
2.58606465161629	382606.891643485\\
2.58616465411635	382622.361503954\\
2.58626465661642	382638.404322218\\
2.58636465911648	382653.874182686\\
2.58646466161654	382669.344043155\\
2.5865646641166	382685.386861418\\
2.58666466661667	382700.856721887\\
2.58676466911673	382716.326582355\\
2.58686467161679	382732.369400619\\
2.58696467411685	382747.839261088\\
2.58706467661692	382763.309121556\\
2.58716467911698	382778.778982025\\
2.58726468161704	382794.821800288\\
2.5873646841171	382810.291660757\\
2.58746468661717	382825.761521225\\
2.58756468911723	382841.231381694\\
2.58766469161729	382856.701242163\\
2.58776469411735	382872.171102631\\
2.58786469661742	382887.6409631\\
2.58796469911748	382903.110823568\\
2.58806470161754	382918.580684037\\
2.5881647041176	382934.050544505\\
2.58826470661767	382949.520404974\\
2.58836470911773	382964.417307647\\
2.58846471161779	382979.887168116\\
2.58856471411785	382995.357028584\\
2.58866471661792	383010.826889053\\
2.58876471911798	383025.723791726\\
2.58886472161804	383041.193652195\\
2.5889647241181	383056.663512663\\
2.58906472661817	383071.560415337\\
2.58916472911823	383087.030275805\\
2.58926473161829	383101.927178479\\
2.58936473411835	383117.397038947\\
2.58946473661842	383132.293941621\\
2.58956473911848	383147.763802089\\
2.58966474161854	383162.660704762\\
2.5897647441186	383178.130565231\\
2.58986474661867	383193.027467904\\
2.58996474911873	383207.924370578\\
2.59006475161879	383222.821273251\\
2.59016475411885	383238.29113372\\
2.59026475661892	383253.188036393\\
2.59036475911898	383268.084939066\\
2.59046476161904	383282.98184174\\
2.5905647641191	383297.878744413\\
2.59066476661917	383313.348604882\\
2.59076476911923	383328.245507555\\
2.59086477161929	383343.142410229\\
2.59096477411935	383358.039312902\\
2.59106477661942	383372.936215575\\
2.59116477911948	383387.833118249\\
2.59126478161954	383402.157063127\\
2.5913647841196	383417.0539658\\
2.59146478661967	383431.950868474\\
2.59156478911973	383446.847771147\\
2.59166479161979	383461.744673821\\
2.59176479411985	383476.068618699\\
2.59186479661992	383490.965521372\\
2.59196479911998	383505.862424046\\
2.59206480162004	383520.759326719\\
2.5921648041201	383535.083271597\\
2.59226480662017	383549.980174271\\
2.59236480912023	383564.304119149\\
2.59246481162029	383579.201021823\\
2.59256481412035	383593.524966701\\
2.59266481662042	383608.421869374\\
2.59276481912048	383622.745814252\\
2.59286482162054	383637.642716926\\
2.5929648241206	383651.966661804\\
2.59306482662067	383666.290606682\\
2.59316482912073	383681.187509356\\
2.59326483162079	383695.511454234\\
2.59336483412085	383709.835399112\\
2.59346483662092	383724.159343991\\
2.59356483912098	383739.056246664\\
2.59366484162104	383753.380191542\\
2.5937648441211	383767.704136421\\
2.59386484662117	383782.028081299\\
2.59396484912123	383796.352026177\\
2.59406485162129	383810.675971055\\
2.59416485412135	383824.999915934\\
2.59426485662142	383839.323860812\\
2.59436485912148	383853.64780569\\
2.59446486162154	383867.971750568\\
2.5945648641216	383882.295695447\\
2.59466486662167	383896.04668253\\
2.59476486912173	383910.370627408\\
2.59486487162179	383924.694572286\\
2.59496487412185	383939.018517165\\
2.59506487662192	383952.769504248\\
2.59516487912198	383967.093449126\\
2.59526488162204	383981.417394004\\
2.5953648841221	383995.168381087\\
2.59546488662217	384009.492325966\\
2.59556488912223	384023.243313049\\
2.59566489162229	384037.567257927\\
2.59576489412235	384051.31824501\\
2.59586489662242	384065.642189889\\
2.59596489912248	384079.393176972\\
2.59606490162254	384093.71712185\\
2.5961649041226	384107.468108933\\
2.59626490662267	384121.219096016\\
2.59636490912273	384135.543040895\\
2.59646491162279	384149.294027978\\
2.59656491412285	384163.045015061\\
2.59666491662292	384176.796002144\\
2.59676491912298	384190.546989227\\
2.59686492162304	384204.870934105\\
2.5969649241231	384218.621921188\\
2.59706492662317	384232.372908272\\
2.59716492912323	384246.123895355\\
2.59726493162329	384259.874882438\\
2.59736493412335	384273.625869521\\
2.59746493662342	384287.376856604\\
2.59756493912348	384301.127843687\\
2.59766494162354	384314.305872975\\
2.5977649441236	384328.056860058\\
2.59786494662367	384341.807847142\\
2.59796494912373	384355.558834225\\
2.59806495162379	384369.309821308\\
2.59816495412385	384382.487850596\\
2.59826495662392	384396.238837679\\
2.59836495912398	384409.989824762\\
2.59846496162404	384423.16785405\\
2.5985649641241	384436.918841133\\
2.59866496662417	384450.096870421\\
2.59876496912423	384463.847857505\\
2.59886497162429	384477.025886792\\
2.59896497412435	384490.776873876\\
2.59906497662442	384503.954903164\\
2.59916497912448	384517.705890247\\
2.59926498162454	384530.883919535\\
2.5993649841246	384544.061948823\\
2.59946498662467	384557.812935906\\
2.59956498912473	384570.990965194\\
2.59966499162479	384584.168994482\\
2.59976499412485	384597.34702377\\
2.59986499662492	384611.098010853\\
2.59996499912498	384624.276040141\\
2.60006500162504	384637.454069429\\
2.6001650041251	384650.632098717\\
2.60026500662517	384663.810128005\\
2.60036500912523	384676.988157293\\
2.60046501162529	384690.166186581\\
2.60056501412535	384703.344215869\\
2.60066501662542	384716.522245157\\
2.60076501912548	384729.700274445\\
2.60086502162554	384742.878303733\\
2.6009650241256	384755.483375226\\
2.60106502662567	384768.661404514\\
2.60116502912573	384781.839433802\\
2.60126503162579	384795.01746309\\
2.60136503412585	384807.622534583\\
2.60146503662592	384820.800563871\\
2.60156503912598	384833.978593159\\
2.60166504162604	384846.583664652\\
2.6017650441261	384859.76169394\\
2.60186504662617	384872.366765433\\
2.60196504912623	384885.544794721\\
2.60206505162629	384898.149866214\\
2.60216505412635	384911.327895502\\
2.60226505662642	384923.932966995\\
2.60236505912648	384937.110996283\\
2.60246506162654	384949.716067775\\
2.6025650641266	384962.321139268\\
2.60266506662667	384974.926210761\\
2.60276506912673	384988.104240049\\
2.60286507162679	385000.709311542\\
2.60296507412685	385013.314383035\\
2.60306507662692	385025.919454528\\
2.60316507912698	385038.524526021\\
2.60326508162704	385051.129597514\\
2.6033650841271	385064.307626802\\
2.60346508662717	385076.912698295\\
2.60356508912723	385089.517769787\\
2.60366509162729	385102.12284128\\
2.60376509412735	385114.154954978\\
2.60386509662742	385126.760026471\\
2.60396509912748	385139.365097964\\
2.60406510162754	385151.970169457\\
2.6041651041276	385164.575240949\\
2.60426510662767	385177.180312442\\
2.60436510912773	385189.21242614\\
2.60446511162779	385201.817497633\\
2.60456511412785	385214.422569126\\
2.60466511662792	385226.454682824\\
2.60476511912798	385239.059754316\\
2.60486512162804	385251.091868014\\
2.6049651241281	385263.696939507\\
2.60506512662817	385276.302011\\
2.60516512912823	385288.334124698\\
2.60526513162829	385300.366238396\\
2.60536513412835	385312.971309888\\
2.60546513662842	385325.003423586\\
2.60556513912848	385337.608495079\\
2.60566514162854	385349.640608777\\
2.6057651441286	385361.672722474\\
2.60586514662867	385373.704836172\\
2.60596514912873	385386.309907665\\
2.60606515162879	385398.342021363\\
2.60616515412885	385410.374135061\\
2.60626515662892	385422.406248758\\
2.60636515912898	385434.438362456\\
2.60646516162904	385446.470476154\\
2.6065651641291	385458.502589852\\
2.60666516662917	385470.534703549\\
2.60676516912923	385482.566817247\\
2.60686517162929	385494.598930945\\
2.60696517412935	385506.631044643\\
2.60706517662942	385518.66315834\\
2.60716517912948	385530.695272038\\
2.60726518162954	385542.727385736\\
2.6073651841296	385554.186541638\\
2.60746518662967	385566.218655336\\
2.60756518912973	385578.250769034\\
2.60766519162979	385589.709924937\\
2.60776519412985	385601.742038634\\
2.60786519662992	385613.774152332\\
2.60796519912998	385625.233308235\\
2.60806520163004	385637.265421932\\
2.6081652041301	385648.724577835\\
2.60826520663017	385660.756691533\\
2.60836520913023	385672.215847435\\
2.60846521163029	385684.247961133\\
2.60856521413035	385695.707117036\\
2.60866521663042	385707.166272938\\
2.60876521913048	385719.198386636\\
2.60886522163054	385730.657542539\\
2.6089652241306	385742.116698441\\
2.60906522663067	385753.575854344\\
2.60916522913073	385765.607968042\\
2.60926523163079	385777.067123944\\
2.60936523413085	385788.526279847\\
2.60946523663092	385799.98543575\\
2.60956523913098	385811.444591652\\
2.60966524163104	385822.903747555\\
2.6097652441311	385834.362903457\\
2.60986524663117	385845.82205936\\
2.60996524913123	385857.281215263\\
2.61006525163129	385868.740371165\\
2.61016525413135	385880.199527068\\
2.61026525663142	385891.65868297\\
2.61036525913148	385902.544881078\\
2.61046526163154	385914.004036981\\
2.6105652641316	385925.463192883\\
2.61066526663167	385936.922348786\\
2.61076526913173	385947.808546893\\
2.61086527163179	385959.267702796\\
2.61096527413185	385970.153900903\\
2.61106527663192	385981.613056806\\
2.61116527913198	385993.072212709\\
2.61126528163204	386003.958410816\\
2.6113652841321	386015.417566719\\
2.61146528663217	386026.303764826\\
2.61156528913223	386037.189962934\\
2.61166529163229	386048.649118836\\
2.61176529413235	386059.535316944\\
2.61186529663242	386070.421515051\\
2.61196529913248	386081.880670954\\
2.61206530163254	386092.766869061\\
2.6121653041326	386103.653067169\\
2.61226530663267	386114.539265276\\
2.61236530913273	386125.998421179\\
2.61246531163279	386136.884619287\\
2.61256531413285	386147.770817394\\
2.61266531663292	386158.657015501\\
2.61276531913298	386169.543213609\\
2.61286532163304	386180.429411716\\
2.6129653241331	386191.315609824\\
2.61306532663317	386202.201807931\\
2.61316532913323	386213.088006039\\
2.61326533163329	386223.401246351\\
2.61336533413335	386234.287444459\\
2.61346533663342	386245.173642566\\
2.61356533913348	386256.059840674\\
2.61366534163354	386266.946038781\\
2.6137653441336	386277.259279094\\
2.61386534663367	386288.145477201\\
2.61396534913373	386299.031675308\\
2.61406535163379	386309.344915621\\
2.61416535413385	386320.231113728\\
2.61426535663392	386330.544354041\\
2.61436535913398	386341.430552148\\
2.61446536163404	386351.743792461\\
2.6145653641341	386362.629990568\\
2.61466536663417	386372.94323088\\
2.61476536913423	386383.256471193\\
2.61486537163429	386394.1426693\\
2.61496537413435	386404.455909613\\
2.61506537663442	386414.769149925\\
2.61516537913448	386425.655348032\\
2.61526538163454	386435.968588345\\
2.6153653841346	386446.281828657\\
2.61546538663467	386456.595068969\\
2.61556538913473	386466.908309282\\
2.61566539163479	386477.221549594\\
2.61576539413485	386487.534789907\\
2.61586539663492	386497.848030219\\
2.61596539913498	386508.161270531\\
2.61606540163504	386518.474510844\\
2.6161654041351	386528.787751156\\
2.61626540663517	386539.100991468\\
2.61636540913523	386549.414231781\\
2.61646541163529	386559.727472093\\
2.61656541413535	386570.040712405\\
2.61666541663542	386579.780994923\\
2.61676541913548	386590.094235235\\
2.61686542163554	386600.407475547\\
2.6169654241356	386610.72071586\\
2.61706542663567	386620.460998377\\
2.61716542913573	386630.774238689\\
2.61726543163579	386640.514521206\\
2.61736543413585	386650.827761519\\
2.61746543663592	386660.568044036\\
2.61756543913598	386670.881284348\\
2.61766544163604	386680.621566866\\
2.6177654441361	386690.934807178\\
2.61786544663617	386700.675089695\\
2.61796544913623	386710.415372212\\
2.61806545163629	386720.728612525\\
2.61816545413635	386730.468895042\\
2.61826545663642	386740.209177559\\
2.61836545913648	386749.949460076\\
2.61846546163654	386760.262700389\\
2.6185654641366	386770.002982906\\
2.61866546663667	386779.743265423\\
2.61876546913673	386789.483547941\\
2.61886547163679	386799.223830458\\
2.61896547413685	386808.964112975\\
2.61906547663692	386818.704395492\\
2.61916547913698	386828.444678009\\
2.61926548163704	386838.184960527\\
2.6193654841371	386847.925243044\\
2.61946548663717	386857.092567766\\
2.61956548913723	386866.832850283\\
2.61966549163729	386876.5731328\\
2.61976549413735	386886.313415318\\
2.61986549663742	386896.053697835\\
2.61996549913748	386905.221022557\\
2.62006550163754	386914.961305074\\
2.6201655041376	386924.128629796\\
2.62026550663767	386933.868912313\\
2.62036550913773	386943.609194831\\
2.62046551163779	386952.776519553\\
2.62056551413785	386962.51680207\\
2.62066551663792	386971.684126792\\
2.62076551913798	386980.851451514\\
2.62086552163804	386990.591734031\\
2.6209655241381	386999.759058754\\
2.62106552663817	387009.499341271\\
2.62116552913823	387018.666665993\\
2.62126553163829	387027.833990715\\
2.62136553413835	387037.001315437\\
2.62146553663842	387046.741597954\\
2.62156553913848	387055.908922676\\
2.62166554163854	387065.076247398\\
2.6217655441386	387074.243572121\\
2.62186554663867	387083.410896843\\
2.62196554913873	387092.578221565\\
2.62206555163879	387101.745546287\\
2.62216555413885	387110.912871009\\
2.62226555663892	387120.080195731\\
2.62236555913898	387129.247520453\\
2.62246556163904	387138.414845175\\
2.6225655641391	387147.582169897\\
2.62266556663917	387156.176536824\\
2.62276556913923	387165.343861546\\
2.62286557163929	387174.511186268\\
2.62296557413935	387183.67851099\\
2.62306557663942	387192.272877917\\
2.62316557913948	387201.44020264\\
2.62326558163954	387210.034569566\\
2.6233655841396	387219.201894289\\
2.62346558663967	387228.369219011\\
2.62356558913973	387236.963585938\\
2.62366559163979	387246.13091066\\
2.62376559413985	387254.725277587\\
2.62386559663992	387263.319644514\\
2.62396559913998	387272.486969236\\
2.62406560164004	387281.081336163\\
2.6241656041401	387289.67570309\\
2.62426560664017	387298.843027812\\
2.62436560914023	387307.437394739\\
2.62446561164029	387316.031761666\\
2.62456561414035	387324.626128593\\
2.62466561664042	387333.793453315\\
2.62476561914048	387342.387820242\\
2.62486562164054	387350.982187169\\
2.6249656241406	387359.576554096\\
2.62506562664067	387368.170921023\\
2.62516562914073	387376.76528795\\
2.62526563164079	387385.359654877\\
2.62536563414085	387393.954021804\\
2.62546563664092	387402.548388731\\
2.62556563914098	387410.569797862\\
2.62566564164104	387419.164164789\\
2.6257656441411	387427.758531716\\
2.62586564664117	387436.352898643\\
2.62596564914123	387444.374307775\\
2.62606565164129	387452.968674702\\
2.62616565414135	387461.563041629\\
2.62626565664142	387469.584450761\\
2.62636565914148	387478.178817688\\
2.62646566164154	387486.773184615\\
2.6265656641416	387494.794593747\\
2.62666566664167	387503.388960674\\
2.62676566914173	387511.410369805\\
2.62686567164179	387519.431778937\\
2.62696567414185	387528.026145864\\
2.62706567664192	387536.047554996\\
2.62716567914198	387544.641921923\\
2.62726568164204	387552.663331055\\
2.6273656841421	387560.684740187\\
2.62746568664217	387568.706149318\\
2.62756568914223	387577.300516245\\
2.62766569164229	387585.321925377\\
2.62776569414235	387593.343334509\\
2.62786569664242	387601.364743641\\
2.62796569914248	387609.386152773\\
2.62806570164254	387617.407561905\\
2.6281657041426	387625.428971036\\
2.62826570664267	387633.450380168\\
2.62836570914273	387641.4717893\\
2.62846571164279	387649.493198432\\
2.62856571414285	387657.514607564\\
2.62866571664292	387665.536016695\\
2.62876571914298	387672.984468032\\
2.62886572164304	387681.005877164\\
2.6289657241431	387689.027286296\\
2.62906572664317	387697.048695428\\
2.62916572914323	387704.497146764\\
2.62926573164329	387712.518555896\\
2.62936573414335	387720.539965028\\
2.62946573664342	387727.988416365\\
2.62956573914348	387736.009825497\\
2.62966574164354	387743.458276833\\
2.6297657441436	387751.479685965\\
2.62986574664367	387758.928137302\\
2.62996574914373	387766.376588639\\
2.63006575164379	387774.39799777\\
2.63016575414385	387781.846449107\\
2.63026575664392	387789.867858239\\
2.63036575914398	387797.316309576\\
2.63046576164404	387804.764760912\\
2.6305657641441	387812.213212249\\
2.63066576664417	387819.661663586\\
2.63076576914423	387827.683072718\\
2.63086577164429	387835.131524054\\
2.63096577414435	387842.579975391\\
2.63106577664442	387850.028426728\\
2.63116577914448	387857.476878064\\
2.63126578164454	387864.925329401\\
2.6313657841446	387872.373780738\\
2.63146578664467	387879.822232074\\
2.63156578914473	387887.270683411\\
2.63166579164479	387894.146176953\\
2.63176579414485	387901.594628289\\
2.63186579664492	387909.043079626\\
2.63196579914498	387916.491530963\\
2.63206580164504	387923.9399823\\
2.6321658041451	387930.815475841\\
2.63226580664517	387938.263927178\\
2.63236580914523	387945.139420719\\
2.63246581164529	387952.587872056\\
2.63256581414535	387960.036323393\\
2.63266581664542	387966.911816934\\
2.63276581914548	387974.360268271\\
2.63286582164554	387981.235761813\\
2.6329658241456	387988.684213149\\
2.63306582664567	387995.559706691\\
2.63316582914573	388002.435200232\\
2.63326583164579	388009.883651569\\
2.63336583414585	388016.759145111\\
2.63346583664592	388023.634638652\\
2.63356583914598	388030.510132194\\
2.63366584164604	388037.958583531\\
2.6337658441461	388044.834077072\\
2.63386584664617	388051.709570614\\
2.63396584914623	388058.585064155\\
2.63406585164629	388065.460557697\\
2.63416585414635	388072.336051238\\
2.63426585664642	388079.21154478\\
2.63436585914648	388086.087038322\\
2.63446586164654	388092.962531863\\
2.6345658641466	388099.838025405\\
2.63466586664667	388106.713518946\\
2.63476586914673	388113.589012488\\
2.63486587164679	388119.891548234\\
2.63496587414685	388126.767041776\\
2.63506587664692	388133.642535317\\
2.63516587914698	388140.518028859\\
2.63526588164704	388146.820564605\\
2.6353658841471	388153.696058147\\
2.63546588664717	388159.998593893\\
2.63556588914723	388166.874087435\\
2.63566589164729	388173.749580977\\
2.63576589414735	388180.052116723\\
2.63586589664742	388186.927610265\\
2.63596589914748	388193.230146011\\
2.63606590164754	388199.532681757\\
2.6361659041476	388206.408175299\\
2.63626590664767	388212.710711045\\
2.63636590914773	388219.013246792\\
2.63646591164779	388225.888740333\\
2.63656591414785	388232.19127608\\
2.63666591664792	388238.493811826\\
2.63676591914798	388244.796347573\\
2.63686592164804	388251.098883319\\
2.6369659241481	388257.401419066\\
2.63706592664817	388264.276912607\\
2.63716592914823	388270.579448354\\
2.63726593164829	388276.8819841\\
2.63736593414835	388283.184519847\\
2.63746593664842	388289.487055593\\
2.63756593914848	388295.216633544\\
2.63766594164854	388301.519169291\\
2.6377659441486	388307.821705037\\
2.63786594664867	388314.124240784\\
2.63796594914873	388320.42677653\\
2.63806595164879	388326.156354481\\
2.63816595414885	388332.458890228\\
2.63826595664892	388338.761425974\\
2.63836595914898	388345.063961721\\
2.63846596164904	388350.793539672\\
2.6385659641491	388357.096075418\\
2.63866596664917	388362.82565337\\
2.63876596914923	388369.128189116\\
2.63886597164929	388374.857767068\\
2.63896597414935	388381.160302814\\
2.63906597664942	388386.889880765\\
2.63916597914948	388392.619458716\\
2.63926598164954	388398.921994463\\
2.6393659841496	388404.651572414\\
2.63946598664967	388410.381150366\\
2.63956598914973	388416.683686112\\
2.63966599164979	388422.413264063\\
2.63976599414985	388428.142842015\\
2.63986599664992	388433.872419966\\
2.63996599914998	388439.601997917\\
2.64006600165004	388445.331575869\\
2.6401660041501	388451.06115382\\
2.64026600665017	388456.790731771\\
2.64036600915023	388462.520309723\\
2.64046601165029	388468.249887674\\
2.64056601415035	388473.979465625\\
2.64066601665042	388479.709043576\\
2.64076601915048	388485.438621528\\
2.64086602165054	388491.168199479\\
2.6409660241506	388496.89777743\\
2.64106602665067	388502.054397587\\
2.64116602915073	388507.783975538\\
2.64126603165079	388513.513553489\\
2.64136603415085	388518.670173645\\
2.64146603665092	388524.399751597\\
2.64156603915098	388530.129329548\\
2.64166604165104	388535.285949704\\
2.6417660441511	388541.015527655\\
2.64186604665117	388546.172147812\\
2.64196604915123	388551.901725763\\
2.64206605165129	388557.058345919\\
2.64216605415135	388562.78792387\\
2.64226605665142	388567.944544027\\
2.64236605915148	388573.101164183\\
2.64246606165154	388578.257784339\\
2.6425660641516	388583.98736229\\
2.64266606665167	388589.143982446\\
2.64276606915173	388594.300602603\\
2.64286607165179	388599.457222759\\
2.64296607415185	388604.613842915\\
2.64306607665192	388610.343420866\\
2.64316607915198	388615.500041022\\
2.64326608165204	388620.656661179\\
2.6433660841521	388625.813281335\\
2.64346608665217	388630.969901491\\
2.64356608915223	388635.553563852\\
2.64366609165229	388640.710184008\\
2.64376609415235	388645.866804164\\
2.64386609665242	388651.023424321\\
2.64396609915248	388656.180044477\\
2.64406610165254	388661.336664633\\
2.6441661041526	388665.920326994\\
2.64426610665267	388671.07694715\\
2.64436610915273	388676.233567306\\
2.64446611165279	388680.817229667\\
2.64456611415285	388685.973849824\\
2.64466611665292	388690.557512185\\
2.64476611915298	388695.714132341\\
2.64486612165304	388700.297794702\\
2.6449661241531	388705.454414858\\
2.64506612665317	388710.038077219\\
2.64516612915323	388715.194697375\\
2.64526613165329	388719.778359736\\
2.64536613415335	388724.362022097\\
2.64546613665342	388729.518642253\\
2.64556613915348	388734.102304614\\
2.64566614165354	388738.685966976\\
2.6457661441536	388743.269629337\\
2.64586614665367	388748.426249493\\
2.64596614915373	388753.009911854\\
2.64606615165379	388757.593574215\\
2.64616615415385	388762.177236576\\
2.64626615665392	388766.760898937\\
2.64636615915398	388771.344561298\\
2.64646616165404	388775.928223659\\
2.6465661641541	388780.51188602\\
2.64666616665417	388785.095548381\\
2.64676616915423	388789.679210742\\
2.64686617165429	388793.689915308\\
2.64696617415435	388798.273577669\\
2.64706617665442	388802.85724003\\
2.64716617915448	388807.440902391\\
2.64726618165454	388811.451606957\\
2.6473661841546	388816.035269318\\
2.64746618665467	388820.618931679\\
2.64756618915473	388824.629636245\\
2.64766619165479	388829.213298606\\
2.64776619415485	388833.224003172\\
2.64786619665492	388837.807665533\\
2.64796619915498	388841.818370099\\
2.64806620165504	388846.40203246\\
2.6481662041551	388850.412737026\\
2.64826620665517	388854.996399387\\
2.64836620915523	388859.007103953\\
2.64846621165529	388863.017808519\\
2.64856621415535	388867.028513085\\
2.64866621665542	388871.612175446\\
2.64876621915548	388875.622880012\\
2.64886622165554	388879.633584578\\
2.6489662241556	388883.644289144\\
2.64906622665567	388887.65499371\\
2.64916622915573	388891.665698275\\
2.64926623165579	388895.676402841\\
2.64936623415585	388899.687107407\\
2.64946623665592	388903.697811973\\
2.64956623915598	388907.708516539\\
2.64966624165604	388911.719221105\\
2.6497662441561	388915.729925671\\
2.64986624665617	388919.740630237\\
2.64996624915623	388923.751334803\\
2.65006625165629	388927.762039369\\
2.65016625415635	388931.199786139\\
2.65026625665642	388935.210490705\\
2.65036625915648	388939.221195271\\
2.65046626165654	388942.658942042\\
2.6505662641566	388946.669646608\\
2.65066626665667	388950.107393379\\
2.65076626915673	388954.118097945\\
2.65086627165679	388957.555844716\\
2.65096627415685	388961.566549281\\
2.65106627665692	388965.004296052\\
2.65116627915698	388969.015000618\\
2.65126628165704	388972.452747389\\
2.6513662841571	388976.463451955\\
2.65146628665717	388979.901198726\\
2.65156628915723	388983.338945496\\
2.65166629165729	388986.776692267\\
2.65176629415735	388990.787396833\\
2.65186629665742	388994.225143604\\
2.65196629915748	388997.662890375\\
2.65206630165754	389001.100637145\\
2.6521663041576	389004.538383916\\
2.65226630665767	389007.976130687\\
2.65236630915773	389011.413877458\\
2.65246631165779	389014.851624229\\
2.65256631415785	389018.289370999\\
2.65266631665792	389021.72711777\\
2.65276631915798	389025.164864541\\
2.65286632165804	389028.602611312\\
2.6529663241581	389031.467400287\\
2.65306632665817	389034.905147058\\
2.65316632915823	389038.342893829\\
2.65326633165829	389041.7806406\\
2.65336633415835	389044.645429575\\
2.65346633665842	389048.083176346\\
2.65356633915848	389051.520923117\\
2.65366634165854	389054.385712093\\
2.6537663441586	389057.823458863\\
2.65386634665867	389060.688247839\\
2.65396634915873	389064.12599461\\
2.65406635165879	389066.990783586\\
2.65416635415885	389070.428530356\\
2.65426635665892	389073.293319332\\
2.65436635915898	389076.158108308\\
2.65446636165904	389079.595855078\\
2.6545663641591	389082.460644054\\
2.65466636665917	389085.32543303\\
2.65476636915923	389088.190222005\\
2.65486637165929	389091.055010981\\
2.65496637415935	389094.492757752\\
2.65506637665942	389097.357546727\\
2.65516637915948	389100.222335703\\
2.65526638165954	389103.087124679\\
2.6553663841596	389105.951913654\\
2.65546638665967	389108.81670263\\
2.65556638915973	389111.681491606\\
2.65566639165979	389114.546280581\\
2.65576639415985	389117.411069557\\
2.65586639665992	389119.702900738\\
2.65596639915998	389122.567689713\\
2.65606640166004	389125.432478689\\
2.6561664041601	389128.297267664\\
2.65626640666017	389130.589098845\\
2.65636640916023	389133.453887821\\
2.65646641166029	389136.318676796\\
2.65656641416035	389138.610507977\\
2.65666641666042	389141.475296953\\
2.65676641916048	389143.767128133\\
2.65686642166054	389146.631917109\\
2.6569664241606	389148.923748289\\
2.65706642666067	389151.788537265\\
2.65716642916073	389154.080368445\\
2.65726643166079	389156.945157421\\
2.65736643416085	389159.236988602\\
2.65746643666092	389161.528819782\\
2.65756643916098	389164.393608758\\
2.65766644166104	389166.685439938\\
2.6577664441611	389168.977271119\\
2.65786644666117	389171.269102299\\
2.65796644916123	389173.56093348\\
2.65806645166129	389176.425722455\\
2.65816645416135	389178.717553636\\
2.65826645666142	389181.009384817\\
2.65836645916148	389183.301215997\\
2.65846646166154	389185.593047178\\
2.6585664641616	389187.884878358\\
2.65866646666167	389190.176709539\\
2.65876646916173	389191.895582924\\
2.65886647166179	389194.187414105\\
2.65896647416185	389196.479245285\\
2.65906647666192	389198.771076466\\
2.65916647916198	389201.062907646\\
2.65926648166204	389202.781781032\\
2.6593664841621	389205.073612212\\
2.65946648666217	389207.365443393\\
2.65956648916223	389209.084316778\\
2.65966649166229	389211.376147959\\
2.65976649416235	389213.095021344\\
2.65986649666242	389215.386852524\\
2.65996649916248	389217.10572591\\
2.66006650166254	389219.39755709\\
2.6601665041626	389221.116430476\\
2.66026650666267	389223.408261656\\
2.66036650916273	389225.127135042\\
2.66046651166279	389226.846008427\\
2.66056651416285	389229.137839607\\
2.66066651666292	389230.856712993\\
2.66076651916298	389232.575586378\\
2.66086652166304	389234.294459764\\
2.6609665241631	389236.013333149\\
2.66106652666317	389237.732206534\\
2.66116652916323	389240.024037715\\
2.66126653166329	389241.7429111\\
2.66136653416335	389243.461784486\\
2.66146653666342	389245.180657871\\
2.66156653916348	389246.899531257\\
2.66166654166354	389248.045446847\\
2.6617665441636	389249.764320232\\
2.66186654666367	389251.483193618\\
2.66196654916373	389253.202067003\\
2.66206655166379	389254.920940388\\
2.66216655416385	389256.639813774\\
2.66226655666392	389257.785729364\\
2.66236655916398	389259.504602749\\
2.66246656166404	389261.223476135\\
2.6625665641641	389262.369391725\\
2.66266656666417	389264.088265111\\
2.66276656916423	389265.234180701\\
2.66286657166429	389266.953054086\\
2.66296657416435	389268.098969676\\
2.66306657666442	389269.817843062\\
2.66316657916448	389270.963758652\\
2.66326658166454	389272.682632037\\
2.6633665841646	389273.828547628\\
2.66346658666467	389274.974463218\\
2.66356658916473	389276.120378808\\
2.66366659166479	389277.839252194\\
2.66376659416485	389278.985167784\\
2.66386659666492	389280.131083374\\
2.66396659916498	389281.276998964\\
2.66406660166504	389282.422914555\\
2.6641666041651	389283.568830145\\
2.66426660666517	389285.28770353\\
2.66436660916523	389286.433619121\\
2.66446661166529	389287.579534711\\
2.66456661416535	389288.152492506\\
2.66466661666542	389289.298408096\\
2.66476661916548	389290.444323687\\
2.66486662166554	389291.590239277\\
2.6649666241656	389292.736154867\\
2.66506662666567	389293.882070457\\
2.66516662916573	389294.455028253\\
2.66526663166579	389295.600943843\\
2.66536663416585	389296.746859433\\
2.66546663666592	389297.892775023\\
2.66556663916598	389298.465732818\\
2.66566664166604	389299.611648409\\
2.6657666441661	389300.184606204\\
2.66586664666617	389301.330521794\\
2.66596664916623	389301.903479589\\
2.66606665166629	389303.049395179\\
2.66616665416635	389303.622352974\\
2.66626665666642	389304.768268565\\
2.66636665916648	389305.34122636\\
2.66646666166654	389305.914184155\\
2.6665666641666	389306.48714195\\
2.66666666666667	389307.63305754\\
2.66676666916673	389308.206015336\\
2.66686667166679	389308.778973131\\
2.66696667416685	389309.351930926\\
2.66706667666692	389309.924888721\\
2.66716667916698	389310.497846516\\
2.66726668166704	389311.643762106\\
2.6673666841671	389312.216719901\\
2.66746668666717	389312.789677697\\
2.66756668916723	389312.789677697\\
2.66766669166729	389313.362635492\\
2.66776669416735	389313.935593287\\
2.66786669666742	389314.508551082\\
2.66796669916748	389315.081508877\\
2.66806670166754	389315.654466672\\
2.6681667041676	389315.654466672\\
2.66826670666767	389316.227424467\\
2.66836670916773	389316.800382263\\
2.66846671166779	389317.373340058\\
2.66856671416785	389317.373340058\\
2.66866671666792	389317.946297853\\
2.66876671916798	389317.946297853\\
2.66886672166804	389318.519255648\\
2.6689667241681	389318.519255648\\
2.66906672666817	389319.092213443\\
2.66916672916823	389319.092213443\\
2.66926673166829	389319.665171238\\
2.66936673416835	389319.665171238\\
2.66946673666842	389319.665171238\\
2.66956673916848	389320.238129033\\
2.66966674166854	389320.238129033\\
2.6697667441686	389320.238129033\\
2.66986674666867	389320.238129033\\
2.66996674916873	389320.238129033\\
2.67006675166879	389320.238129033\\
2.67016675416885	389320.811086829\\
2.67026675666892	389320.811086829\\
2.67036675916898	389320.811086829\\
2.67046676166904	389320.811086829\\
2.6705667641691	389320.811086829\\
2.67066676666917	389320.238129033\\
2.67076676916923	389320.238129033\\
2.67086677166929	389320.238129033\\
2.67096677416935	389320.238129033\\
2.67106677666942	389320.238129033\\
2.67116677916948	389320.238129033\\
2.67126678166954	389319.665171238\\
2.6713667841696	389319.665171238\\
2.67146678666967	389319.665171238\\
2.67156678916973	389319.092213443\\
2.67166679166979	389319.092213443\\
2.67176679416985	389318.519255648\\
2.67186679666992	389318.519255648\\
2.67196679916998	389317.946297853\\
2.67206680167004	389317.946297853\\
2.6721668041701	389317.373340058\\
2.67226680667017	389317.373340058\\
2.67236680917023	389316.800382263\\
2.67246681167029	389316.227424467\\
2.67256681417035	389316.227424467\\
2.67266681667042	389315.654466672\\
2.67276681917048	389315.081508877\\
2.67286682167054	389314.508551082\\
2.6729668241706	389313.935593287\\
2.67306682667067	389313.362635492\\
2.67316682917073	389313.362635492\\
2.67326683167079	389312.789677697\\
2.67336683417085	389312.216719901\\
2.67346683667092	389311.643762106\\
2.67356683917098	389311.070804311\\
2.67366684167104	389309.924888721\\
2.6737668441711	389309.351930926\\
2.67386684667117	389308.778973131\\
2.67396684917123	389308.206015336\\
2.67406685167129	389307.63305754\\
2.67416685417135	389307.060099745\\
2.67426685667142	389305.914184155\\
2.67436685917148	389305.34122636\\
2.67446686167154	389304.768268565\\
2.6745668641716	389303.622352974\\
2.67466686667167	389303.049395179\\
2.67476686917173	389301.903479589\\
2.67486687167179	389301.330521794\\
2.67496687417185	389300.184606204\\
2.67506687667192	389299.611648409\\
2.67516687917198	389298.465732818\\
2.67526688167204	389297.892775023\\
2.6753668841721	389296.746859433\\
2.67546688667217	389295.600943843\\
2.67556688917223	389295.027986048\\
2.67566689167229	389293.882070457\\
2.67576689417235	389292.736154867\\
2.67586689667242	389291.590239277\\
2.67596689917248	389290.444323687\\
2.67606690167254	389289.871365891\\
2.6761669041726	389288.725450301\\
2.67626690667267	389287.579534711\\
2.67636690917273	389286.433619121\\
2.67646691167279	389285.28770353\\
2.67656691417285	389284.14178794\\
2.67666691667292	389282.99587235\\
2.67676691917298	389281.276998964\\
2.67686692167304	389280.131083374\\
2.6769669241731	389278.985167784\\
2.67706692667317	389277.839252194\\
2.67716692917323	389276.693336603\\
2.67726693167329	389274.974463218\\
2.67736693417335	389273.828547628\\
2.67746693667342	389272.682632037\\
2.67756693917348	389270.963758652\\
2.67766694167354	389269.817843062\\
2.6777669441736	389268.671927472\\
2.67786694667367	389266.953054086\\
2.67796694917373	389265.807138496\\
2.67806695167379	389264.088265111\\
2.67816695417385	389262.369391725\\
2.67826695667392	389261.223476135\\
2.67836695917398	389259.504602749\\
2.67846696167404	389257.785729364\\
2.6785669641741	389256.639813774\\
2.67866696667417	389254.920940388\\
2.67876696917423	389253.202067003\\
2.67886697167429	389251.483193618\\
2.67896697417435	389250.337278027\\
2.67906697667442	389248.618404642\\
2.67916697917448	389246.899531257\\
2.67926698167454	389245.180657871\\
2.6793669841746	389243.461784486\\
2.67946698667467	389241.7429111\\
2.67956698917473	389240.024037715\\
2.67966699167479	389238.30516433\\
2.67976699417485	389236.586290944\\
2.67986699667492	389234.294459764\\
2.67996699917498	389232.575586378\\
2.68006700167504	389230.856712993\\
2.6801670041751	389229.137839607\\
2.68026700667517	389227.418966222\\
2.68036700917523	389225.127135042\\
2.68046701167529	389223.408261656\\
2.68056701417535	389221.689388271\\
2.68066701667542	389219.39755709\\
2.68076701917548	389217.678683705\\
2.68086702167554	389215.386852524\\
2.6809670241756	389213.667979139\\
2.68106702667567	389211.376147959\\
2.68116702917573	389209.657274573\\
2.68126703167579	389207.365443393\\
2.68136703417585	389205.073612212\\
2.68146703667592	389203.354738827\\
2.68156703917598	389201.062907646\\
2.68166704167604	389198.771076466\\
2.6817670441761	389196.479245285\\
2.68186704667617	389194.7603719\\
2.68196704917623	389192.468540719\\
2.68206705167629	389190.176709539\\
2.68216705417635	389187.884878358\\
2.68226705667642	389185.593047178\\
2.68236705917648	389183.301215997\\
2.68246706167654	389181.009384817\\
2.6825670641766	389178.717553636\\
2.68266706667667	389176.425722455\\
2.68276706917673	389174.133891275\\
2.68286707167679	389171.842060094\\
2.68296707417685	389168.977271119\\
2.68306707667692	389166.685439938\\
2.68316707917698	389164.393608758\\
2.68326708167704	389162.101777577\\
2.6833670841771	389159.236988602\\
2.68346708667717	389156.945157421\\
2.68356708917723	389154.65332624\\
2.68366709167729	389151.788537265\\
2.68376709417735	389149.496706084\\
2.68386709667742	389146.631917109\\
2.68396709917748	389144.340085928\\
2.68406710167754	389141.475296953\\
2.6841671041776	389139.183465772\\
2.68426710667767	389136.318676796\\
2.68436710917773	389134.026845616\\
2.68446711167779	389131.16205664\\
2.68456711417785	389128.297267664\\
2.68466711667792	389125.432478689\\
2.68476711917798	389123.140647508\\
2.68486712167804	389120.275858533\\
2.6849671241781	389117.411069557\\
2.68506712667817	389114.546280581\\
2.68516712917823	389111.681491606\\
2.68526713167829	389108.81670263\\
2.68536713417835	389105.951913654\\
2.68546713667842	389103.087124679\\
2.68556713917848	389100.222335703\\
2.68566714167854	389097.357546727\\
2.6857671441786	389094.492757752\\
2.68586714667867	389091.627968776\\
2.68596714917873	389088.7631798\\
2.68606715167879	389085.898390825\\
2.68616715417885	389082.460644054\\
2.68626715667892	389079.595855078\\
2.68636715917898	389076.731066103\\
2.68646716167904	389073.293319332\\
2.6865671641791	389070.428530356\\
2.68666716667917	389067.563741381\\
2.68676716917923	389064.12599461\\
2.68686717167929	389061.261205634\\
2.68696717417935	389057.823458863\\
2.68706717667942	389054.958669888\\
2.68716717917948	389051.520923117\\
2.68726718167954	389048.083176346\\
2.6873671841796	389045.218387371\\
2.68746718667967	389041.7806406\\
2.68756718917973	389038.342893829\\
2.68766719167979	389035.478104853\\
2.68776719417985	389032.040358083\\
2.68786719667992	389028.602611312\\
2.68796719917998	389025.164864541\\
2.68806720168004	389021.72711777\\
2.6881672041801	389018.289370999\\
2.68826720668017	389015.424582024\\
2.68836720918023	389011.986835253\\
2.68846721168029	389008.549088482\\
2.68856721418035	389005.111341711\\
2.68866721668042	389001.100637145\\
2.68876721918048	388997.662890375\\
2.68886722168054	388994.225143604\\
2.6889672241806	388990.787396833\\
2.68906722668067	388987.349650062\\
2.68916722918073	388983.911903292\\
2.68926723168079	388979.901198726\\
2.68936723418085	388976.463451955\\
2.68946723668092	388973.025705184\\
2.68956723918098	388969.015000618\\
2.68966724168104	388965.577253847\\
2.6897672441811	388961.566549281\\
2.68986724668117	388958.128802511\\
2.68996724918123	388954.69105574\\
2.69006725168129	388950.680351174\\
2.69016725418135	388946.669646608\\
2.69026725668142	388943.231899837\\
2.69036725918148	388939.221195271\\
2.69046726168154	388935.783448501\\
2.6905672641816	388931.772743935\\
2.69066726668167	388927.762039369\\
2.69076726918173	388923.751334803\\
2.69086727168179	388920.313588032\\
2.69096727418185	388916.302883466\\
2.69106727668192	388912.2921789\\
2.69116727918198	388908.281474334\\
2.69126728168204	388904.270769768\\
2.6913672841821	388900.260065202\\
2.69146728668217	388896.249360637\\
2.69156728918223	388892.238656071\\
2.69166729168229	388888.227951505\\
2.69176729418235	388884.217246939\\
2.69186729668242	388880.206542373\\
2.69196729918248	388876.195837807\\
2.69206730168254	388871.612175446\\
2.6921673041826	388867.60147088\\
2.69226730668267	388863.590766314\\
2.69236730918273	388859.580061748\\
2.69246731168279	388854.996399387\\
2.69256731418285	388850.985694821\\
2.69266731668292	388846.40203246\\
2.69276731918298	388842.391327894\\
2.69286732168304	388838.380623328\\
2.6929673241831	388833.796960967\\
2.69306732668317	388829.213298606\\
2.69316732918323	388825.20259404\\
2.69326733168329	388820.618931679\\
2.69336733418335	388816.608227113\\
2.69346733668342	388812.024564752\\
2.69356733918348	388807.440902391\\
2.69366734168354	388803.430197825\\
2.6937673441836	388798.846535464\\
2.69386734668367	388794.262873103\\
2.69396734918373	388789.679210742\\
2.69406735168379	388785.095548381\\
2.69416735418385	388780.51188602\\
2.69426735668392	388776.501181454\\
2.69436735918398	388771.917519093\\
2.69446736168404	388767.333856732\\
2.6945673641841	388762.750194371\\
2.69466736668417	388757.593574215\\
2.69476736918423	388753.009911854\\
2.69486737168429	388748.426249493\\
2.69496737418435	388743.842587132\\
2.69506737668442	388739.258924771\\
2.69516737918448	388734.67526241\\
2.69526738168454	388729.518642253\\
2.6953673841846	388724.934979892\\
2.69546738668467	388720.351317531\\
2.69556738918473	388715.194697375\\
2.69566739168479	388710.611035014\\
2.69576739418485	388706.027372653\\
2.69586739668492	388700.870752497\\
2.69596739918498	388696.287090136\\
2.69606740168504	388691.13046998\\
2.6961674041851	388686.546807619\\
2.69626740668517	388681.390187463\\
2.69636740918523	388676.233567306\\
2.69646741168529	388671.649904945\\
2.69656741418535	388666.493284789\\
2.69666741668542	388661.336664633\\
2.69676741918548	388656.753002272\\
2.69686742168554	388651.596382116\\
2.6969674241856	388646.439761959\\
2.69706742668567	388641.283141803\\
2.69716742918573	388636.126521647\\
2.69726743168579	388630.969901491\\
2.69736743418585	388625.813281335\\
2.69746743668592	388620.656661179\\
2.69756743918598	388615.500041022\\
2.69766744168604	388610.343420866\\
2.6977674441861	388605.18680071\\
2.69786744668617	388600.030180554\\
2.69796744918623	388594.873560398\\
2.69806745168629	388589.716940242\\
2.69816745418635	388583.98736229\\
2.69826745668642	388578.830742134\\
2.69836745918648	388573.674121978\\
2.69846746168654	388568.517501822\\
2.6985674641866	388562.78792387\\
2.69866746668667	388557.631303714\\
2.69876746918673	388551.901725763\\
2.69886747168679	388546.745105607\\
2.69896747418685	388541.588485451\\
2.69906747668692	388535.858907499\\
2.69916747918698	388530.129329548\\
2.69926748168704	388524.972709392\\
2.6993674841871	388519.24313144\\
2.69946748668717	388514.086511284\\
2.69956748918723	388508.356933333\\
2.69966749168729	388502.627355382\\
2.69976749418735	388496.89777743\\
2.69986749668742	388491.741157274\\
2.69996749918748	388486.011579323\\
2.70006750168754	388480.282001372\\
2.7001675041876	388474.55242342\\
2.70026750668767	388468.822845469\\
2.70036750918773	388463.093267518\\
2.70046751168779	388457.363689566\\
2.70056751418785	388451.634111615\\
2.70066751668792	388445.904533664\\
2.70076751918798	388440.174955712\\
2.70086752168804	388434.445377761\\
2.7009675241881	388428.71579981\\
2.70106752668817	388422.986221858\\
2.70116752918823	388417.256643907\\
2.70126753168829	388410.954108161\\
2.70136753418835	388405.224530209\\
2.70146753668842	388399.494952258\\
2.70156753918848	388393.192416512\\
2.70166754168854	388387.46283856\\
2.7017675441886	388381.733260609\\
2.70186754668867	388375.430724863\\
2.70196754918873	388369.701146911\\
2.70206755168879	388363.398611165\\
2.70216755418885	388357.669033214\\
2.70226755668892	388351.366497467\\
2.70236755918898	388345.063961721\\
2.70246756168904	388339.334383769\\
2.7025675641891	388333.031848023\\
2.70266756668917	388326.729312276\\
2.70276756918923	388320.999734325\\
2.70286757168929	388314.697198579\\
2.70296757418935	388308.394662832\\
2.70306757668942	388302.092127086\\
2.70316757918948	388295.789591339\\
2.70326758168954	388289.487055593\\
2.7033675841896	388283.184519847\\
2.70346758668967	388277.454941895\\
2.70356758918973	388271.152406149\\
2.70366759168979	388264.276912607\\
2.70376759418985	388257.974376861\\
2.70386759668992	388251.671841114\\
2.70396759918998	388245.369305368\\
2.70406760169004	388239.066769621\\
2.7041676041901	388232.764233875\\
2.70426760669017	388226.461698129\\
2.70436760919023	388219.586204587\\
2.70446761169029	388213.283668841\\
2.70456761419035	388206.981133094\\
2.70466761669042	388200.105639553\\
2.70476761919048	388193.803103806\\
2.70486762169054	388186.927610265\\
2.7049676241906	388180.625074518\\
2.70506762669067	388173.749580977\\
2.70516762919073	388167.44704523\\
2.70526763169079	388160.571551689\\
2.70536763419085	388154.269015942\\
2.70546763669092	388147.393522401\\
2.70556763919098	388140.518028859\\
2.70566764169104	388134.215493113\\
2.7057676441911	388127.339999571\\
2.70586764669117	388120.464506029\\
2.70596764919123	388114.161970283\\
2.70606765169129	388107.286476741\\
2.70616765419135	388100.4109832\\
2.70626765669142	388093.535489658\\
2.70636765919148	388086.659996117\\
2.70646766169154	388079.784502575\\
2.7065676641916	388072.909009034\\
2.70666766669167	388066.033515492\\
2.70676766919173	388059.15802195\\
2.70686767169179	388052.282528409\\
2.70696767419185	388045.407034867\\
2.70706767669192	388038.531541326\\
2.70716767919198	388031.083089989\\
2.70726768169204	388024.207596447\\
2.7073676841921	388017.332102906\\
2.70746768669217	388010.456609364\\
2.70756768919223	388003.008158028\\
2.70766769169229	387996.132664486\\
2.70776769419235	387989.257170944\\
2.70786769669242	387981.808719608\\
2.70796769919248	387974.933226066\\
2.70806770169254	387967.484774729\\
2.7081677041926	387960.609281188\\
2.70826770669267	387953.160829851\\
2.70836770919273	387946.28533631\\
2.70846771169279	387938.836884973\\
2.70856771419286	387931.388433636\\
2.70866771669292	387924.512940095\\
2.70876771919298	387917.064488758\\
2.70886772169304	387909.616037421\\
2.7089677241931	387902.167586085\\
2.70906772669317	387895.292092543\\
2.70916772919323	387887.843641206\\
2.70926773169329	387880.39518987\\
2.70936773419335	387872.946738533\\
2.70946773669342	387865.498287196\\
2.70956773919348	387858.04983586\\
2.70966774169354	387850.601384523\\
2.7097677441936	387843.152933186\\
2.70986774669367	387835.704481849\\
2.70996774919373	387828.256030513\\
2.71006775169379	387820.807579176\\
2.71016775419385	387812.786170044\\
2.71026775669392	387805.337718707\\
2.71036775919398	387797.889267371\\
2.71046776169404	387790.440816034\\
2.71056776419411	387782.419406902\\
2.71066776669417	387774.970955566\\
2.71076776919423	387767.522504229\\
2.71086777169429	387759.501095097\\
2.71096777419435	387752.05264376\\
2.71106777669442	387744.031234628\\
2.71116777919448	387736.582783292\\
2.71126778169454	387728.56137416\\
2.7113677841946	387721.112922823\\
2.71146778669467	387713.091513691\\
2.71156778919473	387705.07010456\\
2.71166779169479	387697.621653223\\
2.71176779419485	387689.600244091\\
2.71186779669492	387681.578834959\\
2.71196779919498	387674.130383622\\
2.71206780169504	387666.108974491\\
2.7121678041951	387658.087565359\\
2.71226780669517	387650.066156227\\
2.71236780919523	387642.044747095\\
2.71246781169529	387634.023337963\\
2.71256781419536	387626.001928832\\
2.71266781669542	387617.9805197\\
2.71276781919548	387609.959110568\\
2.71286782169554	387601.937701436\\
2.71296782419561	387593.916292304\\
2.71306782669567	387585.894883172\\
2.71316782919573	387577.873474041\\
2.71326783169579	387569.852064909\\
2.71336783419585	387561.257697982\\
2.71346783669592	387553.23628885\\
2.71356783919598	387545.214879718\\
2.71366784169604	387536.620512791\\
2.7137678441961	387528.599103659\\
2.71386784669617	387520.577694527\\
2.71396784919623	387511.9833276\\
2.71406785169629	387503.961918469\\
2.71416785419635	387495.367551542\\
2.71426785669642	387487.34614241\\
2.71436785919648	387478.751775483\\
2.71446786169654	387470.730366351\\
2.71456786419661	387462.135999424\\
2.71466786669667	387453.541632497\\
2.71476786919673	387445.520223365\\
2.71486787169679	387436.925856438\\
2.71496787419686	387428.331489511\\
2.71506787669692	387419.737122584\\
2.71516787919698	387411.715713453\\
2.71526788169704	387403.121346526\\
2.7153678841971	387394.526979599\\
2.71546788669717	387385.932612672\\
2.71556788919723	387377.338245745\\
2.71566789169729	387368.743878818\\
2.71576789419735	387360.149511891\\
2.71586789669742	387351.555144964\\
2.71596789919748	387342.960778037\\
2.71606790169754	387334.36641111\\
2.7161679041976	387325.772044183\\
2.71626790669767	387316.604719461\\
2.71636790919773	387308.010352534\\
2.71646791169779	387299.415985607\\
2.71656791419786	387290.82161868\\
2.71666791669792	387281.654293958\\
2.71676791919798	387273.059927031\\
2.71686792169804	387264.465560104\\
2.71696792419811	387255.298235382\\
2.71706792669817	387246.703868455\\
2.71716792919823	387237.536543733\\
2.71726793169829	387228.942176806\\
2.71736793419835	387219.774852084\\
2.71746793669842	387211.180485157\\
2.71756793919848	387202.013160435\\
2.71766794169854	387193.418793508\\
2.7177679441986	387184.251468786\\
2.71786794669867	387175.084144064\\
2.71796794919873	387165.916819341\\
2.71806795169879	387157.322452414\\
2.71816795419885	387148.155127692\\
2.71826795669892	387138.98780297\\
2.71836795919898	387129.820478248\\
2.71846796169904	387120.653153526\\
2.71856796419911	387111.485828804\\
2.71866796669917	387102.318504082\\
2.71876796919923	387093.15117936\\
2.71886797169929	387083.983854638\\
2.71896797419936	387074.816529916\\
2.71906797669942	387065.649205194\\
2.71916797919948	387056.481880471\\
2.71926798169954	387047.314555749\\
2.71936798419961	387038.147231027\\
2.71946798669967	387028.40694851\\
2.71956798919973	387019.239623788\\
2.71966799169979	387010.072299066\\
2.71976799419985	387000.904974344\\
2.71986799669992	386991.164691827\\
2.71996799919998	386981.997367104\\
2.72006800170004	386972.257084587\\
2.7201680042001	386963.089759865\\
2.72026800670017	386953.349477348\\
2.72036800920023	386944.182152626\\
2.72046801170029	386934.441870109\\
2.72056801420036	386925.274545387\\
2.72066801670042	386915.534262869\\
2.72076801920048	386906.366938147\\
2.72086802170054	386896.62665563\\
2.72096802420061	386886.886373113\\
2.72106802670067	386877.146090595\\
2.72116802920073	386867.978765873\\
2.72126803170079	386858.238483356\\
2.72136803420086	386848.498200839\\
2.72146803670092	386838.757918322\\
2.72156803920098	386829.017635804\\
2.72166804170104	386819.277353287\\
2.7217680442011	386809.53707077\\
2.72186804670117	386799.796788253\\
2.72196804920123	386790.056505736\\
2.72206805170129	386780.316223218\\
2.72216805420135	386770.575940701\\
2.72226805670142	386760.835658184\\
2.72236805920148	386751.095375667\\
2.72246806170154	386741.35509315\\
2.72256806420161	386731.041852837\\
2.72266806670167	386721.30157032\\
2.72276806920173	386711.561287803\\
2.72286807170179	386701.24804749\\
2.72296807420186	386691.507764973\\
2.72306807670192	386681.767482456\\
2.72316807920198	386671.454242144\\
2.72326808170204	386661.713959626\\
2.72336808420211	386651.400719314\\
2.72346808670217	386641.660436797\\
2.72356808920223	386631.347196484\\
2.72366809170229	386621.606913967\\
2.72376809420235	386611.293673655\\
2.72386809670242	386600.980433342\\
2.72396809920248	386591.240150825\\
2.72406810170254	386580.926910513\\
2.7241681042026	386570.613670201\\
2.72426810670267	386560.300429888\\
2.72436810920273	386550.560147371\\
2.72446811170279	386540.246907059\\
2.72456811420286	386529.933666746\\
2.72466811670292	386519.620426434\\
2.72476811920298	386509.307186122\\
2.72486812170304	386498.993945809\\
2.72496812420311	386488.680705497\\
2.72506812670317	386478.367465184\\
2.72516812920323	386468.054224872\\
2.72526813170329	386457.74098456\\
2.72536813420336	386447.427744247\\
2.72546813670342	386436.54154614\\
2.72556813920348	386426.228305828\\
2.72566814170354	386415.915065515\\
2.72576814420361	386405.601825203\\
2.72586814670367	386394.715627095\\
2.72596814920373	386384.402386783\\
2.72606815170379	386374.089146471\\
2.72616815420385	386363.202948363\\
2.72626815670392	386352.889708051\\
2.72636815920398	386342.003509943\\
2.72646816170404	386331.690269631\\
2.72656816420411	386320.804071523\\
2.72666816670417	386310.490831211\\
2.72676816920423	386299.604633104\\
2.72686817170429	386288.718434996\\
2.72696817420436	386278.405194684\\
2.72706817670442	386267.518996576\\
2.72716817920448	386256.632798469\\
2.72726818170454	386246.319558156\\
2.72736818420461	386235.433360049\\
2.72746818670467	386224.547161941\\
2.72756818920473	386213.660963834\\
2.72766819170479	386202.774765727\\
2.72776819420486	386191.888567619\\
2.72786819670492	386181.002369512\\
2.72796819920498	386170.116171404\\
2.72806820170504	386159.229973297\\
2.7281682042051	386148.343775189\\
2.72826820670517	386137.457577082\\
2.72836820920523	386126.571378974\\
2.72846821170529	386115.685180867\\
2.72856821420536	386104.798982759\\
2.72866821670542	386093.912784652\\
2.72876821920548	386082.453628749\\
2.72886822170554	386071.567430642\\
2.72896822420561	386060.681232534\\
2.72906822670567	386049.222076631\\
2.72916822920573	386038.335878524\\
2.72926823170579	386027.449680416\\
2.72936823420586	386015.990524514\\
2.72946823670592	386005.104326406\\
2.72956823920598	385993.645170504\\
2.72966824170604	385982.758972396\\
2.72976824420611	385971.299816494\\
2.72986824670617	385959.840660591\\
2.72996824920623	385948.954462484\\
2.73006825170629	385937.495306581\\
2.73016825420636	385926.036150678\\
2.73026825670642	385915.149952571\\
2.73036825920648	385903.690796668\\
2.73046826170654	385892.231640766\\
2.73056826420661	385880.772484863\\
2.73066826670667	385869.31332896\\
2.73076826920673	385858.427130853\\
2.73086827170679	385846.96797495\\
2.73096827420686	385835.508819048\\
2.73106827670692	385824.049663145\\
2.73116827920698	385812.590507242\\
2.73126828170704	385801.13135134\\
2.73136828420711	385789.672195437\\
2.73146828670717	385777.640081739\\
2.73156828920723	385766.180925837\\
2.73166829170729	385754.721769934\\
2.73176829420736	385743.262614032\\
2.73186829670742	385731.803458129\\
2.73196829920748	385719.771344431\\
2.73206830170754	385708.312188529\\
2.73216830420761	385696.853032626\\
2.73226830670767	385684.820918928\\
2.73236830920773	385673.361763026\\
2.73246831170779	385661.329649328\\
2.73256831420786	385649.870493425\\
2.73266831670792	385637.838379728\\
2.73276831920798	385626.379223825\\
2.73286832170804	385614.347110127\\
2.73296832420811	385602.887954225\\
2.73306832670817	385590.855840527\\
2.73316832920823	385578.823726829\\
2.73326833170829	385567.364570926\\
2.73336833420836	385555.332457229\\
2.73346833670842	385543.300343531\\
2.73356833920848	385531.268229833\\
2.73366834170854	385519.809073931\\
2.73376834420861	385507.776960233\\
2.73386834670867	385495.744846535\\
2.73396834920873	385483.712732837\\
2.73406835170879	385471.68061914\\
2.73416835420886	385459.648505442\\
2.73426835670892	385447.616391744\\
2.73436835920898	385435.584278046\\
2.73446836170904	385423.552164349\\
2.73456836420911	385411.520050651\\
2.73466836670917	385399.487936953\\
2.73476836920923	385386.88286546\\
2.73486837170929	385374.850751763\\
2.73496837420936	385362.818638065\\
2.73506837670942	385350.786524367\\
2.73516837920948	385338.181452874\\
2.73526838170954	385326.149339176\\
2.73536838420961	385314.117225479\\
2.73546838670967	385301.512153986\\
2.73556838920973	385289.480040288\\
2.73566839170979	385276.874968795\\
2.73576839420986	385264.842855097\\
2.73586839670992	385252.237783605\\
2.73596839920998	385240.205669907\\
2.73606840171004	385227.600598414\\
2.73616840421011	385214.995526921\\
2.73626840671017	385202.963413223\\
2.73636840921023	385190.35834173\\
2.73646841171029	385177.753270238\\
2.73656841421036	385165.72115654\\
2.73666841671042	385153.116085047\\
2.73676841921048	385140.511013554\\
2.73686842171054	385127.905942061\\
2.73696842421061	385115.300870568\\
2.73706842671067	385102.695799075\\
2.73716842921073	385090.090727582\\
2.73726843171079	385077.48565609\\
2.73736843421086	385064.880584597\\
2.73746843671092	385052.275513104\\
2.73756843921098	385039.670441611\\
2.73766844171104	385027.065370118\\
2.73776844421111	385014.460298625\\
2.73786844671117	385001.855227132\\
2.73796844921123	384989.250155639\\
2.73806845171129	384976.072126351\\
2.73816845421136	384963.467054859\\
2.73826845671142	384950.861983366\\
2.73836845921148	384937.683954078\\
2.73846846171154	384925.078882585\\
2.73856846421161	384912.473811092\\
2.73866846671167	384899.295781804\\
2.73876846921173	384886.690710311\\
2.73886847171179	384873.512681023\\
2.73896847421186	384860.90760953\\
2.73906847671192	384847.729580242\\
2.73916847921198	384835.124508749\\
2.73926848171204	384821.946479461\\
2.73936848421211	384808.768450173\\
2.73946848671217	384796.16337868\\
2.73956848921223	384782.985349392\\
2.73966849171229	384769.807320104\\
2.73976849421236	384756.629290816\\
2.73986849671242	384743.451261528\\
2.73996849921248	384730.846190035\\
2.74006850171254	384717.668160747\\
2.74016850421261	384704.490131459\\
2.74026850671267	384691.312102171\\
2.74036850921273	384678.134072883\\
2.74046851171279	384664.956043595\\
2.74056851421286	384651.778014307\\
2.74066851671292	384638.599985019\\
2.74076851921298	384625.421955731\\
2.74086852171304	384612.243926443\\
2.74096852421311	384598.49293936\\
2.74106852671317	384585.314910072\\
2.74116852921323	384572.136880784\\
2.74126853171329	384558.958851496\\
2.74136853421336	384545.207864413\\
2.74146853671342	384532.029835125\\
2.74156853921348	384518.851805837\\
2.74166854171354	384505.100818754\\
2.74176854421361	384491.922789466\\
2.74186854671367	384478.171802383\\
2.74196854921373	384464.993773095\\
2.74206855171379	384451.242786012\\
2.74216855421386	384438.064756724\\
2.74226855671392	384424.31376964\\
2.74236855921398	384411.135740352\\
2.74246856171404	384397.384753269\\
2.74256856421411	384383.633766186\\
2.74266856671417	384370.455736898\\
2.74276856921423	384356.704749815\\
2.74286857171429	384342.953762732\\
2.74296857421436	384329.202775649\\
2.74306857671442	384315.451788566\\
2.74316857921448	384302.273759278\\
2.74326858171454	384288.522772194\\
2.74336858421461	384274.771785111\\
2.74346858671467	384261.020798028\\
2.74356858921473	384247.269810945\\
2.74366859171479	384233.518823862\\
2.74376859421486	384219.767836779\\
2.74386859671492	384206.016849696\\
2.74396859921498	384191.692904817\\
2.74406860171504	384177.941917734\\
2.74416860421511	384164.190930651\\
2.74426860671517	384150.439943568\\
2.74436860921523	384136.688956485\\
2.74446861171529	384122.365011606\\
2.74456861421536	384108.614024523\\
2.74466861671542	384094.86303744\\
2.74476861921548	384080.539092562\\
2.74486862171554	384066.788105479\\
2.74496862421561	384052.464160601\\
2.74506862671567	384038.713173517\\
2.74516862921573	384024.389228639\\
2.74526863171579	384010.638241556\\
2.74536863421586	383996.314296678\\
2.74546863671592	383982.563309595\\
2.74556863921598	383968.239364716\\
2.74566864171604	383953.915419838\\
2.74576864421611	383940.164432755\\
2.74586864671617	383925.840487877\\
2.74596864921623	383911.516542998\\
2.74606865171629	383897.19259812\\
2.74616865421636	383883.441611037\\
2.74626865671642	383869.117666159\\
2.74636865921648	383854.79372128\\
2.74646866171654	383840.469776402\\
2.74656866421661	383826.145831524\\
2.74666866671667	383811.821886646\\
2.74676866921673	383797.497941767\\
2.74686867171679	383783.173996889\\
2.74696867421686	383768.850052011\\
2.74706867671692	383754.526107132\\
2.74716867921698	383740.202162254\\
2.74726868171704	383725.305259581\\
2.74736868421711	383710.981314703\\
2.74746868671717	383696.657369824\\
2.74756868921723	383682.333424946\\
2.74766869171729	383667.436522273\\
2.74776869421736	383653.112577394\\
2.74786869671742	383638.788632516\\
2.74796869921748	383623.891729843\\
2.74806870171754	383609.567784964\\
2.74816870421761	383594.670882291\\
2.74826870671767	383580.346937413\\
2.74836870921773	383565.450034739\\
2.74846871171779	383551.126089861\\
2.74856871421786	383536.229187188\\
2.74866871671792	383521.905242309\\
2.74876871921798	383507.008339636\\
2.74886872171804	383492.111436963\\
2.74896872421811	383477.787492084\\
2.74906872671817	383462.890589411\\
2.74916872921823	383447.993686738\\
2.74926873171829	383433.096784064\\
2.74936873421836	383418.199881391\\
2.74946873671842	383403.875936512\\
2.74956873921848	383388.979033839\\
2.74966874171854	383374.082131166\\
2.74976874421861	383359.185228492\\
2.74986874671867	383344.288325819\\
2.74996874921873	383329.391423145\\
2.75006875171879	383314.494520472\\
2.75016875421886	383299.597617799\\
2.75026875671892	383284.12775733\\
2.75036875921898	383269.230854657\\
2.75046876171904	383254.333951983\\
2.75056876421911	383239.43704931\\
2.75066876671917	383224.540146636\\
2.75076876921923	383209.070286168\\
2.75086877171929	383194.173383495\\
2.75096877421936	383179.276480821\\
2.75106877671942	383163.806620353\\
2.75116877921948	383148.909717679\\
2.75126878171954	383133.439857211\\
2.75136878421961	383118.542954537\\
2.75146878671967	383103.646051864\\
2.75156878921973	383088.176191395\\
2.75166879171979	383072.706330927\\
2.75176879421986	383057.809428253\\
2.75186879671992	383042.339567785\\
2.75196879921998	383027.442665112\\
2.75206880172004	383011.972804643\\
2.75216880422011	382996.502944174\\
2.75226880672017	382981.033083706\\
2.75236880922023	382966.136181033\\
2.75246881172029	382950.666320564\\
2.75256881422036	382935.196460095\\
2.75266881672042	382919.726599627\\
2.75276881922048	382904.256739158\\
2.75286882172054	382888.78687869\\
2.75296882422061	382873.317018221\\
2.75306882672067	382857.847157753\\
2.75316882922073	382842.377297284\\
2.75326883172079	382826.907436816\\
2.75336883422086	382811.437576347\\
2.75346883672092	382795.967715879\\
2.75356883922098	382780.49785541\\
2.75366884172104	382765.027994942\\
2.75376884422111	382748.985176678\\
2.75386884672117	382733.515316209\\
2.75396884922123	382718.045455741\\
2.75406885172129	382702.002637477\\
2.75416885422136	382686.532777009\\
2.75426885672142	382671.06291654\\
2.75436885922148	382655.020098276\\
2.75446886172154	382639.550237808\\
2.75456886422161	382623.507419544\\
2.75466886672167	382608.037559076\\
2.75476886922173	382591.994740812\\
2.75486887172179	382576.524880344\\
2.75496887422186	382560.48206208\\
2.75506887672192	382544.439243816\\
2.75516887922198	382528.969383348\\
2.75526888172204	382512.926565084\\
2.75536888422211	382496.88374682\\
2.75546888672217	382481.413886352\\
2.75556888922223	382465.371068088\\
2.75566889172229	382449.328249825\\
2.75576889422236	382433.285431561\\
2.75586889672242	382417.242613297\\
2.75596889922248	382401.199795034\\
2.75606890172254	382385.15697677\\
2.75616890422261	382369.114158506\\
2.75626890672267	382353.071340242\\
2.75636890922273	382337.028521979\\
2.75646891172279	382320.985703715\\
2.75656891422286	382304.942885451\\
2.75666891672292	382288.900067188\\
2.75676891922298	382272.857248924\\
2.75686892172304	382256.814430661\\
2.75696892422311	382240.771612397\\
2.75706892672317	382224.155836338\\
2.75716892922323	382208.113018074\\
2.75726893172329	382192.070199811\\
2.75736893422336	382175.454423752\\
2.75746893672342	382159.411605488\\
2.75756893922348	382142.79582943\\
2.75766894172354	382126.753011166\\
2.75776894422361	382110.710192902\\
2.75786894672367	382094.094416843\\
2.75796894922373	382078.05159858\\
2.75806895172379	382061.435822521\\
2.75816895422386	382044.820046462\\
2.75826895672392	382028.777228198\\
2.75836895922398	382012.16145214\\
2.75846896172404	381995.545676081\\
2.75856896422411	381979.502857817\\
2.75866896672417	381962.887081758\\
2.75876896922423	381946.2713057\\
2.75886897172429	381929.655529641\\
2.75896897422436	381913.039753582\\
2.75906897672442	381896.996935318\\
2.75916897922448	381880.38115926\\
2.75926898172454	381863.765383201\\
2.75936898422461	381847.149607142\\
2.75946898672467	381830.533831083\\
2.75956898922473	381813.918055024\\
2.75966899172479	381797.302278966\\
2.75976899422486	381780.113545112\\
2.75986899672492	381763.497769053\\
2.75996899922498	381746.881992994\\
2.76006900172504	381730.266216935\\
2.76016900422511	381713.650440877\\
2.76026900672517	381696.461707023\\
2.76036900922523	381679.845930964\\
2.76046901172529	381663.230154905\\
2.76056901422536	381646.614378846\\
2.76066901672542	381629.425644992\\
2.76076901922548	381612.809868934\\
2.76086902172554	381595.62113508\\
2.76096902422561	381579.005359021\\
2.76106902672567	381561.816625167\\
2.76116902922573	381545.200849108\\
2.76126903172579	381528.012115254\\
2.76136903422586	381511.396339195\\
2.76146903672592	381494.207605341\\
2.76156903922598	381477.018871487\\
2.76166904172604	381460.403095429\\
2.76176904422611	381443.214361575\\
2.76186904672617	381426.025627721\\
2.76196904922623	381408.836893867\\
2.76206905172629	381392.221117808\\
2.76216905422636	381375.032383954\\
2.76226905672642	381357.8436501\\
2.76236905922648	381340.654916246\\
2.76246906172654	381323.466182392\\
2.76256906422661	381306.277448539\\
2.76266906672667	381289.088714685\\
2.76276906922673	381271.899980831\\
2.76286907172679	381254.711246977\\
2.76296907422686	381237.522513123\\
2.76306907672692	381220.333779269\\
2.76316907922698	381203.145045415\\
2.76326908172704	381185.383353766\\
2.76336908422711	381168.194619912\\
2.76346908672717	381151.005886058\\
2.76356908922723	381133.817152204\\
2.76366909172729	381116.055460555\\
2.76376909422736	381098.866726701\\
2.76386909672742	381081.677992847\\
2.76396909922748	381063.916301198\\
2.76406910172754	381046.727567344\\
2.76416910422761	381028.965875695\\
2.76426910672767	381011.777141841\\
2.76436910922773	380994.015450192\\
2.76446911172779	380976.826716338\\
2.76456911422786	380959.065024689\\
2.76466911672792	380941.30333304\\
2.76476911922798	380924.114599186\\
2.76486912172804	380906.352907537\\
2.76496912422811	380888.591215888\\
2.76506912672817	380871.402482034\\
2.76516912922823	380853.640790385\\
2.76526913172829	380835.879098736\\
2.76536913422836	380818.117407087\\
2.76546913672842	380800.355715438\\
2.76556913922848	380782.594023789\\
2.76566914172854	380765.405289935\\
2.76576914422861	380747.643598286\\
2.76586914672867	380729.881906637\\
2.76596914922873	380712.120214988\\
2.76606915172879	380693.785565544\\
2.76616915422886	380676.023873895\\
2.76626915672892	380658.262182246\\
2.76636915922898	380640.500490597\\
2.76646916172904	380622.738798947\\
2.76656916422911	380604.977107298\\
2.76666916672917	380587.215415649\\
2.76676916922923	380568.880766205\\
2.76686917172929	380551.119074556\\
2.76696917422936	380533.357382907\\
2.76706917672942	380515.022733463\\
2.76716917922948	380497.261041814\\
2.76726918172954	380478.92639237\\
2.76736918422961	380461.164700721\\
2.76746918672967	380443.403009071\\
2.76756918922973	380425.068359627\\
2.76766919172979	380406.733710183\\
2.76776919422986	380388.972018534\\
2.76786919672992	380370.63736909\\
2.76796919922998	380352.875677441\\
2.76806920173004	380334.541027997\\
2.76816920423011	380316.206378552\\
2.76826920673017	380298.444686903\\
2.76836920923023	380280.110037459\\
2.76846921173029	380261.775388015\\
2.76856921423036	380243.440738571\\
2.76866921673042	380225.106089127\\
2.76876921923048	380206.771439682\\
2.76886922173054	380188.436790238\\
2.76896922423061	380170.675098589\\
2.76906922673067	380152.340449145\\
2.76916922923073	380134.005799701\\
2.76926923173079	380115.671150257\\
2.76936923423086	380096.763543017\\
2.76946923673092	380078.428893573\\
2.76956923923098	380060.094244129\\
2.76966924173104	380041.759594685\\
2.76976924423111	380023.424945241\\
2.76986924673117	380005.090295796\\
2.76996924923123	379986.182688557\\
2.77006925173129	379967.848039113\\
2.77016925423136	379949.513389669\\
2.77026925673142	379930.605782429\\
2.77036925923148	379912.271132985\\
2.77046926173154	379893.936483541\\
2.77056926423161	379875.028876302\\
2.77066926673167	379856.694226858\\
2.77076926923173	379837.786619618\\
2.77086927173179	379819.451970174\\
2.77096927423186	379800.544362935\\
2.77106927673192	379782.209713491\\
2.77116927923198	379763.302106251\\
2.77126928173204	379744.394499012\\
2.77136928423211	379726.059849568\\
2.77146928673217	379707.152242328\\
2.77156928923223	379688.244635089\\
2.77166929173229	379669.33702785\\
2.77176929423236	379651.002378406\\
2.77186929673242	379632.094771166\\
2.77196929923248	379613.187163927\\
2.77206930173254	379594.279556688\\
2.77216930423261	379575.371949448\\
2.77226930673267	379556.464342209\\
2.77236930923273	379537.55673497\\
2.77246931173279	379518.64912773\\
2.77256931423286	379499.741520491\\
2.77266931673292	379480.833913252\\
2.77276931923298	379461.926306012\\
2.77286932173304	379443.018698773\\
2.77296932423311	379424.111091534\\
2.77306932673317	379404.630526499\\
2.77316932923323	379385.72291926\\
2.77326933173329	379366.815312021\\
2.77336933423336	379347.907704781\\
2.77346933673342	379328.427139747\\
2.77356933923348	379309.519532508\\
2.77366934173354	379290.611925268\\
2.77376934423361	379271.131360234\\
2.77386934673367	379252.223752994\\
2.77396934923373	379232.74318796\\
2.77406935173379	379213.835580721\\
2.77416935423386	379194.355015686\\
2.77426935673392	379175.447408447\\
2.77436935923398	379155.966843413\\
2.77446936173404	379136.486278378\\
2.77456936423411	379117.578671139\\
2.77466936673417	379098.098106104\\
2.77476936923423	379078.61754107\\
2.77486937173429	379059.709933831\\
2.77496937423436	379040.229368796\\
2.77506937673442	379020.748803762\\
2.77516937923448	379001.268238727\\
2.77526938173454	378981.787673693\\
2.77536938423461	378962.307108658\\
2.77546938673467	378943.399501419\\
2.77556938923473	378923.918936385\\
2.77566939173479	378904.43837135\\
2.77576939423486	378884.957806316\\
2.77586939673492	378865.477241281\\
2.77596939923498	378845.423718452\\
2.77606940173504	378825.943153417\\
2.77616940423511	378806.462588383\\
2.77626940673517	378786.982023348\\
2.77636940923523	378767.501458314\\
2.77646941173529	378748.020893279\\
2.77656941423536	378727.96737045\\
2.77666941673542	378708.486805415\\
2.77676941923548	378689.006240381\\
2.77686942173554	378668.952717551\\
2.77696942423561	378649.472152517\\
2.77706942673567	378629.991587482\\
2.77716942923573	378609.938064653\\
2.77726943173579	378590.457499618\\
2.77736943423586	378570.403976789\\
2.77746943673592	378550.923411754\\
2.77756943923598	378530.869888925\\
2.77766944173604	378510.816366095\\
2.77776944423611	378491.335801061\\
2.77786944673617	378471.282278231\\
2.77796944923623	378451.801713197\\
2.77806945173629	378431.748190367\\
2.77816945423636	378411.694667538\\
2.77826945673642	378391.641144708\\
2.77836945923648	378371.587621878\\
2.77846946173654	378352.107056844\\
2.77856946423661	378332.053534014\\
2.77866946673667	378312.000011185\\
2.77876946923673	378291.946488355\\
2.77886947173679	378271.892965526\\
2.77896947423686	378251.839442696\\
2.77906947673692	378231.785919867\\
2.77916947923698	378211.732397037\\
2.77926948173704	378191.678874207\\
2.77936948423711	378171.625351378\\
2.77946948673717	378151.571828548\\
2.77956948923723	378130.945347923\\
2.77966949173729	378110.891825094\\
2.77976949423736	378090.838302264\\
2.77986949673742	378070.784779435\\
2.77996949923748	378050.15829881\\
2.78006950173754	378030.104775981\\
2.78016950423761	378010.051253151\\
2.78026950673767	377989.424772526\\
2.78036950923773	377969.371249697\\
2.78046951173779	377949.317726867\\
2.78056951423786	377928.691246242\\
2.78066951673792	377908.637723413\\
2.78076951923798	377888.011242788\\
2.78086952173804	377867.957719958\\
2.78096952423811	377847.331239334\\
2.78106952673817	377826.704758709\\
2.78116952923823	377806.651235879\\
2.78126953173829	377786.024755255\\
2.78136953423836	377765.39827463\\
2.78146953673842	377745.3447518\\
2.78156953923848	377724.718271176\\
2.78166954173854	377704.091790551\\
2.78176954423861	377683.465309926\\
2.78186954673867	377662.838829302\\
2.78196954923873	377642.212348677\\
2.78206955173879	377622.158825847\\
2.78216955423886	377601.532345223\\
2.78226955673892	377580.905864598\\
2.78236955923898	377560.279383973\\
2.78246956173904	377539.652903348\\
2.78256956423911	377518.453464929\\
2.78266956673917	377497.826984304\\
2.78276956923923	377477.200503679\\
2.78286957173929	377456.574023055\\
2.78296957423936	377435.94754243\\
2.78306957673942	377415.321061805\\
2.78316957923948	377394.121623385\\
2.78326958173954	377373.495142761\\
2.78336958423961	377352.868662136\\
2.78346958673967	377332.242181511\\
2.78356958923973	377311.042743091\\
2.78366959173979	377290.416262467\\
2.78376959423986	377269.216824047\\
2.78386959673992	377248.590343422\\
2.78396959923998	377227.390905002\\
2.78406960174004	377206.764424377\\
2.78416960424011	377185.564985958\\
2.78426960674017	377164.938505333\\
2.78436960924023	377143.739066913\\
2.78446961174029	377123.112586288\\
2.78456961424036	377101.913147869\\
2.78466961674042	377080.713709449\\
2.78476961924048	377059.514271029\\
2.78486962174054	377038.887790404\\
2.78496962424061	377017.688351984\\
2.78506962674067	376996.488913564\\
2.78516962924073	376975.289475145\\
2.78526963174079	376954.090036725\\
2.78536963424086	376932.890598305\\
2.78546963674092	376912.26411768\\
2.78556963924098	376891.06467926\\
2.78566964174104	376869.865240841\\
2.78576964424111	376848.665802421\\
2.78586964674117	376827.466364001\\
2.78596964924123	376805.693967786\\
2.78606965174129	376784.494529366\\
2.78616965424136	376763.295090946\\
2.78626965674142	376742.095652526\\
2.78636965924148	376720.896214107\\
2.78646966174154	376699.696775687\\
2.78656966424161	376677.924379472\\
2.78666966674167	376656.724941052\\
2.78676966924173	376635.525502632\\
2.78686967174179	376613.753106417\\
2.78696967424186	376592.553667997\\
2.78706967674192	376571.354229577\\
2.78716967924198	376549.581833362\\
2.78726968174204	376528.382394943\\
2.78736968424211	376506.609998728\\
2.78746968674217	376485.410560308\\
2.78756968924223	376463.638164093\\
2.78766969174229	376442.438725673\\
2.78776969424236	376420.666329458\\
2.78786969674242	376398.893933243\\
2.78796969924248	376377.694494823\\
2.78806970174254	376355.922098608\\
2.78816970424261	376334.149702393\\
2.78826970674267	376312.377306178\\
2.78836970924273	376291.177867758\\
2.78846971174279	376269.405471543\\
2.78856971424286	376247.633075328\\
2.78866971674292	376225.860679113\\
2.78876971924298	376204.088282899\\
2.78886972174304	376182.315886684\\
2.78896972424311	376160.543490469\\
2.78906972674317	376138.771094254\\
2.78916972924323	376116.998698039\\
2.78926973174329	376095.226301824\\
2.78936973424336	376073.453905609\\
2.78946973674342	376051.681509394\\
2.78956973924348	376029.909113179\\
2.78966974174354	376008.136716964\\
2.78976974424361	375985.791362954\\
2.78986974674367	375964.018966739\\
2.78996974924373	375942.246570524\\
2.79006975174379	375920.474174309\\
2.79016975424386	375898.128820299\\
2.79026975674392	375876.356424084\\
2.79036975924398	375854.584027869\\
2.79046976174404	375832.238673859\\
2.79056976424411	375810.466277644\\
2.79066976674417	375788.120923634\\
2.79076976924423	375766.348527419\\
2.79086977174429	375744.003173409\\
2.79096977424436	375722.230777194\\
2.79106977674442	375699.885423183\\
2.79116977924448	375677.540069173\\
2.79126978174454	375655.767672958\\
2.79136978424461	375633.422318948\\
2.79146978674467	375611.076964938\\
2.79156978924473	375589.304568723\\
2.79166979174479	375566.959214713\\
2.79176979424486	375544.613860703\\
2.79186979674492	375522.268506693\\
2.79196979924498	375499.923152683\\
2.79206980174504	375477.577798673\\
2.79216980424511	375455.805402458\\
2.79226980674517	375433.460048448\\
2.79236980924523	375411.114694437\\
2.79246981174529	375388.769340427\\
2.79256981424536	375366.423986417\\
2.79266981674542	375344.078632407\\
2.79276981924548	375321.160320602\\
2.79286982174554	375298.814966592\\
2.79296982424561	375276.469612582\\
2.79306982674567	375254.124258572\\
2.79316982924573	375231.778904561\\
2.79326983174579	375209.433550551\\
2.79336983424586	375186.515238746\\
2.79346983674592	375164.169884736\\
2.79356983924598	375141.824530726\\
2.79366984174604	375118.906218921\\
2.79376984424611	375096.560864911\\
2.79386984674617	375074.215510901\\
2.79396984924623	375051.297199095\\
2.79406985174629	375028.951845085\\
2.79416985424636	375006.03353328\\
2.79426985674642	374983.68817927\\
2.79436985924648	374960.769867465\\
2.79446986174654	374937.851555659\\
2.79456986424661	374915.506201649\\
2.79466986674667	374892.587889844\\
2.79476986924673	374870.242535834\\
2.79486987174679	374847.324224029\\
2.79496987424686	374824.405912224\\
2.79506987674692	374801.487600418\\
2.79516987924698	374779.142246408\\
2.79526988174704	374756.223934603\\
2.79536988424711	374733.305622798\\
2.79546988674717	374710.387310993\\
2.79556988924723	374687.468999187\\
2.79566989174729	374664.550687382\\
2.79576989424736	374641.632375577\\
2.79586989674742	374618.714063772\\
2.79596989924748	374595.795751966\\
2.79606990174754	374572.877440161\\
2.79616990424761	374549.959128356\\
2.79626990674767	374527.040816551\\
2.79636990924773	374504.122504745\\
2.79646991174779	374481.20419294\\
2.79656991424786	374457.71292334\\
2.79666991674792	374434.794611535\\
2.79676991924798	374411.876299729\\
2.79686992174804	374388.957987924\\
2.79696992424811	374365.466718324\\
2.79706992674817	374342.548406519\\
2.79716992924823	374319.630094713\\
2.79726993174829	374296.138825113\\
2.79736993424836	374273.220513308\\
2.79746993674842	374249.729243707\\
2.79756993924848	374226.810931902\\
2.79766994174854	374203.319662302\\
2.79776994424861	374180.401350497\\
2.79786994674867	374156.910080896\\
2.79796994924873	374133.991769091\\
2.79806995174879	374110.500499491\\
2.79816995424886	374087.00922989\\
2.79826995674892	374064.090918085\\
2.79836995924898	374040.599648485\\
2.79846996174904	374017.108378884\\
2.79856996424911	373993.617109284\\
2.79866996674917	373970.698797479\\
2.79876996924923	373947.207527878\\
2.79886997174929	373923.716258278\\
2.79896997424936	373900.224988677\\
2.79906997674942	373876.733719077\\
2.79916997924948	373853.242449477\\
2.79926998174954	373829.751179876\\
2.79936998424961	373806.259910276\\
2.79946998674967	373782.768640676\\
2.79956998924973	373759.277371075\\
2.79966999174979	373735.786101475\\
2.79976999424986	373712.294831875\\
2.79986999674992	373688.803562274\\
2.79996999924998	373664.739334879\\
2.80007000175004	373641.248065278\\
};
\addplot [color=mycolor1,solid,forget plot]
  table[row sep=crcr]{%
2.80007000175004	373641.248065278\\
2.80017000425011	373617.756795678\\
2.80027000675017	373594.265526078\\
2.80037000925023	373570.201298682\\
2.80047001175029	373546.710029082\\
2.80057001425036	373523.218759481\\
2.80067001675042	373499.154532086\\
2.80077001925048	373475.663262486\\
2.80087002175054	373451.59903509\\
2.80097002425061	373428.10776549\\
2.80107002675067	373404.043538094\\
2.80117002925073	373380.552268494\\
2.80127003175079	373356.488041098\\
2.80137003425086	373332.996771498\\
2.80147003675092	373308.932544102\\
2.80157003925098	373285.441274502\\
2.80167004175104	373261.377047107\\
2.80177004425111	373237.312819711\\
2.80187004675117	373213.248592316\\
2.80197004925123	373189.757322715\\
2.80207005175129	373165.69309532\\
2.80217005425136	373141.628867924\\
2.80227005675142	373117.564640529\\
2.80237005925148	373093.500413133\\
2.80247006175154	373069.436185738\\
2.80257006425161	373045.944916137\\
2.80267006675167	373021.880688742\\
2.80277006925173	372997.816461346\\
2.80287007175179	372973.752233951\\
2.80297007425186	372949.688006555\\
2.80307007675192	372925.050821365\\
2.80317007925198	372900.986593969\\
2.80327008175204	372876.922366574\\
2.80337008425211	372852.858139178\\
2.80347008675217	372828.793911783\\
2.80357008925223	372804.729684387\\
2.80367009175229	372780.092499197\\
2.80377009425236	372756.028271801\\
2.80387009675242	372731.964044406\\
2.80397009925248	372707.326859215\\
2.80407010175254	372683.26263182\\
2.80417010425261	372659.198404424\\
2.80427010675267	372634.561219233\\
2.80437010925273	372610.496991838\\
2.80447011175279	372585.859806647\\
2.80457011425286	372561.795579252\\
2.80467011675292	372537.158394061\\
2.80477011925298	372513.094166666\\
2.80487012175304	372488.456981475\\
2.80497012425311	372464.39275408\\
2.80507012675317	372439.755568889\\
2.80517012925323	372415.118383698\\
2.80527013175329	372391.054156303\\
2.80537013425336	372366.416971112\\
2.80547013675342	372341.779785922\\
2.80557013925348	372317.142600731\\
2.80567014175354	372292.50541554\\
2.80577014425361	372268.441188145\\
2.80587014675367	372243.804002954\\
2.80597014925373	372219.166817764\\
2.80607015175379	372194.529632573\\
2.80617015425386	372169.892447382\\
2.80627015675392	372145.255262192\\
2.80637015925398	372120.618077001\\
2.80647016175404	372095.980891811\\
2.80657016425411	372071.34370662\\
2.80667016675417	372046.706521429\\
2.80677016925423	372022.069336239\\
2.80687017175429	371996.859193253\\
2.80697017425436	371972.222008062\\
2.80707017675442	371947.584822872\\
2.80717017925448	371922.947637681\\
2.80727018175454	371898.31045249\\
2.80737018425461	371873.100309505\\
2.80747018675467	371848.463124314\\
2.80757018925473	371823.825939123\\
2.80767019175479	371798.615796138\\
2.80777019425486	371773.978610947\\
2.80787019675492	371748.768467961\\
2.80797019925498	371724.131282771\\
2.80807020175504	371698.921139785\\
2.80817020425511	371674.283954594\\
2.80827020675517	371649.073811608\\
2.80837020925523	371624.436626418\\
2.80847021175529	371599.226483432\\
2.80857021425536	371574.016340446\\
2.80867021675542	371549.379155256\\
2.80877021925548	371524.16901227\\
2.80887022175554	371498.958869284\\
2.80897022425561	371474.321684094\\
2.80907022675567	371449.111541108\\
2.80917022925573	371423.901398122\\
2.80927023175579	371398.691255136\\
2.80937023425586	371373.481112151\\
2.80947023675592	371348.270969165\\
2.80957023925598	371323.060826179\\
2.80967024175604	371298.423640988\\
2.80977024425611	371273.213498003\\
2.80987024675617	371248.003355017\\
2.80997024925623	371222.220254236\\
2.81007025175629	371197.01011125\\
2.81017025425636	371171.799968265\\
2.81027025675642	371146.589825279\\
2.81037025925648	371121.379682293\\
2.81047026175654	371096.169539307\\
2.81057026425661	371070.959396322\\
2.81067026675667	371045.176295541\\
2.81077026925673	371019.966152555\\
2.81087027175679	370994.756009569\\
2.81097027425686	370969.545866583\\
2.81107027675692	370943.762765802\\
2.81117027925698	370918.552622817\\
2.81127028175704	370892.769522036\\
2.81137028425711	370867.55937905\\
2.81147028675717	370842.349236064\\
2.81157028925723	370816.566135283\\
2.81167029175729	370791.355992298\\
2.81177029425736	370765.572891517\\
2.81187029675742	370739.789790736\\
2.81197029925748	370714.57964775\\
2.81207030175754	370688.796546969\\
2.81217030425761	370663.586403983\\
2.81227030675767	370637.803303203\\
2.81237030925773	370612.020202422\\
2.81247031175779	370586.237101641\\
2.81257031425786	370561.026958655\\
2.81267031675792	370535.243857874\\
2.81277031925798	370509.460757093\\
2.81287032175804	370483.677656312\\
2.81297032425811	370457.894555532\\
2.81307032675817	370432.111454751\\
2.81317032925823	370406.901311765\\
2.81327033175829	370381.118210984\\
2.81337033425836	370355.335110203\\
2.81347033675842	370329.552009422\\
2.81357033925848	370303.195950846\\
2.81367034175854	370277.412850065\\
2.81377034425861	370251.629749284\\
2.81387034675867	370225.846648504\\
2.81397034925873	370200.063547723\\
2.81407035175879	370174.280446942\\
2.81417035425886	370148.497346161\\
2.81427035675892	370122.141287585\\
2.81437035925898	370096.358186804\\
2.81447036175904	370070.575086023\\
2.81457036425911	370044.791985242\\
2.81467036675917	370018.435926666\\
2.81477036925923	369992.652825885\\
2.81487037175929	369966.296767309\\
2.81497037425936	369940.513666528\\
2.81507037675942	369914.157607952\\
2.81517037925948	369888.374507172\\
2.81527038175954	369862.018448595\\
2.81537038425961	369836.235347815\\
2.81547038675967	369809.879289239\\
2.81557038925973	369784.096188458\\
2.81567039175979	369757.740129882\\
2.81577039425986	369731.384071306\\
2.81587039675992	369705.600970525\\
2.81597039925998	369679.244911949\\
2.81607040176004	369652.888853373\\
2.81617040426011	369627.105752592\\
2.81627040676017	369600.749694016\\
2.81637040926023	369574.39363544\\
2.81647041176029	369548.037576864\\
2.81657041426036	369521.681518288\\
2.81667041676042	369495.325459712\\
2.81677041926048	369468.969401136\\
2.81687042176054	369442.61334256\\
2.81697042426061	369416.257283984\\
2.81707042676067	369389.901225408\\
2.81717042926073	369363.545166832\\
2.81727043176079	369337.189108256\\
2.81737043426086	369310.83304968\\
2.81747043676092	369284.476991104\\
2.81757043926098	369258.120932528\\
2.81767044176104	369231.764873952\\
2.81777044426111	369204.83585758\\
2.81787044676117	369178.479799004\\
2.81797044926123	369152.123740428\\
2.81807045176129	369125.767681852\\
2.81817045426136	369098.838665481\\
2.81827045676142	369072.482606905\\
2.81837045926148	369046.126548329\\
2.81847046176154	369019.197531958\\
2.81857046426161	368992.841473382\\
2.81867046676167	368965.912457011\\
2.81877046926173	368939.556398435\\
2.81887047176179	368912.627382064\\
2.81897047426186	368886.271323488\\
2.81907047676192	368859.342307117\\
2.81917047926198	368832.986248541\\
2.81927048176204	368806.057232169\\
2.81937048426211	368779.128215798\\
2.81947048676217	368752.772157222\\
2.81957048926223	368725.843140851\\
2.81967049176229	368698.91412448\\
2.81977049426236	368671.985108109\\
2.81987049676242	368645.629049533\\
2.81997049926248	368618.700033162\\
2.82007050176254	368591.77101679\\
2.82017050426261	368564.842000419\\
2.82027050676267	368537.912984048\\
2.82037050926273	368510.983967677\\
2.82047051176279	368484.054951306\\
2.82057051426286	368457.125934935\\
2.82067051676292	368430.196918563\\
2.82077051926298	368403.267902192\\
2.82087052176304	368376.338885821\\
2.82097052426311	368349.40986945\\
2.82107052676317	368322.480853079\\
2.82117052926323	368295.551836708\\
2.82127053176329	368268.622820337\\
2.82137053426336	368241.12084617\\
2.82147053676342	368214.191829799\\
2.82157053926348	368187.262813428\\
2.82167054176354	368160.333797057\\
2.82177054426361	368132.831822891\\
2.82187054676367	368105.902806519\\
2.82197054926373	368078.973790148\\
2.82207055176379	368051.471815982\\
2.82217055426386	368024.542799611\\
2.82227055676392	367997.040825445\\
2.82237055926398	367970.111809073\\
2.82247056176404	367942.609834907\\
2.82257056426411	367915.680818536\\
2.82267056676417	367888.17884437\\
2.82277056926423	367861.249827999\\
2.82287057176429	367833.747853832\\
2.82297057426436	367806.818837461\\
2.82307057676442	367779.316863295\\
2.82317057926448	367751.814889129\\
2.82327058176454	367724.312914962\\
2.82337058426461	367697.383898591\\
2.82347058676467	367669.881924425\\
2.82357058926473	367642.379950259\\
2.82367059176479	367614.877976092\\
2.82377059426486	367587.376001926\\
2.82387059676492	367560.446985555\\
2.82397059926498	367532.945011389\\
2.82407060176504	367505.443037222\\
2.82417060426511	367477.941063056\\
2.82427060676517	367450.43908889\\
2.82437060926523	367422.937114724\\
2.82447061176529	367395.435140557\\
2.82457061426536	367367.933166391\\
2.82467061676542	367340.431192225\\
2.82477061926548	367312.356260263\\
2.82487062176554	367284.854286097\\
2.82497062426561	367257.352311931\\
2.82507062676567	367229.850337764\\
2.82517062926573	367202.348363598\\
2.82527063176579	367174.273431637\\
2.82537063426586	367146.77145747\\
2.82547063676592	367119.269483304\\
2.82557063926598	367091.194551343\\
2.82567064176604	367063.692577177\\
2.82577064426611	367036.19060301\\
2.82587064676617	367008.115671049\\
2.82597064926623	366980.613696883\\
2.82607065176629	366952.538764921\\
2.82617065426636	366925.036790755\\
2.82627065676642	366896.961858793\\
2.82637065926648	366869.459884627\\
2.82647066176654	366841.384952666\\
2.82657066426661	366813.310020704\\
2.82667066676667	366785.808046538\\
2.82677066926673	366757.733114577\\
2.82687067176679	366729.658182615\\
2.82697067426686	366702.156208449\\
2.82707067676692	366674.081276488\\
2.82717067926698	366646.006344526\\
2.82727068176704	366617.931412565\\
2.82737068426711	366590.429438398\\
2.82747068676717	366562.354506437\\
2.82757068926723	366534.279574476\\
2.82767069176729	366506.204642514\\
2.82777069426736	366478.129710553\\
2.82787069676742	366450.054778591\\
2.82797069926748	366421.97984663\\
2.82807070176754	366393.904914669\\
2.82817070426761	366365.829982707\\
2.82827070676767	366337.755050746\\
2.82837070926773	366309.680118784\\
2.82847071176779	366281.605186823\\
2.82857071426786	366252.957297066\\
2.82867071676792	366224.882365105\\
2.82877071926798	366196.807433144\\
2.82887072176804	366168.732501182\\
2.82897072426811	366140.657569221\\
2.82907072676817	366112.009679464\\
2.82917072926823	366083.934747503\\
2.82927073176829	366055.859815541\\
2.82937073426836	366027.211925785\\
2.82947073676842	365999.136993823\\
2.82957073926848	365970.489104067\\
2.82967074176854	365942.414172105\\
2.82977074426861	365914.339240144\\
2.82987074676867	365885.691350388\\
2.82997074926873	365857.616418426\\
2.83007075176879	365828.96852867\\
2.83017075426886	365800.320638913\\
2.83027075676892	365772.245706952\\
2.83037075926898	365743.597817195\\
2.83047076176904	365714.949927439\\
2.83057076426911	365686.874995477\\
2.83067076676917	365658.227105721\\
2.83077076926923	365629.579215964\\
2.83087077176929	365601.504284003\\
2.83097077426936	365572.856394246\\
2.83107077676942	365544.20850449\\
2.83117077926948	365515.560614733\\
2.83127078176954	365486.912724976\\
2.83137078426961	365458.26483522\\
2.83147078676967	365429.616945463\\
2.83157078926973	365400.969055707\\
2.83167079176979	365372.32116595\\
2.83177079426986	365343.673276194\\
2.83187079676992	365315.025386437\\
2.83197079926998	365286.377496681\\
2.83207080177004	365257.729606924\\
2.83217080427011	365229.081717168\\
2.83227080677017	365200.433827411\\
2.83237080927023	365171.785937655\\
2.83247081177029	365143.138047898\\
2.83257081427036	365113.917200346\\
2.83267081677042	365085.26931059\\
2.83277081927048	365056.621420833\\
2.83287082177054	365027.973531077\\
2.83297082427061	364998.752683525\\
2.83307082677067	364970.104793769\\
2.83317082927073	364941.456904012\\
2.83327083177079	364912.23605646\\
2.83337083427086	364883.588166704\\
2.83347083677092	364854.367319152\\
2.83357083927098	364825.719429396\\
2.83367084177104	364796.498581844\\
2.83377084427111	364767.850692087\\
2.83387084677117	364738.629844536\\
2.83397084927123	364709.981954779\\
2.83407085177129	364680.761107227\\
2.83417085427136	364651.540259676\\
2.83427085677142	364622.892369919\\
2.83437085927148	364593.671522368\\
2.83447086177154	364564.450674816\\
2.83457086427161	364535.229827264\\
2.83467086677167	364506.581937508\\
2.83477086927173	364477.361089956\\
2.83487087177179	364448.140242404\\
2.83497087427186	364418.919394853\\
2.83507087677192	364389.698547301\\
2.83517087927198	364360.477699749\\
2.83527088177204	364331.256852198\\
2.83537088427211	364302.608962441\\
2.83547088677217	364273.388114889\\
2.83557088927223	364244.167267338\\
2.83567089177229	364214.373461991\\
2.83577089427236	364185.152614439\\
2.83587089677242	364155.931766888\\
2.83597089927248	364126.710919336\\
2.83607090177254	364097.490071784\\
2.83617090427261	364068.269224233\\
2.83627090677267	364039.048376681\\
2.83637090927273	364009.254571334\\
2.83647091177279	363980.033723782\\
2.83657091427286	363950.812876231\\
2.83667091677292	363921.592028679\\
2.83677091927298	363891.798223332\\
2.83687092177304	363862.577375781\\
2.83697092427311	363833.356528229\\
2.83707092677317	363803.562722882\\
2.83717092927323	363774.34187533\\
2.83727093177329	363744.548069984\\
2.83737093427336	363715.327222432\\
2.83747093677342	363685.533417085\\
2.83757093927348	363656.312569534\\
2.83767094177354	363626.518764187\\
2.83777094427361	363597.297916635\\
2.83787094677367	363567.504111288\\
2.83797094927373	363537.710305941\\
2.83807095177379	363508.48945839\\
2.83817095427386	363478.695653043\\
2.83827095677392	363448.901847696\\
2.83837095927398	363419.681000145\\
2.83847096177404	363389.887194798\\
2.83857096427411	363360.093389451\\
2.83867096677417	363330.299584104\\
2.83877096927423	363300.505778757\\
2.83887097177429	363270.711973411\\
2.83897097427436	363241.491125859\\
2.83907097677442	363211.697320512\\
2.83917097927448	363181.903515165\\
2.83927098177454	363152.109709818\\
2.83937098427461	363122.315904472\\
2.83947098677467	363092.522099125\\
2.83957098927473	363062.728293778\\
2.83967099177479	363032.361530636\\
2.83977099427486	363002.567725289\\
2.83987099677492	362972.773919942\\
2.83997099927498	362942.980114596\\
2.84007100177504	362913.186309249\\
2.84017100427511	362883.392503902\\
2.84027100677517	362853.02574076\\
2.84037100927523	362823.231935413\\
2.84047101177529	362793.438130066\\
2.84057101427536	362763.64432472\\
2.84067101677542	362733.277561578\\
2.84077101927548	362703.483756231\\
2.84087102177554	362673.116993089\\
2.84097102427561	362643.323187742\\
2.84107102677567	362613.529382395\\
2.84117102927573	362583.162619254\\
2.84127103177579	362553.368813907\\
2.84137103427586	362523.002050765\\
2.84147103677592	362493.208245418\\
2.84157103927598	362462.841482276\\
2.84167104177604	362432.474719134\\
2.84177104427611	362402.680913787\\
2.84187104677617	362372.314150645\\
2.84197104927623	362341.947387503\\
2.84207105177629	362312.153582157\\
2.84217105427636	362281.786819015\\
2.84227105677642	362251.420055873\\
2.84237105927648	362221.053292731\\
2.84247106177654	362191.259487384\\
2.84257106427661	362160.892724242\\
2.84267106677667	362130.5259611\\
2.84277106927673	362100.159197958\\
2.84287107177679	362069.792434816\\
2.84297107427686	362039.425671674\\
2.84307107677692	362009.058908532\\
2.84317107927698	361978.69214539\\
2.84327108177704	361948.325382248\\
2.84337108427711	361917.958619107\\
2.84347108677717	361887.591855965\\
2.84357108927723	361857.225092823\\
2.84367109177729	361826.858329681\\
2.84377109427736	361796.491566539\\
2.84387109677742	361766.124803397\\
2.84397109927748	361735.18508246\\
2.84407110177754	361704.818319318\\
2.84417110427761	361674.451556176\\
2.84427110677767	361644.084793034\\
2.84437110927773	361613.145072097\\
2.84447111177779	361582.778308955\\
2.84457111427786	361552.411545813\\
2.84467111677792	361521.471824876\\
2.84477111927798	361491.105061734\\
2.84487112177804	361460.738298592\\
2.84497112427811	361429.798577655\\
2.84507112677817	361399.431814513\\
2.84517112927823	361368.492093576\\
2.84527113177829	361338.125330434\\
2.84537113427836	361307.185609497\\
2.84547113677842	361276.24588856\\
2.84557113927848	361245.879125418\\
2.84567114177854	361214.939404481\\
2.84577114427861	361184.572641339\\
2.84587114677867	361153.632920402\\
2.84597114927873	361122.693199465\\
2.84607115177879	361091.753478528\\
2.84617115427886	361061.386715386\\
2.84627115677892	361030.446994449\\
2.84637115927898	360999.507273512\\
2.84647116177904	360968.567552575\\
2.84657116427911	360937.627831638\\
2.84667116677917	360907.261068496\\
2.84677116927923	360876.321347559\\
2.84687117177929	360845.381626622\\
2.84697117427936	360814.441905685\\
2.84707117677942	360783.502184748\\
2.84717117927948	360752.56246381\\
2.84727118177954	360721.622742873\\
2.84737118427961	360690.683021936\\
2.84747118677967	360659.743300999\\
2.84757118927973	360628.803580062\\
2.84767119177979	360597.29090133\\
2.84777119427986	360566.351180393\\
2.84787119677992	360535.411459456\\
2.84797119927998	360504.471738519\\
2.84807120178004	360473.532017582\\
2.84817120428011	360442.01933885\\
2.84827120678017	360411.079617913\\
2.84837120928023	360380.139896975\\
2.84847121178029	360348.627218243\\
2.84857121428036	360317.687497306\\
2.84867121678042	360286.747776369\\
2.84877121928048	360255.235097637\\
2.84887122178054	360224.2953767\\
2.84897122428061	360192.782697968\\
2.84907122678067	360161.842977031\\
2.84917122928073	360130.330298298\\
2.84927123178079	360099.390577361\\
2.84937123428086	360067.877898629\\
2.84947123678092	360036.938177692\\
2.84957123928098	360005.42549896\\
2.84967124178104	359973.912820228\\
2.84977124428111	359942.973099291\\
2.84987124678117	359911.460420558\\
2.84997124928123	359879.947741826\\
2.85007125178129	359849.008020889\\
2.85017125428136	359817.495342157\\
2.85027125678142	359785.982663425\\
2.85037125928148	359754.469984693\\
2.85047126178154	359722.95730596\\
2.85057126428161	359691.444627228\\
2.85067126678167	359660.504906291\\
2.85077126928173	359628.992227559\\
2.85087127178179	359597.479548827\\
2.85097127428186	359565.966870095\\
2.85107127678192	359534.454191362\\
2.85117127928198	359502.94151263\\
2.85127128178204	359471.428833898\\
2.85137128428211	359439.916155166\\
2.85147128678217	359408.403476434\\
2.85157128928223	359376.317839906\\
2.85167129178229	359344.805161174\\
2.85177129428236	359313.292482442\\
2.85187129678242	359281.77980371\\
2.85197129928248	359250.267124977\\
2.85207130178254	359218.18148845\\
2.85217130428261	359186.668809718\\
2.85227130678267	359155.156130986\\
2.85237130928273	359123.643452254\\
2.85247131178279	359091.557815726\\
2.85257131428286	359060.045136994\\
2.85267131678292	359028.532458262\\
2.85277131928298	358996.446821735\\
2.85287132178304	358964.934143002\\
2.85297132428311	358932.848506475\\
2.85307132678317	358901.335827743\\
2.85317132928323	358869.250191215\\
2.85327133178329	358837.737512483\\
2.85337133428336	358805.651875956\\
2.85347133678342	358774.139197224\\
2.85357133928348	358742.053560696\\
2.85367134178354	358709.967924169\\
2.85377134428361	358678.455245437\\
2.85387134678367	358646.36960891\\
2.85397134928373	358614.283972382\\
2.85407135178379	358582.198335855\\
2.85417135428386	358550.685657123\\
2.85427135678392	358518.600020595\\
2.85437135928398	358486.514384068\\
2.85447136178404	358454.428747541\\
2.85457136428411	358422.343111013\\
2.85467136678417	358390.830432281\\
2.85477136928423	358358.744795754\\
2.85487137178429	358326.659159227\\
2.85497137428436	358294.573522699\\
2.85507137678442	358262.487886172\\
2.85517137928448	358230.402249645\\
2.85527138178454	358198.316613117\\
2.85537138428461	358166.23097659\\
2.85547138678467	358134.145340063\\
2.85557138928473	358101.48674574\\
2.85567139178479	358069.401109213\\
2.85577139428486	358037.315472686\\
2.85587139678492	358005.229836158\\
2.85597139928498	357973.144199631\\
2.85607140178504	357941.058563104\\
2.85617140428511	357908.399968781\\
2.85627140678517	357876.314332254\\
2.85637140928523	357844.228695726\\
2.85647141178529	357811.570101404\\
2.85657141428536	357779.484464877\\
2.85667141678542	357747.398828349\\
2.85677141928548	357714.740234027\\
2.85687142178554	357682.6545975\\
2.85697142428561	357649.996003177\\
2.85707142678567	357617.91036665\\
2.85717142928573	357585.251772327\\
2.85727143178579	357553.1661358\\
2.85737143428586	357520.507541478\\
2.85747143678592	357488.42190495\\
2.85757143928598	357455.763310628\\
2.85767144178604	357423.104716305\\
2.85777144428611	357391.019079778\\
2.85787144678617	357358.360485455\\
2.85797144928623	357325.701891133\\
2.85807145178629	357293.616254606\\
2.85817145428636	357260.957660283\\
2.85827145678642	357228.299065961\\
2.85837145928648	357195.640471638\\
2.85847146178654	357163.554835111\\
2.85857146428661	357130.896240789\\
2.85867146678667	357098.237646466\\
2.85877146928673	357065.579052144\\
2.85887147178679	357032.920457821\\
2.85897147428686	357000.261863499\\
2.85907147678692	356967.603269176\\
2.85917147928698	356934.944674854\\
2.85927148178704	356902.286080531\\
2.85937148428711	356869.627486209\\
2.85947148678717	356836.968891886\\
2.85957148928723	356804.310297564\\
2.85967149178729	356771.651703242\\
2.85977149428736	356738.993108919\\
2.85987149678742	356706.334514597\\
2.85997149928748	356673.102962479\\
2.86007150178754	356640.444368157\\
2.86017150428761	356607.785773834\\
2.86027150678767	356575.127179512\\
2.86037150928773	356541.895627394\\
2.86047151178779	356509.237033072\\
2.86057151428786	356476.578438749\\
2.86067151678792	356443.346886632\\
2.86077151928798	356410.688292309\\
2.86087152178804	356378.029697987\\
2.86097152428811	356344.798145869\\
2.86107152678817	356312.139551547\\
2.86117152928823	356278.907999429\\
2.86127153178829	356246.249405107\\
2.86137153428836	356213.017852989\\
2.86147153678842	356180.359258667\\
2.86157153928848	356147.127706549\\
2.86167154178854	356113.896154431\\
2.86177154428861	356081.237560109\\
2.86187154678867	356048.006007991\\
2.86197154928873	356015.347413669\\
2.86207155178879	355982.115861551\\
2.86217155428886	355948.884309434\\
2.86227155678892	355915.652757316\\
2.86237155928898	355882.994162994\\
2.86247156178904	355849.762610876\\
2.86257156428911	355816.531058758\\
2.86267156678917	355783.299506641\\
2.86277156928923	355750.067954523\\
2.86287157178929	355716.836402406\\
2.86297157428936	355684.177808083\\
2.86307157678942	355650.946255966\\
2.86317157928948	355617.714703848\\
2.86327158178954	355584.48315173\\
2.86337158428961	355551.251599613\\
2.86347158678967	355518.020047495\\
2.86357158928973	355484.788495378\\
2.86367159178979	355450.983985465\\
2.86377159428986	355417.752433347\\
2.86387159678992	355384.52088123\\
2.86397159928998	355351.289329112\\
2.86407160179004	355318.057776995\\
2.86417160429011	355284.826224877\\
2.86427160679017	355251.594672759\\
2.86437160929023	355217.790162847\\
2.86447161179029	355184.558610729\\
2.86457161429036	355151.327058612\\
2.86467161679042	355117.522548699\\
2.86477161929048	355084.290996581\\
2.86487162179054	355051.059444464\\
2.86497162429061	355017.254934551\\
2.86507162679067	354984.023382433\\
2.86517162929073	354950.218872521\\
2.86527163179079	354916.987320403\\
2.86537163429086	354883.18281049\\
2.86547163679092	354849.951258373\\
2.86557163929098	354816.14674846\\
2.86567164179104	354782.915196342\\
2.86577164429111	354749.11068643\\
2.86587164679117	354715.879134312\\
2.86597164929123	354682.074624399\\
2.86607165179129	354648.270114487\\
2.86617165429136	354615.038562369\\
2.86627165679142	354581.234052456\\
2.86637165929148	354547.429542544\\
2.86647166179154	354514.197990426\\
2.86657166429161	354480.393480513\\
2.86667166679167	354446.588970601\\
2.86677166929173	354412.784460688\\
2.86687167179179	354378.979950775\\
2.86697167429186	354345.748398658\\
2.86707167679192	354311.943888745\\
2.86717167929198	354278.139378832\\
2.86727168179204	354244.334868919\\
2.86737168429211	354210.530359007\\
2.86747168679217	354176.725849094\\
2.86757168929223	354142.921339181\\
2.86767169179229	354109.116829269\\
2.86777169429236	354075.312319356\\
2.86787169679242	354041.507809443\\
2.86797169929248	354007.70329953\\
2.86807170179255	353973.325831823\\
2.86817170429261	353939.52132191\\
2.86827170679267	353905.716811997\\
2.86837170929273	353871.912302084\\
2.86847171179279	353838.107792172\\
2.86857171429286	353803.730324464\\
2.86867171679292	353769.925814551\\
2.86877171929298	353736.121304638\\
2.86887172179304	353702.316794726\\
2.86897172429311	353667.939327018\\
2.86907172679317	353634.134817105\\
2.86917172929323	353600.330307192\\
2.86927173179329	353565.952839485\\
2.86937173429336	353532.148329572\\
2.86947173679342	353497.770861864\\
2.86957173929348	353463.966351951\\
2.86967174179354	353429.588884243\\
2.86977174429361	353395.784374331\\
2.86987174679367	353361.406906623\\
2.86997174929373	353327.60239671\\
2.8700717517938	353293.224929002\\
2.87017175429386	353258.847461294\\
2.87027175679392	353225.042951382\\
2.87037175929398	353190.665483674\\
2.87047176179404	353156.288015966\\
2.87057176429411	353122.483506053\\
2.87067176679417	353088.106038345\\
2.87077176929423	353053.728570638\\
2.87087177179429	353019.35110293\\
2.87097177429436	352985.546593017\\
2.87107177679442	352951.169125309\\
2.87117177929448	352916.791657601\\
2.87127178179454	352882.414189894\\
2.87137178429461	352848.036722186\\
2.87147178679467	352813.659254478\\
2.87157178929473	352779.28178677\\
2.87167179179479	352744.904319062\\
2.87177179429486	352710.526851354\\
2.87187179679492	352676.149383646\\
2.87197179929498	352641.771915939\\
2.87207180179505	352607.394448231\\
2.87217180429511	352573.016980523\\
2.87227180679517	352538.639512815\\
2.87237180929523	352504.262045107\\
2.87247181179529	352469.884577399\\
2.87257181429536	352435.507109692\\
2.87267181679542	352400.556684188\\
2.87277181929548	352366.179216481\\
2.87287182179554	352331.801748773\\
2.87297182429561	352297.424281065\\
2.87307182679567	352262.473855562\\
2.87317182929573	352228.096387854\\
2.87327183179579	352193.718920146\\
2.87337183429586	352158.768494643\\
2.87347183679592	352124.391026935\\
2.87357183929598	352090.013559228\\
2.87367184179604	352055.063133725\\
2.87377184429611	352020.685666017\\
2.87387184679617	351985.735240514\\
2.87397184929623	351951.357772806\\
2.8740718517963	351916.407347303\\
2.87417185429636	351882.029879595\\
2.87427185679642	351847.079454092\\
2.87437185929648	351812.129028589\\
2.87447186179655	351777.751560881\\
2.87457186429661	351742.801135378\\
2.87467186679667	351708.42366767\\
2.87477186929673	351673.473242167\\
2.87487187179679	351638.522816664\\
2.87497187429686	351603.572391162\\
2.87507187679692	351569.194923454\\
2.87517187929698	351534.244497951\\
2.87527188179704	351499.294072448\\
2.87537188429711	351464.343646945\\
2.87547188679717	351429.393221442\\
2.87557188929723	351395.015753734\\
2.87567189179729	351360.065328231\\
2.87577189429736	351325.114902728\\
2.87587189679742	351290.164477225\\
2.87597189929748	351255.214051722\\
2.87607190179755	351220.263626219\\
2.87617190429761	351185.313200716\\
2.87627190679767	351150.362775213\\
2.87637190929773	351115.41234971\\
2.8764719117978	351080.461924207\\
2.87657191429786	351045.511498704\\
2.87667191679792	351009.988115406\\
2.87677191929798	350975.037689903\\
2.87687192179804	350940.0872644\\
2.87697192429811	350905.136838897\\
2.87707192679817	350870.186413394\\
2.87717192929823	350835.235987891\\
2.87727193179829	350799.712604593\\
2.87737193429836	350764.76217909\\
2.87747193679842	350729.811753587\\
2.87757193929848	350694.288370289\\
2.87767194179854	350659.337944786\\
2.87777194429861	350624.387519283\\
2.87787194679867	350588.864135985\\
2.87797194929873	350553.913710482\\
2.8780719517988	350518.390327184\\
2.87817195429886	350483.439901681\\
2.87827195679892	350447.916518383\\
2.87837195929898	350412.96609288\\
2.87847196179905	350377.442709582\\
2.87857196429911	350342.492284079\\
2.87867196679917	350306.96890078\\
2.87877196929923	350272.018475277\\
2.87887197179929	350236.495091979\\
2.87897197429936	350200.971708681\\
2.87907197679942	350166.021283178\\
2.87917197929948	350130.49789988\\
2.87927198179954	350094.974516582\\
2.87937198429961	350060.024091079\\
2.87947198679967	350024.500707781\\
2.87957198929973	349988.977324483\\
2.87967199179979	349953.453941185\\
2.87977199429986	349918.503515682\\
2.87987199679992	349882.980132384\\
2.87997199929998	349847.456749086\\
2.88007200180005	349811.933365787\\
2.88017200430011	349776.409982489\\
2.88027200680017	349740.886599191\\
2.88037200930023	349705.363215893\\
2.8804720118003	349669.839832595\\
2.88057201430036	349634.316449297\\
2.88067201680042	349598.793065999\\
2.88077201930048	349563.269682701\\
2.88087202180055	349527.746299403\\
2.88097202430061	349492.222916104\\
2.88107202680067	349456.699532806\\
2.88117202930073	349421.176149508\\
2.88127203180079	349385.079808415\\
2.88137203430086	349349.556425117\\
2.88147203680092	349314.033041819\\
2.88157203930098	349278.509658521\\
2.88167204180104	349242.986275223\\
2.88177204430111	349206.889934129\\
2.88187204680117	349171.366550831\\
2.88197204930123	349135.843167533\\
2.8820720518013	349099.74682644\\
2.88217205430136	349064.223443142\\
2.88227205680142	349028.700059844\\
2.88237205930148	348992.60371875\\
2.88247206180155	348957.080335452\\
2.88257206430161	348920.983994359\\
2.88267206680167	348885.460611061\\
2.88277206930173	348849.364269968\\
2.8828720718018	348813.84088667\\
2.88297207430186	348777.744545576\\
2.88307207680192	348742.221162278\\
2.88317207930198	348706.124821185\\
2.88327208180204	348670.601437887\\
2.88337208430211	348634.505096794\\
2.88347208680217	348598.4087557\\
2.88357208930223	348562.885372402\\
2.88367209180229	348526.789031309\\
2.88377209430236	348490.692690216\\
2.88387209680242	348454.596349123\\
2.88397209930248	348419.072965824\\
2.88407210180255	348382.976624731\\
2.88417210430261	348346.880283638\\
2.88427210680267	348310.783942545\\
2.88437210930273	348274.687601451\\
2.8844721118028	348239.164218153\\
2.88457211430286	348203.06787706\\
2.88467211680292	348166.971535967\\
2.88477211930298	348130.875194874\\
2.88487212180305	348094.77885378\\
2.88497212430311	348058.682512687\\
2.88507212680317	348022.586171594\\
2.88517212930323	347986.489830501\\
2.8852721318033	347950.393489407\\
2.88537213430336	347914.297148314\\
2.88547213680342	347878.200807221\\
2.88557213930348	347842.104466128\\
2.88567214180354	347805.435167239\\
2.88577214430361	347769.338826146\\
2.88587214680367	347733.242485053\\
2.88597214930373	347697.14614396\\
2.8860721518038	347661.049802866\\
2.88617215430386	347624.380503978\\
2.88627215680392	347588.284162885\\
2.88637215930398	347552.187821792\\
2.88647216180405	347516.091480698\\
2.88657216430411	347479.42218181\\
2.88667216680417	347443.325840717\\
2.88677216930423	347406.656541828\\
2.8868721718043	347370.560200735\\
2.88697217430436	347334.463859642\\
2.88707217680442	347297.794560753\\
2.88717217930448	347261.69821966\\
2.88727218180455	347225.028920772\\
2.88737218430461	347188.932579678\\
2.88747218680467	347152.26328079\\
2.88757218930473	347116.166939697\\
2.88767219180479	347079.497640809\\
2.88777219430486	347042.82834192\\
2.88787219680492	347006.732000827\\
2.88797219930498	346970.062701939\\
2.88807220180505	346933.966360845\\
2.88817220430511	346897.297061957\\
2.88827220680517	346860.627763069\\
2.88837220930523	346823.95846418\\
2.8884722118053	346787.862123087\\
2.88857221430536	346751.192824199\\
2.88867221680542	346714.52352531\\
2.88877221930548	346677.854226422\\
2.88887222180555	346641.184927533\\
2.88897222430561	346605.08858644\\
2.88907222680567	346568.419287552\\
2.88917222930573	346531.749988663\\
2.8892722318058	346495.080689775\\
2.88937223430586	346458.411390887\\
2.88947223680592	346421.742091998\\
2.88957223930598	346385.07279311\\
2.88967224180604	346348.403494222\\
2.88977224430611	346311.734195333\\
2.88987224680617	346275.064896445\\
2.88997224930623	346238.395597556\\
2.8900722518063	346201.726298668\\
2.89017225430636	346165.05699978\\
2.89027225680642	346127.814743096\\
2.89037225930648	346091.145444208\\
2.89047226180655	346054.476145319\\
2.89057226430661	346017.806846431\\
2.89067226680667	345981.137547543\\
2.89077226930673	345943.895290859\\
2.8908722718068	345907.225991971\\
2.89097227430686	345870.556693082\\
2.89107227680692	345833.887394194\\
2.89117227930698	345796.645137511\\
2.89127228180705	345759.975838622\\
2.89137228430711	345723.306539734\\
2.89147228680717	345686.06428305\\
2.89157228930723	345649.394984162\\
2.8916722918073	345612.152727478\\
2.89177229430736	345575.48342859\\
2.89187229680742	345538.241171907\\
2.89197229930748	345501.571873018\\
2.89207230180755	345464.329616335\\
2.89217230430761	345427.660317446\\
2.89227230680767	345390.418060763\\
2.89237230930773	345353.748761875\\
2.8924723118078	345316.506505191\\
2.89257231430786	345279.264248508\\
2.89267231680792	345242.594949619\\
2.89277231930798	345205.352692936\\
2.89287232180805	345168.110436252\\
2.89297232430811	345131.441137364\\
2.89307232680817	345094.19888068\\
2.89317232930823	345056.956623997\\
2.8932723318083	345019.714367313\\
2.89337233430836	344983.045068425\\
2.89347233680842	344945.802811741\\
2.89357233930848	344908.560555058\\
2.89367234180855	344871.318298374\\
2.89377234430861	344834.076041691\\
2.89387234680867	344796.833785007\\
2.89397234930873	344759.591528324\\
2.8940723518088	344722.34927164\\
2.89417235430886	344685.107014957\\
2.89427235680892	344647.864758273\\
2.89437235930898	344610.62250159\\
2.89447236180905	344573.380244906\\
2.89457236430911	344536.137988223\\
2.89467236680917	344498.895731539\\
2.89477236930923	344461.653474856\\
2.8948723718093	344424.411218172\\
2.89497237430936	344387.168961489\\
2.89507237680942	344349.926704805\\
2.89517237930948	344312.111490327\\
2.89527238180955	344274.869233643\\
2.89537238430961	344237.62697696\\
2.89547238680967	344200.384720276\\
2.89557238930973	344162.569505798\\
2.8956723918098	344125.327249114\\
2.89577239430986	344088.084992431\\
2.89587239680992	344050.842735747\\
2.89597239930998	344013.027521268\\
2.89607240181005	343975.785264585\\
2.89617240431011	343937.970050106\\
2.89627240681017	343900.727793423\\
2.89637240931023	343863.485536739\\
2.8964724118103	343825.670322261\\
2.89657241431036	343788.428065577\\
2.89667241681042	343750.612851098\\
2.89677241931048	343713.370594415\\
2.89687242181055	343675.555379936\\
2.89697242431061	343637.740165458\\
2.89707242681067	343600.497908774\\
2.89717242931073	343562.682694296\\
2.8972724318108	343525.440437612\\
2.89737243431086	343487.625223133\\
2.89747243681092	343449.810008655\\
2.89757243931098	343412.567751971\\
2.89767244181105	343374.752537493\\
2.89777244431111	343336.937323014\\
2.89787244681117	343299.122108535\\
2.89797244931123	343261.879851852\\
2.8980724518113	343224.064637373\\
2.89817245431136	343186.249422895\\
2.89827245681142	343148.434208416\\
2.89837245931148	343110.618993937\\
2.89847246181155	343072.803779459\\
2.89857246431161	343035.561522775\\
2.89867246681167	342997.746308297\\
2.89877246931173	342959.931093818\\
2.8988724718118	342922.115879339\\
2.89897247431186	342884.300664861\\
2.89907247681192	342846.485450382\\
2.89917247931198	342808.670235903\\
2.89927248181205	342770.855021425\\
2.89937248431211	342733.039806946\\
2.89947248681217	342695.224592467\\
2.89957248931223	342656.836420194\\
2.8996724918123	342619.021205715\\
2.89977249431236	342581.205991236\\
2.89987249681242	342543.390776758\\
2.89997249931248	342505.575562279\\
2.90007250181255	342467.760347801\\
2.90017250431261	342429.372175527\\
2.90027250681267	342391.556961048\\
2.90037250931273	342353.74174657\\
2.9004725118128	342315.926532091\\
2.90057251431286	342277.538359817\\
2.90067251681292	342239.723145338\\
2.90077251931298	342201.90793086\\
2.90087252181305	342163.519758586\\
2.90097252431311	342125.704544107\\
2.90107252681317	342087.316371834\\
2.90117252931323	342049.501157355\\
2.9012725318133	342011.685942876\\
2.90137253431336	341973.297770603\\
2.90147253681342	341935.482556124\\
2.90157253931348	341897.09438385\\
2.90167254181355	341858.706211576\\
2.90177254431361	341820.890997098\\
2.90187254681367	341782.502824824\\
2.90197254931373	341744.687610345\\
2.9020725518138	341706.299438072\\
2.90217255431386	341667.911265798\\
2.90227255681392	341630.096051319\\
2.90237255931398	341591.707879046\\
2.90247256181405	341553.319706772\\
2.90257256431411	341515.504492293\\
2.90267256681417	341477.116320019\\
2.90277256931423	341438.728147746\\
2.9028725718143	341400.339975472\\
2.90297257431436	341362.524760993\\
2.90307257681442	341324.136588719\\
2.90317257931448	341285.748416446\\
2.90327258181455	341247.360244172\\
2.90337258431461	341208.972071898\\
2.90347258681467	341170.583899624\\
2.90357258931473	341132.195727351\\
2.9036725918148	341093.807555077\\
2.90377259431486	341055.419382803\\
2.90387259681492	341017.031210529\\
2.90397259931498	340978.643038256\\
2.90407260181505	340940.254865982\\
2.90417260431511	340901.866693708\\
2.90427260681517	340863.478521434\\
2.90437260931523	340825.09034916\\
2.9044726118153	340786.702176887\\
2.90457261431536	340748.314004613\\
2.90467261681542	340709.925832339\\
2.90477261931548	340671.537660065\\
2.90487262181555	340632.576529997\\
2.90497262431561	340594.188357723\\
2.90507262681567	340555.800185449\\
2.90517262931573	340517.412013175\\
2.9052726318158	340478.450883106\\
2.90537263431586	340440.062710833\\
2.90547263681592	340401.674538559\\
2.90557263931598	340363.286366285\\
2.90567264181605	340324.325236216\\
2.90577264431611	340285.937063942\\
2.90587264681617	340246.975933874\\
2.90597264931623	340208.5877616\\
2.9060726518163	340170.199589326\\
2.90617265431636	340131.238459257\\
2.90627265681642	340092.850286983\\
2.90637265931648	340053.889156914\\
2.90647266181655	340015.500984641\\
2.90657266431661	339976.539854572\\
2.90667266681667	339938.151682298\\
2.90677266931673	339899.190552229\\
2.9068726718168	339860.22942216\\
2.90697267431686	339821.841249886\\
2.90707267681692	339782.880119817\\
2.90717267931698	339744.491947544\\
2.90727268181705	339705.530817475\\
2.90737268431711	339666.569687406\\
2.90747268681717	339627.608557337\\
2.90757268931723	339589.220385063\\
2.9076726918173	339550.259254994\\
2.90777269431736	339511.298124925\\
2.90787269681742	339472.336994857\\
2.90797269931748	339433.948822583\\
2.90807270181755	339394.987692514\\
2.90817270431761	339356.026562445\\
2.90827270681767	339317.065432376\\
2.90837270931773	339278.104302307\\
2.9084727118178	339239.143172238\\
2.90857271431786	339200.182042169\\
2.90867271681792	339161.220912101\\
2.90877271931798	339122.259782032\\
2.90887272181805	339083.298651963\\
2.90897272431811	339044.337521894\\
2.90907272681817	339005.376391825\\
2.90917272931823	338966.415261756\\
2.9092727318183	338927.454131687\\
2.90937273431836	338888.493001618\\
2.90947273681842	338849.531871549\\
2.90957273931848	338810.570741481\\
2.90967274181855	338771.609611412\\
2.90977274431861	338732.648481343\\
2.90987274681867	338693.114393479\\
2.90997274931873	338654.15326341\\
2.9100727518188	338615.192133341\\
2.91017275431886	338576.231003272\\
2.91027275681892	338536.696915408\\
2.91037275931898	338497.735785339\\
2.91047276181905	338458.77465527\\
2.91057276431911	338419.240567406\\
2.91067276681917	338380.279437337\\
2.91077276931923	338341.318307268\\
2.9108727718193	338301.784219404\\
2.91097277431936	338262.823089335\\
2.91107277681942	338223.861959267\\
2.91117277931948	338184.327871403\\
2.91127278181955	338145.366741334\\
2.91137278431961	338105.83265347\\
2.91147278681967	338066.871523401\\
2.91157278931973	338027.337435537\\
2.9116727918198	337988.376305468\\
2.91177279431986	337948.842217604\\
2.91187279681992	337909.30812974\\
2.91197279931998	337870.346999671\\
2.91207280182005	337830.812911807\\
2.91217280432011	337791.851781738\\
2.91227280682017	337752.317693874\\
2.91237280932023	337712.78360601\\
2.9124728118203	337673.822475941\\
2.91257281432036	337634.288388077\\
2.91267281682042	337594.754300213\\
2.91277281932048	337555.220212349\\
2.91287282182055	337516.25908228\\
2.91297282432061	337476.724994416\\
2.91307282682067	337437.190906552\\
2.91317282932073	337397.656818688\\
2.9132728318208	337358.122730824\\
2.91337283432086	337318.58864296\\
2.91347283682092	337279.054555096\\
2.91357283932098	337240.093425027\\
2.91367284182105	337200.559337163\\
2.91377284432111	337161.025249299\\
2.91387284682117	337121.491161435\\
2.91397284932123	337081.957073571\\
2.9140728518213	337042.422985707\\
2.91417285432136	337002.888897843\\
2.91427285682142	336963.354809979\\
2.91437285932148	336923.820722115\\
2.91447286182155	336883.713676455\\
2.91457286432161	336844.179588592\\
2.91467286682167	336804.645500727\\
2.91477286932173	336765.111412863\\
2.9148728718218	336725.577324999\\
2.91497287432186	336686.043237135\\
2.91507287682192	336646.509149271\\
2.91517287932198	336606.402103612\\
2.91527288182205	336566.868015748\\
2.91537288432211	336527.333927884\\
2.91547288682217	336487.79984002\\
2.91557288932223	336447.692794361\\
2.9156728918223	336408.158706497\\
2.91577289432236	336368.624618633\\
2.91587289682242	336328.517572974\\
2.91597289932248	336288.98348511\\
2.91607290182255	336248.876439451\\
2.91617290432261	336209.342351587\\
2.91627290682267	336169.808263723\\
2.91637290932273	336129.701218063\\
2.9164729118228	336090.167130199\\
2.91657291432286	336050.06008454\\
2.91667291682292	336010.525996676\\
2.91677291932298	335970.418951017\\
2.91687292182305	335930.884863153\\
2.91697292432311	335890.777817494\\
2.91707292682317	335850.670771835\\
2.91717292932323	335811.136683971\\
2.9172729318233	335771.029638311\\
2.91737293432336	335731.495550447\\
2.91747293682342	335691.388504788\\
2.91757293932348	335651.281459129\\
2.91767294182355	335611.747371265\\
2.91777294432361	335571.640325606\\
2.91787294682367	335531.533279947\\
2.91797294932373	335491.426234288\\
2.9180729518238	335451.892146424\\
2.91817295432386	335411.785100764\\
2.91827295682392	335371.678055105\\
2.91837295932398	335331.571009446\\
2.91847296182405	335291.463963787\\
2.91857296432411	335251.356918128\\
2.91867296682417	335211.822830264\\
2.91877296932423	335171.715784605\\
2.9188729718243	335131.608738945\\
2.91897297432436	335091.501693286\\
2.91907297682442	335051.394647627\\
2.91917297932448	335011.287601968\\
2.91927298182455	334971.180556309\\
2.91937298432461	334931.07351065\\
2.91947298682467	334890.966464991\\
2.91957298932473	334850.859419331\\
2.9196729918248	334810.752373672\\
2.91977299432486	334770.072370218\\
2.91987299682492	334729.965324559\\
2.91997299932498	334689.8582789\\
2.92007300182505	334649.75123324\\
2.92017300432511	334609.644187581\\
2.92027300682517	334569.537141922\\
2.92037300932523	334529.430096263\\
2.9204730118253	334488.750092809\\
2.92057301432536	334448.64304715\\
2.92067301682542	334408.53600149\\
2.92077301932548	334367.855998036\\
2.92087302182555	334327.748952377\\
2.92097302432561	334287.641906718\\
2.92107302682567	334247.534861059\\
2.92117302932573	334206.854857604\\
2.9212730318258	334166.747811945\\
2.92137303432586	334126.067808491\\
2.92147303682592	334085.960762832\\
2.92157303932598	334045.853717173\\
2.92167304182605	334005.173713718\\
2.92177304432611	333965.066668059\\
2.92187304682617	333924.386664605\\
2.92197304932623	333884.279618946\\
2.9220730518263	333843.599615491\\
2.92217305432636	333803.492569832\\
2.92227305682642	333762.812566378\\
2.92237305932648	333722.132562924\\
2.92247306182655	333682.025517264\\
2.92257306432661	333641.34551381\\
2.92267306682667	333601.238468151\\
2.92277306932673	333560.558464697\\
2.9228730718268	333519.878461242\\
2.92297307432686	333479.771415583\\
2.92307307682692	333439.091412129\\
2.92317307932698	333398.411408675\\
2.92327308182705	333357.73140522\\
2.92337308432711	333317.624359561\\
2.92347308682717	333276.944356107\\
2.92357308932723	333236.264352653\\
2.9236730918273	333195.584349198\\
2.92377309432736	333154.904345744\\
2.92387309682742	333114.22434229\\
2.92397309932748	333074.117296631\\
2.92407310182755	333033.437293176\\
2.92417310432761	332992.757289722\\
2.92427310682767	332952.077286268\\
2.92437310932773	332911.397282814\\
2.9244731118278	332870.717279359\\
2.92457311432786	332830.037275905\\
2.92467311682792	332789.357272451\\
2.92477311932798	332748.677268996\\
2.92487312182805	332707.997265542\\
2.92497312432811	332667.317262088\\
2.92507312682817	332626.637258634\\
2.92517312932823	332585.957255179\\
2.9252731318283	332545.277251725\\
2.92537313432836	332504.024290476\\
2.92547313682842	332463.344287021\\
2.92557313932848	332422.664283567\\
2.92567314182855	332381.984280113\\
2.92577314432861	332341.304276658\\
2.92587314682867	332300.051315409\\
2.92597314932873	332259.371311955\\
2.9260731518288	332218.6913085\\
2.92617315432886	332178.011305046\\
2.92627315682892	332136.758343797\\
2.92637315932898	332096.078340342\\
2.92647316182905	332055.398336888\\
2.92657316432911	332014.145375639\\
2.92667316682917	331973.465372184\\
2.92677316932923	331932.78536873\\
2.9268731718293	331891.532407481\\
2.92697317432936	331850.852404026\\
2.92707317682942	331809.599442777\\
2.92717317932948	331768.919439323\\
2.92727318182955	331727.666478073\\
2.92737318432961	331686.986474619\\
2.92747318682967	331645.73351337\\
2.92757318932973	331605.053509915\\
2.9276731918298	331563.800548666\\
2.92777319432986	331523.120545212\\
2.92787319682992	331481.867583962\\
2.92797319932998	331441.187580508\\
2.92807320183005	331399.934619258\\
2.92817320433011	331358.681658009\\
2.92827320683017	331318.001654555\\
2.92837320933023	331276.748693305\\
2.9284732118303	331235.495732056\\
2.92857321433036	331194.815728602\\
2.92867321683042	331153.562767352\\
2.92877321933048	331112.309806103\\
2.92887322183055	331071.056844853\\
2.92897322433061	331030.376841399\\
2.92907322683067	330989.12388015\\
2.92917322933073	330947.8709189\\
2.9292732318308	330906.617957651\\
2.92937323433086	330865.364996401\\
2.92947323683092	330824.112035152\\
2.92957323933098	330782.859073903\\
2.92967324183105	330742.179070448\\
2.92977324433111	330700.926109199\\
2.92987324683117	330659.673147949\\
2.92997324933123	330618.4201867\\
2.9300732518313	330577.167225451\\
2.93017325433136	330535.914264201\\
2.93027325683142	330494.661302952\\
2.93037325933148	330453.408341702\\
2.93047326183155	330412.155380453\\
2.93057326433161	330370.902419203\\
2.93067326683167	330329.076500159\\
2.93077326933173	330287.823538909\\
2.9308732718318	330246.57057766\\
2.93097327433186	330205.317616411\\
2.93107327683192	330164.064655161\\
2.93117327933198	330122.811693912\\
2.93127328183205	330081.558732662\\
2.93137328433211	330039.732813618\\
2.93147328683217	329998.479852368\\
2.93157328933223	329957.226891119\\
2.9316732918323	329915.97392987\\
2.93177329433236	329874.148010825\\
2.93187329683242	329832.895049576\\
2.93197329933248	329791.642088326\\
2.93207330183255	329749.816169282\\
2.93217330433261	329708.563208032\\
2.93227330683267	329667.310246783\\
2.93237330933273	329625.484327738\\
2.9324733118328	329584.231366489\\
2.93257331433286	329542.405447444\\
2.93267331683292	329501.152486195\\
2.93277331933298	329459.899524945\\
2.93287332183305	329418.073605901\\
2.93297332433311	329376.820644651\\
2.93307332683317	329334.994725607\\
2.93317332933323	329293.741764358\\
2.9332733318333	329251.915845313\\
2.93337333433336	329210.089926268\\
2.93347333683342	329168.836965019\\
2.93357333933348	329127.011045975\\
2.93367334183355	329085.758084725\\
2.93377334433361	329043.93216568\\
2.93387334683367	329002.106246636\\
2.93397334933373	328960.853285387\\
2.9340733518338	328919.027366342\\
2.93417335433386	328877.201447297\\
2.93427335683392	328835.948486048\\
2.93437335933398	328794.122567003\\
2.93447336183405	328752.296647959\\
2.93457336433411	328710.470728914\\
2.93467336683417	328669.217767665\\
2.93477336933423	328627.39184862\\
2.9348733718343	328585.565929576\\
2.93497337433436	328543.740010531\\
2.93507337683442	328501.914091487\\
2.93517337933448	328460.088172442\\
2.93527338183455	328418.835211193\\
2.93537338433461	328377.009292148\\
2.93547338683467	328335.183373104\\
2.93557338933473	328293.357454059\\
2.9356733918348	328251.531535015\\
2.93577339433486	328209.70561597\\
2.93587339683492	328167.879696925\\
2.93597339933498	328126.053777881\\
2.93607340183505	328084.227858836\\
2.93617340433511	328042.401939792\\
2.93627340683517	328000.576020747\\
2.93637340933523	327958.750101703\\
2.9364734118353	327916.351224863\\
2.93657341433536	327874.525305818\\
2.93667341683542	327832.699386774\\
2.93677341933548	327790.873467729\\
2.93687342183555	327749.047548685\\
2.93697342433561	327707.22162964\\
2.93707342683567	327665.395710596\\
2.93717342933573	327622.996833756\\
2.9372734318358	327581.170914711\\
2.93737343433586	327539.344995667\\
2.93747343683592	327497.519076622\\
2.93757343933598	327455.120199783\\
2.93767344183605	327413.294280738\\
2.93777344433611	327371.468361694\\
2.93787344683617	327329.069484854\\
2.93797344933623	327287.243565809\\
2.9380734518363	327245.417646765\\
2.93817345433636	327203.018769925\\
2.93827345683642	327161.192850881\\
2.93837345933648	327118.793974041\\
2.93847346183655	327076.968054996\\
2.93857346433661	327034.569178157\\
2.93867346683667	326992.743259112\\
2.93877346933673	326950.917340068\\
2.9388734718368	326908.518463228\\
2.93897347433686	326866.692544183\\
2.93907347683692	326824.293667344\\
2.93917347933698	326781.894790504\\
2.93927348183705	326740.068871459\\
2.93937348433711	326697.66999462\\
2.93947348683717	326655.844075575\\
2.93957348933723	326613.445198736\\
2.9396734918373	326571.046321896\\
2.93977349433736	326529.220402851\\
2.93987349683742	326486.821526012\\
2.93997349933748	326444.422649172\\
2.94007350183755	326402.596730127\\
2.94017350433761	326360.197853288\\
2.94027350683767	326317.798976448\\
2.94037350933773	326275.400099608\\
2.9404735118378	326233.574180564\\
2.94057351433786	326191.175303724\\
2.94067351683792	326148.776426884\\
2.94077351933798	326106.377550045\\
2.94087352183805	326063.978673205\\
2.94097352433811	326021.579796365\\
2.94107352683817	325979.753877321\\
2.94117352933823	325937.355000481\\
2.9412735318383	325894.956123641\\
2.94137353433836	325852.557246802\\
2.94147353683842	325810.158369962\\
2.94157353933848	325767.759493122\\
2.94167354183855	325725.360616283\\
2.94177354433861	325682.961739443\\
2.94187354683867	325640.562862603\\
2.94197354933873	325598.163985764\\
2.9420735518388	325555.765108924\\
2.94217355433886	325513.366232084\\
2.94227355683892	325470.967355245\\
2.94237355933898	325428.568478405\\
2.94247356183905	325386.169601565\\
2.94257356433911	325343.197766931\\
2.94267356683917	325300.798890091\\
2.94277356933923	325258.400013251\\
2.9428735718393	325216.001136411\\
2.94297357433936	325173.602259572\\
2.94307357683942	325131.203382732\\
2.94317357933948	325088.231548097\\
2.94327358183955	325045.832671258\\
2.94337358433961	325003.433794418\\
2.94347358683967	324961.034917578\\
2.94357358933973	324918.063082943\\
2.9436735918398	324875.664206104\\
2.94377359433986	324833.265329264\\
2.94387359683992	324790.293494629\\
2.94397359933998	324747.89461779\\
2.94407360184005	324705.49574095\\
2.94417360434011	324662.523906315\\
2.94427360684017	324620.125029475\\
2.94437360934023	324577.153194841\\
2.9444736118403	324534.754318001\\
2.94457361434036	324491.782483366\\
2.94467361684042	324449.383606526\\
2.94477361934048	324406.984729687\\
2.94487362184055	324364.012895052\\
2.94497362434061	324321.614018212\\
2.94507362684067	324278.642183577\\
2.94517362934073	324235.670348943\\
2.9452736318408	324193.271472103\\
2.94537363434086	324150.299637468\\
2.94547363684092	324107.900760628\\
2.94557363934098	324064.928925994\\
2.94567364184105	324021.957091359\\
2.94577364434111	323979.558214519\\
2.94587364684117	323936.586379884\\
2.94597364934123	323893.614545249\\
2.9460736518413	323851.21566841\\
2.94617365434136	323808.243833775\\
2.94627365684142	323765.27199914\\
2.94637365934148	323722.8731223\\
2.94647366184155	323679.901287666\\
2.94657366434161	323636.929453031\\
2.94667366684167	323593.957618396\\
2.94677366934173	323550.985783761\\
2.9468736718418	323508.586906922\\
2.94697367434186	323465.615072287\\
2.94707367684192	323422.643237652\\
2.94717367934198	323379.671403017\\
2.94727368184205	323336.699568382\\
2.94737368434211	323293.727733748\\
2.94747368684217	323250.755899113\\
2.94757368934223	323207.784064478\\
2.9476736918423	323164.812229843\\
2.94777369434236	323121.840395208\\
2.94787369684242	323078.868560573\\
2.94797369934248	323035.896725939\\
2.94807370184255	322992.924891304\\
2.94817370434261	322949.953056669\\
2.94827370684267	322906.981222034\\
2.94837370934273	322864.009387399\\
2.9484737118428	322821.037552765\\
2.94857371434286	322778.06571813\\
2.94867371684292	322735.093883495\\
2.94877371934298	322692.12204886\\
2.94887372184305	322649.150214225\\
2.94897372434311	322605.605421795\\
2.94907372684317	322562.633587161\\
2.94917372934323	322519.661752526\\
2.9492737318433	322476.689917891\\
2.94937373434336	322433.718083256\\
2.94947373684342	322390.173290826\\
2.94957373934348	322347.201456191\\
2.94967374184355	322304.229621557\\
2.94977374434361	322261.257786922\\
2.94987374684367	322217.712994492\\
2.94997374934373	322174.741159857\\
2.9500737518438	322131.769325222\\
2.95017375434386	322088.224532792\\
2.95027375684392	322045.252698157\\
2.95037375934398	322001.707905727\\
2.95047376184405	321958.736071093\\
2.95057376434411	321915.764236458\\
2.95067376684417	321872.219444028\\
2.95077376934423	321829.247609393\\
2.9508737718443	321785.702816963\\
2.95097377434436	321742.730982328\\
2.95107377684442	321699.186189898\\
2.95117377934448	321656.214355264\\
2.95127378184455	321612.669562834\\
2.95137378434461	321569.697728199\\
2.95147378684467	321526.152935769\\
2.95157378934473	321483.181101134\\
2.9516737918448	321439.636308704\\
2.95177379434486	321396.091516274\\
2.95187379684492	321353.119681639\\
2.95197379934498	321309.574889209\\
2.95207380184505	321266.03009678\\
2.95217380434511	321223.058262145\\
2.95227380684517	321179.513469715\\
2.95237380934523	321135.968677285\\
2.9524738118453	321092.99684265\\
2.95257381434536	321049.45205022\\
2.95267381684542	321005.90725779\\
2.95277381934548	320962.36246536\\
2.95287382184555	320919.390630725\\
2.95297382434561	320875.845838295\\
2.95307382684567	320832.301045866\\
2.95317382934573	320788.756253436\\
2.9532738318458	320745.211461006\\
2.95337383434586	320702.239626371\\
2.95347383684592	320658.694833941\\
2.95357383934598	320615.150041511\\
2.95367384184605	320571.605249081\\
2.95377384434611	320528.060456651\\
2.95387384684617	320484.515664221\\
2.95397384934623	320440.970871791\\
2.9540738518463	320397.426079361\\
2.95417385434636	320353.881286931\\
2.95427385684642	320310.336494501\\
2.95437385934648	320266.791702071\\
2.95447386184655	320223.246909641\\
2.95457386434661	320179.702117211\\
2.95467386684667	320136.157324782\\
2.95477386934673	320092.612532352\\
2.9548738718468	320049.067739922\\
2.95497387434686	320005.522947492\\
2.95507387684692	319961.978155062\\
2.95517387934698	319917.860404837\\
2.95527388184705	319874.315612407\\
2.95537388434711	319830.770819977\\
2.95547388684717	319787.226027547\\
2.95557388934723	319743.681235117\\
2.9556738918473	319700.136442687\\
2.95577389434736	319656.018692462\\
2.95587389684742	319612.473900032\\
2.95597389934748	319568.929107602\\
2.95607390184755	319525.384315172\\
2.95617390434761	319481.266564947\\
2.95627390684767	319437.721772517\\
2.95637390934773	319394.176980087\\
2.9564739118478	319350.059229862\\
2.95657391434786	319306.514437432\\
2.95667391684792	319262.969645002\\
2.95677391934798	319218.851894777\\
2.95687392184805	319175.307102347\\
2.95697392434811	319131.189352122\\
2.95707392684817	319087.644559692\\
2.95717392934823	319044.099767262\\
2.9572739318483	318999.982017037\\
2.95737393434836	318956.437224607\\
2.95747393684842	318912.319474382\\
2.95757393934848	318868.774681952\\
2.95767394184855	318824.656931727\\
2.95777394434861	318781.112139297\\
2.95787394684867	318736.994389072\\
2.95797394934873	318693.449596642\\
2.9580739518488	318649.331846417\\
2.95817395434886	318605.214096192\\
2.95827395684892	318561.669303762\\
2.95837395934898	318517.551553537\\
2.95847396184905	318474.006761107\\
2.95857396434911	318429.889010882\\
2.95867396684917	318385.771260657\\
2.95877396934923	318342.226468227\\
2.9588739718493	318298.108718002\\
2.95897397434936	318253.990967777\\
2.95907397684942	318209.873217552\\
2.95917397934948	318166.328425122\\
2.95927398184955	318122.210674897\\
2.95937398434961	318078.092924672\\
2.95947398684967	318033.975174447\\
2.95957398934973	317990.430382017\\
2.9596739918498	317946.312631792\\
2.95977399434986	317902.194881566\\
2.95987399684992	317858.077131341\\
2.95997399934998	317813.959381116\\
2.96007400185005	317769.841630891\\
2.96017400435011	317725.723880666\\
2.96027400685017	317682.179088236\\
2.96037400935023	317638.061338011\\
2.9604740118503	317593.943587786\\
2.96057401435036	317549.825837561\\
2.96067401685042	317505.708087336\\
2.96077401935048	317461.590337111\\
2.96087402185055	317417.472586886\\
2.96097402435061	317373.354836661\\
2.96107402685067	317329.237086436\\
2.96117402935073	317285.119336211\\
2.9612740318508	317241.001585985\\
2.96137403435086	317196.88383576\\
2.96147403685092	317152.766085535\\
2.96157403935098	317108.64833531\\
2.96167404185105	317063.95762729\\
2.96177404435111	317019.839877065\\
2.96187404685117	316975.72212684\\
2.96197404935123	316931.604376615\\
2.9620740518513	316887.48662639\\
2.96217405435136	316843.368876165\\
2.96227405685142	316798.678168145\\
2.96237405935148	316754.560417919\\
2.96247406185155	316710.442667694\\
2.96257406435161	316666.324917469\\
2.96267406685167	316622.207167244\\
2.96277406935173	316577.516459224\\
2.9628740718518	316533.398708999\\
2.96297407435186	316489.280958774\\
2.96307407685192	316444.590250754\\
2.96317407935198	316400.472500529\\
2.96327408185205	316356.354750304\\
2.96337408435211	316311.664042283\\
2.96347408685217	316267.546292058\\
2.96357408935223	316223.428541833\\
2.9636740918523	316178.737833813\\
2.96377409435236	316134.620083588\\
2.96387409685242	316089.929375568\\
2.96397409935248	316045.811625343\\
2.96407410185255	316001.120917322\\
2.96417410435261	315957.003167097\\
2.96427410685267	315912.885416872\\
2.96437410935273	315868.194708852\\
2.9644741118528	315823.504000832\\
2.96457411435286	315779.386250607\\
2.96467411685292	315734.695542587\\
2.96477411935298	315690.577792362\\
2.96487412185305	315645.887084341\\
2.96497412435311	315601.769334116\\
2.96507412685317	315557.078626096\\
2.96517412935323	315512.387918076\\
2.9652741318533	315468.270167851\\
2.96537413435336	315423.579459831\\
2.96547413685342	315379.461709605\\
2.96557413935348	315334.771001585\\
2.96567414185355	315290.080293565\\
2.96577414435361	315245.96254334\\
2.96587414685367	315201.27183532\\
2.96597414935373	315156.5811273\\
2.9660741518538	315111.890419279\\
2.96617415435386	315067.772669054\\
2.96627415685392	315023.081961034\\
2.96637415935398	314978.391253014\\
2.96647416185405	314933.700544994\\
2.96657416435411	314889.009836974\\
2.96667416685417	314844.892086748\\
2.96677416935423	314800.201378728\\
2.9668741718543	314755.510670708\\
2.96697417435436	314710.819962688\\
2.96707417685442	314666.129254668\\
2.96717417935448	314621.438546647\\
2.96727418185455	314576.747838627\\
2.96737418435461	314532.057130607\\
2.96747418685467	314487.366422587\\
2.96757418935473	314442.675714567\\
2.9676741918548	314397.985006546\\
2.96777419435486	314353.867256321\\
2.96787419685492	314309.176548301\\
2.96797419935498	314264.485840281\\
2.96807420185505	314219.222174466\\
2.96817420435511	314174.531466445\\
2.96827420685517	314129.840758425\\
2.96837420935523	314085.150050405\\
2.9684742118553	314040.459342385\\
2.96857421435536	313995.768634365\\
2.96867421685542	313951.077926344\\
2.96877421935548	313906.387218324\\
2.96887422185555	313861.696510304\\
2.96897422435561	313817.005802284\\
2.96907422685567	313771.742136468\\
2.96917422935573	313727.051428448\\
2.9692742318558	313682.360720428\\
2.96937423435586	313637.670012408\\
2.96947423685592	313592.979304388\\
2.96957423935598	313547.715638572\\
2.96967424185605	313503.024930552\\
2.96977424435611	313458.334222532\\
2.96987424685617	313413.643514512\\
2.96997424935623	313368.379848696\\
2.9700742518563	313323.689140676\\
2.97017425435636	313278.998432656\\
2.97027425685642	313233.734766841\\
2.97037425935648	313189.04405882\\
2.97047426185655	313144.3533508\\
2.97057426435661	313099.089684985\\
2.97067426685667	313054.398976965\\
2.97077426935673	313009.708268944\\
2.9708742718568	312964.444603129\\
2.97097427435686	312919.753895109\\
2.97107427685692	312874.490229293\\
2.97117427935698	312829.799521273\\
2.97127428185705	312784.535855458\\
2.97137428435711	312739.845147438\\
2.97147428685717	312694.581481622\\
2.97157428935723	312649.890773602\\
2.9716742918573	312604.627107787\\
2.97177429435736	312559.936399767\\
2.97187429685742	312514.672733951\\
2.97197429935748	312469.982025931\\
2.97207430185755	312424.718360116\\
2.97217430435761	312379.4546943\\
2.97227430685767	312334.76398628\\
2.97237430935773	312289.500320465\\
2.9724743118578	312244.809612445\\
2.97257431435786	312199.545946629\\
2.97267431685792	312154.282280814\\
2.97277431935798	312109.591572794\\
2.97287432185805	312064.327906979\\
2.97297432435811	312019.064241163\\
2.97307432685817	311974.373533143\\
2.97317432935823	311929.109867328\\
2.9732743318583	311883.846201512\\
2.97337433435836	311838.582535697\\
2.97347433685842	311793.891827677\\
2.97357433935848	311748.628161861\\
2.97367434185855	311703.364496046\\
2.97377434435861	311658.100830231\\
2.97387434685867	311612.837164415\\
2.97397434935873	311567.5734986\\
2.9740743518588	311522.88279058\\
2.97417435435886	311477.619124765\\
2.97427435685892	311432.355458949\\
2.97437435935898	311387.091793134\\
2.97447436185905	311341.828127319\\
2.97457436435911	311296.564461503\\
2.97467436685917	311251.300795688\\
2.97477436935923	311206.037129873\\
2.9748743718593	311160.773464057\\
2.97497437435936	311115.509798242\\
2.97507437685942	311070.246132427\\
2.97517437935948	311024.982466611\\
2.97527438185955	310979.718800796\\
2.97537438435961	310934.455134981\\
2.97547438685967	310889.191469165\\
2.97557438935973	310843.92780335\\
2.9756743918598	310798.664137535\\
2.97577439435986	310753.400471719\\
2.97587439685992	310708.136805904\\
2.97597439935998	310662.873140089\\
2.97607440186005	310617.609474273\\
2.97617440436011	310572.345808458\\
2.97627440686017	310527.082142643\\
2.97637440936023	310481.245519032\\
2.9764744118603	310435.981853217\\
2.97657441436036	310390.718187401\\
2.97667441686042	310345.454521586\\
2.97677441936048	310300.190855771\\
2.97687442186055	310254.35423216\\
2.97697442436061	310209.090566345\\
2.97707442686067	310163.82690053\\
2.97717442936073	310118.563234714\\
2.9772744318608	310072.726611104\\
2.97737443436086	310027.462945288\\
2.97747443686092	309982.199279473\\
2.97757443936098	309936.362655863\\
2.97767444186105	309891.098990047\\
2.97777444436111	309845.835324232\\
2.97787444686117	309799.998700621\\
2.97797444936123	309754.735034806\\
2.9780744518613	309709.471368991\\
2.97817445436136	309663.63474538\\
2.97827445686142	309618.371079565\\
2.97837445936148	309572.534455955\\
2.97847446186155	309527.270790139\\
2.97857446436161	309482.007124324\\
2.97867446686167	309436.170500713\\
2.97877446936173	309390.906834898\\
2.9788744718618	309345.070211288\\
2.97897447436186	309299.806545472\\
2.97907447686192	309253.969921862\\
2.97917447936198	309208.706256047\\
2.97927448186205	309162.869632436\\
2.97937448436211	309117.605966621\\
2.97947448686217	309071.76934301\\
2.97957448936223	309025.9327194\\
2.9796744918623	308980.669053584\\
2.97977449436236	308934.832429974\\
2.97987449686242	308889.568764159\\
2.97997449936248	308843.732140548\\
2.98007450186255	308797.895516938\\
2.98017450436261	308752.631851122\\
2.98027450686267	308706.795227512\\
2.98037450936273	308660.958603901\\
2.9804745118628	308615.694938086\\
2.98057451436286	308569.858314476\\
2.98067451686292	308524.021690865\\
2.98077451936298	308478.75802505\\
2.98087452186305	308432.921401439\\
2.98097452436311	308387.084777829\\
2.98107452686317	308341.248154218\\
2.98117452936323	308295.984488403\\
2.9812745318633	308250.147864793\\
2.98137453436336	308204.311241182\\
2.98147453686342	308158.474617572\\
2.98157453936348	308112.637993961\\
2.98167454186355	308066.801370351\\
2.98177454436361	308021.537704535\\
2.98187454686367	307975.701080925\\
2.98197454936373	307929.864457315\\
2.9820745518638	307884.027833704\\
2.98217455436386	307838.191210094\\
2.98227455686392	307792.354586483\\
2.98237455936398	307746.517962873\\
2.98247456186405	307700.681339262\\
2.98257456436411	307654.844715652\\
2.98267456686417	307609.008092041\\
2.98277456936423	307563.171468431\\
2.9828745718643	307517.33484482\\
2.98297457436436	307471.49822121\\
2.98307457686442	307425.661597599\\
2.98317457936448	307379.824973989\\
2.98327458186455	307333.988350378\\
2.98337458436461	307288.151726768\\
2.98347458686467	307242.315103158\\
2.98357458936473	307196.478479547\\
2.9836745918648	307150.641855937\\
2.98377459436486	307104.805232326\\
2.98387459686492	307058.968608716\\
2.98397459936498	307013.131985105\\
2.98407460186505	306966.7224037\\
2.98417460436511	306920.885780089\\
2.98427460686517	306875.049156479\\
2.98437460936523	306829.212532868\\
2.9844746118653	306783.375909258\\
2.98457461436536	306737.539285647\\
2.98467461686542	306691.129704242\\
2.98477461936548	306645.293080631\\
2.98487462186555	306599.456457021\\
2.98497462436561	306553.61983341\\
2.98507462686567	306507.210252005\\
2.98517462936573	306461.373628394\\
2.9852746318658	306415.537004784\\
2.98537463436586	306369.127423378\\
2.98547463686592	306323.290799768\\
2.98557463936598	306277.454176157\\
2.98567464186605	306231.044594752\\
2.98577464436611	306185.207971141\\
2.98587464686617	306139.371347531\\
2.98597464936623	306092.961766125\\
2.9860746518663	306047.125142515\\
2.98617465436636	306001.288518904\\
2.98627465686642	305954.878937499\\
2.98637465936648	305909.042313888\\
2.98647466186655	305862.632732483\\
2.98657466436661	305816.796108872\\
2.98667466686667	305770.386527466\\
2.98677466936673	305724.549903856\\
2.9868746718668	305678.14032245\\
2.98697467436686	305632.30369884\\
2.98707467686692	305585.894117434\\
2.98717467936698	305540.057493824\\
2.98727468186705	305493.647912418\\
2.98737468436711	305447.811288808\\
2.98747468686717	305401.401707402\\
2.98757468936723	305355.565083792\\
2.9876746918673	305309.155502386\\
2.98777469436736	305262.745920981\\
2.98787469686742	305216.90929737\\
2.98797469936748	305170.499715964\\
2.98807470186755	305124.663092354\\
2.98817470436761	305078.253510948\\
2.98827470686767	305031.843929543\\
2.98837470936773	304986.007305932\\
2.9884747118678	304939.597724527\\
2.98857471436786	304893.188143121\\
2.98867471686792	304847.351519511\\
2.98877471936798	304800.941938105\\
2.98887472186805	304754.5323567\\
2.98897472436811	304708.122775294\\
2.98907472686817	304662.286151683\\
2.98917472936823	304615.876570278\\
2.9892747318683	304569.466988872\\
2.98937473436836	304523.057407467\\
2.98947473686842	304476.647826061\\
2.98957473936848	304430.811202451\\
2.98967474186855	304384.401621045\\
2.98977474436861	304337.992039639\\
2.98987474686867	304291.582458234\\
2.98997474936873	304245.172876828\\
2.9900747518688	304198.763295423\\
2.99017475436886	304152.926671812\\
2.99027475686892	304106.517090407\\
2.99037475936898	304060.107509001\\
2.99047476186905	304013.697927595\\
2.99057476436911	303967.28834619\\
2.99067476686917	303920.878764784\\
2.99077476936923	303874.469183379\\
2.9908747718693	303828.059601973\\
2.99097477436936	303781.650020567\\
2.99107477686942	303735.240439162\\
2.99117477936948	303688.830857756\\
2.99127478186955	303642.421276351\\
2.99137478436961	303596.011694945\\
2.99147478686967	303549.602113539\\
2.99157478936973	303503.192532134\\
2.9916747918698	303456.782950728\\
2.99177479436986	303410.373369323\\
2.99187479686992	303363.963787917\\
2.99197479936998	303317.554206511\\
2.99207480187005	303270.571667311\\
2.99217480437011	303224.162085905\\
2.99227480687017	303177.752504499\\
2.99237480937023	303131.342923094\\
2.9924748118703	303084.933341688\\
2.99257481437036	303038.523760283\\
2.99267481687042	302992.114178877\\
2.99277481937048	302945.131639676\\
2.99287482187055	302898.722058271\\
2.99297482437061	302852.312476865\\
2.99307482687067	302805.90289546\\
2.99317482937073	302758.920356259\\
2.9932748318708	302712.510774853\\
2.99337483437086	302666.101193448\\
2.99347483687092	302619.691612042\\
2.99357483937098	302572.709072841\\
2.99367484187105	302526.299491436\\
2.99377484437111	302479.88991003\\
2.99387484687117	302433.480328625\\
2.99397484937123	302386.497789424\\
2.9940748518713	302340.088208018\\
2.99417485437136	302293.678626613\\
2.99427485687142	302246.696087412\\
2.99437485937148	302200.286506006\\
2.99447486187155	302153.303966806\\
2.99457486437161	302106.8943854\\
2.99467486687167	302060.484803994\\
2.99477486937173	302013.502264794\\
2.9948748718718	301967.092683388\\
2.99497487437186	301920.110144187\\
2.99507487687192	301873.700562782\\
2.99517487937198	301826.718023581\\
2.99527488187205	301780.308442175\\
2.99537488437211	301733.325902975\\
2.99547488687217	301686.916321569\\
2.99557488937223	301639.933782368\\
2.9956748918723	301593.524200963\\
2.99577489437236	301546.541661762\\
2.99587489687242	301500.132080356\\
2.99597489937248	301453.149541156\\
2.99607490187255	301406.73995975\\
2.99617490437261	301359.757420549\\
2.99627490687267	301313.347839144\\
2.99637490937273	301266.365299943\\
2.9964749118728	301219.382760742\\
2.99657491437286	301172.973179337\\
2.99667491687292	301125.990640136\\
2.99677491937298	301079.008100935\\
2.99687492187305	301032.59851953\\
2.99697492437311	300985.615980329\\
2.99707492687317	300938.633441128\\
2.99717492937323	300892.223859723\\
2.9972749318733	300845.241320522\\
2.99737493437336	300798.258781321\\
2.99747493687342	300751.849199916\\
2.99757493937348	300704.866660715\\
2.99767494187355	300657.884121514\\
2.99777494437361	300610.901582313\\
2.99787494687367	300564.492000908\\
2.99797494937373	300517.509461707\\
2.9980749518738	300470.526922506\\
2.99817495437386	300423.544383306\\
2.99827495687392	300376.561844105\\
2.99837495937398	300330.152262699\\
2.99847496187405	300283.169723499\\
2.99857496437411	300236.187184298\\
2.99867496687417	300189.204645097\\
2.99877496937423	300142.222105896\\
2.9988749718743	300095.239566696\\
2.99897497437436	300048.257027495\\
2.99907497687442	300001.274488294\\
2.99917497937448	299954.864906889\\
2.99927498187455	299907.882367688\\
2.99937498437461	299860.899828487\\
2.99947498687467	299813.917289286\\
2.99957498937473	299766.934750086\\
2.9996749918748	299719.952210885\\
2.99977499437486	299672.969671684\\
2.99987499687492	299625.987132484\\
2.99997499937498	299579.004593283\\
3.00007500187505	299532.022054082\\
3.00017500437511	299485.039514881\\
3.00027500687517	299438.056975681\\
3.00037500937523	299391.07443648\\
3.0004750118753	299344.091897279\\
3.00057501437536	299297.109358078\\
3.00067501687542	299250.126818878\\
3.00077501937548	299203.144279677\\
3.00087502187555	299155.588782681\\
3.00097502437561	299108.60624348\\
3.00107502687567	299061.62370428\\
3.00117502937573	299014.641165079\\
3.0012750318758	298967.658625878\\
3.00137503437586	298920.676086677\\
3.00147503687592	298873.693547477\\
3.00157503937598	298826.138050481\\
3.00167504187605	298779.15551128\\
3.00177504437611	298732.172972079\\
3.00187504687617	298685.190432879\\
3.00197504937623	298638.207893678\\
3.0020750518763	298590.652396682\\
3.00217505437636	298543.669857481\\
3.00227505687642	298496.687318281\\
3.00237505937648	298449.70477908\\
3.00247506187655	298402.149282084\\
3.00257506437661	298355.166742883\\
3.00267506687667	298308.184203683\\
3.00277506937673	298261.201664482\\
3.0028750718768	298213.646167486\\
3.00297507437686	298166.663628285\\
3.00307507687692	298119.681089085\\
3.00317507937698	298072.125592089\\
3.00327508187705	298025.143052888\\
3.00337508437711	297978.160513687\\
3.00347508687717	297930.605016691\\
3.00357508937723	297883.622477491\\
3.0036750918773	297836.066980495\\
3.00377509437736	297789.084441294\\
3.00387509687742	297742.101902093\\
3.00397509937748	297694.546405097\\
3.00407510187755	297647.563865897\\
3.00417510437761	297600.008368901\\
3.00427510687767	297553.0258297\\
3.00437510937773	297505.470332704\\
3.0044751118778	297458.487793504\\
3.00457511437786	297410.932296508\\
3.00467511687792	297363.949757307\\
3.00477511937798	297316.394260311\\
3.00487512187805	297269.41172111\\
3.00497512437811	297221.856224115\\
3.00507512687817	297174.873684914\\
3.00517512937823	297127.318187918\\
3.0052751318783	297080.335648717\\
3.00537513437836	297032.780151721\\
3.00547513687842	296985.224654726\\
3.00557513937848	296938.242115525\\
3.00567514187855	296890.686618529\\
3.00577514437861	296843.704079328\\
3.00587514687867	296796.148582332\\
3.00597514937873	296748.593085336\\
3.0060751518788	296701.610546136\\
3.00617515437886	296654.05504914\\
3.00627515687892	296606.499552144\\
3.00637515937898	296559.517012943\\
3.00647516187905	296511.961515947\\
3.00657516437911	296464.406018952\\
3.00667516687917	296417.423479751\\
3.00677516937923	296369.867982755\\
3.0068751718793	296322.312485759\\
3.00697517437936	296275.329946558\\
3.00707517687942	296227.774449563\\
3.00717517937948	296180.218952567\\
3.00727518187955	296132.663455571\\
3.00737518437961	296085.107958575\\
3.00747518687967	296038.125419374\\
3.00757518937973	295990.569922378\\
3.0076751918798	295943.014425383\\
3.00777519437986	295895.458928387\\
3.00787519687992	295847.903431391\\
3.00797519937998	295800.92089219\\
3.00807520188005	295753.365395194\\
3.00817520438011	295705.809898198\\
3.00827520688017	295658.254401203\\
3.00837520938023	295610.698904207\\
3.0084752118803	295563.143407211\\
3.00857521438036	295515.587910215\\
3.00867521688042	295468.032413219\\
3.00877521938048	295421.049874018\\
3.00887522188055	295373.494377022\\
3.00897522438061	295325.938880027\\
3.00907522688067	295278.383383031\\
3.00917522938073	295230.827886035\\
3.0092752318808	295183.272389039\\
3.00937523438086	295135.716892043\\
3.00947523688092	295088.161395047\\
3.00957523938098	295040.605898052\\
3.00967524188105	294993.050401056\\
3.00977524438111	294945.49490406\\
3.00987524688117	294897.939407064\\
3.00997524938123	294850.383910068\\
3.0100752518813	294802.828413072\\
3.01017525438136	294755.272916076\\
3.01027525688142	294707.71741908\\
3.01037525938148	294659.588964289\\
3.01047526188155	294612.033467294\\
3.01057526438161	294564.477970298\\
3.01067526688167	294516.922473302\\
3.01077526938173	294469.366976306\\
3.0108752718818	294421.81147931\\
3.01097527438186	294374.255982314\\
3.01107527688192	294326.700485319\\
3.01117527938198	294279.144988323\\
3.01127528188205	294231.016533532\\
3.01137528438211	294183.461036536\\
3.01147528688217	294135.90553954\\
3.01157528938223	294088.350042544\\
3.0116752918823	294040.794545548\\
3.01177529438236	293992.666090757\\
3.01187529688242	293945.110593761\\
3.01197529938248	293897.555096765\\
3.01207530188255	293849.99959977\\
3.01217530438261	293801.871144979\\
3.01227530688267	293754.315647983\\
3.01237530938273	293706.760150987\\
3.0124753118828	293659.204653991\\
3.01257531438286	293611.0761992\\
3.01267531688292	293563.520702204\\
3.01277531938298	293515.965205208\\
3.01287532188305	293467.836750417\\
3.01297532438311	293420.281253422\\
3.01307532688317	293372.725756426\\
3.01317532938323	293324.597301635\\
3.0132753318833	293277.041804639\\
3.01337533438336	293229.486307643\\
3.01347533688342	293181.357852852\\
3.01357533938348	293133.802355856\\
3.01367534188355	293085.673901065\\
3.01377534438361	293038.118404069\\
3.01387534688367	292990.562907073\\
3.01397534938373	292942.434452282\\
3.0140753518838	292894.878955287\\
3.01417535438386	292846.750500496\\
3.01427535688392	292799.1950035\\
3.01437535938398	292751.066548709\\
3.01447536188405	292703.511051713\\
3.01457536438411	292655.382596922\\
3.01467536688417	292607.827099926\\
3.01477536938423	292559.698645135\\
3.0148753718843	292512.143148139\\
3.01497537438436	292464.014693348\\
3.01507537688442	292416.459196352\\
3.01517537938448	292368.330741561\\
3.01527538188455	292320.775244566\\
3.01537538438461	292272.646789774\\
3.01547538688467	292225.091292779\\
3.01557538938473	292176.962837988\\
3.0156753918848	292129.407340992\\
3.01577539438486	292081.278886201\\
3.01587539688492	292033.15043141\\
3.01597539938498	291985.594934414\\
3.01607540188505	291937.466479623\\
3.01617540438511	291889.338024832\\
3.01627540688517	291841.782527836\\
3.01637540938523	291793.654073045\\
3.0164754118853	291745.525618254\\
3.01657541438536	291697.970121258\\
3.01667541688542	291649.841666467\\
3.01677541938548	291601.713211676\\
3.01687542188555	291554.15771468\\
3.01697542438561	291506.029259889\\
3.01707542688567	291457.900805098\\
3.01717542938573	291410.345308103\\
3.0172754318858	291362.216853312\\
3.01737543438586	291314.088398521\\
3.01747543688592	291265.95994373\\
3.01757543938598	291218.404446734\\
3.01767544188605	291170.275991943\\
3.01777544438611	291122.147537152\\
3.01787544688617	291074.019082361\\
3.01797544938623	291025.89062757\\
3.0180754518863	290978.335130574\\
3.01817545438636	290930.206675783\\
3.01827545688642	290882.078220992\\
3.01837545938648	290833.949766201\\
3.01847546188655	290785.82131141\\
3.01857546438661	290737.692856619\\
3.01867546688667	290690.137359623\\
3.01877546938673	290642.008904832\\
3.0188754718868	290593.880450041\\
3.01897547438686	290545.75199525\\
3.01907547688692	290497.623540459\\
3.01917547938698	290449.495085668\\
3.01927548188705	290401.366630877\\
3.01937548438711	290353.238176086\\
3.01947548688717	290305.109721295\\
3.01957548938723	290256.981266504\\
3.0196754918873	290208.852811713\\
3.01977549438736	290161.297314717\\
3.01987549688742	290113.168859926\\
3.01997549938748	290065.040405135\\
3.02007550188755	290016.911950344\\
3.02017550438761	289968.783495553\\
3.02027550688767	289920.655040762\\
3.02037550938773	289872.526585971\\
3.0204755118878	289824.39813118\\
3.02057551438786	289775.696718594\\
3.02067551688792	289727.568263803\\
3.02077551938798	289679.439809012\\
3.02087552188805	289631.311354221\\
3.02097552438811	289583.18289943\\
3.02107552688817	289535.054444639\\
3.02117552938823	289486.925989848\\
3.0212755318883	289438.797535057\\
3.02137553438836	289390.669080266\\
3.02147553688842	289342.540625475\\
3.02157553938848	289294.412170684\\
3.02167554188855	289246.283715894\\
3.02177554438861	289197.582303307\\
3.02187554688867	289149.453848516\\
3.02197554938873	289101.325393725\\
3.0220755518888	289053.196938934\\
3.02217555438886	289005.068484143\\
3.02227555688892	288956.940029352\\
3.02237555938898	288908.238616766\\
3.02247556188905	288860.110161975\\
3.02257556438911	288811.981707184\\
3.02267556688917	288763.853252393\\
3.02277556938923	288715.724797602\\
3.0228755718893	288667.023385016\\
3.02297557438936	288618.894930225\\
3.02307557688942	288570.766475434\\
3.02317557938948	288522.638020643\\
3.02327558188955	288473.936608057\\
3.02337558438961	288425.808153266\\
3.02347558688967	288377.679698475\\
3.02357558938973	288328.978285889\\
3.0236755918898	288280.849831098\\
3.02377559438986	288232.721376307\\
3.02387559688992	288184.019963721\\
3.02397559938998	288135.89150893\\
3.02407560189005	288087.763054139\\
3.02417560439011	288039.061641553\\
3.02427560689017	287990.933186762\\
3.02437560939023	287942.804731971\\
3.0244756118903	287894.103319385\\
3.02457561439036	287845.974864594\\
3.02467561689042	287797.846409803\\
3.02477561939048	287749.144997217\\
3.02487562189055	287701.016542426\\
3.02497562439061	287652.31512984\\
3.02507562689067	287604.186675049\\
3.02517562939073	287555.485262462\\
3.0252756318908	287507.356807671\\
3.02537563439086	287459.22835288\\
3.02547563689092	287410.526940294\\
3.02557563939098	287362.398485503\\
3.02567564189105	287313.697072917\\
3.02577564439111	287265.568618126\\
3.02587564689117	287216.86720554\\
3.02597564939123	287168.738750749\\
3.0260756518913	287120.037338163\\
3.02617565439136	287071.908883372\\
3.02627565689142	287023.207470786\\
3.02637565939148	286975.079015995\\
3.02647566189155	286926.377603409\\
3.02657566439161	286878.249148618\\
3.02667566689167	286829.547736032\\
3.02677566939173	286780.846323446\\
3.0268756718918	286732.717868655\\
3.02697567439186	286684.016456068\\
3.02707567689192	286635.888001277\\
3.02717567939198	286587.186588691\\
3.02727568189205	286538.485176105\\
3.02737568439211	286490.356721314\\
3.02747568689217	286441.655308728\\
3.02757568939223	286393.526853937\\
3.0276756918923	286344.825441351\\
3.02777569439236	286296.124028765\\
3.02787569689242	286247.995573974\\
3.02797569939248	286199.294161388\\
3.02807570189255	286150.592748802\\
3.02817570439261	286102.464294011\\
3.02827570689267	286053.762881425\\
3.02837570939273	286005.061468838\\
3.0284757118928	285956.933014047\\
3.02857571439286	285908.231601461\\
3.02867571689292	285859.530188875\\
3.02877571939298	285810.828776289\\
3.02887572189305	285762.700321498\\
3.02897572439311	285713.998908912\\
3.02907572689317	285665.297496326\\
3.02917572939323	285616.59608374\\
3.0292757318933	285568.467628949\\
3.02937573439336	285519.766216363\\
3.02947573689342	285471.064803776\\
3.02957573939349	285422.36339119\\
3.02967574189355	285373.661978604\\
3.02977574439361	285325.533523813\\
3.02987574689367	285276.832111227\\
3.02997574939373	285228.130698641\\
3.0300757518938	285179.429286055\\
3.03017575439386	285130.727873469\\
3.03027575689392	285082.026460883\\
3.03037575939398	285033.898006092\\
3.03047576189405	284985.196593506\\
3.03057576439411	284936.495180919\\
3.03067576689417	284887.793768333\\
3.03077576939423	284839.092355747\\
3.0308757718943	284790.390943161\\
3.03097577439436	284741.689530575\\
3.03107577689442	284692.988117989\\
3.03117577939448	284644.286705403\\
3.03127578189455	284595.585292817\\
3.03137578439461	284547.456838026\\
3.03147578689467	284498.755425439\\
3.03157578939474	284450.054012853\\
3.0316757918948	284401.352600267\\
3.03177579439486	284352.651187681\\
3.03187579689492	284303.949775095\\
3.03197579939498	284255.248362509\\
3.03207580189505	284206.546949923\\
3.03217580439511	284157.845537337\\
3.03227580689517	284109.144124751\\
3.03237580939523	284060.442712164\\
3.0324758118953	284011.741299578\\
3.03257581439536	283963.039886992\\
3.03267581689542	283914.338474406\\
3.03277581939548	283865.63706182\\
3.03287582189555	283816.935649234\\
3.03297582439561	283767.661278853\\
3.03307582689567	283718.959866266\\
3.03317582939573	283670.25845368\\
3.0332758318958	283621.557041094\\
3.03337583439586	283572.855628508\\
3.03347583689592	283524.154215922\\
3.03357583939599	283475.452803336\\
3.03367584189605	283426.75139075\\
3.03377584439611	283378.049978164\\
3.03387584689617	283329.348565577\\
3.03397584939623	283280.074195196\\
3.0340758518963	283231.37278261\\
3.03417585439636	283182.671370024\\
3.03427585689642	283133.969957438\\
3.03437585939648	283085.268544852\\
3.03447586189655	283036.567132266\\
3.03457586439661	282987.292761884\\
3.03467586689667	282938.591349298\\
3.03477586939673	282889.889936712\\
3.0348758718968	282841.188524126\\
3.03497587439686	282792.48711154\\
3.03507587689692	282743.212741159\\
3.03517587939698	282694.511328573\\
3.03527588189705	282645.809915986\\
3.03537588439711	282597.1085034\\
3.03547588689717	282547.834133019\\
3.03557588939724	282499.132720433\\
3.0356758918973	282450.431307847\\
3.03577589439736	282401.729895261\\
3.03587589689742	282352.455524879\\
3.03597589939749	282303.754112293\\
3.03607590189755	282255.052699707\\
3.03617590439761	282205.778329326\\
3.03627590689767	282157.07691674\\
3.03637590939773	282108.375504154\\
3.0364759118978	282059.674091568\\
3.03657591439786	282010.399721186\\
3.03667591689792	281961.6983086\\
3.03677591939798	281912.423938219\\
3.03687592189805	281863.722525633\\
3.03697592439811	281815.021113047\\
3.03707592689817	281765.746742665\\
3.03717592939823	281717.045330079\\
3.0372759318983	281668.343917493\\
3.03737593439836	281619.069547112\\
3.03747593689842	281570.368134526\\
3.03757593939849	281521.093764145\\
3.03767594189855	281472.392351558\\
3.03777594439861	281423.690938972\\
3.03787594689867	281374.416568591\\
3.03797594939874	281325.715156005\\
3.0380759518988	281276.440785624\\
3.03817595439886	281227.739373038\\
3.03827595689892	281178.465002656\\
3.03837595939898	281129.76359007\\
3.03847596189905	281080.489219689\\
3.03857596439911	281031.787807103\\
3.03867596689917	280982.513436722\\
3.03877596939923	280933.812024136\\
3.0388759718993	280885.110611549\\
3.03897597439936	280835.836241168\\
3.03907597689942	280786.561870787\\
3.03917597939948	280737.860458201\\
3.03927598189955	280688.586087819\\
3.03937598439961	280639.884675233\\
3.03947598689967	280590.610304852\\
3.03957598939974	280541.908892266\\
3.0396759918998	280492.634521885\\
3.03977599439986	280443.933109299\\
3.03987599689992	280394.658738917\\
3.03997599939999	280345.957326331\\
3.04007600190005	280296.68295595\\
3.04017600440011	280247.408585569\\
3.04027600690017	280198.707172983\\
3.04037600940024	280149.432802601\\
3.0404760119003	280100.731390015\\
3.04057601440036	280051.457019634\\
3.04067601690042	280002.182649253\\
3.04077601940048	279953.481236667\\
3.04087602190055	279904.206866285\\
3.04097602440061	279854.932495904\\
3.04107602690067	279806.231083318\\
3.04117602940073	279756.956712937\\
3.0412760319008	279707.682342556\\
3.04137603440086	279658.980929969\\
3.04147603690092	279609.706559588\\
3.04157603940099	279560.432189207\\
3.04167604190105	279511.730776621\\
3.04177604440111	279462.45640624\\
3.04187604690117	279413.182035858\\
3.04197604940124	279363.907665477\\
3.0420760519013	279315.206252891\\
3.04217605440136	279265.93188251\\
3.04227605690142	279216.657512128\\
3.04237605940149	279167.956099542\\
3.04247606190155	279118.681729161\\
3.04257606440161	279069.40735878\\
3.04267606690167	279020.132988399\\
3.04277606940173	278970.858618017\\
3.0428760719018	278922.157205431\\
3.04297607440186	278872.88283505\\
3.04307607690192	278823.608464669\\
3.04317607940198	278774.334094287\\
3.04327608190205	278725.632681701\\
3.04337608440211	278676.35831132\\
3.04347608690217	278627.083940939\\
3.04357608940224	278577.809570558\\
3.0436760919023	278528.535200176\\
3.04377609440236	278479.260829795\\
3.04387609690242	278430.559417209\\
3.04397609940249	278381.285046828\\
3.04407610190255	278332.010676446\\
3.04417610440261	278282.736306065\\
3.04427610690267	278233.461935684\\
3.04437610940274	278184.187565303\\
3.0444761119028	278134.913194921\\
3.04457611440286	278085.63882454\\
3.04467611690292	278036.937411954\\
3.04477611940298	277987.663041573\\
3.04487612190305	277938.388671192\\
3.04497612440311	277889.11430081\\
3.04507612690317	277839.839930429\\
3.04517612940323	277790.565560048\\
3.0452761319033	277741.291189667\\
3.04537613440336	277692.016819285\\
3.04547613690342	277642.742448904\\
3.04557613940349	277593.468078523\\
3.04567614190355	277544.193708142\\
3.04577614440361	277494.91933776\\
3.04587614690367	277445.644967379\\
3.04597614940374	277396.370596998\\
3.0460761519038	277347.096226617\\
3.04617615440386	277297.821856235\\
3.04627615690392	277248.547485854\\
3.04637615940399	277199.273115473\\
3.04647616190405	277149.998745092\\
3.04657616440411	277100.72437471\\
3.04667616690417	277051.450004329\\
3.04677616940424	277002.175633948\\
3.0468761719043	276952.901263567\\
3.04697617440436	276903.626893185\\
3.04707617690442	276854.352522804\\
3.04717617940448	276805.078152423\\
3.04727618190455	276755.803782042\\
3.04737618440461	276706.52941166\\
3.04747618690467	276657.255041279\\
3.04757618940474	276607.980670898\\
3.0476761919048	276558.706300516\\
3.04777619440486	276508.85897234\\
3.04787619690492	276459.584601959\\
3.04797619940499	276410.310231578\\
3.04807620190505	276361.035861196\\
3.04817620440511	276311.761490815\\
3.04827620690517	276262.487120434\\
3.04837620940524	276213.212750053\\
3.0484762119053	276163.938379671\\
3.04857621440536	276114.091051495\\
3.04867621690542	276064.816681114\\
3.04877621940549	276015.542310733\\
3.04887622190555	275966.267940351\\
3.04897622440561	275916.99356997\\
3.04907622690567	275867.719199589\\
3.04917622940573	275817.871871412\\
3.0492762319058	275768.597501031\\
3.04937623440586	275719.32313065\\
3.04947623690592	275670.048760269\\
3.04957623940599	275620.774389887\\
3.04967624190605	275570.927061711\\
3.04977624440611	275521.65269133\\
3.04987624690617	275472.378320948\\
3.04997624940624	275423.103950567\\
3.0500762519063	275373.256622391\\
3.05017625440636	275323.98225201\\
3.05027625690642	275274.707881628\\
3.05037625940649	275225.433511247\\
3.05047626190655	275175.586183071\\
3.05057626440661	275126.311812689\\
3.05067626690667	275077.037442308\\
3.05077626940674	275027.763071927\\
3.0508762719068	274977.915743751\\
3.05097627440686	274928.641373369\\
3.05107627690692	274879.367002988\\
3.05117627940698	274829.519674812\\
3.05127628190705	274780.24530443\\
3.05137628440711	274730.970934049\\
3.05147628690717	274681.123605873\\
3.05157628940724	274631.849235492\\
3.0516762919073	274582.57486511\\
3.05177629440736	274532.727536934\\
3.05187629690742	274483.453166553\\
3.05197629940749	274434.178796171\\
3.05207630190755	274384.331467995\\
3.05217630440761	274335.057097614\\
3.05227630690767	274285.209769437\\
3.05237630940774	274235.935399056\\
3.0524763119078	274186.661028675\\
3.05257631440786	274136.813700499\\
3.05267631690792	274087.539330117\\
3.05277631940799	274038.264959736\\
3.05287632190805	273988.41763156\\
3.05297632440811	273939.143261178\\
3.05307632690817	273889.295933002\\
3.05317632940824	273840.021562621\\
3.0532763319083	273790.174234444\\
3.05337633440836	273740.899864063\\
3.05347633690842	273691.625493682\\
3.05357633940849	273641.778165506\\
3.05367634190855	273592.503795124\\
3.05377634440861	273542.656466948\\
3.05387634690867	273493.382096567\\
3.05397634940874	273443.53476839\\
3.0540763519088	273394.260398009\\
3.05417635440886	273344.413069833\\
3.05427635690892	273295.138699451\\
3.05437635940899	273245.291371275\\
3.05447636190905	273196.017000894\\
3.05457636440911	273146.169672717\\
3.05467636690917	273096.895302336\\
3.05477636940924	273047.04797416\\
3.0548763719093	272997.773603779\\
3.05497637440936	272947.926275602\\
3.05507637690942	272898.651905221\\
3.05517637940949	272848.804577044\\
3.05527638190955	272799.530206663\\
3.05537638440961	272749.682878487\\
3.05547638690967	272699.83555031\\
3.05557638940974	272650.561179929\\
3.0556763919098	272600.713851753\\
3.05577639440986	272551.439481372\\
3.05587639690992	272501.592153195\\
3.05597639940999	272452.317782814\\
3.05607640191005	272402.470454638\\
3.05617640441011	272352.623126461\\
3.05627640691017	272303.34875608\\
3.05637640941024	272253.501427904\\
3.0564764119103	272204.227057522\\
3.05657641441036	272154.379729346\\
3.05667641691042	272104.53240117\\
3.05677641941049	272055.258030788\\
3.05687642191055	272005.410702612\\
3.05697642441061	271955.563374436\\
3.05707642691067	271906.289004054\\
3.05717642941074	271856.441675878\\
3.0572764319108	271806.594347701\\
3.05737643441086	271757.31997732\\
3.05747643691092	271707.472649144\\
3.05757643941099	271657.625320967\\
3.05767644191105	271608.350950586\\
3.05777644441111	271558.50362241\\
3.05787644691117	271508.656294234\\
3.05797644941124	271459.381923852\\
3.0580764519113	271409.534595676\\
3.05817645441136	271359.687267499\\
3.05827645691142	271309.839939323\\
3.05837645941149	271260.565568942\\
3.05847646191155	271210.718240765\\
3.05857646441161	271160.870912589\\
3.05867646691167	271111.596542208\\
3.05877646941174	271061.749214031\\
3.0588764719118	271011.901885855\\
3.05897647441186	270962.054557679\\
3.05907647691192	270912.780187297\\
3.05917647941199	270862.932859121\\
3.05927648191205	270813.085530945\\
3.05937648441211	270763.238202768\\
3.05947648691217	270713.390874592\\
3.05957648941224	270664.116504211\\
3.0596764919123	270614.269176034\\
3.05977649441236	270564.421847858\\
3.05987649691242	270514.574519681\\
3.05997649941249	270465.3001493\\
3.06007650191255	270415.452821124\\
3.06017650441261	270365.605492948\\
3.06027650691267	270315.758164771\\
3.06037650941274	270265.910836595\\
3.0604765119128	270216.063508418\\
3.06057651441286	270166.789138037\\
3.06067651691292	270116.941809861\\
3.06077651941299	270067.094481684\\
3.06087652191305	270017.247153508\\
3.06097652441311	269967.399825332\\
3.06107652691317	269917.552497155\\
3.06117652941324	269867.705168979\\
3.0612765319133	269818.430798598\\
3.06137653441336	269768.583470421\\
3.06147653691342	269718.736142245\\
3.06157653941349	269668.888814068\\
3.06167654191355	269619.041485892\\
3.06177654441361	269569.194157716\\
3.06187654691367	269519.346829539\\
3.06197654941374	269469.499501363\\
3.0620765519138	269419.652173186\\
3.06217655441386	269370.377802805\\
3.06227655691392	269320.530474629\\
3.06237655941399	269270.683146452\\
3.06247656191405	269220.835818276\\
3.06257656441411	269170.9884901\\
3.06267656691417	269121.141161923\\
3.06277656941424	269071.293833747\\
3.0628765719143	269021.446505571\\
3.06297657441436	268971.599177394\\
3.06307657691442	268921.751849218\\
3.06317657941449	268871.904521041\\
3.06327658191455	268822.057192865\\
3.06337658441461	268772.209864689\\
3.06347658691467	268722.362536512\\
3.06357658941474	268672.515208336\\
3.0636765919148	268622.66788016\\
3.06377659441486	268572.820551983\\
3.06387659691492	268522.973223807\\
3.06397659941499	268473.12589563\\
3.06407660191505	268423.278567454\\
3.06417660441511	268373.431239278\\
3.06427660691517	268323.583911101\\
3.06437660941524	268273.736582925\\
3.0644766119153	268223.889254748\\
3.06457661441536	268174.041926572\\
3.06467661691542	268124.194598396\\
3.06477661941549	268074.347270219\\
3.06487662191555	268024.499942043\\
3.06497662441561	267974.652613867\\
3.06507662691567	267924.80528569\\
3.06517662941574	267874.957957514\\
3.0652766319158	267825.110629337\\
3.06537663441586	267775.263301161\\
3.06547663691592	267725.415972985\\
3.06557663941599	267674.995687013\\
3.06567664191605	267625.148358837\\
3.06577664441611	267575.30103066\\
3.06587664691617	267525.453702484\\
3.06597664941624	267475.606374308\\
3.0660766519163	267425.759046131\\
3.06617665441636	267375.911717955\\
3.06627665691642	267326.064389778\\
3.06637665941649	267276.217061602\\
3.06647666191655	267226.369733426\\
3.06657666441661	267175.949447454\\
3.06667666691667	267126.102119278\\
3.06677666941674	267076.254791101\\
3.0668766719168	267026.407462925\\
3.06697667441686	266976.560134749\\
3.06707667691692	266926.712806572\\
3.06717667941699	266876.865478396\\
3.06727668191705	266826.445192424\\
3.06737668441711	266776.597864248\\
3.06747668691717	266726.750536072\\
3.06757668941724	266676.903207895\\
3.0676766919173	266627.055879719\\
3.06777669441736	266576.635593747\\
3.06787669691742	266526.788265571\\
3.06797669941749	266476.940937395\\
3.06807670191755	266427.093609218\\
3.06817670441761	266377.246281042\\
3.06827670691767	266327.398952865\\
3.06837670941774	266276.978666894\\
3.0684767119178	266227.131338718\\
3.06857671441786	266177.284010541\\
3.06867671691792	266127.436682365\\
3.06877671941799	266077.016396393\\
3.06887672191805	266027.169068217\\
3.06897672441811	265977.32174004\\
3.06907672691817	265927.474411864\\
3.06917672941824	265877.054125893\\
3.0692767319183	265827.206797716\\
3.06937673441836	265777.35946954\\
3.06947673691842	265727.512141363\\
3.06957673941849	265677.091855392\\
3.06967674191855	265627.244527216\\
3.06977674441861	265577.397199039\\
3.06987674691867	265527.549870863\\
3.06997674941874	265477.129584891\\
3.0700767519188	265427.282256715\\
3.07017675441886	265377.434928539\\
3.07027675691892	265327.014642567\\
3.07037675941899	265277.167314391\\
3.07047676191905	265227.319986214\\
3.07057676441911	265177.472658038\\
3.07067676691917	265127.052372066\\
3.07077676941924	265077.20504389\\
3.0708767719193	265027.357715714\\
3.07097677441936	264976.937429742\\
3.07107677691942	264927.090101566\\
3.07117677941949	264877.242773389\\
3.07127678191955	264826.822487418\\
3.07137678441961	264776.975159241\\
3.07147678691967	264727.127831065\\
3.07157678941974	264676.707545093\\
3.0716767919198	264626.860216917\\
3.07177679441986	264577.012888741\\
3.07187679691992	264526.592602769\\
3.07197679941999	264476.745274593\\
3.07207680192005	264426.324988621\\
3.07217680442011	264376.477660445\\
3.07227680692017	264326.630332269\\
3.07237680942024	264276.210046297\\
3.0724768119203	264226.362718121\\
3.07257681442036	264176.515389944\\
3.07267681692042	264126.095103973\\
3.07277681942049	264076.247775796\\
3.07287682192055	264025.827489825\\
3.07297682442061	263975.980161649\\
3.07307682692067	263926.132833472\\
3.07317682942074	263875.712547501\\
3.0732768319208	263825.865219324\\
3.07337683442086	263775.444933353\\
3.07347683692092	263725.597605176\\
3.07357683942099	263675.177319205\\
3.07367684192105	263625.329991028\\
3.07377684442111	263575.482662852\\
3.07387684692117	263525.062376881\\
3.07397684942124	263475.215048704\\
3.0740768519213	263424.794762733\\
3.07417685442136	263374.947434556\\
3.07427685692142	263324.527148585\\
3.07437685942149	263274.679820408\\
3.07447686192155	263224.259534437\\
3.07457686442161	263174.412206261\\
3.07467686692167	263123.991920289\\
3.07477686942174	263074.144592113\\
3.0748768719218	263023.724306141\\
3.07497687442186	262973.876977965\\
3.07507687692192	262924.029649788\\
3.07517687942199	262873.609363817\\
3.07527688192205	262823.76203564\\
3.07537688442211	262773.341749669\\
3.07547688692217	262723.494421493\\
3.07557688942224	262673.074135521\\
3.0756768919223	262622.65384955\\
3.07577689442236	262572.806521373\\
3.07587689692242	262522.386235402\\
3.07597689942249	262472.538907225\\
3.07607690192255	262422.118621254\\
3.07617690442261	262372.271293077\\
3.07627690692267	262321.851007106\\
3.07637690942274	262272.003678929\\
3.0764769119228	262221.583392958\\
3.07657691442286	262171.736064782\\
3.07667691692292	262121.31577881\\
3.07677691942299	262071.468450634\\
3.07687692192305	262021.048164662\\
3.07697692442311	261970.627878691\\
3.07707692692317	261920.780550514\\
3.07717692942324	261870.360264543\\
3.0772769319233	261820.512936366\\
3.07737693442336	261770.092650395\\
3.07747693692342	261720.245322218\\
3.07757693942349	261669.825036247\\
3.07767694192355	261619.404750275\\
3.07777694442361	261569.557422099\\
3.07787694692367	261519.137136128\\
3.07797694942374	261469.289807951\\
3.0780769519238	261418.86952198\\
3.07817695442386	261368.449236008\\
3.07827695692392	261318.601907832\\
3.07837695942399	261268.18162186\\
3.07847696192405	261218.334293684\\
3.07857696442411	261167.914007712\\
3.07867696692417	261117.493721741\\
3.07877696942424	261067.646393564\\
3.0788769719243	261017.226107593\\
3.07897697442436	260966.805821621\\
3.07907697692442	260916.958493445\\
3.07917697942449	260866.538207474\\
3.07927698192455	260816.690879297\\
3.07937698442461	260766.270593326\\
3.07947698692467	260715.850307354\\
3.07957698942474	260666.002979178\\
3.0796769919248	260615.582693206\\
3.07977699442486	260565.162407235\\
3.07987699692492	260515.315079058\\
3.07997699942499	260464.894793087\\
3.08007700192505	260414.474507115\\
3.08017700442511	260364.627178939\\
3.08027700692517	260314.206892967\\
3.08037700942524	260263.786606996\\
3.0804770119253	260213.93927882\\
3.08057701442536	260163.518992848\\
3.08067701692542	260113.098706877\\
3.08077701942549	260062.678420905\\
3.08087702192555	260012.831092729\\
3.08097702442561	259962.410806757\\
3.08107702692567	259911.990520786\\
3.08117702942574	259862.143192609\\
3.0812770319258	259811.722906638\\
3.08137703442586	259761.302620666\\
3.08147703692592	259711.45529249\\
3.08157703942599	259661.035006518\\
3.08167704192605	259610.614720547\\
3.08177704442611	259560.194434575\\
3.08187704692617	259510.347106399\\
3.08197704942624	259459.926820427\\
3.0820770519263	259409.506534456\\
3.08217705442636	259359.086248484\\
3.08227705692642	259309.238920308\\
3.08237705942649	259258.818634336\\
3.08247706192655	259208.398348365\\
3.08257706442661	259157.978062393\\
3.08267706692667	259108.130734217\\
3.08277706942674	259057.710448246\\
3.0828770719268	259007.290162274\\
3.08297707442686	258956.869876302\\
3.08307707692692	258907.022548126\\
3.08317707942699	258856.602262155\\
3.08327708192705	258806.181976183\\
3.08337708442711	258755.761690212\\
3.08347708692717	258705.34140424\\
3.08357708942724	258655.494076064\\
3.0836770919273	258605.073790092\\
3.08377709442736	258554.653504121\\
3.08387709692742	258504.233218149\\
3.08397709942749	258453.812932178\\
3.08407710192755	258403.965604001\\
3.08417710442761	258353.54531803\\
3.08427710692767	258303.125032058\\
3.08437710942774	258252.704746087\\
3.0844771119278	258202.284460115\\
3.08457711442786	258152.437131939\\
3.08467711692792	258102.016845967\\
3.08477711942799	258051.596559996\\
3.08487712192805	258001.176274024\\
3.08497712442811	257950.755988053\\
3.08507712692817	257900.335702081\\
3.08517712942824	257850.488373905\\
3.0852771319283	257800.068087933\\
3.08537713442836	257749.647801962\\
3.08547713692842	257699.22751599\\
3.08557713942849	257648.807230019\\
3.08567714192855	257598.386944047\\
3.08577714442861	257548.539615871\\
3.08587714692867	257498.119329899\\
3.08597714942874	257447.699043928\\
3.0860771519288	257397.278757956\\
3.08617715442886	257346.858471985\\
3.08627715692892	257296.438186013\\
3.08637715942899	257246.017900042\\
3.08647716192905	257196.170571865\\
3.08657716442911	257145.750285894\\
3.08667716692917	257095.329999922\\
3.08677716942924	257044.909713951\\
3.0868771719293	256994.489427979\\
3.08697717442936	256944.069142008\\
3.08707717692942	256893.648856036\\
3.08717717942949	256843.228570065\\
3.08727718192955	256792.808284093\\
3.08737718442961	256742.960955917\\
3.08747718692967	256692.540669946\\
3.08757718942974	256642.120383974\\
3.0876771919298	256591.700098002\\
3.08777719442986	256541.279812031\\
3.08787719692992	256490.859526059\\
3.08797719942999	256440.439240088\\
3.08807720193005	256390.018954116\\
3.08817720443011	256339.598668145\\
3.08827720693017	256289.178382173\\
3.08837720943024	256238.758096202\\
3.0884772119303	256188.33781023\\
3.08857721443036	256137.917524259\\
3.08867721693042	256088.070196082\\
3.08877721943049	256037.649910111\\
3.08887722193055	255987.229624139\\
3.08897722443061	255936.809338168\\
3.08907722693067	255886.389052196\\
3.08917722943074	255835.968766225\\
3.0892772319308	255785.548480253\\
3.08937723443086	255735.128194282\\
3.08947723693092	255684.70790831\\
3.08957723943099	255634.287622339\\
3.08967724193105	255583.867336367\\
3.08977724443111	255533.447050396\\
3.08987724693117	255483.026764424\\
3.08997724943124	255432.606478453\\
3.0900772519313	255382.186192481\\
3.09017725443136	255331.76590651\\
3.09027725693142	255281.345620538\\
3.09037725943149	255230.925334567\\
3.09047726193155	255180.505048595\\
3.09057726443161	255130.084762624\\
3.09067726693167	255079.664476652\\
3.09077726943174	255029.244190681\\
3.0908772719318	254978.823904709\\
3.09097727443186	254928.403618738\\
3.09107727693192	254877.983332766\\
3.09117727943199	254827.563046795\\
3.09127728193205	254777.142760823\\
3.09137728443211	254726.722474852\\
3.09147728693217	254676.30218888\\
3.09157728943224	254625.881902909\\
3.0916772919323	254575.461616937\\
3.09177729443236	254525.041330966\\
3.09187729693242	254474.621044994\\
3.09197729943249	254424.200759023\\
3.09207730193255	254373.780473051\\
3.09217730443261	254323.36018708\\
3.09227730693267	254272.939901108\\
3.09237730943274	254222.519615137\\
3.0924773119328	254172.099329165\\
3.09257731443286	254121.679043194\\
3.09267731693292	254071.258757222\\
3.09277731943299	254020.83847125\\
3.09287732193305	253970.418185279\\
3.09297732443311	253919.997899307\\
3.09307732693317	253869.577613336\\
3.09317732943324	253819.157327364\\
3.0932773319333	253768.737041393\\
3.09337733443336	253718.316755421\\
3.09347733693342	253667.89646945\\
3.09357733943349	253617.476183478\\
3.09367734193355	253566.482939712\\
3.09377734443361	253516.06265374\\
3.09387734693367	253465.642367769\\
3.09397734943374	253415.222081797\\
3.0940773519338	253364.801795826\\
3.09417735443386	253314.381509854\\
3.09427735693392	253263.961223883\\
3.09437735943399	253213.540937911\\
3.09447736193405	253163.12065194\\
3.09457736443411	253112.700365968\\
3.09467736693417	253062.280079997\\
3.09477736943424	253011.859794025\\
3.0948773719343	252961.439508054\\
3.09497737443436	252910.446264287\\
3.09507737693442	252860.025978315\\
3.09517737943449	252809.605692344\\
3.09527738193455	252759.185406372\\
3.09537738443461	252708.765120401\\
3.09547738693467	252658.344834429\\
3.09557738943474	252607.924548458\\
3.0956773919348	252557.504262486\\
3.09577739443486	252507.083976515\\
3.09587739693492	252456.663690543\\
3.09597739943499	252405.670446777\\
3.09607740193505	252355.250160805\\
3.09617740443511	252304.829874834\\
3.09627740693517	252254.409588862\\
3.09637740943524	252203.989302891\\
3.0964774119353	252153.569016919\\
3.09657741443536	252103.148730948\\
3.09667741693542	252052.728444976\\
3.09677741943549	252002.308159005\\
3.09687742193555	251951.314915238\\
3.09697742443561	251900.894629266\\
3.09707742693567	251850.474343295\\
3.09717742943574	251800.054057323\\
3.0972774319358	251749.633771352\\
3.09737743443586	251699.21348538\\
3.09747743693592	251648.793199409\\
3.09757743943599	251597.799955642\\
3.09767744193605	251547.379669671\\
3.09777744443611	251496.959383699\\
3.09787744693617	251446.539097728\\
3.09797744943624	251396.118811756\\
3.0980774519363	251345.698525785\\
3.09817745443636	251295.278239813\\
3.09827745693642	251244.284996046\\
3.09837745943649	251193.864710075\\
3.09847746193655	251143.444424103\\
3.09857746443661	251093.024138132\\
3.09867746693667	251042.60385216\\
3.09877746943674	250992.183566189\\
3.0988774719368	250941.190322422\\
3.09897747443686	250890.770036451\\
3.09907747693692	250840.349750479\\
3.09917747943699	250789.929464508\\
3.09927748193705	250739.509178536\\
3.09937748443711	250689.088892565\\
3.09947748693717	250638.095648798\\
3.09957748943724	250587.675362827\\
3.0996774919373	250537.255076855\\
3.09977749443736	250486.834790884\\
3.09987749693742	250436.414504912\\
3.09997749943749	250385.421261145\\
3.10007750193755	250335.000975174\\
3.10017750443761	250284.580689202\\
3.10027750693767	250234.160403231\\
3.10037750943774	250183.740117259\\
3.1004775119378	250132.746873493\\
3.10057751443786	250082.326587521\\
3.10067751693792	250031.90630155\\
3.10077751943799	249981.486015578\\
3.10087752193805	249931.065729607\\
3.10097752443811	249880.07248584\\
3.10107752693817	249829.652199869\\
3.10117752943824	249779.231913897\\
3.1012775319383	249728.811627926\\
3.10137753443836	249678.391341954\\
3.10147753693842	249627.398098187\\
3.10157753943849	249576.977812216\\
3.10167754193855	249526.557526244\\
3.10177754443861	249476.137240273\\
3.10187754693867	249425.716954301\\
3.10197754943874	249374.723710535\\
3.1020775519388	249324.303424563\\
3.10217755443886	249273.883138592\\
3.10227755693892	249223.46285262\\
3.10237755943899	249172.469608853\\
3.10247756193905	249122.049322882\\
3.10257756443911	249071.62903691\\
3.10267756693917	249021.208750939\\
3.10277756943924	248970.215507172\\
3.1028775719393	248919.795221201\\
3.10297757443936	248869.374935229\\
3.10307757693942	248818.954649258\\
3.10317757943949	248768.534363286\\
3.10327758193955	248717.54111952\\
3.10337758443961	248667.120833548\\
3.10347758693967	248616.700547577\\
3.10357758943974	248566.280261605\\
3.1036775919398	248515.287017838\\
3.10377759443986	248464.866731867\\
3.10387759693992	248414.446445895\\
3.10397759943999	248364.026159924\\
3.10407760194005	248313.032916157\\
3.10417760444011	248262.612630186\\
3.10427760694017	248212.192344214\\
3.10437760944024	248161.772058243\\
3.1044776119403	248110.778814476\\
3.10457761444036	248060.358528505\\
3.10467761694042	248009.938242533\\
3.10477761944049	247958.944998766\\
3.10487762194055	247908.524712795\\
3.10497762444061	247858.104426823\\
3.10507762694067	247807.684140852\\
3.10517762944074	247756.690897085\\
3.1052776319408	247706.270611114\\
3.10537763444086	247655.850325142\\
3.10547763694092	247605.430039171\\
3.10557763944099	247554.436795404\\
3.10567764194105	247504.016509433\\
3.10577764444111	247453.596223461\\
3.10587764694117	247402.602979694\\
3.10597764944124	247352.182693723\\
3.1060776519413	247301.762407751\\
3.10617765444136	247251.34212178\\
3.10627765694142	247200.348878013\\
3.10637765944149	247149.928592042\\
3.10647766194155	247099.50830607\\
3.10657766444161	247049.088020099\\
3.10667766694167	246998.094776332\\
3.10677766944174	246947.67449036\\
3.1068776719418	246897.254204389\\
3.10697767444186	246846.260960622\\
3.10707767694192	246795.840674651\\
3.10717767944199	246745.420388679\\
3.10727768194205	246694.427144913\\
3.10737768444211	246644.006858941\\
3.10747768694217	246593.58657297\\
3.10757768944224	246543.166286998\\
3.1076776919423	246492.173043232\\
3.10777769444236	246441.75275726\\
3.10787769694242	246391.332471288\\
3.10797769944249	246340.339227522\\
3.10807770194255	246289.91894155\\
3.10817770444261	246239.498655579\\
3.10827770694267	246188.505411812\\
3.10837770944274	246138.085125841\\
3.1084777119428	246087.664839869\\
3.10857771444286	246037.244553898\\
3.10867771694292	245986.251310131\\
3.10877771944299	245935.831024159\\
3.10887772194305	245885.410738188\\
3.10897772444311	245834.417494421\\
3.10907772694317	245783.99720845\\
3.10917772944324	245733.576922478\\
3.1092777319433	245682.583678712\\
3.10937773444336	245632.16339274\\
3.10947773694342	245581.743106769\\
3.10957773944349	245530.749863002\\
3.10967774194355	245480.32957703\\
3.10977774444361	245429.909291059\\
3.10987774694367	245378.916047292\\
3.10997774944374	245328.495761321\\
3.1100777519438	245278.075475349\\
3.11017775444386	245227.082231583\\
3.11027775694392	245176.661945611\\
3.11037775944399	245126.24165964\\
3.11047776194405	245075.248415873\\
3.11057776444411	245024.828129901\\
3.11067776694417	244974.40784393\\
3.11077776944424	244923.414600163\\
3.1108777719443	244872.994314192\\
3.11097777444436	244822.57402822\\
3.11107777694442	244771.580784454\\
3.11117777944449	244721.160498482\\
3.11127778194455	244670.740212511\\
3.11137778444461	244619.746968744\\
3.11147778694467	244569.326682772\\
3.11157778944474	244518.906396801\\
3.1116777919448	244467.913153034\\
3.11177779444486	244417.492867063\\
3.11187779694492	244367.072581091\\
3.11197779944499	244316.079337325\\
3.11207780194505	244265.659051353\\
3.11217780444511	244215.238765382\\
3.11227780694517	244164.245521615\\
3.11237780944524	244113.825235643\\
3.1124778119453	244063.404949672\\
3.11257781444536	244012.411705905\\
3.11267781694542	243961.991419934\\
3.11277781944549	243911.571133962\\
3.11287782194555	243860.577890196\\
3.11297782444561	243810.157604224\\
3.11307782694567	243759.737318253\\
3.11317782944574	243708.744074486\\
3.1132778319458	243658.323788514\\
3.11337783444586	243607.903502543\\
3.11347783694592	243556.910258776\\
3.11357783944599	243506.489972805\\
3.11367784194605	243456.069686833\\
3.11377784444611	243405.076443067\\
3.11387784694617	243354.656157095\\
3.11397784944624	243303.662913328\\
3.1140778519463	243253.242627357\\
3.11417785444636	243202.822341385\\
3.11427785694642	243151.829097619\\
3.11437785944649	243101.408811647\\
3.11447786194655	243050.988525676\\
3.11457786444661	242999.995281909\\
3.11467786694667	242949.574995938\\
3.11477786944674	242899.154709966\\
3.1148778719468	242848.161466199\\
3.11497787444686	242797.741180228\\
3.11507787694692	242747.320894256\\
3.11517787944699	242696.32765049\\
3.11527788194705	242645.907364518\\
3.11537788444711	242594.914120752\\
3.11547788694717	242544.49383478\\
3.11557788944724	242494.073548809\\
3.1156778919473	242443.080305042\\
3.11577789444736	242392.66001907\\
3.11587789694742	242342.239733099\\
3.11597789944749	242291.246489332\\
3.11607790194755	242240.826203361\\
3.11617790444761	242190.405917389\\
3.11627790694767	242139.412673623\\
3.11637790944774	242088.992387651\\
3.1164779119478	242037.999143884\\
3.11657791444786	241987.578857913\\
3.11667791694792	241937.158571941\\
3.11677791944799	241886.165328175\\
3.11687792194805	241835.745042203\\
3.11697792444811	241785.324756232\\
3.11707792694817	241734.331512465\\
3.11717792944824	241683.911226494\\
3.1172779319483	241632.917982727\\
3.11737793444836	241582.497696755\\
3.11747793694842	241532.077410784\\
3.11757793944849	241481.084167017\\
3.11767794194855	241430.663881046\\
3.11777794444861	241380.243595074\\
3.11787794694867	241329.250351308\\
3.11797794944874	241278.830065336\\
3.1180779519488	241227.836821569\\
3.11817795444886	241177.416535598\\
3.11827795694892	241126.996249626\\
3.11837795944899	241076.00300586\\
3.11847796194905	241025.582719888\\
3.11857796444911	240975.162433917\\
3.11867796694917	240924.16919015\\
3.11877796944924	240873.748904179\\
3.1188779719493	240822.755660412\\
3.11897797444936	240772.33537444\\
3.11907797694942	240721.915088469\\
3.11917797944949	240670.921844702\\
3.11927798194955	240620.501558731\\
3.11937798444961	240570.081272759\\
3.11947798694967	240519.088028993\\
3.11957798944974	240468.667743021\\
3.1196779919498	240417.674499254\\
3.11977799444986	240367.254213283\\
3.11987799694992	240316.833927311\\
3.11997799944999	240265.840683545\\
3.12007800195005	240215.420397573\\
3.12017800445011	240165.000111602\\
3.12027800695017	240114.006867835\\
3.12037800945024	240063.586581864\\
3.1204780119503	240012.593338097\\
3.12057801445036	239962.173052125\\
3.12067801695042	239911.752766154\\
3.12077801945049	239860.759522387\\
3.12087802195055	239810.339236416\\
3.12097802445061	239759.345992649\\
3.12107802695067	239708.925706678\\
3.12117802945074	239658.505420706\\
3.1212780319508	239607.512176939\\
3.12137803445086	239557.091890968\\
3.12147803695092	239506.671604996\\
3.12157803945099	239455.67836123\\
3.12167804195105	239405.258075258\\
3.12177804445111	239354.264831492\\
3.12187804695117	239303.84454552\\
3.12197804945124	239253.424259549\\
3.1220780519513	239202.431015782\\
3.12217805445136	239152.01072981\\
3.12227805695142	239101.590443839\\
3.12237805945149	239050.597200072\\
3.12247806195155	239000.176914101\\
3.12257806445161	238949.183670334\\
3.12267806695167	238898.763384363\\
3.12277806945174	238848.343098391\\
3.1228780719518	238797.349854625\\
3.12297807445186	238746.929568653\\
3.12307807695192	238696.509282681\\
3.12317807945199	238645.516038915\\
3.12327808195205	238595.095752943\\
3.12337808445211	238544.102509177\\
3.12347808695217	238493.682223205\\
3.12357808945224	238443.261937234\\
3.1236780919523	238392.268693467\\
3.12377809445236	238341.848407496\\
3.12387809695242	238290.855163729\\
3.12397809945249	238240.434877757\\
3.12407810195255	238190.014591786\\
3.12417810445261	238139.021348019\\
3.12427810695267	238088.601062048\\
3.12437810945274	238038.180776076\\
3.1244781119528	237987.18753231\\
3.12457811445286	237936.767246338\\
3.12467811695292	237885.774002571\\
3.12477811945299	237835.3537166\\
3.12487812195305	237784.933430628\\
3.12497812445311	237733.940186862\\
3.12507812695317	237683.51990089\\
3.12517812945324	237633.099614919\\
3.1252781319533	237582.106371152\\
3.12537813445336	237531.686085181\\
3.12547813695342	237480.692841414\\
3.12557813945349	237430.272555442\\
3.12567814195355	237379.852269471\\
3.12577814445361	237328.859025704\\
3.12587814695367	237278.438739733\\
3.12597814945374	237228.018453761\\
3.1260781519538	237177.025209995\\
3.12617815445386	237126.604924023\\
3.12627815695392	237075.611680256\\
3.12637815945399	237025.191394285\\
3.12647816195405	236974.771108313\\
3.12657816445411	236923.777864547\\
3.12667816695417	236873.357578575\\
3.12677816945424	236822.937292604\\
3.1268781719543	236771.944048837\\
3.12697817445436	236721.523762866\\
3.12707817695442	236670.530519099\\
3.12717817945449	236620.110233127\\
3.12727818195455	236569.689947156\\
3.12737818445461	236518.696703389\\
3.12747818695467	236468.276417418\\
3.12757818945474	236417.856131446\\
3.1276781919548	236366.86288768\\
3.12777819445486	236316.442601708\\
3.12787819695492	236265.449357941\\
3.12797819945499	236215.02907197\\
3.12807820195505	236164.608785998\\
3.12817820445511	236113.615542232\\
3.12827820695517	236063.19525626\\
3.12837820945524	236012.774970289\\
3.1284782119553	235961.781726522\\
3.12857821445536	235911.361440551\\
3.12867821695542	235860.941154579\\
3.12877821945549	235809.947910812\\
3.12887822195555	235759.527624841\\
3.12897822445561	235708.534381074\\
3.12907822695567	235658.114095103\\
3.12917822945574	235607.693809131\\
3.1292782319558	235556.700565365\\
3.12937823445586	235506.280279393\\
3.12947823695592	235455.859993422\\
3.12957823945599	235404.866749655\\
3.12967824195605	235354.446463683\\
3.12977824445611	235304.026177712\\
3.12987824695617	235253.032933945\\
3.12997824945624	235202.612647974\\
3.1300782519563	235152.192362002\\
3.13017825445636	235101.199118236\\
3.13027825695642	235050.778832264\\
3.13037825945649	234999.785588497\\
3.13047826195655	234949.365302526\\
3.13057826445661	234898.945016554\\
3.13067826695667	234847.951772788\\
3.13077826945674	234797.531486816\\
3.1308782719568	234747.111200845\\
3.13097827445686	234696.117957078\\
3.13107827695692	234645.697671107\\
3.13117827945699	234595.277385135\\
3.13127828195705	234544.284141368\\
3.13137828445711	234493.863855397\\
3.13147828695717	234443.443569425\\
3.13157828945724	234392.450325659\\
3.1316782919573	234342.030039687\\
3.13177829445736	234291.609753716\\
3.13187829695742	234240.616509949\\
3.13197829945749	234190.196223978\\
3.13207830195755	234139.775938006\\
3.13217830445761	234088.782694239\\
3.13227830695767	234038.362408268\\
3.13237830945774	233987.942122296\\
3.1324783119578	233936.94887853\\
3.13257831445786	233886.528592558\\
3.13267831695792	233835.535348792\\
3.13277831945799	233785.11506282\\
3.13287832195805	233734.694776849\\
3.13297832445811	233683.701533082\\
3.13307832695817	233633.28124711\\
3.13317832945824	233582.860961139\\
3.1332783319583	233531.867717372\\
3.13337833445836	233481.447431401\\
3.13347833695842	233431.027145429\\
3.13357833945849	233380.033901663\\
3.13367834195855	233329.613615691\\
3.13377834445861	233279.19332972\\
3.13387834695867	233228.200085953\\
3.13397834945874	233177.779799981\\
3.1340783519588	233127.35951401\\
3.13417835445886	233076.939228038\\
3.13427835695892	233025.945984272\\
3.13437835945899	232975.5256983\\
3.13447836195905	232925.105412329\\
3.13457836445911	232874.112168562\\
3.13467836695917	232823.691882591\\
3.13477836945924	232773.271596619\\
3.1348783719593	232722.278352852\\
3.13497837445936	232671.858066881\\
3.13507837695942	232621.437780909\\
3.13517837945949	232570.444537143\\
3.13527838195955	232520.024251171\\
3.13537838445961	232469.6039652\\
3.13547838695967	232418.610721433\\
3.13557838945974	232368.190435462\\
3.1356783919598	232317.77014949\\
3.13577839445986	232266.776905723\\
3.13587839695992	232216.356619752\\
3.13597839945999	232165.93633378\\
3.13607840196005	232115.516047809\\
3.13617840446011	232064.522804042\\
3.13627840696017	232014.102518071\\
3.13637840946024	231963.682232099\\
3.1364784119603	231912.688988333\\
3.13657841446036	231862.268702361\\
3.13667841696042	231811.84841639\\
3.13677841946049	231760.855172623\\
3.13687842196055	231710.434886651\\
3.13697842446061	231660.01460068\\
3.13707842696067	231609.594314708\\
3.13717842946074	231558.601070942\\
3.1372784319608	231508.18078497\\
3.13737843446086	231457.760498999\\
3.13747843696092	231406.767255232\\
3.13757843946099	231356.346969261\\
3.13767844196105	231305.926683289\\
3.13777844446111	231254.933439522\\
3.13787844696117	231204.513153551\\
3.13797844946124	231154.092867579\\
3.1380784519613	231103.672581608\\
3.13817845446136	231052.679337841\\
3.13827845696142	231002.25905187\\
3.13837845946149	230951.838765898\\
3.13847846196155	230901.418479927\\
3.13857846446161	230850.42523616\\
3.13867846696167	230800.004950188\\
3.13877846946174	230749.584664217\\
3.1388784719618	230698.59142045\\
3.13897847446186	230648.171134479\\
3.13907847696192	230597.750848507\\
3.13917847946199	230547.330562536\\
3.13927848196205	230496.337318769\\
3.13937848446211	230445.917032798\\
3.13947848696217	230395.496746826\\
3.13957848946224	230345.076460855\\
3.1396784919623	230294.083217088\\
3.13977849446236	230243.662931116\\
3.13987849696242	230193.242645145\\
3.13997849946249	230142.249401378\\
3.14007850196255	230091.829115407\\
3.14017850446261	230041.408829435\\
3.14027850696267	229990.988543464\\
3.14037850946274	229939.995299697\\
3.1404785119628	229889.575013726\\
3.14057851446286	229839.154727754\\
3.14067851696292	229788.734441783\\
3.14077851946299	229737.741198016\\
3.14087852196305	229687.320912044\\
3.14097852446311	229636.900626073\\
3.14107852696317	229586.480340101\\
3.14117852946324	229535.487096335\\
3.1412785319633	229485.066810363\\
3.14137853446336	229434.646524392\\
3.14147853696342	229384.22623842\\
3.14157853946349	229333.805952449\\
3.14167854196355	229282.812708682\\
3.14177854446361	229232.392422711\\
3.14187854696367	229181.972136739\\
3.14197854946374	229131.551850768\\
3.1420785519638	229080.558607001\\
3.14217855446386	229030.138321029\\
3.14227855696392	228979.718035058\\
3.14237855946399	228929.297749086\\
3.14247856196405	228878.30450532\\
3.14257856446411	228827.884219348\\
3.14267856696417	228777.463933377\\
3.14277856946424	228727.043647405\\
3.1428785719643	228676.623361434\\
3.14297857446436	228625.630117667\\
3.14307857696442	228575.209831695\\
3.14317857946449	228524.789545724\\
3.14327858196455	228474.369259752\\
3.14337858446461	228423.376015986\\
3.14347858696467	228372.955730014\\
3.14357858946474	228322.535444043\\
3.1436785919648	228272.115158071\\
3.14377859446486	228221.6948721\\
3.14387859696492	228170.701628333\\
3.14397859946499	228120.281342362\\
3.14407860196505	228069.86105639\\
3.14417860446511	228019.440770419\\
3.14427860696517	227969.020484447\\
3.14437860946524	227918.02724068\\
3.1444786119653	227867.606954709\\
3.14457861446536	227817.186668737\\
3.14467861696542	227766.766382766\\
3.14477861946549	227716.346096794\\
3.14487862196555	227665.925810823\\
3.14497862446561	227614.932567056\\
3.14507862696567	227564.512281085\\
3.14517862946574	227514.091995113\\
3.1452786319658	227463.671709142\\
3.14537863446586	227413.25142317\\
3.14547863696592	227362.258179404\\
3.14557863946599	227311.837893432\\
3.14567864196605	227261.417607461\\
3.14577864446611	227210.997321489\\
3.14587864696617	227160.577035517\\
3.14597864946624	227110.156749546\\
3.1460786519663	227059.163505779\\
3.14617865446636	227008.743219808\\
3.14627865696642	226958.322933836\\
3.14637865946649	226907.902647865\\
3.14647866196655	226857.482361893\\
3.14657866446661	226807.062075922\\
3.14667866696667	226756.64178995\\
3.14677866946674	226705.648546184\\
3.1468786719668	226655.228260212\\
3.14697867446686	226604.807974241\\
3.14707867696692	226554.387688269\\
3.14717867946699	226503.967402298\\
3.14727868196705	226453.547116326\\
3.14737868446711	226402.553872559\\
3.14747868696717	226352.133586588\\
3.14757868946724	226301.713300616\\
3.1476786919673	226251.293014645\\
3.14777869446736	226200.872728673\\
3.14787869696742	226150.452442702\\
3.14797869946749	226100.03215673\\
3.14807870196755	226049.611870759\\
3.14817870446761	225998.618626992\\
3.14827870696767	225948.198341021\\
3.14837870946774	225897.778055049\\
3.1484787119678	225847.357769078\\
3.14857871446786	225796.937483106\\
3.14867871696792	225746.517197135\\
3.14877871946799	225696.096911163\\
3.14887872196805	225645.676625192\\
3.14897872446811	225594.683381425\\
3.14907872696817	225544.263095453\\
3.14917872946824	225493.842809482\\
3.1492787319683	225443.42252351\\
3.14937873446836	225393.002237539\\
3.14947873696842	225342.581951567\\
3.14957873946849	225292.161665596\\
3.14967874196855	225241.741379624\\
3.14977874446861	225191.321093653\\
3.14987874696867	225140.900807681\\
3.14997874946874	225090.48052171\\
3.1500787519688	225039.487277943\\
3.15017875446886	224989.066991972\\
3.15027875696892	224938.646706\\
3.15037875946899	224888.226420029\\
3.15047876196905	224837.806134057\\
3.15057876446911	224787.385848086\\
3.15067876696917	224736.965562114\\
3.15077876946924	224686.545276143\\
3.1508787719693	224636.124990171\\
3.15097877446936	224585.7047042\\
3.15107877696942	224535.284418228\\
3.15117877946949	224484.864132257\\
3.15127878196955	224434.443846285\\
3.15137878446961	224383.450602518\\
3.15147878696967	224333.030316547\\
3.15157878946974	224282.610030575\\
3.1516787919698	224232.189744604\\
3.15177879446986	224181.769458632\\
3.15187879696992	224131.349172661\\
3.15197879946999	224080.928886689\\
3.15207880197005	224030.508600718\\
3.15217880447011	223980.088314746\\
3.15227880697017	223929.668028775\\
3.15237880947024	223879.247742803\\
3.1524788119703	223828.827456832\\
3.15257881447036	223778.40717086\\
3.15267881697042	223727.986884889\\
3.15277881947049	223677.566598917\\
3.15287882197055	223627.146312946\\
3.15297882447061	223576.726026974\\
3.15307882697067	223526.305741003\\
3.15317882947074	223475.885455031\\
3.1532788319708	223425.46516906\\
3.15337883447086	223375.044883088\\
3.15347883697092	223324.624597117\\
3.15357883947099	223274.204311145\\
3.15367884197105	223223.784025174\\
3.15377884447111	223173.363739202\\
3.15387884697117	223122.943453231\\
3.15397884947124	223072.523167259\\
3.1540788519713	223022.102881288\\
3.15417885447136	222971.682595316\\
3.15427885697142	222921.262309345\\
3.15437885947149	222870.842023373\\
3.15447886197155	222820.421737402\\
3.15457886447161	222770.00145143\\
3.15467886697167	222719.581165458\\
3.15477886947174	222669.160879487\\
3.1548788719718	222618.740593515\\
3.15497887447186	222568.320307544\\
3.15507887697192	222517.900021572\\
3.15517887947199	222467.479735601\\
3.15527888197205	222417.059449629\\
3.15537888447211	222366.639163658\\
3.15547888697217	222316.218877686\\
3.15557888947224	222265.798591715\\
3.1556788919723	222215.378305743\\
3.15577889447236	222164.958019772\\
3.15587889697242	222114.5377338\\
3.15597889947249	222064.117447829\\
3.15607890197255	222013.697161857\\
3.15617890447261	221963.276875886\\
3.15627890697267	221912.856589914\\
3.15637890947274	221862.436303943\\
3.1564789119728	221812.016017971\\
3.15657891447286	221761.595732\\
3.15667891697292	221711.748403823\\
3.15677891947299	221661.328117852\\
3.15687892197305	221610.90783188\\
3.15697892447311	221560.487545909\\
3.15707892697317	221510.067259937\\
3.15717892947324	221459.646973966\\
3.1572789319733	221409.226687994\\
3.15737893447336	221358.806402023\\
3.15747893697342	221308.386116051\\
3.15757893947349	221257.96583008\\
3.15767894197355	221207.545544108\\
3.15777894447361	221157.125258137\\
3.15787894697367	221106.704972165\\
3.15797894947374	221056.857643989\\
3.1580789519738	221006.437358017\\
3.15817895447386	220956.017072046\\
3.15827895697392	220905.596786074\\
3.15837895947399	220855.176500103\\
3.15847896197405	220804.756214131\\
3.15857896447411	220754.33592816\\
3.15867896697417	220703.915642188\\
3.15877896947424	220653.495356217\\
3.1588789719743	220603.64802804\\
3.15897897447436	220553.227742069\\
3.15907897697442	220502.807456097\\
3.15917897947449	220452.387170126\\
3.15927898197455	220401.966884154\\
3.15937898447461	220351.546598183\\
3.15947898697467	220301.126312211\\
3.15957898947474	220251.278984035\\
3.1596789919748	220200.858698063\\
3.15977899447486	220150.438412092\\
3.15987899697492	220100.01812612\\
3.15997899947499	220049.597840149\\
3.16007900197505	219999.177554177\\
3.16017900447511	219948.757268206\\
3.16027900697517	219898.909940029\\
3.16037900947524	219848.489654058\\
3.1604790119753	219798.069368086\\
3.16057901447536	219747.649082115\\
3.16067901697542	219697.228796143\\
3.16077901947549	219646.808510172\\
3.16087902197555	219596.961181995\\
3.16097902447561	219546.540896024\\
3.16107902697567	219496.120610052\\
3.16117902947574	219445.700324081\\
3.1612790319758	219395.280038109\\
3.16137903447586	219344.859752138\\
3.16147903697592	219295.012423962\\
3.16157903947599	219244.59213799\\
3.16167904197605	219194.171852019\\
3.16177904447611	219143.751566047\\
3.16187904697617	219093.331280076\\
3.16197904947624	219043.483951899\\
3.1620790519763	218993.063665928\\
3.16217905447636	218942.643379956\\
3.16227905697642	218892.223093985\\
3.16237905947649	218842.375765808\\
3.16247906197655	218791.955479837\\
3.16257906447661	218741.535193865\\
3.16267906697667	218691.114907894\\
3.16277906947674	218640.694621922\\
3.1628790719768	218590.847293746\\
3.16297907447686	218540.427007774\\
3.16307907697692	218490.006721803\\
3.16317907947699	218439.586435831\\
3.16327908197705	218389.739107655\\
3.16337908447711	218339.318821683\\
3.16347908697717	218288.898535712\\
3.16357908947724	218238.47824974\\
3.1636790919773	218188.630921564\\
3.16377909447736	218138.210635592\\
3.16387909697742	218087.790349621\\
3.16397909947749	218037.370063649\\
3.16407910197755	217987.522735473\\
3.16417910447761	217937.102449501\\
3.16427910697767	217886.68216353\\
3.16437910947774	217836.834835354\\
3.1644791119778	217786.414549382\\
3.16457911447786	217735.994263411\\
3.16467911697792	217685.573977439\\
3.16477911947799	217635.726649263\\
3.16487912197805	217585.306363291\\
3.16497912447811	217534.88607732\\
3.16507912697817	217485.038749143\\
3.16517912947824	217434.618463172\\
3.1652791319783	217384.1981772\\
3.16537913447836	217333.777891229\\
3.16547913697842	217283.930563052\\
3.16557913947849	217233.510277081\\
3.16567914197855	217183.089991109\\
3.16577914447861	217133.242662933\\
3.16587914697867	217082.822376961\\
3.16597914947874	217032.40209099\\
3.1660791519788	216982.554762814\\
3.16617915447886	216932.134476842\\
3.16627915697892	216881.714190871\\
3.16637915947899	216831.866862694\\
3.16647916197905	216781.446576723\\
3.16657916447911	216731.026290751\\
3.16667916697917	216681.178962575\\
3.16677916947924	216630.758676603\\
3.1668791719793	216580.338390632\\
3.16697917447936	216530.491062455\\
3.16707917697942	216480.070776484\\
3.16717917947949	216430.223448307\\
3.16727918197955	216379.803162336\\
3.16737918447961	216329.382876364\\
3.16747918697967	216279.535548188\\
3.16757918947974	216229.115262216\\
3.1676791919798	216178.694976245\\
3.16777919447986	216128.847648069\\
3.16787919697992	216078.427362097\\
3.16797919947999	216028.580033921\\
3.16807920198005	215978.159747949\\
3.16817920448011	215927.739461978\\
3.16827920698017	215877.892133801\\
3.16837920948024	215827.47184783\\
3.1684792119803	215777.624519653\\
3.16857921448036	215727.204233682\\
3.16867921698042	215676.78394771\\
3.16877921948049	215626.936619534\\
3.16887922198055	215576.516333562\\
3.16897922448061	215526.669005386\\
3.16907922698067	215476.248719415\\
3.16917922948074	215426.401391238\\
3.1692792319808	215375.981105267\\
3.16937923448086	215325.560819295\\
3.16947923698092	215275.713491119\\
3.16957923948099	215225.293205147\\
3.16967924198105	215175.445876971\\
3.16977924448111	215125.025590999\\
3.16987924698117	215075.178262823\\
3.16997924948124	215024.757976852\\
3.1700792519813	214974.910648675\\
3.17017925448136	214924.490362704\\
3.17027925698142	214874.643034527\\
3.17037925948149	214824.222748556\\
3.17047926198155	214773.802462584\\
3.17057926448161	214723.955134408\\
3.17067926698167	214673.534848436\\
3.17077926948174	214623.68752026\\
3.1708792719818	214573.267234288\\
3.17097927448186	214523.419906112\\
3.17107927698192	214472.999620141\\
3.17117927948199	214423.152291964\\
3.17127928198205	214372.732005993\\
3.17137928448211	214322.884677816\\
3.17147928698217	214272.464391845\\
3.17157928948224	214222.617063668\\
3.1716792919823	214172.196777697\\
3.17177929448236	214122.34944952\\
3.17187929698242	214071.929163549\\
3.17197929948249	214022.081835373\\
3.17207930198255	213972.234507196\\
3.17217930448261	213921.814221225\\
3.17227930698267	213871.966893048\\
3.17237930948274	213821.546607077\\
3.1724793119828	213771.6992789\\
3.17257931448286	213721.278992929\\
3.17267931698292	213671.431664753\\
3.17277931948299	213621.011378781\\
3.17287932198305	213571.164050605\\
3.17297932448311	213520.743764633\\
3.17307932698317	213470.896436457\\
3.17317932948324	213421.04910828\\
3.1732793319833	213370.628822309\\
3.17337933448336	213320.781494132\\
3.17347933698342	213270.361208161\\
3.17357933948349	213220.513879985\\
3.17367934198355	213170.666551808\\
3.17377934448361	213120.246265837\\
3.17387934698367	213070.39893766\\
3.17397934948374	213019.978651689\\
3.1740793519838	212970.131323512\\
3.17417935448386	212920.283995336\\
3.17427935698392	212869.863709364\\
3.17437935948399	212820.016381188\\
3.17447936198405	212769.596095217\\
3.17457936448411	212719.74876704\\
3.17467936698417	212669.901438864\\
3.17477936948424	212619.481152892\\
3.1748793719843	212569.633824716\\
3.17497937448436	212519.78649654\\
3.17507937698442	212469.366210568\\
3.17517937948449	212419.518882392\\
3.17527938198455	212369.09859642\\
3.17537938448461	212319.251268244\\
3.17547938698467	212269.403940067\\
3.17557938948474	212218.983654096\\
3.1756793919848	212169.13632592\\
3.17577939448486	212119.288997743\\
3.17587939698492	212068.868711772\\
3.17597939948499	212019.021383595\\
3.17607940198505	211969.174055419\\
3.17617940448511	211918.753769447\\
3.17627940698517	211868.906441271\\
3.17637940948524	211819.059113095\\
3.1764794119853	211769.211784918\\
3.17657941448536	211718.791498947\\
3.17667941698542	211668.94417077\\
3.17677941948549	211619.096842594\\
3.17687942198555	211568.676556622\\
3.17697942448561	211518.829228446\\
3.17707942698567	211468.98190027\\
3.17717942948574	211418.561614298\\
3.1772794319858	211368.714286122\\
3.17737943448586	211318.866957945\\
3.17747943698592	211269.019629769\\
3.17757943948599	211218.599343797\\
3.17767944198605	211168.752015621\\
3.17777944448611	211118.904687445\\
3.17787944698617	211069.057359268\\
3.17797944948624	211018.637073297\\
3.1780794519863	210968.78974512\\
3.17817945448636	210918.942416944\\
3.17827945698642	210869.095088768\\
3.17837945948649	210818.674802796\\
3.17847946198655	210768.82747462\\
3.17857946448661	210718.980146443\\
3.17867946698667	210669.132818267\\
3.17877946948674	210619.285490091\\
3.1788794719868	210568.865204119\\
3.17897947448686	210519.017875943\\
3.17907947698692	210469.170547766\\
3.17917947948699	210419.32321959\\
3.17927948198705	210369.475891414\\
3.17937948448711	210319.055605442\\
3.17947948698717	210269.208277266\\
3.17957948948724	210219.360949089\\
3.1796794919873	210169.513620913\\
3.17977949448736	210119.666292737\\
3.17987949698742	210069.246006765\\
3.17997949948749	210019.398678589\\
3.18007950198755	209969.551350412\\
3.18017950448761	209919.704022236\\
3.18027950698767	209869.85669406\\
3.18037950948774	209820.009365883\\
3.1804795119878	209769.589079912\\
3.18057951448786	209719.741751735\\
3.18067951698792	209669.894423559\\
3.18077951948799	209620.047095382\\
3.18087952198805	209570.199767206\\
3.18097952448811	209520.35243903\\
3.18107952698817	209470.505110853\\
3.18117952948824	209420.657782677\\
3.1812795319883	209370.237496705\\
3.18137953448836	209320.390168529\\
3.18147953698842	209270.542840353\\
3.18157953948849	209220.695512176\\
3.18167954198855	209170.848184\\
3.18177954448861	209121.000855824\\
3.18187954698867	209071.153527647\\
3.18197954948874	209021.306199471\\
3.1820795519888	208971.458871294\\
3.18217955448886	208921.611543118\\
3.18227955698892	208871.764214942\\
3.18237955948899	208821.34392897\\
3.18247956198905	208771.496600794\\
3.18257956448911	208721.649272617\\
3.18267956698917	208671.801944441\\
3.18277956948924	208621.954616265\\
3.1828795719893	208572.107288088\\
3.18297957448936	208522.259959912\\
3.18307957698942	208472.412631735\\
3.18317957948949	208422.565303559\\
3.18327958198955	208372.717975383\\
3.18337958448961	208322.870647206\\
3.18347958698967	208273.02331903\\
3.18357958948974	208223.175990854\\
3.1836795919898	208173.328662677\\
3.18377959448986	208123.481334501\\
3.18387959698992	208073.634006324\\
3.18397959948999	208023.786678148\\
3.18407960199005	207973.939349972\\
3.18417960449011	207924.092021795\\
3.18427960699017	207874.244693619\\
3.18437960949024	207824.397365442\\
3.1844796119903	207774.550037266\\
3.18457961449036	207724.70270909\\
3.18467961699042	207674.855380913\\
3.18477961949049	207625.008052737\\
3.18487962199055	207575.160724561\\
3.18497962449061	207525.313396384\\
3.18507962699067	207475.466068208\\
3.18517962949074	207425.618740031\\
3.1852796319908	207375.771411855\\
3.18537963449086	207325.924083679\\
3.18547963699092	207276.076755502\\
3.18557963949099	207226.229427326\\
3.18567964199105	207176.955056945\\
3.18577964449111	207127.107728768\\
3.18587964699117	207077.260400592\\
3.18597964949124	207027.413072416\\
3.1860796519913	206977.565744239\\
3.18617965449136	206927.718416063\\
3.18627965699142	206877.871087886\\
3.18637965949149	206828.02375971\\
3.18647966199155	206778.176431534\\
3.18657966449161	206728.329103357\\
3.18667966699167	206678.481775181\\
3.18677966949174	206629.2074048\\
3.1868796719918	206579.360076623\\
3.18697967449186	206529.512748447\\
3.18707967699192	206479.66542027\\
3.18717967949199	206429.818092094\\
3.18727968199205	206379.970763918\\
3.18737968449211	206330.123435741\\
3.18747968699217	206280.84906536\\
3.18757968949224	206231.001737184\\
3.1876796919923	206181.154409007\\
3.18777969449236	206131.307080831\\
3.18787969699242	206081.459752655\\
3.18797969949249	206031.612424478\\
3.18807970199255	205982.338054097\\
3.18817970449261	205932.49072592\\
3.18827970699267	205882.643397744\\
3.18837970949274	205832.796069568\\
3.1884797119928	205782.948741391\\
3.18857971449286	205733.67437101\\
3.18867971699292	205683.827042834\\
3.18877971949299	205633.979714657\\
3.18887972199305	205584.132386481\\
3.18897972449311	205534.285058305\\
3.18907972699317	205485.010687923\\
3.18917972949324	205435.163359747\\
3.1892797319933	205385.316031571\\
3.18937973449336	205335.468703394\\
3.18947973699342	205286.194333013\\
3.18957973949349	205236.347004837\\
3.18967974199355	205186.49967666\\
3.18977974449361	205136.652348484\\
3.18987974699367	205087.377978103\\
3.18997974949374	205037.530649926\\
3.1900797519938	204987.68332175\\
3.19017975449386	204937.835993573\\
3.19027975699392	204888.561623192\\
3.19037975949399	204838.714295016\\
3.19047976199405	204788.866966839\\
3.19057976449411	204739.592596458\\
3.19067976699417	204689.745268282\\
3.19077976949424	204639.897940105\\
3.1908797719943	204590.050611929\\
3.19097977449436	204540.776241548\\
3.19107977699443	204490.928913371\\
3.19117977949449	204441.081585195\\
3.19127978199455	204391.807214814\\
3.19137978449461	204341.959886637\\
3.19147978699467	204292.112558461\\
3.19157978949474	204242.83818808\\
3.1916797919948	204192.990859903\\
3.19177979449486	204143.143531727\\
3.19187979699492	204093.869161346\\
3.19197979949499	204044.021833169\\
3.19207980199505	203994.747462788\\
3.19217980449511	203944.900134612\\
3.19227980699517	203895.052806435\\
3.19237980949524	203845.778436054\\
3.1924798119953	203795.931107878\\
3.19257981449536	203746.083779701\\
3.19267981699542	203696.80940932\\
3.19277981949549	203646.962081144\\
3.19287982199555	203597.687710762\\
3.19297982449561	203547.840382586\\
3.19307982699568	203497.99305441\\
3.19317982949574	203448.718684028\\
3.1932798319958	203398.871355852\\
3.19337983449586	203349.596985471\\
3.19347983699592	203299.749657294\\
3.19357983949599	203250.475286913\\
3.19367984199605	203200.627958737\\
3.19377984449611	203151.353588355\\
3.19387984699617	203101.506260179\\
3.19397984949624	203051.658932003\\
3.1940798519963	203002.384561621\\
3.19417985449636	202952.537233445\\
3.19427985699642	202903.262863064\\
3.19437985949649	202853.415534887\\
3.19447986199655	202804.141164506\\
3.19457986449661	202754.29383633\\
3.19467986699667	202705.019465949\\
3.19477986949674	202655.172137772\\
3.1948798719968	202605.897767391\\
3.19497987449686	202556.050439215\\
3.19507987699693	202506.776068833\\
3.19517987949699	202456.928740657\\
3.19527988199705	202407.654370276\\
3.19537988449711	202357.807042099\\
3.19547988699718	202308.532671718\\
3.19557988949724	202258.685343542\\
3.1956798919973	202209.41097316\\
3.19577989449736	202160.136602779\\
3.19587989699742	202110.289274603\\
3.19597989949749	202061.014904222\\
3.19607990199755	202011.167576045\\
3.19617990449761	201961.893205664\\
3.19627990699767	201912.045877488\\
3.19637990949774	201862.771507106\\
3.1964799119978	201812.92417893\\
3.19657991449786	201763.649808549\\
3.19667991699792	201714.375438167\\
3.19677991949799	201664.528109991\\
3.19687992199805	201615.25373961\\
3.19697992449811	201565.406411433\\
3.19707992699818	201516.132041052\\
3.19717992949824	201466.857670671\\
3.1972799319983	201417.010342494\\
3.19737993449836	201367.735972113\\
3.19747993699843	201318.461601732\\
3.19757993949849	201268.614273556\\
3.19767994199855	201219.339903174\\
3.19777994449861	201170.065532793\\
3.19787994699867	201120.218204617\\
3.19797994949874	201070.943834235\\
3.1980799519988	201021.096506059\\
3.19817995449886	200971.822135678\\
3.19827995699892	200922.547765297\\
3.19837995949899	200873.273394915\\
3.19847996199905	200823.426066739\\
3.19857996449911	200774.151696358\\
3.19867996699917	200724.877325976\\
3.19877996949924	200675.0299978\\
3.1988799719993	200625.755627419\\
3.19897997449936	200576.481257038\\
3.19907997699943	200526.633928861\\
3.19917997949949	200477.35955848\\
3.19927998199955	200428.085188099\\
3.19937998449961	200378.810817717\\
3.19947998699968	200328.963489541\\
3.19957998949974	200279.68911916\\
3.1996799919998	200230.414748779\\
3.19977999449986	200181.140378397\\
3.19987999699992	200131.293050221\\
3.19997999949999	200082.01867984\\
3.20008000200005	200032.744309458\\
};
\addplot [color=mycolor1,solid,forget plot]
  table[row sep=crcr]{%
3.20008000200005	200032.744309458\\
3.20018000450011	199983.469939077\\
3.20028000700017	199933.622610901\\
3.20038000950024	199884.348240519\\
3.2004800120003	199835.073870138\\
3.20058001450036	199785.799499757\\
3.20068001700042	199736.525129376\\
3.20078001950049	199686.677801199\\
3.20088002200055	199637.403430818\\
3.20098002450061	199588.129060437\\
3.20108002700068	199538.854690056\\
3.20118002950074	199489.580319674\\
3.2012800320008	199439.732991498\\
3.20138003450086	199390.458621117\\
3.20148003700093	199341.184250735\\
3.20158003950099	199291.909880354\\
3.20168004200105	199242.635509973\\
3.20178004450111	199193.361139592\\
3.20188004700118	199144.08676921\\
3.20198004950124	199094.239441034\\
3.2020800520013	199044.965070653\\
3.20218005450136	198995.690700272\\
3.20228005700142	198946.41632989\\
3.20238005950149	198897.141959509\\
3.20248006200155	198847.867589128\\
3.20258006450161	198798.593218747\\
3.20268006700167	198749.318848365\\
3.20278006950174	198700.044477984\\
3.2028800720018	198650.197149808\\
3.20298007450186	198600.922779426\\
3.20308007700193	198551.648409045\\
3.20318007950199	198502.374038664\\
3.20328008200205	198453.099668283\\
3.20338008450211	198403.825297901\\
3.20348008700218	198354.55092752\\
3.20358008950224	198305.276557139\\
3.2036800920023	198256.002186758\\
3.20378009450236	198206.727816376\\
3.20388009700243	198157.453445995\\
3.20398009950249	198108.179075614\\
3.20408010200255	198058.904705233\\
3.20418010450261	198009.630334851\\
3.20428010700267	197960.35596447\\
3.20438010950274	197911.081594089\\
3.2044801120028	197861.807223708\\
3.20458011450286	197812.532853326\\
3.20468011700292	197763.258482945\\
3.20478011950299	197713.984112564\\
3.20488012200305	197664.709742183\\
3.20498012450311	197615.435371801\\
3.20508012700318	197566.16100142\\
3.20518012950324	197516.886631039\\
3.2052801320033	197467.612260658\\
3.20538013450336	197418.337890276\\
3.20548013700343	197369.063519895\\
3.20558013950349	197319.789149514\\
3.20568014200355	197270.514779133\\
3.20578014450361	197221.240408751\\
3.20588014700368	197172.538996165\\
3.20598014950374	197123.264625784\\
3.2060801520038	197073.990255403\\
3.20618015450386	197024.715885022\\
3.20628015700392	196975.44151464\\
3.20638015950399	196926.167144259\\
3.20648016200405	196876.892773878\\
3.20658016450411	196827.618403497\\
3.20668016700417	196778.344033115\\
3.20678016950424	196729.069662734\\
3.2068801720043	196680.368250148\\
3.20698017450436	196631.093879767\\
3.20708017700443	196581.819509385\\
3.20718017950449	196532.545139004\\
3.20728018200455	196483.270768623\\
3.20738018450461	196433.996398242\\
3.20748018700468	196385.294985656\\
3.20758018950474	196336.020615274\\
3.2076801920048	196286.746244893\\
3.20778019450486	196237.471874512\\
3.20788019700493	196188.197504131\\
3.20798019950499	196139.496091544\\
3.20808020200505	196090.221721163\\
3.20818020450511	196040.947350782\\
3.20828020700518	195991.672980401\\
3.20838020950524	195942.398610019\\
3.2084802120053	195893.697197433\\
3.20858021450536	195844.422827052\\
3.20868021700542	195795.148456671\\
3.20878021950549	195745.87408629\\
3.20888022200555	195697.172673703\\
3.20898022450561	195647.898303322\\
3.20908022700568	195598.623932941\\
3.20918022950574	195549.34956256\\
3.2092802320058	195500.648149974\\
3.20938023450586	195451.373779592\\
3.20948023700593	195402.099409211\\
3.20958023950599	195352.82503883\\
3.20968024200605	195304.123626244\\
3.20978024450611	195254.849255863\\
3.20988024700618	195205.574885481\\
3.20998024950624	195156.873472895\\
3.2100802520063	195107.599102514\\
3.21018025450636	195058.324732133\\
3.21028025700643	195009.623319546\\
3.21038025950649	194960.348949165\\
3.21048026200655	194911.074578784\\
3.21058026450661	194862.373166198\\
3.21068026700667	194813.098795817\\
3.21078026950674	194763.824425435\\
3.2108802720068	194715.123012849\\
3.21098027450686	194665.848642468\\
3.21108027700693	194617.147229882\\
3.21118027950699	194567.872859501\\
3.21128028200705	194518.598489119\\
3.21138028450711	194469.897076533\\
3.21148028700718	194420.622706152\\
3.21158028950724	194371.921293566\\
3.2116802920073	194322.646923185\\
3.21178029450736	194273.372552803\\
3.21188029700743	194224.671140217\\
3.21198029950749	194175.396769836\\
3.21208030200755	194126.69535725\\
3.21218030450761	194077.420986869\\
3.21228030700768	194028.719574282\\
3.21238030950774	193979.445203901\\
3.2124803120078	193930.17083352\\
3.21258031450786	193881.469420934\\
3.21268031700793	193832.195050553\\
3.21278031950799	193783.493637966\\
3.21288032200805	193734.219267585\\
3.21298032450811	193685.517854999\\
3.21308032700818	193636.243484618\\
3.21318032950824	193587.542072032\\
3.2132803320083	193538.267701651\\
3.21338033450836	193489.566289064\\
3.21348033700843	193440.291918683\\
3.21358033950849	193391.590506097\\
3.21368034200855	193342.316135716\\
3.21378034450861	193293.61472313\\
3.21388034700868	193244.913310544\\
3.21398034950874	193195.638940162\\
3.2140803520088	193146.937527576\\
3.21418035450886	193097.663157195\\
3.21428035700893	193048.961744609\\
3.21438035950899	192999.687374228\\
3.21448036200905	192950.985961641\\
3.21458036450911	192902.284549055\\
3.21468036700918	192853.010178674\\
3.21478036950924	192804.308766088\\
3.2148803720093	192755.034395707\\
3.21498037450936	192706.332983121\\
3.21508037700943	192657.631570534\\
3.21518037950949	192608.357200153\\
3.21528038200955	192559.655787567\\
3.21538038450961	192510.381417186\\
3.21548038700968	192461.6800046\\
3.21558038950974	192412.978592014\\
3.2156803920098	192363.704221632\\
3.21578039450986	192315.002809046\\
3.21588039700993	192266.30139646\\
3.21598039950999	192217.027026079\\
3.21608040201005	192168.325613493\\
3.21618040451011	192119.624200907\\
3.21628040701018	192070.349830525\\
3.21638040951024	192021.648417939\\
3.2164804120103	191972.947005353\\
3.21658041451036	191924.245592767\\
3.21668041701043	191874.971222386\\
3.21678041951049	191826.2698098\\
3.21688042201055	191777.568397214\\
3.21698042451061	191728.294026832\\
3.21708042701068	191679.592614246\\
3.21718042951074	191630.89120166\\
3.2172804320108	191582.189789074\\
3.21738043451086	191532.915418693\\
3.21748043701093	191484.214006107\\
3.21758043951099	191435.51259352\\
3.21768044201105	191386.811180934\\
3.21778044451111	191338.109768348\\
3.21788044701118	191288.835397967\\
3.21798044951124	191240.133985381\\
3.2180804520113	191191.432572795\\
3.21818045451136	191142.731160209\\
3.21828045701143	191094.029747622\\
3.21838045951149	191044.755377241\\
3.21848046201155	190996.053964655\\
3.21858046451161	190947.352552069\\
3.21868046701168	190898.651139483\\
3.21878046951174	190849.949726897\\
3.2188804720118	190801.248314311\\
3.21898047451186	190751.973943929\\
3.21908047701193	190703.272531343\\
3.21918047951199	190654.571118757\\
3.21928048201205	190605.869706171\\
3.21938048451211	190557.168293585\\
3.21948048701218	190508.466880999\\
3.21958048951224	190459.765468413\\
3.2196804920123	190411.064055826\\
3.21978049451236	190362.36264324\\
3.21988049701243	190313.088272859\\
3.21998049951249	190264.386860273\\
3.22008050201255	190215.685447687\\
3.22018050451261	190166.984035101\\
3.22028050701268	190118.282622515\\
3.22038050951274	190069.581209929\\
3.2204805120128	190020.879797342\\
3.22058051451286	189972.178384756\\
3.22068051701293	189923.47697217\\
3.22078051951299	189874.775559584\\
3.22088052201305	189826.074146998\\
3.22098052451311	189777.372734412\\
3.22108052701318	189728.671321826\\
3.22118052951324	189679.96990924\\
3.2212805320133	189631.268496653\\
3.22138053451336	189582.567084067\\
3.22148053701343	189533.865671481\\
3.22158053951349	189485.164258895\\
3.22168054201355	189436.462846309\\
3.22178054451361	189387.761433723\\
3.22188054701368	189339.060021137\\
3.22198054951374	189290.358608551\\
3.2220805520138	189241.657195964\\
3.22218055451386	189192.955783378\\
3.22228055701393	189144.254370792\\
3.22238055951399	189095.552958206\\
3.22248056201405	189047.424503415\\
3.22258056451411	188998.723090829\\
3.22268056701418	188950.021678243\\
3.22278056951424	188901.320265657\\
3.2228805720143	188852.618853071\\
3.22298057451436	188803.917440485\\
3.22308057701443	188755.216027898\\
3.22318057951449	188706.514615312\\
3.22328058201455	188657.813202726\\
3.22338058451461	188609.684747935\\
3.22348058701468	188560.983335349\\
3.22358058951474	188512.281922763\\
3.2236805920148	188463.580510177\\
3.22378059451486	188414.879097591\\
3.22388059701493	188366.177685005\\
3.22398059951499	188318.049230214\\
3.22408060201505	188269.347817627\\
3.22418060451511	188220.646405041\\
3.22428060701518	188171.944992455\\
3.22438060951524	188123.243579869\\
3.2244806120153	188075.115125078\\
3.22458061451536	188026.413712492\\
3.22468061701543	187977.712299906\\
3.22478061951549	187929.01088732\\
3.22488062201555	187880.309474734\\
3.22498062451561	187832.181019943\\
3.22508062701568	187783.479607357\\
3.22518062951574	187734.77819477\\
3.2252806320158	187686.076782184\\
3.22538063451586	187637.948327393\\
3.22548063701593	187589.246914807\\
3.22558063951599	187540.545502221\\
3.22568064201605	187492.41704743\\
3.22578064451611	187443.715634844\\
3.22588064701618	187395.014222258\\
3.22598064951624	187346.312809672\\
3.2260806520163	187298.184354881\\
3.22618065451636	187249.482942295\\
3.22628065701643	187200.781529708\\
3.22638065951649	187152.653074917\\
3.22648066201655	187103.951662331\\
3.22658066451661	187055.250249745\\
3.22668066701668	187007.121794954\\
3.22678066951674	186958.420382368\\
3.2268806720168	186910.291927577\\
3.22698067451686	186861.590514991\\
3.22708067701693	186812.889102405\\
3.22718067951699	186764.760647614\\
3.22728068201705	186716.059235028\\
3.22738068451711	186667.930780237\\
3.22748068701718	186619.229367651\\
3.22758068951724	186570.527955065\\
3.2276806920173	186522.399500274\\
3.22778069451736	186473.698087687\\
3.22788069701743	186425.569632896\\
3.22798069951749	186376.86822031\\
3.22808070201755	186328.739765519\\
3.22818070451761	186280.038352933\\
3.22828070701768	186231.336940347\\
3.22838070951774	186183.208485556\\
3.2284807120178	186134.50707297\\
3.22858071451786	186086.378618179\\
3.22868071701793	186037.677205593\\
3.22878071951799	185989.548750802\\
3.22888072201805	185940.847338216\\
3.22898072451811	185892.718883425\\
3.22908072701818	185844.017470839\\
3.22918072951824	185795.889016048\\
3.2292807320183	185747.760561257\\
3.22938073451836	185699.059148671\\
3.22948073701843	185650.93069388\\
3.22958073951849	185602.229281293\\
3.22968074201855	185554.100826503\\
3.22978074451861	185505.399413916\\
3.22988074701868	185457.270959125\\
3.22998074951874	185408.569546539\\
3.2300807520188	185360.441091748\\
3.23018075451886	185312.312636957\\
3.23028075701893	185263.611224371\\
3.23038075951899	185215.48276958\\
3.23048076201905	185166.781356994\\
3.23058076451911	185118.652902203\\
3.23068076701918	185070.524447412\\
3.23078076951924	185021.823034826\\
3.2308807720193	184973.694580035\\
3.23098077451936	184925.566125244\\
3.23108077701943	184876.864712658\\
3.23118077951949	184828.736257867\\
3.23128078201955	184780.607803076\\
3.23138078451961	184731.90639049\\
3.23148078701968	184683.777935699\\
3.23158078951974	184635.649480908\\
3.2316807920198	184586.948068322\\
3.23178079451986	184538.819613531\\
3.23188079701993	184490.69115874\\
3.23198079951999	184442.562703949\\
3.23208080202005	184393.861291363\\
3.23218080452011	184345.732836572\\
3.23228080702018	184297.604381781\\
3.23238080952024	184248.902969194\\
3.2324808120203	184200.774514403\\
3.23258081452036	184152.646059612\\
3.23268081702043	184104.517604822\\
3.23278081952049	184056.389150031\\
3.23288082202055	184007.687737444\\
3.23298082452061	183959.559282653\\
3.23308082702068	183911.430827862\\
3.23318082952074	183863.302373071\\
3.2332808320208	183815.17391828\\
3.23338083452086	183766.472505694\\
3.23348083702093	183718.344050903\\
3.23358083952099	183670.215596112\\
3.23368084202105	183622.087141321\\
3.23378084452111	183573.95868653\\
3.23388084702118	183525.830231739\\
3.23398084952124	183477.128819153\\
3.2340808520213	183429.000364362\\
3.23418085452136	183380.871909571\\
3.23428085702143	183332.74345478\\
3.23438085952149	183284.614999989\\
3.23448086202155	183236.486545198\\
3.23458086452161	183188.358090407\\
3.23468086702168	183140.229635616\\
3.23478086952174	183091.52822303\\
3.2348808720218	183043.399768239\\
3.23498087452186	182995.271313448\\
3.23508087702193	182947.142858657\\
3.23518087952199	182899.014403866\\
3.23528088202205	182850.885949075\\
3.23538088452211	182802.757494284\\
3.23548088702218	182754.629039493\\
3.23558088952224	182706.500584702\\
3.2356808920223	182658.372129911\\
3.23578089452236	182610.24367512\\
3.23588089702243	182562.115220329\\
3.23598089952249	182513.986765538\\
3.23608090202255	182465.858310747\\
3.23618090452261	182417.729855956\\
3.23628090702268	182369.601401165\\
3.23638090952274	182321.472946374\\
3.2364809120228	182273.344491583\\
3.23658091452286	182225.216036792\\
3.23668091702293	182177.087582001\\
3.23678091952299	182128.95912721\\
3.23688092202305	182080.830672419\\
3.23698092452311	182032.702217628\\
3.23708092702318	181985.146720633\\
3.23718092952324	181937.018265842\\
3.2372809320233	181888.889811051\\
3.23738093452336	181840.76135626\\
3.23748093702343	181792.632901469\\
3.23758093952349	181744.504446678\\
3.23768094202355	181696.375991887\\
3.23778094452361	181648.247537096\\
3.23788094702368	181600.119082305\\
3.23798094952374	181552.563585309\\
3.2380809520238	181504.435130518\\
3.23818095452386	181456.306675727\\
3.23828095702393	181408.178220936\\
3.23838095952399	181360.049766145\\
3.23848096202405	181311.921311354\\
3.23858096452411	181264.365814358\\
3.23868096702418	181216.237359567\\
3.23878096952424	181168.108904776\\
3.2388809720243	181119.980449985\\
3.23898097452436	181071.851995194\\
3.23908097702443	181024.296498198\\
3.23918097952449	180976.168043407\\
3.23928098202455	180928.039588616\\
3.23938098452461	180879.911133825\\
3.23948098702468	180832.355636829\\
3.23958098952474	180784.227182038\\
3.2396809920248	180736.098727247\\
3.23978099452486	180687.970272456\\
3.23988099702493	180640.414775461\\
3.23998099952499	180592.28632067\\
3.24008100202505	180544.157865879\\
3.24018100452511	180496.602368883\\
3.24028100702518	180448.473914092\\
3.24038100952524	180400.345459301\\
3.2404810120253	180352.789962305\\
3.24058101452536	180304.661507514\\
3.24068101702543	180256.533052723\\
3.24078101952549	180208.977555727\\
3.24088102202555	180160.849100936\\
3.24098102452561	180112.720646145\\
3.24108102702568	180065.165149149\\
3.24118102952574	180017.036694358\\
3.2412810320258	179968.908239567\\
3.24138103452586	179921.352742571\\
3.24148103702593	179873.22428778\\
3.24158103952599	179825.668790785\\
3.24168104202605	179777.540335994\\
3.24178104452611	179729.411881203\\
3.24188104702618	179681.856384207\\
3.24198104952624	179633.727929416\\
3.2420810520263	179586.17243242\\
3.24218105452636	179538.043977629\\
3.24228105702643	179490.488480633\\
3.24238105952649	179442.360025842\\
3.24248106202655	179394.804528846\\
3.24258106452661	179346.676074055\\
3.24268106702668	179299.120577059\\
3.24278106952674	179250.992122268\\
3.2428810720268	179203.436625272\\
3.24298107452686	179155.308170481\\
3.24308107702693	179107.752673486\\
3.24318107952699	179059.624218695\\
3.24328108202705	179012.068721699\\
3.24338108452711	178963.940266908\\
3.24348108702718	178916.384769912\\
3.24358108952724	178868.256315121\\
3.2436810920273	178820.700818125\\
3.24378109452736	178772.572363334\\
3.24388109702743	178725.016866338\\
3.24398109952749	178677.461369342\\
3.24408110202755	178629.332914551\\
3.24418110452761	178581.777417556\\
3.24428110702768	178533.648962765\\
3.24438110952774	178486.093465769\\
3.2444811120278	178438.537968773\\
3.24458111452786	178390.409513982\\
3.24468111702793	178342.854016986\\
3.24478111952799	178295.29851999\\
3.24488112202805	178247.170065199\\
3.24498112452811	178199.614568203\\
3.24508112702818	178152.059071207\\
3.24518112952824	178103.930616416\\
3.2452811320283	178056.375119421\\
3.24538113452836	178008.819622425\\
3.24548113702843	177960.691167634\\
3.24558113952849	177913.135670638\\
3.24568114202855	177865.580173642\\
3.24578114452861	177817.451718851\\
3.24588114702868	177769.896221855\\
3.24598114952874	177722.340724859\\
3.2460811520288	177674.785227863\\
3.24618115452886	177626.656773072\\
3.24628115702893	177579.101276077\\
3.24638115952899	177531.545779081\\
3.24648116202905	177483.990282085\\
3.24658116452911	177436.434785089\\
3.24668116702918	177388.306330298\\
3.24678116952924	177340.750833302\\
3.2468811720293	177293.195336306\\
3.24698117452936	177245.63983931\\
3.24708117702943	177198.084342315\\
3.24718117952949	177149.955887524\\
3.24728118202955	177102.400390528\\
3.24738118452961	177054.844893532\\
3.24748118702968	177007.289396536\\
3.24758118952974	176959.73389954\\
3.2476811920298	176912.178402544\\
3.24778119452986	176864.622905548\\
3.24788119702993	176816.494450757\\
3.24798119952999	176768.938953762\\
3.24808120203005	176721.383456766\\
3.24818120453011	176673.82795977\\
3.24828120703018	176626.272462774\\
3.24838120953024	176578.716965778\\
3.2484812120303	176531.161468782\\
3.24858121453036	176483.605971786\\
3.24868121703043	176436.050474791\\
3.24878121953049	176388.494977795\\
3.24888122203055	176340.939480799\\
3.24898122453061	176293.383983803\\
3.24908122703068	176245.828486807\\
3.24918122953074	176198.272989811\\
3.2492812320308	176150.717492815\\
3.24938123453086	176103.16199582\\
3.24948123703093	176055.606498824\\
3.24958123953099	176008.051001828\\
3.24968124203105	175960.495504832\\
3.24978124453111	175912.940007836\\
3.24988124703118	175865.38451084\\
3.24998124953124	175817.829013844\\
3.2500812520313	175770.273516849\\
3.25018125453136	175722.718019853\\
3.25028125703143	175675.162522857\\
3.25038125953149	175627.607025861\\
3.25048126203155	175580.051528865\\
3.25058126453161	175532.496031869\\
3.25068126703168	175484.940534873\\
3.25078126953174	175437.385037878\\
3.2508812720318	175390.402498677\\
3.25098127453186	175342.847001681\\
3.25108127703193	175295.291504685\\
3.25118127953199	175247.736007689\\
3.25128128203205	175200.180510693\\
3.25138128453211	175152.625013698\\
3.25148128703218	175105.069516702\\
3.25158128953224	175058.086977501\\
3.2516812920323	175010.531480505\\
3.25178129453236	174962.975983509\\
3.25188129703243	174915.420486513\\
3.25198129953249	174867.864989518\\
3.25208130203255	174820.882450317\\
3.25218130453261	174773.326953321\\
3.25228130703268	174725.771456325\\
3.25238130953274	174678.215959329\\
3.2524813120328	174630.660462333\\
3.25258131453286	174583.677923133\\
3.25268131703293	174536.122426137\\
3.25278131953299	174488.566929141\\
3.25288132203305	174441.011432145\\
3.25298132453311	174394.028892944\\
3.25308132703318	174346.473395948\\
3.25318132953324	174298.917898953\\
3.2532813320333	174251.935359752\\
3.25338133453336	174204.379862756\\
3.25348133703343	174156.82436576\\
3.25358133953349	174109.841826559\\
3.25368134203355	174062.286329564\\
3.25378134453361	174014.730832568\\
3.25388134703368	173967.748293367\\
3.25398134953374	173920.192796371\\
3.2540813520338	173872.637299375\\
3.25418135453386	173825.654760175\\
3.25428135703393	173778.099263179\\
3.25438135953399	173731.116723978\\
3.25448136203405	173683.561226982\\
3.25458136453411	173636.005729986\\
3.25468136703418	173589.023190786\\
3.25478136953424	173541.46769379\\
3.2548813720343	173494.485154589\\
3.25498137453436	173446.929657593\\
3.25508137703443	173399.374160597\\
3.25518137953449	173352.391621396\\
3.25528138203455	173304.836124401\\
3.25538138453461	173257.8535852\\
3.25548138703468	173210.298088204\\
3.25558138953474	173163.315549003\\
3.2556813920348	173115.760052007\\
3.25578139453486	173068.777512807\\
3.25588139703493	173021.222015811\\
3.25598139953499	172974.23947661\\
3.25608140203505	172926.683979614\\
3.25618140453511	172879.701440414\\
3.25628140703518	172832.145943418\\
3.25638140953524	172785.163404217\\
3.2564814120353	172738.180865016\\
3.25658141453536	172690.62536802\\
3.25668141703543	172643.64282882\\
3.25678141953549	172596.087331824\\
3.25688142203555	172549.104792623\\
3.25698142453561	172502.122253422\\
3.25708142703568	172454.566756426\\
3.25718142953574	172407.584217226\\
3.2572814320358	172360.02872023\\
3.25738143453586	172313.046181029\\
3.25748143703593	172266.063641828\\
3.25758143953599	172218.508144833\\
3.25768144203605	172171.525605632\\
3.25778144453611	172124.543066431\\
3.25788144703618	172076.987569435\\
3.25798144953624	172030.005030235\\
3.2580814520363	171983.022491034\\
3.25818145453636	171935.466994038\\
3.25828145703643	171888.484454837\\
3.25838145953649	171841.501915637\\
3.25848146203655	171794.519376436\\
3.25858146453661	171746.96387944\\
3.25868146703668	171699.981340239\\
3.25878146953674	171652.998801038\\
3.2588814720368	171605.443304043\\
3.25898147453686	171558.460764842\\
3.25908147703693	171511.478225641\\
3.25918147953699	171464.49568644\\
3.25928148203705	171417.51314724\\
3.25938148453711	171369.957650244\\
3.25948148703718	171322.975111043\\
3.25958148953724	171275.992571842\\
3.2596814920373	171229.010032642\\
3.25978149453736	171182.027493441\\
3.25988149703743	171135.04495424\\
3.25998149953749	171087.489457244\\
3.26008150203755	171040.506918044\\
3.26018150453761	170993.524378843\\
3.26028150703768	170946.541839642\\
3.26038150953774	170899.559300441\\
3.2604815120378	170852.576761241\\
3.26058151453786	170805.59422204\\
3.26068151703793	170758.611682839\\
3.26078151953799	170711.056185843\\
3.26088152203805	170664.073646643\\
3.26098152453811	170617.091107442\\
3.26108152703818	170570.108568241\\
3.26118152953824	170523.12602904\\
3.2612815320383	170476.14348984\\
3.26138153453836	170429.160950639\\
3.26148153703843	170382.178411438\\
3.26158153953849	170335.195872238\\
3.26168154203855	170288.213333037\\
3.26178154453861	170241.230793836\\
3.26188154703868	170194.248254635\\
3.26198154953874	170147.265715435\\
3.2620815520388	170100.283176234\\
3.26218155453886	170053.300637033\\
3.26228155703893	170006.318097832\\
3.26238155953899	169959.335558632\\
3.26248156203905	169912.353019431\\
3.26258156453911	169865.37048023\\
3.26268156703918	169818.38794103\\
3.26278156953924	169771.405401829\\
3.2628815720393	169724.422862628\\
3.26298157453936	169678.013281222\\
3.26308157703943	169631.030742022\\
3.26318157953949	169584.048202821\\
3.26328158203955	169537.06566362\\
3.26338158453961	169490.08312442\\
3.26348158703968	169443.100585219\\
3.26358158953974	169396.118046018\\
3.2636815920398	169349.135506817\\
3.26378159453986	169302.725925412\\
3.26388159703993	169255.743386211\\
3.26398159953999	169208.76084701\\
3.26408160204005	169161.77830781\\
3.26418160454011	169114.795768609\\
3.26428160704018	169068.386187203\\
3.26438160954024	169021.403648003\\
3.2644816120403	168974.421108802\\
3.26458161454036	168927.438569601\\
3.26468161704043	168880.4560304\\
3.26478161954049	168834.046448995\\
3.26488162204055	168787.063909794\\
3.26498162454061	168740.081370593\\
3.26508162704068	168693.098831393\\
3.26518162954074	168646.689249987\\
3.2652816320408	168599.706710786\\
3.26538163454086	168552.724171586\\
3.26548163704093	168506.31459018\\
3.26558163954099	168459.332050979\\
3.26568164204105	168412.349511779\\
3.26578164454111	168365.939930373\\
3.26588164704118	168318.957391172\\
3.26598164954124	168271.974851971\\
3.2660816520413	168225.565270566\\
3.26618165454136	168178.582731365\\
3.26628165704143	168131.600192164\\
3.26638165954149	168085.190610759\\
3.26648166204155	168038.208071558\\
3.26658166454161	167991.798490153\\
3.26668166704168	167944.815950952\\
3.26678166954174	167897.833411751\\
3.2668816720418	167851.423830345\\
3.26698167454186	167804.441291145\\
3.26708167704193	167758.031709739\\
3.26718167954199	167711.049170538\\
3.26728168204205	167664.639589133\\
3.26738168454211	167617.657049932\\
3.26748168704218	167570.674510731\\
3.26758168954224	167524.264929326\\
3.2676816920423	167477.282390125\\
3.26778169454236	167430.872808719\\
3.26788169704243	167383.890269519\\
3.26798169954249	167337.480688113\\
3.26808170204255	167291.071106708\\
3.26818170454261	167244.088567507\\
3.26828170704268	167197.678986101\\
3.26838170954274	167150.6964469\\
3.2684817120428	167104.286865495\\
3.26858171454286	167057.304326294\\
3.26868171704293	167010.894744889\\
3.26878171954299	166963.912205688\\
3.26888172204305	166917.502624282\\
3.26898172454311	166871.093042877\\
3.26908172704318	166824.110503676\\
3.26918172954324	166777.70092227\\
3.2692817320433	166730.71838307\\
3.26938173454336	166684.308801664\\
3.26948173704343	166637.899220258\\
3.26958173954349	166590.916681058\\
3.26968174204355	166544.507099652\\
3.26978174454361	166498.097518246\\
3.26988174704368	166451.114979046\\
3.26998174954374	166404.70539764\\
3.2700817520438	166358.295816235\\
3.27018175454386	166311.886234829\\
3.27028175704393	166264.903695628\\
3.27038175954399	166218.494114223\\
3.27048176204405	166172.084532817\\
3.27058176454411	166125.101993616\\
3.27068176704418	166078.692412211\\
3.27078176954424	166032.282830805\\
3.2708817720443	165985.873249399\\
3.27098177454436	165938.890710199\\
3.27108177704443	165892.481128793\\
3.27118177954449	165846.071547388\\
3.27128178204455	165799.661965982\\
3.27138178454461	165753.252384576\\
3.27148178704468	165706.269845376\\
3.27158178954474	165659.86026397\\
3.2716817920448	165613.450682564\\
3.27178179454486	165567.041101159\\
3.27188179704493	165520.631519753\\
3.27198179954499	165474.221938348\\
3.27208180204505	165427.812356942\\
3.27218180454511	165380.829817741\\
3.27228180704518	165334.420236336\\
3.27238180954524	165288.01065493\\
3.2724818120453	165241.601073525\\
3.27258181454536	165195.191492119\\
3.27268181704543	165148.781910713\\
3.27278181954549	165102.372329308\\
3.27288182204555	165055.962747902\\
3.27298182454561	165009.553166497\\
3.27308182704568	164963.143585091\\
3.27318182954574	164916.734003685\\
3.2732818320458	164870.32442228\\
3.27338183454586	164823.914840874\\
3.27348183704593	164777.505259469\\
3.27358183954599	164731.095678063\\
3.27368184204605	164684.686096657\\
3.27378184454611	164638.276515252\\
3.27388184704618	164591.866933846\\
3.27398184954624	164545.457352441\\
3.2740818520463	164499.047771035\\
3.27418185454636	164452.638189629\\
3.27428185704643	164406.228608224\\
3.27438185954649	164359.819026818\\
3.27448186204655	164313.409445413\\
3.27458186454661	164266.999864007\\
3.27468186704668	164220.590282601\\
3.27478186954674	164174.180701196\\
3.2748818720468	164127.77111979\\
3.27498187454686	164081.93449618\\
3.27508187704693	164035.524914774\\
3.27518187954699	163989.115333369\\
3.27528188204705	163942.705751963\\
3.27538188454711	163896.296170557\\
3.27548188704718	163849.886589152\\
3.27558188954724	163804.049965541\\
3.2756818920473	163757.640384136\\
3.27578189454736	163711.23080273\\
3.27588189704743	163664.821221324\\
3.27598189954749	163618.411639919\\
3.27608190204755	163572.575016308\\
3.27618190454761	163526.165434903\\
3.27628190704768	163479.755853497\\
3.27638190954774	163433.346272092\\
3.2764819120478	163386.936690686\\
3.27658191454786	163341.100067076\\
3.27668191704793	163294.69048567\\
3.27678191954799	163248.280904264\\
3.27688192204805	163202.444280654\\
3.27698192454811	163156.034699248\\
3.27708192704818	163109.625117843\\
3.27718192954824	163063.788494232\\
3.2772819320483	163017.378912827\\
3.27738193454836	162970.969331421\\
3.27748193704843	162925.132707811\\
3.27758193954849	162878.723126405\\
3.27768194204855	162832.313544999\\
3.27778194454861	162786.476921389\\
3.27788194704868	162740.067339983\\
3.27798194954874	162693.657758578\\
3.2780819520488	162647.821134967\\
3.27818195454886	162601.411553562\\
3.27828195704893	162555.574929951\\
3.27838195954899	162509.165348546\\
3.27848196204905	162462.75576714\\
3.27858196454911	162416.91914353\\
3.27868196704918	162370.509562124\\
3.27878196954924	162324.672938514\\
3.2788819720493	162278.263357108\\
3.27898197454936	162232.426733497\\
3.27908197704943	162186.017152092\\
3.27918197954949	162140.180528481\\
3.27928198204955	162093.770947076\\
3.27938198454961	162047.934323465\\
3.27948198704968	162001.52474206\\
3.27958198954974	161955.688118449\\
3.2796819920498	161909.278537044\\
3.27978199454986	161863.441913433\\
3.27988199704993	161817.605289823\\
3.27998199954999	161771.195708417\\
3.28008200205005	161725.359084807\\
3.28018200455011	161678.949503401\\
3.28028200705018	161633.112879791\\
3.28038200955024	161586.703298385\\
3.2804820120503	161540.866674775\\
3.28058201455036	161495.030051164\\
3.28068201705043	161448.620469758\\
3.28078201955049	161402.783846148\\
3.28088202205055	161356.947222538\\
3.28098202455061	161310.537641132\\
3.28108202705068	161264.701017521\\
3.28118202955074	161218.864393911\\
3.2812820320508	161172.454812505\\
3.28138203455086	161126.618188895\\
3.28148203705093	161080.781565285\\
3.28158203955099	161034.371983879\\
3.28168204205105	160988.535360268\\
3.28178204455111	160942.698736658\\
3.28188204705118	160896.862113048\\
3.28198204955124	160850.452531642\\
3.2820820520513	160804.615908031\\
3.28218205455136	160758.779284421\\
3.28228205705143	160712.942660811\\
3.28238205955149	160666.533079405\\
3.28248206205155	160620.696455794\\
3.28258206455161	160574.859832184\\
3.28268206705168	160529.023208574\\
3.28278206955174	160483.186584963\\
3.2828820720518	160437.349961353\\
3.28298207455186	160390.940379947\\
3.28308207705193	160345.103756337\\
3.28318207955199	160299.267132726\\
3.28328208205205	160253.430509116\\
3.28338208455211	160207.593885505\\
3.28348208705218	160161.757261895\\
3.28358208955224	160115.920638284\\
3.2836820920523	160070.084014674\\
3.28378209455236	160023.674433268\\
3.28388209705243	159977.837809658\\
3.28398209955249	159932.001186047\\
3.28408210205255	159886.164562437\\
3.28418210455261	159840.327938826\\
3.28428210705268	159794.491315216\\
3.28438210955274	159748.654691605\\
3.2844821120528	159702.818067995\\
3.28458211455286	159656.981444384\\
3.28468211705293	159611.144820774\\
3.28478211955299	159565.308197163\\
3.28488212205305	159519.471573553\\
3.28498212455311	159473.634949943\\
3.28508212705318	159427.798326332\\
3.28518212955324	159381.961702722\\
3.2852821320533	159336.125079111\\
3.28538213455336	159290.288455501\\
3.28548213705343	159244.45183189\\
3.28558213955349	159198.61520828\\
3.28568214205355	159153.351542464\\
3.28578214455361	159107.514918854\\
3.28588214705368	159061.678295244\\
3.28598214955374	159015.841671633\\
3.2860821520538	158970.005048023\\
3.28618215455386	158924.168424412\\
3.28628215705393	158878.331800802\\
3.28638215955399	158832.495177191\\
3.28648216205405	158787.231511376\\
3.28658216455411	158741.394887765\\
3.28668216705418	158695.558264155\\
3.28678216955424	158649.721640544\\
3.2868821720543	158603.885016934\\
3.28698217455436	158558.621351119\\
3.28708217705443	158512.784727508\\
3.28718217955449	158466.948103898\\
3.28728218205455	158421.111480287\\
3.28738218455461	158375.274856677\\
3.28748218705468	158330.011190861\\
3.28758218955474	158284.174567251\\
3.2876821920548	158238.337943641\\
3.28778219455486	158193.074277825\\
3.28788219705493	158147.237654215\\
3.28798219955499	158101.401030604\\
3.28808220205505	158055.564406994\\
3.28818220455511	158010.300741178\\
3.28828220705518	157964.464117568\\
3.28838220955524	157918.627493957\\
3.2884822120553	157873.363828142\\
3.28858221455536	157827.527204532\\
3.28868221705543	157781.690580921\\
3.28878221955549	157736.426915106\\
3.28888222205555	157690.590291495\\
3.28898222455561	157645.32662568\\
3.28908222705568	157599.49000207\\
3.28918222955574	157553.653378459\\
3.2892822320558	157508.389712644\\
3.28938223455586	157462.553089033\\
3.28948223705593	157417.289423218\\
3.28958223955599	157371.452799608\\
3.28968224205605	157326.189133792\\
3.28978224455611	157280.352510182\\
3.28988224705618	157235.088844366\\
3.28998224955624	157189.252220756\\
3.2900822520563	157143.988554941\\
3.29018225455636	157098.15193133\\
3.29028225705643	157052.888265515\\
3.29038225955649	157007.051641904\\
3.29048226205655	156961.787976089\\
3.29058226455661	156915.951352479\\
3.29068226705668	156870.687686663\\
3.29078226955674	156824.851063053\\
3.2908822720568	156779.587397237\\
3.29098227455686	156733.750773627\\
3.29108227705693	156688.487107812\\
3.29118227955699	156643.223441996\\
3.29128228205705	156597.386818386\\
3.29138228455711	156552.12315257\\
3.29148228705718	156506.859486755\\
3.29158228955724	156461.022863145\\
3.2916822920573	156415.759197329\\
3.29178229455736	156369.922573719\\
3.29188229705743	156324.658907904\\
3.29198229955749	156279.395242088\\
3.29208230205755	156233.558618478\\
3.29218230455761	156188.294952662\\
3.29228230705768	156143.031286847\\
3.29238230955774	156097.767621032\\
3.2924823120578	156051.930997421\\
3.29258231455786	156006.667331606\\
3.29268231705793	155961.403665791\\
3.29278231955799	155916.139999975\\
3.29288232205805	155870.303376365\\
3.29298232455811	155825.039710549\\
3.29308232705818	155779.776044734\\
3.29318232955824	155734.512378919\\
3.2932823320583	155688.675755308\\
3.29338233455836	155643.412089493\\
3.29348233705843	155598.148423678\\
3.29358233955849	155552.884757862\\
3.29368234205855	155507.621092047\\
3.29378234455861	155462.357426232\\
3.29388234705868	155416.520802621\\
3.29398234955874	155371.257136806\\
3.2940823520588	155325.993470991\\
3.29418235455886	155280.729805175\\
3.29428235705893	155235.46613936\\
3.29438235955899	155190.202473545\\
3.29448236205905	155144.938807729\\
3.29458236455911	155099.675141914\\
3.29468236705918	155054.411476098\\
3.29478236955924	155009.147810283\\
3.2948823720593	154963.884144468\\
3.29498237455936	154918.047520857\\
3.29508237705943	154872.783855042\\
3.29518237955949	154827.520189227\\
3.29528238205955	154782.256523411\\
3.29538238455961	154736.992857596\\
3.29548238705968	154691.729191781\\
3.29558238955974	154646.465525965\\
3.2956823920598	154601.20186015\\
3.29578239455986	154556.51115213\\
3.29588239705993	154511.247486315\\
3.29598239955999	154465.983820499\\
3.29608240206005	154420.720154684\\
3.29618240456011	154375.456488869\\
3.29628240706018	154330.192823053\\
3.29638240956024	154284.929157238\\
3.2964824120603	154239.665491422\\
3.29658241456036	154194.401825607\\
3.29668241706043	154149.138159792\\
3.29678241956049	154103.874493976\\
3.29688242206055	154059.183785956\\
3.29698242456061	154013.920120141\\
3.29708242706068	153968.656454326\\
3.29718242956074	153923.39278851\\
3.2972824320608	153878.129122695\\
3.29738243456086	153832.86545688\\
3.29748243706093	153788.174748859\\
3.29758243956099	153742.911083044\\
3.29768244206105	153697.647417229\\
3.29778244456111	153652.383751413\\
3.29788244706118	153607.120085598\\
3.29798244956124	153562.429377578\\
3.2980824520613	153517.165711763\\
3.29818245456136	153471.902045947\\
3.29828245706143	153427.211337927\\
3.29838245956149	153381.947672112\\
3.29848246206155	153336.684006296\\
3.29858246456161	153291.420340481\\
3.29868246706168	153246.729632461\\
3.29878246956174	153201.465966645\\
3.2988824720618	153156.20230083\\
3.29898247456186	153111.51159281\\
3.29908247706193	153066.247926995\\
3.29918247956199	153020.984261179\\
3.29928248206205	152976.293553159\\
3.29938248456211	152931.029887344\\
3.29948248706218	152886.339179323\\
3.29958248956224	152841.075513508\\
3.2996824920623	152795.811847693\\
3.29978249456236	152751.121139673\\
3.29988249706243	152705.857473857\\
3.29998249956249	152661.166765837\\
3.30008250206255	152615.903100022\\
3.30018250456261	152571.212392002\\
3.30028250706268	152525.948726186\\
3.30038250956274	152481.258018166\\
3.3004825120628	152435.994352351\\
3.30058251456286	152391.30364433\\
3.30068251706293	152346.039978515\\
3.30078251956299	152301.349270495\\
3.30088252206305	152256.08560468\\
3.30098252456311	152211.394896659\\
3.30108252706318	152166.131230844\\
3.30118252956324	152121.440522824\\
3.3012825320633	152076.176857009\\
3.30138253456336	152031.486148988\\
3.30148253706343	151986.222483173\\
3.30158253956349	151941.531775153\\
3.30168254206355	151896.841067133\\
3.30178254456361	151851.577401317\\
3.30188254706368	151806.886693297\\
3.30198254956374	151762.195985277\\
3.3020825520638	151716.932319461\\
3.30218255456386	151672.241611441\\
3.30228255706393	151627.550903421\\
3.30238255956399	151582.287237606\\
3.30248256206405	151537.596529586\\
3.30258256456411	151492.905821565\\
3.30268256706418	151447.64215575\\
3.30278256956424	151402.95144773\\
3.3028825720643	151358.26073971\\
3.30298257456436	151312.997073894\\
3.30308257706443	151268.306365874\\
3.30318257956449	151223.615657854\\
3.30328258206455	151178.924949834\\
3.30338258456461	151133.661284018\\
3.30348258706468	151088.970575998\\
3.30358258956474	151044.279867978\\
3.3036825920648	150999.589159958\\
3.30378259456486	150954.898451937\\
3.30388259706493	150909.634786122\\
3.30398259956499	150864.944078102\\
3.30408260206505	150820.253370082\\
3.30418260456511	150775.562662062\\
3.30428260706518	150730.871954041\\
3.30438260956524	150686.181246021\\
3.3044826120653	150641.490538001\\
3.30458261456536	150596.799829981\\
3.30468261706543	150551.536164165\\
3.30478261956549	150506.845456145\\
3.30488262206555	150462.154748125\\
3.30498262456561	150417.464040105\\
3.30508262706568	150372.773332085\\
3.30518262956574	150328.082624064\\
3.3052826320658	150283.391916044\\
3.30538263456586	150238.701208024\\
3.30548263706593	150194.010500004\\
3.30558263956599	150149.319791984\\
3.30568264206605	150104.629083963\\
3.30578264456611	150059.938375943\\
3.30588264706618	150015.247667923\\
3.30598264956624	149970.556959903\\
3.3060826520663	149925.866251882\\
3.30618265456636	149881.175543862\\
3.30628265706643	149836.484835842\\
3.30638265956649	149791.794127822\\
3.30648266206655	149747.103419802\\
3.30658266456661	149702.412711782\\
3.30668266706668	149657.722003761\\
3.30678266956674	149613.604253536\\
3.3068826720668	149568.913545516\\
3.30698267456686	149524.222837496\\
3.30708267706693	149479.532129476\\
3.30718267956699	149434.841421455\\
3.30728268206705	149390.150713435\\
3.30738268456711	149345.460005415\\
3.30748268706718	149301.34225519\\
3.30758268956724	149256.65154717\\
3.3076826920673	149211.960839149\\
3.30778269456736	149167.270131129\\
3.30788269706743	149122.579423109\\
3.30798269956749	149078.461672884\\
3.30808270206755	149033.770964864\\
3.30818270456761	148989.080256844\\
3.30828270706768	148944.389548823\\
3.30838270956774	148900.271798598\\
3.3084827120678	148855.581090578\\
3.30858271456786	148810.890382558\\
3.30868271706793	148766.772632333\\
3.30878271956799	148722.081924313\\
3.30888272206805	148677.391216292\\
3.30898272456811	148632.700508272\\
3.30908272706818	148588.582758047\\
3.30918272956824	148543.892050027\\
3.3092827320683	148499.774299802\\
3.30938273456836	148455.083591782\\
3.30948273706843	148410.392883761\\
3.30958273956849	148366.275133536\\
3.30968274206855	148321.584425516\\
3.30978274456861	148276.893717496\\
3.30988274706868	148232.775967271\\
3.30998274956874	148188.085259251\\
3.3100827520688	148143.967509026\\
3.31018275456886	148099.276801005\\
3.31028275706893	148055.15905078\\
3.31038275956899	148010.46834276\\
3.31048276206905	147966.350592535\\
3.31058276456911	147921.659884515\\
3.31068276706918	147877.54213429\\
3.31078276956924	147832.85142627\\
3.3108827720693	147788.733676045\\
3.31098277456936	147744.042968024\\
3.31108277706943	147699.925217799\\
3.31118277956949	147655.234509779\\
3.31128278206955	147611.116759554\\
3.31138278456961	147566.426051534\\
3.31148278706968	147522.308301309\\
3.31158278956974	147477.617593289\\
3.3116827920698	147433.499843063\\
3.31178279456986	147389.382092838\\
3.31188279706993	147344.691384818\\
3.31198279956999	147300.573634593\\
3.31208280207005	147256.455884368\\
3.31218280457011	147211.765176348\\
3.31228280707018	147167.647426123\\
3.31238280957024	147123.529675898\\
3.3124828120703	147078.838967877\\
3.31258281457036	147034.721217652\\
3.31268281707043	146990.603467427\\
3.31278281957049	146945.912759407\\
3.31288282207055	146901.795009182\\
3.31298282457061	146857.677258957\\
3.31308282707068	146812.986550937\\
3.31318282957074	146768.868800712\\
3.3132828320708	146724.751050487\\
3.31338283457086	146680.633300262\\
3.31348283707093	146636.515550036\\
3.31358283957099	146591.824842016\\
3.31368284207105	146547.707091791\\
3.31378284457111	146503.589341566\\
3.31388284707118	146459.471591341\\
3.31398284957124	146415.353841116\\
3.3140828520713	146370.663133096\\
3.31418285457136	146326.545382871\\
3.31428285707143	146282.427632646\\
3.31438285957149	146238.309882421\\
3.31448286207155	146194.192132195\\
3.31458286457161	146150.07438197\\
3.31468286707168	146105.956631745\\
3.31478286957174	146061.83888152\\
3.3148828720718	146017.721131295\\
3.31498287457186	145973.030423275\\
3.31508287707193	145928.91267305\\
3.31518287957199	145884.794922825\\
3.31528288207205	145840.6771726\\
3.31538288457211	145796.559422375\\
3.31548288707218	145752.44167215\\
3.31558288957224	145708.323921925\\
3.3156828920723	145664.206171699\\
3.31578289457236	145620.088421474\\
3.31588289707243	145575.970671249\\
3.31598289957249	145531.852921024\\
3.31608290207255	145487.735170799\\
3.31618290457261	145443.617420574\\
3.31628290707268	145399.499670349\\
3.31638290957274	145355.954877919\\
3.3164829120728	145311.837127694\\
3.31658291457286	145267.719377469\\
3.31668291707293	145223.601627244\\
3.31678291957299	145179.483877019\\
3.31688292207305	145135.366126794\\
3.31698292457311	145091.248376569\\
3.31708292707318	145047.130626344\\
3.31718292957324	145003.012876118\\
3.3172829320733	144959.468083689\\
3.31738293457336	144915.350333463\\
3.31748293707343	144871.232583238\\
3.31758293957349	144827.114833013\\
3.31768294207355	144782.997082788\\
3.31778294457361	144739.452290358\\
3.31788294707368	144695.334540133\\
3.31798294957374	144651.216789908\\
3.3180829520738	144607.099039683\\
3.31818295457386	144563.554247253\\
3.31828295707393	144519.436497028\\
3.31838295957399	144475.318746803\\
3.31848296207405	144431.200996578\\
3.31858296457411	144387.656204148\\
3.31868296707418	144343.538453923\\
3.31878296957424	144299.420703698\\
3.3188829720743	144255.875911268\\
3.31898297457436	144211.758161043\\
3.31908297707443	144167.640410818\\
3.31918297957449	144124.095618388\\
3.31928298207455	144079.977868163\\
3.31938298457461	144035.860117938\\
3.31948298707468	143992.315325508\\
3.31958298957474	143948.197575283\\
3.3196829920748	143904.652782853\\
3.31978299457486	143860.535032628\\
3.31988299707493	143816.990240198\\
3.31998299957499	143772.872489973\\
3.32008300207505	143728.754739748\\
3.32018300457511	143685.209947318\\
3.32028300707518	143641.092197093\\
3.32038300957524	143597.547404663\\
3.3204830120753	143553.429654437\\
3.32058301457536	143509.884862008\\
3.32068301707543	143465.767111782\\
3.32078301957549	143422.222319353\\
3.32088302207555	143378.104569127\\
3.32098302457561	143334.559776698\\
3.32108302707568	143291.014984268\\
3.32118302957574	143246.897234043\\
3.3212830320758	143203.352441613\\
3.32138303457586	143159.234691387\\
3.32148303707593	143115.689898958\\
3.32158303957599	143071.572148732\\
3.32168304207605	143028.027356303\\
3.32178304457611	142984.482563873\\
3.32188304707618	142940.364813648\\
3.32198304957624	142896.820021218\\
3.3220830520763	142853.275228788\\
3.32218305457636	142809.157478563\\
3.32228305707643	142765.612686133\\
3.32238305957649	142722.067893703\\
3.32248306207655	142677.950143478\\
3.32258306457661	142634.405351048\\
3.32268306707668	142590.860558618\\
3.32278306957674	142547.315766188\\
3.3228830720768	142503.198015963\\
3.32298307457686	142459.653223533\\
3.32308307707693	142416.108431103\\
3.32318307957699	142372.563638673\\
3.32328308207705	142328.445888448\\
3.32338308457711	142284.901096018\\
3.32348308707718	142241.356303588\\
3.32358308957724	142197.811511158\\
3.3236830920773	142154.266718728\\
3.32378309457736	142110.148968503\\
3.32388309707743	142066.604176073\\
3.32398309957749	142023.059383643\\
3.32408310207755	141979.514591213\\
3.32418310457761	141935.969798783\\
3.32428310707768	141892.425006353\\
3.32438310957774	141848.880213923\\
3.3244831120778	141805.335421493\\
3.32458311457786	141761.217671268\\
3.32468311707793	141717.672878838\\
3.32478311957799	141674.128086408\\
3.32488312207805	141630.583293978\\
3.32498312457811	141587.038501549\\
3.32508312707818	141543.493709119\\
3.32518312957824	141499.948916689\\
3.3252831320783	141456.404124259\\
3.32538313457836	141412.859331829\\
3.32548313707843	141369.314539399\\
3.32558313957849	141325.769746969\\
3.32568314207855	141282.224954539\\
3.32578314457861	141238.680162109\\
3.32588314707868	141195.135369679\\
3.32598314957874	141151.590577249\\
3.3260831520788	141108.618742614\\
3.32618315457886	141065.073950184\\
3.32628315707893	141021.529157754\\
3.32638315957899	140977.984365324\\
3.32648316207905	140934.439572895\\
3.32658316457911	140890.894780465\\
3.32668316707918	140847.349988035\\
3.32678316957924	140803.805195605\\
3.3268831720793	140760.260403175\\
3.32698317457936	140717.28856854\\
3.32708317707943	140673.74377611\\
3.32718317957949	140630.19898368\\
3.32728318207955	140586.65419125\\
3.32738318457961	140543.10939882\\
3.32748318707968	140500.137564185\\
3.32758318957974	140456.592771755\\
3.3276831920798	140413.047979325\\
3.32778319457986	140369.503186896\\
3.32788319707993	140326.531352261\\
3.32798319957999	140282.986559831\\
3.32808320208005	140239.441767401\\
3.32818320458011	140196.469932766\\
3.32828320708018	140152.925140336\\
3.32838320958024	140109.380347906\\
3.3284832120803	140065.835555476\\
3.32858321458036	140022.863720841\\
3.32868321708043	139979.318928411\\
3.32878321958049	139936.347093777\\
3.32888322208055	139892.802301347\\
3.32898322458061	139849.257508917\\
3.32908322708068	139806.285674282\\
3.32918322958074	139762.740881852\\
3.3292832320808	139719.196089422\\
3.32938323458086	139676.224254787\\
3.32948323708093	139632.679462357\\
3.32958323958099	139589.707627722\\
3.32968324208105	139546.162835293\\
3.32978324458111	139503.191000658\\
3.32988324708118	139459.646208228\\
3.32998324958124	139416.674373593\\
3.3300832520813	139373.129581163\\
3.33018325458136	139330.157746528\\
3.33028325708143	139286.612954098\\
3.33038325958149	139243.641119463\\
3.33048326208155	139200.096327034\\
3.33058326458161	139157.124492399\\
3.33068326708168	139113.579699969\\
3.33078326958174	139070.607865334\\
3.3308832720818	139027.063072904\\
3.33098327458186	138984.091238269\\
3.33108327708193	138941.119403634\\
3.33118327958199	138897.574611204\\
3.33128328208205	138854.60277657\\
3.33138328458211	138811.630941935\\
3.33148328708218	138768.086149505\\
3.33158328958224	138725.11431487\\
3.3316832920823	138682.142480235\\
3.33178329458236	138638.597687805\\
3.33188329708243	138595.62585317\\
3.33198329958249	138552.654018536\\
3.33208330208255	138509.109226106\\
3.33218330458261	138466.137391471\\
3.33228330708268	138423.165556836\\
3.33238330958274	138379.620764406\\
3.3324833120828	138336.648929771\\
3.33258331458286	138293.677095137\\
3.33268331708293	138250.705260502\\
3.33278331958299	138207.733425867\\
3.33288332208305	138164.188633437\\
3.33298332458311	138121.216798802\\
3.33308332708318	138078.244964167\\
3.33318332958324	138035.273129533\\
3.3332833320833	137992.301294898\\
3.33338333458336	137948.756502468\\
3.33348333708343	137905.784667833\\
3.33358333958349	137862.812833198\\
3.33368334208355	137819.840998563\\
3.33378334458361	137776.869163929\\
3.33388334708368	137733.897329294\\
3.33398334958374	137690.925494659\\
3.3340833520838	137647.953660024\\
3.33418335458386	137604.981825389\\
3.33428335708393	137562.009990754\\
3.33438335958399	137519.03815612\\
3.33448336208405	137475.49336369\\
3.33458336458411	137432.521529055\\
3.33468336708418	137389.54969442\\
3.33478336958424	137346.577859785\\
3.3348833720843	137303.60602515\\
3.33498337458436	137260.634190516\\
3.33508337708443	137217.662355881\\
3.33518337958449	137175.263479041\\
3.33528338208455	137132.291644406\\
3.33538338458461	137089.319809772\\
3.33548338708468	137046.347975137\\
3.33558338958474	137003.376140502\\
3.3356833920848	136960.404305867\\
3.33578339458486	136917.432471232\\
3.33588339708493	136874.460636597\\
3.33598339958499	136831.488801963\\
3.33608340208505	136788.516967328\\
3.33618340458511	136745.545132693\\
3.33628340708518	136703.146255853\\
3.33638340958524	136660.174421219\\
3.3364834120853	136617.202586584\\
3.33658341458536	136574.230751949\\
3.33668341708543	136531.258917314\\
3.33678341958549	136488.860040474\\
3.33688342208555	136445.88820584\\
3.33698342458561	136402.916371205\\
3.33708342708568	136359.94453657\\
3.33718342958574	136317.54565973\\
3.3372834320858	136274.573825096\\
3.33738343458586	136231.601990461\\
3.33748343708593	136188.630155826\\
3.33758343958599	136146.231278986\\
3.33768344208605	136103.259444351\\
3.33778344458611	136060.287609717\\
3.33788344708618	136017.888732877\\
3.33798344958624	135974.916898242\\
3.3380834520863	135931.945063607\\
3.33818345458636	135889.546186768\\
3.33828345708643	135846.574352133\\
3.33838345958649	135803.602517498\\
3.33848346208655	135761.203640658\\
3.33858346458661	135718.231806024\\
3.33868346708668	135675.832929184\\
3.33878346958674	135632.861094549\\
3.3388834720868	135589.889259914\\
3.33898347458686	135547.490383075\\
3.33908347708693	135504.51854844\\
3.33918347958699	135462.1196716\\
3.33928348208705	135419.147836965\\
3.33938348458711	135376.748960126\\
3.33948348708718	135333.777125491\\
3.33958348958724	135291.378248651\\
3.3396834920873	135248.406414016\\
3.33978349458736	135206.007537177\\
3.33988349708743	135163.035702542\\
3.33998349958749	135120.636825702\\
3.34008350208755	135078.237948862\\
3.34018350458761	135035.266114228\\
3.34028350708768	134992.867237388\\
3.34038350958774	134949.895402753\\
3.3404835120878	134907.496525913\\
3.34058351458786	134865.097649074\\
3.34068351708793	134822.125814439\\
3.34078351958799	134779.726937599\\
3.34088352208805	134736.755102964\\
3.34098352458811	134694.356226125\\
3.34108352708818	134651.957349285\\
3.34118352958824	134608.98551465\\
3.3412835320883	134566.586637811\\
3.34138353458836	134524.187760971\\
3.34148353708843	134481.788884131\\
3.34158353958849	134438.817049496\\
3.34168354208855	134396.418172657\\
3.34178354458861	134354.019295817\\
3.34188354708868	134311.620418977\\
3.34198354958874	134268.648584343\\
3.3420835520888	134226.249707503\\
3.34218355458886	134183.850830663\\
3.34228355708893	134141.451953823\\
3.34238355958899	134099.053076984\\
3.34248356208905	134056.081242349\\
3.34258356458911	134013.682365509\\
3.34268356708918	133971.28348867\\
3.34278356958924	133928.88461183\\
3.3428835720893	133886.48573499\\
3.34298357458936	133844.086858151\\
3.34308357708943	133801.687981311\\
3.34318357958949	133758.716146676\\
3.34328358208955	133716.317269836\\
3.34338358458961	133673.918392997\\
3.34348358708968	133631.519516157\\
3.34358358958974	133589.120639317\\
3.3436835920898	133546.721762478\\
3.34378359458986	133504.322885638\\
3.34388359708993	133461.924008798\\
3.34398359958999	133419.525131959\\
3.34408360209005	133377.126255119\\
3.34418360459011	133334.727378279\\
3.34428360709018	133292.32850144\\
3.34438360959024	133249.9296246\\
3.3444836120903	133207.53074776\\
3.34458361459036	133165.131870921\\
3.34468361709043	133122.732994081\\
3.34478361959049	133080.334117241\\
3.34488362209055	133037.935240401\\
3.34498362459061	132996.109321357\\
3.34508362709068	132953.710444517\\
3.34518362959074	132911.311567678\\
3.3452836320908	132868.912690838\\
3.34538363459086	132826.513813998\\
3.34548363709093	132784.114937159\\
3.34558363959099	132741.716060319\\
3.34568364209105	132699.890141274\\
3.34578364459111	132657.491264435\\
3.34588364709118	132615.092387595\\
3.34598364959124	132572.693510755\\
3.3460836520913	132530.294633916\\
3.34618365459136	132488.468714871\\
3.34628365709143	132446.069838031\\
3.34638365959149	132403.670961192\\
3.34648366209155	132361.272084352\\
3.34658366459161	132319.446165307\\
3.34668366709168	132277.047288468\\
3.34678366959174	132234.648411628\\
3.3468836720918	132192.822492584\\
3.34698367459186	132150.423615744\\
3.34708367709193	132108.024738904\\
3.34718367959199	132066.19881986\\
3.34728368209205	132023.79994302\\
3.34738368459211	131981.40106618\\
3.34748368709218	131939.575147136\\
3.34758368959224	131897.176270296\\
3.3476836920923	131854.777393456\\
3.34778369459236	131812.951474412\\
3.34788369709243	131770.552597572\\
3.34798369959249	131728.726678528\\
3.34808370209255	131686.327801688\\
3.34818370459261	131644.501882643\\
3.34828370709268	131602.103005804\\
3.34838370959274	131559.704128964\\
3.3484837120928	131517.878209919\\
3.34858371459286	131475.47933308\\
3.34868371709293	131433.653414035\\
3.34878371959299	131391.254537196\\
3.34888372209305	131349.428618151\\
3.34898372459311	131307.602699106\\
3.34908372709318	131265.203822267\\
3.34918372959324	131223.377903222\\
3.3492837320933	131180.979026382\\
3.34938373459336	131139.153107338\\
3.34948373709343	131096.754230498\\
3.34958373959349	131054.928311454\\
3.34968374209355	131013.102392409\\
3.34978374459361	130970.703515569\\
3.34988374709368	130928.877596525\\
3.34998374959374	130886.478719685\\
3.3500837520938	130844.652800641\\
3.35018375459386	130802.826881596\\
3.35028375709393	130760.428004756\\
3.35038375959399	130718.602085712\\
3.35048376209405	130676.776166667\\
3.35058376459411	130634.950247623\\
3.35068376709418	130592.551370783\\
3.35078376959424	130550.725451739\\
3.3508837720943	130508.899532694\\
3.35098377459436	130467.073613649\\
3.35108377709443	130424.67473681\\
3.35118377959449	130382.848817765\\
3.35128378209455	130341.022898721\\
3.35138378459461	130299.196979676\\
3.35148378709468	130257.371060632\\
3.35158378959474	130214.972183792\\
3.3516837920948	130173.146264747\\
3.35178379459486	130131.320345703\\
3.35188379709493	130089.494426658\\
3.35198379959499	130047.668507614\\
3.35208380209505	130005.842588569\\
3.35218380459511	129964.016669525\\
3.35228380709518	129921.617792685\\
3.35238380959524	129879.79187364\\
3.3524838120953	129837.965954596\\
3.35258381459537	129796.140035551\\
3.35268381709543	129754.314116507\\
3.35278381959549	129712.488197462\\
3.35288382209555	129670.662278418\\
3.35298382459561	129628.836359373\\
3.35308382709568	129587.010440329\\
3.35318382959574	129545.184521284\\
3.3532838320958	129503.358602239\\
3.35338383459586	129461.532683195\\
3.35348383709593	129419.70676415\\
3.35358383959599	129377.880845106\\
3.35368384209605	129336.054926061\\
3.35378384459611	129294.229007017\\
3.35388384709618	129252.403087972\\
3.35398384959624	129210.577168928\\
3.3540838520963	129169.324207678\\
3.35418385459636	129127.498288634\\
3.35428385709643	129085.672369589\\
3.35438385959649	129043.846450545\\
3.35448386209655	129002.0205315\\
3.35458386459662	128960.194612455\\
3.35468386709668	128918.368693411\\
3.35478386959674	128877.115732161\\
3.3548838720968	128835.289813117\\
3.35498387459686	128793.463894072\\
3.35508387709693	128751.637975028\\
3.35518387959699	128709.812055983\\
3.35528388209705	128668.559094734\\
3.35538388459711	128626.733175689\\
3.35548388709718	128584.907256645\\
3.35558388959724	128543.0813376\\
3.3556838920973	128501.828376351\\
3.35578389459736	128460.002457306\\
3.35588389709743	128418.176538262\\
3.35598389959749	128376.923577012\\
3.35608390209755	128335.097657968\\
3.35618390459761	128293.271738923\\
3.35628390709768	128252.018777674\\
3.35638390959774	128210.192858629\\
3.3564839120978	128168.366939585\\
3.35658391459787	128127.113978335\\
3.35668391709793	128085.288059291\\
3.35678391959799	128044.035098041\\
3.35688392209805	128002.209178997\\
3.35698392459812	127960.383259952\\
3.35708392709818	127919.130298703\\
3.35718392959824	127877.304379658\\
3.3572839320983	127836.051418409\\
3.35738393459836	127794.225499364\\
3.35748393709843	127752.972538115\\
3.35758393959849	127711.14661907\\
3.35768394209855	127669.893657821\\
3.35778394459861	127628.067738776\\
3.35788394709868	127586.814777527\\
3.35798394959874	127544.988858482\\
3.3580839520988	127503.735897233\\
3.35818395459886	127461.909978188\\
3.35828395709893	127420.657016939\\
3.35838395959899	127379.404055689\\
3.35848396209905	127337.578136645\\
3.35858396459912	127296.325175396\\
3.35868396709918	127254.499256351\\
3.35878396959924	127213.246295102\\
3.3588839720993	127171.993333852\\
3.35898397459937	127130.167414808\\
3.35908397709943	127088.914453558\\
3.35918397959949	127047.661492309\\
3.35928398209955	127005.835573264\\
3.35938398459961	126964.582612015\\
3.35948398709968	126923.329650765\\
3.35958398959974	126881.503731721\\
3.3596839920998	126840.250770471\\
3.35978399459986	126798.997809222\\
3.35988399709993	126757.744847973\\
3.35998399959999	126715.918928928\\
3.36008400210005	126674.665967679\\
3.36018400460011	126633.413006429\\
3.36028400710018	126592.16004518\\
3.36038400960024	126550.90708393\\
3.3604840121003	126509.081164886\\
3.36058401460037	126467.828203636\\
3.36068401710043	126426.575242387\\
3.36078401960049	126385.322281138\\
3.36088402210055	126344.069319888\\
3.36098402460062	126302.816358639\\
3.36108402710068	126261.563397389\\
3.36118402960074	126219.737478345\\
3.3612840321008	126178.484517095\\
3.36138403460086	126137.231555846\\
3.36148403710093	126095.978594596\\
3.36158403960099	126054.725633347\\
3.36168404210105	126013.472672098\\
3.36178404460111	125972.219710848\\
3.36188404710118	125930.966749599\\
3.36198404960124	125889.713788349\\
3.3620840521013	125848.4608271\\
3.36218405460136	125807.207865851\\
3.36228405710143	125765.954904601\\
3.36238405960149	125724.701943352\\
3.36248406210155	125683.448982102\\
3.36258406460162	125642.196020853\\
3.36268406710168	125600.943059603\\
3.36278406960174	125559.690098354\\
3.3628840721018	125518.437137105\\
3.36298407460187	125477.75713365\\
3.36308407710193	125436.504172401\\
3.36318407960199	125395.251211151\\
3.36328408210205	125353.998249902\\
3.36338408460212	125312.745288653\\
3.36348408710218	125271.492327403\\
3.36358408960224	125230.239366154\\
3.3636840921023	125189.559362699\\
3.36378409460236	125148.30640145\\
3.36388409710243	125107.053440201\\
3.36398409960249	125065.800478951\\
3.36408410210255	125024.547517702\\
3.36418410460261	124983.867514248\\
3.36428410710268	124942.614552998\\
3.36438410960274	124901.361591749\\
3.3644841121028	124860.108630499\\
3.36458411460287	124819.428627045\\
3.36468411710293	124778.175665796\\
3.36478411960299	124736.922704546\\
3.36488412210305	124696.242701092\\
3.36498412460312	124654.989739842\\
3.36508412710318	124613.736778593\\
3.36518412960324	124573.056775139\\
3.3652841321033	124531.803813889\\
3.36538413460337	124490.55085264\\
3.36548413710343	124449.870849186\\
3.36558413960349	124408.617887936\\
3.36568414210355	124367.937884482\\
3.36578414460361	124326.684923232\\
3.36588414710368	124285.431961983\\
3.36598414960374	124244.751958529\\
3.3660841521038	124203.498997279\\
3.36618415460386	124162.818993825\\
3.36628415710393	124121.566032576\\
3.36638415960399	124080.886029121\\
3.36648416210405	124039.633067872\\
3.36658416460412	123998.953064418\\
3.36668416710418	123957.700103168\\
3.36678416960424	123917.020099714\\
3.3668841721043	123875.767138465\\
3.36698417460437	123835.08713501\\
3.36708417710443	123793.834173761\\
3.36718417960449	123753.154170307\\
3.36728418210455	123712.474166852\\
3.36738418460462	123671.221205603\\
3.36748418710468	123630.541202149\\
3.36758418960474	123589.288240899\\
3.3676841921048	123548.608237445\\
3.36778419460487	123507.928233991\\
3.36788419710493	123466.675272741\\
3.36798419960499	123425.995269287\\
3.36808420210505	123385.315265833\\
3.36818420460511	123344.062304583\\
3.36828420710518	123303.382301129\\
3.36838420960524	123262.702297675\\
3.3684842121053	123222.02229422\\
3.36858421460537	123180.769332971\\
3.36868421710543	123140.089329517\\
3.36878421960549	123099.409326062\\
3.36888422210555	123058.729322608\\
3.36898422460562	123017.476361359\\
3.36908422710568	122976.796357904\\
3.36918422960574	122936.11635445\\
3.3692842321058	122895.436350996\\
3.36938423460587	122854.756347541\\
3.36948423710593	122813.503386292\\
3.36958423960599	122772.823382838\\
3.36968424210605	122732.143379383\\
3.36978424460612	122691.463375929\\
3.36988424710618	122650.783372475\\
3.36998424960624	122610.10336902\\
3.3700842521063	122569.423365566\\
3.37018425460636	122528.743362112\\
3.37028425710643	122488.063358658\\
3.37038425960649	122447.383355203\\
3.37048426210655	122406.130393954\\
3.37058426460662	122365.4503905\\
3.37068426710668	122324.770387045\\
3.37078426960674	122284.090383591\\
3.3708842721068	122243.410380137\\
3.37098427460687	122202.730376683\\
3.37108427710693	122162.050373228\\
3.37118427960699	122121.370369774\\
3.37128428210705	122081.263324115\\
3.37138428460712	122040.58332066\\
3.37148428710718	121999.903317206\\
3.37158428960724	121959.223313752\\
3.3716842921073	121918.543310298\\
3.37178429460737	121877.863306843\\
3.37188429710743	121837.183303389\\
3.37198429960749	121796.503299935\\
3.37208430210755	121755.82329648\\
3.37218430460761	121715.143293026\\
3.37228430710768	121675.036247367\\
3.37238430960774	121634.356243913\\
3.3724843121078	121593.676240458\\
3.37258431460787	121552.996237004\\
3.37268431710793	121512.31623355\\
3.37278431960799	121472.209187891\\
3.37288432210805	121431.529184436\\
3.37298432460812	121390.849180982\\
3.37308432710818	121350.169177528\\
3.37318432960824	121310.062131869\\
3.3732843321083	121269.382128414\\
3.37338433460837	121228.70212496\\
3.37348433710843	121188.022121506\\
3.37358433960849	121147.915075847\\
3.37368434210855	121107.235072392\\
3.37378434460862	121066.555068938\\
3.37388434710868	121026.448023279\\
3.37398434960874	120985.768019825\\
3.3740843521088	120945.08801637\\
3.37418435460887	120904.980970711\\
3.37428435710893	120864.300967257\\
3.37438435960899	120824.193921598\\
3.37448436210905	120783.513918143\\
3.37458436460912	120742.833914689\\
3.37468436710918	120702.72686903\\
3.37478436960924	120662.046865576\\
3.3748843721093	120621.939819917\\
3.37498437460937	120581.259816462\\
3.37508437710943	120541.152770803\\
3.37518437960949	120500.472767349\\
3.37528438210955	120460.36572169\\
3.37538438460962	120419.685718235\\
3.37548438710968	120379.578672576\\
3.37558438960974	120338.898669122\\
3.3756843921098	120298.791623463\\
3.37578439460987	120258.111620008\\
3.37588439710993	120218.004574349\\
3.37598439960999	120177.89752869\\
3.37608440211005	120137.217525236\\
3.37618440461012	120097.110479577\\
3.37628440711018	120056.430476122\\
3.37638440961024	120016.323430463\\
3.3764844121103	119976.216384804\\
3.37658441461037	119935.53638135\\
3.37668441711043	119895.429335691\\
3.37678441961049	119855.322290032\\
3.37688442211055	119814.642286577\\
3.37698442461062	119774.535240918\\
3.37708442711068	119734.428195259\\
3.37718442961074	119694.3211496\\
3.3772844321108	119653.641146145\\
3.37738443461087	119613.534100486\\
3.37748443711093	119573.427054827\\
3.37758443961099	119533.320009168\\
3.37768444211105	119492.640005714\\
3.37778444461112	119452.532960055\\
3.37788444711118	119412.425914395\\
3.37798444961124	119372.318868736\\
3.3780844521113	119332.211823077\\
3.37818445461137	119291.531819623\\
3.37828445711143	119251.424773964\\
3.37838445961149	119211.317728304\\
3.37848446211155	119171.210682645\\
3.37858446461162	119131.103636986\\
3.37868446711168	119090.996591327\\
3.37878446961174	119050.889545668\\
3.3788844721118	119010.782500009\\
3.37898447461187	118970.67545435\\
3.37908447711193	118930.56840869\\
3.37918447961199	118890.461363031\\
3.37928448211205	118849.781359577\\
3.37938448461212	118809.674313918\\
3.37948448711218	118769.567268259\\
3.37958448961224	118729.460222599\\
3.3796844921123	118689.35317694\\
3.37978449461237	118649.819089076\\
3.37988449711243	118609.712043417\\
3.37998449961249	118569.604997758\\
3.38008450211255	118529.497952099\\
3.38018450461262	118489.39090644\\
3.38028450711268	118449.28386078\\
3.38038450961274	118409.176815121\\
3.3804845121128	118369.069769462\\
3.38058451461287	118328.962723803\\
3.38068451711293	118288.855678144\\
3.38078451961299	118248.748632485\\
3.38088452211305	118209.214544621\\
3.38098452461312	118169.107498962\\
3.38108452711318	118129.000453302\\
3.38118452961324	118088.893407643\\
3.3812845321133	118048.786361984\\
3.38138453461337	118009.25227412\\
3.38148453711343	117969.145228461\\
3.38158453961349	117929.038182802\\
3.38168454211355	117888.931137143\\
3.38178454461362	117848.824091483\\
3.38188454711368	117809.290003619\\
3.38198454961374	117769.18295796\\
3.3820845521138	117729.075912301\\
3.38218455461387	117689.541824437\\
3.38228455711393	117649.434778778\\
3.38238455961399	117609.327733119\\
3.38248456211405	117569.793645255\\
3.38258456461412	117529.686599596\\
3.38268456711418	117489.579553936\\
3.38278456961424	117450.045466072\\
3.3828845721143	117409.938420413\\
3.38298457461437	117369.831374754\\
3.38308457711443	117330.29728689\\
3.38318457961449	117290.190241231\\
3.38328458211455	117250.656153367\\
3.38338458461462	117210.549107708\\
3.38348458711468	117171.015019844\\
3.38358458961474	117130.907974184\\
3.3836845921148	117091.37388632\\
3.38378459461487	117051.266840661\\
3.38388459711493	117011.732752797\\
3.38398459961499	116971.625707138\\
3.38408460211505	116932.091619274\\
3.38418460461512	116891.984573615\\
3.38428460711518	116852.450485751\\
3.38438460961524	116812.343440092\\
3.3844846121153	116772.809352228\\
3.38458461461537	116732.702306569\\
3.38468461711543	116693.168218705\\
3.38478461961549	116653.63413084\\
3.38488462211555	116613.527085181\\
3.38498462461562	116573.992997317\\
3.38508462711568	116534.458909453\\
3.38518462961574	116494.351863794\\
3.3852846321158	116454.81777593\\
3.38538463461587	116415.283688066\\
3.38548463711593	116375.176642407\\
3.38558463961599	116335.642554543\\
3.38568464211605	116296.108466679\\
3.38578464461612	116256.00142102\\
3.38588464711618	116216.467333156\\
3.38598464961624	116176.933245292\\
3.3860846521163	116137.399157428\\
3.38618465461637	116097.292111768\\
3.38628465711643	116057.758023904\\
3.38638465961649	116018.22393604\\
3.38648466211655	115978.689848176\\
3.38658466461662	115939.155760312\\
3.38668466711668	115899.048714653\\
3.38678466961674	115859.514626789\\
3.3868846721168	115819.980538925\\
3.38698467461687	115780.446451061\\
3.38708467711693	115740.912363197\\
3.38718467961699	115701.378275333\\
3.38728468211705	115661.844187469\\
3.38738468461712	115622.310099605\\
3.38748468711718	115582.776011741\\
3.38758468961724	115542.668966082\\
3.3876846921173	115503.134878218\\
3.38778469461737	115463.600790354\\
3.38788469711743	115424.06670249\\
3.38798469961749	115384.532614626\\
3.38808470211755	115344.998526762\\
3.38818470461762	115305.464438898\\
3.38828470711768	115265.930351034\\
3.38838470961774	115226.39626317\\
3.3884847121178	115186.862175306\\
3.38858471461787	115147.328087442\\
3.38868471711793	115108.366957373\\
3.38878471961799	115068.832869509\\
3.38888472211805	115029.298781645\\
3.38898472461812	114989.764693781\\
3.38908472711818	114950.230605917\\
3.38918472961824	114910.696518053\\
3.3892847321183	114871.162430189\\
3.38938473461837	114831.628342324\\
3.38948473711843	114792.09425446\\
3.38958473961849	114753.133124392\\
3.38968474211855	114713.599036528\\
3.38978474461862	114674.064948663\\
3.38988474711868	114634.530860799\\
3.38998474961874	114594.996772935\\
3.3900847521188	114556.035642867\\
3.39018475461887	114516.501555003\\
3.39028475711893	114476.967467138\\
3.39038475961899	114437.433379274\\
3.39048476211905	114398.472249206\\
3.39058476461912	114358.938161342\\
3.39068476711918	114319.404073478\\
3.39078476961924	114280.442943409\\
3.3908847721193	114240.908855545\\
3.39098477461937	114201.374767681\\
3.39108477711943	114162.413637612\\
3.39118477961949	114122.879549748\\
3.39128478211955	114083.345461884\\
3.39138478461962	114044.384331815\\
3.39148478711968	114004.850243951\\
3.39158478961974	113965.889113882\\
3.3916847921198	113926.355026018\\
3.39178479461987	113886.820938154\\
3.39188479711993	113847.859808085\\
3.39198479961999	113808.325720221\\
3.39208480212005	113769.364590152\\
3.39218480462012	113729.830502288\\
3.39228480712018	113690.869372219\\
3.39238480962024	113651.335284355\\
3.3924848121203	113612.374154286\\
3.39258481462037	113572.840066422\\
3.39268481712043	113533.878936353\\
3.39278481962049	113494.344848489\\
3.39288482212055	113455.38371842\\
3.39298482462062	113415.849630556\\
3.39308482712068	113376.888500487\\
3.39318482962074	113337.927370418\\
3.3932848321208	113298.393282554\\
3.39338483462087	113259.432152485\\
3.39348483712093	113220.471022417\\
3.39358483962099	113180.936934553\\
3.39368484212105	113141.975804484\\
3.39378484462112	113102.44171662\\
3.39388484712118	113063.480586551\\
3.39398484962124	113024.519456482\\
3.3940848521213	112985.558326413\\
3.39418485462137	112946.024238549\\
3.39428485712143	112907.06310848\\
3.39438485962149	112868.101978411\\
3.39448486212155	112828.567890547\\
3.39458486462162	112789.606760478\\
3.39468486712168	112750.645630409\\
3.39478486962174	112711.68450034\\
3.3948848721218	112672.723370272\\
3.39498487462187	112633.189282407\\
3.39508487712193	112594.228152339\\
3.39518487962199	112555.26702227\\
3.39528488212205	112516.305892201\\
3.39538488462212	112477.344762132\\
3.39548488712218	112438.383632063\\
3.39558488962224	112399.422501994\\
3.3956848921223	112359.88841413\\
3.39578489462237	112320.927284061\\
3.39588489712243	112281.966153992\\
3.39598489962249	112243.005023923\\
3.39608490212255	112204.043893854\\
3.39618490462262	112165.082763786\\
3.39628490712268	112126.121633717\\
3.39638490962274	112087.160503648\\
3.3964849121228	112048.199373579\\
3.39658491462287	112009.23824351\\
3.39668491712293	111970.277113441\\
3.39678491962299	111931.315983372\\
3.39688492212305	111892.354853303\\
3.39698492462312	111853.393723234\\
3.39708492712318	111814.432593166\\
3.39718492962324	111775.471463097\\
3.3972849321233	111736.510333028\\
3.39738493462337	111697.549202959\\
3.39748493712343	111658.58807289\\
3.39758493962349	111620.199900616\\
3.39768494212355	111581.238770547\\
3.39778494462362	111542.277640478\\
3.39788494712368	111503.31651041\\
3.39798494962374	111464.355380341\\
3.3980849521238	111425.394250272\\
3.39818495462387	111387.006077998\\
3.39828495712393	111348.044947929\\
3.39838495962399	111309.08381786\\
3.39848496212405	111270.122687791\\
3.39858496462412	111231.161557722\\
3.39868496712418	111192.773385449\\
3.39878496962424	111153.81225538\\
3.3988849721243	111114.851125311\\
3.39898497462437	111075.889995242\\
3.39908497712443	111037.501822968\\
3.39918497962449	110998.540692899\\
3.39928498212455	110959.57956283\\
3.39938498462462	110921.191390557\\
3.39948498712468	110882.230260488\\
3.39958498962474	110843.269130419\\
3.3996849921248	110804.880958145\\
3.39978499462487	110765.919828076\\
3.39988499712493	110727.531655802\\
3.39998499962499	110688.570525733\\
3.40008500212505	110649.609395665\\
3.40018500462512	110611.221223391\\
3.40028500712518	110572.260093322\\
3.40038500962524	110533.871921048\\
3.4004850121253	110494.910790979\\
3.40058501462537	110456.522618705\\
3.40068501712543	110417.561488637\\
3.40078501962549	110379.173316363\\
3.40088502212555	110340.212186294\\
3.40098502462562	110301.82401402\\
3.40108502712568	110262.862883951\\
3.40118502962574	110224.474711678\\
3.4012850321258	110185.513581609\\
3.40138503462587	110147.125409335\\
3.40148503712593	110108.164279266\\
3.40158503962599	110069.776106992\\
3.40168504212605	110031.387934718\\
3.40178504462612	109992.42680465\\
3.40188504712618	109954.038632376\\
3.40198504962624	109915.077502307\\
3.4020850521263	109876.689330033\\
3.40218505462637	109838.301157759\\
3.40228505712643	109799.34002769\\
3.40238505962649	109760.951855417\\
3.40248506212655	109722.563683143\\
3.40258506462662	109684.175510869\\
3.40268506712668	109645.2143808\\
3.40278506962674	109606.826208526\\
3.4028850721268	109568.438036253\\
3.40298507462687	109530.049863979\\
3.40308507712693	109491.08873391\\
3.40318507962699	109452.700561636\\
3.40328508212705	109414.312389363\\
3.40338508462712	109375.924217089\\
3.40348508712718	109336.96308702\\
3.40358508962724	109298.574914746\\
3.4036850921273	109260.186742472\\
3.40378509462737	109221.798570199\\
3.40388509712743	109183.410397925\\
3.40398509962749	109145.022225651\\
3.40408510212755	109106.634053377\\
3.40418510462762	109068.245881104\\
3.40428510712768	109029.284751035\\
3.40438510962774	108990.896578761\\
3.4044851121278	108952.508406487\\
3.40458511462787	108914.120234213\\
3.40468511712793	108875.73206194\\
3.40478511962799	108837.343889666\\
3.40488512212805	108798.955717392\\
3.40498512462812	108760.567545118\\
3.40508512712818	108722.179372844\\
3.40518512962824	108683.791200571\\
3.4052851321283	108645.403028297\\
3.40538513462837	108607.014856023\\
3.40548513712843	108568.626683749\\
3.40558513962849	108530.238511476\\
3.40568514212855	108492.423296997\\
3.40578514462862	108454.035124723\\
3.40588514712868	108415.64695245\\
3.40598514962874	108377.258780176\\
3.4060851521288	108338.870607902\\
3.40618515462887	108300.482435628\\
3.40628515712893	108262.094263354\\
3.40638515962899	108223.706091081\\
3.40648516212905	108185.890876602\\
3.40658516462912	108147.502704328\\
3.40668516712918	108109.114532055\\
3.40678516962924	108070.726359781\\
3.4068851721293	108032.338187507\\
3.40698517462937	107994.522973028\\
3.40708517712943	107956.134800755\\
3.40718517962949	107917.746628481\\
3.40728518212955	107879.358456207\\
3.40738518462962	107841.543241728\\
3.40748518712968	107803.155069455\\
3.40758518962974	107764.766897181\\
3.4076851921298	107726.951682702\\
3.40778519462987	107688.563510428\\
3.40788519712993	107650.175338155\\
3.40798519962999	107612.360123676\\
3.40808520213005	107573.971951402\\
3.40818520463012	107535.583779129\\
3.40828520713018	107497.76856465\\
3.40838520963024	107459.380392376\\
3.4084852121303	107421.565177898\\
3.40858521463037	107383.177005624\\
3.40868521713043	107345.361791145\\
3.40878521963049	107306.973618871\\
3.40888522213055	107268.585446598\\
3.40898522463062	107230.770232119\\
3.40908522713068	107192.382059845\\
3.40918522963074	107154.566845367\\
3.4092852321308	107116.178673093\\
3.40938523463087	107078.363458614\\
3.40948523713093	107040.548244136\\
3.40958523963099	107002.160071862\\
3.40968524213105	106964.344857383\\
3.40978524463112	106925.956685109\\
3.40988524713118	106888.141470631\\
3.40998524963124	106849.753298357\\
3.4100852521313	106811.938083878\\
3.41018525463137	106774.1228694\\
3.41028525713143	106735.734697126\\
3.41038525963149	106697.919482647\\
3.41048526213155	106660.104268169\\
3.41058526463162	106621.716095895\\
3.41068526713168	106583.900881416\\
3.41078526963174	106546.085666938\\
3.4108852721318	106507.697494664\\
3.41098527463187	106469.882280185\\
3.41108527713193	106432.067065707\\
3.41118527963199	106393.678893433\\
3.41128528213205	106355.863678954\\
3.41138528463212	106318.048464476\\
3.41148528713218	106280.233249997\\
3.41158528963224	106242.418035518\\
3.4116852921323	106204.029863245\\
3.41178529463237	106166.214648766\\
3.41188529713243	106128.399434287\\
3.41198529963249	106090.584219809\\
3.41208530213255	106052.76900533\\
3.41218530463262	106014.953790851\\
3.41228530713268	105976.565618578\\
3.41238530963274	105938.750404099\\
3.4124853121328	105900.93518962\\
3.41258531463287	105863.119975142\\
3.41268531713293	105825.304760663\\
3.41278531963299	105787.489546184\\
3.41288532213305	105749.674331706\\
3.41298532463312	105711.859117227\\
3.41308532713318	105674.043902749\\
3.41318532963324	105636.22868827\\
3.4132853321333	105598.413473791\\
3.41338533463337	105560.598259313\\
3.41348533713343	105522.783044834\\
3.41358533963349	105484.967830355\\
3.41368534213355	105447.152615877\\
3.41378534463362	105409.337401398\\
3.41388534713368	105371.522186919\\
3.41398534963374	105333.706972441\\
3.4140853521338	105295.891757962\\
3.41418535463387	105258.076543484\\
3.41428535713393	105220.261329005\\
3.41438535963399	105183.019072321\\
3.41448536213405	105145.203857843\\
3.41458536463412	105107.388643364\\
3.41468536713418	105069.573428885\\
3.41478536963424	105031.758214407\\
3.4148853721343	104993.942999928\\
3.41498537463437	104956.700743245\\
3.41508537713443	104918.885528766\\
3.41518537963449	104881.070314287\\
3.41528538213455	104843.255099809\\
3.41538538463462	104806.012843125\\
3.41548538713468	104768.197628647\\
3.41558538963474	104730.382414168\\
3.4156853921348	104692.567199689\\
3.41578539463487	104655.324943006\\
3.41588539713493	104617.509728527\\
3.41598539963499	104579.694514049\\
3.41608540213505	104542.452257365\\
3.41618540463512	104504.637042887\\
3.41628540713518	104466.821828408\\
3.41638540963524	104429.579571724\\
3.4164854121353	104391.764357246\\
3.41658541463537	104353.949142767\\
3.41668541713543	104316.706886084\\
3.41678541963549	104278.891671605\\
3.41688542213555	104241.649414921\\
3.41698542463562	104203.834200443\\
3.41708542713568	104166.591943759\\
3.41718542963574	104128.776729281\\
3.4172854321358	104091.534472597\\
3.41738543463587	104053.719258119\\
3.41748543713593	104016.477001435\\
3.41758543963599	103978.661786956\\
3.41768544213605	103941.419530273\\
3.41778544463612	103903.604315794\\
3.41788544713618	103866.362059111\\
3.41798544963624	103828.546844632\\
3.4180854521363	103791.304587949\\
3.41818545463637	103753.48937347\\
3.41828545713643	103716.247116787\\
3.41838545963649	103679.004860103\\
3.41848546213655	103641.189645624\\
3.41858546463662	103603.947388941\\
3.41868546713668	103566.132174462\\
3.41878546963674	103528.889917779\\
3.4188854721368	103491.647661095\\
3.41898547463687	103453.832446617\\
3.41908547713693	103416.590189933\\
3.41918547963699	103379.34793325\\
3.41928548213705	103342.105676566\\
3.41938548463712	103304.290462087\\
3.41948548713718	103267.048205404\\
3.41958548963724	103229.80594872\\
3.4196854921373	103192.563692037\\
3.41978549463737	103154.748477558\\
3.41988549713743	103117.506220875\\
3.41998549963749	103080.263964191\\
3.42008550213755	103043.021707508\\
3.42018550463762	103005.779450824\\
3.42028550713768	102967.964236346\\
3.42038550963774	102930.721979662\\
3.4204855121378	102893.479722979\\
3.42058551463787	102856.237466295\\
3.42068551713793	102818.995209612\\
3.42078551963799	102781.752952928\\
3.42088552213805	102744.510696245\\
3.42098552463812	102707.268439561\\
3.42108552713818	102670.026182878\\
3.42118552963824	102632.783926194\\
3.4212855321383	102595.541669511\\
3.42138553463837	102558.299412827\\
3.42148553713843	102521.057156144\\
3.42158553963849	102483.81489946\\
3.42168554213855	102446.572642777\\
3.42178554463862	102409.330386093\\
3.42188554713868	102372.08812941\\
3.42198554963874	102334.845872726\\
3.4220855521388	102297.603616043\\
3.42218555463887	102260.361359359\\
3.42228555713893	102223.119102676\\
3.42238555963899	102185.876845992\\
3.42248556213905	102148.634589309\\
3.42258556463912	102111.392332625\\
3.42268556713918	102074.150075942\\
3.42278556963924	102036.907819258\\
3.4228855721393	102000.23852037\\
3.42298557463937	101962.996263686\\
3.42308557713943	101925.754007003\\
3.42318557963949	101888.511750319\\
3.42328558213955	101851.269493636\\
3.42338558463962	101814.600194747\\
3.42348558713968	101777.357938064\\
3.42358558963974	101740.11568138\\
3.4236855921398	101702.873424697\\
3.42378559463987	101666.204125808\\
3.42388559713993	101628.961869125\\
3.42398559963999	101591.719612441\\
3.42408560214005	101554.477355758\\
3.42418560464012	101517.80805687\\
3.42428560714018	101480.565800186\\
3.42438560964024	101443.323543503\\
3.4244856121403	101406.654244614\\
3.42458561464037	101369.411987931\\
3.42468561714043	101332.169731247\\
3.42478561964049	101295.500432359\\
3.42488562214055	101258.258175675\\
3.42498562464062	101221.588876787\\
3.42508562714068	101184.346620103\\
3.42518562964074	101147.677321215\\
3.4252856321408	101110.435064531\\
3.42538563464087	101073.192807848\\
3.42548563714093	101036.52350896\\
3.42558563964099	100999.281252276\\
3.42568564214105	100962.611953388\\
3.42578564464112	100925.369696704\\
3.42588564714118	100888.700397816\\
3.42598564964124	100851.458141132\\
3.4260856521413	100814.788842244\\
3.42618565464137	100778.119543356\\
3.42628565714143	100740.877286672\\
3.42638565964149	100704.207987784\\
3.42648566214155	100666.9657311\\
3.42658566464162	100630.296432212\\
3.42668566714168	100593.627133324\\
3.42678566964174	100556.38487664\\
3.4268856721418	100519.715577752\\
3.42698567464187	100482.473321068\\
3.42708567714193	100445.80402218\\
3.42718567964199	100409.134723291\\
3.42728568214205	100372.465424403\\
3.42738568464212	100335.22316772\\
3.42748568714218	100298.553868831\\
3.42758568964224	100261.884569943\\
3.4276856921423	100224.642313259\\
3.42778569464237	100187.973014371\\
3.42788569714243	100151.303715482\\
3.42798569964249	100114.634416594\\
3.42808570214255	100077.392159911\\
3.42818570464262	100040.722861022\\
3.42828570714268	100004.053562134\\
3.42838570964274	99967.3842632455\\
3.4284857121428	99930.7149643571\\
3.42858571464287	99894.0456654688\\
3.42868571714293	99856.8034087853\\
3.42878571964299	99820.1341098969\\
3.42888572214305	99783.4648110085\\
3.42898572464312	99746.7955121201\\
3.42908572714318	99710.1262132318\\
3.42918572964324	99673.4569143434\\
3.4292857321433	99636.787615455\\
3.42938573464337	99600.1183165667\\
3.42948573714343	99563.4490176783\\
3.42958573964349	99526.7797187899\\
3.42968574214355	99490.1104199015\\
3.42978574464362	99453.4411210132\\
3.42988574714368	99416.7718221248\\
3.42998574964374	99380.1025232364\\
3.4300857521438	99343.433224348\\
3.43018575464387	99306.7639254597\\
3.43028575714393	99270.0946265713\\
3.43038575964399	99233.4253276829\\
3.43048576214405	99196.7560287946\\
3.43058576464412	99160.0867299062\\
3.43068576714418	99123.4174310178\\
3.43078576964424	99086.7481321294\\
3.4308857721443	99050.6517910362\\
3.43098577464437	99013.9824921478\\
3.43108577714443	98977.3131932595\\
3.43118577964449	98940.6438943711\\
3.43128578214455	98903.9745954827\\
3.43138578464462	98867.3052965943\\
3.43148578714468	98831.2089555011\\
3.43158578964474	98794.5396566127\\
3.4316857921448	98757.8703577243\\
3.43178579464487	98721.201058836\\
3.43188579714493	98685.1047177427\\
3.43198579964499	98648.4354188544\\
3.43208580214505	98611.766119966\\
3.43218580464512	98575.0968210776\\
3.43228580714518	98539.0004799844\\
3.43238580964524	98502.331181096\\
3.4324858121453	98465.6618822076\\
3.43258581464537	98429.5655411144\\
3.43268581714543	98392.896242226\\
3.43278581964549	98356.2269433376\\
3.43288582214555	98320.1306022444\\
3.43298582464562	98283.461303356\\
3.43308582714568	98247.3649622628\\
3.43318582964574	98210.6956633744\\
3.4332858321458	98174.026364486\\
3.43338583464587	98137.9300233928\\
3.43348583714593	98101.2607245044\\
3.43358583964599	98065.1643834112\\
3.43368584214605	98028.4950845228\\
3.43378584464612	97992.3987434296\\
3.43388584714618	97955.7294445412\\
3.43398584964624	97919.6331034479\\
3.4340858521463	97882.9638045596\\
3.43418585464637	97846.8674634663\\
3.43428585714643	97810.198164578\\
3.43438585964649	97774.1018234847\\
3.43448586214655	97737.4325245964\\
3.43458586464662	97701.3361835031\\
3.43468586714668	97665.2398424099\\
3.43478586964674	97628.5705435215\\
3.4348858721468	97592.4742024282\\
3.43498587464687	97555.8049035399\\
3.43508587714693	97519.7085624466\\
3.43518587964699	97483.6122213534\\
3.43528588214705	97446.942922465\\
3.43538588464712	97410.8465813718\\
3.43548588714718	97374.7502402785\\
3.43558588964724	97338.0809413902\\
3.4356858921473	97301.9846002969\\
3.43578589464737	97265.8882592037\\
3.43588589714743	97229.7919181105\\
3.43598589964749	97193.1226192221\\
3.43608590214755	97157.0262781288\\
3.43618590464762	97120.9299370356\\
3.43628590714768	97084.8335959423\\
3.43638590964774	97048.7372548491\\
3.4364859121478	97012.0679559607\\
3.43658591464787	96975.9716148675\\
3.43668591714793	96939.8752737743\\
3.43678591964799	96903.778932681\\
3.43688592214805	96867.6825915878\\
3.43698592464812	96831.5862504945\\
3.43708592714818	96795.4899094013\\
3.43718592964824	96758.8206105129\\
3.4372859321483	96722.7242694197\\
3.43738593464837	96686.6279283264\\
3.43748593714843	96650.5315872332\\
3.43758593964849	96614.4352461399\\
3.43768594214855	96578.3389050467\\
3.43778594464862	96542.2425639535\\
3.43788594714868	96506.1462228602\\
3.43798594964874	96470.049881767\\
3.4380859521488	96433.9535406737\\
3.43818595464887	96397.8571995805\\
3.43828595714893	96361.7608584872\\
3.43838595964899	96325.664517394\\
3.43848596214905	96289.5681763007\\
3.43858596464912	96253.4718352075\\
3.43868596714918	96217.3754941143\\
3.43878596964924	96181.8521108162\\
3.4388859721493	96145.7557697229\\
3.43898597464937	96109.6594286297\\
3.43908597714943	96073.5630875364\\
3.43918597964949	96037.4667464432\\
3.43928598214955	96001.37040535\\
3.43938598464962	95965.2740642567\\
3.43948598714968	95929.7506809586\\
3.43958598964974	95893.6543398654\\
3.4396859921498	95857.5579987721\\
3.43978599464987	95821.4616576789\\
3.43988599714993	95785.3653165856\\
3.43998599964999	95749.8419332875\\
3.44008600215005	95713.7455921943\\
3.44018600465012	95677.649251101\\
3.44028600715018	95642.1258678029\\
3.44038600965024	95606.0295267097\\
3.4404860121503	95569.9331856164\\
3.44058601465037	95533.8368445232\\
3.44068601715043	95498.3134612251\\
3.44078601965049	95462.2171201319\\
3.44088602215055	95426.1207790386\\
3.44098602465062	95390.5973957405\\
3.44108602715068	95354.5010546473\\
3.44118602965074	95318.9776713492\\
3.4412860321508	95282.8813302559\\
3.44138603465087	95246.7849891627\\
3.44148603715093	95211.2616058646\\
3.44158603965099	95175.1652647713\\
3.44168604215105	95139.6418814732\\
3.44178604465112	95103.5455403799\\
3.44188604715118	95068.0221570818\\
3.44198604965124	95031.9258159886\\
3.4420860521513	94996.4024326905\\
3.44218605465137	94960.3060915972\\
3.44228605715143	94924.7827082991\\
3.44238605965149	94888.6863672059\\
3.44248606215155	94853.1629839078\\
3.44258606465162	94817.0666428145\\
3.44268606715168	94781.5432595164\\
3.44278606965174	94746.0198762183\\
3.4428860721518	94709.9235351251\\
3.44298607465187	94674.400151827\\
3.44308607715193	94638.3038107337\\
3.44318607965199	94602.7804274356\\
3.44328608215205	94567.2570441375\\
3.44338608465212	94531.1607030443\\
3.44348608715218	94495.6373197462\\
3.44358608965224	94460.113936448\\
3.4436860921523	94424.0175953548\\
3.44378609465237	94388.4942120567\\
3.44388609715243	94352.9708287586\\
3.44398609965249	94317.4474454605\\
3.44408610215255	94281.3511043672\\
3.44418610465262	94245.8277210691\\
3.44428610715268	94210.304337771\\
3.44438610965274	94174.7809544729\\
3.4444861121528	94138.6846133796\\
3.44458611465287	94103.1612300815\\
3.44468611715293	94067.6378467834\\
3.44478611965299	94032.1144634853\\
3.44488612215305	93996.5910801872\\
3.44498612465312	93961.0676968891\\
3.44508612715318	93925.544313591\\
3.44518612965324	93889.4479724977\\
3.4452861321533	93853.9245891996\\
3.44538613465337	93818.4012059015\\
3.44548613715343	93782.8778226034\\
3.44558613965349	93747.3544393053\\
3.44568614215355	93711.8310560072\\
3.44578614465362	93676.3076727091\\
3.44588614715368	93640.784289411\\
3.44598614965374	93605.2609061128\\
3.4460861521538	93569.7375228147\\
3.44618615465387	93534.2141395166\\
3.44628615715393	93498.6907562185\\
3.44638615965399	93463.1673729204\\
3.44648616215405	93427.6439896223\\
3.44658616465412	93392.1206063242\\
3.44668616715418	93356.5972230261\\
3.44678616965424	93321.073839728\\
3.4468861721543	93285.5504564299\\
3.44698617465437	93250.0270731317\\
3.44708617715443	93215.0766476288\\
3.44718617965449	93179.5532643307\\
3.44728618215455	93144.0298810325\\
3.44738618465462	93108.5064977344\\
3.44748618715468	93072.9831144363\\
3.44758618965474	93037.4597311382\\
3.4476861921548	93002.5093056352\\
3.44778619465487	92966.9859223371\\
3.44788619715493	92931.462539039\\
3.44798619965499	92895.9391557409\\
3.44808620215505	92860.4157724428\\
3.44818620465512	92825.4653469398\\
3.44828620715518	92789.9419636417\\
3.44838620965524	92754.4185803436\\
3.4484862121553	92719.4681548406\\
3.44858621465537	92683.9447715425\\
3.44868621715543	92648.4213882444\\
3.44878621965549	92613.4709627414\\
3.44888622215555	92577.9475794433\\
3.44898622465562	92542.4241961452\\
3.44908622715568	92507.4737706422\\
3.44918622965574	92471.9503873441\\
3.4492862321558	92436.427004046\\
3.44938623465587	92401.476578543\\
3.44948623715593	92365.9531952449\\
3.44958623965599	92331.0027697419\\
3.44968624215605	92295.4793864438\\
3.44978624465612	92260.5289609408\\
3.44988624715618	92225.0055776427\\
3.44998624965624	92189.4821943446\\
3.4500862521563	92154.5317688416\\
3.45018625465637	92119.0083855435\\
3.45028625715643	92084.0579600405\\
3.45038625965649	92048.5345767424\\
3.45048626215655	92013.5841512394\\
3.45058626465662	91978.6337257364\\
3.45068626715668	91943.1103424383\\
3.45078626965674	91908.1599169354\\
3.4508862721568	91872.6365336373\\
3.45098627465687	91837.6861081343\\
3.45108627715693	91802.7356826313\\
3.45118627965699	91767.2122993332\\
3.45128628215705	91732.2618738302\\
3.45138628465712	91696.7384905321\\
3.45148628715718	91661.7880650291\\
3.45158628965724	91626.8376395261\\
3.4516862921573	91591.314256228\\
3.45178629465737	91556.363830725\\
3.45188629715743	91521.413405222\\
3.45198629965749	91486.4629797191\\
3.45208630215755	91450.939596421\\
3.45218630465762	91415.989170918\\
3.45228630715768	91381.038745415\\
3.45238630965774	91346.088319912\\
3.4524863121578	91310.5649366139\\
3.45258631465787	91275.6145111109\\
3.45268631715793	91240.6640856079\\
3.45278631965799	91205.713660105\\
3.45288632215805	91170.763234602\\
3.45298632465812	91135.812809099\\
3.45308632715818	91100.2894258009\\
3.45318632965824	91065.3390002979\\
3.4532863321583	91030.3885747949\\
3.45338633465837	90995.438149292\\
3.45348633715843	90960.487723789\\
3.45358633965849	90925.537298286\\
3.45368634215855	90890.586872783\\
3.45378634465862	90855.63644728\\
3.45388634715868	90820.686021777\\
3.45398634965874	90785.7355962741\\
3.4540863521588	90750.7851707711\\
3.45418635465887	90715.8347452681\\
3.45428635715893	90680.8843197651\\
3.45438635965899	90645.9338942621\\
3.45448636215905	90610.9834687592\\
3.45458636465912	90576.0330432562\\
3.45468636715918	90541.0826177532\\
3.45478636965924	90506.1321922502\\
3.4548863721593	90471.1817667472\\
3.45498637465937	90436.2313412443\\
3.45508637715943	90401.2809157413\\
3.45518637965949	90366.3304902383\\
3.45528638215955	90331.3800647353\\
3.45538638465962	90297.0025970275\\
3.45548638715968	90262.0521715245\\
3.45558638965974	90227.1017460215\\
3.4556863921598	90192.1513205185\\
3.45578639465987	90157.2008950156\\
3.45588639715993	90122.8234273077\\
3.45598639965999	90087.8730018047\\
3.45608640216005	90052.9225763017\\
3.45618640466012	90017.9721507988\\
3.45628640716018	89983.0217252958\\
3.45638640966024	89948.6442575879\\
3.4564864121603	89913.693832085\\
3.45658641466037	89878.743406582\\
3.45668641716043	89844.3659388741\\
3.45678641966049	89809.4155133711\\
3.45688642216055	89774.4650878682\\
3.45698642466062	89740.0876201603\\
3.45708642716068	89705.1371946573\\
3.45718642966074	89670.1867691544\\
3.4572864321608	89635.8093014465\\
3.45738643466087	89600.8588759435\\
3.45748643716093	89566.4814082357\\
3.45758643966099	89531.5309827327\\
3.45768644216105	89496.5805572297\\
3.45778644466112	89462.2030895219\\
3.45788644716118	89427.2526640189\\
3.45798644966124	89392.875196311\\
3.4580864521613	89357.9247708081\\
3.45818645466137	89323.5473031002\\
3.45828645716143	89288.5968775972\\
3.45838645966149	89254.2194098894\\
3.45848646216155	89219.2689843864\\
3.45858646466162	89184.8915166785\\
3.45868646716168	89149.9410911756\\
3.45878646966174	89115.5636234677\\
3.4588864721618	89080.6131979647\\
3.45898647466187	89046.2357302569\\
3.45908647716193	89011.858262549\\
3.45918647966199	88976.9078370461\\
3.45928648216205	88942.5303693382\\
3.45938648466212	88908.1529016304\\
3.45948648716218	88873.2024761274\\
3.45958648966224	88838.8250084195\\
3.4596864921623	88804.4475407117\\
3.45978649466237	88769.4971152087\\
3.45988649716243	88735.1196475008\\
3.45998649966249	88700.742179793\\
3.46008650216255	88665.79175429\\
3.46018650466262	88631.4142865822\\
3.46028650716268	88597.0368188743\\
3.46038650966274	88562.6593511665\\
3.4604865121628	88527.7089256635\\
3.46058651466287	88493.3314579556\\
3.46068651716293	88458.9539902478\\
3.46078651966299	88424.57652254\\
3.46088652216305	88390.1990548321\\
3.46098652466312	88355.8215871243\\
3.46108652716318	88320.8711616213\\
3.46118652966324	88286.4936939134\\
3.4612865321633	88252.1162262056\\
3.46138653466337	88217.7387584977\\
3.46148653716343	88183.3612907899\\
3.46158653966349	88148.983823082\\
3.46168654216355	88114.6063553742\\
3.46178654466362	88080.2288876663\\
3.46188654716368	88045.8514199585\\
3.46198654966374	88011.4739522506\\
3.4620865521638	87977.0964845428\\
3.46218655466387	87942.7190168349\\
3.46228655716393	87908.3415491271\\
3.46238655966399	87873.9640814192\\
3.46248656216405	87839.5866137114\\
3.46258656466412	87805.2091460035\\
3.46268656716418	87770.8316782957\\
3.46278656966424	87736.4542105878\\
3.4628865721643	87702.07674288\\
3.46298657466437	87667.6992751721\\
3.46308657716443	87633.3218074643\\
3.46318657966449	87598.9443397564\\
3.46328658216455	87564.5668720486\\
3.46338658466462	87530.1894043407\\
3.46348658716468	87496.384894428\\
3.46358658966474	87462.0074267202\\
3.4636865921648	87427.6299590123\\
3.46378659466487	87393.2524913045\\
3.46388659716493	87358.8750235966\\
3.46398659966499	87324.4975558888\\
3.46408660216505	87290.693045976\\
3.46418660466512	87256.3155782682\\
3.46428660716518	87221.9381105604\\
3.46438660966524	87187.5606428525\\
3.4644866121653	87153.7561329398\\
3.46458661466537	87119.3786652319\\
3.46468661716543	87085.0011975241\\
3.46478661966549	87051.1966876114\\
3.46488662216555	87016.8192199035\\
3.46498662466562	86982.4417521957\\
3.46508662716568	86948.637242283\\
3.46518662966574	86914.2597745751\\
3.4652866321658	86879.8823068673\\
3.46538663466587	86846.0777969545\\
3.46548663716593	86811.7003292467\\
3.46558663966599	86777.895819334\\
3.46568664216605	86743.5183516261\\
3.46578664466612	86709.1408839183\\
3.46588664716618	86675.3363740055\\
3.46598664966624	86640.9589062977\\
3.4660866521663	86607.154396385\\
3.46618665466637	86572.7769286771\\
3.46628665716643	86538.9724187644\\
3.46638665966649	86504.5949510566\\
3.46648666216655	86470.7904411438\\
3.46658666466662	86436.412973436\\
3.46668666716668	86402.6084635233\\
3.46678666966674	86368.2309958154\\
3.4668866721668	86334.4264859027\\
3.46698667466687	86300.62197599\\
3.46708667716693	86266.2445082822\\
3.46718667966699	86232.4399983694\\
3.46728668216705	86198.0625306616\\
3.46738668466712	86164.2580207488\\
3.46748668716718	86130.4535108361\\
3.46758668966724	86096.0760431283\\
3.4676866921673	86062.2715332156\\
3.46778669466737	86028.4670233029\\
3.46788669716743	85994.089555595\\
3.46798669966749	85960.2850456823\\
3.46808670216755	85926.4805357696\\
3.46818670466762	85892.6760258568\\
3.46828670716768	85858.298558149\\
3.46838670966774	85824.4940482363\\
3.4684867121678	85790.6895383235\\
3.46858671466787	85756.8850284108\\
3.46868671716793	85722.507560703\\
3.46878671966799	85688.7030507903\\
3.46888672216805	85654.8985408775\\
3.46898672466812	85621.0940309648\\
3.46908672716818	85587.2895210521\\
3.46918672966824	85553.4850111394\\
3.4692867321683	85519.1075434315\\
3.46938673466837	85485.3030335188\\
3.46948673716843	85451.4985236061\\
3.46958673966849	85417.6940136934\\
3.46968674216855	85383.8895037807\\
3.46978674466862	85350.084993868\\
3.46988674716868	85316.2804839552\\
3.46998674966874	85282.4759740425\\
3.4700867521688	85248.6714641298\\
3.47018675466887	85214.8669542171\\
3.47028675716893	85181.0624443044\\
3.47038675966899	85147.2579343916\\
3.47048676216905	85113.4534244789\\
3.47058676466912	85079.6489145662\\
3.47068676716918	85045.8444046535\\
3.47078676966924	85012.0398947408\\
3.4708867721693	84978.235384828\\
3.47098677466937	84944.4308749153\\
3.47108677716943	84910.6263650026\\
3.47118677966949	84876.8218550899\\
3.47128678216955	84843.0173451772\\
3.47138678466962	84809.7857930596\\
3.47148678716968	84775.9812831469\\
3.47158678966974	84742.1767732342\\
3.4716867921698	84708.3722633214\\
3.47178679466987	84674.5677534087\\
3.47188679716993	84640.763243496\\
3.47198679966999	84607.5316913784\\
3.47208680217005	84573.7271814657\\
3.47218680467012	84539.922671553\\
3.47228680717018	84506.1181616402\\
3.47238680967024	84472.8866095227\\
3.4724868121703	84439.08209961\\
3.47258681467037	84405.2775896972\\
3.47268681717043	84371.4730797845\\
3.47278681967049	84338.2415276669\\
3.47288682217055	84304.4370177542\\
3.47298682467062	84270.6325078415\\
3.47308682717068	84237.4009557239\\
3.47318682967074	84203.5964458112\\
3.4732868321708	84169.7919358985\\
3.47338683467087	84136.5603837809\\
3.47348683717093	84102.7558738681\\
3.47358683967099	84069.5243217506\\
3.47368684217105	84035.7198118378\\
3.47378684467112	84001.9153019251\\
3.47388684717118	83968.6837498075\\
3.47398684967124	83934.8792398948\\
3.4740868521713	83901.6476877772\\
3.47418685467137	83867.8431778645\\
3.47428685717143	83834.6116257469\\
3.47438685967149	83800.8071158342\\
3.47448686217155	83767.5755637166\\
3.47458686467162	83733.7710538039\\
3.47468686717168	83700.5395016863\\
3.47478686967174	83666.7349917736\\
3.4748868721718	83633.503439656\\
3.47498687467187	83600.2718875384\\
3.47508687717193	83566.4673776257\\
3.47518687967199	83533.2358255081\\
3.47528688217205	83499.4313155954\\
3.47538688467212	83466.1997634778\\
3.47548688717218	83432.9682113602\\
3.47558688967224	83399.1637014475\\
3.4756868921723	83365.9321493299\\
3.47578689467237	83332.7005972123\\
3.47588689717243	83298.8960872996\\
3.47598689967249	83265.664535182\\
3.47608690217255	83232.4329830644\\
3.47618690467262	83198.6284731517\\
3.47628690717268	83165.3969210341\\
3.47638690967274	83132.1653689165\\
3.4764869121728	83098.9338167989\\
3.47658691467287	83065.7022646814\\
3.47668691717293	83031.8977547687\\
3.47678691967299	82998.666202651\\
3.47688692217305	82965.4346505335\\
3.47698692467312	82932.2030984159\\
3.47708692717318	82898.9715462983\\
3.47718692967324	82865.1670363856\\
3.4772869321733	82831.935484268\\
3.47738693467337	82798.7039321504\\
3.47748693717343	82765.4723800328\\
3.47758693967349	82732.2408279152\\
3.47768694217355	82699.0092757976\\
3.47778694467362	82665.77772368\\
3.47788694717368	82632.5461715625\\
3.47798694967374	82599.3146194449\\
3.4780869521738	82566.0830673273\\
3.47818695467387	82532.8515152097\\
3.47828695717393	82499.6199630921\\
3.47838695967399	82466.3884109745\\
3.47848696217405	82433.1568588569\\
3.47858696467412	82399.9253067394\\
3.47868696717418	82366.6937546218\\
3.47878696967424	82333.4622025042\\
3.4788869721743	82300.2306503866\\
3.47898697467437	82266.999098269\\
3.47908697717443	82233.7675461514\\
3.47918697967449	82200.5359940338\\
3.47928698217455	82167.3044419162\\
3.47938698467462	82134.0728897986\\
3.47948698717468	82101.4142954762\\
3.47958698967474	82068.1827433586\\
3.4796869921748	82034.951191241\\
3.47978699467487	82001.7196391234\\
3.47988699717493	81968.4880870058\\
3.47998699967499	81935.2565348882\\
3.48008700217505	81902.5979405658\\
3.48018700467512	81869.3663884482\\
3.48028700717518	81836.1348363306\\
3.48038700967524	81802.903284213\\
3.4804870121753	81770.2446898906\\
3.48058701467537	81737.013137773\\
3.48068701717543	81703.7815856554\\
3.48078701967549	81670.5500335378\\
3.48088702217555	81637.8914392153\\
3.48098702467562	81604.6598870978\\
3.48108702717568	81571.4283349802\\
3.48118702967574	81538.7697406577\\
3.4812870321758	81505.5381885401\\
3.48138703467587	81472.8795942177\\
3.48148703717593	81439.6480421001\\
3.48158703967599	81406.4164899825\\
3.48168704217605	81373.75789566\\
3.48178704467612	81340.5263435425\\
3.48188704717618	81307.86774922\\
3.48198704967624	81274.6361971024\\
3.4820870521763	81241.97760278\\
3.48218705467637	81208.7460506624\\
3.48228705717643	81175.5144985448\\
3.48238705967649	81142.8559042223\\
3.48248706217655	81109.6243521047\\
3.48258706467662	81076.9657577823\\
3.48268706717668	81044.3071634598\\
3.48278706967674	81011.0756113422\\
3.4828870721768	80978.4170170198\\
3.48298707467687	80945.1854649022\\
3.48308707717693	80912.5268705797\\
3.48318707967699	80879.2953184621\\
3.48328708217705	80846.6367241397\\
3.48338708467712	80813.9781298172\\
3.48348708717718	80780.7465776996\\
3.48358708967724	80748.0879833772\\
3.4836870921773	80715.4293890547\\
3.48378709467737	80682.1978369371\\
3.48388709717743	80649.5392426147\\
3.48398709967749	80616.8806482922\\
3.48408710217755	80583.6490961746\\
3.48418710467762	80550.9905018522\\
3.48428710717768	80518.3319075297\\
3.48438710967774	80485.6733132073\\
3.4844871121778	80452.4417610897\\
3.48458711467787	80419.7831667672\\
3.48468711717793	80387.1245724448\\
3.48478711967799	80354.4659781223\\
3.48488712217805	80321.8073837999\\
3.48498712467812	80288.5758316823\\
3.48508712717818	80255.9172373598\\
3.48518712967824	80223.2586430373\\
3.4852871321783	80190.6000487149\\
3.48538713467837	80157.9414543924\\
3.48548713717843	80125.28286007\\
3.48558713967849	80092.6242657475\\
3.48568714217855	80059.3927136299\\
3.48578714467862	80026.7341193075\\
3.48588714717868	79994.075524985\\
3.48598714967874	79961.4169306626\\
3.4860871521788	79928.7583363401\\
3.48618715467887	79896.0997420176\\
3.48628715717893	79863.4411476952\\
3.48638715967899	79830.7825533727\\
3.48648716217905	79798.1239590503\\
3.48658716467912	79765.4653647278\\
3.48668716717918	79732.8067704054\\
3.48678716967924	79700.1481760829\\
3.4868871721793	79667.4895817605\\
3.48698717467937	79634.830987438\\
3.48708717717943	79602.7453509107\\
3.48718717967949	79570.0867565882\\
3.48728718217955	79537.4281622658\\
3.48738718467962	79504.7695679433\\
3.48748718717968	79472.1109736208\\
3.48758718967974	79439.4523792984\\
3.4876871921798	79406.7937849759\\
3.48778719467987	79374.7081484486\\
3.48788719717993	79342.0495541261\\
3.48798719967999	79309.3909598037\\
3.48808720218005	79276.7323654812\\
3.48818720468012	79244.0737711588\\
3.48828720718018	79211.9881346314\\
3.48838720968024	79179.329540309\\
3.4884872121803	79146.6709459865\\
3.48858721468037	79114.0123516641\\
3.48868721718043	79081.9267151368\\
3.48878721968049	79049.2681208143\\
3.48888722218055	79016.6095264918\\
3.48898722468062	78984.5238899645\\
3.48908722718068	78951.8652956421\\
3.48918722968074	78919.2067013196\\
3.4892872321808	78887.1210647923\\
3.48938723468087	78854.4624704698\\
3.48948723718093	78821.8038761474\\
3.48958723968099	78789.71823962\\
3.48968724218105	78757.0596452976\\
3.48978724468112	78724.9740087702\\
3.48988724718118	78692.3154144478\\
3.48998724968124	78660.2297779205\\
3.4900872521813	78627.571183598\\
3.49018725468137	78594.9125892756\\
3.49028725718143	78562.8269527482\\
3.49038725968149	78530.1683584258\\
3.49048726218155	78498.0827218984\\
3.49058726468162	78465.424127576\\
3.49068726718168	78433.3384910486\\
3.49078726968174	78401.2528545213\\
3.4908872721818	78368.5942601989\\
3.49098727468187	78336.5086236715\\
3.49108727718193	78303.8500293491\\
3.49118727968199	78271.7643928218\\
3.49128728218205	78239.1057984993\\
3.49138728468212	78207.020161972\\
3.49148728718218	78174.9345254447\\
3.49158728968224	78142.2759311222\\
3.4916872921823	78110.1902945949\\
3.49178729468237	78078.1046580675\\
3.49188729718243	78045.4460637451\\
3.49198729968249	78013.3604272178\\
3.49208730218255	77981.2747906904\\
3.49218730468262	77949.1891541631\\
3.49228730718268	77916.5305598407\\
3.49238730968274	77884.4449233133\\
3.4924873121828	77852.359286786\\
3.49258731468287	77820.2736502587\\
3.49268731718293	77787.6150559362\\
3.49278731968299	77755.5294194089\\
3.49288732218305	77723.4437828816\\
3.49298732468312	77691.3581463542\\
3.49308732718318	77659.2725098269\\
3.49318732968324	77626.6139155045\\
3.4932873321833	77594.5282789771\\
3.49338733468337	77562.4426424498\\
3.49348733718343	77530.3570059225\\
3.49358733968349	77498.2713693951\\
3.49368734218355	77466.1857328678\\
3.49378734468362	77434.1000963405\\
3.49388734718368	77402.0144598132\\
3.49398734968374	77369.9288232858\\
3.4940873521838	77337.8431867585\\
3.49418735468387	77305.7575502312\\
3.49428735718393	77273.6719137039\\
3.49438735968399	77241.5862771765\\
3.49448736218405	77209.5006406492\\
3.49458736468412	77177.4150041219\\
3.49468736718418	77145.3293675946\\
3.49478736968424	77113.2437310672\\
3.4948873721843	77081.1580945399\\
3.49498737468437	77049.0724580126\\
3.49508737718443	77016.9868214853\\
3.49518737968449	76984.9011849579\\
3.49528738218455	76952.8155484306\\
3.49538738468462	76920.7299119033\\
3.49548738718468	76888.644275376\\
3.49558738968474	76857.1315966438\\
3.4956873921848	76825.0459601164\\
3.49578739468487	76792.9603235891\\
3.49588739718493	76760.8746870618\\
3.49598739968499	76728.7890505345\\
3.49608740218505	76696.7034140071\\
3.49618740468512	76665.1907352749\\
3.49628740718518	76633.1050987476\\
3.49638740968524	76601.0194622203\\
3.4964874121853	76568.933825693\\
3.49658741468537	76537.4211469608\\
3.49668741718543	76505.3355104334\\
3.49678741968549	76473.2498739061\\
3.49688742218555	76441.7371951739\\
3.49698742468562	76409.6515586466\\
3.49708742718568	76377.5659221192\\
3.49718742968574	76346.0532433871\\
3.4972874321858	76313.9676068597\\
3.49738743468587	76281.8819703324\\
3.49748743718593	76250.3692916002\\
3.49758743968599	76218.2836550729\\
3.49768744218605	76186.7709763407\\
3.49778744468612	76154.6853398134\\
3.49788744718618	76122.599703286\\
3.49798744968624	76091.0870245538\\
3.4980874521863	76059.0013880265\\
3.49818745468637	76027.4887092943\\
3.49828745718643	75995.403072767\\
3.49838745968649	75963.8903940348\\
3.49848746218655	75931.8047575075\\
3.49858746468662	75900.2920787753\\
3.49868746718668	75868.206442248\\
3.49878746968674	75836.6937635158\\
3.4988874721868	75804.6081269884\\
3.49898747468687	75773.0954482563\\
3.49908747718693	75741.582769524\\
3.49918747968699	75709.4971329967\\
3.49928748218705	75677.9844542645\\
3.49938748468712	75645.8988177372\\
3.49948748718718	75614.386139005\\
3.49958748968724	75582.8734602728\\
3.4996874921873	75550.7878237455\\
3.49978749468737	75519.2751450133\\
3.49988749718743	75487.7624662811\\
3.49998749968749	75455.6768297538\\
3.50008750218755	75424.1641510216\\
3.50018750468762	75392.6514722894\\
3.50028750718768	75361.1387935572\\
3.50038750968774	75329.0531570299\\
3.5004875121878	75297.5404782977\\
3.50058751468787	75266.0277995655\\
3.50068751718793	75234.5151208333\\
3.50078751968799	75202.4294843059\\
3.50088752218805	75170.9168055737\\
3.50098752468812	75139.4041268416\\
3.50108752718818	75107.8914481094\\
3.50118752968824	75076.3787693772\\
3.5012875321883	75044.866090645\\
3.50138753468837	75012.7804541176\\
3.50148753718843	74981.2677753855\\
3.50158753968849	74949.7550966532\\
3.50168754218855	74918.2424179211\\
3.50178754468862	74886.7297391889\\
3.50188754718868	74855.2170604567\\
3.50198754968874	74823.7043817245\\
3.5020875521888	74792.1917029923\\
3.50218755468887	74760.6790242601\\
3.50228755718893	74729.1663455279\\
3.50238755968899	74697.6536667957\\
3.50248756218905	74666.1409880635\\
3.50258756468912	74634.6283093313\\
3.50268756718918	74603.1156305991\\
3.50278756968924	74571.6029518669\\
3.5028875721893	74540.0902731347\\
3.50298757468937	74508.5775944025\\
3.50308757718943	74477.0649156703\\
3.50318757968949	74445.5522369381\\
3.50328758218955	74414.0395582059\\
3.50338758468962	74383.0998372689\\
3.50348758718968	74351.5871585367\\
3.50358758968974	74320.0744798045\\
3.5036875921898	74288.5618010723\\
3.50378759468987	74257.0491223401\\
3.50388759718993	74225.5364436079\\
3.50398759968999	74194.0237648757\\
3.50408760219005	74163.0840439386\\
3.50418760469012	74131.5713652064\\
3.50428760719018	74100.0586864742\\
3.50438760969024	74068.546007742\\
3.5044876121903	74037.606286805\\
3.50458761469037	74006.0936080728\\
3.50468761719043	73974.5809293406\\
3.50478761969049	73943.6412084035\\
3.50488762219055	73912.1285296713\\
3.50498762469062	73880.6158509391\\
3.50508762719068	73849.1031722069\\
3.50518762969074	73818.1634512699\\
3.5052876321908	73786.6507725377\\
3.50538763469087	73755.7110516006\\
3.50548763719093	73724.1983728684\\
3.50558763969099	73692.6856941362\\
3.50568764219105	73661.7459731992\\
3.50578764469112	73630.233294467\\
3.50588764719118	73599.2935735299\\
3.50598764969124	73567.7808947977\\
3.5060876521913	73536.2682160655\\
3.50618765469137	73505.3284951284\\
3.50628765719143	73473.8158163963\\
3.50638765969149	73442.8760954592\\
3.50648766219155	73411.363416727\\
3.50658766469162	73380.4236957899\\
3.50668766719168	73349.4839748529\\
3.50678766969174	73317.9712961207\\
3.5068876721918	73287.0315751836\\
3.50698767469187	73255.5188964514\\
3.50708767719193	73224.5791755143\\
3.50718767969199	73193.0664967822\\
3.50728768219205	73162.1267758451\\
3.50738768469212	73131.187054908\\
3.50748768719218	73099.6743761758\\
3.50758768969224	73068.7346552387\\
3.50768769219231	73037.7949343017\\
3.50778769469237	73006.2822555695\\
3.50788769719243	72975.3425346324\\
3.50798769969249	72944.4028136954\\
3.50808770219255	72912.8901349632\\
3.50818770469262	72881.9504140261\\
3.50828770719268	72851.010693089\\
3.50838770969274	72819.4980143569\\
3.5084877121928	72788.5582934198\\
3.50858771469287	72757.6185724827\\
3.50868771719293	72726.6788515456\\
3.50878771969299	72695.7391306086\\
3.50888772219305	72664.2264518764\\
3.50898772469312	72633.2867309393\\
3.50908772719318	72602.3470100023\\
3.50918772969324	72571.4072890652\\
3.5092877321933	72540.4675681281\\
3.50938773469337	72509.5278471911\\
3.50948773719343	72478.0151684589\\
3.50958773969349	72447.0754475218\\
3.50968774219356	72416.1357265848\\
3.50978774469362	72385.1960056477\\
3.50988774719368	72354.2562847106\\
3.50998774969374	72323.3165637736\\
3.5100877521938	72292.3768428365\\
3.51018775469387	72261.4371218994\\
3.51028775719393	72230.4974009624\\
3.51038775969399	72199.5576800253\\
3.51048776219405	72168.6179590882\\
3.51058776469412	72137.6782381512\\
3.51068776719418	72106.7385172141\\
3.51078776969424	72075.798796277\\
3.5108877721943	72044.85907534\\
3.51098777469437	72013.9193544029\\
3.51108777719443	71982.9796334658\\
3.51118777969449	71952.0399125288\\
3.51128778219455	71921.6731493868\\
3.51138778469462	71890.7334284498\\
3.51148778719468	71859.7937075127\\
3.51158778969474	71828.8539865757\\
3.51168779219481	71797.9142656386\\
3.51178779469487	71766.9745447015\\
3.51188779719493	71736.0348237645\\
3.51198779969499	71705.6680606225\\
3.51208780219506	71674.7283396855\\
3.51218780469512	71643.7886187484\\
3.51228780719518	71612.8488978113\\
3.51238780969524	71582.4821346694\\
3.5124878121953	71551.5424137323\\
3.51258781469537	71520.6026927953\\
3.51268781719543	71489.6629718582\\
3.51278781969549	71459.2962087163\\
3.51288782219555	71428.3564877792\\
3.51298782469562	71397.4167668421\\
3.51308782719568	71367.0500037002\\
3.51318782969574	71336.1102827631\\
3.5132878321958	71305.1705618261\\
3.51338783469587	71274.8037986842\\
3.51348783719593	71243.8640777471\\
3.51358783969599	71212.92435681\\
3.51368784219606	71182.5575936681\\
3.51378784469612	71151.617872731\\
3.51388784719618	71121.2511095891\\
3.51398784969624	71090.311388652\\
3.51408785219631	71059.9446255101\\
3.51418785469637	71029.004904573\\
3.51428785719643	70998.6381414311\\
3.51438785969649	70967.698420494\\
3.51448786219655	70937.3316573521\\
3.51458786469662	70906.391936415\\
3.51468786719668	70876.0251732731\\
3.51478786969674	70845.085452336\\
3.5148878721968	70814.7186891941\\
3.51498787469687	70783.778968257\\
3.51508787719693	70753.4122051151\\
3.51518787969699	70722.472484178\\
3.51528788219705	70692.1057210361\\
3.51538788469712	70661.7389578942\\
3.51548788719718	70630.7992369571\\
3.51558788969724	70600.4324738152\\
3.51568789219731	70570.0657106732\\
3.51578789469737	70539.1259897362\\
3.51588789719743	70508.7592265942\\
3.51598789969749	70478.3924634523\\
3.51608790219756	70447.4527425152\\
3.51618790469762	70417.0859793733\\
3.51628790719768	70386.7192162314\\
3.51638790969774	70355.7794952943\\
3.5164879121978	70325.4127321524\\
3.51658791469787	70295.0459690104\\
3.51668791719793	70264.6792058685\\
3.51678791969799	70234.3124427266\\
3.51688792219805	70203.3727217895\\
3.51698792469812	70173.0059586476\\
3.51708792719818	70142.6391955056\\
3.51718792969824	70112.2724323637\\
3.5172879321983	70081.9056692218\\
3.51738793469837	70051.5389060798\\
3.51748793719843	70020.5991851428\\
3.51758793969849	69990.2324220008\\
3.51768794219856	69959.8656588589\\
3.51778794469862	69929.498895717\\
3.51788794719868	69899.132132575\\
3.51798794969874	69868.7653694331\\
3.51808795219881	69838.3986062912\\
3.51818795469887	69808.0318431492\\
3.51828795719893	69777.6650800073\\
3.51838795969899	69747.2983168654\\
3.51848796219906	69716.9315537234\\
3.51858796469912	69686.5647905815\\
3.51868796719918	69656.1980274396\\
3.51878796969924	69625.8312642976\\
3.5188879721993	69595.4645011557\\
3.51898797469937	69565.0977380138\\
3.51908797719943	69534.7309748718\\
3.51918797969949	69504.3642117299\\
3.51928798219955	69473.997448588\\
3.51938798469962	69444.2036432412\\
3.51948798719968	69413.8368800992\\
3.51958798969974	69383.4701169573\\
3.51968799219981	69353.1033538154\\
3.51978799469987	69322.7365906734\\
3.51988799719993	69292.3698275315\\
3.51998799969999	69262.5760221847\\
3.52008800220006	69232.2092590428\\
3.52018800470012	69201.8424959008\\
3.52028800720018	69171.4757327589\\
3.52038800970024	69141.108969617\\
3.52048801220031	69111.3151642702\\
3.52058801470037	69080.9484011282\\
3.52068801720043	69050.5816379863\\
3.52078801970049	69020.7878326395\\
3.52088802220055	68990.4210694975\\
3.52098802470062	68960.0543063556\\
3.52108802720068	68929.6875432137\\
3.52118802970074	68899.8937378669\\
3.5212880322008	68869.526974725\\
3.52138803470087	68839.7331693781\\
3.52148803720093	68809.3664062362\\
3.52158803970099	68778.9996430943\\
3.52168804220106	68749.2058377475\\
3.52178804470112	68718.8390746055\\
3.52188804720118	68689.0452692587\\
3.52198804970124	68658.6785061168\\
3.52208805220131	68628.3117429749\\
3.52218805470137	68598.5179376281\\
3.52228805720143	68568.1511744861\\
3.52238805970149	68538.3573691393\\
3.52248806220156	68507.9906059974\\
3.52258806470162	68478.1968006506\\
3.52268806720168	68447.8300375087\\
3.52278806970174	68418.0362321619\\
3.5228880722018	68388.242426815\\
3.52298807470187	68357.8756636731\\
3.52308807720193	68328.0818583263\\
3.52318807970199	68297.7150951844\\
3.52328808220205	68267.9212898376\\
3.52338808470212	68237.5545266957\\
3.52348808720218	68207.7607213489\\
3.52358808970224	68177.9669160021\\
3.52368809220231	68147.6001528601\\
3.52378809470237	68117.8063475133\\
3.52388809720243	68088.0125421665\\
3.52398809970249	68057.6457790246\\
3.52408810220256	68027.8519736778\\
3.52418810470262	67998.058168331\\
3.52428810720268	67968.2643629842\\
3.52438810970274	67937.8975998422\\
3.52448811220281	67908.1037944954\\
3.52458811470287	67878.3099891486\\
3.52468811720293	67848.5161838018\\
3.52478811970299	67818.1494206599\\
3.52488812220306	67788.3556153131\\
3.52498812470312	67758.5618099663\\
3.52508812720318	67728.7680046195\\
3.52518812970324	67698.9741992727\\
3.5252881322033	67669.1803939259\\
3.52538813470337	67638.8136307839\\
3.52548813720343	67609.0198254371\\
3.52558813970349	67579.2260200903\\
3.52568814220356	67549.4322147435\\
3.52578814470362	67519.6384093967\\
3.52588814720368	67489.8446040499\\
3.52598814970374	67460.0507987031\\
3.52608815220381	67430.2569933563\\
3.52618815470387	67400.4631880095\\
3.52628815720393	67370.6693826627\\
3.52638815970399	67340.8755773159\\
3.52648816220406	67311.0817719691\\
3.52658816470412	67281.2879666223\\
3.52668816720418	67251.4941612755\\
3.52678816970424	67221.7003559287\\
3.52688817220431	67191.9065505819\\
3.52698817470437	67162.1127452351\\
3.52708817720443	67132.3189398883\\
3.52718817970449	67102.5251345415\\
3.52728818220455	67072.7313291947\\
3.52738818470462	67042.9375238479\\
3.52748818720468	67013.1437185011\\
3.52758818970474	66983.9228709494\\
3.52768819220481	66954.1290656026\\
3.52778819470487	66924.3352602558\\
3.52788819720493	66894.541454909\\
3.52798819970499	66864.7476495622\\
3.52808820220506	66834.9538442154\\
3.52818820470512	66805.7329966637\\
3.52828820720518	66775.9391913169\\
3.52838820970524	66746.1453859701\\
3.52848821220531	66716.3515806233\\
3.52858821470537	66687.1307330717\\
3.52868821720543	66657.3369277248\\
3.52878821970549	66627.543122378\\
3.52888822220556	66598.3222748264\\
3.52898822470562	66568.5284694796\\
3.52908822720568	66538.7346641328\\
3.52918822970574	66508.940858786\\
3.52928823220581	66479.7200112343\\
3.52938823470587	66449.9262058875\\
3.52948823720593	66420.7053583358\\
3.52958823970599	66390.911552989\\
3.52968824220606	66361.1177476422\\
3.52978824470612	66331.8969000905\\
3.52988824720618	66302.1030947437\\
3.52998824970624	66272.8822471921\\
3.53008825220631	66243.0884418453\\
3.53018825470637	66213.8675942936\\
3.53028825720643	66184.0737889468\\
3.53038825970649	66154.8529413951\\
3.53048826220656	66125.0591360483\\
3.53058826470662	66095.8382884966\\
3.53068826720668	66066.0444831498\\
3.53078826970674	66036.8236355982\\
3.53088827220681	66007.0298302514\\
3.53098827470687	65977.8089826997\\
3.53108827720693	65948.0151773529\\
3.53118827970699	65918.7943298012\\
3.53128828220706	65889.0005244544\\
3.53138828470712	65859.7796769027\\
3.53148828720718	65830.5588293511\\
3.53158828970724	65800.7650240043\\
3.53168829220731	65771.5441764526\\
3.53178829470737	65742.3233289009\\
3.53188829720743	65712.5295235541\\
3.53198829970749	65683.3086760024\\
3.53208830220756	65654.0878284508\\
3.53218830470762	65624.294023104\\
3.53228830720768	65595.0731755523\\
3.53238830970774	65565.8523280006\\
3.53248831220781	65536.631480449\\
3.53258831470787	65506.8376751021\\
3.53268831720793	65477.6168275505\\
3.53278831970799	65448.3959799988\\
3.53288832220806	65419.1751324471\\
3.53298832470812	65389.9542848955\\
3.53308832720818	65360.1604795487\\
3.53318832970824	65330.939631997\\
3.53328833220831	65301.7187844453\\
3.53338833470837	65272.4979368936\\
3.53348833720843	65243.277089342\\
3.53358833970849	65214.0562417903\\
3.53368834220856	65184.8353942386\\
3.53378834470862	65155.0415888918\\
3.53388834720868	65125.8207413402\\
3.53398834970874	65096.5998937885\\
3.53408835220881	65067.3790462368\\
3.53418835470887	65038.1581986851\\
3.53428835720893	65008.9373511335\\
3.53438835970899	64979.7165035818\\
3.53448836220906	64950.4956560301\\
3.53458836470912	64921.2748084784\\
3.53468836720918	64892.0539609268\\
3.53478836970924	64862.8331133751\\
3.53488837220931	64833.6122658234\\
3.53498837470937	64804.3914182718\\
3.53508837720943	64775.1705707201\\
3.53518837970949	64746.5226809635\\
3.53528838220956	64717.3018334119\\
3.53538838470962	64688.0809858602\\
3.53548838720968	64658.8601383085\\
3.53558838970974	64629.6392907569\\
3.53568839220981	64600.4184432052\\
3.53578839470987	64571.1975956535\\
3.53588839720993	64541.9767481018\\
3.53598839970999	64513.3288583453\\
3.53608840221006	64484.1080107936\\
3.53618840471012	64454.887163242\\
3.53628840721018	64425.6663156903\\
3.53638840971024	64397.0184259337\\
3.53648841221031	64367.7975783821\\
3.53658841471037	64338.5767308304\\
3.53668841721043	64309.3558832787\\
3.53678841971049	64280.7079935222\\
3.53688842221056	64251.4871459705\\
3.53698842471062	64222.2662984189\\
3.53708842721068	64193.6184086623\\
3.53718842971074	64164.3975611106\\
3.53728843221081	64135.176713559\\
3.53738843471087	64106.5288238024\\
3.53748843721093	64077.3079762507\\
3.53758843971099	64048.0871286991\\
3.53768844221106	64019.4392389425\\
3.53778844471112	63990.2183913909\\
3.53788844721118	63961.5705016343\\
3.53798844971124	63932.3496540826\\
3.53808845221131	63903.128806531\\
3.53818845471137	63874.4809167744\\
3.53828845721143	63845.2600692228\\
3.53838845971149	63816.6121794662\\
3.53848846221156	63787.3913319146\\
3.53858846471162	63758.743442158\\
3.53868846721168	63729.5225946063\\
3.53878846971174	63700.8747048498\\
3.53888847221181	63671.6538572981\\
3.53898847471187	63643.0059675416\\
3.53908847721193	63614.358077785\\
3.53918847971199	63585.1372302334\\
3.53928848221206	63556.4893404768\\
3.53938848471212	63527.2684929252\\
3.53948848721218	63498.6206031686\\
3.53958848971224	63469.9727134121\\
3.53968849221231	63440.7518658604\\
3.53978849471237	63412.1039761039\\
3.53988849721243	63383.4560863473\\
3.53998849971249	63354.2352387956\\
3.54008850221256	63325.5873490391\\
3.54018850471262	63296.9394592826\\
3.54028850721268	63267.7186117309\\
3.54038850971274	63239.0707219743\\
3.54048851221281	63210.4228322178\\
3.54058851471287	63181.7749424613\\
3.54068851721293	63152.5540949096\\
3.54078851971299	63123.9062051531\\
3.54088852221306	63095.2583153965\\
3.54098852471312	63066.61042564\\
3.54108852721318	63037.9625358834\\
3.54118852971324	63008.7416883318\\
3.54128853221331	62980.0937985752\\
3.54138853471337	62951.4459088187\\
3.54148853721343	62922.7980190621\\
3.54158853971349	62894.1501293056\\
3.54168854221356	62865.5022395491\\
3.54178854471362	62836.8543497925\\
3.54188854721368	62807.6335022408\\
3.54198854971374	62778.9856124843\\
3.54208855221381	62750.3377227278\\
3.54218855471387	62721.6898329712\\
3.54228855721393	62693.0419432147\\
3.54238855971399	62664.3940534581\\
3.54248856221406	62635.7461637016\\
3.54258856471412	62607.0982739451\\
3.54268856721418	62578.4503841885\\
3.54278856971424	62549.802494432\\
3.54288857221431	62521.1546046754\\
3.54298857471437	62492.5067149189\\
3.54308857721443	62463.8588251624\\
3.54318857971449	62435.7838932009\\
3.54328858221456	62407.1360034444\\
3.54338858471462	62378.4881136879\\
3.54348858721468	62349.8402239313\\
3.54358858971474	62321.1923341748\\
3.54368859221481	62292.5444444182\\
3.54378859471487	62263.8965546617\\
3.54388859721493	62235.2486649052\\
3.54398859971499	62207.1737329437\\
3.54408860221506	62178.5258431872\\
3.54418860471512	62149.8779534307\\
3.54428860721518	62121.2300636741\\
3.54438860971524	62092.5821739176\\
3.54448861221531	62064.5072419562\\
3.54458861471537	62035.8593521996\\
3.54468861721543	62007.2114624431\\
3.54478861971549	61978.5635726865\\
3.54488862221556	61950.4886407251\\
3.54498862471562	61921.8407509686\\
3.54508862721568	61893.1928612121\\
3.54518862971574	61865.1179292506\\
3.54528863221581	61836.4700394941\\
3.54538863471587	61807.8221497376\\
3.54548863721593	61779.7472177761\\
3.54558863971599	61751.0993280196\\
3.54568864221606	61723.0243960582\\
3.54578864471612	61694.3765063017\\
3.54588864721618	61665.7286165451\\
3.54598864971624	61637.6536845837\\
3.54608865221631	61609.0057948272\\
3.54618865471637	61580.9308628657\\
3.54628865721643	61552.2829731092\\
3.54638865971649	61524.2080411478\\
3.54648866221656	61495.5601513913\\
3.54658866471662	61467.4852194298\\
3.54668866721668	61438.8373296733\\
3.54678866971674	61410.7623977119\\
3.54688867221681	61382.1145079553\\
3.54698867471687	61354.0395759939\\
3.54708867721693	61325.3916862374\\
3.54718867971699	61297.316754276\\
3.54728868221706	61268.6688645194\\
3.54738868471712	61240.593932558\\
3.54748868721718	61212.5190005966\\
3.54758868971724	61183.8711108401\\
3.54768869221731	61155.7961788787\\
3.54778869471737	61127.7212469173\\
3.54788869721743	61099.0733571607\\
3.54798869971749	61070.9984251993\\
3.54808870221756	61042.9234932379\\
3.54818870471762	61014.2756034814\\
3.54828870721768	60986.20067152\\
3.54838870971774	60958.1257395585\\
3.54848871221781	60929.477849802\\
3.54858871471787	60901.4029178406\\
3.54868871721793	60873.3279858792\\
3.54878871971799	60845.2530539178\\
3.54888872221806	60816.6051641612\\
3.54898872471812	60788.5302321998\\
3.54908872721818	60760.4553002384\\
3.54918872971824	60732.380368277\\
3.54928873221831	60704.3054363156\\
3.54938873471837	60676.2305043542\\
3.54948873721843	60647.5826145976\\
3.54958873971849	60619.5076826362\\
3.54968874221856	60591.4327506748\\
3.54978874471862	60563.3578187134\\
3.54988874721868	60535.282886752\\
3.54998874971874	60507.2079547906\\
3.55008875221881	60479.1330228292\\
3.55018875471887	60451.0580908678\\
3.55028875721893	60422.9831589064\\
3.55038875971899	60394.9082269449\\
3.55048876221906	60366.8332949835\\
3.55058876471912	60338.7583630221\\
3.55068876721918	60310.6834310607\\
3.55078876971924	60282.6084990993\\
3.55088877221931	60254.5335671379\\
3.55098877471937	60226.4586351765\\
3.55108877721943	60198.3837032151\\
3.55118877971949	60170.3087712537\\
3.55128878221956	60142.2338392923\\
3.55138878471962	60114.1589073308\\
3.55148878721968	60086.0839753694\\
3.55158878971974	60058.009043408\\
3.55168879221981	60029.9341114466\\
3.55178879471987	60002.4321372803\\
3.55188879721993	59974.3572053189\\
3.55198879971999	59946.2822733575\\
3.55208880222006	59918.2073413961\\
3.55218880472012	59890.1324094347\\
3.55228880722018	59862.0574774733\\
3.55238880972024	59834.555503307\\
3.55248881222031	59806.4805713456\\
3.55258881472037	59778.4056393842\\
3.55268881722043	59750.3307074228\\
3.55278881972049	59722.8287332565\\
3.55288882222056	59694.7538012951\\
3.55298882472062	59666.6788693337\\
3.55308882722068	59639.1768951674\\
3.55318882972074	59611.101963206\\
3.55328883222081	59583.0270312446\\
3.55338883472087	59555.5250570783\\
3.55348883722093	59527.4501251169\\
3.55358883972099	59499.3751931555\\
3.55368884222106	59471.8732189892\\
3.55378884472112	59443.7982870278\\
3.55388884722118	59415.7233550664\\
3.55398884972124	59388.2213809001\\
3.55408885222131	59360.1464489387\\
3.55418885472137	59332.6444747724\\
3.55428885722143	59304.569542811\\
3.55438885972149	59277.0675686447\\
3.55448886222156	59248.9926366833\\
3.55458886472162	59221.490662517\\
3.55468886722168	59193.4157305556\\
3.55478886972174	59165.9137563893\\
3.55488887222181	59137.8388244279\\
3.55498887472187	59110.3368502616\\
3.55508887722193	59082.2619183002\\
3.55518887972199	59054.7599441339\\
3.55528888222206	59026.6850121725\\
3.55538888472212	58999.1830380063\\
3.55548888722218	58971.68106384\\
3.55558888972224	58943.6061318786\\
3.55568889222231	58916.1041577123\\
3.55578889472237	58888.0292257509\\
3.55588889722243	58860.5272515846\\
3.55598889972249	58833.0252774183\\
3.55608890222256	58804.9503454569\\
3.55618890472262	58777.4483712906\\
3.55628890722268	58749.9463971244\\
3.55638890972274	58722.4444229581\\
3.55648891222281	58694.3694909967\\
3.55658891472287	58666.8675168304\\
3.55668891722293	58639.3655426641\\
3.55678891972299	58611.8635684978\\
3.55688892222306	58583.7886365364\\
3.55698892472312	58556.2866623701\\
3.55708892722318	58528.7846882039\\
3.55718892972324	58501.2827140376\\
3.55728893222331	58473.7807398713\\
3.55738893472337	58445.7058079099\\
3.55748893722343	58418.2038337436\\
3.55758893972349	58390.7018595773\\
3.55768894222356	58363.199885411\\
3.55778894472362	58335.6979112448\\
3.55788894722368	58308.1959370785\\
3.55798894972374	58280.6939629122\\
3.55808895222381	58253.1919887459\\
3.55818895472387	58225.1170567845\\
3.55828895722393	58197.6150826182\\
3.55838895972399	58170.113108452\\
3.55848896222406	58142.6111342857\\
3.55858896472412	58115.1091601194\\
3.55868896722418	58087.6071859531\\
3.55878896972424	58060.1052117868\\
3.55888897222431	58032.6032376206\\
3.55898897472437	58005.1012634543\\
3.55908897722443	57977.599289288\\
3.55918897972449	57950.0973151217\\
3.55928898222456	57923.1682987506\\
3.55938898472462	57895.6663245843\\
3.55948898722468	57868.164350418\\
3.55958898972474	57840.6623762517\\
3.55968899222481	57813.1604020855\\
3.55978899472487	57785.6584279192\\
3.55988899722493	57758.1564537529\\
3.55998899972499	57730.6544795866\\
3.56008900222506	57703.7254632155\\
3.56018900472512	57676.2234890492\\
3.56028900722518	57648.7215148829\\
3.56038900972524	57621.2195407166\\
3.56048901222531	57593.7175665504\\
3.56058901472537	57566.7885501792\\
3.56068901722543	57539.2865760129\\
3.56078901972549	57511.7846018466\\
3.56088902222556	57484.2826276804\\
3.56098902472562	57457.3536113092\\
3.56108902722568	57429.8516371429\\
3.56118902972574	57402.3496629767\\
3.56128903222581	57375.4206466055\\
3.56138903472587	57347.9186724392\\
3.56148903722593	57320.4166982729\\
3.56158903972599	57293.2584987838\\
3.56168904222606	57265.9857077355\\
3.56178904472612	57238.6556209078\\
3.56188904722618	57211.3828298596\\
3.56198904972624	57184.1100388113\\
3.56208905222631	57156.8372477631\\
3.56218905472637	57129.6217524944\\
3.56228905722643	57102.3489614462\\
3.56238905972649	57075.1334661774\\
3.56248906222656	57047.8606751292\\
3.56258906472662	57020.6451798605\\
3.56268906722668	56993.4296845918\\
3.56278906972674	56966.2141893231\\
3.56288907222681	56938.9986940544\\
3.56298907472687	56911.7831987856\\
3.56308907722693	56884.6249992964\\
3.56318907972699	56857.4095040277\\
3.56328908222706	56830.2513045385\\
3.56338908472712	56803.0931050493\\
3.56348908722718	56775.9349055601\\
3.56358908972724	56748.7767060709\\
3.56368909222731	56721.6185065817\\
3.56378909472737	56694.4603070925\\
3.56388909722743	56667.3021076033\\
3.56398909972749	56640.2012038936\\
3.56408910222756	56613.0430044044\\
3.56418910472762	56585.9421006948\\
3.56428910722768	56558.8411969851\\
3.56438910972774	56531.7402932754\\
3.56448911222781	56504.6393895657\\
3.56458911472787	56477.538485856\\
3.56468911722793	56450.4948779258\\
3.56478911972799	56423.3939742161\\
3.56488912222806	56396.350366286\\
3.56498912472812	56369.3067583558\\
3.56508912722818	56342.2058546461\\
3.56518912972824	56315.1622467159\\
3.56528913222831	56288.1759345653\\
3.56538913472837	56261.1323266351\\
3.56548913722843	56234.0887187049\\
3.56558913972849	56207.1024065543\\
3.56568914222856	56180.0587986241\\
3.56578914472862	56153.0724864734\\
3.56588914722868	56126.0861743228\\
3.56598914972874	56099.0998621721\\
3.56608915222881	56072.1135500214\\
3.56618915472887	56045.1272378708\\
3.56628915722893	56018.1409257201\\
3.56638915972899	55991.2119093489\\
3.56648916222906	55964.2255971983\\
3.56658916472912	55937.2965808271\\
3.56668916722918	55910.367564456\\
3.56678916972924	55883.4385480848\\
3.56688917222931	55856.5095317137\\
3.56698917472937	55829.5805153425\\
3.56708917722943	55802.6514989714\\
3.56718917972949	55775.7797783798\\
3.56728918222956	55748.8507620086\\
3.56738918472962	55721.979041417\\
3.56748918722968	55695.1073208253\\
3.56758918972974	55668.2356002337\\
3.56768919222981	55641.3638796421\\
3.56778919472987	55614.4921590504\\
3.56788919722993	55587.6204384588\\
3.56798919972999	55560.8060136467\\
3.56808920223006	55533.934293055\\
3.56818920473012	55507.1198682429\\
3.56828920723018	55480.3054434308\\
3.56838920973024	55453.4910186187\\
3.56848921223031	55426.6765938066\\
3.56858921473037	55399.8621689944\\
3.56868921723043	55373.0477441823\\
3.56878921973049	55346.2906151497\\
3.56888922223056	55319.4761903376\\
3.56898922473062	55292.719061305\\
3.56908922723068	55265.9619322724\\
3.56918922973074	55239.1475074602\\
3.56928923223081	55212.4476742071\\
3.56938923473087	55185.6905451745\\
3.56948923723093	55158.9334161419\\
3.56958923973099	55132.1762871093\\
3.56968924223106	55105.4764538562\\
3.56978924473112	55078.7193248236\\
3.56988924723118	55052.0194915705\\
3.56998924973124	55025.3196583174\\
3.57008925223131	54998.6198250643\\
3.57018925473137	54971.9199918112\\
3.57028925723143	54945.2201585581\\
3.57038925973149	54918.5776210845\\
3.57048926223156	54891.8777878314\\
3.57058926473162	54865.2352503579\\
3.57068926723168	54838.5927128843\\
3.57078926973174	54811.9501754107\\
3.57088927223181	54785.3076379371\\
3.57098927473187	54758.6651004635\\
3.57108927723193	54732.0225629899\\
3.57118927973199	54705.3800255164\\
3.57128928223206	54678.7947838223\\
3.57138928473212	54652.1522463487\\
3.57148928723218	54625.5670046546\\
3.57158928973224	54598.9817629606\\
3.57168929223231	54572.3965212665\\
3.57178929473237	54545.8112795724\\
3.57188929723243	54519.2260378783\\
3.57198929973249	54492.6980919638\\
3.57208930223256	54466.1128502697\\
3.57218930473262	54439.5849043552\\
3.57228930723268	54412.9996626611\\
3.57238930973274	54386.4717167465\\
3.57248931223281	54359.943770832\\
3.57258931473287	54333.4158249174\\
3.57268931723293	54306.8878790029\\
3.57278931973299	54280.4172288678\\
3.57288932223306	54253.8892829533\\
3.57298932473312	54227.4186328182\\
3.57308932723318	54200.8906869037\\
3.57318932973324	54174.4200367686\\
3.57328933223331	54147.9493866336\\
3.57338933473337	54121.4787364985\\
3.57348933723343	54095.0080863635\\
3.57358933973349	54068.594732008\\
3.57368934223356	54042.1240818729\\
3.57378934473362	54015.7107275174\\
3.57388934723368	53989.2400773823\\
3.57398934973374	53962.8267230268\\
3.57408935223381	53936.4133686713\\
3.57418935473387	53910.0000143158\\
3.57428935723393	53883.5866599602\\
3.57438935973399	53857.2306013842\\
3.57448936223406	53830.8172470287\\
3.57458936473412	53804.4611884527\\
3.57468936723418	53778.0478340971\\
3.57478936973424	53751.6917755211\\
3.57488937223431	53725.3357169451\\
3.57498937473437	53698.9796583691\\
3.57508937723443	53672.623599793\\
3.57518937973449	53646.267541217\\
3.57528938223456	53619.9687784205\\
3.57538938473462	53593.6127198445\\
3.57548938723468	53567.313957048\\
3.57558938973474	53541.0151942515\\
3.57568939223481	53514.6591356755\\
3.57578939473487	53488.360372879\\
3.57588939723493	53462.118905862\\
3.57598939973499	53435.8201430655\\
3.57608940223506	53409.521380269\\
3.57618940473512	53383.279913252\\
3.57628940723518	53356.9811504555\\
3.57638940973524	53330.7396834385\\
3.57648941223531	53304.4982164215\\
3.57658941473537	53278.2567494045\\
3.57668941723543	53252.0152823875\\
3.57678941973549	53225.7738153705\\
3.57688942223556	53199.589644133\\
3.57698942473562	53173.3481771161\\
3.57708942723568	53147.1640058786\\
3.57718942973574	53120.9225388616\\
3.57728943223581	53094.7383676241\\
3.57738943473587	53068.5541963866\\
3.57748943723593	53042.3700251491\\
3.57758943973599	53016.2431496912\\
3.57768944223606	52990.0589784537\\
3.57778944473612	52963.8748072162\\
3.57788944723618	52937.7479317583\\
3.57798944973624	52911.6210563003\\
3.57808945223631	52885.4368850628\\
3.57818945473637	52859.3100096048\\
3.57828945723643	52833.1831341469\\
3.57838945973649	52807.1135544684\\
3.57848946223656	52780.9866790105\\
3.57858946473662	52754.8598035525\\
3.57868946723668	52728.790223874\\
3.57878946973674	52702.7206441956\\
3.57888947223681	52676.5937687376\\
3.57898947473687	52650.5241890592\\
3.57908947723693	52624.4546093807\\
3.57918947973699	52598.4423254818\\
3.57928948223706	52572.3727458033\\
3.57938948473712	52546.3031661249\\
3.57948948723718	52520.2908822259\\
3.57958948973724	52494.2213025475\\
3.57968949223731	52468.2090186486\\
3.57978949473737	52442.1967347496\\
3.57988949723743	52416.1844508507\\
3.57998949973749	52390.1721669517\\
3.58008950223756	52364.2171788323\\
3.58018950473762	52338.2048949334\\
3.58028950723768	52312.2499068139\\
3.58038950973774	52286.237622915\\
3.58048951223781	52260.2826347956\\
3.58058951473787	52234.3276466761\\
3.58068951723793	52208.3726585567\\
3.58078951973799	52182.4176704373\\
3.58088952223806	52156.4626823179\\
3.58098952473812	52130.564989978\\
3.58108952723818	52104.6100018585\\
3.58118952973824	52078.7123095186\\
3.58128953223831	52052.7573213992\\
3.58138953473837	52026.8596290593\\
3.58148953723843	52000.9619367194\\
3.58158953973849	51975.0642443794\\
3.58168954223856	51949.223847819\\
3.58178954473862	51923.3261554791\\
3.58188954723868	51897.4284631392\\
3.58198954973874	51871.5880665788\\
3.58208955223881	51845.7476700184\\
3.58218955473887	51819.907273458\\
3.58228955723893	51794.0668768976\\
3.58238955973899	51768.2264803372\\
3.58248956223906	51742.3860837768\\
3.58258956473912	51716.5456872164\\
3.58268956723918	51690.7625864355\\
3.58278956973924	51664.9221898751\\
3.58288957223931	51639.1390890942\\
3.58298957473937	51613.3559883133\\
3.58308957723943	51587.5728875325\\
3.58318957973949	51561.7897867516\\
3.58328958223956	51536.0066859707\\
3.58338958473962	51510.2235851898\\
3.58348958723968	51484.4977801884\\
3.58358958973974	51458.7146794076\\
3.58368959223981	51432.9888744062\\
3.58378959473987	51407.2630694048\\
3.58388959723993	51381.5372644034\\
3.58398959973999	51355.8114594021\\
3.58408960224006	51330.0856544007\\
3.58418960474012	51304.3598493993\\
3.58428960724018	51278.6913401774\\
3.58438960974024	51252.9655351761\\
3.58448961224031	51227.2970259542\\
3.58458961474037	51201.5712209528\\
3.58468961724043	51175.902711731\\
3.58478961974049	51150.2342025091\\
3.58488962224056	51124.6229890668\\
3.58498962474062	51098.9544798449\\
3.58508962724068	51073.2859706231\\
3.58518962974074	51047.6747571807\\
3.58528963224081	51022.0062479588\\
3.58538963474087	50996.3950345165\\
3.58548963724093	50970.7838210741\\
3.58558963974099	50945.1726076318\\
3.58568964224106	50919.5613941894\\
3.58578964474112	50893.9501807471\\
3.58588964724118	50868.3962630843\\
3.58598964974124	50842.7850496419\\
3.58608965224131	50817.2311319791\\
3.58618965474137	50791.6199185367\\
3.58628965724143	50766.0660008739\\
3.58638965974149	50740.5120832111\\
3.58648966224156	50714.9581655482\\
3.58658966474162	50689.4615436649\\
3.58668966724168	50663.9076260021\\
3.58678966974174	50638.3537083392\\
3.58688967224181	50612.8570864559\\
3.58698967474187	50587.3604645726\\
3.58708967724193	50561.8638426893\\
3.58718967974199	50536.3099250264\\
3.58728968224206	50510.8705989226\\
3.58738968474212	50485.3739770393\\
3.58748968724218	50459.877355156\\
3.58758968974224	50434.3807332727\\
3.58768969224231	50408.9414071689\\
3.58778969474237	50383.502081065\\
3.58788969724243	50358.0627549612\\
3.58798969974249	50332.5661330779\\
3.58808970224256	50307.1268069741\\
3.58818970474262	50281.7447766498\\
3.58828970724268	50256.305450546\\
3.58838970974274	50230.8661244422\\
3.58848971224281	50205.4840941179\\
3.58858971474287	50180.1020637936\\
3.58868971724293	50154.6627376898\\
3.58878971974299	50129.2807073655\\
3.58888972224306	50103.8986770412\\
3.58898972474312	50078.5166467169\\
3.58908972724318	50053.1919121721\\
3.58918972974324	50027.8098818478\\
3.58928973224331	50002.4851473031\\
3.58938973474337	49977.1031169788\\
3.58948973724343	49951.778382434\\
3.58958973974349	49926.4536478892\\
3.58968974224356	49901.1289133444\\
3.58978974474362	49875.8041787996\\
3.58988974724368	49850.4794442548\\
3.58998974974374	49825.2120054896\\
3.59008975224381	49799.8872709448\\
3.59018975474387	49774.6198321795\\
3.59028975724393	49749.2950976347\\
3.59038975974399	49724.0276588695\\
3.59048976224406	49698.7602201042\\
3.59058976474412	49673.4927813389\\
3.59068976724418	49648.2253425737\\
3.59078976974424	49623.0151995879\\
3.59088977224431	49597.7477608226\\
3.59098977474437	49572.5376178369\\
3.59108977724443	49547.3274748511\\
3.59118977974449	49522.0600360858\\
3.59128978224456	49496.8498931001\\
3.59138978474462	49471.6397501143\\
3.59148978724468	49446.4869029081\\
3.59158978974474	49421.2767599223\\
3.59168979224481	49396.0666169366\\
3.59178979474487	49370.9137697303\\
3.59188979724493	49345.7609225241\\
3.59198979974499	49320.5507795383\\
3.59208980224506	49295.3979323321\\
3.59218980474512	49270.2450851259\\
3.59228980724518	49245.0922379196\\
3.59238980974524	49219.9966864929\\
3.59248981224531	49194.8438392866\\
3.59258981474537	49169.7482878599\\
3.59268981724543	49144.5954406537\\
3.59278981974549	49119.4998892269\\
3.59288982224556	49094.4043378002\\
3.59298982474562	49069.3087863735\\
3.59308982724568	49044.2132349468\\
3.59318982974574	49019.11768352\\
3.59328983224581	48994.0794278728\\
3.59338983474587	48968.9838764461\\
3.59348983724593	48943.9456207989\\
3.59358983974599	48918.9073651516\\
3.59368984224606	48893.8691095044\\
3.59378984474612	48868.7735580777\\
3.59388984724618	48843.79259821\\
3.59398984974624	48818.7543425628\\
3.59408985224631	48793.7160869156\\
3.59418985474637	48768.7351270478\\
3.59428985724643	48743.6968714006\\
3.59438985974649	48718.7159115329\\
3.59448986224656	48693.7349516652\\
3.59458986474662	48668.7539917975\\
3.59468986724668	48643.7730319298\\
3.59478986974674	48618.7920720621\\
3.59488987224681	48593.8111121944\\
3.59498987474687	48568.8874481062\\
3.59508987724693	48543.9064882385\\
3.59518987974699	48518.9828241503\\
3.59528988224706	48494.0591600621\\
3.59538988474712	48469.1354959739\\
3.59548988724718	48444.2118318857\\
3.59558988974724	48419.2881677976\\
3.59568989224731	48394.3645037094\\
3.59578989474737	48369.4981354007\\
3.59588989724743	48344.5744713125\\
3.59598989974749	48319.7081030038\\
3.59608990224756	48294.8417346951\\
3.59618990474762	48269.9753663865\\
3.59628990724768	48245.1089980778\\
3.59638990974774	48220.2426297691\\
3.59648991224781	48195.3762614604\\
3.59658991474787	48170.5098931517\\
3.59668991724793	48145.7008206226\\
3.59678991974799	48120.8917480934\\
3.59688992224806	48096.0253797847\\
3.59698992474812	48071.2163072556\\
3.59708992724818	48046.4072347264\\
3.59718992974824	48021.5981621972\\
3.59728993224831	47996.8463854476\\
3.59738993474837	47972.0373129184\\
3.59748993724843	47947.2282403893\\
3.59758993974849	47922.4764636396\\
3.59768994224856	47897.72468689\\
3.59778994474862	47872.9729101403\\
3.59788994724868	47848.2211333907\\
3.59798994974874	47823.469356641\\
3.59808995224881	47798.7175798914\\
3.59818995474887	47773.9658031417\\
3.59828995724893	47749.2713221716\\
3.59838995974899	47724.5195454219\\
3.59848996224906	47699.8250644518\\
3.59858996474912	47675.1305834816\\
3.59868996724918	47650.4361025115\\
3.59878996974924	47625.7416215414\\
3.59888997224931	47601.0471405712\\
3.59898997474937	47576.3526596011\\
3.59908997724943	47551.7154744105\\
3.59918997974949	47527.0209934403\\
3.59928998224956	47502.3838082497\\
3.59938998474962	47477.7466230591\\
3.59948998724968	47453.1094378685\\
3.59958998974974	47428.4722526778\\
3.59968999224981	47403.8350674872\\
3.59978999474987	47379.1978822966\\
3.59988999724993	47354.560697106\\
3.59998999974999	47329.9808076948\\
3.60009000225006	47305.4009182837\\
};
\addplot [color=mycolor1,solid]
  table[row sep=crcr]{%
3.60009000225006	47305.4009182837\\
3.60019000475012	47280.7637330931\\
3.60029000725018	47256.183843682\\
3.60039000975024	47231.6039542709\\
3.60049001225031	47207.0240648598\\
3.60059001475037	47182.5014712282\\
3.60069001725043	47157.9215818171\\
3.60079001975049	47133.3416924059\\
3.60089002225056	47108.8190987743\\
3.60099002475062	47084.2965051427\\
3.60109002725068	47059.7739115111\\
3.60119002975074	47035.2513178795\\
3.60129003225081	47010.7287242479\\
3.60139003475087	46986.2061306163\\
3.60149003725093	46961.6835369847\\
3.60159003975099	46937.2182391327\\
3.60169004225106	46912.6956455011\\
3.60179004475112	46888.230347649\\
3.60189004725118	46863.7650497969\\
3.60199004975124	46839.2997519448\\
3.60209005225131	46814.8344540927\\
3.60219005475137	46790.3691562406\\
3.60229005725143	46765.9038583885\\
3.60239005975149	46741.495856316\\
3.60249006225156	46717.0305584639\\
3.60259006475162	46692.6225563913\\
3.60269006725168	46668.2145543187\\
3.60279006975174	46643.7492564667\\
3.60289007225181	46619.3985501736\\
3.60299007475187	46594.990548101\\
3.60309007725193	46570.5825460284\\
3.60319007975199	46546.1745439559\\
3.60329008225206	46521.8238376628\\
3.60339008475212	46497.4158355902\\
3.60349008725218	46473.0651292972\\
3.60359008975224	46448.7144230041\\
3.60369009225231	46424.3637167111\\
3.60379009475237	46400.013010418\\
3.60389009725243	46375.6623041249\\
3.60399009975249	46351.3688936114\\
3.60409010225256	46327.0181873183\\
3.60419010475262	46302.7247768048\\
3.60429010725268	46278.4313662912\\
3.60439010975274	46254.0806599982\\
3.60449011225281	46229.7872494846\\
3.60459011475287	46205.4938389711\\
3.60469011725293	46181.2577242371\\
3.60479011975299	46156.9643137235\\
3.60489012225306	46132.67090321\\
3.60499012475312	46108.4347884759\\
3.60509012725318	46084.1986737419\\
3.60519012975324	46059.9625590079\\
3.60529013225331	46035.6691484943\\
3.60539013475337	46011.4903295398\\
3.60549013725343	45987.2542148058\\
3.60559013975349	45963.0181000717\\
3.60569014225356	45938.7819853377\\
3.60579014475362	45914.6031663832\\
3.60589014725368	45890.4243474286\\
3.60599014975374	45866.1882326946\\
3.60609015225381	45842.0094137401\\
3.60619015475387	45817.8305947856\\
3.60629015725393	45793.651775831\\
3.60639015975399	45769.530252656\\
3.60649016225406	45745.3514337015\\
3.60659016475412	45721.2299105265\\
3.60669016725418	45697.051091572\\
3.60679016975424	45672.929568397\\
3.60689017225431	45648.808045222\\
3.60699017475437	45624.686522047\\
3.60709017725443	45600.564998872\\
3.60719017975449	45576.4434756969\\
3.60729018225456	45552.3219525219\\
3.60739018475462	45528.2577251265\\
3.60749018725468	45504.1362019514\\
3.60759018975474	45480.071974556\\
3.60769019225481	45456.0077471605\\
3.60779019475487	45431.943519765\\
3.60789019725493	45407.8792923695\\
3.60799019975499	45383.815064974\\
3.60809020225506	45359.808133358\\
3.60819020475512	45335.7439059625\\
3.60829020725518	45311.7369743465\\
3.60839020975524	45287.672746951\\
3.60849021225531	45263.665815335\\
3.60859021475537	45239.6588837191\\
3.60869021725543	45215.6519521031\\
3.60879021975549	45191.6450204871\\
3.60889022225556	45167.6380888711\\
3.60899022475562	45143.6884530346\\
3.60909022725568	45119.6815214187\\
3.60919022975574	45095.7318855822\\
3.60929023225581	45071.7822497457\\
3.60939023475587	45047.8326139093\\
3.60949023725593	45023.8829780728\\
3.60959023975599	44999.9333422363\\
3.60969024225606	44975.9837063998\\
3.60979024475612	44952.0340705634\\
3.60989024725618	44928.1417305064\\
3.60999024975624	44904.19209467\\
3.61009025225631	44880.299754613\\
3.61019025475637	44856.407414556\\
3.61029025725643	44832.5150744991\\
3.61039025975649	44808.6227344421\\
3.61049026225656	44784.7303943852\\
3.61059026475662	44760.8953501077\\
3.61069026725668	44737.0030100508\\
3.61079026975674	44713.1679657733\\
3.61089027225681	44689.2756257164\\
3.61099027475687	44665.4405814389\\
3.61109027725693	44641.6055371615\\
3.61119027975699	44617.7704928841\\
3.61129028225706	44593.9354486066\\
3.61139028475712	44570.1004043292\\
3.61149028725718	44546.3226558312\\
3.61159028975724	44522.4876115538\\
3.61169029225731	44498.7098630559\\
3.61179029475737	44474.9321145579\\
3.61189029725743	44451.15436606\\
3.61199029975749	44427.3766175621\\
3.61209030225756	44403.5988690642\\
3.61219030475762	44379.8211205662\\
3.61229030725768	44356.0433720683\\
3.61239030975774	44332.3229193499\\
3.61249031225781	44308.6024666315\\
3.61259031475787	44284.8247181335\\
3.61269031725793	44261.1042654151\\
3.61279031975799	44237.3838126967\\
3.61289032225806	44213.6633599783\\
3.61299032475812	44189.9429072599\\
3.61309032725818	44166.279750321\\
3.61319032975824	44142.5592976026\\
3.61329033225831	44118.8961406637\\
3.61339033475837	44095.2329837248\\
3.61349033725843	44071.5125310063\\
3.61359033975849	44047.8493740674\\
3.61369034225856	44024.1862171285\\
3.61379034475862	44000.5803559691\\
3.61389034725868	43976.9171990302\\
3.61399034975874	43953.2540420913\\
3.61409035225881	43929.6481809319\\
3.61419035475887	43905.985023993\\
3.61429035725893	43882.3791628336\\
3.61439035975899	43858.7733016743\\
3.61449036225906	43835.1674405149\\
3.61459036475912	43811.5615793555\\
3.61469036725918	43788.0130139756\\
3.61479036975924	43764.4071528162\\
3.61489037225931	43740.8012916568\\
3.61499037475937	43717.2527262769\\
3.61509037725943	43693.7041608971\\
3.61519037975949	43670.1555955172\\
3.61529038225956	43646.6070301373\\
3.61539038475962	43623.0584647574\\
3.61549038725968	43599.5098993776\\
3.61559038975974	43575.9613339977\\
3.61569039225981	43552.4700643973\\
3.61579039475987	43528.9214990174\\
3.61589039725993	43505.4302294171\\
3.61599039975999	43481.9389598167\\
3.61609040226006	43458.4476902164\\
3.61619040476012	43434.956420616\\
3.61629040726018	43411.4651510156\\
3.61639040976024	43387.9738814153\\
3.61649041226031	43364.5399075944\\
3.61659041476037	43341.048637994\\
3.61669041726043	43317.6146641732\\
3.61679041976049	43294.1806903523\\
3.61689042226056	43270.7467165315\\
3.61699042476062	43247.3127427106\\
3.61709042726068	43223.8787688898\\
3.61719042976074	43200.4447950689\\
3.61729043226081	43177.0681170276\\
3.61739043476087	43153.6341432068\\
3.61749043726093	43130.2574651654\\
3.61759043976099	43106.8234913446\\
3.61769044226106	43083.4468133032\\
3.61779044476112	43060.0701352619\\
3.61789044726118	43036.6934572205\\
3.61799044976124	43013.3740749587\\
3.61809045226131	42989.9973969174\\
3.61819045476137	42966.6207188761\\
3.61829045726143	42943.3013366142\\
3.61839045976149	42919.9819543524\\
3.61849046226156	42896.6625720906\\
3.61859046476162	42873.2858940492\\
3.61869046726168	42850.0238075669\\
3.61879046976174	42826.7044253051\\
3.61889047226181	42803.3850430433\\
3.61899047476187	42780.0656607815\\
3.61909047726193	42756.8035742991\\
3.61919047976199	42733.5414878168\\
3.61929048226206	42710.222105555\\
3.61939048476212	42686.9600190727\\
3.61949048726218	42663.6979325904\\
3.61959048976224	42640.4931418876\\
3.61969049226231	42617.2310554053\\
3.61979049476237	42593.968968923\\
3.61989049726243	42570.7641782202\\
3.61999049976249	42547.5020917379\\
3.62009050226256	42524.2973010351\\
3.62019050476262	42501.0925103323\\
3.62029050726268	42477.8877196295\\
3.62039050976274	42454.6829289267\\
3.62049051226281	42431.4781382239\\
3.62059051476287	42408.3306433006\\
3.62069051726293	42385.1258525978\\
3.62079051976299	42361.9783576745\\
3.62089052226306	42338.7735669717\\
3.62099052476312	42315.6260720484\\
3.62109052726318	42292.4785771251\\
3.62119052976324	42269.3310822018\\
3.62129053226331	42246.1835872786\\
3.62139053476337	42223.0933881348\\
3.62149053726343	42199.9458932115\\
3.62159053976349	42176.8556940677\\
3.62169054226356	42153.7081991444\\
3.62179054476362	42130.6180000007\\
3.62189054726368	42107.5278008569\\
3.62199054976374	42084.4376017131\\
3.62209055226381	42061.3474025694\\
3.62219055476387	42038.2572034256\\
3.62229055726393	42015.2243000613\\
3.62239055976399	41992.1341009175\\
3.62249056226406	41969.1011975533\\
3.62259056476412	41946.068294189\\
3.62269056726418	41922.9780950453\\
3.62279056976424	41899.945191681\\
3.62289057226431	41876.9695840963\\
3.62299057476437	41853.936680732\\
3.62309057726443	41830.9037773677\\
3.62319057976449	41807.928169783\\
3.62329058226456	41784.8952664187\\
3.62339058476462	41761.919658834\\
3.62349058726468	41738.9440512492\\
3.62359058976474	41715.9684436645\\
3.62369059226481	41692.9928360797\\
3.62379059476487	41670.017228495\\
3.62389059726493	41647.0416209103\\
3.62399059976499	41624.0660133255\\
3.62409060226506	41601.1477015203\\
3.62419060476512	41578.229389715\\
3.62429060726518	41555.2537821303\\
3.62439060976524	41532.3354703251\\
3.62449061226531	41509.4171585198\\
3.62459061476537	41486.4988467146\\
3.62469061726543	41463.6378306889\\
3.62479061976549	41440.7195188836\\
3.62489062226556	41417.8585028579\\
3.62499062476562	41394.9401910527\\
3.62509062726568	41372.079175027\\
3.62519062976574	41349.2181590012\\
3.62529063226581	41326.3571429755\\
3.62539063476587	41303.4961269498\\
3.62549063726593	41280.6351109241\\
3.62559063976599	41257.7740948984\\
3.62569064226606	41234.9703746522\\
3.62579064476612	41212.1093586264\\
3.62589064726618	41189.3056383802\\
3.62599064976624	41166.501918134\\
3.62609065226631	41143.6981978878\\
3.62619065476637	41120.8944776416\\
3.62629065726643	41098.0907573954\\
3.62639065976649	41075.2870371492\\
3.62649066226656	41052.5406126825\\
3.62659066476662	41029.7368924363\\
3.62669066726668	41006.9904679696\\
3.62679066976674	40984.2440435029\\
3.62689067226681	40961.4403232567\\
3.62699067476687	40938.69389879\\
3.62709067726693	40916.0047701028\\
3.62719067976699	40893.2583456361\\
3.62729068226706	40870.5119211695\\
3.62739068476712	40847.8227924823\\
3.62749068726718	40825.0763680156\\
3.62759068976724	40802.3872393284\\
3.62769069226731	40779.6981106412\\
3.62779069476737	40757.008981954\\
3.62789069726743	40734.3198532669\\
3.62799069976749	40711.6307245797\\
3.62809070226756	40688.9415958925\\
3.62819070476762	40666.3097629848\\
3.62829070726768	40643.6206342976\\
3.62839070976774	40620.98880139\\
3.62849071226781	40598.3569684823\\
3.62859071476787	40575.7251355746\\
3.62869071726793	40553.093302667\\
3.62879071976799	40530.4614697593\\
3.62889072226806	40507.8296368516\\
3.62899072476812	40485.197803944\\
3.62909072726818	40462.6232668158\\
3.62919072976824	40440.0487296877\\
3.62929073226831	40417.41689678\\
3.62939073476837	40394.8423596518\\
3.62949073726843	40372.2678225237\\
3.62959073976849	40349.6932853955\\
3.62969074226856	40327.1187482674\\
3.62979074476862	40304.6015069187\\
3.62989074726868	40282.0269697906\\
3.62999074976874	40259.5097284419\\
3.63009075226881	40236.9924870933\\
3.63019075476887	40214.4179499651\\
3.63029075726893	40191.9007086165\\
3.63039075976899	40169.3834672679\\
3.63049076226906	40146.9235216987\\
3.63059076476912	40124.4062803501\\
3.63069076726918	40101.8890390014\\
3.63079076976924	40079.4290934323\\
3.63089077226931	40056.9118520837\\
3.63099077476937	40034.4519065146\\
3.63109077726943	40011.9919609454\\
3.63119077976949	39989.5320153763\\
3.63129078226956	39967.0720698072\\
3.63139078476962	39944.612124238\\
3.63149078726968	39922.2094744484\\
3.63159078976974	39899.7495288793\\
3.63169079226981	39877.3468790897\\
3.63179079476987	39854.9442293001\\
3.63189079726993	39832.4842837309\\
3.63199079976999	39810.0816339413\\
3.63209080227006	39787.6789841517\\
3.63219080477012	39765.3336301416\\
3.63229080727018	39742.930980352\\
3.63239080977024	39720.5283305624\\
3.63249081227031	39698.1829765523\\
3.63259081477037	39675.8376225422\\
3.63269081727043	39653.4349727525\\
3.63279081977049	39631.0896187425\\
3.63289082227056	39608.7442647323\\
3.63299082477062	39586.3989107222\\
3.63309082727068	39564.1108524917\\
3.63319082977074	39541.7654984816\\
3.63329083227081	39519.477440251\\
3.63339083477087	39497.1320862409\\
3.63349083727093	39474.8440280103\\
3.63359083977099	39452.5559697797\\
3.63369084227106	39430.2679115491\\
3.63379084477112	39407.9798533185\\
3.63389084727118	39385.6917950879\\
3.63399084977124	39363.4037368573\\
3.63409085227131	39341.1729744063\\
3.63419085477137	39318.8849161757\\
3.63429085727143	39296.6541537246\\
3.63439085977149	39274.4233912735\\
3.63449086227156	39252.1926288224\\
3.63459086477162	39229.9618663714\\
3.63469086727168	39207.7311039203\\
3.63479086977174	39185.5003414692\\
3.63489087227181	39163.3268747976\\
3.63499087477187	39141.0961123466\\
3.63509087727193	39118.922645675\\
3.63519087977199	39096.7491790034\\
3.63529088227206	39074.5757123319\\
3.63539088477212	39052.4022456603\\
3.63549088727218	39030.2287789888\\
3.63559088977224	39008.0553123172\\
3.63569089227231	38985.8818456456\\
3.63579089477237	38963.7656747536\\
3.63589089727243	38941.592208082\\
3.63599089977249	38919.47603719\\
3.63609090227256	38897.3598662979\\
3.63619090477262	38875.2436954059\\
3.63629090727268	38853.1275245138\\
3.63639090977274	38831.0113536218\\
3.63649091227281	38808.8951827297\\
3.63659091477287	38786.8363076172\\
3.63669091727293	38764.7201367251\\
3.63679091977299	38742.6612616126\\
3.63689092227306	38720.6023865001\\
3.63699092477312	38698.5435113875\\
3.63709092727318	38676.484636275\\
3.63719092977324	38654.4257611624\\
3.63729093227331	38632.3668860499\\
3.63739093477337	38610.3653067169\\
3.63749093727343	38588.3064316044\\
3.63759093977349	38566.3048522713\\
3.63769094227356	38544.2459771588\\
3.63779094477362	38522.2443978258\\
3.63789094727368	38500.2428184927\\
3.63799094977374	38478.2412391597\\
3.63809095227381	38456.2396598267\\
3.63819095477387	38434.2953762732\\
3.63829095727393	38412.2937969402\\
3.63839095977399	38390.3495133867\\
3.63849096227406	38368.3479340536\\
3.63859096477412	38346.4036505001\\
3.63869096727418	38324.4593669466\\
3.63879096977424	38302.5150833931\\
3.63889097227431	38280.5707998396\\
3.63899097477437	38258.6838120656\\
3.63909097727443	38236.7395285121\\
3.63919097977449	38214.8525407381\\
3.63929098227456	38192.9082571846\\
3.63939098477462	38171.0212694106\\
3.63949098727468	38149.1342816366\\
3.63959098977474	38127.2472938626\\
3.63969099227481	38105.3603060886\\
3.63979099477487	38083.4733183146\\
3.63989099727493	38061.5863305406\\
3.63999099977499	38039.7566385461\\
3.64009100227506	38017.8696507721\\
3.64019100477512	37996.0399587776\\
3.64029100727518	37974.2102667831\\
3.64039100977524	37952.3805747887\\
3.64049101227531	37930.5508827942\\
3.64059101477537	37908.7211907997\\
3.64069101727543	37886.8914988052\\
3.64079101977549	37865.1191025902\\
3.64089102227556	37843.2894105957\\
3.64099102477562	37821.5170143808\\
3.64109102727568	37799.7446181658\\
3.64119102977574	37777.9149261713\\
3.64129103227581	37756.1425299563\\
3.64139103477587	37734.4274295209\\
3.64149103727593	37712.6550333059\\
3.64159103977599	37690.8826370909\\
3.64169104227606	37669.1675366555\\
3.64179104477612	37647.3951404405\\
3.64189104727618	37625.6800400051\\
3.64199104977624	37603.9649395696\\
3.64209105227631	37582.2498391341\\
3.64219105477637	37560.5347386987\\
3.64229105727643	37538.8196382632\\
3.64239105977649	37517.1045378278\\
3.64249106227656	37495.4467331718\\
3.64259106477662	37473.7316327364\\
3.64269106727668	37452.0738280804\\
3.64279106977674	37430.4160234245\\
3.64289107227681	37408.700922989\\
3.64299107477687	37387.0431183331\\
3.64309107727693	37365.4426094566\\
3.64319107977699	37343.7848048007\\
3.64329108227706	37322.1270001447\\
3.64339108477712	37300.5264912683\\
3.64349108727718	37278.8686866124\\
3.64359108977724	37257.2681777359\\
3.64369109227731	37235.6676688595\\
3.64379109477737	37214.0671599831\\
3.64389109727743	37192.4666511066\\
3.64399109977749	37170.8661422302\\
3.64409110227756	37149.2656333538\\
3.64419110477762	37127.7224202569\\
3.64429110727768	37106.1219113804\\
3.64439110977774	37084.5786982835\\
3.64449111227781	37063.0354851866\\
3.64459111477787	37041.4922720897\\
3.64469111727793	37019.9490589928\\
3.64479111977799	36998.4058458958\\
3.64489112227806	36976.8626327989\\
3.64499112477812	36955.319419702\\
3.64509112727818	36933.8335023846\\
3.64519112977824	36912.2902892877\\
3.64529113227831	36890.8043719703\\
3.64539113477837	36869.3184546529\\
3.64549113727843	36847.8325373355\\
3.64559113977849	36826.346620018\\
3.64569114227856	36804.8607027006\\
3.64579114477862	36783.3747853832\\
3.64589114727868	36761.9461638453\\
3.64599114977874	36740.4602465279\\
3.64609115227881	36719.03162499\\
3.64619115477887	36697.6030034521\\
3.64629115727893	36676.1743819143\\
3.64639115977899	36654.7457603764\\
3.64649116227906	36633.3171388385\\
3.64659116477912	36611.8885173006\\
3.64669116727918	36590.5171915422\\
3.64679116977924	36569.0885700043\\
3.64689117227931	36547.7172442459\\
3.64699117477937	36526.288622708\\
3.64709117727943	36504.9172969497\\
3.64719117977949	36483.5459711913\\
3.64729118227956	36462.1746454329\\
3.64739118477962	36440.8033196745\\
3.64749118727968	36419.4892896956\\
3.64759118977974	36398.1179639373\\
3.64769119227981	36376.8039339584\\
3.64779119477987	36355.4326082\\
3.64789119727993	36334.1185782212\\
3.64799119977999	36312.8045482423\\
3.64809120228006	36291.4905182634\\
3.64819120478012	36270.1764882846\\
3.64829120728018	36248.8624583057\\
3.64839120978024	36227.6057241063\\
3.64849121228031	36206.2916941275\\
3.64859121478037	36185.0349599281\\
3.64869121728043	36163.7782257288\\
3.64879121978049	36142.5214915294\\
3.64889122228056	36121.2074615505\\
3.64899122478062	36100.0080231307\\
3.64909122728068	36078.7512889313\\
3.64919122978074	36057.494554732\\
3.64929123228081	36036.2378205326\\
3.64939123478087	36015.0383821128\\
3.64949123728093	35993.838943693\\
3.64959123978099	35972.5822094936\\
3.64969124228106	35951.3827710738\\
3.64979124478112	35930.1833326539\\
3.64989124728118	35909.0411900136\\
3.64999124978124	35887.8417515938\\
3.65009125228131	35866.6423131739\\
3.65019125478137	35845.5001705336\\
3.65029125728143	35824.3007321137\\
3.65039125978149	35803.1585894734\\
3.65049126228156	35782.0164468331\\
3.65059126478162	35760.8743041928\\
3.65069126728168	35739.7321615524\\
3.65079126978174	35718.5900189121\\
3.65089127228181	35697.4478762718\\
3.65099127478187	35676.363029411\\
3.65109127728193	35655.2208867706\\
3.65119127978199	35634.1360399098\\
3.65129128228206	35613.051193049\\
3.65139128478212	35591.9663461882\\
3.65149128728218	35570.8814993274\\
3.65159128978224	35549.7966524666\\
3.65169129228231	35528.7118056058\\
3.65179129478237	35507.6842545245\\
3.65189129728243	35486.5994076636\\
3.65199129978249	35465.5718565823\\
3.65209130228256	35444.4870097215\\
3.65219130478262	35423.4594586402\\
3.65229130728268	35402.4319075589\\
3.65239130978274	35381.4043564776\\
3.65249131228281	35360.3768053963\\
3.65259131478287	35339.4065500945\\
3.65269131728293	35318.3789990132\\
3.65279131978299	35297.4087437114\\
3.65289132228306	35276.3811926301\\
3.65299132478312	35255.4109373284\\
3.65309132728318	35234.4406820266\\
3.65319132978324	35213.4704267248\\
3.65329133228331	35192.500171423\\
3.65339133478337	35171.5299161212\\
3.65349133728343	35150.6169565989\\
3.65359133978349	35129.6467012971\\
3.65369134228356	35108.7337417749\\
3.65379134478362	35087.8207822526\\
3.65389134728368	35066.8505269508\\
3.65399134978374	35045.9375674285\\
3.65409135228381	35025.0819036858\\
3.65419135478387	35004.1689441635\\
3.65429135728393	34983.2559846412\\
3.65439135978399	34962.3430251189\\
3.65449136228406	34941.4873613762\\
3.65459136478412	34920.6316976334\\
3.65469136728418	34899.7187381111\\
3.65479136978424	34878.8630743684\\
3.65489137228431	34858.0074106256\\
3.65499137478437	34837.1517468829\\
3.65509137728443	34816.3533789196\\
3.65519137978449	34795.4977151768\\
3.65529138228456	34774.6993472136\\
3.65539138478462	34753.8436834708\\
3.65549138728468	34733.0453155076\\
3.65559138978474	34712.2469475443\\
3.65569139228481	34691.4485795811\\
3.65579139478487	34670.6502116178\\
3.65589139728493	34649.8518436546\\
3.65599139978499	34629.0534756913\\
3.65609140228506	34608.3124035076\\
3.65619140478512	34587.5140355444\\
3.65629140728518	34566.7729633606\\
3.65639140978524	34546.0318911769\\
3.65649141228531	34525.2335232136\\
3.65659141478537	34504.4924510299\\
3.65669141728543	34483.8086746257\\
3.65679141978549	34463.0676024419\\
3.65689142228556	34442.3265302582\\
3.65699142478562	34421.642753854\\
3.65709142728568	34400.9016816702\\
3.65719142978574	34380.217905266\\
3.65729143228581	34359.5341288618\\
3.65739143478587	34338.8503524576\\
3.65749143728593	34318.1665760534\\
3.65759143978599	34297.4827996491\\
3.65769144228606	34276.7990232449\\
3.65779144478612	34256.1725426202\\
3.65789144728618	34235.488766216\\
3.65799144978624	34214.8622855913\\
3.65809145228631	34194.2358049666\\
3.65819145478637	34173.5520285623\\
3.65829145728643	34152.9255479376\\
3.65839145978649	34132.3563630924\\
3.65849146228656	34111.7298824677\\
3.65859146478662	34091.103401843\\
3.65869146728668	34070.5342169978\\
3.65879146978674	34049.9077363731\\
3.65889147228681	34029.3385515279\\
3.65899147478687	34008.7693666827\\
3.65909147728693	33988.2001818375\\
3.65919147978699	33967.6309969923\\
3.65929148228706	33947.0618121471\\
3.65939148478712	33926.4926273019\\
3.65949148728718	33905.9234424567\\
3.65959148978724	33885.411553391\\
3.65969149228731	33864.8996643254\\
3.65979149478737	33844.3304794802\\
3.65989149728743	33823.8185904145\\
3.65999149978749	33803.3067013488\\
3.66009150228756	33782.7948122831\\
3.66019150478762	33762.2829232174\\
3.66029150728768	33741.8283299313\\
3.66039150978774	33721.3164408656\\
3.66049151228781	33700.8618475794\\
3.66059151478787	33680.4072542932\\
3.66069151728793	33659.8953652276\\
3.66079151978799	33639.4407719414\\
3.66089152228806	33618.9861786552\\
3.66099152478812	33598.531585369\\
3.66109152728818	33578.1342878624\\
3.66119152978824	33557.6796945762\\
3.66129153228831	33537.2823970696\\
3.66139153478837	33516.8278037834\\
3.66149153728843	33496.4305062767\\
3.66159153978849	33476.0332087701\\
3.66169154228856	33455.6359112634\\
3.66179154478862	33435.2386137568\\
3.66189154728868	33414.8413162501\\
3.66199154978874	33394.4440187434\\
3.66209155228881	33374.1040170163\\
3.66219155478887	33353.7067195096\\
3.66229155728893	33333.3667177825\\
3.66239155978899	33313.0267160554\\
3.66249156228906	33292.6867143282\\
3.66259156478912	33272.3467126011\\
3.66269156728918	33252.0067108739\\
3.66279156978924	33231.6667091468\\
3.66289157228931	33211.3267074196\\
3.66299157478937	33191.044001472\\
3.66309157728943	33170.7612955244\\
3.66319157978949	33150.4212937972\\
3.66329158228956	33130.1385878496\\
3.66339158478962	33109.855881902\\
3.66349158728968	33089.5731759543\\
3.66359158978974	33069.2904700067\\
3.66369159228981	33049.0650598386\\
3.66379159478987	33028.782353891\\
3.66389159728993	33008.5569437228\\
3.66399159978999	32988.2742377752\\
3.66409160229006	32968.0488276071\\
3.66419160479012	32947.823417439\\
3.66429160729018	32927.5980072708\\
3.66439160979024	32907.3725971027\\
3.66449161229031	32887.1471869346\\
3.66459161479037	32866.979072546\\
3.66469161729043	32846.7536623779\\
3.66479161979049	32826.5855479893\\
3.66489162229056	32806.3601378212\\
3.66499162479062	32786.1920234326\\
3.66509162729068	32766.023909044\\
3.66519162979074	32745.8557946553\\
3.66529163229081	32725.6876802667\\
3.66539163479087	32705.5195658781\\
3.66549163729093	32685.408747269\\
3.66559163979099	32665.2406328804\\
3.66569164229106	32645.1298142713\\
3.66579164479112	32625.0189956623\\
3.66589164729118	32604.9081770532\\
3.66599164979124	32584.7973584441\\
3.66609165229131	32564.686539835\\
3.66619165479137	32544.5757212259\\
3.66629165729143	32524.4649026168\\
3.66639165979149	32504.4113797872\\
3.66649166229156	32484.3005611781\\
3.66659166479162	32464.2470383485\\
3.66669166729168	32444.193515519\\
3.66679166979174	32424.1399926894\\
3.66689167229181	32404.0864698598\\
3.66699167479187	32384.0329470302\\
3.66709167729193	32363.9794242007\\
3.667191679792	32343.9259013711\\
3.66729168229206	32323.929674321\\
3.66739168479212	32303.9334472709\\
3.66749168729218	32283.8799244414\\
3.66759168979224	32263.8836973913\\
3.66769169229231	32243.8874703412\\
3.66779169479237	32223.8912432912\\
3.66789169729243	32203.8950162411\\
3.66799169979249	32183.9560849706\\
3.66809170229256	32163.9598579205\\
3.66819170479262	32144.0209266499\\
3.66829170729268	32124.0246995999\\
3.66839170979274	32104.0857683293\\
3.66849171229281	32084.1468370588\\
3.66859171479287	32064.2079057882\\
3.66869171729293	32044.2689745177\\
3.66879171979299	32024.3300432471\\
3.66889172229306	32004.4484077561\\
3.66899172479312	31984.5094764855\\
3.66909172729318	31964.6278409945\\
3.66919172979325	31944.6889097239\\
3.66929173229331	31924.8072742329\\
3.66939173479337	31904.9256387418\\
3.66949173729343	31885.0440032508\\
3.66959173979349	31865.1623677598\\
3.66969174229356	31845.3380280482\\
3.66979174479362	31825.4563925572\\
3.66989174729368	31805.5747570662\\
3.66999174979374	31785.7504173546\\
3.67009175229381	31765.9260776431\\
3.67019175479387	31746.1017379316\\
3.67029175729393	31726.27739822\\
3.67039175979399	31706.4530585085\\
3.67049176229406	31686.628718797\\
3.67059176479412	31666.8043790855\\
3.67069176729418	31647.0373351535\\
3.67079176979424	31627.2129954419\\
3.67089177229431	31607.4459515099\\
3.67099177479437	31587.6789075779\\
3.67109177729443	31567.9118636459\\
3.6711917797945	31548.1448197139\\
3.67129178229456	31528.3777757819\\
3.67139178479462	31508.6107318499\\
3.67149178729468	31488.8436879178\\
3.67159178979474	31469.1339397653\\
3.67169179229481	31449.4241916128\\
3.67179179479487	31429.6571476808\\
3.67189179729493	31409.9473995283\\
3.67199179979499	31390.2376513758\\
3.67209180229506	31370.5279032233\\
3.67219180479512	31350.8181550708\\
3.67229180729518	31331.1657026978\\
3.67239180979524	31311.4559545453\\
3.67249181229531	31291.7462063928\\
3.67259181479537	31272.0937540198\\
3.67269181729543	31252.4413016469\\
3.67279181979549	31232.7888492739\\
3.67289182229556	31213.1363969009\\
3.67299182479562	31193.4839445279\\
3.67309182729568	31173.8314921549\\
3.67319182979575	31154.1790397819\\
3.67329183229581	31134.5838831884\\
3.67339183479587	31114.9314308155\\
3.67349183729593	31095.336274222\\
3.673591839796	31075.7411176285\\
3.67369184229606	31056.145961035\\
3.67379184479612	31036.5508044416\\
3.67389184729618	31016.9556478481\\
3.67399184979624	30997.3604912546\\
3.67409185229631	30977.8226304407\\
3.67419185479637	30958.2274738472\\
3.67429185729643	30938.6896130332\\
3.67439185979649	30919.1517522193\\
3.67449186229656	30899.5565956258\\
3.67459186479662	30880.0187348118\\
3.67469186729668	30860.4808739979\\
3.67479186979674	30841.0003089634\\
3.67489187229681	30821.4624481495\\
3.67499187479687	30801.9245873355\\
3.67509187729693	30782.444022301\\
3.675191879797	30762.9061614871\\
3.67529188229706	30743.4255964526\\
3.67539188479712	30723.9450314182\\
3.67549188729718	30704.4644663837\\
3.67559188979725	30684.9839013493\\
3.67569189229731	30665.5033363148\\
3.67579189479737	30646.0800670599\\
3.67589189729743	30626.5995020255\\
3.67599189979749	30607.1762327705\\
3.67609190229756	30587.7529635156\\
3.67619190479762	30568.2723984811\\
3.67629190729768	30548.8491292262\\
3.67639190979774	30529.4258599713\\
3.67649191229781	30510.0025907163\\
3.67659191479787	30490.6366172409\\
3.67669191729793	30471.213347986\\
3.67679191979799	30451.8473745106\\
3.67689192229806	30432.4241052556\\
3.67699192479812	30413.0581317802\\
3.67709192729818	30393.6921583048\\
3.67719192979825	30374.3261848294\\
3.67729193229831	30354.9602113539\\
3.67739193479837	30335.5942378785\\
3.67749193729843	30316.2282644031\\
3.6775919397985	30296.9195867072\\
3.67769194229856	30277.5536132318\\
3.67779194479862	30258.2449355359\\
3.67789194729868	30238.9362578399\\
3.67799194979874	30219.627580144\\
3.67809195229881	30200.3189024481\\
3.67819195479887	30181.0102247522\\
3.67829195729893	30161.7015470563\\
3.67839195979899	30142.3928693604\\
3.67849196229906	30123.141487444\\
3.67859196479912	30103.8328097481\\
3.67869196729918	30084.5814278317\\
3.67879196979924	30065.3300459153\\
3.67889197229931	30046.0786639989\\
3.67899197479937	30026.8272820825\\
3.67909197729943	30007.5759001661\\
3.6791919797995	29988.3245182497\\
3.67929198229956	29969.1304321128\\
3.67939198479962	29949.8790501964\\
3.67949198729968	29930.6849640596\\
3.67959198979975	29911.4908779227\\
3.67969199229981	29892.2394960063\\
3.67979199479987	29873.0454098694\\
3.67989199729993	29853.8513237325\\
3.6799919998	29834.7145333752\\
3.68009200230006	29815.5204472383\\
3.68019200480012	29796.3263611014\\
3.68029200730018	29777.189570744\\
3.68039200980024	29758.0527803866\\
3.68049201230031	29738.8586942498\\
3.68059201480037	29719.7219038924\\
3.68069201730043	29700.585113535\\
3.68079201980049	29681.4483231777\\
3.68089202230056	29662.3688285998\\
3.68099202480062	29643.2320382424\\
3.68109202730068	29624.0952478851\\
3.68119202980075	29605.0157533072\\
3.68129203230081	29585.9362587293\\
3.68139203480087	29566.799468372\\
3.68149203730093	29547.7199737941\\
3.681592039801	29528.6404792163\\
3.68169204230106	29509.6182804179\\
3.68179204480112	29490.5387858401\\
3.68189204730118	29471.4592912622\\
3.68199204980125	29452.4370924639\\
3.68209205230131	29433.357597886\\
3.68219205480137	29414.3353990877\\
3.68229205730143	29395.3132002893\\
3.68239205980149	29376.291001491\\
3.68249206230156	29357.2688026926\\
3.68259206480162	29338.2466038943\\
3.68269206730168	29319.2244050959\\
3.68279206980174	29300.2595020771\\
3.68289207230181	29281.2373032788\\
3.68299207480187	29262.2724002599\\
3.68309207730193	29243.3074972411\\
3.683192079802	29224.3425942223\\
3.68329208230206	29205.3776912035\\
3.68339208480212	29186.4127881846\\
3.68349208730218	29167.4478851658\\
3.68359208980225	29148.482982147\\
3.68369209230231	29129.5753749076\\
3.68379209480237	29110.6104718888\\
3.68389209730243	29091.7028646495\\
3.6839920998025	29072.7952574102\\
3.68409210230256	29053.8876501709\\
3.68419210480262	29034.9800429315\\
3.68429210730268	29016.0724356922\\
3.68439210980275	28997.1648284529\\
3.68449211230281	28978.3145169931\\
3.68459211480287	28959.4069097538\\
3.68469211730293	28940.556598294\\
3.68479211980299	28921.6489910547\\
3.68489212230306	28902.7986795949\\
3.68499212480312	28883.9483681351\\
3.68509212730318	28865.0980566753\\
3.68519212980325	28846.2477452155\\
3.68529213230331	28827.4547295352\\
3.68539213480337	28808.6044180754\\
3.68549213730343	28789.8114023951\\
3.6855921398035	28770.9610909353\\
3.68569214230356	28752.168075255\\
3.68579214480362	28733.3750595747\\
3.68589214730368	28714.5820438944\\
3.68599214980375	28695.7890282141\\
3.68609215230381	28676.9960125338\\
3.68619215480387	28658.2029968535\\
3.68629215730393	28639.4672769527\\
3.686392159804	28620.6742612724\\
3.68649216230406	28601.9385413717\\
3.68659216480412	28583.2028214709\\
3.68669216730418	28564.4671015701\\
3.68679216980424	28545.7313816693\\
3.68689217230431	28526.9956617686\\
3.68699217480437	28508.2599418678\\
3.68709217730443	28489.5815177465\\
3.6871921798045	28470.8457978457\\
3.68729218230456	28452.1673737245\\
3.68739218480462	28433.4316538237\\
3.68749218730468	28414.7532297024\\
3.68759218980475	28396.0748055812\\
3.68769219230481	28377.3963814599\\
3.68779219480487	28358.7179573386\\
3.68789219730493	28340.0968289969\\
3.687992199805	28321.4184048756\\
3.68809220230506	28302.7399807544\\
3.68819220480512	28284.1188524126\\
3.68829220730518	28265.4977240709\\
3.68839220980525	28246.8765957291\\
3.68849221230531	28228.2554673873\\
3.68859221480537	28209.6343390456\\
3.68869221730543	28191.0132107038\\
3.68879221980549	28172.3920823621\\
3.68889222230556	28153.8282497999\\
3.68899222480562	28135.2071214581\\
3.68909222730568	28116.6432888959\\
3.68919222980575	28098.0794563336\\
3.68929223230581	28079.4583279919\\
3.68939223480587	28060.8944954296\\
3.68949223730593	28042.3879586469\\
3.689592239806	28023.8241260847\\
3.68969224230606	28005.2602935224\\
3.68979224480612	27986.7537567397\\
3.68989224730618	27968.1899241775\\
3.68999224980625	27949.6833873947\\
3.69009225230631	27931.176850612\\
3.69019225480637	27912.6130180498\\
3.69029225730643	27894.1637770466\\
3.6903922598065	27875.6572402638\\
3.69049226230656	27857.1507034811\\
3.69059226480662	27838.6441666984\\
3.69069226730668	27820.1949256952\\
3.69079226980675	27801.6883889125\\
3.69089227230681	27783.2391479092\\
3.69099227480687	27764.789906906\\
3.69109227730693	27746.3406659028\\
3.691192279807	27727.8914248996\\
3.69129228230706	27709.4421838964\\
3.69139228480712	27690.9929428932\\
3.69149228730718	27672.6009976695\\
3.69159228980725	27654.1517566663\\
3.69169229230731	27635.7598114426\\
3.69179229480737	27617.3678662189\\
3.69189229730743	27598.9186252157\\
3.6919922998075	27580.526679992\\
3.69209230230756	27562.1347347683\\
3.69219230480762	27543.8000853241\\
3.69229230730768	27525.4081401004\\
3.69239230980775	27507.0161948767\\
3.69249231230781	27488.6815454325\\
3.69259231480787	27470.3468959883\\
3.69269231730793	27451.9549507646\\
3.692792319808	27433.6203013204\\
3.69289232230806	27415.2856518762\\
3.69299232480812	27396.951002432\\
3.69309232730818	27378.6736487674\\
3.69319232980825	27360.3389993232\\
3.69329233230831	27342.004349879\\
3.69339233480837	27323.7269962143\\
3.69349233730843	27305.4496425496\\
3.6935923398085	27287.1149931055\\
3.69369234230856	27268.8376394408\\
3.69379234480862	27250.5602857761\\
3.69389234730868	27232.2829321114\\
3.69399234980875	27214.0628742263\\
3.69409235230881	27195.7855205616\\
3.69419235480887	27177.5654626764\\
3.69429235730893	27159.2881090118\\
3.694392359809	27141.0680511266\\
3.69449236230906	27122.8479932414\\
3.69459236480912	27104.6279353563\\
3.69469236730918	27086.4078774711\\
3.69479236980925	27068.187819586\\
3.69489237230931	27049.9677617008\\
3.69499237480937	27031.8049995952\\
3.69509237730943	27013.58494171\\
3.6951923798095	26995.4221796044\\
3.69529238230956	26977.2021217192\\
3.69539238480962	26959.0393596136\\
3.69549238730968	26940.8765975079\\
3.69559238980975	26922.7138354023\\
3.69569239230981	26904.5510732966\\
3.69579239480987	26886.4456069705\\
3.69589239730993	26868.2828448648\\
3.69599239981	26850.1773785387\\
3.69609240231006	26832.014616433\\
3.69619240481012	26813.9091501069\\
3.69629240731018	26795.8036837808\\
3.69639240981025	26777.6982174546\\
3.69649241231031	26759.5927511285\\
3.69659241481037	26741.4872848024\\
3.69669241731043	26723.4391142558\\
3.6967924198105	26705.3336479296\\
3.69689242231056	26687.285477383\\
3.69699242481062	26669.1800110569\\
3.69709242731068	26651.1318405102\\
3.69719242981075	26633.0836699636\\
3.69729243231081	26615.035499417\\
3.69739243481087	26596.9873288704\\
3.69749243731093	26578.9391583238\\
3.697592439811	26560.9482835567\\
3.69769244231106	26542.90011301\\
3.69779244481112	26524.9092382429\\
3.69789244731118	26506.8610676963\\
3.69799244981125	26488.8701929292\\
3.69809245231131	26470.8793181621\\
3.69819245481137	26452.888443395\\
3.69829245731143	26434.8975686279\\
3.6983924598115	26416.9639896403\\
3.69849246231156	26398.9731148732\\
3.69859246481162	26381.0395358856\\
3.69869246731168	26363.0486611185\\
3.69879246981175	26345.1150821309\\
3.69889247231181	26327.1815031433\\
3.69899247481187	26309.2479241557\\
3.69909247731193	26291.3143451681\\
3.699192479812	26273.3807661805\\
3.69929248231206	26255.4471871929\\
3.69939248481212	26237.5709039848\\
3.69949248731218	26219.6373249972\\
3.69959248981225	26201.7610417891\\
3.69969249231231	26183.8274628015\\
3.69979249481237	26165.9511795935\\
3.69989249731243	26148.0748963854\\
3.6999924998125	26130.1986131773\\
3.70009250231256	26112.3796257487\\
3.70019250481262	26094.5033425406\\
3.70029250731268	26076.6270593326\\
3.70039250981275	26058.808071904\\
3.70049251231281	26040.9890844754\\
3.70059251481287	26023.1128012673\\
3.70069251731293	26005.2938138388\\
3.700792519813	25987.4748264102\\
3.70089252231306	25969.6558389816\\
3.70099252481312	25951.8368515531\\
3.70109252731318	25934.075159904\\
3.70119252981325	25916.2561724755\\
3.70129253231331	25898.4944808264\\
3.70139253481337	25880.7327891773\\
3.70149253731343	25862.9138017488\\
3.7015925398135	25845.1521100997\\
3.70169254231356	25827.3904184507\\
3.70179254481362	25809.6287268016\\
3.70189254731368	25791.9243309321\\
3.70199254981375	25774.162639283\\
3.70209255231381	25756.4009476339\\
3.70219255481387	25738.6965517644\\
3.70229255731393	25720.9921558949\\
3.702392559814	25703.2304642458\\
3.70249256231406	25685.5260683763\\
3.70259256481412	25667.8216725067\\
3.70269256731418	25650.1745724167\\
3.70279256981425	25632.4701765472\\
3.70289257231431	25614.7657806776\\
3.70299257481437	25597.1186805876\\
3.70309257731443	25579.414284718\\
3.7031925798145	25561.767184628\\
3.70329258231456	25544.120084538\\
3.70339258481462	25526.4729844479\\
3.70349258731468	25508.8258843579\\
3.70359258981475	25491.1787842679\\
3.70369259231481	25473.5316841779\\
3.70379259481487	25455.9418798673\\
3.70389259731493	25438.2947797773\\
3.703992599815	25420.7049754668\\
3.70409260231506	25403.1151711563\\
3.70419260481512	25385.4680710663\\
3.70429260731518	25367.8782667557\\
3.70439260981525	25350.2884624452\\
3.70449261231531	25332.7559539142\\
3.70459261481537	25315.1661496037\\
3.70469261731543	25297.5763452932\\
3.7047926198155	25280.0438367622\\
3.70489262231556	25262.5113282312\\
3.70499262481562	25244.9215239207\\
3.70509262731568	25227.3890153897\\
3.70519262981575	25209.8565068587\\
3.70529263231581	25192.3239983277\\
3.70539263481587	25174.7914897967\\
3.70549263731593	25157.3162770452\\
3.705592639816	25139.7837685142\\
3.70569264231606	25122.3085557627\\
3.70579264481612	25104.7760472317\\
3.70589264731618	25087.3008344802\\
3.70599264981625	25069.8256217287\\
3.70609265231631	25052.3504089772\\
3.70619265481637	25034.8751962257\\
3.70629265731643	25017.3999834742\\
3.7063926598165	24999.9820665022\\
3.70649266231656	24982.5068537507\\
3.70659266481662	24965.0889367788\\
3.70669266731668	24947.6710198068\\
3.70679266981675	24930.1958070553\\
3.70689267231681	24912.7778900833\\
3.70699267481687	24895.3599731113\\
3.70709267731693	24877.9420561394\\
3.707192679817	24860.5814349469\\
3.70729268231706	24843.1635179749\\
3.70739268481712	24825.745601003\\
3.70749268731718	24808.3849798105\\
3.70759268981725	24791.024358618\\
3.70769269231731	24773.6637374256\\
3.70779269481737	24756.2458204536\\
3.70789269731743	24738.8851992611\\
3.7079926998175	24721.5818738482\\
3.70809270231756	24704.2212526557\\
3.70819270481762	24686.8606314632\\
3.70829270731768	24669.5573060503\\
3.70839270981775	24652.1966848578\\
3.70849271231781	24634.8933594449\\
3.70859271481787	24617.5900340319\\
3.70869271731793	24600.286708619\\
3.708792719818	24582.983383206\\
3.70889272231806	24565.6800577931\\
3.70899272481812	24548.3767323801\\
3.70909272731818	24531.1307027467\\
3.70919272981825	24513.8273773337\\
3.70929273231831	24496.5813477003\\
3.70939273481837	24479.3353180669\\
3.70949273731843	24462.0319926539\\
3.7095927398185	24444.7859630205\\
3.70969274231856	24427.539933387\\
3.70979274481862	24410.3511995331\\
3.70989274731868	24393.1051698997\\
3.70999274981875	24375.8591402662\\
3.71009275231881	24358.6704064123\\
3.71019275481887	24341.4816725584\\
3.71029275731893	24324.2356429249\\
3.710392759819	24307.046909071\\
3.71049276231906	24289.8581752171\\
3.71059276481912	24272.6694413632\\
3.71069276731918	24255.4807075092\\
3.71079276981925	24238.3492694348\\
3.71089277231931	24221.1605355809\\
3.71099277481937	24204.0290975065\\
3.71109277731943	24186.8403636526\\
3.7111927798195	24169.7089255782\\
3.71129278231956	24152.5774875037\\
3.71139278481962	24135.4460494293\\
3.71149278731968	24118.3146113549\\
3.71159278981975	24101.1831732805\\
3.71169279231981	24084.1090309856\\
3.71179279481987	24066.9775929112\\
3.71189279731993	24049.9034506163\\
3.71199279982	24032.7720125419\\
3.71209280232006	24015.697870247\\
3.71219280482012	23998.6237279521\\
3.71229280732018	23981.5495856572\\
3.71239280982025	23964.4754433623\\
3.71249281232031	23947.4013010674\\
3.71259281482037	23930.384454552\\
3.71269281732043	23913.3103122571\\
3.7127928198205	23896.2934657417\\
3.71289282232056	23879.2193234468\\
3.71299282482062	23862.2024769314\\
3.71309282732068	23845.1856304161\\
3.71319282982075	23828.1687839007\\
3.71329283232081	23811.1519373853\\
3.71339283482087	23794.1350908699\\
3.71349283732093	23777.175540134\\
3.713592839821	23760.1586936186\\
3.71369284232106	23743.1991428828\\
3.71379284482112	23726.1822963674\\
3.71389284732118	23709.2227456315\\
3.71399284982125	23692.2631948956\\
3.71409285232131	23675.3036441598\\
3.71419285482137	23658.3440934239\\
3.71429285732143	23641.4418384675\\
3.7143928598215	23624.4822877317\\
3.71449286232156	23607.5227369958\\
3.71459286482162	23590.6204820394\\
3.71469286732168	23573.7182270831\\
3.71479286982175	23556.7586763472\\
3.71489287232181	23539.8564213908\\
3.71499287482187	23522.9541664345\\
3.71509287732193	23506.1092072576\\
3.715192879822	23489.2069523013\\
3.71529288232206	23472.3046973449\\
3.71539288482212	23455.4597381681\\
3.71549288732218	23438.5574832117\\
3.71559288982225	23421.7125240349\\
3.71569289232231	23404.867564858\\
3.71579289482237	23388.0226056812\\
3.71589289732243	23371.1776465043\\
3.7159928998225	23354.3326873275\\
3.71609290232256	23337.4877281506\\
3.71619290482262	23320.7000647533\\
3.71629290732268	23303.8551055765\\
3.71639290982275	23287.0674421791\\
3.71649291232281	23270.2224830023\\
3.71659291482287	23253.4348196049\\
3.71669291732293	23236.6471562076\\
3.716792919823	23219.8594928103\\
3.71689292232306	23203.1291251925\\
3.71699292482312	23186.3414617951\\
3.71709292732318	23169.5537983978\\
3.71719292982325	23152.82343078\\
3.71729293232331	23136.0357673826\\
3.71739293482337	23119.3053997648\\
3.71749293732343	23102.575032147\\
3.7175929398235	23085.8446645292\\
3.71769294232356	23069.1142969114\\
3.71779294482362	23052.3839292935\\
3.71789294732368	23035.6535616757\\
3.71799294982375	23018.9804898374\\
3.71809295232381	23002.2501222196\\
3.71819295482387	22985.5770503813\\
3.71829295732393	22968.903978543\\
3.718392959824	22952.2309067047\\
3.71849296232406	22935.5578348664\\
3.71859296482412	22918.8847630281\\
3.71869296732418	22902.2116911898\\
3.71879296982425	22885.5386193514\\
3.71889297232431	22868.9228432927\\
3.71899297482437	22852.2497714543\\
3.71909297732443	22835.6339953956\\
3.7191929798245	22818.9609235572\\
3.71929298232456	22802.3451474984\\
3.71939298482462	22785.7293714397\\
3.71949298732468	22769.1135953809\\
3.71959298982475	22752.5551151016\\
3.71969299232481	22735.9393390428\\
3.71979299482487	22719.323562984\\
3.71989299732493	22702.7650827047\\
3.719992999825	22686.2066024254\\
3.72009300232506	22669.5908263666\\
3.72019300482512	22653.0323460874\\
3.72029300732518	22636.4738658081\\
3.72039300982525	22619.9153855288\\
3.72049301232531	22603.3569052495\\
3.72059301482537	22586.8557207497\\
3.72069301732543	22570.2972404705\\
3.7207930198255	22553.7960559707\\
3.72089302232556	22537.2375756914\\
3.72099302482562	22520.7363911917\\
3.72109302732568	22504.2352066919\\
3.72119302982575	22487.7340221921\\
3.72129303232581	22471.2328376924\\
3.72139303482587	22454.7316531926\\
3.72149303732593	22438.2877644723\\
3.721593039826	22421.7865799726\\
3.72169304232606	22405.3426912523\\
3.72179304482612	22388.8415067525\\
3.72189304732618	22372.3976180323\\
3.72199304982625	22355.953729312\\
3.72209305232631	22339.5098405918\\
3.72219305482637	22323.0659518715\\
3.72229305732643	22306.6220631513\\
3.7223930598265	22290.2354702105\\
3.72249306232656	22273.7915814903\\
3.72259306482662	22257.4049885495\\
3.72269306732668	22240.9610998293\\
3.72279306982675	22224.5745068885\\
3.72289307232681	22208.1879139478\\
3.72299307482687	22191.801321007\\
3.72309307732693	22175.4147280663\\
3.723193079827	22159.0281351256\\
3.72329308232706	22142.6988379643\\
3.72339308482712	22126.3122450236\\
3.72349308732718	22109.9829478624\\
3.72359308982725	22093.5963549216\\
3.72369309232731	22077.2670577604\\
3.72379309482737	22060.9377605992\\
3.72389309732743	22044.6084634379\\
3.7239930998275	22028.2791662767\\
3.72409310232756	22012.007164895\\
3.72419310482762	21995.6778677338\\
3.72429310732768	21979.3485705725\\
3.72439310982775	21963.0765691908\\
3.72449311232781	21946.8045678091\\
3.72459311482787	21930.4752706479\\
3.72469311732793	21914.2032692662\\
3.724793119828	21897.9312678844\\
3.72489312232806	21881.6592665027\\
3.72499312482812	21865.4445609005\\
3.72509312732818	21849.1725595188\\
3.72519312982825	21832.9005581371\\
3.72529313232831	21816.6858525349\\
3.72539313482837	21800.4711469327\\
3.72549313732843	21784.2564413305\\
3.7255931398285	21767.9844399488\\
3.72569314232856	21751.7697343466\\
3.72579314482862	21735.6123245239\\
3.72589314732868	21719.3976189217\\
3.72599314982875	21703.1829133195\\
3.72609315232881	21687.0255034968\\
3.72619315482887	21670.8107978946\\
3.72629315732893	21654.6533880719\\
3.726393159829	21638.4959782492\\
3.72649316232906	21622.3385684265\\
3.72659316482912	21606.1811586038\\
3.72669316732918	21590.0237487811\\
3.72679316982925	21573.8663389585\\
3.72689317232931	21557.7089291358\\
3.72699317482937	21541.6088150926\\
3.72709317732943	21525.4514052699\\
3.7271931798295	21509.3512912267\\
3.72729318232956	21493.2511771835\\
3.72739318482962	21477.1510631404\\
3.72749318732968	21461.0509490972\\
3.72759318982975	21444.950835054\\
3.72769319232981	21428.8507210108\\
3.72779319482987	21412.8079027472\\
3.72789319732993	21396.707788704\\
3.72799319983	21380.6649704403\\
3.72809320233006	21364.5648563972\\
3.72819320483012	21348.5220381335\\
3.72829320733018	21332.4792198698\\
3.72839320983025	21316.4364016062\\
3.72849321233031	21300.3935833425\\
3.72859321483037	21284.4080608584\\
3.72869321733043	21268.3652425947\\
3.7287932198305	21252.322424331\\
3.72889322233056	21236.3369018469\\
3.72899322483062	21220.3513793627\\
3.72909322733068	21204.3658568786\\
3.72919322983075	21188.3230386149\\
3.72929323233081	21172.3375161308\\
3.72939323483087	21156.4092894261\\
3.72949323733093	21140.423766942\\
3.729593239831	21124.4382444578\\
3.72969324233106	21108.5100177532\\
3.72979324483112	21092.524495269\\
3.72989324733118	21076.5962685644\\
3.72999324983125	21060.6680418598\\
3.73009325233131	21044.7398151551\\
3.73019325483137	21028.8115884505\\
3.73029325733143	21012.8833617459\\
3.7303932598315	20996.9551350412\\
3.73049326233156	20981.0842041161\\
3.73059326483162	20965.1559774115\\
3.73069326733168	20949.2850464863\\
3.73079326983175	20933.3568197817\\
3.73089327233181	20917.4858888566\\
3.73099327483187	20901.6149579315\\
3.73109327733193	20885.7440270063\\
3.731193279832	20869.8730960812\\
3.73129328233206	20854.0021651561\\
3.73139328483212	20838.1885300105\\
3.73149328733218	20822.3175990854\\
3.73159328983225	20806.5039639397\\
3.73169329233231	20790.6903287941\\
3.73179329483237	20774.819397869\\
3.73189329733243	20759.0057627234\\
3.7319932998325	20743.1921275778\\
3.73209330233256	20727.4357882117\\
3.73219330483262	20711.6221530661\\
3.73229330733268	20695.8085179205\\
3.73239330983275	20680.0521785544\\
3.73249331233281	20664.2385434088\\
3.73259331483287	20648.4822040427\\
3.73269331733293	20632.7258646766\\
3.732793319833	20616.9695253105\\
3.73289332233306	20601.2131859444\\
3.73299332483312	20585.4568465783\\
3.73309332733318	20569.7005072122\\
3.73319332983325	20553.9441678461\\
3.73329333233331	20538.2451242595\\
3.73339333483337	20522.4887848934\\
3.73349333733343	20506.7897413068\\
3.7335933398335	20491.0906977202\\
3.73369334233356	20475.3916541336\\
3.73379334483362	20459.6926105471\\
3.73389334733368	20443.9935669605\\
3.73399334983375	20428.2945233739\\
3.73409335233381	20412.6527755668\\
3.73419335483387	20396.9537319802\\
3.73429335733393	20381.3119841732\\
3.734393359834	20365.6129405866\\
3.73449336233406	20349.9711927795\\
3.73459336483412	20334.3294449724\\
3.73469336733418	20318.6876971654\\
3.73479336983425	20303.0459493583\\
3.73489337233431	20287.4614973307\\
3.73499337483437	20271.8197495237\\
3.73509337733443	20256.1780017166\\
3.7351933798345	20240.593549689\\
3.73529338233456	20225.0090976615\\
3.73539338483462	20209.4246456339\\
3.73549338733468	20193.7828978268\\
3.73559338983475	20178.1984457993\\
3.73569339233481	20162.6712895512\\
3.73579339483487	20147.0868375237\\
3.73589339733493	20131.5023854961\\
3.735993399835	20115.9752292481\\
3.73609340233506	20100.3907772205\\
3.73619340483512	20084.8636209725\\
3.73629340733518	20069.3364647244\\
3.73639340983525	20053.8093084764\\
3.73649341233531	20038.2821522283\\
3.73659341483537	20022.7549959803\\
3.73669341733543	20007.2278397322\\
3.7367934198355	19991.7006834842\\
3.73689342233556	19976.2308230157\\
3.73699342483562	19960.7036667676\\
3.73709342733568	19945.2338062991\\
3.73719342983575	19929.7639458306\\
3.73729343233581	19914.294085362\\
3.73739343483587	19898.8242248935\\
3.73749343733593	19883.354364425\\
3.737593439836	19867.8845039564\\
3.73769344233606	19852.4719392674\\
3.73779344483612	19837.0020787989\\
3.73789344733618	19821.5895141099\\
3.73799344983625	19806.1196536413\\
3.73809345233631	19790.7070889523\\
3.73819345483637	19775.2945242633\\
3.73829345733643	19759.8819595743\\
3.7383934598365	19744.4693948852\\
3.73849346233656	19729.0568301962\\
3.73859346483662	19713.7015612867\\
3.73869346733668	19698.2889965977\\
3.73879346983675	19682.9337276882\\
3.73889347233681	19667.5211629992\\
3.73899347483687	19652.1658940897\\
3.73909347733693	19636.8106251802\\
3.739193479837	19621.4553562707\\
3.73929348233706	19606.1000873612\\
3.73939348483712	19590.7448184516\\
3.73949348733718	19575.4468453217\\
3.73959348983725	19560.0915764121\\
3.73969349233731	19544.7936032822\\
3.73979349483737	19529.4383343726\\
3.73989349733743	19514.1403612427\\
3.7399934998375	19498.8423881127\\
3.74009350233756	19483.5444149827\\
3.74019350483762	19468.2464418527\\
3.74029350733768	19452.9484687227\\
3.74039350983775	19437.7077913722\\
3.74049351233781	19422.4098182422\\
3.74059351483787	19407.1691408917\\
3.74069351733793	19391.8711677617\\
3.740793519838	19376.6304904113\\
3.74089352233806	19361.3898130608\\
3.74099352483812	19346.1491357103\\
3.74109352733818	19330.9084583598\\
3.74119352983825	19315.6677810093\\
3.74129353233831	19300.4843994384\\
3.74139353483837	19285.2437220879\\
3.74149353733843	19270.0603405169\\
3.7415935398385	19254.8196631664\\
3.74169354233856	19239.6362815955\\
3.74179354483862	19224.4529000245\\
3.74189354733868	19209.2695184535\\
3.74199354983875	19194.0861368826\\
3.74209355233881	19178.9027553116\\
3.74219355483887	19163.7766695202\\
3.74229355733893	19148.5932879492\\
3.742393559839	19133.4672021577\\
3.74249356233906	19118.2838205868\\
3.74259356483912	19103.1577347953\\
3.74269356733918	19088.0316490039\\
3.74279356983925	19072.9055632124\\
3.74289357233931	19057.779477421\\
3.74299357483937	19042.6533916295\\
3.74309357733943	19027.5846016176\\
3.7431935798395	19012.4585158261\\
3.74329358233956	18997.3324300347\\
3.74339358483962	18982.2636400227\\
3.74349358733968	18967.1948500108\\
3.74359358983975	18952.1260599988\\
3.74369359233981	18937.0572699869\\
3.74379359483987	18921.9884799749\\
3.74389359733993	18906.919689963\\
3.74399359984	18891.8508999511\\
3.74409360234006	18876.8394057186\\
3.74419360484012	18861.7706157067\\
3.74429360734018	18846.7591214743\\
3.74439360984025	18831.7476272418\\
3.74449361234031	18816.6788372299\\
3.74459361484037	18801.6673429975\\
3.74469361734043	18786.6558487651\\
3.7447936198405	18771.7016503121\\
3.74489362234056	18756.6901560797\\
3.74499362484062	18741.6786618473\\
3.74509362734068	18726.7244633944\\
3.74519362984075	18711.7702649415\\
3.74529363234081	18696.758770709\\
3.74539363484087	18681.8045722561\\
3.74549363734093	18666.8503738032\\
3.745593639841	18651.8961753503\\
3.74569364234106	18636.9419768974\\
3.74579364484112	18622.045074224\\
3.74589364734118	18607.090875771\\
3.74599364984125	18592.1366773181\\
3.74609365234131	18577.2397746447\\
3.74619365484137	18562.3428719713\\
3.74629365734143	18547.4459692979\\
3.7463936598415	18532.5490666245\\
3.74649366234156	18517.6521639511\\
3.74659366484162	18502.7552612777\\
3.74669366734168	18487.8583586043\\
3.74679366984175	18472.9614559309\\
3.74689367234181	18458.121849037\\
3.74699367484187	18443.2822421431\\
3.74709367734193	18428.3853394697\\
3.747193679842	18413.5457325759\\
3.74729368234206	18398.706125682\\
3.74739368484212	18383.8665187881\\
3.74749368734218	18369.0269118942\\
3.74759368984225	18354.1873050003\\
3.74769369234231	18339.4049938859\\
3.74779369484237	18324.565386992\\
3.74789369734243	18309.7830758777\\
3.7479936998425	18295.0007647633\\
3.74809370234256	18280.1611578694\\
3.74819370484262	18265.378846755\\
3.74829370734268	18250.5965356407\\
3.74839370984275	18235.8142245263\\
3.74849371234281	18221.0892091914\\
3.74859371484287	18206.306898077\\
3.74869371734293	18191.5818827422\\
3.748793719843	18176.7995716278\\
3.74889372234306	18162.0745562929\\
3.74899372484312	18147.3495409581\\
3.74909372734318	18132.6245256232\\
3.74919372984325	18117.8995102884\\
3.74929373234331	18103.1744949535\\
3.74939373484337	18088.4494796186\\
3.74949373734343	18073.7244642838\\
3.7495937398435	18059.0567447284\\
3.74969374234356	18044.3317293936\\
3.74979374484362	18029.6640098382\\
3.74989374734368	18014.9962902829\\
3.74999374984375	18000.3285707275\\
3.75009375234381	17985.6608511722\\
3.75019375484387	17970.9931316168\\
3.75029375734393	17956.3254120615\\
3.750393759844	17941.6576925061\\
3.75049376234406	17927.0472687303\\
3.75059376484412	17912.3795491749\\
3.75069376734418	17897.7691253991\\
3.75079376984425	17883.1587016233\\
3.75089377234431	17868.5482778474\\
3.75099377484437	17853.9378540716\\
3.75109377734443	17839.3274302957\\
3.7511937798445	17824.7170065199\\
3.75129378234456	17810.1065827441\\
3.75139378484462	17795.5534547478\\
3.75149378734468	17780.9430309719\\
3.75159378984475	17766.3899029756\\
3.75169379234481	17751.8367749793\\
3.75179379484487	17737.2263512034\\
3.75189379734493	17722.6732232071\\
3.751993799845	17708.1773909903\\
3.75209380234506	17693.624262994\\
3.75219380484512	17679.0711349977\\
3.75229380734518	17664.5180070013\\
3.75239380984525	17650.0221747845\\
3.75249381234531	17635.5263425677\\
3.75259381484537	17620.9732145714\\
3.75269381734543	17606.4773823546\\
3.7527938198455	17591.9815501378\\
3.75289382234556	17577.485717921\\
3.75299382484562	17562.9898857041\\
3.75309382734568	17548.5513492669\\
3.75319382984575	17534.05551705\\
3.75329383234581	17519.6169806127\\
3.75339383484587	17505.1211483959\\
3.75349383734593	17490.6826119586\\
3.753593839846	17476.2440755213\\
3.75369384234606	17461.805539084\\
3.75379384484612	17447.3670026468\\
3.75389384734618	17432.9284662095\\
3.75399384984625	17418.4899297722\\
3.75409385234631	17404.1086891144\\
3.75419385484637	17389.6701526771\\
3.75429385734643	17375.2889120193\\
3.7543938598465	17360.850375582\\
3.75449386234656	17346.4691349242\\
3.75459386484662	17332.0878942664\\
3.75469386734668	17317.7066536086\\
3.75479386984675	17303.3254129509\\
3.75489387234681	17289.0014680726\\
3.75499387484687	17274.6202274148\\
3.75509387734693	17260.238986757\\
3.755193879847	17245.9150418788\\
3.75529388234706	17231.5910970005\\
3.75539388484712	17217.2098563427\\
3.75549388734718	17202.8859114644\\
3.75559388984725	17188.5619665862\\
3.75569389234731	17174.2953174874\\
3.75579389484737	17159.9713726091\\
3.75589389734743	17145.6474277309\\
3.7559938998475	17131.3807786321\\
3.75609390234756	17117.0568337538\\
3.75619390484762	17102.7901846551\\
3.75629390734768	17088.5235355563\\
3.75639390984775	17074.2568864576\\
3.75649391234781	17059.9902373588\\
3.75659391484787	17045.72358826\\
3.75669391734793	17031.4569391613\\
3.756793919848	17017.1902900625\\
3.75689392234806	17002.9809367433\\
3.75699392484812	16988.7142876445\\
3.75709392734818	16974.5049343253\\
3.75719392984825	16960.295581006\\
3.75729393234831	16946.0289319073\\
3.75739393484837	16931.819578588\\
3.75749393734843	16917.6675210483\\
3.7575939398485	16903.4581677291\\
3.75769394234856	16889.2488144098\\
3.75779394484862	16875.0394610906\\
3.75789394734868	16860.8874035508\\
3.75799394984875	16846.7353460111\\
3.75809395234881	16832.5259926919\\
3.75819395484887	16818.3739351521\\
3.75829395734893	16804.2218776124\\
3.758393959849	16790.0698200727\\
3.75849396234906	16775.9177625329\\
3.75859396484912	16761.8230007727\\
3.75869396734918	16747.670943233\\
3.75879396984925	16733.5761814728\\
3.75889397234931	16719.424123933\\
3.75899397484937	16705.3293621728\\
3.75909397734943	16691.2346004126\\
3.7591939798495	16677.1398386524\\
3.75929398234956	16663.0450768922\\
3.75939398484962	16648.9503151319\\
3.75949398734968	16634.8555533717\\
3.75959398984975	16620.818087391\\
3.75969399234981	16606.7233256308\\
3.75979399484987	16592.6858596501\\
3.75989399734993	16578.5910978899\\
3.75999399985	16564.5536319092\\
3.76009400235006	16550.5161659285\\
3.76019400485012	16536.4786999478\\
3.76029400735018	16522.4412339671\\
3.76039400985025	16508.4610637659\\
3.76049401235031	16494.4235977852\\
3.76059401485037	16480.443427584\\
3.76069401735043	16466.4059616033\\
3.7607940198505	16452.4257914021\\
3.76089402235056	16438.4456212009\\
3.76099402485062	16424.4081552202\\
3.76109402735068	16410.427985019\\
3.76119402985075	16396.5051105973\\
3.76129403235081	16382.5249403961\\
3.76139403485087	16368.5447701949\\
3.76149403735093	16354.6218957732\\
3.761594039851	16340.6417255721\\
3.76169404235106	16326.7188511504\\
3.76179404485112	16312.7959767287\\
3.76189404735118	16298.873102307\\
3.76199404985125	16284.9502278853\\
3.76209405235131	16271.0273534637\\
3.76219405485137	16257.104479042\\
3.76229405735143	16243.1816046203\\
3.7623940598515	16229.3160259781\\
3.76249406235156	16215.3931515565\\
3.76259406485162	16201.5275729143\\
3.76269406735168	16187.6619942721\\
3.76279406985175	16173.7391198504\\
3.76289407235181	16159.8735412083\\
3.76299407485187	16146.0079625661\\
3.76309407735193	16132.1996797035\\
3.763194079852	16118.3341010613\\
3.76329408235206	16104.4685224191\\
3.76339408485212	16090.6602395565\\
3.76349408735218	16076.7946609143\\
3.76359408985225	16062.9863780517\\
3.76369409235231	16049.178095189\\
3.76379409485237	16035.3698123263\\
3.76389409735243	16021.5615294637\\
3.7639940998525	16007.753246601\\
3.76409410235256	15993.9449637384\\
3.76419410485262	15980.1939766552\\
3.76429410735268	15966.3856937926\\
3.76439410985275	15952.6347067095\\
3.76449411235281	15938.8837196263\\
3.76459411485287	15925.0754367637\\
3.76469411735293	15911.3244496805\\
3.764794119853	15897.5734625974\\
3.76489412235306	15883.8797712938\\
3.76499412485312	15870.1287842106\\
3.76509412735318	15856.3777971275\\
3.76519412985325	15842.6841058239\\
3.76529413235331	15828.9331187407\\
3.76539413485337	15815.2394274371\\
3.76549413735343	15801.5457361335\\
3.7655941398535	15787.8520448298\\
3.76569414235356	15774.1583535262\\
3.76579414485362	15760.4646622226\\
3.76589414735368	15746.770970919\\
3.76599414985375	15733.0772796153\\
3.76609415235381	15719.4408840912\\
3.76619415485387	15705.7471927876\\
3.76629415735393	15692.1107972635\\
3.766394159854	15678.4744017394\\
3.76649416235406	15664.8380062152\\
3.76659416485412	15651.2016106911\\
3.76669416735418	15637.565215167\\
3.76679416985425	15623.9288196429\\
3.76689417235431	15610.2924241188\\
3.76699417485437	15596.7133243742\\
3.76709417735443	15583.0769288501\\
3.7671941798545	15569.4978291055\\
3.76729418235456	15555.9187293609\\
3.76739418485462	15542.2823338368\\
3.76749418735468	15528.7032340922\\
3.76759418985475	15515.1241343476\\
3.76769419235481	15501.6023303825\\
3.76779419485487	15488.0232306379\\
3.76789419735493	15474.4441308933\\
3.767994199855	15460.9223269282\\
3.76809420235506	15447.3432271836\\
3.76819420485512	15433.8214232185\\
3.76829420735518	15420.2996192534\\
3.76839420985525	15406.7778152883\\
3.76849421235531	15393.2560113232\\
3.76859421485537	15379.7342073581\\
3.76869421735543	15366.2124033931\\
3.7687942198555	15352.7478952075\\
3.76889422235556	15339.2260912424\\
3.76899422485562	15325.7615830568\\
3.76909422735568	15312.2397790917\\
3.76919422985575	15298.7752709062\\
3.76929423235581	15285.3107627206\\
3.76939423485587	15271.846254535\\
3.76949423735593	15258.3817463494\\
3.769594239856	15244.9745339434\\
3.76969424235606	15231.5100257578\\
3.76979424485612	15218.0455175722\\
3.76989424735618	15204.6383051662\\
3.76999424985625	15191.2310927601\\
3.77009425235631	15177.7665845745\\
3.77019425485637	15164.3593721685\\
3.77029425735643	15150.9521597624\\
3.7703942598565	15137.5449473563\\
3.77049426235656	15124.1950307298\\
3.77059426485662	15110.7878183237\\
3.77069426735668	15097.3806059177\\
3.77079426985675	15084.0306892911\\
3.77089427235681	15070.6234768851\\
3.77099427485687	15057.2735602585\\
3.77109427735693	15043.923643632\\
3.771194279857	15030.5737270054\\
3.77129428235706	15017.2238103789\\
3.77139428485712	15003.8738937523\\
3.77149428735718	14990.5812729053\\
3.77159428985725	14977.2313562787\\
3.77169429235731	14963.8814396522\\
3.77179429485737	14950.5888188052\\
3.77189429735743	14937.2961979581\\
3.7719942998575	14924.0035771111\\
3.77209430235756	14910.6536604845\\
3.77219430485762	14897.418335417\\
3.77229430735768	14884.12571457\\
3.77239430985775	14870.833093723\\
3.77249431235781	14857.5404728759\\
3.77259431485787	14844.3051478084\\
3.77269431735793	14831.0125269614\\
3.772794319858	14817.7772018938\\
3.77289432235806	14804.5418768263\\
3.77299432485812	14791.3065517588\\
3.77309432735818	14778.0712266913\\
3.77319432985825	14764.8359016237\\
3.77329433235831	14751.6005765562\\
3.77339433485837	14738.3652514887\\
3.77349433735843	14725.1872222007\\
3.7735943398585	14711.9518971332\\
3.77369434235856	14698.7738678452\\
3.77379434485862	14685.5958385572\\
3.77389434735868	14672.4178092691\\
3.77399434985875	14659.2397799811\\
3.77409435235881	14646.0617506931\\
3.77419435485887	14632.8837214051\\
3.77429435735893	14619.7056921171\\
3.774394359859	14606.5849586086\\
3.77449436235906	14593.4069293206\\
3.77459436485912	14580.2861958121\\
3.77469436735918	14567.1654623036\\
3.77479436985925	14553.9874330156\\
3.77489437235931	14540.8666995071\\
3.77499437485937	14527.8032617781\\
3.77509437735943	14514.6825282696\\
3.7751943798595	14501.5617947611\\
3.77529438235956	14488.4410612526\\
3.77539438485962	14475.3776235237\\
3.77549438735968	14462.2568900152\\
3.77559438985975	14449.1934522862\\
3.77569439235981	14436.1300145572\\
3.77579439485987	14423.0665768282\\
3.77589439735993	14410.0031390992\\
3.77599439986	14396.9397013702\\
3.77609440236006	14383.8762636413\\
3.77619440486012	14370.8701216918\\
3.77629440736018	14357.8066839628\\
3.77639440986025	14344.8005420133\\
3.77649441236031	14331.7371042844\\
3.77659441486037	14318.7309623349\\
3.77669441736043	14305.7248203854\\
3.7767944198605	14292.718678436\\
3.77689442236056	14279.7125364865\\
3.77699442486062	14266.706394537\\
3.77709442736068	14253.7575483671\\
3.77719442986075	14240.7514064176\\
3.77729443236081	14227.8025602476\\
3.77739443486087	14214.7964182982\\
3.77749443736093	14201.8475721282\\
3.777594439861	14188.8987259582\\
3.77769444236106	14175.9498797883\\
3.77779444486112	14163.0010336183\\
3.77789444736118	14150.0521874484\\
3.77799444986125	14137.1033412784\\
3.77809445236131	14124.211790888\\
3.77819445486137	14111.262944718\\
3.77829445736143	14098.3713943276\\
3.7783944598615	14085.4798439371\\
3.77849446236156	14072.5309977672\\
3.77859446486162	14059.6394473767\\
3.77869446736168	14046.7478969863\\
3.77879446986175	14033.9136423754\\
3.77889447236181	14021.0220919849\\
3.77899447486187	14008.1305415945\\
3.77909447736193	13995.2962869835\\
3.779194479862	13982.4047365931\\
3.77929448236206	13969.5704819822\\
3.77939448486212	13956.7362273712\\
3.77949448736218	13943.9019727603\\
3.77959448986225	13931.0677181494\\
3.77969449236231	13918.2334635384\\
3.77979449486237	13905.3992089275\\
3.77989449736243	13892.5649543166\\
3.7799944998625	13879.7879954852\\
3.78009450236256	13866.9537408742\\
3.78019450486262	13854.1767820428\\
3.78029450736268	13841.3998232114\\
3.78039450986275	13828.62286438\\
3.78049451236281	13815.8459055486\\
3.78059451486287	13803.0689467171\\
3.78069451736293	13790.2919878857\\
3.780794519863	13777.5150290543\\
3.78089452236306	13764.7953660024\\
3.78099452486312	13752.018407171\\
3.78109452736318	13739.2987441191\\
3.78119452986325	13726.5217852877\\
3.78129453236331	13713.8021222358\\
3.78139453486337	13701.0824591839\\
3.78149453736343	13688.362796132\\
3.7815945398635	13675.6431330801\\
3.78169454236356	13662.9807658077\\
3.78179454486362	13650.2611027558\\
3.78189454736368	13637.5414397039\\
3.78199454986375	13624.8790724315\\
3.78209455236381	13612.2167051591\\
3.78219455486387	13599.5543378867\\
3.78229455736393	13586.8346748348\\
3.782394559864	13574.1723075624\\
3.78249456236406	13561.5672360695\\
3.78259456486412	13548.9048687971\\
3.78269456736418	13536.2425015247\\
3.78279456986425	13523.6374300318\\
3.78289457236431	13510.9750627595\\
3.78299457486437	13498.3699912666\\
3.78309457736443	13485.7649197737\\
3.7831945798645	13473.1025525013\\
3.78329458236456	13460.4974810084\\
3.78339458486462	13447.9497052951\\
3.78349458736468	13435.3446338022\\
3.78359458986475	13422.7395623093\\
3.78369459236481	13410.1344908164\\
3.78379459486487	13397.5867151031\\
3.78389459736493	13385.0389393897\\
3.783994599865	13372.4338678968\\
3.78409460236506	13359.8860921835\\
3.78419460486512	13347.3383164701\\
3.78429460736518	13334.7905407567\\
3.78439460986525	13322.2427650434\\
3.78449461236531	13309.7522851095\\
3.78459461486537	13297.2045093961\\
3.78469461736543	13284.6567336828\\
3.7847946198655	13272.1662537489\\
3.78489462236556	13259.6757738151\\
3.78499462486562	13247.1279981017\\
3.78509462736568	13234.6375181679\\
3.78519462986575	13222.147038234\\
3.78529463236581	13209.6565583002\\
3.78539463486587	13197.2233741458\\
3.78549463736593	13184.732894212\\
3.785594639866	13172.2424142781\\
3.78569464236606	13159.8092301238\\
3.78579464486612	13147.3760459694\\
3.78589464736618	13134.8855660356\\
3.78599464986625	13122.4523818812\\
3.78609465236631	13110.0191977269\\
3.78619465486637	13097.5860135726\\
3.78629465736643	13085.2101251977\\
3.7863946598665	13072.7769410434\\
3.78649466236656	13060.3437568891\\
3.78659466486662	13047.9678685142\\
3.78669466736668	13035.5346843599\\
3.78679466986675	13023.1587959851\\
3.78689467236681	13010.7829076102\\
3.78699467486687	12998.4070192354\\
3.78709467736693	12986.0311308606\\
3.787194679867	12973.6552424858\\
3.78729468236706	12961.2793541109\\
3.78739468486712	12948.9607615156\\
3.78749468736718	12936.5848731408\\
3.78759468986725	12924.2662805455\\
3.78769469236731	12911.8903921707\\
3.78779469486737	12899.5717995754\\
3.78789469736743	12887.25320698\\
3.7879946998675	12874.9346143847\\
3.78809470236756	12862.6160217894\\
3.78819470486762	12850.2974291941\\
3.78829470736768	12838.0361323783\\
3.78839470986775	12825.717539783\\
3.78849471236781	12813.4562429672\\
3.78859471486787	12801.1376503719\\
3.78869471736793	12788.8763535561\\
3.788794719868	12776.6150567403\\
3.78889472236806	12764.3537599245\\
3.78899472486812	12752.0924631087\\
3.78909472736818	12739.8311662929\\
3.78919472986825	12727.5698694771\\
3.78929473236831	12715.3658684408\\
3.78939473486837	12703.104571625\\
3.78949473736843	12690.9005705887\\
3.7895947398685	12678.6965695524\\
3.78969474236856	12666.4352727366\\
3.78979474486862	12654.2312717003\\
3.78989474736868	12642.027270664\\
3.78999474986875	12629.8805654073\\
3.79009475236881	12617.676564371\\
3.79019475486887	12605.4725633347\\
3.79029475736893	12593.3258580779\\
3.790394759869	12581.1218570416\\
3.79049476236906	12568.9751517849\\
3.79059476486912	12556.8284465281\\
3.79069476736918	12544.6817412713\\
3.79079476986925	12532.5350360145\\
3.79089477236931	12520.3883307578\\
3.79099477486937	12508.241625501\\
3.79109477736943	12496.0949202442\\
3.7911947798695	12484.005510767\\
3.79129478236956	12471.8588055102\\
3.79139478486962	12459.7693960329\\
3.79149478736968	12447.6226907762\\
3.79159478986975	12435.5332812989\\
3.79169479236981	12423.4438718216\\
3.79179479486987	12411.3544623444\\
3.79189479736993	12399.2650528671\\
3.79199479987	12387.2329391694\\
3.79209480237006	12375.1435296921\\
3.79219480487012	12363.1114159944\\
3.79229480737018	12351.0220065171\\
3.79239480987025	12338.9898928194\\
3.79249481237031	12326.9577791216\\
3.79259481487037	12314.9256654239\\
3.79269481737043	12302.8935517261\\
3.7927948198705	12290.8614380284\\
3.79289482237056	12278.8293243306\\
3.79299482487062	12266.7972106329\\
3.79309482737068	12254.8223927146\\
3.79319482987075	12242.7902790169\\
3.79329483237081	12230.8154610987\\
3.79339483487087	12218.8406431804\\
3.79349483737093	12206.8658252622\\
3.793594839871	12194.891007344\\
3.79369484237106	12182.9161894257\\
3.79379484487112	12170.9413715075\\
3.79389484737118	12158.9665535893\\
3.79399484987125	12146.991735671\\
3.79409485237131	12135.0742135323\\
3.79419485487137	12123.1566913936\\
3.79429485737143	12111.1818734753\\
3.7943948598715	12099.2643513366\\
3.79449486237156	12087.3468291979\\
3.79459486487162	12075.4293070592\\
3.79469486737168	12063.5117849205\\
3.79479486987175	12051.5942627817\\
3.79489487237181	12039.7340364225\\
3.79499487487187	12027.8165142838\\
3.79509487737193	12015.9562879246\\
3.795194879872	12004.0960615654\\
3.79529488237206	11992.1785394267\\
3.79539488487212	11980.3183130675\\
3.79549488737218	11968.4580867083\\
3.79559488987225	11956.597860349\\
3.79569489237231	11944.7949297694\\
3.79579489487237	11932.9347034101\\
3.79589489737243	11921.0744770509\\
3.7959948998725	11909.2715464712\\
3.79609490237256	11897.411320112\\
3.79619490487262	11885.6083895323\\
3.79629490737268	11873.8054589526\\
3.79639490987275	11862.0025283729\\
3.79649491237281	11850.1995977933\\
3.79659491487287	11838.3966672136\\
3.79669491737293	11826.5937366339\\
3.796794919873	11814.8481018337\\
3.79689492237306	11803.045171254\\
3.79699492487312	11791.2995364538\\
3.79709492737318	11779.5539016536\\
3.79719492987325	11767.7509710739\\
3.79729493237331	11756.0053362737\\
3.79739493487337	11744.2597014736\\
3.79749493737343	11732.5140666734\\
3.7975949398735	11720.8257276527\\
3.79769494237356	11709.0800928525\\
3.79779494487362	11697.3344580523\\
3.79789494737368	11685.6461190317\\
3.79799494987375	11673.957780011\\
3.79809495237381	11662.2121452108\\
3.79819495487387	11650.5238061902\\
3.79829495737393	11638.8354671695\\
3.798394959874	11627.1471281488\\
3.79849496237406	11615.4587891282\\
3.79859496487412	11603.827745887\\
3.79869496737418	11592.1394068663\\
3.79879496987425	11580.5083636252\\
3.79889497237431	11568.8200246045\\
3.79899497487437	11557.1889813633\\
3.79909497737443	11545.5579381222\\
3.7991949798745	11533.926894881\\
3.79929498237456	11522.2958516399\\
3.79939498487462	11510.6648083987\\
3.79949498737468	11499.0337651576\\
3.79959498987475	11487.4027219164\\
3.79969499237481	11475.8289744548\\
3.79979499487487	11464.1979312136\\
3.79989499737493	11452.624183752\\
3.799994999875	11441.0504362903\\
3.80009500237506	11429.4766888287\\
3.80019500487512	11417.902941367\\
3.80029500737518	11406.3291939054\\
3.80039500987525	11394.7554464438\\
3.80049501237531	11383.1816989821\\
3.80059501487537	11371.6652473\\
3.80069501737543	11360.0914998383\\
3.8007950198755	11348.5750481562\\
3.80089502237556	11337.0013006946\\
3.80099502487562	11325.4848490124\\
3.80109502737568	11313.9683973303\\
3.80119502987575	11302.4519456482\\
3.80129503237581	11290.9354939661\\
3.80139503487587	11279.4763380634\\
3.80149503737593	11267.9598863813\\
3.801595039876	11256.4434346992\\
3.80169504237606	11244.9842787966\\
3.80179504487612	11233.5251228939\\
3.80189504737618	11222.0086712118\\
3.80199504987625	11210.5495153092\\
3.80209505237631	11199.0903594066\\
3.80219505487637	11187.631203504\\
3.80229505737643	11176.2293433809\\
3.8023950598765	11164.7701874782\\
3.80249506237656	11153.3110315756\\
3.80259506487662	11141.9091714525\\
3.80269506737668	11130.5073113294\\
3.80279506987675	11119.0481554268\\
3.80289507237681	11107.6462953037\\
3.80299507487687	11096.2444351806\\
3.80309507737693	11084.8425750575\\
3.803195079877	11073.4407149344\\
3.80329508237706	11062.0388548113\\
3.80339508487712	11050.6942904677\\
3.80349508737718	11039.2924303446\\
3.80359508987725	11027.947866001\\
3.80369509237731	11016.6033016574\\
3.80379509487737	11005.2014415343\\
3.80389509737743	10993.8568771907\\
3.8039950998775	10982.5123128471\\
3.80409510237756	10971.1677485035\\
3.80419510487762	10959.8804799395\\
3.80429510737768	10948.5359155959\\
3.80439510987775	10937.1913512523\\
3.80449511237781	10925.9040826882\\
3.80459511487787	10914.5595183446\\
3.80469511737793	10903.2722497805\\
3.804795119878	10891.9849812165\\
3.80489512237806	10880.6977126524\\
3.80499512487812	10869.4104440883\\
3.80509512737818	10858.1231755242\\
3.80519512987825	10846.8932027397\\
3.80529513237831	10835.6059341756\\
3.80539513487837	10824.3186656115\\
3.80549513737843	10813.0886928269\\
3.8055951398785	10801.8587200424\\
3.80569514237856	10790.5714514783\\
3.80579514487862	10779.3414786937\\
3.80589514737868	10768.1115059092\\
3.80599514987875	10756.8815331246\\
3.80609515237881	10745.7088561196\\
3.80619515487887	10734.478883335\\
3.80629515737893	10723.2489105504\\
3.806395159879	10712.0762335454\\
3.80649516237906	10700.9035565403\\
3.80659516487912	10689.6735837558\\
3.80669516737918	10678.5009067507\\
3.80679516987925	10667.3282297457\\
3.80689517237931	10656.1555527406\\
3.80699517487937	10644.9828757356\\
3.80709517737943	10633.86749451\\
3.8071951798795	10622.694817505\\
3.80729518237956	10611.5221404999\\
3.80739518487962	10600.4067592744\\
3.80749518737968	10589.2913780489\\
3.80759518987975	10578.1759968233\\
3.80769519237981	10567.0033198183\\
3.80779519487987	10555.8879385927\\
3.80789519737993	10544.8298531467\\
3.80799519988	10533.7144719212\\
3.80809520238006	10522.5990906956\\
3.80819520488012	10511.5410052496\\
3.80829520738018	10500.4256240241\\
3.80839520988025	10489.367538578\\
3.80849521238031	10478.309453132\\
3.80859521488037	10467.1940719065\\
3.80869521738043	10456.1359864604\\
3.8087952198805	10445.0779010144\\
3.80889522238056	10434.0771113479\\
3.80899522488062	10423.0190259019\\
3.80909522738068	10411.9609404559\\
3.80919522988075	10400.9601507893\\
3.80929523238081	10389.9020653433\\
3.80939523488087	10378.9012756768\\
3.80949523738093	10367.9004860103\\
3.809595239881	10356.8996963438\\
3.80969524238106	10345.8989066773\\
3.80979524488112	10334.8981170108\\
3.80989524738118	10323.8973273443\\
3.80999524988125	10312.8965376777\\
3.81009525238131	10301.9530437907\\
3.81019525488137	10290.9522541242\\
3.81029525738143	10280.0087602372\\
3.8103952598815	10269.0652663502\\
3.81049526238156	10258.1217724632\\
3.81059526488162	10247.1782785762\\
3.81069526738168	10236.2347846892\\
3.81079526988175	10225.2912908022\\
3.81089527238181	10214.3477969152\\
3.81099527488187	10203.4043030282\\
3.81109527738193	10192.5181049208\\
3.811195279882	10181.6319068133\\
3.81129528238206	10170.6884129263\\
3.81139528488212	10159.8022148188\\
3.81149528738218	10148.9160167113\\
3.81159528988225	10138.0298186038\\
3.81169529238231	10127.1436204963\\
3.81179529488237	10116.2574223888\\
3.81189529738243	10105.4285200609\\
3.8119952998825	10094.5423219534\\
3.81209530238256	10083.7134196254\\
3.81219530488262	10072.8272215179\\
3.81229530738268	10061.99831919\\
3.81239530988275	10051.169416862\\
3.81249531238281	10040.340514534\\
3.81259531488287	10029.511612206\\
3.81269531738293	10018.6827098781\\
3.812795319883	10007.8538075501\\
3.81289532238306	9997.08220100163\\
3.81299532488312	9986.25329867366\\
3.81309532738318	9975.4816921252\\
3.81319532988325	9964.65278979723\\
3.81329533238331	9953.88118324877\\
3.81339533488337	9943.10957670031\\
3.81349533738343	9932.33797015185\\
3.8135953398835	9921.56636360339\\
3.81369534238356	9910.79475705493\\
3.81379534488362	9900.08044628598\\
3.81389534738368	9889.30883973752\\
3.81399534988375	9878.59452896858\\
3.81409535238381	9867.82292242012\\
3.81419535488387	9857.10861165117\\
3.81429535738393	9846.39430088222\\
3.814395359884	9835.67999011328\\
3.81449536238406	9824.96567934433\\
3.81459536488412	9814.25136857538\\
3.81469536738418	9803.53705780644\\
3.81479536988425	9792.88004281701\\
3.81489537238431	9782.16573204806\\
3.81499537488437	9771.50871705862\\
3.81509537738443	9760.85170206919\\
3.8151953798845	9750.13739130024\\
3.81529538238456	9739.48037631081\\
3.81539538488462	9728.82336132138\\
3.81549538738468	9718.16634633195\\
3.81559538988475	9707.56662712203\\
3.81569539238481	9696.90961213259\\
3.81579539488487	9686.25259714316\\
3.81589539738493	9675.65287793324\\
3.815995399885	9664.9958629438\\
3.81609540238506	9654.39614373388\\
3.81619540488512	9643.79642452397\\
3.81629540738518	9633.19670531405\\
3.81639540988525	9622.59698610412\\
3.81649541238531	9611.9972668942\\
3.81659541488537	9601.39754768428\\
3.81669541738543	9590.85512425388\\
3.8167954198855	9580.25540504396\\
3.81689542238556	9569.71298161355\\
3.81699542488562	9559.17055818314\\
3.81709542738568	9548.57083897322\\
3.81719542988575	9538.02841554281\\
3.81729543238581	9527.48599211241\\
3.81739543488587	9516.943568682\\
3.81749543738593	9506.45844103111\\
3.817595439886	9495.9160176007\\
3.81769544238606	9485.37359417029\\
3.81779544488612	9474.8884665194\\
3.81789544738618	9464.34604308899\\
3.81799544988625	9453.8609154381\\
3.81809545238631	9443.3757877872\\
3.81819545488637	9432.89066013631\\
3.81829545738643	9422.40553248541\\
3.8183954598865	9411.92040483452\\
3.81849546238656	9401.43527718363\\
3.81859546488662	9391.00744531225\\
3.81869546738668	9380.52231766135\\
3.81879546988675	9370.09448578997\\
3.81889547238681	9359.60935813908\\
3.81899547488687	9349.1815262677\\
3.81909547738693	9338.75369439631\\
3.819195479887	9328.32586252493\\
3.81929548238706	9317.89803065355\\
3.81939548488712	9307.47019878217\\
3.81949548738718	9297.0996626903\\
3.81959548988725	9286.67183081892\\
3.81969549238731	9276.30129472705\\
3.81979549488737	9265.87346285567\\
3.81989549738743	9255.5029267638\\
3.8199954998875	9245.13239067194\\
3.82009550238756	9234.76185458007\\
3.82019550488762	9224.3913184882\\
3.82029550738768	9214.02078239633\\
3.82039550988775	9203.65024630447\\
3.82049551238781	9193.2797102126\\
3.82059551488787	9182.96646990024\\
3.82069551738793	9172.59593380838\\
3.820795519888	9162.28269349602\\
3.82089552238806	9151.96945318366\\
3.82099552488812	9141.65621287131\\
3.82109552738818	9131.34297255896\\
3.82119552988825	9121.0297322466\\
3.82129553238831	9110.71649193425\\
3.82139553488837	9100.40325162189\\
3.82149553738843	9090.14730708905\\
3.8215955398885	9079.8340667767\\
3.82169554238856	9069.57812224385\\
3.82179554488862	9059.2648819315\\
3.82189554738868	9049.00893739866\\
3.82199554988875	9038.75299286582\\
3.82209555238881	9028.49704833297\\
3.82219555488887	9018.24110380013\\
3.82229555738893	9007.98515926729\\
3.822395559889	8997.78651051396\\
3.82249556238906	8987.53056598112\\
3.82259556488912	8977.33191722779\\
3.82269556738918	8967.07597269495\\
3.82279556988925	8956.87732394162\\
3.82289557238931	8946.67867518829\\
3.82299557488937	8936.48002643496\\
3.82309557738943	8926.28137768163\\
3.8231955798895	8916.08272892831\\
3.82329558238956	8905.94137595449\\
3.82339558488962	8895.74272720116\\
3.82349558738968	8885.54407844783\\
3.82359558988975	8875.40272547402\\
3.82369559238981	8865.2613725002\\
3.82379559488987	8855.06272374687\\
3.82389559738993	8844.92137077306\\
3.82399559989	8834.78001779924\\
3.82409560239006	8824.63866482543\\
3.82419560489012	8814.55460763112\\
3.82429560739018	8804.41325465731\\
3.82439560989025	8794.27190168349\\
3.82449561239031	8784.18784448919\\
3.82459561489037	8774.10378729489\\
3.82469561739043	8763.96243432107\\
3.8247956198905	8753.87837712677\\
3.82489562239056	8743.79431993247\\
3.82499562489062	8733.71026273816\\
3.82509562739068	8723.62620554386\\
3.82519562989075	8713.59944412907\\
3.82529563239081	8703.51538693477\\
3.82539563489087	8693.43132974047\\
3.82549563739093	8683.40456832568\\
3.825595639891	8673.37780691089\\
3.82569564239106	8663.29374971659\\
3.82579564489112	8653.2669883018\\
3.82589564739118	8643.24022688701\\
3.82599564989125	8633.21346547222\\
3.82609565239131	8623.18670405743\\
3.82619565489137	8613.21723842215\\
3.82629565739143	8603.19047700736\\
3.8263956598915	8593.22101137209\\
3.82649566239156	8583.1942499573\\
3.82659566489162	8573.22478432202\\
3.82669566739168	8563.25531868674\\
3.82679566989175	8553.28585305147\\
3.82689567239181	8543.31638741619\\
3.82699567489187	8533.34692178092\\
3.82709567739193	8523.37745614564\\
3.827195679892	8513.40799051036\\
3.82729568239206	8503.4958206546\\
3.82739568489212	8493.52635501932\\
3.82749568739218	8483.61418516356\\
3.82759568989225	8473.7020153078\\
3.82769569239231	8463.78984545203\\
3.82779569489237	8453.87767559627\\
3.82789569739243	8443.96550574051\\
3.8279956998925	8434.05333588474\\
3.82809570239256	8424.14116602898\\
3.82819570489262	8414.28629195273\\
3.82829570739268	8404.37412209697\\
3.82839570989275	8394.51924802072\\
3.82849571239281	8384.60707816495\\
3.82859571489287	8374.75220408871\\
3.82869571739294	8364.89733001245\\
3.828795719893	8355.04245593621\\
3.82889572239306	8345.18758185995\\
3.82899572489312	8335.3327077837\\
3.82909572739318	8325.53512948697\\
3.82919572989325	8315.68025541072\\
3.82929573239331	8305.88267711398\\
3.82939573489337	8296.02780303773\\
3.82949573739343	8286.23022474099\\
3.8295957398935	8276.43264644425\\
3.82969574239356	8266.63506814752\\
3.82979574489362	8256.83748985078\\
3.82989574739368	8247.03991155404\\
3.82999574989375	8237.24233325731\\
3.83009575239381	8227.50205074008\\
3.83019575489387	8217.70447244334\\
3.83029575739393	8207.96418992612\\
3.830395759894	8198.16661162939\\
3.83049576239406	8188.42632911216\\
3.83059576489412	8178.68604659494\\
3.83069576739419	8168.94576407771\\
3.83079576989425	8159.20548156049\\
3.83089577239431	8149.46519904326\\
3.83099577489437	8139.72491652604\\
3.83109577739443	8130.04192978833\\
3.8311957798945	8120.30164727111\\
3.83129578239456	8110.61866053339\\
3.83139578489462	8100.93567379568\\
3.83149578739468	8091.25268705797\\
3.83159578989475	8081.51240454075\\
3.83169579239481	8071.82941780304\\
3.83179579489487	8062.20372684484\\
3.83189579739493	8052.52074010713\\
3.831995799895	8042.83775336942\\
3.83209580239506	8033.21206241122\\
3.83219580489512	8023.52907567351\\
3.83229580739518	8013.90338471531\\
3.83239580989525	8004.27769375711\\
3.83249581239531	7994.5947070194\\
3.83259581489537	7984.96901606121\\
3.83269581739544	7975.34332510301\\
3.8327958198955	7965.77492992432\\
3.83289582239556	7956.14923896612\\
3.83299582489562	7946.52354800793\\
3.83309582739568	7936.95515282924\\
3.83319582989575	7927.32946187105\\
3.83329583239581	7917.76106669236\\
3.83339583489587	7908.19267151367\\
3.83349583739593	7898.62427633499\\
3.833595839896	7889.05588115631\\
3.83369584239606	7879.48748597762\\
3.83379584489612	7869.91909079894\\
3.83389584739618	7860.35069562025\\
3.83399584989625	7850.83959622108\\
3.83409585239631	7841.27120104239\\
3.83419585489637	7831.76010164322\\
3.83429585739643	7822.19170646454\\
3.8343958598965	7812.68060706537\\
3.83449586239656	7803.1695076662\\
3.83459586489662	7793.65840826702\\
3.83469586739669	7784.14730886785\\
3.83479586989675	7774.69350524819\\
3.83489587239681	7765.18240584902\\
3.83499587489687	7755.67130644985\\
3.83509587739694	7746.21750283019\\
3.835195879897	7736.76369921053\\
3.83529588239706	7727.25259981136\\
3.83539588489712	7717.7987961917\\
3.83549588739718	7708.34499257204\\
3.83559588989725	7698.89118895239\\
3.83569589239731	7689.43738533273\\
3.83579589489737	7680.04087749258\\
3.83589589739743	7670.58707387292\\
3.8359958998975	7661.19056603278\\
3.83609590239756	7651.73676241312\\
3.83619590489762	7642.34025457297\\
3.83629590739768	7632.94374673283\\
3.83639590989775	7623.54723889268\\
3.83649591239781	7614.15073105254\\
3.83659591489787	7604.75422321239\\
3.83669591739794	7595.35771537225\\
3.836795919898	7585.9612075321\\
3.83689592239806	7576.62199547147\\
3.83699592489812	7567.22548763132\\
3.83709592739819	7557.88627557069\\
3.83719592989825	7548.48976773054\\
3.83729593239831	7539.15055566991\\
3.83739593489837	7529.81134360928\\
3.83749593739843	7520.47213154865\\
3.8375959398985	7511.13291948801\\
3.83769594239856	7501.85100320689\\
3.83779594489862	7492.51179114626\\
3.83789594739868	7483.17257908563\\
3.83799594989875	7473.89066280451\\
3.83809595239881	7464.60874652339\\
3.83819595489887	7455.26953446276\\
3.83829595739893	7445.98761818164\\
3.838395959899	7436.70570190052\\
3.83849596239906	7427.4237856194\\
3.83859596489912	7418.14186933828\\
3.83869596739919	7408.91724883667\\
3.83879596989925	7399.63533255556\\
3.83889597239931	7390.41071205395\\
3.83899597489937	7381.12879577283\\
3.83909597739944	7371.90417527122\\
3.8391959798995	7362.67955476962\\
3.83929598239956	7353.45493426801\\
3.83939598489962	7344.23031376641\\
3.83949598739969	7335.0056932648\\
3.83959598989975	7325.78107276319\\
3.83969599239981	7316.55645226159\\
3.83979599489987	7307.38912753949\\
3.83989599739993	7298.16450703789\\
3.8399959999	7288.99718231579\\
3.84009600240006	7279.8298575937\\
3.84019600490012	7270.60523709209\\
3.84029600740018	7261.43791237\\
3.84039600990025	7252.27058764791\\
3.84049601240031	7243.10326292582\\
3.84059601490037	7233.99323398324\\
3.84069601740044	7224.82590926114\\
3.8407960199005	7215.65858453905\\
3.84089602240056	7206.54855559647\\
3.84099602490062	7197.43852665389\\
3.84109602740069	7188.2712019318\\
3.84119602990075	7179.16117298921\\
3.84129603240081	7170.05114404664\\
3.84139603490087	7160.94111510405\\
3.84149603740094	7151.88838194099\\
3.841596039901	7142.77835299841\\
3.84169604240106	7133.66832405583\\
3.84179604490112	7124.61559089276\\
3.84189604740118	7115.50556195018\\
3.84199604990125	7106.45282878711\\
3.84209605240131	7097.40009562405\\
3.84219605490137	7088.34736246098\\
3.84229605740143	7079.29462929791\\
3.8423960599015	7070.24189613485\\
3.84249606240156	7061.18916297178\\
3.84259606490162	7052.13642980871\\
3.84269606740169	7043.14099242516\\
3.84279606990175	7034.08825926209\\
3.84289607240181	7025.09282187854\\
3.84299607490187	7016.09738449498\\
3.84309607740194	7007.04465133192\\
3.843196079902	6998.04921394836\\
3.84329608240206	6989.05377656481\\
3.84339608490212	6980.05833918125\\
3.84349608740219	6971.12019757721\\
3.84359608990225	6962.12476019366\\
3.84369609240231	6953.12932281011\\
3.84379609490237	6944.19118120606\\
3.84389609740243	6935.25303960202\\
3.8439960999025	6926.25760221847\\
3.84409610240256	6917.31946061443\\
3.84419610490262	6908.38131901039\\
3.84429610740268	6899.44317740635\\
3.84439610990275	6890.50503580231\\
3.84449611240281	6881.62418997778\\
3.84459611490287	6872.68604837374\\
3.84469611740294	6863.80520254921\\
3.844796119903	6854.86706094517\\
3.84489612240306	6845.98621512064\\
3.84499612490312	6837.10536929611\\
3.84509612740319	6828.22452347159\\
3.84519612990325	6819.34367764706\\
3.84529613240331	6810.46283182253\\
3.84539613490337	6801.581985998\\
3.84549613740344	6792.70114017348\\
3.8455961399035	6783.87759012846\\
3.84569614240356	6774.99674430393\\
3.84579614490362	6766.17319425892\\
3.84589614740369	6757.29234843439\\
3.84599614990375	6748.46879838937\\
3.84609615240381	6739.64524834436\\
3.84619615490387	6730.82169829935\\
3.84629615740393	6721.99814825433\\
3.846396159904	6713.23189398883\\
3.84649616240406	6704.40834394382\\
3.84659616490412	6695.5847938988\\
3.84669616740419	6686.8185396333\\
3.84679616990425	6677.99498958828\\
3.84689617240431	6669.22873532278\\
3.84699617490437	6660.46248105728\\
3.84709617740444	6651.69622679178\\
3.8471961799045	6642.92997252628\\
3.84729618240456	6634.16371826078\\
3.84739618490462	6625.39746399527\\
3.84749618740469	6616.68850550929\\
3.84759618990475	6607.92225124378\\
3.84769619240481	6599.2132927578\\
3.84779619490487	6590.50433427181\\
3.84789619740494	6581.73808000631\\
3.847996199905	6573.02912152032\\
3.84809620240506	6564.32016303433\\
3.84819620490512	6555.61120454834\\
3.84829620740518	6546.90224606235\\
3.84839620990525	6538.25058335588\\
3.84849621240531	6529.54162486989\\
3.84859621490537	6520.88996216341\\
3.84869621740544	6512.18100367742\\
3.8487962199055	6503.52934097095\\
3.84889622240556	6494.87767826447\\
3.84899622490562	6486.226015558\\
3.84909622740569	6477.57435285152\\
3.84919622990575	6468.92269014505\\
3.84929623240581	6460.27102743857\\
3.84939623490587	6451.6193647321\\
3.84949623740594	6443.02499780513\\
3.849596239906	6434.37333509866\\
3.84969624240606	6425.7789681717\\
3.84979624490612	6417.18460124473\\
3.84989624740619	6408.53293853826\\
3.84999624990625	6399.9385716113\\
3.85009625240631	6391.34420468433\\
3.85019625490637	6382.80713353688\\
3.85029625740643	6374.21276660992\\
3.8503962599065	6365.61839968296\\
3.85049626240656	6357.08132853551\\
3.85059626490662	6348.48696160855\\
3.85069626740669	6339.9498904611\\
3.85079626990675	6331.41281931365\\
3.85089627240681	6322.81845238669\\
3.85099627490687	6314.28138123924\\
3.85109627740694	6305.74431009179\\
3.851196279907	6297.26453472385\\
3.85129628240706	6288.7274635764\\
3.85139628490712	6280.19039242895\\
3.85149628740719	6271.71061706102\\
3.85159628990725	6263.17354591357\\
3.85169629240731	6254.69377054563\\
3.85179629490737	6246.2139951777\\
3.85189629740744	6237.73421980976\\
3.8519962999075	6229.25444444182\\
3.85209630240756	6220.77466907389\\
3.85219630490762	6212.29489370595\\
3.85229630740769	6203.81511833802\\
3.85239630990775	6195.39263874959\\
3.85249631240781	6186.91286338165\\
3.85259631490787	6178.49038379323\\
3.85269631740794	6170.0106084253\\
3.852796319908	6161.58812883687\\
3.85289632240806	6153.16564924845\\
3.85299632490812	6144.74316966003\\
3.85309632740819	6136.3206900716\\
3.85319632990825	6127.95550626269\\
3.85329633240831	6119.53302667427\\
3.85339633490837	6111.11054708585\\
3.85349633740844	6102.74536327694\\
3.8535963399085	6094.32288368851\\
3.85369634240856	6085.9576998796\\
3.85379634490862	6077.59251607069\\
3.85389634740869	6069.22733226178\\
3.85399634990875	6060.86214845287\\
3.85409635240881	6052.49696464396\\
3.85419635490887	6044.13178083505\\
3.85429635740894	6035.82389280566\\
3.854396359909	6027.45870899675\\
3.85449636240906	6019.15082096735\\
3.85459636490912	6010.78563715844\\
3.85469636740919	6002.47774912904\\
3.85479636990925	5994.16986109965\\
3.85489637240931	5985.86197307025\\
3.85499637490937	5977.55408504085\\
3.85509637740944	5969.24619701146\\
3.8551963799095	5960.99560476157\\
3.85529638240956	5952.68771673218\\
3.85539638490962	5944.37982870278\\
3.85549638740969	5936.12923645289\\
3.85559638990975	5927.87864420301\\
3.85569639240981	5919.62805195313\\
3.85579639490987	5911.32016392373\\
3.85589639740994	5903.06957167385\\
3.85599639991	5894.81897942396\\
3.85609640241006	5886.62568295359\\
3.85619640491012	5878.37509070371\\
3.85629640741019	5870.12449845382\\
3.85639640991025	5861.93120198345\\
3.85649641241031	5853.73790551308\\
3.85659641491037	5845.4873132632\\
3.85669641741044	5837.29401679283\\
3.8567964199105	5829.10072032246\\
3.85689642241056	5820.90742385209\\
3.85699642491062	5812.71412738172\\
3.85709642741069	5804.52083091135\\
3.85719642991075	5796.38483022049\\
3.85729643241081	5788.19153375012\\
3.85739643491087	5780.05553305926\\
3.85749643741094	5771.86223658889\\
3.857596439911	5763.72623589803\\
3.85769644241106	5755.59023520717\\
3.85779644491112	5747.45423451631\\
3.85789644741119	5739.31823382546\\
3.85799644991125	5731.1822331346\\
3.85809645241131	5723.05769159964\\
3.85819645491137	5714.93887964264\\
3.85829645741144	5706.83152684154\\
3.8583964599115	5698.72990361839\\
3.85849646241156	5690.62828039524\\
3.85859646491162	5682.53811632799\\
3.85869646741169	5674.44795226074\\
3.85879646991175	5666.3692473494\\
3.85889647241181	5658.29627201601\\
3.85899647491187	5650.22329668261\\
3.85909647741194	5642.16178050512\\
3.859196479912	5634.10599390558\\
3.85929648241206	5626.055936884\\
3.85939648491212	5618.00587986241\\
3.85949648741219	5609.96728199672\\
3.85959648991225	5601.93441370899\\
3.85969649241231	5593.9072749992\\
3.85979649491237	5585.88586586737\\
3.85989649741244	5577.87018631349\\
3.8599964999125	5569.86023633756\\
3.86009650241256	5561.85601593959\\
3.86019650491262	5553.85752511956\\
3.86029650741269	5545.86476387748\\
3.86039650991275	5537.87773221336\\
3.86049651241281	5529.89643012719\\
3.86059651491287	5521.92085761897\\
3.86069651741294	5513.9510146887\\
3.860796519913	5505.98690133638\\
3.86089652241306	5498.02851756201\\
3.86099652491312	5490.0758633656\\
3.86109652741319	5482.12893874713\\
3.86119652991325	5474.19347328457\\
3.86129653241331	5466.25800782201\\
3.86139653491337	5458.3282719374\\
3.86149653741344	5450.40426563074\\
3.8615965399135	5442.49171847998\\
3.86169654241356	5434.57917132923\\
3.86179654491362	5426.67235375642\\
3.86189654741369	5418.77699533952\\
3.86199654991375	5410.88163692261\\
3.86209655241381	5402.99200808366\\
3.86219655491387	5395.11383840061\\
3.86229655741394	5387.23566871757\\
3.862396559914	5379.36895819042\\
3.86249656241406	5371.50224766327\\
3.86259656491412	5363.64699629203\\
3.86269656741419	5355.79174492079\\
3.86279656991425	5347.94795270545\\
3.86289657241431	5340.1041604901\\
3.86299657491437	5332.27182743067\\
3.86309657741444	5324.44522394918\\
3.8631965799145	5316.61862046769\\
3.86329658241456	5308.80347614211\\
3.86339658491462	5300.99406139447\\
3.86349658741469	5293.18464664684\\
3.86359658991475	5285.38669105511\\
3.86369659241481	5277.59446504133\\
3.86379659491487	5269.8079686055\\
3.86389659741494	5262.02147216968\\
3.863996599915	5254.24643488975\\
3.86409660241506	5246.47712718778\\
3.86419660491512	5238.71354906375\\
3.86429660741519	5230.95570051768\\
3.86439660991525	5223.20358154956\\
3.86449661241531	5215.45719215939\\
3.86459661491537	5207.71080276922\\
3.86469661741544	5199.97587253496\\
3.8647966199155	5192.24667187864\\
3.86489662241556	5184.52320080028\\
3.86499662491562	5176.80545929987\\
3.86509662741569	5169.09344737741\\
3.86519662991575	5161.39289461085\\
3.86529663241581	5153.69234184429\\
3.86539663491587	5145.99751865568\\
3.86549663741594	5138.30842504503\\
3.865596639916	5130.62506101232\\
3.86569664241606	5122.94742655757\\
3.86579664491612	5115.27552168077\\
3.86589664741619	5107.61507595987\\
3.86599664991625	5099.95463023897\\
3.86609665241631	5092.29991409602\\
3.86619665491637	5084.65092753103\\
3.86629665741644	5077.01340012193\\
3.8663966599165	5069.37587271284\\
3.86649666241656	5061.7440748817\\
3.86659666491662	5054.11800662851\\
3.86669666741669	5046.50339753122\\
3.86679666991675	5038.88878843393\\
3.86689667241681	5031.28563849254\\
3.86699667491687	5023.68248855116\\
3.86709667741694	5016.08506818772\\
3.867196679917	5008.49910698019\\
3.86729668241706	5000.91314577266\\
3.86739668491712	4993.33864372103\\
3.86749668741719	4985.7641416694\\
3.86759668991725	4978.20109877367\\
3.86769669241731	4970.6437854559\\
3.86779669491737	4963.08647213812\\
3.86789669741744	4955.54061797625\\
3.8679966999175	4947.99476381437\\
3.86809670241756	4940.4603688084\\
3.86819670491762	4932.93170338038\\
3.86829670741769	4925.40303795237\\
3.86839670991775	4917.88583168025\\
3.86849671241781	4910.37435498608\\
3.86859671491787	4902.86287829192\\
3.86869671741794	4895.36286075366\\
3.868796719918	4887.86857279335\\
3.86889672241806	4880.38001441099\\
3.86899672491812	4872.89145602862\\
3.86909672741819	4865.41435680217\\
3.86919672991825	4857.94298715366\\
3.86929673241831	4850.47734708311\\
3.86939673491837	4843.0174365905\\
3.86949673741844	4835.56325567585\\
3.8695967399185	4828.11480433915\\
3.86969674241856	4820.6720825804\\
3.86979674491862	4813.2350903996\\
3.86989674741869	4805.80382779676\\
3.86999674991875	4798.37829477186\\
3.87009675241881	4790.95849132492\\
3.87019675491887	4783.54441745593\\
3.87029675741894	4776.13607316488\\
3.870396759919	4768.73345845179\\
3.87049676241906	4761.33657331665\\
3.87059676491912	4753.94541775947\\
3.87069676741919	4746.55999178023\\
3.87079676991925	4739.18029537894\\
3.87089677241931	4731.80632855561\\
3.87099677491937	4724.43809131023\\
3.87109677741944	4717.08131322075\\
3.8711967799195	4709.72453513127\\
3.87129678241956	4702.37348661974\\
3.87139678491962	4695.02816768616\\
3.87149678741969	4687.68857833054\\
3.87159678991975	4680.36044813082\\
3.87169679241981	4673.03231793109\\
3.87179679491987	4665.70991730932\\
3.87189679741994	4658.39897584345\\
3.87199679992	4651.08803437758\\
3.87209680242006	4643.78282248966\\
3.87219680492012	4636.48906975765\\
3.87229680742019	4629.19531702563\\
3.87239680992025	4621.90729387157\\
3.87249681242031	4614.63072987341\\
3.87259681492037	4607.35416587525\\
3.87269681742044	4600.08906103299\\
3.8727968199205	4592.82395619073\\
3.87289682242056	4585.57031050437\\
3.87299682492062	4578.31666481801\\
3.87309682742069	4571.07447828756\\
3.87319682992075	4563.83229175711\\
3.87329683242081	4556.60156438256\\
3.87339683492087	4549.37083700801\\
3.87349683742094	4542.15156878936\\
3.873596839921	4534.93803014866\\
3.87369684242106	4527.72449150796\\
3.87379684492112	4520.52241202317\\
3.87389684742119	4513.32033253837\\
3.87399684992125	4506.12971220948\\
3.87409685242131	4498.94482145854\\
3.87419685492137	4491.76566028555\\
3.87429685742144	4484.58649911256\\
3.8743968599215	4477.41879709548\\
3.87449686242156	4470.25682465634\\
3.87459686492162	4463.10058179516\\
3.87469686742169	4455.94433893397\\
3.87479686992175	4448.79955522869\\
3.87489687242181	4441.66050110136\\
3.87499687492187	4434.52717655198\\
3.87509687742194	4427.39958158056\\
3.875196879922	4420.27198660913\\
3.87529688242206	4413.1558507936\\
3.87539688492212	4406.04544455603\\
3.87549688742219	4398.94076789641\\
3.87559688992225	4391.84182081474\\
3.87569689242231	4384.74860331102\\
3.87579689492237	4377.66111538525\\
3.87589689742244	4370.57935703743\\
3.8759968999225	4363.50332826757\\
3.87609690242256	4356.43302907565\\
3.87619690492262	4349.36845946169\\
3.87629690742269	4342.30961942568\\
3.87639690992275	4335.25650896762\\
3.87649691242281	4328.20912808751\\
3.87659691492287	4321.16747678535\\
3.87669691742294	4314.13155506114\\
3.876796919923	4307.10136291489\\
3.87689692242306	4300.07690034658\\
3.87699692492312	4293.05816735623\\
3.87709692742319	4286.05089352178\\
3.87719692992325	4279.04361968733\\
3.87729693242331	4272.04207543083\\
3.87739693492337	4265.04626075229\\
3.87749693742344	4258.05617565169\\
3.8775969399235	4251.077549707\\
3.87769694242356	4244.0989237623\\
3.87779694492362	4237.12602739556\\
3.87789694742369	4230.15886060677\\
3.87799694992375	4223.20315297388\\
3.87809695242381	4216.24744534099\\
3.87819695492387	4209.29746728606\\
3.87829695742394	4202.35894838702\\
3.878396959924	4195.42042948799\\
3.87849696242406	4188.48764016691\\
3.87859696492412	4181.56631000173\\
3.87869696742419	4174.64497983654\\
3.87879696992425	4167.72937924932\\
3.87889697242431	4160.82523781799\\
3.87899697492437	4153.92109638666\\
3.87909697742444	4147.02841411124\\
3.8791969799245	4140.13573183582\\
3.87929698242456	4133.25450871629\\
3.87939698492462	4126.37328559677\\
3.87949698742469	4119.4977920552\\
3.87959698992475	4112.63375766954\\
3.87969699242481	4105.76972328387\\
3.87979699492487	4098.9171480541\\
3.87989699742494	4092.07030240229\\
3.879996999925	4085.22345675048\\
3.88009700242506	4078.38807025457\\
3.88019700492512	4071.55268375866\\
3.88029700742519	4064.72875641865\\
3.88039700992525	4057.90482907864\\
3.88049701242531	4051.09236089453\\
3.88059701492537	4044.28562228838\\
3.88069701742544	4037.47888368223\\
3.8807970199255	4030.68360423197\\
3.88089702242556	4023.89405435967\\
3.88099702492562	4017.10450448737\\
3.88109702742569	4010.32641377098\\
3.88119702992575	4003.55405263253\\
3.88129703242581	3996.78169149408\\
3.88139703492587	3990.02078951154\\
3.88149703742594	3983.26561710695\\
3.881597039926	3976.51617428031\\
3.88169704242606	3969.77246103162\\
3.88179704492612	3963.02874778293\\
3.88189704742619	3956.29649369014\\
3.88199704992625	3949.5699691753\\
3.88209705242631	3942.84917423842\\
3.88219705492637	3936.13410887949\\
3.88229705742644	3929.41904352055\\
3.8823970599265	3922.71543731752\\
3.88249706242656	3916.01756069244\\
3.88259706492662	3909.32541364531\\
3.88269706742669	3902.63899617614\\
3.88279706992675	3895.95830828491\\
3.88289707242681	3889.28334997164\\
3.88299707492687	3882.61412123632\\
3.88309707742694	3875.95062207894\\
3.883197079927	3869.29285249952\\
3.88329708242706	3862.64081249806\\
3.88339708492712	3855.99450207454\\
3.88349708742719	3849.35392122897\\
3.88359708992725	3842.71906996136\\
3.88369709242731	3836.08994827169\\
3.88379709492737	3829.46655615998\\
3.88389709742744	3822.84889362622\\
3.8839970999275	3816.23696067041\\
3.88409710242756	3809.63075729255\\
3.88419710492762	3803.03028349264\\
3.88429710742769	3796.43553927069\\
3.88439710992775	3789.84652462668\\
3.88449711242781	3783.26323956063\\
3.88459711492787	3776.68568407253\\
3.88469711742794	3770.11385816238\\
3.884797119928	3763.55349140813\\
3.88489712242806	3756.99312465388\\
3.88499712492812	3750.43848747759\\
3.88509712742819	3743.88957987924\\
3.88519712992825	3737.34640185885\\
3.88529713242831	3730.8089534164\\
3.88539713492837	3724.28296412986\\
3.88549713742844	3717.75697484332\\
3.8855971399285	3711.23671513474\\
3.88569714242856	3704.7221850041\\
3.88579714492862	3698.21911402936\\
3.88589714742869	3691.71604305463\\
3.88599714992875	3685.21870165784\\
3.88609715242881	3678.72708983901\\
3.88619715492887	3672.24693717608\\
3.88629715742894	3665.76678451315\\
3.886397159929	3659.29236142817\\
3.88649716242906	3652.8293974991\\
3.88659716492912	3646.36643357002\\
3.88669716742919	3639.9091992189\\
3.88679716992925	3633.46342402368\\
3.88689717242931	3627.01764882846\\
3.88699717492937	3620.58333278914\\
3.88709717742944	3614.14901674982\\
3.8871971799295	3607.72043028845\\
3.88729718242956	3601.30330298298\\
3.88739718492962	3594.88617567752\\
3.88749718742969	3588.48050752796\\
3.88759718992975	3582.07483937839\\
3.88769719242981	3575.67490080678\\
3.88779719492987	3569.28642139107\\
3.88789719742994	3562.89794197537\\
3.88799719993	3556.52092171556\\
3.88809720243006	3550.14390145575\\
3.88819720493012	3543.77834035185\\
3.88829720743019	3537.41277924795\\
3.88839720993025	3531.05867729994\\
3.88849721243031	3524.70457535194\\
3.88859721493037	3518.36193255985\\
3.88869721743044	3512.0250193457\\
3.8887972199305	3505.68810613155\\
3.88889722243056	3499.36265207331\\
3.88899722493062	3493.03719801506\\
3.88909722743069	3486.72320311272\\
3.88919722993075	3480.40920821038\\
3.88929723243081	3474.10667246394\\
3.88939723493087	3467.80986629545\\
3.88949723743094	3461.51306012697\\
3.889597239931	3455.22771311438\\
3.88969724243106	3448.94809567975\\
3.88979724493112	3442.66847824511\\
3.88989724743119	3436.40031996638\\
3.88999724993125	3430.1378912656\\
3.89009725243131	3423.87546256482\\
3.89019725493137	3417.62449301994\\
3.89029725743144	3411.37925305302\\
3.8903972599315	3405.13401308609\\
3.89049726243156	3398.90023227507\\
3.89059726493162	3392.672181042\\
3.89069726743169	3386.44985938688\\
3.89079726993175	3380.22753773176\\
3.89089727243181	3374.01667523254\\
3.89099727493187	3367.81154231127\\
3.89109727743194	3361.61213896796\\
3.891197279932	3355.41273562464\\
3.89129728243206	3349.22479143723\\
3.89139728493212	3343.04257682777\\
3.89149728743219	3336.86609179626\\
3.89159728993225	3330.68960676474\\
3.89169729243231	3324.52458088914\\
3.89179729493237	3318.36528459148\\
3.89189729743244	3312.21171787178\\
3.8919972999325	3306.06388073002\\
3.89209730243256	3299.92177316622\\
3.89219730493262	3293.78539518037\\
3.89229730743269	3287.64901719452\\
3.89239730993275	3281.52409836457\\
3.89249731243281	3275.40490911257\\
3.89259731493287	3269.29144943853\\
3.89269731743294	3263.18371934243\\
3.892797319933	3257.08171882429\\
3.89289732243306	3250.9854478841\\
3.89299732493312	3244.89490652186\\
3.89309732743319	3238.81009473757\\
3.89319732993325	3232.73101253123\\
3.89329733243331	3226.65765990284\\
3.89339733493337	3220.58430727446\\
3.89349733743344	3214.52241380197\\
3.8935973399335	3208.46624990744\\
3.89369734243356	3202.41581559086\\
3.89379734493362	3196.37111085223\\
3.89389734743369	3190.33213569155\\
3.89399734993375	3184.29889010882\\
3.89409735243381	3178.27137410404\\
3.89419735493387	3172.24958767722\\
3.89429735743394	3166.23353082834\\
3.894397359934	3160.22320355742\\
3.89449736243406	3154.21860586445\\
3.89459736493412	3148.22546732738\\
3.89469736743419	3142.23232879031\\
3.89479736993425	3136.2449198312\\
3.89489737243431	3130.26324045003\\
3.89499737493437	3124.28729064682\\
3.89509737743444	3118.31707042155\\
3.8951973799345	3112.35257977424\\
3.89529738243456	3106.39381870488\\
3.89539738493462	3100.44078721347\\
3.89549738743469	3094.49348530001\\
3.89559738993475	3088.55191296451\\
3.89569739243481	3082.61607020695\\
3.89579739493487	3076.6916866053\\
3.89589739743494	3070.76730300365\\
3.895997399935	3064.84864897995\\
3.89609740243506	3058.9357245342\\
3.89619740493512	3053.0285296664\\
3.89629740743519	3047.12706437655\\
3.89639740993525	3041.23132866465\\
3.89649741243531	3035.34705210866\\
3.89659741493537	3029.46277555267\\
3.89669741743544	3023.58422857462\\
3.8967974199355	3017.71141117453\\
3.89689742243556	3011.84432335239\\
3.89699742493562	3005.98869468616\\
3.89709742743569	3000.13306601992\\
3.89719742993575	2994.28316693163\\
3.89729743243581	2988.4389974213\\
3.89739743493587	2982.60055748892\\
3.89749743743594	2976.77357671244\\
3.897597439936	2970.94659593596\\
3.89769744243606	2965.12534473743\\
3.89779744493612	2959.30982311685\\
3.89789744743619	2953.50576065217\\
3.89799744993625	2947.7016981875\\
3.89809745243631	2941.90336530077\\
3.89819745493637	2936.11649156995\\
3.89829745743644	2930.32961783913\\
3.8983974599365	2924.54847368626\\
3.89849746243656	2918.77305911134\\
3.89859746493662	2913.00910369233\\
3.89869746743669	2907.24514827331\\
3.89879746993675	2901.48692243225\\
3.89889747243681	2895.74015574708\\
3.89899747493687	2889.99338906192\\
3.89909747743694	2884.25235195471\\
3.899197479937	2878.5227740034\\
3.89929748243706	2872.79319605209\\
3.89939748493712	2867.06934767874\\
3.89949748743719	2861.35695846128\\
3.89959748993725	2855.64456924383\\
3.89969749243731	2849.93790960432\\
3.89979749493737	2844.24270912072\\
3.89989749743744	2838.54750863712\\
3.8999974999375	2832.85803773148\\
3.90009750243756	2827.18002598173\\
3.90019750493762	2821.50201423198\\
3.90029750743769	2815.83546163814\\
3.90039750993775	2810.1689090443\\
3.90049751243781	2804.5080860284\\
3.90059751493787	2798.85872216841\\
3.90069751743794	2793.20935830842\\
3.900797519938	2787.57145360433\\
3.90089752243806	2781.93354890025\\
3.90099752493812	2776.30137377411\\
3.90109752743819	2770.68065780388\\
3.90119752993825	2765.05994183364\\
3.90129753243831	2759.45068501931\\
3.90139753493837	2753.84142820498\\
3.90149753743844	2748.24363054656\\
3.9015975399385	2742.64583288813\\
3.90169754243856	2737.0594943856\\
3.90179754493862	2731.47315588308\\
3.90189754743869	2725.8925469585\\
3.90199754993875	2720.32339718983\\
3.90209755243881	2714.75424742116\\
3.90219755493887	2709.19655680839\\
3.90229755743894	2703.63886619562\\
3.902397559939	2698.09263473875\\
3.90249756243906	2692.54640328189\\
3.90259756493912	2687.01163098092\\
3.90269756743919	2681.47685867996\\
3.90279756993925	2675.9535455349\\
3.90289757243931	2670.43023238984\\
3.90299757493937	2664.91837840068\\
3.90309757743944	2659.40652441152\\
3.9031975799395	2653.90612957827\\
3.90329758243956	2648.40573474501\\
3.90339758493962	2642.91679906766\\
3.90349758743969	2637.4278633903\\
3.90359758993975	2631.95038686885\\
3.90369759243981	2626.4729103474\\
3.90379759493987	2621.00689298185\\
3.90389759743994	2615.54660519426\\
3.90399759994	2610.08631740666\\
3.90409760244006	2604.63748877497\\
3.90419760494012	2599.18866014327\\
3.90429760744019	2593.75129066748\\
3.90439760994025	2588.31392119169\\
3.90449761244031	2582.8880108718\\
3.90459761494037	2577.46210055191\\
3.90469761744044	2572.04764938792\\
3.9047976199405	2566.63892780189\\
3.90489762244056	2561.23020621585\\
3.90499762494062	2555.83294378572\\
3.90509762744069	2550.43568135559\\
3.90519762994075	2545.04987808136\\
3.90529763244081	2539.66407480713\\
3.90539763494087	2534.2897306888\\
3.90549763744094	2528.92111614843\\
3.905597639941	2523.55250160805\\
3.90569764244106	2518.19534622358\\
3.90579764494112	2512.83819083911\\
3.90589764744119	2507.49249461053\\
3.90599764994125	2502.15252795992\\
3.90609765244131	2496.8125613093\\
3.90619765494137	2491.48405381458\\
3.90629765744144	2486.15554631986\\
3.9063976599415	2480.83849798105\\
3.90649766244156	2475.52717922019\\
3.90659766494162	2470.21586045932\\
3.90669766744169	2464.91600085436\\
3.90679766994175	2459.62187082735\\
3.90689767244181	2454.32774080035\\
3.90699767494187	2449.04506992924\\
3.90709767744194	2443.76239905813\\
3.907197679942	2438.49118734293\\
3.90729768244206	2433.22570520568\\
3.90739768494212	2427.96022306843\\
3.90749768744219	2422.70620008708\\
3.90759768994225	2417.45790668368\\
3.90769769244231	2412.20961328028\\
3.90779769494237	2406.97277903278\\
3.90789769744244	2401.74167436324\\
3.9079976999425	2396.51056969369\\
3.90809770244256	2391.29092418005\\
3.90819770494262	2386.07700824436\\
3.90829770744269	2380.86309230867\\
3.90839770994275	2375.66063552888\\
3.90849771244281	2370.46390832705\\
3.90859771494287	2365.26718112521\\
3.90869771744294	2360.08191307928\\
3.908797719943	2354.90237461129\\
3.90889772244306	2349.72283614331\\
3.90899772494312	2344.55475683123\\
3.90909772744319	2339.3924070971\\
3.90919772994325	2334.23005736297\\
3.90929773244331	2329.07916678475\\
3.90939773494337	2323.93400578447\\
3.90949773744344	2318.7888447842\\
3.9095977399435	2313.65514293983\\
3.90969774244356	2308.52717067341\\
3.90979774494362	2303.39919840698\\
3.90989774744369	2298.28268529647\\
3.90999774994375	2293.1719017639\\
3.91009775244381	2288.06111823133\\
3.91019775494387	2282.96179385467\\
3.91029775744394	2277.86819905595\\
3.910397759944	2272.78033383519\\
3.91049776244406	2267.69246861443\\
3.91059776494412	2262.61606254957\\
3.91069776744419	2257.54538606266\\
3.91079776994425	2252.47470957576\\
3.91089777244431	2247.41549224475\\
3.91099777494437	2242.3620044917\\
3.91109777744444	2237.30851673864\\
3.9111977799445	2232.26648814149\\
3.91129778244456	2227.23018912229\\
3.91139778494462	2222.19961968104\\
3.91149778744469	2217.1690502398\\
3.91159778994475	2212.14993995445\\
3.91169779244481	2207.13655924705\\
3.91179779494487	2202.12317853966\\
3.91189779744494	2197.12125698817\\
3.911997799945	2192.12506501463\\
3.91209780244506	2187.13460261904\\
3.91219780494512	2182.14414022345\\
3.91229780744519	2177.16513698376\\
3.91239780994525	2172.19186332203\\
3.91249781244531	2167.21858966029\\
3.91259781494537	2162.25677515446\\
3.91269781744544	2157.30069022658\\
3.9127978199455	2152.35033487665\\
3.91289782244556	2147.39997952672\\
3.91299782494562	2142.46108333269\\
3.91309782744569	2137.52791671661\\
3.91319782994575	2132.59475010053\\
3.91329783244581	2127.67304264036\\
3.91339783494587	2122.75706475814\\
3.91349783744594	2117.84681645387\\
3.913597839946	2112.9365681496\\
3.91369784244606	2108.03777900123\\
3.91379784494612	2103.14471943081\\
3.91389784744619	2098.25165986039\\
3.91399784994625	2093.37005944588\\
3.91409785244631	2088.49418860931\\
3.91419785494637	2083.6240473507\\
3.91429785744644	2078.75390609209\\
3.9143978599465	2073.89522398938\\
3.91449786244656	2069.04227146462\\
3.91459786494662	2064.19504851782\\
3.91469786744669	2059.34782557101\\
3.91479786994675	2054.51206178011\\
3.91489787244681	2049.68202756715\\
3.91499787494687	2044.8519933542\\
3.91509787744694	2040.03341829715\\
3.915197879947	2035.22057281805\\
3.91529788244706	2030.4134569169\\
3.91539788494712	2025.60634101576\\
3.91549788744719	2020.81068427051\\
3.91559788994725	2016.02075710322\\
3.91569789244731	2011.23082993592\\
3.91579789494737	2006.45236192453\\
3.91589789744744	2001.67962349109\\
3.9159978999475	1996.9126146356\\
3.91609790244756	1992.14560578012\\
3.91619790494762	1987.39005608053\\
3.91629790744769	1982.6402359589\\
3.91639790994775	1977.89041583726\\
3.91649791244781	1973.15205487153\\
3.91659791494787	1968.41942348375\\
3.91669791744794	1963.69252167392\\
3.916797919948	1958.96561986409\\
3.91689792244806	1954.25017721016\\
3.91699792494812	1949.54046413419\\
3.91709792744819	1944.83075105821\\
3.91719792994825	1940.13249713814\\
3.91729793244831	1935.43997279602\\
3.91739793494837	1930.7474484539\\
3.91749793744844	1926.06638326768\\
3.9175979399485	1921.39104765941\\
3.91769794244856	1916.72144162909\\
3.91779794494862	1912.05183559878\\
3.91789794744869	1907.39368872436\\
3.91799794994875	1902.7412714279\\
3.91809795244881	1898.08885413144\\
3.91819795494887	1893.44789599088\\
3.91829795744894	1888.81266742827\\
3.918397959949	1884.17743886566\\
3.91849796244906	1879.55366945896\\
3.91859796494912	1874.9356296302\\
3.91869796744919	1870.31758980145\\
3.91879796994925	1865.7110091286\\
3.91889797244931	1861.1101580337\\
3.91899797494937	1856.51503651675\\
3.91909797744944	1851.9199149998\\
3.9191979799495	1847.33625263875\\
3.91929798244956	1842.75831985566\\
3.91939798494962	1838.18038707256\\
3.91949798744969	1833.61391344537\\
3.91959798994975	1829.05316939613\\
3.91969799244981	1824.49242534689\\
3.91979799494987	1819.94314045355\\
3.91989799744994	1815.39958513816\\
3.91999799995	1810.85602982277\\
3.92009800245006	1806.32393366329\\
3.92019800495012	1801.79756708175\\
3.92029800745019	1797.27120050022\\
3.92039800995025	1792.75629307459\\
3.92049801245031	1788.24711522691\\
3.92059801495037	1783.73793737923\\
3.92069801745044	1779.24021868745\\
3.9207980199505	1774.74249999568\\
3.92089802245056	1770.2562404598\\
3.92099802495062	1765.77571050188\\
3.92109802745069	1761.29518054396\\
3.92119802995075	1756.82610974194\\
3.92129803245081	1752.36276851787\\
3.92139803495087	1747.8994272938\\
3.92149803745094	1743.44754522563\\
3.921598039951	1739.00139273542\\
3.92169804245106	1734.5552402452\\
3.92179804495112	1730.12054691089\\
3.92189804745119	1725.68585357658\\
3.92199804995125	1721.26261939817\\
3.92209805245131	1716.84511479771\\
3.92219805495137	1712.42761019725\\
3.92229805745144	1708.02156475269\\
3.9223980599515	1703.62124888609\\
3.92249806245156	1699.22093301948\\
3.92259806495162	1694.83207630878\\
3.92269806745169	1690.44321959808\\
3.92279806995175	1686.06582204328\\
3.92289807245181	1681.69415406643\\
3.92299807495187	1677.32248608958\\
3.92309807745194	1672.96227726864\\
3.923198079952	1668.60206844769\\
3.92329808245206	1664.25331878265\\
3.92339808495212	1659.90456911761\\
3.92349808745219	1655.56727860846\\
3.92359808995225	1651.23571767728\\
3.92369809245231	1646.90415674609\\
3.92379809495237	1642.5840549708\\
3.92389809745244	1638.26395319551\\
3.9239980999525	1633.95531057613\\
3.92409810245256	1629.64666795675\\
3.92419810495262	1625.34948449327\\
3.92429810745269	1621.05803060774\\
3.92439810995275	1616.76657672221\\
3.92449811245281	1612.48658199258\\
3.92459811495287	1608.20658726295\\
3.92469811745294	1603.93805168923\\
3.924798119953	1599.6695161155\\
3.92489812245306	1595.41243969768\\
3.92499812495312	1591.15536327986\\
3.92509812745319	1586.90974601794\\
3.92519812995325	1582.66412875602\\
3.92529813245331	1578.42997065\\
3.92539813495337	1574.19581254399\\
3.92549813745344	1569.97311359387\\
3.9255981399535	1565.75041464376\\
3.92569814245356	1561.53917484955\\
3.92579814495362	1557.33366463329\\
3.92589814745369	1553.12815441702\\
3.92599814995375	1548.93410335667\\
3.92609815245381	1544.74005229631\\
3.92619815495387	1540.5517308139\\
3.92629815745394	1536.3748684874\\
3.926398159954	1532.1980061609\\
3.92649816245406	1528.0326029903\\
3.92659816495412	1523.86719981969\\
3.92669816745419	1519.713255805\\
3.92679816995425	1515.5593117903\\
3.92689817245431	1511.4168269315\\
3.92699817495437	1507.27434207271\\
3.92709817745444	1503.14331636981\\
3.9271981799545	1499.01229066692\\
3.92729818245456	1494.89272411993\\
3.92739818495462	1490.77315757294\\
3.92749818745469	1486.6593206039\\
3.92759818995475	1482.55694279076\\
3.92769819245481	1478.45456497762\\
3.92779819495487	1474.36364632039\\
3.92789819745494	1470.27272766316\\
3.927998199955	1466.19326816183\\
3.92809820245506	1462.11380866049\\
3.92819820495512	1458.04007873711\\
3.92829820745519	1453.97780796964\\
3.92839820995525	1449.91553720216\\
3.92849821245531	1445.86472559058\\
3.92859821495537	1441.81391397901\\
3.92869821745544	1437.76883194539\\
3.9287982199555	1433.73520906766\\
3.92889822245556	1429.70158618994\\
3.92899822495562	1425.67942246812\\
3.92909822745569	1421.65725874631\\
3.92919822995575	1417.64082460244\\
3.92929823245581	1413.63584961448\\
3.92939823495587	1409.63087462651\\
3.92949823745594	1405.6316292165\\
3.929598239956	1401.64384296239\\
3.92969824245606	1397.65605670828\\
3.92979824495612	1393.67400003212\\
3.92989824745619	1389.70340251186\\
3.92999824995625	1385.7328049916\\
3.93009825245631	1381.7679370493\\
3.93019825495637	1377.8145282629\\
3.93029825745644	1373.86111947649\\
3.9303982599565	1369.91344026804\\
3.93049826245656	1365.97722021549\\
3.93059826495662	1362.04100016294\\
3.93069826745669	1358.11050968835\\
3.93079826995675	1354.1857487917\\
3.93089827245681	1350.27244705096\\
3.93099827495687	1346.35914531021\\
3.93109827745694	1342.45157314742\\
3.931198279957	1338.55546014053\\
3.93129828245706	1334.65934713364\\
3.93139828495712	1330.7689637047\\
3.93149828745719	1326.88430985372\\
3.93159828995725	1323.01111515863\\
3.93169829245731	1319.13792046355\\
3.93179829495737	1315.27045534642\\
3.93189829745744	1311.40871980723\\
3.9319982999575	1307.55844342395\\
3.93209830245756	1303.70816704068\\
3.93219830495762	1299.86362023535\\
3.93229830745769	1296.02480300797\\
3.93239830995775	1292.19171535855\\
3.93249831245781	1288.37008686502\\
3.93259831495787	1284.5484583715\\
3.93269831745794	1280.73255945593\\
3.932798319958	1276.92239011831\\
3.93289832245806	1273.11795035864\\
3.93299832495812	1269.31924017692\\
3.93309832745819	1265.53198915111\\
3.93319832995825	1261.74473812529\\
3.93329833245831	1257.96321667743\\
3.93339833495837	1254.18742480752\\
3.93349833745844	1250.41736251556\\
3.9335983399585	1246.65302980155\\
3.93369834245856	1242.90015624344\\
3.93379834495862	1239.14728268533\\
3.93389834745869	1235.40013870518\\
3.93399834995875	1231.65872430297\\
3.93409835245881	1227.92303947872\\
3.93419835495887	1224.19308423242\\
3.93429835745894	1220.46885856407\\
3.934398359959	1216.75036247367\\
3.93449836245906	1213.03759596122\\
3.93459836495912	1209.33055902673\\
3.93469836745919	1205.63498124813\\
3.93479836995925	1201.93940346954\\
3.93489837245931	1198.2495552689\\
3.93499837495937	1194.5654366462\\
3.93509837745944	1190.88704760146\\
3.9351983799595	1187.21438813468\\
3.93529838245956	1183.54745824584\\
3.93539838495962	1179.88625793495\\
3.93549838745969	1176.23078720202\\
3.93559838995975	1172.58104604703\\
3.93569839245981	1168.93703447\\
3.93579839495987	1165.29875247092\\
3.93589839745994	1161.66620004979\\
3.93599839996	1158.03937720661\\
3.93609840246006	1154.41828394139\\
3.93619840496012	1150.80292025411\\
3.93629840746019	1147.19328614479\\
3.93639840996025	1143.58938161342\\
3.93649841246031	1139.99120665999\\
3.93659841496037	1136.39876128452\\
3.93669841746044	1132.812045487\\
3.9367984199605	1129.23105926744\\
3.93689842246056	1125.65580262582\\
3.93699842496062	1122.0805459842\\
3.93709842746069	1118.51674849849\\
3.93719842996075	1114.95868059073\\
3.93729843246081	1111.40634226092\\
3.93739843496087	1107.85973350906\\
3.93749843746094	1104.31885433515\\
3.937598439961	1100.78370473919\\
3.93769844246106	1097.25428472119\\
3.93779844496112	1093.73059428113\\
3.93789844746119	1090.21263341903\\
3.93799844996125	1086.69467255692\\
3.93809845246131	1083.18817085072\\
3.93819845496137	1079.68739872247\\
3.93829845746144	1076.19235617218\\
3.9383984599615	1072.70304319983\\
3.93849846246156	1069.21945980543\\
3.93859846496162	1065.73587641104\\
3.93869846746169	1062.26375217255\\
3.93879846996175	1058.797357512\\
3.93889847246181	1055.33669242941\\
3.93899847496187	1051.88175692478\\
3.93909847746194	1048.42682142014\\
3.939198479962	1044.9833450714\\
3.93929848246206	1041.54559830062\\
3.93939848496212	1038.11358110778\\
3.93949848746219	1034.68156391495\\
3.93959848996225	1031.26100587802\\
3.93969849246231	1027.84617741904\\
3.93979849496237	1024.43707853801\\
3.93989849746244	1021.02797965698\\
3.9399984999625	1017.63033993186\\
3.94009850246256	1014.23842978468\\
3.94019850496262	1010.85224921546\\
3.94029850746269	1007.46606864623\\
3.94039850996275	1004.09134723291\\
3.94049851246281	1000.72235539754\\
3.94059851496287	997.353363562175\\
3.94069851746294	993.995830882709\\
3.940798519963	990.644027781193\\
3.94089852246306	987.292224679678\\
3.94099852496312	983.951880734066\\
3.94109852746319	980.617266366404\\
3.94119852996325	977.282651998742\\
3.94129853246331	973.959496786984\\
3.94139853496337	970.636341575225\\
3.94149853746344	967.324645519369\\
3.9415985399635	964.018679041464\\
3.94169854246356	960.712712563559\\
3.94179854496362	957.418205241557\\
3.94189854746369	954.123697919555\\
3.94199854996375	950.840649753455\\
3.94209855246381	947.557601587356\\
3.94219855496387	944.286012577158\\
3.94229855746394	941.020153144913\\
3.942398559964	937.754293712667\\
3.94249856246406	934.499893436324\\
3.94259856496412	931.245493159981\\
3.94269856746419	928.00255203954\\
3.94279856996425	924.7596109191\\
3.94289857246431	921.528128954562\\
3.94299857496437	918.296646990024\\
3.94309857746444	915.076624181389\\
3.9431985799645	911.856601372754\\
3.94329858246456	908.648037720021\\
3.94339858496462	905.439474067289\\
3.94349858746469	902.236639992507\\
3.94359858996475	899.045265073629\\
3.94369859246481	895.85389015475\\
3.94379859496487	892.673974391774\\
3.94389859746494	889.494058628798\\
3.943998599965	886.325602021724\\
3.94409860246506	883.157145414651\\
3.94419860496512	879.994418385529\\
3.94429860746519	876.843150512309\\
3.94439860996525	873.69188263909\\
3.94449861246531	870.552073921773\\
3.94459861496537	867.412265204456\\
3.94469861746544	864.27818606509\\
3.9447986199655	861.155566081627\\
3.94489862246556	858.032946098164\\
3.94499862496562	854.916055692653\\
3.94509862746569	851.810624443044\\
3.94519862996575	848.705193193435\\
3.94529863246581	845.605491521777\\
3.94539863496587	842.51151942807\\
3.94549863746594	839.429006490267\\
3.945598639966	836.346493552463\\
3.94569864246606	833.26971019261\\
3.94579864496612	830.20438598866\\
3.94589864746619	827.13906178471\\
3.94599864996625	824.079467158712\\
3.94609865246631	821.025602110664\\
3.94619865496637	817.977466640568\\
3.94629865746644	814.940790326375\\
3.9463986599665	811.904114012182\\
3.94649866246656	808.87316727594\\
3.94659866496662	805.847950117649\\
3.94669866746669	802.82846253731\\
3.94679866996675	799.820434112873\\
3.94689867246681	796.812405688436\\
3.94699867496687	793.81010684195\\
3.94709867746694	790.813537573416\\
3.947198679967	787.822697882833\\
3.94729868246706	784.837587770202\\
3.94739868496712	781.858207235521\\
3.94749868746719	778.890285856744\\
3.94759868996725	775.922364477966\\
3.94769869246731	772.96017267714\\
3.94779869496737	770.003710454265\\
3.94789869746744	767.052977809341\\
3.9479986999675	764.107974742368\\
3.94809870246756	761.168701253347\\
3.94819870496762	758.235157342278\\
3.94829870746769	755.307343009159\\
3.94839870996775	752.385258253992\\
3.94849871246781	749.468903076776\\
3.94859871496787	746.558277477511\\
3.94869871746794	743.653381456198\\
3.948798719968	740.754215012836\\
3.94889872246806	737.860778147425\\
3.94899872496812	734.973070859966\\
3.94909872746819	732.091093150458\\
3.94919872996825	729.214845018901\\
3.94929873246831	726.344326465296\\
3.94939873496837	723.479537489642\\
3.94949873746844	720.620478091939\\
3.9495987399685	717.767148272187\\
3.94969874246856	714.919548030387\\
3.94979874496862	712.071947788587\\
3.94989874746869	709.23580670269\\
3.94999874996875	706.405395194743\\
3.95009875246881	703.580713264748\\
3.95019875496887	700.761760912705\\
3.95029875746894	697.948538138612\\
3.950398759969	695.141044942471\\
3.95049876246906	692.33355174633\\
3.95059876496912	689.537517706092\\
3.95069876746919	686.747213243805\\
3.95079876996925	683.962638359469\\
3.95089877246931	681.183793053084\\
3.95099877496937	678.4049477467\\
3.95109877746944	675.637561596218\\
3.9511987799695	672.875905023688\\
3.95129878246956	670.119978029108\\
3.95139878496962	667.364051034529\\
3.95149878746969	664.619583195852\\
3.95159878996975	661.880844935127\\
3.95169879246981	659.147836252353\\
3.95179879496987	656.414827569579\\
3.95189879746994	653.693278042708\\
3.95199879997	650.977458093788\\
3.95209880247006	648.261638144867\\
3.95219880497012	645.55727735185\\
3.95229880747019	642.858646136784\\
3.95239880997025	640.160014921718\\
3.95249881247031	637.472842862554\\
3.95259881497037	634.791400381342\\
3.95269881747044	632.109957900129\\
3.9527988199705	629.43997457482\\
3.95289882247056	626.76999124951\\
3.95299882497062	624.111467080103\\
3.95309882747069	621.458672488647\\
3.95319882997075	618.805877897192\\
3.95329883247081	616.164542461639\\
3.95339883497087	613.523207026086\\
3.95349883747094	610.893330746435\\
3.953598839971	608.263454466785\\
3.95369884247106	605.645037343037\\
3.95379884497112	603.026620219289\\
3.95389884747119	600.419662251444\\
3.95399884997125	597.812704283598\\
3.95409885247131	595.217205471656\\
3.95419885497137	592.621706659713\\
3.95429885747144	590.037667003673\\
3.9543988599715	587.453627347633\\
3.95449886247156	584.881046847496\\
3.95459886497162	582.308466347358\\
3.95469886747169	579.741615425172\\
3.95479886997175	577.186223658889\\
3.95489887247181	574.630831892605\\
3.95499887497187	572.084034493249\\
3.95509887747194	569.542393714048\\
3.955198879972	567.006482512799\\
3.95529888247206	564.475727931707\\
3.95539888497212	561.95127588636\\
3.95549888747219	559.43198046117\\
3.95559888997225	556.918414613931\\
3.95569889247231	554.410578344643\\
3.95579889497237	551.908471653307\\
3.95589889747244	549.412094539922\\
3.9559988999725	546.921447004488\\
3.95609890247256	544.435956089211\\
3.95619890497262	541.95676770968\\
3.95629890747269	539.482735950305\\
3.95639890997275	537.015006726676\\
3.95649891247281	534.552434123204\\
3.95659891497287	532.095591097683\\
3.95669891747294	529.644477650114\\
3.956798919973	527.199093780495\\
3.95689892247306	524.759439488828\\
3.95699892497312	522.324941817317\\
3.95709892747319	519.896746681553\\
3.95719892997325	517.473708165945\\
3.95729893247331	515.056972186083\\
3.95739893497337	512.645392826377\\
3.95749893747344	510.239543044623\\
3.9575989399735	507.83942284082\\
3.95769894247356	505.445032214968\\
3.95779894497362	503.056371167068\\
3.95789894747369	500.672866739323\\
3.95799894997375	498.295664847326\\
3.95809895247381	495.923619575484\\
3.95819895497387	493.557876839389\\
3.95829895747394	491.19729072345\\
3.958398959974	488.842434185462\\
3.95849896247406	486.493307225426\\
3.95859896497412	484.149909843341\\
3.95869896747419	481.812242039207\\
3.95879896997425	479.480303813025\\
3.95889897247431	477.153522206998\\
3.95899897497437	474.833043136718\\
3.95909897747444	472.517720686595\\
3.9591989799745	470.208700772218\\
3.95929898247456	467.904837477997\\
3.95939898497462	465.606703761727\\
3.95949898747469	463.314299623408\\
3.95959898997475	461.027625063041\\
3.95969899247481	458.74610712283\\
3.95979899497487	456.470891718366\\
3.95989899747494	454.200832934057\\
3.959998999975	451.937076685496\\
3.96009900247506	449.67847705709\\
3.96019900497512	447.425607006635\\
3.96029900747519	445.178466534132\\
3.96039900997525	442.937055639581\\
3.96049901247531	440.70137432298\\
3.96059901497537	438.471422584331\\
3.96069901747544	436.246627465838\\
3.9607990199755	434.028134883091\\
3.96089902247556	431.814798920501\\
3.96099902497562	429.607192535862\\
3.96109902747569	427.405315729174\\
3.96119902997575	425.209168500438\\
3.96129903247581	423.018750849653\\
3.96139903497587	420.834062776819\\
3.96149903747594	418.655104281936\\
3.961599039976	416.48130240721\\
3.96169904247606	414.31380306823\\
3.96179904497612	412.151460349406\\
3.96189904747619	409.994847208534\\
3.96199904997625	407.843963645613\\
3.96209905247631	405.698809660643\\
3.96219905497637	403.559385253625\\
3.96229905747644	401.425690424557\\
3.9623990599765	399.297725173441\\
3.96249906247656	397.174916542482\\
3.96259906497662	395.058410447268\\
3.96269906747669	392.947060972211\\
3.96279906997675	390.841441075106\\
3.96289907247681	388.741550755951\\
3.96299907497687	386.647390014748\\
3.96309907747694	384.558958851496\\
3.963199079977	382.476257266196\\
3.96329908247706	380.398712301051\\
3.96339908497712	378.327469871653\\
3.96349908747719	376.261384062412\\
3.96359908997725	374.201027831121\\
3.96369909247731	372.146401177782\\
3.96379909497737	370.097504102394\\
3.96389909747744	368.054336604958\\
3.9639990999775	366.016898685472\\
3.96409910247756	363.985190343939\\
3.96419910497762	361.958638622561\\
3.96429910747769	359.93838943693\\
3.96439910997775	357.923296871455\\
3.96449911247781	355.913933883931\\
3.96459911497787	353.910300474358\\
3.96469911747794	351.912396642737\\
3.964799119978	349.920222389067\\
3.96489912247806	347.933777713349\\
3.96499912497812	345.952489657786\\
3.96509912747819	343.97750413797\\
3.96519912997825	342.00767523831\\
3.96529913247831	340.043575916602\\
3.96539913497837	338.085206172845\\
3.96549913747844	336.132566007039\\
3.9655991399785	334.185655419184\\
3.96569914247856	332.244474409281\\
3.96579914497862	330.308450019534\\
3.96589914747869	328.378728165534\\
3.96599914997875	326.454162931689\\
3.96609915247881	324.535327275796\\
3.96619915497887	322.622794155649\\
3.96629915747894	320.715417655659\\
3.966399159979	318.813197775825\\
3.96649916247906	316.917280431737\\
3.96659916497912	315.0270926656\\
3.96669916747919	313.14206151962\\
3.96679916997925	311.263332909386\\
3.96689917247931	309.389760919308\\
3.96699917497937	307.521918507181\\
3.96709917747944	305.659805673006\\
3.9671991799795	303.803422416782\\
3.96729918247956	301.95276873851\\
3.96739918497962	300.107844638189\\
3.96749918747969	298.268077158023\\
3.96759918997975	296.434612213605\\
3.96769919247981	294.606303889342\\
3.96779919497987	292.783725143031\\
3.96789919747994	290.966875974671\\
3.96799919998	289.155756384263\\
3.96809920248006	287.350366371806\\
3.96819920498012	285.550132979505\\
3.96829920748019	283.75620212295\\
3.96839920998025	281.967427886552\\
3.96849921248031	280.1849561859\\
3.96859921498037	278.407641105404\\
3.96869921748044	276.636055602859\\
3.9687992199805	274.870199678266\\
3.96889922248056	273.110073331624\\
3.96899922498062	271.355103605138\\
3.96909922748069	269.606436414399\\
3.96919922998075	267.862925843816\\
3.96929923248081	266.125717808979\\
3.96939923498087	264.393666394299\\
3.96949923748094	262.66734455757\\
3.969599239981	260.946752298792\\
3.96969924248106	259.231889617965\\
3.96979924498112	257.522183557295\\
3.96989924748119	255.818780032371\\
3.96999924998125	254.120533127603\\
3.97009925248131	252.428588758582\\
3.97019925498137	250.741801009717\\
3.97029925748144	249.060742838803\\
3.9703992599815	247.38541424584\\
3.97049926248156	245.715242273034\\
3.97059926498162	244.051372835974\\
3.97069926748169	242.393232976866\\
3.97079926998175	240.740249737913\\
3.97089927248181	239.092996076912\\
3.97099927498187	237.452044951657\\
3.97109927748194	235.816250446559\\
3.971199279982	234.185612561617\\
3.97129928248206	232.561277212421\\
3.97139928498212	230.942671441176\\
3.97149928748219	229.329222290088\\
3.97159928998225	227.722075674746\\
3.97169929248231	226.12008567956\\
3.97179929498237	224.523825262325\\
3.97189929748244	222.933294423042\\
3.9719992999825	221.34849316171\\
3.97209930248256	219.76942147833\\
3.97219930498262	218.195506415105\\
3.97229930748269	216.627893887628\\
3.97239930998275	215.065437980306\\
3.97249931248281	213.509284608731\\
3.97259931498287	211.958287857311\\
3.97269931748294	210.413020683844\\
3.972799319983	208.872910130532\\
3.97289932248306	207.339102112967\\
3.97299932498312	205.811023673353\\
3.97309932748319	204.288101853895\\
3.97319932998325	202.771482570184\\
3.97329933248331	201.260019906629\\
3.97339933498337	199.754286821025\\
3.97349933748344	198.254283313372\\
3.9735993399835	196.760009383671\\
3.97369934248356	195.270892074126\\
3.97379934498362	193.788077300328\\
3.97389934748369	192.310419146685\\
3.97399934998375	190.839063528789\\
3.97409935248381	189.372864531049\\
3.97419935498387	187.912395111261\\
3.97429935748394	186.457655269424\\
3.974399359984	185.008072047743\\
3.97449936248406	183.564791361808\\
3.97459936498412	182.127240253825\\
3.97469936748419	180.694845765998\\
3.97479936998425	179.268180856122\\
3.97489937248431	177.847245524198\\
3.97499937498437	176.432039770225\\
3.97509937748444	175.022563594203\\
3.9751993799845	173.618816996132\\
3.97529938248456	172.220227018218\\
3.97539938498462	170.82793957605\\
3.97549938748469	169.440808754038\\
3.97559938998475	168.059407509978\\
3.97569939248481	166.683735843869\\
3.97579939498487	165.313793755711\\
3.97589939748494	163.949581245505\\
3.975999399985	162.591098313249\\
3.97609940248506	161.23777200115\\
3.97619940498512	159.890748224798\\
3.97629940748519	158.548881068601\\
3.97639940998525	157.212743490356\\
3.97649941248531	155.882335490063\\
3.97659941498537	154.55765706772\\
3.97669941748544	153.238708223329\\
3.9767994199855	151.924915999094\\
3.97689942248556	150.617426310605\\
3.97699942498562	149.315093242273\\
3.97709942748569	148.018489751892\\
3.97719942998575	146.727615839462\\
3.97729943248581	145.442471504984\\
3.97739943498587	144.163056748457\\
3.97749943748594	142.889371569881\\
3.977599439986	141.620843011461\\
3.97769944248606	140.358616988788\\
3.97779944498612	139.101547586271\\
3.97789944748619	137.850207761705\\
3.97799944998625	136.604597515091\\
3.97809945248631	135.364716846428\\
3.97819945498637	134.130565755716\\
3.97829945748644	132.90157128516\\
3.9783994599865	131.678879350351\\
3.97849946248656	130.461344035698\\
3.97859946498662	129.249538298996\\
3.97869946748669	128.043462140246\\
3.97879946998675	126.843115559447\\
3.97889947248681	125.648498556599\\
3.97899947498687	124.459611131703\\
3.97909947748694	123.275880326963\\
3.979199479987	122.097879100174\\
3.97929948248706	120.926180409131\\
3.97939948498712	119.759638338245\\
3.97949948748719	118.59882584531\\
3.97959948998725	117.443742930326\\
3.97969949248731	116.293816635498\\
3.97979949498737	115.150192876417\\
3.97989949748744	114.011725737492\\
3.9799994999875	112.879561134314\\
3.98009950248756	111.752553151291\\
3.98019950498762	110.63127474622\\
3.98029950748769	109.515725919101\\
3.98039950998775	108.405333712137\\
3.98049951248781	107.30124404092\\
3.98059951498787	106.202310989859\\
3.98069951748794	105.109680474545\\
3.980799519988	104.022206579386\\
3.98089952248806	102.940462262179\\
3.98099952498812	101.864447522924\\
3.98109952748819	100.794162361619\\
3.98119952998825	99.7290338204711\\
3.98129953248831	98.6702078150693\\
3.98139953498837	97.6165384298237\\
3.98149953748844	96.5685986225295\\
3.9815995399885	95.5269613509816\\
3.98169954248856	94.49048069959\\
3.98179954498862	93.4591566683545\\
3.98189954748869	92.4341351728655\\
3.98199954998875	91.4148432553277\\
3.98209955248881	90.4007079579462\\
3.98219955498887	89.3923022385159\\
3.98229955748894	88.389626097037\\
3.982399559989	87.3926795335094\\
3.98249956248906	86.401462547933\\
3.98259956498912	85.415975140308\\
3.98269956748919	84.4362173106343\\
3.98279956998925	83.4616161011168\\
3.98289957248931	82.4927444695505\\
3.98299957498937	81.5301753737308\\
3.98309957748944	80.5727628980672\\
3.9831995799895	79.6210800003549\\
3.98329958248956	78.6745537227987\\
3.98339958498962	77.734329980989\\
3.98349958748969	76.7992628593356\\
3.98359958998975	75.8704982734285\\
3.98369959248981	74.9468903076776\\
3.98379959498987	74.029011919878\\
3.98389959748994	73.1168631100297\\
3.98399959999	72.2104438781328\\
3.98409960249006	71.309181266392\\
3.98419960499012	70.4142211903977\\
3.98429960749019	69.5244177345595\\
3.98439960999025	68.6409168144677\\
3.98449961249031	67.7625725145322\\
3.98459961499037	66.889957792548\\
3.98469961749044	66.0224996907199\\
3.9847996199905	65.1613441246383\\
3.98489962249056	64.3059181365079\\
3.98499962499062	63.4556487685338\\
3.98509962749069	62.611108978511\\
3.98519962999075	61.7722987664394\\
3.98529963249081	60.9392181323192\\
3.98539963499087	60.1118670761503\\
3.98549963749094	59.2902455979327\\
3.985599639991	58.4743536976664\\
3.98569964249106	57.6636184175563\\
3.98579964499112	56.8587273069565\\
3.98589964749119	56.0595657743081\\
3.98599964999125	55.2660192280519\\
3.98609965249131	54.478087668188\\
3.98619965499137	53.6958283904959\\
3.98629965749144	52.9192413949755\\
3.9863996599915	52.148326681627\\
3.98649966249156	51.3830269546708\\
3.98659966499162	50.6233422141068\\
3.98669966749169	49.8693870514942\\
3.98679966999175	49.1209895794943\\
3.98689967249181	48.3783216854457\\
3.98699967499187	47.6412687777894\\
3.98709967749194	46.9098308565254\\
3.987199679992	46.1841225132127\\
3.98729968249206	45.4639718605128\\
3.98739968499212	44.7495507857641\\
3.98749968749219	44.0407446974078\\
3.98759968999225	43.3375535954438\\
3.98769969249231	42.640092071431\\
3.98779969499237	41.948188238031\\
3.98789969749244	41.2620139825823\\
3.9879996999925	40.581454713526\\
3.98809970249256	39.9065104308618\\
3.98819970499262	39.2372384303695\\
3.98829970749269	38.573638712049\\
3.98839970999275	37.9156539801208\\
3.98849971249281	37.2633415303643\\
3.98859971499287	36.6167013627797\\
3.98869971749294	35.9756761815873\\
3.988799719993	35.3402659867872\\
3.98889972249306	34.7105280741589\\
3.98899972499312	34.0864624437025\\
3.98909972749319	33.4680690954178\\
3.98919972999325	32.8552907335253\\
3.98929973249331	32.2481273580252\\
3.98939973499337	31.6466362646969\\
3.98949973749344	31.0508174535403\\
3.9895997399935	30.4606136287761\\
3.98969974249356	29.8760820861836\\
3.98979974499362	29.2972228257629\\
3.98989974749369	28.7239785517345\\
3.98999974999375	28.1564065598779\\
3.99009975249381	27.5944495544136\\
3.99019975499388	27.0381648311211\\
3.99029975749394	26.4874950942209\\
3.990399759994	25.9424976394924\\
3.99049976249406	25.4031151711563\\
3.99059976499412	24.8694622807714\\
3.99069976749419	24.3413670809994\\
3.99079976999425	23.8190014591786\\
3.99089977249431	23.3021935279706\\
3.99099977499437	22.7911151747139\\
3.99109977749444	22.2856518078495\\
3.9911997799945	21.7858034273773\\
3.99129978249456	21.2916846248565\\
3.99139978499462	20.8031235129485\\
3.99149978749469	20.3202919789917\\
3.99159978999475	19.8430754314273\\
3.99169979249481	19.3714738702551\\
3.99179979499487	18.9055445912547\\
3.99189979749494	18.4452875944261\\
3.991999799995	17.9906455839898\\
3.99209980249506	17.5416758557253\\
3.99219980499513	17.0983211138531\\
3.99229980749519	16.6606386541526\\
3.99239980999525	16.2285711808445\\
3.99249981249531	15.8022332854876\\
3.99259981499537	15.3814530807435\\
3.99269981749544	14.9663451581713\\
3.9927998199955	14.5569095177708\\
3.99289982249556	14.1530888637626\\
3.99299982499562	13.7549404919262\\
3.99309982749569	13.3624644022615\\
3.99319982999575	12.9756032989892\\
3.99329983249581	12.5944144778887\\
3.99339983499587	12.2188406431804\\
3.99349983749594	11.848939090644\\
3.993599839996	11.4846525244998\\
3.99369984249606	11.1260382405274\\
3.99379984499612	10.7730389429473\\
3.99389984749619	10.425711927539\\
3.99399984999625	10.0840571943025\\
3.99409985249631	9.74801744745826\\
3.99419985499638	9.41764998278583\\
3.99429985749644	9.09289750450568\\
3.9943998599965	8.77381730839732\\
3.99449986249656	8.46040939446076\\
3.99459986499663	8.15261646691648\\
3.99469986749669	7.85043852576449\\
3.99479986999675	7.55393286678429\\
3.99489987249681	7.26309948997588\\
3.99499987499687	6.97788109955976\\
3.99509987749694	6.69833499131543\\
3.995199879997	6.42446116524289\\
3.99529988249706	6.15620232556264\\
3.99539988499712	5.89355847227467\\
3.99549988749719	5.63660408989235\\
3.99559988999725	5.38528761221412\\
3.99569989249731	5.13961476881793\\
3.99579989499737	4.89959128928172\\
3.99589989749744	4.66521717360551\\
3.9959998999975	4.43649242178928\\
3.99609990249756	4.2134113042551\\
3.99619990499763	3.99597382100295\\
3.99629990749769	3.78419143118874\\
3.99639990999775	3.57805267565658\\
3.99649991249781	3.3775632839844\\
3.99659991499788	3.18271752659426\\
3.99669991749794	2.99352113306411\\
3.996799919998	2.809968373816\\
3.99689992249806	2.63207070800583\\
3.99699992499812	2.45981094689975\\
3.99709992749819	2.29320627923161\\
3.99719992999825	2.13224524584551\\
3.99729993249831	1.97693357631939\\
3.99739993499837	1.82726554107532\\
3.99749993749844	1.68324686969124\\
3.9975999399985	1.54487756216714\\
3.99769994249856	1.41215188892509\\
3.99779994499862	1.28507557954302\\
3.99789994749869	1.16364290444299\\
3.99799994999875	1.04785959320296\\
3.99809995249881	0.937725645822911\\
3.99819995499888	0.833235332724902\\
3.99829995749894	0.734394383486884\\
3.998399959999	0.641202798108856\\
3.99849996249906	0.553654274055071\\
3.99859996499913	0.471754540903481\\
3.99869996749919	0.3955018797807\\
3.99879996999925	0.324896863644524\\
3.99889997249931	0.259938346579362\\
3.99899997499937	0.200627474500804\\
3.99909997749944	0.146963674451056\\
3.9991999799995	0.0989469464301175\\
3.99929998249956	0.0565770612548702\\
3.99939998499962	0.0198546491788889\\
3.99949998749969	-0.0112206335725035\\
3.99959998999975	-0.036648786999307\\
3.99969999249981	-0.0564297538057421\\
3.99979999499987	-0.0705637631749268\\
3.99989999749994	-0.0790504140364046\\
4	-0.0818899928690729\\
};
\addlegendentry{c1};

\addplot [color=mycolor2,solid,forget plot]
  table[row sep=crcr]{%
0	0\\
0.000100002500062502	0\\
0.000200005000125003	0\\
0.000300007500187505	0\\
0.000400010000250006	0\\
0.000500012500312508	0\\
0.000600015000375009	0\\
0.000700017500437511	0\\
0.000800020000500012	0\\
0.000900022500562514	0\\
0.00100002500062502	0\\
0.00110002750068752	0\\
0.00120003000075002	0\\
0.00130003250081252	0\\
0.00140003500087502	0\\
0.00150003750093752	0\\
0.00160004000100002	0\\
0.00170004250106253	0\\
0.00180004500112503	0\\
0.00190004750118753	0\\
0.00200005000125003	0\\
0.00210005250131253	0\\
0.00220005500137503	0\\
0.00230005750143754	0\\
0.00240006000150004	0\\
0.00250006250156254	0\\
0.00260006500162504	0\\
0.00270006750168754	0\\
0.00280007000175004	0\\
0.00290007250181255	0\\
0.00300007500187505	0\\
0.00310007750193755	0\\
0.00320008000200005	0\\
0.00330008250206255	0\\
0.00340008500212505	0\\
0.00350008750218755	0\\
0.00360009000225006	0\\
0.00370009250231256	0\\
0.00380009500237506	0\\
0.00390009750243756	0\\
0.00400010000250006	0\\
0.00410010250256256	0\\
0.00420010500262507	0\\
0.00430010750268757	0\\
0.00440011000275007	0\\
0.00450011250281257	0\\
0.00460011500287507	0\\
0.00470011750293757	0\\
0.00480012000300007	0\\
0.00490012250306258	0\\
0.00500012500312508	0\\
0.00510012750318758	0\\
0.00520013000325008	0\\
0.00530013250331258	0\\
0.00540013500337508	0\\
0.00550013750343759	0\\
0.00560014000350009	0\\
0.00570014250356259	0\\
0.00580014500362509	0\\
0.00590014750368759	0\\
0.00600015000375009	0\\
0.0061001525038126	0\\
0.0062001550038751	0\\
0.0063001575039376	0\\
0.0064001600040001	0\\
0.0065001625040626	0\\
0.0066001650041251	0\\
0.0067001675041876	0\\
0.00680017000425011	0\\
0.00690017250431261	0\\
0.00700017500437511	0\\
0.00710017750443761	0\\
0.00720018000450011	0\\
0.00730018250456261	0\\
0.00740018500462512	0\\
0.00750018750468762	0\\
0.00760019000475012	0\\
0.00770019250481262	0\\
0.00780019500487512	0\\
0.00790019750493762	0\\
0.00800020000500012	0\\
0.00810020250506263	0\\
0.00820020500512513	0\\
0.00830020750518763	0\\
0.00840021000525013	0\\
0.00850021250531263	0\\
0.00860021500537513	0\\
0.00870021750543764	0\\
0.00880022000550014	0\\
0.00890022250556264	0\\
0.00900022500562514	0\\
0.00910022750568764	0\\
0.00920023000575014	0\\
0.00930023250581265	0\\
0.00940023500587515	0\\
0.00950023750593765	0\\
0.00960024000600015	0\\
0.00970024250606265	0\\
0.00980024500612515	0\\
0.00990024750618766	0\\
0.0100002500062502	0\\
0.0101002525063127	0\\
0.0102002550063752	0\\
0.0103002575064377	0\\
0.0104002600065002	0\\
0.0105002625065627	0\\
0.0106002650066252	0\\
0.0107002675066877	0\\
0.0108002700067502	0\\
0.0109002725068127	0\\
0.0110002750068752	0\\
0.0111002775069377	0\\
0.0112002800070002	0\\
0.0113002825070627	0\\
0.0114002850071252	0\\
0.0115002875071877	0\\
0.0116002900072502	0\\
0.0117002925073127	0\\
0.0118002950073752	0\\
0.0119002975074377	0\\
0.0120003000075002	0\\
0.0121003025075627	0\\
0.0122003050076252	0\\
0.0123003075076877	0\\
0.0124003100077502	0\\
0.0125003125078127	0\\
0.0126003150078752	0\\
0.0127003175079377	0\\
0.0128003200080002	0\\
0.0129003225080627	0\\
0.0130003250081252	0\\
0.0131003275081877	0\\
0.0132003300082502	0\\
0.0133003325083127	0\\
0.0134003350083752	0\\
0.0135003375084377	0\\
0.0136003400085002	0\\
0.0137003425085627	0\\
0.0138003450086252	0\\
0.0139003475086877	0\\
0.0140003500087502	0\\
0.0141003525088127	0\\
0.0142003550088752	0\\
0.0143003575089377	0\\
0.0144003600090002	0\\
0.0145003625090627	0\\
0.0146003650091252	0\\
0.0147003675091877	0\\
0.0148003700092502	0\\
0.0149003725093127	0\\
0.0150003750093752	0\\
0.0151003775094377	0\\
0.0152003800095002	0\\
0.0153003825095627	0\\
0.0154003850096252	0\\
0.0155003875096877	0\\
0.0156003900097502	0\\
0.0157003925098127	0\\
0.0158003950098752	0\\
0.0159003975099377	0\\
0.0160004000100002	0\\
0.0161004025100628	0\\
0.0162004050101253	0\\
0.0163004075101878	0\\
0.0164004100102503	0\\
0.0165004125103128	0\\
0.0166004150103753	0\\
0.0167004175104378	0\\
0.0168004200105003	0\\
0.0169004225105628	0\\
0.0170004250106253	0\\
0.0171004275106878	0\\
0.0172004300107503	0\\
0.0173004325108128	0\\
0.0174004350108753	0\\
0.0175004375109378	0\\
0.0176004400110003	0\\
0.0177004425110628	0\\
0.0178004450111253	0\\
0.0179004475111878	0\\
0.0180004500112503	0\\
0.0181004525113128	0\\
0.0182004550113753	0\\
0.0183004575114378	0\\
0.0184004600115003	0\\
0.0185004625115628	0\\
0.0186004650116253	0\\
0.0187004675116878	0\\
0.0188004700117503	0\\
0.0189004725118128	0\\
0.0190004750118753	0\\
0.0191004775119378	0\\
0.0192004800120003	0\\
0.0193004825120628	0\\
0.0194004850121253	0\\
0.0195004875121878	0\\
0.0196004900122503	0\\
0.0197004925123128	0\\
0.0198004950123753	0\\
0.0199004975124378	0\\
0.0200005000125003	0\\
0.0201005025125628	0\\
0.0202005050126253	0\\
0.0203005075126878	0\\
0.0204005100127503	0\\
0.0205005125128128	0\\
0.0206005150128753	0\\
0.0207005175129378	0\\
0.0208005200130003	0\\
0.0209005225130628	0\\
0.0210005250131253	0\\
0.0211005275131878	0\\
0.0212005300132503	0\\
0.0213005325133128	0\\
0.0214005350133753	0\\
0.0215005375134378	0\\
0.0216005400135003	0\\
0.0217005425135628	0\\
0.0218005450136253	0\\
0.0219005475136878	0\\
0.0220005500137503	0\\
0.0221005525138128	0\\
0.0222005550138753	0\\
0.0223005575139378	0\\
0.0224005600140003	0\\
0.0225005625140629	0\\
0.0226005650141254	0\\
0.0227005675141879	0\\
0.0228005700142504	0\\
0.0229005725143129	0\\
0.0230005750143754	0\\
0.0231005775144379	0\\
0.0232005800145004	0\\
0.0233005825145629	0\\
0.0234005850146254	0\\
0.0235005875146879	0\\
0.0236005900147504	0\\
0.0237005925148129	0\\
0.0238005950148754	0\\
0.0239005975149379	0\\
0.0240006000150004	0\\
0.0241006025150629	0\\
0.0242006050151254	0\\
0.0243006075151879	0\\
0.0244006100152504	0\\
0.0245006125153129	0\\
0.0246006150153754	0\\
0.0247006175154379	0\\
0.0248006200155004	0\\
0.0249006225155629	0\\
0.0250006250156254	0\\
0.0251006275156879	0\\
0.0252006300157504	0\\
0.0253006325158129	0\\
0.0254006350158754	0\\
0.0255006375159379	0\\
0.0256006400160004	0\\
0.0257006425160629	0\\
0.0258006450161254	0\\
0.0259006475161879	0\\
0.0260006500162504	0\\
0.0261006525163129	0\\
0.0262006550163754	0\\
0.0263006575164379	0\\
0.0264006600165004	0\\
0.0265006625165629	0\\
0.0266006650166254	0\\
0.0267006675166879	0\\
0.0268006700167504	0\\
0.0269006725168129	0\\
0.0270006750168754	0\\
0.0271006775169379	0\\
0.0272006800170004	0\\
0.0273006825170629	0\\
0.0274006850171254	0\\
0.0275006875171879	0\\
0.0276006900172504	0\\
0.0277006925173129	0\\
0.0278006950173754	0\\
0.0279006975174379	0\\
0.0280007000175004	0\\
0.0281007025175629	0\\
0.0282007050176254	0\\
0.0283007075176879	0\\
0.0284007100177504	0\\
0.0285007125178129	0\\
0.0286007150178754	0\\
0.028700717517938	0\\
0.0288007200180005	0\\
0.028900722518063	0\\
0.0290007250181255	0\\
0.029100727518188	0\\
0.0292007300182505	0\\
0.029300732518313	0\\
0.0294007350183755	0\\
0.029500737518438	0\\
0.0296007400185005	0\\
0.029700742518563	0\\
0.0298007450186255	0\\
0.029900747518688	0\\
0.0300007500187505	0\\
0.030100752518813	0\\
0.0302007550188755	0\\
0.030300757518938	0\\
0.0304007600190005	0\\
0.030500762519063	0\\
0.0306007650191255	0\\
0.030700767519188	0\\
0.0308007700192505	0\\
0.030900772519313	0\\
0.0310007750193755	0\\
0.031100777519438	0\\
0.0312007800195005	0\\
0.031300782519563	0\\
0.0314007850196255	0\\
0.031500787519688	0\\
0.0316007900197505	0\\
0.031700792519813	0\\
0.0318007950198755	0\\
0.031900797519938	0\\
0.0320008000200005	0\\
0.032100802520063	0\\
0.0322008050201255	0\\
0.032300807520188	0\\
0.0324008100202505	0\\
0.032500812520313	0\\
0.0326008150203755	0\\
0.032700817520438	0\\
0.0328008200205005	0\\
0.032900822520563	0\\
0.0330008250206255	0\\
0.033100827520688	0\\
0.0332008300207505	0\\
0.033300832520813	0\\
0.0334008350208755	0\\
0.033500837520938	0\\
0.0336008400210005	0\\
0.033700842521063	0\\
0.0338008450211255	0\\
0.033900847521188	0\\
0.0340008500212505	0\\
0.034100852521313	0\\
0.0342008550213755	0\\
0.034300857521438	0\\
0.0344008600215005	0\\
0.034500862521563	0\\
0.0346008650216255	0\\
0.034700867521688	0\\
0.0348008700217505	0\\
0.034900872521813	0\\
0.0350008750218755	0\\
0.0351008775219381	0\\
0.0352008800220006	0\\
0.0353008825220631	0\\
0.0354008850221256	0\\
0.0355008875221881	0\\
0.0356008900222506	0\\
0.0357008925223131	0\\
0.0358008950223756	0\\
0.0359008975224381	0\\
0.0360009000225006	0\\
0.0361009025225631	0\\
0.0362009050226256	0\\
0.0363009075226881	0\\
0.0364009100227506	0\\
0.0365009125228131	0\\
0.0366009150228756	0\\
0.0367009175229381	0\\
0.0368009200230006	0\\
0.0369009225230631	0\\
0.0370009250231256	0\\
0.0371009275231881	0\\
0.0372009300232506	0\\
0.0373009325233131	0\\
0.0374009350233756	0\\
0.0375009375234381	0\\
0.0376009400235006	0\\
0.0377009425235631	0\\
0.0378009450236256	0\\
0.0379009475236881	0\\
0.0380009500237506	0\\
0.0381009525238131	0\\
0.0382009550238756	0\\
0.0383009575239381	0\\
0.0384009600240006	0\\
0.0385009625240631	0\\
0.0386009650241256	0\\
0.0387009675241881	0\\
0.0388009700242506	0\\
0.0389009725243131	0\\
0.0390009750243756	0\\
0.0391009775244381	0\\
0.0392009800245006	0\\
0.0393009825245631	0\\
0.0394009850246256	0\\
0.0395009875246881	0\\
0.0396009900247506	0\\
0.0397009925248131	0\\
0.0398009950248756	0\\
0.0399009975249381	0\\
0.0400010000250006	0\\
0.0401010025250631	0\\
0.0402010050251256	0\\
0.0403010075251881	0\\
0.0404010100252506	0\\
0.0405010125253131	0\\
0.0406010150253756	0\\
0.0407010175254381	0\\
0.0408010200255006	0\\
0.0409010225255631	0\\
0.0410010250256256	0\\
0.0411010275256881	0\\
0.0412010300257506	0\\
0.0413010325258131	0\\
0.0414010350258756	0\\
0.0415010375259382	0\\
0.0416010400260007	0\\
0.0417010425260632	0\\
0.0418010450261257	0\\
0.0419010475261882	0\\
0.0420010500262507	0\\
0.0421010525263132	0\\
0.0422010550263757	0\\
0.0423010575264382	0\\
0.0424010600265007	0\\
0.0425010625265632	0\\
0.0426010650266257	0\\
0.0427010675266882	0\\
0.0428010700267507	0\\
0.0429010725268132	0\\
0.0430010750268757	0\\
0.0431010775269382	0\\
0.0432010800270007	0\\
0.0433010825270632	0\\
0.0434010850271257	0\\
0.0435010875271882	0\\
0.0436010900272507	0\\
0.0437010925273132	0\\
0.0438010950273757	0\\
0.0439010975274382	0\\
0.0440011000275007	0\\
0.0441011025275632	0\\
0.0442011050276257	0\\
0.0443011075276882	0\\
0.0444011100277507	0\\
0.0445011125278132	0\\
0.0446011150278757	0\\
0.0447011175279382	0\\
0.0448011200280007	0\\
0.0449011225280632	0\\
0.0450011250281257	0\\
0.0451011275281882	0\\
0.0452011300282507	0\\
0.0453011325283132	0\\
0.0454011350283757	0\\
0.0455011375284382	0\\
0.0456011400285007	0\\
0.0457011425285632	0\\
0.0458011450286257	0\\
0.0459011475286882	0\\
0.0460011500287507	0\\
0.0461011525288132	0\\
0.0462011550288757	0\\
0.0463011575289382	0\\
0.0464011600290007	0\\
0.0465011625290632	0\\
0.0466011650291257	0\\
0.0467011675291882	0\\
0.0468011700292507	0\\
0.0469011725293132	0\\
0.0470011750293757	0\\
0.0471011775294382	0\\
0.0472011800295007	0\\
0.0473011825295632	0\\
0.0474011850296257	0\\
0.0475011875296882	0\\
0.0476011900297507	0\\
0.0477011925298132	0\\
0.0478011950298757	0\\
0.0479011975299382	0\\
0.0480012000300008	0\\
0.0481012025300633	0\\
0.0482012050301258	0\\
0.0483012075301883	0\\
0.0484012100302508	0\\
0.0485012125303133	0\\
0.0486012150303758	0\\
0.0487012175304383	0\\
0.0488012200305008	0\\
0.0489012225305633	0\\
0.0490012250306258	0\\
0.0491012275306883	0\\
0.0492012300307508	0\\
0.0493012325308133	0\\
0.0494012350308758	0\\
0.0495012375309383	0\\
0.0496012400310008	0\\
0.0497012425310633	0\\
0.0498012450311258	0\\
0.0499012475311883	0\\
0.0500012500312508	0\\
0.0501012525313133	0\\
0.0502012550313758	0\\
0.0503012575314383	0\\
0.0504012600315008	0\\
0.0505012625315633	0\\
0.0506012650316258	0\\
0.0507012675316883	0\\
0.0508012700317508	0\\
0.0509012725318133	0\\
0.0510012750318758	0\\
0.0511012775319383	0\\
0.0512012800320008	0\\
0.0513012825320633	0\\
0.0514012850321258	0\\
0.0515012875321883	0\\
0.0516012900322508	0\\
0.0517012925323133	0\\
0.0518012950323758	0\\
0.0519012975324383	0\\
0.0520013000325008	0\\
0.0521013025325633	0\\
0.0522013050326258	0\\
0.0523013075326883	0\\
0.0524013100327508	0\\
0.0525013125328133	0\\
0.0526013150328758	0\\
0.0527013175329383	0\\
0.0528013200330008	0\\
0.0529013225330633	0\\
0.0530013250331258	0\\
0.0531013275331883	0\\
0.0532013300332508	0\\
0.0533013325333133	0\\
0.0534013350333758	0\\
0.0535013375334383	0\\
0.0536013400335008	0\\
0.0537013425335633	0\\
0.0538013450336258	0\\
0.0539013475336883	0\\
0.0540013500337508	0\\
0.0541013525338133	0\\
0.0542013550338758	0\\
0.0543013575339383	0\\
0.0544013600340008	0\\
0.0545013625340634	0\\
0.0546013650341259	0\\
0.0547013675341884	0\\
0.0548013700342509	0\\
0.0549013725343134	0\\
0.0550013750343759	0\\
0.0551013775344384	0\\
0.0552013800345009	0\\
0.0553013825345634	0\\
0.0554013850346259	0\\
0.0555013875346884	0\\
0.0556013900347509	0\\
0.0557013925348134	0\\
0.0558013950348759	0\\
0.0559013975349384	0\\
0.0560014000350009	0\\
0.0561014025350634	0\\
0.0562014050351259	0\\
0.0563014075351884	0\\
0.0564014100352509	0\\
0.0565014125353134	0\\
0.0566014150353759	0\\
0.0567014175354384	0\\
0.0568014200355009	0\\
0.0569014225355634	0\\
0.0570014250356259	0\\
0.0571014275356884	0\\
0.0572014300357509	0\\
0.0573014325358134	0\\
0.0574014350358759	0\\
0.0575014375359384	0\\
0.0576014400360009	0\\
0.0577014425360634	0\\
0.0578014450361259	0\\
0.0579014475361884	0\\
0.0580014500362509	0\\
0.0581014525363134	0\\
0.0582014550363759	0\\
0.0583014575364384	0\\
0.0584014600365009	0\\
0.0585014625365634	0\\
0.0586014650366259	0\\
0.0587014675366884	0\\
0.0588014700367509	0\\
0.0589014725368134	0\\
0.0590014750368759	0\\
0.0591014775369384	0\\
0.0592014800370009	0\\
0.0593014825370634	0\\
0.0594014850371259	0\\
0.0595014875371884	0\\
0.0596014900372509	0\\
0.0597014925373134	0\\
0.0598014950373759	0\\
0.0599014975374384	0\\
0.0600015000375009	0\\
0.0601015025375634	0\\
0.0602015050376259	0\\
0.0603015075376884	0\\
0.0604015100377509	0\\
0.0605015125378134	0\\
0.0606015150378759	0\\
0.0607015175379384	0\\
0.0608015200380009	0\\
0.0609015225380635	0\\
0.061001525038126	0\\
0.0611015275381885	0\\
0.061201530038251	0\\
0.0613015325383135	0\\
0.061401535038376	0\\
0.0615015375384385	0\\
0.061601540038501	0\\
0.0617015425385635	0\\
0.061801545038626	0\\
0.0619015475386885	0\\
0.062001550038751	0\\
0.0621015525388135	0\\
0.062201555038876	0\\
0.0623015575389385	0\\
0.062401560039001	0\\
0.0625015625390635	0\\
0.062601565039126	0\\
0.0627015675391885	0\\
0.062801570039251	0\\
0.0629015725393135	0\\
0.063001575039376	0\\
0.0631015775394385	0\\
0.063201580039501	0\\
0.0633015825395635	0\\
0.063401585039626	0\\
0.0635015875396885	0\\
0.063601590039751	0\\
0.0637015925398135	0\\
0.063801595039876	0\\
0.0639015975399385	0\\
0.064001600040001	0\\
0.0641016025400635	0\\
0.064201605040126	0\\
0.0643016075401885	0\\
0.064401610040251	0\\
0.0645016125403135	0\\
0.064601615040376	0\\
0.0647016175404385	0\\
0.064801620040501	0\\
0.0649016225405635	0\\
0.065001625040626	0\\
0.0651016275406885	0\\
0.065201630040751	0\\
0.0653016325408135	0\\
0.065401635040876	0\\
0.0655016375409385	0\\
0.065601640041001	0\\
0.0657016425410635	0\\
0.065801645041126	0\\
0.0659016475411885	0\\
0.066001650041251	0\\
0.0661016525413135	0\\
0.066201655041376	0\\
0.0663016575414385	0\\
0.066401660041501	0\\
0.0665016625415635	0\\
0.066601665041626	0\\
0.0667016675416885	0\\
0.066801670041751	0\\
0.0669016725418135	0\\
0.067001675041876	0\\
0.0671016775419385	0\\
0.0672016800420011	0\\
0.0673016825420635	0\\
0.0674016850421261	0\\
0.0675016875421885	0\\
0.0676016900422511	0\\
0.0677016925423136	0\\
0.0678016950423761	0\\
0.0679016975424386	0\\
0.0680017000425011	0\\
0.0681017025425636	0\\
0.0682017050426261	0\\
0.0683017075426886	0\\
0.0684017100427511	0\\
0.0685017125428136	0\\
0.0686017150428761	0\\
0.0687017175429386	0\\
0.0688017200430011	0\\
0.0689017225430636	0\\
0.0690017250431261	0\\
0.0691017275431886	0\\
0.0692017300432511	0\\
0.0693017325433136	0\\
0.0694017350433761	0\\
0.0695017375434386	0\\
0.0696017400435011	0\\
0.0697017425435636	0\\
0.0698017450436261	0\\
0.0699017475436886	0\\
0.0700017500437511	0\\
0.0701017525438136	0\\
0.0702017550438761	0\\
0.0703017575439386	0\\
0.0704017600440011	0\\
0.0705017625440636	0\\
0.0706017650441261	0\\
0.0707017675441886	0\\
0.0708017700442511	0\\
0.0709017725443136	0\\
0.0710017750443761	0\\
0.0711017775444386	0\\
0.0712017800445011	0\\
0.0713017825445636	0\\
0.0714017850446261	0\\
0.0715017875446886	0\\
0.0716017900447511	0\\
0.0717017925448136	0\\
0.0718017950448761	0\\
0.0719017975449386	0\\
0.0720018000450011	0\\
0.0721018025450636	0\\
0.0722018050451261	0\\
0.0723018075451886	0\\
0.0724018100452511	0\\
0.0725018125453136	0\\
0.0726018150453761	0\\
0.0727018175454386	0\\
0.0728018200455011	0\\
0.0729018225455636	0\\
0.0730018250456261	0\\
0.0731018275456886	0\\
0.0732018300457511	0\\
0.0733018325458136	0\\
0.0734018350458761	0\\
0.0735018375459386	0\\
0.0736018400460011	0\\
0.0737018425460637	0\\
0.0738018450461261	0\\
0.0739018475461887	0\\
0.0740018500462511	0\\
0.0741018525463137	0\\
0.0742018550463762	0\\
0.0743018575464387	0\\
0.0744018600465012	0\\
0.0745018625465637	0\\
0.0746018650466262	0\\
0.0747018675466887	0\\
0.0748018700467512	0\\
0.0749018725468137	0\\
0.0750018750468762	0\\
0.0751018775469387	0\\
0.0752018800470012	0\\
0.0753018825470637	0\\
0.0754018850471262	0\\
0.0755018875471887	0\\
0.0756018900472512	0\\
0.0757018925473137	0\\
0.0758018950473762	0\\
0.0759018975474387	0\\
0.0760019000475012	0\\
0.0761019025475637	0\\
0.0762019050476262	0\\
0.0763019075476887	0\\
0.0764019100477512	0\\
0.0765019125478137	0\\
0.0766019150478762	0\\
0.0767019175479387	0\\
0.0768019200480012	0\\
0.0769019225480637	0\\
0.0770019250481262	0\\
0.0771019275481887	0\\
0.0772019300482512	0\\
0.0773019325483137	0\\
0.0774019350483762	0\\
0.0775019375484387	0\\
0.0776019400485012	0\\
0.0777019425485637	0\\
0.0778019450486262	0\\
0.0779019475486887	0\\
0.0780019500487512	0\\
0.0781019525488137	0\\
0.0782019550488762	0\\
0.0783019575489387	0\\
0.0784019600490012	0\\
0.0785019625490637	0\\
0.0786019650491262	0\\
0.0787019675491887	0\\
0.0788019700492512	0\\
0.0789019725493137	0\\
0.0790019750493762	0\\
0.0791019775494387	0\\
0.0792019800495012	0\\
0.0793019825495637	0\\
0.0794019850496262	0\\
0.0795019875496887	0\\
0.0796019900497512	0\\
0.0797019925498137	0\\
0.0798019950498762	0\\
0.0799019975499387	0\\
0.0800020000500013	0\\
0.0801020025500638	0\\
0.0802020050501263	0\\
0.0803020075501888	0\\
0.0804020100502513	0\\
0.0805020125503138	0\\
0.0806020150503763	0\\
0.0807020175504388	0\\
0.0808020200505013	0\\
0.0809020225505638	0\\
0.0810020250506263	0\\
0.0811020275506888	0\\
0.0812020300507513	0\\
0.0813020325508138	0\\
0.0814020350508763	0\\
0.0815020375509388	0\\
0.0816020400510013	0\\
0.0817020425510638	0\\
0.0818020450511263	0\\
0.0819020475511888	0\\
0.0820020500512513	0\\
0.0821020525513138	0\\
0.0822020550513763	0\\
0.0823020575514388	0\\
0.0824020600515013	0\\
0.0825020625515638	0\\
0.0826020650516263	0\\
0.0827020675516888	0\\
0.0828020700517513	0\\
0.0829020725518138	0\\
0.0830020750518763	0\\
0.0831020775519388	0\\
0.0832020800520013	0\\
0.0833020825520638	0\\
0.0834020850521263	0\\
0.0835020875521888	0\\
0.0836020900522513	0\\
0.0837020925523138	0\\
0.0838020950523763	0\\
0.0839020975524388	0\\
0.0840021000525013	0\\
0.0841021025525638	0\\
0.0842021050526263	0\\
0.0843021075526888	0\\
0.0844021100527513	0\\
0.0845021125528138	0\\
0.0846021150528763	0\\
0.0847021175529388	0\\
0.0848021200530013	0\\
0.0849021225530638	0\\
0.0850021250531263	0\\
0.0851021275531888	0\\
0.0852021300532513	0\\
0.0853021325533138	0\\
0.0854021350533763	0\\
0.0855021375534388	0\\
0.0856021400535013	0\\
0.0857021425535638	0\\
0.0858021450536263	0\\
0.0859021475536888	0\\
0.0860021500537514	0\\
0.0861021525538138	0\\
0.0862021550538764	0\\
0.0863021575539388	0\\
0.0864021600540014	0\\
0.0865021625540639	0\\
0.0866021650541264	0\\
0.0867021675541889	0\\
0.0868021700542514	0\\
0.0869021725543139	0\\
0.0870021750543764	0\\
0.0871021775544389	0\\
0.0872021800545014	0\\
0.0873021825545639	0\\
0.0874021850546264	0\\
0.0875021875546889	0\\
0.0876021900547514	0\\
0.0877021925548139	0\\
0.0878021950548764	0\\
0.0879021975549389	0\\
0.0880022000550014	0\\
0.0881022025550639	0\\
0.0882022050551264	0\\
0.0883022075551889	0\\
0.0884022100552514	0\\
0.0885022125553139	0\\
0.0886022150553764	0\\
0.0887022175554389	0\\
0.0888022200555014	0\\
0.0889022225555639	0\\
0.0890022250556264	0\\
0.0891022275556889	0\\
0.0892022300557514	0\\
0.0893022325558139	0\\
0.0894022350558764	0\\
0.0895022375559389	0\\
0.0896022400560014	0\\
0.0897022425560639	0\\
0.0898022450561264	0\\
0.0899022475561889	0\\
0.0900022500562514	0\\
0.0901022525563139	0\\
0.0902022550563764	0\\
0.0903022575564389	0\\
0.0904022600565014	0\\
0.0905022625565639	0\\
0.0906022650566264	0\\
0.0907022675566889	0\\
0.0908022700567514	0\\
0.0909022725568139	0\\
0.0910022750568764	0\\
0.0911022775569389	0\\
0.0912022800570014	0\\
0.0913022825570639	0\\
0.0914022850571264	0\\
0.0915022875571889	0\\
0.0916022900572514	0\\
0.0917022925573139	0\\
0.0918022950573764	0\\
0.0919022975574389	0\\
0.0920023000575014	0\\
0.0921023025575639	0\\
0.0922023050576264	0\\
0.0923023075576889	0\\
0.0924023100577514	0\\
0.092502312557814	0\\
0.0926023150578764	0\\
0.092702317557939	0\\
0.0928023200580014	0\\
0.092902322558064	0\\
0.0930023250581265	0\\
0.093102327558189	0\\
0.0932023300582515	0\\
0.093302332558314	0\\
0.0934023350583765	0\\
0.093502337558439	0\\
0.0936023400585015	0\\
0.093702342558564	0\\
0.0938023450586265	0\\
0.093902347558689	0\\
0.0940023500587515	0\\
0.094102352558814	0\\
0.0942023550588765	0\\
0.094302357558939	0\\
0.0944023600590015	0\\
0.094502362559064	0\\
0.0946023650591265	0\\
0.094702367559189	0\\
0.0948023700592515	0\\
0.094902372559314	0\\
0.0950023750593765	0\\
0.095102377559439	0\\
0.0952023800595015	0\\
0.095302382559564	0\\
0.0954023850596265	0\\
0.095502387559689	0\\
0.0956023900597515	0\\
0.095702392559814	0\\
0.0958023950598765	0\\
0.095902397559939	0\\
0.0960024000600015	0\\
0.096102402560064	0\\
0.0962024050601265	0\\
0.096302407560189	0\\
0.0964024100602515	0\\
0.096502412560314	0\\
0.0966024150603765	0\\
0.096702417560439	0\\
0.0968024200605015	0\\
0.096902422560564	0\\
0.0970024250606265	0\\
0.097102427560689	0\\
0.0972024300607515	0\\
0.097302432560814	0\\
0.0974024350608765	0\\
0.097502437560939	0\\
0.0976024400610015	0\\
0.097702442561064	0\\
0.0978024450611265	0\\
0.097902447561189	0\\
0.0980024500612515	0\\
0.098102452561314	0\\
0.0982024550613765	0\\
0.098302457561439	0\\
0.0984024600615015	0\\
0.098502462561564	0\\
0.0986024650616265	0\\
0.098702467561689	0\\
0.0988024700617515	0\\
0.098902472561814	0\\
0.0990024750618766	0\\
0.099102477561939	0\\
0.0992024800620016	0\\
0.099302482562064	0\\
0.0994024850621266	0\\
0.0995024875621891	0\\
0.0996024900622516	0\\
0.0997024925623141	0\\
0.0998024950623766	0\\
0.0999024975624391	0\\
0.100002500062502	0\\
0.100102502562564	0\\
0.100202505062627	0\\
0.100302507562689	0\\
0.100402510062752	0\\
0.100502512562814	0\\
0.100602515062877	0\\
0.100702517562939	0\\
0.100802520063002	0\\
0.100902522563064	0\\
0.101002525063127	0\\
0.101102527563189	0\\
0.101202530063252	0\\
0.101302532563314	0\\
0.101402535063377	0\\
0.101502537563439	0\\
0.101602540063502	0\\
0.101702542563564	0\\
0.101802545063627	0\\
0.101902547563689	0\\
0.102002550063752	0\\
0.102102552563814	0\\
0.102202555063877	0\\
0.102302557563939	0\\
0.102402560064002	0\\
0.102502562564064	0\\
0.102602565064127	0\\
0.102702567564189	0\\
0.102802570064252	0\\
0.102902572564314	0\\
0.103002575064377	0\\
0.103102577564439	0\\
0.103202580064502	0\\
0.103302582564564	0\\
0.103402585064627	0\\
0.103502587564689	0\\
0.103602590064752	0\\
0.103702592564814	0\\
0.103802595064877	0\\
0.103902597564939	0\\
0.104002600065002	0\\
0.104102602565064	0\\
0.104202605065127	0\\
0.104302607565189	0\\
0.104402610065252	0\\
0.104502612565314	0\\
0.104602615065377	0\\
0.104702617565439	0\\
0.104802620065502	0\\
0.104902622565564	0\\
0.105002625065627	0\\
0.105102627565689	0\\
0.105202630065752	0\\
0.105302632565814	0\\
0.105402635065877	0\\
0.105502637565939	0\\
0.105602640066002	0\\
0.105702642566064	0\\
0.105802645066127	0\\
0.105902647566189	0\\
0.106002650066252	0\\
0.106102652566314	0\\
0.106202655066377	0\\
0.106302657566439	0\\
0.106402660066502	0\\
0.106502662566564	0\\
0.106602665066627	0\\
0.106702667566689	0\\
0.106802670066752	0\\
0.106902672566814	0\\
0.107002675066877	0\\
0.107102677566939	0\\
0.107202680067002	0\\
0.107302682567064	0\\
0.107402685067127	0\\
0.107502687567189	0\\
0.107602690067252	0\\
0.107702692567314	0\\
0.107802695067377	0\\
0.107902697567439	0\\
0.108002700067502	0\\
0.108102702567564	0\\
0.108202705067627	0\\
0.108302707567689	0\\
0.108402710067752	0\\
0.108502712567814	0\\
0.108602715067877	0\\
0.108702717567939	0\\
0.108802720068002	0\\
0.108902722568064	0\\
0.109002725068127	0\\
0.109102727568189	0\\
0.109202730068252	0\\
0.109302732568314	0\\
0.109402735068377	0\\
0.109502737568439	0\\
0.109602740068502	0\\
0.109702742568564	0\\
0.109802745068627	0\\
0.109902747568689	0\\
0.110002750068752	0\\
0.110102752568814	0\\
0.110202755068877	0\\
0.110302757568939	0\\
0.110402760069002	0\\
0.110502762569064	0\\
0.110602765069127	0\\
0.110702767569189	0\\
0.110802770069252	0\\
0.110902772569314	0\\
0.111002775069377	0\\
0.111102777569439	0\\
0.111202780069502	0\\
0.111302782569564	0\\
0.111402785069627	0\\
0.111502787569689	0\\
0.111602790069752	0\\
0.111702792569814	0\\
0.111802795069877	0\\
0.111902797569939	0\\
0.112002800070002	0\\
0.112102802570064	0\\
0.112202805070127	0\\
0.112302807570189	0\\
0.112402810070252	0\\
0.112502812570314	0\\
0.112602815070377	0\\
0.112702817570439	0\\
0.112802820070502	0\\
0.112902822570564	0\\
0.113002825070627	0\\
0.113102827570689	0\\
0.113202830070752	0\\
0.113302832570814	0\\
0.113402835070877	0\\
0.113502837570939	0\\
0.113602840071002	0\\
0.113702842571064	0\\
0.113802845071127	0\\
0.113902847571189	0\\
0.114002850071252	0\\
0.114102852571314	0\\
0.114202855071377	0\\
0.114302857571439	0\\
0.114402860071502	0\\
0.114502862571564	0\\
0.114602865071627	0\\
0.114702867571689	0\\
0.114802870071752	0\\
0.114902872571814	0\\
0.115002875071877	0\\
0.115102877571939	0\\
0.115202880072002	0\\
0.115302882572064	0\\
0.115402885072127	0\\
0.115502887572189	0\\
0.115602890072252	0\\
0.115702892572314	0\\
0.115802895072377	0\\
0.115902897572439	0\\
0.116002900072502	0\\
0.116102902572564	0\\
0.116202905072627	0\\
0.116302907572689	0\\
0.116402910072752	0\\
0.116502912572814	0\\
0.116602915072877	0\\
0.116702917572939	0\\
0.116802920073002	0\\
0.116902922573064	0\\
0.117002925073127	0\\
0.117102927573189	0\\
0.117202930073252	0\\
0.117302932573314	0\\
0.117402935073377	0\\
0.117502937573439	0\\
0.117602940073502	0\\
0.117702942573564	0\\
0.117802945073627	0\\
0.117902947573689	0\\
0.118002950073752	0\\
0.118102952573814	0\\
0.118202955073877	0\\
0.118302957573939	0\\
0.118402960074002	0\\
0.118502962574064	0\\
0.118602965074127	0\\
0.118702967574189	0\\
0.118802970074252	0\\
0.118902972574314	0\\
0.119002975074377	0\\
0.119102977574439	0\\
0.119202980074502	0\\
0.119302982574564	0\\
0.119402985074627	0\\
0.119502987574689	0\\
0.119602990074752	0\\
0.119702992574814	0\\
0.119802995074877	0\\
0.119902997574939	0\\
0.120003000075002	0\\
0.120103002575064	0\\
0.120203005075127	0\\
0.120303007575189	0\\
0.120403010075252	0\\
0.120503012575314	0\\
0.120603015075377	0\\
0.120703017575439	0\\
0.120803020075502	0\\
0.120903022575564	0\\
0.121003025075627	0\\
0.121103027575689	0\\
0.121203030075752	0\\
0.121303032575814	0\\
0.121403035075877	0\\
0.121503037575939	0\\
0.121603040076002	0\\
0.121703042576064	0\\
0.121803045076127	0\\
0.121903047576189	0\\
0.122003050076252	0\\
0.122103052576314	0\\
0.122203055076377	0\\
0.122303057576439	0\\
0.122403060076502	0\\
0.122503062576564	0\\
0.122603065076627	0\\
0.122703067576689	0\\
0.122803070076752	0\\
0.122903072576814	0\\
0.123003075076877	0\\
0.123103077576939	0\\
0.123203080077002	0\\
0.123303082577064	0\\
0.123403085077127	0\\
0.123503087577189	0\\
0.123603090077252	0\\
0.123703092577314	0\\
0.123803095077377	0\\
0.123903097577439	0\\
0.124003100077502	0\\
0.124103102577564	0\\
0.124203105077627	0\\
0.124303107577689	0\\
0.124403110077752	0\\
0.124503112577814	0\\
0.124603115077877	0\\
0.124703117577939	0\\
0.124803120078002	0\\
0.124903122578064	0\\
0.125003125078127	0\\
0.125103127578189	0\\
0.125203130078252	0\\
0.125303132578314	0\\
0.125403135078377	0\\
0.125503137578439	0\\
0.125603140078502	0\\
0.125703142578564	0\\
0.125803145078627	0\\
0.125903147578689	0\\
0.126003150078752	0\\
0.126103152578814	0\\
0.126203155078877	0\\
0.126303157578939	0\\
0.126403160079002	0\\
0.126503162579064	0\\
0.126603165079127	0\\
0.126703167579189	0\\
0.126803170079252	0\\
0.126903172579314	0\\
0.127003175079377	0\\
0.127103177579439	0\\
0.127203180079502	0\\
0.127303182579564	0\\
0.127403185079627	0\\
0.127503187579689	0\\
0.127603190079752	0\\
0.127703192579814	0\\
0.127803195079877	0\\
0.12790319757994	0\\
0.128003200080002	0\\
0.128103202580064	0\\
0.128203205080127	0\\
0.12830320758019	0\\
0.128403210080252	0\\
0.128503212580314	0\\
0.128603215080377	0\\
0.12870321758044	0\\
0.128803220080502	0\\
0.128903222580565	0\\
0.129003225080627	0\\
0.12910322758069	0\\
0.129203230080752	0\\
0.129303232580815	0\\
0.129403235080877	0\\
0.12950323758094	0\\
0.129603240081002	0\\
0.129703242581065	0\\
0.129803245081127	0\\
0.12990324758119	0\\
0.130003250081252	0\\
0.130103252581315	0\\
0.130203255081377	0\\
0.13030325758144	0\\
0.130403260081502	0\\
0.130503262581565	0\\
0.130603265081627	0\\
0.13070326758169	0\\
0.130803270081752	0\\
0.130903272581815	0\\
0.131003275081877	0\\
0.13110327758194	0\\
0.131203280082002	0\\
0.131303282582065	0\\
0.131403285082127	0\\
0.13150328758219	0\\
0.131603290082252	0\\
0.131703292582315	0\\
0.131803295082377	0\\
0.13190329758244	0\\
0.132003300082502	0\\
0.132103302582565	0\\
0.132203305082627	0\\
0.13230330758269	0\\
0.132403310082752	0\\
0.132503312582815	0\\
0.132603315082877	0\\
0.13270331758294	0\\
0.132803320083002	0\\
0.132903322583065	0\\
0.133003325083127	0\\
0.13310332758319	0\\
0.133203330083252	0\\
0.133303332583315	0\\
0.133403335083377	0\\
0.13350333758344	0\\
0.133603340083502	0\\
0.133703342583565	0\\
0.133803345083627	0\\
0.13390334758369	0\\
0.134003350083752	0\\
0.134103352583815	0\\
0.134203355083877	0\\
0.13430335758394	0\\
0.134403360084002	0\\
0.134503362584065	0\\
0.134603365084127	0\\
0.13470336758419	0\\
0.134803370084252	0\\
0.134903372584315	0\\
0.135003375084377	0\\
0.13510337758444	0\\
0.135203380084502	0\\
0.135303382584565	0\\
0.135403385084627	0\\
0.13550338758469	0\\
0.135603390084752	0\\
0.135703392584815	0\\
0.135803395084877	0\\
0.13590339758494	0\\
0.136003400085002	0\\
0.136103402585065	0\\
0.136203405085127	0\\
0.13630340758519	0\\
0.136403410085252	0\\
0.136503412585315	0\\
0.136603415085377	0\\
0.13670341758544	0\\
0.136803420085502	0\\
0.136903422585565	0\\
0.137003425085627	0\\
0.13710342758569	0\\
0.137203430085752	0\\
0.137303432585815	0\\
0.137403435085877	0\\
0.13750343758594	0\\
0.137603440086002	0\\
0.137703442586065	0\\
0.137803445086127	0\\
0.13790344758619	0\\
0.138003450086252	0\\
0.138103452586315	0\\
0.138203455086377	0\\
0.13830345758644	0\\
0.138403460086502	0\\
0.138503462586565	0\\
0.138603465086627	0\\
0.13870346758669	0\\
0.138803470086752	0\\
0.138903472586815	0\\
0.139003475086877	0\\
0.13910347758694	0\\
0.139203480087002	0\\
0.139303482587065	0\\
0.139403485087127	0\\
0.13950348758719	0\\
0.139603490087252	0\\
0.139703492587315	0\\
0.139803495087377	0\\
0.13990349758744	0\\
0.140003500087502	0\\
0.140103502587565	0\\
0.140203505087627	0\\
0.14030350758769	0\\
0.140403510087752	0\\
0.140503512587815	0\\
0.140603515087877	0\\
0.14070351758794	0\\
0.140803520088002	0\\
0.140903522588065	0\\
0.141003525088127	0\\
0.14110352758819	0\\
0.141203530088252	0\\
0.141303532588315	0\\
0.141403535088377	0\\
0.14150353758844	0\\
0.141603540088502	0\\
0.141703542588565	0\\
0.141803545088627	0\\
0.14190354758869	0\\
0.142003550088752	0\\
0.142103552588815	0\\
0.142203555088877	0\\
0.14230355758894	0\\
0.142403560089002	0\\
0.142503562589065	0\\
0.142603565089127	0\\
0.14270356758919	0\\
0.142803570089252	0\\
0.142903572589315	0\\
0.143003575089377	0\\
0.14310357758944	0\\
0.143203580089502	0\\
0.143303582589565	0\\
0.143403585089627	0\\
0.14350358758969	0\\
0.143603590089752	0\\
0.143703592589815	0\\
0.143803595089877	0\\
0.14390359758994	0\\
0.144003600090002	0\\
0.144103602590065	0\\
0.144203605090127	0\\
0.14430360759019	0\\
0.144403610090252	0\\
0.144503612590315	0\\
0.144603615090377	0\\
0.14470361759044	0\\
0.144803620090502	0\\
0.144903622590565	0\\
0.145003625090627	0\\
0.14510362759069	0\\
0.145203630090752	0\\
0.145303632590815	0\\
0.145403635090877	0\\
0.14550363759094	0\\
0.145603640091002	0\\
0.145703642591065	0\\
0.145803645091127	0\\
0.14590364759119	0\\
0.146003650091252	0\\
0.146103652591315	0\\
0.146203655091377	0\\
0.14630365759144	0\\
0.146403660091502	0\\
0.146503662591565	0\\
0.146603665091627	0\\
0.14670366759169	0\\
0.146803670091752	0\\
0.146903672591815	0\\
0.147003675091877	0\\
0.14710367759194	0\\
0.147203680092002	0\\
0.147303682592065	0\\
0.147403685092127	0\\
0.14750368759219	0\\
0.147603690092252	0\\
0.147703692592315	0\\
0.147803695092377	0\\
0.14790369759244	0\\
0.148003700092502	0\\
0.148103702592565	0\\
0.148203705092627	0\\
0.14830370759269	0\\
0.148403710092752	0\\
0.148503712592815	0\\
0.148603715092877	0\\
0.14870371759294	0\\
0.148803720093002	0\\
0.148903722593065	0\\
0.149003725093127	0\\
0.14910372759319	0\\
0.149203730093252	0\\
0.149303732593315	0\\
0.149403735093377	0\\
0.14950373759344	0\\
0.149603740093502	0\\
0.149703742593565	0\\
0.149803745093627	0\\
0.14990374759369	0\\
0.150003750093752	0\\
0.150103752593815	0\\
0.150203755093877	0\\
0.15030375759394	0\\
0.150403760094002	0\\
0.150503762594065	0\\
0.150603765094127	0\\
0.15070376759419	0\\
0.150803770094252	0\\
0.150903772594315	0\\
0.151003775094377	0\\
0.15110377759444	0\\
0.151203780094502	0\\
0.151303782594565	0\\
0.151403785094627	0\\
0.15150378759469	0\\
0.151603790094752	0\\
0.151703792594815	0\\
0.151803795094877	0\\
0.15190379759494	0\\
0.152003800095002	0\\
0.152103802595065	0\\
0.152203805095127	0\\
0.15230380759519	0\\
0.152403810095252	0\\
0.152503812595315	0\\
0.152603815095377	0\\
0.15270381759544	0\\
0.152803820095502	0\\
0.152903822595565	0\\
0.153003825095627	0\\
0.15310382759569	0\\
0.153203830095752	0\\
0.153303832595815	0\\
0.153403835095877	0\\
0.15350383759594	0\\
0.153603840096002	0\\
0.153703842596065	0\\
0.153803845096127	0\\
0.15390384759619	0\\
0.154003850096252	0\\
0.154103852596315	0\\
0.154203855096377	0\\
0.15430385759644	0\\
0.154403860096502	0\\
0.154503862596565	0\\
0.154603865096627	0\\
0.15470386759669	0\\
0.154803870096752	0\\
0.154903872596815	0\\
0.155003875096877	0\\
0.15510387759694	0\\
0.155203880097002	0\\
0.155303882597065	0\\
0.155403885097127	0\\
0.15550388759719	0\\
0.155603890097252	0\\
0.155703892597315	0\\
0.155803895097377	0\\
0.15590389759744	0\\
0.156003900097502	0\\
0.156103902597565	0\\
0.156203905097627	0\\
0.15630390759769	0\\
0.156403910097752	0\\
0.156503912597815	0\\
0.156603915097877	0\\
0.15670391759794	0\\
0.156803920098002	0\\
0.156903922598065	0\\
0.157003925098127	0\\
0.15710392759819	0\\
0.157203930098252	0\\
0.157303932598315	0\\
0.157403935098377	0\\
0.15750393759844	0\\
0.157603940098502	0\\
0.157703942598565	0\\
0.157803945098627	0\\
0.15790394759869	0\\
0.158003950098752	0\\
0.158103952598815	0\\
0.158203955098877	0\\
0.15830395759894	0\\
0.158403960099002	0\\
0.158503962599065	0\\
0.158603965099127	0\\
0.15870396759919	0\\
0.158803970099252	0\\
0.158903972599315	0\\
0.159003975099377	0\\
0.15910397759944	0\\
0.159203980099502	0\\
0.159303982599565	0\\
0.159403985099627	0\\
0.15950398759969	0\\
0.159603990099752	0\\
0.159703992599815	0\\
0.159803995099877	0\\
0.15990399759994	0\\
0.160004000100003	0\\
0.160104002600065	0\\
0.160204005100128	0\\
0.16030400760019	0\\
0.160404010100253	0\\
0.160504012600315	0\\
0.160604015100378	0\\
0.16070401760044	0\\
0.160804020100503	0\\
0.160904022600565	0\\
0.161004025100628	0\\
0.16110402760069	0\\
0.161204030100753	0\\
0.161304032600815	0\\
0.161404035100878	0\\
0.16150403760094	0\\
0.161604040101003	0\\
0.161704042601065	0\\
0.161804045101128	0\\
0.16190404760119	0\\
0.162004050101253	0\\
0.162104052601315	0\\
0.162204055101378	0\\
0.16230405760144	0\\
0.162404060101503	0\\
0.162504062601565	0\\
0.162604065101628	0\\
0.16270406760169	0\\
0.162804070101753	0\\
0.162904072601815	0\\
0.163004075101878	0\\
0.16310407760194	0\\
0.163204080102003	0\\
0.163304082602065	0\\
0.163404085102128	0\\
0.16350408760219	0\\
0.163604090102253	0\\
0.163704092602315	0\\
0.163804095102378	0\\
0.16390409760244	0\\
0.164004100102503	0\\
0.164104102602565	0\\
0.164204105102628	0\\
0.16430410760269	0\\
0.164404110102753	0\\
0.164504112602815	0\\
0.164604115102878	0\\
0.16470411760294	0\\
0.164804120103003	0\\
0.164904122603065	0\\
0.165004125103128	0\\
0.16510412760319	0\\
0.165204130103253	0\\
0.165304132603315	0\\
0.165404135103378	0\\
0.16550413760344	0\\
0.165604140103503	0\\
0.165704142603565	0\\
0.165804145103628	0\\
0.16590414760369	0\\
0.166004150103753	0\\
0.166104152603815	0\\
0.166204155103878	0\\
0.16630415760394	0\\
0.166404160104003	0\\
0.166504162604065	0\\
0.166604165104128	0\\
0.16670416760419	0\\
0.166804170104253	0\\
0.166904172604315	0\\
0.167004175104378	0\\
0.16710417760444	0\\
0.167204180104503	0\\
0.167304182604565	0\\
0.167404185104628	0\\
0.16750418760469	0\\
0.167604190104753	0\\
0.167704192604815	0\\
0.167804195104878	0\\
0.16790419760494	0\\
0.168004200105003	0\\
0.168104202605065	0\\
0.168204205105128	0\\
0.16830420760519	0\\
0.168404210105253	0\\
0.168504212605315	0\\
0.168604215105378	0\\
0.16870421760544	0\\
0.168804220105503	0\\
0.168904222605565	0\\
0.169004225105628	0\\
0.16910422760569	0\\
0.169204230105753	0\\
0.169304232605815	0\\
0.169404235105878	0\\
0.16950423760594	0\\
0.169604240106003	0\\
0.169704242606065	0\\
0.169804245106128	0\\
0.16990424760619	0\\
0.170004250106253	0\\
0.170104252606315	0\\
0.170204255106378	0\\
0.17030425760644	0\\
0.170404260106503	0\\
0.170504262606565	0\\
0.170604265106628	0\\
0.17070426760669	0\\
0.170804270106753	0\\
0.170904272606815	0\\
0.171004275106878	0\\
0.17110427760694	0\\
0.171204280107003	0\\
0.171304282607065	0\\
0.171404285107128	0\\
0.17150428760719	0\\
0.171604290107253	0\\
0.171704292607315	0\\
0.171804295107378	0\\
0.17190429760744	0\\
0.172004300107503	0\\
0.172104302607565	0\\
0.172204305107628	0\\
0.17230430760769	0\\
0.172404310107753	0\\
0.172504312607815	0\\
0.172604315107878	0\\
0.17270431760794	0\\
0.172804320108003	0\\
0.172904322608065	0\\
0.173004325108128	0\\
0.17310432760819	0\\
0.173204330108253	0\\
0.173304332608315	0\\
0.173404335108378	0\\
0.17350433760844	0\\
0.173604340108503	0\\
0.173704342608565	0\\
0.173804345108628	0\\
0.17390434760869	0\\
0.174004350108753	0\\
0.174104352608815	0\\
0.174204355108878	0\\
0.17430435760894	0\\
0.174404360109003	0\\
0.174504362609065	0\\
0.174604365109128	0\\
0.17470436760919	0\\
0.174804370109253	0\\
0.174904372609315	0\\
0.175004375109378	0\\
0.17510437760944	0\\
0.175204380109503	0\\
0.175304382609565	0\\
0.175404385109628	0\\
0.17550438760969	0\\
0.175604390109753	0\\
0.175704392609815	0\\
0.175804395109878	0\\
0.17590439760994	0\\
0.176004400110003	0\\
0.176104402610065	0\\
0.176204405110128	0\\
0.17630440761019	0\\
0.176404410110253	0\\
0.176504412610315	0\\
0.176604415110378	0\\
0.17670441761044	0\\
0.176804420110503	0\\
0.176904422610565	0\\
0.177004425110628	0\\
0.17710442761069	0\\
0.177204430110753	0\\
0.177304432610815	0\\
0.177404435110878	0\\
0.17750443761094	0\\
0.177604440111003	0\\
0.177704442611065	0\\
0.177804445111128	0\\
0.17790444761119	0\\
0.178004450111253	0\\
0.178104452611315	0\\
0.178204455111378	0\\
0.17830445761144	0\\
0.178404460111503	0\\
0.178504462611565	0\\
0.178604465111628	0\\
0.17870446761169	0\\
0.178804470111753	0\\
0.178904472611815	0\\
0.179004475111878	0\\
0.17910447761194	0\\
0.179204480112003	0\\
0.179304482612065	0\\
0.179404485112128	0\\
0.17950448761219	0\\
0.179604490112253	0\\
0.179704492612315	0\\
0.179804495112378	0\\
0.17990449761244	0\\
0.180004500112503	0\\
0.180104502612565	0\\
0.180204505112628	0\\
0.18030450761269	0\\
0.180404510112753	0\\
0.180504512612815	0\\
0.180604515112878	0\\
0.18070451761294	0\\
0.180804520113003	0\\
0.180904522613065	0\\
0.181004525113128	0\\
0.18110452761319	0\\
0.181204530113253	0\\
0.181304532613315	0\\
0.181404535113378	0\\
0.18150453761344	0\\
0.181604540113503	0\\
0.181704542613565	0\\
0.181804545113628	0\\
0.18190454761369	0\\
0.182004550113753	0\\
0.182104552613815	0\\
0.182204555113878	0\\
0.18230455761394	0\\
0.182404560114003	0\\
0.182504562614065	0\\
0.182604565114128	0\\
0.18270456761419	0\\
0.182804570114253	0\\
0.182904572614315	0\\
0.183004575114378	0\\
0.18310457761444	0\\
0.183204580114503	0\\
0.183304582614565	0\\
0.183404585114628	0\\
0.18350458761469	0\\
0.183604590114753	0\\
0.183704592614815	0\\
0.183804595114878	0\\
0.18390459761494	0\\
0.184004600115003	0\\
0.184104602615065	0\\
0.184204605115128	0\\
0.18430460761519	0\\
0.184404610115253	0\\
0.184504612615315	0\\
0.184604615115378	0\\
0.18470461761544	0\\
0.184804620115503	0\\
0.184904622615565	0\\
0.185004625115628	0\\
0.18510462761569	0\\
0.185204630115753	0\\
0.185304632615815	0\\
0.185404635115878	0\\
0.18550463761594	0\\
0.185604640116003	0\\
0.185704642616065	0\\
0.185804645116128	0\\
0.18590464761619	0\\
0.186004650116253	0\\
0.186104652616315	0\\
0.186204655116378	0\\
0.18630465761644	0\\
0.186404660116503	0\\
0.186504662616565	0\\
0.186604665116628	0\\
0.18670466761669	0\\
0.186804670116753	0\\
0.186904672616815	0\\
0.187004675116878	0\\
0.18710467761694	0\\
0.187204680117003	0\\
0.187304682617065	0\\
0.187404685117128	0\\
0.18750468761719	0\\
0.187604690117253	0\\
0.187704692617315	0\\
0.187804695117378	0\\
0.18790469761744	0\\
0.188004700117503	0\\
0.188104702617565	0\\
0.188204705117628	0\\
0.18830470761769	0\\
0.188404710117753	0\\
0.188504712617815	0\\
0.188604715117878	0\\
0.18870471761794	0\\
0.188804720118003	0\\
0.188904722618065	0\\
0.189004725118128	0\\
0.18910472761819	0\\
0.189204730118253	0\\
0.189304732618315	0\\
0.189404735118378	0\\
0.18950473761844	0\\
0.189604740118503	0\\
0.189704742618565	0\\
0.189804745118628	0\\
0.18990474761869	0\\
0.190004750118753	0\\
0.190104752618815	0\\
0.190204755118878	0\\
0.19030475761894	0\\
0.190404760119003	0\\
0.190504762619065	0\\
0.190604765119128	0\\
0.19070476761919	0\\
0.190804770119253	0\\
0.190904772619315	0\\
0.191004775119378	0\\
0.19110477761944	0\\
0.191204780119503	0\\
0.191304782619565	0\\
0.191404785119628	0\\
0.191504787619691	0\\
0.191604790119753	0\\
0.191704792619815	0\\
0.191804795119878	0\\
0.191904797619941	0\\
0.192004800120003	0\\
0.192104802620065	0\\
0.192204805120128	0\\
0.192304807620191	0\\
0.192404810120253	0\\
0.192504812620316	0\\
0.192604815120378	0\\
0.192704817620441	0\\
0.192804820120503	0\\
0.192904822620566	0\\
0.193004825120628	0\\
0.193104827620691	0\\
0.193204830120753	0\\
0.193304832620816	0\\
0.193404835120878	0\\
0.193504837620941	0\\
0.193604840121003	0\\
0.193704842621066	0\\
0.193804845121128	0\\
0.193904847621191	0\\
0.194004850121253	0\\
0.194104852621316	0\\
0.194204855121378	0\\
0.194304857621441	0\\
0.194404860121503	0\\
0.194504862621566	0\\
0.194604865121628	0\\
0.194704867621691	0\\
0.194804870121753	0\\
0.194904872621816	0\\
0.195004875121878	0\\
0.195104877621941	0\\
0.195204880122003	0\\
0.195304882622066	0\\
0.195404885122128	0\\
0.195504887622191	0\\
0.195604890122253	0\\
0.195704892622316	0\\
0.195804895122378	0\\
0.195904897622441	0\\
0.196004900122503	0\\
0.196104902622566	0\\
0.196204905122628	0\\
0.196304907622691	0\\
0.196404910122753	0\\
0.196504912622816	0\\
0.196604915122878	0\\
0.196704917622941	0\\
0.196804920123003	0\\
0.196904922623066	0\\
0.197004925123128	0\\
0.197104927623191	0\\
0.197204930123253	0\\
0.197304932623316	0\\
0.197404935123378	0\\
0.197504937623441	0\\
0.197604940123503	0\\
0.197704942623566	0\\
0.197804945123628	0\\
0.197904947623691	0\\
0.198004950123753	0\\
0.198104952623816	0\\
0.198204955123878	0\\
0.198304957623941	0\\
0.198404960124003	0\\
0.198504962624066	0\\
0.198604965124128	0\\
0.198704967624191	0\\
0.198804970124253	0\\
0.198904972624316	0\\
0.199004975124378	0\\
0.199104977624441	0\\
0.199204980124503	0\\
0.199304982624566	0\\
0.199404985124628	0\\
0.199504987624691	0\\
0.199604990124753	0\\
0.199704992624816	0\\
0.199804995124878	0\\
0.199904997624941	0\\
0.200005000125003	0\\
0.200105002625066	0\\
0.200205005125128	0\\
0.200305007625191	0\\
0.200405010125253	0\\
0.200505012625316	0\\
0.200605015125378	0\\
0.200705017625441	0\\
0.200805020125503	0\\
0.200905022625566	0\\
0.201005025125628	0\\
0.201105027625691	0\\
0.201205030125753	0\\
0.201305032625816	0\\
0.201405035125878	0\\
0.201505037625941	0\\
0.201605040126003	0\\
0.201705042626066	0\\
0.201805045126128	0\\
0.201905047626191	0\\
0.202005050126253	0\\
0.202105052626316	0\\
0.202205055126378	0\\
0.202305057626441	0\\
0.202405060126503	0\\
0.202505062626566	0\\
0.202605065126628	0\\
0.202705067626691	0\\
0.202805070126753	0\\
0.202905072626816	0\\
0.203005075126878	0\\
0.203105077626941	0\\
0.203205080127003	0\\
0.203305082627066	0\\
0.203405085127128	0\\
0.203505087627191	0\\
0.203605090127253	0\\
0.203705092627316	0\\
0.203805095127378	0\\
0.203905097627441	0\\
0.204005100127503	0\\
0.204105102627566	0\\
0.204205105127628	0\\
0.204305107627691	0\\
0.204405110127753	0\\
0.204505112627816	0\\
0.204605115127878	0\\
0.204705117627941	0\\
0.204805120128003	0\\
0.204905122628066	0\\
0.205005125128128	0\\
0.205105127628191	0\\
0.205205130128253	0\\
0.205305132628316	0\\
0.205405135128378	0\\
0.205505137628441	0\\
0.205605140128503	0\\
0.205705142628566	0\\
0.205805145128628	0\\
0.205905147628691	0\\
0.206005150128753	0\\
0.206105152628816	0\\
0.206205155128878	0\\
0.206305157628941	0\\
0.206405160129003	0\\
0.206505162629066	0\\
0.206605165129128	0\\
0.206705167629191	0\\
0.206805170129253	0\\
0.206905172629316	0\\
0.207005175129378	0\\
0.207105177629441	0\\
0.207205180129503	0\\
0.207305182629566	0\\
0.207405185129628	0\\
0.207505187629691	0\\
0.207605190129753	0\\
0.207705192629816	0\\
0.207805195129878	0\\
0.207905197629941	0\\
0.208005200130003	0\\
0.208105202630066	0\\
0.208205205130128	0\\
0.208305207630191	0\\
0.208405210130253	0\\
0.208505212630316	0\\
0.208605215130378	0\\
0.208705217630441	0\\
0.208805220130503	0\\
0.208905222630566	0\\
0.209005225130628	0\\
0.209105227630691	0\\
0.209205230130753	0\\
0.209305232630816	0\\
0.209405235130878	0\\
0.209505237630941	0\\
0.209605240131003	0\\
0.209705242631066	0\\
0.209805245131128	0\\
0.209905247631191	0\\
0.210005250131253	0\\
0.210105252631316	0\\
0.210205255131378	0\\
0.210305257631441	0\\
0.210405260131503	0\\
0.210505262631566	0\\
0.210605265131628	0\\
0.210705267631691	0\\
0.210805270131753	0\\
0.210905272631816	0\\
0.211005275131878	0\\
0.211105277631941	0\\
0.211205280132003	0\\
0.211305282632066	0\\
0.211405285132128	0\\
0.211505287632191	0\\
0.211605290132253	0\\
0.211705292632316	0\\
0.211805295132378	0\\
0.211905297632441	0\\
0.212005300132503	0\\
0.212105302632566	0\\
0.212205305132628	0\\
0.212305307632691	0\\
0.212405310132753	0\\
0.212505312632816	0\\
0.212605315132878	0\\
0.212705317632941	0\\
0.212805320133003	0\\
0.212905322633066	0\\
0.213005325133128	0\\
0.213105327633191	0\\
0.213205330133253	0\\
0.213305332633316	0\\
0.213405335133378	0\\
0.213505337633441	0\\
0.213605340133503	0\\
0.213705342633566	0\\
0.213805345133628	0\\
0.213905347633691	0\\
0.214005350133753	0\\
0.214105352633816	0\\
0.214205355133878	0\\
0.214305357633941	0\\
0.214405360134003	0\\
0.214505362634066	0\\
0.214605365134128	0\\
0.214705367634191	0\\
0.214805370134253	0\\
0.214905372634316	0\\
0.215005375134378	0\\
0.215105377634441	0\\
0.215205380134503	0\\
0.215305382634566	0\\
0.215405385134628	0\\
0.215505387634691	0\\
0.215605390134753	0\\
0.215705392634816	0\\
0.215805395134878	0\\
0.215905397634941	0\\
0.216005400135003	0\\
0.216105402635066	0\\
0.216205405135128	0\\
0.216305407635191	0\\
0.216405410135253	0\\
0.216505412635316	0\\
0.216605415135378	0\\
0.216705417635441	0\\
0.216805420135503	0\\
0.216905422635566	0\\
0.217005425135628	0\\
0.217105427635691	0\\
0.217205430135753	0\\
0.217305432635816	0\\
0.217405435135878	0\\
0.217505437635941	0\\
0.217605440136003	0\\
0.217705442636066	0\\
0.217805445136128	0\\
0.217905447636191	0\\
0.218005450136253	0\\
0.218105452636316	0\\
0.218205455136378	0\\
0.218305457636441	0\\
0.218405460136503	0\\
0.218505462636566	0\\
0.218605465136628	0\\
0.218705467636691	0\\
0.218805470136753	0\\
0.218905472636816	0\\
0.219005475136878	0\\
0.219105477636941	0\\
0.219205480137003	0\\
0.219305482637066	0\\
0.219405485137128	0\\
0.219505487637191	0\\
0.219605490137253	0\\
0.219705492637316	0\\
0.219805495137378	0\\
0.219905497637441	0\\
0.220005500137503	0\\
0.220105502637566	0\\
0.220205505137628	0\\
0.220305507637691	0\\
0.220405510137753	0\\
0.220505512637816	0\\
0.220605515137878	0\\
0.220705517637941	0\\
0.220805520138003	0\\
0.220905522638066	0\\
0.221005525138128	0\\
0.221105527638191	0\\
0.221205530138253	0\\
0.221305532638316	0\\
0.221405535138378	0\\
0.221505537638441	0\\
0.221605540138503	0\\
0.221705542638566	0\\
0.221805545138628	0\\
0.221905547638691	0\\
0.222005550138753	0\\
0.222105552638816	0\\
0.222205555138878	0\\
0.222305557638941	0\\
0.222405560139003	0\\
0.222505562639066	0\\
0.222605565139128	0\\
0.222705567639191	0\\
0.222805570139253	0\\
0.222905572639316	0\\
0.223005575139378	0\\
0.223105577639441	0\\
0.223205580139503	0\\
0.223305582639566	0\\
0.223405585139628	0\\
0.223505587639691	0\\
0.223605590139754	0\\
0.223705592639816	0\\
0.223805595139878	0\\
0.223905597639941	0\\
0.224005600140004	0\\
0.224105602640066	0\\
0.224205605140128	0\\
0.224305607640191	0\\
0.224405610140254	0\\
0.224505612640316	0\\
0.224605615140379	0\\
0.224705617640441	0\\
0.224805620140504	0\\
0.224905622640566	0\\
0.225005625140629	0\\
0.225105627640691	0\\
0.225205630140754	0\\
0.225305632640816	0\\
0.225405635140879	0\\
0.225505637640941	0\\
0.225605640141004	0\\
0.225705642641066	0\\
0.225805645141129	0\\
0.225905647641191	0\\
0.226005650141254	0\\
0.226105652641316	0\\
0.226205655141379	0\\
0.226305657641441	0\\
0.226405660141504	0\\
0.226505662641566	0\\
0.226605665141629	0\\
0.226705667641691	0\\
0.226805670141754	0\\
0.226905672641816	0\\
0.227005675141879	0\\
0.227105677641941	0\\
0.227205680142004	0\\
0.227305682642066	0\\
0.227405685142129	0\\
0.227505687642191	0\\
0.227605690142254	0\\
0.227705692642316	0\\
0.227805695142379	0\\
0.227905697642441	0\\
0.228005700142504	0\\
0.228105702642566	0\\
0.228205705142629	0\\
0.228305707642691	0\\
0.228405710142754	0\\
0.228505712642816	0\\
0.228605715142879	0\\
0.228705717642941	0\\
0.228805720143004	0\\
0.228905722643066	0\\
0.229005725143129	0\\
0.229105727643191	0\\
0.229205730143254	0\\
0.229305732643316	0\\
0.229405735143379	0\\
0.229505737643441	0\\
0.229605740143504	0\\
0.229705742643566	0\\
0.229805745143629	0\\
0.229905747643691	0\\
0.230005750143754	0\\
0.230105752643816	0\\
0.230205755143879	0\\
0.230305757643941	0\\
0.230405760144004	0\\
0.230505762644066	0\\
0.230605765144129	0\\
0.230705767644191	0\\
0.230805770144254	0\\
0.230905772644316	0\\
0.231005775144379	0\\
0.231105777644441	0\\
0.231205780144504	0\\
0.231305782644566	0\\
0.231405785144629	0\\
0.231505787644691	0\\
0.231605790144754	0\\
0.231705792644816	0\\
0.231805795144879	0\\
0.231905797644941	0\\
0.232005800145004	0\\
0.232105802645066	0\\
0.232205805145129	0\\
0.232305807645191	0\\
0.232405810145254	0\\
0.232505812645316	0\\
0.232605815145379	0\\
0.232705817645441	0\\
0.232805820145504	0\\
0.232905822645566	0\\
0.233005825145629	0\\
0.233105827645691	0\\
0.233205830145754	0\\
0.233305832645816	0\\
0.233405835145879	0\\
0.233505837645941	0\\
0.233605840146004	0\\
0.233705842646066	0\\
0.233805845146129	0\\
0.233905847646191	0\\
0.234005850146254	0\\
0.234105852646316	0\\
0.234205855146379	0\\
0.234305857646441	0\\
0.234405860146504	0\\
0.234505862646566	0\\
0.234605865146629	0\\
0.234705867646691	0\\
0.234805870146754	0\\
0.234905872646816	0\\
0.235005875146879	0\\
0.235105877646941	0\\
0.235205880147004	0\\
0.235305882647066	0\\
0.235405885147129	0\\
0.235505887647191	0\\
0.235605890147254	0\\
0.235705892647316	0\\
0.235805895147379	0\\
0.235905897647441	0\\
0.236005900147504	0\\
0.236105902647566	0\\
0.236205905147629	0\\
0.236305907647691	0\\
0.236405910147754	0\\
0.236505912647816	0\\
0.236605915147879	0\\
0.236705917647941	0\\
0.236805920148004	0\\
0.236905922648066	0\\
0.237005925148129	0\\
0.237105927648191	0\\
0.237205930148254	0\\
0.237305932648316	0\\
0.237405935148379	0\\
0.237505937648441	0\\
0.237605940148504	0\\
0.237705942648566	0\\
0.237805945148629	0\\
0.237905947648691	0\\
0.238005950148754	0\\
0.238105952648816	0\\
0.238205955148879	0\\
0.238305957648941	0\\
0.238405960149004	0\\
0.238505962649066	0\\
0.238605965149129	0\\
0.238705967649191	0\\
0.238805970149254	0\\
0.238905972649316	0\\
0.239005975149379	0\\
0.239105977649441	0\\
0.239205980149504	0\\
0.239305982649566	0\\
0.239405985149629	0\\
0.239505987649691	0\\
0.239605990149754	0\\
0.239705992649816	0\\
0.239805995149879	0\\
0.239905997649941	0\\
0.240006000150004	0\\
0.240106002650066	0\\
0.240206005150129	0\\
0.240306007650191	0\\
0.240406010150254	0\\
0.240506012650316	0\\
0.240606015150379	0\\
0.240706017650441	0\\
0.240806020150504	0\\
0.240906022650566	0\\
0.241006025150629	0\\
0.241106027650691	0\\
0.241206030150754	0\\
0.241306032650816	0\\
0.241406035150879	0\\
0.241506037650941	0\\
0.241606040151004	0\\
0.241706042651066	0\\
0.241806045151129	0\\
0.241906047651191	0\\
0.242006050151254	0\\
0.242106052651316	0\\
0.242206055151379	0\\
0.242306057651441	0\\
0.242406060151504	0\\
0.242506062651566	0\\
0.242606065151629	0\\
0.242706067651691	0\\
0.242806070151754	0\\
0.242906072651816	0\\
0.243006075151879	0\\
0.243106077651941	0\\
0.243206080152004	0\\
0.243306082652066	0\\
0.243406085152129	0\\
0.243506087652191	0\\
0.243606090152254	0\\
0.243706092652316	0\\
0.243806095152379	0\\
0.243906097652441	0\\
0.244006100152504	0\\
0.244106102652566	0\\
0.244206105152629	0\\
0.244306107652691	0\\
0.244406110152754	0\\
0.244506112652816	0\\
0.244606115152879	0\\
0.244706117652941	0\\
0.244806120153004	0\\
0.244906122653066	0\\
0.245006125153129	0\\
0.245106127653191	0\\
0.245206130153254	0\\
0.245306132653316	0\\
0.245406135153379	0\\
0.245506137653441	0\\
0.245606140153504	0\\
0.245706142653566	0\\
0.245806145153629	0\\
0.245906147653691	0\\
0.246006150153754	0\\
0.246106152653816	0\\
0.246206155153879	0\\
0.246306157653941	0\\
0.246406160154004	0\\
0.246506162654066	0\\
0.246606165154129	0\\
0.246706167654191	0\\
0.246806170154254	0\\
0.246906172654316	0\\
0.247006175154379	0\\
0.247106177654441	0\\
0.247206180154504	0\\
0.247306182654566	0\\
0.247406185154629	0\\
0.247506187654691	0\\
0.247606190154754	0\\
0.247706192654816	0\\
0.247806195154879	0\\
0.247906197654941	0\\
0.248006200155004	0\\
0.248106202655066	0\\
0.248206205155129	0\\
0.248306207655191	0\\
0.248406210155254	0\\
0.248506212655316	0\\
0.248606215155379	0\\
0.248706217655441	0\\
0.248806220155504	0\\
0.248906222655566	0\\
0.249006225155629	0\\
0.249106227655691	0\\
0.249206230155754	0\\
0.249306232655816	0\\
0.249406235155879	0\\
0.249506237655941	0\\
0.249606240156004	0\\
0.249706242656066	0\\
0.249806245156129	0\\
0.249906247656191	0\\
0.250006250156254	0\\
0.250106252656316	0\\
0.250206255156379	0\\
0.250306257656441	0\\
0.250406260156504	0\\
0.250506262656566	0\\
0.250606265156629	0\\
0.250706267656691	0\\
0.250806270156754	0\\
0.250906272656816	0\\
0.251006275156879	0\\
0.251106277656941	0\\
0.251206280157004	0\\
0.251306282657066	0\\
0.251406285157129	0\\
0.251506287657191	0\\
0.251606290157254	0\\
0.251706292657316	0\\
0.251806295157379	0\\
0.251906297657441	0\\
0.252006300157504	0\\
0.252106302657566	0\\
0.252206305157629	0\\
0.252306307657691	0\\
0.252406310157754	0\\
0.252506312657816	0\\
0.252606315157879	0\\
0.252706317657941	0\\
0.252806320158004	0\\
0.252906322658066	0\\
0.253006325158129	0\\
0.253106327658191	0\\
0.253206330158254	0\\
0.253306332658316	0\\
0.253406335158379	0\\
0.253506337658441	0\\
0.253606340158504	0\\
0.253706342658566	0\\
0.253806345158629	0\\
0.253906347658691	0\\
0.254006350158754	0\\
0.254106352658816	0\\
0.254206355158879	0\\
0.254306357658941	0\\
0.254406360159004	0\\
0.254506362659066	0\\
0.254606365159129	0\\
0.254706367659191	0\\
0.254806370159254	0\\
0.254906372659316	0\\
0.255006375159379	0\\
0.255106377659441	0\\
0.255206380159504	0\\
0.255306382659567	0\\
0.255406385159629	0\\
0.255506387659691	0\\
0.255606390159754	0\\
0.255706392659816	0\\
0.255806395159879	0\\
0.255906397659941	0\\
0.256006400160004	0\\
0.256106402660067	0\\
0.256206405160129	0\\
0.256306407660192	0\\
0.256406410160254	0\\
0.256506412660316	0\\
0.256606415160379	0\\
0.256706417660441	0\\
0.256806420160504	0\\
0.256906422660567	0\\
0.257006425160629	0\\
0.257106427660692	0\\
0.257206430160754	0\\
0.257306432660817	0\\
0.257406435160879	0\\
0.257506437660942	0\\
0.257606440161004	0\\
0.257706442661067	0\\
0.257806445161129	0\\
0.257906447661192	0\\
0.258006450161254	0\\
0.258106452661317	0\\
0.258206455161379	0\\
0.258306457661442	0\\
0.258406460161504	0\\
0.258506462661567	0\\
0.258606465161629	0\\
0.258706467661692	0\\
0.258806470161754	0\\
0.258906472661817	0\\
0.259006475161879	0\\
0.259106477661942	0\\
0.259206480162004	0\\
0.259306482662067	0\\
0.259406485162129	0\\
0.259506487662192	0\\
0.259606490162254	0\\
0.259706492662317	0\\
0.259806495162379	0\\
0.259906497662442	0\\
0.260006500162504	0\\
0.260106502662567	0\\
0.260206505162629	0\\
0.260306507662692	0\\
0.260406510162754	0\\
0.260506512662817	0\\
0.260606515162879	0\\
0.260706517662942	0\\
0.260806520163004	0\\
0.260906522663067	0\\
0.261006525163129	0\\
0.261106527663192	0\\
0.261206530163254	0\\
0.261306532663317	0\\
0.261406535163379	0\\
0.261506537663442	0\\
0.261606540163504	0\\
0.261706542663567	0\\
0.261806545163629	0\\
0.261906547663692	0\\
0.262006550163754	0\\
0.262106552663817	0\\
0.262206555163879	0\\
0.262306557663942	0\\
0.262406560164004	0\\
0.262506562664067	0\\
0.262606565164129	0\\
0.262706567664192	0\\
0.262806570164254	0\\
0.262906572664317	0\\
0.263006575164379	0\\
0.263106577664442	0\\
0.263206580164504	0\\
0.263306582664567	0\\
0.263406585164629	0\\
0.263506587664692	0\\
0.263606590164754	0\\
0.263706592664817	0\\
0.263806595164879	0\\
0.263906597664942	0\\
0.264006600165004	0\\
0.264106602665067	0\\
0.264206605165129	0\\
0.264306607665192	0\\
0.264406610165254	0\\
0.264506612665317	0\\
0.264606615165379	0\\
0.264706617665442	0\\
0.264806620165504	0\\
0.264906622665567	0\\
0.265006625165629	0\\
0.265106627665692	0\\
0.265206630165754	0\\
0.265306632665817	0\\
0.265406635165879	0\\
0.265506637665942	0\\
0.265606640166004	0\\
0.265706642666067	0\\
0.265806645166129	0\\
0.265906647666192	0\\
0.266006650166254	0\\
0.266106652666317	0\\
0.266206655166379	0\\
0.266306657666442	0\\
0.266406660166504	0\\
0.266506662666567	0\\
0.266606665166629	0\\
0.266706667666692	0\\
0.266806670166754	0\\
0.266906672666817	0\\
0.267006675166879	0\\
0.267106677666942	0\\
0.267206680167004	0\\
0.267306682667067	0\\
0.267406685167129	0\\
0.267506687667192	0\\
0.267606690167254	0\\
0.267706692667317	0\\
0.267806695167379	0\\
0.267906697667442	0\\
0.268006700167504	0\\
0.268106702667567	0\\
0.268206705167629	0\\
0.268306707667692	0\\
0.268406710167754	0\\
0.268506712667817	0\\
0.268606715167879	0\\
0.268706717667942	0\\
0.268806720168004	0\\
0.268906722668067	0\\
0.269006725168129	0\\
0.269106727668192	0\\
0.269206730168254	0\\
0.269306732668317	0\\
0.269406735168379	0\\
0.269506737668442	0\\
0.269606740168504	0\\
0.269706742668567	0\\
0.269806745168629	0\\
0.269906747668692	0\\
0.270006750168754	0\\
0.270106752668817	0\\
0.270206755168879	0\\
0.270306757668942	0\\
0.270406760169004	0\\
0.270506762669067	0\\
0.270606765169129	0\\
0.270706767669192	0\\
0.270806770169254	0\\
0.270906772669317	0\\
0.271006775169379	0\\
0.271106777669442	0\\
0.271206780169504	0\\
0.271306782669567	0\\
0.271406785169629	0\\
0.271506787669692	0\\
0.271606790169754	0\\
0.271706792669817	0\\
0.271806795169879	0\\
0.271906797669942	0\\
0.272006800170004	0\\
0.272106802670067	0\\
0.272206805170129	0\\
0.272306807670192	0\\
0.272406810170254	0\\
0.272506812670317	0\\
0.272606815170379	0\\
0.272706817670442	0\\
0.272806820170504	0\\
0.272906822670567	0\\
0.273006825170629	0\\
0.273106827670692	0\\
0.273206830170754	0\\
0.273306832670817	0\\
0.273406835170879	0\\
0.273506837670942	0\\
0.273606840171004	0\\
0.273706842671067	0\\
0.273806845171129	0\\
0.273906847671192	0\\
0.274006850171254	0\\
0.274106852671317	0\\
0.274206855171379	0\\
0.274306857671442	0\\
0.274406860171504	0\\
0.274506862671567	0\\
0.274606865171629	0\\
0.274706867671692	0\\
0.274806870171754	0\\
0.274906872671817	0\\
0.275006875171879	0\\
0.275106877671942	0\\
0.275206880172004	0\\
0.275306882672067	0\\
0.275406885172129	0\\
0.275506887672192	0\\
0.275606890172254	0\\
0.275706892672317	0\\
0.275806895172379	0\\
0.275906897672442	0\\
0.276006900172504	0\\
0.276106902672567	0\\
0.276206905172629	0\\
0.276306907672692	0\\
0.276406910172754	0\\
0.276506912672817	0\\
0.276606915172879	0\\
0.276706917672942	0\\
0.276806920173004	0\\
0.276906922673067	0\\
0.277006925173129	0\\
0.277106927673192	0\\
0.277206930173254	0\\
0.277306932673317	0\\
0.277406935173379	0\\
0.277506937673442	0\\
0.277606940173504	0\\
0.277706942673567	0\\
0.277806945173629	0\\
0.277906947673692	0\\
0.278006950173754	0\\
0.278106952673817	0\\
0.278206955173879	0\\
0.278306957673942	0\\
0.278406960174004	0\\
0.278506962674067	0\\
0.278606965174129	0\\
0.278706967674192	0\\
0.278806970174254	0\\
0.278906972674317	0\\
0.279006975174379	0\\
0.279106977674442	0\\
0.279206980174504	0\\
0.279306982674567	0\\
0.279406985174629	0\\
0.279506987674692	0\\
0.279606990174754	0\\
0.279706992674817	0\\
0.279806995174879	0\\
0.279906997674942	0\\
0.280007000175004	0\\
0.280107002675067	0\\
0.280207005175129	0\\
0.280307007675192	0\\
0.280407010175254	0\\
0.280507012675317	0\\
0.280607015175379	0\\
0.280707017675442	0\\
0.280807020175504	0\\
0.280907022675567	0\\
0.281007025175629	0\\
0.281107027675692	0\\
0.281207030175754	0\\
0.281307032675817	0\\
0.281407035175879	0\\
0.281507037675942	0\\
0.281607040176004	0\\
0.281707042676067	0\\
0.281807045176129	0\\
0.281907047676192	0\\
0.282007050176254	0\\
0.282107052676317	0\\
0.282207055176379	0\\
0.282307057676442	0\\
0.282407060176504	0\\
0.282507062676567	0\\
0.282607065176629	0\\
0.282707067676692	0\\
0.282807070176754	0\\
0.282907072676817	0\\
0.283007075176879	0\\
0.283107077676942	0\\
0.283207080177004	0\\
0.283307082677067	0\\
0.283407085177129	0\\
0.283507087677192	0\\
0.283607090177254	0\\
0.283707092677317	0\\
0.283807095177379	0\\
0.283907097677442	0\\
0.284007100177504	0\\
0.284107102677567	0\\
0.284207105177629	0\\
0.284307107677692	0\\
0.284407110177754	0\\
0.284507112677817	0\\
0.284607115177879	0\\
0.284707117677942	0\\
0.284807120178004	0\\
0.284907122678067	0\\
0.285007125178129	0\\
0.285107127678192	0\\
0.285207130178254	0\\
0.285307132678317	0\\
0.285407135178379	0\\
0.285507137678442	0\\
0.285607140178504	0\\
0.285707142678567	0\\
0.285807145178629	0\\
0.285907147678692	0\\
0.286007150178754	0\\
0.286107152678817	0\\
0.286207155178879	0\\
0.286307157678942	0\\
0.286407160179004	0\\
0.286507162679067	0\\
0.286607165179129	0\\
0.286707167679192	0\\
0.286807170179255	0\\
0.286907172679317	0\\
0.287007175179379	0\\
0.287107177679442	0\\
0.287207180179504	0\\
0.287307182679567	0\\
0.287407185179629	0\\
0.287507187679692	0\\
0.287607190179755	0\\
0.287707192679817	0\\
0.28780719517988	0\\
0.287907197679942	0\\
0.288007200180004	0\\
0.288107202680067	0\\
0.288207205180129	0\\
0.288307207680192	0\\
0.288407210180255	0\\
0.288507212680317	0\\
0.28860721518038	0\\
0.288707217680442	0\\
0.288807220180505	0\\
0.288907222680567	0\\
0.289007225180629	0\\
0.289107227680692	0\\
0.289207230180755	0\\
0.289307232680817	0\\
0.28940723518088	0\\
0.289507237680942	0\\
0.289607240181005	0\\
0.289707242681067	0\\
0.28980724518113	0\\
0.289907247681192	0\\
0.290007250181255	0\\
0.290107252681317	0\\
0.29020725518138	0\\
0.290307257681442	0\\
0.290407260181505	0\\
0.290507262681567	0\\
0.29060726518163	0\\
0.290707267681692	0\\
0.290807270181755	0\\
0.290907272681817	0\\
0.29100727518188	0\\
0.291107277681942	0\\
0.291207280182005	0\\
0.291307282682067	0\\
0.29140728518213	0\\
0.291507287682192	0\\
0.291607290182255	0\\
0.291707292682317	0\\
0.29180729518238	0\\
0.291907297682442	0\\
0.292007300182505	0\\
0.292107302682567	0\\
0.29220730518263	0\\
0.292307307682692	0\\
0.292407310182755	0\\
0.292507312682817	0\\
0.29260731518288	0\\
0.292707317682942	0\\
0.292807320183005	0\\
0.292907322683067	0\\
0.29300732518313	0\\
0.293107327683192	0\\
0.293207330183255	0\\
0.293307332683317	0\\
0.29340733518338	0\\
0.293507337683442	0\\
0.293607340183505	0\\
0.293707342683567	0\\
0.29380734518363	0\\
0.293907347683692	0\\
0.294007350183755	0\\
0.294107352683817	0\\
0.29420735518388	0\\
0.294307357683942	0\\
0.294407360184005	0\\
0.294507362684067	0\\
0.29460736518413	0\\
0.294707367684192	0\\
0.294807370184255	0\\
0.294907372684317	0\\
0.29500737518438	0\\
0.295107377684442	0\\
0.295207380184505	0\\
0.295307382684567	0\\
0.29540738518463	0\\
0.295507387684692	0\\
0.295607390184755	0\\
0.295707392684817	0\\
0.29580739518488	0\\
0.295907397684942	0\\
0.296007400185005	0\\
0.296107402685067	0\\
0.29620740518513	0\\
0.296307407685192	0\\
0.296407410185255	0\\
0.296507412685317	0\\
0.29660741518538	0\\
0.296707417685442	0\\
0.296807420185505	0\\
0.296907422685567	0\\
0.29700742518563	0\\
0.297107427685692	0\\
0.297207430185755	0\\
0.297307432685817	0\\
0.29740743518588	0\\
0.297507437685942	0\\
0.297607440186005	0\\
0.297707442686067	0\\
0.29780744518613	0\\
0.297907447686192	0\\
0.298007450186255	0\\
0.298107452686317	0\\
0.29820745518638	0\\
0.298307457686442	0\\
0.298407460186505	0\\
0.298507462686567	0\\
0.29860746518663	0\\
0.298707467686692	0\\
0.298807470186755	0\\
0.298907472686817	0\\
0.29900747518688	0\\
0.299107477686942	0\\
0.299207480187005	0\\
0.299307482687067	0\\
0.29940748518713	0\\
0.299507487687192	0\\
0.299607490187255	0\\
0.299707492687317	0\\
0.29980749518738	0\\
0.299907497687442	0\\
0.300007500187505	0\\
0.300107502687567	0\\
0.30020750518763	0\\
0.300307507687692	0\\
0.300407510187755	0\\
0.300507512687817	0\\
0.30060751518788	0\\
0.300707517687942	0\\
0.300807520188005	0\\
0.300907522688067	0\\
0.30100752518813	0\\
0.301107527688192	0\\
0.301207530188255	0\\
0.301307532688317	0\\
0.30140753518838	0\\
0.301507537688442	0\\
0.301607540188505	0\\
0.301707542688567	0\\
0.30180754518863	0\\
0.301907547688692	0\\
0.302007550188755	0\\
0.302107552688817	0\\
0.30220755518888	0\\
0.302307557688942	0\\
0.302407560189005	0\\
0.302507562689067	0\\
0.30260756518913	0\\
0.302707567689192	0\\
0.302807570189255	0\\
0.302907572689317	0\\
0.30300757518938	0\\
0.303107577689442	0\\
0.303207580189505	0\\
0.303307582689567	0\\
0.30340758518963	0\\
0.303507587689692	0\\
0.303607590189755	0\\
0.303707592689817	0\\
0.30380759518988	0\\
0.303907597689942	0\\
0.304007600190005	0\\
0.304107602690067	0\\
0.30420760519013	0\\
0.304307607690192	0\\
0.304407610190255	0\\
0.304507612690317	0\\
0.30460761519038	0\\
0.304707617690442	0\\
0.304807620190505	0\\
0.304907622690567	0\\
0.30500762519063	0\\
0.305107627690692	0\\
0.305207630190755	0\\
0.305307632690817	0\\
0.30540763519088	0\\
0.305507637690942	0\\
0.305607640191005	0\\
0.305707642691067	0\\
0.30580764519113	0\\
0.305907647691192	0\\
0.306007650191255	0\\
0.306107652691317	0\\
0.30620765519138	0\\
0.306307657691442	0\\
0.306407660191505	0\\
0.306507662691567	0\\
0.30660766519163	0\\
0.306707667691692	0\\
0.306807670191755	0\\
0.306907672691817	0\\
0.30700767519188	0\\
0.307107677691942	0\\
0.307207680192005	0\\
0.307307682692067	0\\
0.30740768519213	0\\
0.307507687692192	0\\
0.307607690192255	0\\
0.307707692692317	0\\
0.30780769519238	0\\
0.307907697692442	0\\
0.308007700192505	0\\
0.308107702692567	0\\
0.30820770519263	0\\
0.308307707692692	0\\
0.308407710192755	0\\
0.308507712692817	0\\
0.30860771519288	0\\
0.308707717692942	0\\
0.308807720193005	0\\
0.308907722693067	0\\
0.30900772519313	0\\
0.309107727693192	0\\
0.309207730193255	0\\
0.309307732693317	0\\
0.30940773519338	0\\
0.309507737693442	0\\
0.309607740193505	0\\
0.309707742693567	0\\
0.30980774519363	0\\
0.309907747693692	0\\
0.310007750193755	0\\
0.310107752693817	0\\
0.31020775519388	0\\
0.310307757693942	0\\
0.310407760194005	0\\
0.310507762694067	0\\
0.31060776519413	0\\
0.310707767694192	0\\
0.310807770194255	0\\
0.310907772694317	0\\
0.31100777519438	0\\
0.311107777694442	0\\
0.311207780194505	0\\
0.311307782694567	0\\
0.31140778519463	0\\
0.311507787694692	0\\
0.311607790194755	0\\
0.311707792694817	0\\
0.31180779519488	0\\
0.311907797694942	0\\
0.312007800195005	0\\
0.312107802695067	0\\
0.31220780519513	0\\
0.312307807695192	0\\
0.312407810195255	0\\
0.312507812695317	0\\
0.31260781519538	0\\
0.312707817695442	0\\
0.312807820195505	0\\
0.312907822695567	0\\
0.31300782519563	0\\
0.313107827695692	0\\
0.313207830195755	0\\
0.313307832695817	0\\
0.31340783519588	0\\
0.313507837695942	0\\
0.313607840196005	0\\
0.313707842696067	0\\
0.31380784519613	0\\
0.313907847696192	0\\
0.314007850196255	0\\
0.314107852696317	0\\
0.31420785519638	0\\
0.314307857696442	0\\
0.314407860196505	0\\
0.314507862696567	0\\
0.31460786519663	0\\
0.314707867696692	0\\
0.314807870196755	0\\
0.314907872696817	0\\
0.31500787519688	0\\
0.315107877696942	0\\
0.315207880197005	0\\
0.315307882697067	0\\
0.31540788519713	0\\
0.315507887697192	0\\
0.315607890197255	0\\
0.315707892697317	0\\
0.31580789519738	0\\
0.315907897697442	0\\
0.316007900197505	0\\
0.316107902697567	0\\
0.31620790519763	0\\
0.316307907697692	0\\
0.316407910197755	0\\
0.316507912697817	0\\
0.31660791519788	0\\
0.316707917697942	0\\
0.316807920198005	0\\
0.316907922698067	0\\
0.31700792519813	0\\
0.317107927698192	0\\
0.317207930198255	0\\
0.317307932698317	0\\
0.31740793519838	0\\
0.317507937698442	0\\
0.317607940198505	0\\
0.317707942698567	0\\
0.31780794519863	0\\
0.317907947698692	0\\
0.318007950198755	0\\
0.318107952698817	0\\
0.31820795519888	0\\
0.318307957698942	0\\
0.318407960199005	0\\
0.318507962699068	0\\
0.31860796519913	0\\
0.318707967699192	0\\
0.318807970199255	0\\
0.318907972699317	0\\
0.31900797519938	0\\
0.319107977699442	0\\
0.319207980199505	0\\
0.319307982699568	0\\
0.31940798519963	0\\
0.319507987699693	0\\
0.319607990199755	0\\
0.319707992699817	0\\
0.31980799519988	0\\
0.319907997699942	0\\
0.320008000200005	0\\
0.320108002700068	0\\
0.32020800520013	0\\
0.320308007700193	0\\
0.320408010200255	0\\
0.320508012700318	0\\
0.32060801520038	0\\
0.320708017700442	0\\
0.320808020200505	0\\
0.320908022700568	0\\
0.32100802520063	0\\
0.321108027700693	0\\
0.321208030200755	0\\
0.321308032700818	0\\
0.32140803520088	0\\
0.321508037700943	0\\
0.321608040201005	0\\
0.321708042701068	0\\
0.32180804520113	0\\
0.321908047701193	0\\
0.322008050201255	0\\
0.322108052701318	0\\
0.32220805520138	0\\
0.322308057701443	0\\
0.322408060201505	0\\
0.322508062701568	0\\
0.32260806520163	0\\
0.322708067701693	0\\
0.322808070201755	0\\
0.322908072701818	0\\
0.32300807520188	0\\
0.323108077701943	0\\
0.323208080202005	0\\
0.323308082702068	0\\
0.32340808520213	0\\
0.323508087702193	0\\
0.323608090202255	0\\
0.323708092702318	0\\
0.32380809520238	0\\
0.323908097702443	0\\
0.324008100202505	0\\
0.324108102702568	0\\
0.32420810520263	0\\
0.324308107702693	0\\
0.324408110202755	0\\
0.324508112702818	0\\
0.32460811520288	0\\
0.324708117702943	0\\
0.324808120203005	0\\
0.324908122703068	0\\
0.32500812520313	0\\
0.325108127703193	0\\
0.325208130203255	0\\
0.325308132703318	0\\
0.32540813520338	0\\
0.325508137703443	0\\
0.325608140203505	0\\
0.325708142703568	0\\
0.32580814520363	0\\
0.325908147703693	0\\
0.326008150203755	0\\
0.326108152703818	0\\
0.32620815520388	0\\
0.326308157703943	0\\
0.326408160204005	0\\
0.326508162704068	0\\
0.32660816520413	0\\
0.326708167704193	0\\
0.326808170204255	0\\
0.326908172704318	0\\
0.32700817520438	0\\
0.327108177704443	0\\
0.327208180204505	0\\
0.327308182704568	0\\
0.32740818520463	0\\
0.327508187704693	0\\
0.327608190204755	0\\
0.327708192704818	0\\
0.32780819520488	0\\
0.327908197704943	0\\
0.328008200205005	0\\
0.328108202705068	0\\
0.32820820520513	0\\
0.328308207705193	0\\
0.328408210205255	0\\
0.328508212705318	0\\
0.32860821520538	0\\
0.328708217705443	0\\
0.328808220205505	0\\
0.328908222705568	0\\
0.32900822520563	0\\
0.329108227705693	0\\
0.329208230205755	0\\
0.329308232705818	0\\
0.32940823520588	0\\
0.329508237705943	0\\
0.329608240206005	0\\
0.329708242706068	0\\
0.32980824520613	0\\
0.329908247706193	0\\
0.330008250206255	0\\
0.330108252706318	0\\
0.33020825520638	0\\
0.330308257706443	0\\
0.330408260206505	0\\
0.330508262706568	0\\
0.33060826520663	0\\
0.330708267706693	0\\
0.330808270206755	0\\
0.330908272706818	0\\
0.33100827520688	0\\
0.331108277706943	0\\
0.331208280207005	0\\
0.331308282707068	0\\
0.33140828520713	0\\
0.331508287707193	0\\
0.331608290207255	0\\
0.331708292707318	0\\
0.33180829520738	0\\
0.331908297707443	0\\
0.332008300207505	0\\
0.332108302707568	0\\
0.33220830520763	0\\
0.332308307707693	0\\
0.332408310207755	0\\
0.332508312707818	0\\
0.33260831520788	0\\
0.332708317707943	0\\
0.332808320208005	0\\
0.332908322708068	0\\
0.33300832520813	0\\
0.333108327708193	0\\
0.333208330208255	0\\
0.333308332708318	0\\
0.33340833520838	0\\
0.333508337708443	0\\
0.333608340208505	0\\
0.333708342708568	0\\
0.33380834520863	0\\
0.333908347708693	0\\
0.334008350208755	0\\
0.334108352708818	0\\
0.33420835520888	0\\
0.334308357708943	0\\
0.334408360209005	0\\
0.334508362709068	0\\
0.33460836520913	0\\
0.334708367709193	0\\
0.334808370209255	0\\
0.334908372709318	0\\
0.33500837520938	0\\
0.335108377709443	0\\
0.335208380209505	0\\
0.335308382709568	0\\
0.33540838520963	0\\
0.335508387709693	0\\
0.335608390209755	0\\
0.335708392709818	0\\
0.33580839520988	0\\
0.335908397709943	0\\
0.336008400210005	0\\
0.336108402710068	0\\
0.33620840521013	0\\
0.336308407710193	0\\
0.336408410210255	0\\
0.336508412710318	0\\
0.33660841521038	0\\
0.336708417710443	0\\
0.336808420210505	0\\
0.336908422710568	0\\
0.33700842521063	0\\
0.337108427710693	0\\
0.337208430210755	0\\
0.337308432710818	0\\
0.33740843521088	0\\
0.337508437710943	0\\
0.337608440211005	0\\
0.337708442711068	0\\
0.33780844521113	0\\
0.337908447711193	0\\
0.338008450211255	0\\
0.338108452711318	0\\
0.33820845521138	0\\
0.338308457711443	0\\
0.338408460211505	0\\
0.338508462711568	0\\
0.33860846521163	0\\
0.338708467711693	0\\
0.338808470211755	0\\
0.338908472711818	0\\
0.33900847521188	0\\
0.339108477711943	0\\
0.339208480212005	0\\
0.339308482712068	0\\
0.33940848521213	0\\
0.339508487712193	0\\
0.339608490212255	0\\
0.339708492712318	0\\
0.33980849521238	0\\
0.339908497712443	0\\
0.340008500212505	0\\
0.340108502712568	0\\
0.34020850521263	0\\
0.340308507712693	0\\
0.340408510212755	0\\
0.340508512712818	0\\
0.34060851521288	0\\
0.340708517712943	0\\
0.340808520213005	0\\
0.340908522713068	0\\
0.34100852521313	0\\
0.341108527713193	0\\
0.341208530213255	0\\
0.341308532713318	0\\
0.34140853521338	0\\
0.341508537713443	0\\
0.341608540213505	0\\
0.341708542713568	0\\
0.34180854521363	0\\
0.341908547713693	0\\
0.342008550213755	0\\
0.342108552713818	0\\
0.34220855521388	0\\
0.342308557713943	0\\
0.342408560214005	0\\
0.342508562714068	0\\
0.34260856521413	0\\
0.342708567714193	0\\
0.342808570214255	0\\
0.342908572714318	0\\
0.34300857521438	0\\
0.343108577714443	0\\
0.343208580214505	0\\
0.343308582714568	0\\
0.34340858521463	0\\
0.343508587714693	0\\
0.343608590214755	0\\
0.343708592714818	0\\
0.34380859521488	0\\
0.343908597714943	0\\
0.344008600215005	0\\
0.344108602715068	0\\
0.34420860521513	0\\
0.344308607715193	0\\
0.344408610215255	0\\
0.344508612715318	0\\
0.34460861521538	0\\
0.344708617715443	0\\
0.344808620215505	0\\
0.344908622715568	0\\
0.34500862521563	0\\
0.345108627715693	0\\
0.345208630215755	0\\
0.345308632715818	0\\
0.34540863521588	0\\
0.345508637715943	0\\
0.345608640216005	0\\
0.345708642716068	0\\
0.34580864521613	0\\
0.345908647716193	0\\
0.346008650216255	0\\
0.346108652716318	0\\
0.34620865521638	0\\
0.346308657716443	0\\
0.346408660216505	0\\
0.346508662716568	0\\
0.34660866521663	0\\
0.346708667716693	0\\
0.346808670216755	0\\
0.346908672716818	0\\
0.34700867521688	0\\
0.347108677716943	0\\
0.347208680217005	0\\
0.347308682717068	0\\
0.34740868521713	0\\
0.347508687717193	0\\
0.347608690217255	0\\
0.347708692717318	0\\
0.34780869521738	0\\
0.347908697717443	0\\
0.348008700217505	0\\
0.348108702717568	0\\
0.34820870521763	0\\
0.348308707717693	0\\
0.348408710217755	0\\
0.348508712717818	0\\
0.34860871521788	0\\
0.348708717717943	0\\
0.348808720218005	0\\
0.348908722718068	0\\
0.34900872521813	0\\
0.349108727718193	0\\
0.349208730218255	0\\
0.349308732718318	0\\
0.34940873521838	0\\
0.349508737718443	0\\
0.349608740218505	0\\
0.349708742718568	0\\
0.34980874521863	0\\
0.349908747718693	0\\
0.350008750218755	0\\
0.350108752718818	0\\
0.35020875521888	0\\
0.350308757718943	0\\
0.350408760219005	0\\
0.350508762719068	0\\
0.35060876521913	0\\
0.350708767719193	0\\
0.350808770219255	0\\
0.350908772719318	0\\
0.351008775219381	0\\
0.351108777719443	0\\
0.351208780219505	0\\
0.351308782719568	0\\
0.35140878521963	0\\
0.351508787719693	0\\
0.351608790219755	0\\
0.351708792719818	0\\
0.351808795219881	0\\
0.351908797719943	0\\
0.352008800220006	0\\
0.352108802720068	0\\
0.35220880522013	0\\
0.352308807720193	0\\
0.352408810220255	0\\
0.352508812720318	0\\
0.352608815220381	0\\
0.352708817720443	0\\
0.352808820220506	0\\
0.352908822720568	0\\
0.353008825220631	0\\
0.353108827720693	0\\
0.353208830220755	0\\
0.353308832720818	0\\
0.353408835220881	0\\
0.353508837720943	0\\
0.353608840221006	0\\
0.353708842721068	0\\
0.353808845221131	0\\
0.353908847721193	0\\
0.354008850221256	0\\
0.354108852721318	0\\
0.354208855221381	0\\
0.354308857721443	0\\
0.354408860221506	0\\
0.354508862721568	0\\
0.354608865221631	0\\
0.354708867721693	0\\
0.354808870221756	0\\
0.354908872721818	0\\
0.355008875221881	0\\
0.355108877721943	0\\
0.355208880222006	0\\
0.355308882722068	0\\
0.355408885222131	0\\
0.355508887722193	0\\
0.355608890222256	0\\
0.355708892722318	0\\
0.355808895222381	0\\
0.355908897722443	0\\
0.356008900222506	0\\
0.356108902722568	0\\
0.356208905222631	0\\
0.356308907722693	0\\
0.356408910222756	0\\
0.356508912722818	0\\
0.356608915222881	0\\
0.356708917722943	0\\
0.356808920223006	0\\
0.356908922723068	0\\
0.357008925223131	0\\
0.357108927723193	0\\
0.357208930223256	0\\
0.357308932723318	0\\
0.357408935223381	0\\
0.357508937723443	0\\
0.357608940223506	0\\
0.357708942723568	0\\
0.357808945223631	0\\
0.357908947723693	0\\
0.358008950223756	0\\
0.358108952723818	0\\
0.358208955223881	0\\
0.358308957723943	0\\
0.358408960224006	0\\
0.358508962724068	0\\
0.358608965224131	0\\
0.358708967724193	0\\
0.358808970224256	0\\
0.358908972724318	0\\
0.359008975224381	0\\
0.359108977724443	0\\
0.359208980224506	0\\
0.359308982724568	0\\
0.359408985224631	0\\
0.359508987724693	0\\
0.359608990224756	0\\
0.359708992724818	0\\
0.359808995224881	0\\
0.359908997724943	0\\
0.360009000225006	0\\
0.360109002725068	0\\
0.360209005225131	0\\
0.360309007725193	0\\
0.360409010225256	0\\
0.360509012725318	0\\
0.360609015225381	0\\
0.360709017725443	0\\
0.360809020225506	0\\
0.360909022725568	0\\
0.361009025225631	0\\
0.361109027725693	0\\
0.361209030225756	0\\
0.361309032725818	0\\
0.361409035225881	0\\
0.361509037725943	0\\
0.361609040226006	0\\
0.361709042726068	0\\
0.361809045226131	0\\
0.361909047726193	0\\
0.362009050226256	0\\
0.362109052726318	0\\
0.362209055226381	0\\
0.362309057726443	0\\
0.362409060226506	0\\
0.362509062726568	0\\
0.362609065226631	0\\
0.362709067726693	0\\
0.362809070226756	0\\
0.362909072726818	0\\
0.363009075226881	0\\
0.363109077726943	0\\
0.363209080227006	0\\
0.363309082727068	0\\
0.363409085227131	0\\
0.363509087727193	0\\
0.363609090227256	0\\
0.363709092727318	0\\
0.363809095227381	0\\
0.363909097727443	0\\
0.364009100227506	0\\
0.364109102727568	0\\
0.364209105227631	0\\
0.364309107727693	0\\
0.364409110227756	0\\
0.364509112727818	0\\
0.364609115227881	0\\
0.364709117727943	0\\
0.364809120228006	0\\
0.364909122728068	0\\
0.365009125228131	0\\
0.365109127728193	0\\
0.365209130228256	0\\
0.365309132728318	0\\
0.365409135228381	0\\
0.365509137728443	0\\
0.365609140228506	0\\
0.365709142728568	0\\
0.365809145228631	0\\
0.365909147728693	0\\
0.366009150228756	0\\
0.366109152728818	0\\
0.366209155228881	0\\
0.366309157728943	0\\
0.366409160229006	0\\
0.366509162729068	0\\
0.366609165229131	0\\
0.366709167729193	0\\
0.366809170229256	0\\
0.366909172729318	0\\
0.367009175229381	0\\
0.367109177729443	0\\
0.367209180229506	0\\
0.367309182729568	0\\
0.367409185229631	0\\
0.367509187729693	0\\
0.367609190229756	0\\
0.367709192729818	0\\
0.367809195229881	0\\
0.367909197729943	0\\
0.368009200230006	0\\
0.368109202730068	0\\
0.368209205230131	0\\
0.368309207730193	0\\
0.368409210230256	0\\
0.368509212730318	0\\
0.368609215230381	0\\
0.368709217730443	0\\
0.368809220230506	0\\
0.368909222730568	0\\
0.369009225230631	0\\
0.369109227730693	0\\
0.369209230230756	0\\
0.369309232730818	0\\
0.369409235230881	0\\
0.369509237730943	0\\
0.369609240231006	0\\
0.369709242731068	0\\
0.369809245231131	0\\
0.369909247731193	0\\
0.370009250231256	0\\
0.370109252731318	0\\
0.370209255231381	0\\
0.370309257731443	0\\
0.370409260231506	0\\
0.370509262731568	0\\
0.370609265231631	0\\
0.370709267731693	0\\
0.370809270231756	0\\
0.370909272731818	0\\
0.371009275231881	0\\
0.371109277731943	0\\
0.371209280232006	0\\
0.371309282732068	0\\
0.371409285232131	0\\
0.371509287732193	0\\
0.371609290232256	0\\
0.371709292732318	0\\
0.371809295232381	0\\
0.371909297732443	0\\
0.372009300232506	0\\
0.372109302732568	0\\
0.372209305232631	0\\
0.372309307732693	0\\
0.372409310232756	0\\
0.372509312732818	0\\
0.372609315232881	0\\
0.372709317732943	0\\
0.372809320233006	0\\
0.372909322733068	0\\
0.373009325233131	0\\
0.373109327733193	0\\
0.373209330233256	0\\
0.373309332733318	0\\
0.373409335233381	0\\
0.373509337733443	0\\
0.373609340233506	0\\
0.373709342733568	0\\
0.373809345233631	0\\
0.373909347733693	0\\
0.374009350233756	0\\
0.374109352733818	0\\
0.374209355233881	0\\
0.374309357733943	0\\
0.374409360234006	0\\
0.374509362734068	0\\
0.374609365234131	0\\
0.374709367734193	0\\
0.374809370234256	0\\
0.374909372734318	0\\
0.375009375234381	0\\
0.375109377734443	0\\
0.375209380234506	0\\
0.375309382734568	0\\
0.375409385234631	0\\
0.375509387734693	0\\
0.375609390234756	0\\
0.375709392734818	0\\
0.375809395234881	0\\
0.375909397734943	0\\
0.376009400235006	0\\
0.376109402735068	0\\
0.376209405235131	0\\
0.376309407735193	0\\
0.376409410235256	0\\
0.376509412735318	0\\
0.376609415235381	0\\
0.376709417735443	0\\
0.376809420235506	0\\
0.376909422735568	0\\
0.377009425235631	0\\
0.377109427735693	0\\
0.377209430235756	0\\
0.377309432735818	0\\
0.377409435235881	0\\
0.377509437735943	0\\
0.377609440236006	0\\
0.377709442736068	0\\
0.377809445236131	0\\
0.377909447736193	0\\
0.378009450236256	0\\
0.378109452736318	0\\
0.378209455236381	0\\
0.378309457736443	0\\
0.378409460236506	0\\
0.378509462736568	0\\
0.378609465236631	0\\
0.378709467736693	0\\
0.378809470236756	0\\
0.378909472736818	0\\
0.379009475236881	0\\
0.379109477736943	0\\
0.379209480237006	0\\
0.379309482737068	0\\
0.379409485237131	0\\
0.379509487737193	0\\
0.379609490237256	0\\
0.379709492737318	0\\
0.379809495237381	0\\
0.379909497737443	0\\
0.380009500237506	0\\
0.380109502737568	0\\
0.380209505237631	0\\
0.380309507737693	0\\
0.380409510237756	0\\
0.380509512737818	0\\
0.380609515237881	0\\
0.380709517737943	0\\
0.380809520238006	0\\
0.380909522738068	0\\
0.381009525238131	0\\
0.381109527738193	0\\
0.381209530238256	0\\
0.381309532738318	0\\
0.381409535238381	0\\
0.381509537738443	0\\
0.381609540238506	0\\
0.381709542738568	0\\
0.381809545238631	0\\
0.381909547738693	0\\
0.382009550238756	0\\
0.382109552738818	0\\
0.382209555238881	0\\
0.382309557738943	0\\
0.382409560239006	0\\
0.382509562739068	0\\
0.382609565239131	0\\
0.382709567739193	0\\
0.382809570239256	0\\
0.382909572739318	0\\
0.383009575239381	0\\
0.383109577739443	0\\
0.383209580239506	0\\
0.383309582739568	0\\
0.383409585239631	0\\
0.383509587739694	0\\
0.383609590239756	0\\
0.383709592739818	0\\
0.383809595239881	0\\
0.383909597739943	0\\
0.384009600240006	0\\
0.384109602740069	0\\
0.384209605240131	0\\
0.384309607740194	0\\
0.384409610240256	0\\
0.384509612740319	0\\
0.384609615240381	0\\
0.384709617740443	0\\
0.384809620240506	0\\
0.384909622740569	0\\
0.385009625240631	0\\
0.385109627740694	0\\
0.385209630240756	0\\
0.385309632740819	0\\
0.385409635240881	0\\
0.385509637740944	0\\
0.385609640241006	0\\
0.385709642741069	0\\
0.385809645241131	0\\
0.385909647741194	0\\
0.386009650241256	0\\
0.386109652741319	0\\
0.386209655241381	0\\
0.386309657741444	0\\
0.386409660241506	0\\
0.386509662741569	0\\
0.386609665241631	0\\
0.386709667741694	0\\
0.386809670241756	0\\
0.386909672741819	0\\
0.387009675241881	0\\
0.387109677741944	0\\
0.387209680242006	0\\
0.387309682742069	0\\
0.387409685242131	0\\
0.387509687742194	0\\
0.387609690242256	0\\
0.387709692742319	0\\
0.387809695242381	0\\
0.387909697742444	0\\
0.388009700242506	0\\
0.388109702742569	0\\
0.388209705242631	0\\
0.388309707742694	0\\
0.388409710242756	0\\
0.388509712742819	0\\
0.388609715242881	0\\
0.388709717742944	0\\
0.388809720243006	0\\
0.388909722743069	0\\
0.389009725243131	0\\
0.389109727743194	0\\
0.389209730243256	0\\
0.389309732743319	0\\
0.389409735243381	0\\
0.389509737743444	0\\
0.389609740243506	0\\
0.389709742743569	0\\
0.389809745243631	0\\
0.389909747743694	0\\
0.390009750243756	0\\
0.390109752743819	0\\
0.390209755243881	0\\
0.390309757743944	0\\
0.390409760244006	0\\
0.390509762744069	0\\
0.390609765244131	0\\
0.390709767744194	0\\
0.390809770244256	0\\
0.390909772744319	0\\
0.391009775244381	0\\
0.391109777744444	0\\
0.391209780244506	0\\
0.391309782744569	0\\
0.391409785244631	0\\
0.391509787744694	0\\
0.391609790244756	0\\
0.391709792744819	0\\
0.391809795244881	0\\
0.391909797744944	0\\
0.392009800245006	0\\
0.392109802745069	0\\
0.392209805245131	0\\
0.392309807745194	0\\
0.392409810245256	0\\
0.392509812745319	0\\
0.392609815245381	0\\
0.392709817745444	0\\
0.392809820245506	0\\
0.392909822745569	0\\
0.393009825245631	0\\
0.393109827745694	0\\
0.393209830245756	0\\
0.393309832745819	0\\
0.393409835245881	0\\
0.393509837745944	0\\
0.393609840246006	0\\
0.393709842746069	0\\
0.393809845246131	0\\
0.393909847746194	0\\
0.394009850246256	0\\
0.394109852746319	0\\
0.394209855246381	0\\
0.394309857746444	0\\
0.394409860246506	0\\
0.394509862746569	0\\
0.394609865246631	0\\
0.394709867746694	0\\
0.394809870246756	0\\
0.394909872746819	0\\
0.395009875246881	0\\
0.395109877746944	0\\
0.395209880247006	0\\
0.395309882747069	0\\
0.395409885247131	0\\
0.395509887747194	0\\
0.395609890247256	0\\
0.395709892747319	0\\
0.395809895247381	0\\
0.395909897747444	0\\
0.396009900247506	0\\
0.396109902747569	0\\
0.396209905247631	0\\
0.396309907747694	0\\
0.396409910247756	0\\
0.396509912747819	0\\
0.396609915247881	0\\
0.396709917747944	0\\
0.396809920248006	0\\
0.396909922748069	0\\
0.397009925248131	0\\
0.397109927748194	0\\
0.397209930248256	0\\
0.397309932748319	0\\
0.397409935248381	0\\
0.397509937748444	0\\
0.397609940248506	0\\
0.397709942748569	0\\
0.397809945248631	0\\
0.397909947748694	0\\
0.398009950248756	0\\
0.398109952748819	0\\
0.398209955248881	0\\
0.398309957748944	0\\
0.398409960249006	0\\
0.398509962749069	0\\
0.398609965249131	0\\
0.398709967749194	0\\
0.398809970249256	0\\
0.398909972749319	0\\
0.399009975249381	0\\
0.399109977749444	0\\
0.399209980249506	0\\
0.399309982749569	0\\
0.399409985249631	0\\
0.399509987749694	0\\
0.399609990249756	0\\
0.399709992749819	0\\
0.399809995249881	0\\
0.399909997749944	0\\
0.400010000250006	0\\
};
\addplot [color=mycolor2,solid,forget plot]
  table[row sep=crcr]{%
0.400010000250006	0\\
0.400110002750069	0\\
0.400210005250131	0\\
0.400310007750194	0\\
0.400410010250256	0\\
0.400510012750319	0\\
0.400610015250381	0\\
0.400710017750444	0\\
0.400810020250506	0\\
0.400910022750569	0\\
0.401010025250631	0\\
0.401110027750694	0\\
0.401210030250756	0\\
0.401310032750819	0\\
0.401410035250881	0\\
0.401510037750944	0\\
0.401610040251006	0\\
0.401710042751069	0\\
0.401810045251131	0\\
0.401910047751194	0\\
0.402010050251256	0\\
0.402110052751319	0\\
0.402210055251381	0\\
0.402310057751444	0\\
0.402410060251506	0\\
0.402510062751569	0\\
0.402610065251631	0\\
0.402710067751694	0\\
0.402810070251756	0\\
0.402910072751819	0\\
0.403010075251881	0\\
0.403110077751944	0\\
0.403210080252006	0\\
0.403310082752069	0\\
0.403410085252131	0\\
0.403510087752194	0\\
0.403610090252256	0\\
0.403710092752319	0\\
0.403810095252381	0\\
0.403910097752444	0\\
0.404010100252506	0\\
0.404110102752569	0\\
0.404210105252631	0\\
0.404310107752694	0\\
0.404410110252756	0\\
0.404510112752819	0\\
0.404610115252881	0\\
0.404710117752944	0\\
0.404810120253006	0\\
0.404910122753069	0\\
0.405010125253131	0\\
0.405110127753194	0\\
0.405210130253256	0\\
0.405310132753319	0\\
0.405410135253381	0\\
0.405510137753444	0\\
0.405610140253506	0\\
0.405710142753569	0\\
0.405810145253631	0\\
0.405910147753694	0\\
0.406010150253756	0\\
0.406110152753819	0\\
0.406210155253881	0\\
0.406310157753944	0\\
0.406410160254006	0\\
0.406510162754069	0\\
0.406610165254131	0\\
0.406710167754194	0\\
0.406810170254256	0\\
0.406910172754319	0\\
0.407010175254381	0\\
0.407110177754444	0\\
0.407210180254506	0\\
0.407310182754569	0\\
0.407410185254631	0\\
0.407510187754694	0\\
0.407610190254756	0\\
0.407710192754819	0\\
0.407810195254881	0\\
0.407910197754944	0\\
0.408010200255006	0\\
0.408110202755069	0\\
0.408210205255131	0\\
0.408310207755194	0\\
0.408410210255256	0\\
0.408510212755319	0\\
0.408610215255381	0\\
0.408710217755444	0\\
0.408810220255506	0\\
0.408910222755569	0\\
0.409010225255631	0\\
0.409110227755694	0\\
0.409210230255756	0\\
0.409310232755819	0\\
0.409410235255881	0\\
0.409510237755944	0\\
0.409610240256006	0\\
0.409710242756069	0\\
0.409810245256131	0\\
0.409910247756194	0\\
0.410010250256256	0\\
0.410110252756319	0\\
0.410210255256381	0\\
0.410310257756444	0\\
0.410410260256506	0\\
0.410510262756569	0\\
0.410610265256631	0\\
0.410710267756694	0\\
0.410810270256756	0\\
0.410910272756819	0\\
0.411010275256881	0\\
0.411110277756944	0\\
0.411210280257006	0\\
0.411310282757069	0\\
0.411410285257131	0\\
0.411510287757194	0\\
0.411610290257256	0\\
0.411710292757319	0\\
0.411810295257381	0\\
0.411910297757444	0\\
0.412010300257506	0\\
0.412110302757569	0\\
0.412210305257631	0\\
0.412310307757694	0\\
0.412410310257756	0\\
0.412510312757819	0\\
0.412610315257881	0\\
0.412710317757944	0\\
0.412810320258006	0\\
0.412910322758069	0\\
0.413010325258131	0\\
0.413110327758194	0\\
0.413210330258256	0\\
0.413310332758319	0\\
0.413410335258381	0\\
0.413510337758444	0\\
0.413610340258506	0\\
0.413710342758569	0\\
0.413810345258631	0\\
0.413910347758694	0\\
0.414010350258756	0\\
0.414110352758819	0\\
0.414210355258881	0\\
0.414310357758944	0\\
0.414410360259006	0\\
0.414510362759069	0\\
0.414610365259131	0\\
0.414710367759194	0\\
0.414810370259256	0\\
0.414910372759319	0\\
0.415010375259382	0\\
0.415110377759444	0\\
0.415210380259506	0\\
0.415310382759569	0\\
0.415410385259631	0\\
0.415510387759694	0\\
0.415610390259756	0\\
0.415710392759819	0\\
0.415810395259882	0\\
0.415910397759944	0\\
0.416010400260007	0\\
0.416110402760069	0\\
0.416210405260131	0\\
0.416310407760194	0\\
0.416410410260256	0\\
0.416510412760319	0\\
0.416610415260382	0\\
0.416710417760444	0\\
0.416810420260507	0\\
0.416910422760569	0\\
0.417010425260632	0\\
0.417110427760694	0\\
0.417210430260756	0\\
0.417310432760819	0\\
0.417410435260882	0\\
0.417510437760944	0\\
0.417610440261007	0\\
0.417710442761069	0\\
0.417810445261132	0\\
0.417910447761194	0\\
0.418010450261257	0\\
0.418110452761319	0\\
0.418210455261382	0\\
0.418310457761444	0\\
0.418410460261507	0\\
0.418510462761569	0\\
0.418610465261632	0\\
0.418710467761694	0\\
0.418810470261757	0\\
0.418910472761819	0\\
0.419010475261882	0\\
0.419110477761944	0\\
0.419210480262007	0\\
0.419310482762069	0\\
0.419410485262132	0\\
0.419510487762194	0\\
0.419610490262257	0\\
0.419710492762319	0\\
0.419810495262382	0\\
0.419910497762444	0\\
0.420010500262507	0\\
0.420110502762569	0\\
0.420210505262632	0\\
0.420310507762694	0\\
0.420410510262757	0\\
0.420510512762819	0\\
0.420610515262882	0\\
0.420710517762944	0\\
0.420810520263007	0\\
0.420910522763069	0\\
0.421010525263132	0\\
0.421110527763194	0\\
0.421210530263257	0\\
0.421310532763319	0\\
0.421410535263382	0\\
0.421510537763444	0\\
0.421610540263507	0\\
0.421710542763569	0\\
0.421810545263632	0\\
0.421910547763694	0\\
0.422010550263757	0\\
0.422110552763819	0\\
0.422210555263882	0\\
0.422310557763944	0\\
0.422410560264007	0\\
0.422510562764069	0\\
0.422610565264132	0\\
0.422710567764194	0\\
0.422810570264257	0\\
0.422910572764319	0\\
0.423010575264382	0\\
0.423110577764444	0\\
0.423210580264507	0\\
0.423310582764569	0\\
0.423410585264632	0\\
0.423510587764694	0\\
0.423610590264757	0\\
0.423710592764819	0\\
0.423810595264882	0\\
0.423910597764944	0\\
0.424010600265007	0\\
0.424110602765069	0\\
0.424210605265132	0\\
0.424310607765194	0\\
0.424410610265257	0\\
0.424510612765319	0\\
0.424610615265382	0\\
0.424710617765444	0\\
0.424810620265507	0\\
0.424910622765569	0\\
0.425010625265632	0\\
0.425110627765694	0\\
0.425210630265757	0\\
0.425310632765819	0\\
0.425410635265882	0\\
0.425510637765944	0\\
0.425610640266007	0\\
0.425710642766069	0\\
0.425810645266132	0\\
0.425910647766194	0\\
0.426010650266257	0\\
0.426110652766319	0\\
0.426210655266382	0\\
0.426310657766444	0\\
0.426410660266507	0\\
0.426510662766569	0\\
0.426610665266632	0\\
0.426710667766694	0\\
0.426810670266757	0\\
0.426910672766819	0\\
0.427010675266882	0\\
0.427110677766944	0\\
0.427210680267007	0\\
0.427310682767069	0\\
0.427410685267132	0\\
0.427510687767194	0\\
0.427610690267257	0\\
0.427710692767319	0\\
0.427810695267382	0\\
0.427910697767444	0\\
0.428010700267507	0\\
0.428110702767569	0\\
0.428210705267632	0\\
0.428310707767694	0\\
0.428410710267757	0\\
0.428510712767819	0\\
0.428610715267882	0\\
0.428710717767944	0\\
0.428810720268007	0\\
0.428910722768069	0\\
0.429010725268132	0\\
0.429110727768194	0\\
0.429210730268257	0\\
0.429310732768319	0\\
0.429410735268382	0\\
0.429510737768444	0\\
0.429610740268507	0\\
0.429710742768569	0\\
0.429810745268632	0\\
0.429910747768694	0\\
0.430010750268757	0\\
0.430110752768819	0\\
0.430210755268882	0\\
0.430310757768944	0\\
0.430410760269007	0\\
0.430510762769069	0\\
0.430610765269132	0\\
0.430710767769194	0\\
0.430810770269257	0\\
0.430910772769319	0\\
0.431010775269382	0\\
0.431110777769444	0\\
0.431210780269507	0\\
0.431310782769569	0\\
0.431410785269632	0\\
0.431510787769694	0\\
0.431610790269757	0\\
0.431710792769819	0\\
0.431810795269882	0\\
0.431910797769944	0\\
0.432010800270007	0\\
0.432110802770069	0\\
0.432210805270132	0\\
0.432310807770194	0\\
0.432410810270257	0\\
0.432510812770319	0\\
0.432610815270382	0\\
0.432710817770444	0\\
0.432810820270507	0\\
0.432910822770569	0\\
0.433010825270632	0\\
0.433110827770694	0\\
0.433210830270757	0\\
0.433310832770819	0\\
0.433410835270882	0\\
0.433510837770944	0\\
0.433610840271007	0\\
0.433710842771069	0\\
0.433810845271132	0\\
0.433910847771194	0\\
0.434010850271257	0\\
0.434110852771319	0\\
0.434210855271382	0\\
0.434310857771444	0\\
0.434410860271507	0\\
0.434510862771569	0\\
0.434610865271632	0\\
0.434710867771694	0\\
0.434810870271757	0\\
0.434910872771819	0\\
0.435010875271882	0\\
0.435110877771944	0\\
0.435210880272007	0\\
0.435310882772069	0\\
0.435410885272132	0\\
0.435510887772194	0\\
0.435610890272257	0\\
0.435710892772319	0\\
0.435810895272382	0\\
0.435910897772444	0\\
0.436010900272507	0\\
0.436110902772569	0\\
0.436210905272632	0\\
0.436310907772694	0\\
0.436410910272757	0\\
0.436510912772819	0\\
0.436610915272882	0\\
0.436710917772944	0\\
0.436810920273007	0\\
0.436910922773069	0\\
0.437010925273132	0\\
0.437110927773194	0\\
0.437210930273257	0\\
0.437310932773319	0\\
0.437410935273382	0\\
0.437510937773444	0\\
0.437610940273507	0\\
0.437710942773569	0\\
0.437810945273632	0\\
0.437910947773694	0\\
0.438010950273757	0\\
0.438110952773819	0\\
0.438210955273882	0\\
0.438310957773944	0\\
0.438410960274007	0\\
0.438510962774069	0\\
0.438610965274132	0\\
0.438710967774194	0\\
0.438810970274257	0\\
0.438910972774319	0\\
0.439010975274382	0\\
0.439110977774444	0\\
0.439210980274507	0\\
0.439310982774569	0\\
0.439410985274632	0\\
0.439510987774694	0\\
0.439610990274757	0\\
0.439710992774819	0\\
0.439810995274882	0\\
0.439910997774944	0\\
0.440011000275007	0\\
0.440111002775069	0\\
0.440211005275132	0\\
0.440311007775194	0\\
0.440411010275257	0\\
0.440511012775319	0\\
0.440611015275382	0\\
0.440711017775444	0\\
0.440811020275507	0\\
0.440911022775569	0\\
0.441011025275632	0\\
0.441111027775694	0\\
0.441211030275757	0\\
0.441311032775819	0\\
0.441411035275882	0\\
0.441511037775944	0\\
0.441611040276007	0\\
0.441711042776069	0\\
0.441811045276132	0\\
0.441911047776194	0\\
0.442011050276257	0\\
0.442111052776319	0\\
0.442211055276382	0\\
0.442311057776444	0\\
0.442411060276507	0\\
0.442511062776569	0\\
0.442611065276632	0\\
0.442711067776694	0\\
0.442811070276757	0\\
0.442911072776819	0\\
0.443011075276882	0\\
0.443111077776944	0\\
0.443211080277007	0\\
0.443311082777069	0\\
0.443411085277132	0\\
0.443511087777194	0\\
0.443611090277257	0\\
0.443711092777319	0\\
0.443811095277382	0\\
0.443911097777444	0\\
0.444011100277507	0\\
0.444111102777569	0\\
0.444211105277632	0\\
0.444311107777694	0\\
0.444411110277757	0\\
0.444511112777819	0\\
0.444611115277882	0\\
0.444711117777944	0\\
0.444811120278007	0\\
0.444911122778069	0\\
0.445011125278132	0\\
0.445111127778194	0\\
0.445211130278257	0\\
0.445311132778319	0\\
0.445411135278382	0\\
0.445511137778444	0\\
0.445611140278507	0\\
0.445711142778569	0\\
0.445811145278632	0\\
0.445911147778694	0\\
0.446011150278757	0\\
0.446111152778819	0\\
0.446211155278882	0\\
0.446311157778944	0\\
0.446411160279007	0\\
0.446511162779069	0\\
0.446611165279132	0\\
0.446711167779195	0\\
0.446811170279257	0\\
0.446911172779319	0\\
0.447011175279382	0\\
0.447111177779444	0\\
0.447211180279507	0\\
0.447311182779569	0\\
0.447411185279632	0\\
0.447511187779695	0\\
0.447611190279757	0\\
0.44771119277982	0\\
0.447811195279882	0\\
0.447911197779944	0\\
0.448011200280007	0\\
0.448111202780069	0\\
0.448211205280132	0\\
0.448311207780195	0\\
0.448411210280257	0\\
0.44851121278032	0\\
0.448611215280382	0\\
0.448711217780445	0\\
0.448811220280507	0\\
0.448911222780569	0\\
0.449011225280632	0\\
0.449111227780695	0\\
0.449211230280757	0\\
0.44931123278082	0\\
0.449411235280882	0\\
0.449511237780945	0\\
0.449611240281007	0\\
0.44971124278107	0\\
0.449811245281132	0\\
0.449911247781195	0\\
0.450011250281257	0\\
0.45011125278132	0\\
0.450211255281382	0\\
0.450311257781445	0\\
0.450411260281507	0\\
0.45051126278157	0\\
0.450611265281632	0\\
0.450711267781695	0\\
0.450811270281757	0\\
0.45091127278182	0\\
0.451011275281882	0\\
0.451111277781945	0\\
0.451211280282007	0\\
0.45131128278207	0\\
0.451411285282132	0\\
0.451511287782195	0\\
0.451611290282257	0\\
0.45171129278232	0\\
0.451811295282382	0\\
0.451911297782445	0\\
0.452011300282507	0\\
0.45211130278257	0\\
0.452211305282632	0\\
0.452311307782695	0\\
0.452411310282757	0\\
0.45251131278282	0\\
0.452611315282882	0\\
0.452711317782945	0\\
0.452811320283007	0\\
0.45291132278307	0\\
0.453011325283132	0\\
0.453111327783195	0\\
0.453211330283257	0\\
0.45331133278332	0\\
0.453411335283382	0\\
0.453511337783445	0\\
0.453611340283507	0\\
0.45371134278357	0\\
0.453811345283632	0\\
0.453911347783695	0\\
0.454011350283757	0\\
0.45411135278382	0\\
0.454211355283882	0\\
0.454311357783945	0\\
0.454411360284007	0\\
0.45451136278407	0\\
0.454611365284132	0\\
0.454711367784195	0\\
0.454811370284257	0\\
0.45491137278432	0\\
0.455011375284382	0\\
0.455111377784445	0\\
0.455211380284507	0\\
0.45531138278457	0\\
0.455411385284632	0\\
0.455511387784695	0\\
0.455611390284757	0\\
0.45571139278482	0\\
0.455811395284882	0\\
0.455911397784945	0\\
0.456011400285007	0\\
0.45611140278507	0\\
0.456211405285132	0\\
0.456311407785195	0\\
0.456411410285257	0\\
0.45651141278532	0\\
0.456611415285382	0\\
0.456711417785445	0\\
0.456811420285507	0\\
0.45691142278557	0\\
0.457011425285632	0\\
0.457111427785695	0\\
0.457211430285757	0\\
0.45731143278582	0\\
0.457411435285882	0\\
0.457511437785945	0\\
0.457611440286007	0\\
0.45771144278607	0\\
0.457811445286132	0\\
0.457911447786195	0\\
0.458011450286257	0\\
0.45811145278632	0\\
0.458211455286382	0\\
0.458311457786445	0\\
0.458411460286507	0\\
0.45851146278657	0\\
0.458611465286632	0\\
0.458711467786695	0\\
0.458811470286757	0\\
0.45891147278682	0\\
0.459011475286882	0\\
0.459111477786945	0\\
0.459211480287007	0\\
0.45931148278707	0\\
0.459411485287132	0\\
0.459511487787195	0\\
0.459611490287257	0\\
0.45971149278732	0\\
0.459811495287382	0\\
0.459911497787445	0\\
0.460011500287507	0\\
0.46011150278757	0\\
0.460211505287632	0\\
0.460311507787695	0\\
0.460411510287757	0\\
0.46051151278782	0\\
0.460611515287882	0\\
0.460711517787945	0\\
0.460811520288007	0\\
0.46091152278807	0\\
0.461011525288132	0\\
0.461111527788195	0\\
0.461211530288257	0\\
0.46131153278832	0\\
0.461411535288382	0\\
0.461511537788445	0\\
0.461611540288507	0\\
0.46171154278857	0\\
0.461811545288632	0\\
0.461911547788695	0\\
0.462011550288757	0\\
0.46211155278882	0\\
0.462211555288882	0\\
0.462311557788945	0\\
0.462411560289007	0\\
0.46251156278907	0\\
0.462611565289132	0\\
0.462711567789195	0\\
0.462811570289257	0\\
0.46291157278932	0\\
0.463011575289382	0\\
0.463111577789445	0\\
0.463211580289507	0\\
0.46331158278957	0\\
0.463411585289632	0\\
0.463511587789695	0\\
0.463611590289757	0\\
0.46371159278982	0\\
0.463811595289882	0\\
0.463911597789945	0\\
0.464011600290007	0\\
0.46411160279007	0\\
0.464211605290132	0\\
0.464311607790195	0\\
0.464411610290257	0\\
0.46451161279032	0\\
0.464611615290382	0\\
0.464711617790445	0\\
0.464811620290507	0\\
0.46491162279057	0\\
0.465011625290632	0\\
0.465111627790695	0\\
0.465211630290757	0\\
0.46531163279082	0\\
0.465411635290882	0\\
0.465511637790945	0\\
0.465611640291007	0\\
0.46571164279107	0\\
0.465811645291132	0\\
0.465911647791195	0\\
0.466011650291257	0\\
0.46611165279132	0\\
0.466211655291382	0\\
0.466311657791445	0\\
0.466411660291507	0\\
0.46651166279157	0\\
0.466611665291632	0\\
0.466711667791695	0\\
0.466811670291757	0\\
0.46691167279182	0\\
0.467011675291882	0\\
0.467111677791945	0\\
0.467211680292007	0\\
0.46731168279207	0\\
0.467411685292132	0\\
0.467511687792195	0\\
0.467611690292257	0\\
0.46771169279232	0\\
0.467811695292382	0\\
0.467911697792445	0\\
0.468011700292507	0\\
0.46811170279257	0\\
0.468211705292632	0\\
0.468311707792695	0\\
0.468411710292757	0\\
0.46851171279282	0\\
0.468611715292882	0\\
0.468711717792945	0\\
0.468811720293007	0\\
0.46891172279307	0\\
0.469011725293132	0\\
0.469111727793195	0\\
0.469211730293257	0\\
0.46931173279332	0\\
0.469411735293382	0\\
0.469511737793445	0\\
0.469611740293507	0\\
0.46971174279357	0\\
0.469811745293632	0\\
0.469911747793695	0\\
0.470011750293757	0\\
0.47011175279382	0\\
0.470211755293882	0\\
0.470311757793945	0\\
0.470411760294007	0\\
0.47051176279407	0\\
0.470611765294132	0\\
0.470711767794195	0\\
0.470811770294257	0\\
0.47091177279432	0\\
0.471011775294382	0\\
0.471111777794445	0\\
0.471211780294507	0\\
0.47131178279457	0\\
0.471411785294632	0\\
0.471511787794695	0\\
0.471611790294757	0\\
0.47171179279482	0\\
0.471811795294882	0\\
0.471911797794945	0\\
0.472011800295007	0\\
0.47211180279507	0\\
0.472211805295132	0\\
0.472311807795195	0\\
0.472411810295257	0\\
0.47251181279532	0\\
0.472611815295382	0\\
0.472711817795445	0\\
0.472811820295507	0\\
0.47291182279557	0\\
0.473011825295632	0\\
0.473111827795695	0\\
0.473211830295757	0\\
0.47331183279582	0\\
0.473411835295882	0\\
0.473511837795945	0\\
0.473611840296007	0\\
0.47371184279607	0\\
0.473811845296132	0\\
0.473911847796195	0\\
0.474011850296257	0\\
0.47411185279632	0\\
0.474211855296382	0\\
0.474311857796445	0\\
0.474411860296507	0\\
0.47451186279657	0\\
0.474611865296632	0\\
0.474711867796695	0\\
0.474811870296757	0\\
0.47491187279682	0\\
0.475011875296882	0\\
0.475111877796945	0\\
0.475211880297007	0\\
0.47531188279707	0\\
0.475411885297132	0\\
0.475511887797195	0\\
0.475611890297257	0\\
0.47571189279732	0\\
0.475811895297382	0\\
0.475911897797445	0\\
0.476011900297507	0\\
0.47611190279757	0\\
0.476211905297632	0\\
0.476311907797695	0\\
0.476411910297757	0\\
0.47651191279782	0\\
0.476611915297882	0\\
0.476711917797945	0\\
0.476811920298007	0\\
0.47691192279807	0\\
0.477011925298132	0\\
0.477111927798195	0\\
0.477211930298257	0\\
0.47731193279832	0\\
0.477411935298382	0\\
0.477511937798445	0\\
0.477611940298507	0\\
0.47771194279857	0\\
0.477811945298632	0\\
0.477911947798695	0\\
0.478011950298757	0\\
0.47811195279882	0\\
0.478211955298882	0\\
0.478311957798945	0\\
0.478411960299007	0\\
0.47851196279907	0\\
0.478611965299132	0\\
0.478711967799195	0\\
0.478811970299257	0\\
0.47891197279932	0\\
0.479011975299382	0\\
0.479111977799445	0\\
0.479211980299508	0\\
0.47931198279957	0\\
0.479411985299632	0\\
0.479511987799695	0\\
0.479611990299757	0\\
0.47971199279982	0\\
0.479811995299882	0\\
0.479911997799945	0\\
0.480012000300008	0\\
0.48011200280007	0\\
0.480212005300133	0\\
0.480312007800195	0\\
0.480412010300257	0\\
0.48051201280032	0\\
0.480612015300383	0\\
0.480712017800445	0\\
0.480812020300508	0\\
0.48091202280057	0\\
0.481012025300633	0\\
0.481112027800695	0\\
0.481212030300758	0\\
0.48131203280082	0\\
0.481412035300883	0\\
0.481512037800945	0\\
0.481612040301008	0\\
0.48171204280107	0\\
0.481812045301133	0\\
0.481912047801195	0\\
0.482012050301258	0\\
0.48211205280132	0\\
0.482212055301383	0\\
0.482312057801445	0\\
0.482412060301508	0\\
0.48251206280157	0\\
0.482612065301633	0\\
0.482712067801695	0\\
0.482812070301758	0\\
0.48291207280182	0\\
0.483012075301883	0\\
0.483112077801945	0\\
0.483212080302008	0\\
0.48331208280207	0\\
0.483412085302133	0\\
0.483512087802195	0\\
0.483612090302258	0\\
0.48371209280232	0\\
0.483812095302383	0\\
0.483912097802445	0\\
0.484012100302508	0\\
0.48411210280257	0\\
0.484212105302633	0\\
0.484312107802695	0\\
0.484412110302758	0\\
0.48451211280282	0\\
0.484612115302883	0\\
0.484712117802945	0\\
0.484812120303008	0\\
0.48491212280307	0\\
0.485012125303133	0\\
0.485112127803195	0\\
0.485212130303258	0\\
0.48531213280332	0\\
0.485412135303383	0\\
0.485512137803445	0\\
0.485612140303508	0\\
0.48571214280357	0\\
0.485812145303633	0\\
0.485912147803695	0\\
0.486012150303758	0\\
0.48611215280382	0\\
0.486212155303883	0\\
0.486312157803945	0\\
0.486412160304008	0\\
0.48651216280407	0\\
0.486612165304133	0\\
0.486712167804195	0\\
0.486812170304258	0\\
0.48691217280432	0\\
0.487012175304383	0\\
0.487112177804445	0\\
0.487212180304508	0\\
0.48731218280457	0\\
0.487412185304633	0\\
0.487512187804695	0\\
0.487612190304758	0\\
0.48771219280482	0\\
0.487812195304883	0\\
0.487912197804945	0\\
0.488012200305008	0\\
0.48811220280507	0\\
0.488212205305133	0\\
0.488312207805195	0\\
0.488412210305258	0\\
0.48851221280532	0\\
0.488612215305383	0\\
0.488712217805445	0\\
0.488812220305508	0\\
0.48891222280557	0\\
0.489012225305633	0\\
0.489112227805695	0\\
0.489212230305758	0\\
0.48931223280582	0\\
0.489412235305883	0\\
0.489512237805945	0\\
0.489612240306008	0\\
0.48971224280607	0\\
0.489812245306133	0\\
0.489912247806195	0\\
0.490012250306258	0\\
0.49011225280632	0\\
0.490212255306383	0\\
0.490312257806445	0\\
0.490412260306508	0\\
0.49051226280657	0\\
0.490612265306633	0\\
0.490712267806695	0\\
0.490812270306758	0\\
0.49091227280682	0\\
0.491012275306883	0\\
0.491112277806945	0\\
0.491212280307008	0\\
0.49131228280707	0\\
0.491412285307133	0\\
0.491512287807195	0\\
0.491612290307258	0\\
0.49171229280732	0\\
0.491812295307383	0\\
0.491912297807445	0\\
0.492012300307508	0\\
0.49211230280757	0\\
0.492212305307633	0\\
0.492312307807695	0\\
0.492412310307758	0\\
0.49251231280782	0\\
0.492612315307883	0\\
0.492712317807945	0\\
0.492812320308008	0\\
0.49291232280807	0\\
0.493012325308133	0\\
0.493112327808195	0\\
0.493212330308258	0\\
0.49331233280832	0\\
0.493412335308383	0\\
0.493512337808445	0\\
0.493612340308508	0\\
0.49371234280857	0\\
0.493812345308633	0\\
0.493912347808695	0\\
0.494012350308758	0\\
0.49411235280882	0\\
0.494212355308883	0\\
0.494312357808945	0\\
0.494412360309008	0\\
0.49451236280907	0\\
0.494612365309133	0\\
0.494712367809195	0\\
0.494812370309258	0\\
0.49491237280932	0\\
0.495012375309383	0\\
0.495112377809445	0\\
0.495212380309508	0\\
0.49531238280957	0\\
0.495412385309633	0\\
0.495512387809695	0\\
0.495612390309758	0\\
0.49571239280982	0\\
0.495812395309883	0\\
0.495912397809945	0\\
0.496012400310008	0\\
0.49611240281007	0\\
0.496212405310133	0\\
0.496312407810195	0\\
0.496412410310258	0\\
0.49651241281032	0\\
0.496612415310383	0\\
0.496712417810445	0\\
0.496812420310508	0\\
0.49691242281057	0\\
0.497012425310633	0\\
0.497112427810695	0\\
0.497212430310758	0\\
0.49731243281082	0\\
0.497412435310883	0\\
0.497512437810945	0\\
0.497612440311008	0\\
0.49771244281107	0\\
0.497812445311133	0\\
0.497912447811195	0\\
0.498012450311258	0\\
0.49811245281132	0\\
0.498212455311383	0\\
0.498312457811445	0\\
0.498412460311508	0\\
0.49851246281157	0\\
0.498612465311633	0\\
0.498712467811695	0\\
0.498812470311758	0\\
0.49891247281182	0\\
0.499012475311883	0\\
0.499112477811945	0\\
0.499212480312008	0\\
0.49931248281207	0\\
0.499412485312133	0\\
0.499512487812195	0\\
0.499612490312258	0\\
0.49971249281232	0\\
0.499812495312383	0\\
0.499912497812445	0\\
0.500012500312508	0\\
0.50011250281257	0\\
0.500212505312633	0\\
0.500312507812695	0\\
0.500412510312758	0\\
0.50051251281282	0\\
0.500612515312883	0\\
0.500712517812945	0\\
0.500812520313008	0\\
0.50091252281307	0\\
0.501012525313133	0\\
0.501112527813195	0\\
0.501212530313258	0\\
0.50131253281332	0\\
0.501412535313383	0\\
0.501512537813445	0\\
0.501612540313508	0\\
0.50171254281357	0\\
0.501812545313633	0\\
0.501912547813695	0\\
0.502012550313758	0\\
0.50211255281382	0\\
0.502212555313883	0\\
0.502312557813945	0\\
0.502412560314008	0\\
0.50251256281407	0\\
0.502612565314133	0\\
0.502712567814195	0\\
0.502812570314258	0\\
0.50291257281432	0\\
0.503012575314383	0\\
0.503112577814445	0\\
0.503212580314508	0\\
0.50331258281457	0\\
0.503412585314633	0\\
0.503512587814695	0\\
0.503612590314758	0\\
0.50371259281482	0\\
0.503812595314883	0\\
0.503912597814945	0\\
0.504012600315008	0\\
0.50411260281507	0\\
0.504212605315133	0\\
0.504312607815195	0\\
0.504412610315258	0\\
0.50451261281532	0\\
0.504612615315383	0\\
0.504712617815445	0\\
0.504812620315508	0\\
0.50491262281557	0\\
0.505012625315633	0\\
0.505112627815695	0\\
0.505212630315758	0\\
0.50531263281582	0\\
0.505412635315883	0\\
0.505512637815945	0\\
0.505612640316008	0\\
0.50571264281607	0\\
0.505812645316133	0\\
0.505912647816195	0\\
0.506012650316258	0\\
0.50611265281632	0\\
0.506212655316383	0\\
0.506312657816445	0\\
0.506412660316508	0\\
0.50651266281657	0\\
0.506612665316633	0\\
0.506712667816695	0\\
0.506812670316758	0\\
0.50691267281682	0\\
0.507012675316883	0\\
0.507112677816945	0\\
0.507212680317008	0\\
0.50731268281707	0\\
0.507412685317133	0\\
0.507512687817195	0\\
0.507612690317258	0\\
0.50771269281732	0\\
0.507812695317383	0\\
0.507912697817445	0\\
0.508012700317508	0\\
0.50811270281757	0\\
0.508212705317633	0\\
0.508312707817695	0\\
0.508412710317758	0\\
0.50851271281782	0\\
0.508612715317883	0\\
0.508712717817945	0\\
0.508812720318008	0\\
0.50891272281807	0\\
0.509012725318133	0\\
0.509112727818195	0\\
0.509212730318258	0\\
0.50931273281832	0\\
0.509412735318383	0\\
0.509512737818445	0\\
0.509612740318508	0\\
0.50971274281857	0\\
0.509812745318633	0\\
0.509912747818695	0\\
0.510012750318758	0\\
0.510112752818821	0\\
0.510212755318883	0\\
0.510312757818945	0\\
0.510412760319008	0\\
0.51051276281907	0\\
0.510612765319133	0\\
0.510712767819196	0\\
0.510812770319258	0\\
0.51091277281932	0\\
0.511012775319383	0\\
0.511112777819446	0\\
0.511212780319508	0\\
0.51131278281957	0\\
0.511412785319633	0\\
0.511512787819695	0\\
0.511612790319758	0\\
0.511712792819821	0\\
0.511812795319883	0\\
0.511912797819945	0\\
0.512012800320008	0\\
0.512112802820071	0\\
0.512212805320133	0\\
0.512312807820196	0\\
0.512412810320258	0\\
0.51251281282032	0\\
0.512612815320383	0\\
0.512712817820446	0\\
0.512812820320508	0\\
0.51291282282057	0\\
0.513012825320633	0\\
0.513112827820696	0\\
0.513212830320758	0\\
0.513312832820821	0\\
0.513412835320883	0\\
0.513512837820945	0\\
0.513612840321008	0\\
0.513712842821071	0\\
0.513812845321133	0\\
0.513912847821196	0\\
0.514012850321258	0\\
0.514112852821321	0\\
0.514212855321383	0\\
0.514312857821446	0\\
0.514412860321508	0\\
0.51451286282157	0\\
0.514612865321633	0\\
0.514712867821696	0\\
0.514812870321758	0\\
0.514912872821821	0\\
0.515012875321883	0\\
0.515112877821946	0\\
0.515212880322008	0\\
0.515312882822071	0\\
0.515412885322133	0\\
0.515512887822196	0\\
0.515612890322258	0\\
0.515712892822321	0\\
0.515812895322383	0\\
0.515912897822446	0\\
0.516012900322508	0\\
0.516112902822571	0\\
0.516212905322633	0\\
0.516312907822696	0\\
0.516412910322758	0\\
0.516512912822821	0\\
0.516612915322883	0\\
0.516712917822946	0\\
0.516812920323008	0\\
0.516912922823071	0\\
0.517012925323133	0\\
0.517112927823196	0\\
0.517212930323258	0\\
0.517312932823321	0\\
0.517412935323383	0\\
0.517512937823446	0\\
0.517612940323508	0\\
0.517712942823571	0\\
0.517812945323633	0\\
0.517912947823696	0\\
0.518012950323758	0\\
0.518112952823821	0\\
0.518212955323883	0\\
0.518312957823946	0\\
0.518412960324008	0\\
0.518512962824071	0\\
0.518612965324133	0\\
0.518712967824196	0\\
0.518812970324258	0\\
0.518912972824321	0\\
0.519012975324383	0\\
0.519112977824446	0\\
0.519212980324508	0\\
0.519312982824571	0\\
0.519412985324633	0\\
0.519512987824696	0\\
0.519612990324758	0\\
0.519712992824821	0\\
0.519812995324883	0\\
0.519912997824946	0\\
0.520013000325008	0\\
0.520113002825071	0\\
0.520213005325133	0\\
0.520313007825196	0\\
0.520413010325258	0\\
0.520513012825321	0\\
0.520613015325383	0\\
0.520713017825446	0\\
0.520813020325508	0\\
0.520913022825571	0\\
0.521013025325633	0\\
0.521113027825696	0\\
0.521213030325758	0\\
0.521313032825821	0\\
0.521413035325883	0\\
0.521513037825946	0\\
0.521613040326008	0\\
0.521713042826071	0\\
0.521813045326133	0\\
0.521913047826196	0\\
0.522013050326258	0\\
0.522113052826321	0\\
0.522213055326383	0\\
0.522313057826446	0\\
0.522413060326508	0\\
0.522513062826571	0\\
0.522613065326633	0\\
0.522713067826696	0\\
0.522813070326758	0\\
0.522913072826821	0\\
0.523013075326883	0\\
0.523113077826946	0\\
0.523213080327008	0\\
0.523313082827071	0\\
0.523413085327133	0\\
0.523513087827196	0\\
0.523613090327258	0\\
0.523713092827321	0\\
0.523813095327383	0\\
0.523913097827446	0\\
0.524013100327508	0\\
0.524113102827571	0\\
0.524213105327633	0\\
0.524313107827696	0\\
0.524413110327758	0\\
0.524513112827821	0\\
0.524613115327883	0\\
0.524713117827946	0\\
0.524813120328008	0\\
0.524913122828071	0\\
0.525013125328133	0\\
0.525113127828196	0\\
0.525213130328258	0\\
0.525313132828321	0\\
0.525413135328383	0\\
0.525513137828446	0\\
0.525613140328508	0\\
0.525713142828571	0\\
0.525813145328633	0\\
0.525913147828696	0\\
0.526013150328758	0\\
0.526113152828821	0\\
0.526213155328883	0\\
0.526313157828946	0\\
0.526413160329008	0\\
0.526513162829071	0\\
0.526613165329133	0\\
0.526713167829196	0\\
0.526813170329258	0\\
0.526913172829321	0\\
0.527013175329383	0\\
0.527113177829446	0\\
0.527213180329508	0\\
0.527313182829571	0\\
0.527413185329633	0\\
0.527513187829696	0\\
0.527613190329758	0\\
0.527713192829821	0\\
0.527813195329883	0\\
0.527913197829946	0\\
0.528013200330008	0\\
0.528113202830071	0\\
0.528213205330133	0\\
0.528313207830196	0\\
0.528413210330258	0\\
0.528513212830321	0\\
0.528613215330383	0\\
0.528713217830446	0\\
0.528813220330508	0\\
0.528913222830571	0\\
0.529013225330633	0\\
0.529113227830696	0\\
0.529213230330758	0\\
0.529313232830821	0\\
0.529413235330883	0\\
0.529513237830946	0\\
0.529613240331008	0\\
0.529713242831071	0\\
0.529813245331133	0\\
0.529913247831196	0\\
0.530013250331258	0\\
0.530113252831321	0\\
0.530213255331383	0\\
0.530313257831446	0\\
0.530413260331508	0\\
0.530513262831571	0\\
0.530613265331633	0\\
0.530713267831696	0\\
0.530813270331758	0\\
0.530913272831821	0\\
0.531013275331883	0\\
0.531113277831946	0\\
0.531213280332008	0\\
0.531313282832071	0\\
0.531413285332133	0\\
0.531513287832196	0\\
0.531613290332258	0\\
0.531713292832321	0\\
0.531813295332383	0\\
0.531913297832446	0\\
0.532013300332508	0\\
0.532113302832571	0\\
0.532213305332633	0\\
0.532313307832696	0\\
0.532413310332758	0\\
0.532513312832821	0\\
0.532613315332883	0\\
0.532713317832946	0\\
0.532813320333008	0\\
0.532913322833071	0\\
0.533013325333133	0\\
0.533113327833196	0\\
0.533213330333258	0\\
0.533313332833321	0\\
0.533413335333383	0\\
0.533513337833446	0\\
0.533613340333508	0\\
0.533713342833571	0\\
0.533813345333633	0\\
0.533913347833696	0\\
0.534013350333758	0\\
0.534113352833821	0\\
0.534213355333883	0\\
0.534313357833946	0\\
0.534413360334008	0\\
0.534513362834071	0\\
0.534613365334133	0\\
0.534713367834196	0\\
0.534813370334258	0\\
0.534913372834321	0\\
0.535013375334383	0\\
0.535113377834446	0\\
0.535213380334508	0\\
0.535313382834571	0\\
0.535413385334633	0\\
0.535513387834696	0\\
0.535613390334758	0\\
0.535713392834821	0\\
0.535813395334883	0\\
0.535913397834946	0\\
0.536013400335008	0\\
0.536113402835071	0\\
0.536213405335133	0\\
0.536313407835196	0\\
0.536413410335258	0\\
0.536513412835321	0\\
0.536613415335383	0\\
0.536713417835446	0\\
0.536813420335508	0\\
0.536913422835571	0\\
0.537013425335633	0\\
0.537113427835696	0\\
0.537213430335758	0\\
0.537313432835821	0\\
0.537413435335883	0\\
0.537513437835946	0\\
0.537613440336008	0\\
0.537713442836071	0\\
0.537813445336133	0\\
0.537913447836196	0\\
0.538013450336258	0\\
0.538113452836321	0\\
0.538213455336383	0\\
0.538313457836446	0\\
0.538413460336508	0\\
0.538513462836571	0\\
0.538613465336633	0\\
0.538713467836696	0\\
0.538813470336758	0\\
0.538913472836821	0\\
0.539013475336883	0\\
0.539113477836946	0\\
0.539213480337008	0\\
0.539313482837071	0\\
0.539413485337133	0\\
0.539513487837196	0\\
0.539613490337258	0\\
0.539713492837321	0\\
0.539813495337383	0\\
0.539913497837446	0\\
0.540013500337508	0\\
0.540113502837571	0\\
0.540213505337633	0\\
0.540313507837696	0\\
0.540413510337758	0\\
0.540513512837821	0\\
0.540613515337883	0\\
0.540713517837946	0\\
0.540813520338008	0\\
0.540913522838071	0\\
0.541013525338133	0\\
0.541113527838196	0\\
0.541213530338258	0\\
0.541313532838321	0\\
0.541413535338383	0\\
0.541513537838446	0\\
0.541613540338509	0\\
0.541713542838571	0\\
0.541813545338633	0\\
0.541913547838696	0\\
0.542013550338758	0\\
0.542113552838821	0\\
0.542213555338883	0\\
0.542313557838946	0\\
0.542413560339008	0\\
0.542513562839071	0\\
0.542613565339134	0\\
0.542713567839196	0\\
0.542813570339258	0\\
0.542913572839321	0\\
0.543013575339383	0\\
0.543113577839446	0\\
0.543213580339509	0\\
0.543313582839571	0\\
0.543413585339633	0\\
0.543513587839696	0\\
0.543613590339759	0\\
0.543713592839821	0\\
0.543813595339883	0\\
0.543913597839946	0\\
0.544013600340008	0\\
0.544113602840071	0\\
0.544213605340134	0\\
0.544313607840196	0\\
0.544413610340258	0\\
0.544513612840321	0\\
0.544613615340384	0\\
0.544713617840446	0\\
0.544813620340509	0\\
0.544913622840571	0\\
0.545013625340633	0\\
0.545113627840696	0\\
0.545213630340759	0\\
0.545313632840821	0\\
0.545413635340883	0\\
0.545513637840946	0\\
0.545613640341009	0\\
0.545713642841071	0\\
0.545813645341134	0\\
0.545913647841196	0\\
0.546013650341258	0\\
0.546113652841321	0\\
0.546213655341384	0\\
0.546313657841446	0\\
0.546413660341509	0\\
0.546513662841571	0\\
0.546613665341634	0\\
0.546713667841696	0\\
0.546813670341759	0\\
0.546913672841821	0\\
0.547013675341883	0\\
0.547113677841946	0\\
0.547213680342009	0\\
0.547313682842071	0\\
0.547413685342134	0\\
0.547513687842196	0\\
0.547613690342259	0\\
0.547713692842321	0\\
0.547813695342384	0\\
0.547913697842446	0\\
0.548013700342509	0\\
0.548113702842571	0\\
0.548213705342634	0\\
0.548313707842696	0\\
0.548413710342759	0\\
0.548513712842821	0\\
0.548613715342884	0\\
0.548713717842946	0\\
0.548813720343009	0\\
0.548913722843071	0\\
0.549013725343134	0\\
0.549113727843196	0\\
0.549213730343259	0\\
0.549313732843321	0\\
0.549413735343384	0\\
0.549513737843446	0\\
0.549613740343509	0\\
0.549713742843571	0\\
0.549813745343634	0\\
0.549913747843696	0\\
0.550013750343759	0\\
0.550113752843821	0\\
0.550213755343884	0\\
0.550313757843946	0\\
0.550413760344009	0\\
0.550513762844071	0\\
0.550613765344134	0\\
0.550713767844196	0\\
0.550813770344259	0\\
0.550913772844321	0\\
0.551013775344384	0\\
0.551113777844446	0\\
0.551213780344509	0\\
0.551313782844571	0\\
0.551413785344634	0\\
0.551513787844696	0\\
0.551613790344759	0\\
0.551713792844821	0\\
0.551813795344884	0\\
0.551913797844946	0\\
0.552013800345009	0\\
0.552113802845071	0\\
0.552213805345134	0\\
0.552313807845196	0\\
0.552413810345259	0\\
0.552513812845321	0\\
0.552613815345384	0\\
0.552713817845446	0\\
0.552813820345509	0\\
0.552913822845571	0\\
0.553013825345634	0\\
0.553113827845696	0\\
0.553213830345759	0\\
0.553313832845821	0\\
0.553413835345884	0\\
0.553513837845946	0\\
0.553613840346009	0\\
0.553713842846071	0\\
0.553813845346134	0\\
0.553913847846196	0\\
0.554013850346259	0\\
0.554113852846321	0\\
0.554213855346384	0\\
0.554313857846446	0\\
0.554413860346509	0\\
0.554513862846571	0\\
0.554613865346634	0\\
0.554713867846696	0\\
0.554813870346759	0\\
0.554913872846821	0\\
0.555013875346884	0\\
0.555113877846946	0\\
0.555213880347009	0\\
0.555313882847071	0\\
0.555413885347134	0\\
0.555513887847196	0\\
0.555613890347259	0\\
0.555713892847321	0\\
0.555813895347384	0\\
0.555913897847446	0\\
0.556013900347509	0\\
0.556113902847571	0\\
0.556213905347634	0\\
0.556313907847696	0\\
0.556413910347759	0\\
0.556513912847821	0\\
0.556613915347884	0\\
0.556713917847946	0\\
0.556813920348009	0\\
0.556913922848071	0\\
0.557013925348134	0\\
0.557113927848196	0\\
0.557213930348259	0\\
0.557313932848321	0\\
0.557413935348384	0\\
0.557513937848446	0\\
0.557613940348509	0\\
0.557713942848571	0\\
0.557813945348634	0\\
0.557913947848696	0\\
0.558013950348759	0\\
0.558113952848821	0\\
0.558213955348884	0\\
0.558313957848946	0\\
0.558413960349009	0\\
0.558513962849071	0\\
0.558613965349134	0\\
0.558713967849196	0\\
0.558813970349259	0\\
0.558913972849321	0\\
0.559013975349384	0\\
0.559113977849446	0\\
0.559213980349509	0\\
0.559313982849571	0\\
0.559413985349634	0\\
0.559513987849696	0\\
0.559613990349759	0\\
0.559713992849821	0\\
0.559813995349884	0\\
0.559913997849946	0\\
0.560014000350009	0\\
0.560114002850071	0\\
0.560214005350134	0\\
0.560314007850196	0\\
0.560414010350259	0\\
0.560514012850321	0\\
0.560614015350384	0\\
0.560714017850446	0\\
0.560814020350509	0\\
0.560914022850571	0\\
0.561014025350634	0\\
0.561114027850696	0\\
0.561214030350759	0\\
0.561314032850821	0\\
0.561414035350884	0\\
0.561514037850946	0\\
0.561614040351009	0\\
0.561714042851071	0\\
0.561814045351134	0\\
0.561914047851196	0\\
0.562014050351259	0\\
0.562114052851321	0\\
0.562214055351384	0\\
0.562314057851446	0\\
0.562414060351509	0\\
0.562514062851571	0\\
0.562614065351634	0\\
0.562714067851696	0\\
0.562814070351759	0\\
0.562914072851821	0\\
0.563014075351884	0\\
0.563114077851946	0\\
0.563214080352009	0\\
0.563314082852071	0\\
0.563414085352134	0\\
0.563514087852196	0\\
0.563614090352259	0\\
0.563714092852321	0\\
0.563814095352384	0\\
0.563914097852446	0\\
0.564014100352509	0\\
0.564114102852571	0\\
0.564214105352634	0\\
0.564314107852696	0\\
0.564414110352759	0\\
0.564514112852821	0\\
0.564614115352884	0\\
0.564714117852946	0\\
0.564814120353009	0\\
0.564914122853071	0\\
0.565014125353134	0\\
0.565114127853196	0\\
0.565214130353259	0\\
0.565314132853321	0\\
0.565414135353384	0\\
0.565514137853446	0\\
0.565614140353509	0\\
0.565714142853571	0\\
0.565814145353634	0\\
0.565914147853696	0\\
0.566014150353759	0\\
0.566114152853821	0\\
0.566214155353884	0\\
0.566314157853946	0\\
0.566414160354009	0\\
0.566514162854071	0\\
0.566614165354134	0\\
0.566714167854196	0\\
0.566814170354259	0\\
0.566914172854321	0\\
0.567014175354384	0\\
0.567114177854446	0\\
0.567214180354509	0\\
0.567314182854571	0\\
0.567414185354634	0\\
0.567514187854696	0\\
0.567614190354759	0\\
0.567714192854821	0\\
0.567814195354884	0\\
0.567914197854946	0\\
0.568014200355009	0\\
0.568114202855071	0\\
0.568214205355134	0\\
0.568314207855196	0\\
0.568414210355259	0\\
0.568514212855321	0\\
0.568614215355384	0\\
0.568714217855446	0\\
0.568814220355509	0\\
0.568914222855571	0\\
0.569014225355634	0\\
0.569114227855696	0\\
0.569214230355759	0\\
0.569314232855821	0\\
0.569414235355884	0\\
0.569514237855946	0\\
0.569614240356009	0\\
0.569714242856071	0\\
0.569814245356134	0\\
0.569914247856196	0\\
0.570014250356259	0\\
0.570114252856321	0\\
0.570214255356384	0\\
0.570314257856446	0\\
0.570414260356509	0\\
0.570514262856571	0\\
0.570614265356634	0\\
0.570714267856696	0\\
0.570814270356759	0\\
0.570914272856821	0\\
0.571014275356884	0\\
0.571114277856946	0\\
0.571214280357009	0\\
0.571314282857071	0\\
0.571414285357134	0\\
0.571514287857196	0\\
0.571614290357259	0\\
0.571714292857321	0\\
0.571814295357384	0\\
0.571914297857446	0\\
0.572014300357509	0\\
0.572114302857571	0\\
0.572214305357634	0\\
0.572314307857696	0\\
0.572414310357759	0\\
0.572514312857821	0\\
0.572614315357884	0\\
0.572714317857946	0\\
0.572814320358009	0\\
0.572914322858071	0\\
0.573014325358134	0\\
0.573114327858196	0\\
0.573214330358259	0\\
0.573314332858321	0\\
0.573414335358384	0\\
0.573514337858446	0\\
0.573614340358509	0\\
0.573714342858571	0\\
0.573814345358634	0\\
0.573914347858696	0\\
0.574014350358759	0\\
0.574114352858822	0\\
0.574214355358884	0\\
0.574314357858946	0\\
0.574414360359009	0\\
0.574514362859071	0\\
0.574614365359134	0\\
0.574714367859196	0\\
0.574814370359259	0\\
0.574914372859321	0\\
0.575014375359384	0\\
0.575114377859447	0\\
0.575214380359509	0\\
0.575314382859571	0\\
0.575414385359634	0\\
0.575514387859696	0\\
0.575614390359759	0\\
0.575714392859822	0\\
0.575814395359884	0\\
0.575914397859946	0\\
0.576014400360009	0\\
0.576114402860072	0\\
0.576214405360134	0\\
0.576314407860196	0\\
0.576414410360259	0\\
0.576514412860321	0\\
0.576614415360384	0\\
0.576714417860447	0\\
0.576814420360509	0\\
0.576914422860571	0\\
0.577014425360634	0\\
0.577114427860697	0\\
0.577214430360759	0\\
0.577314432860822	0\\
0.577414435360884	0\\
0.577514437860946	0\\
0.577614440361009	0\\
0.577714442861072	0\\
0.577814445361134	0\\
0.577914447861197	0\\
0.578014450361259	0\\
0.578114452861322	0\\
0.578214455361384	0\\
0.578314457861447	0\\
0.578414460361509	0\\
0.578514462861571	0\\
0.578614465361634	0\\
0.578714467861697	0\\
0.578814470361759	0\\
0.578914472861822	0\\
0.579014475361884	0\\
0.579114477861947	0\\
0.579214480362009	0\\
0.579314482862072	0\\
0.579414485362134	0\\
0.579514487862197	0\\
0.579614490362259	0\\
0.579714492862322	0\\
0.579814495362384	0\\
0.579914497862447	0\\
0.580014500362509	0\\
0.580114502862572	0\\
0.580214505362634	0\\
0.580314507862697	0\\
0.580414510362759	0\\
0.580514512862822	0\\
0.580614515362884	0\\
0.580714517862947	0\\
0.580814520363009	0\\
0.580914522863072	0\\
0.581014525363134	0\\
0.581114527863197	0\\
0.581214530363259	0\\
0.581314532863322	0\\
0.581414535363384	0\\
0.581514537863447	0\\
0.581614540363509	0\\
0.581714542863572	0\\
0.581814545363634	0\\
0.581914547863697	0\\
0.582014550363759	0\\
0.582114552863822	0\\
0.582214555363884	0\\
0.582314557863947	0\\
0.582414560364009	0\\
0.582514562864072	0\\
0.582614565364134	0\\
0.582714567864197	0\\
0.582814570364259	0\\
0.582914572864322	0\\
0.583014575364384	0\\
0.583114577864447	0\\
0.583214580364509	0\\
0.583314582864572	0\\
0.583414585364634	0\\
0.583514587864697	0\\
0.583614590364759	0\\
0.583714592864822	0\\
0.583814595364884	0\\
0.583914597864947	0\\
0.584014600365009	0\\
0.584114602865072	0\\
0.584214605365134	0\\
0.584314607865197	0\\
0.584414610365259	0\\
0.584514612865322	0\\
0.584614615365384	0\\
0.584714617865447	0\\
0.584814620365509	0\\
0.584914622865572	0\\
0.585014625365634	0\\
0.585114627865697	0\\
0.585214630365759	0\\
0.585314632865822	0\\
0.585414635365884	0\\
0.585514637865947	0\\
0.585614640366009	0\\
0.585714642866072	0\\
0.585814645366134	0\\
0.585914647866197	0\\
0.586014650366259	0\\
0.586114652866322	0\\
0.586214655366384	0\\
0.586314657866447	0\\
0.586414660366509	0\\
0.586514662866572	0\\
0.586614665366634	0\\
0.586714667866697	0\\
0.586814670366759	0\\
0.586914672866822	0\\
0.587014675366884	0\\
0.587114677866947	0\\
0.587214680367009	0\\
0.587314682867072	0\\
0.587414685367134	0\\
0.587514687867197	0\\
0.587614690367259	0\\
0.587714692867322	0\\
0.587814695367384	0\\
0.587914697867447	0\\
0.588014700367509	0\\
0.588114702867572	0\\
0.588214705367634	0\\
0.588314707867697	0\\
0.588414710367759	0\\
0.588514712867822	0\\
0.588614715367884	0\\
0.588714717867947	0\\
0.588814720368009	0\\
0.588914722868072	0\\
0.589014725368134	0\\
0.589114727868197	0\\
0.589214730368259	0\\
0.589314732868322	0\\
0.589414735368384	0\\
0.589514737868447	0\\
0.589614740368509	0\\
0.589714742868572	0\\
0.589814745368634	0\\
0.589914747868697	0\\
0.590014750368759	0\\
0.590114752868822	0\\
0.590214755368884	0\\
0.590314757868947	0\\
0.590414760369009	0\\
0.590514762869072	0\\
0.590614765369134	0\\
0.590714767869197	0\\
0.590814770369259	0\\
0.590914772869322	0\\
0.591014775369384	0\\
0.591114777869447	0\\
0.591214780369509	0\\
0.591314782869572	0\\
0.591414785369634	0\\
0.591514787869697	0\\
0.591614790369759	0\\
0.591714792869822	0\\
0.591814795369884	0\\
0.591914797869947	0\\
0.592014800370009	0\\
0.592114802870072	0\\
0.592214805370134	0\\
0.592314807870197	0\\
0.592414810370259	0\\
0.592514812870322	0\\
0.592614815370384	0\\
0.592714817870447	0\\
0.592814820370509	0\\
0.592914822870572	0\\
0.593014825370634	0\\
0.593114827870697	0\\
0.593214830370759	0\\
0.593314832870822	0\\
0.593414835370884	0\\
0.593514837870947	0\\
0.593614840371009	0\\
0.593714842871072	0\\
0.593814845371134	0\\
0.593914847871197	0\\
0.594014850371259	0\\
0.594114852871322	0\\
0.594214855371384	0\\
0.594314857871447	0\\
0.594414860371509	0\\
0.594514862871572	0\\
0.594614865371634	0\\
0.594714867871697	0\\
0.594814870371759	0\\
0.594914872871822	0\\
0.595014875371884	0\\
0.595114877871947	0\\
0.595214880372009	0\\
0.595314882872072	0\\
0.595414885372134	0\\
0.595514887872197	0\\
0.595614890372259	0\\
0.595714892872322	0\\
0.595814895372384	0\\
0.595914897872447	0\\
0.596014900372509	0\\
0.596114902872572	0\\
0.596214905372634	0\\
0.596314907872697	0\\
0.596414910372759	0\\
0.596514912872822	0\\
0.596614915372884	0\\
0.596714917872947	0\\
0.596814920373009	0\\
0.596914922873072	0\\
0.597014925373134	0\\
0.597114927873197	0\\
0.597214930373259	0\\
0.597314932873322	0\\
0.597414935373384	0\\
0.597514937873447	0\\
0.597614940373509	0\\
0.597714942873572	0\\
0.597814945373634	0\\
0.597914947873697	0\\
0.598014950373759	0\\
0.598114952873822	0\\
0.598214955373884	0\\
0.598314957873947	0\\
0.598414960374009	0\\
0.598514962874072	0\\
0.598614965374134	0\\
0.598714967874197	0\\
0.598814970374259	0\\
0.598914972874322	0\\
0.599014975374384	0\\
0.599114977874447	0\\
0.599214980374509	0\\
0.599314982874572	0\\
0.599414985374634	0\\
0.599514987874697	0\\
0.599614990374759	0\\
0.599714992874822	0\\
0.599814995374884	0\\
0.599914997874947	0\\
0.600015000375009	0\\
0.600115002875072	0\\
0.600215005375134	0\\
0.600315007875197	0\\
0.600415010375259	0\\
0.600515012875322	0\\
0.600615015375384	0\\
0.600715017875447	0\\
0.600815020375509	0\\
0.600915022875572	0\\
0.601015025375634	0\\
0.601115027875697	0\\
0.601215030375759	0\\
0.601315032875822	0\\
0.601415035375884	0\\
0.601515037875947	0\\
0.601615040376009	0\\
0.601715042876072	0\\
0.601815045376134	0\\
0.601915047876197	0\\
0.602015050376259	0\\
0.602115052876322	0\\
0.602215055376384	0\\
0.602315057876447	0\\
0.602415060376509	0\\
0.602515062876572	0\\
0.602615065376634	0\\
0.602715067876697	0\\
0.602815070376759	0\\
0.602915072876822	0\\
0.603015075376884	0\\
0.603115077876947	0\\
0.603215080377009	0\\
0.603315082877072	0\\
0.603415085377134	0\\
0.603515087877197	0\\
0.603615090377259	0\\
0.603715092877322	0\\
0.603815095377384	0\\
0.603915097877447	0\\
0.604015100377509	0\\
0.604115102877572	0\\
0.604215105377634	0\\
0.604315107877697	0\\
0.604415110377759	0\\
0.604515112877822	0\\
0.604615115377884	0\\
0.604715117877947	0\\
0.604815120378009	0\\
0.604915122878072	0\\
0.605015125378134	0\\
0.605115127878197	0\\
0.605215130378259	0\\
0.605315132878322	0\\
0.605415135378384	0\\
0.605515137878447	0\\
0.605615140378509	0\\
0.605715142878572	0\\
0.605815145378634	0\\
0.605915147878697	0\\
0.606015150378759	0\\
0.606115152878822	0\\
0.606215155378884	0\\
0.606315157878947	0\\
0.606415160379009	0\\
0.606515162879072	0\\
0.606615165379135	0\\
0.606715167879197	0\\
0.606815170379259	0\\
0.606915172879322	0\\
0.607015175379384	0\\
0.607115177879447	0\\
0.60721518037951	0\\
0.607315182879572	0\\
0.607415185379634	0\\
0.607515187879697	0\\
0.60761519037976	0\\
0.607715192879822	0\\
0.607815195379884	0\\
0.607915197879947	0\\
0.608015200380009	0\\
0.608115202880072	0\\
0.608215205380135	0\\
0.608315207880197	0\\
0.608415210380259	0\\
0.608515212880322	0\\
0.608615215380385	0\\
0.608715217880447	0\\
0.60881522038051	0\\
0.608915222880572	0\\
0.609015225380634	0\\
0.609115227880697	0\\
0.60921523038076	0\\
0.609315232880822	0\\
0.609415235380884	0\\
0.609515237880947	0\\
0.60961524038101	0\\
0.609715242881072	0\\
0.609815245381135	0\\
0.609915247881197	0\\
0.610015250381259	0\\
0.610115252881322	0\\
0.610215255381385	0\\
0.610315257881447	0\\
0.61041526038151	0\\
0.610515262881572	0\\
0.610615265381635	0\\
0.610715267881697	0\\
0.61081527038176	0\\
0.610915272881822	0\\
0.611015275381884	0\\
0.611115277881947	0\\
0.61121528038201	0\\
0.611315282882072	0\\
0.611415285382135	0\\
0.611515287882197	0\\
0.61161529038226	0\\
0.611715292882322	0\\
0.611815295382385	0\\
0.611915297882447	0\\
0.61201530038251	0\\
0.612115302882572	0\\
0.612215305382635	0\\
0.612315307882697	0\\
0.61241531038276	0\\
0.612515312882822	0\\
0.612615315382885	0\\
0.612715317882947	0\\
0.61281532038301	0\\
0.612915322883072	0\\
0.613015325383135	0\\
0.613115327883197	0\\
0.61321533038326	0\\
0.613315332883322	0\\
0.613415335383385	0\\
0.613515337883447	0\\
0.61361534038351	0\\
0.613715342883572	0\\
0.613815345383635	0\\
0.613915347883697	0\\
0.61401535038376	0\\
0.614115352883822	0\\
0.614215355383885	0\\
0.614315357883947	0\\
0.61441536038401	0\\
0.614515362884072	0\\
0.614615365384135	0\\
0.614715367884197	0\\
0.61481537038426	0\\
0.614915372884322	0\\
0.615015375384385	0\\
0.615115377884447	0\\
0.61521538038451	0\\
0.615315382884572	0\\
0.615415385384635	0\\
0.615515387884697	0\\
0.61561539038476	0\\
0.615715392884822	0\\
0.615815395384885	0\\
0.615915397884947	0\\
0.61601540038501	0\\
0.616115402885072	0\\
0.616215405385135	0\\
0.616315407885197	0\\
0.61641541038526	0\\
0.616515412885322	0\\
0.616615415385385	0\\
0.616715417885447	0\\
0.61681542038551	0\\
0.616915422885572	0\\
0.617015425385635	0\\
0.617115427885697	0\\
0.61721543038576	0\\
0.617315432885822	0\\
0.617415435385885	0\\
0.617515437885947	0\\
0.61761544038601	0\\
0.617715442886072	0\\
0.617815445386135	0\\
0.617915447886197	0\\
0.61801545038626	0\\
0.618115452886322	0\\
0.618215455386385	0\\
0.618315457886447	0\\
0.61841546038651	0\\
0.618515462886572	0\\
0.618615465386635	0\\
0.618715467886697	0\\
0.61881547038676	0\\
0.618915472886822	0\\
0.619015475386885	0\\
0.619115477886947	0\\
0.61921548038701	0\\
0.619315482887072	0\\
0.619415485387135	0\\
0.619515487887197	0\\
0.61961549038726	0\\
0.619715492887322	0\\
0.619815495387385	0\\
0.619915497887447	0\\
0.62001550038751	0\\
0.620115502887572	0\\
0.620215505387635	0\\
0.620315507887697	0\\
0.62041551038776	0\\
0.620515512887822	0\\
0.620615515387885	0\\
0.620715517887947	0\\
0.62081552038801	0\\
0.620915522888072	0\\
0.621015525388135	0\\
0.621115527888197	0\\
0.62121553038826	0\\
0.621315532888322	0\\
0.621415535388385	0\\
0.621515537888447	0\\
0.62161554038851	0\\
0.621715542888572	0\\
0.621815545388635	0\\
0.621915547888697	0\\
0.62201555038876	0\\
0.622115552888822	0\\
0.622215555388885	0\\
0.622315557888947	0\\
0.62241556038901	0\\
0.622515562889072	0\\
0.622615565389135	0\\
0.622715567889197	0\\
0.62281557038926	0\\
0.622915572889322	0\\
0.623015575389385	0\\
0.623115577889447	0\\
0.62321558038951	0\\
0.623315582889572	0\\
0.623415585389635	0\\
0.623515587889697	0\\
0.62361559038976	0\\
0.623715592889822	0\\
0.623815595389885	0\\
0.623915597889947	0\\
0.62401560039001	0\\
0.624115602890072	0\\
0.624215605390135	0\\
0.624315607890197	0\\
0.62441561039026	0\\
0.624515612890322	0\\
0.624615615390385	0\\
0.624715617890447	0\\
0.62481562039051	0\\
0.624915622890572	0\\
0.625015625390635	0\\
0.625115627890697	0\\
0.62521563039076	0\\
0.625315632890822	0\\
0.625415635390885	0\\
0.625515637890947	0\\
0.62561564039101	0\\
0.625715642891072	0\\
0.625815645391135	0\\
0.625915647891197	0\\
0.62601565039126	0\\
0.626115652891322	0\\
0.626215655391385	0\\
0.626315657891447	0\\
0.62641566039151	0\\
0.626515662891572	0\\
0.626615665391635	0\\
0.626715667891697	0\\
0.62681567039176	0\\
0.626915672891822	0\\
0.627015675391885	0\\
0.627115677891947	0\\
0.62721568039201	0\\
0.627315682892072	0\\
0.627415685392135	0\\
0.627515687892197	0\\
0.62761569039226	0\\
0.627715692892322	0\\
0.627815695392385	0\\
0.627915697892447	0\\
0.62801570039251	0\\
0.628115702892572	0\\
0.628215705392635	0\\
0.628315707892697	0\\
0.62841571039276	0\\
0.628515712892822	0\\
0.628615715392885	0\\
0.628715717892947	0\\
0.62881572039301	0\\
0.628915722893072	0\\
0.629015725393135	0\\
0.629115727893197	0\\
0.62921573039326	0\\
0.629315732893322	0\\
0.629415735393385	0\\
0.629515737893447	0\\
0.62961574039351	0\\
0.629715742893572	0\\
0.629815745393635	0\\
0.629915747893697	0\\
0.63001575039376	0\\
0.630115752893822	0\\
0.630215755393885	0\\
0.630315757893947	0\\
0.63041576039401	0\\
0.630515762894072	0\\
0.630615765394135	0\\
0.630715767894197	0\\
0.63081577039426	0\\
0.630915772894322	0\\
0.631015775394385	0\\
0.631115777894447	0\\
0.63121578039451	0\\
0.631315782894572	0\\
0.631415785394635	0\\
0.631515787894697	0\\
0.63161579039476	0\\
0.631715792894822	0\\
0.631815795394885	0\\
0.631915797894947	0\\
0.63201580039501	0\\
0.632115802895072	0\\
0.632215805395135	0\\
0.632315807895197	0\\
0.63241581039526	0\\
0.632515812895322	0\\
0.632615815395385	0\\
0.632715817895447	0\\
0.63281582039551	0\\
0.632915822895572	0\\
0.633015825395635	0\\
0.633115827895697	0\\
0.63321583039576	0\\
0.633315832895822	0\\
0.633415835395885	0\\
0.633515837895947	0\\
0.63361584039601	0\\
0.633715842896072	0\\
0.633815845396135	0\\
0.633915847896197	0\\
0.63401585039626	0\\
0.634115852896322	0\\
0.634215855396385	0\\
0.634315857896447	0\\
0.63441586039651	0\\
0.634515862896572	0\\
0.634615865396635	0\\
0.634715867896697	0\\
0.63481587039676	0\\
0.634915872896822	0\\
0.635015875396885	0\\
0.635115877896947	0\\
0.63521588039701	0\\
0.635315882897072	0\\
0.635415885397135	0\\
0.635515887897197	0\\
0.63561589039726	0\\
0.635715892897322	0\\
0.635815895397385	0\\
0.635915897897447	0\\
0.63601590039751	0\\
0.636115902897572	0\\
0.636215905397635	0\\
0.636315907897697	0\\
0.63641591039776	0\\
0.636515912897823	0\\
0.636615915397885	0\\
0.636715917897947	0\\
0.63681592039801	0\\
0.636915922898072	0\\
0.637015925398135	0\\
0.637115927898197	0\\
0.63721593039826	0\\
0.637315932898322	0\\
0.637415935398385	0\\
0.637515937898448	0\\
0.63761594039851	0\\
0.637715942898572	0\\
0.637815945398635	0\\
0.637915947898697	0\\
0.63801595039876	0\\
0.638115952898823	0\\
0.638215955398885	0\\
0.638315957898947	0\\
0.63841596039901	0\\
0.638515962899073	0\\
0.638615965399135	0\\
0.638715967899197	0\\
0.63881597039926	0\\
0.638915972899322	0\\
0.639015975399385	0\\
0.639115977899448	0\\
0.63921598039951	0\\
0.639315982899572	0\\
0.639415985399635	0\\
0.639515987899698	0\\
0.63961599039976	0\\
0.639715992899823	0\\
0.639815995399885	0\\
0.639915997899947	0\\
0.64001600040001	0\\
0.640116002900073	0\\
0.640216005400135	0\\
0.640316007900197	0\\
0.64041601040026	0\\
0.640516012900323	0\\
0.640616015400385	0\\
0.640716017900448	0\\
0.64081602040051	0\\
0.640916022900572	0\\
0.641016025400635	0\\
0.641116027900698	0\\
0.64121603040076	0\\
0.641316032900823	0\\
0.641416035400885	0\\
0.641516037900948	0\\
0.64161604040101	0\\
0.641716042901073	0\\
0.641816045401135	0\\
0.641916047901197	0\\
0.64201605040126	0\\
0.642116052901323	0\\
0.642216055401385	0\\
0.642316057901448	0\\
0.64241606040151	0\\
0.642516062901573	0\\
0.642616065401635	0\\
0.642716067901698	0\\
0.64281607040176	0\\
0.642916072901823	0\\
0.643016075401885	0\\
0.643116077901948	0\\
0.64321608040201	0\\
0.643316082902073	0\\
0.643416085402135	0\\
0.643516087902198	0\\
0.64361609040226	0\\
0.643716092902323	0\\
0.643816095402385	0\\
0.643916097902448	0\\
0.64401610040251	0\\
0.644116102902573	0\\
0.644216105402635	0\\
0.644316107902698	0\\
0.64441611040276	0\\
0.644516112902823	0\\
0.644616115402885	0\\
0.644716117902948	0\\
0.64481612040301	0\\
0.644916122903073	0\\
0.645016125403135	0\\
0.645116127903198	0\\
0.64521613040326	0\\
0.645316132903323	0\\
0.645416135403385	0\\
0.645516137903448	0\\
0.64561614040351	0\\
0.645716142903573	0\\
0.645816145403635	0\\
0.645916147903698	0\\
0.64601615040376	0\\
0.646116152903823	0\\
0.646216155403885	0\\
0.646316157903948	0\\
0.64641616040401	0\\
0.646516162904073	0\\
0.646616165404135	0\\
0.646716167904198	0\\
0.64681617040426	0\\
0.646916172904323	0\\
0.647016175404385	0\\
0.647116177904448	0\\
0.64721618040451	0\\
0.647316182904573	0\\
0.647416185404635	0\\
0.647516187904698	0\\
0.64761619040476	0\\
0.647716192904823	0\\
0.647816195404885	0\\
0.647916197904948	0\\
0.64801620040501	0\\
0.648116202905073	0\\
0.648216205405135	0\\
0.648316207905198	0\\
0.64841621040526	0\\
0.648516212905323	0\\
0.648616215405385	0\\
0.648716217905448	0\\
0.64881622040551	0\\
0.648916222905573	0\\
0.649016225405635	0\\
0.649116227905698	0\\
0.64921623040576	0\\
0.649316232905823	0\\
0.649416235405885	0\\
0.649516237905948	0\\
0.64961624040601	0\\
0.649716242906073	0\\
0.649816245406135	0\\
0.649916247906198	0\\
0.65001625040626	0\\
0.650116252906323	0\\
0.650216255406385	0\\
0.650316257906448	0\\
0.65041626040651	0\\
0.650516262906573	0\\
0.650616265406635	0\\
0.650716267906698	0\\
0.65081627040676	0\\
0.650916272906823	0\\
0.651016275406885	0\\
0.651116277906948	0\\
0.65121628040701	0\\
0.651316282907073	0\\
0.651416285407135	0\\
0.651516287907198	0\\
0.65161629040726	0\\
0.651716292907323	0\\
0.651816295407385	0\\
0.651916297907448	0\\
0.65201630040751	0\\
0.652116302907573	0\\
0.652216305407635	0\\
0.652316307907698	0\\
0.65241631040776	0\\
0.652516312907823	0\\
0.652616315407885	0\\
0.652716317907948	0\\
0.65281632040801	0\\
0.652916322908073	0\\
0.653016325408135	0\\
0.653116327908198	0\\
0.65321633040826	0\\
0.653316332908323	0\\
0.653416335408385	0\\
0.653516337908448	0\\
0.65361634040851	0\\
0.653716342908573	0\\
0.653816345408635	0\\
0.653916347908698	0\\
0.65401635040876	0\\
0.654116352908823	0\\
0.654216355408885	0\\
0.654316357908948	0\\
0.65441636040901	0\\
0.654516362909073	0\\
0.654616365409135	0\\
0.654716367909198	0\\
0.65481637040926	0\\
0.654916372909323	0\\
0.655016375409385	0\\
0.655116377909448	0\\
0.65521638040951	0\\
0.655316382909573	0\\
0.655416385409635	0\\
0.655516387909698	0\\
0.65561639040976	0\\
0.655716392909823	0\\
0.655816395409885	0\\
0.655916397909948	0\\
0.65601640041001	0\\
0.656116402910073	0\\
0.656216405410135	0\\
0.656316407910198	0\\
0.65641641041026	0\\
0.656516412910323	0\\
0.656616415410385	0\\
0.656716417910448	0\\
0.65681642041051	0\\
0.656916422910573	0\\
0.657016425410635	0\\
0.657116427910698	0\\
0.65721643041076	0\\
0.657316432910823	0\\
0.657416435410885	0\\
0.657516437910948	0\\
0.65761644041101	0\\
0.657716442911073	0\\
0.657816445411135	0\\
0.657916447911198	0\\
0.65801645041126	0\\
0.658116452911323	0\\
0.658216455411385	0\\
0.658316457911448	0\\
0.65841646041151	0\\
0.658516462911573	0\\
0.658616465411635	0\\
0.658716467911698	0\\
0.65881647041176	0\\
0.658916472911823	0\\
0.659016475411885	0\\
0.659116477911948	0\\
0.65921648041201	0\\
0.659316482912073	0\\
0.659416485412135	0\\
0.659516487912198	0\\
0.65961649041226	0\\
0.659716492912323	0\\
0.659816495412385	0\\
0.659916497912448	0\\
0.66001650041251	0\\
0.660116502912573	0\\
0.660216505412635	0\\
0.660316507912698	0\\
0.66041651041276	0\\
0.660516512912823	0\\
0.660616515412885	0\\
0.660716517912948	0\\
0.66081652041301	0\\
0.660916522913073	0\\
0.661016525413135	0\\
0.661116527913198	0\\
0.66121653041326	0\\
0.661316532913323	0\\
0.661416535413385	0\\
0.661516537913448	0\\
0.66161654041351	0\\
0.661716542913573	0\\
0.661816545413635	0\\
0.661916547913698	0\\
0.66201655041376	0\\
0.662116552913823	0\\
0.662216555413885	0\\
0.662316557913948	0\\
0.66241656041401	0\\
0.662516562914073	0\\
0.662616565414135	0\\
0.662716567914198	0\\
0.66281657041426	0\\
0.662916572914323	0\\
0.663016575414385	0\\
0.663116577914448	0\\
0.66321658041451	0\\
0.663316582914573	0\\
0.663416585414635	0\\
0.663516587914698	0\\
0.66361659041476	0\\
0.663716592914823	0\\
0.663816595414885	0\\
0.663916597914948	0\\
0.66401660041501	0\\
0.664116602915073	0\\
0.664216605415135	0\\
0.664316607915198	0\\
0.66441661041526	0\\
0.664516612915323	0\\
0.664616615415385	0\\
0.664716617915448	0\\
0.66481662041551	0\\
0.664916622915573	0\\
0.665016625415635	0\\
0.665116627915698	0\\
0.66521663041576	0\\
0.665316632915823	0\\
0.665416635415885	0\\
0.665516637915948	0\\
0.66561664041601	0\\
0.665716642916073	0\\
0.665816645416135	0\\
0.665916647916198	0\\
0.66601665041626	0\\
0.666116652916323	0\\
0.666216655416385	0\\
0.666316657916448	0\\
0.66641666041651	0\\
0.666516662916573	0\\
0.666616665416635	0\\
0.666716667916698	0\\
0.66681667041676	0\\
0.666916672916823	0\\
0.667016675416885	0\\
0.667116677916948	0\\
0.66721668041701	0\\
0.667316682917073	0\\
0.667416685417135	0\\
0.667516687917198	0\\
0.66761669041726	0\\
0.667716692917323	0\\
0.667816695417385	0\\
0.667916697917448	0\\
0.66801670041751	0\\
0.668116702917573	0\\
0.668216705417635	0\\
0.668316707917698	0\\
0.66841671041776	0\\
0.668516712917823	0\\
0.668616715417885	0\\
0.668716717917948	0\\
0.66881672041801	0\\
0.668916722918073	0\\
0.669016725418136	0\\
0.669116727918198	0\\
0.66921673041826	0\\
0.669316732918323	0\\
0.669416735418385	0\\
0.669516737918448	0\\
0.66961674041851	0\\
0.669716742918573	0\\
0.669816745418635	0\\
0.669916747918698	0\\
0.670016750418761	0\\
0.670116752918823	0\\
0.670216755418885	0\\
0.670316757918948	0\\
0.67041676041901	0\\
0.670516762919073	0\\
0.670616765419136	0\\
0.670716767919198	0\\
0.67081677041926	0\\
0.670916772919323	0\\
0.671016775419386	0\\
0.671116777919448	0\\
0.67121678041951	0\\
0.671316782919573	0\\
0.671416785419635	0\\
0.671516787919698	0\\
0.671616790419761	0\\
0.671716792919823	0\\
0.671816795419885	0\\
0.671916797919948	0\\
0.672016800420011	0\\
0.672116802920073	0\\
0.672216805420136	0\\
0.672316807920198	0\\
0.67241681042026	0\\
0.672516812920323	0\\
0.672616815420386	0\\
0.672716817920448	0\\
0.67281682042051	0\\
0.672916822920573	0\\
0.673016825420636	0\\
0.673116827920698	0\\
0.673216830420761	0\\
0.673316832920823	0\\
0.673416835420885	0\\
0.673516837920948	0\\
0.673616840421011	0\\
0.673716842921073	0\\
0.673816845421136	0\\
0.673916847921198	0\\
0.674016850421261	0\\
0.674116852921323	0\\
0.674216855421386	0\\
0.674316857921448	0\\
0.674416860421511	0\\
0.674516862921573	0\\
0.674616865421636	0\\
0.674716867921698	0\\
0.674816870421761	0\\
0.674916872921823	0\\
0.675016875421886	0\\
0.675116877921948	0\\
0.675216880422011	0\\
0.675316882922073	0\\
0.675416885422136	0\\
0.675516887922198	0\\
0.675616890422261	0\\
0.675716892922323	0\\
0.675816895422386	0\\
0.675916897922448	0\\
0.676016900422511	0\\
0.676116902922573	0\\
0.676216905422636	0\\
0.676316907922698	0\\
0.676416910422761	0\\
0.676516912922823	0\\
0.676616915422886	0\\
0.676716917922948	0\\
0.676816920423011	0\\
0.676916922923073	0\\
0.677016925423136	0\\
0.677116927923198	0\\
0.677216930423261	0\\
0.677316932923323	0\\
0.677416935423386	0\\
0.677516937923448	0\\
0.677616940423511	0\\
0.677716942923573	0\\
0.677816945423636	0\\
0.677916947923698	0\\
0.678016950423761	0\\
0.678116952923823	0\\
0.678216955423886	0\\
0.678316957923948	0\\
0.678416960424011	0\\
0.678516962924073	0\\
0.678616965424136	0\\
0.678716967924198	0\\
0.678816970424261	0\\
0.678916972924323	0\\
0.679016975424386	0\\
0.679116977924448	0\\
0.679216980424511	0\\
0.679316982924573	0\\
0.679416985424636	0\\
0.679516987924698	0\\
0.679616990424761	0\\
0.679716992924823	0\\
0.679816995424886	0\\
0.679916997924948	0\\
0.680017000425011	0\\
0.680117002925073	0\\
0.680217005425136	0\\
0.680317007925198	0\\
0.680417010425261	0\\
0.680517012925323	0\\
0.680617015425386	0\\
0.680717017925448	0\\
0.680817020425511	0\\
0.680917022925573	0\\
0.681017025425636	0\\
0.681117027925698	0\\
0.681217030425761	0\\
0.681317032925823	0\\
0.681417035425886	0\\
0.681517037925948	0\\
0.681617040426011	0\\
0.681717042926073	0\\
0.681817045426136	0\\
0.681917047926198	0\\
0.682017050426261	0\\
0.682117052926323	0\\
0.682217055426386	0\\
0.682317057926448	0\\
0.682417060426511	0\\
0.682517062926573	0\\
0.682617065426636	0\\
0.682717067926698	0\\
0.682817070426761	0\\
0.682917072926823	0\\
0.683017075426886	0\\
0.683117077926948	0\\
0.683217080427011	0\\
0.683317082927073	0\\
0.683417085427136	0\\
0.683517087927198	0\\
0.683617090427261	0\\
0.683717092927323	0\\
0.683817095427386	0\\
0.683917097927448	0\\
0.684017100427511	0\\
0.684117102927573	0\\
0.684217105427636	0\\
0.684317107927698	0\\
0.684417110427761	0\\
0.684517112927823	0\\
0.684617115427886	0\\
0.684717117927948	0\\
0.684817120428011	0\\
0.684917122928073	0\\
0.685017125428136	0\\
0.685117127928198	0\\
0.685217130428261	0\\
0.685317132928323	0\\
0.685417135428386	0\\
0.685517137928448	0\\
0.685617140428511	0\\
0.685717142928573	0\\
0.685817145428636	0\\
0.685917147928698	0\\
0.686017150428761	0\\
0.686117152928823	0\\
0.686217155428886	0\\
0.686317157928948	0\\
0.686417160429011	0\\
0.686517162929073	0\\
0.686617165429136	0\\
0.686717167929198	0\\
0.686817170429261	0\\
0.686917172929323	0\\
0.687017175429386	0\\
0.687117177929448	0\\
0.687217180429511	0\\
0.687317182929573	0\\
0.687417185429636	0\\
0.687517187929698	0\\
0.687617190429761	0\\
0.687717192929823	0\\
0.687817195429886	0\\
0.687917197929948	0\\
0.688017200430011	0\\
0.688117202930073	0\\
0.688217205430136	0\\
0.688317207930198	0\\
0.688417210430261	0\\
0.688517212930323	0\\
0.688617215430386	0\\
0.688717217930448	0\\
0.688817220430511	0\\
0.688917222930573	0\\
0.689017225430636	0\\
0.689117227930698	0\\
0.689217230430761	0\\
0.689317232930823	0\\
0.689417235430886	0\\
0.689517237930948	0\\
0.689617240431011	0\\
0.689717242931073	0\\
0.689817245431136	0\\
0.689917247931198	0\\
0.690017250431261	0\\
0.690117252931323	0\\
0.690217255431386	0\\
0.690317257931448	0\\
0.690417260431511	0\\
0.690517262931573	0\\
0.690617265431636	0\\
0.690717267931698	0\\
0.690817270431761	0\\
0.690917272931823	0\\
0.691017275431886	0\\
0.691117277931948	0\\
0.691217280432011	0\\
0.691317282932073	0\\
0.691417285432136	0\\
0.691517287932198	0\\
0.691617290432261	0\\
0.691717292932323	0\\
0.691817295432386	0\\
0.691917297932448	0\\
0.692017300432511	0\\
0.692117302932573	0\\
0.692217305432636	0\\
0.692317307932698	0\\
0.692417310432761	0\\
0.692517312932823	0\\
0.692617315432886	0\\
0.692717317932948	0\\
0.692817320433011	0\\
0.692917322933073	0\\
0.693017325433136	0\\
0.693117327933198	0\\
0.693217330433261	0\\
0.693317332933323	0\\
0.693417335433386	0\\
0.693517337933448	0\\
0.693617340433511	0\\
0.693717342933573	0\\
0.693817345433636	0\\
0.693917347933698	0\\
0.694017350433761	0\\
0.694117352933823	0\\
0.694217355433886	0\\
0.694317357933948	0\\
0.694417360434011	0\\
0.694517362934073	0\\
0.694617365434136	0\\
0.694717367934198	0\\
0.694817370434261	0\\
0.694917372934323	0\\
0.695017375434386	0\\
0.695117377934448	0\\
0.695217380434511	0\\
0.695317382934573	0\\
0.695417385434636	0\\
0.695517387934698	0\\
0.695617390434761	0\\
0.695717392934823	0\\
0.695817395434886	0\\
0.695917397934948	0\\
0.696017400435011	0\\
0.696117402935073	0\\
0.696217405435136	0\\
0.696317407935198	0\\
0.696417410435261	0\\
0.696517412935323	0\\
0.696617415435386	0\\
0.696717417935448	0\\
0.696817420435511	0\\
0.696917422935573	0\\
0.697017425435636	0\\
0.697117427935698	0\\
0.697217430435761	0\\
0.697317432935823	0\\
0.697417435435886	0\\
0.697517437935948	0\\
0.697617440436011	0\\
0.697717442936073	0\\
0.697817445436136	0\\
0.697917447936198	0\\
0.698017450436261	0\\
0.698117452936323	0\\
0.698217455436386	0\\
0.698317457936448	0\\
0.698417460436511	0\\
0.698517462936573	0\\
0.698617465436636	0\\
0.698717467936698	0\\
0.698817470436761	0\\
0.698917472936823	0\\
0.699017475436886	0\\
0.699117477936948	0\\
0.699217480437011	0\\
0.699317482937073	0\\
0.699417485437136	0\\
0.699517487937198	0\\
0.699617490437261	0\\
0.699717492937323	0\\
0.699817495437386	0\\
0.699917497937448	0\\
0.700017500437511	0\\
0.700117502937573	0\\
0.700217505437636	0\\
0.700317507937698	0\\
0.700417510437761	0\\
0.700517512937823	0\\
0.700617515437886	0\\
0.700717517937948	0\\
0.700817520438011	0\\
0.700917522938073	0\\
0.701017525438136	0\\
0.701117527938198	0\\
0.701217530438261	0\\
0.701317532938323	0\\
0.701417535438386	0\\
0.701517537938449	0\\
0.701617540438511	0\\
0.701717542938573	0\\
0.701817545438636	0\\
0.701917547938698	0\\
0.702017550438761	0\\
0.702117552938823	0\\
0.702217555438886	0\\
0.702317557938948	0\\
0.702417560439011	0\\
0.702517562939074	0\\
0.702617565439136	0\\
0.702717567939198	0\\
0.702817570439261	0\\
0.702917572939323	0\\
0.703017575439386	0\\
0.703117577939449	0\\
0.703217580439511	0\\
0.703317582939573	0\\
0.703417585439636	0\\
0.703517587939699	0\\
0.703617590439761	0\\
0.703717592939823	0\\
0.703817595439886	0\\
0.703917597939948	0\\
0.704017600440011	0\\
0.704117602940074	0\\
0.704217605440136	0\\
0.704317607940198	0\\
0.704417610440261	0\\
0.704517612940324	0\\
0.704617615440386	0\\
0.704717617940449	0\\
0.704817620440511	0\\
0.704917622940573	0\\
0.705017625440636	0\\
0.705117627940699	0\\
0.705217630440761	0\\
0.705317632940824	0\\
0.705417635440886	0\\
0.705517637940949	0\\
0.705617640441011	0\\
0.705717642941074	0\\
0.705817645441136	0\\
0.705917647941198	0\\
0.706017650441261	0\\
0.706117652941324	0\\
0.706217655441386	0\\
0.706317657941449	0\\
0.706417660441511	0\\
0.706517662941574	0\\
0.706617665441636	0\\
0.706717667941699	0\\
0.706817670441761	0\\
0.706917672941824	0\\
0.707017675441886	0\\
0.707117677941949	0\\
0.707217680442011	0\\
0.707317682942074	0\\
0.707417685442136	0\\
0.707517687942199	0\\
0.707617690442261	0\\
0.707717692942324	0\\
0.707817695442386	0\\
0.707917697942449	0\\
0.708017700442511	0\\
0.708117702942574	0\\
0.708217705442636	0\\
0.708317707942699	0\\
0.708417710442761	0\\
0.708517712942824	0\\
0.708617715442886	0\\
0.708717717942949	0\\
0.708817720443011	0\\
0.708917722943074	0\\
0.709017725443136	0\\
0.709117727943199	0\\
0.709217730443261	0\\
0.709317732943324	0\\
0.709417735443386	0\\
0.709517737943449	0\\
0.709617740443511	0\\
0.709717742943574	0\\
0.709817745443636	0\\
0.709917747943699	0\\
0.710017750443761	0\\
0.710117752943824	0\\
0.710217755443886	0\\
0.710317757943949	0\\
0.710417760444011	0\\
0.710517762944074	0\\
0.710617765444136	0\\
0.710717767944199	0\\
0.710817770444261	0\\
0.710917772944324	0\\
0.711017775444386	0\\
0.711117777944449	0\\
0.711217780444511	0\\
0.711317782944574	0\\
0.711417785444636	0\\
0.711517787944699	0\\
0.711617790444761	0\\
0.711717792944824	0\\
0.711817795444886	0\\
0.711917797944949	0\\
0.712017800445011	0\\
0.712117802945074	0\\
0.712217805445136	0\\
0.712317807945199	0\\
0.712417810445261	0\\
0.712517812945324	0\\
0.712617815445386	0\\
0.712717817945449	0\\
0.712817820445511	0\\
0.712917822945574	0\\
0.713017825445636	0\\
0.713117827945699	0\\
0.713217830445761	0\\
0.713317832945824	0\\
0.713417835445886	0\\
0.713517837945949	0\\
0.713617840446011	0\\
0.713717842946074	0\\
0.713817845446136	0\\
0.713917847946199	0\\
0.714017850446261	0\\
0.714117852946324	0\\
0.714217855446386	0\\
0.714317857946449	0\\
0.714417860446511	0\\
0.714517862946574	0\\
0.714617865446636	0\\
0.714717867946699	0\\
0.714817870446761	0\\
0.714917872946824	0\\
0.715017875446886	0\\
0.715117877946949	0\\
0.715217880447011	0\\
0.715317882947074	0\\
0.715417885447136	0\\
0.715517887947199	0\\
0.715617890447261	0\\
0.715717892947324	0\\
0.715817895447386	0\\
0.715917897947449	0\\
0.716017900447511	0\\
0.716117902947574	0\\
0.716217905447636	0\\
0.716317907947699	0\\
0.716417910447761	0\\
0.716517912947824	0\\
0.716617915447886	0\\
0.716717917947949	0\\
0.716817920448011	0\\
0.716917922948074	0\\
0.717017925448136	0\\
0.717117927948199	0\\
0.717217930448261	0\\
0.717317932948324	0\\
0.717417935448386	0\\
0.717517937948449	0\\
0.717617940448511	0\\
0.717717942948574	0\\
0.717817945448636	0\\
0.717917947948699	0\\
0.718017950448761	0\\
0.718117952948824	0\\
0.718217955448886	0\\
0.718317957948949	0\\
0.718417960449011	0\\
0.718517962949074	0\\
0.718617965449136	0\\
0.718717967949199	0\\
0.718817970449261	0\\
0.718917972949324	0\\
0.719017975449386	0\\
0.719117977949449	0\\
0.719217980449511	0\\
0.719317982949574	0\\
0.719417985449636	0\\
0.719517987949699	0\\
0.719617990449761	0\\
0.719717992949824	0\\
0.719817995449886	0\\
0.719917997949949	0\\
0.720018000450011	0\\
0.720118002950074	0\\
0.720218005450136	0\\
0.720318007950199	0\\
0.720418010450261	0\\
0.720518012950324	0\\
0.720618015450386	0\\
0.720718017950449	0\\
0.720818020450511	0\\
0.720918022950574	0\\
0.721018025450636	0\\
0.721118027950699	0\\
0.721218030450761	0\\
0.721318032950824	0\\
0.721418035450886	0\\
0.721518037950949	0\\
0.721618040451011	0\\
0.721718042951074	0\\
0.721818045451136	0\\
0.721918047951199	0\\
0.722018050451261	0\\
0.722118052951324	0\\
0.722218055451386	0\\
0.722318057951449	0\\
0.722418060451511	0\\
0.722518062951574	0\\
0.722618065451636	0\\
0.722718067951699	0\\
0.722818070451761	0\\
0.722918072951824	0\\
0.723018075451886	0\\
0.723118077951949	0\\
0.723218080452011	0\\
0.723318082952074	0\\
0.723418085452136	0\\
0.723518087952199	0\\
0.723618090452261	0\\
0.723718092952324	0\\
0.723818095452386	0\\
0.723918097952449	0\\
0.724018100452511	0\\
0.724118102952574	0\\
0.724218105452636	0\\
0.724318107952699	0\\
0.724418110452761	0\\
0.724518112952824	0\\
0.724618115452886	0\\
0.724718117952949	0\\
0.724818120453011	0\\
0.724918122953074	0\\
0.725018125453136	0\\
0.725118127953199	0\\
0.725218130453261	0\\
0.725318132953324	0\\
0.725418135453386	0\\
0.725518137953449	0\\
0.725618140453511	0\\
0.725718142953574	0\\
0.725818145453636	0\\
0.725918147953699	0\\
0.726018150453761	0\\
0.726118152953824	0\\
0.726218155453886	0\\
0.726318157953949	0\\
0.726418160454011	0\\
0.726518162954074	0\\
0.726618165454136	0\\
0.726718167954199	0\\
0.726818170454261	0\\
0.726918172954324	0\\
0.727018175454386	0\\
0.727118177954449	0\\
0.727218180454511	0\\
0.727318182954574	0\\
0.727418185454636	0\\
0.727518187954699	0\\
0.727618190454761	0\\
0.727718192954824	0\\
0.727818195454886	0\\
0.727918197954949	0\\
0.728018200455011	0\\
0.728118202955074	0\\
0.728218205455136	0\\
0.728318207955199	0\\
0.728418210455261	0\\
0.728518212955324	0\\
0.728618215455386	0\\
0.728718217955449	0\\
0.728818220455511	0\\
0.728918222955574	0\\
0.729018225455636	0\\
0.729118227955699	0\\
0.729218230455761	0\\
0.729318232955824	0\\
0.729418235455886	0\\
0.729518237955949	0\\
0.729618240456011	0\\
0.729718242956074	0\\
0.729818245456136	0\\
0.729918247956199	0\\
0.730018250456261	0\\
0.730118252956324	0\\
0.730218255456386	0\\
0.730318257956449	0\\
0.730418260456511	0\\
0.730518262956574	0\\
0.730618265456636	0\\
0.730718267956699	0\\
0.730818270456761	0\\
0.730918272956824	0\\
0.731018275456886	0\\
0.731118277956949	0\\
0.731218280457011	0\\
0.731318282957074	0\\
0.731418285457136	0\\
0.731518287957199	0\\
0.731618290457261	0\\
0.731718292957324	0\\
0.731818295457386	0\\
0.731918297957449	0\\
0.732018300457511	0\\
0.732118302957574	0\\
0.732218305457636	0\\
0.732318307957699	0\\
0.732418310457761	0\\
0.732518312957824	0\\
0.732618315457886	0\\
0.732718317957949	0\\
0.732818320458011	0\\
0.732918322958074	0\\
0.733018325458136	0\\
0.733118327958199	0\\
0.733218330458261	0\\
0.733318332958324	0\\
0.733418335458386	0\\
0.733518337958449	0\\
0.733618340458511	0\\
0.733718342958574	0\\
0.733818345458636	0\\
0.733918347958699	0\\
0.734018350458762	0\\
0.734118352958824	0\\
0.734218355458886	0\\
0.734318357958949	0\\
0.734418360459011	0\\
0.734518362959074	0\\
0.734618365459137	0\\
0.734718367959199	0\\
0.734818370459261	0\\
0.734918372959324	0\\
0.735018375459387	0\\
0.735118377959449	0\\
0.735218380459511	0\\
0.735318382959574	0\\
0.735418385459636	0\\
0.735518387959699	0\\
0.735618390459762	0\\
0.735718392959824	0\\
0.735818395459886	0\\
0.735918397959949	0\\
0.736018400460012	0\\
0.736118402960074	0\\
0.736218405460137	0\\
0.736318407960199	0\\
0.736418410460261	0\\
0.736518412960324	0\\
0.736618415460387	0\\
0.736718417960449	0\\
0.736818420460511	0\\
0.736918422960574	0\\
0.737018425460637	0\\
0.737118427960699	0\\
0.737218430460762	0\\
0.737318432960824	0\\
0.737418435460886	0\\
0.737518437960949	0\\
0.737618440461012	0\\
0.737718442961074	0\\
0.737818445461137	0\\
0.737918447961199	0\\
0.738018450461262	0\\
0.738118452961324	0\\
0.738218455461387	0\\
0.738318457961449	0\\
0.738418460461511	0\\
0.738518462961574	0\\
0.738618465461637	0\\
0.738718467961699	0\\
0.738818470461762	0\\
0.738918472961824	0\\
0.739018475461887	0\\
0.739118477961949	0\\
0.739218480462012	0\\
0.739318482962074	0\\
0.739418485462137	0\\
0.739518487962199	0\\
0.739618490462262	0\\
0.739718492962324	0\\
0.739818495462387	0\\
0.739918497962449	0\\
0.740018500462512	0\\
0.740118502962574	0\\
0.740218505462637	0\\
0.740318507962699	0\\
0.740418510462762	0\\
0.740518512962824	0\\
0.740618515462887	0\\
0.740718517962949	0\\
0.740818520463012	0\\
0.740918522963074	0\\
0.741018525463137	0\\
0.741118527963199	0\\
0.741218530463262	0\\
0.741318532963324	0\\
0.741418535463387	0\\
0.741518537963449	0\\
0.741618540463512	0\\
0.741718542963574	0\\
0.741818545463637	0\\
0.741918547963699	0\\
0.742018550463762	0\\
0.742118552963824	0\\
0.742218555463887	0\\
0.742318557963949	0\\
0.742418560464012	0\\
0.742518562964074	0\\
0.742618565464137	0\\
0.742718567964199	0\\
0.742818570464262	0\\
0.742918572964324	0\\
0.743018575464387	0\\
0.743118577964449	0\\
0.743218580464512	0\\
0.743318582964574	0\\
0.743418585464637	0\\
0.743518587964699	0\\
0.743618590464762	0\\
0.743718592964824	0\\
0.743818595464887	0\\
0.743918597964949	0\\
0.744018600465012	0\\
0.744118602965074	0\\
0.744218605465137	0\\
0.744318607965199	0\\
0.744418610465262	0\\
0.744518612965324	0\\
0.744618615465387	0\\
0.744718617965449	0\\
0.744818620465512	0\\
0.744918622965574	0\\
0.745018625465637	0\\
0.745118627965699	0\\
0.745218630465762	0\\
0.745318632965824	0\\
0.745418635465887	0\\
0.745518637965949	0\\
0.745618640466012	0\\
0.745718642966074	0\\
0.745818645466137	0\\
0.745918647966199	0\\
0.746018650466262	0\\
0.746118652966324	0\\
0.746218655466387	0\\
0.746318657966449	0\\
0.746418660466512	0\\
0.746518662966574	0\\
0.746618665466637	0\\
0.746718667966699	0\\
0.746818670466762	0\\
0.746918672966824	0\\
0.747018675466887	0\\
0.747118677966949	0\\
0.747218680467012	0\\
0.747318682967074	0\\
0.747418685467137	0\\
0.747518687967199	0\\
0.747618690467262	0\\
0.747718692967324	0\\
0.747818695467387	0\\
0.747918697967449	0\\
0.748018700467512	0\\
0.748118702967574	0\\
0.748218705467637	0\\
0.748318707967699	0\\
0.748418710467762	0\\
0.748518712967824	0\\
0.748618715467887	0\\
0.748718717967949	0\\
0.748818720468012	0\\
0.748918722968074	0\\
0.749018725468137	0\\
0.749118727968199	0\\
0.749218730468262	0\\
0.749318732968324	0\\
0.749418735468387	0\\
0.749518737968449	0\\
0.749618740468512	0\\
0.749718742968574	0\\
0.749818745468637	0\\
0.749918747968699	0\\
0.750018750468762	0\\
0.750118752968824	0\\
0.750218755468887	0\\
0.750318757968949	0\\
0.750418760469012	0\\
0.750518762969074	0\\
0.750618765469137	0\\
0.750718767969199	0\\
0.750818770469262	0\\
0.750918772969324	0\\
0.751018775469387	0\\
0.751118777969449	0\\
0.751218780469512	0\\
0.751318782969574	0\\
0.751418785469637	0\\
0.751518787969699	0\\
0.751618790469762	0\\
0.751718792969824	0\\
0.751818795469887	0\\
0.751918797969949	0\\
0.752018800470012	0\\
0.752118802970074	0\\
0.752218805470137	0\\
0.752318807970199	0\\
0.752418810470262	0\\
0.752518812970324	0\\
0.752618815470387	0\\
0.752718817970449	0\\
0.752818820470512	0\\
0.752918822970574	0\\
0.753018825470637	0\\
0.753118827970699	0\\
0.753218830470762	0\\
0.753318832970824	0\\
0.753418835470887	0\\
0.753518837970949	0\\
0.753618840471012	0\\
0.753718842971074	0\\
0.753818845471137	0\\
0.753918847971199	0\\
0.754018850471262	0\\
0.754118852971324	0\\
0.754218855471387	0\\
0.754318857971449	0\\
0.754418860471512	0\\
0.754518862971574	0\\
0.754618865471637	0\\
0.754718867971699	0\\
0.754818870471762	0\\
0.754918872971824	0\\
0.755018875471887	0\\
0.755118877971949	0\\
0.755218880472012	0\\
0.755318882972074	0\\
0.755418885472137	0\\
0.755518887972199	0\\
0.755618890472262	0\\
0.755718892972324	0\\
0.755818895472387	0\\
0.755918897972449	0\\
0.756018900472512	0\\
0.756118902972574	0\\
0.756218905472637	0\\
0.756318907972699	0\\
0.756418910472762	0\\
0.756518912972824	0\\
0.756618915472887	0\\
0.756718917972949	0\\
0.756818920473012	0\\
0.756918922973074	0\\
0.757018925473137	0\\
0.757118927973199	0\\
0.757218930473262	0\\
0.757318932973324	0\\
0.757418935473387	0\\
0.757518937973449	0\\
0.757618940473512	0\\
0.757718942973574	0\\
0.757818945473637	0\\
0.757918947973699	0\\
0.758018950473762	0\\
0.758118952973824	0\\
0.758218955473887	0\\
0.758318957973949	0\\
0.758418960474012	0\\
0.758518962974074	0\\
0.758618965474137	0\\
0.758718967974199	0\\
0.758818970474262	0\\
0.758918972974324	0\\
0.759018975474387	0\\
0.759118977974449	0\\
0.759218980474512	0\\
0.759318982974574	0\\
0.759418985474637	0\\
0.759518987974699	0\\
0.759618990474762	0\\
0.759718992974824	0\\
0.759818995474887	0\\
0.759918997974949	0\\
0.760019000475012	0\\
0.760119002975074	0\\
0.760219005475137	0\\
0.760319007975199	0\\
0.760419010475262	0\\
0.760519012975324	0\\
0.760619015475387	0\\
0.760719017975449	0\\
0.760819020475512	0\\
0.760919022975574	0\\
0.761019025475637	0\\
0.761119027975699	0\\
0.761219030475762	0\\
0.761319032975824	0\\
0.761419035475887	0\\
0.761519037975949	0\\
0.761619040476012	0\\
0.761719042976074	0\\
0.761819045476137	0\\
0.761919047976199	0\\
0.762019050476262	0\\
0.762119052976324	0\\
0.762219055476387	0\\
0.762319057976449	0\\
0.762419060476512	0\\
0.762519062976574	0\\
0.762619065476637	0\\
0.762719067976699	0\\
0.762819070476762	0\\
0.762919072976824	0\\
0.763019075476887	0\\
0.763119077976949	0\\
0.763219080477012	0\\
0.763319082977074	0\\
0.763419085477137	0\\
0.763519087977199	0\\
0.763619090477262	0\\
0.763719092977324	0\\
0.763819095477387	0\\
0.763919097977449	0\\
0.764019100477512	0\\
0.764119102977574	0\\
0.764219105477637	0\\
0.764319107977699	0\\
0.764419110477762	0\\
0.764519112977824	0\\
0.764619115477887	0\\
0.764719117977949	0\\
0.764819120478012	0\\
0.764919122978074	0\\
0.765019125478137	0\\
0.765119127978199	0\\
0.765219130478262	0\\
0.765319132978324	0\\
0.765419135478387	0\\
0.76551913797845	0\\
0.765619140478512	0\\
0.765719142978574	0\\
0.765819145478637	0\\
0.765919147978699	0\\
0.766019150478762	0\\
0.766119152978824	0\\
0.766219155478887	0\\
0.766319157978949	0\\
0.766419160479012	0\\
0.766519162979075	0\\
0.766619165479137	0\\
0.766719167979199	0\\
0.766819170479262	0\\
0.766919172979324	0\\
0.767019175479387	0\\
0.76711917797945	0\\
0.767219180479512	0\\
0.767319182979574	0\\
0.767419185479637	0\\
0.7675191879797	0\\
0.767619190479762	0\\
0.767719192979824	0\\
0.767819195479887	0\\
0.767919197979949	0\\
0.768019200480012	0\\
0.768119202980075	0\\
0.768219205480137	0\\
0.768319207980199	0\\
0.768419210480262	0\\
0.768519212980325	0\\
0.768619215480387	0\\
0.76871921798045	0\\
0.768819220480512	0\\
0.768919222980574	0\\
0.769019225480637	0\\
0.7691192279807	0\\
0.769219230480762	0\\
0.769319232980824	0\\
0.769419235480887	0\\
0.76951923798095	0\\
0.769619240481012	0\\
0.769719242981075	0\\
0.769819245481137	0\\
0.769919247981199	0\\
0.770019250481262	0\\
0.770119252981325	0\\
0.770219255481387	0\\
0.77031925798145	0\\
0.770419260481512	0\\
0.770519262981575	0\\
0.770619265481637	0\\
0.7707192679817	0\\
0.770819270481762	0\\
0.770919272981824	0\\
0.771019275481887	0\\
0.77111927798195	0\\
0.771219280482012	0\\
0.771319282982075	0\\
0.771419285482137	0\\
0.7715192879822	0\\
0.771619290482262	0\\
0.771719292982325	0\\
0.771819295482387	0\\
0.77191929798245	0\\
0.772019300482512	0\\
0.772119302982575	0\\
0.772219305482637	0\\
0.7723193079827	0\\
0.772419310482762	0\\
0.772519312982825	0\\
0.772619315482887	0\\
0.77271931798295	0\\
0.772819320483012	0\\
0.772919322983075	0\\
0.773019325483137	0\\
0.7731193279832	0\\
0.773219330483262	0\\
0.773319332983325	0\\
0.773419335483387	0\\
0.77351933798345	0\\
0.773619340483512	0\\
0.773719342983575	0\\
0.773819345483637	0\\
0.7739193479837	0\\
0.774019350483762	0\\
0.774119352983825	0\\
0.774219355483887	0\\
0.77431935798395	0\\
0.774419360484012	0\\
0.774519362984075	0\\
0.774619365484137	0\\
0.7747193679842	0\\
0.774819370484262	0\\
0.774919372984325	0\\
0.775019375484387	0\\
0.77511937798445	0\\
0.775219380484512	0\\
0.775319382984575	0\\
0.775419385484637	0\\
0.7755193879847	0\\
0.775619390484762	0\\
0.775719392984825	0\\
0.775819395484887	0\\
0.77591939798495	0\\
0.776019400485012	0\\
0.776119402985075	0\\
0.776219405485137	0\\
0.7763194079852	0\\
0.776419410485262	0\\
0.776519412985325	0\\
0.776619415485387	0\\
0.77671941798545	0\\
0.776819420485512	0\\
0.776919422985575	0\\
0.777019425485637	0\\
0.7771194279857	0\\
0.777219430485762	0\\
0.777319432985825	0\\
0.777419435485887	0\\
0.77751943798595	0\\
0.777619440486012	0\\
0.777719442986075	0\\
0.777819445486137	0\\
0.7779194479862	0\\
0.778019450486262	0\\
0.778119452986325	0\\
0.778219455486387	0\\
0.77831945798645	0\\
0.778419460486512	0\\
0.778519462986575	0\\
0.778619465486637	0\\
0.7787194679867	0\\
0.778819470486762	0\\
0.778919472986825	0\\
0.779019475486887	0\\
0.77911947798695	0\\
0.779219480487012	0\\
0.779319482987075	0\\
0.779419485487137	0\\
0.7795194879872	0\\
0.779619490487262	0\\
0.779719492987325	0\\
0.779819495487387	0\\
0.77991949798745	0\\
0.780019500487512	0\\
0.780119502987575	0\\
0.780219505487637	0\\
0.7803195079877	0\\
0.780419510487762	0\\
0.780519512987825	0\\
0.780619515487887	0\\
0.78071951798795	0\\
0.780819520488012	0\\
0.780919522988075	0\\
0.781019525488137	0\\
0.7811195279882	0\\
0.781219530488262	0\\
0.781319532988325	0\\
0.781419535488387	0\\
0.78151953798845	0\\
0.781619540488512	0\\
0.781719542988575	0\\
0.781819545488637	0\\
0.7819195479887	0\\
0.782019550488762	0\\
0.782119552988825	0\\
0.782219555488887	0\\
0.78231955798895	0\\
0.782419560489012	0\\
0.782519562989075	0\\
0.782619565489137	0\\
0.7827195679892	0\\
0.782819570489262	0\\
0.782919572989325	0\\
0.783019575489387	0\\
0.78311957798945	0\\
0.783219580489512	0\\
0.783319582989575	0\\
0.783419585489637	0\\
0.7835195879897	0\\
0.783619590489762	0\\
0.783719592989825	0\\
0.783819595489887	0\\
0.78391959798995	0\\
0.784019600490012	0\\
0.784119602990075	0\\
0.784219605490137	0\\
0.7843196079902	0\\
0.784419610490262	0\\
0.784519612990325	0\\
0.784619615490387	0\\
0.78471961799045	0\\
0.784819620490512	0\\
0.784919622990575	0\\
0.785019625490637	0\\
0.7851196279907	0\\
0.785219630490762	0\\
0.785319632990825	0\\
0.785419635490887	0\\
0.78551963799095	0\\
0.785619640491012	0\\
0.785719642991075	0\\
0.785819645491137	0\\
0.7859196479912	0\\
0.786019650491262	0\\
0.786119652991325	0\\
0.786219655491387	0\\
0.78631965799145	0\\
0.786419660491512	0\\
0.786519662991575	0\\
0.786619665491637	0\\
0.7867196679917	0\\
0.786819670491762	0\\
0.786919672991825	0\\
0.787019675491887	0\\
0.78711967799195	0\\
0.787219680492012	0\\
0.787319682992075	0\\
0.787419685492137	0\\
0.7875196879922	0\\
0.787619690492262	0\\
0.787719692992325	0\\
0.787819695492387	0\\
0.78791969799245	0\\
0.788019700492512	0\\
0.788119702992575	0\\
0.788219705492637	0\\
0.7883197079927	0\\
0.788419710492762	0\\
0.788519712992825	0\\
0.788619715492887	0\\
0.78871971799295	0\\
0.788819720493012	0\\
0.788919722993075	0\\
0.789019725493137	0\\
0.7891197279932	0\\
0.789219730493262	0\\
0.789319732993325	0\\
0.789419735493387	0\\
0.78951973799345	0\\
0.789619740493512	0\\
0.789719742993575	0\\
0.789819745493637	0\\
0.7899197479937	0\\
0.790019750493762	0\\
0.790119752993825	0\\
0.790219755493887	0\\
0.79031975799395	0\\
0.790419760494012	0\\
0.790519762994075	0\\
0.790619765494137	0\\
0.7907197679942	0\\
0.790819770494262	0\\
0.790919772994325	0\\
0.791019775494387	0\\
0.79111977799445	0\\
0.791219780494512	0\\
0.791319782994575	0\\
0.791419785494637	0\\
0.7915197879947	0\\
0.791619790494762	0\\
0.791719792994825	0\\
0.791819795494887	0\\
0.79191979799495	0\\
0.792019800495012	0\\
0.792119802995075	0\\
0.792219805495137	0\\
0.7923198079952	0\\
0.792419810495262	0\\
0.792519812995325	0\\
0.792619815495387	0\\
0.79271981799545	0\\
0.792819820495512	0\\
0.792919822995575	0\\
0.793019825495637	0\\
0.7931198279957	0\\
0.793219830495762	0\\
0.793319832995825	0\\
0.793419835495887	0\\
0.79351983799595	0\\
0.793619840496012	0\\
0.793719842996075	0\\
0.793819845496137	0\\
0.7939198479962	0\\
0.794019850496262	0\\
0.794119852996325	0\\
0.794219855496387	0\\
0.79431985799645	0\\
0.794419860496512	0\\
0.794519862996575	0\\
0.794619865496637	0\\
0.7947198679967	0\\
0.794819870496762	0\\
0.794919872996825	0\\
0.795019875496887	0\\
0.79511987799695	0\\
0.795219880497012	0\\
0.795319882997075	0\\
0.795419885497137	0\\
0.7955198879972	0\\
0.795619890497262	0\\
0.795719892997325	0\\
0.795819895497387	0\\
0.79591989799745	0\\
0.796019900497512	0\\
0.796119902997575	0\\
0.796219905497637	0\\
0.7963199079977	0\\
0.796419910497762	0\\
0.796519912997825	0\\
0.796619915497887	0\\
0.79671991799795	0\\
0.796819920498012	0\\
0.796919922998075	0\\
0.797019925498137	0\\
0.7971199279982	0\\
0.797219930498262	0\\
0.797319932998325	0\\
0.797419935498387	0\\
0.79751993799845	0\\
0.797619940498512	0\\
0.797719942998575	0\\
0.797819945498637	0\\
0.7979199479987	0\\
0.798019950498763	0\\
0.798119952998825	0\\
0.798219955498887	0\\
0.79831995799895	0\\
0.798419960499012	0\\
0.798519962999075	0\\
0.798619965499137	0\\
0.7987199679992	0\\
0.798819970499262	0\\
0.798919972999325	0\\
0.799019975499388	0\\
0.79911997799945	0\\
0.799219980499512	0\\
0.799319982999575	0\\
0.799419985499637	0\\
0.7995199879997	0\\
0.799619990499763	0\\
0.799719992999825	0\\
0.799819995499887	0\\
0.79991999799995	0\\
0.800020000500013	0\\
};
\addplot [color=mycolor2,solid,forget plot]
  table[row sep=crcr]{%
0.800020000500013	0\\
0.800120003000075	0\\
0.800220005500137	0\\
0.8003200080002	0\\
0.800420010500262	0\\
0.800520013000325	0\\
0.800620015500388	0\\
0.80072001800045	0\\
0.800820020500512	0\\
0.800920023000575	0\\
0.801020025500638	0\\
0.8011200280007	0\\
0.801220030500763	0\\
0.801320033000825	0\\
0.801420035500887	0\\
0.80152003800095	0\\
0.801620040501013	0\\
0.801720043001075	0\\
0.801820045501138	0\\
0.8019200480012	0\\
0.802020050501263	0\\
0.802120053001325	0\\
0.802220055501388	0\\
0.80232005800145	0\\
0.802420060501512	0\\
0.802520063001575	0\\
0.802620065501638	0\\
0.8027200680017	0\\
0.802820070501763	0\\
0.802920073001825	0\\
0.803020075501888	0\\
0.80312007800195	0\\
0.803220080502013	0\\
0.803320083002075	0\\
0.803420085502138	0\\
0.8035200880022	0\\
0.803620090502263	0\\
0.803720093002325	0\\
0.803820095502388	0\\
0.80392009800245	0\\
0.804020100502513	0\\
0.804120103002575	0\\
0.804220105502638	0\\
0.8043201080027	0\\
0.804420110502763	0\\
0.804520113002825	0\\
0.804620115502888	0\\
0.80472011800295	0\\
0.804820120503013	0\\
0.804920123003075	0\\
0.805020125503138	0\\
0.8051201280032	0\\
0.805220130503263	0\\
0.805320133003325	0\\
0.805420135503388	0\\
0.80552013800345	0\\
0.805620140503513	0\\
0.805720143003575	0\\
0.805820145503638	0\\
0.8059201480037	0\\
0.806020150503763	0\\
0.806120153003825	0\\
0.806220155503888	0\\
0.80632015800395	0\\
0.806420160504013	0\\
0.806520163004075	0\\
0.806620165504138	0\\
0.8067201680042	0\\
0.806820170504263	0\\
0.806920173004325	0\\
0.807020175504388	0\\
0.80712017800445	0\\
0.807220180504513	0\\
0.807320183004575	0\\
0.807420185504638	0\\
0.8075201880047	0\\
0.807620190504763	0\\
0.807720193004825	0\\
0.807820195504888	0\\
0.80792019800495	0\\
0.808020200505013	0\\
0.808120203005075	0\\
0.808220205505138	0\\
0.8083202080052	0\\
0.808420210505263	0\\
0.808520213005325	0\\
0.808620215505388	0\\
0.80872021800545	0\\
0.808820220505513	0\\
0.808920223005575	0\\
0.809020225505638	0\\
0.8091202280057	0\\
0.809220230505763	0\\
0.809320233005825	0\\
0.809420235505888	0\\
0.80952023800595	0\\
0.809620240506013	0\\
0.809720243006075	0\\
0.809820245506138	0\\
0.8099202480062	0\\
0.810020250506263	0\\
0.810120253006325	0\\
0.810220255506388	0\\
0.81032025800645	0\\
0.810420260506513	0\\
0.810520263006575	0\\
0.810620265506638	0\\
0.8107202680067	0\\
0.810820270506763	0\\
0.810920273006825	0\\
0.811020275506888	0\\
0.81112027800695	0\\
0.811220280507013	0\\
0.811320283007075	0\\
0.811420285507138	0\\
0.8115202880072	0\\
0.811620290507263	0\\
0.811720293007325	0\\
0.811820295507388	0\\
0.81192029800745	0\\
0.812020300507513	0\\
0.812120303007575	0\\
0.812220305507638	0\\
0.8123203080077	0\\
0.812420310507763	0\\
0.812520313007825	0\\
0.812620315507888	0\\
0.81272031800795	0\\
0.812820320508013	0\\
0.812920323008075	0\\
0.813020325508138	0\\
0.8131203280082	0\\
0.813220330508263	0\\
0.813320333008325	0\\
0.813420335508388	0\\
0.81352033800845	0\\
0.813620340508513	0\\
0.813720343008575	0\\
0.813820345508638	0\\
0.8139203480087	0\\
0.814020350508763	0\\
0.814120353008825	0\\
0.814220355508888	0\\
0.81432035800895	0\\
0.814420360509013	0\\
0.814520363009075	0\\
0.814620365509138	0\\
0.8147203680092	0\\
0.814820370509263	0\\
0.814920373009325	0\\
0.815020375509388	0\\
0.81512037800945	0\\
0.815220380509513	0\\
0.815320383009575	0\\
0.815420385509638	0\\
0.8155203880097	0\\
0.815620390509763	0\\
0.815720393009825	0\\
0.815820395509888	0\\
0.81592039800995	0\\
0.816020400510013	0\\
0.816120403010075	0\\
0.816220405510138	0\\
0.8163204080102	0\\
0.816420410510263	0\\
0.816520413010325	0\\
0.816620415510388	0\\
0.81672041801045	0\\
0.816820420510513	0\\
0.816920423010575	0\\
0.817020425510638	0\\
0.8171204280107	0\\
0.817220430510763	0\\
0.817320433010825	0\\
0.817420435510888	0\\
0.81752043801095	0\\
0.817620440511013	0\\
0.817720443011075	0\\
0.817820445511138	0\\
0.8179204480112	0\\
0.818020450511263	0\\
0.818120453011325	0\\
0.818220455511388	0\\
0.81832045801145	0\\
0.818420460511513	0\\
0.818520463011575	0\\
0.818620465511638	0\\
0.8187204680117	0\\
0.818820470511763	0\\
0.818920473011825	0\\
0.819020475511888	0\\
0.81912047801195	0\\
0.819220480512013	0\\
0.819320483012075	0\\
0.819420485512138	0\\
0.8195204880122	0\\
0.819620490512263	0\\
0.819720493012325	0\\
0.819820495512388	0\\
0.81992049801245	0\\
0.820020500512513	0\\
0.820120503012575	0\\
0.820220505512638	0\\
0.8203205080127	0\\
0.820420510512763	0\\
0.820520513012825	0\\
0.820620515512888	0\\
0.82072051801295	0\\
0.820820520513013	0\\
0.820920523013075	0\\
0.821020525513138	0\\
0.8211205280132	0\\
0.821220530513263	0\\
0.821320533013325	0\\
0.821420535513388	0\\
0.82152053801345	0\\
0.821620540513513	0\\
0.821720543013575	0\\
0.821820545513638	0\\
0.8219205480137	0\\
0.822020550513763	0\\
0.822120553013825	0\\
0.822220555513888	0\\
0.82232055801395	0\\
0.822420560514013	0\\
0.822520563014075	0\\
0.822620565514138	0\\
0.8227205680142	0\\
0.822820570514263	0\\
0.822920573014325	0\\
0.823020575514388	0\\
0.82312057801445	0\\
0.823220580514513	0\\
0.823320583014575	0\\
0.823420585514638	0\\
0.8235205880147	0\\
0.823620590514763	0\\
0.823720593014825	0\\
0.823820595514888	0\\
0.82392059801495	0\\
0.824020600515013	0\\
0.824120603015075	0\\
0.824220605515138	0\\
0.8243206080152	0\\
0.824420610515263	0\\
0.824520613015325	0\\
0.824620615515388	0\\
0.82472061801545	0\\
0.824820620515513	0\\
0.824920623015575	0\\
0.825020625515638	0\\
0.8251206280157	0\\
0.825220630515763	0\\
0.825320633015825	0\\
0.825420635515888	0\\
0.82552063801595	0\\
0.825620640516013	0\\
0.825720643016075	0\\
0.825820645516138	0\\
0.8259206480162	0\\
0.826020650516263	0\\
0.826120653016325	0\\
0.826220655516388	0\\
0.82632065801645	0\\
0.826420660516513	0\\
0.826520663016575	0\\
0.826620665516638	0\\
0.8267206680167	0\\
0.826820670516763	0\\
0.826920673016825	0\\
0.827020675516888	0\\
0.82712067801695	0\\
0.827220680517013	0\\
0.827320683017075	0\\
0.827420685517138	0\\
0.8275206880172	0\\
0.827620690517263	0\\
0.827720693017325	0\\
0.827820695517388	0\\
0.82792069801745	0\\
0.828020700517513	0\\
0.828120703017575	0\\
0.828220705517638	0\\
0.8283207080177	0\\
0.828420710517763	0\\
0.828520713017825	0\\
0.828620715517888	0\\
0.82872071801795	0\\
0.828820720518013	0\\
0.828920723018075	0\\
0.829020725518138	0\\
0.8291207280182	0\\
0.829220730518263	0\\
0.829320733018325	0\\
0.829420735518388	0\\
0.82952073801845	0\\
0.829620740518513	0\\
0.829720743018575	0\\
0.829820745518638	0\\
0.8299207480187	0\\
0.830020750518763	0\\
0.830120753018825	0\\
0.830220755518888	0\\
0.83032075801895	0\\
0.830420760519013	0\\
0.830520763019076	0\\
0.830620765519138	0\\
0.8307207680192	0\\
0.830820770519263	0\\
0.830920773019325	0\\
0.831020775519388	0\\
0.831120778019451	0\\
0.831220780519513	0\\
0.831320783019575	0\\
0.831420785519638	0\\
0.831520788019701	0\\
0.831620790519763	0\\
0.831720793019825	0\\
0.831820795519888	0\\
0.83192079801995	0\\
0.832020800520013	0\\
0.832120803020076	0\\
0.832220805520138	0\\
0.8323208080202	0\\
0.832420810520263	0\\
0.832520813020326	0\\
0.832620815520388	0\\
0.832720818020451	0\\
0.832820820520513	0\\
0.832920823020575	0\\
0.833020825520638	0\\
0.833120828020701	0\\
0.833220830520763	0\\
0.833320833020825	0\\
0.833420835520888	0\\
0.833520838020951	0\\
0.833620840521013	0\\
0.833720843021076	0\\
0.833820845521138	0\\
0.8339208480212	0\\
0.834020850521263	0\\
0.834120853021326	0\\
0.834220855521388	0\\
0.834320858021451	0\\
0.834420860521513	0\\
0.834520863021576	0\\
0.834620865521638	0\\
0.834720868021701	0\\
0.834820870521763	0\\
0.834920873021825	0\\
0.835020875521888	0\\
0.835120878021951	0\\
0.835220880522013	0\\
0.835320883022076	0\\
0.835420885522138	0\\
0.835520888022201	0\\
0.835620890522263	0\\
0.835720893022326	0\\
0.835820895522388	0\\
0.835920898022451	0\\
0.836020900522513	0\\
0.836120903022576	0\\
0.836220905522638	0\\
0.836320908022701	0\\
0.836420910522763	0\\
0.836520913022826	0\\
0.836620915522888	0\\
0.836720918022951	0\\
0.836820920523013	0\\
0.836920923023076	0\\
0.837020925523138	0\\
0.837120928023201	0\\
0.837220930523263	0\\
0.837320933023326	0\\
0.837420935523388	0\\
0.837520938023451	0\\
0.837620940523513	0\\
0.837720943023576	0\\
0.837820945523638	0\\
0.837920948023701	0\\
0.838020950523763	0\\
0.838120953023826	0\\
0.838220955523888	0\\
0.838320958023951	0\\
0.838420960524013	0\\
0.838520963024076	0\\
0.838620965524138	0\\
0.838720968024201	0\\
0.838820970524263	0\\
0.838920973024326	0\\
0.839020975524388	0\\
0.839120978024451	0\\
0.839220980524513	0\\
0.839320983024576	0\\
0.839420985524638	0\\
0.839520988024701	0\\
0.839620990524763	0\\
0.839720993024826	0\\
0.839820995524888	0\\
0.839920998024951	0\\
0.840021000525013	0\\
0.840121003025076	0\\
0.840221005525138	0\\
0.840321008025201	0\\
0.840421010525263	0\\
0.840521013025326	0\\
0.840621015525388	0\\
0.840721018025451	0\\
0.840821020525513	0\\
0.840921023025576	0\\
0.841021025525638	0\\
0.841121028025701	0\\
0.841221030525763	0\\
0.841321033025826	0\\
0.841421035525888	0\\
0.841521038025951	0\\
0.841621040526013	0\\
0.841721043026076	0\\
0.841821045526138	0\\
0.841921048026201	0\\
0.842021050526263	0\\
0.842121053026326	0\\
0.842221055526388	0\\
0.842321058026451	0\\
0.842421060526513	0\\
0.842521063026576	0\\
0.842621065526638	0\\
0.842721068026701	0\\
0.842821070526763	0\\
0.842921073026826	0\\
0.843021075526888	0\\
0.843121078026951	0\\
0.843221080527013	0\\
0.843321083027076	0\\
0.843421085527138	0\\
0.843521088027201	0\\
0.843621090527263	0\\
0.843721093027326	0\\
0.843821095527388	0\\
0.843921098027451	0\\
0.844021100527513	0\\
0.844121103027576	0\\
0.844221105527638	0\\
0.844321108027701	0\\
0.844421110527763	0\\
0.844521113027826	0\\
0.844621115527888	0\\
0.844721118027951	0\\
0.844821120528013	0\\
0.844921123028076	0\\
0.845021125528138	0\\
0.845121128028201	0\\
0.845221130528263	0\\
0.845321133028326	0\\
0.845421135528388	0\\
0.845521138028451	0\\
0.845621140528513	0\\
0.845721143028576	0\\
0.845821145528638	0\\
0.845921148028701	0\\
0.846021150528763	0\\
0.846121153028826	0\\
0.846221155528888	0\\
0.846321158028951	0\\
0.846421160529013	0\\
0.846521163029076	0\\
0.846621165529138	0\\
0.846721168029201	0\\
0.846821170529263	0\\
0.846921173029326	0\\
0.847021175529388	0\\
0.847121178029451	0\\
0.847221180529513	0\\
0.847321183029576	0\\
0.847421185529638	0\\
0.847521188029701	0\\
0.847621190529763	0\\
0.847721193029826	0\\
0.847821195529888	0\\
0.847921198029951	0\\
0.848021200530013	0\\
0.848121203030076	0\\
0.848221205530138	0\\
0.848321208030201	0\\
0.848421210530263	0\\
0.848521213030326	0\\
0.848621215530388	0\\
0.848721218030451	0\\
0.848821220530513	0\\
0.848921223030576	0\\
0.849021225530638	0\\
0.849121228030701	0\\
0.849221230530763	0\\
0.849321233030826	0\\
0.849421235530888	0\\
0.849521238030951	0\\
0.849621240531013	0\\
0.849721243031076	0\\
0.849821245531138	0\\
0.849921248031201	0\\
0.850021250531263	0\\
0.850121253031326	0\\
0.850221255531388	0\\
0.850321258031451	0\\
0.850421260531513	0\\
0.850521263031576	0\\
0.850621265531638	0\\
0.850721268031701	0\\
0.850821270531763	0\\
0.850921273031826	0\\
0.851021275531888	0\\
0.851121278031951	0\\
0.851221280532013	0\\
0.851321283032076	0\\
0.851421285532138	0\\
0.851521288032201	0\\
0.851621290532263	0\\
0.851721293032326	0\\
0.851821295532388	0\\
0.851921298032451	0\\
0.852021300532513	0\\
0.852121303032576	0\\
0.852221305532638	0\\
0.852321308032701	0\\
0.852421310532763	0\\
0.852521313032826	0\\
0.852621315532888	0\\
0.852721318032951	0\\
0.852821320533013	0\\
0.852921323033076	0\\
0.853021325533138	0\\
0.853121328033201	0\\
0.853221330533263	0\\
0.853321333033326	0\\
0.853421335533388	0\\
0.853521338033451	0\\
0.853621340533513	0\\
0.853721343033576	0\\
0.853821345533638	0\\
0.853921348033701	0\\
0.854021350533763	0\\
0.854121353033826	0\\
0.854221355533888	0\\
0.854321358033951	0\\
0.854421360534013	0\\
0.854521363034076	0\\
0.854621365534138	0\\
0.854721368034201	0\\
0.854821370534263	0\\
0.854921373034326	0\\
0.855021375534388	0\\
0.855121378034451	0\\
0.855221380534513	0\\
0.855321383034576	0\\
0.855421385534638	0\\
0.855521388034701	0\\
0.855621390534763	0\\
0.855721393034826	0\\
0.855821395534888	0\\
0.855921398034951	0\\
0.856021400535013	0\\
0.856121403035076	0\\
0.856221405535138	0\\
0.856321408035201	0\\
0.856421410535263	0\\
0.856521413035326	0\\
0.856621415535388	0\\
0.856721418035451	0\\
0.856821420535513	0\\
0.856921423035576	0\\
0.857021425535638	0\\
0.857121428035701	0\\
0.857221430535763	0\\
0.857321433035826	0\\
0.857421435535888	0\\
0.857521438035951	0\\
0.857621440536013	0\\
0.857721443036076	0\\
0.857821445536138	0\\
0.857921448036201	0\\
0.858021450536263	0\\
0.858121453036326	0\\
0.858221455536388	0\\
0.858321458036451	0\\
0.858421460536513	0\\
0.858521463036576	0\\
0.858621465536638	0\\
0.858721468036701	0\\
0.858821470536763	0\\
0.858921473036826	0\\
0.859021475536888	0\\
0.859121478036951	0\\
0.859221480537013	0\\
0.859321483037076	0\\
0.859421485537138	0\\
0.859521488037201	0\\
0.859621490537263	0\\
0.859721493037326	0\\
0.859821495537388	0\\
0.859921498037451	0\\
0.860021500537513	0\\
0.860121503037576	0\\
0.860221505537638	0\\
0.860321508037701	0\\
0.860421510537763	0\\
0.860521513037826	0\\
0.860621515537888	0\\
0.860721518037951	0\\
0.860821520538013	0\\
0.860921523038076	0\\
0.861021525538138	0\\
0.861121528038201	0\\
0.861221530538263	0\\
0.861321533038326	0\\
0.861421535538388	0\\
0.861521538038451	0\\
0.861621540538513	0\\
0.861721543038576	0\\
0.861821545538638	0\\
0.861921548038701	0\\
0.862021550538764	0\\
0.862121553038826	0\\
0.862221555538888	0\\
0.862321558038951	0\\
0.862421560539013	0\\
0.862521563039076	0\\
0.862621565539138	0\\
0.862721568039201	0\\
0.862821570539263	0\\
0.862921573039326	0\\
0.863021575539389	0\\
0.863121578039451	0\\
0.863221580539513	0\\
0.863321583039576	0\\
0.863421585539638	0\\
0.863521588039701	0\\
0.863621590539764	0\\
0.863721593039826	0\\
0.863821595539888	0\\
0.863921598039951	0\\
0.864021600540014	0\\
0.864121603040076	0\\
0.864221605540138	0\\
0.864321608040201	0\\
0.864421610540263	0\\
0.864521613040326	0\\
0.864621615540389	0\\
0.864721618040451	0\\
0.864821620540513	0\\
0.864921623040576	0\\
0.865021625540639	0\\
0.865121628040701	0\\
0.865221630540764	0\\
0.865321633040826	0\\
0.865421635540888	0\\
0.865521638040951	0\\
0.865621640541014	0\\
0.865721643041076	0\\
0.865821645541138	0\\
0.865921648041201	0\\
0.866021650541264	0\\
0.866121653041326	0\\
0.866221655541389	0\\
0.866321658041451	0\\
0.866421660541513	0\\
0.866521663041576	0\\
0.866621665541639	0\\
0.866721668041701	0\\
0.866821670541764	0\\
0.866921673041826	0\\
0.867021675541889	0\\
0.867121678041951	0\\
0.867221680542014	0\\
0.867321683042076	0\\
0.867421685542138	0\\
0.867521688042201	0\\
0.867621690542264	0\\
0.867721693042326	0\\
0.867821695542389	0\\
0.867921698042451	0\\
0.868021700542514	0\\
0.868121703042576	0\\
0.868221705542639	0\\
0.868321708042701	0\\
0.868421710542764	0\\
0.868521713042826	0\\
0.868621715542889	0\\
0.868721718042951	0\\
0.868821720543014	0\\
0.868921723043076	0\\
0.869021725543139	0\\
0.869121728043201	0\\
0.869221730543264	0\\
0.869321733043326	0\\
0.869421735543389	0\\
0.869521738043451	0\\
0.869621740543514	0\\
0.869721743043576	0\\
0.869821745543639	0\\
0.869921748043701	0\\
0.870021750543764	0\\
0.870121753043826	0\\
0.870221755543889	0\\
0.870321758043951	0\\
0.870421760544014	0\\
0.870521763044076	0\\
0.870621765544139	0\\
0.870721768044201	0\\
0.870821770544264	0\\
0.870921773044326	0\\
0.871021775544389	0\\
0.871121778044451	0\\
0.871221780544514	0\\
0.871321783044576	0\\
0.871421785544639	0\\
0.871521788044701	0\\
0.871621790544764	0\\
0.871721793044826	0\\
0.871821795544889	0\\
0.871921798044951	0\\
0.872021800545014	0\\
0.872121803045076	0\\
0.872221805545139	0\\
0.872321808045201	0\\
0.872421810545264	0\\
0.872521813045326	0\\
0.872621815545389	0\\
0.872721818045451	0\\
0.872821820545514	0\\
0.872921823045576	0\\
0.873021825545639	0\\
0.873121828045701	0\\
0.873221830545764	0\\
0.873321833045826	0\\
0.873421835545889	0\\
0.873521838045951	0\\
0.873621840546014	0\\
0.873721843046076	0\\
0.873821845546139	0\\
0.873921848046201	0\\
0.874021850546264	0\\
0.874121853046326	0\\
0.874221855546389	0\\
0.874321858046451	0\\
0.874421860546514	0\\
0.874521863046576	0\\
0.874621865546639	0\\
0.874721868046701	0\\
0.874821870546764	0\\
0.874921873046826	0\\
0.875021875546889	0\\
0.875121878046951	0\\
0.875221880547014	0\\
0.875321883047076	0\\
0.875421885547139	0\\
0.875521888047201	0\\
0.875621890547264	0\\
0.875721893047326	0\\
0.875821895547389	0\\
0.875921898047451	0\\
0.876021900547514	0\\
0.876121903047576	0\\
0.876221905547639	0\\
0.876321908047701	0\\
0.876421910547764	0\\
0.876521913047826	0\\
0.876621915547889	0\\
0.876721918047951	0\\
0.876821920548014	0\\
0.876921923048076	0\\
0.877021925548139	0\\
0.877121928048201	0\\
0.877221930548264	0\\
0.877321933048326	0\\
0.877421935548389	0\\
0.877521938048451	0\\
0.877621940548514	0\\
0.877721943048576	0\\
0.877821945548639	0\\
0.877921948048701	0\\
0.878021950548764	0\\
0.878121953048826	0\\
0.878221955548889	0\\
0.878321958048951	0\\
0.878421960549014	0\\
0.878521963049076	0\\
0.878621965549139	0\\
0.878721968049201	0\\
0.878821970549264	0\\
0.878921973049326	0\\
0.879021975549389	0\\
0.879121978049451	0\\
0.879221980549514	0\\
0.879321983049576	0\\
0.879421985549639	0\\
0.879521988049701	0\\
0.879621990549764	0\\
0.879721993049826	0\\
0.879821995549889	0\\
0.879921998049951	0\\
0.880022000550014	0\\
0.880122003050076	0\\
0.880222005550139	0\\
0.880322008050201	0\\
0.880422010550264	0\\
0.880522013050326	0\\
0.880622015550389	0\\
0.880722018050451	0\\
0.880822020550514	0\\
0.880922023050576	0\\
0.881022025550639	0\\
0.881122028050701	0\\
0.881222030550764	0\\
0.881322033050826	0\\
0.881422035550889	0\\
0.881522038050951	0\\
0.881622040551014	0\\
0.881722043051076	0\\
0.881822045551139	0\\
0.881922048051201	0\\
0.882022050551264	0\\
0.882122053051326	0\\
0.882222055551389	0\\
0.882322058051451	0\\
0.882422060551514	0\\
0.882522063051576	0\\
0.882622065551639	0\\
0.882722068051701	0\\
0.882822070551764	0\\
0.882922073051826	0\\
0.883022075551889	0\\
0.883122078051951	0\\
0.883222080552014	0\\
0.883322083052076	0\\
0.883422085552139	0\\
0.883522088052201	0\\
0.883622090552264	0\\
0.883722093052326	0\\
0.883822095552389	0\\
0.883922098052451	0\\
0.884022100552514	0\\
0.884122103052576	0\\
0.884222105552639	0\\
0.884322108052701	0\\
0.884422110552764	0\\
0.884522113052826	0\\
0.884622115552889	0\\
0.884722118052951	0\\
0.884822120553014	0\\
0.884922123053076	0\\
0.885022125553139	0\\
0.885122128053201	0\\
0.885222130553264	0\\
0.885322133053326	0\\
0.885422135553389	0\\
0.885522138053451	0\\
0.885622140553514	0\\
0.885722143053576	0\\
0.885822145553639	0\\
0.885922148053701	0\\
0.886022150553764	0\\
0.886122153053826	0\\
0.886222155553889	0\\
0.886322158053951	0\\
0.886422160554014	0\\
0.886522163054076	0\\
0.886622165554139	0\\
0.886722168054201	0\\
0.886822170554264	0\\
0.886922173054326	0\\
0.887022175554389	0\\
0.887122178054451	0\\
0.887222180554514	0\\
0.887322183054576	0\\
0.887422185554639	0\\
0.887522188054701	0\\
0.887622190554764	0\\
0.887722193054826	0\\
0.887822195554889	0\\
0.887922198054951	0\\
0.888022200555014	0\\
0.888122203055076	0\\
0.888222205555139	0\\
0.888322208055201	0\\
0.888422210555264	0\\
0.888522213055326	0\\
0.888622215555389	0\\
0.888722218055451	0\\
0.888822220555514	0\\
0.888922223055576	0\\
0.889022225555639	0\\
0.889122228055701	0\\
0.889222230555764	0\\
0.889322233055826	0\\
0.889422235555889	0\\
0.889522238055951	0\\
0.889622240556014	0\\
0.889722243056076	0\\
0.889822245556139	0\\
0.889922248056201	0\\
0.890022250556264	0\\
0.890122253056326	0\\
0.890222255556389	0\\
0.890322258056451	0\\
0.890422260556514	0\\
0.890522263056576	0\\
0.890622265556639	0\\
0.890722268056701	0\\
0.890822270556764	0\\
0.890922273056826	0\\
0.891022275556889	0\\
0.891122278056951	0\\
0.891222280557014	0\\
0.891322283057076	0\\
0.891422285557139	0\\
0.891522288057201	0\\
0.891622290557264	0\\
0.891722293057326	0\\
0.891822295557389	0\\
0.891922298057451	0\\
0.892022300557514	0\\
0.892122303057576	0\\
0.892222305557639	0\\
0.892322308057701	0\\
0.892422310557764	0\\
0.892522313057826	0\\
0.892622315557889	0\\
0.892722318057951	0\\
0.892822320558014	0\\
0.892922323058077	0\\
0.893022325558139	0\\
0.893122328058201	0\\
0.893222330558264	0\\
0.893322333058326	0\\
0.893422335558389	0\\
0.893522338058451	0\\
0.893622340558514	0\\
0.893722343058576	0\\
0.893822345558639	0\\
0.893922348058702	0\\
0.894022350558764	0\\
0.894122353058826	0\\
0.894222355558889	0\\
0.894322358058951	0\\
0.894422360559014	0\\
0.894522363059077	0\\
0.894622365559139	0\\
0.894722368059201	0\\
0.894822370559264	0\\
0.894922373059327	0\\
0.895022375559389	0\\
0.895122378059451	0\\
0.895222380559514	0\\
0.895322383059576	0\\
0.895422385559639	0\\
0.895522388059702	0\\
0.895622390559764	0\\
0.895722393059826	0\\
0.895822395559889	0\\
0.895922398059952	0\\
0.896022400560014	0\\
0.896122403060077	0\\
0.896222405560139	0\\
0.896322408060201	0\\
0.896422410560264	0\\
0.896522413060327	0\\
0.896622415560389	0\\
0.896722418060451	0\\
0.896822420560514	0\\
0.896922423060577	0\\
0.897022425560639	0\\
0.897122428060702	0\\
0.897222430560764	0\\
0.897322433060826	0\\
0.897422435560889	0\\
0.897522438060952	0\\
0.897622440561014	0\\
0.897722443061077	0\\
0.897822445561139	0\\
0.897922448061202	0\\
0.898022450561264	0\\
0.898122453061327	0\\
0.898222455561389	0\\
0.898322458061452	0\\
0.898422460561514	0\\
0.898522463061577	0\\
0.898622465561639	0\\
0.898722468061702	0\\
0.898822470561764	0\\
0.898922473061827	0\\
0.899022475561889	0\\
0.899122478061952	0\\
0.899222480562014	0\\
0.899322483062077	0\\
0.899422485562139	0\\
0.899522488062202	0\\
0.899622490562264	0\\
0.899722493062327	0\\
0.899822495562389	0\\
0.899922498062452	0\\
0.900022500562514	0\\
0.900122503062577	0\\
0.900222505562639	0\\
0.900322508062702	0\\
0.900422510562764	0\\
0.900522513062827	0\\
0.900622515562889	0\\
0.900722518062952	0\\
0.900822520563014	0\\
0.900922523063077	0\\
0.901022525563139	0\\
0.901122528063202	0\\
0.901222530563264	0\\
0.901322533063327	0\\
0.901422535563389	0\\
0.901522538063452	0\\
0.901622540563514	0\\
0.901722543063577	0\\
0.901822545563639	0\\
0.901922548063702	0\\
0.902022550563764	0\\
0.902122553063827	0\\
0.902222555563889	0\\
0.902322558063952	0\\
0.902422560564014	0\\
0.902522563064077	0\\
0.902622565564139	0\\
0.902722568064202	0\\
0.902822570564264	0\\
0.902922573064327	0\\
0.903022575564389	0\\
0.903122578064452	0\\
0.903222580564514	0\\
0.903322583064577	0\\
0.903422585564639	0\\
0.903522588064702	0\\
0.903622590564764	0\\
0.903722593064827	0\\
0.903822595564889	0\\
0.903922598064952	0\\
0.904022600565014	0\\
0.904122603065077	0\\
0.904222605565139	0\\
0.904322608065202	0\\
0.904422610565264	0\\
0.904522613065327	0\\
0.904622615565389	0\\
0.904722618065452	0\\
0.904822620565514	0\\
0.904922623065577	0\\
0.905022625565639	0\\
0.905122628065702	0\\
0.905222630565764	0\\
0.905322633065827	0\\
0.905422635565889	0\\
0.905522638065952	0\\
0.905622640566014	0\\
0.905722643066077	0\\
0.905822645566139	0\\
0.905922648066202	0\\
0.906022650566264	0\\
0.906122653066327	0\\
0.906222655566389	0\\
0.906322658066452	0\\
0.906422660566514	0\\
0.906522663066577	0\\
0.906622665566639	0\\
0.906722668066702	0\\
0.906822670566764	0\\
0.906922673066827	0\\
0.907022675566889	0\\
0.907122678066952	0\\
0.907222680567014	0\\
0.907322683067077	0\\
0.907422685567139	0\\
0.907522688067202	0\\
0.907622690567264	0\\
0.907722693067327	0\\
0.907822695567389	0\\
0.907922698067452	0\\
0.908022700567514	0\\
0.908122703067577	0\\
0.908222705567639	0\\
0.908322708067702	0\\
0.908422710567764	0\\
0.908522713067827	0\\
0.908622715567889	0\\
0.908722718067952	0\\
0.908822720568014	0\\
0.908922723068077	0\\
0.909022725568139	0\\
0.909122728068202	0\\
0.909222730568264	0\\
0.909322733068327	0\\
0.909422735568389	0\\
0.909522738068452	0\\
0.909622740568514	0\\
0.909722743068577	0\\
0.909822745568639	0\\
0.909922748068702	0\\
0.910022750568764	0\\
0.910122753068827	0\\
0.910222755568889	0\\
0.910322758068952	0\\
0.910422760569014	0\\
0.910522763069077	0\\
0.910622765569139	0\\
0.910722768069202	0\\
0.910822770569264	0\\
0.910922773069327	0\\
0.911022775569389	0\\
0.911122778069452	0\\
0.911222780569514	0\\
0.911322783069577	0\\
0.911422785569639	0\\
0.911522788069702	0\\
0.911622790569764	0\\
0.911722793069827	0\\
0.911822795569889	0\\
0.911922798069952	0\\
0.912022800570014	0\\
0.912122803070077	0\\
0.912222805570139	0\\
0.912322808070202	0\\
0.912422810570264	0\\
0.912522813070327	0\\
0.912622815570389	0\\
0.912722818070452	0\\
0.912822820570514	0\\
0.912922823070577	0\\
0.913022825570639	0\\
0.913122828070702	0\\
0.913222830570764	0\\
0.913322833070827	0\\
0.913422835570889	0\\
0.913522838070952	0\\
0.913622840571014	0\\
0.913722843071077	0\\
0.913822845571139	0\\
0.913922848071202	0\\
0.914022850571264	0\\
0.914122853071327	0\\
0.914222855571389	0\\
0.914322858071452	0\\
0.914422860571514	0\\
0.914522863071577	0\\
0.914622865571639	0\\
0.914722868071702	0\\
0.914822870571764	0\\
0.914922873071827	0\\
0.915022875571889	0\\
0.915122878071952	0\\
0.915222880572014	0\\
0.915322883072077	0\\
0.915422885572139	0\\
0.915522888072202	0\\
0.915622890572264	0\\
0.915722893072327	0\\
0.915822895572389	0\\
0.915922898072452	0\\
0.916022900572514	0\\
0.916122903072577	0\\
0.916222905572639	0\\
0.916322908072702	0\\
0.916422910572764	0\\
0.916522913072827	0\\
0.916622915572889	0\\
0.916722918072952	0\\
0.916822920573014	0\\
0.916922923073077	0\\
0.917022925573139	0\\
0.917122928073202	0\\
0.917222930573264	0\\
0.917322933073327	0\\
0.917422935573389	0\\
0.917522938073452	0\\
0.917622940573514	0\\
0.917722943073577	0\\
0.917822945573639	0\\
0.917922948073702	0\\
0.918022950573764	0\\
0.918122953073827	0\\
0.918222955573889	0\\
0.918322958073952	0\\
0.918422960574014	0\\
0.918522963074077	0\\
0.918622965574139	0\\
0.918722968074202	0\\
0.918822970574264	0\\
0.918922973074327	0\\
0.919022975574389	0\\
0.919122978074452	0\\
0.919222980574514	0\\
0.919322983074577	0\\
0.919422985574639	0\\
0.919522988074702	0\\
0.919622990574764	0\\
0.919722993074827	0\\
0.919822995574889	0\\
0.919922998074952	0\\
0.920023000575014	0\\
0.920123003075077	0\\
0.920223005575139	0\\
0.920323008075202	0\\
0.920423010575264	0\\
0.920523013075327	0\\
0.920623015575389	0\\
0.920723018075452	0\\
0.920823020575514	0\\
0.920923023075577	0\\
0.921023025575639	0\\
0.921123028075702	0\\
0.921223030575764	0\\
0.921323033075827	0\\
0.921423035575889	0\\
0.921523038075952	0\\
0.921623040576014	0\\
0.921723043076077	0\\
0.921823045576139	0\\
0.921923048076202	0\\
0.922023050576264	0\\
0.922123053076327	0\\
0.922223055576389	0\\
0.922323058076452	0\\
0.922423060576514	0\\
0.922523063076577	0\\
0.922623065576639	0\\
0.922723068076702	0\\
0.922823070576764	0\\
0.922923073076827	0\\
0.923023075576889	0\\
0.923123078076952	0\\
0.923223080577014	0\\
0.923323083077077	0\\
0.923423085577139	0\\
0.923523088077202	0\\
0.923623090577264	0\\
0.923723093077327	0\\
0.923823095577389	0\\
0.923923098077452	0\\
0.924023100577514	0\\
0.924123103077577	0\\
0.924223105577639	0\\
0.924323108077702	0\\
0.924423110577764	0\\
0.924523113077827	0\\
0.924623115577889	0\\
0.924723118077952	0\\
0.924823120578014	0\\
0.924923123078077	0\\
0.925023125578139	0\\
0.925123128078202	0\\
0.925223130578264	0\\
0.925323133078327	0\\
0.92542313557839	0\\
0.925523138078452	0\\
0.925623140578514	0\\
0.925723143078577	0\\
0.925823145578639	0\\
0.925923148078702	0\\
0.926023150578764	0\\
0.926123153078827	0\\
0.926223155578889	0\\
0.926323158078952	0\\
0.926423160579015	0\\
0.926523163079077	0\\
0.926623165579139	0\\
0.926723168079202	0\\
0.926823170579264	0\\
0.926923173079327	0\\
0.92702317557939	0\\
0.927123178079452	0\\
0.927223180579514	0\\
0.927323183079577	0\\
0.92742318557964	0\\
0.927523188079702	0\\
0.927623190579765	0\\
0.927723193079827	0\\
0.927823195579889	0\\
0.927923198079952	0\\
0.928023200580015	0\\
0.928123203080077	0\\
0.928223205580139	0\\
0.928323208080202	0\\
0.928423210580265	0\\
0.928523213080327	0\\
0.92862321558039	0\\
0.928723218080452	0\\
0.928823220580514	0\\
0.928923223080577	0\\
0.92902322558064	0\\
0.929123228080702	0\\
0.929223230580765	0\\
0.929323233080827	0\\
0.92942323558089	0\\
0.929523238080952	0\\
0.929623240581015	0\\
0.929723243081077	0\\
0.929823245581139	0\\
0.929923248081202	0\\
0.930023250581265	0\\
0.930123253081327	0\\
0.93022325558139	0\\
0.930323258081452	0\\
0.930423260581515	0\\
0.930523263081577	0\\
0.93062326558164	0\\
0.930723268081702	0\\
0.930823270581765	0\\
0.930923273081827	0\\
0.93102327558189	0\\
0.931123278081952	0\\
0.931223280582015	0\\
0.931323283082077	0\\
0.93142328558214	0\\
0.931523288082202	0\\
0.931623290582265	0\\
0.931723293082327	0\\
0.93182329558239	0\\
0.931923298082452	0\\
0.932023300582515	0\\
0.932123303082577	0\\
0.93222330558264	0\\
0.932323308082702	0\\
0.932423310582765	0\\
0.932523313082827	0\\
0.93262331558289	0\\
0.932723318082952	0\\
0.932823320583015	0\\
0.932923323083077	0\\
0.93302332558314	0\\
0.933123328083202	0\\
0.933223330583265	0\\
0.933323333083327	0\\
0.93342333558339	0\\
0.933523338083452	0\\
0.933623340583515	0\\
0.933723343083577	0\\
0.93382334558364	0\\
0.933923348083702	0\\
0.934023350583765	0\\
0.934123353083827	0\\
0.93422335558389	0\\
0.934323358083952	0\\
0.934423360584015	0\\
0.934523363084077	0\\
0.93462336558414	0\\
0.934723368084202	0\\
0.934823370584265	0\\
0.934923373084327	0\\
0.93502337558439	0\\
0.935123378084452	0\\
0.935223380584515	0\\
0.935323383084577	0\\
0.93542338558464	0\\
0.935523388084702	0\\
0.935623390584765	0\\
0.935723393084827	0\\
0.93582339558489	0\\
0.935923398084952	0\\
0.936023400585015	0\\
0.936123403085077	0\\
0.93622340558514	0\\
0.936323408085202	0\\
0.936423410585265	0\\
0.936523413085327	0\\
0.93662341558539	0\\
0.936723418085452	0\\
0.936823420585515	0\\
0.936923423085577	0\\
0.93702342558564	0\\
0.937123428085702	0\\
0.937223430585765	0\\
0.937323433085827	0\\
0.93742343558589	0\\
0.937523438085952	0\\
0.937623440586015	0\\
0.937723443086077	0\\
0.93782344558614	0\\
0.937923448086202	0\\
0.938023450586265	0\\
0.938123453086327	0\\
0.93822345558639	0\\
0.938323458086452	0\\
0.938423460586515	0\\
0.938523463086577	0\\
0.93862346558664	0\\
0.938723468086702	0\\
0.938823470586765	0\\
0.938923473086827	0\\
0.93902347558689	0\\
0.939123478086952	0\\
0.939223480587015	0\\
0.939323483087077	0\\
0.93942348558714	0\\
0.939523488087202	0\\
0.939623490587265	0\\
0.939723493087327	0\\
0.93982349558739	0\\
0.939923498087452	0\\
0.940023500587515	0\\
0.940123503087577	0\\
0.94022350558764	0\\
0.940323508087702	0\\
0.940423510587765	0\\
0.940523513087827	0\\
0.94062351558789	0\\
0.940723518087952	0\\
0.940823520588015	0\\
0.940923523088077	0\\
0.94102352558814	0\\
0.941123528088202	0\\
0.941223530588265	0\\
0.941323533088327	0\\
0.94142353558839	0\\
0.941523538088452	0\\
0.941623540588515	0\\
0.941723543088577	0\\
0.94182354558864	0\\
0.941923548088702	0\\
0.942023550588765	0\\
0.942123553088827	0\\
0.94222355558889	0\\
0.942323558088952	0\\
0.942423560589015	0\\
0.942523563089077	0\\
0.94262356558914	0\\
0.942723568089202	0\\
0.942823570589265	0\\
0.942923573089327	0\\
0.94302357558939	0\\
0.943123578089452	0\\
0.943223580589515	0\\
0.943323583089577	0\\
0.94342358558964	0\\
0.943523588089702	0\\
0.943623590589765	0\\
0.943723593089827	0\\
0.94382359558989	0\\
0.943923598089952	0\\
0.944023600590015	0\\
0.944123603090077	0\\
0.94422360559014	0\\
0.944323608090202	0\\
0.944423610590265	0\\
0.944523613090327	0\\
0.94462361559039	0\\
0.944723618090452	0\\
0.944823620590515	0\\
0.944923623090577	0\\
0.94502362559064	0\\
0.945123628090702	0\\
0.945223630590765	0\\
0.945323633090827	0\\
0.94542363559089	0\\
0.945523638090952	0\\
0.945623640591015	0\\
0.945723643091077	0\\
0.94582364559114	0\\
0.945923648091202	0\\
0.946023650591265	0\\
0.946123653091327	0\\
0.94622365559139	0\\
0.946323658091452	0\\
0.946423660591515	0\\
0.946523663091577	0\\
0.94662366559164	0\\
0.946723668091702	0\\
0.946823670591765	0\\
0.946923673091827	0\\
0.94702367559189	0\\
0.947123678091952	0\\
0.947223680592015	0\\
0.947323683092077	0\\
0.94742368559214	0\\
0.947523688092202	0\\
0.947623690592265	0\\
0.947723693092327	0\\
0.94782369559239	0\\
0.947923698092452	0\\
0.948023700592515	0\\
0.948123703092577	0\\
0.94822370559264	0\\
0.948323708092702	0\\
0.948423710592765	0\\
0.948523713092827	0\\
0.94862371559289	0\\
0.948723718092952	0\\
0.948823720593015	0\\
0.948923723093077	0\\
0.94902372559314	0\\
0.949123728093202	0\\
0.949223730593265	0\\
0.949323733093327	0\\
0.94942373559339	0\\
0.949523738093452	0\\
0.949623740593515	0\\
0.949723743093577	0\\
0.94982374559364	0\\
0.949923748093702	0\\
0.950023750593765	0\\
0.950123753093827	0\\
0.95022375559389	0\\
0.950323758093952	0\\
0.950423760594015	0\\
0.950523763094077	0\\
0.95062376559414	0\\
0.950723768094202	0\\
0.950823770594265	0\\
0.950923773094327	0\\
0.95102377559439	0\\
0.951123778094452	0\\
0.951223780594515	0\\
0.951323783094577	0\\
0.95142378559464	0\\
0.951523788094702	0\\
0.951623790594765	0\\
0.951723793094827	0\\
0.95182379559489	0\\
0.951923798094952	0\\
0.952023800595015	0\\
0.952123803095077	0\\
0.95222380559514	0\\
0.952323808095202	0\\
0.952423810595265	0\\
0.952523813095327	0\\
0.95262381559539	0\\
0.952723818095452	0\\
0.952823820595515	0\\
0.952923823095577	0\\
0.95302382559564	0\\
0.953123828095702	0\\
0.953223830595765	0\\
0.953323833095827	0\\
0.95342383559589	0\\
0.953523838095952	0\\
0.953623840596015	0\\
0.953723843096077	0\\
0.95382384559614	0\\
0.953923848096202	0\\
0.954023850596265	0\\
0.954123853096327	0\\
0.95422385559639	0\\
0.954323858096452	0\\
0.954423860596515	0\\
0.954523863096577	0\\
0.95462386559664	0\\
0.954723868096702	0\\
0.954823870596765	0\\
0.954923873096827	0\\
0.95502387559689	0\\
0.955123878096952	0\\
0.955223880597015	0\\
0.955323883097077	0\\
0.95542388559714	0\\
0.955523888097202	0\\
0.955623890597265	0\\
0.955723893097327	0\\
0.95582389559739	0\\
0.955923898097452	0\\
0.956023900597515	0\\
0.956123903097577	0\\
0.95622390559764	0\\
0.956323908097702	0\\
0.956423910597765	0\\
0.956523913097827	0\\
0.95662391559789	0\\
0.956723918097952	0\\
0.956823920598015	0\\
0.956923923098078	0\\
0.95702392559814	0\\
0.957123928098202	0\\
0.957223930598265	0\\
0.957323933098327	0\\
0.95742393559839	0\\
0.957523938098452	0\\
0.957623940598515	0\\
0.957723943098577	0\\
0.95782394559864	0\\
0.957923948098703	0\\
0.958023950598765	0\\
0.958123953098827	0\\
0.95822395559889	0\\
0.958323958098952	0\\
0.958423960599015	0\\
0.958523963099078	0\\
0.95862396559914	0\\
0.958723968099202	0\\
0.958823970599265	0\\
0.958923973099328	0\\
0.95902397559939	0\\
0.959123978099452	0\\
0.959223980599515	0\\
0.959323983099577	0\\
0.95942398559964	0\\
0.959523988099703	0\\
0.959623990599765	0\\
0.959723993099827	0\\
0.95982399559989	0\\
0.959923998099953	0\\
0.960024000600015	0\\
0.960124003100078	0\\
0.96022400560014	0\\
0.960324008100202	0\\
0.960424010600265	0\\
0.960524013100328	0\\
0.96062401560039	0\\
0.960724018100452	0\\
0.960824020600515	0\\
0.960924023100578	0\\
0.96102402560064	0\\
0.961124028100703	0\\
0.961224030600765	0\\
0.961324033100827	0\\
0.96142403560089	0\\
0.961524038100953	0\\
0.961624040601015	0\\
0.961724043101078	0\\
0.96182404560114	0\\
0.961924048101203	0\\
0.962024050601265	0\\
0.962124053101328	0\\
0.96222405560139	0\\
0.962324058101452	0\\
0.962424060601515	0\\
0.962524063101578	0\\
0.96262406560164	0\\
0.962724068101703	0\\
0.962824070601765	0\\
0.962924073101828	0\\
0.96302407560189	0\\
0.963124078101953	0\\
0.963224080602015	0\\
0.963324083102078	0\\
0.96342408560214	0\\
0.963524088102203	0\\
0.963624090602265	0\\
0.963724093102328	0\\
0.96382409560239	0\\
0.963924098102453	0\\
0.964024100602515	0\\
0.964124103102578	0\\
0.96422410560264	0\\
0.964324108102703	0\\
0.964424110602765	0\\
0.964524113102828	0\\
0.96462411560289	0\\
0.964724118102953	0\\
0.964824120603015	0\\
0.964924123103078	0\\
0.96502412560314	0\\
0.965124128103203	0\\
0.965224130603265	0\\
0.965324133103328	0\\
0.96542413560339	0\\
0.965524138103453	0\\
0.965624140603515	0\\
0.965724143103578	0\\
0.96582414560364	0\\
0.965924148103703	0\\
0.966024150603765	0\\
0.966124153103828	0\\
0.96622415560389	0\\
0.966324158103953	0\\
0.966424160604015	0\\
0.966524163104078	0\\
0.96662416560414	0\\
0.966724168104203	0\\
0.966824170604265	0\\
0.966924173104328	0\\
0.96702417560439	0\\
0.967124178104453	0\\
0.967224180604515	0\\
0.967324183104578	0\\
0.96742418560464	0\\
0.967524188104703	0\\
0.967624190604765	0\\
0.967724193104828	0\\
0.96782419560489	0\\
0.967924198104953	0\\
0.968024200605015	0\\
0.968124203105078	0\\
0.96822420560514	0\\
0.968324208105203	0\\
0.968424210605265	0\\
0.968524213105328	0\\
0.96862421560539	0\\
0.968724218105453	0\\
0.968824220605515	0\\
0.968924223105578	0\\
0.96902422560564	0\\
0.969124228105703	0\\
0.969224230605765	0\\
0.969324233105828	0\\
0.96942423560589	0\\
0.969524238105953	0\\
0.969624240606015	0\\
0.969724243106078	0\\
0.96982424560614	0\\
0.969924248106203	0\\
0.970024250606265	0\\
0.970124253106328	0\\
0.97022425560639	0\\
0.970324258106453	0\\
0.970424260606515	0\\
0.970524263106578	0\\
0.97062426560664	0\\
0.970724268106703	0\\
0.970824270606765	0\\
0.970924273106828	0\\
0.97102427560689	0\\
0.971124278106953	0\\
0.971224280607015	0\\
0.971324283107078	0\\
0.97142428560714	0\\
0.971524288107203	0\\
0.971624290607265	0\\
0.971724293107328	0\\
0.97182429560739	0\\
0.971924298107453	0\\
0.972024300607515	0\\
0.972124303107578	0\\
0.97222430560764	0\\
0.972324308107703	0\\
0.972424310607765	0\\
0.972524313107828	0\\
0.97262431560789	0\\
0.972724318107953	0\\
0.972824320608015	0\\
0.972924323108078	0\\
0.97302432560814	0\\
0.973124328108203	0\\
0.973224330608265	0\\
0.973324333108328	0\\
0.97342433560839	0\\
0.973524338108453	0\\
0.973624340608515	0\\
0.973724343108578	0\\
0.97382434560864	0\\
0.973924348108703	0\\
0.974024350608765	0\\
0.974124353108828	0\\
0.97422435560889	0\\
0.974324358108953	0\\
0.974424360609015	0\\
0.974524363109078	0\\
0.97462436560914	0\\
0.974724368109203	0\\
0.974824370609265	0\\
0.974924373109328	0\\
0.97502437560939	0\\
0.975124378109453	0\\
0.975224380609515	0\\
0.975324383109578	0\\
0.97542438560964	0\\
0.975524388109703	0\\
0.975624390609765	0\\
0.975724393109828	0\\
0.97582439560989	0\\
0.975924398109953	0\\
0.976024400610015	0\\
0.976124403110078	0\\
0.97622440561014	0\\
0.976324408110203	0\\
0.976424410610265	0\\
0.976524413110328	0\\
0.97662441561039	0\\
0.976724418110453	0\\
0.976824420610515	0\\
0.976924423110578	0\\
0.97702442561064	0\\
0.977124428110703	0\\
0.977224430610765	0\\
0.977324433110828	0\\
0.97742443561089	0\\
0.977524438110953	0\\
0.977624440611015	0\\
0.977724443111078	0\\
0.97782444561114	0\\
0.977924448111203	0\\
0.978024450611265	0\\
0.978124453111328	0\\
0.97822445561139	0\\
0.978324458111453	0\\
0.978424460611515	0\\
0.978524463111578	0\\
0.97862446561164	0\\
0.978724468111703	0\\
0.978824470611765	0\\
0.978924473111828	0\\
0.97902447561189	0\\
0.979124478111953	0\\
0.979224480612015	0\\
0.979324483112078	0\\
0.97942448561214	0\\
0.979524488112203	0\\
0.979624490612265	0\\
0.979724493112328	0\\
0.97982449561239	0\\
0.979924498112453	0\\
0.980024500612515	0\\
0.980124503112578	0\\
0.98022450561264	0\\
0.980324508112703	0\\
0.980424510612765	0\\
0.980524513112828	0\\
0.98062451561289	0\\
0.980724518112953	0\\
0.980824520613015	0\\
0.980924523113078	0\\
0.98102452561314	0\\
0.981124528113203	0\\
0.981224530613265	0\\
0.981324533113328	0\\
0.98142453561339	0\\
0.981524538113453	0\\
0.981624540613515	0\\
0.981724543113578	0\\
0.98182454561364	0\\
0.981924548113703	0\\
0.982024550613765	0\\
0.982124553113828	0\\
0.98222455561389	0\\
0.982324558113953	0\\
0.982424560614015	0\\
0.982524563114078	0\\
0.98262456561414	0\\
0.982724568114203	0\\
0.982824570614265	0\\
0.982924573114328	0\\
0.98302457561439	0\\
0.983124578114453	0\\
0.983224580614515	0\\
0.983324583114578	0\\
0.98342458561464	0\\
0.983524588114703	0\\
0.983624590614765	0\\
0.983724593114828	0\\
0.98382459561489	0\\
0.983924598114953	0\\
0.984024600615015	0\\
0.984124603115078	0\\
0.98422460561514	0\\
0.984324608115203	0\\
0.984424610615265	0\\
0.984524613115328	0\\
0.98462461561539	0\\
0.984724618115453	0\\
0.984824620615515	0\\
0.984924623115578	0\\
0.98502462561564	0\\
0.985124628115703	0\\
0.985224630615765	0\\
0.985324633115828	0\\
0.98542463561589	0\\
0.985524638115953	0\\
0.985624640616015	0\\
0.985724643116078	0\\
0.98582464561614	0\\
0.985924648116203	0\\
0.986024650616265	0\\
0.986124653116328	0\\
0.98622465561639	0\\
0.986324658116453	0\\
0.986424660616515	0\\
0.986524663116578	0\\
0.98662466561664	0\\
0.986724668116703	0\\
0.986824670616765	0\\
0.986924673116828	0\\
0.98702467561689	0\\
0.987124678116953	0\\
0.987224680617015	0\\
0.987324683117078	0\\
0.98742468561714	0\\
0.987524688117203	0\\
0.987624690617265	0\\
0.987724693117328	0\\
0.98782469561739	0\\
0.987924698117453	0\\
0.988024700617515	0\\
0.988124703117578	0\\
0.98822470561764	0\\
0.988324708117703	0\\
0.988424710617765	0\\
0.988524713117828	0\\
0.98862471561789	0\\
0.988724718117953	0\\
0.988824720618015	0\\
0.988924723118078	0\\
0.98902472561814	0\\
0.989124728118203	0\\
0.989224730618265	0\\
0.989324733118328	0\\
0.989424735618391	0\\
0.989524738118453	0\\
0.989624740618515	0\\
0.989724743118578	0\\
0.98982474561864	0\\
0.989924748118703	0\\
0.990024750618765	0\\
0.990124753118828	0\\
0.99022475561889	0\\
0.990324758118953	0\\
0.990424760619016	0\\
0.990524763119078	0\\
0.99062476561914	0\\
0.990724768119203	0\\
0.990824770619265	0\\
0.990924773119328	0\\
0.991024775619391	0\\
0.991124778119453	0\\
0.991224780619515	0\\
0.991324783119578	0\\
0.991424785619641	0\\
0.991524788119703	0\\
0.991624790619765	0\\
0.991724793119828	0\\
0.99182479561989	0\\
0.991924798119953	0\\
0.992024800620016	0\\
0.992124803120078	0\\
0.99222480562014	0\\
0.992324808120203	0\\
0.992424810620266	0\\
0.992524813120328	0\\
0.992624815620391	0\\
0.992724818120453	0\\
0.992824820620515	0\\
0.992924823120578	0\\
0.993024825620641	0\\
0.993124828120703	0\\
0.993224830620765	0\\
0.993324833120828	0\\
0.993424835620891	0\\
0.993524838120953	0\\
0.993624840621016	0\\
0.993724843121078	0\\
0.99382484562114	0\\
0.993924848121203	0\\
0.994024850621266	0\\
0.994124853121328	0\\
0.994224855621391	0\\
0.994324858121453	0\\
0.994424860621516	0\\
0.994524863121578	0\\
0.994624865621641	0\\
0.994724868121703	0\\
0.994824870621766	0\\
0.994924873121828	0\\
0.995024875621891	0\\
0.995124878121953	0\\
0.995224880622016	0\\
0.995324883122078	0\\
0.995424885622141	0\\
0.995524888122203	0\\
0.995624890622266	0\\
0.995724893122328	0\\
0.995824895622391	0\\
0.995924898122453	0\\
0.996024900622516	0\\
0.996124903122578	0\\
0.996224905622641	0\\
0.996324908122703	0\\
0.996424910622766	0\\
0.996524913122828	0\\
0.996624915622891	0\\
0.996724918122953	0\\
0.996824920623016	0\\
0.996924923123078	0\\
0.997024925623141	0\\
0.997124928123203	0\\
0.997224930623266	0\\
0.997324933123328	0\\
0.997424935623391	0\\
0.997524938123453	0\\
0.997624940623516	0\\
0.997724943123578	0\\
0.997824945623641	0\\
0.997924948123703	0\\
0.998024950623766	0\\
0.998124953123828	0\\
0.998224955623891	0\\
0.998324958123953	0\\
0.998424960624016	0\\
0.998524963124078	0\\
0.998624965624141	0\\
0.998724968124203	0\\
0.998824970624266	0\\
0.998924973124328	0\\
0.999024975624391	0\\
0.999124978124453	0\\
0.999224980624516	0\\
0.999324983124578	0\\
0.999424985624641	0\\
0.999524988124703	0\\
0.999624990624766	0\\
0.999724993124828	0\\
0.999824995624891	0\\
0.999924998124953	0\\
1.00002500062502	0\\
1.00012500312508	0\\
1.00022500562514	0\\
1.0003250081252	0\\
1.00042501062527	0\\
1.00052501312533	0\\
1.00062501562539	0\\
1.00072501812545	0\\
1.00082502062552	0\\
1.00092502312558	0\\
1.00102502562564	0\\
1.0011250281257	0\\
1.00122503062577	0\\
1.00132503312583	0\\
1.00142503562589	0\\
1.00152503812595	0\\
1.00162504062602	0\\
1.00172504312608	0\\
1.00182504562614	0\\
1.0019250481262	0\\
1.00202505062627	0\\
1.00212505312633	0\\
1.00222505562639	0\\
1.00232505812645	0\\
1.00242506062652	0\\
1.00252506312658	0\\
1.00262506562664	0\\
1.0027250681267	0\\
1.00282507062677	0\\
1.00292507312683	0\\
1.00302507562689	0\\
1.00312507812695	0\\
1.00322508062702	0\\
1.00332508312708	0\\
1.00342508562714	0\\
1.0035250881272	0\\
1.00362509062727	0\\
1.00372509312733	0\\
1.00382509562739	0\\
1.00392509812745	0\\
1.00402510062752	0\\
1.00412510312758	0\\
1.00422510562764	0\\
1.0043251081277	0\\
1.00442511062777	0\\
1.00452511312783	0\\
1.00462511562789	0\\
1.00472511812795	0\\
1.00482512062802	0\\
1.00492512312808	0\\
1.00502512562814	0\\
1.0051251281282	0\\
1.00522513062827	0\\
1.00532513312833	0\\
1.00542513562839	0\\
1.00552513812845	0\\
1.00562514062852	0\\
1.00572514312858	0\\
1.00582514562864	0\\
1.0059251481287	0\\
1.00602515062877	0\\
1.00612515312883	0\\
1.00622515562889	0\\
1.00632515812895	0\\
1.00642516062902	0\\
1.00652516312908	0\\
1.00662516562914	0\\
1.0067251681292	0\\
1.00682517062927	0\\
1.00692517312933	0\\
1.00702517562939	0\\
1.00712517812945	0\\
1.00722518062952	0\\
1.00732518312958	0\\
1.00742518562964	0\\
1.0075251881297	0\\
1.00762519062977	0\\
1.00772519312983	0\\
1.00782519562989	0\\
1.00792519812995	0\\
1.00802520063002	0\\
1.00812520313008	0\\
1.00822520563014	0\\
1.0083252081302	0\\
1.00842521063027	0\\
1.00852521313033	0\\
1.00862521563039	0\\
1.00872521813045	0\\
1.00882522063052	0\\
1.00892522313058	0\\
1.00902522563064	0\\
1.0091252281307	0\\
1.00922523063077	0\\
1.00932523313083	0\\
1.00942523563089	0\\
1.00952523813095	0\\
1.00962524063102	0\\
1.00972524313108	0\\
1.00982524563114	0\\
1.0099252481312	0\\
1.01002525063127	0\\
1.01012525313133	0\\
1.01022525563139	0\\
1.01032525813145	0\\
1.01042526063152	0\\
1.01052526313158	0\\
1.01062526563164	0\\
1.0107252681317	0\\
1.01082527063177	0\\
1.01092527313183	0\\
1.01102527563189	0\\
1.01112527813195	0\\
1.01122528063202	0\\
1.01132528313208	0\\
1.01142528563214	0\\
1.0115252881322	0\\
1.01162529063227	0\\
1.01172529313233	0\\
1.01182529563239	0\\
1.01192529813245	0\\
1.01202530063252	0\\
1.01212530313258	0\\
1.01222530563264	0\\
1.0123253081327	0\\
1.01242531063277	0\\
1.01252531313283	0\\
1.01262531563289	0\\
1.01272531813295	0\\
1.01282532063302	0\\
1.01292532313308	0\\
1.01302532563314	0\\
1.0131253281332	0\\
1.01322533063327	0\\
1.01332533313333	0\\
1.01342533563339	0\\
1.01352533813345	0\\
1.01362534063352	0\\
1.01372534313358	0\\
1.01382534563364	0\\
1.0139253481337	0\\
1.01402535063377	0\\
1.01412535313383	0\\
1.01422535563389	0\\
1.01432535813395	0\\
1.01442536063402	0\\
1.01452536313408	0\\
1.01462536563414	0\\
1.0147253681342	0\\
1.01482537063427	0\\
1.01492537313433	0\\
1.01502537563439	0\\
1.01512537813445	0\\
1.01522538063452	0\\
1.01532538313458	0\\
1.01542538563464	0\\
1.0155253881347	0\\
1.01562539063477	0\\
1.01572539313483	0\\
1.01582539563489	0\\
1.01592539813495	0\\
1.01602540063502	0\\
1.01612540313508	0\\
1.01622540563514	0\\
1.0163254081352	0\\
1.01642541063527	0\\
1.01652541313533	0\\
1.01662541563539	0\\
1.01672541813545	0\\
1.01682542063552	0\\
1.01692542313558	0\\
1.01702542563564	0\\
1.0171254281357	0\\
1.01722543063577	0\\
1.01732543313583	0\\
1.01742543563589	0\\
1.01752543813595	0\\
1.01762544063602	0\\
1.01772544313608	0\\
1.01782544563614	0\\
1.0179254481362	0\\
1.01802545063627	0\\
1.01812545313633	0\\
1.01822545563639	0\\
1.01832545813645	0\\
1.01842546063652	0\\
1.01852546313658	0\\
1.01862546563664	0\\
1.0187254681367	0\\
1.01882547063677	0\\
1.01892547313683	0\\
1.01902547563689	0\\
1.01912547813695	0\\
1.01922548063702	0\\
1.01932548313708	0\\
1.01942548563714	0\\
1.0195254881372	0\\
1.01962549063727	0\\
1.01972549313733	0\\
1.01982549563739	0\\
1.01992549813745	0\\
1.02002550063752	0\\
1.02012550313758	0\\
1.02022550563764	0\\
1.0203255081377	0\\
1.02042551063777	0\\
1.02052551313783	0\\
1.02062551563789	0\\
1.02072551813795	0\\
1.02082552063802	0\\
1.02092552313808	0\\
1.02102552563814	0\\
1.0211255281382	0\\
1.02122553063827	0\\
1.02132553313833	0\\
1.02142553563839	0\\
1.02152553813845	0\\
1.02162554063852	0\\
1.02172554313858	0\\
1.02182554563864	0\\
1.0219255481387	0\\
1.02202555063877	0\\
1.02212555313883	0\\
1.02222555563889	0\\
1.02232555813895	0\\
1.02242556063902	0\\
1.02252556313908	0\\
1.02262556563914	0\\
1.0227255681392	0\\
1.02282557063927	0\\
1.02292557313933	0\\
1.02302557563939	0\\
1.02312557813945	0\\
1.02322558063952	0\\
1.02332558313958	0\\
1.02342558563964	0\\
1.0235255881397	0\\
1.02362559063977	0\\
1.02372559313983	0\\
1.02382559563989	0\\
1.02392559813995	0\\
1.02402560064002	0\\
1.02412560314008	0\\
1.02422560564014	0\\
1.0243256081402	0\\
1.02442561064027	0\\
1.02452561314033	0\\
1.02462561564039	0\\
1.02472561814045	0\\
1.02482562064052	0\\
1.02492562314058	0\\
1.02502562564064	0\\
1.0251256281407	0\\
1.02522563064077	0\\
1.02532563314083	0\\
1.02542563564089	0\\
1.02552563814095	0\\
1.02562564064102	0\\
1.02572564314108	0\\
1.02582564564114	0\\
1.0259256481412	0\\
1.02602565064127	0\\
1.02612565314133	0\\
1.02622565564139	0\\
1.02632565814145	0\\
1.02642566064152	0\\
1.02652566314158	0\\
1.02662566564164	0\\
1.0267256681417	0\\
1.02682567064177	0\\
1.02692567314183	0\\
1.02702567564189	0\\
1.02712567814195	0\\
1.02722568064202	0\\
1.02732568314208	0\\
1.02742568564214	0\\
1.0275256881422	0\\
1.02762569064227	0\\
1.02772569314233	0\\
1.02782569564239	0\\
1.02792569814245	0\\
1.02802570064252	0\\
1.02812570314258	0\\
1.02822570564264	0\\
1.0283257081427	0\\
1.02842571064277	0\\
1.02852571314283	0\\
1.02862571564289	0\\
1.02872571814295	0\\
1.02882572064302	0\\
1.02892572314308	0\\
1.02902572564314	0\\
1.0291257281432	0\\
1.02922573064327	0\\
1.02932573314333	0\\
1.02942573564339	0\\
1.02952573814345	0\\
1.02962574064352	0\\
1.02972574314358	0\\
1.02982574564364	0\\
1.0299257481437	0\\
1.03002575064377	0\\
1.03012575314383	0\\
1.03022575564389	0\\
1.03032575814395	0\\
1.03042576064402	0\\
1.03052576314408	0\\
1.03062576564414	0\\
1.0307257681442	0\\
1.03082577064427	0\\
1.03092577314433	0\\
1.03102577564439	0\\
1.03112577814445	0\\
1.03122578064452	0\\
1.03132578314458	0\\
1.03142578564464	0\\
1.0315257881447	0\\
1.03162579064477	0\\
1.03172579314483	0\\
1.03182579564489	0\\
1.03192579814495	0\\
1.03202580064502	0\\
1.03212580314508	0\\
1.03222580564514	0\\
1.0323258081452	0\\
1.03242581064527	0\\
1.03252581314533	0\\
1.03262581564539	0\\
1.03272581814545	0\\
1.03282582064552	0\\
1.03292582314558	0\\
1.03302582564564	0\\
1.0331258281457	0\\
1.03322583064577	0\\
1.03332583314583	0\\
1.03342583564589	0\\
1.03352583814595	0\\
1.03362584064602	0\\
1.03372584314608	0\\
1.03382584564614	0\\
1.0339258481462	0\\
1.03402585064627	0\\
1.03412585314633	0\\
1.03422585564639	0\\
1.03432585814645	0\\
1.03442586064652	0\\
1.03452586314658	0\\
1.03462586564664	0\\
1.0347258681467	0\\
1.03482587064677	0\\
1.03492587314683	0\\
1.03502587564689	0\\
1.03512587814695	0\\
1.03522588064702	0\\
1.03532588314708	0\\
1.03542588564714	0\\
1.0355258881472	0\\
1.03562589064727	0\\
1.03572589314733	0\\
1.03582589564739	0\\
1.03592589814745	0\\
1.03602590064752	0\\
1.03612590314758	0\\
1.03622590564764	0\\
1.0363259081477	0\\
1.03642591064777	0\\
1.03652591314783	0\\
1.03662591564789	0\\
1.03672591814795	0\\
1.03682592064802	0\\
1.03692592314808	0\\
1.03702592564814	0\\
1.0371259281482	0\\
1.03722593064827	0\\
1.03732593314833	0\\
1.03742593564839	0\\
1.03752593814845	0\\
1.03762594064852	0\\
1.03772594314858	0\\
1.03782594564864	0\\
1.0379259481487	0\\
1.03802595064877	0\\
1.03812595314883	0\\
1.03822595564889	0\\
1.03832595814895	0\\
1.03842596064902	0\\
1.03852596314908	0\\
1.03862596564914	0\\
1.0387259681492	0\\
1.03882597064927	0\\
1.03892597314933	0\\
1.03902597564939	0\\
1.03912597814945	0\\
1.03922598064952	0\\
1.03932598314958	0\\
1.03942598564964	0\\
1.0395259881497	0\\
1.03962599064977	0\\
1.03972599314983	0\\
1.03982599564989	0\\
1.03992599814995	0\\
1.04002600065002	0\\
1.04012600315008	0\\
1.04022600565014	0\\
1.0403260081502	0\\
1.04042601065027	0\\
1.04052601315033	0\\
1.04062601565039	0\\
1.04072601815045	0\\
1.04082602065052	0\\
1.04092602315058	0\\
1.04102602565064	0\\
1.0411260281507	0\\
1.04122603065077	0\\
1.04132603315083	0\\
1.04142603565089	0\\
1.04152603815095	0\\
1.04162604065102	0\\
1.04172604315108	0\\
1.04182604565114	0\\
1.0419260481512	0\\
1.04202605065127	0\\
1.04212605315133	0\\
1.04222605565139	0\\
1.04232605815145	0\\
1.04242606065152	0\\
1.04252606315158	0\\
1.04262606565164	0\\
1.0427260681517	0\\
1.04282607065177	0\\
1.04292607315183	0\\
1.04302607565189	0\\
1.04312607815195	0\\
1.04322608065202	0\\
1.04332608315208	0\\
1.04342608565214	0\\
1.0435260881522	0\\
1.04362609065227	0\\
1.04372609315233	0\\
1.04382609565239	0\\
1.04392609815245	0\\
1.04402610065252	0\\
1.04412610315258	0\\
1.04422610565264	0\\
1.0443261081527	0\\
1.04442611065277	0\\
1.04452611315283	0\\
1.04462611565289	0\\
1.04472611815295	0\\
1.04482612065302	0\\
1.04492612315308	0\\
1.04502612565314	0\\
1.0451261281532	0\\
1.04522613065327	0\\
1.04532613315333	0\\
1.04542613565339	0\\
1.04552613815345	0\\
1.04562614065352	0\\
1.04572614315358	0\\
1.04582614565364	0\\
1.0459261481537	0\\
1.04602615065377	0\\
1.04612615315383	0\\
1.04622615565389	0\\
1.04632615815395	0\\
1.04642616065402	0\\
1.04652616315408	0\\
1.04662616565414	0\\
1.0467261681542	0\\
1.04682617065427	0\\
1.04692617315433	0\\
1.04702617565439	0\\
1.04712617815445	0\\
1.04722618065452	0\\
1.04732618315458	0\\
1.04742618565464	0\\
1.0475261881547	0\\
1.04762619065477	0\\
1.04772619315483	0\\
1.04782619565489	0\\
1.04792619815495	0\\
1.04802620065502	0\\
1.04812620315508	0\\
1.04822620565514	0\\
1.0483262081552	0\\
1.04842621065527	0\\
1.04852621315533	0\\
1.04862621565539	0\\
1.04872621815545	0\\
1.04882622065552	0\\
1.04892622315558	0\\
1.04902622565564	0\\
1.0491262281557	0\\
1.04922623065577	0\\
1.04932623315583	0\\
1.04942623565589	0\\
1.04952623815595	0\\
1.04962624065602	0\\
1.04972624315608	0\\
1.04982624565614	0\\
1.0499262481562	0\\
1.05002625065627	0\\
1.05012625315633	0\\
1.05022625565639	0\\
1.05032625815645	0\\
1.05042626065652	0\\
1.05052626315658	0\\
1.05062626565664	0\\
1.0507262681567	0\\
1.05082627065677	0\\
1.05092627315683	0\\
1.05102627565689	0\\
1.05112627815695	0\\
1.05122628065702	0\\
1.05132628315708	0\\
1.05142628565714	0\\
1.0515262881572	0\\
1.05162629065727	0\\
1.05172629315733	0\\
1.05182629565739	0\\
1.05192629815745	0\\
1.05202630065752	0\\
1.05212630315758	0\\
1.05222630565764	0\\
1.0523263081577	0\\
1.05242631065777	0\\
1.05252631315783	0\\
1.05262631565789	0\\
1.05272631815795	0\\
1.05282632065802	0\\
1.05292632315808	0\\
1.05302632565814	0\\
1.0531263281582	0\\
1.05322633065827	0\\
1.05332633315833	0\\
1.05342633565839	0\\
1.05352633815845	0\\
1.05362634065852	0\\
1.05372634315858	0\\
1.05382634565864	0\\
1.0539263481587	0\\
1.05402635065877	0\\
1.05412635315883	0\\
1.05422635565889	0\\
1.05432635815895	0\\
1.05442636065902	0\\
1.05452636315908	0\\
1.05462636565914	0\\
1.0547263681592	0\\
1.05482637065927	0\\
1.05492637315933	0\\
1.05502637565939	0\\
1.05512637815945	0\\
1.05522638065952	0\\
1.05532638315958	0\\
1.05542638565964	0\\
1.0555263881597	0\\
1.05562639065977	0\\
1.05572639315983	0\\
1.05582639565989	0\\
1.05592639815995	0\\
1.05602640066002	0\\
1.05612640316008	0\\
1.05622640566014	0\\
1.0563264081602	0\\
1.05642641066027	0\\
1.05652641316033	0\\
1.05662641566039	0\\
1.05672641816045	0\\
1.05682642066052	0\\
1.05692642316058	0\\
1.05702642566064	0\\
1.0571264281607	0\\
1.05722643066077	0\\
1.05732643316083	0\\
1.05742643566089	0\\
1.05752643816095	0\\
1.05762644066102	0\\
1.05772644316108	0\\
1.05782644566114	0\\
1.0579264481612	0\\
1.05802645066127	0\\
1.05812645316133	0\\
1.05822645566139	0\\
1.05832645816145	0\\
1.05842646066152	0\\
1.05852646316158	0\\
1.05862646566164	0\\
1.0587264681617	0\\
1.05882647066177	0\\
1.05892647316183	0\\
1.05902647566189	0\\
1.05912647816195	0\\
1.05922648066202	0\\
1.05932648316208	0\\
1.05942648566214	0\\
1.0595264881622	0\\
1.05962649066227	0\\
1.05972649316233	0\\
1.05982649566239	0\\
1.05992649816245	0\\
1.06002650066252	0\\
1.06012650316258	0\\
1.06022650566264	0\\
1.0603265081627	0\\
1.06042651066277	0\\
1.06052651316283	0\\
1.06062651566289	0\\
1.06072651816295	0\\
1.06082652066302	0\\
1.06092652316308	0\\
1.06102652566314	0\\
1.0611265281632	0\\
1.06122653066327	0\\
1.06132653316333	0\\
1.06142653566339	0\\
1.06152653816345	0\\
1.06162654066352	0\\
1.06172654316358	0\\
1.06182654566364	0\\
1.0619265481637	0\\
1.06202655066377	0\\
1.06212655316383	0\\
1.06222655566389	0\\
1.06232655816395	0\\
1.06242656066402	0\\
1.06252656316408	0\\
1.06262656566414	0\\
1.0627265681642	0\\
1.06282657066427	0\\
1.06292657316433	0\\
1.06302657566439	0\\
1.06312657816445	0\\
1.06322658066452	0\\
1.06332658316458	0\\
1.06342658566464	0\\
1.0635265881647	0\\
1.06362659066477	0\\
1.06372659316483	0\\
1.06382659566489	0\\
1.06392659816495	0\\
1.06402660066502	0\\
1.06412660316508	0\\
1.06422660566514	0\\
1.0643266081652	0\\
1.06442661066527	0\\
1.06452661316533	0\\
1.06462661566539	0\\
1.06472661816545	0\\
1.06482662066552	0\\
1.06492662316558	0\\
1.06502662566564	0\\
1.0651266281657	0\\
1.06522663066577	0\\
1.06532663316583	0\\
1.06542663566589	0\\
1.06552663816595	0\\
1.06562664066602	0\\
1.06572664316608	0\\
1.06582664566614	0\\
1.0659266481662	0\\
1.06602665066627	0\\
1.06612665316633	0\\
1.06622665566639	0\\
1.06632665816645	0\\
1.06642666066652	0\\
1.06652666316658	0\\
1.06662666566664	0\\
1.0667266681667	0\\
1.06682667066677	0\\
1.06692667316683	0\\
1.06702667566689	0\\
1.06712667816695	0\\
1.06722668066702	0\\
1.06732668316708	0\\
1.06742668566714	0\\
1.0675266881672	0\\
1.06762669066727	0\\
1.06772669316733	0\\
1.06782669566739	0\\
1.06792669816745	0\\
1.06802670066752	0\\
1.06812670316758	0\\
1.06822670566764	0\\
1.0683267081677	0\\
1.06842671066777	0\\
1.06852671316783	0\\
1.06862671566789	0\\
1.06872671816795	0\\
1.06882672066802	0\\
1.06892672316808	0\\
1.06902672566814	0\\
1.0691267281682	0\\
1.06922673066827	0\\
1.06932673316833	0\\
1.06942673566839	0\\
1.06952673816845	0\\
1.06962674066852	0\\
1.06972674316858	0\\
1.06982674566864	0\\
1.0699267481687	0\\
1.07002675066877	0\\
1.07012675316883	0\\
1.07022675566889	0\\
1.07032675816895	0\\
1.07042676066902	0\\
1.07052676316908	0\\
1.07062676566914	0\\
1.0707267681692	0\\
1.07082677066927	0\\
1.07092677316933	0\\
1.07102677566939	0\\
1.07112677816945	0\\
1.07122678066952	0\\
1.07132678316958	0\\
1.07142678566964	0\\
1.0715267881697	0\\
1.07162679066977	0\\
1.07172679316983	0\\
1.07182679566989	0\\
1.07192679816995	0\\
1.07202680067002	0\\
1.07212680317008	0\\
1.07222680567014	0\\
1.0723268081702	0\\
1.07242681067027	0\\
1.07252681317033	0\\
1.07262681567039	0\\
1.07272681817045	0\\
1.07282682067052	0\\
1.07292682317058	0\\
1.07302682567064	0\\
1.0731268281707	0\\
1.07322683067077	0\\
1.07332683317083	0\\
1.07342683567089	0\\
1.07352683817095	0\\
1.07362684067102	0\\
1.07372684317108	0\\
1.07382684567114	0\\
1.0739268481712	0\\
1.07402685067127	0\\
1.07412685317133	0\\
1.07422685567139	0\\
1.07432685817145	0\\
1.07442686067152	0\\
1.07452686317158	0\\
1.07462686567164	0\\
1.0747268681717	0\\
1.07482687067177	0\\
1.07492687317183	0\\
1.07502687567189	0\\
1.07512687817195	0\\
1.07522688067202	0\\
1.07532688317208	0\\
1.07542688567214	0\\
1.0755268881722	0\\
1.07562689067227	0\\
1.07572689317233	0\\
1.07582689567239	0\\
1.07592689817245	0\\
1.07602690067252	0\\
1.07612690317258	0\\
1.07622690567264	0\\
1.0763269081727	0\\
1.07642691067277	0\\
1.07652691317283	0\\
1.07662691567289	0\\
1.07672691817295	0\\
1.07682692067302	0\\
1.07692692317308	0\\
1.07702692567314	0\\
1.0771269281732	0\\
1.07722693067327	0\\
1.07732693317333	0\\
1.07742693567339	0\\
1.07752693817345	0\\
1.07762694067352	0\\
1.07772694317358	0\\
1.07782694567364	0\\
1.0779269481737	0\\
1.07802695067377	0\\
1.07812695317383	0\\
1.07822695567389	0\\
1.07832695817395	0\\
1.07842696067402	0\\
1.07852696317408	0\\
1.07862696567414	0\\
1.0787269681742	0\\
1.07882697067427	0\\
1.07892697317433	0\\
1.07902697567439	0\\
1.07912697817445	0\\
1.07922698067452	0\\
1.07932698317458	0\\
1.07942698567464	0\\
1.0795269881747	0\\
1.07962699067477	0\\
1.07972699317483	0\\
1.07982699567489	0\\
1.07992699817495	0\\
1.08002700067502	0\\
1.08012700317508	0\\
1.08022700567514	0\\
1.0803270081752	0\\
1.08042701067527	0\\
1.08052701317533	0\\
1.08062701567539	0\\
1.08072701817545	0\\
1.08082702067552	0\\
1.08092702317558	0\\
1.08102702567564	0\\
1.0811270281757	0\\
1.08122703067577	0\\
1.08132703317583	0\\
1.08142703567589	0\\
1.08152703817595	0\\
1.08162704067602	0\\
1.08172704317608	0\\
1.08182704567614	0\\
1.0819270481762	0\\
1.08202705067627	0\\
1.08212705317633	0\\
1.08222705567639	0\\
1.08232705817645	0\\
1.08242706067652	0\\
1.08252706317658	0\\
1.08262706567664	0\\
1.0827270681767	0\\
1.08282707067677	0\\
1.08292707317683	0\\
1.08302707567689	0\\
1.08312707817695	0\\
1.08322708067702	0\\
1.08332708317708	0\\
1.08342708567714	0\\
1.0835270881772	0\\
1.08362709067727	0\\
1.08372709317733	0\\
1.08382709567739	0\\
1.08392709817745	0\\
1.08402710067752	0\\
1.08412710317758	0\\
1.08422710567764	0\\
1.0843271081777	0\\
1.08442711067777	0\\
1.08452711317783	0\\
1.08462711567789	0\\
1.08472711817795	0\\
1.08482712067802	0\\
1.08492712317808	0\\
1.08502712567814	0\\
1.0851271281782	0\\
1.08522713067827	0\\
1.08532713317833	0\\
1.08542713567839	0\\
1.08552713817845	0\\
1.08562714067852	0\\
1.08572714317858	0\\
1.08582714567864	0\\
1.0859271481787	0\\
1.08602715067877	0\\
1.08612715317883	0\\
1.08622715567889	0\\
1.08632715817895	0\\
1.08642716067902	0\\
1.08652716317908	0\\
1.08662716567914	0\\
1.0867271681792	0\\
1.08682717067927	0\\
1.08692717317933	0\\
1.08702717567939	0\\
1.08712717817945	0\\
1.08722718067952	0\\
1.08732718317958	0\\
1.08742718567964	0\\
1.0875271881797	0\\
1.08762719067977	0\\
1.08772719317983	0\\
1.08782719567989	0\\
1.08792719817995	0\\
1.08802720068002	0\\
1.08812720318008	0\\
1.08822720568014	0\\
1.0883272081802	0\\
1.08842721068027	0\\
1.08852721318033	0\\
1.08862721568039	0\\
1.08872721818045	0\\
1.08882722068052	0\\
1.08892722318058	0\\
1.08902722568064	0\\
1.0891272281807	0\\
1.08922723068077	0\\
1.08932723318083	0\\
1.08942723568089	0\\
1.08952723818095	0\\
1.08962724068102	0\\
1.08972724318108	0\\
1.08982724568114	0\\
1.0899272481812	0\\
1.09002725068127	0\\
1.09012725318133	0\\
1.09022725568139	0\\
1.09032725818145	0\\
1.09042726068152	0\\
1.09052726318158	0\\
1.09062726568164	0\\
1.0907272681817	0\\
1.09082727068177	0\\
1.09092727318183	0\\
1.09102727568189	0\\
1.09112727818195	0\\
1.09122728068202	0\\
1.09132728318208	0\\
1.09142728568214	0\\
1.0915272881822	0\\
1.09162729068227	0\\
1.09172729318233	0\\
1.09182729568239	0\\
1.09192729818245	0\\
1.09202730068252	0\\
1.09212730318258	0\\
1.09222730568264	0\\
1.0923273081827	0\\
1.09242731068277	0\\
1.09252731318283	0\\
1.09262731568289	0\\
1.09272731818295	0\\
1.09282732068302	0\\
1.09292732318308	0\\
1.09302732568314	0\\
1.0931273281832	0\\
1.09322733068327	0\\
1.09332733318333	0\\
1.09342733568339	0\\
1.09352733818345	0\\
1.09362734068352	0\\
1.09372734318358	0\\
1.09382734568364	0\\
1.0939273481837	0\\
1.09402735068377	0\\
1.09412735318383	0\\
1.09422735568389	0\\
1.09432735818395	0\\
1.09442736068402	0\\
1.09452736318408	0\\
1.09462736568414	0\\
1.0947273681842	0\\
1.09482737068427	0\\
1.09492737318433	0\\
1.09502737568439	0\\
1.09512737818445	0\\
1.09522738068452	0\\
1.09532738318458	0\\
1.09542738568464	0\\
1.0955273881847	0\\
1.09562739068477	0\\
1.09572739318483	0\\
1.09582739568489	0\\
1.09592739818495	0\\
1.09602740068502	0\\
1.09612740318508	0\\
1.09622740568514	0\\
1.0963274081852	0\\
1.09642741068527	0\\
1.09652741318533	0\\
1.09662741568539	0\\
1.09672741818545	0\\
1.09682742068552	0\\
1.09692742318558	0\\
1.09702742568564	0\\
1.0971274281857	0\\
1.09722743068577	0\\
1.09732743318583	0\\
1.09742743568589	0\\
1.09752743818595	0\\
1.09762744068602	0\\
1.09772744318608	0\\
1.09782744568614	0\\
1.0979274481862	0\\
1.09802745068627	0\\
1.09812745318633	0\\
1.09822745568639	0\\
1.09832745818645	0\\
1.09842746068652	0\\
1.09852746318658	0\\
1.09862746568664	0\\
1.0987274681867	0\\
1.09882747068677	0\\
1.09892747318683	0\\
1.09902747568689	0\\
1.09912747818695	0\\
1.09922748068702	0\\
1.09932748318708	0\\
1.09942748568714	0\\
1.0995274881872	0\\
1.09962749068727	0\\
1.09972749318733	0\\
1.09982749568739	0\\
1.09992749818745	0\\
1.10002750068752	0\\
1.10012750318758	0\\
1.10022750568764	0\\
1.1003275081877	0\\
1.10042751068777	0\\
1.10052751318783	0\\
1.10062751568789	0\\
1.10072751818795	0\\
1.10082752068802	0\\
1.10092752318808	0\\
1.10102752568814	0\\
1.1011275281882	0\\
1.10122753068827	0\\
1.10132753318833	0\\
1.10142753568839	0\\
1.10152753818845	0\\
1.10162754068852	0\\
1.10172754318858	0\\
1.10182754568864	0\\
1.1019275481887	0\\
1.10202755068877	0\\
1.10212755318883	0\\
1.10222755568889	0\\
1.10232755818895	0\\
1.10242756068902	0\\
1.10252756318908	0\\
1.10262756568914	0\\
1.1027275681892	0\\
1.10282757068927	0\\
1.10292757318933	0\\
1.10302757568939	0\\
1.10312757818945	0\\
1.10322758068952	0\\
1.10332758318958	0\\
1.10342758568964	0\\
1.1035275881897	0\\
1.10362759068977	0\\
1.10372759318983	0\\
1.10382759568989	0\\
1.10392759818995	0\\
1.10402760069002	0\\
1.10412760319008	0\\
1.10422760569014	0\\
1.1043276081902	0\\
1.10442761069027	0\\
1.10452761319033	0\\
1.10462761569039	0\\
1.10472761819045	0\\
1.10482762069052	0\\
1.10492762319058	0\\
1.10502762569064	0\\
1.1051276281907	0\\
1.10522763069077	0\\
1.10532763319083	0\\
1.10542763569089	0\\
1.10552763819095	0\\
1.10562764069102	0\\
1.10572764319108	0\\
1.10582764569114	0\\
1.1059276481912	0\\
1.10602765069127	0\\
1.10612765319133	0\\
1.10622765569139	0\\
1.10632765819145	0\\
1.10642766069152	0\\
1.10652766319158	0\\
1.10662766569164	0\\
1.1067276681917	0\\
1.10682767069177	0\\
1.10692767319183	0\\
1.10702767569189	0\\
1.10712767819195	0\\
1.10722768069202	0\\
1.10732768319208	0\\
1.10742768569214	0\\
1.1075276881922	0\\
1.10762769069227	0\\
1.10772769319233	0\\
1.10782769569239	0\\
1.10792769819245	0\\
1.10802770069252	0\\
1.10812770319258	0\\
1.10822770569264	0\\
1.1083277081927	0\\
1.10842771069277	0\\
1.10852771319283	0\\
1.10862771569289	0\\
1.10872771819295	0\\
1.10882772069302	0\\
1.10892772319308	0\\
1.10902772569314	0\\
1.1091277281932	0\\
1.10922773069327	0\\
1.10932773319333	0\\
1.10942773569339	0\\
1.10952773819345	0\\
1.10962774069352	0\\
1.10972774319358	0\\
1.10982774569364	0\\
1.1099277481937	0\\
1.11002775069377	0\\
1.11012775319383	0\\
1.11022775569389	0\\
1.11032775819395	0\\
1.11042776069402	0\\
1.11052776319408	0\\
1.11062776569414	0\\
1.1107277681942	0\\
1.11082777069427	0\\
1.11092777319433	0\\
1.11102777569439	0\\
1.11112777819445	0\\
1.11122778069452	0\\
1.11132778319458	0\\
1.11142778569464	0\\
1.1115277881947	0\\
1.11162779069477	0\\
1.11172779319483	0\\
1.11182779569489	0\\
1.11192779819495	0\\
1.11202780069502	0\\
1.11212780319508	0\\
1.11222780569514	0\\
1.1123278081952	0\\
1.11242781069527	0\\
1.11252781319533	0\\
1.11262781569539	0\\
1.11272781819545	0\\
1.11282782069552	0\\
1.11292782319558	0\\
1.11302782569564	0\\
1.1131278281957	0\\
1.11322783069577	0\\
1.11332783319583	0\\
1.11342783569589	0\\
1.11352783819595	0\\
1.11362784069602	0\\
1.11372784319608	0\\
1.11382784569614	0\\
1.1139278481962	0\\
1.11402785069627	0\\
1.11412785319633	0\\
1.11422785569639	0\\
1.11432785819645	0\\
1.11442786069652	0\\
1.11452786319658	0\\
1.11462786569664	0\\
1.1147278681967	0\\
1.11482787069677	0\\
1.11492787319683	0\\
1.11502787569689	0\\
1.11512787819695	0\\
1.11522788069702	0\\
1.11532788319708	0\\
1.11542788569714	0\\
1.1155278881972	0\\
1.11562789069727	0\\
1.11572789319733	0\\
1.11582789569739	0\\
1.11592789819745	0\\
1.11602790069752	0\\
1.11612790319758	0\\
1.11622790569764	0\\
1.1163279081977	0\\
1.11642791069777	0\\
1.11652791319783	0\\
1.11662791569789	0\\
1.11672791819796	0\\
1.11682792069802	0\\
1.11692792319808	0\\
1.11702792569814	0\\
1.1171279281982	0\\
1.11722793069827	0\\
1.11732793319833	0\\
1.11742793569839	0\\
1.11752793819845	0\\
1.11762794069852	0\\
1.11772794319858	0\\
1.11782794569864	0\\
1.11792794819871	0\\
1.11802795069877	0\\
1.11812795319883	0\\
1.11822795569889	0\\
1.11832795819895	0\\
1.11842796069902	0\\
1.11852796319908	0\\
1.11862796569914	0\\
1.11872796819921	0\\
1.11882797069927	0\\
1.11892797319933	0\\
1.11902797569939	0\\
1.11912797819945	0\\
1.11922798069952	0\\
1.11932798319958	0\\
1.11942798569964	0\\
1.1195279881997	0\\
1.11962799069977	0\\
1.11972799319983	0\\
1.11982799569989	0\\
1.11992799819996	0\\
1.12002800070002	0\\
1.12012800320008	0\\
1.12022800570014	0\\
1.1203280082002	0\\
1.12042801070027	0\\
1.12052801320033	0\\
1.12062801570039	0\\
1.12072801820046	0\\
1.12082802070052	0\\
1.12092802320058	0\\
1.12102802570064	0\\
1.12112802820071	0\\
1.12122803070077	0\\
1.12132803320083	0\\
1.12142803570089	0\\
1.12152803820095	0\\
1.12162804070102	0\\
1.12172804320108	0\\
1.12182804570114	0\\
1.12192804820121	0\\
1.12202805070127	0\\
1.12212805320133	0\\
1.12222805570139	0\\
1.12232805820145	0\\
1.12242806070152	0\\
1.12252806320158	0\\
1.12262806570164	0\\
1.12272806820171	0\\
1.12282807070177	0\\
1.12292807320183	0\\
1.12302807570189	0\\
1.12312807820196	0\\
1.12322808070202	0\\
1.12332808320208	0\\
1.12342808570214	0\\
1.1235280882022	0\\
1.12362809070227	0\\
1.12372809320233	0\\
1.12382809570239	0\\
1.12392809820246	0\\
1.12402810070252	0\\
1.12412810320258	0\\
1.12422810570264	0\\
1.12432810820271	0\\
1.12442811070277	0\\
1.12452811320283	0\\
1.12462811570289	0\\
1.12472811820296	0\\
1.12482812070302	0\\
1.12492812320308	0\\
1.12502812570314	0\\
1.12512812820321	0\\
1.12522813070327	0\\
1.12532813320333	0\\
1.12542813570339	0\\
1.12552813820345	0\\
1.12562814070352	0\\
1.12572814320358	0\\
1.12582814570364	0\\
1.12592814820371	0\\
1.12602815070377	0\\
1.12612815320383	0\\
1.12622815570389	0\\
1.12632815820396	0\\
1.12642816070402	0\\
1.12652816320408	0\\
1.12662816570414	0\\
1.12672816820421	0\\
1.12682817070427	0\\
1.12692817320433	0\\
1.12702817570439	0\\
1.12712817820446	0\\
1.12722818070452	0\\
1.12732818320458	0\\
1.12742818570464	0\\
1.12752818820471	0\\
1.12762819070477	0\\
1.12772819320483	0\\
1.12782819570489	0\\
1.12792819820496	0\\
1.12802820070502	0\\
1.12812820320508	0\\
1.12822820570514	0\\
1.12832820820521	0\\
1.12842821070527	0\\
1.12852821320533	0\\
1.12862821570539	0\\
1.12872821820546	0\\
1.12882822070552	0\\
1.12892822320558	0\\
1.12902822570564	0\\
1.12912822820571	0\\
1.12922823070577	0\\
1.12932823320583	0\\
1.12942823570589	0\\
1.12952823820596	0\\
1.12962824070602	0\\
1.12972824320608	0\\
1.12982824570614	0\\
1.12992824820621	0\\
1.13002825070627	0\\
1.13012825320633	0\\
1.13022825570639	0\\
1.13032825820646	0\\
1.13042826070652	0\\
1.13052826320658	0\\
1.13062826570664	0\\
1.13072826820671	0\\
1.13082827070677	0\\
1.13092827320683	0\\
1.13102827570689	0\\
1.13112827820696	0\\
1.13122828070702	0\\
1.13132828320708	0\\
1.13142828570714	0\\
1.13152828820721	0\\
1.13162829070727	0\\
1.13172829320733	0\\
1.13182829570739	0\\
1.13192829820746	0\\
1.13202830070752	0\\
1.13212830320758	0\\
1.13222830570764	0\\
1.13232830820771	0\\
1.13242831070777	0\\
1.13252831320783	0\\
1.13262831570789	0\\
1.13272831820796	0\\
1.13282832070802	0\\
1.13292832320808	0\\
1.13302832570814	0\\
1.13312832820821	0\\
1.13322833070827	0\\
1.13332833320833	0\\
1.13342833570839	0\\
1.13352833820846	0\\
1.13362834070852	0\\
1.13372834320858	0\\
1.13382834570864	0\\
1.13392834820871	0\\
1.13402835070877	0\\
1.13412835320883	0\\
1.13422835570889	0\\
1.13432835820896	0\\
1.13442836070902	0\\
1.13452836320908	0\\
1.13462836570914	0\\
1.13472836820921	0\\
1.13482837070927	0\\
1.13492837320933	0\\
1.13502837570939	0\\
1.13512837820946	0\\
1.13522838070952	0\\
1.13532838320958	0\\
1.13542838570964	0\\
1.13552838820971	0\\
1.13562839070977	0\\
1.13572839320983	0\\
1.13582839570989	0\\
1.13592839820996	0\\
1.13602840071002	0\\
1.13612840321008	0\\
1.13622840571014	0\\
1.13632840821021	0\\
1.13642841071027	0\\
1.13652841321033	0\\
1.13662841571039	0\\
1.13672841821046	0\\
1.13682842071052	0\\
1.13692842321058	0\\
1.13702842571064	0\\
1.13712842821071	0\\
1.13722843071077	0\\
1.13732843321083	0\\
1.13742843571089	0\\
1.13752843821096	0\\
1.13762844071102	0\\
1.13772844321108	0\\
1.13782844571114	0\\
1.13792844821121	0\\
1.13802845071127	0\\
1.13812845321133	0\\
1.13822845571139	0\\
1.13832845821146	0\\
1.13842846071152	0\\
1.13852846321158	0\\
1.13862846571164	0\\
1.13872846821171	0\\
1.13882847071177	0\\
1.13892847321183	0\\
1.13902847571189	0\\
1.13912847821196	0\\
1.13922848071202	0\\
1.13932848321208	0\\
1.13942848571214	0\\
1.13952848821221	0\\
1.13962849071227	0\\
1.13972849321233	0\\
1.13982849571239	0\\
1.13992849821246	0\\
1.14002850071252	0\\
1.14012850321258	0\\
1.14022850571264	0\\
1.14032850821271	0\\
1.14042851071277	0\\
1.14052851321283	0\\
1.14062851571289	0\\
1.14072851821296	0\\
1.14082852071302	0\\
1.14092852321308	0\\
1.14102852571314	0\\
1.14112852821321	0\\
1.14122853071327	0\\
1.14132853321333	0\\
1.14142853571339	0\\
1.14152853821346	0\\
1.14162854071352	0\\
1.14172854321358	0\\
1.14182854571364	0\\
1.14192854821371	0\\
1.14202855071377	0\\
1.14212855321383	0\\
1.14222855571389	0\\
1.14232855821396	0\\
1.14242856071402	0\\
1.14252856321408	0\\
1.14262856571414	0\\
1.14272856821421	0\\
1.14282857071427	0\\
1.14292857321433	0\\
1.14302857571439	0\\
1.14312857821446	0\\
1.14322858071452	0\\
1.14332858321458	0\\
1.14342858571464	0\\
1.14352858821471	0\\
1.14362859071477	0\\
1.14372859321483	0\\
1.14382859571489	0\\
1.14392859821496	0\\
1.14402860071502	0\\
1.14412860321508	0\\
1.14422860571514	0\\
1.14432860821521	0\\
1.14442861071527	0\\
1.14452861321533	0\\
1.14462861571539	0\\
1.14472861821546	0\\
1.14482862071552	0\\
1.14492862321558	0\\
1.14502862571564	0\\
1.14512862821571	0\\
1.14522863071577	0\\
1.14532863321583	0\\
1.14542863571589	0\\
1.14552863821596	0\\
1.14562864071602	0\\
1.14572864321608	0\\
1.14582864571614	0\\
1.14592864821621	0\\
1.14602865071627	0\\
1.14612865321633	0\\
1.14622865571639	0\\
1.14632865821646	0\\
1.14642866071652	0\\
1.14652866321658	0\\
1.14662866571664	0\\
1.14672866821671	0\\
1.14682867071677	0\\
1.14692867321683	0\\
1.14702867571689	0\\
1.14712867821696	0\\
1.14722868071702	0\\
1.14732868321708	0\\
1.14742868571714	0\\
1.14752868821721	0\\
1.14762869071727	0\\
1.14772869321733	0\\
1.14782869571739	0\\
1.14792869821746	0\\
1.14802870071752	0\\
1.14812870321758	0\\
1.14822870571764	0\\
1.14832870821771	0\\
1.14842871071777	0\\
1.14852871321783	0\\
1.14862871571789	0\\
1.14872871821796	0\\
1.14882872071802	0\\
1.14892872321808	0\\
1.14902872571814	0\\
1.14912872821821	0\\
1.14922873071827	0\\
1.14932873321833	0\\
1.14942873571839	0\\
1.14952873821846	0\\
1.14962874071852	0\\
1.14972874321858	0\\
1.14982874571864	0\\
1.14992874821871	0\\
1.15002875071877	0\\
1.15012875321883	0\\
1.15022875571889	0\\
1.15032875821896	0\\
1.15042876071902	0\\
1.15052876321908	0\\
1.15062876571914	0\\
1.15072876821921	0\\
1.15082877071927	0\\
1.15092877321933	0\\
1.15102877571939	0\\
1.15112877821946	0\\
1.15122878071952	0\\
1.15132878321958	0\\
1.15142878571964	0\\
1.15152878821971	0\\
1.15162879071977	0\\
1.15172879321983	0\\
1.15182879571989	0\\
1.15192879821996	0\\
1.15202880072002	0\\
1.15212880322008	0\\
1.15222880572014	0\\
1.15232880822021	0\\
1.15242881072027	0\\
1.15252881322033	0\\
1.15262881572039	0\\
1.15272881822046	0\\
1.15282882072052	0\\
1.15292882322058	0\\
1.15302882572064	0\\
1.15312882822071	0\\
1.15322883072077	0\\
1.15332883322083	0\\
1.15342883572089	0\\
1.15352883822096	0\\
1.15362884072102	0\\
1.15372884322108	0\\
1.15382884572114	0\\
1.15392884822121	0\\
1.15402885072127	0\\
1.15412885322133	0\\
1.15422885572139	0\\
1.15432885822146	0\\
1.15442886072152	0\\
1.15452886322158	0\\
1.15462886572164	0\\
1.15472886822171	0\\
1.15482887072177	0\\
1.15492887322183	0\\
1.15502887572189	0\\
1.15512887822196	0\\
1.15522888072202	0\\
1.15532888322208	0\\
1.15542888572214	0\\
1.15552888822221	0\\
1.15562889072227	0\\
1.15572889322233	0\\
1.15582889572239	0\\
1.15592889822246	0\\
1.15602890072252	0\\
1.15612890322258	0\\
1.15622890572264	0\\
1.15632890822271	0\\
1.15642891072277	0\\
1.15652891322283	0\\
1.15662891572289	0\\
1.15672891822296	0\\
1.15682892072302	0\\
1.15692892322308	0\\
1.15702892572314	0\\
1.15712892822321	0\\
1.15722893072327	0\\
1.15732893322333	0\\
1.15742893572339	0\\
1.15752893822346	0\\
1.15762894072352	0\\
1.15772894322358	0\\
1.15782894572364	0\\
1.15792894822371	0\\
1.15802895072377	0\\
1.15812895322383	0\\
1.15822895572389	0\\
1.15832895822396	0\\
1.15842896072402	0\\
1.15852896322408	0\\
1.15862896572414	0\\
1.15872896822421	0\\
1.15882897072427	0\\
1.15892897322433	0\\
1.15902897572439	0\\
1.15912897822446	0\\
1.15922898072452	0\\
1.15932898322458	0\\
1.15942898572464	0\\
1.15952898822471	0\\
1.15962899072477	0\\
1.15972899322483	0\\
1.15982899572489	0\\
1.15992899822496	0\\
1.16002900072502	0\\
1.16012900322508	0\\
1.16022900572514	0\\
1.16032900822521	0\\
1.16042901072527	0\\
1.16052901322533	0\\
1.16062901572539	0\\
1.16072901822546	0\\
1.16082902072552	0\\
1.16092902322558	0\\
1.16102902572564	0\\
1.16112902822571	0\\
1.16122903072577	0\\
1.16132903322583	0\\
1.16142903572589	0\\
1.16152903822596	0\\
1.16162904072602	0\\
1.16172904322608	0\\
1.16182904572614	0\\
1.16192904822621	0\\
1.16202905072627	0\\
1.16212905322633	0\\
1.16222905572639	0\\
1.16232905822646	0\\
1.16242906072652	0\\
1.16252906322658	0\\
1.16262906572664	0\\
1.16272906822671	0\\
1.16282907072677	0\\
1.16292907322683	0\\
1.16302907572689	0\\
1.16312907822696	0\\
1.16322908072702	0\\
1.16332908322708	0\\
1.16342908572714	0\\
1.16352908822721	0\\
1.16362909072727	0\\
1.16372909322733	0\\
1.16382909572739	0\\
1.16392909822746	0\\
1.16402910072752	0\\
1.16412910322758	0\\
1.16422910572764	0\\
1.16432910822771	0\\
1.16442911072777	0\\
1.16452911322783	0\\
1.16462911572789	0\\
1.16472911822796	0\\
1.16482912072802	0\\
1.16492912322808	0\\
1.16502912572814	0\\
1.16512912822821	0\\
1.16522913072827	0\\
1.16532913322833	0\\
1.16542913572839	0\\
1.16552913822846	0\\
1.16562914072852	0\\
1.16572914322858	0\\
1.16582914572864	0\\
1.16592914822871	0\\
1.16602915072877	0\\
1.16612915322883	0\\
1.16622915572889	0\\
1.16632915822896	0\\
1.16642916072902	0\\
1.16652916322908	0\\
1.16662916572914	0\\
1.16672916822921	0\\
1.16682917072927	0\\
1.16692917322933	0\\
1.16702917572939	0\\
1.16712917822946	0\\
1.16722918072952	0\\
1.16732918322958	0\\
1.16742918572964	0\\
1.16752918822971	0\\
1.16762919072977	0\\
1.16772919322983	0\\
1.16782919572989	0\\
1.16792919822996	0\\
1.16802920073002	0\\
1.16812920323008	0\\
1.16822920573014	0\\
1.16832920823021	0\\
1.16842921073027	0\\
1.16852921323033	0\\
1.16862921573039	0\\
1.16872921823046	0\\
1.16882922073052	0\\
1.16892922323058	0\\
1.16902922573064	0\\
1.16912922823071	0\\
1.16922923073077	0\\
1.16932923323083	0\\
1.16942923573089	0\\
1.16952923823096	0\\
1.16962924073102	0\\
1.16972924323108	0\\
1.16982924573114	0\\
1.16992924823121	0\\
1.17002925073127	0\\
1.17012925323133	0\\
1.17022925573139	0\\
1.17032925823146	0\\
1.17042926073152	0\\
1.17052926323158	0\\
1.17062926573164	0\\
1.17072926823171	0\\
1.17082927073177	0\\
1.17092927323183	0\\
1.17102927573189	0\\
1.17112927823196	0\\
1.17122928073202	0\\
1.17132928323208	0\\
1.17142928573214	0\\
1.17152928823221	0\\
1.17162929073227	0\\
1.17172929323233	0\\
1.17182929573239	0\\
1.17192929823246	0\\
1.17202930073252	0\\
1.17212930323258	0\\
1.17222930573264	0\\
1.17232930823271	0\\
1.17242931073277	0\\
1.17252931323283	0\\
1.17262931573289	0\\
1.17272931823296	0\\
1.17282932073302	0\\
1.17292932323308	0\\
1.17302932573314	0\\
1.17312932823321	0\\
1.17322933073327	0\\
1.17332933323333	0\\
1.17342933573339	0\\
1.17352933823346	0\\
1.17362934073352	0\\
1.17372934323358	0\\
1.17382934573364	0\\
1.17392934823371	0\\
1.17402935073377	0\\
1.17412935323383	0\\
1.17422935573389	0\\
1.17432935823396	0\\
1.17442936073402	0\\
1.17452936323408	0\\
1.17462936573414	0\\
1.17472936823421	0\\
1.17482937073427	0\\
1.17492937323433	0\\
1.17502937573439	0\\
1.17512937823446	0\\
1.17522938073452	0\\
1.17532938323458	0\\
1.17542938573464	0\\
1.17552938823471	0\\
1.17562939073477	0\\
1.17572939323483	0\\
1.17582939573489	0\\
1.17592939823496	0\\
1.17602940073502	0\\
1.17612940323508	0\\
1.17622940573514	0\\
1.17632940823521	0\\
1.17642941073527	0\\
1.17652941323533	0\\
1.17662941573539	0\\
1.17672941823546	0\\
1.17682942073552	0\\
1.17692942323558	0\\
1.17702942573564	0\\
1.17712942823571	0\\
1.17722943073577	0\\
1.17732943323583	0\\
1.17742943573589	0\\
1.17752943823596	0\\
1.17762944073602	0\\
1.17772944323608	0\\
1.17782944573614	0\\
1.17792944823621	0\\
1.17802945073627	0\\
1.17812945323633	0\\
1.17822945573639	0\\
1.17832945823646	0\\
1.17842946073652	0\\
1.17852946323658	0\\
1.17862946573664	0\\
1.17872946823671	0\\
1.17882947073677	0\\
1.17892947323683	0\\
1.17902947573689	0\\
1.17912947823696	0\\
1.17922948073702	0\\
1.17932948323708	0\\
1.17942948573714	0\\
1.17952948823721	0\\
1.17962949073727	0\\
1.17972949323733	0\\
1.17982949573739	0\\
1.17992949823746	0\\
1.18002950073752	0\\
1.18012950323758	0\\
1.18022950573764	0\\
1.18032950823771	0\\
1.18042951073777	0\\
1.18052951323783	0\\
1.18062951573789	0\\
1.18072951823796	0\\
1.18082952073802	0\\
1.18092952323808	0\\
1.18102952573814	0\\
1.18112952823821	0\\
1.18122953073827	0\\
1.18132953323833	0\\
1.18142953573839	0\\
1.18152953823846	0\\
1.18162954073852	0\\
1.18172954323858	0\\
1.18182954573864	0\\
1.18192954823871	0\\
1.18202955073877	0\\
1.18212955323883	0\\
1.18222955573889	0\\
1.18232955823896	0\\
1.18242956073902	0\\
1.18252956323908	0\\
1.18262956573914	0\\
1.18272956823921	0\\
1.18282957073927	0\\
1.18292957323933	0\\
1.18302957573939	0\\
1.18312957823946	0\\
1.18322958073952	0\\
1.18332958323958	0\\
1.18342958573964	0\\
1.18352958823971	0\\
1.18362959073977	0\\
1.18372959323983	0\\
1.18382959573989	0\\
1.18392959823996	0\\
1.18402960074002	0\\
1.18412960324008	0\\
1.18422960574014	0\\
1.18432960824021	0\\
1.18442961074027	0\\
1.18452961324033	0\\
1.18462961574039	0\\
1.18472961824046	0\\
1.18482962074052	0\\
1.18492962324058	0\\
1.18502962574064	0\\
1.18512962824071	0\\
1.18522963074077	0\\
1.18532963324083	0\\
1.18542963574089	0\\
1.18552963824096	0\\
1.18562964074102	0\\
1.18572964324108	0\\
1.18582964574114	0\\
1.18592964824121	0\\
1.18602965074127	0\\
1.18612965324133	0\\
1.18622965574139	0\\
1.18632965824146	0\\
1.18642966074152	0\\
1.18652966324158	0\\
1.18662966574164	0\\
1.18672966824171	0\\
1.18682967074177	0\\
1.18692967324183	0\\
1.18702967574189	0\\
1.18712967824196	0\\
1.18722968074202	0\\
1.18732968324208	0\\
1.18742968574214	0\\
1.18752968824221	0\\
1.18762969074227	0\\
1.18772969324233	0\\
1.18782969574239	0\\
1.18792969824246	0\\
1.18802970074252	0\\
1.18812970324258	0\\
1.18822970574264	0\\
1.18832970824271	0\\
1.18842971074277	0\\
1.18852971324283	0\\
1.18862971574289	0\\
1.18872971824296	0\\
1.18882972074302	0\\
1.18892972324308	0\\
1.18902972574314	0\\
1.18912972824321	0\\
1.18922973074327	0\\
1.18932973324333	0\\
1.18942973574339	0\\
1.18952973824346	0\\
1.18962974074352	0\\
1.18972974324358	0\\
1.18982974574364	0\\
1.18992974824371	0\\
1.19002975074377	0\\
1.19012975324383	0\\
1.19022975574389	0\\
1.19032975824396	0\\
1.19042976074402	0\\
1.19052976324408	0\\
1.19062976574414	0\\
1.19072976824421	0\\
1.19082977074427	0\\
1.19092977324433	0\\
1.19102977574439	0\\
1.19112977824446	0\\
1.19122978074452	0\\
1.19132978324458	0\\
1.19142978574464	0\\
1.19152978824471	0\\
1.19162979074477	0\\
1.19172979324483	0\\
1.19182979574489	0\\
1.19192979824496	0\\
1.19202980074502	0\\
1.19212980324508	0\\
1.19222980574514	0\\
1.19232980824521	0\\
1.19242981074527	0\\
1.19252981324533	0\\
1.19262981574539	0\\
1.19272981824546	0\\
1.19282982074552	0\\
1.19292982324558	0\\
1.19302982574564	0\\
1.19312982824571	0\\
1.19322983074577	0\\
1.19332983324583	0\\
1.19342983574589	0\\
1.19352983824596	0\\
1.19362984074602	0\\
1.19372984324608	0\\
1.19382984574614	0\\
1.19392984824621	0\\
1.19402985074627	0\\
1.19412985324633	0\\
1.19422985574639	0\\
1.19432985824646	0\\
1.19442986074652	0\\
1.19452986324658	0\\
1.19462986574664	0\\
1.19472986824671	0\\
1.19482987074677	0\\
1.19492987324683	0\\
1.19502987574689	0\\
1.19512987824696	0\\
1.19522988074702	0\\
1.19532988324708	0\\
1.19542988574714	0\\
1.19552988824721	0\\
1.19562989074727	0\\
1.19572989324733	0\\
1.19582989574739	0\\
1.19592989824746	0\\
1.19602990074752	0\\
1.19612990324758	0\\
1.19622990574764	0\\
1.19632990824771	0\\
1.19642991074777	0\\
1.19652991324783	0\\
1.19662991574789	0\\
1.19672991824796	0\\
1.19682992074802	0\\
1.19692992324808	0\\
1.19702992574814	0\\
1.19712992824821	0\\
1.19722993074827	0\\
1.19732993324833	0\\
1.19742993574839	0\\
1.19752993824846	0\\
1.19762994074852	0\\
1.19772994324858	0\\
1.19782994574864	0\\
1.19792994824871	0\\
1.19802995074877	0\\
1.19812995324883	0\\
1.19822995574889	0\\
1.19832995824896	0\\
1.19842996074902	0\\
1.19852996324908	0\\
1.19862996574914	0\\
1.19872996824921	0\\
1.19882997074927	0\\
1.19892997324933	0\\
1.19902997574939	0\\
1.19912997824946	0\\
1.19922998074952	0\\
1.19932998324958	0\\
1.19942998574964	0\\
1.19952998824971	0\\
1.19962999074977	0\\
1.19972999324983	0\\
1.19982999574989	0\\
1.19992999824996	0\\
1.20003000075002	0\\
};
\addplot [color=mycolor2,solid,forget plot]
  table[row sep=crcr]{%
1.20003000075002	0\\
1.20013000325008	0\\
1.20023000575014	0\\
1.20033000825021	0\\
1.20043001075027	0\\
1.20053001325033	0\\
1.20063001575039	0\\
1.20073001825046	0\\
1.20083002075052	0\\
1.20093002325058	0\\
1.20103002575064	0\\
1.20113002825071	0\\
1.20123003075077	0\\
1.20133003325083	0\\
1.20143003575089	0\\
1.20153003825096	0\\
1.20163004075102	0\\
1.20173004325108	0\\
1.20183004575114	0\\
1.20193004825121	0\\
1.20203005075127	0\\
1.20213005325133	0\\
1.20223005575139	0\\
1.20233005825146	0\\
1.20243006075152	0\\
1.20253006325158	0\\
1.20263006575164	0\\
1.20273006825171	0\\
1.20283007075177	0\\
1.20293007325183	0\\
1.20303007575189	0\\
1.20313007825196	0\\
1.20323008075202	0\\
1.20333008325208	0\\
1.20343008575214	0\\
1.20353008825221	0\\
1.20363009075227	0\\
1.20373009325233	0\\
1.20383009575239	0\\
1.20393009825246	0\\
1.20403010075252	0\\
1.20413010325258	0\\
1.20423010575264	0\\
1.20433010825271	0\\
1.20443011075277	0\\
1.20453011325283	0\\
1.20463011575289	0\\
1.20473011825296	0\\
1.20483012075302	0\\
1.20493012325308	0\\
1.20503012575314	0\\
1.20513012825321	0\\
1.20523013075327	0\\
1.20533013325333	0\\
1.20543013575339	0\\
1.20553013825346	0\\
1.20563014075352	0\\
1.20573014325358	0\\
1.20583014575364	0\\
1.20593014825371	0\\
1.20603015075377	0\\
1.20613015325383	0\\
1.20623015575389	0\\
1.20633015825396	0\\
1.20643016075402	0\\
1.20653016325408	0\\
1.20663016575414	0\\
1.20673016825421	0\\
1.20683017075427	0\\
1.20693017325433	0\\
1.20703017575439	0\\
1.20713017825446	0\\
1.20723018075452	0\\
1.20733018325458	0\\
1.20743018575464	0\\
1.20753018825471	0\\
1.20763019075477	0\\
1.20773019325483	0\\
1.20783019575489	0\\
1.20793019825496	0\\
1.20803020075502	0\\
1.20813020325508	0\\
1.20823020575514	0\\
1.20833020825521	0\\
1.20843021075527	0\\
1.20853021325533	0\\
1.20863021575539	0\\
1.20873021825546	0\\
1.20883022075552	0\\
1.20893022325558	0\\
1.20903022575564	0\\
1.20913022825571	0\\
1.20923023075577	0\\
1.20933023325583	0\\
1.20943023575589	0\\
1.20953023825596	0\\
1.20963024075602	0\\
1.20973024325608	0\\
1.20983024575614	0\\
1.20993024825621	0\\
1.21003025075627	0\\
1.21013025325633	0\\
1.21023025575639	0\\
1.21033025825646	0\\
1.21043026075652	0\\
1.21053026325658	0\\
1.21063026575664	0\\
1.21073026825671	0\\
1.21083027075677	0\\
1.21093027325683	0\\
1.21103027575689	0\\
1.21113027825696	0\\
1.21123028075702	0\\
1.21133028325708	0\\
1.21143028575714	0\\
1.21153028825721	0\\
1.21163029075727	0\\
1.21173029325733	0\\
1.21183029575739	0\\
1.21193029825746	0\\
1.21203030075752	0\\
1.21213030325758	0\\
1.21223030575764	0\\
1.21233030825771	0\\
1.21243031075777	0\\
1.21253031325783	0\\
1.21263031575789	0\\
1.21273031825796	0\\
1.21283032075802	0\\
1.21293032325808	0\\
1.21303032575814	0\\
1.21313032825821	0\\
1.21323033075827	0\\
1.21333033325833	0\\
1.21343033575839	0\\
1.21353033825846	0\\
1.21363034075852	0\\
1.21373034325858	0\\
1.21383034575864	0\\
1.21393034825871	0\\
1.21403035075877	0\\
1.21413035325883	0\\
1.21423035575889	0\\
1.21433035825896	0\\
1.21443036075902	0\\
1.21453036325908	0\\
1.21463036575914	0\\
1.21473036825921	0\\
1.21483037075927	0\\
1.21493037325933	0\\
1.21503037575939	0\\
1.21513037825946	0\\
1.21523038075952	0\\
1.21533038325958	0\\
1.21543038575964	0\\
1.21553038825971	0\\
1.21563039075977	0\\
1.21573039325983	0\\
1.21583039575989	0\\
1.21593039825996	0\\
1.21603040076002	0\\
1.21613040326008	0\\
1.21623040576014	0\\
1.21633040826021	0\\
1.21643041076027	0\\
1.21653041326033	0\\
1.21663041576039	0\\
1.21673041826046	0\\
1.21683042076052	0\\
1.21693042326058	0\\
1.21703042576064	0\\
1.21713042826071	0\\
1.21723043076077	0\\
1.21733043326083	0\\
1.21743043576089	0\\
1.21753043826096	0\\
1.21763044076102	0\\
1.21773044326108	0\\
1.21783044576114	0\\
1.21793044826121	0\\
1.21803045076127	0\\
1.21813045326133	0\\
1.21823045576139	0\\
1.21833045826146	0\\
1.21843046076152	0\\
1.21853046326158	0\\
1.21863046576164	0\\
1.21873046826171	0\\
1.21883047076177	0\\
1.21893047326183	0\\
1.21903047576189	0\\
1.21913047826196	0\\
1.21923048076202	0\\
1.21933048326208	0\\
1.21943048576214	0\\
1.21953048826221	0\\
1.21963049076227	0\\
1.21973049326233	0\\
1.21983049576239	0\\
1.21993049826246	0\\
1.22003050076252	0\\
1.22013050326258	0\\
1.22023050576264	0\\
1.22033050826271	0\\
1.22043051076277	0\\
1.22053051326283	0\\
1.22063051576289	0\\
1.22073051826296	0\\
1.22083052076302	0\\
1.22093052326308	0\\
1.22103052576314	0\\
1.22113052826321	0\\
1.22123053076327	0\\
1.22133053326333	0\\
1.22143053576339	0\\
1.22153053826346	0\\
1.22163054076352	0\\
1.22173054326358	0\\
1.22183054576364	0\\
1.22193054826371	0\\
1.22203055076377	0\\
1.22213055326383	0\\
1.22223055576389	0\\
1.22233055826396	0\\
1.22243056076402	0\\
1.22253056326408	0\\
1.22263056576414	0\\
1.22273056826421	0\\
1.22283057076427	0\\
1.22293057326433	0\\
1.22303057576439	0\\
1.22313057826446	0\\
1.22323058076452	0\\
1.22333058326458	0\\
1.22343058576464	0\\
1.22353058826471	0\\
1.22363059076477	0\\
1.22373059326483	0\\
1.22383059576489	0\\
1.22393059826496	0\\
1.22403060076502	0\\
1.22413060326508	0\\
1.22423060576514	0\\
1.22433060826521	0\\
1.22443061076527	0\\
1.22453061326533	0\\
1.22463061576539	0\\
1.22473061826546	0\\
1.22483062076552	0\\
1.22493062326558	0\\
1.22503062576564	0\\
1.22513062826571	0\\
1.22523063076577	0\\
1.22533063326583	0\\
1.22543063576589	0\\
1.22553063826596	0\\
1.22563064076602	0\\
1.22573064326608	0\\
1.22583064576614	0\\
1.22593064826621	0\\
1.22603065076627	0\\
1.22613065326633	0\\
1.22623065576639	0\\
1.22633065826646	0\\
1.22643066076652	0\\
1.22653066326658	0\\
1.22663066576664	0\\
1.22673066826671	0\\
1.22683067076677	0\\
1.22693067326683	0\\
1.22703067576689	0\\
1.22713067826696	0\\
1.22723068076702	0\\
1.22733068326708	0\\
1.22743068576714	0\\
1.22753068826721	0\\
1.22763069076727	0\\
1.22773069326733	0\\
1.22783069576739	0\\
1.22793069826746	0\\
1.22803070076752	0\\
1.22813070326758	0\\
1.22823070576764	0\\
1.22833070826771	0\\
1.22843071076777	0\\
1.22853071326783	0\\
1.22863071576789	0\\
1.22873071826796	0\\
1.22883072076802	0\\
1.22893072326808	0\\
1.22903072576814	0\\
1.22913072826821	0\\
1.22923073076827	0\\
1.22933073326833	0\\
1.22943073576839	0\\
1.22953073826846	0\\
1.22963074076852	0\\
1.22973074326858	0\\
1.22983074576864	0\\
1.22993074826871	0\\
1.23003075076877	0\\
1.23013075326883	0\\
1.23023075576889	0\\
1.23033075826896	0\\
1.23043076076902	0\\
1.23053076326908	0\\
1.23063076576914	0\\
1.23073076826921	0\\
1.23083077076927	0\\
1.23093077326933	0\\
1.23103077576939	0\\
1.23113077826946	0\\
1.23123078076952	0\\
1.23133078326958	0\\
1.23143078576964	0\\
1.23153078826971	0\\
1.23163079076977	0\\
1.23173079326983	0\\
1.23183079576989	0\\
1.23193079826996	0\\
1.23203080077002	0\\
1.23213080327008	0\\
1.23223080577014	0\\
1.23233080827021	0\\
1.23243081077027	0\\
1.23253081327033	0\\
1.23263081577039	0\\
1.23273081827046	0\\
1.23283082077052	0\\
1.23293082327058	0\\
1.23303082577064	0\\
1.23313082827071	0\\
1.23323083077077	0\\
1.23333083327083	0\\
1.23343083577089	0\\
1.23353083827096	0\\
1.23363084077102	0\\
1.23373084327108	0\\
1.23383084577114	0\\
1.23393084827121	0\\
1.23403085077127	0\\
1.23413085327133	0\\
1.23423085577139	0\\
1.23433085827146	0\\
1.23443086077152	0\\
1.23453086327158	0\\
1.23463086577164	0\\
1.23473086827171	0\\
1.23483087077177	0\\
1.23493087327183	0\\
1.23503087577189	0\\
1.23513087827196	0\\
1.23523088077202	0\\
1.23533088327208	0\\
1.23543088577214	0\\
1.23553088827221	0\\
1.23563089077227	0\\
1.23573089327233	0\\
1.23583089577239	0\\
1.23593089827246	0\\
1.23603090077252	0\\
1.23613090327258	0\\
1.23623090577264	0\\
1.23633090827271	0\\
1.23643091077277	0\\
1.23653091327283	0\\
1.23663091577289	0\\
1.23673091827296	0\\
1.23683092077302	0\\
1.23693092327308	0\\
1.23703092577314	0\\
1.23713092827321	0\\
1.23723093077327	0\\
1.23733093327333	0\\
1.23743093577339	0\\
1.23753093827346	0\\
1.23763094077352	0\\
1.23773094327358	0\\
1.23783094577364	0\\
1.23793094827371	0\\
1.23803095077377	0\\
1.23813095327383	0\\
1.23823095577389	0\\
1.23833095827396	0\\
1.23843096077402	0\\
1.23853096327408	0\\
1.23863096577414	0\\
1.23873096827421	0\\
1.23883097077427	0\\
1.23893097327433	0\\
1.23903097577439	0\\
1.23913097827446	0\\
1.23923098077452	0\\
1.23933098327458	0\\
1.23943098577464	0\\
1.23953098827471	0\\
1.23963099077477	0\\
1.23973099327483	0\\
1.23983099577489	0\\
1.23993099827496	0\\
1.24003100077502	0\\
1.24013100327508	0\\
1.24023100577514	0\\
1.24033100827521	0\\
1.24043101077527	0\\
1.24053101327533	0\\
1.24063101577539	0\\
1.24073101827546	0\\
1.24083102077552	0\\
1.24093102327558	0\\
1.24103102577564	0\\
1.24113102827571	0\\
1.24123103077577	0\\
1.24133103327583	0\\
1.24143103577589	0\\
1.24153103827596	0\\
1.24163104077602	0\\
1.24173104327608	0\\
1.24183104577614	0\\
1.24193104827621	0\\
1.24203105077627	0\\
1.24213105327633	0\\
1.24223105577639	0\\
1.24233105827646	0\\
1.24243106077652	0\\
1.24253106327658	0\\
1.24263106577664	0\\
1.24273106827671	0\\
1.24283107077677	0\\
1.24293107327683	0\\
1.24303107577689	0\\
1.24313107827696	0\\
1.24323108077702	0\\
1.24333108327708	0\\
1.24343108577714	0\\
1.24353108827721	0\\
1.24363109077727	0\\
1.24373109327733	0\\
1.24383109577739	0\\
1.24393109827746	0\\
1.24403110077752	0\\
1.24413110327758	0\\
1.24423110577764	0\\
1.24433110827771	0\\
1.24443111077777	0\\
1.24453111327783	0\\
1.24463111577789	0\\
1.24473111827796	0\\
1.24483112077802	0\\
1.24493112327808	0\\
1.24503112577814	0\\
1.24513112827821	0\\
1.24523113077827	0\\
1.24533113327833	0\\
1.24543113577839	0\\
1.24553113827846	0\\
1.24563114077852	0\\
1.24573114327858	0\\
1.24583114577864	0\\
1.24593114827871	0\\
1.24603115077877	0\\
1.24613115327883	0\\
1.24623115577889	0\\
1.24633115827896	0\\
1.24643116077902	0\\
1.24653116327908	0\\
1.24663116577914	0\\
1.24673116827921	0\\
1.24683117077927	0\\
1.24693117327933	0\\
1.24703117577939	0\\
1.24713117827946	0\\
1.24723118077952	0\\
1.24733118327958	0\\
1.24743118577964	0\\
1.24753118827971	0\\
1.24763119077977	0\\
1.24773119327983	0\\
1.24783119577989	0\\
1.24793119827996	0\\
1.24803120078002	0\\
1.24813120328008	0\\
1.24823120578014	0\\
1.24833120828021	0\\
1.24843121078027	0\\
1.24853121328033	0\\
1.24863121578039	0\\
1.24873121828046	0\\
1.24883122078052	0\\
1.24893122328058	0\\
1.24903122578064	0\\
1.24913122828071	0\\
1.24923123078077	0\\
1.24933123328083	0\\
1.24943123578089	0\\
1.24953123828096	0\\
1.24963124078102	0\\
1.24973124328108	0\\
1.24983124578114	0\\
1.24993124828121	0\\
1.25003125078127	0\\
1.25013125328133	0\\
1.25023125578139	0\\
1.25033125828146	0\\
1.25043126078152	0\\
1.25053126328158	0\\
1.25063126578164	0\\
1.25073126828171	0\\
1.25083127078177	0\\
1.25093127328183	0\\
1.25103127578189	0\\
1.25113127828196	0\\
1.25123128078202	0\\
1.25133128328208	0\\
1.25143128578214	0\\
1.25153128828221	0\\
1.25163129078227	0\\
1.25173129328233	0\\
1.25183129578239	0\\
1.25193129828246	0\\
1.25203130078252	0\\
1.25213130328258	0\\
1.25223130578264	0\\
1.25233130828271	0\\
1.25243131078277	0\\
1.25253131328283	0\\
1.25263131578289	0\\
1.25273131828296	0\\
1.25283132078302	0\\
1.25293132328308	0\\
1.25303132578314	0\\
1.25313132828321	0\\
1.25323133078327	0\\
1.25333133328333	0\\
1.25343133578339	0\\
1.25353133828346	0\\
1.25363134078352	0\\
1.25373134328358	0\\
1.25383134578364	0\\
1.25393134828371	0\\
1.25403135078377	0\\
1.25413135328383	0\\
1.25423135578389	0\\
1.25433135828396	0\\
1.25443136078402	0\\
1.25453136328408	0\\
1.25463136578414	0\\
1.25473136828421	0\\
1.25483137078427	0\\
1.25493137328433	0\\
1.25503137578439	0\\
1.25513137828446	0\\
1.25523138078452	0\\
1.25533138328458	0\\
1.25543138578464	0\\
1.25553138828471	0\\
1.25563139078477	0\\
1.25573139328483	0\\
1.25583139578489	0\\
1.25593139828496	0\\
1.25603140078502	0\\
1.25613140328508	0\\
1.25623140578514	0\\
1.25633140828521	0\\
1.25643141078527	0\\
1.25653141328533	0\\
1.25663141578539	0\\
1.25673141828546	0\\
1.25683142078552	0\\
1.25693142328558	0\\
1.25703142578564	0\\
1.25713142828571	0\\
1.25723143078577	0\\
1.25733143328583	0\\
1.25743143578589	0\\
1.25753143828596	0\\
1.25763144078602	0\\
1.25773144328608	0\\
1.25783144578614	0\\
1.25793144828621	0\\
1.25803145078627	0\\
1.25813145328633	0\\
1.25823145578639	0\\
1.25833145828646	0\\
1.25843146078652	0\\
1.25853146328658	0\\
1.25863146578664	0\\
1.25873146828671	0\\
1.25883147078677	0\\
1.25893147328683	0\\
1.25903147578689	0\\
1.25913147828696	0\\
1.25923148078702	0\\
1.25933148328708	0\\
1.25943148578714	0\\
1.25953148828721	0\\
1.25963149078727	0\\
1.25973149328733	0\\
1.25983149578739	0\\
1.25993149828746	0\\
1.26003150078752	0\\
1.26013150328758	0\\
1.26023150578764	0\\
1.26033150828771	0\\
1.26043151078777	0\\
1.26053151328783	0\\
1.26063151578789	0\\
1.26073151828796	0\\
1.26083152078802	0\\
1.26093152328808	0\\
1.26103152578814	0\\
1.26113152828821	0\\
1.26123153078827	0\\
1.26133153328833	0\\
1.26143153578839	0\\
1.26153153828846	0\\
1.26163154078852	0\\
1.26173154328858	0\\
1.26183154578864	0\\
1.26193154828871	0\\
1.26203155078877	0\\
1.26213155328883	0\\
1.26223155578889	0\\
1.26233155828896	0\\
1.26243156078902	0\\
1.26253156328908	0\\
1.26263156578914	0\\
1.26273156828921	0\\
1.26283157078927	0\\
1.26293157328933	0\\
1.26303157578939	0\\
1.26313157828946	0\\
1.26323158078952	0\\
1.26333158328958	0\\
1.26343158578964	0\\
1.26353158828971	0\\
1.26363159078977	0\\
1.26373159328983	0\\
1.26383159578989	0\\
1.26393159828996	0\\
1.26403160079002	0\\
1.26413160329008	0\\
1.26423160579014	0\\
1.26433160829021	0\\
1.26443161079027	0\\
1.26453161329033	0\\
1.26463161579039	0\\
1.26473161829046	0\\
1.26483162079052	0\\
1.26493162329058	0\\
1.26503162579064	0\\
1.26513162829071	0\\
1.26523163079077	0\\
1.26533163329083	0\\
1.26543163579089	0\\
1.26553163829096	0\\
1.26563164079102	0\\
1.26573164329108	0\\
1.26583164579114	0\\
1.26593164829121	0\\
1.26603165079127	0\\
1.26613165329133	0\\
1.26623165579139	0\\
1.26633165829146	0\\
1.26643166079152	0\\
1.26653166329158	0\\
1.26663166579164	0\\
1.26673166829171	0\\
1.26683167079177	0\\
1.26693167329183	0\\
1.26703167579189	0\\
1.26713167829196	0\\
1.26723168079202	0\\
1.26733168329208	0\\
1.26743168579214	0\\
1.26753168829221	0\\
1.26763169079227	0\\
1.26773169329233	0\\
1.26783169579239	0\\
1.26793169829246	0\\
1.26803170079252	0\\
1.26813170329258	0\\
1.26823170579264	0\\
1.26833170829271	0\\
1.26843171079277	0\\
1.26853171329283	0\\
1.26863171579289	0\\
1.26873171829296	0\\
1.26883172079302	0\\
1.26893172329308	0\\
1.26903172579314	0\\
1.26913172829321	0\\
1.26923173079327	0\\
1.26933173329333	0\\
1.26943173579339	0\\
1.26953173829346	0\\
1.26963174079352	0\\
1.26973174329358	0\\
1.26983174579364	0\\
1.26993174829371	0\\
1.27003175079377	0\\
1.27013175329383	0\\
1.27023175579389	0\\
1.27033175829396	0\\
1.27043176079402	0\\
1.27053176329408	0\\
1.27063176579414	0\\
1.27073176829421	0\\
1.27083177079427	0\\
1.27093177329433	0\\
1.27103177579439	0\\
1.27113177829446	0\\
1.27123178079452	0\\
1.27133178329458	0\\
1.27143178579464	0\\
1.27153178829471	0\\
1.27163179079477	0\\
1.27173179329483	0\\
1.27183179579489	0\\
1.27193179829496	0\\
1.27203180079502	0\\
1.27213180329508	0\\
1.27223180579514	0\\
1.27233180829521	0\\
1.27243181079527	0\\
1.27253181329533	0\\
1.27263181579539	0\\
1.27273181829546	0\\
1.27283182079552	0\\
1.27293182329558	0\\
1.27303182579565	0\\
1.27313182829571	0\\
1.27323183079577	0\\
1.27333183329583	0\\
1.27343183579589	0\\
1.27353183829596	0\\
1.27363184079602	0\\
1.27373184329608	0\\
1.27383184579614	0\\
1.27393184829621	0\\
1.27403185079627	0\\
1.27413185329633	0\\
1.27423185579639	0\\
1.27433185829646	0\\
1.27443186079652	0\\
1.27453186329658	0\\
1.27463186579664	0\\
1.27473186829671	0\\
1.27483187079677	0\\
1.27493187329683	0\\
1.2750318757969	0\\
1.27513187829696	0\\
1.27523188079702	0\\
1.27533188329708	0\\
1.27543188579714	0\\
1.27553188829721	0\\
1.27563189079727	0\\
1.27573189329733	0\\
1.27583189579739	0\\
1.27593189829746	0\\
1.27603190079752	0\\
1.27613190329758	0\\
1.27623190579765	0\\
1.27633190829771	0\\
1.27643191079777	0\\
1.27653191329783	0\\
1.27663191579789	0\\
1.27673191829796	0\\
1.27683192079802	0\\
1.27693192329808	0\\
1.27703192579815	0\\
1.27713192829821	0\\
1.27723193079827	0\\
1.27733193329833	0\\
1.27743193579839	0\\
1.27753193829846	0\\
1.27763194079852	0\\
1.27773194329858	0\\
1.27783194579864	0\\
1.27793194829871	0\\
1.27803195079877	0\\
1.27813195329883	0\\
1.2782319557989	0\\
1.27833195829896	0\\
1.27843196079902	0\\
1.27853196329908	0\\
1.27863196579914	0\\
1.27873196829921	0\\
1.27883197079927	0\\
1.27893197329933	0\\
1.2790319757994	0\\
1.27913197829946	0\\
1.27923198079952	0\\
1.27933198329958	0\\
1.27943198579965	0\\
1.27953198829971	0\\
1.27963199079977	0\\
1.27973199329983	0\\
1.27983199579989	0\\
1.27993199829996	0\\
1.28003200080002	0\\
1.28013200330008	0\\
1.28023200580015	0\\
1.28033200830021	0\\
1.28043201080027	0\\
1.28053201330033	0\\
1.28063201580039	0\\
1.28073201830046	0\\
1.28083202080052	0\\
1.28093202330058	0\\
1.28103202580065	0\\
1.28113202830071	0\\
1.28123203080077	0\\
1.28133203330083	0\\
1.2814320358009	0\\
1.28153203830096	0\\
1.28163204080102	0\\
1.28173204330108	0\\
1.28183204580114	0\\
1.28193204830121	0\\
1.28203205080127	0\\
1.28213205330133	0\\
1.2822320558014	0\\
1.28233205830146	0\\
1.28243206080152	0\\
1.28253206330158	0\\
1.28263206580165	0\\
1.28273206830171	0\\
1.28283207080177	0\\
1.28293207330183	0\\
1.2830320758019	0\\
1.28313207830196	0\\
1.28323208080202	0\\
1.28333208330208	0\\
1.28343208580215	0\\
1.28353208830221	0\\
1.28363209080227	0\\
1.28373209330233	0\\
1.28383209580239	0\\
1.28393209830246	0\\
1.28403210080252	0\\
1.28413210330258	0\\
1.28423210580265	0\\
1.28433210830271	0\\
1.28443211080277	0\\
1.28453211330283	0\\
1.2846321158029	0\\
1.28473211830296	0\\
1.28483212080302	0\\
1.28493212330308	0\\
1.28503212580315	0\\
1.28513212830321	0\\
1.28523213080327	0\\
1.28533213330333	0\\
1.2854321358034	0\\
1.28553213830346	0\\
1.28563214080352	0\\
1.28573214330358	0\\
1.28583214580365	0\\
1.28593214830371	0\\
1.28603215080377	0\\
1.28613215330383	0\\
1.2862321558039	0\\
1.28633215830396	0\\
1.28643216080402	0\\
1.28653216330408	0\\
1.28663216580415	0\\
1.28673216830421	0\\
1.28683217080427	0\\
1.28693217330433	0\\
1.2870321758044	0\\
1.28713217830446	0\\
1.28723218080452	0\\
1.28733218330458	0\\
1.28743218580465	0\\
1.28753218830471	0\\
1.28763219080477	0\\
1.28773219330483	0\\
1.2878321958049	0\\
1.28793219830496	0\\
1.28803220080502	0\\
1.28813220330508	0\\
1.28823220580515	0\\
1.28833220830521	0\\
1.28843221080527	0\\
1.28853221330533	0\\
1.2886322158054	0\\
1.28873221830546	0\\
1.28883222080552	0\\
1.28893222330558	0\\
1.28903222580565	0\\
1.28913222830571	0\\
1.28923223080577	0\\
1.28933223330583	0\\
1.2894322358059	0\\
1.28953223830596	0\\
1.28963224080602	0\\
1.28973224330608	0\\
1.28983224580615	0\\
1.28993224830621	0\\
1.29003225080627	0\\
1.29013225330633	0\\
1.2902322558064	0\\
1.29033225830646	0\\
1.29043226080652	0\\
1.29053226330658	0\\
1.29063226580665	0\\
1.29073226830671	0\\
1.29083227080677	0\\
1.29093227330683	0\\
1.2910322758069	0\\
1.29113227830696	0\\
1.29123228080702	0\\
1.29133228330708	0\\
1.29143228580715	0\\
1.29153228830721	0\\
1.29163229080727	0\\
1.29173229330733	0\\
1.2918322958074	0\\
1.29193229830746	0\\
1.29203230080752	0\\
1.29213230330758	0\\
1.29223230580765	0\\
1.29233230830771	0\\
1.29243231080777	0\\
1.29253231330783	0\\
1.2926323158079	0\\
1.29273231830796	0\\
1.29283232080802	0\\
1.29293232330808	0\\
1.29303232580815	0\\
1.29313232830821	0\\
1.29323233080827	0\\
1.29333233330833	0\\
1.2934323358084	0\\
1.29353233830846	0\\
1.29363234080852	0\\
1.29373234330858	0\\
1.29383234580865	0\\
1.29393234830871	0\\
1.29403235080877	0\\
1.29413235330883	0\\
1.2942323558089	0\\
1.29433235830896	0\\
1.29443236080902	0\\
1.29453236330908	0\\
1.29463236580915	0\\
1.29473236830921	0\\
1.29483237080927	0\\
1.29493237330933	0\\
1.2950323758094	0\\
1.29513237830946	0\\
1.29523238080952	0\\
1.29533238330958	0\\
1.29543238580965	0\\
1.29553238830971	0\\
1.29563239080977	0\\
1.29573239330983	0\\
1.2958323958099	0\\
1.29593239830996	0\\
1.29603240081002	0\\
1.29613240331008	0\\
1.29623240581015	0\\
1.29633240831021	0\\
1.29643241081027	0\\
1.29653241331033	0\\
1.2966324158104	0\\
1.29673241831046	0\\
1.29683242081052	0\\
1.29693242331058	0\\
1.29703242581065	0\\
1.29713242831071	0\\
1.29723243081077	0\\
1.29733243331083	0\\
1.2974324358109	0\\
1.29753243831096	0\\
1.29763244081102	0\\
1.29773244331108	0\\
1.29783244581115	0\\
1.29793244831121	0\\
1.29803245081127	0\\
1.29813245331133	0\\
1.2982324558114	0\\
1.29833245831146	0\\
1.29843246081152	0\\
1.29853246331158	0\\
1.29863246581165	0\\
1.29873246831171	0\\
1.29883247081177	0\\
1.29893247331183	0\\
1.2990324758119	0\\
1.29913247831196	0\\
1.29923248081202	0\\
1.29933248331208	0\\
1.29943248581215	0\\
1.29953248831221	0\\
1.29963249081227	0\\
1.29973249331233	0\\
1.2998324958124	0\\
1.29993249831246	0\\
1.30003250081252	0\\
1.30013250331258	0\\
1.30023250581265	0\\
1.30033250831271	0\\
1.30043251081277	0\\
1.30053251331283	0\\
1.3006325158129	0\\
1.30073251831296	0\\
1.30083252081302	0\\
1.30093252331308	0\\
1.30103252581315	0\\
1.30113252831321	0\\
1.30123253081327	0\\
1.30133253331333	0\\
1.3014325358134	0\\
1.30153253831346	0\\
1.30163254081352	0\\
1.30173254331358	0\\
1.30183254581365	0\\
1.30193254831371	0\\
1.30203255081377	0\\
1.30213255331383	0\\
1.3022325558139	0\\
1.30233255831396	0\\
1.30243256081402	0\\
1.30253256331408	0\\
1.30263256581415	0\\
1.30273256831421	0\\
1.30283257081427	0\\
1.30293257331433	0\\
1.3030325758144	0\\
1.30313257831446	0\\
1.30323258081452	0\\
1.30333258331458	0\\
1.30343258581465	0\\
1.30353258831471	0\\
1.30363259081477	0\\
1.30373259331483	0\\
1.3038325958149	0\\
1.30393259831496	0\\
1.30403260081502	0\\
1.30413260331508	0\\
1.30423260581515	0\\
1.30433260831521	0\\
1.30443261081527	0\\
1.30453261331533	0\\
1.3046326158154	0\\
1.30473261831546	0\\
1.30483262081552	0\\
1.30493262331558	0\\
1.30503262581565	0\\
1.30513262831571	0\\
1.30523263081577	0\\
1.30533263331583	0\\
1.3054326358159	0\\
1.30553263831596	0\\
1.30563264081602	0\\
1.30573264331608	0\\
1.30583264581615	0\\
1.30593264831621	0\\
1.30603265081627	0\\
1.30613265331633	0\\
1.3062326558164	0\\
1.30633265831646	0\\
1.30643266081652	0\\
1.30653266331658	0\\
1.30663266581665	0\\
1.30673266831671	0\\
1.30683267081677	0\\
1.30693267331683	0\\
1.3070326758169	0\\
1.30713267831696	0\\
1.30723268081702	0\\
1.30733268331708	0\\
1.30743268581715	0\\
1.30753268831721	0\\
1.30763269081727	0\\
1.30773269331733	0\\
1.3078326958174	0\\
1.30793269831746	0\\
1.30803270081752	0\\
1.30813270331758	0\\
1.30823270581765	0\\
1.30833270831771	0\\
1.30843271081777	0\\
1.30853271331783	0\\
1.3086327158179	0\\
1.30873271831796	0\\
1.30883272081802	0\\
1.30893272331808	0\\
1.30903272581815	0\\
1.30913272831821	0\\
1.30923273081827	0\\
1.30933273331833	0\\
1.3094327358184	0\\
1.30953273831846	0\\
1.30963274081852	0\\
1.30973274331858	0\\
1.30983274581865	0\\
1.30993274831871	0\\
1.31003275081877	0\\
1.31013275331883	0\\
1.3102327558189	0\\
1.31033275831896	0\\
1.31043276081902	0\\
1.31053276331908	0\\
1.31063276581915	0\\
1.31073276831921	0\\
1.31083277081927	0\\
1.31093277331933	0\\
1.3110327758194	0\\
1.31113277831946	0\\
1.31123278081952	0\\
1.31133278331958	0\\
1.31143278581965	0\\
1.31153278831971	0\\
1.31163279081977	0\\
1.31173279331983	0\\
1.3118327958199	0\\
1.31193279831996	0\\
1.31203280082002	0\\
1.31213280332008	0\\
1.31223280582015	0\\
1.31233280832021	0\\
1.31243281082027	0\\
1.31253281332033	0\\
1.3126328158204	0\\
1.31273281832046	0\\
1.31283282082052	0\\
1.31293282332058	0\\
1.31303282582065	0\\
1.31313282832071	0\\
1.31323283082077	0\\
1.31333283332083	0\\
1.3134328358209	0\\
1.31353283832096	0\\
1.31363284082102	0\\
1.31373284332108	0\\
1.31383284582115	0\\
1.31393284832121	0\\
1.31403285082127	0\\
1.31413285332133	0\\
1.3142328558214	0\\
1.31433285832146	0\\
1.31443286082152	0\\
1.31453286332158	0\\
1.31463286582165	0\\
1.31473286832171	0\\
1.31483287082177	0\\
1.31493287332183	0\\
1.3150328758219	0\\
1.31513287832196	0\\
1.31523288082202	0\\
1.31533288332208	0\\
1.31543288582215	0\\
1.31553288832221	0\\
1.31563289082227	0\\
1.31573289332233	0\\
1.3158328958224	0\\
1.31593289832246	0\\
1.31603290082252	0\\
1.31613290332258	0\\
1.31623290582265	0\\
1.31633290832271	0\\
1.31643291082277	0\\
1.31653291332283	0\\
1.3166329158229	0\\
1.31673291832296	0\\
1.31683292082302	0\\
1.31693292332308	0\\
1.31703292582315	0\\
1.31713292832321	0\\
1.31723293082327	0\\
1.31733293332333	0\\
1.3174329358234	0\\
1.31753293832346	0\\
1.31763294082352	0\\
1.31773294332358	0\\
1.31783294582365	0\\
1.31793294832371	0\\
1.31803295082377	0\\
1.31813295332383	0\\
1.3182329558239	0\\
1.31833295832396	0\\
1.31843296082402	0\\
1.31853296332408	0\\
1.31863296582415	0\\
1.31873296832421	0\\
1.31883297082427	0\\
1.31893297332433	0\\
1.3190329758244	0\\
1.31913297832446	0\\
1.31923298082452	0\\
1.31933298332458	0\\
1.31943298582465	0\\
1.31953298832471	0\\
1.31963299082477	0\\
1.31973299332483	0\\
1.3198329958249	0\\
1.31993299832496	0\\
1.32003300082502	0\\
1.32013300332508	0\\
1.32023300582515	0\\
1.32033300832521	0\\
1.32043301082527	0\\
1.32053301332533	0\\
1.3206330158254	0\\
1.32073301832546	0\\
1.32083302082552	0\\
1.32093302332558	0\\
1.32103302582565	0\\
1.32113302832571	0\\
1.32123303082577	0\\
1.32133303332583	0\\
1.3214330358259	0\\
1.32153303832596	0\\
1.32163304082602	0\\
1.32173304332608	0\\
1.32183304582615	0\\
1.32193304832621	0\\
1.32203305082627	0\\
1.32213305332633	0\\
1.3222330558264	0\\
1.32233305832646	0\\
1.32243306082652	0\\
1.32253306332658	0\\
1.32263306582665	0\\
1.32273306832671	0\\
1.32283307082677	0\\
1.32293307332683	0\\
1.3230330758269	0\\
1.32313307832696	0\\
1.32323308082702	0\\
1.32333308332708	0\\
1.32343308582715	0\\
1.32353308832721	0\\
1.32363309082727	0\\
1.32373309332733	0\\
1.3238330958274	0\\
1.32393309832746	0\\
1.32403310082752	0\\
1.32413310332758	0\\
1.32423310582765	0\\
1.32433310832771	0\\
1.32443311082777	0\\
1.32453311332783	0\\
1.3246331158279	0\\
1.32473311832796	0\\
1.32483312082802	0\\
1.32493312332808	0\\
1.32503312582815	0\\
1.32513312832821	0\\
1.32523313082827	0\\
1.32533313332833	0\\
1.3254331358284	0\\
1.32553313832846	0\\
1.32563314082852	0\\
1.32573314332858	0\\
1.32583314582865	0\\
1.32593314832871	0\\
1.32603315082877	0\\
1.32613315332883	0\\
1.3262331558289	0\\
1.32633315832896	0\\
1.32643316082902	0\\
1.32653316332908	0\\
1.32663316582915	0\\
1.32673316832921	0\\
1.32683317082927	0\\
1.32693317332933	0\\
1.3270331758294	0\\
1.32713317832946	0\\
1.32723318082952	0\\
1.32733318332958	0\\
1.32743318582965	0\\
1.32753318832971	0\\
1.32763319082977	0\\
1.32773319332983	0\\
1.3278331958299	0\\
1.32793319832996	0\\
1.32803320083002	0\\
1.32813320333008	0\\
1.32823320583015	0\\
1.32833320833021	0\\
1.32843321083027	0\\
1.32853321333033	0\\
1.3286332158304	0\\
1.32873321833046	0\\
1.32883322083052	0\\
1.32893322333058	0\\
1.32903322583065	0\\
1.32913322833071	0\\
1.32923323083077	0\\
1.32933323333083	0\\
1.3294332358309	0\\
1.32953323833096	0\\
1.32963324083102	0\\
1.32973324333108	0\\
1.32983324583115	0\\
1.32993324833121	0\\
1.33003325083127	0\\
1.33013325333133	0\\
1.3302332558314	0\\
1.33033325833146	0\\
1.33043326083152	0\\
1.33053326333158	0\\
1.33063326583165	0\\
1.33073326833171	0\\
1.33083327083177	0\\
1.33093327333183	0\\
1.3310332758319	0\\
1.33113327833196	0\\
1.33123328083202	0\\
1.33133328333208	0\\
1.33143328583215	0\\
1.33153328833221	0\\
1.33163329083227	0\\
1.33173329333233	0\\
1.3318332958324	0\\
1.33193329833246	0\\
1.33203330083252	0\\
1.33213330333258	0\\
1.33223330583265	0\\
1.33233330833271	0\\
1.33243331083277	0\\
1.33253331333283	0\\
1.3326333158329	0\\
1.33273331833296	0\\
1.33283332083302	0\\
1.33293332333308	0\\
1.33303332583315	0\\
1.33313332833321	0\\
1.33323333083327	0\\
1.33333333333333	0\\
1.3334333358334	0\\
1.33353333833346	0\\
1.33363334083352	0\\
1.33373334333358	0\\
1.33383334583365	0\\
1.33393334833371	0\\
1.33403335083377	0\\
1.33413335333383	0\\
1.3342333558339	0\\
1.33433335833396	0\\
1.33443336083402	0\\
1.33453336333408	0\\
1.33463336583415	0\\
1.33473336833421	0\\
1.33483337083427	0\\
1.33493337333433	0\\
1.3350333758344	0\\
1.33513337833446	0\\
1.33523338083452	0\\
1.33533338333458	0\\
1.33543338583465	0\\
1.33553338833471	0\\
1.33563339083477	0\\
1.33573339333483	0\\
1.3358333958349	0\\
1.33593339833496	0\\
1.33603340083502	0\\
1.33613340333508	0\\
1.33623340583515	0\\
1.33633340833521	0\\
1.33643341083527	0\\
1.33653341333533	0\\
1.3366334158354	0\\
1.33673341833546	0\\
1.33683342083552	0\\
1.33693342333558	0\\
1.33703342583565	0\\
1.33713342833571	0\\
1.33723343083577	0\\
1.33733343333583	0\\
1.3374334358359	0\\
1.33753343833596	0\\
1.33763344083602	0\\
1.33773344333608	0\\
1.33783344583615	0\\
1.33793344833621	0\\
1.33803345083627	0\\
1.33813345333633	0\\
1.3382334558364	0\\
1.33833345833646	0\\
1.33843346083652	0\\
1.33853346333658	0\\
1.33863346583665	0\\
1.33873346833671	0\\
1.33883347083677	0\\
1.33893347333683	0\\
1.3390334758369	0\\
1.33913347833696	0\\
1.33923348083702	0\\
1.33933348333708	0\\
1.33943348583715	0\\
1.33953348833721	0\\
1.33963349083727	0\\
1.33973349333733	0\\
1.3398334958374	0\\
1.33993349833746	0\\
1.34003350083752	0\\
1.34013350333758	0\\
1.34023350583765	0\\
1.34033350833771	0\\
1.34043351083777	0\\
1.34053351333783	0\\
1.3406335158379	0\\
1.34073351833796	0\\
1.34083352083802	0\\
1.34093352333808	0\\
1.34103352583815	0\\
1.34113352833821	0\\
1.34123353083827	0\\
1.34133353333833	0\\
1.3414335358384	0\\
1.34153353833846	0\\
1.34163354083852	0\\
1.34173354333858	0\\
1.34183354583865	0\\
1.34193354833871	0\\
1.34203355083877	0\\
1.34213355333883	0\\
1.3422335558389	0\\
1.34233355833896	0\\
1.34243356083902	0\\
1.34253356333908	0\\
1.34263356583915	0\\
1.34273356833921	0\\
1.34283357083927	0\\
1.34293357333933	0\\
1.3430335758394	0\\
1.34313357833946	0\\
1.34323358083952	0\\
1.34333358333958	0\\
1.34343358583965	0\\
1.34353358833971	0\\
1.34363359083977	0\\
1.34373359333983	0\\
1.3438335958399	0\\
1.34393359833996	0\\
1.34403360084002	0\\
1.34413360334008	0\\
1.34423360584015	0\\
1.34433360834021	0\\
1.34443361084027	0\\
1.34453361334033	0\\
1.3446336158404	0\\
1.34473361834046	0\\
1.34483362084052	0\\
1.34493362334058	0\\
1.34503362584065	0\\
1.34513362834071	0\\
1.34523363084077	0\\
1.34533363334083	0\\
1.3454336358409	0\\
1.34553363834096	0\\
1.34563364084102	0\\
1.34573364334108	0\\
1.34583364584115	0\\
1.34593364834121	0\\
1.34603365084127	0\\
1.34613365334133	0\\
1.3462336558414	0\\
1.34633365834146	0\\
1.34643366084152	0\\
1.34653366334158	0\\
1.34663366584165	0\\
1.34673366834171	0\\
1.34683367084177	0\\
1.34693367334183	0\\
1.3470336758419	0\\
1.34713367834196	0\\
1.34723368084202	0\\
1.34733368334208	0\\
1.34743368584215	0\\
1.34753368834221	0\\
1.34763369084227	0\\
1.34773369334233	0\\
1.3478336958424	0\\
1.34793369834246	0\\
1.34803370084252	0\\
1.34813370334258	0\\
1.34823370584265	0\\
1.34833370834271	0\\
1.34843371084277	0\\
1.34853371334283	0\\
1.3486337158429	0\\
1.34873371834296	0\\
1.34883372084302	0\\
1.34893372334308	0\\
1.34903372584315	0\\
1.34913372834321	0\\
1.34923373084327	0\\
1.34933373334333	0\\
1.3494337358434	0\\
1.34953373834346	0\\
1.34963374084352	0\\
1.34973374334358	0\\
1.34983374584365	0\\
1.34993374834371	0\\
1.35003375084377	0\\
1.35013375334383	0\\
1.3502337558439	0\\
1.35033375834396	0\\
1.35043376084402	0\\
1.35053376334408	0\\
1.35063376584415	0\\
1.35073376834421	0\\
1.35083377084427	0\\
1.35093377334433	0\\
1.3510337758444	0\\
1.35113377834446	0\\
1.35123378084452	0\\
1.35133378334458	0\\
1.35143378584465	0\\
1.35153378834471	0\\
1.35163379084477	0\\
1.35173379334483	0\\
1.3518337958449	0\\
1.35193379834496	0\\
1.35203380084502	0\\
1.35213380334508	0\\
1.35223380584515	0\\
1.35233380834521	0\\
1.35243381084527	0\\
1.35253381334533	0\\
1.3526338158454	0\\
1.35273381834546	0\\
1.35283382084552	0\\
1.35293382334558	0\\
1.35303382584565	0\\
1.35313382834571	0\\
1.35323383084577	0\\
1.35333383334583	0\\
1.3534338358459	0\\
1.35353383834596	0\\
1.35363384084602	0\\
1.35373384334608	0\\
1.35383384584615	0\\
1.35393384834621	0\\
1.35403385084627	0\\
1.35413385334633	0\\
1.3542338558464	0\\
1.35433385834646	0\\
1.35443386084652	0\\
1.35453386334658	0\\
1.35463386584665	0\\
1.35473386834671	0\\
1.35483387084677	0\\
1.35493387334683	0\\
1.3550338758469	0\\
1.35513387834696	0\\
1.35523388084702	0\\
1.35533388334708	0\\
1.35543388584715	0\\
1.35553388834721	0\\
1.35563389084727	0\\
1.35573389334733	0\\
1.3558338958474	0\\
1.35593389834746	0\\
1.35603390084752	0\\
1.35613390334758	0\\
1.35623390584765	0\\
1.35633390834771	0\\
1.35643391084777	0\\
1.35653391334783	0\\
1.3566339158479	0\\
1.35673391834796	0\\
1.35683392084802	0\\
1.35693392334808	0\\
1.35703392584815	0\\
1.35713392834821	0\\
1.35723393084827	0\\
1.35733393334833	0\\
1.3574339358484	0\\
1.35753393834846	0\\
1.35763394084852	0\\
1.35773394334858	0\\
1.35783394584865	0\\
1.35793394834871	0\\
1.35803395084877	0\\
1.35813395334883	0\\
1.3582339558489	0\\
1.35833395834896	0\\
1.35843396084902	0\\
1.35853396334908	0\\
1.35863396584915	0\\
1.35873396834921	0\\
1.35883397084927	0\\
1.35893397334933	0\\
1.3590339758494	0\\
1.35913397834946	0\\
1.35923398084952	0\\
1.35933398334958	0\\
1.35943398584965	0\\
1.35953398834971	0\\
1.35963399084977	0\\
1.35973399334983	0\\
1.3598339958499	0\\
1.35993399834996	0\\
1.36003400085002	0\\
1.36013400335008	0\\
1.36023400585015	0\\
1.36033400835021	0\\
1.36043401085027	0\\
1.36053401335033	0\\
1.3606340158504	0\\
1.36073401835046	0\\
1.36083402085052	0\\
1.36093402335058	0\\
1.36103402585065	0\\
1.36113402835071	0\\
1.36123403085077	0\\
1.36133403335083	0\\
1.3614340358509	0\\
1.36153403835096	0\\
1.36163404085102	0\\
1.36173404335108	0\\
1.36183404585115	0\\
1.36193404835121	0\\
1.36203405085127	0\\
1.36213405335133	0\\
1.3622340558514	0\\
1.36233405835146	0\\
1.36243406085152	0\\
1.36253406335158	0\\
1.36263406585165	0\\
1.36273406835171	0\\
1.36283407085177	0\\
1.36293407335183	0\\
1.3630340758519	0\\
1.36313407835196	0\\
1.36323408085202	0\\
1.36333408335208	0\\
1.36343408585215	0\\
1.36353408835221	0\\
1.36363409085227	0\\
1.36373409335233	0\\
1.3638340958524	0\\
1.36393409835246	0\\
1.36403410085252	0\\
1.36413410335258	0\\
1.36423410585265	0\\
1.36433410835271	0\\
1.36443411085277	0\\
1.36453411335283	0\\
1.3646341158529	0\\
1.36473411835296	0\\
1.36483412085302	0\\
1.36493412335308	0\\
1.36503412585315	0\\
1.36513412835321	0\\
1.36523413085327	0\\
1.36533413335333	0\\
1.3654341358534	0\\
1.36553413835346	0\\
1.36563414085352	0\\
1.36573414335358	0\\
1.36583414585365	0\\
1.36593414835371	0\\
1.36603415085377	0\\
1.36613415335383	0\\
1.3662341558539	0\\
1.36633415835396	0\\
1.36643416085402	0\\
1.36653416335408	0\\
1.36663416585415	0\\
1.36673416835421	0\\
1.36683417085427	0\\
1.36693417335433	0\\
1.3670341758544	0\\
1.36713417835446	0\\
1.36723418085452	0\\
1.36733418335458	0\\
1.36743418585465	0\\
1.36753418835471	0\\
1.36763419085477	0\\
1.36773419335483	0\\
1.3678341958549	0\\
1.36793419835496	0\\
1.36803420085502	0\\
1.36813420335508	0\\
1.36823420585515	0\\
1.36833420835521	0\\
1.36843421085527	0\\
1.36853421335533	0\\
1.3686342158554	0\\
1.36873421835546	0\\
1.36883422085552	0\\
1.36893422335558	0\\
1.36903422585565	0\\
1.36913422835571	0\\
1.36923423085577	0\\
1.36933423335583	0\\
1.3694342358559	0\\
1.36953423835596	0\\
1.36963424085602	0\\
1.36973424335608	0\\
1.36983424585615	0\\
1.36993424835621	0\\
1.37003425085627	0\\
1.37013425335633	0\\
1.3702342558564	0\\
1.37033425835646	0\\
1.37043426085652	0\\
1.37053426335658	0\\
1.37063426585665	0\\
1.37073426835671	0\\
1.37083427085677	0\\
1.37093427335683	0\\
1.3710342758569	0\\
1.37113427835696	0\\
1.37123428085702	0\\
1.37133428335708	0\\
1.37143428585715	0\\
1.37153428835721	0\\
1.37163429085727	0\\
1.37173429335733	0\\
1.3718342958574	0\\
1.37193429835746	0\\
1.37203430085752	0\\
1.37213430335758	0\\
1.37223430585765	0\\
1.37233430835771	0\\
1.37243431085777	0\\
1.37253431335783	0\\
1.3726343158579	0\\
1.37273431835796	0\\
1.37283432085802	0\\
1.37293432335808	0\\
1.37303432585815	0\\
1.37313432835821	0\\
1.37323433085827	0\\
1.37333433335833	0\\
1.3734343358584	0\\
1.37353433835846	0\\
1.37363434085852	0\\
1.37373434335858	0\\
1.37383434585865	0\\
1.37393434835871	0\\
1.37403435085877	0\\
1.37413435335883	0\\
1.3742343558589	0\\
1.37433435835896	0\\
1.37443436085902	0\\
1.37453436335908	0\\
1.37463436585915	0\\
1.37473436835921	0\\
1.37483437085927	0\\
1.37493437335933	0\\
1.3750343758594	0\\
1.37513437835946	0\\
1.37523438085952	0\\
1.37533438335958	0\\
1.37543438585965	0\\
1.37553438835971	0\\
1.37563439085977	0\\
1.37573439335983	0\\
1.3758343958599	0\\
1.37593439835996	0\\
1.37603440086002	0\\
1.37613440336008	0\\
1.37623440586015	0\\
1.37633440836021	0\\
1.37643441086027	0\\
1.37653441336033	0\\
1.3766344158604	0\\
1.37673441836046	0\\
1.37683442086052	0\\
1.37693442336058	0\\
1.37703442586065	0\\
1.37713442836071	0\\
1.37723443086077	0\\
1.37733443336083	0\\
1.3774344358609	0\\
1.37753443836096	0\\
1.37763444086102	0\\
1.37773444336108	0\\
1.37783444586115	0\\
1.37793444836121	0\\
1.37803445086127	0\\
1.37813445336133	0\\
1.3782344558614	0\\
1.37833445836146	0\\
1.37843446086152	0\\
1.37853446336158	0\\
1.37863446586165	0\\
1.37873446836171	0\\
1.37883447086177	0\\
1.37893447336183	0\\
1.3790344758619	0\\
1.37913447836196	0\\
1.37923448086202	0\\
1.37933448336208	0\\
1.37943448586215	0\\
1.37953448836221	0\\
1.37963449086227	0\\
1.37973449336233	0\\
1.3798344958624	0\\
1.37993449836246	0\\
1.38003450086252	0\\
1.38013450336258	0\\
1.38023450586265	0\\
1.38033450836271	0\\
1.38043451086277	0\\
1.38053451336283	0\\
1.3806345158629	0\\
1.38073451836296	0\\
1.38083452086302	0\\
1.38093452336308	0\\
1.38103452586315	0\\
1.38113452836321	0\\
1.38123453086327	0\\
1.38133453336333	0\\
1.3814345358634	0\\
1.38153453836346	0\\
1.38163454086352	0\\
1.38173454336358	0\\
1.38183454586365	0\\
1.38193454836371	0\\
1.38203455086377	0\\
1.38213455336383	0\\
1.3822345558639	0\\
1.38233455836396	0\\
1.38243456086402	0\\
1.38253456336408	0\\
1.38263456586415	0\\
1.38273456836421	0\\
1.38283457086427	0\\
1.38293457336433	0\\
1.3830345758644	0\\
1.38313457836446	0\\
1.38323458086452	0\\
1.38333458336458	0\\
1.38343458586465	0\\
1.38353458836471	0\\
1.38363459086477	0\\
1.38373459336483	0\\
1.3838345958649	0\\
1.38393459836496	0\\
1.38403460086502	0\\
1.38413460336508	0\\
1.38423460586515	0\\
1.38433460836521	0\\
1.38443461086527	0\\
1.38453461336533	0\\
1.3846346158654	0\\
1.38473461836546	0\\
1.38483462086552	0\\
1.38493462336558	0\\
1.38503462586565	0\\
1.38513462836571	0\\
1.38523463086577	0\\
1.38533463336583	0\\
1.3854346358659	0\\
1.38553463836596	0\\
1.38563464086602	0\\
1.38573464336608	0\\
1.38583464586615	0\\
1.38593464836621	0\\
1.38603465086627	0\\
1.38613465336633	0\\
1.3862346558664	0\\
1.38633465836646	0\\
1.38643466086652	0\\
1.38653466336658	0\\
1.38663466586665	0\\
1.38673466836671	0\\
1.38683467086677	0\\
1.38693467336683	0\\
1.3870346758669	0\\
1.38713467836696	0\\
1.38723468086702	0\\
1.38733468336708	0\\
1.38743468586715	0\\
1.38753468836721	0\\
1.38763469086727	0\\
1.38773469336733	0\\
1.3878346958674	0\\
1.38793469836746	0\\
1.38803470086752	0\\
1.38813470336758	0\\
1.38823470586765	0\\
1.38833470836771	0\\
1.38843471086777	0\\
1.38853471336783	0\\
1.3886347158679	0\\
1.38873471836796	0\\
1.38883472086802	0\\
1.38893472336808	0\\
1.38903472586815	0\\
1.38913472836821	0\\
1.38923473086827	0\\
1.38933473336833	0\\
1.3894347358684	0\\
1.38953473836846	0\\
1.38963474086852	0\\
1.38973474336858	0\\
1.38983474586865	0\\
1.38993474836871	0\\
1.39003475086877	0\\
1.39013475336883	0\\
1.3902347558689	0\\
1.39033475836896	0\\
1.39043476086902	0\\
1.39053476336908	0\\
1.39063476586915	0\\
1.39073476836921	0\\
1.39083477086927	0\\
1.39093477336933	0\\
1.3910347758694	0\\
1.39113477836946	0\\
1.39123478086952	0\\
1.39133478336958	0\\
1.39143478586965	0\\
1.39153478836971	0\\
1.39163479086977	0\\
1.39173479336983	0\\
1.3918347958699	0\\
1.39193479836996	0\\
1.39203480087002	0\\
1.39213480337008	0\\
1.39223480587015	0\\
1.39233480837021	0\\
1.39243481087027	0\\
1.39253481337033	0\\
1.3926348158704	0\\
1.39273481837046	0\\
1.39283482087052	0\\
1.39293482337058	0\\
1.39303482587065	0\\
1.39313482837071	0\\
1.39323483087077	0\\
1.39333483337083	0\\
1.3934348358709	0\\
1.39353483837096	0\\
1.39363484087102	0\\
1.39373484337108	0\\
1.39383484587115	0\\
1.39393484837121	0\\
1.39403485087127	0\\
1.39413485337133	0\\
1.3942348558714	0\\
1.39433485837146	0\\
1.39443486087152	0\\
1.39453486337158	0\\
1.39463486587165	0\\
1.39473486837171	0\\
1.39483487087177	0\\
1.39493487337183	0\\
1.3950348758719	0\\
1.39513487837196	0\\
1.39523488087202	0\\
1.39533488337208	0\\
1.39543488587215	0\\
1.39553488837221	0\\
1.39563489087227	0\\
1.39573489337233	0\\
1.3958348958724	0\\
1.39593489837246	0\\
1.39603490087252	0\\
1.39613490337258	0\\
1.39623490587265	0\\
1.39633490837271	0\\
1.39643491087277	0\\
1.39653491337283	0\\
1.3966349158729	0\\
1.39673491837296	0\\
1.39683492087302	0\\
1.39693492337308	0\\
1.39703492587315	0\\
1.39713492837321	0\\
1.39723493087327	0\\
1.39733493337333	0\\
1.3974349358734	0\\
1.39753493837346	0\\
1.39763494087352	0\\
1.39773494337358	0\\
1.39783494587365	0\\
1.39793494837371	0\\
1.39803495087377	0\\
1.39813495337383	0\\
1.3982349558739	0\\
1.39833495837396	0\\
1.39843496087402	0\\
1.39853496337408	0\\
1.39863496587415	0\\
1.39873496837421	0\\
1.39883497087427	0\\
1.39893497337433	0\\
1.3990349758744	0\\
1.39913497837446	0\\
1.39923498087452	0\\
1.39933498337458	0\\
1.39943498587465	0\\
1.39953498837471	0\\
1.39963499087477	0\\
1.39973499337483	0\\
1.3998349958749	0\\
1.39993499837496	0\\
1.40003500087502	0\\
1.40013500337508	0\\
1.40023500587515	0\\
1.40033500837521	0\\
1.40043501087527	0\\
1.40053501337533	0\\
1.4006350158754	0\\
1.40073501837546	0\\
1.40083502087552	0\\
1.40093502337558	0\\
1.40103502587565	0\\
1.40113502837571	0\\
1.40123503087577	0\\
1.40133503337583	0\\
1.4014350358759	0\\
1.40153503837596	0\\
1.40163504087602	0\\
1.40173504337608	0\\
1.40183504587615	0\\
1.40193504837621	0\\
1.40203505087627	0\\
1.40213505337633	0\\
1.4022350558764	0\\
1.40233505837646	0\\
1.40243506087652	0\\
1.40253506337658	0\\
1.40263506587665	0\\
1.40273506837671	0\\
1.40283507087677	0\\
1.40293507337683	0\\
1.4030350758769	0\\
1.40313507837696	0\\
1.40323508087702	0\\
1.40333508337708	0\\
1.40343508587715	0\\
1.40353508837721	0\\
1.40363509087727	0\\
1.40373509337733	0\\
1.4038350958774	0\\
1.40393509837746	0\\
1.40403510087752	0\\
1.40413510337758	0\\
1.40423510587765	0\\
1.40433510837771	0\\
1.40443511087777	0\\
1.40453511337783	0\\
1.4046351158779	0\\
1.40473511837796	0\\
1.40483512087802	0\\
1.40493512337808	0\\
1.40503512587815	0\\
1.40513512837821	0\\
1.40523513087827	0\\
1.40533513337833	0\\
1.4054351358784	0\\
1.40553513837846	0\\
1.40563514087852	0\\
1.40573514337858	0\\
1.40583514587865	0\\
1.40593514837871	0\\
1.40603515087877	0\\
1.40613515337883	0\\
1.4062351558789	0\\
1.40633515837896	0\\
1.40643516087902	0\\
1.40653516337908	0\\
1.40663516587915	0\\
1.40673516837921	0\\
1.40683517087927	0\\
1.40693517337933	0\\
1.4070351758794	0\\
1.40713517837946	0\\
1.40723518087952	0\\
1.40733518337958	0\\
1.40743518587965	0\\
1.40753518837971	0\\
1.40763519087977	0\\
1.40773519337983	0\\
1.4078351958799	0\\
1.40793519837996	0\\
1.40803520088002	0\\
1.40813520338008	0\\
1.40823520588015	0\\
1.40833520838021	0\\
1.40843521088027	0\\
1.40853521338033	0\\
1.4086352158804	0\\
1.40873521838046	0\\
1.40883522088052	0\\
1.40893522338058	0\\
1.40903522588065	0\\
1.40913522838071	0\\
1.40923523088077	0\\
1.40933523338083	0\\
1.4094352358809	0\\
1.40953523838096	0\\
1.40963524088102	0\\
1.40973524338108	0\\
1.40983524588115	0\\
1.40993524838121	0\\
1.41003525088127	0\\
1.41013525338133	0\\
1.4102352558814	0\\
1.41033525838146	0\\
1.41043526088152	0\\
1.41053526338158	0\\
1.41063526588165	0\\
1.41073526838171	0\\
1.41083527088177	0\\
1.41093527338183	0\\
1.4110352758819	0\\
1.41113527838196	0\\
1.41123528088202	0\\
1.41133528338208	0\\
1.41143528588215	0\\
1.41153528838221	0\\
1.41163529088227	0\\
1.41173529338233	0\\
1.4118352958824	0\\
1.41193529838246	0\\
1.41203530088252	0\\
1.41213530338258	0\\
1.41223530588265	0\\
1.41233530838271	0\\
1.41243531088277	0\\
1.41253531338283	0\\
1.4126353158829	0\\
1.41273531838296	0\\
1.41283532088302	0\\
1.41293532338308	0\\
1.41303532588315	0\\
1.41313532838321	0\\
1.41323533088327	0\\
1.41333533338333	0\\
1.4134353358834	0\\
1.41353533838346	0\\
1.41363534088352	0\\
1.41373534338358	0\\
1.41383534588365	0\\
1.41393534838371	0\\
1.41403535088377	0\\
1.41413535338383	0\\
1.4142353558839	0\\
1.41433535838396	0\\
1.41443536088402	0\\
1.41453536338408	0\\
1.41463536588415	0\\
1.41473536838421	0\\
1.41483537088427	0\\
1.41493537338433	0\\
1.4150353758844	0\\
1.41513537838446	0\\
1.41523538088452	0\\
1.41533538338458	0\\
1.41543538588465	0\\
1.41553538838471	0\\
1.41563539088477	0\\
1.41573539338483	0\\
1.4158353958849	0\\
1.41593539838496	0\\
1.41603540088502	0\\
1.41613540338508	0\\
1.41623540588515	0\\
1.41633540838521	0\\
1.41643541088527	0\\
1.41653541338533	0\\
1.4166354158854	0\\
1.41673541838546	0\\
1.41683542088552	0\\
1.41693542338558	0\\
1.41703542588565	0\\
1.41713542838571	0\\
1.41723543088577	0\\
1.41733543338583	0\\
1.4174354358859	0\\
1.41753543838596	0\\
1.41763544088602	0\\
1.41773544338608	0\\
1.41783544588615	0\\
1.41793544838621	0\\
1.41803545088627	0\\
1.41813545338633	0\\
1.4182354558864	0\\
1.41833545838646	0\\
1.41843546088652	0\\
1.41853546338658	0\\
1.41863546588665	0\\
1.41873546838671	0\\
1.41883547088677	0\\
1.41893547338683	0\\
1.4190354758869	0\\
1.41913547838696	0\\
1.41923548088702	0\\
1.41933548338708	0\\
1.41943548588715	0\\
1.41953548838721	0\\
1.41963549088727	0\\
1.41973549338733	0\\
1.4198354958874	0\\
1.41993549838746	0\\
1.42003550088752	0\\
1.42013550338758	0\\
1.42023550588765	0\\
1.42033550838771	0\\
1.42043551088777	0\\
1.42053551338783	0\\
1.4206355158879	0\\
1.42073551838796	0\\
1.42083552088802	0\\
1.42093552338808	0\\
1.42103552588815	0\\
1.42113552838821	0\\
1.42123553088827	0\\
1.42133553338833	0\\
1.4214355358884	0\\
1.42153553838846	0\\
1.42163554088852	0\\
1.42173554338858	0\\
1.42183554588865	0\\
1.42193554838871	0\\
1.42203555088877	0\\
1.42213555338883	0\\
1.4222355558889	0\\
1.42233555838896	0\\
1.42243556088902	0\\
1.42253556338908	0\\
1.42263556588915	0\\
1.42273556838921	0\\
1.42283557088927	0\\
1.42293557338933	0\\
1.4230355758894	0\\
1.42313557838946	0\\
1.42323558088952	0\\
1.42333558338958	0\\
1.42343558588965	0\\
1.42353558838971	0\\
1.42363559088977	0\\
1.42373559338983	0\\
1.4238355958899	0\\
1.42393559838996	0\\
1.42403560089002	0\\
1.42413560339008	0\\
1.42423560589015	0\\
1.42433560839021	0\\
1.42443561089027	0\\
1.42453561339033	0\\
1.4246356158904	0\\
1.42473561839046	0\\
1.42483562089052	0\\
1.42493562339058	0\\
1.42503562589065	0\\
1.42513562839071	0\\
1.42523563089077	0\\
1.42533563339083	0\\
1.4254356358909	0\\
1.42553563839096	0\\
1.42563564089102	0\\
1.42573564339108	0\\
1.42583564589115	0\\
1.42593564839121	0\\
1.42603565089127	0\\
1.42613565339133	0\\
1.4262356558914	0\\
1.42633565839146	0\\
1.42643566089152	0\\
1.42653566339158	0\\
1.42663566589165	0\\
1.42673566839171	0\\
1.42683567089177	0\\
1.42693567339183	0\\
1.4270356758919	0\\
1.42713567839196	0\\
1.42723568089202	0\\
1.42733568339208	0\\
1.42743568589215	0\\
1.42753568839221	0\\
1.42763569089227	0\\
1.42773569339233	0\\
1.4278356958924	0\\
1.42793569839246	0\\
1.42803570089252	0\\
1.42813570339258	0\\
1.42823570589265	0\\
1.42833570839271	0\\
1.42843571089277	0\\
1.42853571339283	0\\
1.4286357158929	0\\
1.42873571839296	0\\
1.42883572089302	0\\
1.42893572339308	0\\
1.42903572589315	0\\
1.42913572839321	0\\
1.42923573089327	0\\
1.42933573339333	0\\
1.4294357358934	0\\
1.42953573839346	0\\
1.42963574089352	0\\
1.42973574339358	0\\
1.42983574589365	0\\
1.42993574839371	0\\
1.43003575089377	0\\
1.43013575339383	0\\
1.4302357558939	0\\
1.43033575839396	0\\
1.43043576089402	0\\
1.43053576339408	0\\
1.43063576589415	0\\
1.43073576839421	0\\
1.43083577089427	0\\
1.43093577339433	0\\
1.4310357758944	0\\
1.43113577839446	0\\
1.43123578089452	0\\
1.43133578339458	0\\
1.43143578589465	0\\
1.43153578839471	0\\
1.43163579089477	0\\
1.43173579339483	0\\
1.4318357958949	0\\
1.43193579839496	0\\
1.43203580089502	0\\
1.43213580339508	0\\
1.43223580589515	0\\
1.43233580839521	0\\
1.43243581089527	0\\
1.43253581339533	0\\
1.4326358158954	0\\
1.43273581839546	0\\
1.43283582089552	0\\
1.43293582339558	0\\
1.43303582589565	0\\
1.43313582839571	0\\
1.43323583089577	0\\
1.43333583339583	0\\
1.4334358358959	0\\
1.43353583839596	0\\
1.43363584089602	0\\
1.43373584339608	0\\
1.43383584589615	0\\
1.43393584839621	0\\
1.43403585089627	0\\
1.43413585339633	0\\
1.4342358558964	0\\
1.43433585839646	0\\
1.43443586089652	0\\
1.43453586339659	0\\
1.43463586589665	0\\
1.43473586839671	0\\
1.43483587089677	0\\
1.43493587339683	0\\
1.4350358758969	0\\
1.43513587839696	0\\
1.43523588089702	0\\
1.43533588339708	0\\
1.43543588589715	0\\
1.43553588839721	0\\
1.43563589089727	0\\
1.43573589339733	0\\
1.4358358958974	0\\
1.43593589839746	0\\
1.43603590089752	0\\
1.43613590339758	0\\
1.43623590589765	0\\
1.43633590839771	0\\
1.43643591089777	0\\
1.43653591339784	0\\
1.4366359158979	0\\
1.43673591839796	0\\
1.43683592089802	0\\
1.43693592339808	0\\
1.43703592589815	0\\
1.43713592839821	0\\
1.43723593089827	0\\
1.43733593339833	0\\
1.4374359358984	0\\
1.43753593839846	0\\
1.43763594089852	0\\
1.43773594339859	0\\
1.43783594589865	0\\
1.43793594839871	0\\
1.43803595089877	0\\
1.43813595339883	0\\
1.4382359558989	0\\
1.43833595839896	0\\
1.43843596089902	0\\
1.43853596339909	0\\
1.43863596589915	0\\
1.43873596839921	0\\
1.43883597089927	0\\
1.43893597339933	0\\
1.4390359758994	0\\
1.43913597839946	0\\
1.43923598089952	0\\
1.43933598339958	0\\
1.43943598589965	0\\
1.43953598839971	0\\
1.43963599089977	0\\
1.43973599339984	0\\
1.4398359958999	0\\
1.43993599839996	0\\
1.44003600090002	0\\
1.44013600340008	0\\
1.44023600590015	0\\
1.44033600840021	0\\
1.44043601090027	0\\
1.44053601340034	0\\
1.4406360159004	0\\
1.44073601840046	0\\
1.44083602090052	0\\
1.44093602340059	0\\
1.44103602590065	0\\
1.44113602840071	0\\
1.44123603090077	0\\
1.44133603340083	0\\
1.4414360359009	0\\
1.44153603840096	0\\
1.44163604090102	0\\
1.44173604340109	0\\
1.44183604590115	0\\
1.44193604840121	0\\
1.44203605090127	0\\
1.44213605340133	0\\
1.4422360559014	0\\
1.44233605840146	0\\
1.44243606090152	0\\
1.44253606340159	0\\
1.44263606590165	0\\
1.44273606840171	0\\
1.44283607090177	0\\
1.44293607340184	0\\
1.4430360759019	0\\
1.44313607840196	0\\
1.44323608090202	0\\
1.44333608340208	0\\
1.44343608590215	0\\
1.44353608840221	0\\
1.44363609090227	0\\
1.44373609340234	0\\
1.4438360959024	0\\
1.44393609840246	0\\
1.44403610090252	0\\
1.44413610340259	0\\
1.44423610590265	0\\
1.44433610840271	0\\
1.44443611090277	0\\
1.44453611340284	0\\
1.4446361159029	0\\
1.44473611840296	0\\
1.44483612090302	0\\
1.44493612340309	0\\
1.44503612590315	0\\
1.44513612840321	0\\
1.44523613090327	0\\
1.44533613340333	0\\
1.4454361359034	0\\
1.44553613840346	0\\
1.44563614090352	0\\
1.44573614340359	0\\
1.44583614590365	0\\
1.44593614840371	0\\
1.44603615090377	0\\
1.44613615340384	0\\
1.4462361559039	0\\
1.44633615840396	0\\
1.44643616090402	0\\
1.44653616340409	0\\
1.44663616590415	0\\
1.44673616840421	0\\
1.44683617090427	0\\
1.44693617340434	0\\
1.4470361759044	0\\
1.44713617840446	0\\
1.44723618090452	0\\
1.44733618340459	0\\
1.44743618590465	0\\
1.44753618840471	0\\
1.44763619090477	0\\
1.44773619340484	0\\
1.4478361959049	0\\
1.44793619840496	0\\
1.44803620090502	0\\
1.44813620340509	0\\
1.44823620590515	0\\
1.44833620840521	0\\
1.44843621090527	0\\
1.44853621340534	0\\
1.4486362159054	0\\
1.44873621840546	0\\
1.44883622090552	0\\
1.44893622340559	0\\
1.44903622590565	0\\
1.44913622840571	0\\
1.44923623090577	0\\
1.44933623340584	0\\
1.4494362359059	0\\
1.44953623840596	0\\
1.44963624090602	0\\
1.44973624340609	0\\
1.44983624590615	0\\
1.44993624840621	0\\
1.45003625090627	0\\
1.45013625340634	0\\
1.4502362559064	0\\
1.45033625840646	0\\
1.45043626090652	0\\
1.45053626340659	0\\
1.45063626590665	0\\
1.45073626840671	0\\
1.45083627090677	0\\
1.45093627340684	0\\
1.4510362759069	0\\
1.45113627840696	0\\
1.45123628090702	0\\
1.45133628340709	0\\
1.45143628590715	0\\
1.45153628840721	0\\
1.45163629090727	0\\
1.45173629340734	0\\
1.4518362959074	0\\
1.45193629840746	0\\
1.45203630090752	0\\
1.45213630340759	0\\
1.45223630590765	0\\
1.45233630840771	0\\
1.45243631090777	0\\
1.45253631340784	0\\
1.4526363159079	0\\
1.45273631840796	0\\
1.45283632090802	0\\
1.45293632340809	0\\
1.45303632590815	0\\
1.45313632840821	0\\
1.45323633090827	0\\
1.45333633340834	0\\
1.4534363359084	0\\
1.45353633840846	0\\
1.45363634090852	0\\
1.45373634340859	0\\
1.45383634590865	0\\
1.45393634840871	0\\
1.45403635090877	0\\
1.45413635340884	0\\
1.4542363559089	0\\
1.45433635840896	0\\
1.45443636090902	0\\
1.45453636340909	0\\
1.45463636590915	0\\
1.45473636840921	0\\
1.45483637090927	0\\
1.45493637340934	0\\
1.4550363759094	0\\
1.45513637840946	0\\
1.45523638090952	0\\
1.45533638340959	0\\
1.45543638590965	0\\
1.45553638840971	0\\
1.45563639090977	0\\
1.45573639340984	0\\
1.4558363959099	0\\
1.45593639840996	0\\
1.45603640091002	0\\
1.45613640341009	0\\
1.45623640591015	0\\
1.45633640841021	0\\
1.45643641091027	0\\
1.45653641341034	0\\
1.4566364159104	0\\
1.45673641841046	0\\
1.45683642091052	0\\
1.45693642341059	0\\
1.45703642591065	0\\
1.45713642841071	0\\
1.45723643091077	0\\
1.45733643341084	0\\
1.4574364359109	0\\
1.45753643841096	0\\
1.45763644091102	0\\
1.45773644341109	0\\
1.45783644591115	0\\
1.45793644841121	0\\
1.45803645091127	0\\
1.45813645341134	0\\
1.4582364559114	0\\
1.45833645841146	0\\
1.45843646091152	0\\
1.45853646341159	0\\
1.45863646591165	0\\
1.45873646841171	0\\
1.45883647091177	0\\
1.45893647341184	0\\
1.4590364759119	0\\
1.45913647841196	0\\
1.45923648091202	0\\
1.45933648341209	0\\
1.45943648591215	0\\
1.45953648841221	0\\
1.45963649091227	0\\
1.45973649341234	0\\
1.4598364959124	0\\
1.45993649841246	0\\
1.46003650091252	0\\
1.46013650341259	0\\
1.46023650591265	0\\
1.46033650841271	0\\
1.46043651091277	0\\
1.46053651341284	0\\
1.4606365159129	0\\
1.46073651841296	0\\
1.46083652091302	0\\
1.46093652341309	0\\
1.46103652591315	0\\
1.46113652841321	0\\
1.46123653091327	0\\
1.46133653341334	0\\
1.4614365359134	0\\
1.46153653841346	0\\
1.46163654091352	0\\
1.46173654341359	0\\
1.46183654591365	0\\
1.46193654841371	0\\
1.46203655091377	0\\
1.46213655341384	0\\
1.4622365559139	0\\
1.46233655841396	0\\
1.46243656091402	0\\
1.46253656341409	0\\
1.46263656591415	0\\
1.46273656841421	0\\
1.46283657091427	0\\
1.46293657341434	0\\
1.4630365759144	0\\
1.46313657841446	0\\
1.46323658091452	0\\
1.46333658341459	0\\
1.46343658591465	0\\
1.46353658841471	0\\
1.46363659091477	0\\
1.46373659341484	0\\
1.4638365959149	0\\
1.46393659841496	0\\
1.46403660091502	0\\
1.46413660341509	0\\
1.46423660591515	0\\
1.46433660841521	0\\
1.46443661091527	0\\
1.46453661341534	0\\
1.4646366159154	0\\
1.46473661841546	0\\
1.46483662091552	0\\
1.46493662341559	0\\
1.46503662591565	0\\
1.46513662841571	0\\
1.46523663091577	0\\
1.46533663341584	0\\
1.4654366359159	0\\
1.46553663841596	0\\
1.46563664091602	0\\
1.46573664341609	0\\
1.46583664591615	0\\
1.46593664841621	0\\
1.46603665091627	0\\
1.46613665341634	0\\
1.4662366559164	0\\
1.46633665841646	0\\
1.46643666091652	0\\
1.46653666341659	0\\
1.46663666591665	0\\
1.46673666841671	0\\
1.46683667091677	0\\
1.46693667341684	0\\
1.4670366759169	0\\
1.46713667841696	0\\
1.46723668091702	0\\
1.46733668341709	0\\
1.46743668591715	0\\
1.46753668841721	0\\
1.46763669091727	0\\
1.46773669341734	0\\
1.4678366959174	0\\
1.46793669841746	0\\
1.46803670091752	0\\
1.46813670341759	0\\
1.46823670591765	0\\
1.46833670841771	0\\
1.46843671091777	0\\
1.46853671341784	0\\
1.4686367159179	0\\
1.46873671841796	0\\
1.46883672091802	0\\
1.46893672341809	0\\
1.46903672591815	0\\
1.46913672841821	0\\
1.46923673091827	0\\
1.46933673341834	0\\
1.4694367359184	0\\
1.46953673841846	0\\
1.46963674091852	0\\
1.46973674341859	0\\
1.46983674591865	0\\
1.46993674841871	0\\
1.47003675091877	0\\
1.47013675341884	0\\
1.4702367559189	0\\
1.47033675841896	0\\
1.47043676091902	0\\
1.47053676341909	0\\
1.47063676591915	0\\
1.47073676841921	0\\
1.47083677091927	0\\
1.47093677341934	0\\
1.4710367759194	0\\
1.47113677841946	0\\
1.47123678091952	0\\
1.47133678341959	0\\
1.47143678591965	0\\
1.47153678841971	0\\
1.47163679091977	0\\
1.47173679341984	0\\
1.4718367959199	0\\
1.47193679841996	0\\
1.47203680092002	0\\
1.47213680342009	0\\
1.47223680592015	0\\
1.47233680842021	0\\
1.47243681092027	0\\
1.47253681342034	0\\
1.4726368159204	0\\
1.47273681842046	0\\
1.47283682092052	0\\
1.47293682342059	0\\
1.47303682592065	0\\
1.47313682842071	0\\
1.47323683092077	0\\
1.47333683342084	0\\
1.4734368359209	0\\
1.47353683842096	0\\
1.47363684092102	0\\
1.47373684342109	0\\
1.47383684592115	0\\
1.47393684842121	0\\
1.47403685092127	0\\
1.47413685342134	0\\
1.4742368559214	0\\
1.47433685842146	0\\
1.47443686092152	0\\
1.47453686342159	0\\
1.47463686592165	0\\
1.47473686842171	0\\
1.47483687092177	0\\
1.47493687342184	0\\
1.4750368759219	0\\
1.47513687842196	0\\
1.47523688092202	0\\
1.47533688342209	0\\
1.47543688592215	0\\
1.47553688842221	0\\
1.47563689092227	0\\
1.47573689342234	0\\
1.4758368959224	0\\
1.47593689842246	0\\
1.47603690092252	0\\
1.47613690342259	0\\
1.47623690592265	0\\
1.47633690842271	0\\
1.47643691092277	0\\
1.47653691342284	0\\
1.4766369159229	0\\
1.47673691842296	0\\
1.47683692092302	0\\
1.47693692342309	0\\
1.47703692592315	0\\
1.47713692842321	0\\
1.47723693092327	0\\
1.47733693342334	0\\
1.4774369359234	0\\
1.47753693842346	0\\
1.47763694092352	0\\
1.47773694342359	0\\
1.47783694592365	0\\
1.47793694842371	0\\
1.47803695092377	0\\
1.47813695342384	0\\
1.4782369559239	0\\
1.47833695842396	0\\
1.47843696092402	0\\
1.47853696342409	0\\
1.47863696592415	0\\
1.47873696842421	0\\
1.47883697092427	0\\
1.47893697342434	0\\
1.4790369759244	0\\
1.47913697842446	0\\
1.47923698092452	0\\
1.47933698342459	0\\
1.47943698592465	0\\
1.47953698842471	0\\
1.47963699092477	0\\
1.47973699342484	0\\
1.4798369959249	0\\
1.47993699842496	0\\
1.48003700092502	0\\
1.48013700342509	0\\
1.48023700592515	0\\
1.48033700842521	0\\
1.48043701092527	0\\
1.48053701342534	0\\
1.4806370159254	0\\
1.48073701842546	0\\
1.48083702092552	0\\
1.48093702342559	0\\
1.48103702592565	0\\
1.48113702842571	0\\
1.48123703092577	0\\
1.48133703342584	0\\
1.4814370359259	0\\
1.48153703842596	0\\
1.48163704092602	0\\
1.48173704342609	0\\
1.48183704592615	0\\
1.48193704842621	0\\
1.48203705092627	0\\
1.48213705342634	0\\
1.4822370559264	0\\
1.48233705842646	0\\
1.48243706092652	0\\
1.48253706342659	0\\
1.48263706592665	0\\
1.48273706842671	0\\
1.48283707092677	0\\
1.48293707342684	0\\
1.4830370759269	0\\
1.48313707842696	0\\
1.48323708092702	0\\
1.48333708342709	0\\
1.48343708592715	0\\
1.48353708842721	0\\
1.48363709092727	0\\
1.48373709342734	0\\
1.4838370959274	0\\
1.48393709842746	0\\
1.48403710092752	0\\
1.48413710342759	0\\
1.48423710592765	0\\
1.48433710842771	0\\
1.48443711092777	0\\
1.48453711342784	0\\
1.4846371159279	0\\
1.48473711842796	0\\
1.48483712092802	0\\
1.48493712342809	0\\
1.48503712592815	0\\
1.48513712842821	0\\
1.48523713092827	0\\
1.48533713342834	0\\
1.4854371359284	0\\
1.48553713842846	0\\
1.48563714092852	0\\
1.48573714342859	0\\
1.48583714592865	0\\
1.48593714842871	0\\
1.48603715092877	0\\
1.48613715342884	0\\
1.4862371559289	0\\
1.48633715842896	0\\
1.48643716092902	0\\
1.48653716342909	0\\
1.48663716592915	0\\
1.48673716842921	0\\
1.48683717092927	0\\
1.48693717342934	0\\
1.4870371759294	0\\
1.48713717842946	0\\
1.48723718092952	0\\
1.48733718342959	0\\
1.48743718592965	0\\
1.48753718842971	0\\
1.48763719092977	0\\
1.48773719342984	0\\
1.4878371959299	0\\
1.48793719842996	0\\
1.48803720093002	0\\
1.48813720343009	0\\
1.48823720593015	0\\
1.48833720843021	0\\
1.48843721093027	0\\
1.48853721343034	0\\
1.4886372159304	0\\
1.48873721843046	0\\
1.48883722093052	0\\
1.48893722343059	0\\
1.48903722593065	0\\
1.48913722843071	0\\
1.48923723093077	0\\
1.48933723343084	0\\
1.4894372359309	0\\
1.48953723843096	0\\
1.48963724093102	0\\
1.48973724343109	0\\
1.48983724593115	0\\
1.48993724843121	0\\
1.49003725093127	0\\
1.49013725343134	0\\
1.4902372559314	0\\
1.49033725843146	0\\
1.49043726093152	0\\
1.49053726343159	0\\
1.49063726593165	0\\
1.49073726843171	0\\
1.49083727093177	0\\
1.49093727343184	0\\
1.4910372759319	0\\
1.49113727843196	0\\
1.49123728093202	0\\
1.49133728343209	0\\
1.49143728593215	0\\
1.49153728843221	0\\
1.49163729093227	0\\
1.49173729343234	0\\
1.4918372959324	0\\
1.49193729843246	0\\
1.49203730093252	0\\
1.49213730343259	0\\
1.49223730593265	0\\
1.49233730843271	0\\
1.49243731093277	0\\
1.49253731343284	0\\
1.4926373159329	0\\
1.49273731843296	0\\
1.49283732093302	0\\
1.49293732343309	0\\
1.49303732593315	0\\
1.49313732843321	0\\
1.49323733093327	0\\
1.49333733343334	0\\
1.4934373359334	0\\
1.49353733843346	0\\
1.49363734093352	0\\
1.49373734343359	0\\
1.49383734593365	0\\
1.49393734843371	0\\
1.49403735093377	0\\
1.49413735343384	0\\
1.4942373559339	0\\
1.49433735843396	0\\
1.49443736093402	0\\
1.49453736343409	0\\
1.49463736593415	0\\
1.49473736843421	0\\
1.49483737093427	0\\
1.49493737343434	0\\
1.4950373759344	0\\
1.49513737843446	0\\
1.49523738093452	0\\
1.49533738343459	0\\
1.49543738593465	0\\
1.49553738843471	0\\
1.49563739093477	0\\
1.49573739343484	0\\
1.4958373959349	0\\
1.49593739843496	0\\
1.49603740093502	0\\
1.49613740343509	0\\
1.49623740593515	0\\
1.49633740843521	0\\
1.49643741093527	0\\
1.49653741343534	0\\
1.4966374159354	0\\
1.49673741843546	0\\
1.49683742093552	0\\
1.49693742343559	0\\
1.49703742593565	0\\
1.49713742843571	0\\
1.49723743093577	0\\
1.49733743343584	0\\
1.4974374359359	0\\
1.49753743843596	0\\
1.49763744093602	0\\
1.49773744343609	0\\
1.49783744593615	0\\
1.49793744843621	0\\
1.49803745093627	0\\
1.49813745343634	0\\
1.4982374559364	0\\
1.49833745843646	0\\
1.49843746093652	0\\
1.49853746343659	0\\
1.49863746593665	0\\
1.49873746843671	0\\
1.49883747093677	0\\
1.49893747343684	0\\
1.4990374759369	0\\
1.49913747843696	0\\
1.49923748093702	0\\
1.49933748343709	0\\
1.49943748593715	0\\
1.49953748843721	0\\
1.49963749093727	0\\
1.49973749343734	0\\
1.4998374959374	0\\
1.49993749843746	0\\
1.50003750093752	0\\
1.50013750343759	0\\
1.50023750593765	0\\
1.50033750843771	0\\
1.50043751093777	0\\
1.50053751343784	0\\
1.5006375159379	0\\
1.50073751843796	0\\
1.50083752093802	0\\
1.50093752343809	0\\
1.50103752593815	0\\
1.50113752843821	0\\
1.50123753093827	0\\
1.50133753343834	0\\
1.5014375359384	0\\
1.50153753843846	0\\
1.50163754093852	0\\
1.50173754343859	0\\
1.50183754593865	0\\
1.50193754843871	0\\
1.50203755093877	0\\
1.50213755343884	0\\
1.5022375559389	0\\
1.50233755843896	0\\
1.50243756093902	0\\
1.50253756343909	0\\
1.50263756593915	0\\
1.50273756843921	0\\
1.50283757093927	0\\
1.50293757343934	0\\
1.5030375759394	0\\
1.50313757843946	0\\
1.50323758093952	0\\
1.50333758343959	0\\
1.50343758593965	0\\
1.50353758843971	0\\
1.50363759093977	0\\
1.50373759343984	0\\
1.5038375959399	0\\
1.50393759843996	0\\
1.50403760094002	0\\
1.50413760344009	0\\
1.50423760594015	0\\
1.50433760844021	0\\
1.50443761094027	0\\
1.50453761344034	0\\
1.5046376159404	0\\
1.50473761844046	0\\
1.50483762094052	0\\
1.50493762344059	0\\
1.50503762594065	0\\
1.50513762844071	0\\
1.50523763094077	0\\
1.50533763344084	0\\
1.5054376359409	0\\
1.50553763844096	0\\
1.50563764094102	0\\
1.50573764344109	0\\
1.50583764594115	0\\
1.50593764844121	0\\
1.50603765094127	0\\
1.50613765344134	0\\
1.5062376559414	0\\
1.50633765844146	0\\
1.50643766094152	0\\
1.50653766344159	0\\
1.50663766594165	0\\
1.50673766844171	0\\
1.50683767094177	0\\
1.50693767344184	0\\
1.5070376759419	0\\
1.50713767844196	0\\
1.50723768094202	0\\
1.50733768344209	0\\
1.50743768594215	0\\
1.50753768844221	0\\
1.50763769094227	0\\
1.50773769344234	0\\
1.5078376959424	0\\
1.50793769844246	0\\
1.50803770094252	0\\
1.50813770344259	0\\
1.50823770594265	0\\
1.50833770844271	0\\
1.50843771094277	0\\
1.50853771344284	0\\
1.5086377159429	0\\
1.50873771844296	0\\
1.50883772094302	0\\
1.50893772344309	0\\
1.50903772594315	0\\
1.50913772844321	0\\
1.50923773094327	0\\
1.50933773344334	0\\
1.5094377359434	0\\
1.50953773844346	0\\
1.50963774094352	0\\
1.50973774344359	0\\
1.50983774594365	0\\
1.50993774844371	0\\
1.51003775094377	0\\
1.51013775344384	0\\
1.5102377559439	0\\
1.51033775844396	0\\
1.51043776094402	0\\
1.51053776344409	0\\
1.51063776594415	0\\
1.51073776844421	0\\
1.51083777094427	0\\
1.51093777344434	0\\
1.5110377759444	0\\
1.51113777844446	0\\
1.51123778094452	0\\
1.51133778344459	0\\
1.51143778594465	0\\
1.51153778844471	0\\
1.51163779094477	0\\
1.51173779344484	0\\
1.5118377959449	0\\
1.51193779844496	0\\
1.51203780094502	0\\
1.51213780344509	0\\
1.51223780594515	0\\
1.51233780844521	0\\
1.51243781094527	0\\
1.51253781344534	0\\
1.5126378159454	0\\
1.51273781844546	0\\
1.51283782094552	0\\
1.51293782344559	0\\
1.51303782594565	0\\
1.51313782844571	0\\
1.51323783094577	0\\
1.51333783344584	0\\
1.5134378359459	0\\
1.51353783844596	0\\
1.51363784094602	0\\
1.51373784344609	0\\
1.51383784594615	0\\
1.51393784844621	0\\
1.51403785094627	0\\
1.51413785344634	0\\
1.5142378559464	0\\
1.51433785844646	0\\
1.51443786094652	0\\
1.51453786344659	0\\
1.51463786594665	0\\
1.51473786844671	0\\
1.51483787094677	0\\
1.51493787344684	0\\
1.5150378759469	0\\
1.51513787844696	0\\
1.51523788094702	0\\
1.51533788344709	0\\
1.51543788594715	0\\
1.51553788844721	0\\
1.51563789094727	0\\
1.51573789344734	0\\
1.5158378959474	0\\
1.51593789844746	0\\
1.51603790094752	0\\
1.51613790344759	0\\
1.51623790594765	0\\
1.51633790844771	0\\
1.51643791094777	0\\
1.51653791344784	0\\
1.5166379159479	0\\
1.51673791844796	0\\
1.51683792094802	0\\
1.51693792344809	0\\
1.51703792594815	0\\
1.51713792844821	0\\
1.51723793094827	0\\
1.51733793344834	0\\
1.5174379359484	0\\
1.51753793844846	0\\
1.51763794094852	0\\
1.51773794344859	0\\
1.51783794594865	0\\
1.51793794844871	0\\
1.51803795094877	0\\
1.51813795344884	0\\
1.5182379559489	0\\
1.51833795844896	0\\
1.51843796094902	0\\
1.51853796344909	0\\
1.51863796594915	0\\
1.51873796844921	0\\
1.51883797094927	0\\
1.51893797344934	0\\
1.5190379759494	0\\
1.51913797844946	0\\
1.51923798094952	0\\
1.51933798344959	0\\
1.51943798594965	0\\
1.51953798844971	0\\
1.51963799094977	0\\
1.51973799344984	0\\
1.5198379959499	0\\
1.51993799844996	0\\
1.52003800095002	0\\
1.52013800345009	0\\
1.52023800595015	0\\
1.52033800845021	0\\
1.52043801095027	0\\
1.52053801345034	0\\
1.5206380159504	0\\
1.52073801845046	0\\
1.52083802095052	0\\
1.52093802345059	0\\
1.52103802595065	0\\
1.52113802845071	0\\
1.52123803095077	0\\
1.52133803345084	0\\
1.5214380359509	0\\
1.52153803845096	0\\
1.52163804095102	0\\
1.52173804345109	0\\
1.52183804595115	0\\
1.52193804845121	0\\
1.52203805095127	0\\
1.52213805345134	0\\
1.5222380559514	0\\
1.52233805845146	0\\
1.52243806095152	0\\
1.52253806345159	0\\
1.52263806595165	0\\
1.52273806845171	0\\
1.52283807095177	0\\
1.52293807345184	0\\
1.5230380759519	0\\
1.52313807845196	0\\
1.52323808095202	0\\
1.52333808345209	0\\
1.52343808595215	0\\
1.52353808845221	0\\
1.52363809095227	0\\
1.52373809345234	0\\
1.5238380959524	0\\
1.52393809845246	0\\
1.52403810095252	0\\
1.52413810345259	0\\
1.52423810595265	0\\
1.52433810845271	0\\
1.52443811095277	0\\
1.52453811345284	0\\
1.5246381159529	0\\
1.52473811845296	0\\
1.52483812095302	0\\
1.52493812345309	0\\
1.52503812595315	0\\
1.52513812845321	0\\
1.52523813095327	0\\
1.52533813345334	0\\
1.5254381359534	0\\
1.52553813845346	0\\
1.52563814095352	0\\
1.52573814345359	0\\
1.52583814595365	0\\
1.52593814845371	0\\
1.52603815095377	0\\
1.52613815345384	0\\
1.5262381559539	0\\
1.52633815845396	0\\
1.52643816095402	0\\
1.52653816345409	0\\
1.52663816595415	0\\
1.52673816845421	0\\
1.52683817095427	0\\
1.52693817345434	0\\
1.5270381759544	0\\
1.52713817845446	0\\
1.52723818095452	0\\
1.52733818345459	0\\
1.52743818595465	0\\
1.52753818845471	0\\
1.52763819095477	0\\
1.52773819345484	0\\
1.5278381959549	0\\
1.52793819845496	0\\
1.52803820095502	0\\
1.52813820345509	0\\
1.52823820595515	0\\
1.52833820845521	0\\
1.52843821095527	0\\
1.52853821345534	0\\
1.5286382159554	0\\
1.52873821845546	0\\
1.52883822095552	0\\
1.52893822345559	0\\
1.52903822595565	0\\
1.52913822845571	0\\
1.52923823095577	0\\
1.52933823345584	0\\
1.5294382359559	0\\
1.52953823845596	0\\
1.52963824095602	0\\
1.52973824345609	0\\
1.52983824595615	0\\
1.52993824845621	0\\
1.53003825095627	0\\
1.53013825345634	0\\
1.5302382559564	0\\
1.53033825845646	0\\
1.53043826095652	0\\
1.53053826345659	0\\
1.53063826595665	0\\
1.53073826845671	0\\
1.53083827095677	0\\
1.53093827345684	0\\
1.5310382759569	0\\
1.53113827845696	0\\
1.53123828095702	0\\
1.53133828345709	0\\
1.53143828595715	0\\
1.53153828845721	0\\
1.53163829095727	0\\
1.53173829345734	0\\
1.5318382959574	0\\
1.53193829845746	0\\
1.53203830095752	0\\
1.53213830345759	0\\
1.53223830595765	0\\
1.53233830845771	0\\
1.53243831095777	0\\
1.53253831345784	0\\
1.5326383159579	0\\
1.53273831845796	0\\
1.53283832095802	0\\
1.53293832345809	0\\
1.53303832595815	0\\
1.53313832845821	0\\
1.53323833095827	0\\
1.53333833345834	0\\
1.5334383359584	0\\
1.53353833845846	0\\
1.53363834095852	0\\
1.53373834345859	0\\
1.53383834595865	0\\
1.53393834845871	0\\
1.53403835095877	0\\
1.53413835345884	0\\
1.5342383559589	0\\
1.53433835845896	0\\
1.53443836095902	0\\
1.53453836345909	0\\
1.53463836595915	0\\
1.53473836845921	0\\
1.53483837095927	0\\
1.53493837345934	0\\
1.5350383759594	0\\
1.53513837845946	0\\
1.53523838095952	0\\
1.53533838345959	0\\
1.53543838595965	0\\
1.53553838845971	0\\
1.53563839095977	0\\
1.53573839345984	0\\
1.5358383959599	0\\
1.53593839845996	0\\
1.53603840096002	0\\
1.53613840346009	0\\
1.53623840596015	0\\
1.53633840846021	0\\
1.53643841096027	0\\
1.53653841346034	0\\
1.5366384159604	0\\
1.53673841846046	0\\
1.53683842096052	0\\
1.53693842346059	0\\
1.53703842596065	0\\
1.53713842846071	0\\
1.53723843096077	0\\
1.53733843346084	0\\
1.5374384359609	0\\
1.53753843846096	0\\
1.53763844096102	0\\
1.53773844346109	0\\
1.53783844596115	0\\
1.53793844846121	0\\
1.53803845096127	0\\
1.53813845346134	0\\
1.5382384559614	0\\
1.53833845846146	0\\
1.53843846096152	0\\
1.53853846346159	0\\
1.53863846596165	0\\
1.53873846846171	0\\
1.53883847096177	0\\
1.53893847346184	0\\
1.5390384759619	0\\
1.53913847846196	0\\
1.53923848096202	0\\
1.53933848346209	0\\
1.53943848596215	0\\
1.53953848846221	0\\
1.53963849096227	0\\
1.53973849346234	0\\
1.5398384959624	0\\
1.53993849846246	0\\
1.54003850096252	0\\
1.54013850346259	0\\
1.54023850596265	0\\
1.54033850846271	0\\
1.54043851096277	0\\
1.54053851346284	0\\
1.5406385159629	0\\
1.54073851846296	0\\
1.54083852096302	0\\
1.54093852346309	0\\
1.54103852596315	0\\
1.54113852846321	0\\
1.54123853096327	0\\
1.54133853346334	0\\
1.5414385359634	0\\
1.54153853846346	0\\
1.54163854096352	0\\
1.54173854346359	0\\
1.54183854596365	0\\
1.54193854846371	0\\
1.54203855096377	0\\
1.54213855346384	0\\
1.5422385559639	0\\
1.54233855846396	0\\
1.54243856096402	0\\
1.54253856346409	0\\
1.54263856596415	0\\
1.54273856846421	0\\
1.54283857096427	0\\
1.54293857346434	0\\
1.5430385759644	0\\
1.54313857846446	0\\
1.54323858096452	0\\
1.54333858346459	0\\
1.54343858596465	0\\
1.54353858846471	0\\
1.54363859096477	0\\
1.54373859346484	0\\
1.5438385959649	0\\
1.54393859846496	0\\
1.54403860096502	0\\
1.54413860346509	0\\
1.54423860596515	0\\
1.54433860846521	0\\
1.54443861096527	0\\
1.54453861346534	0\\
1.5446386159654	0\\
1.54473861846546	0\\
1.54483862096552	0\\
1.54493862346559	0\\
1.54503862596565	0\\
1.54513862846571	0\\
1.54523863096577	0\\
1.54533863346584	0\\
1.5454386359659	0\\
1.54553863846596	0\\
1.54563864096602	0\\
1.54573864346609	0\\
1.54583864596615	0\\
1.54593864846621	0\\
1.54603865096627	0\\
1.54613865346634	0\\
1.5462386559664	0\\
1.54633865846646	0\\
1.54643866096652	0\\
1.54653866346659	0\\
1.54663866596665	0\\
1.54673866846671	0\\
1.54683867096677	0\\
1.54693867346684	0\\
1.5470386759669	0\\
1.54713867846696	0\\
1.54723868096702	0\\
1.54733868346709	0\\
1.54743868596715	0\\
1.54753868846721	0\\
1.54763869096727	0\\
1.54773869346734	0\\
1.5478386959674	0\\
1.54793869846746	0\\
1.54803870096752	0\\
1.54813870346759	0\\
1.54823870596765	0\\
1.54833870846771	0\\
1.54843871096777	0\\
1.54853871346784	0\\
1.5486387159679	0\\
1.54873871846796	0\\
1.54883872096802	0\\
1.54893872346809	0\\
1.54903872596815	0\\
1.54913872846821	0\\
1.54923873096827	0\\
1.54933873346834	0\\
1.5494387359684	0\\
1.54953873846846	0\\
1.54963874096852	0\\
1.54973874346859	0\\
1.54983874596865	0\\
1.54993874846871	0\\
1.55003875096877	0\\
1.55013875346884	0\\
1.5502387559689	0\\
1.55033875846896	0\\
1.55043876096902	0\\
1.55053876346909	0\\
1.55063876596915	0\\
1.55073876846921	0\\
1.55083877096927	0\\
1.55093877346934	0\\
1.5510387759694	0\\
1.55113877846946	0\\
1.55123878096952	0\\
1.55133878346959	0\\
1.55143878596965	0\\
1.55153878846971	0\\
1.55163879096977	0\\
1.55173879346984	0\\
1.5518387959699	0\\
1.55193879846996	0\\
1.55203880097002	0\\
1.55213880347009	0\\
1.55223880597015	0\\
1.55233880847021	0\\
1.55243881097027	0\\
1.55253881347034	0\\
1.5526388159704	0\\
1.55273881847046	0\\
1.55283882097052	0\\
1.55293882347059	0\\
1.55303882597065	0\\
1.55313882847071	0\\
1.55323883097077	0\\
1.55333883347084	0\\
1.5534388359709	0\\
1.55353883847096	0\\
1.55363884097102	0\\
1.55373884347109	0\\
1.55383884597115	0\\
1.55393884847121	0\\
1.55403885097127	0\\
1.55413885347134	0\\
1.5542388559714	0\\
1.55433885847146	0\\
1.55443886097152	0\\
1.55453886347159	0\\
1.55463886597165	0\\
1.55473886847171	0\\
1.55483887097177	0\\
1.55493887347184	0\\
1.5550388759719	0\\
1.55513887847196	0\\
1.55523888097202	0\\
1.55533888347209	0\\
1.55543888597215	0\\
1.55553888847221	0\\
1.55563889097227	0\\
1.55573889347234	0\\
1.5558388959724	0\\
1.55593889847246	0\\
1.55603890097252	0\\
1.55613890347259	0\\
1.55623890597265	0\\
1.55633890847271	0\\
1.55643891097277	0\\
1.55653891347284	0\\
1.5566389159729	0\\
1.55673891847296	0\\
1.55683892097302	0\\
1.55693892347309	0\\
1.55703892597315	0\\
1.55713892847321	0\\
1.55723893097327	0\\
1.55733893347334	0\\
1.5574389359734	0\\
1.55753893847346	0\\
1.55763894097352	0\\
1.55773894347359	0\\
1.55783894597365	0\\
1.55793894847371	0\\
1.55803895097377	0\\
1.55813895347384	0\\
1.5582389559739	0\\
1.55833895847396	0\\
1.55843896097402	0\\
1.55853896347409	0\\
1.55863896597415	0\\
1.55873896847421	0\\
1.55883897097427	0\\
1.55893897347434	0\\
1.5590389759744	0\\
1.55913897847446	0\\
1.55923898097452	0\\
1.55933898347459	0\\
1.55943898597465	0\\
1.55953898847471	0\\
1.55963899097477	0\\
1.55973899347484	0\\
1.5598389959749	0\\
1.55993899847496	0\\
1.56003900097502	0\\
1.56013900347509	0\\
1.56023900597515	0\\
1.56033900847521	0\\
1.56043901097527	0\\
1.56053901347534	0\\
1.5606390159754	0\\
1.56073901847546	0\\
1.56083902097552	0\\
1.56093902347559	0\\
1.56103902597565	0\\
1.56113902847571	0\\
1.56123903097577	0\\
1.56133903347584	0\\
1.5614390359759	0\\
1.56153903847596	0\\
1.56163904097602	0\\
1.56173904347609	0\\
1.56183904597615	0\\
1.56193904847621	0\\
1.56203905097627	0\\
1.56213905347634	0\\
1.5622390559764	0\\
1.56233905847646	0\\
1.56243906097652	0\\
1.56253906347659	0\\
1.56263906597665	0\\
1.56273906847671	0\\
1.56283907097677	0\\
1.56293907347684	0\\
1.5630390759769	0\\
1.56313907847696	0\\
1.56323908097702	0\\
1.56333908347709	0\\
1.56343908597715	0\\
1.56353908847721	0\\
1.56363909097727	0\\
1.56373909347734	0\\
1.5638390959774	0\\
1.56393909847746	0\\
1.56403910097752	0\\
1.56413910347759	0\\
1.56423910597765	0\\
1.56433910847771	0\\
1.56443911097777	0\\
1.56453911347784	0\\
1.5646391159779	0\\
1.56473911847796	0\\
1.56483912097802	0\\
1.56493912347809	0\\
1.56503912597815	0\\
1.56513912847821	0\\
1.56523913097827	0\\
1.56533913347834	0\\
1.5654391359784	0\\
1.56553913847846	0\\
1.56563914097852	0\\
1.56573914347859	0\\
1.56583914597865	0\\
1.56593914847871	0\\
1.56603915097877	0\\
1.56613915347884	0\\
1.5662391559789	0\\
1.56633915847896	0\\
1.56643916097902	0\\
1.56653916347909	0\\
1.56663916597915	0\\
1.56673916847921	0\\
1.56683917097927	0\\
1.56693917347934	0\\
1.5670391759794	0\\
1.56713917847946	0\\
1.56723918097952	0\\
1.56733918347959	0\\
1.56743918597965	0\\
1.56753918847971	0\\
1.56763919097977	0\\
1.56773919347984	0\\
1.5678391959799	0\\
1.56793919847996	0\\
1.56803920098002	0\\
1.56813920348009	0\\
1.56823920598015	0\\
1.56833920848021	0\\
1.56843921098027	0\\
1.56853921348034	0\\
1.5686392159804	0\\
1.56873921848046	0\\
1.56883922098052	0\\
1.56893922348059	0\\
1.56903922598065	0\\
1.56913922848071	0\\
1.56923923098077	0\\
1.56933923348084	0\\
1.5694392359809	0\\
1.56953923848096	0\\
1.56963924098102	0\\
1.56973924348109	0\\
1.56983924598115	0\\
1.56993924848121	0\\
1.57003925098127	0\\
1.57013925348134	0\\
1.5702392559814	0\\
1.57033925848146	0\\
1.57043926098152	0\\
1.57053926348159	0\\
1.57063926598165	0\\
1.57073926848171	0\\
1.57083927098177	0\\
1.57093927348184	0\\
1.5710392759819	0\\
1.57113927848196	0\\
1.57123928098202	0\\
1.57133928348209	0\\
1.57143928598215	0\\
1.57153928848221	0\\
1.57163929098227	0\\
1.57173929348234	0\\
1.5718392959824	0\\
1.57193929848246	0\\
1.57203930098252	0\\
1.57213930348259	0\\
1.57223930598265	0\\
1.57233930848271	0\\
1.57243931098277	0\\
1.57253931348284	0\\
1.5726393159829	0\\
1.57273931848296	0\\
1.57283932098302	0\\
1.57293932348309	0\\
1.57303932598315	0\\
1.57313932848321	0\\
1.57323933098327	0\\
1.57333933348334	0\\
1.5734393359834	0\\
1.57353933848346	0\\
1.57363934098352	0\\
1.57373934348359	0\\
1.57383934598365	0\\
1.57393934848371	0\\
1.57403935098377	0\\
1.57413935348384	0\\
1.5742393559839	0\\
1.57433935848396	0\\
1.57443936098402	0\\
1.57453936348409	0\\
1.57463936598415	0\\
1.57473936848421	0\\
1.57483937098427	0\\
1.57493937348434	0\\
1.5750393759844	0\\
1.57513937848446	0\\
1.57523938098452	0\\
1.57533938348459	0\\
1.57543938598465	0\\
1.57553938848471	0\\
1.57563939098477	0\\
1.57573939348484	0\\
1.5758393959849	0\\
1.57593939848496	0\\
1.57603940098502	0\\
1.57613940348509	0\\
1.57623940598515	0\\
1.57633940848521	0\\
1.57643941098527	0\\
1.57653941348534	0\\
1.5766394159854	0\\
1.57673941848546	0\\
1.57683942098552	0\\
1.57693942348559	0\\
1.57703942598565	0\\
1.57713942848571	0\\
1.57723943098577	0\\
1.57733943348584	0\\
1.5774394359859	0\\
1.57753943848596	0\\
1.57763944098602	0\\
1.57773944348609	0\\
1.57783944598615	0\\
1.57793944848621	0\\
1.57803945098627	0\\
1.57813945348634	0\\
1.5782394559864	0\\
1.57833945848646	0\\
1.57843946098652	0\\
1.57853946348659	0\\
1.57863946598665	0\\
1.57873946848671	0\\
1.57883947098677	0\\
1.57893947348684	0\\
1.5790394759869	0\\
1.57913947848696	0\\
1.57923948098702	0\\
1.57933948348709	0\\
1.57943948598715	0\\
1.57953948848721	0\\
1.57963949098727	0\\
1.57973949348734	0\\
1.5798394959874	0\\
1.57993949848746	0\\
1.58003950098752	0\\
1.58013950348759	0\\
1.58023950598765	0\\
1.58033950848771	0\\
1.58043951098777	0\\
1.58053951348784	0\\
1.5806395159879	0\\
1.58073951848796	0\\
1.58083952098802	0\\
1.58093952348809	0\\
1.58103952598815	0\\
1.58113952848821	0\\
1.58123953098827	0\\
1.58133953348834	0\\
1.5814395359884	0\\
1.58153953848846	0\\
1.58163954098852	0\\
1.58173954348859	0\\
1.58183954598865	0\\
1.58193954848871	0\\
1.58203955098877	0\\
1.58213955348884	0\\
1.5822395559889	0\\
1.58233955848896	0\\
1.58243956098902	0\\
1.58253956348909	0\\
1.58263956598915	0\\
1.58273956848921	0\\
1.58283957098927	0\\
1.58293957348934	0\\
1.5830395759894	0\\
1.58313957848946	0\\
1.58323958098952	0\\
1.58333958348959	0\\
1.58343958598965	0\\
1.58353958848971	0\\
1.58363959098977	0\\
1.58373959348984	0\\
1.5838395959899	0\\
1.58393959848996	0\\
1.58403960099002	0\\
1.58413960349009	0\\
1.58423960599015	0\\
1.58433960849021	0\\
1.58443961099027	0\\
1.58453961349034	0\\
1.5846396159904	0\\
1.58473961849046	0\\
1.58483962099052	0\\
1.58493962349059	0\\
1.58503962599065	0\\
1.58513962849071	0\\
1.58523963099077	0\\
1.58533963349084	0\\
1.5854396359909	0\\
1.58553963849096	0\\
1.58563964099102	0\\
1.58573964349109	0\\
1.58583964599115	0\\
1.58593964849121	0\\
1.58603965099127	0\\
1.58613965349134	0\\
1.5862396559914	0\\
1.58633965849146	0\\
1.58643966099152	0\\
1.58653966349159	0\\
1.58663966599165	0\\
1.58673966849171	0\\
1.58683967099177	0\\
1.58693967349184	0\\
1.5870396759919	0\\
1.58713967849196	0\\
1.58723968099202	0\\
1.58733968349209	0\\
1.58743968599215	0\\
1.58753968849221	0\\
1.58763969099227	0\\
1.58773969349234	0\\
1.5878396959924	0\\
1.58793969849246	0\\
1.58803970099252	0\\
1.58813970349259	0\\
1.58823970599265	0\\
1.58833970849271	0\\
1.58843971099277	0\\
1.58853971349284	0\\
1.5886397159929	0\\
1.58873971849296	0\\
1.58883972099302	0\\
1.58893972349309	0\\
1.58903972599315	0\\
1.58913972849321	0\\
1.58923973099327	0\\
1.58933973349334	0\\
1.5894397359934	0\\
1.58953973849346	0\\
1.58963974099352	0\\
1.58973974349359	0\\
1.58983974599365	0\\
1.58993974849371	0\\
1.59003975099377	0\\
1.59013975349384	0\\
1.5902397559939	0\\
1.59033975849396	0\\
1.59043976099402	0\\
1.59053976349409	0\\
1.59063976599415	0\\
1.59073976849421	0\\
1.59083977099427	0\\
1.59093977349434	0\\
1.5910397759944	0\\
1.59113977849446	0\\
1.59123978099452	0\\
1.59133978349459	0\\
1.59143978599465	0\\
1.59153978849471	0\\
1.59163979099477	0\\
1.59173979349484	0\\
1.5918397959949	0\\
1.59193979849496	0\\
1.59203980099502	0\\
1.59213980349509	0\\
1.59223980599515	0\\
1.59233980849521	0\\
1.59243981099527	0\\
1.59253981349534	0\\
1.5926398159954	0\\
1.59273981849546	0\\
1.59283982099552	0\\
1.59293982349559	0\\
1.59303982599565	0\\
1.59313982849571	0\\
1.59323983099577	0\\
1.59333983349584	0\\
1.5934398359959	0\\
1.59353983849596	0\\
1.59363984099602	0\\
1.59373984349609	0\\
1.59383984599615	0\\
1.59393984849621	0\\
1.59403985099627	0\\
1.59413985349634	0\\
1.5942398559964	0\\
1.59433985849646	0\\
1.59443986099652	0\\
1.59453986349659	0\\
1.59463986599665	0\\
1.59473986849671	0\\
1.59483987099677	0\\
1.59493987349684	0\\
1.5950398759969	0\\
1.59513987849696	0\\
1.59523988099702	0\\
1.59533988349709	0\\
1.59543988599715	0\\
1.59553988849721	0\\
1.59563989099727	0\\
1.59573989349734	0\\
1.5958398959974	0\\
1.59593989849746	0\\
1.59603990099753	0\\
1.59613990349759	0\\
1.59623990599765	0\\
1.59633990849771	0\\
1.59643991099777	0\\
1.59653991349784	0\\
1.5966399159979	0\\
1.59673991849796	0\\
1.59683992099802	0\\
1.59693992349809	0\\
1.59703992599815	0\\
1.59713992849821	0\\
1.59723993099827	0\\
1.59733993349834	0\\
1.5974399359984	0\\
1.59753993849846	0\\
1.59763994099852	0\\
1.59773994349859	0\\
1.59783994599865	0\\
1.59793994849871	0\\
1.59803995099878	0\\
1.59813995349884	0\\
1.5982399559989	0\\
1.59833995849896	0\\
1.59843996099902	0\\
1.59853996349909	0\\
1.59863996599915	0\\
1.59873996849921	0\\
1.59883997099927	0\\
1.59893997349934	0\\
1.5990399759994	0\\
1.59913997849946	0\\
1.59923998099953	0\\
1.59933998349959	0\\
1.59943998599965	0\\
1.59953998849971	0\\
1.59963999099977	0\\
1.59973999349984	0\\
1.5998399959999	0\\
1.59993999849996	0\\
1.60004000100003	0\\
};
\addplot [color=mycolor2,solid,forget plot]
  table[row sep=crcr]{%
1.60004000100003	0\\
1.60014000350009	0\\
1.60024000600015	0\\
1.60034000850021	0\\
1.60044001100027	0\\
1.60054001350034	0\\
1.6006400160004	0\\
1.60074001850046	0\\
1.60084002100052	0\\
1.60094002350059	0\\
1.60104002600065	0\\
1.60114002850071	0\\
1.60124003100078	0\\
1.60134003350084	0\\
1.6014400360009	0\\
1.60154003850096	0\\
1.60164004100102	0\\
1.60174004350109	0\\
1.60184004600115	0\\
1.60194004850121	0\\
1.60204005100128	0\\
1.60214005350134	0\\
1.6022400560014	0\\
1.60234005850146	0\\
1.60244006100153	0\\
1.60254006350159	0\\
1.60264006600165	0\\
1.60274006850171	0\\
1.60284007100177	0\\
1.60294007350184	0\\
1.6030400760019	0\\
1.60314007850196	0\\
1.60324008100203	0\\
1.60334008350209	0\\
1.60344008600215	0\\
1.60354008850221	0\\
1.60364009100228	0\\
1.60374009350234	0\\
1.6038400960024	0\\
1.60394009850246	0\\
1.60404010100253	0\\
1.60414010350259	0\\
1.60424010600265	0\\
1.60434010850271	0\\
1.60444011100278	0\\
1.60454011350284	0\\
1.6046401160029	0\\
1.60474011850296	0\\
1.60484012100302	0\\
1.60494012350309	0\\
1.60504012600315	0\\
1.60514012850321	0\\
1.60524013100328	0\\
1.60534013350334	0\\
1.6054401360034	0\\
1.60554013850346	0\\
1.60564014100353	0\\
1.60574014350359	0\\
1.60584014600365	0\\
1.60594014850371	0\\
1.60604015100378	0\\
1.60614015350384	0\\
1.6062401560039	0\\
1.60634015850396	0\\
1.60644016100403	0\\
1.60654016350409	0\\
1.60664016600415	0\\
1.60674016850421	0\\
1.60684017100428	0\\
1.60694017350434	0\\
1.6070401760044	0\\
1.60714017850446	0\\
1.60724018100453	0\\
1.60734018350459	0\\
1.60744018600465	0\\
1.60754018850471	0\\
1.60764019100478	0\\
1.60774019350484	0\\
1.6078401960049	0\\
1.60794019850496	0\\
1.60804020100503	0\\
1.60814020350509	0\\
1.60824020600515	0\\
1.60834020850521	0\\
1.60844021100528	0\\
1.60854021350534	0\\
1.6086402160054	0\\
1.60874021850546	0\\
1.60884022100553	0\\
1.60894022350559	0\\
1.60904022600565	0\\
1.60914022850571	0\\
1.60924023100578	0\\
1.60934023350584	0\\
1.6094402360059	0\\
1.60954023850596	0\\
1.60964024100603	0\\
1.60974024350609	0\\
1.60984024600615	0\\
1.60994024850621	0\\
1.61004025100628	0\\
1.61014025350634	0\\
1.6102402560064	0\\
1.61034025850646	0\\
1.61044026100653	0\\
1.61054026350659	0\\
1.61064026600665	0\\
1.61074026850671	0\\
1.61084027100678	0\\
1.61094027350684	0\\
1.6110402760069	0\\
1.61114027850696	0\\
1.61124028100703	0\\
1.61134028350709	0\\
1.61144028600715	0\\
1.61154028850721	0\\
1.61164029100728	0\\
1.61174029350734	0\\
1.6118402960074	0\\
1.61194029850746	0\\
1.61204030100753	0\\
1.61214030350759	0\\
1.61224030600765	0\\
1.61234030850771	0\\
1.61244031100778	0\\
1.61254031350784	0\\
1.6126403160079	0\\
1.61274031850796	0\\
1.61284032100803	0\\
1.61294032350809	0\\
1.61304032600815	0\\
1.61314032850821	0\\
1.61324033100828	0\\
1.61334033350834	0\\
1.6134403360084	0\\
1.61354033850846	0\\
1.61364034100853	0\\
1.61374034350859	0\\
1.61384034600865	0\\
1.61394034850871	0\\
1.61404035100878	0\\
1.61414035350884	0\\
1.6142403560089	0\\
1.61434035850896	0\\
1.61444036100903	0\\
1.61454036350909	0\\
1.61464036600915	0\\
1.61474036850921	0\\
1.61484037100928	0\\
1.61494037350934	0\\
1.6150403760094	0\\
1.61514037850946	0\\
1.61524038100953	0\\
1.61534038350959	0\\
1.61544038600965	0\\
1.61554038850971	0\\
1.61564039100978	0\\
1.61574039350984	0\\
1.6158403960099	0\\
1.61594039850996	0\\
1.61604040101003	0\\
1.61614040351009	0\\
1.61624040601015	0\\
1.61634040851021	0\\
1.61644041101028	0\\
1.61654041351034	0\\
1.6166404160104	0\\
1.61674041851046	0\\
1.61684042101053	0\\
1.61694042351059	0\\
1.61704042601065	0\\
1.61714042851071	0\\
1.61724043101078	0\\
1.61734043351084	0\\
1.6174404360109	0\\
1.61754043851096	0\\
1.61764044101103	0\\
1.61774044351109	0\\
1.61784044601115	0\\
1.61794044851121	0\\
1.61804045101128	0\\
1.61814045351134	0\\
1.6182404560114	0\\
1.61834045851146	0\\
1.61844046101153	0\\
1.61854046351159	0\\
1.61864046601165	0\\
1.61874046851171	0\\
1.61884047101178	0\\
1.61894047351184	0\\
1.6190404760119	0\\
1.61914047851196	0\\
1.61924048101203	0\\
1.61934048351209	0\\
1.61944048601215	0\\
1.61954048851221	0\\
1.61964049101228	0\\
1.61974049351234	0\\
1.6198404960124	0\\
1.61994049851246	0\\
1.62004050101253	0\\
1.62014050351259	0\\
1.62024050601265	0\\
1.62034050851271	0\\
1.62044051101278	0\\
1.62054051351284	0\\
1.6206405160129	0\\
1.62074051851296	0\\
1.62084052101303	0\\
1.62094052351309	0\\
1.62104052601315	0\\
1.62114052851321	0\\
1.62124053101328	0\\
1.62134053351334	0\\
1.6214405360134	0\\
1.62154053851346	0\\
1.62164054101353	0\\
1.62174054351359	0\\
1.62184054601365	0\\
1.62194054851371	0\\
1.62204055101378	0\\
1.62214055351384	0\\
1.6222405560139	0\\
1.62234055851396	0\\
1.62244056101403	0\\
1.62254056351409	0\\
1.62264056601415	0\\
1.62274056851421	0\\
1.62284057101428	0\\
1.62294057351434	0\\
1.6230405760144	0\\
1.62314057851446	0\\
1.62324058101453	0\\
1.62334058351459	0\\
1.62344058601465	0\\
1.62354058851471	0\\
1.62364059101478	0\\
1.62374059351484	0\\
1.6238405960149	0\\
1.62394059851496	0\\
1.62404060101503	0\\
1.62414060351509	0\\
1.62424060601515	0\\
1.62434060851521	0\\
1.62444061101528	0\\
1.62454061351534	0\\
1.6246406160154	0\\
1.62474061851546	0\\
1.62484062101553	0\\
1.62494062351559	0\\
1.62504062601565	0\\
1.62514062851571	0\\
1.62524063101578	0\\
1.62534063351584	0\\
1.6254406360159	0\\
1.62554063851596	0\\
1.62564064101603	0\\
1.62574064351609	0\\
1.62584064601615	0\\
1.62594064851621	0\\
1.62604065101628	0\\
1.62614065351634	0\\
1.6262406560164	0\\
1.62634065851646	0\\
1.62644066101653	0\\
1.62654066351659	0\\
1.62664066601665	0\\
1.62674066851671	0\\
1.62684067101678	0\\
1.62694067351684	0\\
1.6270406760169	0\\
1.62714067851696	0\\
1.62724068101703	0\\
1.62734068351709	0\\
1.62744068601715	0\\
1.62754068851721	0\\
1.62764069101728	0\\
1.62774069351734	0\\
1.6278406960174	0\\
1.62794069851746	0\\
1.62804070101753	0\\
1.62814070351759	0\\
1.62824070601765	0\\
1.62834070851771	0\\
1.62844071101778	0\\
1.62854071351784	0\\
1.6286407160179	0\\
1.62874071851796	0\\
1.62884072101803	0\\
1.62894072351809	0\\
1.62904072601815	0\\
1.62914072851821	0\\
1.62924073101828	0\\
1.62934073351834	0\\
1.6294407360184	0\\
1.62954073851846	0\\
1.62964074101853	0\\
1.62974074351859	0\\
1.62984074601865	0\\
1.62994074851871	0\\
1.63004075101878	0\\
1.63014075351884	0\\
1.6302407560189	0\\
1.63034075851896	0\\
1.63044076101903	0\\
1.63054076351909	0\\
1.63064076601915	0\\
1.63074076851921	0\\
1.63084077101928	0\\
1.63094077351934	0\\
1.6310407760194	0\\
1.63114077851946	0\\
1.63124078101953	0\\
1.63134078351959	0\\
1.63144078601965	0\\
1.63154078851971	0\\
1.63164079101978	0\\
1.63174079351984	0\\
1.6318407960199	0\\
1.63194079851996	0\\
1.63204080102003	0\\
1.63214080352009	0\\
1.63224080602015	0\\
1.63234080852021	0\\
1.63244081102028	0\\
1.63254081352034	0\\
1.6326408160204	0\\
1.63274081852046	0\\
1.63284082102053	0\\
1.63294082352059	0\\
1.63304082602065	0\\
1.63314082852071	0\\
1.63324083102078	0\\
1.63334083352084	0\\
1.6334408360209	0\\
1.63354083852096	0\\
1.63364084102103	0\\
1.63374084352109	0\\
1.63384084602115	0\\
1.63394084852121	0\\
1.63404085102128	0\\
1.63414085352134	0\\
1.6342408560214	0\\
1.63434085852146	0\\
1.63444086102153	0\\
1.63454086352159	0\\
1.63464086602165	0\\
1.63474086852171	0\\
1.63484087102178	0\\
1.63494087352184	0\\
1.6350408760219	0\\
1.63514087852196	0\\
1.63524088102203	0\\
1.63534088352209	0\\
1.63544088602215	0\\
1.63554088852221	0\\
1.63564089102228	0\\
1.63574089352234	0\\
1.6358408960224	0\\
1.63594089852246	0\\
1.63604090102253	0\\
1.63614090352259	0\\
1.63624090602265	0\\
1.63634090852271	0\\
1.63644091102278	0\\
1.63654091352284	0\\
1.6366409160229	0\\
1.63674091852296	0\\
1.63684092102303	0\\
1.63694092352309	0\\
1.63704092602315	0\\
1.63714092852321	0\\
1.63724093102328	0\\
1.63734093352334	0\\
1.6374409360234	0\\
1.63754093852346	0\\
1.63764094102353	0\\
1.63774094352359	0\\
1.63784094602365	0\\
1.63794094852371	0\\
1.63804095102378	0\\
1.63814095352384	0\\
1.6382409560239	0\\
1.63834095852396	0\\
1.63844096102403	0\\
1.63854096352409	0\\
1.63864096602415	0\\
1.63874096852421	0\\
1.63884097102428	0\\
1.63894097352434	0\\
1.6390409760244	0\\
1.63914097852446	0\\
1.63924098102453	0\\
1.63934098352459	0\\
1.63944098602465	0\\
1.63954098852471	0\\
1.63964099102478	0\\
1.63974099352484	0\\
1.6398409960249	0\\
1.63994099852496	0\\
1.64004100102503	0\\
1.64014100352509	0\\
1.64024100602515	0\\
1.64034100852521	0\\
1.64044101102528	0\\
1.64054101352534	0\\
1.6406410160254	0\\
1.64074101852546	0\\
1.64084102102553	0\\
1.64094102352559	0\\
1.64104102602565	0\\
1.64114102852571	0\\
1.64124103102578	0\\
1.64134103352584	0\\
1.6414410360259	0\\
1.64154103852596	0\\
1.64164104102603	0\\
1.64174104352609	0\\
1.64184104602615	0\\
1.64194104852621	0\\
1.64204105102628	0\\
1.64214105352634	0\\
1.6422410560264	0\\
1.64234105852646	0\\
1.64244106102653	0\\
1.64254106352659	0\\
1.64264106602665	0\\
1.64274106852671	0\\
1.64284107102678	0\\
1.64294107352684	0\\
1.6430410760269	0\\
1.64314107852696	0\\
1.64324108102703	0\\
1.64334108352709	0\\
1.64344108602715	0\\
1.64354108852721	0\\
1.64364109102728	0\\
1.64374109352734	0\\
1.6438410960274	0\\
1.64394109852746	0\\
1.64404110102753	0\\
1.64414110352759	0\\
1.64424110602765	0\\
1.64434110852771	0\\
1.64444111102778	0\\
1.64454111352784	0\\
1.6446411160279	0\\
1.64474111852796	0\\
1.64484112102803	0\\
1.64494112352809	0\\
1.64504112602815	0\\
1.64514112852821	0\\
1.64524113102828	0\\
1.64534113352834	0\\
1.6454411360284	0\\
1.64554113852846	0\\
1.64564114102853	0\\
1.64574114352859	0\\
1.64584114602865	0\\
1.64594114852871	0\\
1.64604115102878	0\\
1.64614115352884	0\\
1.6462411560289	0\\
1.64634115852896	0\\
1.64644116102903	0\\
1.64654116352909	0\\
1.64664116602915	0\\
1.64674116852921	0\\
1.64684117102928	0\\
1.64694117352934	0\\
1.6470411760294	0\\
1.64714117852946	0\\
1.64724118102953	0\\
1.64734118352959	0\\
1.64744118602965	0\\
1.64754118852971	0\\
1.64764119102978	0\\
1.64774119352984	0\\
1.6478411960299	0\\
1.64794119852996	0\\
1.64804120103003	0\\
1.64814120353009	0\\
1.64824120603015	0\\
1.64834120853021	0\\
1.64844121103028	0\\
1.64854121353034	0\\
1.6486412160304	0\\
1.64874121853046	0\\
1.64884122103053	0\\
1.64894122353059	0\\
1.64904122603065	0\\
1.64914122853071	0\\
1.64924123103078	0\\
1.64934123353084	0\\
1.6494412360309	0\\
1.64954123853096	0\\
1.64964124103103	0\\
1.64974124353109	0\\
1.64984124603115	0\\
1.64994124853121	0\\
1.65004125103128	0\\
1.65014125353134	0\\
1.6502412560314	0\\
1.65034125853146	0\\
1.65044126103153	0\\
1.65054126353159	0\\
1.65064126603165	0\\
1.65074126853171	0\\
1.65084127103178	0\\
1.65094127353184	0\\
1.6510412760319	0\\
1.65114127853196	0\\
1.65124128103203	0\\
1.65134128353209	0\\
1.65144128603215	0\\
1.65154128853221	0\\
1.65164129103228	0\\
1.65174129353234	0\\
1.6518412960324	0\\
1.65194129853246	0\\
1.65204130103253	0\\
1.65214130353259	0\\
1.65224130603265	0\\
1.65234130853271	0\\
1.65244131103278	0\\
1.65254131353284	0\\
1.6526413160329	0\\
1.65274131853296	0\\
1.65284132103303	0\\
1.65294132353309	0\\
1.65304132603315	0\\
1.65314132853321	0\\
1.65324133103328	0\\
1.65334133353334	0\\
1.6534413360334	0\\
1.65354133853346	0\\
1.65364134103353	0\\
1.65374134353359	0\\
1.65384134603365	0\\
1.65394134853371	0\\
1.65404135103378	0\\
1.65414135353384	0\\
1.6542413560339	0\\
1.65434135853396	0\\
1.65444136103403	0\\
1.65454136353409	0\\
1.65464136603415	0\\
1.65474136853421	0\\
1.65484137103428	0\\
1.65494137353434	0\\
1.6550413760344	0\\
1.65514137853446	0\\
1.65524138103453	0\\
1.65534138353459	0\\
1.65544138603465	0\\
1.65554138853471	0\\
1.65564139103478	0\\
1.65574139353484	0\\
1.6558413960349	0\\
1.65594139853496	0\\
1.65604140103503	0\\
1.65614140353509	0\\
1.65624140603515	0\\
1.65634140853521	0\\
1.65644141103528	0\\
1.65654141353534	0\\
1.6566414160354	0\\
1.65674141853546	0\\
1.65684142103553	0\\
1.65694142353559	0\\
1.65704142603565	0\\
1.65714142853571	0\\
1.65724143103578	0\\
1.65734143353584	0\\
1.6574414360359	0\\
1.65754143853596	0\\
1.65764144103603	0\\
1.65774144353609	0\\
1.65784144603615	0\\
1.65794144853621	0\\
1.65804145103628	0\\
1.65814145353634	0\\
1.6582414560364	0\\
1.65834145853646	0\\
1.65844146103653	0\\
1.65854146353659	0\\
1.65864146603665	0\\
1.65874146853671	0\\
1.65884147103678	0\\
1.65894147353684	0\\
1.6590414760369	0\\
1.65914147853696	0\\
1.65924148103703	0\\
1.65934148353709	0\\
1.65944148603715	0\\
1.65954148853721	0\\
1.65964149103728	0\\
1.65974149353734	0\\
1.6598414960374	0\\
1.65994149853746	0\\
1.66004150103753	0\\
1.66014150353759	0\\
1.66024150603765	0\\
1.66034150853771	0\\
1.66044151103778	0\\
1.66054151353784	0\\
1.6606415160379	0\\
1.66074151853796	0\\
1.66084152103803	0\\
1.66094152353809	0\\
1.66104152603815	0\\
1.66114152853821	0\\
1.66124153103828	0\\
1.66134153353834	0\\
1.6614415360384	0\\
1.66154153853846	0\\
1.66164154103853	0\\
1.66174154353859	0\\
1.66184154603865	0\\
1.66194154853871	0\\
1.66204155103878	0\\
1.66214155353884	0\\
1.6622415560389	0\\
1.66234155853896	0\\
1.66244156103903	0\\
1.66254156353909	0\\
1.66264156603915	0\\
1.66274156853921	0\\
1.66284157103928	0\\
1.66294157353934	0\\
1.6630415760394	0\\
1.66314157853946	0\\
1.66324158103953	0\\
1.66334158353959	0\\
1.66344158603965	0\\
1.66354158853971	0\\
1.66364159103978	0\\
1.66374159353984	0\\
1.6638415960399	0\\
1.66394159853996	0\\
1.66404160104003	0\\
1.66414160354009	0\\
1.66424160604015	0\\
1.66434160854021	0\\
1.66444161104028	0\\
1.66454161354034	0\\
1.6646416160404	0\\
1.66474161854046	0\\
1.66484162104053	0\\
1.66494162354059	0\\
1.66504162604065	0\\
1.66514162854071	0\\
1.66524163104078	0\\
1.66534163354084	0\\
1.6654416360409	0\\
1.66554163854096	0\\
1.66564164104103	0\\
1.66574164354109	0\\
1.66584164604115	0\\
1.66594164854121	0\\
1.66604165104128	0\\
1.66614165354134	0\\
1.6662416560414	0\\
1.66634165854146	0\\
1.66644166104153	0\\
1.66654166354159	0\\
1.66664166604165	0\\
1.66674166854171	0\\
1.66684167104178	0\\
1.66694167354184	0\\
1.6670416760419	0\\
1.66714167854196	0\\
1.66724168104203	0\\
1.66734168354209	0\\
1.66744168604215	0\\
1.66754168854221	0\\
1.66764169104228	0\\
1.66774169354234	0\\
1.6678416960424	0\\
1.66794169854246	0\\
1.66804170104253	0\\
1.66814170354259	0\\
1.66824170604265	0\\
1.66834170854271	0\\
1.66844171104278	0\\
1.66854171354284	0\\
1.6686417160429	0\\
1.66874171854296	0\\
1.66884172104303	0\\
1.66894172354309	0\\
1.66904172604315	0\\
1.66914172854321	0\\
1.66924173104328	0\\
1.66934173354334	0\\
1.6694417360434	0\\
1.66954173854346	0\\
1.66964174104353	0\\
1.66974174354359	0\\
1.66984174604365	0\\
1.66994174854371	0\\
1.67004175104378	0\\
1.67014175354384	0\\
1.6702417560439	0\\
1.67034175854396	0\\
1.67044176104403	0\\
1.67054176354409	0\\
1.67064176604415	0\\
1.67074176854421	0\\
1.67084177104428	0\\
1.67094177354434	0\\
1.6710417760444	0\\
1.67114177854446	0\\
1.67124178104453	0\\
1.67134178354459	0\\
1.67144178604465	0\\
1.67154178854471	0\\
1.67164179104478	0\\
1.67174179354484	0\\
1.6718417960449	0\\
1.67194179854496	0\\
1.67204180104503	0\\
1.67214180354509	0\\
1.67224180604515	0\\
1.67234180854521	0\\
1.67244181104528	0\\
1.67254181354534	0\\
1.6726418160454	0\\
1.67274181854546	0\\
1.67284182104553	0\\
1.67294182354559	0\\
1.67304182604565	0\\
1.67314182854571	0\\
1.67324183104578	0\\
1.67334183354584	0\\
1.6734418360459	0\\
1.67354183854596	0\\
1.67364184104603	0\\
1.67374184354609	0\\
1.67384184604615	0\\
1.67394184854621	0\\
1.67404185104628	0\\
1.67414185354634	0\\
1.6742418560464	0\\
1.67434185854646	0\\
1.67444186104653	0\\
1.67454186354659	0\\
1.67464186604665	0\\
1.67474186854671	0\\
1.67484187104678	0\\
1.67494187354684	0\\
1.6750418760469	0\\
1.67514187854696	0\\
1.67524188104703	0\\
1.67534188354709	0\\
1.67544188604715	0\\
1.67554188854721	0\\
1.67564189104728	0\\
1.67574189354734	0\\
1.6758418960474	0\\
1.67594189854746	0\\
1.67604190104753	0\\
1.67614190354759	0\\
1.67624190604765	0\\
1.67634190854771	0\\
1.67644191104778	0\\
1.67654191354784	0\\
1.6766419160479	0\\
1.67674191854796	0\\
1.67684192104803	0\\
1.67694192354809	0\\
1.67704192604815	0\\
1.67714192854821	0\\
1.67724193104828	0\\
1.67734193354834	0\\
1.6774419360484	0\\
1.67754193854846	0\\
1.67764194104853	0\\
1.67774194354859	0\\
1.67784194604865	0\\
1.67794194854871	0\\
1.67804195104878	0\\
1.67814195354884	0\\
1.6782419560489	0\\
1.67834195854896	0\\
1.67844196104903	0\\
1.67854196354909	0\\
1.67864196604915	0\\
1.67874196854921	0\\
1.67884197104928	0\\
1.67894197354934	0\\
1.6790419760494	0\\
1.67914197854946	0\\
1.67924198104953	0\\
1.67934198354959	0\\
1.67944198604965	0\\
1.67954198854971	0\\
1.67964199104978	0\\
1.67974199354984	0\\
1.6798419960499	0\\
1.67994199854996	0\\
1.68004200105003	0\\
1.68014200355009	0\\
1.68024200605015	0\\
1.68034200855021	0\\
1.68044201105028	0\\
1.68054201355034	0\\
1.6806420160504	0\\
1.68074201855046	0\\
1.68084202105053	0\\
1.68094202355059	0\\
1.68104202605065	0\\
1.68114202855071	0\\
1.68124203105078	0\\
1.68134203355084	0\\
1.6814420360509	0\\
1.68154203855096	0\\
1.68164204105103	0\\
1.68174204355109	0\\
1.68184204605115	0\\
1.68194204855121	0\\
1.68204205105128	0\\
1.68214205355134	0\\
1.6822420560514	0\\
1.68234205855146	0\\
1.68244206105153	0\\
1.68254206355159	0\\
1.68264206605165	0\\
1.68274206855171	0\\
1.68284207105178	0\\
1.68294207355184	0\\
1.6830420760519	0\\
1.68314207855196	0\\
1.68324208105203	0\\
1.68334208355209	0\\
1.68344208605215	0\\
1.68354208855221	0\\
1.68364209105228	0\\
1.68374209355234	0\\
1.6838420960524	0\\
1.68394209855246	0\\
1.68404210105253	0\\
1.68414210355259	0\\
1.68424210605265	0\\
1.68434210855271	0\\
1.68444211105278	0\\
1.68454211355284	0\\
1.6846421160529	0\\
1.68474211855296	0\\
1.68484212105303	0\\
1.68494212355309	0\\
1.68504212605315	0\\
1.68514212855321	0\\
1.68524213105328	0\\
1.68534213355334	0\\
1.6854421360534	0\\
1.68554213855346	0\\
1.68564214105353	0\\
1.68574214355359	0\\
1.68584214605365	0\\
1.68594214855371	0\\
1.68604215105378	0\\
1.68614215355384	0\\
1.6862421560539	0\\
1.68634215855396	0\\
1.68644216105403	0\\
1.68654216355409	0\\
1.68664216605415	0\\
1.68674216855421	0\\
1.68684217105428	0\\
1.68694217355434	0\\
1.6870421760544	0\\
1.68714217855446	0\\
1.68724218105453	0\\
1.68734218355459	0\\
1.68744218605465	0\\
1.68754218855471	0\\
1.68764219105478	0\\
1.68774219355484	0\\
1.6878421960549	0\\
1.68794219855496	0\\
1.68804220105503	0\\
1.68814220355509	0\\
1.68824220605515	0\\
1.68834220855521	0\\
1.68844221105528	0\\
1.68854221355534	0\\
1.6886422160554	0\\
1.68874221855546	0\\
1.68884222105553	0\\
1.68894222355559	0\\
1.68904222605565	0\\
1.68914222855571	0\\
1.68924223105578	0\\
1.68934223355584	0\\
1.6894422360559	0\\
1.68954223855596	0\\
1.68964224105603	0\\
1.68974224355609	0\\
1.68984224605615	0\\
1.68994224855621	0\\
1.69004225105628	0\\
1.69014225355634	0\\
1.6902422560564	0\\
1.69034225855646	0\\
1.69044226105653	0\\
1.69054226355659	0\\
1.69064226605665	0\\
1.69074226855671	0\\
1.69084227105678	0\\
1.69094227355684	0\\
1.6910422760569	0\\
1.69114227855696	0\\
1.69124228105703	0\\
1.69134228355709	0\\
1.69144228605715	0\\
1.69154228855721	0\\
1.69164229105728	0\\
1.69174229355734	0\\
1.6918422960574	0\\
1.69194229855746	0\\
1.69204230105753	0\\
1.69214230355759	0\\
1.69224230605765	0\\
1.69234230855771	0\\
1.69244231105778	0\\
1.69254231355784	0\\
1.6926423160579	0\\
1.69274231855796	0\\
1.69284232105803	0\\
1.69294232355809	0\\
1.69304232605815	0\\
1.69314232855821	0\\
1.69324233105828	0\\
1.69334233355834	0\\
1.6934423360584	0\\
1.69354233855846	0\\
1.69364234105853	0\\
1.69374234355859	0\\
1.69384234605865	0\\
1.69394234855871	0\\
1.69404235105878	0\\
1.69414235355884	0\\
1.6942423560589	0\\
1.69434235855896	0\\
1.69444236105903	0\\
1.69454236355909	0\\
1.69464236605915	0\\
1.69474236855921	0\\
1.69484237105928	0\\
1.69494237355934	0\\
1.6950423760594	0\\
1.69514237855946	0\\
1.69524238105953	0\\
1.69534238355959	0\\
1.69544238605965	0\\
1.69554238855971	0\\
1.69564239105978	0\\
1.69574239355984	0\\
1.6958423960599	0\\
1.69594239855996	0\\
1.69604240106003	0\\
1.69614240356009	0\\
1.69624240606015	0\\
1.69634240856021	0\\
1.69644241106028	0\\
1.69654241356034	0\\
1.6966424160604	0\\
1.69674241856046	0\\
1.69684242106053	0\\
1.69694242356059	0\\
1.69704242606065	0\\
1.69714242856071	0\\
1.69724243106078	0\\
1.69734243356084	0\\
1.6974424360609	0\\
1.69754243856096	0\\
1.69764244106103	0\\
1.69774244356109	0\\
1.69784244606115	0\\
1.69794244856121	0\\
1.69804245106128	0\\
1.69814245356134	0\\
1.6982424560614	0\\
1.69834245856146	0\\
1.69844246106153	0\\
1.69854246356159	0\\
1.69864246606165	0\\
1.69874246856171	0\\
1.69884247106178	0\\
1.69894247356184	0\\
1.6990424760619	0\\
1.69914247856196	0\\
1.69924248106203	0\\
1.69934248356209	0\\
1.69944248606215	0\\
1.69954248856221	0\\
1.69964249106228	0\\
1.69974249356234	0\\
1.6998424960624	0\\
1.69994249856246	0\\
1.70004250106253	0\\
1.70014250356259	0\\
1.70024250606265	0\\
1.70034250856271	0\\
1.70044251106278	0\\
1.70054251356284	0\\
1.7006425160629	0\\
1.70074251856296	0\\
1.70084252106303	0\\
1.70094252356309	0\\
1.70104252606315	0\\
1.70114252856321	0\\
1.70124253106328	0\\
1.70134253356334	0\\
1.7014425360634	0\\
1.70154253856346	0\\
1.70164254106353	0\\
1.70174254356359	0\\
1.70184254606365	0\\
1.70194254856371	0\\
1.70204255106378	0\\
1.70214255356384	0\\
1.7022425560639	0\\
1.70234255856396	0\\
1.70244256106403	0\\
1.70254256356409	0\\
1.70264256606415	0\\
1.70274256856421	0\\
1.70284257106428	0\\
1.70294257356434	0\\
1.7030425760644	0\\
1.70314257856446	0\\
1.70324258106453	0\\
1.70334258356459	0\\
1.70344258606465	0\\
1.70354258856471	0\\
1.70364259106478	0\\
1.70374259356484	0\\
1.7038425960649	0\\
1.70394259856496	0\\
1.70404260106503	0\\
1.70414260356509	0\\
1.70424260606515	0\\
1.70434260856521	0\\
1.70444261106528	0\\
1.70454261356534	0\\
1.7046426160654	0\\
1.70474261856546	0\\
1.70484262106553	0\\
1.70494262356559	0\\
1.70504262606565	0\\
1.70514262856571	0\\
1.70524263106578	0\\
1.70534263356584	0\\
1.7054426360659	0\\
1.70554263856596	0\\
1.70564264106603	0\\
1.70574264356609	0\\
1.70584264606615	0\\
1.70594264856621	0\\
1.70604265106628	0\\
1.70614265356634	0\\
1.7062426560664	0\\
1.70634265856646	0\\
1.70644266106653	0\\
1.70654266356659	0\\
1.70664266606665	0\\
1.70674266856671	0\\
1.70684267106678	0\\
1.70694267356684	0\\
1.7070426760669	0\\
1.70714267856696	0\\
1.70724268106703	0\\
1.70734268356709	0\\
1.70744268606715	0\\
1.70754268856721	0\\
1.70764269106728	0\\
1.70774269356734	0\\
1.7078426960674	0\\
1.70794269856746	0\\
1.70804270106753	0\\
1.70814270356759	0\\
1.70824270606765	0\\
1.70834270856771	0\\
1.70844271106778	0\\
1.70854271356784	0\\
1.7086427160679	0\\
1.70874271856796	0\\
1.70884272106803	0\\
1.70894272356809	0\\
1.70904272606815	0\\
1.70914272856821	0\\
1.70924273106828	0\\
1.70934273356834	0\\
1.7094427360684	0\\
1.70954273856846	0\\
1.70964274106853	0\\
1.70974274356859	0\\
1.70984274606865	0\\
1.70994274856871	0\\
1.71004275106878	0\\
1.71014275356884	0\\
1.7102427560689	0\\
1.71034275856896	0\\
1.71044276106903	0\\
1.71054276356909	0\\
1.71064276606915	0\\
1.71074276856921	0\\
1.71084277106928	0\\
1.71094277356934	0\\
1.7110427760694	0\\
1.71114277856946	0\\
1.71124278106953	0\\
1.71134278356959	0\\
1.71144278606965	0\\
1.71154278856971	0\\
1.71164279106978	0\\
1.71174279356984	0\\
1.7118427960699	0\\
1.71194279856996	0\\
1.71204280107003	0\\
1.71214280357009	0\\
1.71224280607015	0\\
1.71234280857021	0\\
1.71244281107028	0\\
1.71254281357034	0\\
1.7126428160704	0\\
1.71274281857046	0\\
1.71284282107053	0\\
1.71294282357059	0\\
1.71304282607065	0\\
1.71314282857071	0\\
1.71324283107078	0\\
1.71334283357084	0\\
1.7134428360709	0\\
1.71354283857096	0\\
1.71364284107103	0\\
1.71374284357109	0\\
1.71384284607115	0\\
1.71394284857121	0\\
1.71404285107128	0\\
1.71414285357134	0\\
1.7142428560714	0\\
1.71434285857146	0\\
1.71444286107153	0\\
1.71454286357159	0\\
1.71464286607165	0\\
1.71474286857171	0\\
1.71484287107178	0\\
1.71494287357184	0\\
1.7150428760719	0\\
1.71514287857196	0\\
1.71524288107203	0\\
1.71534288357209	0\\
1.71544288607215	0\\
1.71554288857221	0\\
1.71564289107228	0\\
1.71574289357234	0\\
1.7158428960724	0\\
1.71594289857246	0\\
1.71604290107253	0\\
1.71614290357259	0\\
1.71624290607265	0\\
1.71634290857271	0\\
1.71644291107278	0\\
1.71654291357284	0\\
1.7166429160729	0\\
1.71674291857296	0\\
1.71684292107303	0\\
1.71694292357309	0\\
1.71704292607315	0\\
1.71714292857321	0\\
1.71724293107328	0\\
1.71734293357334	0\\
1.7174429360734	0\\
1.71754293857346	0\\
1.71764294107353	0\\
1.71774294357359	0\\
1.71784294607365	0\\
1.71794294857371	0\\
1.71804295107378	0\\
1.71814295357384	0\\
1.7182429560739	0\\
1.71834295857396	0\\
1.71844296107403	0\\
1.71854296357409	0\\
1.71864296607415	0\\
1.71874296857421	0\\
1.71884297107428	0\\
1.71894297357434	0\\
1.7190429760744	0\\
1.71914297857446	0\\
1.71924298107453	0\\
1.71934298357459	0\\
1.71944298607465	0\\
1.71954298857471	0\\
1.71964299107478	0\\
1.71974299357484	0\\
1.7198429960749	0\\
1.71994299857496	0\\
1.72004300107503	0\\
1.72014300357509	0\\
1.72024300607515	0\\
1.72034300857521	0\\
1.72044301107528	0\\
1.72054301357534	0\\
1.7206430160754	0\\
1.72074301857546	0\\
1.72084302107553	0\\
1.72094302357559	0\\
1.72104302607565	0\\
1.72114302857571	0\\
1.72124303107578	0\\
1.72134303357584	0\\
1.7214430360759	0\\
1.72154303857596	0\\
1.72164304107603	0\\
1.72174304357609	0\\
1.72184304607615	0\\
1.72194304857621	0\\
1.72204305107628	0\\
1.72214305357634	0\\
1.7222430560764	0\\
1.72234305857646	0\\
1.72244306107653	0\\
1.72254306357659	0\\
1.72264306607665	0\\
1.72274306857671	0\\
1.72284307107678	0\\
1.72294307357684	0\\
1.7230430760769	0\\
1.72314307857696	0\\
1.72324308107703	0\\
1.72334308357709	0\\
1.72344308607715	0\\
1.72354308857721	0\\
1.72364309107728	0\\
1.72374309357734	0\\
1.7238430960774	0\\
1.72394309857746	0\\
1.72404310107753	0\\
1.72414310357759	0\\
1.72424310607765	0\\
1.72434310857771	0\\
1.72444311107778	0\\
1.72454311357784	0\\
1.7246431160779	0\\
1.72474311857796	0\\
1.72484312107803	0\\
1.72494312357809	0\\
1.72504312607815	0\\
1.72514312857821	0\\
1.72524313107828	0\\
1.72534313357834	0\\
1.7254431360784	0\\
1.72554313857846	0\\
1.72564314107853	0\\
1.72574314357859	0\\
1.72584314607865	0\\
1.72594314857871	0\\
1.72604315107878	0\\
1.72614315357884	0\\
1.7262431560789	0\\
1.72634315857896	0\\
1.72644316107903	0\\
1.72654316357909	0\\
1.72664316607915	0\\
1.72674316857921	0\\
1.72684317107928	0\\
1.72694317357934	0\\
1.7270431760794	0\\
1.72714317857946	0\\
1.72724318107953	0\\
1.72734318357959	0\\
1.72744318607965	0\\
1.72754318857971	0\\
1.72764319107978	0\\
1.72774319357984	0\\
1.7278431960799	0\\
1.72794319857996	0\\
1.72804320108003	0\\
1.72814320358009	0\\
1.72824320608015	0\\
1.72834320858021	0\\
1.72844321108028	0\\
1.72854321358034	0\\
1.7286432160804	0\\
1.72874321858046	0\\
1.72884322108053	0\\
1.72894322358059	0\\
1.72904322608065	0\\
1.72914322858071	0\\
1.72924323108078	0\\
1.72934323358084	0\\
1.7294432360809	0\\
1.72954323858096	0\\
1.72964324108103	0\\
1.72974324358109	0\\
1.72984324608115	0\\
1.72994324858121	0\\
1.73004325108128	0\\
1.73014325358134	0\\
1.7302432560814	0\\
1.73034325858146	0\\
1.73044326108153	0\\
1.73054326358159	0\\
1.73064326608165	0\\
1.73074326858171	0\\
1.73084327108178	0\\
1.73094327358184	0\\
1.7310432760819	0\\
1.73114327858196	0\\
1.73124328108203	0\\
1.73134328358209	0\\
1.73144328608215	0\\
1.73154328858221	0\\
1.73164329108228	0\\
1.73174329358234	0\\
1.7318432960824	0\\
1.73194329858246	0\\
1.73204330108253	0\\
1.73214330358259	0\\
1.73224330608265	0\\
1.73234330858271	0\\
1.73244331108278	0\\
1.73254331358284	0\\
1.7326433160829	0\\
1.73274331858296	0\\
1.73284332108303	0\\
1.73294332358309	0\\
1.73304332608315	0\\
1.73314332858321	0\\
1.73324333108328	0\\
1.73334333358334	0\\
1.7334433360834	0\\
1.73354333858346	0\\
1.73364334108353	0\\
1.73374334358359	0\\
1.73384334608365	0\\
1.73394334858371	0\\
1.73404335108378	0\\
1.73414335358384	0\\
1.7342433560839	0\\
1.73434335858396	0\\
1.73444336108403	0\\
1.73454336358409	0\\
1.73464336608415	0\\
1.73474336858421	0\\
1.73484337108428	0\\
1.73494337358434	0\\
1.7350433760844	0\\
1.73514337858446	0\\
1.73524338108453	0\\
1.73534338358459	0\\
1.73544338608465	0\\
1.73554338858471	0\\
1.73564339108478	0\\
1.73574339358484	0\\
1.7358433960849	0\\
1.73594339858496	0\\
1.73604340108503	0\\
1.73614340358509	0\\
1.73624340608515	0\\
1.73634340858521	0\\
1.73644341108528	0\\
1.73654341358534	0\\
1.7366434160854	0\\
1.73674341858546	0\\
1.73684342108553	0\\
1.73694342358559	0\\
1.73704342608565	0\\
1.73714342858571	0\\
1.73724343108578	0\\
1.73734343358584	0\\
1.7374434360859	0\\
1.73754343858596	0\\
1.73764344108603	0\\
1.73774344358609	0\\
1.73784344608615	0\\
1.73794344858621	0\\
1.73804345108628	0\\
1.73814345358634	0\\
1.7382434560864	0\\
1.73834345858646	0\\
1.73844346108653	0\\
1.73854346358659	0\\
1.73864346608665	0\\
1.73874346858671	0\\
1.73884347108678	0\\
1.73894347358684	0\\
1.7390434760869	0\\
1.73914347858696	0\\
1.73924348108703	0\\
1.73934348358709	0\\
1.73944348608715	0\\
1.73954348858721	0\\
1.73964349108728	0\\
1.73974349358734	0\\
1.7398434960874	0\\
1.73994349858746	0\\
1.74004350108753	0\\
1.74014350358759	0\\
1.74024350608765	0\\
1.74034350858771	0\\
1.74044351108778	0\\
1.74054351358784	0\\
1.7406435160879	0\\
1.74074351858796	0\\
1.74084352108803	0\\
1.74094352358809	0\\
1.74104352608815	0\\
1.74114352858821	0\\
1.74124353108828	0\\
1.74134353358834	0\\
1.7414435360884	0\\
1.74154353858846	0\\
1.74164354108853	0\\
1.74174354358859	0\\
1.74184354608865	0\\
1.74194354858871	0\\
1.74204355108878	0\\
1.74214355358884	0\\
1.7422435560889	0\\
1.74234355858896	0\\
1.74244356108903	0\\
1.74254356358909	0\\
1.74264356608915	0\\
1.74274356858921	0\\
1.74284357108928	0\\
1.74294357358934	0\\
1.7430435760894	0\\
1.74314357858946	0\\
1.74324358108953	0\\
1.74334358358959	0\\
1.74344358608965	0\\
1.74354358858971	0\\
1.74364359108978	0\\
1.74374359358984	0\\
1.7438435960899	0\\
1.74394359858996	0\\
1.74404360109003	0\\
1.74414360359009	0\\
1.74424360609015	0\\
1.74434360859021	0\\
1.74444361109028	0\\
1.74454361359034	0\\
1.7446436160904	0\\
1.74474361859046	0\\
1.74484362109053	0\\
1.74494362359059	0\\
1.74504362609065	0\\
1.74514362859071	0\\
1.74524363109078	0\\
1.74534363359084	0\\
1.7454436360909	0\\
1.74554363859096	0\\
1.74564364109103	0\\
1.74574364359109	0\\
1.74584364609115	0\\
1.74594364859121	0\\
1.74604365109128	0\\
1.74614365359134	0\\
1.7462436560914	0\\
1.74634365859146	0\\
1.74644366109153	0\\
1.74654366359159	0\\
1.74664366609165	0\\
1.74674366859171	0\\
1.74684367109178	0\\
1.74694367359184	0\\
1.7470436760919	0\\
1.74714367859196	0\\
1.74724368109203	0\\
1.74734368359209	0\\
1.74744368609215	0\\
1.74754368859221	0\\
1.74764369109228	0\\
1.74774369359234	0\\
1.7478436960924	0\\
1.74794369859246	0\\
1.74804370109253	0\\
1.74814370359259	0\\
1.74824370609265	0\\
1.74834370859271	0\\
1.74844371109278	0\\
1.74854371359284	0\\
1.7486437160929	0\\
1.74874371859296	0\\
1.74884372109303	0\\
1.74894372359309	0\\
1.74904372609315	0\\
1.74914372859321	0\\
1.74924373109328	0\\
1.74934373359334	0\\
1.7494437360934	0\\
1.74954373859346	0\\
1.74964374109353	0\\
1.74974374359359	0\\
1.74984374609365	0\\
1.74994374859371	0\\
1.75004375109378	0\\
1.75014375359384	0\\
1.7502437560939	0\\
1.75034375859396	0\\
1.75044376109403	0\\
1.75054376359409	0\\
1.75064376609415	0\\
1.75074376859421	0\\
1.75084377109428	0\\
1.75094377359434	0\\
1.7510437760944	0\\
1.75114377859446	0\\
1.75124378109453	0\\
1.75134378359459	0\\
1.75144378609465	0\\
1.75154378859471	0\\
1.75164379109478	0\\
1.75174379359484	0\\
1.7518437960949	0\\
1.75194379859496	0\\
1.75204380109503	0\\
1.75214380359509	0\\
1.75224380609515	0\\
1.75234380859521	0\\
1.75244381109528	0\\
1.75254381359534	0\\
1.7526438160954	0\\
1.75274381859546	0\\
1.75284382109553	0\\
1.75294382359559	0\\
1.75304382609565	0\\
1.75314382859571	0\\
1.75324383109578	0\\
1.75334383359584	0\\
1.7534438360959	0\\
1.75354383859596	0\\
1.75364384109603	0\\
1.75374384359609	0\\
1.75384384609615	0\\
1.75394384859621	0\\
1.75404385109628	0\\
1.75414385359634	0\\
1.7542438560964	0\\
1.75434385859647	0\\
1.75444386109653	0\\
1.75454386359659	0\\
1.75464386609665	0\\
1.75474386859671	0\\
1.75484387109678	0\\
1.75494387359684	0\\
1.7550438760969	0\\
1.75514387859696	0\\
1.75524388109703	0\\
1.75534388359709	0\\
1.75544388609715	0\\
1.75554388859721	0\\
1.75564389109728	0\\
1.75574389359734	0\\
1.7558438960974	0\\
1.75594389859746	0\\
1.75604390109753	0\\
1.75614390359759	0\\
1.75624390609765	0\\
1.75634390859772	0\\
1.75644391109778	0\\
1.75654391359784	0\\
1.7566439160979	0\\
1.75674391859796	0\\
1.75684392109803	0\\
1.75694392359809	0\\
1.75704392609815	0\\
1.75714392859821	0\\
1.75724393109828	0\\
1.75734393359834	0\\
1.7574439360984	0\\
1.75754393859847	0\\
1.75764394109853	0\\
1.75774394359859	0\\
1.75784394609865	0\\
1.75794394859871	0\\
1.75804395109878	0\\
1.75814395359884	0\\
1.7582439560989	0\\
1.75834395859897	0\\
1.75844396109903	0\\
1.75854396359909	0\\
1.75864396609915	0\\
1.75874396859922	0\\
1.75884397109928	0\\
1.75894397359934	0\\
1.7590439760994	0\\
1.75914397859946	0\\
1.75924398109953	0\\
1.75934398359959	0\\
1.75944398609965	0\\
1.75954398859972	0\\
1.75964399109978	0\\
1.75974399359984	0\\
1.7598439960999	0\\
1.75994399859996	0\\
1.76004400110003	0\\
1.76014400360009	0\\
1.76024400610015	0\\
1.76034400860022	0\\
1.76044401110028	0\\
1.76054401360034	0\\
1.7606440161004	0\\
1.76074401860047	0\\
1.76084402110053	0\\
1.76094402360059	0\\
1.76104402610065	0\\
1.76114402860071	0\\
1.76124403110078	0\\
1.76134403360084	0\\
1.7614440361009	0\\
1.76154403860097	0\\
1.76164404110103	0\\
1.76174404360109	0\\
1.76184404610115	0\\
1.76194404860122	0\\
1.76204405110128	0\\
1.76214405360134	0\\
1.7622440561014	0\\
1.76234405860147	0\\
1.76244406110153	0\\
1.76254406360159	0\\
1.76264406610165	0\\
1.76274406860172	0\\
1.76284407110178	0\\
1.76294407360184	0\\
1.7630440761019	0\\
1.76314407860196	0\\
1.76324408110203	0\\
1.76334408360209	0\\
1.76344408610215	0\\
1.76354408860222	0\\
1.76364409110228	0\\
1.76374409360234	0\\
1.7638440961024	0\\
1.76394409860247	0\\
1.76404410110253	0\\
1.76414410360259	0\\
1.76424410610265	0\\
1.76434410860272	0\\
1.76444411110278	0\\
1.76454411360284	0\\
1.7646441161029	0\\
1.76474411860297	0\\
1.76484412110303	0\\
1.76494412360309	0\\
1.76504412610315	0\\
1.76514412860322	0\\
1.76524413110328	0\\
1.76534413360334	0\\
1.7654441361034	0\\
1.76554413860347	0\\
1.76564414110353	0\\
1.76574414360359	0\\
1.76584414610365	0\\
1.76594414860372	0\\
1.76604415110378	0\\
1.76614415360384	0\\
1.7662441561039	0\\
1.76634415860397	0\\
1.76644416110403	0\\
1.76654416360409	0\\
1.76664416610415	0\\
1.76674416860422	0\\
1.76684417110428	0\\
1.76694417360434	0\\
1.7670441761044	0\\
1.76714417860447	0\\
1.76724418110453	0\\
1.76734418360459	0\\
1.76744418610465	0\\
1.76754418860472	0\\
1.76764419110478	0\\
1.76774419360484	0\\
1.7678441961049	0\\
1.76794419860497	0\\
1.76804420110503	0\\
1.76814420360509	0\\
1.76824420610515	0\\
1.76834420860522	0\\
1.76844421110528	0\\
1.76854421360534	0\\
1.7686442161054	0\\
1.76874421860547	0\\
1.76884422110553	0\\
1.76894422360559	0\\
1.76904422610565	0\\
1.76914422860572	0\\
1.76924423110578	0\\
1.76934423360584	0\\
1.7694442361059	0\\
1.76954423860597	0\\
1.76964424110603	0\\
1.76974424360609	0\\
1.76984424610615	0\\
1.76994424860622	0\\
1.77004425110628	0\\
1.77014425360634	0\\
1.7702442561064	0\\
1.77034425860647	0\\
1.77044426110653	0\\
1.77054426360659	0\\
1.77064426610665	0\\
1.77074426860672	0\\
1.77084427110678	0\\
1.77094427360684	0\\
1.7710442761069	0\\
1.77114427860697	0\\
1.77124428110703	0\\
1.77134428360709	0\\
1.77144428610715	0\\
1.77154428860722	0\\
1.77164429110728	0\\
1.77174429360734	0\\
1.7718442961074	0\\
1.77194429860747	0\\
1.77204430110753	0\\
1.77214430360759	0\\
1.77224430610765	0\\
1.77234430860772	0\\
1.77244431110778	0\\
1.77254431360784	0\\
1.7726443161079	0\\
1.77274431860797	0\\
1.77284432110803	0\\
1.77294432360809	0\\
1.77304432610815	0\\
1.77314432860822	0\\
1.77324433110828	0\\
1.77334433360834	0\\
1.7734443361084	0\\
1.77354433860847	0\\
1.77364434110853	0\\
1.77374434360859	0\\
1.77384434610865	0\\
1.77394434860872	0\\
1.77404435110878	0\\
1.77414435360884	0\\
1.7742443561089	0\\
1.77434435860897	0\\
1.77444436110903	0\\
1.77454436360909	0\\
1.77464436610915	0\\
1.77474436860922	0\\
1.77484437110928	0\\
1.77494437360934	0\\
1.7750443761094	0\\
1.77514437860947	0\\
1.77524438110953	0\\
1.77534438360959	0\\
1.77544438610965	0\\
1.77554438860972	0\\
1.77564439110978	0\\
1.77574439360984	0\\
1.7758443961099	0\\
1.77594439860997	0\\
1.77604440111003	0\\
1.77614440361009	0\\
1.77624440611015	0\\
1.77634440861022	0\\
1.77644441111028	0\\
1.77654441361034	0\\
1.7766444161104	0\\
1.77674441861047	0\\
1.77684442111053	0\\
1.77694442361059	0\\
1.77704442611065	0\\
1.77714442861072	0\\
1.77724443111078	0\\
1.77734443361084	0\\
1.7774444361109	0\\
1.77754443861097	0\\
1.77764444111103	0\\
1.77774444361109	0\\
1.77784444611115	0\\
1.77794444861122	0\\
1.77804445111128	0\\
1.77814445361134	0\\
1.7782444561114	0\\
1.77834445861147	0\\
1.77844446111153	0\\
1.77854446361159	0\\
1.77864446611165	0\\
1.77874446861172	0\\
1.77884447111178	0\\
1.77894447361184	0\\
1.7790444761119	0\\
1.77914447861197	0\\
1.77924448111203	0\\
1.77934448361209	0\\
1.77944448611215	0\\
1.77954448861222	0\\
1.77964449111228	0\\
1.77974449361234	0\\
1.7798444961124	0\\
1.77994449861247	0\\
1.78004450111253	0\\
1.78014450361259	0\\
1.78024450611265	0\\
1.78034450861272	0\\
1.78044451111278	0\\
1.78054451361284	0\\
1.7806445161129	0\\
1.78074451861297	0\\
1.78084452111303	0\\
1.78094452361309	0\\
1.78104452611315	0\\
1.78114452861322	0\\
1.78124453111328	0\\
1.78134453361334	0\\
1.7814445361134	0\\
1.78154453861347	0\\
1.78164454111353	0\\
1.78174454361359	0\\
1.78184454611365	0\\
1.78194454861372	0\\
1.78204455111378	0\\
1.78214455361384	0\\
1.7822445561139	0\\
1.78234455861397	0\\
1.78244456111403	0\\
1.78254456361409	0\\
1.78264456611415	0\\
1.78274456861422	0\\
1.78284457111428	0\\
1.78294457361434	0\\
1.7830445761144	0\\
1.78314457861447	0\\
1.78324458111453	0\\
1.78334458361459	0\\
1.78344458611465	0\\
1.78354458861472	0\\
1.78364459111478	0\\
1.78374459361484	0\\
1.7838445961149	0\\
1.78394459861497	0\\
1.78404460111503	0\\
1.78414460361509	0\\
1.78424460611515	0\\
1.78434460861522	0\\
1.78444461111528	0\\
1.78454461361534	0\\
1.7846446161154	0\\
1.78474461861547	0\\
1.78484462111553	0\\
1.78494462361559	0\\
1.78504462611565	0\\
1.78514462861572	0\\
1.78524463111578	0\\
1.78534463361584	0\\
1.7854446361159	0\\
1.78554463861597	0\\
1.78564464111603	0\\
1.78574464361609	0\\
1.78584464611615	0\\
1.78594464861622	0\\
1.78604465111628	0\\
1.78614465361634	0\\
1.7862446561164	0\\
1.78634465861647	0\\
1.78644466111653	0\\
1.78654466361659	0\\
1.78664466611665	0\\
1.78674466861672	0\\
1.78684467111678	0\\
1.78694467361684	0\\
1.7870446761169	0\\
1.78714467861697	0\\
1.78724468111703	0\\
1.78734468361709	0\\
1.78744468611715	0\\
1.78754468861722	0\\
1.78764469111728	0\\
1.78774469361734	0\\
1.7878446961174	0\\
1.78794469861747	0\\
1.78804470111753	0\\
1.78814470361759	0\\
1.78824470611765	0\\
1.78834470861772	0\\
1.78844471111778	0\\
1.78854471361784	0\\
1.7886447161179	0\\
1.78874471861797	0\\
1.78884472111803	0\\
1.78894472361809	0\\
1.78904472611815	0\\
1.78914472861822	0\\
1.78924473111828	0\\
1.78934473361834	0\\
1.7894447361184	0\\
1.78954473861847	0\\
1.78964474111853	0\\
1.78974474361859	0\\
1.78984474611865	0\\
1.78994474861872	0\\
1.79004475111878	0\\
1.79014475361884	0\\
1.7902447561189	0\\
1.79034475861897	0\\
1.79044476111903	0\\
1.79054476361909	0\\
1.79064476611915	0\\
1.79074476861922	0\\
1.79084477111928	0\\
1.79094477361934	0\\
1.7910447761194	0\\
1.79114477861947	0\\
1.79124478111953	0\\
1.79134478361959	0\\
1.79144478611965	0\\
1.79154478861972	0\\
1.79164479111978	0\\
1.79174479361984	0\\
1.7918447961199	0\\
1.79194479861997	0\\
1.79204480112003	0\\
1.79214480362009	0\\
1.79224480612015	0\\
1.79234480862022	0\\
1.79244481112028	0\\
1.79254481362034	0\\
1.7926448161204	0\\
1.79274481862047	0\\
1.79284482112053	0\\
1.79294482362059	0\\
1.79304482612065	0\\
1.79314482862072	0\\
1.79324483112078	0\\
1.79334483362084	0\\
1.7934448361209	0\\
1.79354483862097	0\\
1.79364484112103	0\\
1.79374484362109	0\\
1.79384484612115	0\\
1.79394484862122	0\\
1.79404485112128	0\\
1.79414485362134	0\\
1.7942448561214	0\\
1.79434485862147	0\\
1.79444486112153	0\\
1.79454486362159	0\\
1.79464486612165	0\\
1.79474486862172	0\\
1.79484487112178	0\\
1.79494487362184	0\\
1.7950448761219	0\\
1.79514487862197	0\\
1.79524488112203	0\\
1.79534488362209	0\\
1.79544488612215	0\\
1.79554488862222	0\\
1.79564489112228	0\\
1.79574489362234	0\\
1.7958448961224	0\\
1.79594489862247	0\\
1.79604490112253	0\\
1.79614490362259	0\\
1.79624490612265	0\\
1.79634490862272	0\\
1.79644491112278	0\\
1.79654491362284	0\\
1.7966449161229	0\\
1.79674491862297	0\\
1.79684492112303	0\\
1.79694492362309	0\\
1.79704492612315	0\\
1.79714492862322	0\\
1.79724493112328	0\\
1.79734493362334	0\\
1.7974449361234	0\\
1.79754493862347	0\\
1.79764494112353	0\\
1.79774494362359	0\\
1.79784494612365	0\\
1.79794494862372	0\\
1.79804495112378	0\\
1.79814495362384	0\\
1.7982449561239	0\\
1.79834495862397	0\\
1.79844496112403	0\\
1.79854496362409	0\\
1.79864496612415	0\\
1.79874496862422	0\\
1.79884497112428	0\\
1.79894497362434	0\\
1.7990449761244	0\\
1.79914497862447	0\\
1.79924498112453	0\\
1.79934498362459	0\\
1.79944498612465	0\\
1.79954498862472	0\\
1.79964499112478	0\\
1.79974499362484	0\\
1.7998449961249	0\\
1.79994499862497	0\\
1.80004500112503	0\\
1.80014500362509	0\\
1.80024500612515	0\\
1.80034500862522	0\\
1.80044501112528	0\\
1.80054501362534	0\\
1.8006450161254	0\\
1.80074501862547	0\\
1.80084502112553	0\\
1.80094502362559	0\\
1.80104502612565	0\\
1.80114502862572	0\\
1.80124503112578	0\\
1.80134503362584	0\\
1.8014450361259	0\\
1.80154503862597	0\\
1.80164504112603	0\\
1.80174504362609	0\\
1.80184504612615	0\\
1.80194504862622	0\\
1.80204505112628	0\\
1.80214505362634	0\\
1.8022450561264	0\\
1.80234505862647	0\\
1.80244506112653	0\\
1.80254506362659	0\\
1.80264506612665	0\\
1.80274506862672	0\\
1.80284507112678	0\\
1.80294507362684	0\\
1.8030450761269	0\\
1.80314507862697	0\\
1.80324508112703	0\\
1.80334508362709	0\\
1.80344508612715	0\\
1.80354508862722	0\\
1.80364509112728	0\\
1.80374509362734	0\\
1.8038450961274	0\\
1.80394509862747	0\\
1.80404510112753	0\\
1.80414510362759	0\\
1.80424510612765	0\\
1.80434510862772	0\\
1.80444511112778	0\\
1.80454511362784	0\\
1.8046451161279	0\\
1.80474511862797	0\\
1.80484512112803	0\\
1.80494512362809	0\\
1.80504512612815	0\\
1.80514512862822	0\\
1.80524513112828	0\\
1.80534513362834	0\\
1.8054451361284	0\\
1.80554513862847	0\\
1.80564514112853	0\\
1.80574514362859	0\\
1.80584514612865	0\\
1.80594514862872	0\\
1.80604515112878	0\\
1.80614515362884	0\\
1.8062451561289	0\\
1.80634515862897	0\\
1.80644516112903	0\\
1.80654516362909	0\\
1.80664516612915	0\\
1.80674516862922	0\\
1.80684517112928	0\\
1.80694517362934	0\\
1.8070451761294	0\\
1.80714517862947	0\\
1.80724518112953	0\\
1.80734518362959	0\\
1.80744518612965	0\\
1.80754518862972	0\\
1.80764519112978	0\\
1.80774519362984	0\\
1.8078451961299	0\\
1.80794519862997	0\\
1.80804520113003	0\\
1.80814520363009	0\\
1.80824520613015	0\\
1.80834520863022	0\\
1.80844521113028	0\\
1.80854521363034	0\\
1.8086452161304	0\\
1.80874521863047	0\\
1.80884522113053	0\\
1.80894522363059	0\\
1.80904522613065	0\\
1.80914522863072	0\\
1.80924523113078	0\\
1.80934523363084	0\\
1.8094452361309	0\\
1.80954523863097	0\\
1.80964524113103	0\\
1.80974524363109	0\\
1.80984524613115	0\\
1.80994524863122	0\\
1.81004525113128	0\\
1.81014525363134	0\\
1.8102452561314	0\\
1.81034525863147	0\\
1.81044526113153	0\\
1.81054526363159	0\\
1.81064526613165	0\\
1.81074526863172	0\\
1.81084527113178	0\\
1.81094527363184	0\\
1.8110452761319	0\\
1.81114527863197	0\\
1.81124528113203	0\\
1.81134528363209	0\\
1.81144528613215	0\\
1.81154528863222	0\\
1.81164529113228	0\\
1.81174529363234	0\\
1.8118452961324	0\\
1.81194529863247	0\\
1.81204530113253	0\\
1.81214530363259	0\\
1.81224530613265	0\\
1.81234530863272	0\\
1.81244531113278	0\\
1.81254531363284	0\\
1.8126453161329	0\\
1.81274531863297	0\\
1.81284532113303	0\\
1.81294532363309	0\\
1.81304532613315	0\\
1.81314532863322	0\\
1.81324533113328	0\\
1.81334533363334	0\\
1.8134453361334	0\\
1.81354533863347	0\\
1.81364534113353	0\\
1.81374534363359	0\\
1.81384534613365	0\\
1.81394534863372	0\\
1.81404535113378	0\\
1.81414535363384	0\\
1.8142453561339	0\\
1.81434535863397	0\\
1.81444536113403	0\\
1.81454536363409	0\\
1.81464536613415	0\\
1.81474536863422	0\\
1.81484537113428	0\\
1.81494537363434	0\\
1.8150453761344	0\\
1.81514537863447	0\\
1.81524538113453	0\\
1.81534538363459	0\\
1.81544538613465	0\\
1.81554538863472	0\\
1.81564539113478	0\\
1.81574539363484	0\\
1.8158453961349	0\\
1.81594539863497	0\\
1.81604540113503	0\\
1.81614540363509	0\\
1.81624540613515	0\\
1.81634540863522	0\\
1.81644541113528	0\\
1.81654541363534	0\\
1.8166454161354	0\\
1.81674541863547	0\\
1.81684542113553	0\\
1.81694542363559	0\\
1.81704542613565	0\\
1.81714542863572	0\\
1.81724543113578	0\\
1.81734543363584	0\\
1.8174454361359	0\\
1.81754543863597	0\\
1.81764544113603	0\\
1.81774544363609	0\\
1.81784544613615	0\\
1.81794544863622	0\\
1.81804545113628	0\\
1.81814545363634	0\\
1.8182454561364	0\\
1.81834545863647	0\\
1.81844546113653	0\\
1.81854546363659	0\\
1.81864546613665	0\\
1.81874546863672	0\\
1.81884547113678	0\\
1.81894547363684	0\\
1.8190454761369	0\\
1.81914547863697	0\\
1.81924548113703	0\\
1.81934548363709	0\\
1.81944548613715	0\\
1.81954548863722	0\\
1.81964549113728	0\\
1.81974549363734	0\\
1.8198454961374	0\\
1.81994549863747	0\\
1.82004550113753	0\\
1.82014550363759	0\\
1.82024550613765	0\\
1.82034550863772	0\\
1.82044551113778	0\\
1.82054551363784	0\\
1.8206455161379	0\\
1.82074551863797	0\\
1.82084552113803	0\\
1.82094552363809	0\\
1.82104552613815	0\\
1.82114552863822	0\\
1.82124553113828	0\\
1.82134553363834	0\\
1.8214455361384	0\\
1.82154553863847	0\\
1.82164554113853	0\\
1.82174554363859	0\\
1.82184554613865	0\\
1.82194554863872	0\\
1.82204555113878	0\\
1.82214555363884	0\\
1.8222455561389	0\\
1.82234555863897	0\\
1.82244556113903	0\\
1.82254556363909	0\\
1.82264556613915	0\\
1.82274556863922	0\\
1.82284557113928	0\\
1.82294557363934	0\\
1.8230455761394	0\\
1.82314557863947	0\\
1.82324558113953	0\\
1.82334558363959	0\\
1.82344558613965	0\\
1.82354558863972	0\\
1.82364559113978	0\\
1.82374559363984	0\\
1.8238455961399	0\\
1.82394559863997	0\\
1.82404560114003	0\\
1.82414560364009	0\\
1.82424560614015	0\\
1.82434560864022	0\\
1.82444561114028	0\\
1.82454561364034	0\\
1.8246456161404	0\\
1.82474561864047	0\\
1.82484562114053	0\\
1.82494562364059	0\\
1.82504562614065	0\\
1.82514562864072	0\\
1.82524563114078	0\\
1.82534563364084	0\\
1.8254456361409	0\\
1.82554563864097	0\\
1.82564564114103	0\\
1.82574564364109	0\\
1.82584564614115	0\\
1.82594564864122	0\\
1.82604565114128	0\\
1.82614565364134	0\\
1.8262456561414	0\\
1.82634565864147	0\\
1.82644566114153	0\\
1.82654566364159	0\\
1.82664566614165	0\\
1.82674566864172	0\\
1.82684567114178	0\\
1.82694567364184	0\\
1.8270456761419	0\\
1.82714567864197	0\\
1.82724568114203	0\\
1.82734568364209	0\\
1.82744568614215	0\\
1.82754568864222	0\\
1.82764569114228	0\\
1.82774569364234	0\\
1.8278456961424	0\\
1.82794569864247	0\\
1.82804570114253	0\\
1.82814570364259	0\\
1.82824570614265	0\\
1.82834570864272	0\\
1.82844571114278	0\\
1.82854571364284	0\\
1.8286457161429	0\\
1.82874571864297	0\\
1.82884572114303	0\\
1.82894572364309	0\\
1.82904572614315	0\\
1.82914572864322	0\\
1.82924573114328	0\\
1.82934573364334	0\\
1.8294457361434	0\\
1.82954573864347	0\\
1.82964574114353	0\\
1.82974574364359	0\\
1.82984574614365	0\\
1.82994574864372	0\\
1.83004575114378	0\\
1.83014575364384	0\\
1.8302457561439	0\\
1.83034575864397	0\\
1.83044576114403	0\\
1.83054576364409	0\\
1.83064576614415	0\\
1.83074576864422	0\\
1.83084577114428	0\\
1.83094577364434	0\\
1.8310457761444	0\\
1.83114577864447	0\\
1.83124578114453	0\\
1.83134578364459	0\\
1.83144578614465	0\\
1.83154578864472	0\\
1.83164579114478	0\\
1.83174579364484	0\\
1.8318457961449	0\\
1.83194579864497	0\\
1.83204580114503	0\\
1.83214580364509	0\\
1.83224580614515	0\\
1.83234580864522	0\\
1.83244581114528	0\\
1.83254581364534	0\\
1.8326458161454	0\\
1.83274581864547	0\\
1.83284582114553	0\\
1.83294582364559	0\\
1.83304582614565	0\\
1.83314582864572	0\\
1.83324583114578	0\\
1.83334583364584	0\\
1.8334458361459	0\\
1.83354583864597	0\\
1.83364584114603	0\\
1.83374584364609	0\\
1.83384584614615	0\\
1.83394584864622	0\\
1.83404585114628	0\\
1.83414585364634	0\\
1.8342458561464	0\\
1.83434585864647	0\\
1.83444586114653	0\\
1.83454586364659	0\\
1.83464586614665	0\\
1.83474586864672	0\\
1.83484587114678	0\\
1.83494587364684	0\\
1.8350458761469	0\\
1.83514587864697	0\\
1.83524588114703	0\\
1.83534588364709	0\\
1.83544588614715	0\\
1.83554588864722	0\\
1.83564589114728	0\\
1.83574589364734	0\\
1.8358458961474	0\\
1.83594589864747	0\\
1.83604590114753	0\\
1.83614590364759	0\\
1.83624590614765	0\\
1.83634590864772	0\\
1.83644591114778	0\\
1.83654591364784	0\\
1.8366459161479	0\\
1.83674591864797	0\\
1.83684592114803	0\\
1.83694592364809	0\\
1.83704592614815	0\\
1.83714592864822	0\\
1.83724593114828	0\\
1.83734593364834	0\\
1.8374459361484	0\\
1.83754593864847	0\\
1.83764594114853	0\\
1.83774594364859	0\\
1.83784594614865	0\\
1.83794594864872	0\\
1.83804595114878	0\\
1.83814595364884	0\\
1.8382459561489	0\\
1.83834595864897	0\\
1.83844596114903	0\\
1.83854596364909	0\\
1.83864596614915	0\\
1.83874596864922	0\\
1.83884597114928	0\\
1.83894597364934	0\\
1.8390459761494	0\\
1.83914597864947	0\\
1.83924598114953	0\\
1.83934598364959	0\\
1.83944598614965	0\\
1.83954598864972	0\\
1.83964599114978	0\\
1.83974599364984	0\\
1.8398459961499	0\\
1.83994599864997	0\\
1.84004600115003	0\\
1.84014600365009	0\\
1.84024600615015	0\\
1.84034600865022	0\\
1.84044601115028	0\\
1.84054601365034	0\\
1.8406460161504	0\\
1.84074601865047	0\\
1.84084602115053	0\\
1.84094602365059	0\\
1.84104602615065	0\\
1.84114602865072	0\\
1.84124603115078	0\\
1.84134603365084	0\\
1.8414460361509	0\\
1.84154603865097	0\\
1.84164604115103	0\\
1.84174604365109	0\\
1.84184604615115	0\\
1.84194604865122	0\\
1.84204605115128	0\\
1.84214605365134	0\\
1.8422460561514	0\\
1.84234605865147	0\\
1.84244606115153	0\\
1.84254606365159	0\\
1.84264606615165	0\\
1.84274606865172	0\\
1.84284607115178	0\\
1.84294607365184	0\\
1.8430460761519	0\\
1.84314607865197	0\\
1.84324608115203	0\\
1.84334608365209	0\\
1.84344608615215	0\\
1.84354608865222	0\\
1.84364609115228	0\\
1.84374609365234	0\\
1.8438460961524	0\\
1.84394609865247	0\\
1.84404610115253	0\\
1.84414610365259	0\\
1.84424610615265	0\\
1.84434610865272	0\\
1.84444611115278	0\\
1.84454611365284	0\\
1.8446461161529	0\\
1.84474611865297	0\\
1.84484612115303	0\\
1.84494612365309	0\\
1.84504612615315	0\\
1.84514612865322	0\\
1.84524613115328	0\\
1.84534613365334	0\\
1.8454461361534	0\\
1.84554613865347	0\\
1.84564614115353	0\\
1.84574614365359	0\\
1.84584614615365	0\\
1.84594614865372	0\\
1.84604615115378	0\\
1.84614615365384	0\\
1.8462461561539	0\\
1.84634615865397	0\\
1.84644616115403	0\\
1.84654616365409	0\\
1.84664616615415	0\\
1.84674616865422	0\\
1.84684617115428	0\\
1.84694617365434	0\\
1.8470461761544	0\\
1.84714617865447	0\\
1.84724618115453	0\\
1.84734618365459	0\\
1.84744618615465	0\\
1.84754618865472	0\\
1.84764619115478	0\\
1.84774619365484	0\\
1.8478461961549	0\\
1.84794619865497	0\\
1.84804620115503	0\\
1.84814620365509	0\\
1.84824620615515	0\\
1.84834620865522	0\\
1.84844621115528	0\\
1.84854621365534	0\\
1.8486462161554	0\\
1.84874621865547	0\\
1.84884622115553	0\\
1.84894622365559	0\\
1.84904622615565	0\\
1.84914622865572	0\\
1.84924623115578	0\\
1.84934623365584	0\\
1.8494462361559	0\\
1.84954623865597	0\\
1.84964624115603	0\\
1.84974624365609	0\\
1.84984624615615	0\\
1.84994624865622	0\\
1.85004625115628	0\\
1.85014625365634	0\\
1.8502462561564	0\\
1.85034625865647	0\\
1.85044626115653	0\\
1.85054626365659	0\\
1.85064626615665	0\\
1.85074626865672	0\\
1.85084627115678	0\\
1.85094627365684	0\\
1.8510462761569	0\\
1.85114627865697	0\\
1.85124628115703	0\\
1.85134628365709	0\\
1.85144628615715	0\\
1.85154628865722	0\\
1.85164629115728	0\\
1.85174629365734	0\\
1.8518462961574	0\\
1.85194629865747	0\\
1.85204630115753	0\\
1.85214630365759	0\\
1.85224630615765	0\\
1.85234630865772	0\\
1.85244631115778	0\\
1.85254631365784	0\\
1.8526463161579	0\\
1.85274631865797	0\\
1.85284632115803	0\\
1.85294632365809	0\\
1.85304632615815	0\\
1.85314632865822	0\\
1.85324633115828	0\\
1.85334633365834	0\\
1.8534463361584	0\\
1.85354633865847	0\\
1.85364634115853	0\\
1.85374634365859	0\\
1.85384634615865	0\\
1.85394634865872	0\\
1.85404635115878	0\\
1.85414635365884	0\\
1.8542463561589	0\\
1.85434635865897	0\\
1.85444636115903	0\\
1.85454636365909	0\\
1.85464636615915	0\\
1.85474636865922	0\\
1.85484637115928	0\\
1.85494637365934	0\\
1.8550463761594	0\\
1.85514637865947	0\\
1.85524638115953	0\\
1.85534638365959	0\\
1.85544638615965	0\\
1.85554638865972	0\\
1.85564639115978	0\\
1.85574639365984	0\\
1.8558463961599	0\\
1.85594639865997	0\\
1.85604640116003	0\\
1.85614640366009	0\\
1.85624640616015	0\\
1.85634640866022	0\\
1.85644641116028	0\\
1.85654641366034	0\\
1.8566464161604	0\\
1.85674641866047	0\\
1.85684642116053	0\\
1.85694642366059	0\\
1.85704642616065	0\\
1.85714642866072	0\\
1.85724643116078	0\\
1.85734643366084	0\\
1.8574464361609	0\\
1.85754643866097	0\\
1.85764644116103	0\\
1.85774644366109	0\\
1.85784644616115	0\\
1.85794644866122	0\\
1.85804645116128	0\\
1.85814645366134	0\\
1.8582464561614	0\\
1.85834645866147	0\\
1.85844646116153	0\\
1.85854646366159	0\\
1.85864646616165	0\\
1.85874646866172	0\\
1.85884647116178	0\\
1.85894647366184	0\\
1.8590464761619	0\\
1.85914647866197	0\\
1.85924648116203	0\\
1.85934648366209	0\\
1.85944648616215	0\\
1.85954648866222	0\\
1.85964649116228	0\\
1.85974649366234	0\\
1.8598464961624	0\\
1.85994649866247	0\\
1.86004650116253	0\\
1.86014650366259	0\\
1.86024650616265	0\\
1.86034650866272	0\\
1.86044651116278	0\\
1.86054651366284	0\\
1.8606465161629	0\\
1.86074651866297	0\\
1.86084652116303	0\\
1.86094652366309	0\\
1.86104652616315	0\\
1.86114652866322	0\\
1.86124653116328	0\\
1.86134653366334	0\\
1.8614465361634	0\\
1.86154653866347	0\\
1.86164654116353	0\\
1.86174654366359	0\\
1.86184654616365	0\\
1.86194654866372	0\\
1.86204655116378	0\\
1.86214655366384	0\\
1.8622465561639	0\\
1.86234655866397	0\\
1.86244656116403	0\\
1.86254656366409	0\\
1.86264656616415	0\\
1.86274656866422	0\\
1.86284657116428	0\\
1.86294657366434	0\\
1.8630465761644	0\\
1.86314657866447	0\\
1.86324658116453	0\\
1.86334658366459	0\\
1.86344658616465	0\\
1.86354658866472	0\\
1.86364659116478	0\\
1.86374659366484	0\\
1.8638465961649	0\\
1.86394659866497	0\\
1.86404660116503	0\\
1.86414660366509	0\\
1.86424660616515	0\\
1.86434660866522	0\\
1.86444661116528	0\\
1.86454661366534	0\\
1.8646466161654	0\\
1.86474661866547	0\\
1.86484662116553	0\\
1.86494662366559	0\\
1.86504662616565	0\\
1.86514662866572	0\\
1.86524663116578	0\\
1.86534663366584	0\\
1.8654466361659	0\\
1.86554663866597	0\\
1.86564664116603	0\\
1.86574664366609	0\\
1.86584664616615	0\\
1.86594664866622	0\\
1.86604665116628	0\\
1.86614665366634	0\\
1.8662466561664	0\\
1.86634665866647	0\\
1.86644666116653	0\\
1.86654666366659	0\\
1.86664666616665	0\\
1.86674666866672	0\\
1.86684667116678	0\\
1.86694667366684	0\\
1.8670466761669	0\\
1.86714667866697	0\\
1.86724668116703	0\\
1.86734668366709	0\\
1.86744668616715	0\\
1.86754668866722	0\\
1.86764669116728	0\\
1.86774669366734	0\\
1.8678466961674	0\\
1.86794669866747	0\\
1.86804670116753	0\\
1.86814670366759	0\\
1.86824670616765	0\\
1.86834670866772	0\\
1.86844671116778	0\\
1.86854671366784	0\\
1.8686467161679	0\\
1.86874671866797	0\\
1.86884672116803	0\\
1.86894672366809	0\\
1.86904672616815	0\\
1.86914672866822	0\\
1.86924673116828	0\\
1.86934673366834	0\\
1.8694467361684	0\\
1.86954673866847	0\\
1.86964674116853	0\\
1.86974674366859	0\\
1.86984674616865	0\\
1.86994674866872	0\\
1.87004675116878	0\\
1.87014675366884	0\\
1.8702467561689	0\\
1.87034675866897	0\\
1.87044676116903	0\\
1.87054676366909	0\\
1.87064676616915	0\\
1.87074676866922	0\\
1.87084677116928	0\\
1.87094677366934	0\\
1.8710467761694	0\\
1.87114677866947	0\\
1.87124678116953	0\\
1.87134678366959	0\\
1.87144678616965	0\\
1.87154678866972	0\\
1.87164679116978	0\\
1.87174679366984	0\\
1.8718467961699	0\\
1.87194679866997	0\\
1.87204680117003	0\\
1.87214680367009	0\\
1.87224680617015	0\\
1.87234680867022	0\\
1.87244681117028	0\\
1.87254681367034	0\\
1.8726468161704	0\\
1.87274681867047	0\\
1.87284682117053	0\\
1.87294682367059	0\\
1.87304682617065	0\\
1.87314682867072	0\\
1.87324683117078	0\\
1.87334683367084	0\\
1.8734468361709	0\\
1.87354683867097	0\\
1.87364684117103	0\\
1.87374684367109	0\\
1.87384684617115	0\\
1.87394684867122	0\\
1.87404685117128	0\\
1.87414685367134	0\\
1.8742468561714	0\\
1.87434685867147	0\\
1.87444686117153	0\\
1.87454686367159	0\\
1.87464686617165	0\\
1.87474686867172	0\\
1.87484687117178	0\\
1.87494687367184	0\\
1.8750468761719	0\\
1.87514687867197	0\\
1.87524688117203	0\\
1.87534688367209	0\\
1.87544688617215	0\\
1.87554688867222	0\\
1.87564689117228	0\\
1.87574689367234	0\\
1.8758468961724	0\\
1.87594689867247	0\\
1.87604690117253	0\\
1.87614690367259	0\\
1.87624690617265	0\\
1.87634690867272	0\\
1.87644691117278	0\\
1.87654691367284	0\\
1.8766469161729	0\\
1.87674691867297	0\\
1.87684692117303	0\\
1.87694692367309	0\\
1.87704692617315	0\\
1.87714692867322	0\\
1.87724693117328	0\\
1.87734693367334	0\\
1.8774469361734	0\\
1.87754693867347	0\\
1.87764694117353	0\\
1.87774694367359	0\\
1.87784694617365	0\\
1.87794694867372	0\\
1.87804695117378	0\\
1.87814695367384	0\\
1.8782469561739	0\\
1.87834695867397	0\\
1.87844696117403	0\\
1.87854696367409	0\\
1.87864696617415	0\\
1.87874696867422	0\\
1.87884697117428	0\\
1.87894697367434	0\\
1.8790469761744	0\\
1.87914697867447	0\\
1.87924698117453	0\\
1.87934698367459	0\\
1.87944698617465	0\\
1.87954698867472	0\\
1.87964699117478	0\\
1.87974699367484	0\\
1.8798469961749	0\\
1.87994699867497	0\\
1.88004700117503	0\\
1.88014700367509	0\\
1.88024700617515	0\\
1.88034700867522	0\\
1.88044701117528	0\\
1.88054701367534	0\\
1.8806470161754	0\\
1.88074701867547	0\\
1.88084702117553	0\\
1.88094702367559	0\\
1.88104702617565	0\\
1.88114702867572	0\\
1.88124703117578	0\\
1.88134703367584	0\\
1.8814470361759	0\\
1.88154703867597	0\\
1.88164704117603	0\\
1.88174704367609	0\\
1.88184704617615	0\\
1.88194704867622	0\\
1.88204705117628	0\\
1.88214705367634	0\\
1.8822470561764	0\\
1.88234705867647	0\\
1.88244706117653	0\\
1.88254706367659	0\\
1.88264706617665	0\\
1.88274706867672	0\\
1.88284707117678	0\\
1.88294707367684	0\\
1.8830470761769	0\\
1.88314707867697	0\\
1.88324708117703	0\\
1.88334708367709	0\\
1.88344708617715	0\\
1.88354708867722	0\\
1.88364709117728	0\\
1.88374709367734	0\\
1.8838470961774	0\\
1.88394709867747	0\\
1.88404710117753	0\\
1.88414710367759	0\\
1.88424710617765	0\\
1.88434710867772	0\\
1.88444711117778	0\\
1.88454711367784	0\\
1.8846471161779	0\\
1.88474711867797	0\\
1.88484712117803	0\\
1.88494712367809	0\\
1.88504712617815	0\\
1.88514712867822	0\\
1.88524713117828	0\\
1.88534713367834	0\\
1.8854471361784	0\\
1.88554713867847	0\\
1.88564714117853	0\\
1.88574714367859	0\\
1.88584714617865	0\\
1.88594714867872	0\\
1.88604715117878	0\\
1.88614715367884	0\\
1.8862471561789	0\\
1.88634715867897	0\\
1.88644716117903	0\\
1.88654716367909	0\\
1.88664716617915	0\\
1.88674716867922	0\\
1.88684717117928	0\\
1.88694717367934	0\\
1.8870471761794	0\\
1.88714717867947	0\\
1.88724718117953	0\\
1.88734718367959	0\\
1.88744718617965	0\\
1.88754718867972	0\\
1.88764719117978	0\\
1.88774719367984	0\\
1.8878471961799	0\\
1.88794719867997	0\\
1.88804720118003	0\\
1.88814720368009	0\\
1.88824720618015	0\\
1.88834720868022	0\\
1.88844721118028	0\\
1.88854721368034	0\\
1.8886472161804	0\\
1.88874721868047	0\\
1.88884722118053	0\\
1.88894722368059	0\\
1.88904722618065	0\\
1.88914722868072	0\\
1.88924723118078	0\\
1.88934723368084	0\\
1.8894472361809	0\\
1.88954723868097	0\\
1.88964724118103	0\\
1.88974724368109	0\\
1.88984724618115	0\\
1.88994724868122	0\\
1.89004725118128	0\\
1.89014725368134	0\\
1.8902472561814	0\\
1.89034725868147	0\\
1.89044726118153	0\\
1.89054726368159	0\\
1.89064726618165	0\\
1.89074726868172	0\\
1.89084727118178	0\\
1.89094727368184	0\\
1.8910472761819	0\\
1.89114727868197	0\\
1.89124728118203	0\\
1.89134728368209	0\\
1.89144728618215	0\\
1.89154728868222	0\\
1.89164729118228	0\\
1.89174729368234	0\\
1.8918472961824	0\\
1.89194729868247	0\\
1.89204730118253	0\\
1.89214730368259	0\\
1.89224730618265	0\\
1.89234730868272	0\\
1.89244731118278	0\\
1.89254731368284	0\\
1.8926473161829	0\\
1.89274731868297	0\\
1.89284732118303	0\\
1.89294732368309	0\\
1.89304732618315	0\\
1.89314732868322	0\\
1.89324733118328	0\\
1.89334733368334	0\\
1.8934473361834	0\\
1.89354733868347	0\\
1.89364734118353	0\\
1.89374734368359	0\\
1.89384734618365	0\\
1.89394734868372	0\\
1.89404735118378	0\\
1.89414735368384	0\\
1.8942473561839	0\\
1.89434735868397	0\\
1.89444736118403	0\\
1.89454736368409	0\\
1.89464736618415	0\\
1.89474736868422	0\\
1.89484737118428	0\\
1.89494737368434	0\\
1.8950473761844	0\\
1.89514737868447	0\\
1.89524738118453	0\\
1.89534738368459	0\\
1.89544738618465	0\\
1.89554738868472	0\\
1.89564739118478	0\\
1.89574739368484	0\\
1.8958473961849	0\\
1.89594739868497	0\\
1.89604740118503	0\\
1.89614740368509	0\\
1.89624740618515	0\\
1.89634740868522	0\\
1.89644741118528	0\\
1.89654741368534	0\\
1.8966474161854	0\\
1.89674741868547	0\\
1.89684742118553	0\\
1.89694742368559	0\\
1.89704742618565	0\\
1.89714742868572	0\\
1.89724743118578	0\\
1.89734743368584	0\\
1.8974474361859	0\\
1.89754743868597	0\\
1.89764744118603	0\\
1.89774744368609	0\\
1.89784744618615	0\\
1.89794744868622	0\\
1.89804745118628	0\\
1.89814745368634	0\\
1.8982474561864	0\\
1.89834745868647	0\\
1.89844746118653	0\\
1.89854746368659	0\\
1.89864746618665	0\\
1.89874746868672	0\\
1.89884747118678	0\\
1.89894747368684	0\\
1.8990474761869	0\\
1.89914747868697	0\\
1.89924748118703	0\\
1.89934748368709	0\\
1.89944748618715	0\\
1.89954748868722	0\\
1.89964749118728	0\\
1.89974749368734	0\\
1.8998474961874	0\\
1.89994749868747	0\\
1.90004750118753	0\\
1.90014750368759	0\\
1.90024750618765	0\\
1.90034750868772	0\\
1.90044751118778	0\\
1.90054751368784	0\\
1.9006475161879	0\\
1.90074751868797	0\\
1.90084752118803	0\\
1.90094752368809	0\\
1.90104752618815	0\\
1.90114752868822	0\\
1.90124753118828	0\\
1.90134753368834	0\\
1.9014475361884	0\\
1.90154753868847	0\\
1.90164754118853	0\\
1.90174754368859	0\\
1.90184754618865	0\\
1.90194754868872	0\\
1.90204755118878	0\\
1.90214755368884	0\\
1.9022475561889	0\\
1.90234755868897	0\\
1.90244756118903	0\\
1.90254756368909	0\\
1.90264756618915	0\\
1.90274756868922	0\\
1.90284757118928	0\\
1.90294757368934	0\\
1.9030475761894	0\\
1.90314757868947	0\\
1.90324758118953	0\\
1.90334758368959	0\\
1.90344758618965	0\\
1.90354758868972	0\\
1.90364759118978	0\\
1.90374759368984	0\\
1.9038475961899	0\\
1.90394759868997	0\\
1.90404760119003	0\\
1.90414760369009	0\\
1.90424760619015	0\\
1.90434760869022	0\\
1.90444761119028	0\\
1.90454761369034	0\\
1.9046476161904	0\\
1.90474761869047	0\\
1.90484762119053	0\\
1.90494762369059	0\\
1.90504762619065	0\\
1.90514762869072	0\\
1.90524763119078	0\\
1.90534763369084	0\\
1.9054476361909	0\\
1.90554763869097	0\\
1.90564764119103	0\\
1.90574764369109	0\\
1.90584764619115	0\\
1.90594764869122	0\\
1.90604765119128	0\\
1.90614765369134	0\\
1.9062476561914	0\\
1.90634765869147	0\\
1.90644766119153	0\\
1.90654766369159	0\\
1.90664766619165	0\\
1.90674766869172	0\\
1.90684767119178	0\\
1.90694767369184	0\\
1.9070476761919	0\\
1.90714767869197	0\\
1.90724768119203	0\\
1.90734768369209	0\\
1.90744768619215	0\\
1.90754768869222	0\\
1.90764769119228	0\\
1.90774769369234	0\\
1.9078476961924	0\\
1.90794769869247	0\\
1.90804770119253	0\\
1.90814770369259	0\\
1.90824770619265	0\\
1.90834770869272	0\\
1.90844771119278	0\\
1.90854771369284	0\\
1.9086477161929	0\\
1.90874771869297	0\\
1.90884772119303	0\\
1.90894772369309	0\\
1.90904772619315	0\\
1.90914772869322	0\\
1.90924773119328	0\\
1.90934773369334	0\\
1.9094477361934	0\\
1.90954773869347	0\\
1.90964774119353	0\\
1.90974774369359	0\\
1.90984774619365	0\\
1.90994774869372	0\\
1.91004775119378	0\\
1.91014775369384	0\\
1.9102477561939	0\\
1.91034775869397	0\\
1.91044776119403	0\\
1.91054776369409	0\\
1.91064776619415	0\\
1.91074776869422	0\\
1.91084777119428	0\\
1.91094777369434	0\\
1.9110477761944	0\\
1.91114777869447	0\\
1.91124778119453	0\\
1.91134778369459	0\\
1.91144778619465	0\\
1.91154778869472	0\\
1.91164779119478	0\\
1.91174779369484	0\\
1.9118477961949	0\\
1.91194779869497	0\\
1.91204780119503	0\\
1.91214780369509	0\\
1.91224780619515	0\\
1.91234780869522	0\\
1.91244781119528	0\\
1.91254781369534	0\\
1.9126478161954	0\\
1.91274781869547	0\\
1.91284782119553	0\\
1.91294782369559	0\\
1.91304782619565	0\\
1.91314782869572	0\\
1.91324783119578	0\\
1.91334783369584	0\\
1.9134478361959	0\\
1.91354783869597	0\\
1.91364784119603	0\\
1.91374784369609	0\\
1.91384784619616	0\\
1.91394784869622	0\\
1.91404785119628	0\\
1.91414785369634	0\\
1.9142478561964	0\\
1.91434785869647	0\\
1.91444786119653	0\\
1.91454786369659	0\\
1.91464786619665	0\\
1.91474786869672	0\\
1.91484787119678	0\\
1.91494787369684	0\\
1.9150478761969	0\\
1.91514787869697	0\\
1.91524788119703	0\\
1.91534788369709	0\\
1.91544788619715	0\\
1.91554788869722	0\\
1.91564789119728	0\\
1.91574789369734	0\\
1.91584789619741	0\\
1.91594789869747	0\\
1.91604790119753	0\\
1.91614790369759	0\\
1.91624790619765	0\\
1.91634790869772	0\\
1.91644791119778	0\\
1.91654791369784	0\\
1.9166479161979	0\\
1.91674791869797	0\\
1.91684792119803	0\\
1.91694792369809	0\\
1.91704792619816	0\\
1.91714792869822	0\\
1.91724793119828	0\\
1.91734793369834	0\\
1.9174479361984	0\\
1.91754793869847	0\\
1.91764794119853	0\\
1.91774794369859	0\\
1.91784794619866	0\\
1.91794794869872	0\\
1.91804795119878	0\\
1.91814795369884	0\\
1.9182479561989	0\\
1.91834795869897	0\\
1.91844796119903	0\\
1.91854796369909	0\\
1.91864796619915	0\\
1.91874796869922	0\\
1.91884797119928	0\\
1.91894797369934	0\\
1.91904797619941	0\\
1.91914797869947	0\\
1.91924798119953	0\\
1.91934798369959	0\\
1.91944798619965	0\\
1.91954798869972	0\\
1.91964799119978	0\\
1.91974799369984	0\\
1.91984799619991	0\\
1.91994799869997	0\\
1.92004800120003	0\\
1.92014800370009	0\\
1.92024800620016	0\\
1.92034800870022	0\\
1.92044801120028	0\\
1.92054801370034	0\\
1.9206480162004	0\\
1.92074801870047	0\\
1.92084802120053	0\\
1.92094802370059	0\\
1.92104802620066	0\\
1.92114802870072	0\\
1.92124803120078	0\\
1.92134803370084	0\\
1.9214480362009	0\\
1.92154803870097	0\\
1.92164804120103	0\\
1.92174804370109	0\\
1.92184804620116	0\\
1.92194804870122	0\\
1.92204805120128	0\\
1.92214805370134	0\\
1.92224805620141	0\\
1.92234805870147	0\\
1.92244806120153	0\\
1.92254806370159	0\\
1.92264806620165	0\\
1.92274806870172	0\\
1.92284807120178	0\\
1.92294807370184	0\\
1.92304807620191	0\\
1.92314807870197	0\\
1.92324808120203	0\\
1.92334808370209	0\\
1.92344808620216	0\\
1.92354808870222	0\\
1.92364809120228	0\\
1.92374809370234	0\\
1.92384809620241	0\\
1.92394809870247	0\\
1.92404810120253	0\\
1.92414810370259	0\\
1.92424810620266	0\\
1.92434810870272	0\\
1.92444811120278	0\\
1.92454811370284	0\\
1.9246481162029	0\\
1.92474811870297	0\\
1.92484812120303	0\\
1.92494812370309	0\\
1.92504812620316	0\\
1.92514812870322	0\\
1.92524813120328	0\\
1.92534813370334	0\\
1.92544813620341	0\\
1.92554813870347	0\\
1.92564814120353	0\\
1.92574814370359	0\\
1.92584814620366	0\\
1.92594814870372	0\\
1.92604815120378	0\\
1.92614815370384	0\\
1.92624815620391	0\\
1.92634815870397	0\\
1.92644816120403	0\\
1.92654816370409	0\\
1.92664816620416	0\\
1.92674816870422	0\\
1.92684817120428	0\\
1.92694817370434	0\\
1.92704817620441	0\\
1.92714817870447	0\\
1.92724818120453	0\\
1.92734818370459	0\\
1.92744818620466	0\\
1.92754818870472	0\\
1.92764819120478	0\\
1.92774819370484	0\\
1.92784819620491	0\\
1.92794819870497	0\\
1.92804820120503	0\\
1.92814820370509	0\\
1.92824820620516	0\\
1.92834820870522	0\\
1.92844821120528	0\\
1.92854821370534	0\\
1.92864821620541	0\\
1.92874821870547	0\\
1.92884822120553	0\\
1.92894822370559	0\\
1.92904822620566	0\\
1.92914822870572	0\\
1.92924823120578	0\\
1.92934823370584	0\\
1.92944823620591	0\\
1.92954823870597	0\\
1.92964824120603	0\\
1.92974824370609	0\\
1.92984824620616	0\\
1.92994824870622	0\\
1.93004825120628	0\\
1.93014825370634	0\\
1.93024825620641	0\\
1.93034825870647	0\\
1.93044826120653	0\\
1.93054826370659	0\\
1.93064826620666	0\\
1.93074826870672	0\\
1.93084827120678	0\\
1.93094827370684	0\\
1.93104827620691	0\\
1.93114827870697	0\\
1.93124828120703	0\\
1.93134828370709	0\\
1.93144828620716	0\\
1.93154828870722	0\\
1.93164829120728	0\\
1.93174829370734	0\\
1.93184829620741	0\\
1.93194829870747	0\\
1.93204830120753	0\\
1.93214830370759	0\\
1.93224830620766	0\\
1.93234830870772	0\\
1.93244831120778	0\\
1.93254831370784	0\\
1.93264831620791	0\\
1.93274831870797	0\\
1.93284832120803	0\\
1.93294832370809	0\\
1.93304832620816	0\\
1.93314832870822	0\\
1.93324833120828	0\\
1.93334833370834	0\\
1.93344833620841	0\\
1.93354833870847	0\\
1.93364834120853	0\\
1.93374834370859	0\\
1.93384834620866	0\\
1.93394834870872	0\\
1.93404835120878	0\\
1.93414835370884	0\\
1.93424835620891	0\\
1.93434835870897	0\\
1.93444836120903	0\\
1.93454836370909	0\\
1.93464836620916	0\\
1.93474836870922	0\\
1.93484837120928	0\\
1.93494837370934	0\\
1.93504837620941	0\\
1.93514837870947	0\\
1.93524838120953	0\\
1.93534838370959	0\\
1.93544838620966	0\\
1.93554838870972	0\\
1.93564839120978	0\\
1.93574839370984	0\\
1.93584839620991	0\\
1.93594839870997	0\\
1.93604840121003	0\\
1.93614840371009	0\\
1.93624840621016	0\\
1.93634840871022	0\\
1.93644841121028	0\\
1.93654841371034	0\\
1.93664841621041	0\\
1.93674841871047	0\\
1.93684842121053	0\\
1.93694842371059	0\\
1.93704842621066	0\\
1.93714842871072	0\\
1.93724843121078	0\\
1.93734843371084	0\\
1.93744843621091	0\\
1.93754843871097	0\\
1.93764844121103	0\\
1.93774844371109	0\\
1.93784844621116	0\\
1.93794844871122	0\\
1.93804845121128	0\\
1.93814845371134	0\\
1.93824845621141	0\\
1.93834845871147	0\\
1.93844846121153	0\\
1.93854846371159	0\\
1.93864846621166	0\\
1.93874846871172	0\\
1.93884847121178	0\\
1.93894847371184	0\\
1.93904847621191	0\\
1.93914847871197	0\\
1.93924848121203	0\\
1.93934848371209	0\\
1.93944848621216	0\\
1.93954848871222	0\\
1.93964849121228	0\\
1.93974849371234	0\\
1.93984849621241	0\\
1.93994849871247	0\\
1.94004850121253	0\\
1.94014850371259	0\\
1.94024850621266	0\\
1.94034850871272	0\\
1.94044851121278	0\\
1.94054851371284	0\\
1.94064851621291	0\\
1.94074851871297	0\\
1.94084852121303	0\\
1.94094852371309	0\\
1.94104852621316	0\\
1.94114852871322	0\\
1.94124853121328	0\\
1.94134853371334	0\\
1.94144853621341	0\\
1.94154853871347	0\\
1.94164854121353	0\\
1.94174854371359	0\\
1.94184854621366	0\\
1.94194854871372	0\\
1.94204855121378	0\\
1.94214855371384	0\\
1.94224855621391	0\\
1.94234855871397	0\\
1.94244856121403	0\\
1.94254856371409	0\\
1.94264856621416	0\\
1.94274856871422	0\\
1.94284857121428	0\\
1.94294857371434	0\\
1.94304857621441	0\\
1.94314857871447	0\\
1.94324858121453	0\\
1.94334858371459	0\\
1.94344858621466	0\\
1.94354858871472	0\\
1.94364859121478	0\\
1.94374859371484	0\\
1.94384859621491	0\\
1.94394859871497	0\\
1.94404860121503	0\\
1.94414860371509	0\\
1.94424860621516	0\\
1.94434860871522	0\\
1.94444861121528	0\\
1.94454861371534	0\\
1.94464861621541	0\\
1.94474861871547	0\\
1.94484862121553	0\\
1.94494862371559	0\\
1.94504862621566	0\\
1.94514862871572	0\\
1.94524863121578	0\\
1.94534863371584	0\\
1.94544863621591	0\\
1.94554863871597	0\\
1.94564864121603	0\\
1.94574864371609	0\\
1.94584864621616	0\\
1.94594864871622	0\\
1.94604865121628	0\\
1.94614865371634	0\\
1.94624865621641	0\\
1.94634865871647	0\\
1.94644866121653	0\\
1.94654866371659	0\\
1.94664866621666	0\\
1.94674866871672	0\\
1.94684867121678	0\\
1.94694867371684	0\\
1.94704867621691	0\\
1.94714867871697	0\\
1.94724868121703	0\\
1.94734868371709	0\\
1.94744868621716	0\\
1.94754868871722	0\\
1.94764869121728	0\\
1.94774869371734	0\\
1.94784869621741	0\\
1.94794869871747	0\\
1.94804870121753	0\\
1.94814870371759	0\\
1.94824870621766	0\\
1.94834870871772	0\\
1.94844871121778	0\\
1.94854871371784	0\\
1.94864871621791	0\\
1.94874871871797	0\\
1.94884872121803	0\\
1.94894872371809	0\\
1.94904872621816	0\\
1.94914872871822	0\\
1.94924873121828	0\\
1.94934873371834	0\\
1.94944873621841	0\\
1.94954873871847	0\\
1.94964874121853	0\\
1.94974874371859	0\\
1.94984874621866	0\\
1.94994874871872	0\\
1.95004875121878	0\\
1.95014875371884	0\\
1.95024875621891	0\\
1.95034875871897	0\\
1.95044876121903	0\\
1.95054876371909	0\\
1.95064876621916	0\\
1.95074876871922	0\\
1.95084877121928	0\\
1.95094877371934	0\\
1.95104877621941	0\\
1.95114877871947	0\\
1.95124878121953	0\\
1.95134878371959	0\\
1.95144878621966	0\\
1.95154878871972	0\\
1.95164879121978	0\\
1.95174879371984	0\\
1.95184879621991	0\\
1.95194879871997	0\\
1.95204880122003	0\\
1.95214880372009	0\\
1.95224880622016	0\\
1.95234880872022	0\\
1.95244881122028	0\\
1.95254881372034	0\\
1.95264881622041	0\\
1.95274881872047	0\\
1.95284882122053	0\\
1.95294882372059	0\\
1.95304882622066	0\\
1.95314882872072	0\\
1.95324883122078	0\\
1.95334883372084	0\\
1.95344883622091	0\\
1.95354883872097	0\\
1.95364884122103	0\\
1.95374884372109	0\\
1.95384884622116	0\\
1.95394884872122	0\\
1.95404885122128	0\\
1.95414885372134	0\\
1.95424885622141	0\\
1.95434885872147	0\\
1.95444886122153	0\\
1.95454886372159	0\\
1.95464886622166	0\\
1.95474886872172	0\\
1.95484887122178	0\\
1.95494887372184	0\\
1.95504887622191	0\\
1.95514887872197	0\\
1.95524888122203	0\\
1.95534888372209	0\\
1.95544888622216	0\\
1.95554888872222	0\\
1.95564889122228	0\\
1.95574889372234	0\\
1.95584889622241	0\\
1.95594889872247	0\\
1.95604890122253	0\\
1.95614890372259	0\\
1.95624890622266	0\\
1.95634890872272	0\\
1.95644891122278	0\\
1.95654891372284	0\\
1.95664891622291	0\\
1.95674891872297	0\\
1.95684892122303	0\\
1.95694892372309	0\\
1.95704892622316	0\\
1.95714892872322	0\\
1.95724893122328	0\\
1.95734893372334	0\\
1.95744893622341	0\\
1.95754893872347	0\\
1.95764894122353	0\\
1.95774894372359	0\\
1.95784894622366	0\\
1.95794894872372	0\\
1.95804895122378	0\\
1.95814895372384	0\\
1.95824895622391	0\\
1.95834895872397	0\\
1.95844896122403	0\\
1.95854896372409	0\\
1.95864896622416	0\\
1.95874896872422	0\\
1.95884897122428	0\\
1.95894897372434	0\\
1.95904897622441	0\\
1.95914897872447	0\\
1.95924898122453	0\\
1.95934898372459	0\\
1.95944898622466	0\\
1.95954898872472	0\\
1.95964899122478	0\\
1.95974899372484	0\\
1.95984899622491	0\\
1.95994899872497	0\\
1.96004900122503	0\\
1.96014900372509	0\\
1.96024900622516	0\\
1.96034900872522	0\\
1.96044901122528	0\\
1.96054901372534	0\\
1.96064901622541	0\\
1.96074901872547	0\\
1.96084902122553	0\\
1.96094902372559	0\\
1.96104902622566	0\\
1.96114902872572	0\\
1.96124903122578	0\\
1.96134903372584	0\\
1.96144903622591	0\\
1.96154903872597	0\\
1.96164904122603	0\\
1.96174904372609	0\\
1.96184904622616	0\\
1.96194904872622	0\\
1.96204905122628	0\\
1.96214905372634	0\\
1.96224905622641	0\\
1.96234905872647	0\\
1.96244906122653	0\\
1.96254906372659	0\\
1.96264906622666	0\\
1.96274906872672	0\\
1.96284907122678	0\\
1.96294907372684	0\\
1.96304907622691	0\\
1.96314907872697	0\\
1.96324908122703	0\\
1.96334908372709	0\\
1.96344908622716	0\\
1.96354908872722	0\\
1.96364909122728	0\\
1.96374909372734	0\\
1.96384909622741	0\\
1.96394909872747	0\\
1.96404910122753	0\\
1.96414910372759	0\\
1.96424910622766	0\\
1.96434910872772	0\\
1.96444911122778	0\\
1.96454911372784	0\\
1.96464911622791	0\\
1.96474911872797	0\\
1.96484912122803	0\\
1.96494912372809	0\\
1.96504912622816	0\\
1.96514912872822	0\\
1.96524913122828	0\\
1.96534913372834	0\\
1.96544913622841	0\\
1.96554913872847	0\\
1.96564914122853	0\\
1.96574914372859	0\\
1.96584914622866	0\\
1.96594914872872	0\\
1.96604915122878	0\\
1.96614915372884	0\\
1.96624915622891	0\\
1.96634915872897	0\\
1.96644916122903	0\\
1.96654916372909	0\\
1.96664916622916	0\\
1.96674916872922	0\\
1.96684917122928	0\\
1.96694917372934	0\\
1.96704917622941	0\\
1.96714917872947	0\\
1.96724918122953	0\\
1.96734918372959	0\\
1.96744918622966	0\\
1.96754918872972	0\\
1.96764919122978	0\\
1.96774919372984	0\\
1.96784919622991	0\\
1.96794919872997	0\\
1.96804920123003	0\\
1.96814920373009	0\\
1.96824920623016	0\\
1.96834920873022	0\\
1.96844921123028	0\\
1.96854921373034	0\\
1.96864921623041	0\\
1.96874921873047	0\\
1.96884922123053	0\\
1.96894922373059	0\\
1.96904922623066	0\\
1.96914922873072	0\\
1.96924923123078	0\\
1.96934923373084	0\\
1.96944923623091	0\\
1.96954923873097	0\\
1.96964924123103	0\\
1.96974924373109	0\\
1.96984924623116	0\\
1.96994924873122	0\\
1.97004925123128	0\\
1.97014925373134	0\\
1.97024925623141	0\\
1.97034925873147	0\\
1.97044926123153	0\\
1.97054926373159	0\\
1.97064926623166	0\\
1.97074926873172	0\\
1.97084927123178	0\\
1.97094927373184	0\\
1.97104927623191	0\\
1.97114927873197	0\\
1.97124928123203	0\\
1.97134928373209	0\\
1.97144928623216	0\\
1.97154928873222	0\\
1.97164929123228	0\\
1.97174929373234	0\\
1.97184929623241	0\\
1.97194929873247	0\\
1.97204930123253	0\\
1.97214930373259	0\\
1.97224930623266	0\\
1.97234930873272	0\\
1.97244931123278	0\\
1.97254931373284	0\\
1.97264931623291	0\\
1.97274931873297	0\\
1.97284932123303	0\\
1.97294932373309	0\\
1.97304932623316	0\\
1.97314932873322	0\\
1.97324933123328	0\\
1.97334933373334	0\\
1.97344933623341	0\\
1.97354933873347	0\\
1.97364934123353	0\\
1.97374934373359	0\\
1.97384934623366	0\\
1.97394934873372	0\\
1.97404935123378	0\\
1.97414935373384	0\\
1.97424935623391	0\\
1.97434935873397	0\\
1.97444936123403	0\\
1.97454936373409	0\\
1.97464936623416	0\\
1.97474936873422	0\\
1.97484937123428	0\\
1.97494937373434	0\\
1.97504937623441	0\\
1.97514937873447	0\\
1.97524938123453	0\\
1.97534938373459	0\\
1.97544938623466	0\\
1.97554938873472	0\\
1.97564939123478	0\\
1.97574939373484	0\\
1.97584939623491	0\\
1.97594939873497	0\\
1.97604940123503	0\\
1.97614940373509	0\\
1.97624940623516	0\\
1.97634940873522	0\\
1.97644941123528	0\\
1.97654941373534	0\\
1.97664941623541	0\\
1.97674941873547	0\\
1.97684942123553	0\\
1.97694942373559	0\\
1.97704942623566	0\\
1.97714942873572	0\\
1.97724943123578	0\\
1.97734943373584	0\\
1.97744943623591	0\\
1.97754943873597	0\\
1.97764944123603	0\\
1.97774944373609	0\\
1.97784944623616	0\\
1.97794944873622	0\\
1.97804945123628	0\\
1.97814945373634	0\\
1.97824945623641	0\\
1.97834945873647	0\\
1.97844946123653	0\\
1.97854946373659	0\\
1.97864946623666	0\\
1.97874946873672	0\\
1.97884947123678	0\\
1.97894947373684	0\\
1.97904947623691	0\\
1.97914947873697	0\\
1.97924948123703	0\\
1.97934948373709	0\\
1.97944948623716	0\\
1.97954948873722	0\\
1.97964949123728	0\\
1.97974949373734	0\\
1.97984949623741	0\\
1.97994949873747	0\\
1.98004950123753	0\\
1.98014950373759	0\\
1.98024950623766	0\\
1.98034950873772	0\\
1.98044951123778	0\\
1.98054951373784	0\\
1.98064951623791	0\\
1.98074951873797	0\\
1.98084952123803	0\\
1.98094952373809	0\\
1.98104952623816	0\\
1.98114952873822	0\\
1.98124953123828	0\\
1.98134953373834	0\\
1.98144953623841	0\\
1.98154953873847	0\\
1.98164954123853	0\\
1.98174954373859	0\\
1.98184954623866	0\\
1.98194954873872	0\\
1.98204955123878	0\\
1.98214955373884	0\\
1.98224955623891	0\\
1.98234955873897	0\\
1.98244956123903	0\\
1.98254956373909	0\\
1.98264956623916	0\\
1.98274956873922	0\\
1.98284957123928	0\\
1.98294957373934	0\\
1.98304957623941	0\\
1.98314957873947	0\\
1.98324958123953	0\\
1.98334958373959	0\\
1.98344958623966	0\\
1.98354958873972	0\\
1.98364959123978	0\\
1.98374959373984	0\\
1.98384959623991	0\\
1.98394959873997	0\\
1.98404960124003	0\\
1.98414960374009	0\\
1.98424960624016	0\\
1.98434960874022	0\\
1.98444961124028	0\\
1.98454961374034	0\\
1.98464961624041	0\\
1.98474961874047	0\\
1.98484962124053	0\\
1.98494962374059	0\\
1.98504962624066	0\\
1.98514962874072	0\\
1.98524963124078	0\\
1.98534963374084	0\\
1.98544963624091	0\\
1.98554963874097	0\\
1.98564964124103	0\\
1.98574964374109	0\\
1.98584964624116	0\\
1.98594964874122	0\\
1.98604965124128	0\\
1.98614965374134	0\\
1.98624965624141	0\\
1.98634965874147	0\\
1.98644966124153	0\\
1.98654966374159	0\\
1.98664966624166	0\\
1.98674966874172	0\\
1.98684967124178	0\\
1.98694967374184	0\\
1.98704967624191	0\\
1.98714967874197	0\\
1.98724968124203	0\\
1.98734968374209	0\\
1.98744968624216	0\\
1.98754968874222	0\\
1.98764969124228	0\\
1.98774969374234	0\\
1.98784969624241	0\\
1.98794969874247	0\\
1.98804970124253	0\\
1.98814970374259	0\\
1.98824970624266	0\\
1.98834970874272	0\\
1.98844971124278	0\\
1.98854971374284	0\\
1.98864971624291	0\\
1.98874971874297	0\\
1.98884972124303	0\\
1.98894972374309	0\\
1.98904972624316	0\\
1.98914972874322	0\\
1.98924973124328	0\\
1.98934973374334	0\\
1.98944973624341	0\\
1.98954973874347	0\\
1.98964974124353	0\\
1.98974974374359	0\\
1.98984974624366	0\\
1.98994974874372	0\\
1.99004975124378	0\\
1.99014975374384	0\\
1.99024975624391	0\\
1.99034975874397	0\\
1.99044976124403	0\\
1.99054976374409	0\\
1.99064976624416	0\\
1.99074976874422	0\\
1.99084977124428	0\\
1.99094977374434	0\\
1.99104977624441	0\\
1.99114977874447	0\\
1.99124978124453	0\\
1.99134978374459	0\\
1.99144978624466	0\\
1.99154978874472	0\\
1.99164979124478	0\\
1.99174979374484	0\\
1.99184979624491	0\\
1.99194979874497	0\\
1.99204980124503	0\\
1.99214980374509	0\\
1.99224980624516	0\\
1.99234980874522	0\\
1.99244981124528	0\\
1.99254981374534	0\\
1.99264981624541	0\\
1.99274981874547	0\\
1.99284982124553	0\\
1.99294982374559	0\\
1.99304982624566	0\\
1.99314982874572	0\\
1.99324983124578	0\\
1.99334983374584	0\\
1.99344983624591	0\\
1.99354983874597	0\\
1.99364984124603	0\\
1.99374984374609	0\\
1.99384984624616	0\\
1.99394984874622	0\\
1.99404985124628	0\\
1.99414985374634	0\\
1.99424985624641	0\\
1.99434985874647	0\\
1.99444986124653	0\\
1.99454986374659	0\\
1.99464986624666	0\\
1.99474986874672	0\\
1.99484987124678	0\\
1.99494987374684	0\\
1.99504987624691	0\\
1.99514987874697	0\\
1.99524988124703	0\\
1.99534988374709	0\\
1.99544988624716	0\\
1.99554988874722	0\\
1.99564989124728	0\\
1.99574989374734	0\\
1.99584989624741	0\\
1.99594989874747	0\\
1.99604990124753	0\\
1.99614990374759	0\\
1.99624990624766	0\\
1.99634990874772	0\\
1.99644991124778	0\\
1.99654991374784	0\\
1.99664991624791	0\\
1.99674991874797	0\\
1.99684992124803	0\\
1.99694992374809	0\\
1.99704992624816	0\\
1.99714992874822	0\\
1.99724993124828	0\\
1.99734993374834	0\\
1.99744993624841	0\\
1.99754993874847	0\\
1.99764994124853	0\\
1.99774994374859	0\\
1.99784994624866	0\\
1.99794994874872	0\\
1.99804995124878	0\\
1.99814995374884	0\\
1.99824995624891	0\\
1.99834995874897	0\\
1.99844996124903	0\\
1.99854996374909	0\\
1.99864996624916	0\\
1.99874996874922	0\\
1.99884997124928	0\\
1.99894997374934	0\\
1.99904997624941	0\\
1.99914997874947	0\\
1.99924998124953	0\\
1.99934998374959	0\\
1.99944998624966	0\\
1.99954998874972	0\\
1.99964999124978	0\\
1.99974999374984	0\\
1.99984999624991	0\\
1.99994999874997	0\\
2.00005000125003	0\\
};
\addplot [color=mycolor2,solid,forget plot]
  table[row sep=crcr]{%
2.00005000125003	0\\
2.00015000375009	0\\
2.00025000625016	0\\
2.00035000875022	0\\
2.00045001125028	0\\
2.00055001375034	0\\
2.00065001625041	0\\
2.00075001875047	0\\
2.00085002125053	0\\
2.00095002375059	0\\
2.00105002625066	0\\
2.00115002875072	0\\
2.00125003125078	0\\
2.00135003375084	0\\
2.00145003625091	0\\
2.00155003875097	0\\
2.00165004125103	0\\
2.00175004375109	0\\
2.00185004625116	0\\
2.00195004875122	0\\
2.00205005125128	0\\
2.00215005375134	0\\
2.00225005625141	0\\
2.00235005875147	0\\
2.00245006125153	0\\
2.00255006375159	0\\
2.00265006625166	0\\
2.00275006875172	0\\
2.00285007125178	0\\
2.00295007375184	0\\
2.00305007625191	0\\
2.00315007875197	0\\
2.00325008125203	0\\
2.00335008375209	0\\
2.00345008625216	0\\
2.00355008875222	0\\
2.00365009125228	0\\
2.00375009375234	0\\
2.00385009625241	0\\
2.00395009875247	0\\
2.00405010125253	0\\
2.00415010375259	0\\
2.00425010625266	0\\
2.00435010875272	0\\
2.00445011125278	0\\
2.00455011375284	0\\
2.00465011625291	0\\
2.00475011875297	0\\
2.00485012125303	0\\
2.00495012375309	0\\
2.00505012625316	0\\
2.00515012875322	0\\
2.00525013125328	0\\
2.00535013375334	0\\
2.00545013625341	0\\
2.00555013875347	0\\
2.00565014125353	0\\
2.00575014375359	0\\
2.00585014625366	0\\
2.00595014875372	0\\
2.00605015125378	0\\
2.00615015375384	0\\
2.00625015625391	0\\
2.00635015875397	0\\
2.00645016125403	0\\
2.00655016375409	0\\
2.00665016625416	0\\
2.00675016875422	0\\
2.00685017125428	0\\
2.00695017375434	0\\
2.00705017625441	0\\
2.00715017875447	0\\
2.00725018125453	0\\
2.00735018375459	0\\
2.00745018625466	0\\
2.00755018875472	0\\
2.00765019125478	0\\
2.00775019375484	0\\
2.00785019625491	0\\
2.00795019875497	0\\
2.00805020125503	0\\
2.00815020375509	0\\
2.00825020625516	0\\
2.00835020875522	0\\
2.00845021125528	0\\
2.00855021375534	0\\
2.00865021625541	0\\
2.00875021875547	0\\
2.00885022125553	0\\
2.00895022375559	0\\
2.00905022625566	0\\
2.00915022875572	0\\
2.00925023125578	0\\
2.00935023375584	0\\
2.00945023625591	0\\
2.00955023875597	0\\
2.00965024125603	0\\
2.00975024375609	0\\
2.00985024625616	0\\
2.00995024875622	0\\
2.01005025125628	0\\
2.01015025375634	0\\
2.01025025625641	0\\
2.01035025875647	0\\
2.01045026125653	0\\
2.01055026375659	0\\
2.01065026625666	0\\
2.01075026875672	0\\
2.01085027125678	0\\
2.01095027375684	0\\
2.01105027625691	0\\
2.01115027875697	0\\
2.01125028125703	0\\
2.01135028375709	0\\
2.01145028625716	0\\
2.01155028875722	0\\
2.01165029125728	0\\
2.01175029375734	0\\
2.01185029625741	0\\
2.01195029875747	0\\
2.01205030125753	0\\
2.01215030375759	0\\
2.01225030625766	0\\
2.01235030875772	0\\
2.01245031125778	0\\
2.01255031375784	0\\
2.01265031625791	0\\
2.01275031875797	0\\
2.01285032125803	0\\
2.01295032375809	0\\
2.01305032625816	0\\
2.01315032875822	0\\
2.01325033125828	0\\
2.01335033375834	0\\
2.01345033625841	0\\
2.01355033875847	0\\
2.01365034125853	0\\
2.01375034375859	0\\
2.01385034625866	0\\
2.01395034875872	0\\
2.01405035125878	0\\
2.01415035375884	0\\
2.01425035625891	0\\
2.01435035875897	0\\
2.01445036125903	0\\
2.01455036375909	0\\
2.01465036625916	0\\
2.01475036875922	0\\
2.01485037125928	0\\
2.01495037375934	0\\
2.01505037625941	0\\
2.01515037875947	0\\
2.01525038125953	0\\
2.01535038375959	0\\
2.01545038625966	0\\
2.01555038875972	0\\
2.01565039125978	0\\
2.01575039375984	0\\
2.01585039625991	0\\
2.01595039875997	0\\
2.01605040126003	0\\
2.01615040376009	0\\
2.01625040626016	0\\
2.01635040876022	0\\
2.01645041126028	0\\
2.01655041376034	0\\
2.01665041626041	0\\
2.01675041876047	0\\
2.01685042126053	0\\
2.01695042376059	0\\
2.01705042626066	0\\
2.01715042876072	0\\
2.01725043126078	0\\
2.01735043376084	0\\
2.01745043626091	0\\
2.01755043876097	0\\
2.01765044126103	0\\
2.01775044376109	0\\
2.01785044626116	0\\
2.01795044876122	0\\
2.01805045126128	0\\
2.01815045376134	0\\
2.01825045626141	0\\
2.01835045876147	0\\
2.01845046126153	0\\
2.01855046376159	0\\
2.01865046626166	0\\
2.01875046876172	0\\
2.01885047126178	0\\
2.01895047376184	0\\
2.01905047626191	0\\
2.01915047876197	0\\
2.01925048126203	0\\
2.01935048376209	0\\
2.01945048626216	0\\
2.01955048876222	0\\
2.01965049126228	0\\
2.01975049376234	0\\
2.01985049626241	0\\
2.01995049876247	0\\
2.02005050126253	0\\
2.02015050376259	0\\
2.02025050626266	0\\
2.02035050876272	0\\
2.02045051126278	0\\
2.02055051376284	0\\
2.02065051626291	0\\
2.02075051876297	0\\
2.02085052126303	0\\
2.02095052376309	0\\
2.02105052626316	0\\
2.02115052876322	0\\
2.02125053126328	0\\
2.02135053376334	0\\
2.02145053626341	0\\
2.02155053876347	0\\
2.02165054126353	0\\
2.02175054376359	0\\
2.02185054626366	0\\
2.02195054876372	0\\
2.02205055126378	0\\
2.02215055376384	0\\
2.02225055626391	0\\
2.02235055876397	0\\
2.02245056126403	0\\
2.02255056376409	0\\
2.02265056626416	0\\
2.02275056876422	0\\
2.02285057126428	0\\
2.02295057376434	0\\
2.02305057626441	0\\
2.02315057876447	0\\
2.02325058126453	0\\
2.02335058376459	0\\
2.02345058626466	0\\
2.02355058876472	0\\
2.02365059126478	0\\
2.02375059376484	0\\
2.02385059626491	0\\
2.02395059876497	0\\
2.02405060126503	0\\
2.02415060376509	0\\
2.02425060626516	0\\
2.02435060876522	0\\
2.02445061126528	0\\
2.02455061376534	0\\
2.02465061626541	0\\
2.02475061876547	0\\
2.02485062126553	0\\
2.02495062376559	0\\
2.02505062626566	0\\
2.02515062876572	0\\
2.02525063126578	0\\
2.02535063376584	0\\
2.02545063626591	0\\
2.02555063876597	0\\
2.02565064126603	0\\
2.02575064376609	0\\
2.02585064626616	0\\
2.02595064876622	0\\
2.02605065126628	0\\
2.02615065376634	0\\
2.02625065626641	0\\
2.02635065876647	0\\
2.02645066126653	0\\
2.02655066376659	0\\
2.02665066626666	0\\
2.02675066876672	0\\
2.02685067126678	0\\
2.02695067376684	0\\
2.02705067626691	0\\
2.02715067876697	0\\
2.02725068126703	0\\
2.02735068376709	0\\
2.02745068626716	0\\
2.02755068876722	0\\
2.02765069126728	0\\
2.02775069376734	0\\
2.02785069626741	0\\
2.02795069876747	0\\
2.02805070126753	0\\
2.02815070376759	0\\
2.02825070626766	0\\
2.02835070876772	0\\
2.02845071126778	0\\
2.02855071376784	0\\
2.02865071626791	0\\
2.02875071876797	0\\
2.02885072126803	0\\
2.02895072376809	0\\
2.02905072626816	0\\
2.02915072876822	0\\
2.02925073126828	0\\
2.02935073376834	0\\
2.02945073626841	0\\
2.02955073876847	0\\
2.02965074126853	0\\
2.02975074376859	0\\
2.02985074626866	0\\
2.02995074876872	0\\
2.03005075126878	0\\
2.03015075376884	0\\
2.03025075626891	0\\
2.03035075876897	0\\
2.03045076126903	0\\
2.03055076376909	0\\
2.03065076626916	0\\
2.03075076876922	0\\
2.03085077126928	0\\
2.03095077376934	0\\
2.03105077626941	0\\
2.03115077876947	0\\
2.03125078126953	0\\
2.03135078376959	0\\
2.03145078626966	0\\
2.03155078876972	0\\
2.03165079126978	0\\
2.03175079376984	0\\
2.03185079626991	0\\
2.03195079876997	0\\
2.03205080127003	0\\
2.03215080377009	0\\
2.03225080627016	0\\
2.03235080877022	0\\
2.03245081127028	0\\
2.03255081377034	0\\
2.03265081627041	0\\
2.03275081877047	0\\
2.03285082127053	0\\
2.03295082377059	0\\
2.03305082627066	0\\
2.03315082877072	0\\
2.03325083127078	0\\
2.03335083377084	0\\
2.03345083627091	0\\
2.03355083877097	0\\
2.03365084127103	0\\
2.03375084377109	0\\
2.03385084627116	0\\
2.03395084877122	0\\
2.03405085127128	0\\
2.03415085377134	0\\
2.03425085627141	0\\
2.03435085877147	0\\
2.03445086127153	0\\
2.03455086377159	0\\
2.03465086627166	0\\
2.03475086877172	0\\
2.03485087127178	0\\
2.03495087377184	0\\
2.03505087627191	0\\
2.03515087877197	0\\
2.03525088127203	0\\
2.03535088377209	0\\
2.03545088627216	0\\
2.03555088877222	0\\
2.03565089127228	0\\
2.03575089377234	0\\
2.03585089627241	0\\
2.03595089877247	0\\
2.03605090127253	0\\
2.03615090377259	0\\
2.03625090627266	0\\
2.03635090877272	0\\
2.03645091127278	0\\
2.03655091377284	0\\
2.03665091627291	0\\
2.03675091877297	0\\
2.03685092127303	0\\
2.03695092377309	0\\
2.03705092627316	0\\
2.03715092877322	0\\
2.03725093127328	0\\
2.03735093377334	0\\
2.03745093627341	0\\
2.03755093877347	0\\
2.03765094127353	0\\
2.03775094377359	0\\
2.03785094627366	0\\
2.03795094877372	0\\
2.03805095127378	0\\
2.03815095377384	0\\
2.03825095627391	0\\
2.03835095877397	0\\
2.03845096127403	0\\
2.03855096377409	0\\
2.03865096627416	0\\
2.03875096877422	0\\
2.03885097127428	0\\
2.03895097377434	0\\
2.03905097627441	0\\
2.03915097877447	0\\
2.03925098127453	0\\
2.03935098377459	0\\
2.03945098627466	0\\
2.03955098877472	0\\
2.03965099127478	0\\
2.03975099377484	0\\
2.03985099627491	0\\
2.03995099877497	0\\
2.04005100127503	0\\
2.04015100377509	0\\
2.04025100627516	0\\
2.04035100877522	0\\
2.04045101127528	0\\
2.04055101377534	0\\
2.04065101627541	0\\
2.04075101877547	0\\
2.04085102127553	0\\
2.04095102377559	0\\
2.04105102627566	0\\
2.04115102877572	0\\
2.04125103127578	0\\
2.04135103377584	0\\
2.04145103627591	0\\
2.04155103877597	0\\
2.04165104127603	0\\
2.04175104377609	0\\
2.04185104627616	0\\
2.04195104877622	0\\
2.04205105127628	0\\
2.04215105377634	0\\
2.04225105627641	0\\
2.04235105877647	0\\
2.04245106127653	0\\
2.04255106377659	0\\
2.04265106627666	0\\
2.04275106877672	0\\
2.04285107127678	0\\
2.04295107377684	0\\
2.04305107627691	0\\
2.04315107877697	0\\
2.04325108127703	0\\
2.04335108377709	0\\
2.04345108627716	0\\
2.04355108877722	0\\
2.04365109127728	0\\
2.04375109377734	0\\
2.04385109627741	0\\
2.04395109877747	0\\
2.04405110127753	0\\
2.04415110377759	0\\
2.04425110627766	0\\
2.04435110877772	0\\
2.04445111127778	0\\
2.04455111377784	0\\
2.04465111627791	0\\
2.04475111877797	0\\
2.04485112127803	0\\
2.04495112377809	0\\
2.04505112627816	0\\
2.04515112877822	0\\
2.04525113127828	0\\
2.04535113377834	0\\
2.04545113627841	0\\
2.04555113877847	0\\
2.04565114127853	0\\
2.04575114377859	0\\
2.04585114627866	0\\
2.04595114877872	0\\
2.04605115127878	0\\
2.04615115377884	0\\
2.04625115627891	0\\
2.04635115877897	0\\
2.04645116127903	0\\
2.04655116377909	0\\
2.04665116627916	0\\
2.04675116877922	0\\
2.04685117127928	0\\
2.04695117377934	0\\
2.04705117627941	0\\
2.04715117877947	0\\
2.04725118127953	0\\
2.04735118377959	0\\
2.04745118627966	0\\
2.04755118877972	0\\
2.04765119127978	0\\
2.04775119377984	0\\
2.04785119627991	0\\
2.04795119877997	0\\
2.04805120128003	0\\
2.04815120378009	0\\
2.04825120628016	0\\
2.04835120878022	0\\
2.04845121128028	0\\
2.04855121378034	0\\
2.04865121628041	0\\
2.04875121878047	0\\
2.04885122128053	0\\
2.04895122378059	0\\
2.04905122628066	0\\
2.04915122878072	0\\
2.04925123128078	0\\
2.04935123378084	0\\
2.04945123628091	0\\
2.04955123878097	0\\
2.04965124128103	0\\
2.04975124378109	0\\
2.04985124628116	0\\
2.04995124878122	0\\
2.05005125128128	0\\
2.05015125378134	0\\
2.05025125628141	0\\
2.05035125878147	0\\
2.05045126128153	0\\
2.05055126378159	0\\
2.05065126628166	0\\
2.05075126878172	0\\
2.05085127128178	0\\
2.05095127378184	0\\
2.05105127628191	0\\
2.05115127878197	0\\
2.05125128128203	0\\
2.05135128378209	0\\
2.05145128628216	0\\
2.05155128878222	0\\
2.05165129128228	0\\
2.05175129378234	0\\
2.05185129628241	0\\
2.05195129878247	0\\
2.05205130128253	0\\
2.05215130378259	0\\
2.05225130628266	0\\
2.05235130878272	0\\
2.05245131128278	0\\
2.05255131378284	0\\
2.05265131628291	0\\
2.05275131878297	0\\
2.05285132128303	0\\
2.05295132378309	0\\
2.05305132628316	0\\
2.05315132878322	0\\
2.05325133128328	0\\
2.05335133378334	0\\
2.05345133628341	0\\
2.05355133878347	0\\
2.05365134128353	0\\
2.05375134378359	0\\
2.05385134628366	0\\
2.05395134878372	0\\
2.05405135128378	0\\
2.05415135378384	0\\
2.05425135628391	0\\
2.05435135878397	0\\
2.05445136128403	0\\
2.05455136378409	0\\
2.05465136628416	0\\
2.05475136878422	0\\
2.05485137128428	0\\
2.05495137378434	0\\
2.05505137628441	0\\
2.05515137878447	0\\
2.05525138128453	0\\
2.05535138378459	0\\
2.05545138628466	0\\
2.05555138878472	0\\
2.05565139128478	0\\
2.05575139378484	0\\
2.05585139628491	0\\
2.05595139878497	0\\
2.05605140128503	0\\
2.05615140378509	0\\
2.05625140628516	0\\
2.05635140878522	0\\
2.05645141128528	0\\
2.05655141378534	0\\
2.05665141628541	0\\
2.05675141878547	0\\
2.05685142128553	0\\
2.05695142378559	0\\
2.05705142628566	0\\
2.05715142878572	0\\
2.05725143128578	0\\
2.05735143378584	0\\
2.05745143628591	0\\
2.05755143878597	0\\
2.05765144128603	0\\
2.05775144378609	0\\
2.05785144628616	0\\
2.05795144878622	0\\
2.05805145128628	0\\
2.05815145378634	0\\
2.05825145628641	0\\
2.05835145878647	0\\
2.05845146128653	0\\
2.05855146378659	0\\
2.05865146628666	0\\
2.05875146878672	0\\
2.05885147128678	0\\
2.05895147378684	0\\
2.05905147628691	0\\
2.05915147878697	0\\
2.05925148128703	0\\
2.05935148378709	0\\
2.05945148628716	0\\
2.05955148878722	0\\
2.05965149128728	0\\
2.05975149378734	0\\
2.05985149628741	0\\
2.05995149878747	0\\
2.06005150128753	0\\
2.06015150378759	0\\
2.06025150628766	0\\
2.06035150878772	0\\
2.06045151128778	0\\
2.06055151378784	0\\
2.06065151628791	0\\
2.06075151878797	0\\
2.06085152128803	0\\
2.06095152378809	0\\
2.06105152628816	0\\
2.06115152878822	0\\
2.06125153128828	0\\
2.06135153378834	0\\
2.06145153628841	0\\
2.06155153878847	0\\
2.06165154128853	0\\
2.06175154378859	0\\
2.06185154628866	0\\
2.06195154878872	0\\
2.06205155128878	0\\
2.06215155378884	0\\
2.06225155628891	0\\
2.06235155878897	0\\
2.06245156128903	0\\
2.06255156378909	0\\
2.06265156628916	0\\
2.06275156878922	0\\
2.06285157128928	0\\
2.06295157378934	0\\
2.06305157628941	0\\
2.06315157878947	0\\
2.06325158128953	0\\
2.06335158378959	0\\
2.06345158628966	0\\
2.06355158878972	0\\
2.06365159128978	0\\
2.06375159378984	0\\
2.06385159628991	0\\
2.06395159878997	0\\
2.06405160129003	0\\
2.06415160379009	0\\
2.06425160629016	0\\
2.06435160879022	0\\
2.06445161129028	0\\
2.06455161379034	0\\
2.06465161629041	0\\
2.06475161879047	0\\
2.06485162129053	0\\
2.06495162379059	0\\
2.06505162629066	0\\
2.06515162879072	0\\
2.06525163129078	0\\
2.06535163379084	0\\
2.06545163629091	0\\
2.06555163879097	0\\
2.06565164129103	0\\
2.06575164379109	0\\
2.06585164629116	0\\
2.06595164879122	0\\
2.06605165129128	0\\
2.06615165379134	0\\
2.06625165629141	0\\
2.06635165879147	0\\
2.06645166129153	0\\
2.06655166379159	0\\
2.06665166629166	0\\
2.06675166879172	0\\
2.06685167129178	0\\
2.06695167379184	0\\
2.06705167629191	0\\
2.06715167879197	0\\
2.06725168129203	0\\
2.06735168379209	0\\
2.06745168629216	0\\
2.06755168879222	0\\
2.06765169129228	0\\
2.06775169379234	0\\
2.06785169629241	0\\
2.06795169879247	0\\
2.06805170129253	0\\
2.06815170379259	0\\
2.06825170629266	0\\
2.06835170879272	0\\
2.06845171129278	0\\
2.06855171379284	0\\
2.06865171629291	0\\
2.06875171879297	0\\
2.06885172129303	0\\
2.06895172379309	0\\
2.06905172629316	0\\
2.06915172879322	0\\
2.06925173129328	0\\
2.06935173379334	0\\
2.06945173629341	0\\
2.06955173879347	0\\
2.06965174129353	0\\
2.06975174379359	0\\
2.06985174629366	0\\
2.06995174879372	0\\
2.07005175129378	0\\
2.07015175379384	0\\
2.07025175629391	0\\
2.07035175879397	0\\
2.07045176129403	0\\
2.07055176379409	0\\
2.07065176629416	0\\
2.07075176879422	0\\
2.07085177129428	0\\
2.07095177379435	0\\
2.07105177629441	0\\
2.07115177879447	0\\
2.07125178129453	0\\
2.07135178379459	0\\
2.07145178629466	0\\
2.07155178879472	0\\
2.07165179129478	0\\
2.07175179379484	0\\
2.07185179629491	0\\
2.07195179879497	0\\
2.07205180129503	0\\
2.07215180379509	0\\
2.07225180629516	0\\
2.07235180879522	0\\
2.07245181129528	0\\
2.07255181379534	0\\
2.07265181629541	0\\
2.07275181879547	0\\
2.07285182129553	0\\
2.0729518237956	0\\
2.07305182629566	0\\
2.07315182879572	0\\
2.07325183129578	0\\
2.07335183379584	0\\
2.07345183629591	0\\
2.07355183879597	0\\
2.07365184129603	0\\
2.07375184379609	0\\
2.07385184629616	0\\
2.07395184879622	0\\
2.07405185129628	0\\
2.07415185379634	0\\
2.07425185629641	0\\
2.07435185879647	0\\
2.07445186129653	0\\
2.07455186379659	0\\
2.07465186629666	0\\
2.07475186879672	0\\
2.07485187129678	0\\
2.07495187379685	0\\
2.07505187629691	0\\
2.07515187879697	0\\
2.07525188129703	0\\
2.0753518837971	0\\
2.07545188629716	0\\
2.07555188879722	0\\
2.07565189129728	0\\
2.07575189379734	0\\
2.07585189629741	0\\
2.07595189879747	0\\
2.07605190129753	0\\
2.07615190379759	0\\
2.07625190629766	0\\
2.07635190879772	0\\
2.07645191129778	0\\
2.07655191379784	0\\
2.07665191629791	0\\
2.07675191879797	0\\
2.07685192129803	0\\
2.0769519237981	0\\
2.07705192629816	0\\
2.07715192879822	0\\
2.07725193129828	0\\
2.07735193379835	0\\
2.07745193629841	0\\
2.07755193879847	0\\
2.07765194129853	0\\
2.07775194379859	0\\
2.07785194629866	0\\
2.07795194879872	0\\
2.07805195129878	0\\
2.07815195379884	0\\
2.07825195629891	0\\
2.07835195879897	0\\
2.07845196129903	0\\
2.07855196379909	0\\
2.07865196629916	0\\
2.07875196879922	0\\
2.07885197129928	0\\
2.07895197379935	0\\
2.07905197629941	0\\
2.07915197879947	0\\
2.07925198129953	0\\
2.0793519837996	0\\
2.07945198629966	0\\
2.07955198879972	0\\
2.07965199129978	0\\
2.07975199379984	0\\
2.07985199629991	0\\
2.07995199879997	0\\
2.08005200130003	0\\
2.08015200380009	0\\
2.08025200630016	0\\
2.08035200880022	0\\
2.08045201130028	0\\
2.08055201380034	0\\
2.08065201630041	0\\
2.08075201880047	0\\
2.08085202130053	0\\
2.0809520238006	0\\
2.08105202630066	0\\
2.08115202880072	0\\
2.08125203130078	0\\
2.08135203380085	0\\
2.08145203630091	0\\
2.08155203880097	0\\
2.08165204130103	0\\
2.0817520438011	0\\
2.08185204630116	0\\
2.08195204880122	0\\
2.08205205130128	0\\
2.08215205380134	0\\
2.08225205630141	0\\
2.08235205880147	0\\
2.08245206130153	0\\
2.08255206380159	0\\
2.08265206630166	0\\
2.08275206880172	0\\
2.08285207130178	0\\
2.08295207380185	0\\
2.08305207630191	0\\
2.08315207880197	0\\
2.08325208130203	0\\
2.0833520838021	0\\
2.08345208630216	0\\
2.08355208880222	0\\
2.08365209130228	0\\
2.08375209380235	0\\
2.08385209630241	0\\
2.08395209880247	0\\
2.08405210130253	0\\
2.08415210380259	0\\
2.08425210630266	0\\
2.08435210880272	0\\
2.08445211130278	0\\
2.08455211380284	0\\
2.08465211630291	0\\
2.08475211880297	0\\
2.08485212130303	0\\
2.0849521238031	0\\
2.08505212630316	0\\
2.08515212880322	0\\
2.08525213130328	0\\
2.08535213380335	0\\
2.08545213630341	0\\
2.08555213880347	0\\
2.08565214130353	0\\
2.0857521438036	0\\
2.08585214630366	0\\
2.08595214880372	0\\
2.08605215130378	0\\
2.08615215380384	0\\
2.08625215630391	0\\
2.08635215880397	0\\
2.08645216130403	0\\
2.08655216380409	0\\
2.08665216630416	0\\
2.08675216880422	0\\
2.08685217130428	0\\
2.08695217380435	0\\
2.08705217630441	0\\
2.08715217880447	0\\
2.08725218130453	0\\
2.0873521838046	0\\
2.08745218630466	0\\
2.08755218880472	0\\
2.08765219130478	0\\
2.08775219380485	0\\
2.08785219630491	0\\
2.08795219880497	0\\
2.08805220130503	0\\
2.0881522038051	0\\
2.08825220630516	0\\
2.08835220880522	0\\
2.08845221130528	0\\
2.08855221380534	0\\
2.08865221630541	0\\
2.08875221880547	0\\
2.08885222130553	0\\
2.0889522238056	0\\
2.08905222630566	0\\
2.08915222880572	0\\
2.08925223130578	0\\
2.08935223380585	0\\
2.08945223630591	0\\
2.08955223880597	0\\
2.08965224130603	0\\
2.0897522438061	0\\
2.08985224630616	0\\
2.08995224880622	0\\
2.09005225130628	0\\
2.09015225380635	0\\
2.09025225630641	0\\
2.09035225880647	0\\
2.09045226130653	0\\
2.09055226380659	0\\
2.09065226630666	0\\
2.09075226880672	0\\
2.09085227130678	0\\
2.09095227380685	0\\
2.09105227630691	0\\
2.09115227880697	0\\
2.09125228130703	0\\
2.0913522838071	0\\
2.09145228630716	0\\
2.09155228880722	0\\
2.09165229130728	0\\
2.09175229380735	0\\
2.09185229630741	0\\
2.09195229880747	0\\
2.09205230130753	0\\
2.0921523038076	0\\
2.09225230630766	0\\
2.09235230880772	0\\
2.09245231130778	0\\
2.09255231380785	0\\
2.09265231630791	0\\
2.09275231880797	0\\
2.09285232130803	0\\
2.0929523238081	0\\
2.09305232630816	0\\
2.09315232880822	0\\
2.09325233130828	0\\
2.09335233380835	0\\
2.09345233630841	0\\
2.09355233880847	0\\
2.09365234130853	0\\
2.0937523438086	0\\
2.09385234630866	0\\
2.09395234880872	0\\
2.09405235130878	0\\
2.09415235380885	0\\
2.09425235630891	0\\
2.09435235880897	0\\
2.09445236130903	0\\
2.0945523638091	0\\
2.09465236630916	0\\
2.09475236880922	0\\
2.09485237130928	0\\
2.09495237380935	0\\
2.09505237630941	0\\
2.09515237880947	0\\
2.09525238130953	0\\
2.0953523838096	0\\
2.09545238630966	0\\
2.09555238880972	0\\
2.09565239130978	0\\
2.09575239380985	0\\
2.09585239630991	0\\
2.09595239880997	0\\
2.09605240131003	0\\
2.0961524038101	0\\
2.09625240631016	0\\
2.09635240881022	0\\
2.09645241131028	0\\
2.09655241381035	0\\
2.09665241631041	0\\
2.09675241881047	0\\
2.09685242131053	0\\
2.0969524238106	0\\
2.09705242631066	0\\
2.09715242881072	0\\
2.09725243131078	0\\
2.09735243381085	0\\
2.09745243631091	0\\
2.09755243881097	0\\
2.09765244131103	0\\
2.0977524438111	0\\
2.09785244631116	0\\
2.09795244881122	0\\
2.09805245131128	0\\
2.09815245381135	0\\
2.09825245631141	0\\
2.09835245881147	0\\
2.09845246131153	0\\
2.0985524638116	0\\
2.09865246631166	0\\
2.09875246881172	0\\
2.09885247131178	0\\
2.09895247381185	0\\
2.09905247631191	0\\
2.09915247881197	0\\
2.09925248131203	0\\
2.0993524838121	0\\
2.09945248631216	0\\
2.09955248881222	0\\
2.09965249131228	0\\
2.09975249381235	0\\
2.09985249631241	0\\
2.09995249881247	0\\
2.10005250131253	0\\
2.1001525038126	0\\
2.10025250631266	0\\
2.10035250881272	0\\
2.10045251131278	0\\
2.10055251381285	0\\
2.10065251631291	0\\
2.10075251881297	0\\
2.10085252131303	0\\
2.1009525238131	0\\
2.10105252631316	0\\
2.10115252881322	0\\
2.10125253131328	0\\
2.10135253381335	0\\
2.10145253631341	0\\
2.10155253881347	0\\
2.10165254131353	0\\
2.1017525438136	0\\
2.10185254631366	0\\
2.10195254881372	0\\
2.10205255131378	0\\
2.10215255381385	0\\
2.10225255631391	0\\
2.10235255881397	0\\
2.10245256131403	0\\
2.1025525638141	0\\
2.10265256631416	0\\
2.10275256881422	0\\
2.10285257131428	0\\
2.10295257381435	0\\
2.10305257631441	0\\
2.10315257881447	0\\
2.10325258131453	0\\
2.1033525838146	0\\
2.10345258631466	0\\
2.10355258881472	0\\
2.10365259131478	0\\
2.10375259381485	0\\
2.10385259631491	0\\
2.10395259881497	0\\
2.10405260131503	0\\
2.1041526038151	0\\
2.10425260631516	0\\
2.10435260881522	0\\
2.10445261131528	0\\
2.10455261381535	0\\
2.10465261631541	0\\
2.10475261881547	0\\
2.10485262131553	0\\
2.1049526238156	0\\
2.10505262631566	0\\
2.10515262881572	0\\
2.10525263131578	0\\
2.10535263381585	0\\
2.10545263631591	0\\
2.10555263881597	0\\
2.10565264131603	0\\
2.1057526438161	0\\
2.10585264631616	0\\
2.10595264881622	0\\
2.10605265131628	0\\
2.10615265381635	0\\
2.10625265631641	0\\
2.10635265881647	0\\
2.10645266131653	0\\
2.1065526638166	0\\
2.10665266631666	0\\
2.10675266881672	0\\
2.10685267131678	0\\
2.10695267381685	0\\
2.10705267631691	0\\
2.10715267881697	0\\
2.10725268131703	0\\
2.1073526838171	0\\
2.10745268631716	0\\
2.10755268881722	0\\
2.10765269131728	0\\
2.10775269381735	0\\
2.10785269631741	0\\
2.10795269881747	0\\
2.10805270131753	0\\
2.1081527038176	0\\
2.10825270631766	0\\
2.10835270881772	0\\
2.10845271131778	0\\
2.10855271381785	0\\
2.10865271631791	0\\
2.10875271881797	0\\
2.10885272131803	0\\
2.1089527238181	0\\
2.10905272631816	0\\
2.10915272881822	0\\
2.10925273131828	0\\
2.10935273381835	0\\
2.10945273631841	0\\
2.10955273881847	0\\
2.10965274131853	0\\
2.1097527438186	0\\
2.10985274631866	0\\
2.10995274881872	0\\
2.11005275131878	0\\
2.11015275381885	0\\
2.11025275631891	0\\
2.11035275881897	0\\
2.11045276131903	0\\
2.1105527638191	0\\
2.11065276631916	0\\
2.11075276881922	0\\
2.11085277131928	0\\
2.11095277381935	0\\
2.11105277631941	0\\
2.11115277881947	0\\
2.11125278131953	0\\
2.1113527838196	0\\
2.11145278631966	0\\
2.11155278881972	0\\
2.11165279131978	0\\
2.11175279381985	0\\
2.11185279631991	0\\
2.11195279881997	0\\
2.11205280132003	0\\
2.1121528038201	0\\
2.11225280632016	0\\
2.11235280882022	0\\
2.11245281132028	0\\
2.11255281382035	0\\
2.11265281632041	0\\
2.11275281882047	0\\
2.11285282132053	0\\
2.1129528238206	0\\
2.11305282632066	0\\
2.11315282882072	0\\
2.11325283132078	0\\
2.11335283382085	0\\
2.11345283632091	0\\
2.11355283882097	0\\
2.11365284132103	0\\
2.1137528438211	0\\
2.11385284632116	0\\
2.11395284882122	0\\
2.11405285132128	0\\
2.11415285382135	0\\
2.11425285632141	0\\
2.11435285882147	0\\
2.11445286132153	0\\
2.1145528638216	0\\
2.11465286632166	0\\
2.11475286882172	0\\
2.11485287132178	0\\
2.11495287382185	0\\
2.11505287632191	0\\
2.11515287882197	0\\
2.11525288132203	0\\
2.1153528838221	0\\
2.11545288632216	0\\
2.11555288882222	0\\
2.11565289132228	0\\
2.11575289382235	0\\
2.11585289632241	0\\
2.11595289882247	0\\
2.11605290132253	0\\
2.1161529038226	0\\
2.11625290632266	0\\
2.11635290882272	0\\
2.11645291132278	0\\
2.11655291382285	0\\
2.11665291632291	0\\
2.11675291882297	0\\
2.11685292132303	0\\
2.1169529238231	0\\
2.11705292632316	0\\
2.11715292882322	0\\
2.11725293132328	0\\
2.11735293382335	0\\
2.11745293632341	0\\
2.11755293882347	0\\
2.11765294132353	0\\
2.1177529438236	0\\
2.11785294632366	0\\
2.11795294882372	0\\
2.11805295132378	0\\
2.11815295382385	0\\
2.11825295632391	0\\
2.11835295882397	0\\
2.11845296132403	0\\
2.1185529638241	0\\
2.11865296632416	0\\
2.11875296882422	0\\
2.11885297132428	0\\
2.11895297382435	0\\
2.11905297632441	0\\
2.11915297882447	0\\
2.11925298132453	0\\
2.1193529838246	0\\
2.11945298632466	0\\
2.11955298882472	0\\
2.11965299132478	0\\
2.11975299382485	0\\
2.11985299632491	0\\
2.11995299882497	0\\
2.12005300132503	0\\
2.1201530038251	0\\
2.12025300632516	0\\
2.12035300882522	0\\
2.12045301132528	0\\
2.12055301382535	0\\
2.12065301632541	0\\
2.12075301882547	0\\
2.12085302132553	0\\
2.1209530238256	0\\
2.12105302632566	0\\
2.12115302882572	0\\
2.12125303132578	0\\
2.12135303382585	0\\
2.12145303632591	0\\
2.12155303882597	0\\
2.12165304132603	0\\
2.1217530438261	0\\
2.12185304632616	0\\
2.12195304882622	0\\
2.12205305132628	0\\
2.12215305382635	0\\
2.12225305632641	0\\
2.12235305882647	0\\
2.12245306132653	0\\
2.1225530638266	0\\
2.12265306632666	0\\
2.12275306882672	0\\
2.12285307132678	0\\
2.12295307382685	0\\
2.12305307632691	0\\
2.12315307882697	0\\
2.12325308132703	0\\
2.1233530838271	0\\
2.12345308632716	0\\
2.12355308882722	0\\
2.12365309132728	0\\
2.12375309382735	0\\
2.12385309632741	0\\
2.12395309882747	0\\
2.12405310132753	0\\
2.1241531038276	0\\
2.12425310632766	0\\
2.12435310882772	0\\
2.12445311132778	0\\
2.12455311382785	0\\
2.12465311632791	0\\
2.12475311882797	0\\
2.12485312132803	0\\
2.1249531238281	0\\
2.12505312632816	0\\
2.12515312882822	0\\
2.12525313132828	0\\
2.12535313382835	0\\
2.12545313632841	0\\
2.12555313882847	0\\
2.12565314132853	0\\
2.1257531438286	0\\
2.12585314632866	0\\
2.12595314882872	0\\
2.12605315132878	0\\
2.12615315382885	0\\
2.12625315632891	0\\
2.12635315882897	0\\
2.12645316132903	0\\
2.1265531638291	0\\
2.12665316632916	0\\
2.12675316882922	0\\
2.12685317132928	0\\
2.12695317382935	0\\
2.12705317632941	0\\
2.12715317882947	0\\
2.12725318132953	0\\
2.1273531838296	0\\
2.12745318632966	0\\
2.12755318882972	0\\
2.12765319132978	0\\
2.12775319382985	0\\
2.12785319632991	0\\
2.12795319882997	0\\
2.12805320133003	0\\
2.1281532038301	0\\
2.12825320633016	0\\
2.12835320883022	0\\
2.12845321133028	0\\
2.12855321383035	0\\
2.12865321633041	0\\
2.12875321883047	0\\
2.12885322133053	0\\
2.1289532238306	0\\
2.12905322633066	0\\
2.12915322883072	0\\
2.12925323133078	0\\
2.12935323383085	0\\
2.12945323633091	0\\
2.12955323883097	0\\
2.12965324133103	0\\
2.1297532438311	0\\
2.12985324633116	0\\
2.12995324883122	0\\
2.13005325133128	0\\
2.13015325383135	0\\
2.13025325633141	0\\
2.13035325883147	0\\
2.13045326133153	0\\
2.1305532638316	0\\
2.13065326633166	0\\
2.13075326883172	0\\
2.13085327133178	0\\
2.13095327383185	0\\
2.13105327633191	0\\
2.13115327883197	0\\
2.13125328133203	0\\
2.1313532838321	0\\
2.13145328633216	0\\
2.13155328883222	0\\
2.13165329133228	0\\
2.13175329383235	0\\
2.13185329633241	0\\
2.13195329883247	0\\
2.13205330133253	0\\
2.1321533038326	0\\
2.13225330633266	0\\
2.13235330883272	0\\
2.13245331133278	0\\
2.13255331383285	0\\
2.13265331633291	0\\
2.13275331883297	0\\
2.13285332133303	0\\
2.1329533238331	0\\
2.13305332633316	0\\
2.13315332883322	0\\
2.13325333133328	0\\
2.13335333383335	0\\
2.13345333633341	0\\
2.13355333883347	0\\
2.13365334133353	0\\
2.1337533438336	0\\
2.13385334633366	0\\
2.13395334883372	0\\
2.13405335133378	0\\
2.13415335383385	0\\
2.13425335633391	0\\
2.13435335883397	0\\
2.13445336133403	0\\
2.1345533638341	0\\
2.13465336633416	0\\
2.13475336883422	0\\
2.13485337133428	0\\
2.13495337383435	0\\
2.13505337633441	0\\
2.13515337883447	0\\
2.13525338133453	0\\
2.1353533838346	0\\
2.13545338633466	0\\
2.13555338883472	0\\
2.13565339133478	0\\
2.13575339383485	0\\
2.13585339633491	0\\
2.13595339883497	0\\
2.13605340133503	0\\
2.1361534038351	0\\
2.13625340633516	0\\
2.13635340883522	0\\
2.13645341133528	0\\
2.13655341383535	0\\
2.13665341633541	0\\
2.13675341883547	0\\
2.13685342133553	0\\
2.1369534238356	0\\
2.13705342633566	0\\
2.13715342883572	0\\
2.13725343133578	0\\
2.13735343383585	0\\
2.13745343633591	0\\
2.13755343883597	0\\
2.13765344133603	0\\
2.1377534438361	0\\
2.13785344633616	0\\
2.13795344883622	0\\
2.13805345133628	0\\
2.13815345383635	0\\
2.13825345633641	0\\
2.13835345883647	0\\
2.13845346133653	0\\
2.1385534638366	0\\
2.13865346633666	0\\
2.13875346883672	0\\
2.13885347133678	0\\
2.13895347383685	0\\
2.13905347633691	0\\
2.13915347883697	0\\
2.13925348133703	0\\
2.1393534838371	0\\
2.13945348633716	0\\
2.13955348883722	0\\
2.13965349133728	0\\
2.13975349383735	0\\
2.13985349633741	0\\
2.13995349883747	0\\
2.14005350133753	0\\
2.1401535038376	0\\
2.14025350633766	0\\
2.14035350883772	0\\
2.14045351133778	0\\
2.14055351383785	0\\
2.14065351633791	0\\
2.14075351883797	0\\
2.14085352133803	0\\
2.1409535238381	0\\
2.14105352633816	0\\
2.14115352883822	0\\
2.14125353133828	0\\
2.14135353383835	0\\
2.14145353633841	0\\
2.14155353883847	0\\
2.14165354133853	0\\
2.1417535438386	0\\
2.14185354633866	0\\
2.14195354883872	0\\
2.14205355133878	0\\
2.14215355383885	0\\
2.14225355633891	0\\
2.14235355883897	0\\
2.14245356133903	0\\
2.1425535638391	0\\
2.14265356633916	0\\
2.14275356883922	0\\
2.14285357133928	0\\
2.14295357383935	0\\
2.14305357633941	0\\
2.14315357883947	0\\
2.14325358133953	0\\
2.1433535838396	0\\
2.14345358633966	0\\
2.14355358883972	0\\
2.14365359133978	0\\
2.14375359383985	0\\
2.14385359633991	0\\
2.14395359883997	0\\
2.14405360134003	0\\
2.1441536038401	0\\
2.14425360634016	0\\
2.14435360884022	0\\
2.14445361134028	0\\
2.14455361384035	0\\
2.14465361634041	0\\
2.14475361884047	0\\
2.14485362134053	0\\
2.1449536238406	0\\
2.14505362634066	0\\
2.14515362884072	0\\
2.14525363134078	0\\
2.14535363384085	0\\
2.14545363634091	0\\
2.14555363884097	0\\
2.14565364134103	0\\
2.1457536438411	0\\
2.14585364634116	0\\
2.14595364884122	0\\
2.14605365134128	0\\
2.14615365384135	0\\
2.14625365634141	0\\
2.14635365884147	0\\
2.14645366134153	0\\
2.1465536638416	0\\
2.14665366634166	0\\
2.14675366884172	0\\
2.14685367134178	0\\
2.14695367384185	0\\
2.14705367634191	0\\
2.14715367884197	0\\
2.14725368134203	0\\
2.1473536838421	0\\
2.14745368634216	0\\
2.14755368884222	0\\
2.14765369134228	0\\
2.14775369384235	0\\
2.14785369634241	0\\
2.14795369884247	0\\
2.14805370134253	0\\
2.1481537038426	0\\
2.14825370634266	0\\
2.14835370884272	0\\
2.14845371134278	0\\
2.14855371384285	0\\
2.14865371634291	0\\
2.14875371884297	0\\
2.14885372134303	0\\
2.1489537238431	0\\
2.14905372634316	0\\
2.14915372884322	0\\
2.14925373134328	0\\
2.14935373384335	0\\
2.14945373634341	0\\
2.14955373884347	0\\
2.14965374134353	0\\
2.1497537438436	0\\
2.14985374634366	0\\
2.14995374884372	0\\
2.15005375134378	0\\
2.15015375384385	0\\
2.15025375634391	0\\
2.15035375884397	0\\
2.15045376134403	0\\
2.1505537638441	0\\
2.15065376634416	0\\
2.15075376884422	0\\
2.15085377134428	0\\
2.15095377384435	0\\
2.15105377634441	0\\
2.15115377884447	0\\
2.15125378134453	0\\
2.1513537838446	0\\
2.15145378634466	0\\
2.15155378884472	0\\
2.15165379134478	0\\
2.15175379384485	0\\
2.15185379634491	0\\
2.15195379884497	0\\
2.15205380134503	0\\
2.1521538038451	0\\
2.15225380634516	0\\
2.15235380884522	0\\
2.15245381134528	0\\
2.15255381384535	0\\
2.15265381634541	0\\
2.15275381884547	0\\
2.15285382134553	0\\
2.1529538238456	0\\
2.15305382634566	0\\
2.15315382884572	0\\
2.15325383134578	0\\
2.15335383384585	0\\
2.15345383634591	0\\
2.15355383884597	0\\
2.15365384134603	0\\
2.1537538438461	0\\
2.15385384634616	0\\
2.15395384884622	0\\
2.15405385134628	0\\
2.15415385384635	0\\
2.15425385634641	0\\
2.15435385884647	0\\
2.15445386134653	0\\
2.1545538638466	0\\
2.15465386634666	0\\
2.15475386884672	0\\
2.15485387134678	0\\
2.15495387384685	0\\
2.15505387634691	0\\
2.15515387884697	0\\
2.15525388134703	0\\
2.1553538838471	0\\
2.15545388634716	0\\
2.15555388884722	0\\
2.15565389134728	0\\
2.15575389384735	0\\
2.15585389634741	0\\
2.15595389884747	0\\
2.15605390134753	0\\
2.1561539038476	0\\
2.15625390634766	0\\
2.15635390884772	0\\
2.15645391134778	0\\
2.15655391384785	0\\
2.15665391634791	0\\
2.15675391884797	0\\
2.15685392134803	0\\
2.1569539238481	0\\
2.15705392634816	0\\
2.15715392884822	0\\
2.15725393134828	0\\
2.15735393384835	0\\
2.15745393634841	0\\
2.15755393884847	0\\
2.15765394134853	0\\
2.1577539438486	0\\
2.15785394634866	0\\
2.15795394884872	0\\
2.15805395134878	0\\
2.15815395384885	0\\
2.15825395634891	0\\
2.15835395884897	0\\
2.15845396134903	0\\
2.1585539638491	0\\
2.15865396634916	0\\
2.15875396884922	0\\
2.15885397134928	0\\
2.15895397384935	0\\
2.15905397634941	0\\
2.15915397884947	0\\
2.15925398134953	0\\
2.1593539838496	0\\
2.15945398634966	0\\
2.15955398884972	0\\
2.15965399134978	0\\
2.15975399384985	0\\
2.15985399634991	0\\
2.15995399884997	0\\
2.16005400135003	0\\
2.1601540038501	0\\
2.16025400635016	0\\
2.16035400885022	0\\
2.16045401135028	0\\
2.16055401385035	0\\
2.16065401635041	0\\
2.16075401885047	0\\
2.16085402135053	0\\
2.1609540238506	0\\
2.16105402635066	0\\
2.16115402885072	0\\
2.16125403135078	0\\
2.16135403385085	0\\
2.16145403635091	0\\
2.16155403885097	0\\
2.16165404135103	0\\
2.1617540438511	0\\
2.16185404635116	0\\
2.16195404885122	0\\
2.16205405135128	0\\
2.16215405385135	0\\
2.16225405635141	0\\
2.16235405885147	0\\
2.16245406135153	0\\
2.1625540638516	0\\
2.16265406635166	0\\
2.16275406885172	0\\
2.16285407135178	0\\
2.16295407385185	0\\
2.16305407635191	0\\
2.16315407885197	0\\
2.16325408135203	0\\
2.1633540838521	0\\
2.16345408635216	0\\
2.16355408885222	0\\
2.16365409135228	0\\
2.16375409385235	0\\
2.16385409635241	0\\
2.16395409885247	0\\
2.16405410135253	0\\
2.1641541038526	0\\
2.16425410635266	0\\
2.16435410885272	0\\
2.16445411135278	0\\
2.16455411385285	0\\
2.16465411635291	0\\
2.16475411885297	0\\
2.16485412135303	0\\
2.1649541238531	0\\
2.16505412635316	0\\
2.16515412885322	0\\
2.16525413135328	0\\
2.16535413385335	0\\
2.16545413635341	0\\
2.16555413885347	0\\
2.16565414135353	0\\
2.1657541438536	0\\
2.16585414635366	0\\
2.16595414885372	0\\
2.16605415135378	0\\
2.16615415385385	0\\
2.16625415635391	0\\
2.16635415885397	0\\
2.16645416135403	0\\
2.1665541638541	0\\
2.16665416635416	0\\
2.16675416885422	0\\
2.16685417135428	0\\
2.16695417385435	0\\
2.16705417635441	0\\
2.16715417885447	0\\
2.16725418135453	0\\
2.1673541838546	0\\
2.16745418635466	0\\
2.16755418885472	0\\
2.16765419135478	0\\
2.16775419385485	0\\
2.16785419635491	0\\
2.16795419885497	0\\
2.16805420135503	0\\
2.1681542038551	0\\
2.16825420635516	0\\
2.16835420885522	0\\
2.16845421135528	0\\
2.16855421385535	0\\
2.16865421635541	0\\
2.16875421885547	0\\
2.16885422135553	0\\
2.1689542238556	0\\
2.16905422635566	0\\
2.16915422885572	0\\
2.16925423135578	0\\
2.16935423385585	0\\
2.16945423635591	0\\
2.16955423885597	0\\
2.16965424135603	0\\
2.1697542438561	0\\
2.16985424635616	0\\
2.16995424885622	0\\
2.17005425135628	0\\
2.17015425385635	0\\
2.17025425635641	0\\
2.17035425885647	0\\
2.17045426135653	0\\
2.1705542638566	0\\
2.17065426635666	0\\
2.17075426885672	0\\
2.17085427135678	0\\
2.17095427385685	0\\
2.17105427635691	0\\
2.17115427885697	0\\
2.17125428135703	0\\
2.1713542838571	0\\
2.17145428635716	0\\
2.17155428885722	0\\
2.17165429135728	0\\
2.17175429385735	0\\
2.17185429635741	0\\
2.17195429885747	0\\
2.17205430135753	0\\
2.1721543038576	0\\
2.17225430635766	0\\
2.17235430885772	0\\
2.17245431135778	0\\
2.17255431385785	0\\
2.17265431635791	0\\
2.17275431885797	0\\
2.17285432135803	0\\
2.1729543238581	0\\
2.17305432635816	0\\
2.17315432885822	0\\
2.17325433135828	0\\
2.17335433385835	0\\
2.17345433635841	0\\
2.17355433885847	0\\
2.17365434135853	0\\
2.1737543438586	0\\
2.17385434635866	0\\
2.17395434885872	0\\
2.17405435135878	0\\
2.17415435385885	0\\
2.17425435635891	0\\
2.17435435885897	0\\
2.17445436135903	0\\
2.1745543638591	0\\
2.17465436635916	0\\
2.17475436885922	0\\
2.17485437135928	0\\
2.17495437385935	0\\
2.17505437635941	0\\
2.17515437885947	0\\
2.17525438135953	0\\
2.1753543838596	0\\
2.17545438635966	0\\
2.17555438885972	0\\
2.17565439135978	0\\
2.17575439385985	0\\
2.17585439635991	0\\
2.17595439885997	0\\
2.17605440136003	0\\
2.1761544038601	0\\
2.17625440636016	0\\
2.17635440886022	0\\
2.17645441136028	0\\
2.17655441386035	0\\
2.17665441636041	0\\
2.17675441886047	0\\
2.17685442136053	0\\
2.1769544238606	0\\
2.17705442636066	0\\
2.17715442886072	0\\
2.17725443136078	0\\
2.17735443386085	0\\
2.17745443636091	0\\
2.17755443886097	0\\
2.17765444136103	0\\
2.1777544438611	0\\
2.17785444636116	0\\
2.17795444886122	0\\
2.17805445136128	0\\
2.17815445386135	0\\
2.17825445636141	0\\
2.17835445886147	0\\
2.17845446136153	0\\
2.1785544638616	0\\
2.17865446636166	0\\
2.17875446886172	0\\
2.17885447136178	0\\
2.17895447386185	0\\
2.17905447636191	0\\
2.17915447886197	0\\
2.17925448136203	0\\
2.1793544838621	0\\
2.17945448636216	0\\
2.17955448886222	0\\
2.17965449136228	0\\
2.17975449386235	0\\
2.17985449636241	0\\
2.17995449886247	0\\
2.18005450136253	0\\
2.1801545038626	0\\
2.18025450636266	0\\
2.18035450886272	0\\
2.18045451136278	0\\
2.18055451386285	0\\
2.18065451636291	0\\
2.18075451886297	0\\
2.18085452136303	0\\
2.1809545238631	0\\
2.18105452636316	0\\
2.18115452886322	0\\
2.18125453136328	0\\
2.18135453386335	0\\
2.18145453636341	0\\
2.18155453886347	0\\
2.18165454136353	0\\
2.1817545438636	0\\
2.18185454636366	0\\
2.18195454886372	0\\
2.18205455136378	0\\
2.18215455386385	0\\
2.18225455636391	0\\
2.18235455886397	0\\
2.18245456136403	0\\
2.1825545638641	0\\
2.18265456636416	0\\
2.18275456886422	0\\
2.18285457136428	0\\
2.18295457386435	0\\
2.18305457636441	0\\
2.18315457886447	0\\
2.18325458136453	0\\
2.1833545838646	0\\
2.18345458636466	0\\
2.18355458886472	0\\
2.18365459136478	0\\
2.18375459386485	0\\
2.18385459636491	0\\
2.18395459886497	0\\
2.18405460136503	0\\
2.1841546038651	0\\
2.18425460636516	0\\
2.18435460886522	0\\
2.18445461136528	0\\
2.18455461386535	0\\
2.18465461636541	0\\
2.18475461886547	0\\
2.18485462136553	0\\
2.1849546238656	0\\
2.18505462636566	0\\
2.18515462886572	0\\
2.18525463136578	0\\
2.18535463386585	0\\
2.18545463636591	0\\
2.18555463886597	0\\
2.18565464136603	0\\
2.1857546438661	0\\
2.18585464636616	0\\
2.18595464886622	0\\
2.18605465136628	0\\
2.18615465386635	0\\
2.18625465636641	0\\
2.18635465886647	0\\
2.18645466136653	0\\
2.1865546638666	0\\
2.18665466636666	0\\
2.18675466886672	0\\
2.18685467136678	0\\
2.18695467386685	0\\
2.18705467636691	0\\
2.18715467886697	0\\
2.18725468136703	0\\
2.1873546838671	0\\
2.18745468636716	0\\
2.18755468886722	0\\
2.18765469136728	0\\
2.18775469386735	0\\
2.18785469636741	0\\
2.18795469886747	0\\
2.18805470136753	0\\
2.1881547038676	0\\
2.18825470636766	0\\
2.18835470886772	0\\
2.18845471136778	0\\
2.18855471386785	0\\
2.18865471636791	0\\
2.18875471886797	0\\
2.18885472136803	0\\
2.1889547238681	0\\
2.18905472636816	0\\
2.18915472886822	0\\
2.18925473136828	0\\
2.18935473386835	0\\
2.18945473636841	0\\
2.18955473886847	0\\
2.18965474136853	0\\
2.1897547438686	0\\
2.18985474636866	0\\
2.18995474886872	0\\
2.19005475136878	0\\
2.19015475386885	0\\
2.19025475636891	0\\
2.19035475886897	0\\
2.19045476136903	0\\
2.1905547638691	0\\
2.19065476636916	0\\
2.19075476886922	0\\
2.19085477136928	0\\
2.19095477386935	0\\
2.19105477636941	0\\
2.19115477886947	0\\
2.19125478136953	0\\
2.1913547838696	0\\
2.19145478636966	0\\
2.19155478886972	0\\
2.19165479136978	0\\
2.19175479386985	0\\
2.19185479636991	0\\
2.19195479886997	0\\
2.19205480137003	0\\
2.1921548038701	0\\
2.19225480637016	0\\
2.19235480887022	0\\
2.19245481137028	0\\
2.19255481387035	0\\
2.19265481637041	0\\
2.19275481887047	0\\
2.19285482137053	0\\
2.1929548238706	0\\
2.19305482637066	0\\
2.19315482887072	0\\
2.19325483137078	0\\
2.19335483387085	0\\
2.19345483637091	0\\
2.19355483887097	0\\
2.19365484137103	0\\
2.1937548438711	0\\
2.19385484637116	0\\
2.19395484887122	0\\
2.19405485137128	0\\
2.19415485387135	0\\
2.19425485637141	0\\
2.19435485887147	0\\
2.19445486137153	0\\
2.1945548638716	0\\
2.19465486637166	0\\
2.19475486887172	0\\
2.19485487137178	0\\
2.19495487387185	0\\
2.19505487637191	0\\
2.19515487887197	0\\
2.19525488137203	0\\
2.1953548838721	0\\
2.19545488637216	0\\
2.19555488887222	0\\
2.19565489137228	0\\
2.19575489387235	0\\
2.19585489637241	0\\
2.19595489887247	0\\
2.19605490137253	0\\
2.1961549038726	0\\
2.19625490637266	0\\
2.19635490887272	0\\
2.19645491137278	0\\
2.19655491387285	0\\
2.19665491637291	0\\
2.19675491887297	0\\
2.19685492137303	0\\
2.1969549238731	0\\
2.19705492637316	0\\
2.19715492887322	0\\
2.19725493137328	0\\
2.19735493387335	0\\
2.19745493637341	0\\
2.19755493887347	0\\
2.19765494137353	0\\
2.1977549438736	0\\
2.19785494637366	0\\
2.19795494887372	0\\
2.19805495137378	0\\
2.19815495387385	0\\
2.19825495637391	0\\
2.19835495887397	0\\
2.19845496137403	0\\
2.1985549638741	0\\
2.19865496637416	0\\
2.19875496887422	0\\
2.19885497137428	0\\
2.19895497387435	0\\
2.19905497637441	0\\
2.19915497887447	0\\
2.19925498137453	0\\
2.1993549838746	0\\
2.19945498637466	0\\
2.19955498887472	0\\
2.19965499137478	0\\
2.19975499387485	0\\
2.19985499637491	0\\
2.19995499887497	0\\
2.20005500137503	0\\
2.2001550038751	0\\
2.20025500637516	0\\
2.20035500887522	0\\
2.20045501137528	0\\
2.20055501387535	0\\
2.20065501637541	0\\
2.20075501887547	0\\
2.20085502137553	0\\
2.2009550238756	0\\
2.20105502637566	0\\
2.20115502887572	0\\
2.20125503137578	0\\
2.20135503387585	0\\
2.20145503637591	0\\
2.20155503887597	0\\
2.20165504137603	0\\
2.2017550438761	0\\
2.20185504637616	0\\
2.20195504887622	0\\
2.20205505137628	0\\
2.20215505387635	0\\
2.20225505637641	0\\
2.20235505887647	0\\
2.20245506137653	0\\
2.2025550638766	0\\
2.20265506637666	0\\
2.20275506887672	0\\
2.20285507137678	0\\
2.20295507387685	0\\
2.20305507637691	0\\
2.20315507887697	0\\
2.20325508137703	0\\
2.2033550838771	0\\
2.20345508637716	0\\
2.20355508887722	0\\
2.20365509137728	0\\
2.20375509387735	0\\
2.20385509637741	0\\
2.20395509887747	0\\
2.20405510137753	0\\
2.2041551038776	0\\
2.20425510637766	0\\
2.20435510887772	0\\
2.20445511137778	0\\
2.20455511387785	0\\
2.20465511637791	0\\
2.20475511887797	0\\
2.20485512137803	0\\
2.2049551238781	0\\
2.20505512637816	0\\
2.20515512887822	0\\
2.20525513137828	0\\
2.20535513387835	0\\
2.20545513637841	0\\
2.20555513887847	0\\
2.20565514137853	0\\
2.2057551438786	0\\
2.20585514637866	0\\
2.20595514887872	0\\
2.20605515137878	0\\
2.20615515387885	0\\
2.20625515637891	0\\
2.20635515887897	0\\
2.20645516137903	0\\
2.2065551638791	0\\
2.20665516637916	0\\
2.20675516887922	0\\
2.20685517137928	0\\
2.20695517387935	0\\
2.20705517637941	0\\
2.20715517887947	0\\
2.20725518137953	0\\
2.2073551838796	0\\
2.20745518637966	0\\
2.20755518887972	0\\
2.20765519137978	0\\
2.20775519387985	0\\
2.20785519637991	0\\
2.20795519887997	0\\
2.20805520138003	0\\
2.2081552038801	0\\
2.20825520638016	0\\
2.20835520888022	0\\
2.20845521138028	0\\
2.20855521388035	0\\
2.20865521638041	0\\
2.20875521888047	0\\
2.20885522138053	0\\
2.2089552238806	0\\
2.20905522638066	0\\
2.20915522888072	0\\
2.20925523138078	0\\
2.20935523388085	0\\
2.20945523638091	0\\
2.20955523888097	0\\
2.20965524138103	0\\
2.2097552438811	0\\
2.20985524638116	0\\
2.20995524888122	0\\
2.21005525138128	0\\
2.21015525388135	0\\
2.21025525638141	0\\
2.21035525888147	0\\
2.21045526138153	0\\
2.2105552638816	0\\
2.21065526638166	0\\
2.21075526888172	0\\
2.21085527138178	0\\
2.21095527388185	0\\
2.21105527638191	0\\
2.21115527888197	0\\
2.21125528138203	0\\
2.2113552838821	0\\
2.21145528638216	0\\
2.21155528888222	0\\
2.21165529138228	0\\
2.21175529388235	0\\
2.21185529638241	0\\
2.21195529888247	0\\
2.21205530138253	0\\
2.2121553038826	0\\
2.21225530638266	0\\
2.21235530888272	0\\
2.21245531138278	0\\
2.21255531388285	0\\
2.21265531638291	0\\
2.21275531888297	0\\
2.21285532138303	0\\
2.2129553238831	0\\
2.21305532638316	0\\
2.21315532888322	0\\
2.21325533138328	0\\
2.21335533388335	0\\
2.21345533638341	0\\
2.21355533888347	0\\
2.21365534138353	0\\
2.2137553438836	0\\
2.21385534638366	0\\
2.21395534888372	0\\
2.21405535138378	0\\
2.21415535388385	0\\
2.21425535638391	0\\
2.21435535888397	0\\
2.21445536138403	0\\
2.2145553638841	0\\
2.21465536638416	0\\
2.21475536888422	0\\
2.21485537138428	0\\
2.21495537388435	0\\
2.21505537638441	0\\
2.21515537888447	0\\
2.21525538138453	0\\
2.2153553838846	0\\
2.21545538638466	0\\
2.21555538888472	0\\
2.21565539138478	0\\
2.21575539388485	0\\
2.21585539638491	0\\
2.21595539888497	0\\
2.21605540138503	0\\
2.2161554038851	0\\
2.21625540638516	0\\
2.21635540888522	0\\
2.21645541138528	0\\
2.21655541388535	0\\
2.21665541638541	0\\
2.21675541888547	0\\
2.21685542138553	0\\
2.2169554238856	0\\
2.21705542638566	0\\
2.21715542888572	0\\
2.21725543138578	0\\
2.21735543388585	0\\
2.21745543638591	0\\
2.21755543888597	0\\
2.21765544138603	0\\
2.2177554438861	0\\
2.21785544638616	0\\
2.21795544888622	0\\
2.21805545138628	0\\
2.21815545388635	0\\
2.21825545638641	0\\
2.21835545888647	0\\
2.21845546138653	0\\
2.2185554638866	0\\
2.21865546638666	0\\
2.21875546888672	0\\
2.21885547138678	0\\
2.21895547388685	0\\
2.21905547638691	0\\
2.21915547888697	0\\
2.21925548138703	0\\
2.2193554838871	0\\
2.21945548638716	0\\
2.21955548888722	0\\
2.21965549138728	0\\
2.21975549388735	0\\
2.21985549638741	0\\
2.21995549888747	0\\
2.22005550138753	0\\
2.2201555038876	0\\
2.22025550638766	0\\
2.22035550888772	0\\
2.22045551138778	0\\
2.22055551388785	0\\
2.22065551638791	0\\
2.22075551888797	0\\
2.22085552138803	0\\
2.2209555238881	0\\
2.22105552638816	0\\
2.22115552888822	0\\
2.22125553138828	0\\
2.22135553388835	0\\
2.22145553638841	0\\
2.22155553888847	0\\
2.22165554138853	0\\
2.2217555438886	0\\
2.22185554638866	0\\
2.22195554888872	0\\
2.22205555138878	0\\
2.22215555388885	0\\
2.22225555638891	0\\
2.22235555888897	0\\
2.22245556138903	0\\
2.2225555638891	0\\
2.22265556638916	0\\
2.22275556888922	0\\
2.22285557138928	0\\
2.22295557388935	0\\
2.22305557638941	0\\
2.22315557888947	0\\
2.22325558138953	0\\
2.2233555838896	0\\
2.22345558638966	0\\
2.22355558888972	0\\
2.22365559138978	0\\
2.22375559388985	0\\
2.22385559638991	0\\
2.22395559888997	0\\
2.22405560139003	0\\
2.2241556038901	0\\
2.22425560639016	0\\
2.22435560889022	0\\
2.22445561139028	0\\
2.22455561389035	0\\
2.22465561639041	0\\
2.22475561889047	0\\
2.22485562139053	0\\
2.2249556238906	0\\
2.22505562639066	0\\
2.22515562889072	0\\
2.22525563139078	0\\
2.22535563389085	0\\
2.22545563639091	0\\
2.22555563889097	0\\
2.22565564139103	0\\
2.2257556438911	0\\
2.22585564639116	0\\
2.22595564889122	0\\
2.22605565139128	0\\
2.22615565389135	0\\
2.22625565639141	0\\
2.22635565889147	0\\
2.22645566139153	0\\
2.2265556638916	0\\
2.22665566639166	0\\
2.22675566889172	0\\
2.22685567139178	0\\
2.22695567389185	0\\
2.22705567639191	0\\
2.22715567889197	0\\
2.22725568139203	0\\
2.2273556838921	0\\
2.22745568639216	0\\
2.22755568889222	0\\
2.22765569139228	0\\
2.22775569389235	0\\
2.22785569639241	0\\
2.22795569889247	0\\
2.22805570139253	0\\
2.2281557038926	0\\
2.22825570639266	0\\
2.22835570889272	0\\
2.22845571139278	0\\
2.22855571389285	0\\
2.22865571639291	0\\
2.22875571889297	0\\
2.22885572139303	0\\
2.2289557238931	0\\
2.22905572639316	0\\
2.22915572889322	0\\
2.22925573139328	0\\
2.22935573389335	0\\
2.22945573639341	0\\
2.22955573889347	0\\
2.22965574139353	0\\
2.2297557438936	0\\
2.22985574639366	0\\
2.22995574889372	0\\
2.23005575139378	0\\
2.23015575389385	0\\
2.23025575639391	0\\
2.23035575889397	0\\
2.23045576139404	0\\
2.2305557638941	0\\
2.23065576639416	0\\
2.23075576889422	0\\
2.23085577139428	0\\
2.23095577389435	0\\
2.23105577639441	0\\
2.23115577889447	0\\
2.23125578139453	0\\
2.2313557838946	0\\
2.23145578639466	0\\
2.23155578889472	0\\
2.23165579139478	0\\
2.23175579389485	0\\
2.23185579639491	0\\
2.23195579889497	0\\
2.23205580139503	0\\
2.2321558038951	0\\
2.23225580639516	0\\
2.23235580889522	0\\
2.23245581139529	0\\
2.23255581389535	0\\
2.23265581639541	0\\
2.23275581889547	0\\
2.23285582139553	0\\
2.2329558238956	0\\
2.23305582639566	0\\
2.23315582889572	0\\
2.23325583139578	0\\
2.23335583389585	0\\
2.23345583639591	0\\
2.23355583889597	0\\
2.23365584139603	0\\
2.2337558438961	0\\
2.23385584639616	0\\
2.23395584889622	0\\
2.23405585139628	0\\
2.23415585389635	0\\
2.23425585639641	0\\
2.23435585889647	0\\
2.23445586139654	0\\
2.2345558638966	0\\
2.23465586639666	0\\
2.23475586889672	0\\
2.23485587139678	0\\
2.23495587389685	0\\
2.23505587639691	0\\
2.23515587889697	0\\
2.23525588139703	0\\
2.2353558838971	0\\
2.23545588639716	0\\
2.23555588889722	0\\
2.23565589139728	0\\
2.23575589389735	0\\
2.23585589639741	0\\
2.23595589889747	0\\
2.23605590139753	0\\
2.2361559038976	0\\
2.23625590639766	0\\
2.23635590889772	0\\
2.23645591139779	0\\
2.23655591389785	0\\
2.23665591639791	0\\
2.23675591889797	0\\
2.23685592139804	0\\
2.2369559238981	0\\
2.23705592639816	0\\
2.23715592889822	0\\
2.23725593139828	0\\
2.23735593389835	0\\
2.23745593639841	0\\
2.23755593889847	0\\
2.23765594139853	0\\
2.2377559438986	0\\
2.23785594639866	0\\
2.23795594889872	0\\
2.23805595139878	0\\
2.23815595389885	0\\
2.23825595639891	0\\
2.23835595889897	0\\
2.23845596139904	0\\
2.2385559638991	0\\
2.23865596639916	0\\
2.23875596889922	0\\
2.23885597139929	0\\
2.23895597389935	0\\
2.23905597639941	0\\
2.23915597889947	0\\
2.23925598139953	0\\
2.2393559838996	0\\
2.23945598639966	0\\
2.23955598889972	0\\
2.23965599139978	0\\
2.23975599389985	0\\
2.23985599639991	0\\
2.23995599889997	0\\
2.24005600140003	0\\
2.2401560039001	0\\
2.24025600640016	0\\
2.24035600890022	0\\
2.24045601140029	0\\
2.24055601390035	0\\
2.24065601640041	0\\
2.24075601890047	0\\
2.24085602140054	0\\
2.2409560239006	0\\
2.24105602640066	0\\
2.24115602890072	0\\
2.24125603140078	0\\
2.24135603390085	0\\
2.24145603640091	0\\
2.24155603890097	0\\
2.24165604140103	0\\
2.2417560439011	0\\
2.24185604640116	0\\
2.24195604890122	0\\
2.24205605140128	0\\
2.24215605390135	0\\
2.24225605640141	0\\
2.24235605890147	0\\
2.24245606140154	0\\
2.2425560639016	0\\
2.24265606640166	0\\
2.24275606890172	0\\
2.24285607140179	0\\
2.24295607390185	0\\
2.24305607640191	0\\
2.24315607890197	0\\
2.24325608140204	0\\
2.2433560839021	0\\
2.24345608640216	0\\
2.24355608890222	0\\
2.24365609140228	0\\
2.24375609390235	0\\
2.24385609640241	0\\
2.24395609890247	0\\
2.24405610140253	0\\
2.2441561039026	0\\
2.24425610640266	0\\
2.24435610890272	0\\
2.24445611140279	0\\
2.24455611390285	0\\
2.24465611640291	0\\
2.24475611890297	0\\
2.24485612140304	0\\
2.2449561239031	0\\
2.24505612640316	0\\
2.24515612890322	0\\
2.24525613140329	0\\
2.24535613390335	0\\
2.24545613640341	0\\
2.24555613890347	0\\
2.24565614140353	0\\
2.2457561439036	0\\
2.24585614640366	0\\
2.24595614890372	0\\
2.24605615140378	0\\
2.24615615390385	0\\
2.24625615640391	0\\
2.24635615890397	0\\
2.24645616140404	0\\
2.2465561639041	0\\
2.24665616640416	0\\
2.24675616890422	0\\
2.24685617140429	0\\
2.24695617390435	0\\
2.24705617640441	0\\
2.24715617890447	0\\
2.24725618140454	0\\
2.2473561839046	0\\
2.24745618640466	0\\
2.24755618890472	0\\
2.24765619140479	0\\
2.24775619390485	0\\
2.24785619640491	0\\
2.24795619890497	0\\
2.24805620140503	0\\
2.2481562039051	0\\
2.24825620640516	0\\
2.24835620890522	0\\
2.24845621140529	0\\
2.24855621390535	0\\
2.24865621640541	0\\
2.24875621890547	0\\
2.24885622140554	0\\
2.2489562239056	0\\
2.24905622640566	0\\
2.24915622890572	0\\
2.24925623140579	0\\
2.24935623390585	0\\
2.24945623640591	0\\
2.24955623890597	0\\
2.24965624140604	0\\
2.2497562439061	0\\
2.24985624640616	0\\
2.24995624890622	0\\
2.25005625140628	0\\
2.25015625390635	0\\
2.25025625640641	0\\
2.25035625890647	0\\
2.25045626140654	0\\
2.2505562639066	0\\
2.25065626640666	0\\
2.25075626890672	0\\
2.25085627140679	0\\
2.25095627390685	0\\
2.25105627640691	0\\
2.25115627890697	0\\
2.25125628140704	0\\
2.2513562839071	0\\
2.25145628640716	0\\
2.25155628890722	0\\
2.25165629140729	0\\
2.25175629390735	0\\
2.25185629640741	0\\
2.25195629890747	0\\
2.25205630140753	0\\
2.2521563039076	0\\
2.25225630640766	0\\
2.25235630890772	0\\
2.25245631140779	0\\
2.25255631390785	0\\
2.25265631640791	0\\
2.25275631890797	0\\
2.25285632140804	0\\
2.2529563239081	0\\
2.25305632640816	0\\
2.25315632890822	0\\
2.25325633140829	0\\
2.25335633390835	0\\
2.25345633640841	0\\
2.25355633890847	0\\
2.25365634140854	0\\
2.2537563439086	0\\
2.25385634640866	0\\
2.25395634890872	0\\
2.25405635140879	0\\
2.25415635390885	0\\
2.25425635640891	0\\
2.25435635890897	0\\
2.25445636140904	0\\
2.2545563639091	0\\
2.25465636640916	0\\
2.25475636890922	0\\
2.25485637140929	0\\
2.25495637390935	0\\
2.25505637640941	0\\
2.25515637890947	0\\
2.25525638140954	0\\
2.2553563839096	0\\
2.25545638640966	0\\
2.25555638890972	0\\
2.25565639140979	0\\
2.25575639390985	0\\
2.25585639640991	0\\
2.25595639890997	0\\
2.25605640141004	0\\
2.2561564039101	0\\
2.25625640641016	0\\
2.25635640891022	0\\
2.25645641141029	0\\
2.25655641391035	0\\
2.25665641641041	0\\
2.25675641891047	0\\
2.25685642141054	0\\
2.2569564239106	0\\
2.25705642641066	0\\
2.25715642891072	0\\
2.25725643141079	0\\
2.25735643391085	0\\
2.25745643641091	0\\
2.25755643891097	0\\
2.25765644141104	0\\
2.2577564439111	0\\
2.25785644641116	0\\
2.25795644891122	0\\
2.25805645141129	0\\
2.25815645391135	0\\
2.25825645641141	0\\
2.25835645891147	0\\
2.25845646141154	0\\
2.2585564639116	0\\
2.25865646641166	0\\
2.25875646891172	0\\
2.25885647141179	0\\
2.25895647391185	0\\
2.25905647641191	0\\
2.25915647891197	0\\
2.25925648141204	0\\
2.2593564839121	0\\
2.25945648641216	0\\
2.25955648891222	0\\
2.25965649141229	0\\
2.25975649391235	0\\
2.25985649641241	0\\
2.25995649891247	0\\
2.26005650141254	0\\
2.2601565039126	0\\
2.26025650641266	0\\
2.26035650891272	0\\
2.26045651141279	0\\
2.26055651391285	0\\
2.26065651641291	0\\
2.26075651891297	0\\
2.26085652141304	0\\
2.2609565239131	0\\
2.26105652641316	0\\
2.26115652891322	0\\
2.26125653141329	0\\
2.26135653391335	0\\
2.26145653641341	0\\
2.26155653891347	0\\
2.26165654141354	0\\
2.2617565439136	0\\
2.26185654641366	0\\
2.26195654891372	0\\
2.26205655141379	0\\
2.26215655391385	0\\
2.26225655641391	0\\
2.26235655891397	0\\
2.26245656141404	0\\
2.2625565639141	0\\
2.26265656641416	0\\
2.26275656891422	0\\
2.26285657141429	0\\
2.26295657391435	0\\
2.26305657641441	0\\
2.26315657891447	0\\
2.26325658141454	0\\
2.2633565839146	0\\
2.26345658641466	0\\
2.26355658891472	0\\
2.26365659141479	0\\
2.26375659391485	0\\
2.26385659641491	0\\
2.26395659891497	0\\
2.26405660141504	0\\
2.2641566039151	0\\
2.26425660641516	0\\
2.26435660891522	0\\
2.26445661141529	0\\
2.26455661391535	0\\
2.26465661641541	0\\
2.26475661891547	0\\
2.26485662141554	0\\
2.2649566239156	0\\
2.26505662641566	0\\
2.26515662891572	0\\
2.26525663141579	0\\
2.26535663391585	0\\
2.26545663641591	0\\
2.26555663891597	0\\
2.26565664141604	0\\
2.2657566439161	0\\
2.26585664641616	0\\
2.26595664891622	0\\
2.26605665141629	0\\
2.26615665391635	0\\
2.26625665641641	0\\
2.26635665891647	0\\
2.26645666141654	0\\
2.2665566639166	0\\
2.26665666641666	0\\
2.26675666891672	0\\
2.26685667141679	0\\
2.26695667391685	0\\
2.26705667641691	0\\
2.26715667891697	0\\
2.26725668141704	0\\
2.2673566839171	0\\
2.26745668641716	0\\
2.26755668891722	0\\
2.26765669141729	0\\
2.26775669391735	0\\
2.26785669641741	0\\
2.26795669891747	0\\
2.26805670141754	0\\
2.2681567039176	0\\
2.26825670641766	0\\
2.26835670891772	0\\
2.26845671141779	0\\
2.26855671391785	0\\
2.26865671641791	0\\
2.26875671891797	0\\
2.26885672141804	0\\
2.2689567239181	0\\
2.26905672641816	0\\
2.26915672891822	0\\
2.26925673141829	0\\
2.26935673391835	0\\
2.26945673641841	0\\
2.26955673891847	0\\
2.26965674141854	0\\
2.2697567439186	0\\
2.26985674641866	0\\
2.26995674891872	0\\
2.27005675141879	0\\
2.27015675391885	0\\
2.27025675641891	0\\
2.27035675891897	0\\
2.27045676141904	0\\
2.2705567639191	0\\
2.27065676641916	0\\
2.27075676891922	0\\
2.27085677141929	0\\
2.27095677391935	0\\
2.27105677641941	0\\
2.27115677891947	0\\
2.27125678141954	0\\
2.2713567839196	0\\
2.27145678641966	0\\
2.27155678891972	0\\
2.27165679141979	0\\
2.27175679391985	0\\
2.27185679641991	0\\
2.27195679891997	0\\
2.27205680142004	0\\
2.2721568039201	0\\
2.27225680642016	0\\
2.27235680892022	0\\
2.27245681142029	0\\
2.27255681392035	0\\
2.27265681642041	0\\
2.27275681892047	0\\
2.27285682142054	0\\
2.2729568239206	0\\
2.27305682642066	0\\
2.27315682892072	0\\
2.27325683142079	0\\
2.27335683392085	0\\
2.27345683642091	0\\
2.27355683892097	0\\
2.27365684142104	0\\
2.2737568439211	0\\
2.27385684642116	0\\
2.27395684892122	0\\
2.27405685142129	0\\
2.27415685392135	0\\
2.27425685642141	0\\
2.27435685892147	0\\
2.27445686142154	0\\
2.2745568639216	0\\
2.27465686642166	0\\
2.27475686892172	0\\
2.27485687142179	0\\
2.27495687392185	0\\
2.27505687642191	0\\
2.27515687892197	0\\
2.27525688142204	0\\
2.2753568839221	0\\
2.27545688642216	0\\
2.27555688892222	0\\
2.27565689142229	0\\
2.27575689392235	0\\
2.27585689642241	0\\
2.27595689892247	0\\
2.27605690142254	0\\
2.2761569039226	0\\
2.27625690642266	0\\
2.27635690892272	0\\
2.27645691142279	0\\
2.27655691392285	0\\
2.27665691642291	0\\
2.27675691892297	0\\
2.27685692142304	0\\
2.2769569239231	0\\
2.27705692642316	0\\
2.27715692892322	0\\
2.27725693142329	0\\
2.27735693392335	0\\
2.27745693642341	0\\
2.27755693892347	0\\
2.27765694142354	0\\
2.2777569439236	0\\
2.27785694642366	0\\
2.27795694892372	0\\
2.27805695142379	0\\
2.27815695392385	0\\
2.27825695642391	0\\
2.27835695892397	0\\
2.27845696142404	0\\
2.2785569639241	0\\
2.27865696642416	0\\
2.27875696892422	0\\
2.27885697142429	0\\
2.27895697392435	0\\
2.27905697642441	0\\
2.27915697892447	0\\
2.27925698142454	0\\
2.2793569839246	0\\
2.27945698642466	0\\
2.27955698892472	0\\
2.27965699142479	0\\
2.27975699392485	0\\
2.27985699642491	0\\
2.27995699892497	0\\
2.28005700142504	0\\
2.2801570039251	0\\
2.28025700642516	0\\
2.28035700892522	0\\
2.28045701142529	0\\
2.28055701392535	0\\
2.28065701642541	0\\
2.28075701892547	0\\
2.28085702142554	0\\
2.2809570239256	0\\
2.28105702642566	0\\
2.28115702892572	0\\
2.28125703142579	0\\
2.28135703392585	0\\
2.28145703642591	0\\
2.28155703892597	0\\
2.28165704142604	0\\
2.2817570439261	0\\
2.28185704642616	0\\
2.28195704892622	0\\
2.28205705142629	0\\
2.28215705392635	0\\
2.28225705642641	0\\
2.28235705892647	0\\
2.28245706142654	0\\
2.2825570639266	0\\
2.28265706642666	0\\
2.28275706892672	0\\
2.28285707142679	0\\
2.28295707392685	0\\
2.28305707642691	0\\
2.28315707892697	0\\
2.28325708142704	0\\
2.2833570839271	0\\
2.28345708642716	0\\
2.28355708892722	0\\
2.28365709142729	0\\
2.28375709392735	0\\
2.28385709642741	0\\
2.28395709892747	0\\
2.28405710142754	0\\
2.2841571039276	0\\
2.28425710642766	0\\
2.28435710892772	0\\
2.28445711142779	0\\
2.28455711392785	0\\
2.28465711642791	0\\
2.28475711892797	0\\
2.28485712142804	0\\
2.2849571239281	0\\
2.28505712642816	0\\
2.28515712892822	0\\
2.28525713142829	0\\
2.28535713392835	0\\
2.28545713642841	0\\
2.28555713892847	0\\
2.28565714142854	0\\
2.2857571439286	0\\
2.28585714642866	0\\
2.28595714892872	0\\
2.28605715142879	0\\
2.28615715392885	0\\
2.28625715642891	0\\
2.28635715892897	0\\
2.28645716142904	0\\
2.2865571639291	0\\
2.28665716642916	0\\
2.28675716892922	0\\
2.28685717142929	0\\
2.28695717392935	0\\
2.28705717642941	0\\
2.28715717892947	0\\
2.28725718142954	0\\
2.2873571839296	0\\
2.28745718642966	0\\
2.28755718892972	0\\
2.28765719142979	0\\
2.28775719392985	0\\
2.28785719642991	0\\
2.28795719892997	0\\
2.28805720143004	0\\
2.2881572039301	0\\
2.28825720643016	0\\
2.28835720893022	0\\
2.28845721143029	0\\
2.28855721393035	0\\
2.28865721643041	0\\
2.28875721893047	0\\
2.28885722143054	0\\
2.2889572239306	0\\
2.28905722643066	0\\
2.28915722893072	0\\
2.28925723143079	0\\
2.28935723393085	0\\
2.28945723643091	0\\
2.28955723893097	0\\
2.28965724143104	0\\
2.2897572439311	0\\
2.28985724643116	0\\
2.28995724893122	0\\
2.29005725143129	0\\
2.29015725393135	0\\
2.29025725643141	0\\
2.29035725893147	0\\
2.29045726143154	0\\
2.2905572639316	0\\
2.29065726643166	0\\
2.29075726893172	0\\
2.29085727143179	0\\
2.29095727393185	0\\
2.29105727643191	0\\
2.29115727893197	0\\
2.29125728143204	0\\
2.2913572839321	0\\
2.29145728643216	0\\
2.29155728893222	0\\
2.29165729143229	0\\
2.29175729393235	0\\
2.29185729643241	0\\
2.29195729893247	0\\
2.29205730143254	0\\
2.2921573039326	0\\
2.29225730643266	0\\
2.29235730893272	0\\
2.29245731143279	0\\
2.29255731393285	0\\
2.29265731643291	0\\
2.29275731893297	0\\
2.29285732143304	0\\
2.2929573239331	0\\
2.29305732643316	0\\
2.29315732893322	0\\
2.29325733143329	0\\
2.29335733393335	0\\
2.29345733643341	0\\
2.29355733893347	0\\
2.29365734143354	0\\
2.2937573439336	0\\
2.29385734643366	0\\
2.29395734893372	0\\
2.29405735143379	0\\
2.29415735393385	0\\
2.29425735643391	0\\
2.29435735893397	0\\
2.29445736143404	0\\
2.2945573639341	0\\
2.29465736643416	0\\
2.29475736893422	0\\
2.29485737143429	0\\
2.29495737393435	0\\
2.29505737643441	0\\
2.29515737893447	0\\
2.29525738143454	0\\
2.2953573839346	0\\
2.29545738643466	0\\
2.29555738893472	0\\
2.29565739143479	0\\
2.29575739393485	0\\
2.29585739643491	0\\
2.29595739893497	0\\
2.29605740143504	0\\
2.2961574039351	0\\
2.29625740643516	0\\
2.29635740893522	0\\
2.29645741143529	0\\
2.29655741393535	0\\
2.29665741643541	0\\
2.29675741893547	0\\
2.29685742143554	0\\
2.2969574239356	0\\
2.29705742643566	0\\
2.29715742893572	0\\
2.29725743143579	0\\
2.29735743393585	0\\
2.29745743643591	0\\
2.29755743893597	0\\
2.29765744143604	0\\
2.2977574439361	0\\
2.29785744643616	0\\
2.29795744893622	0\\
2.29805745143629	0\\
2.29815745393635	0\\
2.29825745643641	0\\
2.29835745893647	0\\
2.29845746143654	0\\
2.2985574639366	0\\
2.29865746643666	0\\
2.29875746893672	0\\
2.29885747143679	0\\
2.29895747393685	0\\
2.29905747643691	0\\
2.29915747893697	0\\
2.29925748143704	0\\
2.2993574839371	0\\
2.29945748643716	0\\
2.29955748893722	0\\
2.29965749143729	0\\
2.29975749393735	0\\
2.29985749643741	0\\
2.29995749893747	0\\
2.30005750143754	0\\
2.3001575039376	0\\
2.30025750643766	0\\
2.30035750893772	0\\
2.30045751143779	0\\
2.30055751393785	0\\
2.30065751643791	0\\
2.30075751893797	0\\
2.30085752143804	0\\
2.3009575239381	0\\
2.30105752643816	0\\
2.30115752893822	0\\
2.30125753143829	0\\
2.30135753393835	0\\
2.30145753643841	0\\
2.30155753893847	0\\
2.30165754143854	0\\
2.3017575439386	0\\
2.30185754643866	0\\
2.30195754893872	0\\
2.30205755143879	0\\
2.30215755393885	0\\
2.30225755643891	0\\
2.30235755893897	0\\
2.30245756143904	0\\
2.3025575639391	0\\
2.30265756643916	0\\
2.30275756893922	0\\
2.30285757143929	0\\
2.30295757393935	0\\
2.30305757643941	0\\
2.30315757893947	0\\
2.30325758143954	0\\
2.3033575839396	0\\
2.30345758643966	0\\
2.30355758893972	0\\
2.30365759143979	0\\
2.30375759393985	0\\
2.30385759643991	0\\
2.30395759893997	0\\
2.30405760144004	0\\
2.3041576039401	0\\
2.30425760644016	0\\
2.30435760894022	0\\
2.30445761144029	0\\
2.30455761394035	0\\
2.30465761644041	0\\
2.30475761894047	0\\
2.30485762144054	0\\
2.3049576239406	0\\
2.30505762644066	0\\
2.30515762894072	0\\
2.30525763144079	0\\
2.30535763394085	0\\
2.30545763644091	0\\
2.30555763894097	0\\
2.30565764144104	0\\
2.3057576439411	0\\
2.30585764644116	0\\
2.30595764894122	0\\
2.30605765144129	0\\
2.30615765394135	0\\
2.30625765644141	0\\
2.30635765894147	0\\
2.30645766144154	0\\
2.3065576639416	0\\
2.30665766644166	0\\
2.30675766894172	0\\
2.30685767144179	0\\
2.30695767394185	0\\
2.30705767644191	0\\
2.30715767894197	0\\
2.30725768144204	0\\
2.3073576839421	0\\
2.30745768644216	0\\
2.30755768894222	0\\
2.30765769144229	0\\
2.30775769394235	0\\
2.30785769644241	0\\
2.30795769894247	0\\
2.30805770144254	0\\
2.3081577039426	0\\
2.30825770644266	0\\
2.30835770894272	0\\
2.30845771144279	0\\
2.30855771394285	0\\
2.30865771644291	0\\
2.30875771894297	0\\
2.30885772144304	0\\
2.3089577239431	0\\
2.30905772644316	0\\
2.30915772894322	0\\
2.30925773144329	0\\
2.30935773394335	0\\
2.30945773644341	0\\
2.30955773894347	0\\
2.30965774144354	0\\
2.3097577439436	0\\
2.30985774644366	0\\
2.30995774894372	0\\
2.31005775144379	0\\
2.31015775394385	0\\
2.31025775644391	0\\
2.31035775894397	0\\
2.31045776144404	0\\
2.3105577639441	0\\
2.31065776644416	0\\
2.31075776894422	0\\
2.31085777144429	0\\
2.31095777394435	0\\
2.31105777644441	0\\
2.31115777894447	0\\
2.31125778144454	0\\
2.3113577839446	0\\
2.31145778644466	0\\
2.31155778894472	0\\
2.31165779144479	0\\
2.31175779394485	0\\
2.31185779644491	0\\
2.31195779894497	0\\
2.31205780144504	0\\
2.3121578039451	0\\
2.31225780644516	0\\
2.31235780894522	0\\
2.31245781144529	0\\
2.31255781394535	0\\
2.31265781644541	0\\
2.31275781894547	0\\
2.31285782144554	0\\
2.3129578239456	0\\
2.31305782644566	0\\
2.31315782894572	0\\
2.31325783144579	0\\
2.31335783394585	0\\
2.31345783644591	0\\
2.31355783894597	0\\
2.31365784144604	0\\
2.3137578439461	0\\
2.31385784644616	0\\
2.31395784894622	0\\
2.31405785144629	0\\
2.31415785394635	0\\
2.31425785644641	0\\
2.31435785894647	0\\
2.31445786144654	0\\
2.3145578639466	0\\
2.31465786644666	0\\
2.31475786894672	0\\
2.31485787144679	0\\
2.31495787394685	0\\
2.31505787644691	0\\
2.31515787894697	0\\
2.31525788144704	0\\
2.3153578839471	0\\
2.31545788644716	0\\
2.31555788894722	0\\
2.31565789144729	0\\
2.31575789394735	0\\
2.31585789644741	0\\
2.31595789894747	0\\
2.31605790144754	0\\
2.3161579039476	0\\
2.31625790644766	0\\
2.31635790894772	0\\
2.31645791144779	0\\
2.31655791394785	0\\
2.31665791644791	0\\
2.31675791894797	0\\
2.31685792144804	0\\
2.3169579239481	0\\
2.31705792644816	0\\
2.31715792894822	0\\
2.31725793144829	0\\
2.31735793394835	0\\
2.31745793644841	0\\
2.31755793894847	0\\
2.31765794144854	0\\
2.3177579439486	0\\
2.31785794644866	0\\
2.31795794894872	0\\
2.31805795144879	0\\
2.31815795394885	0\\
2.31825795644891	0\\
2.31835795894897	0\\
2.31845796144904	0\\
2.3185579639491	0\\
2.31865796644916	0\\
2.31875796894922	0\\
2.31885797144929	0\\
2.31895797394935	0\\
2.31905797644941	0\\
2.31915797894947	0\\
2.31925798144954	0\\
2.3193579839496	0\\
2.31945798644966	0\\
2.31955798894972	0\\
2.31965799144979	0\\
2.31975799394985	0\\
2.31985799644991	0\\
2.31995799894997	0\\
2.32005800145004	0\\
2.3201580039501	0\\
2.32025800645016	0\\
2.32035800895022	0\\
2.32045801145029	0\\
2.32055801395035	0\\
2.32065801645041	0\\
2.32075801895047	0\\
2.32085802145054	0\\
2.3209580239506	0\\
2.32105802645066	0\\
2.32115802895072	0\\
2.32125803145079	0\\
2.32135803395085	0\\
2.32145803645091	0\\
2.32155803895097	0\\
2.32165804145104	0\\
2.3217580439511	0\\
2.32185804645116	0\\
2.32195804895122	0\\
2.32205805145129	0\\
2.32215805395135	0\\
2.32225805645141	0\\
2.32235805895147	0\\
2.32245806145154	0\\
2.3225580639516	0\\
2.32265806645166	0\\
2.32275806895172	0\\
2.32285807145179	0\\
2.32295807395185	0\\
2.32305807645191	0\\
2.32315807895197	0\\
2.32325808145204	0\\
2.3233580839521	0\\
2.32345808645216	0\\
2.32355808895222	0\\
2.32365809145229	0\\
2.32375809395235	0\\
2.32385809645241	0\\
2.32395809895247	0\\
2.32405810145254	0\\
2.3241581039526	0\\
2.32425810645266	0\\
2.32435810895272	0\\
2.32445811145279	0\\
2.32455811395285	0\\
2.32465811645291	0\\
2.32475811895297	0\\
2.32485812145304	0\\
2.3249581239531	0\\
2.32505812645316	0\\
2.32515812895322	0\\
2.32525813145329	0\\
2.32535813395335	0\\
2.32545813645341	0\\
2.32555813895347	0\\
2.32565814145354	0\\
2.3257581439536	0\\
2.32585814645366	0\\
2.32595814895372	0\\
2.32605815145379	0\\
2.32615815395385	0\\
2.32625815645391	0\\
2.32635815895397	0\\
2.32645816145404	0\\
2.3265581639541	0\\
2.32665816645416	0\\
2.32675816895422	0\\
2.32685817145429	0\\
2.32695817395435	0\\
2.32705817645441	0\\
2.32715817895447	0\\
2.32725818145454	0\\
2.3273581839546	0\\
2.32745818645466	0\\
2.32755818895472	0\\
2.32765819145479	0\\
2.32775819395485	0\\
2.32785819645491	0\\
2.32795819895497	0\\
2.32805820145504	0\\
2.3281582039551	0\\
2.32825820645516	0\\
2.32835820895522	0\\
2.32845821145529	0\\
2.32855821395535	0\\
2.32865821645541	0\\
2.32875821895547	0\\
2.32885822145554	0\\
2.3289582239556	0\\
2.32905822645566	0\\
2.32915822895572	0\\
2.32925823145579	0\\
2.32935823395585	0\\
2.32945823645591	0\\
2.32955823895597	0\\
2.32965824145604	0\\
2.3297582439561	0\\
2.32985824645616	0\\
2.32995824895622	0\\
2.33005825145629	0\\
2.33015825395635	0\\
2.33025825645641	0\\
2.33035825895647	0\\
2.33045826145654	0\\
2.3305582639566	0\\
2.33065826645666	0\\
2.33075826895672	0\\
2.33085827145679	0\\
2.33095827395685	0\\
2.33105827645691	0\\
2.33115827895697	0\\
2.33125828145704	0\\
2.3313582839571	0\\
2.33145828645716	0\\
2.33155828895722	0\\
2.33165829145729	0\\
2.33175829395735	0\\
2.33185829645741	0\\
2.33195829895747	0\\
2.33205830145754	0\\
2.3321583039576	0\\
2.33225830645766	0\\
2.33235830895772	0\\
2.33245831145779	0\\
2.33255831395785	0\\
2.33265831645791	0\\
2.33275831895797	0\\
2.33285832145804	0\\
2.3329583239581	0\\
2.33305832645816	0\\
2.33315832895822	0\\
2.33325833145829	0\\
2.33335833395835	0\\
2.33345833645841	0\\
2.33355833895847	0\\
2.33365834145854	0\\
2.3337583439586	0\\
2.33385834645866	0\\
2.33395834895872	0\\
2.33405835145879	0\\
2.33415835395885	0\\
2.33425835645891	0\\
2.33435835895897	0\\
2.33445836145904	0\\
2.3345583639591	0\\
2.33465836645916	0\\
2.33475836895922	0\\
2.33485837145929	0\\
2.33495837395935	0\\
2.33505837645941	0\\
2.33515837895947	0\\
2.33525838145954	0\\
2.3353583839596	0\\
2.33545838645966	0\\
2.33555838895972	0\\
2.33565839145979	0\\
2.33575839395985	0\\
2.33585839645991	0\\
2.33595839895997	0\\
2.33605840146004	0\\
2.3361584039601	0\\
2.33625840646016	0\\
2.33635840896022	0\\
2.33645841146029	0\\
2.33655841396035	0\\
2.33665841646041	0\\
2.33675841896047	0\\
2.33685842146054	0\\
2.3369584239606	0\\
2.33705842646066	0\\
2.33715842896072	0\\
2.33725843146079	0\\
2.33735843396085	0\\
2.33745843646091	0\\
2.33755843896097	0\\
2.33765844146104	0\\
2.3377584439611	0\\
2.33785844646116	0\\
2.33795844896122	0\\
2.33805845146129	0\\
2.33815845396135	0\\
2.33825845646141	0\\
2.33835845896147	0\\
2.33845846146154	0\\
2.3385584639616	0\\
2.33865846646166	0\\
2.33875846896172	0\\
2.33885847146179	0\\
2.33895847396185	0\\
2.33905847646191	0\\
2.33915847896197	0\\
2.33925848146204	0\\
2.3393584839621	0\\
2.33945848646216	0\\
2.33955848896222	0\\
2.33965849146229	0\\
2.33975849396235	0\\
2.33985849646241	0\\
2.33995849896247	0\\
2.34005850146254	0\\
2.3401585039626	0\\
2.34025850646266	0\\
2.34035850896272	0\\
2.34045851146279	0\\
2.34055851396285	0\\
2.34065851646291	0\\
2.34075851896297	0\\
2.34085852146304	0\\
2.3409585239631	0\\
2.34105852646316	0\\
2.34115852896322	0\\
2.34125853146329	0\\
2.34135853396335	0\\
2.34145853646341	0\\
2.34155853896347	0\\
2.34165854146354	0\\
2.3417585439636	0\\
2.34185854646366	0\\
2.34195854896372	0\\
2.34205855146379	0\\
2.34215855396385	0\\
2.34225855646391	0\\
2.34235855896397	0\\
2.34245856146404	0\\
2.3425585639641	0\\
2.34265856646416	0\\
2.34275856896422	0\\
2.34285857146429	0\\
2.34295857396435	0\\
2.34305857646441	0\\
2.34315857896447	0\\
2.34325858146454	0\\
2.3433585839646	0\\
2.34345858646466	0\\
2.34355858896472	0\\
2.34365859146479	0\\
2.34375859396485	0\\
2.34385859646491	0\\
2.34395859896497	0\\
2.34405860146504	0\\
2.3441586039651	0\\
2.34425860646516	0\\
2.34435860896522	0\\
2.34445861146529	0\\
2.34455861396535	0\\
2.34465861646541	0\\
2.34475861896547	0\\
2.34485862146554	0\\
2.3449586239656	0\\
2.34505862646566	0\\
2.34515862896572	0\\
2.34525863146579	0\\
2.34535863396585	0\\
2.34545863646591	0\\
2.34555863896597	0\\
2.34565864146604	0\\
2.3457586439661	0\\
2.34585864646616	0\\
2.34595864896622	0\\
2.34605865146629	0\\
2.34615865396635	0\\
2.34625865646641	0\\
2.34635865896647	0\\
2.34645866146654	0\\
2.3465586639666	0\\
2.34665866646666	0\\
2.34675866896672	0\\
2.34685867146679	0\\
2.34695867396685	0\\
2.34705867646691	0\\
2.34715867896697	0\\
2.34725868146704	0\\
2.3473586839671	0\\
2.34745868646716	0\\
2.34755868896722	0\\
2.34765869146729	0\\
2.34775869396735	0\\
2.34785869646741	0\\
2.34795869896747	0\\
2.34805870146754	0\\
2.3481587039676	0\\
2.34825870646766	0\\
2.34835870896772	0\\
2.34845871146779	0\\
2.34855871396785	0\\
2.34865871646791	0\\
2.34875871896797	0\\
2.34885872146804	0\\
2.3489587239681	0\\
2.34905872646816	0\\
2.34915872896822	0\\
2.34925873146829	0\\
2.34935873396835	0\\
2.34945873646841	0\\
2.34955873896847	0\\
2.34965874146854	0\\
2.3497587439686	0\\
2.34985874646866	0\\
2.34995874896872	0\\
2.35005875146879	0\\
2.35015875396885	0\\
2.35025875646891	0\\
2.35035875896897	0\\
2.35045876146904	0\\
2.3505587639691	0\\
2.35065876646916	0\\
2.35075876896922	0\\
2.35085877146929	0\\
2.35095877396935	0\\
2.35105877646941	0\\
2.35115877896947	0\\
2.35125878146954	0\\
2.3513587839696	0\\
2.35145878646966	0\\
2.35155878896972	0\\
2.35165879146979	0\\
2.35175879396985	0\\
2.35185879646991	0\\
2.35195879896997	0\\
2.35205880147004	0\\
2.3521588039701	0\\
2.35225880647016	0\\
2.35235880897022	0\\
2.35245881147029	0\\
2.35255881397035	0\\
2.35265881647041	0\\
2.35275881897047	0\\
2.35285882147054	0\\
2.3529588239706	0\\
2.35305882647066	0\\
2.35315882897072	0\\
2.35325883147079	0\\
2.35335883397085	0\\
2.35345883647091	0\\
2.35355883897097	0\\
2.35365884147104	0\\
2.3537588439711	0\\
2.35385884647116	0\\
2.35395884897122	0\\
2.35405885147129	0\\
2.35415885397135	0\\
2.35425885647141	0\\
2.35435885897147	0\\
2.35445886147154	0\\
2.3545588639716	0\\
2.35465886647166	0\\
2.35475886897172	0\\
2.35485887147179	0\\
2.35495887397185	0\\
2.35505887647191	0\\
2.35515887897197	0\\
2.35525888147204	0\\
2.3553588839721	0\\
2.35545888647216	0\\
2.35555888897222	0\\
2.35565889147229	0\\
2.35575889397235	0\\
2.35585889647241	0\\
2.35595889897247	0\\
2.35605890147254	0\\
2.3561589039726	0\\
2.35625890647266	0\\
2.35635890897272	0\\
2.35645891147279	0\\
2.35655891397285	0\\
2.35665891647291	0\\
2.35675891897297	0\\
2.35685892147304	0\\
2.3569589239731	0\\
2.35705892647316	0\\
2.35715892897322	0\\
2.35725893147329	0\\
2.35735893397335	0\\
2.35745893647341	0\\
2.35755893897347	0\\
2.35765894147354	0\\
2.3577589439736	0\\
2.35785894647366	0\\
2.35795894897372	0\\
2.35805895147379	0\\
2.35815895397385	0\\
2.35825895647391	0\\
2.35835895897397	0\\
2.35845896147404	0\\
2.3585589639741	0\\
2.35865896647416	0\\
2.35875896897422	0\\
2.35885897147429	0\\
2.35895897397435	0\\
2.35905897647441	0\\
2.35915897897447	0\\
2.35925898147454	0\\
2.3593589839746	0\\
2.35945898647466	0\\
2.35955898897472	0\\
2.35965899147479	0\\
2.35975899397485	0\\
2.35985899647491	0\\
2.35995899897497	0\\
2.36005900147504	0\\
2.3601590039751	0\\
2.36025900647516	0\\
2.36035900897522	0\\
2.36045901147529	0\\
2.36055901397535	0\\
2.36065901647541	0\\
2.36075901897547	0\\
2.36085902147554	0\\
2.3609590239756	0\\
2.36105902647566	0\\
2.36115902897572	0\\
2.36125903147579	0\\
2.36135903397585	0\\
2.36145903647591	0\\
2.36155903897597	0\\
2.36165904147604	0\\
2.3617590439761	0\\
2.36185904647616	0\\
2.36195904897622	0\\
2.36205905147629	0\\
2.36215905397635	0\\
2.36225905647641	0\\
2.36235905897647	0\\
2.36245906147654	0\\
2.3625590639766	0\\
2.36265906647666	0\\
2.36275906897672	0\\
2.36285907147679	0\\
2.36295907397685	0\\
2.36305907647691	0\\
2.36315907897697	0\\
2.36325908147704	0\\
2.3633590839771	0\\
2.36345908647716	0\\
2.36355908897722	0\\
2.36365909147729	0\\
2.36375909397735	0\\
2.36385909647741	0\\
2.36395909897747	0\\
2.36405910147754	0\\
2.3641591039776	0\\
2.36425910647766	0\\
2.36435910897772	0\\
2.36445911147779	0\\
2.36455911397785	0\\
2.36465911647791	0\\
2.36475911897797	0\\
2.36485912147804	0\\
2.3649591239781	0\\
2.36505912647816	0\\
2.36515912897822	0\\
2.36525913147829	0\\
2.36535913397835	0\\
2.36545913647841	0\\
2.36555913897847	0\\
2.36565914147854	0\\
2.3657591439786	0\\
2.36585914647866	0\\
2.36595914897872	0\\
2.36605915147879	0\\
2.36615915397885	0\\
2.36625915647891	0\\
2.36635915897897	0\\
2.36645916147904	0\\
2.3665591639791	0\\
2.36665916647916	0\\
2.36675916897922	0\\
2.36685917147929	0\\
2.36695917397935	0\\
2.36705917647941	0\\
2.36715917897947	0\\
2.36725918147954	0\\
2.3673591839796	0\\
2.36745918647966	0\\
2.36755918897972	0\\
2.36765919147979	0\\
2.36775919397985	0\\
2.36785919647991	0\\
2.36795919897997	0\\
2.36805920148004	0\\
2.3681592039801	0\\
2.36825920648016	0\\
2.36835920898022	0\\
2.36845921148029	0\\
2.36855921398035	0\\
2.36865921648041	0\\
2.36875921898047	0\\
2.36885922148054	0\\
2.3689592239806	0\\
2.36905922648066	0\\
2.36915922898072	0\\
2.36925923148079	0\\
2.36935923398085	0\\
2.36945923648091	0\\
2.36955923898097	0\\
2.36965924148104	0\\
2.3697592439811	0\\
2.36985924648116	0\\
2.36995924898122	0\\
2.37005925148129	0\\
2.37015925398135	0\\
2.37025925648141	0\\
2.37035925898147	0\\
2.37045926148154	0\\
2.3705592639816	0\\
2.37065926648166	0\\
2.37075926898172	0\\
2.37085927148179	0\\
2.37095927398185	0\\
2.37105927648191	0\\
2.37115927898197	0\\
2.37125928148204	0\\
2.3713592839821	0\\
2.37145928648216	0\\
2.37155928898222	0\\
2.37165929148229	0\\
2.37175929398235	0\\
2.37185929648241	0\\
2.37195929898247	0\\
2.37205930148254	0\\
2.3721593039826	0\\
2.37225930648266	0\\
2.37235930898272	0\\
2.37245931148279	0\\
2.37255931398285	0\\
2.37265931648291	0\\
2.37275931898297	0\\
2.37285932148304	0\\
2.3729593239831	0\\
2.37305932648316	0\\
2.37315932898322	0\\
2.37325933148329	0\\
2.37335933398335	0\\
2.37345933648341	0\\
2.37355933898347	0\\
2.37365934148354	0\\
2.3737593439836	0\\
2.37385934648366	0\\
2.37395934898372	0\\
2.37405935148379	0\\
2.37415935398385	0\\
2.37425935648391	0\\
2.37435935898397	0\\
2.37445936148404	0\\
2.3745593639841	0\\
2.37465936648416	0\\
2.37475936898422	0\\
2.37485937148429	0\\
2.37495937398435	0\\
2.37505937648441	0\\
2.37515937898447	0\\
2.37525938148454	0\\
2.3753593839846	0\\
2.37545938648466	0\\
2.37555938898472	0\\
2.37565939148479	0\\
2.37575939398485	0\\
2.37585939648491	0\\
2.37595939898497	0\\
2.37605940148504	0\\
2.3761594039851	0\\
2.37625940648516	0\\
2.37635940898522	0\\
2.37645941148529	0\\
2.37655941398535	0\\
2.37665941648541	0\\
2.37675941898547	0\\
2.37685942148554	0\\
2.3769594239856	0\\
2.37705942648566	0\\
2.37715942898572	0\\
2.37725943148579	0\\
2.37735943398585	0\\
2.37745943648591	0\\
2.37755943898597	0\\
2.37765944148604	0\\
2.3777594439861	0\\
2.37785944648616	0\\
2.37795944898622	0\\
2.37805945148629	0\\
2.37815945398635	0\\
2.37825945648641	0\\
2.37835945898647	0\\
2.37845946148654	0\\
2.3785594639866	0\\
2.37865946648666	0\\
2.37875946898672	0\\
2.37885947148679	0\\
2.37895947398685	0\\
2.37905947648691	0\\
2.37915947898697	0\\
2.37925948148704	0\\
2.3793594839871	0\\
2.37945948648716	0\\
2.37955948898722	0\\
2.37965949148729	0\\
2.37975949398735	0\\
2.37985949648741	0\\
2.37995949898747	0\\
2.38005950148754	0\\
2.3801595039876	0\\
2.38025950648766	0\\
2.38035950898772	0\\
2.38045951148779	0\\
2.38055951398785	0\\
2.38065951648791	0\\
2.38075951898797	0\\
2.38085952148804	0\\
2.3809595239881	0\\
2.38105952648816	0\\
2.38115952898822	0\\
2.38125953148829	0\\
2.38135953398835	0\\
2.38145953648841	0\\
2.38155953898847	0\\
2.38165954148854	0\\
2.3817595439886	0\\
2.38185954648866	0\\
2.38195954898872	0\\
2.38205955148879	0\\
2.38215955398885	0\\
2.38225955648891	0\\
2.38235955898897	0\\
2.38245956148904	0\\
2.3825595639891	0\\
2.38265956648916	0\\
2.38275956898922	0\\
2.38285957148929	0\\
2.38295957398935	0\\
2.38305957648941	0\\
2.38315957898947	0\\
2.38325958148954	0\\
2.3833595839896	0\\
2.38345958648966	0\\
2.38355958898972	0\\
2.38365959148979	0\\
2.38375959398985	0\\
2.38385959648991	0\\
2.38395959898997	0\\
2.38405960149004	0\\
2.3841596039901	0\\
2.38425960649016	0\\
2.38435960899022	0\\
2.38445961149029	0\\
2.38455961399035	0\\
2.38465961649041	0\\
2.38475961899047	0\\
2.38485962149054	0\\
2.3849596239906	0\\
2.38505962649066	0\\
2.38515962899072	0\\
2.38525963149079	0\\
2.38535963399085	0\\
2.38545963649091	0\\
2.38555963899097	0\\
2.38565964149104	0\\
2.3857596439911	0\\
2.38585964649116	0\\
2.38595964899122	0\\
2.38605965149129	0\\
2.38615965399135	0\\
2.38625965649141	0\\
2.38635965899147	0\\
2.38645966149154	0\\
2.3865596639916	0\\
2.38665966649166	0\\
2.38675966899172	0\\
2.38685967149179	0\\
2.38695967399185	0\\
2.38705967649191	0\\
2.38715967899197	0\\
2.38725968149204	0\\
2.3873596839921	0\\
2.38745968649216	0\\
2.38755968899222	0\\
2.38765969149229	0\\
2.38775969399235	0\\
2.38785969649241	0\\
2.38795969899247	0\\
2.38805970149254	0\\
2.3881597039926	0\\
2.38825970649266	0\\
2.38835970899272	0\\
2.38845971149279	0\\
2.38855971399285	0\\
2.38865971649291	0\\
2.38875971899297	0\\
2.38885972149304	0\\
2.3889597239931	0\\
2.38905972649316	0\\
2.38915972899322	0\\
2.38925973149329	0\\
2.38935973399335	0\\
2.38945973649341	0\\
2.38955973899347	0\\
2.38965974149354	0\\
2.3897597439936	0\\
2.38985974649366	0\\
2.38995974899372	0\\
2.39005975149379	0\\
2.39015975399385	0\\
2.39025975649391	0\\
2.39035975899397	0\\
2.39045976149404	0\\
2.3905597639941	0\\
2.39065976649416	0\\
2.39075976899422	0\\
2.39085977149429	0\\
2.39095977399435	0\\
2.39105977649441	0\\
2.39115977899447	0\\
2.39125978149454	0\\
2.3913597839946	0\\
2.39145978649466	0\\
2.39155978899472	0\\
2.39165979149479	0\\
2.39175979399485	0\\
2.39185979649491	0\\
2.39195979899498	0\\
2.39205980149504	0\\
2.3921598039951	0\\
2.39225980649516	0\\
2.39235980899522	0\\
2.39245981149529	0\\
2.39255981399535	0\\
2.39265981649541	0\\
2.39275981899547	0\\
2.39285982149554	0\\
2.3929598239956	0\\
2.39305982649566	0\\
2.39315982899572	0\\
2.39325983149579	0\\
2.39335983399585	0\\
2.39345983649591	0\\
2.39355983899597	0\\
2.39365984149604	0\\
2.3937598439961	0\\
2.39385984649616	0\\
2.39395984899623	0\\
2.39405985149629	0\\
2.39415985399635	0\\
2.39425985649641	0\\
2.39435985899647	0\\
2.39445986149654	0\\
2.3945598639966	0\\
2.39465986649666	0\\
2.39475986899672	0\\
2.39485987149679	0\\
2.39495987399685	0\\
2.39505987649691	0\\
2.39515987899697	0\\
2.39525988149704	0\\
2.3953598839971	0\\
2.39545988649716	0\\
2.39555988899722	0\\
2.39565989149729	0\\
2.39575989399735	0\\
2.39585989649741	0\\
2.39595989899748	0\\
2.39605990149754	0\\
2.3961599039976	0\\
2.39625990649766	0\\
2.39635990899772	0\\
2.39645991149779	0\\
2.39655991399785	0\\
2.39665991649791	0\\
2.39675991899797	0\\
2.39685992149804	0\\
2.3969599239981	0\\
2.39705992649816	0\\
2.39715992899822	0\\
2.39725993149829	0\\
2.39735993399835	0\\
2.39745993649841	0\\
2.39755993899847	0\\
2.39765994149854	0\\
2.3977599439986	0\\
2.39785994649866	0\\
2.39795994899873	0\\
2.39805995149879	0\\
2.39815995399885	0\\
2.39825995649891	0\\
2.39835995899898	0\\
2.39845996149904	0\\
2.3985599639991	0\\
2.39865996649916	0\\
2.39875996899922	0\\
2.39885997149929	0\\
2.39895997399935	0\\
2.39905997649941	0\\
2.39915997899947	0\\
2.39925998149954	0\\
2.3993599839996	0\\
2.39945998649966	0\\
2.39955998899972	0\\
2.39965999149979	0\\
2.39975999399985	0\\
2.39985999649991	0\\
2.39995999899998	0\\
2.40006000150004	0\\
};
\addplot [color=mycolor2,solid,forget plot]
  table[row sep=crcr]{%
2.40006000150004	0\\
2.4001600040001	0\\
2.40026000650016	0\\
2.40036000900023	0\\
2.40046001150029	0\\
2.40056001400035	0\\
2.40066001650041	0\\
2.40076001900047	0\\
2.40086002150054	0\\
2.4009600240006	0\\
2.40106002650066	0\\
2.40116002900072	0\\
2.40126003150079	0\\
2.40136003400085	0\\
2.40146003650091	0\\
2.40156003900097	0\\
2.40166004150104	0\\
2.4017600440011	0\\
2.40186004650116	0\\
2.40196004900123	0\\
2.40206005150129	0\\
2.40216005400135	0\\
2.40226005650141	0\\
2.40236005900148	0\\
2.40246006150154	0\\
2.4025600640016	0\\
2.40266006650166	0\\
2.40276006900173	0\\
2.40286007150179	0\\
2.40296007400185	0\\
2.40306007650191	0\\
2.40316007900197	0\\
2.40326008150204	0\\
2.4033600840021	0\\
2.40346008650216	0\\
2.40356008900222	0\\
2.40366009150229	0\\
2.40376009400235	0\\
2.40386009650241	0\\
2.40396009900248	0\\
2.40406010150254	0\\
2.4041601040026	0\\
2.40426010650266	0\\
2.40436010900273	0\\
2.40446011150279	0\\
2.40456011400285	0\\
2.40466011650291	0\\
2.40476011900298	0\\
2.40486012150304	0\\
2.4049601240031	0\\
2.40506012650316	0\\
2.40516012900322	0\\
2.40526013150329	0\\
2.40536013400335	0\\
2.40546013650341	0\\
2.40556013900347	0\\
2.40566014150354	0\\
2.4057601440036	0\\
2.40586014650366	0\\
2.40596014900373	0\\
2.40606015150379	0\\
2.40616015400385	0\\
2.40626015650391	0\\
2.40636015900398	0\\
2.40646016150404	0\\
2.4065601640041	0\\
2.40666016650416	0\\
2.40676016900423	0\\
2.40686017150429	0\\
2.40696017400435	0\\
2.40706017650441	0\\
2.40716017900447	0\\
2.40726018150454	0\\
2.4073601840046	0\\
2.40746018650466	0\\
2.40756018900472	0\\
2.40766019150479	0\\
2.40776019400485	0\\
2.40786019650491	0\\
2.40796019900498	0\\
2.40806020150504	0\\
2.4081602040051	0\\
2.40826020650516	0\\
2.40836020900523	0\\
2.40846021150529	0\\
2.40856021400535	0\\
2.40866021650541	0\\
2.40876021900548	0\\
2.40886022150554	0\\
2.4089602240056	0\\
2.40906022650566	0\\
2.40916022900573	0\\
2.40926023150579	0\\
2.40936023400585	0\\
2.40946023650591	0\\
2.40956023900597	0\\
2.40966024150604	0\\
2.4097602440061	0\\
2.40986024650616	0\\
2.40996024900623	0\\
2.41006025150629	0\\
2.41016025400635	0\\
2.41026025650641	0\\
2.41036025900648	0\\
2.41046026150654	0\\
2.4105602640066	0\\
2.41066026650666	0\\
2.41076026900673	0\\
2.41086027150679	0\\
2.41096027400685	0\\
2.41106027650691	0\\
2.41116027900698	0\\
2.41126028150704	0\\
2.4113602840071	0\\
2.41146028650716	0\\
2.41156028900722	0\\
2.41166029150729	0\\
2.41176029400735	0\\
2.41186029650741	0\\
2.41196029900748	0\\
2.41206030150754	0\\
2.4121603040076	0\\
2.41226030650766	0\\
2.41236030900773	0\\
2.41246031150779	0\\
2.41256031400785	0\\
2.41266031650791	0\\
2.41276031900798	0\\
2.41286032150804	0\\
2.4129603240081	0\\
2.41306032650816	0\\
2.41316032900823	0\\
2.41326033150829	0\\
2.41336033400835	0\\
2.41346033650841	0\\
2.41356033900847	0\\
2.41366034150854	0\\
2.4137603440086	0\\
2.41386034650866	0\\
2.41396034900873	0\\
2.41406035150879	0\\
2.41416035400885	0\\
2.41426035650891	0\\
2.41436035900898	0\\
2.41446036150904	0\\
2.4145603640091	0\\
2.41466036650916	0\\
2.41476036900923	0\\
2.41486037150929	0\\
2.41496037400935	0\\
2.41506037650941	0\\
2.41516037900948	0\\
2.41526038150954	0\\
2.4153603840096	0\\
2.41546038650966	0\\
2.41556038900973	0\\
2.41566039150979	0\\
2.41576039400985	0\\
2.41586039650991	0\\
2.41596039900998	0\\
2.41606040151004	0\\
2.4161604040101	0\\
2.41626040651016	0\\
2.41636040901023	0\\
2.41646041151029	0\\
2.41656041401035	0\\
2.41666041651041	0\\
2.41676041901048	0\\
2.41686042151054	0\\
2.4169604240106	0\\
2.41706042651066	0\\
2.41716042901073	0\\
2.41726043151079	0\\
2.41736043401085	0\\
2.41746043651091	0\\
2.41756043901098	0\\
2.41766044151104	0\\
2.4177604440111	0\\
2.41786044651116	0\\
2.41796044901123	0\\
2.41806045151129	0\\
2.41816045401135	0\\
2.41826045651141	0\\
2.41836045901148	0\\
2.41846046151154	0\\
2.4185604640116	0\\
2.41866046651166	0\\
2.41876046901173	0\\
2.41886047151179	0\\
2.41896047401185	0\\
2.41906047651191	0\\
2.41916047901198	0\\
2.41926048151204	0\\
2.4193604840121	0\\
2.41946048651216	0\\
2.41956048901223	0\\
2.41966049151229	0\\
2.41976049401235	0\\
2.41986049651241	0\\
2.41996049901248	0\\
2.42006050151254	0\\
2.4201605040126	0\\
2.42026050651266	0\\
2.42036050901273	0\\
2.42046051151279	0\\
2.42056051401285	0\\
2.42066051651291	0\\
2.42076051901298	0\\
2.42086052151304	0\\
2.4209605240131	0\\
2.42106052651316	0\\
2.42116052901323	0\\
2.42126053151329	0\\
2.42136053401335	0\\
2.42146053651341	0\\
2.42156053901348	0\\
2.42166054151354	0\\
2.4217605440136	0\\
2.42186054651366	0\\
2.42196054901373	0\\
2.42206055151379	0\\
2.42216055401385	0\\
2.42226055651391	0\\
2.42236055901398	0\\
2.42246056151404	0\\
2.4225605640141	0\\
2.42266056651416	0\\
2.42276056901423	0\\
2.42286057151429	0\\
2.42296057401435	0\\
2.42306057651441	0\\
2.42316057901448	0\\
2.42326058151454	0\\
2.4233605840146	0\\
2.42346058651466	0\\
2.42356058901473	0\\
2.42366059151479	0\\
2.42376059401485	0\\
2.42386059651491	0\\
2.42396059901498	0\\
2.42406060151504	0\\
2.4241606040151	0\\
2.42426060651516	0\\
2.42436060901523	0\\
2.42446061151529	0\\
2.42456061401535	0\\
2.42466061651541	0\\
2.42476061901548	0\\
2.42486062151554	0\\
2.4249606240156	0\\
2.42506062651566	0\\
2.42516062901573	0\\
2.42526063151579	0\\
2.42536063401585	0\\
2.42546063651591	0\\
2.42556063901598	0\\
2.42566064151604	0\\
2.4257606440161	0\\
2.42586064651616	0\\
2.42596064901623	0\\
2.42606065151629	0\\
2.42616065401635	0\\
2.42626065651641	0\\
2.42636065901648	0\\
2.42646066151654	0\\
2.4265606640166	0\\
2.42666066651666	0\\
2.42676066901673	0\\
2.42686067151679	0\\
2.42696067401685	0\\
2.42706067651691	0\\
2.42716067901698	0\\
2.42726068151704	0\\
2.4273606840171	0\\
2.42746068651716	0\\
2.42756068901723	0\\
2.42766069151729	0\\
2.42776069401735	0\\
2.42786069651741	0\\
2.42796069901748	0\\
2.42806070151754	0\\
2.4281607040176	0\\
2.42826070651766	0\\
2.42836070901773	0\\
2.42846071151779	0\\
2.42856071401785	0\\
2.42866071651791	0\\
2.42876071901798	0\\
2.42886072151804	0\\
2.4289607240181	0\\
2.42906072651816	0\\
2.42916072901823	0\\
2.42926073151829	0\\
2.42936073401835	0\\
2.42946073651841	0\\
2.42956073901848	0\\
2.42966074151854	0\\
2.4297607440186	0\\
2.42986074651866	0\\
2.42996074901873	0\\
2.43006075151879	0\\
2.43016075401885	0\\
2.43026075651891	0\\
2.43036075901898	0\\
2.43046076151904	0\\
2.4305607640191	0\\
2.43066076651916	0\\
2.43076076901923	0\\
2.43086077151929	0\\
2.43096077401935	0\\
2.43106077651941	0\\
2.43116077901948	0\\
2.43126078151954	0\\
2.4313607840196	0\\
2.43146078651966	0\\
2.43156078901973	0\\
2.43166079151979	0\\
2.43176079401985	0\\
2.43186079651991	0\\
2.43196079901998	0\\
2.43206080152004	0\\
2.4321608040201	0\\
2.43226080652016	0\\
2.43236080902023	0\\
2.43246081152029	0\\
2.43256081402035	0\\
2.43266081652041	0\\
2.43276081902048	0\\
2.43286082152054	0\\
2.4329608240206	0\\
2.43306082652066	0\\
2.43316082902073	0\\
2.43326083152079	0\\
2.43336083402085	0\\
2.43346083652091	0\\
2.43356083902098	0\\
2.43366084152104	0\\
2.4337608440211	0\\
2.43386084652116	0\\
2.43396084902123	0\\
2.43406085152129	0\\
2.43416085402135	0\\
2.43426085652141	0\\
2.43436085902148	0\\
2.43446086152154	0\\
2.4345608640216	0\\
2.43466086652166	0\\
2.43476086902173	0\\
2.43486087152179	0\\
2.43496087402185	0\\
2.43506087652191	0\\
2.43516087902198	0\\
2.43526088152204	0\\
2.4353608840221	0\\
2.43546088652216	0\\
2.43556088902223	0\\
2.43566089152229	0\\
2.43576089402235	0\\
2.43586089652241	0\\
2.43596089902248	0\\
2.43606090152254	0\\
2.4361609040226	0\\
2.43626090652266	0\\
2.43636090902273	0\\
2.43646091152279	0\\
2.43656091402285	0\\
2.43666091652291	0\\
2.43676091902298	0\\
2.43686092152304	0\\
2.4369609240231	0\\
2.43706092652316	0\\
2.43716092902323	0\\
2.43726093152329	0\\
2.43736093402335	0\\
2.43746093652341	0\\
2.43756093902348	0\\
2.43766094152354	0\\
2.4377609440236	0\\
2.43786094652366	0\\
2.43796094902373	0\\
2.43806095152379	0\\
2.43816095402385	0\\
2.43826095652391	0\\
2.43836095902398	0\\
2.43846096152404	0\\
2.4385609640241	0\\
2.43866096652416	0\\
2.43876096902423	0\\
2.43886097152429	0\\
2.43896097402435	0\\
2.43906097652441	0\\
2.43916097902448	0\\
2.43926098152454	0\\
2.4393609840246	0\\
2.43946098652466	0\\
2.43956098902473	0\\
2.43966099152479	0\\
2.43976099402485	0\\
2.43986099652491	0\\
2.43996099902498	0\\
2.44006100152504	0\\
2.4401610040251	0\\
2.44026100652516	0\\
2.44036100902523	0\\
2.44046101152529	0\\
2.44056101402535	0\\
2.44066101652541	0\\
2.44076101902548	0\\
2.44086102152554	0\\
2.4409610240256	0\\
2.44106102652566	0\\
2.44116102902573	0\\
2.44126103152579	0\\
2.44136103402585	0\\
2.44146103652591	0\\
2.44156103902598	0\\
2.44166104152604	0\\
2.4417610440261	0\\
2.44186104652616	0\\
2.44196104902623	0\\
2.44206105152629	0\\
2.44216105402635	0\\
2.44226105652641	0\\
2.44236105902648	0\\
2.44246106152654	0\\
2.4425610640266	0\\
2.44266106652666	0\\
2.44276106902673	0\\
2.44286107152679	0\\
2.44296107402685	0\\
2.44306107652691	0\\
2.44316107902698	0\\
2.44326108152704	0\\
2.4433610840271	0\\
2.44346108652716	0\\
2.44356108902723	0\\
2.44366109152729	0\\
2.44376109402735	0\\
2.44386109652741	0\\
2.44396109902748	0\\
2.44406110152754	0\\
2.4441611040276	0\\
2.44426110652766	0\\
2.44436110902773	0\\
2.44446111152779	0\\
2.44456111402785	0\\
2.44466111652791	0\\
2.44476111902798	0\\
2.44486112152804	0\\
2.4449611240281	0\\
2.44506112652816	0\\
2.44516112902823	0\\
2.44526113152829	0\\
2.44536113402835	0\\
2.44546113652841	0\\
2.44556113902848	0\\
2.44566114152854	0\\
2.4457611440286	0\\
2.44586114652866	0\\
2.44596114902873	0\\
2.44606115152879	0\\
2.44616115402885	0\\
2.44626115652891	0\\
2.44636115902898	0\\
2.44646116152904	0\\
2.4465611640291	0\\
2.44666116652916	0\\
2.44676116902923	0\\
2.44686117152929	0\\
2.44696117402935	0\\
2.44706117652941	0\\
2.44716117902948	0\\
2.44726118152954	0\\
2.4473611840296	0\\
2.44746118652966	0\\
2.44756118902973	0\\
2.44766119152979	0\\
2.44776119402985	0\\
2.44786119652991	0\\
2.44796119902998	0\\
2.44806120153004	0\\
2.4481612040301	0\\
2.44826120653016	0\\
2.44836120903023	0\\
2.44846121153029	0\\
2.44856121403035	0\\
2.44866121653041	0\\
2.44876121903048	0\\
2.44886122153054	0\\
2.4489612240306	0\\
2.44906122653066	0\\
2.44916122903073	0\\
2.44926123153079	0\\
2.44936123403085	0\\
2.44946123653091	0\\
2.44956123903098	0\\
2.44966124153104	0\\
2.4497612440311	0\\
2.44986124653116	0\\
2.44996124903123	0\\
2.45006125153129	0\\
2.45016125403135	0\\
2.45026125653141	0\\
2.45036125903148	0\\
2.45046126153154	0\\
2.4505612640316	0\\
2.45066126653166	0\\
2.45076126903173	0\\
2.45086127153179	0\\
2.45096127403185	0\\
2.45106127653191	0\\
2.45116127903198	0\\
2.45126128153204	0\\
2.4513612840321	0\\
2.45146128653216	0\\
2.45156128903223	0\\
2.45166129153229	0\\
2.45176129403235	0\\
2.45186129653241	0\\
2.45196129903248	0\\
2.45206130153254	0\\
2.4521613040326	0\\
2.45226130653266	0\\
2.45236130903273	0\\
2.45246131153279	0\\
2.45256131403285	0\\
2.45266131653291	0\\
2.45276131903298	0\\
2.45286132153304	0\\
2.4529613240331	0\\
2.45306132653316	0\\
2.45316132903323	0\\
2.45326133153329	0\\
2.45336133403335	0\\
2.45346133653341	0\\
2.45356133903348	0\\
2.45366134153354	0\\
2.4537613440336	0\\
2.45386134653366	0\\
2.45396134903373	0\\
2.45406135153379	0\\
2.45416135403385	0\\
2.45426135653391	0\\
2.45436135903398	0\\
2.45446136153404	0\\
2.4545613640341	0\\
2.45466136653416	0\\
2.45476136903423	0\\
2.45486137153429	0\\
2.45496137403435	0\\
2.45506137653441	0\\
2.45516137903448	0\\
2.45526138153454	0\\
2.4553613840346	0\\
2.45546138653466	0\\
2.45556138903473	0\\
2.45566139153479	0\\
2.45576139403485	0\\
2.45586139653491	0\\
2.45596139903498	0\\
2.45606140153504	0\\
2.4561614040351	0\\
2.45626140653516	0\\
2.45636140903523	0\\
2.45646141153529	0\\
2.45656141403535	0\\
2.45666141653541	0\\
2.45676141903548	0\\
2.45686142153554	0\\
2.4569614240356	0\\
2.45706142653566	0\\
2.45716142903573	0\\
2.45726143153579	0\\
2.45736143403585	0\\
2.45746143653591	0\\
2.45756143903598	0\\
2.45766144153604	0\\
2.4577614440361	0\\
2.45786144653616	0\\
2.45796144903623	0\\
2.45806145153629	0\\
2.45816145403635	0\\
2.45826145653641	0\\
2.45836145903648	0\\
2.45846146153654	0\\
2.4585614640366	0\\
2.45866146653666	0\\
2.45876146903673	0\\
2.45886147153679	0\\
2.45896147403685	0\\
2.45906147653691	0\\
2.45916147903698	0\\
2.45926148153704	0\\
2.4593614840371	0\\
2.45946148653716	0\\
2.45956148903723	0\\
2.45966149153729	0\\
2.45976149403735	0\\
2.45986149653741	0\\
2.45996149903748	0\\
2.46006150153754	0\\
2.4601615040376	0\\
2.46026150653766	0\\
2.46036150903773	0\\
2.46046151153779	0\\
2.46056151403785	0\\
2.46066151653791	0\\
2.46076151903798	0\\
2.46086152153804	0\\
2.4609615240381	0\\
2.46106152653816	0\\
2.46116152903823	0\\
2.46126153153829	0\\
2.46136153403835	0\\
2.46146153653841	0\\
2.46156153903848	0\\
2.46166154153854	0\\
2.4617615440386	0\\
2.46186154653866	0\\
2.46196154903873	0\\
2.46206155153879	0\\
2.46216155403885	0\\
2.46226155653891	0\\
2.46236155903898	0\\
2.46246156153904	0\\
2.4625615640391	0\\
2.46266156653916	0\\
2.46276156903923	0\\
2.46286157153929	0\\
2.46296157403935	0\\
2.46306157653941	0\\
2.46316157903948	0\\
2.46326158153954	0\\
2.4633615840396	0\\
2.46346158653966	0\\
2.46356158903973	0\\
2.46366159153979	0\\
2.46376159403985	0\\
2.46386159653991	0\\
2.46396159903998	0\\
2.46406160154004	0\\
2.4641616040401	0\\
2.46426160654016	0\\
2.46436160904023	0\\
2.46446161154029	0\\
2.46456161404035	0\\
2.46466161654041	0\\
2.46476161904048	0\\
2.46486162154054	0\\
2.4649616240406	0\\
2.46506162654066	0\\
2.46516162904073	0\\
2.46526163154079	0\\
2.46536163404085	0\\
2.46546163654091	0\\
2.46556163904098	0\\
2.46566164154104	0\\
2.4657616440411	0\\
2.46586164654116	0\\
2.46596164904123	0\\
2.46606165154129	0\\
2.46616165404135	0\\
2.46626165654141	0\\
2.46636165904148	0\\
2.46646166154154	0\\
2.4665616640416	0\\
2.46666166654166	0\\
2.46676166904173	0\\
2.46686167154179	0\\
2.46696167404185	0\\
2.46706167654191	0\\
2.46716167904198	0\\
2.46726168154204	0\\
2.4673616840421	0\\
2.46746168654216	0\\
2.46756168904223	0\\
2.46766169154229	0\\
2.46776169404235	0\\
2.46786169654241	0\\
2.46796169904248	0\\
2.46806170154254	0\\
2.4681617040426	0\\
2.46826170654266	0\\
2.46836170904273	0\\
2.46846171154279	0\\
2.46856171404285	0\\
2.46866171654291	0\\
2.46876171904298	0\\
2.46886172154304	0\\
2.4689617240431	0\\
2.46906172654316	0\\
2.46916172904323	0\\
2.46926173154329	0\\
2.46936173404335	0\\
2.46946173654341	0\\
2.46956173904348	0\\
2.46966174154354	0\\
2.4697617440436	0\\
2.46986174654366	0\\
2.46996174904373	0\\
2.47006175154379	0\\
2.47016175404385	0\\
2.47026175654391	0\\
2.47036175904398	0\\
2.47046176154404	0\\
2.4705617640441	0\\
2.47066176654416	0\\
2.47076176904423	0\\
2.47086177154429	0\\
2.47096177404435	0\\
2.47106177654441	0\\
2.47116177904448	0\\
2.47126178154454	0\\
2.4713617840446	0\\
2.47146178654466	0\\
2.47156178904473	0\\
2.47166179154479	0\\
2.47176179404485	0\\
2.47186179654491	0\\
2.47196179904498	0\\
2.47206180154504	0\\
2.4721618040451	0\\
2.47226180654516	0\\
2.47236180904523	0\\
2.47246181154529	0\\
2.47256181404535	0\\
2.47266181654541	0\\
2.47276181904548	0\\
2.47286182154554	0\\
2.4729618240456	0\\
2.47306182654566	0\\
2.47316182904573	0\\
2.47326183154579	0\\
2.47336183404585	0\\
2.47346183654591	0\\
2.47356183904598	0\\
2.47366184154604	0\\
2.4737618440461	0\\
2.47386184654616	0\\
2.47396184904623	0\\
2.47406185154629	0\\
2.47416185404635	0\\
2.47426185654641	0\\
2.47436185904648	0\\
2.47446186154654	0\\
2.4745618640466	0\\
2.47466186654666	0\\
2.47476186904673	0\\
2.47486187154679	0\\
2.47496187404685	0\\
2.47506187654691	0\\
2.47516187904698	0\\
2.47526188154704	0\\
2.4753618840471	0\\
2.47546188654716	0\\
2.47556188904723	0\\
2.47566189154729	0\\
2.47576189404735	0\\
2.47586189654741	0\\
2.47596189904748	0\\
2.47606190154754	0\\
2.4761619040476	0\\
2.47626190654766	0\\
2.47636190904773	0\\
2.47646191154779	0\\
2.47656191404785	0\\
2.47666191654791	0\\
2.47676191904798	0\\
2.47686192154804	0\\
2.4769619240481	0\\
2.47706192654816	0\\
2.47716192904823	0\\
2.47726193154829	0\\
2.47736193404835	0\\
2.47746193654841	0\\
2.47756193904848	0\\
2.47766194154854	0\\
2.4777619440486	0\\
2.47786194654866	0\\
2.47796194904873	0\\
2.47806195154879	0\\
2.47816195404885	0\\
2.47826195654891	0\\
2.47836195904898	0\\
2.47846196154904	0\\
2.4785619640491	0\\
2.47866196654916	0\\
2.47876196904923	0\\
2.47886197154929	0\\
2.47896197404935	0\\
2.47906197654941	0\\
2.47916197904948	0\\
2.47926198154954	0\\
2.4793619840496	0\\
2.47946198654966	0\\
2.47956198904973	0\\
2.47966199154979	0\\
2.47976199404985	0\\
2.47986199654991	0\\
2.47996199904998	0\\
2.48006200155004	0\\
2.4801620040501	0\\
2.48026200655016	0\\
2.48036200905023	0\\
2.48046201155029	0\\
2.48056201405035	0\\
2.48066201655041	0\\
2.48076201905048	0\\
2.48086202155054	0\\
2.4809620240506	0\\
2.48106202655066	0\\
2.48116202905073	0\\
2.48126203155079	0\\
2.48136203405085	0\\
2.48146203655091	0\\
2.48156203905098	0\\
2.48166204155104	0\\
2.4817620440511	0\\
2.48186204655116	0\\
2.48196204905123	0\\
2.48206205155129	0\\
2.48216205405135	0\\
2.48226205655141	0\\
2.48236205905148	0\\
2.48246206155154	0\\
2.4825620640516	0\\
2.48266206655166	0\\
2.48276206905173	0\\
2.48286207155179	0\\
2.48296207405185	0\\
2.48306207655191	0\\
2.48316207905198	0\\
2.48326208155204	0\\
2.4833620840521	0\\
2.48346208655216	0\\
2.48356208905223	0\\
2.48366209155229	0\\
2.48376209405235	0\\
2.48386209655241	0\\
2.48396209905248	0\\
2.48406210155254	0\\
2.4841621040526	0\\
2.48426210655266	0\\
2.48436210905273	0\\
2.48446211155279	0\\
2.48456211405285	0\\
2.48466211655291	0\\
2.48476211905298	0\\
2.48486212155304	0\\
2.4849621240531	0\\
2.48506212655316	0\\
2.48516212905323	0\\
2.48526213155329	0\\
2.48536213405335	0\\
2.48546213655341	0\\
2.48556213905348	0\\
2.48566214155354	0\\
2.4857621440536	0\\
2.48586214655366	0\\
2.48596214905373	0\\
2.48606215155379	0\\
2.48616215405385	0\\
2.48626215655391	0\\
2.48636215905398	0\\
2.48646216155404	0\\
2.4865621640541	0\\
2.48666216655416	0\\
2.48676216905423	0\\
2.48686217155429	0\\
2.48696217405435	0\\
2.48706217655441	0\\
2.48716217905448	0\\
2.48726218155454	0\\
2.4873621840546	0\\
2.48746218655466	0\\
2.48756218905473	0\\
2.48766219155479	0\\
2.48776219405485	0\\
2.48786219655491	0\\
2.48796219905498	0\\
2.48806220155504	0\\
2.4881622040551	0\\
2.48826220655516	0\\
2.48836220905523	0\\
2.48846221155529	0\\
2.48856221405535	0\\
2.48866221655541	0\\
2.48876221905548	0\\
2.48886222155554	0\\
2.4889622240556	0\\
2.48906222655566	0\\
2.48916222905573	0\\
2.48926223155579	0\\
2.48936223405585	0\\
2.48946223655591	0\\
2.48956223905598	0\\
2.48966224155604	0\\
2.4897622440561	0\\
2.48986224655616	0\\
2.48996224905623	0\\
2.49006225155629	0\\
2.49016225405635	0\\
2.49026225655641	0\\
2.49036225905648	0\\
2.49046226155654	0\\
2.4905622640566	0\\
2.49066226655666	0\\
2.49076226905673	0\\
2.49086227155679	0\\
2.49096227405685	0\\
2.49106227655691	0\\
2.49116227905698	0\\
2.49126228155704	0\\
2.4913622840571	0\\
2.49146228655716	0\\
2.49156228905723	0\\
2.49166229155729	0\\
2.49176229405735	0\\
2.49186229655741	0\\
2.49196229905748	0\\
2.49206230155754	0\\
2.4921623040576	0\\
2.49226230655766	0\\
2.49236230905773	0\\
2.49246231155779	0\\
2.49256231405785	0\\
2.49266231655791	0\\
2.49276231905798	0\\
2.49286232155804	0\\
2.4929623240581	0\\
2.49306232655816	0\\
2.49316232905823	0\\
2.49326233155829	0\\
2.49336233405835	0\\
2.49346233655841	0\\
2.49356233905848	0\\
2.49366234155854	0\\
2.4937623440586	0\\
2.49386234655866	0\\
2.49396234905873	0\\
2.49406235155879	0\\
2.49416235405885	0\\
2.49426235655891	0\\
2.49436235905898	0\\
2.49446236155904	0\\
2.4945623640591	0\\
2.49466236655916	0\\
2.49476236905923	0\\
2.49486237155929	0\\
2.49496237405935	0\\
2.49506237655941	0\\
2.49516237905948	0\\
2.49526238155954	0\\
2.4953623840596	0\\
2.49546238655966	0\\
2.49556238905973	0\\
2.49566239155979	0\\
2.49576239405985	0\\
2.49586239655991	0\\
2.49596239905998	0\\
2.49606240156004	0\\
2.4961624040601	0\\
2.49626240656016	0\\
2.49636240906023	0\\
2.49646241156029	0\\
2.49656241406035	0\\
2.49666241656041	0\\
2.49676241906048	0\\
2.49686242156054	0\\
2.4969624240606	0\\
2.49706242656066	0\\
2.49716242906073	0\\
2.49726243156079	0\\
2.49736243406085	0\\
2.49746243656091	0\\
2.49756243906098	0\\
2.49766244156104	0\\
2.4977624440611	0\\
2.49786244656116	0\\
2.49796244906123	0\\
2.49806245156129	0\\
2.49816245406135	0\\
2.49826245656141	0\\
2.49836245906148	0\\
2.49846246156154	0\\
2.4985624640616	0\\
2.49866246656166	0\\
2.49876246906173	0\\
2.49886247156179	0\\
2.49896247406185	0\\
2.49906247656191	0\\
2.49916247906198	0\\
2.49926248156204	0\\
2.4993624840621	0\\
2.49946248656216	0\\
2.49956248906223	0\\
2.49966249156229	0\\
2.49976249406235	0\\
2.49986249656241	0\\
2.49996249906248	0\\
2.50006250156254	0\\
2.5001625040626	0\\
2.50026250656266	0\\
2.50036250906273	0\\
2.50046251156279	0\\
2.50056251406285	0\\
2.50066251656291	0\\
2.50076251906298	0\\
2.50086252156304	0\\
2.5009625240631	0\\
2.50106252656316	0\\
2.50116252906323	0\\
2.50126253156329	0\\
2.50136253406335	0\\
2.50146253656341	0\\
2.50156253906348	0\\
2.50166254156354	0\\
2.5017625440636	0\\
2.50186254656366	0\\
2.50196254906373	0\\
2.50206255156379	0\\
2.50216255406385	0\\
2.50226255656391	0\\
2.50236255906398	0\\
2.50246256156404	0\\
2.5025625640641	0\\
2.50266256656416	0\\
2.50276256906423	0\\
2.50286257156429	0\\
2.50296257406435	0\\
2.50306257656441	0\\
2.50316257906448	0\\
2.50326258156454	0\\
2.5033625840646	0\\
2.50346258656466	0\\
2.50356258906473	0\\
2.50366259156479	0\\
2.50376259406485	0\\
2.50386259656491	0\\
2.50396259906498	0\\
2.50406260156504	0\\
2.5041626040651	0\\
2.50426260656516	0\\
2.50436260906523	0\\
2.50446261156529	0\\
2.50456261406535	0\\
2.50466261656541	0\\
2.50476261906548	0\\
2.50486262156554	0\\
2.5049626240656	0\\
2.50506262656566	0\\
2.50516262906573	0\\
2.50526263156579	0\\
2.50536263406585	0\\
2.50546263656591	0\\
2.50556263906598	0\\
2.50566264156604	0\\
2.5057626440661	0\\
2.50586264656616	0\\
2.50596264906623	0\\
2.50606265156629	0\\
2.50616265406635	0\\
2.50626265656641	0\\
2.50636265906648	0\\
2.50646266156654	0\\
2.5065626640666	0\\
2.50666266656666	0\\
2.50676266906673	0\\
2.50686267156679	0\\
2.50696267406685	0\\
2.50706267656691	0\\
2.50716267906698	0\\
2.50726268156704	0\\
2.5073626840671	0\\
2.50746268656716	0\\
2.50756268906723	0\\
2.50766269156729	0\\
2.50776269406735	0\\
2.50786269656741	0\\
2.50796269906748	0\\
2.50806270156754	0\\
2.5081627040676	0\\
2.50826270656766	0\\
2.50836270906773	0\\
2.50846271156779	0\\
2.50856271406785	0\\
2.50866271656791	0\\
2.50876271906798	0\\
2.50886272156804	0\\
2.5089627240681	0\\
2.50906272656816	0\\
2.50916272906823	0\\
2.50926273156829	0\\
2.50936273406835	0\\
2.50946273656841	0\\
2.50956273906848	0\\
2.50966274156854	0\\
2.5097627440686	0\\
2.50986274656866	0\\
2.50996274906873	0\\
2.51006275156879	0\\
2.51016275406885	0\\
2.51026275656891	0\\
2.51036275906898	0\\
2.51046276156904	0\\
2.5105627640691	0\\
2.51066276656916	0\\
2.51076276906923	0\\
2.51086277156929	0\\
2.51096277406935	0\\
2.51106277656941	0\\
2.51116277906948	0\\
2.51126278156954	0\\
2.5113627840696	0\\
2.51146278656966	0\\
2.51156278906973	0\\
2.51166279156979	0\\
2.51176279406985	0\\
2.51186279656991	0\\
2.51196279906998	0\\
2.51206280157004	0\\
2.5121628040701	0\\
2.51226280657016	0\\
2.51236280907023	0\\
2.51246281157029	0\\
2.51256281407035	0\\
2.51266281657041	0\\
2.51276281907048	0\\
2.51286282157054	0\\
2.5129628240706	0\\
2.51306282657066	0\\
2.51316282907073	0\\
2.51326283157079	0\\
2.51336283407085	0\\
2.51346283657091	0\\
2.51356283907098	0\\
2.51366284157104	0\\
2.5137628440711	0\\
2.51386284657116	0\\
2.51396284907123	0\\
2.51406285157129	0\\
2.51416285407135	0\\
2.51426285657141	0\\
2.51436285907148	0\\
2.51446286157154	0\\
2.5145628640716	0\\
2.51466286657166	0\\
2.51476286907173	0\\
2.51486287157179	0\\
2.51496287407185	0\\
2.51506287657191	0\\
2.51516287907198	0\\
2.51526288157204	0\\
2.5153628840721	0\\
2.51546288657216	0\\
2.51556288907223	0\\
2.51566289157229	0\\
2.51576289407235	0\\
2.51586289657241	0\\
2.51596289907248	0\\
2.51606290157254	0\\
2.5161629040726	0\\
2.51626290657266	0\\
2.51636290907273	0\\
2.51646291157279	0\\
2.51656291407285	0\\
2.51666291657291	0\\
2.51676291907298	0\\
2.51686292157304	0\\
2.5169629240731	0\\
2.51706292657316	0\\
2.51716292907323	0\\
2.51726293157329	0\\
2.51736293407335	0\\
2.51746293657341	0\\
2.51756293907348	0\\
2.51766294157354	0\\
2.5177629440736	0\\
2.51786294657366	0\\
2.51796294907373	0\\
2.51806295157379	0\\
2.51816295407385	0\\
2.51826295657391	0\\
2.51836295907398	0\\
2.51846296157404	0\\
2.5185629640741	0\\
2.51866296657416	0\\
2.51876296907423	0\\
2.51886297157429	0\\
2.51896297407435	0\\
2.51906297657441	0\\
2.51916297907448	0\\
2.51926298157454	0\\
2.5193629840746	0\\
2.51946298657466	0\\
2.51956298907473	0\\
2.51966299157479	0\\
2.51976299407485	0\\
2.51986299657491	0\\
2.51996299907498	0\\
2.52006300157504	0\\
2.5201630040751	0\\
2.52026300657516	0\\
2.52036300907523	0\\
2.52046301157529	0\\
2.52056301407535	0\\
2.52066301657541	0\\
2.52076301907548	0\\
2.52086302157554	0\\
2.5209630240756	0\\
2.52106302657566	0\\
2.52116302907573	0\\
2.52126303157579	0\\
2.52136303407585	0\\
2.52146303657591	0\\
2.52156303907598	0\\
2.52166304157604	0\\
2.5217630440761	0\\
2.52186304657616	0\\
2.52196304907623	0\\
2.52206305157629	0\\
2.52216305407635	0\\
2.52226305657641	0\\
2.52236305907648	0\\
2.52246306157654	0\\
2.5225630640766	0\\
2.52266306657666	0\\
2.52276306907673	0\\
2.52286307157679	0\\
2.52296307407685	0\\
2.52306307657691	0\\
2.52316307907698	0\\
2.52326308157704	0\\
2.5233630840771	0\\
2.52346308657716	0\\
2.52356308907723	0\\
2.52366309157729	0\\
2.52376309407735	0\\
2.52386309657741	0\\
2.52396309907748	0\\
2.52406310157754	0\\
2.5241631040776	0\\
2.52426310657766	0\\
2.52436310907773	0\\
2.52446311157779	0\\
2.52456311407785	0\\
2.52466311657791	0\\
2.52476311907798	0\\
2.52486312157804	0\\
2.5249631240781	0\\
2.52506312657816	0\\
2.52516312907823	0\\
2.52526313157829	0\\
2.52536313407835	0\\
2.52546313657841	0\\
2.52556313907848	0\\
2.52566314157854	0\\
2.5257631440786	0\\
2.52586314657866	0\\
2.52596314907873	0\\
2.52606315157879	0\\
2.52616315407885	0\\
2.52626315657891	0\\
2.52636315907898	0\\
2.52646316157904	0\\
2.5265631640791	0\\
2.52666316657916	0\\
2.52676316907923	0\\
2.52686317157929	0\\
2.52696317407935	0\\
2.52706317657941	0\\
2.52716317907948	0\\
2.52726318157954	0\\
2.5273631840796	0\\
2.52746318657966	0\\
2.52756318907973	0\\
2.52766319157979	0\\
2.52776319407985	0\\
2.52786319657991	0\\
2.52796319907998	0\\
2.52806320158004	0\\
2.5281632040801	0\\
2.52826320658016	0\\
2.52836320908023	0\\
2.52846321158029	0\\
2.52856321408035	0\\
2.52866321658041	0\\
2.52876321908048	0\\
2.52886322158054	0\\
2.5289632240806	0\\
2.52906322658066	0\\
2.52916322908073	0\\
2.52926323158079	0\\
2.52936323408085	0\\
2.52946323658091	0\\
2.52956323908098	0\\
2.52966324158104	0\\
2.5297632440811	0\\
2.52986324658116	0\\
2.52996324908123	0\\
2.53006325158129	0\\
2.53016325408135	0\\
2.53026325658141	0\\
2.53036325908148	0\\
2.53046326158154	0\\
2.5305632640816	0\\
2.53066326658166	0\\
2.53076326908173	0\\
2.53086327158179	0\\
2.53096327408185	0\\
2.53106327658191	0\\
2.53116327908198	0\\
2.53126328158204	0\\
2.5313632840821	0\\
2.53146328658216	0\\
2.53156328908223	0\\
2.53166329158229	0\\
2.53176329408235	0\\
2.53186329658241	0\\
2.53196329908248	0\\
2.53206330158254	0\\
2.5321633040826	0\\
2.53226330658266	0\\
2.53236330908273	0\\
2.53246331158279	0\\
2.53256331408285	0\\
2.53266331658291	0\\
2.53276331908298	0\\
2.53286332158304	0\\
2.5329633240831	0\\
2.53306332658316	0\\
2.53316332908323	0\\
2.53326333158329	0\\
2.53336333408335	0\\
2.53346333658341	0\\
2.53356333908348	0\\
2.53366334158354	0\\
2.5337633440836	0\\
2.53386334658366	0\\
2.53396334908373	0\\
2.53406335158379	0\\
2.53416335408385	0\\
2.53426335658391	0\\
2.53436335908398	0\\
2.53446336158404	0\\
2.5345633640841	0\\
2.53466336658416	0\\
2.53476336908423	0\\
2.53486337158429	0\\
2.53496337408435	0\\
2.53506337658441	0\\
2.53516337908448	0\\
2.53526338158454	0\\
2.5353633840846	0\\
2.53546338658466	0\\
2.53556338908473	0\\
2.53566339158479	0\\
2.53576339408485	0\\
2.53586339658491	0\\
2.53596339908498	0\\
2.53606340158504	0\\
2.5361634040851	0\\
2.53626340658516	0\\
2.53636340908523	0\\
2.53646341158529	0\\
2.53656341408535	0\\
2.53666341658541	0\\
2.53676341908548	0\\
2.53686342158554	0\\
2.5369634240856	0\\
2.53706342658566	0\\
2.53716342908573	0\\
2.53726343158579	0\\
2.53736343408585	0\\
2.53746343658591	0\\
2.53756343908598	0\\
2.53766344158604	0\\
2.5377634440861	0\\
2.53786344658616	0\\
2.53796344908623	0\\
2.53806345158629	0\\
2.53816345408635	0\\
2.53826345658641	0\\
2.53836345908648	0\\
2.53846346158654	0\\
2.5385634640866	0\\
2.53866346658666	0\\
2.53876346908673	0\\
2.53886347158679	0\\
2.53896347408685	0\\
2.53906347658691	0\\
2.53916347908698	0\\
2.53926348158704	0\\
2.5393634840871	0\\
2.53946348658716	0\\
2.53956348908723	0\\
2.53966349158729	0\\
2.53976349408735	0\\
2.53986349658741	0\\
2.53996349908748	0\\
2.54006350158754	0\\
2.5401635040876	0\\
2.54026350658766	0\\
2.54036350908773	0\\
2.54046351158779	0\\
2.54056351408785	0\\
2.54066351658791	0\\
2.54076351908798	0\\
2.54086352158804	0\\
2.5409635240881	0\\
2.54106352658816	0\\
2.54116352908823	0\\
2.54126353158829	0\\
2.54136353408835	0\\
2.54146353658841	0\\
2.54156353908848	0\\
2.54166354158854	0\\
2.5417635440886	0\\
2.54186354658866	0\\
2.54196354908873	0\\
2.54206355158879	0\\
2.54216355408885	0\\
2.54226355658891	0\\
2.54236355908898	0\\
2.54246356158904	0\\
2.5425635640891	0\\
2.54266356658916	0\\
2.54276356908923	0\\
2.54286357158929	0\\
2.54296357408935	0\\
2.54306357658941	0\\
2.54316357908948	0\\
2.54326358158954	0\\
2.5433635840896	0\\
2.54346358658966	0\\
2.54356358908973	0\\
2.54366359158979	0\\
2.54376359408985	0\\
2.54386359658991	0\\
2.54396359908998	0\\
2.54406360159004	0\\
2.5441636040901	0\\
2.54426360659016	0\\
2.54436360909023	0\\
2.54446361159029	0\\
2.54456361409035	0\\
2.54466361659041	0\\
2.54476361909048	0\\
2.54486362159054	0\\
2.5449636240906	0\\
2.54506362659066	0\\
2.54516362909073	0\\
2.54526363159079	0\\
2.54536363409085	0\\
2.54546363659091	0\\
2.54556363909098	0\\
2.54566364159104	0\\
2.5457636440911	0\\
2.54586364659116	0\\
2.54596364909123	0\\
2.54606365159129	0\\
2.54616365409135	0\\
2.54626365659141	0\\
2.54636365909148	0\\
2.54646366159154	0\\
2.5465636640916	0\\
2.54666366659166	0\\
2.54676366909173	0\\
2.54686367159179	0\\
2.54696367409185	0\\
2.54706367659192	0\\
2.54716367909198	0\\
2.54726368159204	0\\
2.5473636840921	0\\
2.54746368659216	0\\
2.54756368909223	0\\
2.54766369159229	0\\
2.54776369409235	0\\
2.54786369659241	0\\
2.54796369909248	0\\
2.54806370159254	0\\
2.5481637040926	0\\
2.54826370659266	0\\
2.54836370909273	0\\
2.54846371159279	0\\
2.54856371409285	0\\
2.54866371659291	0\\
2.54876371909298	0\\
2.54886372159304	0\\
2.5489637240931	0\\
2.54906372659317	0\\
2.54916372909323	0\\
2.54926373159329	0\\
2.54936373409335	0\\
2.54946373659341	0\\
2.54956373909348	0\\
2.54966374159354	0\\
2.5497637440936	0\\
2.54986374659366	0\\
2.54996374909373	0\\
2.55006375159379	0\\
2.55016375409385	0\\
2.55026375659391	0\\
2.55036375909398	0\\
2.55046376159404	0\\
2.5505637640941	0\\
2.55066376659416	0\\
2.55076376909423	0\\
2.55086377159429	0\\
2.55096377409435	0\\
2.55106377659442	0\\
2.55116377909448	0\\
2.55126378159454	0\\
2.5513637840946	0\\
2.55146378659466	0\\
2.55156378909473	0\\
2.55166379159479	0\\
2.55176379409485	0\\
2.55186379659491	0\\
2.55196379909498	0\\
2.55206380159504	0\\
2.5521638040951	0\\
2.55226380659516	0\\
2.55236380909523	0\\
2.55246381159529	0\\
2.55256381409535	0\\
2.55266381659541	0\\
2.55276381909548	0\\
2.55286382159554	0\\
2.5529638240956	0\\
2.55306382659567	0\\
2.55316382909573	0\\
2.55326383159579	0\\
2.55336383409585	0\\
2.55346383659592	0\\
2.55356383909598	0\\
2.55366384159604	0\\
2.5537638440961	0\\
2.55386384659616	0\\
2.55396384909623	0\\
2.55406385159629	0\\
2.55416385409635	0\\
2.55426385659641	0\\
2.55436385909648	0\\
2.55446386159654	0\\
2.5545638640966	0\\
2.55466386659666	0\\
2.55476386909673	0\\
2.55486387159679	0\\
2.55496387409685	0\\
2.55506387659692	0\\
2.55516387909698	0\\
2.55526388159704	0\\
2.5553638840971	0\\
2.55546388659717	0\\
2.55556388909723	0\\
2.55566389159729	0\\
2.55576389409735	0\\
2.55586389659741	0\\
2.55596389909748	0\\
2.55606390159754	0\\
2.5561639040976	0\\
2.55626390659766	0\\
2.55636390909773	0\\
2.55646391159779	0\\
2.55656391409785	0\\
2.55666391659791	0\\
2.55676391909798	0\\
2.55686392159804	0\\
2.5569639240981	0\\
2.55706392659817	0\\
2.55716392909823	0\\
2.55726393159829	0\\
2.55736393409835	0\\
2.55746393659842	0\\
2.55756393909848	0\\
2.55766394159854	0\\
2.5577639440986	0\\
2.55786394659867	0\\
2.55796394909873	0\\
2.55806395159879	0\\
2.55816395409885	0\\
2.55826395659891	0\\
2.55836395909898	0\\
2.55846396159904	0\\
2.5585639640991	0\\
2.55866396659916	0\\
2.55876396909923	0\\
2.55886397159929	0\\
2.55896397409935	0\\
2.55906397659942	0\\
2.55916397909948	0\\
2.55926398159954	0\\
2.5593639840996	0\\
2.55946398659967	0\\
2.55956398909973	0\\
2.55966399159979	0\\
2.55976399409985	0\\
2.55986399659992	0\\
2.55996399909998	0\\
2.56006400160004	0\\
2.5601640041001	0\\
2.56026400660016	0\\
2.56036400910023	0\\
2.56046401160029	0\\
2.56056401410035	0\\
2.56066401660041	0\\
2.56076401910048	0\\
2.56086402160054	0\\
2.5609640241006	0\\
2.56106402660067	0\\
2.56116402910073	0\\
2.56126403160079	0\\
2.56136403410085	0\\
2.56146403660092	0\\
2.56156403910098	0\\
2.56166404160104	0\\
2.5617640441011	0\\
2.56186404660117	0\\
2.56196404910123	0\\
2.56206405160129	0\\
2.56216405410135	0\\
2.56226405660141	0\\
2.56236405910148	0\\
2.56246406160154	0\\
2.5625640641016	0\\
2.56266406660166	0\\
2.56276406910173	0\\
2.56286407160179	0\\
2.56296407410185	0\\
2.56306407660192	0\\
2.56316407910198	0\\
2.56326408160204	0\\
2.5633640841021	0\\
2.56346408660217	0\\
2.56356408910223	0\\
2.56366409160229	0\\
2.56376409410235	0\\
2.56386409660242	0\\
2.56396409910248	0\\
2.56406410160254	0\\
2.5641641041026	0\\
2.56426410660267	0\\
2.56436410910273	0\\
2.56446411160279	0\\
2.56456411410285	0\\
2.56466411660291	0\\
2.56476411910298	0\\
2.56486412160304	0\\
2.5649641241031	0\\
2.56506412660317	0\\
2.56516412910323	0\\
2.56526413160329	0\\
2.56536413410335	0\\
2.56546413660342	0\\
2.56556413910348	0\\
2.56566414160354	0\\
2.5657641441036	0\\
2.56586414660367	0\\
2.56596414910373	0\\
2.56606415160379	0\\
2.56616415410385	0\\
2.56626415660392	0\\
2.56636415910398	0\\
2.56646416160404	0\\
2.5665641641041	0\\
2.56666416660416	0\\
2.56676416910423	0\\
2.56686417160429	0\\
2.56696417410435	0\\
2.56706417660442	0\\
2.56716417910448	0\\
2.56726418160454	0\\
2.5673641841046	0\\
2.56746418660467	0\\
2.56756418910473	0\\
2.56766419160479	0\\
2.56776419410485	0\\
2.56786419660492	0\\
2.56796419910498	0\\
2.56806420160504	0\\
2.5681642041051	0\\
2.56826420660517	0\\
2.56836420910523	0\\
2.56846421160529	0\\
2.56856421410535	0\\
2.56866421660541	0\\
2.56876421910548	0\\
2.56886422160554	0\\
2.5689642241056	0\\
2.56906422660567	0\\
2.56916422910573	0\\
2.56926423160579	0\\
2.56936423410585	0\\
2.56946423660592	0\\
2.56956423910598	0\\
2.56966424160604	0\\
2.5697642441061	0\\
2.56986424660617	0\\
2.56996424910623	0\\
2.57006425160629	0\\
2.57016425410635	0\\
2.57026425660642	0\\
2.57036425910648	0\\
2.57046426160654	0\\
2.5705642641066	0\\
2.57066426660667	0\\
2.57076426910673	0\\
2.57086427160679	0\\
2.57096427410685	0\\
2.57106427660692	0\\
2.57116427910698	0\\
2.57126428160704	0\\
2.5713642841071	0\\
2.57146428660717	0\\
2.57156428910723	0\\
2.57166429160729	0\\
2.57176429410735	0\\
2.57186429660742	0\\
2.57196429910748	0\\
2.57206430160754	0\\
2.5721643041076	0\\
2.57226430660767	0\\
2.57236430910773	0\\
2.57246431160779	0\\
2.57256431410785	0\\
2.57266431660792	0\\
2.57276431910798	0\\
2.57286432160804	0\\
2.5729643241081	0\\
2.57306432660817	0\\
2.57316432910823	0\\
2.57326433160829	0\\
2.57336433410835	0\\
2.57346433660842	0\\
2.57356433910848	0\\
2.57366434160854	0\\
2.5737643441086	0\\
2.57386434660867	0\\
2.57396434910873	0\\
2.57406435160879	0\\
2.57416435410885	0\\
2.57426435660892	0\\
2.57436435910898	0\\
2.57446436160904	0\\
2.5745643641091	0\\
2.57466436660917	0\\
2.57476436910923	0\\
2.57486437160929	0\\
2.57496437410935	0\\
2.57506437660942	0\\
2.57516437910948	0\\
2.57526438160954	0\\
2.5753643841096	0\\
2.57546438660967	0\\
2.57556438910973	0\\
2.57566439160979	0\\
2.57576439410985	0\\
2.57586439660992	0\\
2.57596439910998	0\\
2.57606440161004	0\\
2.5761644041101	0\\
2.57626440661017	0\\
2.57636440911023	0\\
2.57646441161029	0\\
2.57656441411035	0\\
2.57666441661042	0\\
2.57676441911048	0\\
2.57686442161054	0\\
2.5769644241106	0\\
2.57706442661067	0\\
2.57716442911073	0\\
2.57726443161079	0\\
2.57736443411085	0\\
2.57746443661092	0\\
2.57756443911098	0\\
2.57766444161104	0\\
2.5777644441111	0\\
2.57786444661117	0\\
2.57796444911123	0\\
2.57806445161129	0\\
2.57816445411135	0\\
2.57826445661142	0\\
2.57836445911148	0\\
2.57846446161154	0\\
2.5785644641116	0\\
2.57866446661167	0\\
2.57876446911173	0\\
2.57886447161179	0\\
2.57896447411185	0\\
2.57906447661192	0\\
2.57916447911198	0\\
2.57926448161204	0\\
2.5793644841121	0\\
2.57946448661217	0\\
2.57956448911223	0\\
2.57966449161229	0\\
2.57976449411235	0\\
2.57986449661242	0\\
2.57996449911248	0\\
2.58006450161254	0\\
2.5801645041126	0\\
2.58026450661267	0\\
2.58036450911273	0\\
2.58046451161279	0\\
2.58056451411285	0\\
2.58066451661292	0\\
2.58076451911298	0\\
2.58086452161304	0\\
2.5809645241131	0\\
2.58106452661317	0\\
2.58116452911323	0\\
2.58126453161329	0\\
2.58136453411335	0\\
2.58146453661342	0\\
2.58156453911348	0\\
2.58166454161354	0\\
2.5817645441136	0\\
2.58186454661367	0\\
2.58196454911373	0\\
2.58206455161379	0\\
2.58216455411385	0\\
2.58226455661392	0\\
2.58236455911398	0\\
2.58246456161404	0\\
2.5825645641141	0\\
2.58266456661417	0\\
2.58276456911423	0\\
2.58286457161429	0\\
2.58296457411435	0\\
2.58306457661442	0\\
2.58316457911448	0\\
2.58326458161454	0\\
2.5833645841146	0\\
2.58346458661467	0\\
2.58356458911473	0\\
2.58366459161479	0\\
2.58376459411485	0\\
2.58386459661492	0\\
2.58396459911498	0\\
2.58406460161504	0\\
2.5841646041151	0\\
2.58426460661517	0\\
2.58436460911523	0\\
2.58446461161529	0\\
2.58456461411535	0\\
2.58466461661542	0\\
2.58476461911548	0\\
2.58486462161554	0\\
2.5849646241156	0\\
2.58506462661567	0\\
2.58516462911573	0\\
2.58526463161579	0\\
2.58536463411585	0\\
2.58546463661592	0\\
2.58556463911598	0\\
2.58566464161604	0\\
2.5857646441161	0\\
2.58586464661617	0\\
2.58596464911623	0\\
2.58606465161629	0\\
2.58616465411635	0\\
2.58626465661642	0\\
2.58636465911648	0\\
2.58646466161654	0\\
2.5865646641166	0\\
2.58666466661667	0\\
2.58676466911673	0\\
2.58686467161679	0\\
2.58696467411685	0\\
2.58706467661692	0\\
2.58716467911698	0\\
2.58726468161704	0\\
2.5873646841171	0\\
2.58746468661717	0\\
2.58756468911723	0\\
2.58766469161729	0\\
2.58776469411735	0\\
2.58786469661742	0\\
2.58796469911748	0\\
2.58806470161754	0\\
2.5881647041176	0\\
2.58826470661767	0\\
2.58836470911773	0\\
2.58846471161779	0\\
2.58856471411785	0\\
2.58866471661792	0\\
2.58876471911798	0\\
2.58886472161804	0\\
2.5889647241181	0\\
2.58906472661817	0\\
2.58916472911823	0\\
2.58926473161829	0\\
2.58936473411835	0\\
2.58946473661842	0\\
2.58956473911848	0\\
2.58966474161854	0\\
2.5897647441186	0\\
2.58986474661867	0\\
2.58996474911873	0\\
2.59006475161879	0\\
2.59016475411885	0\\
2.59026475661892	0\\
2.59036475911898	0\\
2.59046476161904	0\\
2.5905647641191	0\\
2.59066476661917	0\\
2.59076476911923	0\\
2.59086477161929	0\\
2.59096477411935	0\\
2.59106477661942	0\\
2.59116477911948	0\\
2.59126478161954	0\\
2.5913647841196	0\\
2.59146478661967	0\\
2.59156478911973	0\\
2.59166479161979	0\\
2.59176479411985	0\\
2.59186479661992	0\\
2.59196479911998	0\\
2.59206480162004	0\\
2.5921648041201	0\\
2.59226480662017	0\\
2.59236480912023	0\\
2.59246481162029	0\\
2.59256481412035	0\\
2.59266481662042	0\\
2.59276481912048	0\\
2.59286482162054	0\\
2.5929648241206	0\\
2.59306482662067	0\\
2.59316482912073	0\\
2.59326483162079	0\\
2.59336483412085	0\\
2.59346483662092	0\\
2.59356483912098	0\\
2.59366484162104	0\\
2.5937648441211	0\\
2.59386484662117	0\\
2.59396484912123	0\\
2.59406485162129	0\\
2.59416485412135	0\\
2.59426485662142	0\\
2.59436485912148	0\\
2.59446486162154	0\\
2.5945648641216	0\\
2.59466486662167	0\\
2.59476486912173	0\\
2.59486487162179	0\\
2.59496487412185	0\\
2.59506487662192	0\\
2.59516487912198	0\\
2.59526488162204	0\\
2.5953648841221	0\\
2.59546488662217	0\\
2.59556488912223	0\\
2.59566489162229	0\\
2.59576489412235	0\\
2.59586489662242	0\\
2.59596489912248	0\\
2.59606490162254	0\\
2.5961649041226	0\\
2.59626490662267	0\\
2.59636490912273	0\\
2.59646491162279	0\\
2.59656491412285	0\\
2.59666491662292	0\\
2.59676491912298	0\\
2.59686492162304	0\\
2.5969649241231	0\\
2.59706492662317	0\\
2.59716492912323	0\\
2.59726493162329	0\\
2.59736493412335	0\\
2.59746493662342	0\\
2.59756493912348	0\\
2.59766494162354	0\\
2.5977649441236	0\\
2.59786494662367	0\\
2.59796494912373	0\\
2.59806495162379	0\\
2.59816495412385	0\\
2.59826495662392	0\\
2.59836495912398	0\\
2.59846496162404	0\\
2.5985649641241	0\\
2.59866496662417	0\\
2.59876496912423	0\\
2.59886497162429	0\\
2.59896497412435	0\\
2.59906497662442	0\\
2.59916497912448	0\\
2.59926498162454	0\\
2.5993649841246	0\\
2.59946498662467	0\\
2.59956498912473	0\\
2.59966499162479	0\\
2.59976499412485	0\\
2.59986499662492	0\\
2.59996499912498	0\\
2.60006500162504	0\\
2.6001650041251	0\\
2.60026500662517	0\\
2.60036500912523	0\\
2.60046501162529	0\\
2.60056501412535	0\\
2.60066501662542	0\\
2.60076501912548	0\\
2.60086502162554	0\\
2.6009650241256	0\\
2.60106502662567	0\\
2.60116502912573	0\\
2.60126503162579	0\\
2.60136503412585	0\\
2.60146503662592	0\\
2.60156503912598	0\\
2.60166504162604	0\\
2.6017650441261	0\\
2.60186504662617	0\\
2.60196504912623	0\\
2.60206505162629	0\\
2.60216505412635	0\\
2.60226505662642	0\\
2.60236505912648	0\\
2.60246506162654	0\\
2.6025650641266	0\\
2.60266506662667	0\\
2.60276506912673	0\\
2.60286507162679	0\\
2.60296507412685	0\\
2.60306507662692	0\\
2.60316507912698	0\\
2.60326508162704	0\\
2.6033650841271	0\\
2.60346508662717	0\\
2.60356508912723	0\\
2.60366509162729	0\\
2.60376509412735	0\\
2.60386509662742	0\\
2.60396509912748	0\\
2.60406510162754	0\\
2.6041651041276	0\\
2.60426510662767	0\\
2.60436510912773	0\\
2.60446511162779	0\\
2.60456511412785	0\\
2.60466511662792	0\\
2.60476511912798	0\\
2.60486512162804	0\\
2.6049651241281	0\\
2.60506512662817	0\\
2.60516512912823	0\\
2.60526513162829	0\\
2.60536513412835	0\\
2.60546513662842	0\\
2.60556513912848	0\\
2.60566514162854	0\\
2.6057651441286	0\\
2.60586514662867	0\\
2.60596514912873	0\\
2.60606515162879	0\\
2.60616515412885	0\\
2.60626515662892	0\\
2.60636515912898	0\\
2.60646516162904	0\\
2.6065651641291	0\\
2.60666516662917	0\\
2.60676516912923	0\\
2.60686517162929	0\\
2.60696517412935	0\\
2.60706517662942	0\\
2.60716517912948	0\\
2.60726518162954	0\\
2.6073651841296	0\\
2.60746518662967	0\\
2.60756518912973	0\\
2.60766519162979	0\\
2.60776519412985	0\\
2.60786519662992	0\\
2.60796519912998	0\\
2.60806520163004	0\\
2.6081652041301	0\\
2.60826520663017	0\\
2.60836520913023	0\\
2.60846521163029	0\\
2.60856521413035	0\\
2.60866521663042	0\\
2.60876521913048	0\\
2.60886522163054	0\\
2.6089652241306	0\\
2.60906522663067	0\\
2.60916522913073	0\\
2.60926523163079	0\\
2.60936523413085	0\\
2.60946523663092	0\\
2.60956523913098	0\\
2.60966524163104	0\\
2.6097652441311	0\\
2.60986524663117	0\\
2.60996524913123	0\\
2.61006525163129	0\\
2.61016525413135	0\\
2.61026525663142	0\\
2.61036525913148	0\\
2.61046526163154	0\\
2.6105652641316	0\\
2.61066526663167	0\\
2.61076526913173	0\\
2.61086527163179	0\\
2.61096527413185	0\\
2.61106527663192	0\\
2.61116527913198	0\\
2.61126528163204	0\\
2.6113652841321	0\\
2.61146528663217	0\\
2.61156528913223	0\\
2.61166529163229	0\\
2.61176529413235	0\\
2.61186529663242	0\\
2.61196529913248	0\\
2.61206530163254	0\\
2.6121653041326	0\\
2.61226530663267	0\\
2.61236530913273	0\\
2.61246531163279	0\\
2.61256531413285	0\\
2.61266531663292	0\\
2.61276531913298	0\\
2.61286532163304	0\\
2.6129653241331	0\\
2.61306532663317	0\\
2.61316532913323	0\\
2.61326533163329	0\\
2.61336533413335	0\\
2.61346533663342	0\\
2.61356533913348	0\\
2.61366534163354	0\\
2.6137653441336	0\\
2.61386534663367	0\\
2.61396534913373	0\\
2.61406535163379	0\\
2.61416535413385	0\\
2.61426535663392	0\\
2.61436535913398	0\\
2.61446536163404	0\\
2.6145653641341	0\\
2.61466536663417	0\\
2.61476536913423	0\\
2.61486537163429	0\\
2.61496537413435	0\\
2.61506537663442	0\\
2.61516537913448	0\\
2.61526538163454	0\\
2.6153653841346	0\\
2.61546538663467	0\\
2.61556538913473	0\\
2.61566539163479	0\\
2.61576539413485	0\\
2.61586539663492	0\\
2.61596539913498	0\\
2.61606540163504	0\\
2.6161654041351	0\\
2.61626540663517	0\\
2.61636540913523	0\\
2.61646541163529	0\\
2.61656541413535	0\\
2.61666541663542	0\\
2.61676541913548	0\\
2.61686542163554	0\\
2.6169654241356	0\\
2.61706542663567	0\\
2.61716542913573	0\\
2.61726543163579	0\\
2.61736543413585	0\\
2.61746543663592	0\\
2.61756543913598	0\\
2.61766544163604	0\\
2.6177654441361	0\\
2.61786544663617	0\\
2.61796544913623	0\\
2.61806545163629	0\\
2.61816545413635	0\\
2.61826545663642	0\\
2.61836545913648	0\\
2.61846546163654	0\\
2.6185654641366	0\\
2.61866546663667	0\\
2.61876546913673	0\\
2.61886547163679	0\\
2.61896547413685	0\\
2.61906547663692	0\\
2.61916547913698	0\\
2.61926548163704	0\\
2.6193654841371	0\\
2.61946548663717	0\\
2.61956548913723	0\\
2.61966549163729	0\\
2.61976549413735	0\\
2.61986549663742	0\\
2.61996549913748	0\\
2.62006550163754	0\\
2.6201655041376	0\\
2.62026550663767	0\\
2.62036550913773	0\\
2.62046551163779	0\\
2.62056551413785	0\\
2.62066551663792	0\\
2.62076551913798	0\\
2.62086552163804	0\\
2.6209655241381	0\\
2.62106552663817	0\\
2.62116552913823	0\\
2.62126553163829	0\\
2.62136553413835	0\\
2.62146553663842	0\\
2.62156553913848	0\\
2.62166554163854	0\\
2.6217655441386	0\\
2.62186554663867	0\\
2.62196554913873	0\\
2.62206555163879	0\\
2.62216555413885	0\\
2.62226555663892	0\\
2.62236555913898	0\\
2.62246556163904	0\\
2.6225655641391	0\\
2.62266556663917	0\\
2.62276556913923	0\\
2.62286557163929	0\\
2.62296557413935	0\\
2.62306557663942	0\\
2.62316557913948	0\\
2.62326558163954	0\\
2.6233655841396	0\\
2.62346558663967	0\\
2.62356558913973	0\\
2.62366559163979	0\\
2.62376559413985	0\\
2.62386559663992	0\\
2.62396559913998	0\\
2.62406560164004	0\\
2.6241656041401	0\\
2.62426560664017	0\\
2.62436560914023	0\\
2.62446561164029	0\\
2.62456561414035	0\\
2.62466561664042	0\\
2.62476561914048	0\\
2.62486562164054	0\\
2.6249656241406	0\\
2.62506562664067	0\\
2.62516562914073	0\\
2.62526563164079	0\\
2.62536563414085	0\\
2.62546563664092	0\\
2.62556563914098	0\\
2.62566564164104	0\\
2.6257656441411	0\\
2.62586564664117	0\\
2.62596564914123	0\\
2.62606565164129	0\\
2.62616565414135	0\\
2.62626565664142	0\\
2.62636565914148	0\\
2.62646566164154	0\\
2.6265656641416	0\\
2.62666566664167	0\\
2.62676566914173	0\\
2.62686567164179	0\\
2.62696567414185	0\\
2.62706567664192	0\\
2.62716567914198	0\\
2.62726568164204	0\\
2.6273656841421	0\\
2.62746568664217	0\\
2.62756568914223	0\\
2.62766569164229	0\\
2.62776569414235	0\\
2.62786569664242	0\\
2.62796569914248	0\\
2.62806570164254	0\\
2.6281657041426	0\\
2.62826570664267	0\\
2.62836570914273	0\\
2.62846571164279	0\\
2.62856571414285	0\\
2.62866571664292	0\\
2.62876571914298	0\\
2.62886572164304	0\\
2.6289657241431	0\\
2.62906572664317	0\\
2.62916572914323	0\\
2.62926573164329	0\\
2.62936573414335	0\\
2.62946573664342	0\\
2.62956573914348	0\\
2.62966574164354	0\\
2.6297657441436	0\\
2.62986574664367	0\\
2.62996574914373	0\\
2.63006575164379	0\\
2.63016575414385	0\\
2.63026575664392	0\\
2.63036575914398	0\\
2.63046576164404	0\\
2.6305657641441	0\\
2.63066576664417	0\\
2.63076576914423	0\\
2.63086577164429	0\\
2.63096577414435	0\\
2.63106577664442	0\\
2.63116577914448	0\\
2.63126578164454	0\\
2.6313657841446	0\\
2.63146578664467	0\\
2.63156578914473	0\\
2.63166579164479	0\\
2.63176579414485	0\\
2.63186579664492	0\\
2.63196579914498	0\\
2.63206580164504	0\\
2.6321658041451	0\\
2.63226580664517	0\\
2.63236580914523	0\\
2.63246581164529	0\\
2.63256581414535	0\\
2.63266581664542	0\\
2.63276581914548	0\\
2.63286582164554	0\\
2.6329658241456	0\\
2.63306582664567	0\\
2.63316582914573	0\\
2.63326583164579	0\\
2.63336583414585	0\\
2.63346583664592	0\\
2.63356583914598	0\\
2.63366584164604	0\\
2.6337658441461	0\\
2.63386584664617	0\\
2.63396584914623	0\\
2.63406585164629	0\\
2.63416585414635	0\\
2.63426585664642	0\\
2.63436585914648	0\\
2.63446586164654	0\\
2.6345658641466	0\\
2.63466586664667	0\\
2.63476586914673	0\\
2.63486587164679	0\\
2.63496587414685	0\\
2.63506587664692	0\\
2.63516587914698	0\\
2.63526588164704	0\\
2.6353658841471	0\\
2.63546588664717	0\\
2.63556588914723	0\\
2.63566589164729	0\\
2.63576589414735	0\\
2.63586589664742	0\\
2.63596589914748	0\\
2.63606590164754	0\\
2.6361659041476	0\\
2.63626590664767	0\\
2.63636590914773	0\\
2.63646591164779	0\\
2.63656591414785	0\\
2.63666591664792	0\\
2.63676591914798	0\\
2.63686592164804	0\\
2.6369659241481	0\\
2.63706592664817	0\\
2.63716592914823	0\\
2.63726593164829	0\\
2.63736593414835	0\\
2.63746593664842	0\\
2.63756593914848	0\\
2.63766594164854	0\\
2.6377659441486	0\\
2.63786594664867	0\\
2.63796594914873	0\\
2.63806595164879	0\\
2.63816595414885	0\\
2.63826595664892	0\\
2.63836595914898	0\\
2.63846596164904	0\\
2.6385659641491	0\\
2.63866596664917	0\\
2.63876596914923	0\\
2.63886597164929	0\\
2.63896597414935	0\\
2.63906597664942	0\\
2.63916597914948	0\\
2.63926598164954	0\\
2.6393659841496	0\\
2.63946598664967	0\\
2.63956598914973	0\\
2.63966599164979	0\\
2.63976599414985	0\\
2.63986599664992	0\\
2.63996599914998	0\\
2.64006600165004	0\\
2.6401660041501	0\\
2.64026600665017	0\\
2.64036600915023	0\\
2.64046601165029	0\\
2.64056601415035	0\\
2.64066601665042	0\\
2.64076601915048	0\\
2.64086602165054	0\\
2.6409660241506	0\\
2.64106602665067	0\\
2.64116602915073	0\\
2.64126603165079	0\\
2.64136603415085	0\\
2.64146603665092	0\\
2.64156603915098	0\\
2.64166604165104	0\\
2.6417660441511	0\\
2.64186604665117	0\\
2.64196604915123	0\\
2.64206605165129	0\\
2.64216605415135	0\\
2.64226605665142	0\\
2.64236605915148	0\\
2.64246606165154	0\\
2.6425660641516	0\\
2.64266606665167	0\\
2.64276606915173	0\\
2.64286607165179	0\\
2.64296607415185	0\\
2.64306607665192	0\\
2.64316607915198	0\\
2.64326608165204	0\\
2.6433660841521	0\\
2.64346608665217	0\\
2.64356608915223	0\\
2.64366609165229	0\\
2.64376609415235	0\\
2.64386609665242	0\\
2.64396609915248	0\\
2.64406610165254	0\\
2.6441661041526	0\\
2.64426610665267	0\\
2.64436610915273	0\\
2.64446611165279	0\\
2.64456611415285	0\\
2.64466611665292	0\\
2.64476611915298	0\\
2.64486612165304	0\\
2.6449661241531	0\\
2.64506612665317	0\\
2.64516612915323	0\\
2.64526613165329	0\\
2.64536613415335	0\\
2.64546613665342	0\\
2.64556613915348	0\\
2.64566614165354	0\\
2.6457661441536	0\\
2.64586614665367	0\\
2.64596614915373	0\\
2.64606615165379	0\\
2.64616615415385	0\\
2.64626615665392	0\\
2.64636615915398	0\\
2.64646616165404	0\\
2.6465661641541	0\\
2.64666616665417	0\\
2.64676616915423	0\\
2.64686617165429	0\\
2.64696617415435	0\\
2.64706617665442	0\\
2.64716617915448	0\\
2.64726618165454	0\\
2.6473661841546	0\\
2.64746618665467	0\\
2.64756618915473	0\\
2.64766619165479	0\\
2.64776619415485	0\\
2.64786619665492	0\\
2.64796619915498	0\\
2.64806620165504	0\\
2.6481662041551	0\\
2.64826620665517	0\\
2.64836620915523	0\\
2.64846621165529	0\\
2.64856621415535	0\\
2.64866621665542	0\\
2.64876621915548	0\\
2.64886622165554	0\\
2.6489662241556	0\\
2.64906622665567	0\\
2.64916622915573	0\\
2.64926623165579	0\\
2.64936623415585	0\\
2.64946623665592	0\\
2.64956623915598	0\\
2.64966624165604	0\\
2.6497662441561	0\\
2.64986624665617	0\\
2.64996624915623	0\\
2.65006625165629	0\\
2.65016625415635	0\\
2.65026625665642	0\\
2.65036625915648	0\\
2.65046626165654	0\\
2.6505662641566	0\\
2.65066626665667	0\\
2.65076626915673	0\\
2.65086627165679	0\\
2.65096627415685	0\\
2.65106627665692	0\\
2.65116627915698	0\\
2.65126628165704	0\\
2.6513662841571	0\\
2.65146628665717	0\\
2.65156628915723	0\\
2.65166629165729	0\\
2.65176629415735	0\\
2.65186629665742	0\\
2.65196629915748	0\\
2.65206630165754	0\\
2.6521663041576	0\\
2.65226630665767	0\\
2.65236630915773	0\\
2.65246631165779	0\\
2.65256631415785	0\\
2.65266631665792	0\\
2.65276631915798	0\\
2.65286632165804	0\\
2.6529663241581	0\\
2.65306632665817	0\\
2.65316632915823	0\\
2.65326633165829	0\\
2.65336633415835	0\\
2.65346633665842	0\\
2.65356633915848	0\\
2.65366634165854	0\\
2.6537663441586	0\\
2.65386634665867	0\\
2.65396634915873	0\\
2.65406635165879	0\\
2.65416635415885	0\\
2.65426635665892	0\\
2.65436635915898	0\\
2.65446636165904	0\\
2.6545663641591	0\\
2.65466636665917	0\\
2.65476636915923	0\\
2.65486637165929	0\\
2.65496637415935	0\\
2.65506637665942	0\\
2.65516637915948	0\\
2.65526638165954	0\\
2.6553663841596	0\\
2.65546638665967	0\\
2.65556638915973	0\\
2.65566639165979	0\\
2.65576639415985	0\\
2.65586639665992	0\\
2.65596639915998	0\\
2.65606640166004	0\\
2.6561664041601	0\\
2.65626640666017	0\\
2.65636640916023	0\\
2.65646641166029	0\\
2.65656641416035	0\\
2.65666641666042	0\\
2.65676641916048	0\\
2.65686642166054	0\\
2.6569664241606	0\\
2.65706642666067	0\\
2.65716642916073	0\\
2.65726643166079	0\\
2.65736643416085	0\\
2.65746643666092	0\\
2.65756643916098	0\\
2.65766644166104	0\\
2.6577664441611	0\\
2.65786644666117	0\\
2.65796644916123	0\\
2.65806645166129	0\\
2.65816645416135	0\\
2.65826645666142	0\\
2.65836645916148	0\\
2.65846646166154	0\\
2.6585664641616	0\\
2.65866646666167	0\\
2.65876646916173	0\\
2.65886647166179	0\\
2.65896647416185	0\\
2.65906647666192	0\\
2.65916647916198	0\\
2.65926648166204	0\\
2.6593664841621	0\\
2.65946648666217	0\\
2.65956648916223	0\\
2.65966649166229	0\\
2.65976649416235	0\\
2.65986649666242	0\\
2.65996649916248	0\\
2.66006650166254	0\\
2.6601665041626	0\\
2.66026650666267	0\\
2.66036650916273	0\\
2.66046651166279	0\\
2.66056651416285	0\\
2.66066651666292	0\\
2.66076651916298	0\\
2.66086652166304	0\\
2.6609665241631	0\\
2.66106652666317	0\\
2.66116652916323	0\\
2.66126653166329	0\\
2.66136653416335	0\\
2.66146653666342	0\\
2.66156653916348	0\\
2.66166654166354	0\\
2.6617665441636	0\\
2.66186654666367	0\\
2.66196654916373	0\\
2.66206655166379	0\\
2.66216655416385	0\\
2.66226655666392	0\\
2.66236655916398	0\\
2.66246656166404	0\\
2.6625665641641	0\\
2.66266656666417	0\\
2.66276656916423	0\\
2.66286657166429	0\\
2.66296657416435	0\\
2.66306657666442	0\\
2.66316657916448	0\\
2.66326658166454	0\\
2.6633665841646	0\\
2.66346658666467	0\\
2.66356658916473	0\\
2.66366659166479	0\\
2.66376659416485	0\\
2.66386659666492	0\\
2.66396659916498	0\\
2.66406660166504	0\\
2.6641666041651	0\\
2.66426660666517	0\\
2.66436660916523	0\\
2.66446661166529	0\\
2.66456661416535	0\\
2.66466661666542	0\\
2.66476661916548	0\\
2.66486662166554	0\\
2.6649666241656	0\\
2.66506662666567	0\\
2.66516662916573	0\\
2.66526663166579	0\\
2.66536663416585	0\\
2.66546663666592	0\\
2.66556663916598	0\\
2.66566664166604	0\\
2.6657666441661	0\\
2.66586664666617	0\\
2.66596664916623	0\\
2.66606665166629	0\\
2.66616665416635	0\\
2.66626665666642	0\\
2.66636665916648	0\\
2.66646666166654	0\\
2.6665666641666	0\\
2.66666666666667	0\\
2.66676666916673	0\\
2.66686667166679	0\\
2.66696667416685	0\\
2.66706667666692	0\\
2.66716667916698	0\\
2.66726668166704	0\\
2.6673666841671	0\\
2.66746668666717	0\\
2.66756668916723	0\\
2.66766669166729	0\\
2.66776669416735	0\\
2.66786669666742	0\\
2.66796669916748	0\\
2.66806670166754	0\\
2.6681667041676	0\\
2.66826670666767	0\\
2.66836670916773	0\\
2.66846671166779	0\\
2.66856671416785	0\\
2.66866671666792	0\\
2.66876671916798	0\\
2.66886672166804	0\\
2.6689667241681	0\\
2.66906672666817	0\\
2.66916672916823	0\\
2.66926673166829	0\\
2.66936673416835	0\\
2.66946673666842	0\\
2.66956673916848	0\\
2.66966674166854	0\\
2.6697667441686	0\\
2.66986674666867	0\\
2.66996674916873	0\\
2.67006675166879	0\\
2.67016675416885	0\\
2.67026675666892	0\\
2.67036675916898	0\\
2.67046676166904	0\\
2.6705667641691	0\\
2.67066676666917	0\\
2.67076676916923	0\\
2.67086677166929	0\\
2.67096677416935	0\\
2.67106677666942	0\\
2.67116677916948	0\\
2.67126678166954	0\\
2.6713667841696	0\\
2.67146678666967	0\\
2.67156678916973	0\\
2.67166679166979	0\\
2.67176679416985	0\\
2.67186679666992	0\\
2.67196679916998	0\\
2.67206680167004	0\\
2.6721668041701	0\\
2.67226680667017	0\\
2.67236680917023	0\\
2.67246681167029	0\\
2.67256681417035	0\\
2.67266681667042	0\\
2.67276681917048	0\\
2.67286682167054	0\\
2.6729668241706	0\\
2.67306682667067	0\\
2.67316682917073	0\\
2.67326683167079	0\\
2.67336683417085	0\\
2.67346683667092	0\\
2.67356683917098	0\\
2.67366684167104	0\\
2.6737668441711	0\\
2.67386684667117	0\\
2.67396684917123	0\\
2.67406685167129	0\\
2.67416685417135	0\\
2.67426685667142	0\\
2.67436685917148	0\\
2.67446686167154	0\\
2.6745668641716	0\\
2.67466686667167	0\\
2.67476686917173	0\\
2.67486687167179	0\\
2.67496687417185	0\\
2.67506687667192	0\\
2.67516687917198	0\\
2.67526688167204	0\\
2.6753668841721	0\\
2.67546688667217	0\\
2.67556688917223	0\\
2.67566689167229	0\\
2.67576689417235	0\\
2.67586689667242	0\\
2.67596689917248	0\\
2.67606690167254	0\\
2.6761669041726	0\\
2.67626690667267	0\\
2.67636690917273	0\\
2.67646691167279	0\\
2.67656691417285	0\\
2.67666691667292	0\\
2.67676691917298	0\\
2.67686692167304	0\\
2.6769669241731	0\\
2.67706692667317	0\\
2.67716692917323	0\\
2.67726693167329	0\\
2.67736693417335	0\\
2.67746693667342	0\\
2.67756693917348	0\\
2.67766694167354	0\\
2.6777669441736	0\\
2.67786694667367	0\\
2.67796694917373	0\\
2.67806695167379	0\\
2.67816695417385	0\\
2.67826695667392	0\\
2.67836695917398	0\\
2.67846696167404	0\\
2.6785669641741	0\\
2.67866696667417	0\\
2.67876696917423	0\\
2.67886697167429	0\\
2.67896697417435	0\\
2.67906697667442	0\\
2.67916697917448	0\\
2.67926698167454	0\\
2.6793669841746	0\\
2.67946698667467	0\\
2.67956698917473	0\\
2.67966699167479	0\\
2.67976699417485	0\\
2.67986699667492	0\\
2.67996699917498	0\\
2.68006700167504	0\\
2.6801670041751	0\\
2.68026700667517	0\\
2.68036700917523	0\\
2.68046701167529	0\\
2.68056701417535	0\\
2.68066701667542	0\\
2.68076701917548	0\\
2.68086702167554	0\\
2.6809670241756	0\\
2.68106702667567	0\\
2.68116702917573	0\\
2.68126703167579	0\\
2.68136703417585	0\\
2.68146703667592	0\\
2.68156703917598	0\\
2.68166704167604	0\\
2.6817670441761	0\\
2.68186704667617	0\\
2.68196704917623	0\\
2.68206705167629	0\\
2.68216705417635	0\\
2.68226705667642	0\\
2.68236705917648	0\\
2.68246706167654	0\\
2.6825670641766	0\\
2.68266706667667	0\\
2.68276706917673	0\\
2.68286707167679	0\\
2.68296707417685	0\\
2.68306707667692	0\\
2.68316707917698	0\\
2.68326708167704	0\\
2.6833670841771	0\\
2.68346708667717	0\\
2.68356708917723	0\\
2.68366709167729	0\\
2.68376709417735	0\\
2.68386709667742	0\\
2.68396709917748	0\\
2.68406710167754	0\\
2.6841671041776	0\\
2.68426710667767	0\\
2.68436710917773	0\\
2.68446711167779	0\\
2.68456711417785	0\\
2.68466711667792	0\\
2.68476711917798	0\\
2.68486712167804	0\\
2.6849671241781	0\\
2.68506712667817	0\\
2.68516712917823	0\\
2.68526713167829	0\\
2.68536713417835	0\\
2.68546713667842	0\\
2.68556713917848	0\\
2.68566714167854	0\\
2.6857671441786	0\\
2.68586714667867	0\\
2.68596714917873	0\\
2.68606715167879	0\\
2.68616715417885	0\\
2.68626715667892	0\\
2.68636715917898	0\\
2.68646716167904	0\\
2.6865671641791	0\\
2.68666716667917	0\\
2.68676716917923	0\\
2.68686717167929	0\\
2.68696717417935	0\\
2.68706717667942	0\\
2.68716717917948	0\\
2.68726718167954	0\\
2.6873671841796	0\\
2.68746718667967	0\\
2.68756718917973	0\\
2.68766719167979	0\\
2.68776719417985	0\\
2.68786719667992	0\\
2.68796719917998	0\\
2.68806720168004	0\\
2.6881672041801	0\\
2.68826720668017	0\\
2.68836720918023	0\\
2.68846721168029	0\\
2.68856721418035	0\\
2.68866721668042	0\\
2.68876721918048	0\\
2.68886722168054	0\\
2.6889672241806	0\\
2.68906722668067	0\\
2.68916722918073	0\\
2.68926723168079	0\\
2.68936723418085	0\\
2.68946723668092	0\\
2.68956723918098	0\\
2.68966724168104	0\\
2.6897672441811	0\\
2.68986724668117	0\\
2.68996724918123	0\\
2.69006725168129	0\\
2.69016725418135	0\\
2.69026725668142	0\\
2.69036725918148	0\\
2.69046726168154	0\\
2.6905672641816	0\\
2.69066726668167	0\\
2.69076726918173	0\\
2.69086727168179	0\\
2.69096727418185	0\\
2.69106727668192	0\\
2.69116727918198	0\\
2.69126728168204	0\\
2.6913672841821	0\\
2.69146728668217	0\\
2.69156728918223	0\\
2.69166729168229	0\\
2.69176729418235	0\\
2.69186729668242	0\\
2.69196729918248	0\\
2.69206730168254	0\\
2.6921673041826	0\\
2.69226730668267	0\\
2.69236730918273	0\\
2.69246731168279	0\\
2.69256731418285	0\\
2.69266731668292	0\\
2.69276731918298	0\\
2.69286732168304	0\\
2.6929673241831	0\\
2.69306732668317	0\\
2.69316732918323	0\\
2.69326733168329	0\\
2.69336733418335	0\\
2.69346733668342	0\\
2.69356733918348	0\\
2.69366734168354	0\\
2.6937673441836	0\\
2.69386734668367	0\\
2.69396734918373	0\\
2.69406735168379	0\\
2.69416735418385	0\\
2.69426735668392	0\\
2.69436735918398	0\\
2.69446736168404	0\\
2.6945673641841	0\\
2.69466736668417	0\\
2.69476736918423	0\\
2.69486737168429	0\\
2.69496737418435	0\\
2.69506737668442	0\\
2.69516737918448	0\\
2.69526738168454	0\\
2.6953673841846	0\\
2.69546738668467	0\\
2.69556738918473	0\\
2.69566739168479	0\\
2.69576739418485	0\\
2.69586739668492	0\\
2.69596739918498	0\\
2.69606740168504	0\\
2.6961674041851	0\\
2.69626740668517	0\\
2.69636740918523	0\\
2.69646741168529	0\\
2.69656741418535	0\\
2.69666741668542	0\\
2.69676741918548	0\\
2.69686742168554	0\\
2.6969674241856	0\\
2.69706742668567	0\\
2.69716742918573	0\\
2.69726743168579	0\\
2.69736743418585	0\\
2.69746743668592	0\\
2.69756743918598	0\\
2.69766744168604	0\\
2.6977674441861	0\\
2.69786744668617	0\\
2.69796744918623	0\\
2.69806745168629	0\\
2.69816745418635	0\\
2.69826745668642	0\\
2.69836745918648	0\\
2.69846746168654	0\\
2.6985674641866	0\\
2.69866746668667	0\\
2.69876746918673	0\\
2.69886747168679	0\\
2.69896747418685	0\\
2.69906747668692	0\\
2.69916747918698	0\\
2.69926748168704	0\\
2.6993674841871	0\\
2.69946748668717	0\\
2.69956748918723	0\\
2.69966749168729	0\\
2.69976749418735	0\\
2.69986749668742	0\\
2.69996749918748	0\\
2.70006750168754	0\\
2.7001675041876	0\\
2.70026750668767	0\\
2.70036750918773	0\\
2.70046751168779	0\\
2.70056751418785	0\\
2.70066751668792	0\\
2.70076751918798	0\\
2.70086752168804	0\\
2.7009675241881	0\\
2.70106752668817	0\\
2.70116752918823	0\\
2.70126753168829	0\\
2.70136753418835	0\\
2.70146753668842	0\\
2.70156753918848	0\\
2.70166754168854	0\\
2.7017675441886	0\\
2.70186754668867	0\\
2.70196754918873	0\\
2.70206755168879	0\\
2.70216755418885	0\\
2.70226755668892	0\\
2.70236755918898	0\\
2.70246756168904	0\\
2.7025675641891	0\\
2.70266756668917	0\\
2.70276756918923	0\\
2.70286757168929	0\\
2.70296757418935	0\\
2.70306757668942	0\\
2.70316757918948	0\\
2.70326758168954	0\\
2.7033675841896	0\\
2.70346758668967	0\\
2.70356758918973	0\\
2.70366759168979	0\\
2.70376759418985	0\\
2.70386759668992	0\\
2.70396759918998	0\\
2.70406760169004	0\\
2.7041676041901	0\\
2.70426760669017	0\\
2.70436760919023	0\\
2.70446761169029	0\\
2.70456761419035	0\\
2.70466761669042	0\\
2.70476761919048	0\\
2.70486762169054	0\\
2.7049676241906	0\\
2.70506762669067	0\\
2.70516762919073	0\\
2.70526763169079	0\\
2.70536763419085	0\\
2.70546763669092	0\\
2.70556763919098	0\\
2.70566764169104	0\\
2.7057676441911	0\\
2.70586764669117	0\\
2.70596764919123	0\\
2.70606765169129	0\\
2.70616765419135	0\\
2.70626765669142	0\\
2.70636765919148	0\\
2.70646766169154	0\\
2.7065676641916	0\\
2.70666766669167	0\\
2.70676766919173	0\\
2.70686767169179	0\\
2.70696767419185	0\\
2.70706767669192	0\\
2.70716767919198	0\\
2.70726768169204	0\\
2.7073676841921	0\\
2.70746768669217	0\\
2.70756768919223	0\\
2.70766769169229	0\\
2.70776769419235	0\\
2.70786769669242	0\\
2.70796769919248	0\\
2.70806770169254	0\\
2.7081677041926	0\\
2.70826770669267	0\\
2.70836770919273	0\\
2.70846771169279	0\\
2.70856771419286	0\\
2.70866771669292	0\\
2.70876771919298	0\\
2.70886772169304	0\\
2.7089677241931	0\\
2.70906772669317	0\\
2.70916772919323	0\\
2.70926773169329	0\\
2.70936773419335	0\\
2.70946773669342	0\\
2.70956773919348	0\\
2.70966774169354	0\\
2.7097677441936	0\\
2.70986774669367	0\\
2.70996774919373	0\\
2.71006775169379	0\\
2.71016775419385	0\\
2.71026775669392	0\\
2.71036775919398	0\\
2.71046776169404	0\\
2.71056776419411	0\\
2.71066776669417	0\\
2.71076776919423	0\\
2.71086777169429	0\\
2.71096777419435	0\\
2.71106777669442	0\\
2.71116777919448	0\\
2.71126778169454	0\\
2.7113677841946	0\\
2.71146778669467	0\\
2.71156778919473	0\\
2.71166779169479	0\\
2.71176779419485	0\\
2.71186779669492	0\\
2.71196779919498	0\\
2.71206780169504	0\\
2.7121678041951	0\\
2.71226780669517	0\\
2.71236780919523	0\\
2.71246781169529	0\\
2.71256781419536	0\\
2.71266781669542	0\\
2.71276781919548	0\\
2.71286782169554	0\\
2.71296782419561	0\\
2.71306782669567	0\\
2.71316782919573	0\\
2.71326783169579	0\\
2.71336783419585	0\\
2.71346783669592	0\\
2.71356783919598	0\\
2.71366784169604	0\\
2.7137678441961	0\\
2.71386784669617	0\\
2.71396784919623	0\\
2.71406785169629	0\\
2.71416785419635	0\\
2.71426785669642	0\\
2.71436785919648	0\\
2.71446786169654	0\\
2.71456786419661	0\\
2.71466786669667	0\\
2.71476786919673	0\\
2.71486787169679	0\\
2.71496787419686	0\\
2.71506787669692	0\\
2.71516787919698	0\\
2.71526788169704	0\\
2.7153678841971	0\\
2.71546788669717	0\\
2.71556788919723	0\\
2.71566789169729	0\\
2.71576789419735	0\\
2.71586789669742	0\\
2.71596789919748	0\\
2.71606790169754	0\\
2.7161679041976	0\\
2.71626790669767	0\\
2.71636790919773	0\\
2.71646791169779	0\\
2.71656791419786	0\\
2.71666791669792	0\\
2.71676791919798	0\\
2.71686792169804	0\\
2.71696792419811	0\\
2.71706792669817	0\\
2.71716792919823	0\\
2.71726793169829	0\\
2.71736793419835	0\\
2.71746793669842	0\\
2.71756793919848	0\\
2.71766794169854	0\\
2.7177679441986	0\\
2.71786794669867	0\\
2.71796794919873	0\\
2.71806795169879	0\\
2.71816795419885	0\\
2.71826795669892	0\\
2.71836795919898	0\\
2.71846796169904	0\\
2.71856796419911	0\\
2.71866796669917	0\\
2.71876796919923	0\\
2.71886797169929	0\\
2.71896797419936	0\\
2.71906797669942	0\\
2.71916797919948	0\\
2.71926798169954	0\\
2.71936798419961	0\\
2.71946798669967	0\\
2.71956798919973	0\\
2.71966799169979	0\\
2.71976799419985	0\\
2.71986799669992	0\\
2.71996799919998	0\\
2.72006800170004	0\\
2.7201680042001	0\\
2.72026800670017	0\\
2.72036800920023	0\\
2.72046801170029	0\\
2.72056801420036	0\\
2.72066801670042	0\\
2.72076801920048	0\\
2.72086802170054	0\\
2.72096802420061	0\\
2.72106802670067	0\\
2.72116802920073	0\\
2.72126803170079	0\\
2.72136803420086	0\\
2.72146803670092	0\\
2.72156803920098	0\\
2.72166804170104	0\\
2.7217680442011	0\\
2.72186804670117	0\\
2.72196804920123	0\\
2.72206805170129	0\\
2.72216805420135	0\\
2.72226805670142	0\\
2.72236805920148	0\\
2.72246806170154	0\\
2.72256806420161	0\\
2.72266806670167	0\\
2.72276806920173	0\\
2.72286807170179	0\\
2.72296807420186	0\\
2.72306807670192	0\\
2.72316807920198	0\\
2.72326808170204	0\\
2.72336808420211	0\\
2.72346808670217	0\\
2.72356808920223	0\\
2.72366809170229	0\\
2.72376809420235	0\\
2.72386809670242	0\\
2.72396809920248	0\\
2.72406810170254	0\\
2.7241681042026	0\\
2.72426810670267	0\\
2.72436810920273	0\\
2.72446811170279	0\\
2.72456811420286	0\\
2.72466811670292	0\\
2.72476811920298	0\\
2.72486812170304	0\\
2.72496812420311	0\\
2.72506812670317	0\\
2.72516812920323	0\\
2.72526813170329	0\\
2.72536813420336	0\\
2.72546813670342	0\\
2.72556813920348	0\\
2.72566814170354	0\\
2.72576814420361	0\\
2.72586814670367	0\\
2.72596814920373	0\\
2.72606815170379	0\\
2.72616815420385	0\\
2.72626815670392	0\\
2.72636815920398	0\\
2.72646816170404	0\\
2.72656816420411	0\\
2.72666816670417	0\\
2.72676816920423	0\\
2.72686817170429	0\\
2.72696817420436	0\\
2.72706817670442	0\\
2.72716817920448	0\\
2.72726818170454	0\\
2.72736818420461	0\\
2.72746818670467	0\\
2.72756818920473	0\\
2.72766819170479	0\\
2.72776819420486	0\\
2.72786819670492	0\\
2.72796819920498	0\\
2.72806820170504	0\\
2.7281682042051	0\\
2.72826820670517	0\\
2.72836820920523	0\\
2.72846821170529	0\\
2.72856821420536	0\\
2.72866821670542	0\\
2.72876821920548	0\\
2.72886822170554	0\\
2.72896822420561	0\\
2.72906822670567	0\\
2.72916822920573	0\\
2.72926823170579	0\\
2.72936823420586	0\\
2.72946823670592	0\\
2.72956823920598	0\\
2.72966824170604	0\\
2.72976824420611	0\\
2.72986824670617	0\\
2.72996824920623	0\\
2.73006825170629	0\\
2.73016825420636	0\\
2.73026825670642	0\\
2.73036825920648	0\\
2.73046826170654	0\\
2.73056826420661	0\\
2.73066826670667	0\\
2.73076826920673	0\\
2.73086827170679	0\\
2.73096827420686	0\\
2.73106827670692	0\\
2.73116827920698	0\\
2.73126828170704	0\\
2.73136828420711	0\\
2.73146828670717	0\\
2.73156828920723	0\\
2.73166829170729	0\\
2.73176829420736	0\\
2.73186829670742	0\\
2.73196829920748	0\\
2.73206830170754	0\\
2.73216830420761	0\\
2.73226830670767	0\\
2.73236830920773	0\\
2.73246831170779	0\\
2.73256831420786	0\\
2.73266831670792	0\\
2.73276831920798	0\\
2.73286832170804	0\\
2.73296832420811	0\\
2.73306832670817	0\\
2.73316832920823	0\\
2.73326833170829	0\\
2.73336833420836	0\\
2.73346833670842	0\\
2.73356833920848	0\\
2.73366834170854	0\\
2.73376834420861	0\\
2.73386834670867	0\\
2.73396834920873	0\\
2.73406835170879	0\\
2.73416835420886	0\\
2.73426835670892	0\\
2.73436835920898	0\\
2.73446836170904	0\\
2.73456836420911	0\\
2.73466836670917	0\\
2.73476836920923	0\\
2.73486837170929	0\\
2.73496837420936	0\\
2.73506837670942	0\\
2.73516837920948	0\\
2.73526838170954	0\\
2.73536838420961	0\\
2.73546838670967	0\\
2.73556838920973	0\\
2.73566839170979	0\\
2.73576839420986	0\\
2.73586839670992	0\\
2.73596839920998	0\\
2.73606840171004	0\\
2.73616840421011	0\\
2.73626840671017	0\\
2.73636840921023	0\\
2.73646841171029	0\\
2.73656841421036	0\\
2.73666841671042	0\\
2.73676841921048	0\\
2.73686842171054	0\\
2.73696842421061	0\\
2.73706842671067	0\\
2.73716842921073	0\\
2.73726843171079	0\\
2.73736843421086	0\\
2.73746843671092	0\\
2.73756843921098	0\\
2.73766844171104	0\\
2.73776844421111	0\\
2.73786844671117	0\\
2.73796844921123	0\\
2.73806845171129	0\\
2.73816845421136	0\\
2.73826845671142	0\\
2.73836845921148	0\\
2.73846846171154	0\\
2.73856846421161	0\\
2.73866846671167	0\\
2.73876846921173	0\\
2.73886847171179	0\\
2.73896847421186	0\\
2.73906847671192	0\\
2.73916847921198	0\\
2.73926848171204	0\\
2.73936848421211	0\\
2.73946848671217	0\\
2.73956848921223	0\\
2.73966849171229	0\\
2.73976849421236	0\\
2.73986849671242	0\\
2.73996849921248	0\\
2.74006850171254	0\\
2.74016850421261	0\\
2.74026850671267	0\\
2.74036850921273	0\\
2.74046851171279	0\\
2.74056851421286	0\\
2.74066851671292	0\\
2.74076851921298	0\\
2.74086852171304	0\\
2.74096852421311	0\\
2.74106852671317	0\\
2.74116852921323	0\\
2.74126853171329	0\\
2.74136853421336	0\\
2.74146853671342	0\\
2.74156853921348	0\\
2.74166854171354	0\\
2.74176854421361	0\\
2.74186854671367	0\\
2.74196854921373	0\\
2.74206855171379	0\\
2.74216855421386	0\\
2.74226855671392	0\\
2.74236855921398	0\\
2.74246856171404	0\\
2.74256856421411	0\\
2.74266856671417	0\\
2.74276856921423	0\\
2.74286857171429	0\\
2.74296857421436	0\\
2.74306857671442	0\\
2.74316857921448	0\\
2.74326858171454	0\\
2.74336858421461	0\\
2.74346858671467	0\\
2.74356858921473	0\\
2.74366859171479	0\\
2.74376859421486	0\\
2.74386859671492	0\\
2.74396859921498	0\\
2.74406860171504	0\\
2.74416860421511	0\\
2.74426860671517	0\\
2.74436860921523	0\\
2.74446861171529	0\\
2.74456861421536	0\\
2.74466861671542	0\\
2.74476861921548	0\\
2.74486862171554	0\\
2.74496862421561	0\\
2.74506862671567	0\\
2.74516862921573	0\\
2.74526863171579	0\\
2.74536863421586	0\\
2.74546863671592	0\\
2.74556863921598	0\\
2.74566864171604	0\\
2.74576864421611	0\\
2.74586864671617	0\\
2.74596864921623	0\\
2.74606865171629	0\\
2.74616865421636	0\\
2.74626865671642	0\\
2.74636865921648	0\\
2.74646866171654	0\\
2.74656866421661	0\\
2.74666866671667	0\\
2.74676866921673	0\\
2.74686867171679	0\\
2.74696867421686	0\\
2.74706867671692	0\\
2.74716867921698	0\\
2.74726868171704	0\\
2.74736868421711	0\\
2.74746868671717	0\\
2.74756868921723	0\\
2.74766869171729	0\\
2.74776869421736	0\\
2.74786869671742	0\\
2.74796869921748	0\\
2.74806870171754	0\\
2.74816870421761	0\\
2.74826870671767	0\\
2.74836870921773	0\\
2.74846871171779	0\\
2.74856871421786	0\\
2.74866871671792	0\\
2.74876871921798	0\\
2.74886872171804	0\\
2.74896872421811	0\\
2.74906872671817	0\\
2.74916872921823	0\\
2.74926873171829	0\\
2.74936873421836	0\\
2.74946873671842	0\\
2.74956873921848	0\\
2.74966874171854	0\\
2.74976874421861	0\\
2.74986874671867	0\\
2.74996874921873	0\\
2.75006875171879	0\\
2.75016875421886	0\\
2.75026875671892	0\\
2.75036875921898	0\\
2.75046876171904	0\\
2.75056876421911	0\\
2.75066876671917	0\\
2.75076876921923	0\\
2.75086877171929	0\\
2.75096877421936	0\\
2.75106877671942	0\\
2.75116877921948	0\\
2.75126878171954	0\\
2.75136878421961	0\\
2.75146878671967	0\\
2.75156878921973	0\\
2.75166879171979	0\\
2.75176879421986	0\\
2.75186879671992	0\\
2.75196879921998	0\\
2.75206880172004	0\\
2.75216880422011	0\\
2.75226880672017	0\\
2.75236880922023	0\\
2.75246881172029	0\\
2.75256881422036	0\\
2.75266881672042	0\\
2.75276881922048	0\\
2.75286882172054	0\\
2.75296882422061	0\\
2.75306882672067	0\\
2.75316882922073	0\\
2.75326883172079	0\\
2.75336883422086	0\\
2.75346883672092	0\\
2.75356883922098	0\\
2.75366884172104	0\\
2.75376884422111	0\\
2.75386884672117	0\\
2.75396884922123	0\\
2.75406885172129	0\\
2.75416885422136	0\\
2.75426885672142	0\\
2.75436885922148	0\\
2.75446886172154	0\\
2.75456886422161	0\\
2.75466886672167	0\\
2.75476886922173	0\\
2.75486887172179	0\\
2.75496887422186	0\\
2.75506887672192	0\\
2.75516887922198	0\\
2.75526888172204	0\\
2.75536888422211	0\\
2.75546888672217	0\\
2.75556888922223	0\\
2.75566889172229	0\\
2.75576889422236	0\\
2.75586889672242	0\\
2.75596889922248	0\\
2.75606890172254	0\\
2.75616890422261	0\\
2.75626890672267	0\\
2.75636890922273	0\\
2.75646891172279	0\\
2.75656891422286	0\\
2.75666891672292	0\\
2.75676891922298	0\\
2.75686892172304	0\\
2.75696892422311	0\\
2.75706892672317	0\\
2.75716892922323	0\\
2.75726893172329	0\\
2.75736893422336	0\\
2.75746893672342	0\\
2.75756893922348	0\\
2.75766894172354	0\\
2.75776894422361	0\\
2.75786894672367	0\\
2.75796894922373	0\\
2.75806895172379	0\\
2.75816895422386	0\\
2.75826895672392	0\\
2.75836895922398	0\\
2.75846896172404	0\\
2.75856896422411	0\\
2.75866896672417	0\\
2.75876896922423	0\\
2.75886897172429	0\\
2.75896897422436	0\\
2.75906897672442	0\\
2.75916897922448	0\\
2.75926898172454	0\\
2.75936898422461	0\\
2.75946898672467	0\\
2.75956898922473	0\\
2.75966899172479	0\\
2.75976899422486	0\\
2.75986899672492	0\\
2.75996899922498	0\\
2.76006900172504	0\\
2.76016900422511	0\\
2.76026900672517	0\\
2.76036900922523	0\\
2.76046901172529	0\\
2.76056901422536	0\\
2.76066901672542	0\\
2.76076901922548	0\\
2.76086902172554	0\\
2.76096902422561	0\\
2.76106902672567	0\\
2.76116902922573	0\\
2.76126903172579	0\\
2.76136903422586	0\\
2.76146903672592	0\\
2.76156903922598	0\\
2.76166904172604	0\\
2.76176904422611	0\\
2.76186904672617	0\\
2.76196904922623	0\\
2.76206905172629	0\\
2.76216905422636	0\\
2.76226905672642	0\\
2.76236905922648	0\\
2.76246906172654	0\\
2.76256906422661	0\\
2.76266906672667	0\\
2.76276906922673	0\\
2.76286907172679	0\\
2.76296907422686	0\\
2.76306907672692	0\\
2.76316907922698	0\\
2.76326908172704	0\\
2.76336908422711	0\\
2.76346908672717	0\\
2.76356908922723	0\\
2.76366909172729	0\\
2.76376909422736	0\\
2.76386909672742	0\\
2.76396909922748	0\\
2.76406910172754	0\\
2.76416910422761	0\\
2.76426910672767	0\\
2.76436910922773	0\\
2.76446911172779	0\\
2.76456911422786	0\\
2.76466911672792	0\\
2.76476911922798	0\\
2.76486912172804	0\\
2.76496912422811	0\\
2.76506912672817	0\\
2.76516912922823	0\\
2.76526913172829	0\\
2.76536913422836	0\\
2.76546913672842	0\\
2.76556913922848	0\\
2.76566914172854	0\\
2.76576914422861	0\\
2.76586914672867	0\\
2.76596914922873	0\\
2.76606915172879	0\\
2.76616915422886	0\\
2.76626915672892	0\\
2.76636915922898	0\\
2.76646916172904	0\\
2.76656916422911	0\\
2.76666916672917	0\\
2.76676916922923	0\\
2.76686917172929	0\\
2.76696917422936	0\\
2.76706917672942	0\\
2.76716917922948	0\\
2.76726918172954	0\\
2.76736918422961	0\\
2.76746918672967	0\\
2.76756918922973	0\\
2.76766919172979	0\\
2.76776919422986	0\\
2.76786919672992	0\\
2.76796919922998	0\\
2.76806920173004	0\\
2.76816920423011	0\\
2.76826920673017	0\\
2.76836920923023	0\\
2.76846921173029	0\\
2.76856921423036	0\\
2.76866921673042	0\\
2.76876921923048	0\\
2.76886922173054	0\\
2.76896922423061	0\\
2.76906922673067	0\\
2.76916922923073	0\\
2.76926923173079	0\\
2.76936923423086	0\\
2.76946923673092	0\\
2.76956923923098	0\\
2.76966924173104	0\\
2.76976924423111	0\\
2.76986924673117	0\\
2.76996924923123	0\\
2.77006925173129	0\\
2.77016925423136	0\\
2.77026925673142	0\\
2.77036925923148	0\\
2.77046926173154	0\\
2.77056926423161	0\\
2.77066926673167	0\\
2.77076926923173	0\\
2.77086927173179	0\\
2.77096927423186	0\\
2.77106927673192	0\\
2.77116927923198	0\\
2.77126928173204	0\\
2.77136928423211	0\\
2.77146928673217	0\\
2.77156928923223	0\\
2.77166929173229	0\\
2.77176929423236	0\\
2.77186929673242	0\\
2.77196929923248	0\\
2.77206930173254	0\\
2.77216930423261	0\\
2.77226930673267	0\\
2.77236930923273	0\\
2.77246931173279	0\\
2.77256931423286	0\\
2.77266931673292	0\\
2.77276931923298	0\\
2.77286932173304	0\\
2.77296932423311	0\\
2.77306932673317	0\\
2.77316932923323	0\\
2.77326933173329	0\\
2.77336933423336	0\\
2.77346933673342	0\\
2.77356933923348	0\\
2.77366934173354	0\\
2.77376934423361	0\\
2.77386934673367	0\\
2.77396934923373	0\\
2.77406935173379	0\\
2.77416935423386	0\\
2.77426935673392	0\\
2.77436935923398	0\\
2.77446936173404	0\\
2.77456936423411	0\\
2.77466936673417	0\\
2.77476936923423	0\\
2.77486937173429	0\\
2.77496937423436	0\\
2.77506937673442	0\\
2.77516937923448	0\\
2.77526938173454	0\\
2.77536938423461	0\\
2.77546938673467	0\\
2.77556938923473	0\\
2.77566939173479	0\\
2.77576939423486	0\\
2.77586939673492	0\\
2.77596939923498	0\\
2.77606940173504	0\\
2.77616940423511	0\\
2.77626940673517	0\\
2.77636940923523	0\\
2.77646941173529	0\\
2.77656941423536	0\\
2.77666941673542	0\\
2.77676941923548	0\\
2.77686942173554	0\\
2.77696942423561	0\\
2.77706942673567	0\\
2.77716942923573	0\\
2.77726943173579	0\\
2.77736943423586	0\\
2.77746943673592	0\\
2.77756943923598	0\\
2.77766944173604	0\\
2.77776944423611	0\\
2.77786944673617	0\\
2.77796944923623	0\\
2.77806945173629	0\\
2.77816945423636	0\\
2.77826945673642	0\\
2.77836945923648	0\\
2.77846946173654	0\\
2.77856946423661	0\\
2.77866946673667	0\\
2.77876946923673	0\\
2.77886947173679	0\\
2.77896947423686	0\\
2.77906947673692	0\\
2.77916947923698	0\\
2.77926948173704	0\\
2.77936948423711	0\\
2.77946948673717	0\\
2.77956948923723	0\\
2.77966949173729	0\\
2.77976949423736	0\\
2.77986949673742	0\\
2.77996949923748	0\\
2.78006950173754	0\\
2.78016950423761	0\\
2.78026950673767	0\\
2.78036950923773	0\\
2.78046951173779	0\\
2.78056951423786	0\\
2.78066951673792	0\\
2.78076951923798	0\\
2.78086952173804	0\\
2.78096952423811	0\\
2.78106952673817	0\\
2.78116952923823	0\\
2.78126953173829	0\\
2.78136953423836	0\\
2.78146953673842	0\\
2.78156953923848	0\\
2.78166954173854	0\\
2.78176954423861	0\\
2.78186954673867	0\\
2.78196954923873	0\\
2.78206955173879	0\\
2.78216955423886	0\\
2.78226955673892	0\\
2.78236955923898	0\\
2.78246956173904	0\\
2.78256956423911	0\\
2.78266956673917	0\\
2.78276956923923	0\\
2.78286957173929	0\\
2.78296957423936	0\\
2.78306957673942	0\\
2.78316957923948	0\\
2.78326958173954	0\\
2.78336958423961	0\\
2.78346958673967	0\\
2.78356958923973	0\\
2.78366959173979	0\\
2.78376959423986	0\\
2.78386959673992	0\\
2.78396959923998	0\\
2.78406960174004	0\\
2.78416960424011	0\\
2.78426960674017	0\\
2.78436960924023	0\\
2.78446961174029	0\\
2.78456961424036	0\\
2.78466961674042	0\\
2.78476961924048	0\\
2.78486962174054	0\\
2.78496962424061	0\\
2.78506962674067	0\\
2.78516962924073	0\\
2.78526963174079	0\\
2.78536963424086	0\\
2.78546963674092	0\\
2.78556963924098	0\\
2.78566964174104	0\\
2.78576964424111	0\\
2.78586964674117	0\\
2.78596964924123	0\\
2.78606965174129	0\\
2.78616965424136	0\\
2.78626965674142	0\\
2.78636965924148	0\\
2.78646966174154	0\\
2.78656966424161	0\\
2.78666966674167	0\\
2.78676966924173	0\\
2.78686967174179	0\\
2.78696967424186	0\\
2.78706967674192	0\\
2.78716967924198	0\\
2.78726968174204	0\\
2.78736968424211	0\\
2.78746968674217	0\\
2.78756968924223	0\\
2.78766969174229	0\\
2.78776969424236	0\\
2.78786969674242	0\\
2.78796969924248	0\\
2.78806970174254	0\\
2.78816970424261	0\\
2.78826970674267	0\\
2.78836970924273	0\\
2.78846971174279	0\\
2.78856971424286	0\\
2.78866971674292	0\\
2.78876971924298	0\\
2.78886972174304	0\\
2.78896972424311	0\\
2.78906972674317	0\\
2.78916972924323	0\\
2.78926973174329	0\\
2.78936973424336	0\\
2.78946973674342	0\\
2.78956973924348	0\\
2.78966974174354	0\\
2.78976974424361	0\\
2.78986974674367	0\\
2.78996974924373	0\\
2.79006975174379	0\\
2.79016975424386	0\\
2.79026975674392	0\\
2.79036975924398	0\\
2.79046976174404	0\\
2.79056976424411	0\\
2.79066976674417	0\\
2.79076976924423	0\\
2.79086977174429	0\\
2.79096977424436	0\\
2.79106977674442	0\\
2.79116977924448	0\\
2.79126978174454	0\\
2.79136978424461	0\\
2.79146978674467	0\\
2.79156978924473	0\\
2.79166979174479	0\\
2.79176979424486	0\\
2.79186979674492	0\\
2.79196979924498	0\\
2.79206980174504	0\\
2.79216980424511	0\\
2.79226980674517	0\\
2.79236980924523	0\\
2.79246981174529	0\\
2.79256981424536	0\\
2.79266981674542	0\\
2.79276981924548	0\\
2.79286982174554	0\\
2.79296982424561	0\\
2.79306982674567	0\\
2.79316982924573	0\\
2.79326983174579	0\\
2.79336983424586	0\\
2.79346983674592	0\\
2.79356983924598	0\\
2.79366984174604	0\\
2.79376984424611	0\\
2.79386984674617	0\\
2.79396984924623	0\\
2.79406985174629	0\\
2.79416985424636	0\\
2.79426985674642	0\\
2.79436985924648	0\\
2.79446986174654	0\\
2.79456986424661	0\\
2.79466986674667	0\\
2.79476986924673	0\\
2.79486987174679	0\\
2.79496987424686	0\\
2.79506987674692	0\\
2.79516987924698	0\\
2.79526988174704	0\\
2.79536988424711	0\\
2.79546988674717	0\\
2.79556988924723	0\\
2.79566989174729	0\\
2.79576989424736	0\\
2.79586989674742	0\\
2.79596989924748	0\\
2.79606990174754	0\\
2.79616990424761	0\\
2.79626990674767	0\\
2.79636990924773	0\\
2.79646991174779	0\\
2.79656991424786	0\\
2.79666991674792	0\\
2.79676991924798	0\\
2.79686992174804	0\\
2.79696992424811	0\\
2.79706992674817	0\\
2.79716992924823	0\\
2.79726993174829	0\\
2.79736993424836	0\\
2.79746993674842	0\\
2.79756993924848	0\\
2.79766994174854	0\\
2.79776994424861	0\\
2.79786994674867	0\\
2.79796994924873	0\\
2.79806995174879	0\\
2.79816995424886	0\\
2.79826995674892	0\\
2.79836995924898	0\\
2.79846996174904	0\\
2.79856996424911	0\\
2.79866996674917	0\\
2.79876996924923	0\\
2.79886997174929	0\\
2.79896997424936	0\\
2.79906997674942	0\\
2.79916997924948	0\\
2.79926998174954	0\\
2.79936998424961	0\\
2.79946998674967	0\\
2.79956998924973	0\\
2.79966999174979	0\\
2.79976999424986	0\\
2.79986999674992	0\\
2.79996999924998	0\\
2.80007000175004	0\\
};
\addplot [color=mycolor2,solid,forget plot]
  table[row sep=crcr]{%
2.80007000175004	0\\
2.80017000425011	0\\
2.80027000675017	0\\
2.80037000925023	0\\
2.80047001175029	0\\
2.80057001425036	0\\
2.80067001675042	0\\
2.80077001925048	0\\
2.80087002175054	0\\
2.80097002425061	0\\
2.80107002675067	0\\
2.80117002925073	0\\
2.80127003175079	0\\
2.80137003425086	0\\
2.80147003675092	0\\
2.80157003925098	0\\
2.80167004175104	0\\
2.80177004425111	0\\
2.80187004675117	0\\
2.80197004925123	0\\
2.80207005175129	0\\
2.80217005425136	0\\
2.80227005675142	0\\
2.80237005925148	0\\
2.80247006175154	0\\
2.80257006425161	0\\
2.80267006675167	0\\
2.80277006925173	0\\
2.80287007175179	0\\
2.80297007425186	0\\
2.80307007675192	0\\
2.80317007925198	0\\
2.80327008175204	0\\
2.80337008425211	0\\
2.80347008675217	0\\
2.80357008925223	0\\
2.80367009175229	0\\
2.80377009425236	0\\
2.80387009675242	0\\
2.80397009925248	0\\
2.80407010175254	0\\
2.80417010425261	0\\
2.80427010675267	0\\
2.80437010925273	0\\
2.80447011175279	0\\
2.80457011425286	0\\
2.80467011675292	0\\
2.80477011925298	0\\
2.80487012175304	0\\
2.80497012425311	0\\
2.80507012675317	0\\
2.80517012925323	0\\
2.80527013175329	0\\
2.80537013425336	0\\
2.80547013675342	0\\
2.80557013925348	0\\
2.80567014175354	0\\
2.80577014425361	0\\
2.80587014675367	0\\
2.80597014925373	0\\
2.80607015175379	0\\
2.80617015425386	0\\
2.80627015675392	0\\
2.80637015925398	0\\
2.80647016175404	0\\
2.80657016425411	0\\
2.80667016675417	0\\
2.80677016925423	0\\
2.80687017175429	0\\
2.80697017425436	0\\
2.80707017675442	0\\
2.80717017925448	0\\
2.80727018175454	0\\
2.80737018425461	0\\
2.80747018675467	0\\
2.80757018925473	0\\
2.80767019175479	0\\
2.80777019425486	0\\
2.80787019675492	0\\
2.80797019925498	0\\
2.80807020175504	0\\
2.80817020425511	0\\
2.80827020675517	0\\
2.80837020925523	0\\
2.80847021175529	0\\
2.80857021425536	0\\
2.80867021675542	0\\
2.80877021925548	0\\
2.80887022175554	0\\
2.80897022425561	0\\
2.80907022675567	0\\
2.80917022925573	0\\
2.80927023175579	0\\
2.80937023425586	0\\
2.80947023675592	0\\
2.80957023925598	0\\
2.80967024175604	0\\
2.80977024425611	0\\
2.80987024675617	0\\
2.80997024925623	0\\
2.81007025175629	0\\
2.81017025425636	0\\
2.81027025675642	0\\
2.81037025925648	0\\
2.81047026175654	0\\
2.81057026425661	0\\
2.81067026675667	0\\
2.81077026925673	0\\
2.81087027175679	0\\
2.81097027425686	0\\
2.81107027675692	0\\
2.81117027925698	0\\
2.81127028175704	0\\
2.81137028425711	0\\
2.81147028675717	0\\
2.81157028925723	0\\
2.81167029175729	0\\
2.81177029425736	0\\
2.81187029675742	0\\
2.81197029925748	0\\
2.81207030175754	0\\
2.81217030425761	0\\
2.81227030675767	0\\
2.81237030925773	0\\
2.81247031175779	0\\
2.81257031425786	0\\
2.81267031675792	0\\
2.81277031925798	0\\
2.81287032175804	0\\
2.81297032425811	0\\
2.81307032675817	0\\
2.81317032925823	0\\
2.81327033175829	0\\
2.81337033425836	0\\
2.81347033675842	0\\
2.81357033925848	0\\
2.81367034175854	0\\
2.81377034425861	0\\
2.81387034675867	0\\
2.81397034925873	0\\
2.81407035175879	0\\
2.81417035425886	0\\
2.81427035675892	0\\
2.81437035925898	0\\
2.81447036175904	0\\
2.81457036425911	0\\
2.81467036675917	0\\
2.81477036925923	0\\
2.81487037175929	0\\
2.81497037425936	0\\
2.81507037675942	0\\
2.81517037925948	0\\
2.81527038175954	0\\
2.81537038425961	0\\
2.81547038675967	0\\
2.81557038925973	0\\
2.81567039175979	0\\
2.81577039425986	0\\
2.81587039675992	0\\
2.81597039925998	0\\
2.81607040176004	0\\
2.81617040426011	0\\
2.81627040676017	0\\
2.81637040926023	0\\
2.81647041176029	0\\
2.81657041426036	0\\
2.81667041676042	0\\
2.81677041926048	0\\
2.81687042176054	0\\
2.81697042426061	0\\
2.81707042676067	0\\
2.81717042926073	0\\
2.81727043176079	0\\
2.81737043426086	0\\
2.81747043676092	0\\
2.81757043926098	0\\
2.81767044176104	0\\
2.81777044426111	0\\
2.81787044676117	0\\
2.81797044926123	0\\
2.81807045176129	0\\
2.81817045426136	0\\
2.81827045676142	0\\
2.81837045926148	0\\
2.81847046176154	0\\
2.81857046426161	0\\
2.81867046676167	0\\
2.81877046926173	0\\
2.81887047176179	0\\
2.81897047426186	0\\
2.81907047676192	0\\
2.81917047926198	0\\
2.81927048176204	0\\
2.81937048426211	0\\
2.81947048676217	0\\
2.81957048926223	0\\
2.81967049176229	0\\
2.81977049426236	0\\
2.81987049676242	0\\
2.81997049926248	0\\
2.82007050176254	0\\
2.82017050426261	0\\
2.82027050676267	0\\
2.82037050926273	0\\
2.82047051176279	0\\
2.82057051426286	0\\
2.82067051676292	0\\
2.82077051926298	0\\
2.82087052176304	0\\
2.82097052426311	0\\
2.82107052676317	0\\
2.82117052926323	0\\
2.82127053176329	0\\
2.82137053426336	0\\
2.82147053676342	0\\
2.82157053926348	0\\
2.82167054176354	0\\
2.82177054426361	0\\
2.82187054676367	0\\
2.82197054926373	0\\
2.82207055176379	0\\
2.82217055426386	0\\
2.82227055676392	0\\
2.82237055926398	0\\
2.82247056176404	0\\
2.82257056426411	0\\
2.82267056676417	0\\
2.82277056926423	0\\
2.82287057176429	0\\
2.82297057426436	0\\
2.82307057676442	0\\
2.82317057926448	0\\
2.82327058176454	0\\
2.82337058426461	0\\
2.82347058676467	0\\
2.82357058926473	0\\
2.82367059176479	0\\
2.82377059426486	0\\
2.82387059676492	0\\
2.82397059926498	0\\
2.82407060176504	0\\
2.82417060426511	0\\
2.82427060676517	0\\
2.82437060926523	0\\
2.82447061176529	0\\
2.82457061426536	0\\
2.82467061676542	0\\
2.82477061926548	0\\
2.82487062176554	0\\
2.82497062426561	0\\
2.82507062676567	0\\
2.82517062926573	0\\
2.82527063176579	0\\
2.82537063426586	0\\
2.82547063676592	0\\
2.82557063926598	0\\
2.82567064176604	0\\
2.82577064426611	0\\
2.82587064676617	0\\
2.82597064926623	0\\
2.82607065176629	0\\
2.82617065426636	0\\
2.82627065676642	0\\
2.82637065926648	0\\
2.82647066176654	0\\
2.82657066426661	0\\
2.82667066676667	0\\
2.82677066926673	0\\
2.82687067176679	0\\
2.82697067426686	0\\
2.82707067676692	0\\
2.82717067926698	0\\
2.82727068176704	0\\
2.82737068426711	0\\
2.82747068676717	0\\
2.82757068926723	0\\
2.82767069176729	0\\
2.82777069426736	0\\
2.82787069676742	0\\
2.82797069926748	0\\
2.82807070176754	0\\
2.82817070426761	0\\
2.82827070676767	0\\
2.82837070926773	0\\
2.82847071176779	0\\
2.82857071426786	0\\
2.82867071676792	0\\
2.82877071926798	0\\
2.82887072176804	0\\
2.82897072426811	0\\
2.82907072676817	0\\
2.82917072926823	0\\
2.82927073176829	0\\
2.82937073426836	0\\
2.82947073676842	0\\
2.82957073926848	0\\
2.82967074176854	0\\
2.82977074426861	0\\
2.82987074676867	0\\
2.82997074926873	0\\
2.83007075176879	0\\
2.83017075426886	0\\
2.83027075676892	0\\
2.83037075926898	0\\
2.83047076176904	0\\
2.83057076426911	0\\
2.83067076676917	0\\
2.83077076926923	0\\
2.83087077176929	0\\
2.83097077426936	0\\
2.83107077676942	0\\
2.83117077926948	0\\
2.83127078176954	0\\
2.83137078426961	0\\
2.83147078676967	0\\
2.83157078926973	0\\
2.83167079176979	0\\
2.83177079426986	0\\
2.83187079676992	0\\
2.83197079926998	0\\
2.83207080177004	0\\
2.83217080427011	0\\
2.83227080677017	0\\
2.83237080927023	0\\
2.83247081177029	0\\
2.83257081427036	0\\
2.83267081677042	0\\
2.83277081927048	0\\
2.83287082177054	0\\
2.83297082427061	0\\
2.83307082677067	0\\
2.83317082927073	0\\
2.83327083177079	0\\
2.83337083427086	0\\
2.83347083677092	0\\
2.83357083927098	0\\
2.83367084177104	0\\
2.83377084427111	0\\
2.83387084677117	0\\
2.83397084927123	0\\
2.83407085177129	0\\
2.83417085427136	0\\
2.83427085677142	0\\
2.83437085927148	0\\
2.83447086177154	0\\
2.83457086427161	0\\
2.83467086677167	0\\
2.83477086927173	0\\
2.83487087177179	0\\
2.83497087427186	0\\
2.83507087677192	0\\
2.83517087927198	0\\
2.83527088177204	0\\
2.83537088427211	0\\
2.83547088677217	0\\
2.83557088927223	0\\
2.83567089177229	0\\
2.83577089427236	0\\
2.83587089677242	0\\
2.83597089927248	0\\
2.83607090177254	0\\
2.83617090427261	0\\
2.83627090677267	0\\
2.83637090927273	0\\
2.83647091177279	0\\
2.83657091427286	0\\
2.83667091677292	0\\
2.83677091927298	0\\
2.83687092177304	0\\
2.83697092427311	0\\
2.83707092677317	0\\
2.83717092927323	0\\
2.83727093177329	0\\
2.83737093427336	0\\
2.83747093677342	0\\
2.83757093927348	0\\
2.83767094177354	0\\
2.83777094427361	0\\
2.83787094677367	0\\
2.83797094927373	0\\
2.83807095177379	0\\
2.83817095427386	0\\
2.83827095677392	0\\
2.83837095927398	0\\
2.83847096177404	0\\
2.83857096427411	0\\
2.83867096677417	0\\
2.83877096927423	0\\
2.83887097177429	0\\
2.83897097427436	0\\
2.83907097677442	0\\
2.83917097927448	0\\
2.83927098177454	0\\
2.83937098427461	0\\
2.83947098677467	0\\
2.83957098927473	0\\
2.83967099177479	0\\
2.83977099427486	0\\
2.83987099677492	0\\
2.83997099927498	0\\
2.84007100177504	0\\
2.84017100427511	0\\
2.84027100677517	0\\
2.84037100927523	0\\
2.84047101177529	0\\
2.84057101427536	0\\
2.84067101677542	0\\
2.84077101927548	0\\
2.84087102177554	0\\
2.84097102427561	0\\
2.84107102677567	0\\
2.84117102927573	0\\
2.84127103177579	0\\
2.84137103427586	0\\
2.84147103677592	0\\
2.84157103927598	0\\
2.84167104177604	0\\
2.84177104427611	0\\
2.84187104677617	0\\
2.84197104927623	0\\
2.84207105177629	0\\
2.84217105427636	0\\
2.84227105677642	0\\
2.84237105927648	0\\
2.84247106177654	0\\
2.84257106427661	0\\
2.84267106677667	0\\
2.84277106927673	0\\
2.84287107177679	0\\
2.84297107427686	0\\
2.84307107677692	0\\
2.84317107927698	0\\
2.84327108177704	0\\
2.84337108427711	0\\
2.84347108677717	0\\
2.84357108927723	0\\
2.84367109177729	0\\
2.84377109427736	0\\
2.84387109677742	0\\
2.84397109927748	0\\
2.84407110177754	0\\
2.84417110427761	0\\
2.84427110677767	0\\
2.84437110927773	0\\
2.84447111177779	0\\
2.84457111427786	0\\
2.84467111677792	0\\
2.84477111927798	0\\
2.84487112177804	0\\
2.84497112427811	0\\
2.84507112677817	0\\
2.84517112927823	0\\
2.84527113177829	0\\
2.84537113427836	0\\
2.84547113677842	0\\
2.84557113927848	0\\
2.84567114177854	0\\
2.84577114427861	0\\
2.84587114677867	0\\
2.84597114927873	0\\
2.84607115177879	0\\
2.84617115427886	0\\
2.84627115677892	0\\
2.84637115927898	0\\
2.84647116177904	0\\
2.84657116427911	0\\
2.84667116677917	0\\
2.84677116927923	0\\
2.84687117177929	0\\
2.84697117427936	0\\
2.84707117677942	0\\
2.84717117927948	0\\
2.84727118177954	0\\
2.84737118427961	0\\
2.84747118677967	0\\
2.84757118927973	0\\
2.84767119177979	0\\
2.84777119427986	0\\
2.84787119677992	0\\
2.84797119927998	0\\
2.84807120178004	0\\
2.84817120428011	0\\
2.84827120678017	0\\
2.84837120928023	0\\
2.84847121178029	0\\
2.84857121428036	0\\
2.84867121678042	0\\
2.84877121928048	0\\
2.84887122178054	0\\
2.84897122428061	0\\
2.84907122678067	0\\
2.84917122928073	0\\
2.84927123178079	0\\
2.84937123428086	0\\
2.84947123678092	0\\
2.84957123928098	0\\
2.84967124178104	0\\
2.84977124428111	0\\
2.84987124678117	0\\
2.84997124928123	0\\
2.85007125178129	0\\
2.85017125428136	0\\
2.85027125678142	0\\
2.85037125928148	0\\
2.85047126178154	0\\
2.85057126428161	0\\
2.85067126678167	0\\
2.85077126928173	0\\
2.85087127178179	0\\
2.85097127428186	0\\
2.85107127678192	0\\
2.85117127928198	0\\
2.85127128178204	0\\
2.85137128428211	0\\
2.85147128678217	0\\
2.85157128928223	0\\
2.85167129178229	0\\
2.85177129428236	0\\
2.85187129678242	0\\
2.85197129928248	0\\
2.85207130178254	0\\
2.85217130428261	0\\
2.85227130678267	0\\
2.85237130928273	0\\
2.85247131178279	0\\
2.85257131428286	0\\
2.85267131678292	0\\
2.85277131928298	0\\
2.85287132178304	0\\
2.85297132428311	0\\
2.85307132678317	0\\
2.85317132928323	0\\
2.85327133178329	0\\
2.85337133428336	0\\
2.85347133678342	0\\
2.85357133928348	0\\
2.85367134178354	0\\
2.85377134428361	0\\
2.85387134678367	0\\
2.85397134928373	0\\
2.85407135178379	0\\
2.85417135428386	0\\
2.85427135678392	0\\
2.85437135928398	0\\
2.85447136178404	0\\
2.85457136428411	0\\
2.85467136678417	0\\
2.85477136928423	0\\
2.85487137178429	0\\
2.85497137428436	0\\
2.85507137678442	0\\
2.85517137928448	0\\
2.85527138178454	0\\
2.85537138428461	0\\
2.85547138678467	0\\
2.85557138928473	0\\
2.85567139178479	0\\
2.85577139428486	0\\
2.85587139678492	0\\
2.85597139928498	0\\
2.85607140178504	0\\
2.85617140428511	0\\
2.85627140678517	0\\
2.85637140928523	0\\
2.85647141178529	0\\
2.85657141428536	0\\
2.85667141678542	0\\
2.85677141928548	0\\
2.85687142178554	0\\
2.85697142428561	0\\
2.85707142678567	0\\
2.85717142928573	0\\
2.85727143178579	0\\
2.85737143428586	0\\
2.85747143678592	0\\
2.85757143928598	0\\
2.85767144178604	0\\
2.85777144428611	0\\
2.85787144678617	0\\
2.85797144928623	0\\
2.85807145178629	0\\
2.85817145428636	0\\
2.85827145678642	0\\
2.85837145928648	0\\
2.85847146178654	0\\
2.85857146428661	0\\
2.85867146678667	0\\
2.85877146928673	0\\
2.85887147178679	0\\
2.85897147428686	0\\
2.85907147678692	0\\
2.85917147928698	0\\
2.85927148178704	0\\
2.85937148428711	0\\
2.85947148678717	0\\
2.85957148928723	0\\
2.85967149178729	0\\
2.85977149428736	0\\
2.85987149678742	0\\
2.85997149928748	0\\
2.86007150178754	0\\
2.86017150428761	0\\
2.86027150678767	0\\
2.86037150928773	0\\
2.86047151178779	0\\
2.86057151428786	0\\
2.86067151678792	0\\
2.86077151928798	0\\
2.86087152178804	0\\
2.86097152428811	0\\
2.86107152678817	0\\
2.86117152928823	0\\
2.86127153178829	0\\
2.86137153428836	0\\
2.86147153678842	0\\
2.86157153928848	0\\
2.86167154178854	0\\
2.86177154428861	0\\
2.86187154678867	0\\
2.86197154928873	0\\
2.86207155178879	0\\
2.86217155428886	0\\
2.86227155678892	0\\
2.86237155928898	0\\
2.86247156178904	0\\
2.86257156428911	0\\
2.86267156678917	0\\
2.86277156928923	0\\
2.86287157178929	0\\
2.86297157428936	0\\
2.86307157678942	0\\
2.86317157928948	0\\
2.86327158178954	0\\
2.86337158428961	0\\
2.86347158678967	0\\
2.86357158928973	0\\
2.86367159178979	0\\
2.86377159428986	0\\
2.86387159678992	0\\
2.86397159928998	0\\
2.86407160179004	0\\
2.86417160429011	0\\
2.86427160679017	0\\
2.86437160929023	0\\
2.86447161179029	0\\
2.86457161429036	0\\
2.86467161679042	0\\
2.86477161929048	0\\
2.86487162179054	0\\
2.86497162429061	0\\
2.86507162679067	0\\
2.86517162929073	0\\
2.86527163179079	0\\
2.86537163429086	0\\
2.86547163679092	0\\
2.86557163929098	0\\
2.86567164179104	0\\
2.86577164429111	0\\
2.86587164679117	0\\
2.86597164929123	0\\
2.86607165179129	0\\
2.86617165429136	0\\
2.86627165679142	0\\
2.86637165929148	0\\
2.86647166179154	0\\
2.86657166429161	0\\
2.86667166679167	0\\
2.86677166929173	0\\
2.86687167179179	0\\
2.86697167429186	0\\
2.86707167679192	0\\
2.86717167929198	0\\
2.86727168179204	0\\
2.86737168429211	0\\
2.86747168679217	0\\
2.86757168929223	0\\
2.86767169179229	0\\
2.86777169429236	0\\
2.86787169679242	0\\
2.86797169929248	0\\
2.86807170179255	0\\
2.86817170429261	0\\
2.86827170679267	0\\
2.86837170929273	0\\
2.86847171179279	0\\
2.86857171429286	0\\
2.86867171679292	0\\
2.86877171929298	0\\
2.86887172179304	0\\
2.86897172429311	0\\
2.86907172679317	0\\
2.86917172929323	0\\
2.86927173179329	0\\
2.86937173429336	0\\
2.86947173679342	0\\
2.86957173929348	0\\
2.86967174179354	0\\
2.86977174429361	0\\
2.86987174679367	0\\
2.86997174929373	0\\
2.8700717517938	0\\
2.87017175429386	0\\
2.87027175679392	0\\
2.87037175929398	0\\
2.87047176179404	0\\
2.87057176429411	0\\
2.87067176679417	0\\
2.87077176929423	0\\
2.87087177179429	0\\
2.87097177429436	0\\
2.87107177679442	0\\
2.87117177929448	0\\
2.87127178179454	0\\
2.87137178429461	0\\
2.87147178679467	0\\
2.87157178929473	0\\
2.87167179179479	0\\
2.87177179429486	0\\
2.87187179679492	0\\
2.87197179929498	0\\
2.87207180179505	0\\
2.87217180429511	0\\
2.87227180679517	0\\
2.87237180929523	0\\
2.87247181179529	0\\
2.87257181429536	0\\
2.87267181679542	0\\
2.87277181929548	0\\
2.87287182179554	0\\
2.87297182429561	0\\
2.87307182679567	0\\
2.87317182929573	0\\
2.87327183179579	0\\
2.87337183429586	0\\
2.87347183679592	0\\
2.87357183929598	0\\
2.87367184179604	0\\
2.87377184429611	0\\
2.87387184679617	0\\
2.87397184929623	0\\
2.8740718517963	0\\
2.87417185429636	0\\
2.87427185679642	0\\
2.87437185929648	0\\
2.87447186179655	0\\
2.87457186429661	0\\
2.87467186679667	0\\
2.87477186929673	0\\
2.87487187179679	0\\
2.87497187429686	0\\
2.87507187679692	0\\
2.87517187929698	0\\
2.87527188179704	0\\
2.87537188429711	0\\
2.87547188679717	0\\
2.87557188929723	0\\
2.87567189179729	0\\
2.87577189429736	0\\
2.87587189679742	0\\
2.87597189929748	0\\
2.87607190179755	0\\
2.87617190429761	0\\
2.87627190679767	0\\
2.87637190929773	0\\
2.8764719117978	0\\
2.87657191429786	0\\
2.87667191679792	0\\
2.87677191929798	0\\
2.87687192179804	0\\
2.87697192429811	0\\
2.87707192679817	0\\
2.87717192929823	0\\
2.87727193179829	0\\
2.87737193429836	0\\
2.87747193679842	0\\
2.87757193929848	0\\
2.87767194179854	0\\
2.87777194429861	0\\
2.87787194679867	0\\
2.87797194929873	0\\
2.8780719517988	0\\
2.87817195429886	0\\
2.87827195679892	0\\
2.87837195929898	0\\
2.87847196179905	0\\
2.87857196429911	0\\
2.87867196679917	0\\
2.87877196929923	0\\
2.87887197179929	0\\
2.87897197429936	0\\
2.87907197679942	0\\
2.87917197929948	0\\
2.87927198179954	0\\
2.87937198429961	0\\
2.87947198679967	0\\
2.87957198929973	0\\
2.87967199179979	0\\
2.87977199429986	0\\
2.87987199679992	0\\
2.87997199929998	0\\
2.88007200180005	0\\
2.88017200430011	0\\
2.88027200680017	0\\
2.88037200930023	0\\
2.8804720118003	0\\
2.88057201430036	0\\
2.88067201680042	0\\
2.88077201930048	0\\
2.88087202180055	0\\
2.88097202430061	0\\
2.88107202680067	0\\
2.88117202930073	0\\
2.88127203180079	0\\
2.88137203430086	0\\
2.88147203680092	0\\
2.88157203930098	0\\
2.88167204180104	0\\
2.88177204430111	0\\
2.88187204680117	0\\
2.88197204930123	0\\
2.8820720518013	0\\
2.88217205430136	0\\
2.88227205680142	0\\
2.88237205930148	0\\
2.88247206180155	0\\
2.88257206430161	0\\
2.88267206680167	0\\
2.88277206930173	0\\
2.8828720718018	0\\
2.88297207430186	0\\
2.88307207680192	0\\
2.88317207930198	0\\
2.88327208180204	0\\
2.88337208430211	0\\
2.88347208680217	0\\
2.88357208930223	0\\
2.88367209180229	0\\
2.88377209430236	0\\
2.88387209680242	0\\
2.88397209930248	0\\
2.88407210180255	0\\
2.88417210430261	0\\
2.88427210680267	0\\
2.88437210930273	0\\
2.8844721118028	0\\
2.88457211430286	0\\
2.88467211680292	0\\
2.88477211930298	0\\
2.88487212180305	0\\
2.88497212430311	0\\
2.88507212680317	0\\
2.88517212930323	0\\
2.8852721318033	0\\
2.88537213430336	0\\
2.88547213680342	0\\
2.88557213930348	0\\
2.88567214180354	0\\
2.88577214430361	0\\
2.88587214680367	0\\
2.88597214930373	0\\
2.8860721518038	0\\
2.88617215430386	0\\
2.88627215680392	0\\
2.88637215930398	0\\
2.88647216180405	0\\
2.88657216430411	0\\
2.88667216680417	0\\
2.88677216930423	0\\
2.8868721718043	0\\
2.88697217430436	0\\
2.88707217680442	0\\
2.88717217930448	0\\
2.88727218180455	0\\
2.88737218430461	0\\
2.88747218680467	0\\
2.88757218930473	0\\
2.88767219180479	0\\
2.88777219430486	0\\
2.88787219680492	0\\
2.88797219930498	0\\
2.88807220180505	0\\
2.88817220430511	0\\
2.88827220680517	0\\
2.88837220930523	0\\
2.8884722118053	0\\
2.88857221430536	0\\
2.88867221680542	0\\
2.88877221930548	0\\
2.88887222180555	0\\
2.88897222430561	0\\
2.88907222680567	0\\
2.88917222930573	0\\
2.8892722318058	0\\
2.88937223430586	0\\
2.88947223680592	0\\
2.88957223930598	0\\
2.88967224180604	0\\
2.88977224430611	0\\
2.88987224680617	0\\
2.88997224930623	0\\
2.8900722518063	0\\
2.89017225430636	0\\
2.89027225680642	0\\
2.89037225930648	0\\
2.89047226180655	0\\
2.89057226430661	0\\
2.89067226680667	0\\
2.89077226930673	0\\
2.8908722718068	0\\
2.89097227430686	0\\
2.89107227680692	0\\
2.89117227930698	0\\
2.89127228180705	0\\
2.89137228430711	0\\
2.89147228680717	0\\
2.89157228930723	0\\
2.8916722918073	0\\
2.89177229430736	0\\
2.89187229680742	0\\
2.89197229930748	0\\
2.89207230180755	0\\
2.89217230430761	0\\
2.89227230680767	0\\
2.89237230930773	0\\
2.8924723118078	0\\
2.89257231430786	0\\
2.89267231680792	0\\
2.89277231930798	0\\
2.89287232180805	0\\
2.89297232430811	0\\
2.89307232680817	0\\
2.89317232930823	0\\
2.8932723318083	0\\
2.89337233430836	0\\
2.89347233680842	0\\
2.89357233930848	0\\
2.89367234180855	0\\
2.89377234430861	0\\
2.89387234680867	0\\
2.89397234930873	0\\
2.8940723518088	0\\
2.89417235430886	0\\
2.89427235680892	0\\
2.89437235930898	0\\
2.89447236180905	0\\
2.89457236430911	0\\
2.89467236680917	0\\
2.89477236930923	0\\
2.8948723718093	0\\
2.89497237430936	0\\
2.89507237680942	0\\
2.89517237930948	0\\
2.89527238180955	0\\
2.89537238430961	0\\
2.89547238680967	0\\
2.89557238930973	0\\
2.8956723918098	0\\
2.89577239430986	0\\
2.89587239680992	0\\
2.89597239930998	0\\
2.89607240181005	0\\
2.89617240431011	0\\
2.89627240681017	0\\
2.89637240931023	0\\
2.8964724118103	0\\
2.89657241431036	0\\
2.89667241681042	0\\
2.89677241931048	0\\
2.89687242181055	0\\
2.89697242431061	0\\
2.89707242681067	0\\
2.89717242931073	0\\
2.8972724318108	0\\
2.89737243431086	0\\
2.89747243681092	0\\
2.89757243931098	0\\
2.89767244181105	0\\
2.89777244431111	0\\
2.89787244681117	0\\
2.89797244931123	0\\
2.8980724518113	0\\
2.89817245431136	0\\
2.89827245681142	0\\
2.89837245931148	0\\
2.89847246181155	0\\
2.89857246431161	0\\
2.89867246681167	0\\
2.89877246931173	0\\
2.8988724718118	0\\
2.89897247431186	0\\
2.89907247681192	0\\
2.89917247931198	0\\
2.89927248181205	0\\
2.89937248431211	0\\
2.89947248681217	0\\
2.89957248931223	0\\
2.8996724918123	0\\
2.89977249431236	0\\
2.89987249681242	0\\
2.89997249931248	0\\
2.90007250181255	0\\
2.90017250431261	0\\
2.90027250681267	0\\
2.90037250931273	0\\
2.9004725118128	0\\
2.90057251431286	0\\
2.90067251681292	0\\
2.90077251931298	0\\
2.90087252181305	0\\
2.90097252431311	0\\
2.90107252681317	0\\
2.90117252931323	0\\
2.9012725318133	0\\
2.90137253431336	0\\
2.90147253681342	0\\
2.90157253931348	0\\
2.90167254181355	0\\
2.90177254431361	0\\
2.90187254681367	0\\
2.90197254931373	0\\
2.9020725518138	0\\
2.90217255431386	0\\
2.90227255681392	0\\
2.90237255931398	0\\
2.90247256181405	0\\
2.90257256431411	0\\
2.90267256681417	0\\
2.90277256931423	0\\
2.9028725718143	0\\
2.90297257431436	0\\
2.90307257681442	0\\
2.90317257931448	0\\
2.90327258181455	0\\
2.90337258431461	0\\
2.90347258681467	0\\
2.90357258931473	0\\
2.9036725918148	0\\
2.90377259431486	0\\
2.90387259681492	0\\
2.90397259931498	0\\
2.90407260181505	0\\
2.90417260431511	0\\
2.90427260681517	0\\
2.90437260931523	0\\
2.9044726118153	0\\
2.90457261431536	0\\
2.90467261681542	0\\
2.90477261931548	0\\
2.90487262181555	0\\
2.90497262431561	0\\
2.90507262681567	0\\
2.90517262931573	0\\
2.9052726318158	0\\
2.90537263431586	0\\
2.90547263681592	0\\
2.90557263931598	0\\
2.90567264181605	0\\
2.90577264431611	0\\
2.90587264681617	0\\
2.90597264931623	0\\
2.9060726518163	0\\
2.90617265431636	0\\
2.90627265681642	0\\
2.90637265931648	0\\
2.90647266181655	0\\
2.90657266431661	0\\
2.90667266681667	0\\
2.90677266931673	0\\
2.9068726718168	0\\
2.90697267431686	0\\
2.90707267681692	0\\
2.90717267931698	0\\
2.90727268181705	0\\
2.90737268431711	0\\
2.90747268681717	0\\
2.90757268931723	0\\
2.9076726918173	0\\
2.90777269431736	0\\
2.90787269681742	0\\
2.90797269931748	0\\
2.90807270181755	0\\
2.90817270431761	0\\
2.90827270681767	0\\
2.90837270931773	0\\
2.9084727118178	0\\
2.90857271431786	0\\
2.90867271681792	0\\
2.90877271931798	0\\
2.90887272181805	0\\
2.90897272431811	0\\
2.90907272681817	0\\
2.90917272931823	0\\
2.9092727318183	0\\
2.90937273431836	0\\
2.90947273681842	0\\
2.90957273931848	0\\
2.90967274181855	0\\
2.90977274431861	0\\
2.90987274681867	0\\
2.90997274931873	0\\
2.9100727518188	0\\
2.91017275431886	0\\
2.91027275681892	0\\
2.91037275931898	0\\
2.91047276181905	0\\
2.91057276431911	0\\
2.91067276681917	0\\
2.91077276931923	0\\
2.9108727718193	0\\
2.91097277431936	0\\
2.91107277681942	0\\
2.91117277931948	0\\
2.91127278181955	0\\
2.91137278431961	0\\
2.91147278681967	0\\
2.91157278931973	0\\
2.9116727918198	0\\
2.91177279431986	0\\
2.91187279681992	0\\
2.91197279931998	0\\
2.91207280182005	0\\
2.91217280432011	0\\
2.91227280682017	0\\
2.91237280932023	0\\
2.9124728118203	0\\
2.91257281432036	0\\
2.91267281682042	0\\
2.91277281932048	0\\
2.91287282182055	0\\
2.91297282432061	0\\
2.91307282682067	0\\
2.91317282932073	0\\
2.9132728318208	0\\
2.91337283432086	0\\
2.91347283682092	0\\
2.91357283932098	0\\
2.91367284182105	0\\
2.91377284432111	0\\
2.91387284682117	0\\
2.91397284932123	0\\
2.9140728518213	0\\
2.91417285432136	0\\
2.91427285682142	0\\
2.91437285932148	0\\
2.91447286182155	0\\
2.91457286432161	0\\
2.91467286682167	0\\
2.91477286932173	0\\
2.9148728718218	0\\
2.91497287432186	0\\
2.91507287682192	0\\
2.91517287932198	0\\
2.91527288182205	0\\
2.91537288432211	0\\
2.91547288682217	0\\
2.91557288932223	0\\
2.9156728918223	0\\
2.91577289432236	0\\
2.91587289682242	0\\
2.91597289932248	0\\
2.91607290182255	0\\
2.91617290432261	0\\
2.91627290682267	0\\
2.91637290932273	0\\
2.9164729118228	0\\
2.91657291432286	0\\
2.91667291682292	0\\
2.91677291932298	0\\
2.91687292182305	0\\
2.91697292432311	0\\
2.91707292682317	0\\
2.91717292932323	0\\
2.9172729318233	0\\
2.91737293432336	0\\
2.91747293682342	0\\
2.91757293932348	0\\
2.91767294182355	0\\
2.91777294432361	0\\
2.91787294682367	0\\
2.91797294932373	0\\
2.9180729518238	0\\
2.91817295432386	0\\
2.91827295682392	0\\
2.91837295932398	0\\
2.91847296182405	0\\
2.91857296432411	0\\
2.91867296682417	0\\
2.91877296932423	0\\
2.9188729718243	0\\
2.91897297432436	0\\
2.91907297682442	0\\
2.91917297932448	0\\
2.91927298182455	0\\
2.91937298432461	0\\
2.91947298682467	0\\
2.91957298932473	0\\
2.9196729918248	0\\
2.91977299432486	0\\
2.91987299682492	0\\
2.91997299932498	0\\
2.92007300182505	0\\
2.92017300432511	0\\
2.92027300682517	0\\
2.92037300932523	0\\
2.9204730118253	0\\
2.92057301432536	0\\
2.92067301682542	0\\
2.92077301932548	0\\
2.92087302182555	0\\
2.92097302432561	0\\
2.92107302682567	0\\
2.92117302932573	0\\
2.9212730318258	0\\
2.92137303432586	0\\
2.92147303682592	0\\
2.92157303932598	0\\
2.92167304182605	0\\
2.92177304432611	0\\
2.92187304682617	0\\
2.92197304932623	0\\
2.9220730518263	0\\
2.92217305432636	0\\
2.92227305682642	0\\
2.92237305932648	0\\
2.92247306182655	0\\
2.92257306432661	0\\
2.92267306682667	0\\
2.92277306932673	0\\
2.9228730718268	0\\
2.92297307432686	0\\
2.92307307682692	0\\
2.92317307932698	0\\
2.92327308182705	0\\
2.92337308432711	0\\
2.92347308682717	0\\
2.92357308932723	0\\
2.9236730918273	0\\
2.92377309432736	0\\
2.92387309682742	0\\
2.92397309932748	0\\
2.92407310182755	0\\
2.92417310432761	0\\
2.92427310682767	0\\
2.92437310932773	0\\
2.9244731118278	0\\
2.92457311432786	0\\
2.92467311682792	0\\
2.92477311932798	0\\
2.92487312182805	0\\
2.92497312432811	0\\
2.92507312682817	0\\
2.92517312932823	0\\
2.9252731318283	0\\
2.92537313432836	0\\
2.92547313682842	0\\
2.92557313932848	0\\
2.92567314182855	0\\
2.92577314432861	0\\
2.92587314682867	0\\
2.92597314932873	0\\
2.9260731518288	0\\
2.92617315432886	0\\
2.92627315682892	0\\
2.92637315932898	0\\
2.92647316182905	0\\
2.92657316432911	0\\
2.92667316682917	0\\
2.92677316932923	0\\
2.9268731718293	0\\
2.92697317432936	0\\
2.92707317682942	0\\
2.92717317932948	0\\
2.92727318182955	0\\
2.92737318432961	0\\
2.92747318682967	0\\
2.92757318932973	0\\
2.9276731918298	0\\
2.92777319432986	0\\
2.92787319682992	0\\
2.92797319932998	0\\
2.92807320183005	0\\
2.92817320433011	0\\
2.92827320683017	0\\
2.92837320933023	0\\
2.9284732118303	0\\
2.92857321433036	0\\
2.92867321683042	0\\
2.92877321933048	0\\
2.92887322183055	0\\
2.92897322433061	0\\
2.92907322683067	0\\
2.92917322933073	0\\
2.9292732318308	0\\
2.92937323433086	0\\
2.92947323683092	0\\
2.92957323933098	0\\
2.92967324183105	0\\
2.92977324433111	0\\
2.92987324683117	0\\
2.92997324933123	0\\
2.9300732518313	0\\
2.93017325433136	0\\
2.93027325683142	0\\
2.93037325933148	0\\
2.93047326183155	0\\
2.93057326433161	0\\
2.93067326683167	0\\
2.93077326933173	0\\
2.9308732718318	0\\
2.93097327433186	0\\
2.93107327683192	0\\
2.93117327933198	0\\
2.93127328183205	0\\
2.93137328433211	0\\
2.93147328683217	0\\
2.93157328933223	0\\
2.9316732918323	0\\
2.93177329433236	0\\
2.93187329683242	0\\
2.93197329933248	0\\
2.93207330183255	0\\
2.93217330433261	0\\
2.93227330683267	0\\
2.93237330933273	0\\
2.9324733118328	0\\
2.93257331433286	0\\
2.93267331683292	0\\
2.93277331933298	0\\
2.93287332183305	0\\
2.93297332433311	0\\
2.93307332683317	0\\
2.93317332933323	0\\
2.9332733318333	0\\
2.93337333433336	0\\
2.93347333683342	0\\
2.93357333933348	0\\
2.93367334183355	0\\
2.93377334433361	0\\
2.93387334683367	0\\
2.93397334933373	0\\
2.9340733518338	0\\
2.93417335433386	0\\
2.93427335683392	0\\
2.93437335933398	0\\
2.93447336183405	0\\
2.93457336433411	0\\
2.93467336683417	0\\
2.93477336933423	0\\
2.9348733718343	0\\
2.93497337433436	0\\
2.93507337683442	0\\
2.93517337933448	0\\
2.93527338183455	0\\
2.93537338433461	0\\
2.93547338683467	0\\
2.93557338933473	0\\
2.9356733918348	0\\
2.93577339433486	0\\
2.93587339683492	0\\
2.93597339933498	0\\
2.93607340183505	0\\
2.93617340433511	0\\
2.93627340683517	0\\
2.93637340933523	0\\
2.9364734118353	0\\
2.93657341433536	0\\
2.93667341683542	0\\
2.93677341933548	0\\
2.93687342183555	0\\
2.93697342433561	0\\
2.93707342683567	0\\
2.93717342933573	0\\
2.9372734318358	0\\
2.93737343433586	0\\
2.93747343683592	0\\
2.93757343933598	0\\
2.93767344183605	0\\
2.93777344433611	0\\
2.93787344683617	0\\
2.93797344933623	0\\
2.9380734518363	0\\
2.93817345433636	0\\
2.93827345683642	0\\
2.93837345933648	0\\
2.93847346183655	0\\
2.93857346433661	0\\
2.93867346683667	0\\
2.93877346933673	0\\
2.9388734718368	0\\
2.93897347433686	0\\
2.93907347683692	0\\
2.93917347933698	0\\
2.93927348183705	0\\
2.93937348433711	0\\
2.93947348683717	0\\
2.93957348933723	0\\
2.9396734918373	0\\
2.93977349433736	0\\
2.93987349683742	0\\
2.93997349933748	0\\
2.94007350183755	0\\
2.94017350433761	0\\
2.94027350683767	0\\
2.94037350933773	0\\
2.9404735118378	0\\
2.94057351433786	0\\
2.94067351683792	0\\
2.94077351933798	0\\
2.94087352183805	0\\
2.94097352433811	0\\
2.94107352683817	0\\
2.94117352933823	0\\
2.9412735318383	0\\
2.94137353433836	0\\
2.94147353683842	0\\
2.94157353933848	0\\
2.94167354183855	0\\
2.94177354433861	0\\
2.94187354683867	0\\
2.94197354933873	0\\
2.9420735518388	0\\
2.94217355433886	0\\
2.94227355683892	0\\
2.94237355933898	0\\
2.94247356183905	0\\
2.94257356433911	0\\
2.94267356683917	0\\
2.94277356933923	0\\
2.9428735718393	0\\
2.94297357433936	0\\
2.94307357683942	0\\
2.94317357933948	0\\
2.94327358183955	0\\
2.94337358433961	0\\
2.94347358683967	0\\
2.94357358933973	0\\
2.9436735918398	0\\
2.94377359433986	0\\
2.94387359683992	0\\
2.94397359933998	0\\
2.94407360184005	0\\
2.94417360434011	0\\
2.94427360684017	0\\
2.94437360934023	0\\
2.9444736118403	0\\
2.94457361434036	0\\
2.94467361684042	0\\
2.94477361934048	0\\
2.94487362184055	0\\
2.94497362434061	0\\
2.94507362684067	0\\
2.94517362934073	0\\
2.9452736318408	0\\
2.94537363434086	0\\
2.94547363684092	0\\
2.94557363934098	0\\
2.94567364184105	0\\
2.94577364434111	0\\
2.94587364684117	0\\
2.94597364934123	0\\
2.9460736518413	0\\
2.94617365434136	0\\
2.94627365684142	0\\
2.94637365934148	0\\
2.94647366184155	0\\
2.94657366434161	0\\
2.94667366684167	0\\
2.94677366934173	0\\
2.9468736718418	0\\
2.94697367434186	0\\
2.94707367684192	0\\
2.94717367934198	0\\
2.94727368184205	0\\
2.94737368434211	0\\
2.94747368684217	0\\
2.94757368934223	0\\
2.9476736918423	0\\
2.94777369434236	0\\
2.94787369684242	0\\
2.94797369934248	0\\
2.94807370184255	0\\
2.94817370434261	0\\
2.94827370684267	0\\
2.94837370934273	0\\
2.9484737118428	0\\
2.94857371434286	0\\
2.94867371684292	0\\
2.94877371934298	0\\
2.94887372184305	0\\
2.94897372434311	0\\
2.94907372684317	0\\
2.94917372934323	0\\
2.9492737318433	0\\
2.94937373434336	0\\
2.94947373684342	0\\
2.94957373934348	0\\
2.94967374184355	0\\
2.94977374434361	0\\
2.94987374684367	0\\
2.94997374934373	0\\
2.9500737518438	0\\
2.95017375434386	0\\
2.95027375684392	0\\
2.95037375934398	0\\
2.95047376184405	0\\
2.95057376434411	0\\
2.95067376684417	0\\
2.95077376934423	0\\
2.9508737718443	0\\
2.95097377434436	0\\
2.95107377684442	0\\
2.95117377934448	0\\
2.95127378184455	0\\
2.95137378434461	0\\
2.95147378684467	0\\
2.95157378934473	0\\
2.9516737918448	0\\
2.95177379434486	0\\
2.95187379684492	0\\
2.95197379934498	0\\
2.95207380184505	0\\
2.95217380434511	0\\
2.95227380684517	0\\
2.95237380934523	0\\
2.9524738118453	0\\
2.95257381434536	0\\
2.95267381684542	0\\
2.95277381934548	0\\
2.95287382184555	0\\
2.95297382434561	0\\
2.95307382684567	0\\
2.95317382934573	0\\
2.9532738318458	0\\
2.95337383434586	0\\
2.95347383684592	0\\
2.95357383934598	0\\
2.95367384184605	0\\
2.95377384434611	0\\
2.95387384684617	0\\
2.95397384934623	0\\
2.9540738518463	0\\
2.95417385434636	0\\
2.95427385684642	0\\
2.95437385934648	0\\
2.95447386184655	0\\
2.95457386434661	0\\
2.95467386684667	0\\
2.95477386934673	0\\
2.9548738718468	0\\
2.95497387434686	0\\
2.95507387684692	0\\
2.95517387934698	0\\
2.95527388184705	0\\
2.95537388434711	0\\
2.95547388684717	0\\
2.95557388934723	0\\
2.9556738918473	0\\
2.95577389434736	0\\
2.95587389684742	0\\
2.95597389934748	0\\
2.95607390184755	0\\
2.95617390434761	0\\
2.95627390684767	0\\
2.95637390934773	0\\
2.9564739118478	0\\
2.95657391434786	0\\
2.95667391684792	0\\
2.95677391934798	0\\
2.95687392184805	0\\
2.95697392434811	0\\
2.95707392684817	0\\
2.95717392934823	0\\
2.9572739318483	0\\
2.95737393434836	0\\
2.95747393684842	0\\
2.95757393934848	0\\
2.95767394184855	0\\
2.95777394434861	0\\
2.95787394684867	0\\
2.95797394934873	0\\
2.9580739518488	0\\
2.95817395434886	0\\
2.95827395684892	0\\
2.95837395934898	0\\
2.95847396184905	0\\
2.95857396434911	0\\
2.95867396684917	0\\
2.95877396934923	0\\
2.9588739718493	0\\
2.95897397434936	0\\
2.95907397684942	0\\
2.95917397934948	0\\
2.95927398184955	0\\
2.95937398434961	0\\
2.95947398684967	0\\
2.95957398934973	0\\
2.9596739918498	0\\
2.95977399434986	0\\
2.95987399684992	0\\
2.95997399934998	0\\
2.96007400185005	0\\
2.96017400435011	0\\
2.96027400685017	0\\
2.96037400935023	0\\
2.9604740118503	0\\
2.96057401435036	0\\
2.96067401685042	0\\
2.96077401935048	0\\
2.96087402185055	0\\
2.96097402435061	0\\
2.96107402685067	0\\
2.96117402935073	0\\
2.9612740318508	0\\
2.96137403435086	0\\
2.96147403685092	0\\
2.96157403935098	0\\
2.96167404185105	0\\
2.96177404435111	0\\
2.96187404685117	0\\
2.96197404935123	0\\
2.9620740518513	0\\
2.96217405435136	0\\
2.96227405685142	0\\
2.96237405935148	0\\
2.96247406185155	0\\
2.96257406435161	0\\
2.96267406685167	0\\
2.96277406935173	0\\
2.9628740718518	0\\
2.96297407435186	0\\
2.96307407685192	0\\
2.96317407935198	0\\
2.96327408185205	0\\
2.96337408435211	0\\
2.96347408685217	0\\
2.96357408935223	0\\
2.9636740918523	0\\
2.96377409435236	0\\
2.96387409685242	0\\
2.96397409935248	0\\
2.96407410185255	0\\
2.96417410435261	0\\
2.96427410685267	0\\
2.96437410935273	0\\
2.9644741118528	0\\
2.96457411435286	0\\
2.96467411685292	0\\
2.96477411935298	0\\
2.96487412185305	0\\
2.96497412435311	0\\
2.96507412685317	0\\
2.96517412935323	0\\
2.9652741318533	0\\
2.96537413435336	0\\
2.96547413685342	0\\
2.96557413935348	0\\
2.96567414185355	0\\
2.96577414435361	0\\
2.96587414685367	0\\
2.96597414935373	0\\
2.9660741518538	0\\
2.96617415435386	0\\
2.96627415685392	0\\
2.96637415935398	0\\
2.96647416185405	0\\
2.96657416435411	0\\
2.96667416685417	0\\
2.96677416935423	0\\
2.9668741718543	0\\
2.96697417435436	0\\
2.96707417685442	0\\
2.96717417935448	0\\
2.96727418185455	0\\
2.96737418435461	0\\
2.96747418685467	0\\
2.96757418935473	0\\
2.9676741918548	0\\
2.96777419435486	0\\
2.96787419685492	0\\
2.96797419935498	0\\
2.96807420185505	0\\
2.96817420435511	0\\
2.96827420685517	0\\
2.96837420935523	0\\
2.9684742118553	0\\
2.96857421435536	0\\
2.96867421685542	0\\
2.96877421935548	0\\
2.96887422185555	0\\
2.96897422435561	0\\
2.96907422685567	0\\
2.96917422935573	0\\
2.9692742318558	0\\
2.96937423435586	0\\
2.96947423685592	0\\
2.96957423935598	0\\
2.96967424185605	0\\
2.96977424435611	0\\
2.96987424685617	0\\
2.96997424935623	0\\
2.9700742518563	0\\
2.97017425435636	0\\
2.97027425685642	0\\
2.97037425935648	0\\
2.97047426185655	0\\
2.97057426435661	0\\
2.97067426685667	0\\
2.97077426935673	0\\
2.9708742718568	0\\
2.97097427435686	0\\
2.97107427685692	0\\
2.97117427935698	0\\
2.97127428185705	0\\
2.97137428435711	0\\
2.97147428685717	0\\
2.97157428935723	0\\
2.9716742918573	0\\
2.97177429435736	0\\
2.97187429685742	0\\
2.97197429935748	0\\
2.97207430185755	0\\
2.97217430435761	0\\
2.97227430685767	0\\
2.97237430935773	0\\
2.9724743118578	0\\
2.97257431435786	0\\
2.97267431685792	0\\
2.97277431935798	0\\
2.97287432185805	0\\
2.97297432435811	0\\
2.97307432685817	0\\
2.97317432935823	0\\
2.9732743318583	0\\
2.97337433435836	0\\
2.97347433685842	0\\
2.97357433935848	0\\
2.97367434185855	0\\
2.97377434435861	0\\
2.97387434685867	0\\
2.97397434935873	0\\
2.9740743518588	0\\
2.97417435435886	0\\
2.97427435685892	0\\
2.97437435935898	0\\
2.97447436185905	0\\
2.97457436435911	0\\
2.97467436685917	0\\
2.97477436935923	0\\
2.9748743718593	0\\
2.97497437435936	0\\
2.97507437685942	0\\
2.97517437935948	0\\
2.97527438185955	0\\
2.97537438435961	0\\
2.97547438685967	0\\
2.97557438935973	0\\
2.9756743918598	0\\
2.97577439435986	0\\
2.97587439685992	0\\
2.97597439935998	0\\
2.97607440186005	0\\
2.97617440436011	0\\
2.97627440686017	0\\
2.97637440936023	0\\
2.9764744118603	0\\
2.97657441436036	0\\
2.97667441686042	0\\
2.97677441936048	0\\
2.97687442186055	0\\
2.97697442436061	0\\
2.97707442686067	0\\
2.97717442936073	0\\
2.9772744318608	0\\
2.97737443436086	0\\
2.97747443686092	0\\
2.97757443936098	0\\
2.97767444186105	0\\
2.97777444436111	0\\
2.97787444686117	0\\
2.97797444936123	0\\
2.9780744518613	0\\
2.97817445436136	0\\
2.97827445686142	0\\
2.97837445936148	0\\
2.97847446186155	0\\
2.97857446436161	0\\
2.97867446686167	0\\
2.97877446936173	0\\
2.9788744718618	0\\
2.97897447436186	0\\
2.97907447686192	0\\
2.97917447936198	0\\
2.97927448186205	0\\
2.97937448436211	0\\
2.97947448686217	0\\
2.97957448936223	0\\
2.9796744918623	0\\
2.97977449436236	0\\
2.97987449686242	0\\
2.97997449936248	0\\
2.98007450186255	0\\
2.98017450436261	0\\
2.98027450686267	0\\
2.98037450936273	0\\
2.9804745118628	0\\
2.98057451436286	0\\
2.98067451686292	0\\
2.98077451936298	0\\
2.98087452186305	0\\
2.98097452436311	0\\
2.98107452686317	0\\
2.98117452936323	0\\
2.9812745318633	0\\
2.98137453436336	0\\
2.98147453686342	0\\
2.98157453936348	0\\
2.98167454186355	0\\
2.98177454436361	0\\
2.98187454686367	0\\
2.98197454936373	0\\
2.9820745518638	0\\
2.98217455436386	0\\
2.98227455686392	0\\
2.98237455936398	0\\
2.98247456186405	0\\
2.98257456436411	0\\
2.98267456686417	0\\
2.98277456936423	0\\
2.9828745718643	0\\
2.98297457436436	0\\
2.98307457686442	0\\
2.98317457936448	0\\
2.98327458186455	0\\
2.98337458436461	0\\
2.98347458686467	0\\
2.98357458936473	0\\
2.9836745918648	0\\
2.98377459436486	0\\
2.98387459686492	0\\
2.98397459936498	0\\
2.98407460186505	0\\
2.98417460436511	0\\
2.98427460686517	0\\
2.98437460936523	0\\
2.9844746118653	0\\
2.98457461436536	0\\
2.98467461686542	0\\
2.98477461936548	0\\
2.98487462186555	0\\
2.98497462436561	0\\
2.98507462686567	0\\
2.98517462936573	0\\
2.9852746318658	0\\
2.98537463436586	0\\
2.98547463686592	0\\
2.98557463936598	0\\
2.98567464186605	0\\
2.98577464436611	0\\
2.98587464686617	0\\
2.98597464936623	0\\
2.9860746518663	0\\
2.98617465436636	0\\
2.98627465686642	0\\
2.98637465936648	0\\
2.98647466186655	0\\
2.98657466436661	0\\
2.98667466686667	0\\
2.98677466936673	0\\
2.9868746718668	0\\
2.98697467436686	0\\
2.98707467686692	0\\
2.98717467936698	0\\
2.98727468186705	0\\
2.98737468436711	0\\
2.98747468686717	0\\
2.98757468936723	0\\
2.9876746918673	0\\
2.98777469436736	0\\
2.98787469686742	0\\
2.98797469936748	0\\
2.98807470186755	0\\
2.98817470436761	0\\
2.98827470686767	0\\
2.98837470936773	0\\
2.9884747118678	0\\
2.98857471436786	0\\
2.98867471686792	0\\
2.98877471936798	0\\
2.98887472186805	0\\
2.98897472436811	0\\
2.98907472686817	0\\
2.98917472936823	0\\
2.9892747318683	0\\
2.98937473436836	0\\
2.98947473686842	0\\
2.98957473936848	0\\
2.98967474186855	0\\
2.98977474436861	0\\
2.98987474686867	0\\
2.98997474936873	0\\
2.9900747518688	0\\
2.99017475436886	0\\
2.99027475686892	0\\
2.99037475936898	0\\
2.99047476186905	0\\
2.99057476436911	0\\
2.99067476686917	0\\
2.99077476936923	0\\
2.9908747718693	0\\
2.99097477436936	0\\
2.99107477686942	0\\
2.99117477936948	0\\
2.99127478186955	0\\
2.99137478436961	0\\
2.99147478686967	0\\
2.99157478936973	0\\
2.9916747918698	0\\
2.99177479436986	0\\
2.99187479686992	0\\
2.99197479936998	0\\
2.99207480187005	0\\
2.99217480437011	0\\
2.99227480687017	0\\
2.99237480937023	0\\
2.9924748118703	0\\
2.99257481437036	0\\
2.99267481687042	0\\
2.99277481937048	0\\
2.99287482187055	0\\
2.99297482437061	0\\
2.99307482687067	0\\
2.99317482937073	0\\
2.9932748318708	0\\
2.99337483437086	0\\
2.99347483687092	0\\
2.99357483937098	0\\
2.99367484187105	0\\
2.99377484437111	0\\
2.99387484687117	0\\
2.99397484937123	0\\
2.9940748518713	0\\
2.99417485437136	0\\
2.99427485687142	0\\
2.99437485937148	0\\
2.99447486187155	0\\
2.99457486437161	0\\
2.99467486687167	0\\
2.99477486937173	0\\
2.9948748718718	0\\
2.99497487437186	0\\
2.99507487687192	0\\
2.99517487937198	0\\
2.99527488187205	0\\
2.99537488437211	0\\
2.99547488687217	0\\
2.99557488937223	0\\
2.9956748918723	0\\
2.99577489437236	0\\
2.99587489687242	0\\
2.99597489937248	0\\
2.99607490187255	0\\
2.99617490437261	0\\
2.99627490687267	0\\
2.99637490937273	0\\
2.9964749118728	0\\
2.99657491437286	0\\
2.99667491687292	0\\
2.99677491937298	0\\
2.99687492187305	0\\
2.99697492437311	0\\
2.99707492687317	0\\
2.99717492937323	0\\
2.9972749318733	0\\
2.99737493437336	0\\
2.99747493687342	0\\
2.99757493937348	0\\
2.99767494187355	0\\
2.99777494437361	0\\
2.99787494687367	0\\
2.99797494937373	0\\
2.9980749518738	0\\
2.99817495437386	0\\
2.99827495687392	0\\
2.99837495937398	0\\
2.99847496187405	0\\
2.99857496437411	0\\
2.99867496687417	0\\
2.99877496937423	0\\
2.9988749718743	0\\
2.99897497437436	0\\
2.99907497687442	0\\
2.99917497937448	0\\
2.99927498187455	0\\
2.99937498437461	0\\
2.99947498687467	0\\
2.99957498937473	0\\
2.9996749918748	0\\
2.99977499437486	0\\
2.99987499687492	0\\
2.99997499937498	0\\
3.00007500187505	0\\
3.00017500437511	0\\
3.00027500687517	0\\
3.00037500937523	0\\
3.0004750118753	0\\
3.00057501437536	0\\
3.00067501687542	0\\
3.00077501937548	0\\
3.00087502187555	0\\
3.00097502437561	0\\
3.00107502687567	0\\
3.00117502937573	0\\
3.0012750318758	0\\
3.00137503437586	0\\
3.00147503687592	0\\
3.00157503937598	0\\
3.00167504187605	0\\
3.00177504437611	0\\
3.00187504687617	0\\
3.00197504937623	0\\
3.0020750518763	0\\
3.00217505437636	0\\
3.00227505687642	0\\
3.00237505937648	0\\
3.00247506187655	0\\
3.00257506437661	0\\
3.00267506687667	0\\
3.00277506937673	0\\
3.0028750718768	0\\
3.00297507437686	0\\
3.00307507687692	0\\
3.00317507937698	0\\
3.00327508187705	0\\
3.00337508437711	0\\
3.00347508687717	0\\
3.00357508937723	0\\
3.0036750918773	0\\
3.00377509437736	0\\
3.00387509687742	0\\
3.00397509937748	0\\
3.00407510187755	0\\
3.00417510437761	0\\
3.00427510687767	0\\
3.00437510937773	0\\
3.0044751118778	0\\
3.00457511437786	0\\
3.00467511687792	0\\
3.00477511937798	0\\
3.00487512187805	0\\
3.00497512437811	0\\
3.00507512687817	0\\
3.00517512937823	0\\
3.0052751318783	0\\
3.00537513437836	0\\
3.00547513687842	0\\
3.00557513937848	0\\
3.00567514187855	0\\
3.00577514437861	0\\
3.00587514687867	0\\
3.00597514937873	0\\
3.0060751518788	0\\
3.00617515437886	0\\
3.00627515687892	0\\
3.00637515937898	0\\
3.00647516187905	0\\
3.00657516437911	0\\
3.00667516687917	0\\
3.00677516937923	0\\
3.0068751718793	0\\
3.00697517437936	0\\
3.00707517687942	0\\
3.00717517937948	0\\
3.00727518187955	0\\
3.00737518437961	0\\
3.00747518687967	0\\
3.00757518937973	0\\
3.0076751918798	0\\
3.00777519437986	0\\
3.00787519687992	0\\
3.00797519937998	0\\
3.00807520188005	0\\
3.00817520438011	0\\
3.00827520688017	0\\
3.00837520938023	0\\
3.0084752118803	0\\
3.00857521438036	0\\
3.00867521688042	0\\
3.00877521938048	0\\
3.00887522188055	0\\
3.00897522438061	0\\
3.00907522688067	0\\
3.00917522938073	0\\
3.0092752318808	0\\
3.00937523438086	0\\
3.00947523688092	0\\
3.00957523938098	0\\
3.00967524188105	0\\
3.00977524438111	0\\
3.00987524688117	0\\
3.00997524938123	0\\
3.0100752518813	0\\
3.01017525438136	0\\
3.01027525688142	0\\
3.01037525938148	0\\
3.01047526188155	0\\
3.01057526438161	0\\
3.01067526688167	0\\
3.01077526938173	0\\
3.0108752718818	0\\
3.01097527438186	0\\
3.01107527688192	0\\
3.01117527938198	0\\
3.01127528188205	0\\
3.01137528438211	0\\
3.01147528688217	0\\
3.01157528938223	0\\
3.0116752918823	0\\
3.01177529438236	0\\
3.01187529688242	0\\
3.01197529938248	0\\
3.01207530188255	0\\
3.01217530438261	0\\
3.01227530688267	0\\
3.01237530938273	0\\
3.0124753118828	0\\
3.01257531438286	0\\
3.01267531688292	0\\
3.01277531938298	0\\
3.01287532188305	0\\
3.01297532438311	0\\
3.01307532688317	0\\
3.01317532938323	0\\
3.0132753318833	0\\
3.01337533438336	0\\
3.01347533688342	0\\
3.01357533938348	0\\
3.01367534188355	0\\
3.01377534438361	0\\
3.01387534688367	0\\
3.01397534938373	0\\
3.0140753518838	0\\
3.01417535438386	0\\
3.01427535688392	0\\
3.01437535938398	0\\
3.01447536188405	0\\
3.01457536438411	0\\
3.01467536688417	0\\
3.01477536938423	0\\
3.0148753718843	0\\
3.01497537438436	0\\
3.01507537688442	0\\
3.01517537938448	0\\
3.01527538188455	0\\
3.01537538438461	0\\
3.01547538688467	0\\
3.01557538938473	0\\
3.0156753918848	0\\
3.01577539438486	0\\
3.01587539688492	0\\
3.01597539938498	0\\
3.01607540188505	0\\
3.01617540438511	0\\
3.01627540688517	0\\
3.01637540938523	0\\
3.0164754118853	0\\
3.01657541438536	0\\
3.01667541688542	0\\
3.01677541938548	0\\
3.01687542188555	0\\
3.01697542438561	0\\
3.01707542688567	0\\
3.01717542938573	0\\
3.0172754318858	0\\
3.01737543438586	0\\
3.01747543688592	0\\
3.01757543938598	0\\
3.01767544188605	0\\
3.01777544438611	0\\
3.01787544688617	0\\
3.01797544938623	0\\
3.0180754518863	0\\
3.01817545438636	0\\
3.01827545688642	0\\
3.01837545938648	0\\
3.01847546188655	0\\
3.01857546438661	0\\
3.01867546688667	0\\
3.01877546938673	0\\
3.0188754718868	0\\
3.01897547438686	0\\
3.01907547688692	0\\
3.01917547938698	0\\
3.01927548188705	0\\
3.01937548438711	0\\
3.01947548688717	0\\
3.01957548938723	0\\
3.0196754918873	0\\
3.01977549438736	0\\
3.01987549688742	0\\
3.01997549938748	0\\
3.02007550188755	0\\
3.02017550438761	0\\
3.02027550688767	0\\
3.02037550938773	0\\
3.0204755118878	0\\
3.02057551438786	0\\
3.02067551688792	0\\
3.02077551938798	0\\
3.02087552188805	0\\
3.02097552438811	0\\
3.02107552688817	0\\
3.02117552938823	0\\
3.0212755318883	0\\
3.02137553438836	0\\
3.02147553688842	0\\
3.02157553938848	0\\
3.02167554188855	0\\
3.02177554438861	0\\
3.02187554688867	0\\
3.02197554938873	0\\
3.0220755518888	0\\
3.02217555438886	0\\
3.02227555688892	0\\
3.02237555938898	0\\
3.02247556188905	0\\
3.02257556438911	0\\
3.02267556688917	0\\
3.02277556938923	0\\
3.0228755718893	0\\
3.02297557438936	0\\
3.02307557688942	0\\
3.02317557938948	0\\
3.02327558188955	0\\
3.02337558438961	0\\
3.02347558688967	0\\
3.02357558938973	0\\
3.0236755918898	0\\
3.02377559438986	0\\
3.02387559688992	0\\
3.02397559938998	0\\
3.02407560189005	0\\
3.02417560439011	0\\
3.02427560689017	0\\
3.02437560939023	0\\
3.0244756118903	0\\
3.02457561439036	0\\
3.02467561689042	0\\
3.02477561939048	0\\
3.02487562189055	0\\
3.02497562439061	0\\
3.02507562689067	0\\
3.02517562939073	0\\
3.0252756318908	0\\
3.02537563439086	0\\
3.02547563689092	0\\
3.02557563939098	0\\
3.02567564189105	0\\
3.02577564439111	0\\
3.02587564689117	0\\
3.02597564939123	0\\
3.0260756518913	0\\
3.02617565439136	0\\
3.02627565689142	0\\
3.02637565939148	0\\
3.02647566189155	0\\
3.02657566439161	0\\
3.02667566689167	0\\
3.02677566939173	0\\
3.0268756718918	0\\
3.02697567439186	0\\
3.02707567689192	0\\
3.02717567939198	0\\
3.02727568189205	0\\
3.02737568439211	0\\
3.02747568689217	0\\
3.02757568939223	0\\
3.0276756918923	0\\
3.02777569439236	0\\
3.02787569689242	0\\
3.02797569939248	0\\
3.02807570189255	0\\
3.02817570439261	0\\
3.02827570689267	0\\
3.02837570939273	0\\
3.0284757118928	0\\
3.02857571439286	0\\
3.02867571689292	0\\
3.02877571939298	0\\
3.02887572189305	0\\
3.02897572439311	0\\
3.02907572689317	0\\
3.02917572939323	0\\
3.0292757318933	0\\
3.02937573439336	0\\
3.02947573689342	0\\
3.02957573939349	0\\
3.02967574189355	0\\
3.02977574439361	0\\
3.02987574689367	0\\
3.02997574939373	0\\
3.0300757518938	0\\
3.03017575439386	0\\
3.03027575689392	0\\
3.03037575939398	0\\
3.03047576189405	0\\
3.03057576439411	0\\
3.03067576689417	0\\
3.03077576939423	0\\
3.0308757718943	0\\
3.03097577439436	0\\
3.03107577689442	0\\
3.03117577939448	0\\
3.03127578189455	0\\
3.03137578439461	0\\
3.03147578689467	0\\
3.03157578939474	0\\
3.0316757918948	0\\
3.03177579439486	0\\
3.03187579689492	0\\
3.03197579939498	0\\
3.03207580189505	0\\
3.03217580439511	0\\
3.03227580689517	0\\
3.03237580939523	0\\
3.0324758118953	0\\
3.03257581439536	0\\
3.03267581689542	0\\
3.03277581939548	0\\
3.03287582189555	0\\
3.03297582439561	0\\
3.03307582689567	0\\
3.03317582939573	0\\
3.0332758318958	0\\
3.03337583439586	0\\
3.03347583689592	0\\
3.03357583939599	0\\
3.03367584189605	0\\
3.03377584439611	0\\
3.03387584689617	0\\
3.03397584939623	0\\
3.0340758518963	0\\
3.03417585439636	0\\
3.03427585689642	0\\
3.03437585939648	0\\
3.03447586189655	0\\
3.03457586439661	0\\
3.03467586689667	0\\
3.03477586939673	0\\
3.0348758718968	0\\
3.03497587439686	0\\
3.03507587689692	0\\
3.03517587939698	0\\
3.03527588189705	0\\
3.03537588439711	0\\
3.03547588689717	0\\
3.03557588939724	0\\
3.0356758918973	0\\
3.03577589439736	0\\
3.03587589689742	0\\
3.03597589939749	0\\
3.03607590189755	0\\
3.03617590439761	0\\
3.03627590689767	0\\
3.03637590939773	0\\
3.0364759118978	0\\
3.03657591439786	0\\
3.03667591689792	0\\
3.03677591939798	0\\
3.03687592189805	0\\
3.03697592439811	0\\
3.03707592689817	0\\
3.03717592939823	0\\
3.0372759318983	0\\
3.03737593439836	0\\
3.03747593689842	0\\
3.03757593939849	0\\
3.03767594189855	0\\
3.03777594439861	0\\
3.03787594689867	0\\
3.03797594939874	0\\
3.0380759518988	0\\
3.03817595439886	0\\
3.03827595689892	0\\
3.03837595939898	0\\
3.03847596189905	0\\
3.03857596439911	0\\
3.03867596689917	0\\
3.03877596939923	0\\
3.0388759718993	0\\
3.03897597439936	0\\
3.03907597689942	0\\
3.03917597939948	0\\
3.03927598189955	0\\
3.03937598439961	0\\
3.03947598689967	0\\
3.03957598939974	0\\
3.0396759918998	0\\
3.03977599439986	0\\
3.03987599689992	0\\
3.03997599939999	0\\
3.04007600190005	0\\
3.04017600440011	0\\
3.04027600690017	0\\
3.04037600940024	0\\
3.0404760119003	0\\
3.04057601440036	0\\
3.04067601690042	0\\
3.04077601940048	0\\
3.04087602190055	0\\
3.04097602440061	0\\
3.04107602690067	0\\
3.04117602940073	0\\
3.0412760319008	0\\
3.04137603440086	0\\
3.04147603690092	0\\
3.04157603940099	0\\
3.04167604190105	0\\
3.04177604440111	0\\
3.04187604690117	0\\
3.04197604940124	0\\
3.0420760519013	0\\
3.04217605440136	0\\
3.04227605690142	0\\
3.04237605940149	0\\
3.04247606190155	0\\
3.04257606440161	0\\
3.04267606690167	0\\
3.04277606940173	0\\
3.0428760719018	0\\
3.04297607440186	0\\
3.04307607690192	0\\
3.04317607940198	0\\
3.04327608190205	0\\
3.04337608440211	0\\
3.04347608690217	0\\
3.04357608940224	0\\
3.0436760919023	0\\
3.04377609440236	0\\
3.04387609690242	0\\
3.04397609940249	0\\
3.04407610190255	0\\
3.04417610440261	0\\
3.04427610690267	0\\
3.04437610940274	0\\
3.0444761119028	0\\
3.04457611440286	0\\
3.04467611690292	0\\
3.04477611940298	0\\
3.04487612190305	0\\
3.04497612440311	0\\
3.04507612690317	0\\
3.04517612940323	0\\
3.0452761319033	0\\
3.04537613440336	0\\
3.04547613690342	0\\
3.04557613940349	0\\
3.04567614190355	0\\
3.04577614440361	0\\
3.04587614690367	0\\
3.04597614940374	0\\
3.0460761519038	0\\
3.04617615440386	0\\
3.04627615690392	0\\
3.04637615940399	0\\
3.04647616190405	0\\
3.04657616440411	0\\
3.04667616690417	0\\
3.04677616940424	0\\
3.0468761719043	0\\
3.04697617440436	0\\
3.04707617690442	0\\
3.04717617940448	0\\
3.04727618190455	0\\
3.04737618440461	0\\
3.04747618690467	0\\
3.04757618940474	0\\
3.0476761919048	0\\
3.04777619440486	0\\
3.04787619690492	0\\
3.04797619940499	0\\
3.04807620190505	0\\
3.04817620440511	0\\
3.04827620690517	0\\
3.04837620940524	0\\
3.0484762119053	0\\
3.04857621440536	0\\
3.04867621690542	0\\
3.04877621940549	0\\
3.04887622190555	0\\
3.04897622440561	0\\
3.04907622690567	0\\
3.04917622940573	0\\
3.0492762319058	0\\
3.04937623440586	0\\
3.04947623690592	0\\
3.04957623940599	0\\
3.04967624190605	0\\
3.04977624440611	0\\
3.04987624690617	0\\
3.04997624940624	0\\
3.0500762519063	0\\
3.05017625440636	0\\
3.05027625690642	0\\
3.05037625940649	0\\
3.05047626190655	0\\
3.05057626440661	0\\
3.05067626690667	0\\
3.05077626940674	0\\
3.0508762719068	0\\
3.05097627440686	0\\
3.05107627690692	0\\
3.05117627940698	0\\
3.05127628190705	0\\
3.05137628440711	0\\
3.05147628690717	0\\
3.05157628940724	0\\
3.0516762919073	0\\
3.05177629440736	0\\
3.05187629690742	0\\
3.05197629940749	0\\
3.05207630190755	0\\
3.05217630440761	0\\
3.05227630690767	0\\
3.05237630940774	0\\
3.0524763119078	0\\
3.05257631440786	0\\
3.05267631690792	0\\
3.05277631940799	0\\
3.05287632190805	0\\
3.05297632440811	0\\
3.05307632690817	0\\
3.05317632940824	0\\
3.0532763319083	0\\
3.05337633440836	0\\
3.05347633690842	0\\
3.05357633940849	0\\
3.05367634190855	0\\
3.05377634440861	0\\
3.05387634690867	0\\
3.05397634940874	0\\
3.0540763519088	0\\
3.05417635440886	0\\
3.05427635690892	0\\
3.05437635940899	0\\
3.05447636190905	0\\
3.05457636440911	0\\
3.05467636690917	0\\
3.05477636940924	0\\
3.0548763719093	0\\
3.05497637440936	0\\
3.05507637690942	0\\
3.05517637940949	0\\
3.05527638190955	0\\
3.05537638440961	0\\
3.05547638690967	0\\
3.05557638940974	0\\
3.0556763919098	0\\
3.05577639440986	0\\
3.05587639690992	0\\
3.05597639940999	0\\
3.05607640191005	0\\
3.05617640441011	0\\
3.05627640691017	0\\
3.05637640941024	0\\
3.0564764119103	0\\
3.05657641441036	0\\
3.05667641691042	0\\
3.05677641941049	0\\
3.05687642191055	0\\
3.05697642441061	0\\
3.05707642691067	0\\
3.05717642941074	0\\
3.0572764319108	0\\
3.05737643441086	0\\
3.05747643691092	0\\
3.05757643941099	0\\
3.05767644191105	0\\
3.05777644441111	0\\
3.05787644691117	0\\
3.05797644941124	0\\
3.0580764519113	0\\
3.05817645441136	0\\
3.05827645691142	0\\
3.05837645941149	0\\
3.05847646191155	0\\
3.05857646441161	0\\
3.05867646691167	0\\
3.05877646941174	0\\
3.0588764719118	0\\
3.05897647441186	0\\
3.05907647691192	0\\
3.05917647941199	0\\
3.05927648191205	0\\
3.05937648441211	0\\
3.05947648691217	0\\
3.05957648941224	0\\
3.0596764919123	0\\
3.05977649441236	0\\
3.05987649691242	0\\
3.05997649941249	0\\
3.06007650191255	0\\
3.06017650441261	0\\
3.06027650691267	0\\
3.06037650941274	0\\
3.0604765119128	0\\
3.06057651441286	0\\
3.06067651691292	0\\
3.06077651941299	0\\
3.06087652191305	0\\
3.06097652441311	0\\
3.06107652691317	0\\
3.06117652941324	0\\
3.0612765319133	0\\
3.06137653441336	0\\
3.06147653691342	0\\
3.06157653941349	0\\
3.06167654191355	0\\
3.06177654441361	0\\
3.06187654691367	0\\
3.06197654941374	0\\
3.0620765519138	0\\
3.06217655441386	0\\
3.06227655691392	0\\
3.06237655941399	0\\
3.06247656191405	0\\
3.06257656441411	0\\
3.06267656691417	0\\
3.06277656941424	0\\
3.0628765719143	0\\
3.06297657441436	0\\
3.06307657691442	0\\
3.06317657941449	0\\
3.06327658191455	0\\
3.06337658441461	0\\
3.06347658691467	0\\
3.06357658941474	0\\
3.0636765919148	0\\
3.06377659441486	0\\
3.06387659691492	0\\
3.06397659941499	0\\
3.06407660191505	0\\
3.06417660441511	0\\
3.06427660691517	0\\
3.06437660941524	0\\
3.0644766119153	0\\
3.06457661441536	0\\
3.06467661691542	0\\
3.06477661941549	0\\
3.06487662191555	0\\
3.06497662441561	0\\
3.06507662691567	0\\
3.06517662941574	0\\
3.0652766319158	0\\
3.06537663441586	0\\
3.06547663691592	0\\
3.06557663941599	0\\
3.06567664191605	0\\
3.06577664441611	0\\
3.06587664691617	0\\
3.06597664941624	0\\
3.0660766519163	0\\
3.06617665441636	0\\
3.06627665691642	0\\
3.06637665941649	0\\
3.06647666191655	0\\
3.06657666441661	0\\
3.06667666691667	0\\
3.06677666941674	0\\
3.0668766719168	0\\
3.06697667441686	0\\
3.06707667691692	0\\
3.06717667941699	0\\
3.06727668191705	0\\
3.06737668441711	0\\
3.06747668691717	0\\
3.06757668941724	0\\
3.0676766919173	0\\
3.06777669441736	0\\
3.06787669691742	0\\
3.06797669941749	0\\
3.06807670191755	0\\
3.06817670441761	0\\
3.06827670691767	0\\
3.06837670941774	0\\
3.0684767119178	0\\
3.06857671441786	0\\
3.06867671691792	0\\
3.06877671941799	0\\
3.06887672191805	0\\
3.06897672441811	0\\
3.06907672691817	0\\
3.06917672941824	0\\
3.0692767319183	0\\
3.06937673441836	0\\
3.06947673691842	0\\
3.06957673941849	0\\
3.06967674191855	0\\
3.06977674441861	0\\
3.06987674691867	0\\
3.06997674941874	0\\
3.0700767519188	0\\
3.07017675441886	0\\
3.07027675691892	0\\
3.07037675941899	0\\
3.07047676191905	0\\
3.07057676441911	0\\
3.07067676691917	0\\
3.07077676941924	0\\
3.0708767719193	0\\
3.07097677441936	0\\
3.07107677691942	0\\
3.07117677941949	0\\
3.07127678191955	0\\
3.07137678441961	0\\
3.07147678691967	0\\
3.07157678941974	0\\
3.0716767919198	0\\
3.07177679441986	0\\
3.07187679691992	0\\
3.07197679941999	0\\
3.07207680192005	0\\
3.07217680442011	0\\
3.07227680692017	0\\
3.07237680942024	0\\
3.0724768119203	0\\
3.07257681442036	0\\
3.07267681692042	0\\
3.07277681942049	0\\
3.07287682192055	0\\
3.07297682442061	0\\
3.07307682692067	0\\
3.07317682942074	0\\
3.0732768319208	0\\
3.07337683442086	0\\
3.07347683692092	0\\
3.07357683942099	0\\
3.07367684192105	0\\
3.07377684442111	0\\
3.07387684692117	0\\
3.07397684942124	0\\
3.0740768519213	0\\
3.07417685442136	0\\
3.07427685692142	0\\
3.07437685942149	0\\
3.07447686192155	0\\
3.07457686442161	0\\
3.07467686692167	0\\
3.07477686942174	0\\
3.0748768719218	0\\
3.07497687442186	0\\
3.07507687692192	0\\
3.07517687942199	0\\
3.07527688192205	0\\
3.07537688442211	0\\
3.07547688692217	0\\
3.07557688942224	0\\
3.0756768919223	0\\
3.07577689442236	0\\
3.07587689692242	0\\
3.07597689942249	0\\
3.07607690192255	0\\
3.07617690442261	0\\
3.07627690692267	0\\
3.07637690942274	0\\
3.0764769119228	0\\
3.07657691442286	0\\
3.07667691692292	0\\
3.07677691942299	0\\
3.07687692192305	0\\
3.07697692442311	0\\
3.07707692692317	0\\
3.07717692942324	0\\
3.0772769319233	0\\
3.07737693442336	0\\
3.07747693692342	0\\
3.07757693942349	0\\
3.07767694192355	0\\
3.07777694442361	0\\
3.07787694692367	0\\
3.07797694942374	0\\
3.0780769519238	0\\
3.07817695442386	0\\
3.07827695692392	0\\
3.07837695942399	0\\
3.07847696192405	0\\
3.07857696442411	0\\
3.07867696692417	0\\
3.07877696942424	0\\
3.0788769719243	0\\
3.07897697442436	0\\
3.07907697692442	0\\
3.07917697942449	0\\
3.07927698192455	0\\
3.07937698442461	0\\
3.07947698692467	0\\
3.07957698942474	0\\
3.0796769919248	0\\
3.07977699442486	0\\
3.07987699692492	0\\
3.07997699942499	0\\
3.08007700192505	0\\
3.08017700442511	0\\
3.08027700692517	0\\
3.08037700942524	0\\
3.0804770119253	0\\
3.08057701442536	0\\
3.08067701692542	0\\
3.08077701942549	0\\
3.08087702192555	0\\
3.08097702442561	0\\
3.08107702692567	0\\
3.08117702942574	0\\
3.0812770319258	0\\
3.08137703442586	0\\
3.08147703692592	0\\
3.08157703942599	0\\
3.08167704192605	0\\
3.08177704442611	0\\
3.08187704692617	0\\
3.08197704942624	0\\
3.0820770519263	0\\
3.08217705442636	0\\
3.08227705692642	0\\
3.08237705942649	0\\
3.08247706192655	0\\
3.08257706442661	0\\
3.08267706692667	0\\
3.08277706942674	0\\
3.0828770719268	0\\
3.08297707442686	0\\
3.08307707692692	0\\
3.08317707942699	0\\
3.08327708192705	0\\
3.08337708442711	0\\
3.08347708692717	0\\
3.08357708942724	0\\
3.0836770919273	0\\
3.08377709442736	0\\
3.08387709692742	0\\
3.08397709942749	0\\
3.08407710192755	0\\
3.08417710442761	0\\
3.08427710692767	0\\
3.08437710942774	0\\
3.0844771119278	0\\
3.08457711442786	0\\
3.08467711692792	0\\
3.08477711942799	0\\
3.08487712192805	0\\
3.08497712442811	0\\
3.08507712692817	0\\
3.08517712942824	0\\
3.0852771319283	0\\
3.08537713442836	0\\
3.08547713692842	0\\
3.08557713942849	0\\
3.08567714192855	0\\
3.08577714442861	0\\
3.08587714692867	0\\
3.08597714942874	0\\
3.0860771519288	0\\
3.08617715442886	0\\
3.08627715692892	0\\
3.08637715942899	0\\
3.08647716192905	0\\
3.08657716442911	0\\
3.08667716692917	0\\
3.08677716942924	0\\
3.0868771719293	0\\
3.08697717442936	0\\
3.08707717692942	0\\
3.08717717942949	0\\
3.08727718192955	0\\
3.08737718442961	0\\
3.08747718692967	0\\
3.08757718942974	0\\
3.0876771919298	0\\
3.08777719442986	0\\
3.08787719692992	0\\
3.08797719942999	0\\
3.08807720193005	0\\
3.08817720443011	0\\
3.08827720693017	0\\
3.08837720943024	0\\
3.0884772119303	0\\
3.08857721443036	0\\
3.08867721693042	0\\
3.08877721943049	0\\
3.08887722193055	0\\
3.08897722443061	0\\
3.08907722693067	0\\
3.08917722943074	0\\
3.0892772319308	0\\
3.08937723443086	0\\
3.08947723693092	0\\
3.08957723943099	0\\
3.08967724193105	0\\
3.08977724443111	0\\
3.08987724693117	0\\
3.08997724943124	0\\
3.0900772519313	0\\
3.09017725443136	0\\
3.09027725693142	0\\
3.09037725943149	0\\
3.09047726193155	0\\
3.09057726443161	0\\
3.09067726693167	0\\
3.09077726943174	0\\
3.0908772719318	0\\
3.09097727443186	0\\
3.09107727693192	0\\
3.09117727943199	0\\
3.09127728193205	0\\
3.09137728443211	0\\
3.09147728693217	0\\
3.09157728943224	0\\
3.0916772919323	0\\
3.09177729443236	0\\
3.09187729693242	0\\
3.09197729943249	0\\
3.09207730193255	0\\
3.09217730443261	0\\
3.09227730693267	0\\
3.09237730943274	0\\
3.0924773119328	0\\
3.09257731443286	0\\
3.09267731693292	0\\
3.09277731943299	0\\
3.09287732193305	0\\
3.09297732443311	0\\
3.09307732693317	0\\
3.09317732943324	0\\
3.0932773319333	0\\
3.09337733443336	0\\
3.09347733693342	0\\
3.09357733943349	0\\
3.09367734193355	0\\
3.09377734443361	0\\
3.09387734693367	0\\
3.09397734943374	0\\
3.0940773519338	0\\
3.09417735443386	0\\
3.09427735693392	0\\
3.09437735943399	0\\
3.09447736193405	0\\
3.09457736443411	0\\
3.09467736693417	0\\
3.09477736943424	0\\
3.0948773719343	0\\
3.09497737443436	0\\
3.09507737693442	0\\
3.09517737943449	0\\
3.09527738193455	0\\
3.09537738443461	0\\
3.09547738693467	0\\
3.09557738943474	0\\
3.0956773919348	0\\
3.09577739443486	0\\
3.09587739693492	0\\
3.09597739943499	0\\
3.09607740193505	0\\
3.09617740443511	0\\
3.09627740693517	0\\
3.09637740943524	0\\
3.0964774119353	0\\
3.09657741443536	0\\
3.09667741693542	0\\
3.09677741943549	0\\
3.09687742193555	0\\
3.09697742443561	0\\
3.09707742693567	0\\
3.09717742943574	0\\
3.0972774319358	0\\
3.09737743443586	0\\
3.09747743693592	0\\
3.09757743943599	0\\
3.09767744193605	0\\
3.09777744443611	0\\
3.09787744693617	0\\
3.09797744943624	0\\
3.0980774519363	0\\
3.09817745443636	0\\
3.09827745693642	0\\
3.09837745943649	0\\
3.09847746193655	0\\
3.09857746443661	0\\
3.09867746693667	0\\
3.09877746943674	0\\
3.0988774719368	0\\
3.09897747443686	0\\
3.09907747693692	0\\
3.09917747943699	0\\
3.09927748193705	0\\
3.09937748443711	0\\
3.09947748693717	0\\
3.09957748943724	0\\
3.0996774919373	0\\
3.09977749443736	0\\
3.09987749693742	0\\
3.09997749943749	0\\
3.10007750193755	0\\
3.10017750443761	0\\
3.10027750693767	0\\
3.10037750943774	0\\
3.1004775119378	0\\
3.10057751443786	0\\
3.10067751693792	0\\
3.10077751943799	0\\
3.10087752193805	0\\
3.10097752443811	0\\
3.10107752693817	0\\
3.10117752943824	0\\
3.1012775319383	0\\
3.10137753443836	0\\
3.10147753693842	0\\
3.10157753943849	0\\
3.10167754193855	0\\
3.10177754443861	0\\
3.10187754693867	0\\
3.10197754943874	0\\
3.1020775519388	0\\
3.10217755443886	0\\
3.10227755693892	0\\
3.10237755943899	0\\
3.10247756193905	0\\
3.10257756443911	0\\
3.10267756693917	0\\
3.10277756943924	0\\
3.1028775719393	0\\
3.10297757443936	0\\
3.10307757693942	0\\
3.10317757943949	0\\
3.10327758193955	0\\
3.10337758443961	0\\
3.10347758693967	0\\
3.10357758943974	0\\
3.1036775919398	0\\
3.10377759443986	0\\
3.10387759693992	0\\
3.10397759943999	0\\
3.10407760194005	0\\
3.10417760444011	0\\
3.10427760694017	0\\
3.10437760944024	0\\
3.1044776119403	0\\
3.10457761444036	0\\
3.10467761694042	0\\
3.10477761944049	0\\
3.10487762194055	0\\
3.10497762444061	0\\
3.10507762694067	0\\
3.10517762944074	0\\
3.1052776319408	0\\
3.10537763444086	0\\
3.10547763694092	0\\
3.10557763944099	0\\
3.10567764194105	0\\
3.10577764444111	0\\
3.10587764694117	0\\
3.10597764944124	0\\
3.1060776519413	0\\
3.10617765444136	0\\
3.10627765694142	0\\
3.10637765944149	0\\
3.10647766194155	0\\
3.10657766444161	0\\
3.10667766694167	0\\
3.10677766944174	0\\
3.1068776719418	0\\
3.10697767444186	0\\
3.10707767694192	0\\
3.10717767944199	0\\
3.10727768194205	0\\
3.10737768444211	0\\
3.10747768694217	0\\
3.10757768944224	0\\
3.1076776919423	0\\
3.10777769444236	0\\
3.10787769694242	0\\
3.10797769944249	0\\
3.10807770194255	0\\
3.10817770444261	0\\
3.10827770694267	0\\
3.10837770944274	0\\
3.1084777119428	0\\
3.10857771444286	0\\
3.10867771694292	0\\
3.10877771944299	0\\
3.10887772194305	0\\
3.10897772444311	0\\
3.10907772694317	0\\
3.10917772944324	0\\
3.1092777319433	0\\
3.10937773444336	0\\
3.10947773694342	0\\
3.10957773944349	0\\
3.10967774194355	0\\
3.10977774444361	0\\
3.10987774694367	0\\
3.10997774944374	0\\
3.1100777519438	0\\
3.11017775444386	0\\
3.11027775694392	0\\
3.11037775944399	0\\
3.11047776194405	0\\
3.11057776444411	0\\
3.11067776694417	0\\
3.11077776944424	0\\
3.1108777719443	0\\
3.11097777444436	0\\
3.11107777694442	0\\
3.11117777944449	0\\
3.11127778194455	0\\
3.11137778444461	0\\
3.11147778694467	0\\
3.11157778944474	0\\
3.1116777919448	0\\
3.11177779444486	0\\
3.11187779694492	0\\
3.11197779944499	0\\
3.11207780194505	0\\
3.11217780444511	0\\
3.11227780694517	0\\
3.11237780944524	0\\
3.1124778119453	0\\
3.11257781444536	0\\
3.11267781694542	0\\
3.11277781944549	0\\
3.11287782194555	0\\
3.11297782444561	0\\
3.11307782694567	0\\
3.11317782944574	0\\
3.1132778319458	0\\
3.11337783444586	0\\
3.11347783694592	0\\
3.11357783944599	0\\
3.11367784194605	0\\
3.11377784444611	0\\
3.11387784694617	0\\
3.11397784944624	0\\
3.1140778519463	0\\
3.11417785444636	0\\
3.11427785694642	0\\
3.11437785944649	0\\
3.11447786194655	0\\
3.11457786444661	0\\
3.11467786694667	0\\
3.11477786944674	0\\
3.1148778719468	0\\
3.11497787444686	0\\
3.11507787694692	0\\
3.11517787944699	0\\
3.11527788194705	0\\
3.11537788444711	0\\
3.11547788694717	0\\
3.11557788944724	0\\
3.1156778919473	0\\
3.11577789444736	0\\
3.11587789694742	0\\
3.11597789944749	0\\
3.11607790194755	0\\
3.11617790444761	0\\
3.11627790694767	0\\
3.11637790944774	0\\
3.1164779119478	0\\
3.11657791444786	0\\
3.11667791694792	0\\
3.11677791944799	0\\
3.11687792194805	0\\
3.11697792444811	0\\
3.11707792694817	0\\
3.11717792944824	0\\
3.1172779319483	0\\
3.11737793444836	0\\
3.11747793694842	0\\
3.11757793944849	0\\
3.11767794194855	0\\
3.11777794444861	0\\
3.11787794694867	0\\
3.11797794944874	0\\
3.1180779519488	0\\
3.11817795444886	0\\
3.11827795694892	0\\
3.11837795944899	0\\
3.11847796194905	0\\
3.11857796444911	0\\
3.11867796694917	0\\
3.11877796944924	0\\
3.1188779719493	0\\
3.11897797444936	0\\
3.11907797694942	0\\
3.11917797944949	0\\
3.11927798194955	0\\
3.11937798444961	0\\
3.11947798694967	0\\
3.11957798944974	0\\
3.1196779919498	0\\
3.11977799444986	0\\
3.11987799694992	0\\
3.11997799944999	0\\
3.12007800195005	0\\
3.12017800445011	0\\
3.12027800695017	0\\
3.12037800945024	0\\
3.1204780119503	0\\
3.12057801445036	0\\
3.12067801695042	0\\
3.12077801945049	0\\
3.12087802195055	0\\
3.12097802445061	0\\
3.12107802695067	0\\
3.12117802945074	0\\
3.1212780319508	0\\
3.12137803445086	0\\
3.12147803695092	0\\
3.12157803945099	0\\
3.12167804195105	0\\
3.12177804445111	0\\
3.12187804695117	0\\
3.12197804945124	0\\
3.1220780519513	0\\
3.12217805445136	0\\
3.12227805695142	0\\
3.12237805945149	0\\
3.12247806195155	0\\
3.12257806445161	0\\
3.12267806695167	0\\
3.12277806945174	0\\
3.1228780719518	0\\
3.12297807445186	0\\
3.12307807695192	0\\
3.12317807945199	0\\
3.12327808195205	0\\
3.12337808445211	0\\
3.12347808695217	0\\
3.12357808945224	0\\
3.1236780919523	0\\
3.12377809445236	0\\
3.12387809695242	0\\
3.12397809945249	0\\
3.12407810195255	0\\
3.12417810445261	0\\
3.12427810695267	0\\
3.12437810945274	0\\
3.1244781119528	0\\
3.12457811445286	0\\
3.12467811695292	0\\
3.12477811945299	0\\
3.12487812195305	0\\
3.12497812445311	0\\
3.12507812695317	0\\
3.12517812945324	0\\
3.1252781319533	0\\
3.12537813445336	0\\
3.12547813695342	0\\
3.12557813945349	0\\
3.12567814195355	0\\
3.12577814445361	0\\
3.12587814695367	0\\
3.12597814945374	0\\
3.1260781519538	0\\
3.12617815445386	0\\
3.12627815695392	0\\
3.12637815945399	0\\
3.12647816195405	0\\
3.12657816445411	0\\
3.12667816695417	0\\
3.12677816945424	0\\
3.1268781719543	0\\
3.12697817445436	0\\
3.12707817695442	0\\
3.12717817945449	0\\
3.12727818195455	0\\
3.12737818445461	0\\
3.12747818695467	0\\
3.12757818945474	0\\
3.1276781919548	0\\
3.12777819445486	0\\
3.12787819695492	0\\
3.12797819945499	0\\
3.12807820195505	0\\
3.12817820445511	0\\
3.12827820695517	0\\
3.12837820945524	0\\
3.1284782119553	0\\
3.12857821445536	0\\
3.12867821695542	0\\
3.12877821945549	0\\
3.12887822195555	0\\
3.12897822445561	0\\
3.12907822695567	0\\
3.12917822945574	0\\
3.1292782319558	0\\
3.12937823445586	0\\
3.12947823695592	0\\
3.12957823945599	0\\
3.12967824195605	0\\
3.12977824445611	0\\
3.12987824695617	0\\
3.12997824945624	0\\
3.1300782519563	0\\
3.13017825445636	0\\
3.13027825695642	0\\
3.13037825945649	0\\
3.13047826195655	0\\
3.13057826445661	0\\
3.13067826695667	0\\
3.13077826945674	0\\
3.1308782719568	0\\
3.13097827445686	0\\
3.13107827695692	0\\
3.13117827945699	0\\
3.13127828195705	0\\
3.13137828445711	0\\
3.13147828695717	0\\
3.13157828945724	0\\
3.1316782919573	0\\
3.13177829445736	0\\
3.13187829695742	0\\
3.13197829945749	0\\
3.13207830195755	0\\
3.13217830445761	0\\
3.13227830695767	0\\
3.13237830945774	0\\
3.1324783119578	0\\
3.13257831445786	0\\
3.13267831695792	0\\
3.13277831945799	0\\
3.13287832195805	0\\
3.13297832445811	0\\
3.13307832695817	0\\
3.13317832945824	0\\
3.1332783319583	0\\
3.13337833445836	0\\
3.13347833695842	0\\
3.13357833945849	0\\
3.13367834195855	0\\
3.13377834445861	0\\
3.13387834695867	0\\
3.13397834945874	0\\
3.1340783519588	0\\
3.13417835445886	0\\
3.13427835695892	0\\
3.13437835945899	0\\
3.13447836195905	0\\
3.13457836445911	0\\
3.13467836695917	0\\
3.13477836945924	0\\
3.1348783719593	0\\
3.13497837445936	0\\
3.13507837695942	0\\
3.13517837945949	0\\
3.13527838195955	0\\
3.13537838445961	0\\
3.13547838695967	0\\
3.13557838945974	0\\
3.1356783919598	0\\
3.13577839445986	0\\
3.13587839695992	0\\
3.13597839945999	0\\
3.13607840196005	0\\
3.13617840446011	0\\
3.13627840696017	0\\
3.13637840946024	0\\
3.1364784119603	0\\
3.13657841446036	0\\
3.13667841696042	0\\
3.13677841946049	0\\
3.13687842196055	0\\
3.13697842446061	0\\
3.13707842696067	0\\
3.13717842946074	0\\
3.1372784319608	0\\
3.13737843446086	0\\
3.13747843696092	0\\
3.13757843946099	0\\
3.13767844196105	0\\
3.13777844446111	0\\
3.13787844696117	0\\
3.13797844946124	0\\
3.1380784519613	0\\
3.13817845446136	0\\
3.13827845696142	0\\
3.13837845946149	0\\
3.13847846196155	0\\
3.13857846446161	0\\
3.13867846696167	0\\
3.13877846946174	0\\
3.1388784719618	0\\
3.13897847446186	0\\
3.13907847696192	0\\
3.13917847946199	0\\
3.13927848196205	0\\
3.13937848446211	0\\
3.13947848696217	0\\
3.13957848946224	0\\
3.1396784919623	0\\
3.13977849446236	0\\
3.13987849696242	0\\
3.13997849946249	0\\
3.14007850196255	0\\
3.14017850446261	0\\
3.14027850696267	0\\
3.14037850946274	0\\
3.1404785119628	0\\
3.14057851446286	0\\
3.14067851696292	0\\
3.14077851946299	0\\
3.14087852196305	0\\
3.14097852446311	0\\
3.14107852696317	0\\
3.14117852946324	0\\
3.1412785319633	0\\
3.14137853446336	0\\
3.14147853696342	0\\
3.14157853946349	0\\
3.14167854196355	0\\
3.14177854446361	0\\
3.14187854696367	0\\
3.14197854946374	0\\
3.1420785519638	0\\
3.14217855446386	0\\
3.14227855696392	0\\
3.14237855946399	0\\
3.14247856196405	0\\
3.14257856446411	0\\
3.14267856696417	0\\
3.14277856946424	0\\
3.1428785719643	0\\
3.14297857446436	0\\
3.14307857696442	0\\
3.14317857946449	0\\
3.14327858196455	0\\
3.14337858446461	0\\
3.14347858696467	0\\
3.14357858946474	0\\
3.1436785919648	0\\
3.14377859446486	0\\
3.14387859696492	0\\
3.14397859946499	0\\
3.14407860196505	0\\
3.14417860446511	0\\
3.14427860696517	0\\
3.14437860946524	0\\
3.1444786119653	0\\
3.14457861446536	0\\
3.14467861696542	0\\
3.14477861946549	0\\
3.14487862196555	0\\
3.14497862446561	0\\
3.14507862696567	0\\
3.14517862946574	0\\
3.1452786319658	0\\
3.14537863446586	0\\
3.14547863696592	0\\
3.14557863946599	0\\
3.14567864196605	0\\
3.14577864446611	0\\
3.14587864696617	0\\
3.14597864946624	0\\
3.1460786519663	0\\
3.14617865446636	0\\
3.14627865696642	0\\
3.14637865946649	0\\
3.14647866196655	0\\
3.14657866446661	0\\
3.14667866696667	0\\
3.14677866946674	0\\
3.1468786719668	0\\
3.14697867446686	0\\
3.14707867696692	0\\
3.14717867946699	0\\
3.14727868196705	0\\
3.14737868446711	0\\
3.14747868696717	0\\
3.14757868946724	0\\
3.1476786919673	0\\
3.14777869446736	0\\
3.14787869696742	0\\
3.14797869946749	0\\
3.14807870196755	0\\
3.14817870446761	0\\
3.14827870696767	0\\
3.14837870946774	0\\
3.1484787119678	0\\
3.14857871446786	0\\
3.14867871696792	0\\
3.14877871946799	0\\
3.14887872196805	0\\
3.14897872446811	0\\
3.14907872696817	0\\
3.14917872946824	0\\
3.1492787319683	0\\
3.14937873446836	0\\
3.14947873696842	0\\
3.14957873946849	0\\
3.14967874196855	0\\
3.14977874446861	0\\
3.14987874696867	0\\
3.14997874946874	0\\
3.1500787519688	0\\
3.15017875446886	0\\
3.15027875696892	0\\
3.15037875946899	0\\
3.15047876196905	0\\
3.15057876446911	0\\
3.15067876696917	0\\
3.15077876946924	0\\
3.1508787719693	0\\
3.15097877446936	0\\
3.15107877696942	0\\
3.15117877946949	0\\
3.15127878196955	0\\
3.15137878446961	0\\
3.15147878696967	0\\
3.15157878946974	0\\
3.1516787919698	0\\
3.15177879446986	0\\
3.15187879696992	0\\
3.15197879946999	0\\
3.15207880197005	0\\
3.15217880447011	0\\
3.15227880697017	0\\
3.15237880947024	0\\
3.1524788119703	0\\
3.15257881447036	0\\
3.15267881697042	0\\
3.15277881947049	0\\
3.15287882197055	0\\
3.15297882447061	0\\
3.15307882697067	0\\
3.15317882947074	0\\
3.1532788319708	0\\
3.15337883447086	0\\
3.15347883697092	0\\
3.15357883947099	0\\
3.15367884197105	0\\
3.15377884447111	0\\
3.15387884697117	0\\
3.15397884947124	0\\
3.1540788519713	0\\
3.15417885447136	0\\
3.15427885697142	0\\
3.15437885947149	0\\
3.15447886197155	0\\
3.15457886447161	0\\
3.15467886697167	0\\
3.15477886947174	0\\
3.1548788719718	0\\
3.15497887447186	0\\
3.15507887697192	0\\
3.15517887947199	0\\
3.15527888197205	0\\
3.15537888447211	0\\
3.15547888697217	0\\
3.15557888947224	0\\
3.1556788919723	0\\
3.15577889447236	0\\
3.15587889697242	0\\
3.15597889947249	0\\
3.15607890197255	0\\
3.15617890447261	0\\
3.15627890697267	0\\
3.15637890947274	0\\
3.1564789119728	0\\
3.15657891447286	0\\
3.15667891697292	0\\
3.15677891947299	0\\
3.15687892197305	0\\
3.15697892447311	0\\
3.15707892697317	0\\
3.15717892947324	0\\
3.1572789319733	0\\
3.15737893447336	0\\
3.15747893697342	0\\
3.15757893947349	0\\
3.15767894197355	0\\
3.15777894447361	0\\
3.15787894697367	0\\
3.15797894947374	0\\
3.1580789519738	0\\
3.15817895447386	0\\
3.15827895697392	0\\
3.15837895947399	0\\
3.15847896197405	0\\
3.15857896447411	0\\
3.15867896697417	0\\
3.15877896947424	0\\
3.1588789719743	0\\
3.15897897447436	0\\
3.15907897697442	0\\
3.15917897947449	0\\
3.15927898197455	0\\
3.15937898447461	0\\
3.15947898697467	0\\
3.15957898947474	0\\
3.1596789919748	0\\
3.15977899447486	0\\
3.15987899697492	0\\
3.15997899947499	0\\
3.16007900197505	0\\
3.16017900447511	0\\
3.16027900697517	0\\
3.16037900947524	0\\
3.1604790119753	0\\
3.16057901447536	0\\
3.16067901697542	0\\
3.16077901947549	0\\
3.16087902197555	0\\
3.16097902447561	0\\
3.16107902697567	0\\
3.16117902947574	0\\
3.1612790319758	0\\
3.16137903447586	0\\
3.16147903697592	0\\
3.16157903947599	0\\
3.16167904197605	0\\
3.16177904447611	0\\
3.16187904697617	0\\
3.16197904947624	0\\
3.1620790519763	0\\
3.16217905447636	0\\
3.16227905697642	0\\
3.16237905947649	0\\
3.16247906197655	0\\
3.16257906447661	0\\
3.16267906697667	0\\
3.16277906947674	0\\
3.1628790719768	0\\
3.16297907447686	0\\
3.16307907697692	0\\
3.16317907947699	0\\
3.16327908197705	0\\
3.16337908447711	0\\
3.16347908697717	0\\
3.16357908947724	0\\
3.1636790919773	0\\
3.16377909447736	0\\
3.16387909697742	0\\
3.16397909947749	0\\
3.16407910197755	0\\
3.16417910447761	0\\
3.16427910697767	0\\
3.16437910947774	0\\
3.1644791119778	0\\
3.16457911447786	0\\
3.16467911697792	0\\
3.16477911947799	0\\
3.16487912197805	0\\
3.16497912447811	0\\
3.16507912697817	0\\
3.16517912947824	0\\
3.1652791319783	0\\
3.16537913447836	0\\
3.16547913697842	0\\
3.16557913947849	0\\
3.16567914197855	0\\
3.16577914447861	0\\
3.16587914697867	0\\
3.16597914947874	0\\
3.1660791519788	0\\
3.16617915447886	0\\
3.16627915697892	0\\
3.16637915947899	0\\
3.16647916197905	0\\
3.16657916447911	0\\
3.16667916697917	0\\
3.16677916947924	0\\
3.1668791719793	0\\
3.16697917447936	0\\
3.16707917697942	0\\
3.16717917947949	0\\
3.16727918197955	0\\
3.16737918447961	0\\
3.16747918697967	0\\
3.16757918947974	0\\
3.1676791919798	0\\
3.16777919447986	0\\
3.16787919697992	0\\
3.16797919947999	0\\
3.16807920198005	0\\
3.16817920448011	0\\
3.16827920698017	0\\
3.16837920948024	0\\
3.1684792119803	0\\
3.16857921448036	0\\
3.16867921698042	0\\
3.16877921948049	0\\
3.16887922198055	0\\
3.16897922448061	0\\
3.16907922698067	0\\
3.16917922948074	0\\
3.1692792319808	0\\
3.16937923448086	0\\
3.16947923698092	0\\
3.16957923948099	0\\
3.16967924198105	0\\
3.16977924448111	0\\
3.16987924698117	0\\
3.16997924948124	0\\
3.1700792519813	0\\
3.17017925448136	0\\
3.17027925698142	0\\
3.17037925948149	0\\
3.17047926198155	0\\
3.17057926448161	0\\
3.17067926698167	0\\
3.17077926948174	0\\
3.1708792719818	0\\
3.17097927448186	0\\
3.17107927698192	0\\
3.17117927948199	0\\
3.17127928198205	0\\
3.17137928448211	0\\
3.17147928698217	0\\
3.17157928948224	0\\
3.1716792919823	0\\
3.17177929448236	0\\
3.17187929698242	0\\
3.17197929948249	0\\
3.17207930198255	0\\
3.17217930448261	0\\
3.17227930698267	0\\
3.17237930948274	0\\
3.1724793119828	0\\
3.17257931448286	0\\
3.17267931698292	0\\
3.17277931948299	0\\
3.17287932198305	0\\
3.17297932448311	0\\
3.17307932698317	0\\
3.17317932948324	0\\
3.1732793319833	0\\
3.17337933448336	0\\
3.17347933698342	0\\
3.17357933948349	0\\
3.17367934198355	0\\
3.17377934448361	0\\
3.17387934698367	0\\
3.17397934948374	0\\
3.1740793519838	0\\
3.17417935448386	0\\
3.17427935698392	0\\
3.17437935948399	0\\
3.17447936198405	0\\
3.17457936448411	0\\
3.17467936698417	0\\
3.17477936948424	0\\
3.1748793719843	0\\
3.17497937448436	0\\
3.17507937698442	0\\
3.17517937948449	0\\
3.17527938198455	0\\
3.17537938448461	0\\
3.17547938698467	0\\
3.17557938948474	0\\
3.1756793919848	0\\
3.17577939448486	0\\
3.17587939698492	0\\
3.17597939948499	0\\
3.17607940198505	0\\
3.17617940448511	0\\
3.17627940698517	0\\
3.17637940948524	0\\
3.1764794119853	0\\
3.17657941448536	0\\
3.17667941698542	0\\
3.17677941948549	0\\
3.17687942198555	0\\
3.17697942448561	0\\
3.17707942698567	0\\
3.17717942948574	0\\
3.1772794319858	0\\
3.17737943448586	0\\
3.17747943698592	0\\
3.17757943948599	0\\
3.17767944198605	0\\
3.17777944448611	0\\
3.17787944698617	0\\
3.17797944948624	0\\
3.1780794519863	0\\
3.17817945448636	0\\
3.17827945698642	0\\
3.17837945948649	0\\
3.17847946198655	0\\
3.17857946448661	0\\
3.17867946698667	0\\
3.17877946948674	0\\
3.1788794719868	0\\
3.17897947448686	0\\
3.17907947698692	0\\
3.17917947948699	0\\
3.17927948198705	0\\
3.17937948448711	0\\
3.17947948698717	0\\
3.17957948948724	0\\
3.1796794919873	0\\
3.17977949448736	0\\
3.17987949698742	0\\
3.17997949948749	0\\
3.18007950198755	0\\
3.18017950448761	0\\
3.18027950698767	0\\
3.18037950948774	0\\
3.1804795119878	0\\
3.18057951448786	0\\
3.18067951698792	0\\
3.18077951948799	0\\
3.18087952198805	0\\
3.18097952448811	0\\
3.18107952698817	0\\
3.18117952948824	0\\
3.1812795319883	0\\
3.18137953448836	0\\
3.18147953698842	0\\
3.18157953948849	0\\
3.18167954198855	0\\
3.18177954448861	0\\
3.18187954698867	0\\
3.18197954948874	0\\
3.1820795519888	0\\
3.18217955448886	0\\
3.18227955698892	0\\
3.18237955948899	0\\
3.18247956198905	0\\
3.18257956448911	0\\
3.18267956698917	0\\
3.18277956948924	0\\
3.1828795719893	0\\
3.18297957448936	0\\
3.18307957698942	0\\
3.18317957948949	0\\
3.18327958198955	0\\
3.18337958448961	0\\
3.18347958698967	0\\
3.18357958948974	0\\
3.1836795919898	0\\
3.18377959448986	0\\
3.18387959698992	0\\
3.18397959948999	0\\
3.18407960199005	0\\
3.18417960449011	0\\
3.18427960699017	0\\
3.18437960949024	0\\
3.1844796119903	0\\
3.18457961449036	0\\
3.18467961699042	0\\
3.18477961949049	0\\
3.18487962199055	0\\
3.18497962449061	0\\
3.18507962699067	0\\
3.18517962949074	0\\
3.1852796319908	0\\
3.18537963449086	0\\
3.18547963699092	0\\
3.18557963949099	0\\
3.18567964199105	0\\
3.18577964449111	0\\
3.18587964699117	0\\
3.18597964949124	0\\
3.1860796519913	0\\
3.18617965449136	0\\
3.18627965699142	0\\
3.18637965949149	0\\
3.18647966199155	0\\
3.18657966449161	0\\
3.18667966699167	0\\
3.18677966949174	0\\
3.1868796719918	0\\
3.18697967449186	0\\
3.18707967699192	0\\
3.18717967949199	0\\
3.18727968199205	0\\
3.18737968449211	0\\
3.18747968699217	0\\
3.18757968949224	0\\
3.1876796919923	0\\
3.18777969449236	0\\
3.18787969699242	0\\
3.18797969949249	0\\
3.18807970199255	0\\
3.18817970449261	0\\
3.18827970699267	0\\
3.18837970949274	0\\
3.1884797119928	0\\
3.18857971449286	0\\
3.18867971699292	0\\
3.18877971949299	0\\
3.18887972199305	0\\
3.18897972449311	0\\
3.18907972699317	0\\
3.18917972949324	0\\
3.1892797319933	0\\
3.18937973449336	0\\
3.18947973699342	0\\
3.18957973949349	0\\
3.18967974199355	0\\
3.18977974449361	0\\
3.18987974699367	0\\
3.18997974949374	0\\
3.1900797519938	0\\
3.19017975449386	0\\
3.19027975699392	0\\
3.19037975949399	0\\
3.19047976199405	0\\
3.19057976449411	0\\
3.19067976699417	0\\
3.19077976949424	0\\
3.1908797719943	0\\
3.19097977449436	0\\
3.19107977699443	0\\
3.19117977949449	0\\
3.19127978199455	0\\
3.19137978449461	0\\
3.19147978699467	0\\
3.19157978949474	0\\
3.1916797919948	0\\
3.19177979449486	0\\
3.19187979699492	0\\
3.19197979949499	0\\
3.19207980199505	0\\
3.19217980449511	0\\
3.19227980699517	0\\
3.19237980949524	0\\
3.1924798119953	0\\
3.19257981449536	0\\
3.19267981699542	0\\
3.19277981949549	0\\
3.19287982199555	0\\
3.19297982449561	0\\
3.19307982699568	0\\
3.19317982949574	0\\
3.1932798319958	0\\
3.19337983449586	0\\
3.19347983699592	0\\
3.19357983949599	0\\
3.19367984199605	0\\
3.19377984449611	0\\
3.19387984699617	0\\
3.19397984949624	0\\
3.1940798519963	0\\
3.19417985449636	0\\
3.19427985699642	0\\
3.19437985949649	0\\
3.19447986199655	0\\
3.19457986449661	0\\
3.19467986699667	0\\
3.19477986949674	0\\
3.1948798719968	0\\
3.19497987449686	0\\
3.19507987699693	0\\
3.19517987949699	0\\
3.19527988199705	0\\
3.19537988449711	0\\
3.19547988699718	0\\
3.19557988949724	0\\
3.1956798919973	0\\
3.19577989449736	0\\
3.19587989699742	0\\
3.19597989949749	0\\
3.19607990199755	0\\
3.19617990449761	0\\
3.19627990699767	0\\
3.19637990949774	0\\
3.1964799119978	0\\
3.19657991449786	0\\
3.19667991699792	0\\
3.19677991949799	0\\
3.19687992199805	0\\
3.19697992449811	0\\
3.19707992699818	0\\
3.19717992949824	0\\
3.1972799319983	0\\
3.19737993449836	0\\
3.19747993699843	0\\
3.19757993949849	0\\
3.19767994199855	0\\
3.19777994449861	0\\
3.19787994699867	0\\
3.19797994949874	0\\
3.1980799519988	0\\
3.19817995449886	0\\
3.19827995699892	0\\
3.19837995949899	0\\
3.19847996199905	0\\
3.19857996449911	0\\
3.19867996699917	0\\
3.19877996949924	0\\
3.1988799719993	0\\
3.19897997449936	0\\
3.19907997699943	0\\
3.19917997949949	0\\
3.19927998199955	0\\
3.19937998449961	0\\
3.19947998699968	0\\
3.19957998949974	0\\
3.1996799919998	0\\
3.19977999449986	0\\
3.19987999699992	0\\
3.19997999949999	0\\
3.20008000200005	0\\
};
\addplot [color=mycolor2,solid,forget plot]
  table[row sep=crcr]{%
3.20008000200005	0\\
3.20018000450011	0\\
3.20028000700017	0\\
3.20038000950024	0\\
3.2004800120003	0\\
3.20058001450036	0\\
3.20068001700042	0\\
3.20078001950049	0\\
3.20088002200055	0\\
3.20098002450061	0\\
3.20108002700068	0\\
3.20118002950074	0\\
3.2012800320008	0\\
3.20138003450086	0\\
3.20148003700093	0\\
3.20158003950099	0\\
3.20168004200105	0\\
3.20178004450111	0\\
3.20188004700118	0\\
3.20198004950124	0\\
3.2020800520013	0\\
3.20218005450136	0\\
3.20228005700142	0\\
3.20238005950149	0\\
3.20248006200155	0\\
3.20258006450161	0\\
3.20268006700167	0\\
3.20278006950174	0\\
3.2028800720018	0\\
3.20298007450186	0\\
3.20308007700193	0\\
3.20318007950199	0\\
3.20328008200205	0\\
3.20338008450211	0\\
3.20348008700218	0\\
3.20358008950224	0\\
3.2036800920023	0\\
3.20378009450236	0\\
3.20388009700243	0\\
3.20398009950249	0\\
3.20408010200255	0\\
3.20418010450261	0\\
3.20428010700267	0\\
3.20438010950274	0\\
3.2044801120028	0\\
3.20458011450286	0\\
3.20468011700292	0\\
3.20478011950299	0\\
3.20488012200305	0\\
3.20498012450311	0\\
3.20508012700318	0\\
3.20518012950324	0\\
3.2052801320033	0\\
3.20538013450336	0\\
3.20548013700343	0\\
3.20558013950349	0\\
3.20568014200355	0\\
3.20578014450361	0\\
3.20588014700368	0\\
3.20598014950374	0\\
3.2060801520038	0\\
3.20618015450386	0\\
3.20628015700392	0\\
3.20638015950399	0\\
3.20648016200405	0\\
3.20658016450411	0\\
3.20668016700417	0\\
3.20678016950424	0\\
3.2068801720043	0\\
3.20698017450436	0\\
3.20708017700443	0\\
3.20718017950449	0\\
3.20728018200455	0\\
3.20738018450461	0\\
3.20748018700468	0\\
3.20758018950474	0\\
3.2076801920048	0\\
3.20778019450486	0\\
3.20788019700493	0\\
3.20798019950499	0\\
3.20808020200505	0\\
3.20818020450511	0\\
3.20828020700518	0\\
3.20838020950524	0\\
3.2084802120053	0\\
3.20858021450536	0\\
3.20868021700542	0\\
3.20878021950549	0\\
3.20888022200555	0\\
3.20898022450561	0\\
3.20908022700568	0\\
3.20918022950574	0\\
3.2092802320058	0\\
3.20938023450586	0\\
3.20948023700593	0\\
3.20958023950599	0\\
3.20968024200605	0\\
3.20978024450611	0\\
3.20988024700618	0\\
3.20998024950624	0\\
3.2100802520063	0\\
3.21018025450636	0\\
3.21028025700643	0\\
3.21038025950649	0\\
3.21048026200655	0\\
3.21058026450661	0\\
3.21068026700667	0\\
3.21078026950674	0\\
3.2108802720068	0\\
3.21098027450686	0\\
3.21108027700693	0\\
3.21118027950699	0\\
3.21128028200705	0\\
3.21138028450711	0\\
3.21148028700718	0\\
3.21158028950724	0\\
3.2116802920073	0\\
3.21178029450736	0\\
3.21188029700743	0\\
3.21198029950749	0\\
3.21208030200755	0\\
3.21218030450761	0\\
3.21228030700768	0\\
3.21238030950774	0\\
3.2124803120078	0\\
3.21258031450786	0\\
3.21268031700793	0\\
3.21278031950799	0\\
3.21288032200805	0\\
3.21298032450811	0\\
3.21308032700818	0\\
3.21318032950824	0\\
3.2132803320083	0\\
3.21338033450836	0\\
3.21348033700843	0\\
3.21358033950849	0\\
3.21368034200855	0\\
3.21378034450861	0\\
3.21388034700868	0\\
3.21398034950874	0\\
3.2140803520088	0\\
3.21418035450886	0\\
3.21428035700893	0\\
3.21438035950899	0\\
3.21448036200905	0\\
3.21458036450911	0\\
3.21468036700918	0\\
3.21478036950924	0\\
3.2148803720093	0\\
3.21498037450936	0\\
3.21508037700943	0\\
3.21518037950949	0\\
3.21528038200955	0\\
3.21538038450961	0\\
3.21548038700968	0\\
3.21558038950974	0\\
3.2156803920098	0\\
3.21578039450986	0\\
3.21588039700993	0\\
3.21598039950999	0\\
3.21608040201005	0\\
3.21618040451011	0\\
3.21628040701018	0\\
3.21638040951024	0\\
3.2164804120103	0\\
3.21658041451036	0\\
3.21668041701043	0\\
3.21678041951049	0\\
3.21688042201055	0\\
3.21698042451061	0\\
3.21708042701068	0\\
3.21718042951074	0\\
3.2172804320108	0\\
3.21738043451086	0\\
3.21748043701093	0\\
3.21758043951099	0\\
3.21768044201105	0\\
3.21778044451111	0\\
3.21788044701118	0\\
3.21798044951124	0\\
3.2180804520113	0\\
3.21818045451136	0\\
3.21828045701143	0\\
3.21838045951149	0\\
3.21848046201155	0\\
3.21858046451161	0\\
3.21868046701168	0\\
3.21878046951174	0\\
3.2188804720118	0\\
3.21898047451186	0\\
3.21908047701193	0\\
3.21918047951199	0\\
3.21928048201205	0\\
3.21938048451211	0\\
3.21948048701218	0\\
3.21958048951224	0\\
3.2196804920123	0\\
3.21978049451236	0\\
3.21988049701243	0\\
3.21998049951249	0\\
3.22008050201255	0\\
3.22018050451261	0\\
3.22028050701268	0\\
3.22038050951274	0\\
3.2204805120128	0\\
3.22058051451286	0\\
3.22068051701293	0\\
3.22078051951299	0\\
3.22088052201305	0\\
3.22098052451311	0\\
3.22108052701318	0\\
3.22118052951324	0\\
3.2212805320133	0\\
3.22138053451336	0\\
3.22148053701343	0\\
3.22158053951349	0\\
3.22168054201355	0\\
3.22178054451361	0\\
3.22188054701368	0\\
3.22198054951374	0\\
3.2220805520138	0\\
3.22218055451386	0\\
3.22228055701393	0\\
3.22238055951399	0\\
3.22248056201405	0\\
3.22258056451411	0\\
3.22268056701418	0\\
3.22278056951424	0\\
3.2228805720143	0\\
3.22298057451436	0\\
3.22308057701443	0\\
3.22318057951449	0\\
3.22328058201455	0\\
3.22338058451461	0\\
3.22348058701468	0\\
3.22358058951474	0\\
3.2236805920148	0\\
3.22378059451486	0\\
3.22388059701493	0\\
3.22398059951499	0\\
3.22408060201505	0\\
3.22418060451511	0\\
3.22428060701518	0\\
3.22438060951524	0\\
3.2244806120153	0\\
3.22458061451536	0\\
3.22468061701543	0\\
3.22478061951549	0\\
3.22488062201555	0\\
3.22498062451561	0\\
3.22508062701568	0\\
3.22518062951574	0\\
3.2252806320158	0\\
3.22538063451586	0\\
3.22548063701593	0\\
3.22558063951599	0\\
3.22568064201605	0\\
3.22578064451611	0\\
3.22588064701618	0\\
3.22598064951624	0\\
3.2260806520163	0\\
3.22618065451636	0\\
3.22628065701643	0\\
3.22638065951649	0\\
3.22648066201655	0\\
3.22658066451661	0\\
3.22668066701668	0\\
3.22678066951674	0\\
3.2268806720168	0\\
3.22698067451686	0\\
3.22708067701693	0\\
3.22718067951699	0\\
3.22728068201705	0\\
3.22738068451711	0\\
3.22748068701718	0\\
3.22758068951724	0\\
3.2276806920173	0\\
3.22778069451736	0\\
3.22788069701743	0\\
3.22798069951749	0\\
3.22808070201755	0\\
3.22818070451761	0\\
3.22828070701768	0\\
3.22838070951774	0\\
3.2284807120178	0\\
3.22858071451786	0\\
3.22868071701793	0\\
3.22878071951799	0\\
3.22888072201805	0\\
3.22898072451811	0\\
3.22908072701818	0\\
3.22918072951824	0\\
3.2292807320183	0\\
3.22938073451836	0\\
3.22948073701843	0\\
3.22958073951849	0\\
3.22968074201855	0\\
3.22978074451861	0\\
3.22988074701868	0\\
3.22998074951874	0\\
3.2300807520188	0\\
3.23018075451886	0\\
3.23028075701893	0\\
3.23038075951899	0\\
3.23048076201905	0\\
3.23058076451911	0\\
3.23068076701918	0\\
3.23078076951924	0\\
3.2308807720193	0\\
3.23098077451936	0\\
3.23108077701943	0\\
3.23118077951949	0\\
3.23128078201955	0\\
3.23138078451961	0\\
3.23148078701968	0\\
3.23158078951974	0\\
3.2316807920198	0\\
3.23178079451986	0\\
3.23188079701993	0\\
3.23198079951999	0\\
3.23208080202005	0\\
3.23218080452011	0\\
3.23228080702018	0\\
3.23238080952024	0\\
3.2324808120203	0\\
3.23258081452036	0\\
3.23268081702043	0\\
3.23278081952049	0\\
3.23288082202055	0\\
3.23298082452061	0\\
3.23308082702068	0\\
3.23318082952074	0\\
3.2332808320208	0\\
3.23338083452086	0\\
3.23348083702093	0\\
3.23358083952099	0\\
3.23368084202105	0\\
3.23378084452111	0\\
3.23388084702118	0\\
3.23398084952124	0\\
3.2340808520213	0\\
3.23418085452136	0\\
3.23428085702143	0\\
3.23438085952149	0\\
3.23448086202155	0\\
3.23458086452161	0\\
3.23468086702168	0\\
3.23478086952174	0\\
3.2348808720218	0\\
3.23498087452186	0\\
3.23508087702193	0\\
3.23518087952199	0\\
3.23528088202205	0\\
3.23538088452211	0\\
3.23548088702218	0\\
3.23558088952224	0\\
3.2356808920223	0\\
3.23578089452236	0\\
3.23588089702243	0\\
3.23598089952249	0\\
3.23608090202255	0\\
3.23618090452261	0\\
3.23628090702268	0\\
3.23638090952274	0\\
3.2364809120228	0\\
3.23658091452286	0\\
3.23668091702293	0\\
3.23678091952299	0\\
3.23688092202305	0\\
3.23698092452311	0\\
3.23708092702318	0\\
3.23718092952324	0\\
3.2372809320233	0\\
3.23738093452336	0\\
3.23748093702343	0\\
3.23758093952349	0\\
3.23768094202355	0\\
3.23778094452361	0\\
3.23788094702368	0\\
3.23798094952374	0\\
3.2380809520238	0\\
3.23818095452386	0\\
3.23828095702393	0\\
3.23838095952399	0\\
3.23848096202405	0\\
3.23858096452411	0\\
3.23868096702418	0\\
3.23878096952424	0\\
3.2388809720243	0\\
3.23898097452436	0\\
3.23908097702443	0\\
3.23918097952449	0\\
3.23928098202455	0\\
3.23938098452461	0\\
3.23948098702468	0\\
3.23958098952474	0\\
3.2396809920248	0\\
3.23978099452486	0\\
3.23988099702493	0\\
3.23998099952499	0\\
3.24008100202505	0\\
3.24018100452511	0\\
3.24028100702518	0\\
3.24038100952524	0\\
3.2404810120253	0\\
3.24058101452536	0\\
3.24068101702543	0\\
3.24078101952549	0\\
3.24088102202555	0\\
3.24098102452561	0\\
3.24108102702568	0\\
3.24118102952574	0\\
3.2412810320258	0\\
3.24138103452586	0\\
3.24148103702593	0\\
3.24158103952599	0\\
3.24168104202605	0\\
3.24178104452611	0\\
3.24188104702618	0\\
3.24198104952624	0\\
3.2420810520263	0\\
3.24218105452636	0\\
3.24228105702643	0\\
3.24238105952649	0\\
3.24248106202655	0\\
3.24258106452661	0\\
3.24268106702668	0\\
3.24278106952674	0\\
3.2428810720268	0\\
3.24298107452686	0\\
3.24308107702693	0\\
3.24318107952699	0\\
3.24328108202705	0\\
3.24338108452711	0\\
3.24348108702718	0\\
3.24358108952724	0\\
3.2436810920273	0\\
3.24378109452736	0\\
3.24388109702743	0\\
3.24398109952749	0\\
3.24408110202755	0\\
3.24418110452761	0\\
3.24428110702768	0\\
3.24438110952774	0\\
3.2444811120278	0\\
3.24458111452786	0\\
3.24468111702793	0\\
3.24478111952799	0\\
3.24488112202805	0\\
3.24498112452811	0\\
3.24508112702818	0\\
3.24518112952824	0\\
3.2452811320283	0\\
3.24538113452836	0\\
3.24548113702843	0\\
3.24558113952849	0\\
3.24568114202855	0\\
3.24578114452861	0\\
3.24588114702868	0\\
3.24598114952874	0\\
3.2460811520288	0\\
3.24618115452886	0\\
3.24628115702893	0\\
3.24638115952899	0\\
3.24648116202905	0\\
3.24658116452911	0\\
3.24668116702918	0\\
3.24678116952924	0\\
3.2468811720293	0\\
3.24698117452936	0\\
3.24708117702943	0\\
3.24718117952949	0\\
3.24728118202955	0\\
3.24738118452961	0\\
3.24748118702968	0\\
3.24758118952974	0\\
3.2476811920298	0\\
3.24778119452986	0\\
3.24788119702993	0\\
3.24798119952999	0\\
3.24808120203005	0\\
3.24818120453011	0\\
3.24828120703018	0\\
3.24838120953024	0\\
3.2484812120303	0\\
3.24858121453036	0\\
3.24868121703043	0\\
3.24878121953049	0\\
3.24888122203055	0\\
3.24898122453061	0\\
3.24908122703068	0\\
3.24918122953074	0\\
3.2492812320308	0\\
3.24938123453086	0\\
3.24948123703093	0\\
3.24958123953099	0\\
3.24968124203105	0\\
3.24978124453111	0\\
3.24988124703118	0\\
3.24998124953124	0\\
3.2500812520313	0\\
3.25018125453136	0\\
3.25028125703143	0\\
3.25038125953149	0\\
3.25048126203155	0\\
3.25058126453161	0\\
3.25068126703168	0\\
3.25078126953174	0\\
3.2508812720318	0\\
3.25098127453186	0\\
3.25108127703193	0\\
3.25118127953199	0\\
3.25128128203205	0\\
3.25138128453211	0\\
3.25148128703218	0\\
3.25158128953224	0\\
3.2516812920323	0\\
3.25178129453236	0\\
3.25188129703243	0\\
3.25198129953249	0\\
3.25208130203255	0\\
3.25218130453261	0\\
3.25228130703268	0\\
3.25238130953274	0\\
3.2524813120328	0\\
3.25258131453286	0\\
3.25268131703293	0\\
3.25278131953299	0\\
3.25288132203305	0\\
3.25298132453311	0\\
3.25308132703318	0\\
3.25318132953324	0\\
3.2532813320333	0\\
3.25338133453336	0\\
3.25348133703343	0\\
3.25358133953349	0\\
3.25368134203355	0\\
3.25378134453361	0\\
3.25388134703368	0\\
3.25398134953374	0\\
3.2540813520338	0\\
3.25418135453386	0\\
3.25428135703393	0\\
3.25438135953399	0\\
3.25448136203405	0\\
3.25458136453411	0\\
3.25468136703418	0\\
3.25478136953424	0\\
3.2548813720343	0\\
3.25498137453436	0\\
3.25508137703443	0\\
3.25518137953449	0\\
3.25528138203455	0\\
3.25538138453461	0\\
3.25548138703468	0\\
3.25558138953474	0\\
3.2556813920348	0\\
3.25578139453486	0\\
3.25588139703493	0\\
3.25598139953499	0\\
3.25608140203505	0\\
3.25618140453511	0\\
3.25628140703518	0\\
3.25638140953524	0\\
3.2564814120353	0\\
3.25658141453536	0\\
3.25668141703543	0\\
3.25678141953549	0\\
3.25688142203555	0\\
3.25698142453561	0\\
3.25708142703568	0\\
3.25718142953574	0\\
3.2572814320358	0\\
3.25738143453586	0\\
3.25748143703593	0\\
3.25758143953599	0\\
3.25768144203605	0\\
3.25778144453611	0\\
3.25788144703618	0\\
3.25798144953624	0\\
3.2580814520363	0\\
3.25818145453636	0\\
3.25828145703643	0\\
3.25838145953649	0\\
3.25848146203655	0\\
3.25858146453661	0\\
3.25868146703668	0\\
3.25878146953674	0\\
3.2588814720368	0\\
3.25898147453686	0\\
3.25908147703693	0\\
3.25918147953699	0\\
3.25928148203705	0\\
3.25938148453711	0\\
3.25948148703718	0\\
3.25958148953724	0\\
3.2596814920373	0\\
3.25978149453736	0\\
3.25988149703743	0\\
3.25998149953749	0\\
3.26008150203755	0\\
3.26018150453761	0\\
3.26028150703768	0\\
3.26038150953774	0\\
3.2604815120378	0\\
3.26058151453786	0\\
3.26068151703793	0\\
3.26078151953799	0\\
3.26088152203805	0\\
3.26098152453811	0\\
3.26108152703818	0\\
3.26118152953824	0\\
3.2612815320383	0\\
3.26138153453836	0\\
3.26148153703843	0\\
3.26158153953849	0\\
3.26168154203855	0\\
3.26178154453861	0\\
3.26188154703868	0\\
3.26198154953874	0\\
3.2620815520388	0\\
3.26218155453886	0\\
3.26228155703893	0\\
3.26238155953899	0\\
3.26248156203905	0\\
3.26258156453911	0\\
3.26268156703918	0\\
3.26278156953924	0\\
3.2628815720393	0\\
3.26298157453936	0\\
3.26308157703943	0\\
3.26318157953949	0\\
3.26328158203955	0\\
3.26338158453961	0\\
3.26348158703968	0\\
3.26358158953974	0\\
3.2636815920398	0\\
3.26378159453986	0\\
3.26388159703993	0\\
3.26398159953999	0\\
3.26408160204005	0\\
3.26418160454011	0\\
3.26428160704018	0\\
3.26438160954024	0\\
3.2644816120403	0\\
3.26458161454036	0\\
3.26468161704043	0\\
3.26478161954049	0\\
3.26488162204055	0\\
3.26498162454061	0\\
3.26508162704068	0\\
3.26518162954074	0\\
3.2652816320408	0\\
3.26538163454086	0\\
3.26548163704093	0\\
3.26558163954099	0\\
3.26568164204105	0\\
3.26578164454111	0\\
3.26588164704118	0\\
3.26598164954124	0\\
3.2660816520413	0\\
3.26618165454136	0\\
3.26628165704143	0\\
3.26638165954149	0\\
3.26648166204155	0\\
3.26658166454161	0\\
3.26668166704168	0\\
3.26678166954174	0\\
3.2668816720418	0\\
3.26698167454186	0\\
3.26708167704193	0\\
3.26718167954199	0\\
3.26728168204205	0\\
3.26738168454211	0\\
3.26748168704218	0\\
3.26758168954224	0\\
3.2676816920423	0\\
3.26778169454236	0\\
3.26788169704243	0\\
3.26798169954249	0\\
3.26808170204255	0\\
3.26818170454261	0\\
3.26828170704268	0\\
3.26838170954274	0\\
3.2684817120428	0\\
3.26858171454286	0\\
3.26868171704293	0\\
3.26878171954299	0\\
3.26888172204305	0\\
3.26898172454311	0\\
3.26908172704318	0\\
3.26918172954324	0\\
3.2692817320433	0\\
3.26938173454336	0\\
3.26948173704343	0\\
3.26958173954349	0\\
3.26968174204355	0\\
3.26978174454361	0\\
3.26988174704368	0\\
3.26998174954374	0\\
3.2700817520438	0\\
3.27018175454386	0\\
3.27028175704393	0\\
3.27038175954399	0\\
3.27048176204405	0\\
3.27058176454411	0\\
3.27068176704418	0\\
3.27078176954424	0\\
3.2708817720443	0\\
3.27098177454436	0\\
3.27108177704443	0\\
3.27118177954449	0\\
3.27128178204455	0\\
3.27138178454461	0\\
3.27148178704468	0\\
3.27158178954474	0\\
3.2716817920448	0\\
3.27178179454486	0\\
3.27188179704493	0\\
3.27198179954499	0\\
3.27208180204505	0\\
3.27218180454511	0\\
3.27228180704518	0\\
3.27238180954524	0\\
3.2724818120453	0\\
3.27258181454536	0\\
3.27268181704543	0\\
3.27278181954549	0\\
3.27288182204555	0\\
3.27298182454561	0\\
3.27308182704568	0\\
3.27318182954574	0\\
3.2732818320458	0\\
3.27338183454586	0\\
3.27348183704593	0\\
3.27358183954599	0\\
3.27368184204605	0\\
3.27378184454611	0\\
3.27388184704618	0\\
3.27398184954624	0\\
3.2740818520463	0\\
3.27418185454636	0\\
3.27428185704643	0\\
3.27438185954649	0\\
3.27448186204655	0\\
3.27458186454661	0\\
3.27468186704668	0\\
3.27478186954674	0\\
3.2748818720468	0\\
3.27498187454686	0\\
3.27508187704693	0\\
3.27518187954699	0\\
3.27528188204705	0\\
3.27538188454711	0\\
3.27548188704718	0\\
3.27558188954724	0\\
3.2756818920473	0\\
3.27578189454736	0\\
3.27588189704743	0\\
3.27598189954749	0\\
3.27608190204755	0\\
3.27618190454761	0\\
3.27628190704768	0\\
3.27638190954774	0\\
3.2764819120478	0\\
3.27658191454786	0\\
3.27668191704793	0\\
3.27678191954799	0\\
3.27688192204805	0\\
3.27698192454811	0\\
3.27708192704818	0\\
3.27718192954824	0\\
3.2772819320483	0\\
3.27738193454836	0\\
3.27748193704843	0\\
3.27758193954849	0\\
3.27768194204855	0\\
3.27778194454861	0\\
3.27788194704868	0\\
3.27798194954874	0\\
3.2780819520488	0\\
3.27818195454886	0\\
3.27828195704893	0\\
3.27838195954899	0\\
3.27848196204905	0\\
3.27858196454911	0\\
3.27868196704918	0\\
3.27878196954924	0\\
3.2788819720493	0\\
3.27898197454936	0\\
3.27908197704943	0\\
3.27918197954949	0\\
3.27928198204955	0\\
3.27938198454961	0\\
3.27948198704968	0\\
3.27958198954974	0\\
3.2796819920498	0\\
3.27978199454986	0\\
3.27988199704993	0\\
3.27998199954999	0\\
3.28008200205005	0\\
3.28018200455011	0\\
3.28028200705018	0\\
3.28038200955024	0\\
3.2804820120503	0\\
3.28058201455036	0\\
3.28068201705043	0\\
3.28078201955049	0\\
3.28088202205055	0\\
3.28098202455061	0\\
3.28108202705068	0\\
3.28118202955074	0\\
3.2812820320508	0\\
3.28138203455086	0\\
3.28148203705093	0\\
3.28158203955099	0\\
3.28168204205105	0\\
3.28178204455111	0\\
3.28188204705118	0\\
3.28198204955124	0\\
3.2820820520513	0\\
3.28218205455136	0\\
3.28228205705143	0\\
3.28238205955149	0\\
3.28248206205155	0\\
3.28258206455161	0\\
3.28268206705168	0\\
3.28278206955174	0\\
3.2828820720518	0\\
3.28298207455186	0\\
3.28308207705193	0\\
3.28318207955199	0\\
3.28328208205205	0\\
3.28338208455211	0\\
3.28348208705218	0\\
3.28358208955224	0\\
3.2836820920523	0\\
3.28378209455236	0\\
3.28388209705243	0\\
3.28398209955249	0\\
3.28408210205255	0\\
3.28418210455261	0\\
3.28428210705268	0\\
3.28438210955274	0\\
3.2844821120528	0\\
3.28458211455286	0\\
3.28468211705293	0\\
3.28478211955299	0\\
3.28488212205305	0\\
3.28498212455311	0\\
3.28508212705318	0\\
3.28518212955324	0\\
3.2852821320533	0\\
3.28538213455336	0\\
3.28548213705343	0\\
3.28558213955349	0\\
3.28568214205355	0\\
3.28578214455361	0\\
3.28588214705368	0\\
3.28598214955374	0\\
3.2860821520538	0\\
3.28618215455386	0\\
3.28628215705393	0\\
3.28638215955399	0\\
3.28648216205405	0\\
3.28658216455411	0\\
3.28668216705418	0\\
3.28678216955424	0\\
3.2868821720543	0\\
3.28698217455436	0\\
3.28708217705443	0\\
3.28718217955449	0\\
3.28728218205455	0\\
3.28738218455461	0\\
3.28748218705468	0\\
3.28758218955474	0\\
3.2876821920548	0\\
3.28778219455486	0\\
3.28788219705493	0\\
3.28798219955499	0\\
3.28808220205505	0\\
3.28818220455511	0\\
3.28828220705518	0\\
3.28838220955524	0\\
3.2884822120553	0\\
3.28858221455536	0\\
3.28868221705543	0\\
3.28878221955549	0\\
3.28888222205555	0\\
3.28898222455561	0\\
3.28908222705568	0\\
3.28918222955574	0\\
3.2892822320558	0\\
3.28938223455586	0\\
3.28948223705593	0\\
3.28958223955599	0\\
3.28968224205605	0\\
3.28978224455611	0\\
3.28988224705618	0\\
3.28998224955624	0\\
3.2900822520563	0\\
3.29018225455636	0\\
3.29028225705643	0\\
3.29038225955649	0\\
3.29048226205655	0\\
3.29058226455661	0\\
3.29068226705668	0\\
3.29078226955674	0\\
3.2908822720568	0\\
3.29098227455686	0\\
3.29108227705693	0\\
3.29118227955699	0\\
3.29128228205705	0\\
3.29138228455711	0\\
3.29148228705718	0\\
3.29158228955724	0\\
3.2916822920573	0\\
3.29178229455736	0\\
3.29188229705743	0\\
3.29198229955749	0\\
3.29208230205755	0\\
3.29218230455761	0\\
3.29228230705768	0\\
3.29238230955774	0\\
3.2924823120578	0\\
3.29258231455786	0\\
3.29268231705793	0\\
3.29278231955799	0\\
3.29288232205805	0\\
3.29298232455811	0\\
3.29308232705818	0\\
3.29318232955824	0\\
3.2932823320583	0\\
3.29338233455836	0\\
3.29348233705843	0\\
3.29358233955849	0\\
3.29368234205855	0\\
3.29378234455861	0\\
3.29388234705868	0\\
3.29398234955874	0\\
3.2940823520588	0\\
3.29418235455886	0\\
3.29428235705893	0\\
3.29438235955899	0\\
3.29448236205905	0\\
3.29458236455911	0\\
3.29468236705918	0\\
3.29478236955924	0\\
3.2948823720593	0\\
3.29498237455936	0\\
3.29508237705943	0\\
3.29518237955949	0\\
3.29528238205955	0\\
3.29538238455961	0\\
3.29548238705968	0\\
3.29558238955974	0\\
3.2956823920598	0\\
3.29578239455986	0\\
3.29588239705993	0\\
3.29598239955999	0\\
3.29608240206005	0\\
3.29618240456011	0\\
3.29628240706018	0\\
3.29638240956024	0\\
3.2964824120603	0\\
3.29658241456036	0\\
3.29668241706043	0\\
3.29678241956049	0\\
3.29688242206055	0\\
3.29698242456061	0\\
3.29708242706068	0\\
3.29718242956074	0\\
3.2972824320608	0\\
3.29738243456086	0\\
3.29748243706093	0\\
3.29758243956099	0\\
3.29768244206105	0\\
3.29778244456111	0\\
3.29788244706118	0\\
3.29798244956124	0\\
3.2980824520613	0\\
3.29818245456136	0\\
3.29828245706143	0\\
3.29838245956149	0\\
3.29848246206155	0\\
3.29858246456161	0\\
3.29868246706168	0\\
3.29878246956174	0\\
3.2988824720618	0\\
3.29898247456186	0\\
3.29908247706193	0\\
3.29918247956199	0\\
3.29928248206205	0\\
3.29938248456211	0\\
3.29948248706218	0\\
3.29958248956224	0\\
3.2996824920623	0\\
3.29978249456236	0\\
3.29988249706243	0\\
3.29998249956249	0\\
3.30008250206255	0\\
3.30018250456261	0\\
3.30028250706268	0\\
3.30038250956274	0\\
3.3004825120628	0\\
3.30058251456286	0\\
3.30068251706293	0\\
3.30078251956299	0\\
3.30088252206305	0\\
3.30098252456311	0\\
3.30108252706318	0\\
3.30118252956324	0\\
3.3012825320633	0\\
3.30138253456336	0\\
3.30148253706343	0\\
3.30158253956349	0\\
3.30168254206355	0\\
3.30178254456361	0\\
3.30188254706368	0\\
3.30198254956374	0\\
3.3020825520638	0\\
3.30218255456386	0\\
3.30228255706393	0\\
3.30238255956399	0\\
3.30248256206405	0\\
3.30258256456411	0\\
3.30268256706418	0\\
3.30278256956424	0\\
3.3028825720643	0\\
3.30298257456436	0\\
3.30308257706443	0\\
3.30318257956449	0\\
3.30328258206455	0\\
3.30338258456461	0\\
3.30348258706468	0\\
3.30358258956474	0\\
3.3036825920648	0\\
3.30378259456486	0\\
3.30388259706493	0\\
3.30398259956499	0\\
3.30408260206505	0\\
3.30418260456511	0\\
3.30428260706518	0\\
3.30438260956524	0\\
3.3044826120653	0\\
3.30458261456536	0\\
3.30468261706543	0\\
3.30478261956549	0\\
3.30488262206555	0\\
3.30498262456561	0\\
3.30508262706568	0\\
3.30518262956574	0\\
3.3052826320658	0\\
3.30538263456586	0\\
3.30548263706593	0\\
3.30558263956599	0\\
3.30568264206605	0\\
3.30578264456611	0\\
3.30588264706618	0\\
3.30598264956624	0\\
3.3060826520663	0\\
3.30618265456636	0\\
3.30628265706643	0\\
3.30638265956649	0\\
3.30648266206655	0\\
3.30658266456661	0\\
3.30668266706668	0\\
3.30678266956674	0\\
3.3068826720668	0\\
3.30698267456686	0\\
3.30708267706693	0\\
3.30718267956699	0\\
3.30728268206705	0\\
3.30738268456711	0\\
3.30748268706718	0\\
3.30758268956724	0\\
3.3076826920673	0\\
3.30778269456736	0\\
3.30788269706743	0\\
3.30798269956749	0\\
3.30808270206755	0\\
3.30818270456761	0\\
3.30828270706768	0\\
3.30838270956774	0\\
3.3084827120678	0\\
3.30858271456786	0\\
3.30868271706793	0\\
3.30878271956799	0\\
3.30888272206805	0\\
3.30898272456811	0\\
3.30908272706818	0\\
3.30918272956824	0\\
3.3092827320683	0\\
3.30938273456836	0\\
3.30948273706843	0\\
3.30958273956849	0\\
3.30968274206855	0\\
3.30978274456861	0\\
3.30988274706868	0\\
3.30998274956874	0\\
3.3100827520688	0\\
3.31018275456886	0\\
3.31028275706893	0\\
3.31038275956899	0\\
3.31048276206905	0\\
3.31058276456911	0\\
3.31068276706918	0\\
3.31078276956924	0\\
3.3108827720693	0\\
3.31098277456936	0\\
3.31108277706943	0\\
3.31118277956949	0\\
3.31128278206955	0\\
3.31138278456961	0\\
3.31148278706968	0\\
3.31158278956974	0\\
3.3116827920698	0\\
3.31178279456986	0\\
3.31188279706993	0\\
3.31198279956999	0\\
3.31208280207005	0\\
3.31218280457011	0\\
3.31228280707018	0\\
3.31238280957024	0\\
3.3124828120703	0\\
3.31258281457036	0\\
3.31268281707043	0\\
3.31278281957049	0\\
3.31288282207055	0\\
3.31298282457061	0\\
3.31308282707068	0\\
3.31318282957074	0\\
3.3132828320708	0\\
3.31338283457086	0\\
3.31348283707093	0\\
3.31358283957099	0\\
3.31368284207105	0\\
3.31378284457111	0\\
3.31388284707118	0\\
3.31398284957124	0\\
3.3140828520713	0\\
3.31418285457136	0\\
3.31428285707143	0\\
3.31438285957149	0\\
3.31448286207155	0\\
3.31458286457161	0\\
3.31468286707168	0\\
3.31478286957174	0\\
3.3148828720718	0\\
3.31498287457186	0\\
3.31508287707193	0\\
3.31518287957199	0\\
3.31528288207205	0\\
3.31538288457211	0\\
3.31548288707218	0\\
3.31558288957224	0\\
3.3156828920723	0\\
3.31578289457236	0\\
3.31588289707243	0\\
3.31598289957249	0\\
3.31608290207255	0\\
3.31618290457261	0\\
3.31628290707268	0\\
3.31638290957274	0\\
3.3164829120728	0\\
3.31658291457286	0\\
3.31668291707293	0\\
3.31678291957299	0\\
3.31688292207305	0\\
3.31698292457311	0\\
3.31708292707318	0\\
3.31718292957324	0\\
3.3172829320733	0\\
3.31738293457336	0\\
3.31748293707343	0\\
3.31758293957349	0\\
3.31768294207355	0\\
3.31778294457361	0\\
3.31788294707368	0\\
3.31798294957374	0\\
3.3180829520738	0\\
3.31818295457386	0\\
3.31828295707393	0\\
3.31838295957399	0\\
3.31848296207405	0\\
3.31858296457411	0\\
3.31868296707418	0\\
3.31878296957424	0\\
3.3188829720743	0\\
3.31898297457436	0\\
3.31908297707443	0\\
3.31918297957449	0\\
3.31928298207455	0\\
3.31938298457461	0\\
3.31948298707468	0\\
3.31958298957474	0\\
3.3196829920748	0\\
3.31978299457486	0\\
3.31988299707493	0\\
3.31998299957499	0\\
3.32008300207505	0\\
3.32018300457511	0\\
3.32028300707518	0\\
3.32038300957524	0\\
3.3204830120753	0\\
3.32058301457536	0\\
3.32068301707543	0\\
3.32078301957549	0\\
3.32088302207555	0\\
3.32098302457561	0\\
3.32108302707568	0\\
3.32118302957574	0\\
3.3212830320758	0\\
3.32138303457586	0\\
3.32148303707593	0\\
3.32158303957599	0\\
3.32168304207605	0\\
3.32178304457611	0\\
3.32188304707618	0\\
3.32198304957624	0\\
3.3220830520763	0\\
3.32218305457636	0\\
3.32228305707643	0\\
3.32238305957649	0\\
3.32248306207655	0\\
3.32258306457661	0\\
3.32268306707668	0\\
3.32278306957674	0\\
3.3228830720768	0\\
3.32298307457686	0\\
3.32308307707693	0\\
3.32318307957699	0\\
3.32328308207705	0\\
3.32338308457711	0\\
3.32348308707718	0\\
3.32358308957724	0\\
3.3236830920773	0\\
3.32378309457736	0\\
3.32388309707743	0\\
3.32398309957749	0\\
3.32408310207755	0\\
3.32418310457761	0\\
3.32428310707768	0\\
3.32438310957774	0\\
3.3244831120778	0\\
3.32458311457786	0\\
3.32468311707793	0\\
3.32478311957799	0\\
3.32488312207805	0\\
3.32498312457811	0\\
3.32508312707818	0\\
3.32518312957824	0\\
3.3252831320783	0\\
3.32538313457836	0\\
3.32548313707843	0\\
3.32558313957849	0\\
3.32568314207855	0\\
3.32578314457861	0\\
3.32588314707868	0\\
3.32598314957874	0\\
3.3260831520788	0\\
3.32618315457886	0\\
3.32628315707893	0\\
3.32638315957899	0\\
3.32648316207905	0\\
3.32658316457911	0\\
3.32668316707918	0\\
3.32678316957924	0\\
3.3268831720793	0\\
3.32698317457936	0\\
3.32708317707943	0\\
3.32718317957949	0\\
3.32728318207955	0\\
3.32738318457961	0\\
3.32748318707968	0\\
3.32758318957974	0\\
3.3276831920798	0\\
3.32778319457986	0\\
3.32788319707993	0\\
3.32798319957999	0\\
3.32808320208005	0\\
3.32818320458011	0\\
3.32828320708018	0\\
3.32838320958024	0\\
3.3284832120803	0\\
3.32858321458036	0\\
3.32868321708043	0\\
3.32878321958049	0\\
3.32888322208055	0\\
3.32898322458061	0\\
3.32908322708068	0\\
3.32918322958074	0\\
3.3292832320808	0\\
3.32938323458086	0\\
3.32948323708093	0\\
3.32958323958099	0\\
3.32968324208105	0\\
3.32978324458111	0\\
3.32988324708118	0\\
3.32998324958124	0\\
3.3300832520813	0\\
3.33018325458136	0\\
3.33028325708143	0\\
3.33038325958149	0\\
3.33048326208155	0\\
3.33058326458161	0\\
3.33068326708168	0\\
3.33078326958174	0\\
3.3308832720818	0\\
3.33098327458186	0\\
3.33108327708193	0\\
3.33118327958199	0\\
3.33128328208205	0\\
3.33138328458211	0\\
3.33148328708218	0\\
3.33158328958224	0\\
3.3316832920823	0\\
3.33178329458236	0\\
3.33188329708243	0\\
3.33198329958249	0\\
3.33208330208255	0\\
3.33218330458261	0\\
3.33228330708268	0\\
3.33238330958274	0\\
3.3324833120828	0\\
3.33258331458286	0\\
3.33268331708293	0\\
3.33278331958299	0\\
3.33288332208305	0\\
3.33298332458311	0\\
3.33308332708318	0\\
3.33318332958324	0\\
3.3332833320833	0\\
3.33338333458336	0\\
3.33348333708343	0\\
3.33358333958349	0\\
3.33368334208355	0\\
3.33378334458361	0\\
3.33388334708368	0\\
3.33398334958374	0\\
3.3340833520838	0\\
3.33418335458386	0\\
3.33428335708393	0\\
3.33438335958399	0\\
3.33448336208405	0\\
3.33458336458411	0\\
3.33468336708418	0\\
3.33478336958424	0\\
3.3348833720843	0\\
3.33498337458436	0\\
3.33508337708443	0\\
3.33518337958449	0\\
3.33528338208455	0\\
3.33538338458461	0\\
3.33548338708468	0\\
3.33558338958474	0\\
3.3356833920848	0\\
3.33578339458486	0\\
3.33588339708493	0\\
3.33598339958499	0\\
3.33608340208505	0\\
3.33618340458511	0\\
3.33628340708518	0\\
3.33638340958524	0\\
3.3364834120853	0\\
3.33658341458536	0\\
3.33668341708543	0\\
3.33678341958549	0\\
3.33688342208555	0\\
3.33698342458561	0\\
3.33708342708568	0\\
3.33718342958574	0\\
3.3372834320858	0\\
3.33738343458586	0\\
3.33748343708593	0\\
3.33758343958599	0\\
3.33768344208605	0\\
3.33778344458611	0\\
3.33788344708618	0\\
3.33798344958624	0\\
3.3380834520863	0\\
3.33818345458636	0\\
3.33828345708643	0\\
3.33838345958649	0\\
3.33848346208655	0\\
3.33858346458661	0\\
3.33868346708668	0\\
3.33878346958674	0\\
3.3388834720868	0\\
3.33898347458686	0\\
3.33908347708693	0\\
3.33918347958699	0\\
3.33928348208705	0\\
3.33938348458711	0\\
3.33948348708718	0\\
3.33958348958724	0\\
3.3396834920873	0\\
3.33978349458736	0\\
3.33988349708743	0\\
3.33998349958749	0\\
3.34008350208755	0\\
3.34018350458761	0\\
3.34028350708768	0\\
3.34038350958774	0\\
3.3404835120878	0\\
3.34058351458786	0\\
3.34068351708793	0\\
3.34078351958799	0\\
3.34088352208805	0\\
3.34098352458811	0\\
3.34108352708818	0\\
3.34118352958824	0\\
3.3412835320883	0\\
3.34138353458836	0\\
3.34148353708843	0\\
3.34158353958849	0\\
3.34168354208855	0\\
3.34178354458861	0\\
3.34188354708868	0\\
3.34198354958874	0\\
3.3420835520888	0\\
3.34218355458886	0\\
3.34228355708893	0\\
3.34238355958899	0\\
3.34248356208905	0\\
3.34258356458911	0\\
3.34268356708918	0\\
3.34278356958924	0\\
3.3428835720893	0\\
3.34298357458936	0\\
3.34308357708943	0\\
3.34318357958949	0\\
3.34328358208955	0\\
3.34338358458961	0\\
3.34348358708968	0\\
3.34358358958974	0\\
3.3436835920898	0\\
3.34378359458986	0\\
3.34388359708993	0\\
3.34398359958999	0\\
3.34408360209005	0\\
3.34418360459011	0\\
3.34428360709018	0\\
3.34438360959024	0\\
3.3444836120903	0\\
3.34458361459036	0\\
3.34468361709043	0\\
3.34478361959049	0\\
3.34488362209055	0\\
3.34498362459061	0\\
3.34508362709068	0\\
3.34518362959074	0\\
3.3452836320908	0\\
3.34538363459086	0\\
3.34548363709093	0\\
3.34558363959099	0\\
3.34568364209105	0\\
3.34578364459111	0\\
3.34588364709118	0\\
3.34598364959124	0\\
3.3460836520913	0\\
3.34618365459136	0\\
3.34628365709143	0\\
3.34638365959149	0\\
3.34648366209155	0\\
3.34658366459161	0\\
3.34668366709168	0\\
3.34678366959174	0\\
3.3468836720918	0\\
3.34698367459186	0\\
3.34708367709193	0\\
3.34718367959199	0\\
3.34728368209205	0\\
3.34738368459211	0\\
3.34748368709218	0\\
3.34758368959224	0\\
3.3476836920923	0\\
3.34778369459236	0\\
3.34788369709243	0\\
3.34798369959249	0\\
3.34808370209255	0\\
3.34818370459261	0\\
3.34828370709268	0\\
3.34838370959274	0\\
3.3484837120928	0\\
3.34858371459286	0\\
3.34868371709293	0\\
3.34878371959299	0\\
3.34888372209305	0\\
3.34898372459311	0\\
3.34908372709318	0\\
3.34918372959324	0\\
3.3492837320933	0\\
3.34938373459336	0\\
3.34948373709343	0\\
3.34958373959349	0\\
3.34968374209355	0\\
3.34978374459361	0\\
3.34988374709368	0\\
3.34998374959374	0\\
3.3500837520938	0\\
3.35018375459386	0\\
3.35028375709393	0\\
3.35038375959399	0\\
3.35048376209405	0\\
3.35058376459411	0\\
3.35068376709418	0\\
3.35078376959424	0\\
3.3508837720943	0\\
3.35098377459436	0\\
3.35108377709443	0\\
3.35118377959449	0\\
3.35128378209455	0\\
3.35138378459461	0\\
3.35148378709468	0\\
3.35158378959474	0\\
3.3516837920948	0\\
3.35178379459486	0\\
3.35188379709493	0\\
3.35198379959499	0\\
3.35208380209505	0\\
3.35218380459511	0\\
3.35228380709518	0\\
3.35238380959524	0\\
3.3524838120953	0\\
3.35258381459537	0\\
3.35268381709543	0\\
3.35278381959549	0\\
3.35288382209555	0\\
3.35298382459561	0\\
3.35308382709568	0\\
3.35318382959574	0\\
3.3532838320958	0\\
3.35338383459586	0\\
3.35348383709593	0\\
3.35358383959599	0\\
3.35368384209605	0\\
3.35378384459611	0\\
3.35388384709618	0\\
3.35398384959624	0\\
3.3540838520963	0\\
3.35418385459636	0\\
3.35428385709643	0\\
3.35438385959649	0\\
3.35448386209655	0\\
3.35458386459662	0\\
3.35468386709668	0\\
3.35478386959674	0\\
3.3548838720968	0\\
3.35498387459686	0\\
3.35508387709693	0\\
3.35518387959699	0\\
3.35528388209705	0\\
3.35538388459711	0\\
3.35548388709718	0\\
3.35558388959724	0\\
3.3556838920973	0\\
3.35578389459736	0\\
3.35588389709743	0\\
3.35598389959749	0\\
3.35608390209755	0\\
3.35618390459761	0\\
3.35628390709768	0\\
3.35638390959774	0\\
3.3564839120978	0\\
3.35658391459787	0\\
3.35668391709793	0\\
3.35678391959799	0\\
3.35688392209805	0\\
3.35698392459812	0\\
3.35708392709818	0\\
3.35718392959824	0\\
3.3572839320983	0\\
3.35738393459836	0\\
3.35748393709843	0\\
3.35758393959849	0\\
3.35768394209855	0\\
3.35778394459861	0\\
3.35788394709868	0\\
3.35798394959874	0\\
3.3580839520988	0\\
3.35818395459886	0\\
3.35828395709893	0\\
3.35838395959899	0\\
3.35848396209905	0\\
3.35858396459912	0\\
3.35868396709918	0\\
3.35878396959924	0\\
3.3588839720993	0\\
3.35898397459937	0\\
3.35908397709943	0\\
3.35918397959949	0\\
3.35928398209955	0\\
3.35938398459961	0\\
3.35948398709968	0\\
3.35958398959974	0\\
3.3596839920998	0\\
3.35978399459986	0\\
3.35988399709993	0\\
3.35998399959999	0\\
3.36008400210005	0\\
3.36018400460011	0\\
3.36028400710018	0\\
3.36038400960024	0\\
3.3604840121003	0\\
3.36058401460037	0\\
3.36068401710043	0\\
3.36078401960049	0\\
3.36088402210055	0\\
3.36098402460062	0\\
3.36108402710068	0\\
3.36118402960074	0\\
3.3612840321008	0\\
3.36138403460086	0\\
3.36148403710093	0\\
3.36158403960099	0\\
3.36168404210105	0\\
3.36178404460111	0\\
3.36188404710118	0\\
3.36198404960124	0\\
3.3620840521013	0\\
3.36218405460136	0\\
3.36228405710143	0\\
3.36238405960149	0\\
3.36248406210155	0\\
3.36258406460162	0\\
3.36268406710168	0\\
3.36278406960174	0\\
3.3628840721018	0\\
3.36298407460187	0\\
3.36308407710193	0\\
3.36318407960199	0\\
3.36328408210205	0\\
3.36338408460212	0\\
3.36348408710218	0\\
3.36358408960224	0\\
3.3636840921023	0\\
3.36378409460236	0\\
3.36388409710243	0\\
3.36398409960249	0\\
3.36408410210255	0\\
3.36418410460261	0\\
3.36428410710268	0\\
3.36438410960274	0\\
3.3644841121028	0\\
3.36458411460287	0\\
3.36468411710293	0\\
3.36478411960299	0\\
3.36488412210305	0\\
3.36498412460312	0\\
3.36508412710318	0\\
3.36518412960324	0\\
3.3652841321033	0\\
3.36538413460337	0\\
3.36548413710343	0\\
3.36558413960349	0\\
3.36568414210355	0\\
3.36578414460361	0\\
3.36588414710368	0\\
3.36598414960374	0\\
3.3660841521038	0\\
3.36618415460386	0\\
3.36628415710393	0\\
3.36638415960399	0\\
3.36648416210405	0\\
3.36658416460412	0\\
3.36668416710418	0\\
3.36678416960424	0\\
3.3668841721043	0\\
3.36698417460437	0\\
3.36708417710443	0\\
3.36718417960449	0\\
3.36728418210455	0\\
3.36738418460462	0\\
3.36748418710468	0\\
3.36758418960474	0\\
3.3676841921048	0\\
3.36778419460487	0\\
3.36788419710493	0\\
3.36798419960499	0\\
3.36808420210505	0\\
3.36818420460511	0\\
3.36828420710518	0\\
3.36838420960524	0\\
3.3684842121053	0\\
3.36858421460537	0\\
3.36868421710543	0\\
3.36878421960549	0\\
3.36888422210555	0\\
3.36898422460562	0\\
3.36908422710568	0\\
3.36918422960574	0\\
3.3692842321058	0\\
3.36938423460587	0\\
3.36948423710593	0\\
3.36958423960599	0\\
3.36968424210605	0\\
3.36978424460612	0\\
3.36988424710618	0\\
3.36998424960624	0\\
3.3700842521063	0\\
3.37018425460636	0\\
3.37028425710643	0\\
3.37038425960649	0\\
3.37048426210655	0\\
3.37058426460662	0\\
3.37068426710668	0\\
3.37078426960674	0\\
3.3708842721068	0\\
3.37098427460687	0\\
3.37108427710693	0\\
3.37118427960699	0\\
3.37128428210705	0\\
3.37138428460712	0\\
3.37148428710718	0\\
3.37158428960724	0\\
3.3716842921073	0\\
3.37178429460737	0\\
3.37188429710743	0\\
3.37198429960749	0\\
3.37208430210755	0\\
3.37218430460761	0\\
3.37228430710768	0\\
3.37238430960774	0\\
3.3724843121078	0\\
3.37258431460787	0\\
3.37268431710793	0\\
3.37278431960799	0\\
3.37288432210805	0\\
3.37298432460812	0\\
3.37308432710818	0\\
3.37318432960824	0\\
3.3732843321083	0\\
3.37338433460837	0\\
3.37348433710843	0\\
3.37358433960849	0\\
3.37368434210855	0\\
3.37378434460862	0\\
3.37388434710868	0\\
3.37398434960874	0\\
3.3740843521088	0\\
3.37418435460887	0\\
3.37428435710893	0\\
3.37438435960899	0\\
3.37448436210905	0\\
3.37458436460912	0\\
3.37468436710918	0\\
3.37478436960924	0\\
3.3748843721093	0\\
3.37498437460937	0\\
3.37508437710943	0\\
3.37518437960949	0\\
3.37528438210955	0\\
3.37538438460962	0\\
3.37548438710968	0\\
3.37558438960974	0\\
3.3756843921098	0\\
3.37578439460987	0\\
3.37588439710993	0\\
3.37598439960999	0\\
3.37608440211005	0\\
3.37618440461012	0\\
3.37628440711018	0\\
3.37638440961024	0\\
3.3764844121103	0\\
3.37658441461037	0\\
3.37668441711043	0\\
3.37678441961049	0\\
3.37688442211055	0\\
3.37698442461062	0\\
3.37708442711068	0\\
3.37718442961074	0\\
3.3772844321108	0\\
3.37738443461087	0\\
3.37748443711093	0\\
3.37758443961099	0\\
3.37768444211105	0\\
3.37778444461112	0\\
3.37788444711118	0\\
3.37798444961124	0\\
3.3780844521113	0\\
3.37818445461137	0\\
3.37828445711143	0\\
3.37838445961149	0\\
3.37848446211155	0\\
3.37858446461162	0\\
3.37868446711168	0\\
3.37878446961174	0\\
3.3788844721118	0\\
3.37898447461187	0\\
3.37908447711193	0\\
3.37918447961199	0\\
3.37928448211205	0\\
3.37938448461212	0\\
3.37948448711218	0\\
3.37958448961224	0\\
3.3796844921123	0\\
3.37978449461237	0\\
3.37988449711243	0\\
3.37998449961249	0\\
3.38008450211255	0\\
3.38018450461262	0\\
3.38028450711268	0\\
3.38038450961274	0\\
3.3804845121128	0\\
3.38058451461287	0\\
3.38068451711293	0\\
3.38078451961299	0\\
3.38088452211305	0\\
3.38098452461312	0\\
3.38108452711318	0\\
3.38118452961324	0\\
3.3812845321133	0\\
3.38138453461337	0\\
3.38148453711343	0\\
3.38158453961349	0\\
3.38168454211355	0\\
3.38178454461362	0\\
3.38188454711368	0\\
3.38198454961374	0\\
3.3820845521138	0\\
3.38218455461387	0\\
3.38228455711393	0\\
3.38238455961399	0\\
3.38248456211405	0\\
3.38258456461412	0\\
3.38268456711418	0\\
3.38278456961424	0\\
3.3828845721143	0\\
3.38298457461437	0\\
3.38308457711443	0\\
3.38318457961449	0\\
3.38328458211455	0\\
3.38338458461462	0\\
3.38348458711468	0\\
3.38358458961474	0\\
3.3836845921148	0\\
3.38378459461487	0\\
3.38388459711493	0\\
3.38398459961499	0\\
3.38408460211505	0\\
3.38418460461512	0\\
3.38428460711518	0\\
3.38438460961524	0\\
3.3844846121153	0\\
3.38458461461537	0\\
3.38468461711543	0\\
3.38478461961549	0\\
3.38488462211555	0\\
3.38498462461562	0\\
3.38508462711568	0\\
3.38518462961574	0\\
3.3852846321158	0\\
3.38538463461587	0\\
3.38548463711593	0\\
3.38558463961599	0\\
3.38568464211605	0\\
3.38578464461612	0\\
3.38588464711618	0\\
3.38598464961624	0\\
3.3860846521163	0\\
3.38618465461637	0\\
3.38628465711643	0\\
3.38638465961649	0\\
3.38648466211655	0\\
3.38658466461662	0\\
3.38668466711668	0\\
3.38678466961674	0\\
3.3868846721168	0\\
3.38698467461687	0\\
3.38708467711693	0\\
3.38718467961699	0\\
3.38728468211705	0\\
3.38738468461712	0\\
3.38748468711718	0\\
3.38758468961724	0\\
3.3876846921173	0\\
3.38778469461737	0\\
3.38788469711743	0\\
3.38798469961749	0\\
3.38808470211755	0\\
3.38818470461762	0\\
3.38828470711768	0\\
3.38838470961774	0\\
3.3884847121178	0\\
3.38858471461787	0\\
3.38868471711793	0\\
3.38878471961799	0\\
3.38888472211805	0\\
3.38898472461812	0\\
3.38908472711818	0\\
3.38918472961824	0\\
3.3892847321183	0\\
3.38938473461837	0\\
3.38948473711843	0\\
3.38958473961849	0\\
3.38968474211855	0\\
3.38978474461862	0\\
3.38988474711868	0\\
3.38998474961874	0\\
3.3900847521188	0\\
3.39018475461887	0\\
3.39028475711893	0\\
3.39038475961899	0\\
3.39048476211905	0\\
3.39058476461912	0\\
3.39068476711918	0\\
3.39078476961924	0\\
3.3908847721193	0\\
3.39098477461937	0\\
3.39108477711943	0\\
3.39118477961949	0\\
3.39128478211955	0\\
3.39138478461962	0\\
3.39148478711968	0\\
3.39158478961974	0\\
3.3916847921198	0\\
3.39178479461987	0\\
3.39188479711993	0\\
3.39198479961999	0\\
3.39208480212005	0\\
3.39218480462012	0\\
3.39228480712018	0\\
3.39238480962024	0\\
3.3924848121203	0\\
3.39258481462037	0\\
3.39268481712043	0\\
3.39278481962049	0\\
3.39288482212055	0\\
3.39298482462062	0\\
3.39308482712068	0\\
3.39318482962074	0\\
3.3932848321208	0\\
3.39338483462087	0\\
3.39348483712093	0\\
3.39358483962099	0\\
3.39368484212105	0\\
3.39378484462112	0\\
3.39388484712118	0\\
3.39398484962124	0\\
3.3940848521213	0\\
3.39418485462137	0\\
3.39428485712143	0\\
3.39438485962149	0\\
3.39448486212155	0\\
3.39458486462162	0\\
3.39468486712168	0\\
3.39478486962174	0\\
3.3948848721218	0\\
3.39498487462187	0\\
3.39508487712193	0\\
3.39518487962199	0\\
3.39528488212205	0\\
3.39538488462212	0\\
3.39548488712218	0\\
3.39558488962224	0\\
3.3956848921223	0\\
3.39578489462237	0\\
3.39588489712243	0\\
3.39598489962249	0\\
3.39608490212255	0\\
3.39618490462262	0\\
3.39628490712268	0\\
3.39638490962274	0\\
3.3964849121228	0\\
3.39658491462287	0\\
3.39668491712293	0\\
3.39678491962299	0\\
3.39688492212305	0\\
3.39698492462312	0\\
3.39708492712318	0\\
3.39718492962324	0\\
3.3972849321233	0\\
3.39738493462337	0\\
3.39748493712343	0\\
3.39758493962349	0\\
3.39768494212355	0\\
3.39778494462362	0\\
3.39788494712368	0\\
3.39798494962374	0\\
3.3980849521238	0\\
3.39818495462387	0\\
3.39828495712393	0\\
3.39838495962399	0\\
3.39848496212405	0\\
3.39858496462412	0\\
3.39868496712418	0\\
3.39878496962424	0\\
3.3988849721243	0\\
3.39898497462437	0\\
3.39908497712443	0\\
3.39918497962449	0\\
3.39928498212455	0\\
3.39938498462462	0\\
3.39948498712468	0\\
3.39958498962474	0\\
3.3996849921248	0\\
3.39978499462487	0\\
3.39988499712493	0\\
3.39998499962499	0\\
3.40008500212505	0\\
3.40018500462512	0\\
3.40028500712518	0\\
3.40038500962524	0\\
3.4004850121253	0\\
3.40058501462537	0\\
3.40068501712543	0\\
3.40078501962549	0\\
3.40088502212555	0\\
3.40098502462562	0\\
3.40108502712568	0\\
3.40118502962574	0\\
3.4012850321258	0\\
3.40138503462587	0\\
3.40148503712593	0\\
3.40158503962599	0\\
3.40168504212605	0\\
3.40178504462612	0\\
3.40188504712618	0\\
3.40198504962624	0\\
3.4020850521263	0\\
3.40218505462637	0\\
3.40228505712643	0\\
3.40238505962649	0\\
3.40248506212655	0\\
3.40258506462662	0\\
3.40268506712668	0\\
3.40278506962674	0\\
3.4028850721268	0\\
3.40298507462687	0\\
3.40308507712693	0\\
3.40318507962699	0\\
3.40328508212705	0\\
3.40338508462712	0\\
3.40348508712718	0\\
3.40358508962724	0\\
3.4036850921273	0\\
3.40378509462737	0\\
3.40388509712743	0\\
3.40398509962749	0\\
3.40408510212755	0\\
3.40418510462762	0\\
3.40428510712768	0\\
3.40438510962774	0\\
3.4044851121278	0\\
3.40458511462787	0\\
3.40468511712793	0\\
3.40478511962799	0\\
3.40488512212805	0\\
3.40498512462812	0\\
3.40508512712818	0\\
3.40518512962824	0\\
3.4052851321283	0\\
3.40538513462837	0\\
3.40548513712843	0\\
3.40558513962849	0\\
3.40568514212855	0\\
3.40578514462862	0\\
3.40588514712868	0\\
3.40598514962874	0\\
3.4060851521288	0\\
3.40618515462887	0\\
3.40628515712893	0\\
3.40638515962899	0\\
3.40648516212905	0\\
3.40658516462912	0\\
3.40668516712918	0\\
3.40678516962924	0\\
3.4068851721293	0\\
3.40698517462937	0\\
3.40708517712943	0\\
3.40718517962949	0\\
3.40728518212955	0\\
3.40738518462962	0\\
3.40748518712968	0\\
3.40758518962974	0\\
3.4076851921298	0\\
3.40778519462987	0\\
3.40788519712993	0\\
3.40798519962999	0\\
3.40808520213005	0\\
3.40818520463012	0\\
3.40828520713018	0\\
3.40838520963024	0\\
3.4084852121303	0\\
3.40858521463037	0\\
3.40868521713043	0\\
3.40878521963049	0\\
3.40888522213055	0\\
3.40898522463062	0\\
3.40908522713068	0\\
3.40918522963074	0\\
3.4092852321308	0\\
3.40938523463087	0\\
3.40948523713093	0\\
3.40958523963099	0\\
3.40968524213105	0\\
3.40978524463112	0\\
3.40988524713118	0\\
3.40998524963124	0\\
3.4100852521313	0\\
3.41018525463137	0\\
3.41028525713143	0\\
3.41038525963149	0\\
3.41048526213155	0\\
3.41058526463162	0\\
3.41068526713168	0\\
3.41078526963174	0\\
3.4108852721318	0\\
3.41098527463187	0\\
3.41108527713193	0\\
3.41118527963199	0\\
3.41128528213205	0\\
3.41138528463212	0\\
3.41148528713218	0\\
3.41158528963224	0\\
3.4116852921323	0\\
3.41178529463237	0\\
3.41188529713243	0\\
3.41198529963249	0\\
3.41208530213255	0\\
3.41218530463262	0\\
3.41228530713268	0\\
3.41238530963274	0\\
3.4124853121328	0\\
3.41258531463287	0\\
3.41268531713293	0\\
3.41278531963299	0\\
3.41288532213305	0\\
3.41298532463312	0\\
3.41308532713318	0\\
3.41318532963324	0\\
3.4132853321333	0\\
3.41338533463337	0\\
3.41348533713343	0\\
3.41358533963349	0\\
3.41368534213355	0\\
3.41378534463362	0\\
3.41388534713368	0\\
3.41398534963374	0\\
3.4140853521338	0\\
3.41418535463387	0\\
3.41428535713393	0\\
3.41438535963399	0\\
3.41448536213405	0\\
3.41458536463412	0\\
3.41468536713418	0\\
3.41478536963424	0\\
3.4148853721343	0\\
3.41498537463437	0\\
3.41508537713443	0\\
3.41518537963449	0\\
3.41528538213455	0\\
3.41538538463462	0\\
3.41548538713468	0\\
3.41558538963474	0\\
3.4156853921348	0\\
3.41578539463487	0\\
3.41588539713493	0\\
3.41598539963499	0\\
3.41608540213505	0\\
3.41618540463512	0\\
3.41628540713518	0\\
3.41638540963524	0\\
3.4164854121353	0\\
3.41658541463537	0\\
3.41668541713543	0\\
3.41678541963549	0\\
3.41688542213555	0\\
3.41698542463562	0\\
3.41708542713568	0\\
3.41718542963574	0\\
3.4172854321358	0\\
3.41738543463587	0\\
3.41748543713593	0\\
3.41758543963599	0\\
3.41768544213605	0\\
3.41778544463612	0\\
3.41788544713618	0\\
3.41798544963624	0\\
3.4180854521363	0\\
3.41818545463637	0\\
3.41828545713643	0\\
3.41838545963649	0\\
3.41848546213655	0\\
3.41858546463662	0\\
3.41868546713668	0\\
3.41878546963674	0\\
3.4188854721368	0\\
3.41898547463687	0\\
3.41908547713693	0\\
3.41918547963699	0\\
3.41928548213705	0\\
3.41938548463712	0\\
3.41948548713718	0\\
3.41958548963724	0\\
3.4196854921373	0\\
3.41978549463737	0\\
3.41988549713743	0\\
3.41998549963749	0\\
3.42008550213755	0\\
3.42018550463762	0\\
3.42028550713768	0\\
3.42038550963774	0\\
3.4204855121378	0\\
3.42058551463787	0\\
3.42068551713793	0\\
3.42078551963799	0\\
3.42088552213805	0\\
3.42098552463812	0\\
3.42108552713818	0\\
3.42118552963824	0\\
3.4212855321383	0\\
3.42138553463837	0\\
3.42148553713843	0\\
3.42158553963849	0\\
3.42168554213855	0\\
3.42178554463862	0\\
3.42188554713868	0\\
3.42198554963874	0\\
3.4220855521388	0\\
3.42218555463887	0\\
3.42228555713893	0\\
3.42238555963899	0\\
3.42248556213905	0\\
3.42258556463912	0\\
3.42268556713918	0\\
3.42278556963924	0\\
3.4228855721393	0\\
3.42298557463937	0\\
3.42308557713943	0\\
3.42318557963949	0\\
3.42328558213955	0\\
3.42338558463962	0\\
3.42348558713968	0\\
3.42358558963974	0\\
3.4236855921398	0\\
3.42378559463987	0\\
3.42388559713993	0\\
3.42398559963999	0\\
3.42408560214005	0\\
3.42418560464012	0\\
3.42428560714018	0\\
3.42438560964024	0\\
3.4244856121403	0\\
3.42458561464037	0\\
3.42468561714043	0\\
3.42478561964049	0\\
3.42488562214055	0\\
3.42498562464062	0\\
3.42508562714068	0\\
3.42518562964074	0\\
3.4252856321408	0\\
3.42538563464087	0\\
3.42548563714093	0\\
3.42558563964099	0\\
3.42568564214105	0\\
3.42578564464112	0\\
3.42588564714118	0\\
3.42598564964124	0\\
3.4260856521413	0\\
3.42618565464137	0\\
3.42628565714143	0\\
3.42638565964149	0\\
3.42648566214155	0\\
3.42658566464162	0\\
3.42668566714168	0\\
3.42678566964174	0\\
3.4268856721418	0\\
3.42698567464187	0\\
3.42708567714193	0\\
3.42718567964199	0\\
3.42728568214205	0\\
3.42738568464212	0\\
3.42748568714218	0\\
3.42758568964224	0\\
3.4276856921423	0\\
3.42778569464237	0\\
3.42788569714243	0\\
3.42798569964249	0\\
3.42808570214255	0\\
3.42818570464262	0\\
3.42828570714268	0\\
3.42838570964274	0\\
3.4284857121428	0\\
3.42858571464287	0\\
3.42868571714293	0\\
3.42878571964299	0\\
3.42888572214305	0\\
3.42898572464312	0\\
3.42908572714318	0\\
3.42918572964324	0\\
3.4292857321433	0\\
3.42938573464337	0\\
3.42948573714343	0\\
3.42958573964349	0\\
3.42968574214355	0\\
3.42978574464362	0\\
3.42988574714368	0\\
3.42998574964374	0\\
3.4300857521438	0\\
3.43018575464387	0\\
3.43028575714393	0\\
3.43038575964399	0\\
3.43048576214405	0\\
3.43058576464412	0\\
3.43068576714418	0\\
3.43078576964424	0\\
3.4308857721443	0\\
3.43098577464437	0\\
3.43108577714443	0\\
3.43118577964449	0\\
3.43128578214455	0\\
3.43138578464462	0\\
3.43148578714468	0\\
3.43158578964474	0\\
3.4316857921448	0\\
3.43178579464487	0\\
3.43188579714493	0\\
3.43198579964499	0\\
3.43208580214505	0\\
3.43218580464512	0\\
3.43228580714518	0\\
3.43238580964524	0\\
3.4324858121453	0\\
3.43258581464537	0\\
3.43268581714543	0\\
3.43278581964549	0\\
3.43288582214555	0\\
3.43298582464562	0\\
3.43308582714568	0\\
3.43318582964574	0\\
3.4332858321458	0\\
3.43338583464587	0\\
3.43348583714593	0\\
3.43358583964599	0\\
3.43368584214605	0\\
3.43378584464612	0\\
3.43388584714618	0\\
3.43398584964624	0\\
3.4340858521463	0\\
3.43418585464637	0\\
3.43428585714643	0\\
3.43438585964649	0\\
3.43448586214655	0\\
3.43458586464662	0\\
3.43468586714668	0\\
3.43478586964674	0\\
3.4348858721468	0\\
3.43498587464687	0\\
3.43508587714693	0\\
3.43518587964699	0\\
3.43528588214705	0\\
3.43538588464712	0\\
3.43548588714718	0\\
3.43558588964724	0\\
3.4356858921473	0\\
3.43578589464737	0\\
3.43588589714743	0\\
3.43598589964749	0\\
3.43608590214755	0\\
3.43618590464762	0\\
3.43628590714768	0\\
3.43638590964774	0\\
3.4364859121478	0\\
3.43658591464787	0\\
3.43668591714793	0\\
3.43678591964799	0\\
3.43688592214805	0\\
3.43698592464812	0\\
3.43708592714818	0\\
3.43718592964824	0\\
3.4372859321483	0\\
3.43738593464837	0\\
3.43748593714843	0\\
3.43758593964849	0\\
3.43768594214855	0\\
3.43778594464862	0\\
3.43788594714868	0\\
3.43798594964874	0\\
3.4380859521488	0\\
3.43818595464887	0\\
3.43828595714893	0\\
3.43838595964899	0\\
3.43848596214905	0\\
3.43858596464912	0\\
3.43868596714918	0\\
3.43878596964924	0\\
3.4388859721493	0\\
3.43898597464937	0\\
3.43908597714943	0\\
3.43918597964949	0\\
3.43928598214955	0\\
3.43938598464962	0\\
3.43948598714968	0\\
3.43958598964974	0\\
3.4396859921498	0\\
3.43978599464987	0\\
3.43988599714993	0\\
3.43998599964999	0\\
3.44008600215005	0\\
3.44018600465012	0\\
3.44028600715018	0\\
3.44038600965024	0\\
3.4404860121503	0\\
3.44058601465037	0\\
3.44068601715043	0\\
3.44078601965049	0\\
3.44088602215055	0\\
3.44098602465062	0\\
3.44108602715068	0\\
3.44118602965074	0\\
3.4412860321508	0\\
3.44138603465087	0\\
3.44148603715093	0\\
3.44158603965099	0\\
3.44168604215105	0\\
3.44178604465112	0\\
3.44188604715118	0\\
3.44198604965124	0\\
3.4420860521513	0\\
3.44218605465137	0\\
3.44228605715143	0\\
3.44238605965149	0\\
3.44248606215155	0\\
3.44258606465162	0\\
3.44268606715168	0\\
3.44278606965174	0\\
3.4428860721518	0\\
3.44298607465187	0\\
3.44308607715193	0\\
3.44318607965199	0\\
3.44328608215205	0\\
3.44338608465212	0\\
3.44348608715218	0\\
3.44358608965224	0\\
3.4436860921523	0\\
3.44378609465237	0\\
3.44388609715243	0\\
3.44398609965249	0\\
3.44408610215255	0\\
3.44418610465262	0\\
3.44428610715268	0\\
3.44438610965274	0\\
3.4444861121528	0\\
3.44458611465287	0\\
3.44468611715293	0\\
3.44478611965299	0\\
3.44488612215305	0\\
3.44498612465312	0\\
3.44508612715318	0\\
3.44518612965324	0\\
3.4452861321533	0\\
3.44538613465337	0\\
3.44548613715343	0\\
3.44558613965349	0\\
3.44568614215355	0\\
3.44578614465362	0\\
3.44588614715368	0\\
3.44598614965374	0\\
3.4460861521538	0\\
3.44618615465387	0\\
3.44628615715393	0\\
3.44638615965399	0\\
3.44648616215405	0\\
3.44658616465412	0\\
3.44668616715418	0\\
3.44678616965424	0\\
3.4468861721543	0\\
3.44698617465437	0\\
3.44708617715443	0\\
3.44718617965449	0\\
3.44728618215455	0\\
3.44738618465462	0\\
3.44748618715468	0\\
3.44758618965474	0\\
3.4476861921548	0\\
3.44778619465487	0\\
3.44788619715493	0\\
3.44798619965499	0\\
3.44808620215505	0\\
3.44818620465512	0\\
3.44828620715518	0\\
3.44838620965524	0\\
3.4484862121553	0\\
3.44858621465537	0\\
3.44868621715543	0\\
3.44878621965549	0\\
3.44888622215555	0\\
3.44898622465562	0\\
3.44908622715568	0\\
3.44918622965574	0\\
3.4492862321558	0\\
3.44938623465587	0\\
3.44948623715593	0\\
3.44958623965599	0\\
3.44968624215605	0\\
3.44978624465612	0\\
3.44988624715618	0\\
3.44998624965624	0\\
3.4500862521563	0\\
3.45018625465637	0\\
3.45028625715643	0\\
3.45038625965649	0\\
3.45048626215655	0\\
3.45058626465662	0\\
3.45068626715668	0\\
3.45078626965674	0\\
3.4508862721568	0\\
3.45098627465687	0\\
3.45108627715693	0\\
3.45118627965699	0\\
3.45128628215705	0\\
3.45138628465712	0\\
3.45148628715718	0\\
3.45158628965724	0\\
3.4516862921573	0\\
3.45178629465737	0\\
3.45188629715743	0\\
3.45198629965749	0\\
3.45208630215755	0\\
3.45218630465762	0\\
3.45228630715768	0\\
3.45238630965774	0\\
3.4524863121578	0\\
3.45258631465787	0\\
3.45268631715793	0\\
3.45278631965799	0\\
3.45288632215805	0\\
3.45298632465812	0\\
3.45308632715818	0\\
3.45318632965824	0\\
3.4532863321583	0\\
3.45338633465837	0\\
3.45348633715843	0\\
3.45358633965849	0\\
3.45368634215855	0\\
3.45378634465862	0\\
3.45388634715868	0\\
3.45398634965874	0\\
3.4540863521588	0\\
3.45418635465887	0\\
3.45428635715893	0\\
3.45438635965899	0\\
3.45448636215905	0\\
3.45458636465912	0\\
3.45468636715918	0\\
3.45478636965924	0\\
3.4548863721593	0\\
3.45498637465937	0\\
3.45508637715943	0\\
3.45518637965949	0\\
3.45528638215955	0\\
3.45538638465962	0\\
3.45548638715968	0\\
3.45558638965974	0\\
3.4556863921598	0\\
3.45578639465987	0\\
3.45588639715993	0\\
3.45598639965999	0\\
3.45608640216005	0\\
3.45618640466012	0\\
3.45628640716018	0\\
3.45638640966024	0\\
3.4564864121603	0\\
3.45658641466037	0\\
3.45668641716043	0\\
3.45678641966049	0\\
3.45688642216055	0\\
3.45698642466062	0\\
3.45708642716068	0\\
3.45718642966074	0\\
3.4572864321608	0\\
3.45738643466087	0\\
3.45748643716093	0\\
3.45758643966099	0\\
3.45768644216105	0\\
3.45778644466112	0\\
3.45788644716118	0\\
3.45798644966124	0\\
3.4580864521613	0\\
3.45818645466137	0\\
3.45828645716143	0\\
3.45838645966149	0\\
3.45848646216155	0\\
3.45858646466162	0\\
3.45868646716168	0\\
3.45878646966174	0\\
3.4588864721618	0\\
3.45898647466187	0\\
3.45908647716193	0\\
3.45918647966199	0\\
3.45928648216205	0\\
3.45938648466212	0\\
3.45948648716218	0\\
3.45958648966224	0\\
3.4596864921623	0\\
3.45978649466237	0\\
3.45988649716243	0\\
3.45998649966249	0\\
3.46008650216255	0\\
3.46018650466262	0\\
3.46028650716268	0\\
3.46038650966274	0\\
3.4604865121628	0\\
3.46058651466287	0\\
3.46068651716293	0\\
3.46078651966299	0\\
3.46088652216305	0\\
3.46098652466312	0\\
3.46108652716318	0\\
3.46118652966324	0\\
3.4612865321633	0\\
3.46138653466337	0\\
3.46148653716343	0\\
3.46158653966349	0\\
3.46168654216355	0\\
3.46178654466362	0\\
3.46188654716368	0\\
3.46198654966374	0\\
3.4620865521638	0\\
3.46218655466387	0\\
3.46228655716393	0\\
3.46238655966399	0\\
3.46248656216405	0\\
3.46258656466412	0\\
3.46268656716418	0\\
3.46278656966424	0\\
3.4628865721643	0\\
3.46298657466437	0\\
3.46308657716443	0\\
3.46318657966449	0\\
3.46328658216455	0\\
3.46338658466462	0\\
3.46348658716468	0\\
3.46358658966474	0\\
3.4636865921648	0\\
3.46378659466487	0\\
3.46388659716493	0\\
3.46398659966499	0\\
3.46408660216505	0\\
3.46418660466512	0\\
3.46428660716518	0\\
3.46438660966524	0\\
3.4644866121653	0\\
3.46458661466537	0\\
3.46468661716543	0\\
3.46478661966549	0\\
3.46488662216555	0\\
3.46498662466562	0\\
3.46508662716568	0\\
3.46518662966574	0\\
3.4652866321658	0\\
3.46538663466587	0\\
3.46548663716593	0\\
3.46558663966599	0\\
3.46568664216605	0\\
3.46578664466612	0\\
3.46588664716618	0\\
3.46598664966624	0\\
3.4660866521663	0\\
3.46618665466637	0\\
3.46628665716643	0\\
3.46638665966649	0\\
3.46648666216655	0\\
3.46658666466662	0\\
3.46668666716668	0\\
3.46678666966674	0\\
3.4668866721668	0\\
3.46698667466687	0\\
3.46708667716693	0\\
3.46718667966699	0\\
3.46728668216705	0\\
3.46738668466712	0\\
3.46748668716718	0\\
3.46758668966724	0\\
3.4676866921673	0\\
3.46778669466737	0\\
3.46788669716743	0\\
3.46798669966749	0\\
3.46808670216755	0\\
3.46818670466762	0\\
3.46828670716768	0\\
3.46838670966774	0\\
3.4684867121678	0\\
3.46858671466787	0\\
3.46868671716793	0\\
3.46878671966799	0\\
3.46888672216805	0\\
3.46898672466812	0\\
3.46908672716818	0\\
3.46918672966824	0\\
3.4692867321683	0\\
3.46938673466837	0\\
3.46948673716843	0\\
3.46958673966849	0\\
3.46968674216855	0\\
3.46978674466862	0\\
3.46988674716868	0\\
3.46998674966874	0\\
3.4700867521688	0\\
3.47018675466887	0\\
3.47028675716893	0\\
3.47038675966899	0\\
3.47048676216905	0\\
3.47058676466912	0\\
3.47068676716918	0\\
3.47078676966924	0\\
3.4708867721693	0\\
3.47098677466937	0\\
3.47108677716943	0\\
3.47118677966949	0\\
3.47128678216955	0\\
3.47138678466962	0\\
3.47148678716968	0\\
3.47158678966974	0\\
3.4716867921698	0\\
3.47178679466987	0\\
3.47188679716993	0\\
3.47198679966999	0\\
3.47208680217005	0\\
3.47218680467012	0\\
3.47228680717018	0\\
3.47238680967024	0\\
3.4724868121703	0\\
3.47258681467037	0\\
3.47268681717043	0\\
3.47278681967049	0\\
3.47288682217055	0\\
3.47298682467062	0\\
3.47308682717068	0\\
3.47318682967074	0\\
3.4732868321708	0\\
3.47338683467087	0\\
3.47348683717093	0\\
3.47358683967099	0\\
3.47368684217105	0\\
3.47378684467112	0\\
3.47388684717118	0\\
3.47398684967124	0\\
3.4740868521713	0\\
3.47418685467137	0\\
3.47428685717143	0\\
3.47438685967149	0\\
3.47448686217155	0\\
3.47458686467162	0\\
3.47468686717168	0\\
3.47478686967174	0\\
3.4748868721718	0\\
3.47498687467187	0\\
3.47508687717193	0\\
3.47518687967199	0\\
3.47528688217205	0\\
3.47538688467212	0\\
3.47548688717218	0\\
3.47558688967224	0\\
3.4756868921723	0\\
3.47578689467237	0\\
3.47588689717243	0\\
3.47598689967249	0\\
3.47608690217255	0\\
3.47618690467262	0\\
3.47628690717268	0\\
3.47638690967274	0\\
3.4764869121728	0\\
3.47658691467287	0\\
3.47668691717293	0\\
3.47678691967299	0\\
3.47688692217305	0\\
3.47698692467312	0\\
3.47708692717318	0\\
3.47718692967324	0\\
3.4772869321733	0\\
3.47738693467337	0\\
3.47748693717343	0\\
3.47758693967349	0\\
3.47768694217355	0\\
3.47778694467362	0\\
3.47788694717368	0\\
3.47798694967374	0\\
3.4780869521738	0\\
3.47818695467387	0\\
3.47828695717393	0\\
3.47838695967399	0\\
3.47848696217405	0\\
3.47858696467412	0\\
3.47868696717418	0\\
3.47878696967424	0\\
3.4788869721743	0\\
3.47898697467437	0\\
3.47908697717443	0\\
3.47918697967449	0\\
3.47928698217455	0\\
3.47938698467462	0\\
3.47948698717468	0\\
3.47958698967474	0\\
3.4796869921748	0\\
3.47978699467487	0\\
3.47988699717493	0\\
3.47998699967499	0\\
3.48008700217505	0\\
3.48018700467512	0\\
3.48028700717518	0\\
3.48038700967524	0\\
3.4804870121753	0\\
3.48058701467537	0\\
3.48068701717543	0\\
3.48078701967549	0\\
3.48088702217555	0\\
3.48098702467562	0\\
3.48108702717568	0\\
3.48118702967574	0\\
3.4812870321758	0\\
3.48138703467587	0\\
3.48148703717593	0\\
3.48158703967599	0\\
3.48168704217605	0\\
3.48178704467612	0\\
3.48188704717618	0\\
3.48198704967624	0\\
3.4820870521763	0\\
3.48218705467637	0\\
3.48228705717643	0\\
3.48238705967649	0\\
3.48248706217655	0\\
3.48258706467662	0\\
3.48268706717668	0\\
3.48278706967674	0\\
3.4828870721768	0\\
3.48298707467687	0\\
3.48308707717693	0\\
3.48318707967699	0\\
3.48328708217705	0\\
3.48338708467712	0\\
3.48348708717718	0\\
3.48358708967724	0\\
3.4836870921773	0\\
3.48378709467737	0\\
3.48388709717743	0\\
3.48398709967749	0\\
3.48408710217755	0\\
3.48418710467762	0\\
3.48428710717768	0\\
3.48438710967774	0\\
3.4844871121778	0\\
3.48458711467787	0\\
3.48468711717793	0\\
3.48478711967799	0\\
3.48488712217805	0\\
3.48498712467812	0\\
3.48508712717818	0\\
3.48518712967824	0\\
3.4852871321783	0\\
3.48538713467837	0\\
3.48548713717843	0\\
3.48558713967849	0\\
3.48568714217855	0\\
3.48578714467862	0\\
3.48588714717868	0\\
3.48598714967874	0\\
3.4860871521788	0\\
3.48618715467887	0\\
3.48628715717893	0\\
3.48638715967899	0\\
3.48648716217905	0\\
3.48658716467912	0\\
3.48668716717918	0\\
3.48678716967924	0\\
3.4868871721793	0\\
3.48698717467937	0\\
3.48708717717943	0\\
3.48718717967949	0\\
3.48728718217955	0\\
3.48738718467962	0\\
3.48748718717968	0\\
3.48758718967974	0\\
3.4876871921798	0\\
3.48778719467987	0\\
3.48788719717993	0\\
3.48798719967999	0\\
3.48808720218005	0\\
3.48818720468012	0\\
3.48828720718018	0\\
3.48838720968024	0\\
3.4884872121803	0\\
3.48858721468037	0\\
3.48868721718043	0\\
3.48878721968049	0\\
3.48888722218055	0\\
3.48898722468062	0\\
3.48908722718068	0\\
3.48918722968074	0\\
3.4892872321808	0\\
3.48938723468087	0\\
3.48948723718093	0\\
3.48958723968099	0\\
3.48968724218105	0\\
3.48978724468112	0\\
3.48988724718118	0\\
3.48998724968124	0\\
3.4900872521813	0\\
3.49018725468137	0\\
3.49028725718143	0\\
3.49038725968149	0\\
3.49048726218155	0\\
3.49058726468162	0\\
3.49068726718168	0\\
3.49078726968174	0\\
3.4908872721818	0\\
3.49098727468187	0\\
3.49108727718193	0\\
3.49118727968199	0\\
3.49128728218205	0\\
3.49138728468212	0\\
3.49148728718218	0\\
3.49158728968224	0\\
3.4916872921823	0\\
3.49178729468237	0\\
3.49188729718243	0\\
3.49198729968249	0\\
3.49208730218255	0\\
3.49218730468262	0\\
3.49228730718268	0\\
3.49238730968274	0\\
3.4924873121828	0\\
3.49258731468287	0\\
3.49268731718293	0\\
3.49278731968299	0\\
3.49288732218305	0\\
3.49298732468312	0\\
3.49308732718318	0\\
3.49318732968324	0\\
3.4932873321833	0\\
3.49338733468337	0\\
3.49348733718343	0\\
3.49358733968349	0\\
3.49368734218355	0\\
3.49378734468362	0\\
3.49388734718368	0\\
3.49398734968374	0\\
3.4940873521838	0\\
3.49418735468387	0\\
3.49428735718393	0\\
3.49438735968399	0\\
3.49448736218405	0\\
3.49458736468412	0\\
3.49468736718418	0\\
3.49478736968424	0\\
3.4948873721843	0\\
3.49498737468437	0\\
3.49508737718443	0\\
3.49518737968449	0\\
3.49528738218455	0\\
3.49538738468462	0\\
3.49548738718468	0\\
3.49558738968474	0\\
3.4956873921848	0\\
3.49578739468487	0\\
3.49588739718493	0\\
3.49598739968499	0\\
3.49608740218505	0\\
3.49618740468512	0\\
3.49628740718518	0\\
3.49638740968524	0\\
3.4964874121853	0\\
3.49658741468537	0\\
3.49668741718543	0\\
3.49678741968549	0\\
3.49688742218555	0\\
3.49698742468562	0\\
3.49708742718568	0\\
3.49718742968574	0\\
3.4972874321858	0\\
3.49738743468587	0\\
3.49748743718593	0\\
3.49758743968599	0\\
3.49768744218605	0\\
3.49778744468612	0\\
3.49788744718618	0\\
3.49798744968624	0\\
3.4980874521863	0\\
3.49818745468637	0\\
3.49828745718643	0\\
3.49838745968649	0\\
3.49848746218655	0\\
3.49858746468662	0\\
3.49868746718668	0\\
3.49878746968674	0\\
3.4988874721868	0\\
3.49898747468687	0\\
3.49908747718693	0\\
3.49918747968699	0\\
3.49928748218705	0\\
3.49938748468712	0\\
3.49948748718718	0\\
3.49958748968724	0\\
3.4996874921873	0\\
3.49978749468737	0\\
3.49988749718743	0\\
3.49998749968749	0\\
3.50008750218755	0\\
3.50018750468762	0\\
3.50028750718768	0\\
3.50038750968774	0\\
3.5004875121878	0\\
3.50058751468787	0\\
3.50068751718793	0\\
3.50078751968799	0\\
3.50088752218805	0\\
3.50098752468812	0\\
3.50108752718818	0\\
3.50118752968824	0\\
3.5012875321883	0\\
3.50138753468837	0\\
3.50148753718843	0\\
3.50158753968849	0\\
3.50168754218855	0\\
3.50178754468862	0\\
3.50188754718868	0\\
3.50198754968874	0\\
3.5020875521888	0\\
3.50218755468887	0\\
3.50228755718893	0\\
3.50238755968899	0\\
3.50248756218905	0\\
3.50258756468912	0\\
3.50268756718918	0\\
3.50278756968924	0\\
3.5028875721893	0\\
3.50298757468937	0\\
3.50308757718943	0\\
3.50318757968949	0\\
3.50328758218955	0\\
3.50338758468962	0\\
3.50348758718968	0\\
3.50358758968974	0\\
3.5036875921898	0\\
3.50378759468987	0\\
3.50388759718993	0\\
3.50398759968999	0\\
3.50408760219005	0\\
3.50418760469012	0\\
3.50428760719018	0\\
3.50438760969024	0\\
3.5044876121903	0\\
3.50458761469037	0\\
3.50468761719043	0\\
3.50478761969049	0\\
3.50488762219055	0\\
3.50498762469062	0\\
3.50508762719068	0\\
3.50518762969074	0\\
3.5052876321908	0\\
3.50538763469087	0\\
3.50548763719093	0\\
3.50558763969099	0\\
3.50568764219105	0\\
3.50578764469112	0\\
3.50588764719118	0\\
3.50598764969124	0\\
3.5060876521913	0\\
3.50618765469137	0\\
3.50628765719143	0\\
3.50638765969149	0\\
3.50648766219155	0\\
3.50658766469162	0\\
3.50668766719168	0\\
3.50678766969174	0\\
3.5068876721918	0\\
3.50698767469187	0\\
3.50708767719193	0\\
3.50718767969199	0\\
3.50728768219205	0\\
3.50738768469212	0\\
3.50748768719218	0\\
3.50758768969224	0\\
3.50768769219231	0\\
3.50778769469237	0\\
3.50788769719243	0\\
3.50798769969249	0\\
3.50808770219255	0\\
3.50818770469262	0\\
3.50828770719268	0\\
3.50838770969274	0\\
3.5084877121928	0\\
3.50858771469287	0\\
3.50868771719293	0\\
3.50878771969299	0\\
3.50888772219305	0\\
3.50898772469312	0\\
3.50908772719318	0\\
3.50918772969324	0\\
3.5092877321933	0\\
3.50938773469337	0\\
3.50948773719343	0\\
3.50958773969349	0\\
3.50968774219356	0\\
3.50978774469362	0\\
3.50988774719368	0\\
3.50998774969374	0\\
3.5100877521938	0\\
3.51018775469387	0\\
3.51028775719393	0\\
3.51038775969399	0\\
3.51048776219405	0\\
3.51058776469412	0\\
3.51068776719418	0\\
3.51078776969424	0\\
3.5108877721943	0\\
3.51098777469437	0\\
3.51108777719443	0\\
3.51118777969449	0\\
3.51128778219455	0\\
3.51138778469462	0\\
3.51148778719468	0\\
3.51158778969474	0\\
3.51168779219481	0\\
3.51178779469487	0\\
3.51188779719493	0\\
3.51198779969499	0\\
3.51208780219506	0\\
3.51218780469512	0\\
3.51228780719518	0\\
3.51238780969524	0\\
3.5124878121953	0\\
3.51258781469537	0\\
3.51268781719543	0\\
3.51278781969549	0\\
3.51288782219555	0\\
3.51298782469562	0\\
3.51308782719568	0\\
3.51318782969574	0\\
3.5132878321958	0\\
3.51338783469587	0\\
3.51348783719593	0\\
3.51358783969599	0\\
3.51368784219606	0\\
3.51378784469612	0\\
3.51388784719618	0\\
3.51398784969624	0\\
3.51408785219631	0\\
3.51418785469637	0\\
3.51428785719643	0\\
3.51438785969649	0\\
3.51448786219655	0\\
3.51458786469662	0\\
3.51468786719668	0\\
3.51478786969674	0\\
3.5148878721968	0\\
3.51498787469687	0\\
3.51508787719693	0\\
3.51518787969699	0\\
3.51528788219705	0\\
3.51538788469712	0\\
3.51548788719718	0\\
3.51558788969724	0\\
3.51568789219731	0\\
3.51578789469737	0\\
3.51588789719743	0\\
3.51598789969749	0\\
3.51608790219756	0\\
3.51618790469762	0\\
3.51628790719768	0\\
3.51638790969774	0\\
3.5164879121978	0\\
3.51658791469787	0\\
3.51668791719793	0\\
3.51678791969799	0\\
3.51688792219805	0\\
3.51698792469812	0\\
3.51708792719818	0\\
3.51718792969824	0\\
3.5172879321983	0\\
3.51738793469837	0\\
3.51748793719843	0\\
3.51758793969849	0\\
3.51768794219856	0\\
3.51778794469862	0\\
3.51788794719868	0\\
3.51798794969874	0\\
3.51808795219881	0\\
3.51818795469887	0\\
3.51828795719893	0\\
3.51838795969899	0\\
3.51848796219906	0\\
3.51858796469912	0\\
3.51868796719918	0\\
3.51878796969924	0\\
3.5188879721993	0\\
3.51898797469937	0\\
3.51908797719943	0\\
3.51918797969949	0\\
3.51928798219955	0\\
3.51938798469962	0\\
3.51948798719968	0\\
3.51958798969974	0\\
3.51968799219981	0\\
3.51978799469987	0\\
3.51988799719993	0\\
3.51998799969999	0\\
3.52008800220006	0\\
3.52018800470012	0\\
3.52028800720018	0\\
3.52038800970024	0\\
3.52048801220031	0\\
3.52058801470037	0\\
3.52068801720043	0\\
3.52078801970049	0\\
3.52088802220055	0\\
3.52098802470062	0\\
3.52108802720068	0\\
3.52118802970074	0\\
3.5212880322008	0\\
3.52138803470087	0\\
3.52148803720093	0\\
3.52158803970099	0\\
3.52168804220106	0\\
3.52178804470112	0\\
3.52188804720118	0\\
3.52198804970124	0\\
3.52208805220131	0\\
3.52218805470137	0\\
3.52228805720143	0\\
3.52238805970149	0\\
3.52248806220156	0\\
3.52258806470162	0\\
3.52268806720168	0\\
3.52278806970174	0\\
3.5228880722018	0\\
3.52298807470187	0\\
3.52308807720193	0\\
3.52318807970199	0\\
3.52328808220205	0\\
3.52338808470212	0\\
3.52348808720218	0\\
3.52358808970224	0\\
3.52368809220231	0\\
3.52378809470237	0\\
3.52388809720243	0\\
3.52398809970249	0\\
3.52408810220256	0\\
3.52418810470262	0\\
3.52428810720268	0\\
3.52438810970274	0\\
3.52448811220281	0\\
3.52458811470287	0\\
3.52468811720293	0\\
3.52478811970299	0\\
3.52488812220306	0\\
3.52498812470312	0\\
3.52508812720318	0\\
3.52518812970324	0\\
3.5252881322033	0\\
3.52538813470337	0\\
3.52548813720343	0\\
3.52558813970349	0\\
3.52568814220356	0\\
3.52578814470362	0\\
3.52588814720368	0\\
3.52598814970374	0\\
3.52608815220381	0\\
3.52618815470387	0\\
3.52628815720393	0\\
3.52638815970399	0\\
3.52648816220406	0\\
3.52658816470412	0\\
3.52668816720418	0\\
3.52678816970424	0\\
3.52688817220431	0\\
3.52698817470437	0\\
3.52708817720443	0\\
3.52718817970449	0\\
3.52728818220455	0\\
3.52738818470462	0\\
3.52748818720468	0\\
3.52758818970474	0\\
3.52768819220481	0\\
3.52778819470487	0\\
3.52788819720493	0\\
3.52798819970499	0\\
3.52808820220506	0\\
3.52818820470512	0\\
3.52828820720518	0\\
3.52838820970524	0\\
3.52848821220531	0\\
3.52858821470537	0\\
3.52868821720543	0\\
3.52878821970549	0\\
3.52888822220556	0\\
3.52898822470562	0\\
3.52908822720568	0\\
3.52918822970574	0\\
3.52928823220581	0\\
3.52938823470587	0\\
3.52948823720593	0\\
3.52958823970599	0\\
3.52968824220606	0\\
3.52978824470612	0\\
3.52988824720618	0\\
3.52998824970624	0\\
3.53008825220631	0\\
3.53018825470637	0\\
3.53028825720643	0\\
3.53038825970649	0\\
3.53048826220656	0\\
3.53058826470662	0\\
3.53068826720668	0\\
3.53078826970674	0\\
3.53088827220681	0\\
3.53098827470687	0\\
3.53108827720693	0\\
3.53118827970699	0\\
3.53128828220706	0\\
3.53138828470712	0\\
3.53148828720718	0\\
3.53158828970724	0\\
3.53168829220731	0\\
3.53178829470737	0\\
3.53188829720743	0\\
3.53198829970749	0\\
3.53208830220756	0\\
3.53218830470762	0\\
3.53228830720768	0\\
3.53238830970774	0\\
3.53248831220781	0\\
3.53258831470787	0\\
3.53268831720793	0\\
3.53278831970799	0\\
3.53288832220806	0\\
3.53298832470812	0\\
3.53308832720818	0\\
3.53318832970824	0\\
3.53328833220831	0\\
3.53338833470837	0\\
3.53348833720843	0\\
3.53358833970849	0\\
3.53368834220856	0\\
3.53378834470862	0\\
3.53388834720868	0\\
3.53398834970874	0\\
3.53408835220881	0\\
3.53418835470887	0\\
3.53428835720893	0\\
3.53438835970899	0\\
3.53448836220906	0\\
3.53458836470912	0\\
3.53468836720918	0\\
3.53478836970924	0\\
3.53488837220931	0\\
3.53498837470937	0\\
3.53508837720943	0\\
3.53518837970949	0\\
3.53528838220956	0\\
3.53538838470962	0\\
3.53548838720968	0\\
3.53558838970974	0\\
3.53568839220981	0\\
3.53578839470987	0\\
3.53588839720993	0\\
3.53598839970999	0\\
3.53608840221006	0\\
3.53618840471012	0\\
3.53628840721018	0\\
3.53638840971024	0\\
3.53648841221031	0\\
3.53658841471037	0\\
3.53668841721043	0\\
3.53678841971049	0\\
3.53688842221056	0\\
3.53698842471062	0\\
3.53708842721068	0\\
3.53718842971074	0\\
3.53728843221081	0\\
3.53738843471087	0\\
3.53748843721093	0\\
3.53758843971099	0\\
3.53768844221106	0\\
3.53778844471112	0\\
3.53788844721118	0\\
3.53798844971124	0\\
3.53808845221131	0\\
3.53818845471137	0\\
3.53828845721143	0\\
3.53838845971149	0\\
3.53848846221156	0\\
3.53858846471162	0\\
3.53868846721168	0\\
3.53878846971174	0\\
3.53888847221181	0\\
3.53898847471187	0\\
3.53908847721193	0\\
3.53918847971199	0\\
3.53928848221206	0\\
3.53938848471212	0\\
3.53948848721218	0\\
3.53958848971224	0\\
3.53968849221231	0\\
3.53978849471237	0\\
3.53988849721243	0\\
3.53998849971249	0\\
3.54008850221256	0\\
3.54018850471262	0\\
3.54028850721268	0\\
3.54038850971274	0\\
3.54048851221281	0\\
3.54058851471287	0\\
3.54068851721293	0\\
3.54078851971299	0\\
3.54088852221306	0\\
3.54098852471312	0\\
3.54108852721318	0\\
3.54118852971324	0\\
3.54128853221331	0\\
3.54138853471337	0\\
3.54148853721343	0\\
3.54158853971349	0\\
3.54168854221356	0\\
3.54178854471362	0\\
3.54188854721368	0\\
3.54198854971374	0\\
3.54208855221381	0\\
3.54218855471387	0\\
3.54228855721393	0\\
3.54238855971399	0\\
3.54248856221406	0\\
3.54258856471412	0\\
3.54268856721418	0\\
3.54278856971424	0\\
3.54288857221431	0\\
3.54298857471437	0\\
3.54308857721443	0\\
3.54318857971449	0\\
3.54328858221456	0\\
3.54338858471462	0\\
3.54348858721468	0\\
3.54358858971474	0\\
3.54368859221481	0\\
3.54378859471487	0\\
3.54388859721493	0\\
3.54398859971499	0\\
3.54408860221506	0\\
3.54418860471512	0\\
3.54428860721518	0\\
3.54438860971524	0\\
3.54448861221531	0\\
3.54458861471537	0\\
3.54468861721543	0\\
3.54478861971549	0\\
3.54488862221556	0\\
3.54498862471562	0\\
3.54508862721568	0\\
3.54518862971574	0\\
3.54528863221581	0\\
3.54538863471587	0\\
3.54548863721593	0\\
3.54558863971599	0\\
3.54568864221606	0\\
3.54578864471612	0\\
3.54588864721618	0\\
3.54598864971624	0\\
3.54608865221631	0\\
3.54618865471637	0\\
3.54628865721643	0\\
3.54638865971649	0\\
3.54648866221656	0\\
3.54658866471662	0\\
3.54668866721668	0\\
3.54678866971674	0\\
3.54688867221681	0\\
3.54698867471687	0\\
3.54708867721693	0\\
3.54718867971699	0\\
3.54728868221706	0\\
3.54738868471712	0\\
3.54748868721718	0\\
3.54758868971724	0\\
3.54768869221731	0\\
3.54778869471737	0\\
3.54788869721743	0\\
3.54798869971749	0\\
3.54808870221756	0\\
3.54818870471762	0\\
3.54828870721768	0\\
3.54838870971774	0\\
3.54848871221781	0\\
3.54858871471787	0\\
3.54868871721793	0\\
3.54878871971799	0\\
3.54888872221806	0\\
3.54898872471812	0\\
3.54908872721818	0\\
3.54918872971824	0\\
3.54928873221831	0\\
3.54938873471837	0\\
3.54948873721843	0\\
3.54958873971849	0\\
3.54968874221856	0\\
3.54978874471862	0\\
3.54988874721868	0\\
3.54998874971874	0\\
3.55008875221881	0\\
3.55018875471887	0\\
3.55028875721893	0\\
3.55038875971899	0\\
3.55048876221906	0\\
3.55058876471912	0\\
3.55068876721918	0\\
3.55078876971924	0\\
3.55088877221931	0\\
3.55098877471937	0\\
3.55108877721943	0\\
3.55118877971949	0\\
3.55128878221956	0\\
3.55138878471962	0\\
3.55148878721968	0\\
3.55158878971974	0\\
3.55168879221981	0\\
3.55178879471987	0\\
3.55188879721993	0\\
3.55198879971999	0\\
3.55208880222006	0\\
3.55218880472012	0\\
3.55228880722018	0\\
3.55238880972024	0\\
3.55248881222031	0\\
3.55258881472037	0\\
3.55268881722043	0\\
3.55278881972049	0\\
3.55288882222056	0\\
3.55298882472062	0\\
3.55308882722068	0\\
3.55318882972074	0\\
3.55328883222081	0\\
3.55338883472087	0\\
3.55348883722093	0\\
3.55358883972099	0\\
3.55368884222106	0\\
3.55378884472112	0\\
3.55388884722118	0\\
3.55398884972124	0\\
3.55408885222131	0\\
3.55418885472137	0\\
3.55428885722143	0\\
3.55438885972149	0\\
3.55448886222156	0\\
3.55458886472162	0\\
3.55468886722168	0\\
3.55478886972174	0\\
3.55488887222181	0\\
3.55498887472187	0\\
3.55508887722193	0\\
3.55518887972199	0\\
3.55528888222206	0\\
3.55538888472212	0\\
3.55548888722218	0\\
3.55558888972224	0\\
3.55568889222231	0\\
3.55578889472237	0\\
3.55588889722243	0\\
3.55598889972249	0\\
3.55608890222256	0\\
3.55618890472262	0\\
3.55628890722268	0\\
3.55638890972274	0\\
3.55648891222281	0\\
3.55658891472287	0\\
3.55668891722293	0\\
3.55678891972299	0\\
3.55688892222306	0\\
3.55698892472312	0\\
3.55708892722318	0\\
3.55718892972324	0\\
3.55728893222331	0\\
3.55738893472337	0\\
3.55748893722343	0\\
3.55758893972349	0\\
3.55768894222356	0\\
3.55778894472362	0\\
3.55788894722368	0\\
3.55798894972374	0\\
3.55808895222381	0\\
3.55818895472387	0\\
3.55828895722393	0\\
3.55838895972399	0\\
3.55848896222406	0\\
3.55858896472412	0\\
3.55868896722418	0\\
3.55878896972424	0\\
3.55888897222431	0\\
3.55898897472437	0\\
3.55908897722443	0\\
3.55918897972449	0\\
3.55928898222456	0\\
3.55938898472462	0\\
3.55948898722468	0\\
3.55958898972474	0\\
3.55968899222481	0\\
3.55978899472487	0\\
3.55988899722493	0\\
3.55998899972499	0\\
3.56008900222506	0\\
3.56018900472512	0\\
3.56028900722518	0\\
3.56038900972524	0\\
3.56048901222531	0\\
3.56058901472537	0\\
3.56068901722543	0\\
3.56078901972549	0\\
3.56088902222556	0\\
3.56098902472562	0\\
3.56108902722568	0\\
3.56118902972574	0\\
3.56128903222581	0\\
3.56138903472587	0\\
3.56148903722593	0\\
3.56158903972599	0\\
3.56168904222606	0\\
3.56178904472612	0\\
3.56188904722618	0\\
3.56198904972624	0\\
3.56208905222631	0\\
3.56218905472637	0\\
3.56228905722643	0\\
3.56238905972649	0\\
3.56248906222656	0\\
3.56258906472662	0\\
3.56268906722668	0\\
3.56278906972674	0\\
3.56288907222681	0\\
3.56298907472687	0\\
3.56308907722693	0\\
3.56318907972699	0\\
3.56328908222706	0\\
3.56338908472712	0\\
3.56348908722718	0\\
3.56358908972724	0\\
3.56368909222731	0\\
3.56378909472737	0\\
3.56388909722743	0\\
3.56398909972749	0\\
3.56408910222756	0\\
3.56418910472762	0\\
3.56428910722768	0\\
3.56438910972774	0\\
3.56448911222781	0\\
3.56458911472787	0\\
3.56468911722793	0\\
3.56478911972799	0\\
3.56488912222806	0\\
3.56498912472812	0\\
3.56508912722818	0\\
3.56518912972824	0\\
3.56528913222831	0\\
3.56538913472837	0\\
3.56548913722843	0\\
3.56558913972849	0\\
3.56568914222856	0\\
3.56578914472862	0\\
3.56588914722868	0\\
3.56598914972874	0\\
3.56608915222881	0\\
3.56618915472887	0\\
3.56628915722893	0\\
3.56638915972899	0\\
3.56648916222906	0\\
3.56658916472912	0\\
3.56668916722918	0\\
3.56678916972924	0\\
3.56688917222931	0\\
3.56698917472937	0\\
3.56708917722943	0\\
3.56718917972949	0\\
3.56728918222956	0\\
3.56738918472962	0\\
3.56748918722968	0\\
3.56758918972974	0\\
3.56768919222981	0\\
3.56778919472987	0\\
3.56788919722993	0\\
3.56798919972999	0\\
3.56808920223006	0\\
3.56818920473012	0\\
3.56828920723018	0\\
3.56838920973024	0\\
3.56848921223031	0\\
3.56858921473037	0\\
3.56868921723043	0\\
3.56878921973049	0\\
3.56888922223056	0\\
3.56898922473062	0\\
3.56908922723068	0\\
3.56918922973074	0\\
3.56928923223081	0\\
3.56938923473087	0\\
3.56948923723093	0\\
3.56958923973099	0\\
3.56968924223106	0\\
3.56978924473112	0\\
3.56988924723118	0\\
3.56998924973124	0\\
3.57008925223131	0\\
3.57018925473137	0\\
3.57028925723143	0\\
3.57038925973149	0\\
3.57048926223156	0\\
3.57058926473162	0\\
3.57068926723168	0\\
3.57078926973174	0\\
3.57088927223181	0\\
3.57098927473187	0\\
3.57108927723193	0\\
3.57118927973199	0\\
3.57128928223206	0\\
3.57138928473212	0\\
3.57148928723218	0\\
3.57158928973224	0\\
3.57168929223231	0\\
3.57178929473237	0\\
3.57188929723243	0\\
3.57198929973249	0\\
3.57208930223256	0\\
3.57218930473262	0\\
3.57228930723268	0\\
3.57238930973274	0\\
3.57248931223281	0\\
3.57258931473287	0\\
3.57268931723293	0\\
3.57278931973299	0\\
3.57288932223306	0\\
3.57298932473312	0\\
3.57308932723318	0\\
3.57318932973324	0\\
3.57328933223331	0\\
3.57338933473337	0\\
3.57348933723343	0\\
3.57358933973349	0\\
3.57368934223356	0\\
3.57378934473362	0\\
3.57388934723368	0\\
3.57398934973374	0\\
3.57408935223381	0\\
3.57418935473387	0\\
3.57428935723393	0\\
3.57438935973399	0\\
3.57448936223406	0\\
3.57458936473412	0\\
3.57468936723418	0\\
3.57478936973424	0\\
3.57488937223431	0\\
3.57498937473437	0\\
3.57508937723443	0\\
3.57518937973449	0\\
3.57528938223456	0\\
3.57538938473462	0\\
3.57548938723468	0\\
3.57558938973474	0\\
3.57568939223481	0\\
3.57578939473487	0\\
3.57588939723493	0\\
3.57598939973499	0\\
3.57608940223506	0\\
3.57618940473512	0\\
3.57628940723518	0\\
3.57638940973524	0\\
3.57648941223531	0\\
3.57658941473537	0\\
3.57668941723543	0\\
3.57678941973549	0\\
3.57688942223556	0\\
3.57698942473562	0\\
3.57708942723568	0\\
3.57718942973574	0\\
3.57728943223581	0\\
3.57738943473587	0\\
3.57748943723593	0\\
3.57758943973599	0\\
3.57768944223606	0\\
3.57778944473612	0\\
3.57788944723618	0\\
3.57798944973624	0\\
3.57808945223631	0\\
3.57818945473637	0\\
3.57828945723643	0\\
3.57838945973649	0\\
3.57848946223656	0\\
3.57858946473662	0\\
3.57868946723668	0\\
3.57878946973674	0\\
3.57888947223681	0\\
3.57898947473687	0\\
3.57908947723693	0\\
3.57918947973699	0\\
3.57928948223706	0\\
3.57938948473712	0\\
3.57948948723718	0\\
3.57958948973724	0\\
3.57968949223731	0\\
3.57978949473737	0\\
3.57988949723743	0\\
3.57998949973749	0\\
3.58008950223756	0\\
3.58018950473762	0\\
3.58028950723768	0\\
3.58038950973774	0\\
3.58048951223781	0\\
3.58058951473787	0\\
3.58068951723793	0\\
3.58078951973799	0\\
3.58088952223806	0\\
3.58098952473812	0\\
3.58108952723818	0\\
3.58118952973824	0\\
3.58128953223831	0\\
3.58138953473837	0\\
3.58148953723843	0\\
3.58158953973849	0\\
3.58168954223856	0\\
3.58178954473862	0\\
3.58188954723868	0\\
3.58198954973874	0\\
3.58208955223881	0\\
3.58218955473887	0\\
3.58228955723893	0\\
3.58238955973899	0\\
3.58248956223906	0\\
3.58258956473912	0\\
3.58268956723918	0\\
3.58278956973924	0\\
3.58288957223931	0\\
3.58298957473937	0\\
3.58308957723943	0\\
3.58318957973949	0\\
3.58328958223956	0\\
3.58338958473962	0\\
3.58348958723968	0\\
3.58358958973974	0\\
3.58368959223981	0\\
3.58378959473987	0\\
3.58388959723993	0\\
3.58398959973999	0\\
3.58408960224006	0\\
3.58418960474012	0\\
3.58428960724018	0\\
3.58438960974024	0\\
3.58448961224031	0\\
3.58458961474037	0\\
3.58468961724043	0\\
3.58478961974049	0\\
3.58488962224056	0\\
3.58498962474062	0\\
3.58508962724068	0\\
3.58518962974074	0\\
3.58528963224081	0\\
3.58538963474087	0\\
3.58548963724093	0\\
3.58558963974099	0\\
3.58568964224106	0\\
3.58578964474112	0\\
3.58588964724118	0\\
3.58598964974124	0\\
3.58608965224131	0\\
3.58618965474137	0\\
3.58628965724143	0\\
3.58638965974149	0\\
3.58648966224156	0\\
3.58658966474162	0\\
3.58668966724168	0\\
3.58678966974174	0\\
3.58688967224181	0\\
3.58698967474187	0\\
3.58708967724193	0\\
3.58718967974199	0\\
3.58728968224206	0\\
3.58738968474212	0\\
3.58748968724218	0\\
3.58758968974224	0\\
3.58768969224231	0\\
3.58778969474237	0\\
3.58788969724243	0\\
3.58798969974249	0\\
3.58808970224256	0\\
3.58818970474262	0\\
3.58828970724268	0\\
3.58838970974274	0\\
3.58848971224281	0\\
3.58858971474287	0\\
3.58868971724293	0\\
3.58878971974299	0\\
3.58888972224306	0\\
3.58898972474312	0\\
3.58908972724318	0\\
3.58918972974324	0\\
3.58928973224331	0\\
3.58938973474337	0\\
3.58948973724343	0\\
3.58958973974349	0\\
3.58968974224356	0\\
3.58978974474362	0\\
3.58988974724368	0\\
3.58998974974374	0\\
3.59008975224381	0\\
3.59018975474387	0\\
3.59028975724393	0\\
3.59038975974399	0\\
3.59048976224406	0\\
3.59058976474412	0\\
3.59068976724418	0\\
3.59078976974424	0\\
3.59088977224431	0\\
3.59098977474437	0\\
3.59108977724443	0\\
3.59118977974449	0\\
3.59128978224456	0\\
3.59138978474462	0\\
3.59148978724468	0\\
3.59158978974474	0\\
3.59168979224481	0\\
3.59178979474487	0\\
3.59188979724493	0\\
3.59198979974499	0\\
3.59208980224506	0\\
3.59218980474512	0\\
3.59228980724518	0\\
3.59238980974524	0\\
3.59248981224531	0\\
3.59258981474537	0\\
3.59268981724543	0\\
3.59278981974549	0\\
3.59288982224556	0\\
3.59298982474562	0\\
3.59308982724568	0\\
3.59318982974574	0\\
3.59328983224581	0\\
3.59338983474587	0\\
3.59348983724593	0\\
3.59358983974599	0\\
3.59368984224606	0\\
3.59378984474612	0\\
3.59388984724618	0\\
3.59398984974624	0\\
3.59408985224631	0\\
3.59418985474637	0\\
3.59428985724643	0\\
3.59438985974649	0\\
3.59448986224656	0\\
3.59458986474662	0\\
3.59468986724668	0\\
3.59478986974674	0\\
3.59488987224681	0\\
3.59498987474687	0\\
3.59508987724693	0\\
3.59518987974699	0\\
3.59528988224706	0\\
3.59538988474712	0\\
3.59548988724718	0\\
3.59558988974724	0\\
3.59568989224731	0\\
3.59578989474737	0\\
3.59588989724743	0\\
3.59598989974749	0\\
3.59608990224756	0\\
3.59618990474762	0\\
3.59628990724768	0\\
3.59638990974774	0\\
3.59648991224781	0\\
3.59658991474787	0\\
3.59668991724793	0\\
3.59678991974799	0\\
3.59688992224806	0\\
3.59698992474812	0\\
3.59708992724818	0\\
3.59718992974824	0\\
3.59728993224831	0\\
3.59738993474837	0\\
3.59748993724843	0\\
3.59758993974849	0\\
3.59768994224856	0\\
3.59778994474862	0\\
3.59788994724868	0\\
3.59798994974874	0\\
3.59808995224881	0\\
3.59818995474887	0\\
3.59828995724893	0\\
3.59838995974899	0\\
3.59848996224906	0\\
3.59858996474912	0\\
3.59868996724918	0\\
3.59878996974924	0\\
3.59888997224931	0\\
3.59898997474937	0\\
3.59908997724943	0\\
3.59918997974949	0\\
3.59928998224956	0\\
3.59938998474962	0\\
3.59948998724968	0\\
3.59958998974974	0\\
3.59968999224981	0\\
3.59978999474987	0\\
3.59988999724993	0\\
3.59998999974999	0\\
3.60009000225006	0\\
};
\addplot [color=mycolor2,solid]
  table[row sep=crcr]{%
3.60009000225006	0\\
3.60019000475012	0\\
3.60029000725018	0\\
3.60039000975024	0\\
3.60049001225031	0\\
3.60059001475037	0\\
3.60069001725043	0\\
3.60079001975049	0\\
3.60089002225056	0\\
3.60099002475062	0\\
3.60109002725068	0\\
3.60119002975074	0\\
3.60129003225081	0\\
3.60139003475087	0\\
3.60149003725093	0\\
3.60159003975099	0\\
3.60169004225106	0\\
3.60179004475112	0\\
3.60189004725118	0\\
3.60199004975124	0\\
3.60209005225131	0\\
3.60219005475137	0\\
3.60229005725143	0\\
3.60239005975149	0\\
3.60249006225156	0\\
3.60259006475162	0\\
3.60269006725168	0\\
3.60279006975174	0\\
3.60289007225181	0\\
3.60299007475187	0\\
3.60309007725193	0\\
3.60319007975199	0\\
3.60329008225206	0\\
3.60339008475212	0\\
3.60349008725218	0\\
3.60359008975224	0\\
3.60369009225231	0\\
3.60379009475237	0\\
3.60389009725243	0\\
3.60399009975249	0\\
3.60409010225256	0\\
3.60419010475262	0\\
3.60429010725268	0\\
3.60439010975274	0\\
3.60449011225281	0\\
3.60459011475287	0\\
3.60469011725293	0\\
3.60479011975299	0\\
3.60489012225306	0\\
3.60499012475312	0\\
3.60509012725318	0\\
3.60519012975324	0\\
3.60529013225331	0\\
3.60539013475337	0\\
3.60549013725343	0\\
3.60559013975349	0\\
3.60569014225356	0\\
3.60579014475362	0\\
3.60589014725368	0\\
3.60599014975374	0\\
3.60609015225381	0\\
3.60619015475387	0\\
3.60629015725393	0\\
3.60639015975399	0\\
3.60649016225406	0\\
3.60659016475412	0\\
3.60669016725418	0\\
3.60679016975424	0\\
3.60689017225431	0\\
3.60699017475437	0\\
3.60709017725443	0\\
3.60719017975449	0\\
3.60729018225456	0\\
3.60739018475462	0\\
3.60749018725468	0\\
3.60759018975474	0\\
3.60769019225481	0\\
3.60779019475487	0\\
3.60789019725493	0\\
3.60799019975499	0\\
3.60809020225506	0\\
3.60819020475512	0\\
3.60829020725518	0\\
3.60839020975524	0\\
3.60849021225531	0\\
3.60859021475537	0\\
3.60869021725543	0\\
3.60879021975549	0\\
3.60889022225556	0\\
3.60899022475562	0\\
3.60909022725568	0\\
3.60919022975574	0\\
3.60929023225581	0\\
3.60939023475587	0\\
3.60949023725593	0\\
3.60959023975599	0\\
3.60969024225606	0\\
3.60979024475612	0\\
3.60989024725618	0\\
3.60999024975624	0\\
3.61009025225631	0\\
3.61019025475637	0\\
3.61029025725643	0\\
3.61039025975649	0\\
3.61049026225656	0\\
3.61059026475662	0\\
3.61069026725668	0\\
3.61079026975674	0\\
3.61089027225681	0\\
3.61099027475687	0\\
3.61109027725693	0\\
3.61119027975699	0\\
3.61129028225706	0\\
3.61139028475712	0\\
3.61149028725718	0\\
3.61159028975724	0\\
3.61169029225731	0\\
3.61179029475737	0\\
3.61189029725743	0\\
3.61199029975749	0\\
3.61209030225756	0\\
3.61219030475762	0\\
3.61229030725768	0\\
3.61239030975774	0\\
3.61249031225781	0\\
3.61259031475787	0\\
3.61269031725793	0\\
3.61279031975799	0\\
3.61289032225806	0\\
3.61299032475812	0\\
3.61309032725818	0\\
3.61319032975824	0\\
3.61329033225831	0\\
3.61339033475837	0\\
3.61349033725843	0\\
3.61359033975849	0\\
3.61369034225856	0\\
3.61379034475862	0\\
3.61389034725868	0\\
3.61399034975874	0\\
3.61409035225881	0\\
3.61419035475887	0\\
3.61429035725893	0\\
3.61439035975899	0\\
3.61449036225906	0\\
3.61459036475912	0\\
3.61469036725918	0\\
3.61479036975924	0\\
3.61489037225931	0\\
3.61499037475937	0\\
3.61509037725943	0\\
3.61519037975949	0\\
3.61529038225956	0\\
3.61539038475962	0\\
3.61549038725968	0\\
3.61559038975974	0\\
3.61569039225981	0\\
3.61579039475987	0\\
3.61589039725993	0\\
3.61599039975999	0\\
3.61609040226006	0\\
3.61619040476012	0\\
3.61629040726018	0\\
3.61639040976024	0\\
3.61649041226031	0\\
3.61659041476037	0\\
3.61669041726043	0\\
3.61679041976049	0\\
3.61689042226056	0\\
3.61699042476062	0\\
3.61709042726068	0\\
3.61719042976074	0\\
3.61729043226081	0\\
3.61739043476087	0\\
3.61749043726093	0\\
3.61759043976099	0\\
3.61769044226106	0\\
3.61779044476112	0\\
3.61789044726118	0\\
3.61799044976124	0\\
3.61809045226131	0\\
3.61819045476137	0\\
3.61829045726143	0\\
3.61839045976149	0\\
3.61849046226156	0\\
3.61859046476162	0\\
3.61869046726168	0\\
3.61879046976174	0\\
3.61889047226181	0\\
3.61899047476187	0\\
3.61909047726193	0\\
3.61919047976199	0\\
3.61929048226206	0\\
3.61939048476212	0\\
3.61949048726218	0\\
3.61959048976224	0\\
3.61969049226231	0\\
3.61979049476237	0\\
3.61989049726243	0\\
3.61999049976249	0\\
3.62009050226256	0\\
3.62019050476262	0\\
3.62029050726268	0\\
3.62039050976274	0\\
3.62049051226281	0\\
3.62059051476287	0\\
3.62069051726293	0\\
3.62079051976299	0\\
3.62089052226306	0\\
3.62099052476312	0\\
3.62109052726318	0\\
3.62119052976324	0\\
3.62129053226331	0\\
3.62139053476337	0\\
3.62149053726343	0\\
3.62159053976349	0\\
3.62169054226356	0\\
3.62179054476362	0\\
3.62189054726368	0\\
3.62199054976374	0\\
3.62209055226381	0\\
3.62219055476387	0\\
3.62229055726393	0\\
3.62239055976399	0\\
3.62249056226406	0\\
3.62259056476412	0\\
3.62269056726418	0\\
3.62279056976424	0\\
3.62289057226431	0\\
3.62299057476437	0\\
3.62309057726443	0\\
3.62319057976449	0\\
3.62329058226456	0\\
3.62339058476462	0\\
3.62349058726468	0\\
3.62359058976474	0\\
3.62369059226481	0\\
3.62379059476487	0\\
3.62389059726493	0\\
3.62399059976499	0\\
3.62409060226506	0\\
3.62419060476512	0\\
3.62429060726518	0\\
3.62439060976524	0\\
3.62449061226531	0\\
3.62459061476537	0\\
3.62469061726543	0\\
3.62479061976549	0\\
3.62489062226556	0\\
3.62499062476562	0\\
3.62509062726568	0\\
3.62519062976574	0\\
3.62529063226581	0\\
3.62539063476587	0\\
3.62549063726593	0\\
3.62559063976599	0\\
3.62569064226606	0\\
3.62579064476612	0\\
3.62589064726618	0\\
3.62599064976624	0\\
3.62609065226631	0\\
3.62619065476637	0\\
3.62629065726643	0\\
3.62639065976649	0\\
3.62649066226656	0\\
3.62659066476662	0\\
3.62669066726668	0\\
3.62679066976674	0\\
3.62689067226681	0\\
3.62699067476687	0\\
3.62709067726693	0\\
3.62719067976699	0\\
3.62729068226706	0\\
3.62739068476712	0\\
3.62749068726718	0\\
3.62759068976724	0\\
3.62769069226731	0\\
3.62779069476737	0\\
3.62789069726743	0\\
3.62799069976749	0\\
3.62809070226756	0\\
3.62819070476762	0\\
3.62829070726768	0\\
3.62839070976774	0\\
3.62849071226781	0\\
3.62859071476787	0\\
3.62869071726793	0\\
3.62879071976799	0\\
3.62889072226806	0\\
3.62899072476812	0\\
3.62909072726818	0\\
3.62919072976824	0\\
3.62929073226831	0\\
3.62939073476837	0\\
3.62949073726843	0\\
3.62959073976849	0\\
3.62969074226856	0\\
3.62979074476862	0\\
3.62989074726868	0\\
3.62999074976874	0\\
3.63009075226881	0\\
3.63019075476887	0\\
3.63029075726893	0\\
3.63039075976899	0\\
3.63049076226906	0\\
3.63059076476912	0\\
3.63069076726918	0\\
3.63079076976924	0\\
3.63089077226931	0\\
3.63099077476937	0\\
3.63109077726943	0\\
3.63119077976949	0\\
3.63129078226956	0\\
3.63139078476962	0\\
3.63149078726968	0\\
3.63159078976974	0\\
3.63169079226981	0\\
3.63179079476987	0\\
3.63189079726993	0\\
3.63199079976999	0\\
3.63209080227006	0\\
3.63219080477012	0\\
3.63229080727018	0\\
3.63239080977024	0\\
3.63249081227031	0\\
3.63259081477037	0\\
3.63269081727043	0\\
3.63279081977049	0\\
3.63289082227056	0\\
3.63299082477062	0\\
3.63309082727068	0\\
3.63319082977074	0\\
3.63329083227081	0\\
3.63339083477087	0\\
3.63349083727093	0\\
3.63359083977099	0\\
3.63369084227106	0\\
3.63379084477112	0\\
3.63389084727118	0\\
3.63399084977124	0\\
3.63409085227131	0\\
3.63419085477137	0\\
3.63429085727143	0\\
3.63439085977149	0\\
3.63449086227156	0\\
3.63459086477162	0\\
3.63469086727168	0\\
3.63479086977174	0\\
3.63489087227181	0\\
3.63499087477187	0\\
3.63509087727193	0\\
3.63519087977199	0\\
3.63529088227206	0\\
3.63539088477212	0\\
3.63549088727218	0\\
3.63559088977224	0\\
3.63569089227231	0\\
3.63579089477237	0\\
3.63589089727243	0\\
3.63599089977249	0\\
3.63609090227256	0\\
3.63619090477262	0\\
3.63629090727268	0\\
3.63639090977274	0\\
3.63649091227281	0\\
3.63659091477287	0\\
3.63669091727293	0\\
3.63679091977299	0\\
3.63689092227306	0\\
3.63699092477312	0\\
3.63709092727318	0\\
3.63719092977324	0\\
3.63729093227331	0\\
3.63739093477337	0\\
3.63749093727343	0\\
3.63759093977349	0\\
3.63769094227356	0\\
3.63779094477362	0\\
3.63789094727368	0\\
3.63799094977374	0\\
3.63809095227381	0\\
3.63819095477387	0\\
3.63829095727393	0\\
3.63839095977399	0\\
3.63849096227406	0\\
3.63859096477412	0\\
3.63869096727418	0\\
3.63879096977424	0\\
3.63889097227431	0\\
3.63899097477437	0\\
3.63909097727443	0\\
3.63919097977449	0\\
3.63929098227456	0\\
3.63939098477462	0\\
3.63949098727468	0\\
3.63959098977474	0\\
3.63969099227481	0\\
3.63979099477487	0\\
3.63989099727493	0\\
3.63999099977499	0\\
3.64009100227506	0\\
3.64019100477512	0\\
3.64029100727518	0\\
3.64039100977524	0\\
3.64049101227531	0\\
3.64059101477537	0\\
3.64069101727543	0\\
3.64079101977549	0\\
3.64089102227556	0\\
3.64099102477562	0\\
3.64109102727568	0\\
3.64119102977574	0\\
3.64129103227581	0\\
3.64139103477587	0\\
3.64149103727593	0\\
3.64159103977599	0\\
3.64169104227606	0\\
3.64179104477612	0\\
3.64189104727618	0\\
3.64199104977624	0\\
3.64209105227631	0\\
3.64219105477637	0\\
3.64229105727643	0\\
3.64239105977649	0\\
3.64249106227656	0\\
3.64259106477662	0\\
3.64269106727668	0\\
3.64279106977674	0\\
3.64289107227681	0\\
3.64299107477687	0\\
3.64309107727693	0\\
3.64319107977699	0\\
3.64329108227706	0\\
3.64339108477712	0\\
3.64349108727718	0\\
3.64359108977724	0\\
3.64369109227731	0\\
3.64379109477737	0\\
3.64389109727743	0\\
3.64399109977749	0\\
3.64409110227756	0\\
3.64419110477762	0\\
3.64429110727768	0\\
3.64439110977774	0\\
3.64449111227781	0\\
3.64459111477787	0\\
3.64469111727793	0\\
3.64479111977799	0\\
3.64489112227806	0\\
3.64499112477812	0\\
3.64509112727818	0\\
3.64519112977824	0\\
3.64529113227831	0\\
3.64539113477837	0\\
3.64549113727843	0\\
3.64559113977849	0\\
3.64569114227856	0\\
3.64579114477862	0\\
3.64589114727868	0\\
3.64599114977874	0\\
3.64609115227881	0\\
3.64619115477887	0\\
3.64629115727893	0\\
3.64639115977899	0\\
3.64649116227906	0\\
3.64659116477912	0\\
3.64669116727918	0\\
3.64679116977924	0\\
3.64689117227931	0\\
3.64699117477937	0\\
3.64709117727943	0\\
3.64719117977949	0\\
3.64729118227956	0\\
3.64739118477962	0\\
3.64749118727968	0\\
3.64759118977974	0\\
3.64769119227981	0\\
3.64779119477987	0\\
3.64789119727993	0\\
3.64799119977999	0\\
3.64809120228006	0\\
3.64819120478012	0\\
3.64829120728018	0\\
3.64839120978024	0\\
3.64849121228031	0\\
3.64859121478037	0\\
3.64869121728043	0\\
3.64879121978049	0\\
3.64889122228056	0\\
3.64899122478062	0\\
3.64909122728068	0\\
3.64919122978074	0\\
3.64929123228081	0\\
3.64939123478087	0\\
3.64949123728093	0\\
3.64959123978099	0\\
3.64969124228106	0\\
3.64979124478112	0\\
3.64989124728118	0\\
3.64999124978124	0\\
3.65009125228131	0\\
3.65019125478137	0\\
3.65029125728143	0\\
3.65039125978149	0\\
3.65049126228156	0\\
3.65059126478162	0\\
3.65069126728168	0\\
3.65079126978174	0\\
3.65089127228181	0\\
3.65099127478187	0\\
3.65109127728193	0\\
3.65119127978199	0\\
3.65129128228206	0\\
3.65139128478212	0\\
3.65149128728218	0\\
3.65159128978224	0\\
3.65169129228231	0\\
3.65179129478237	0\\
3.65189129728243	0\\
3.65199129978249	0\\
3.65209130228256	0\\
3.65219130478262	0\\
3.65229130728268	0\\
3.65239130978274	0\\
3.65249131228281	0\\
3.65259131478287	0\\
3.65269131728293	0\\
3.65279131978299	0\\
3.65289132228306	0\\
3.65299132478312	0\\
3.65309132728318	0\\
3.65319132978324	0\\
3.65329133228331	0\\
3.65339133478337	0\\
3.65349133728343	0\\
3.65359133978349	0\\
3.65369134228356	0\\
3.65379134478362	0\\
3.65389134728368	0\\
3.65399134978374	0\\
3.65409135228381	0\\
3.65419135478387	0\\
3.65429135728393	0\\
3.65439135978399	0\\
3.65449136228406	0\\
3.65459136478412	0\\
3.65469136728418	0\\
3.65479136978424	0\\
3.65489137228431	0\\
3.65499137478437	0\\
3.65509137728443	0\\
3.65519137978449	0\\
3.65529138228456	0\\
3.65539138478462	0\\
3.65549138728468	0\\
3.65559138978474	0\\
3.65569139228481	0\\
3.65579139478487	0\\
3.65589139728493	0\\
3.65599139978499	0\\
3.65609140228506	0\\
3.65619140478512	0\\
3.65629140728518	0\\
3.65639140978524	0\\
3.65649141228531	0\\
3.65659141478537	0\\
3.65669141728543	0\\
3.65679141978549	0\\
3.65689142228556	0\\
3.65699142478562	0\\
3.65709142728568	0\\
3.65719142978574	0\\
3.65729143228581	0\\
3.65739143478587	0\\
3.65749143728593	0\\
3.65759143978599	0\\
3.65769144228606	0\\
3.65779144478612	0\\
3.65789144728618	0\\
3.65799144978624	0\\
3.65809145228631	0\\
3.65819145478637	0\\
3.65829145728643	0\\
3.65839145978649	0\\
3.65849146228656	0\\
3.65859146478662	0\\
3.65869146728668	0\\
3.65879146978674	0\\
3.65889147228681	0\\
3.65899147478687	0\\
3.65909147728693	0\\
3.65919147978699	0\\
3.65929148228706	0\\
3.65939148478712	0\\
3.65949148728718	0\\
3.65959148978724	0\\
3.65969149228731	0\\
3.65979149478737	0\\
3.65989149728743	0\\
3.65999149978749	0\\
3.66009150228756	0\\
3.66019150478762	0\\
3.66029150728768	0\\
3.66039150978774	0\\
3.66049151228781	0\\
3.66059151478787	0\\
3.66069151728793	0\\
3.66079151978799	0\\
3.66089152228806	0\\
3.66099152478812	0\\
3.66109152728818	0\\
3.66119152978824	0\\
3.66129153228831	0\\
3.66139153478837	0\\
3.66149153728843	0\\
3.66159153978849	0\\
3.66169154228856	0\\
3.66179154478862	0\\
3.66189154728868	0\\
3.66199154978874	0\\
3.66209155228881	0\\
3.66219155478887	0\\
3.66229155728893	0\\
3.66239155978899	0\\
3.66249156228906	0\\
3.66259156478912	0\\
3.66269156728918	0\\
3.66279156978924	0\\
3.66289157228931	0\\
3.66299157478937	0\\
3.66309157728943	0\\
3.66319157978949	0\\
3.66329158228956	0\\
3.66339158478962	0\\
3.66349158728968	0\\
3.66359158978974	0\\
3.66369159228981	0\\
3.66379159478987	0\\
3.66389159728993	0\\
3.66399159978999	0\\
3.66409160229006	0\\
3.66419160479012	0\\
3.66429160729018	0\\
3.66439160979024	0\\
3.66449161229031	0\\
3.66459161479037	0\\
3.66469161729043	0\\
3.66479161979049	0\\
3.66489162229056	0\\
3.66499162479062	0\\
3.66509162729068	0\\
3.66519162979074	0\\
3.66529163229081	0\\
3.66539163479087	0\\
3.66549163729093	0\\
3.66559163979099	0\\
3.66569164229106	0\\
3.66579164479112	0\\
3.66589164729118	0\\
3.66599164979124	0\\
3.66609165229131	0\\
3.66619165479137	0\\
3.66629165729143	0\\
3.66639165979149	0\\
3.66649166229156	0\\
3.66659166479162	0\\
3.66669166729168	0\\
3.66679166979174	0\\
3.66689167229181	0\\
3.66699167479187	0\\
3.66709167729193	0\\
3.667191679792	0\\
3.66729168229206	0\\
3.66739168479212	0\\
3.66749168729218	0\\
3.66759168979224	0\\
3.66769169229231	0\\
3.66779169479237	0\\
3.66789169729243	0\\
3.66799169979249	0\\
3.66809170229256	0\\
3.66819170479262	0\\
3.66829170729268	0\\
3.66839170979274	0\\
3.66849171229281	0\\
3.66859171479287	0\\
3.66869171729293	0\\
3.66879171979299	0\\
3.66889172229306	0\\
3.66899172479312	0\\
3.66909172729318	0\\
3.66919172979325	0\\
3.66929173229331	0\\
3.66939173479337	0\\
3.66949173729343	0\\
3.66959173979349	0\\
3.66969174229356	0\\
3.66979174479362	0\\
3.66989174729368	0\\
3.66999174979374	0\\
3.67009175229381	0\\
3.67019175479387	0\\
3.67029175729393	0\\
3.67039175979399	0\\
3.67049176229406	0\\
3.67059176479412	0\\
3.67069176729418	0\\
3.67079176979424	0\\
3.67089177229431	0\\
3.67099177479437	0\\
3.67109177729443	0\\
3.6711917797945	0\\
3.67129178229456	0\\
3.67139178479462	0\\
3.67149178729468	0\\
3.67159178979474	0\\
3.67169179229481	0\\
3.67179179479487	0\\
3.67189179729493	0\\
3.67199179979499	0\\
3.67209180229506	0\\
3.67219180479512	0\\
3.67229180729518	0\\
3.67239180979524	0\\
3.67249181229531	0\\
3.67259181479537	0\\
3.67269181729543	0\\
3.67279181979549	0\\
3.67289182229556	0\\
3.67299182479562	0\\
3.67309182729568	0\\
3.67319182979575	0\\
3.67329183229581	0\\
3.67339183479587	0\\
3.67349183729593	0\\
3.673591839796	0\\
3.67369184229606	0\\
3.67379184479612	0\\
3.67389184729618	0\\
3.67399184979624	0\\
3.67409185229631	0\\
3.67419185479637	0\\
3.67429185729643	0\\
3.67439185979649	0\\
3.67449186229656	0\\
3.67459186479662	0\\
3.67469186729668	0\\
3.67479186979674	0\\
3.67489187229681	0\\
3.67499187479687	0\\
3.67509187729693	0\\
3.675191879797	0\\
3.67529188229706	0\\
3.67539188479712	0\\
3.67549188729718	0\\
3.67559188979725	0\\
3.67569189229731	0\\
3.67579189479737	0\\
3.67589189729743	0\\
3.67599189979749	0\\
3.67609190229756	0\\
3.67619190479762	0\\
3.67629190729768	0\\
3.67639190979774	0\\
3.67649191229781	0\\
3.67659191479787	0\\
3.67669191729793	0\\
3.67679191979799	0\\
3.67689192229806	0\\
3.67699192479812	0\\
3.67709192729818	0\\
3.67719192979825	0\\
3.67729193229831	0\\
3.67739193479837	0\\
3.67749193729843	0\\
3.6775919397985	0\\
3.67769194229856	0\\
3.67779194479862	0\\
3.67789194729868	0\\
3.67799194979874	0\\
3.67809195229881	0\\
3.67819195479887	0\\
3.67829195729893	0\\
3.67839195979899	0\\
3.67849196229906	0\\
3.67859196479912	0\\
3.67869196729918	0\\
3.67879196979924	0\\
3.67889197229931	0\\
3.67899197479937	0\\
3.67909197729943	0\\
3.6791919797995	0\\
3.67929198229956	0\\
3.67939198479962	0\\
3.67949198729968	0\\
3.67959198979975	0\\
3.67969199229981	0\\
3.67979199479987	0\\
3.67989199729993	0\\
3.6799919998	0\\
3.68009200230006	0\\
3.68019200480012	0\\
3.68029200730018	0\\
3.68039200980024	0\\
3.68049201230031	0\\
3.68059201480037	0\\
3.68069201730043	0\\
3.68079201980049	0\\
3.68089202230056	0\\
3.68099202480062	0\\
3.68109202730068	0\\
3.68119202980075	0\\
3.68129203230081	0\\
3.68139203480087	0\\
3.68149203730093	0\\
3.681592039801	0\\
3.68169204230106	0\\
3.68179204480112	0\\
3.68189204730118	0\\
3.68199204980125	0\\
3.68209205230131	0\\
3.68219205480137	0\\
3.68229205730143	0\\
3.68239205980149	0\\
3.68249206230156	0\\
3.68259206480162	0\\
3.68269206730168	0\\
3.68279206980174	0\\
3.68289207230181	0\\
3.68299207480187	0\\
3.68309207730193	0\\
3.683192079802	0\\
3.68329208230206	0\\
3.68339208480212	0\\
3.68349208730218	0\\
3.68359208980225	0\\
3.68369209230231	0\\
3.68379209480237	0\\
3.68389209730243	0\\
3.6839920998025	0\\
3.68409210230256	0\\
3.68419210480262	0\\
3.68429210730268	0\\
3.68439210980275	0\\
3.68449211230281	0\\
3.68459211480287	0\\
3.68469211730293	0\\
3.68479211980299	0\\
3.68489212230306	0\\
3.68499212480312	0\\
3.68509212730318	0\\
3.68519212980325	0\\
3.68529213230331	0\\
3.68539213480337	0\\
3.68549213730343	0\\
3.6855921398035	0\\
3.68569214230356	0\\
3.68579214480362	0\\
3.68589214730368	0\\
3.68599214980375	0\\
3.68609215230381	0\\
3.68619215480387	0\\
3.68629215730393	0\\
3.686392159804	0\\
3.68649216230406	0\\
3.68659216480412	0\\
3.68669216730418	0\\
3.68679216980424	0\\
3.68689217230431	0\\
3.68699217480437	0\\
3.68709217730443	0\\
3.6871921798045	0\\
3.68729218230456	0\\
3.68739218480462	0\\
3.68749218730468	0\\
3.68759218980475	0\\
3.68769219230481	0\\
3.68779219480487	0\\
3.68789219730493	0\\
3.687992199805	0\\
3.68809220230506	0\\
3.68819220480512	0\\
3.68829220730518	0\\
3.68839220980525	0\\
3.68849221230531	0\\
3.68859221480537	0\\
3.68869221730543	0\\
3.68879221980549	0\\
3.68889222230556	0\\
3.68899222480562	0\\
3.68909222730568	0\\
3.68919222980575	0\\
3.68929223230581	0\\
3.68939223480587	0\\
3.68949223730593	0\\
3.689592239806	0\\
3.68969224230606	0\\
3.68979224480612	0\\
3.68989224730618	0\\
3.68999224980625	0\\
3.69009225230631	0\\
3.69019225480637	0\\
3.69029225730643	0\\
3.6903922598065	0\\
3.69049226230656	0\\
3.69059226480662	0\\
3.69069226730668	0\\
3.69079226980675	0\\
3.69089227230681	0\\
3.69099227480687	0\\
3.69109227730693	0\\
3.691192279807	0\\
3.69129228230706	0\\
3.69139228480712	0\\
3.69149228730718	0\\
3.69159228980725	0\\
3.69169229230731	0\\
3.69179229480737	0\\
3.69189229730743	0\\
3.6919922998075	0\\
3.69209230230756	0\\
3.69219230480762	0\\
3.69229230730768	0\\
3.69239230980775	0\\
3.69249231230781	0\\
3.69259231480787	0\\
3.69269231730793	0\\
3.692792319808	0\\
3.69289232230806	0\\
3.69299232480812	0\\
3.69309232730818	0\\
3.69319232980825	0\\
3.69329233230831	0\\
3.69339233480837	0\\
3.69349233730843	0\\
3.6935923398085	0\\
3.69369234230856	0\\
3.69379234480862	0\\
3.69389234730868	0\\
3.69399234980875	0\\
3.69409235230881	0\\
3.69419235480887	0\\
3.69429235730893	0\\
3.694392359809	0\\
3.69449236230906	0\\
3.69459236480912	0\\
3.69469236730918	0\\
3.69479236980925	0\\
3.69489237230931	0\\
3.69499237480937	0\\
3.69509237730943	0\\
3.6951923798095	0\\
3.69529238230956	0\\
3.69539238480962	0\\
3.69549238730968	0\\
3.69559238980975	0\\
3.69569239230981	0\\
3.69579239480987	0\\
3.69589239730993	0\\
3.69599239981	0\\
3.69609240231006	0\\
3.69619240481012	0\\
3.69629240731018	0\\
3.69639240981025	0\\
3.69649241231031	0\\
3.69659241481037	0\\
3.69669241731043	0\\
3.6967924198105	0\\
3.69689242231056	0\\
3.69699242481062	0\\
3.69709242731068	0\\
3.69719242981075	0\\
3.69729243231081	0\\
3.69739243481087	0\\
3.69749243731093	0\\
3.697592439811	0\\
3.69769244231106	0\\
3.69779244481112	0\\
3.69789244731118	0\\
3.69799244981125	0\\
3.69809245231131	0\\
3.69819245481137	0\\
3.69829245731143	0\\
3.6983924598115	0\\
3.69849246231156	0\\
3.69859246481162	0\\
3.69869246731168	0\\
3.69879246981175	0\\
3.69889247231181	0\\
3.69899247481187	0\\
3.69909247731193	0\\
3.699192479812	0\\
3.69929248231206	0\\
3.69939248481212	0\\
3.69949248731218	0\\
3.69959248981225	0\\
3.69969249231231	0\\
3.69979249481237	0\\
3.69989249731243	0\\
3.6999924998125	0\\
3.70009250231256	0\\
3.70019250481262	0\\
3.70029250731268	0\\
3.70039250981275	0\\
3.70049251231281	0\\
3.70059251481287	0\\
3.70069251731293	0\\
3.700792519813	0\\
3.70089252231306	0\\
3.70099252481312	0\\
3.70109252731318	0\\
3.70119252981325	0\\
3.70129253231331	0\\
3.70139253481337	0\\
3.70149253731343	0\\
3.7015925398135	0\\
3.70169254231356	0\\
3.70179254481362	0\\
3.70189254731368	0\\
3.70199254981375	0\\
3.70209255231381	0\\
3.70219255481387	0\\
3.70229255731393	0\\
3.702392559814	0\\
3.70249256231406	0\\
3.70259256481412	0\\
3.70269256731418	0\\
3.70279256981425	0\\
3.70289257231431	0\\
3.70299257481437	0\\
3.70309257731443	0\\
3.7031925798145	0\\
3.70329258231456	0\\
3.70339258481462	0\\
3.70349258731468	0\\
3.70359258981475	0\\
3.70369259231481	0\\
3.70379259481487	0\\
3.70389259731493	0\\
3.703992599815	0\\
3.70409260231506	0\\
3.70419260481512	0\\
3.70429260731518	0\\
3.70439260981525	0\\
3.70449261231531	0\\
3.70459261481537	0\\
3.70469261731543	0\\
3.7047926198155	0\\
3.70489262231556	0\\
3.70499262481562	0\\
3.70509262731568	0\\
3.70519262981575	0\\
3.70529263231581	0\\
3.70539263481587	0\\
3.70549263731593	0\\
3.705592639816	0\\
3.70569264231606	0\\
3.70579264481612	0\\
3.70589264731618	0\\
3.70599264981625	0\\
3.70609265231631	0\\
3.70619265481637	0\\
3.70629265731643	0\\
3.7063926598165	0\\
3.70649266231656	0\\
3.70659266481662	0\\
3.70669266731668	0\\
3.70679266981675	0\\
3.70689267231681	0\\
3.70699267481687	0\\
3.70709267731693	0\\
3.707192679817	0\\
3.70729268231706	0\\
3.70739268481712	0\\
3.70749268731718	0\\
3.70759268981725	0\\
3.70769269231731	0\\
3.70779269481737	0\\
3.70789269731743	0\\
3.7079926998175	0\\
3.70809270231756	0\\
3.70819270481762	0\\
3.70829270731768	0\\
3.70839270981775	0\\
3.70849271231781	0\\
3.70859271481787	0\\
3.70869271731793	0\\
3.708792719818	0\\
3.70889272231806	0\\
3.70899272481812	0\\
3.70909272731818	0\\
3.70919272981825	0\\
3.70929273231831	0\\
3.70939273481837	0\\
3.70949273731843	0\\
3.7095927398185	0\\
3.70969274231856	0\\
3.70979274481862	0\\
3.70989274731868	0\\
3.70999274981875	0\\
3.71009275231881	0\\
3.71019275481887	0\\
3.71029275731893	0\\
3.710392759819	0\\
3.71049276231906	0\\
3.71059276481912	0\\
3.71069276731918	0\\
3.71079276981925	0\\
3.71089277231931	0\\
3.71099277481937	0\\
3.71109277731943	0\\
3.7111927798195	0\\
3.71129278231956	0\\
3.71139278481962	0\\
3.71149278731968	0\\
3.71159278981975	0\\
3.71169279231981	0\\
3.71179279481987	0\\
3.71189279731993	0\\
3.71199279982	0\\
3.71209280232006	0\\
3.71219280482012	0\\
3.71229280732018	0\\
3.71239280982025	0\\
3.71249281232031	0\\
3.71259281482037	0\\
3.71269281732043	0\\
3.7127928198205	0\\
3.71289282232056	0\\
3.71299282482062	0\\
3.71309282732068	0\\
3.71319282982075	0\\
3.71329283232081	0\\
3.71339283482087	0\\
3.71349283732093	0\\
3.713592839821	0\\
3.71369284232106	0\\
3.71379284482112	0\\
3.71389284732118	0\\
3.71399284982125	0\\
3.71409285232131	0\\
3.71419285482137	0\\
3.71429285732143	0\\
3.7143928598215	0\\
3.71449286232156	0\\
3.71459286482162	0\\
3.71469286732168	0\\
3.71479286982175	0\\
3.71489287232181	0\\
3.71499287482187	0\\
3.71509287732193	0\\
3.715192879822	0\\
3.71529288232206	0\\
3.71539288482212	0\\
3.71549288732218	0\\
3.71559288982225	0\\
3.71569289232231	0\\
3.71579289482237	0\\
3.71589289732243	0\\
3.7159928998225	0\\
3.71609290232256	0\\
3.71619290482262	0\\
3.71629290732268	0\\
3.71639290982275	0\\
3.71649291232281	0\\
3.71659291482287	0\\
3.71669291732293	0\\
3.716792919823	0\\
3.71689292232306	0\\
3.71699292482312	0\\
3.71709292732318	0\\
3.71719292982325	0\\
3.71729293232331	0\\
3.71739293482337	0\\
3.71749293732343	0\\
3.7175929398235	0\\
3.71769294232356	0\\
3.71779294482362	0\\
3.71789294732368	0\\
3.71799294982375	0\\
3.71809295232381	0\\
3.71819295482387	0\\
3.71829295732393	0\\
3.718392959824	0\\
3.71849296232406	0\\
3.71859296482412	0\\
3.71869296732418	0\\
3.71879296982425	0\\
3.71889297232431	0\\
3.71899297482437	0\\
3.71909297732443	0\\
3.7191929798245	0\\
3.71929298232456	0\\
3.71939298482462	0\\
3.71949298732468	0\\
3.71959298982475	0\\
3.71969299232481	0\\
3.71979299482487	0\\
3.71989299732493	0\\
3.719992999825	0\\
3.72009300232506	0\\
3.72019300482512	0\\
3.72029300732518	0\\
3.72039300982525	0\\
3.72049301232531	0\\
3.72059301482537	0\\
3.72069301732543	0\\
3.7207930198255	0\\
3.72089302232556	0\\
3.72099302482562	0\\
3.72109302732568	0\\
3.72119302982575	0\\
3.72129303232581	0\\
3.72139303482587	0\\
3.72149303732593	0\\
3.721593039826	0\\
3.72169304232606	0\\
3.72179304482612	0\\
3.72189304732618	0\\
3.72199304982625	0\\
3.72209305232631	0\\
3.72219305482637	0\\
3.72229305732643	0\\
3.7223930598265	0\\
3.72249306232656	0\\
3.72259306482662	0\\
3.72269306732668	0\\
3.72279306982675	0\\
3.72289307232681	0\\
3.72299307482687	0\\
3.72309307732693	0\\
3.723193079827	0\\
3.72329308232706	0\\
3.72339308482712	0\\
3.72349308732718	0\\
3.72359308982725	0\\
3.72369309232731	0\\
3.72379309482737	0\\
3.72389309732743	0\\
3.7239930998275	0\\
3.72409310232756	0\\
3.72419310482762	0\\
3.72429310732768	0\\
3.72439310982775	0\\
3.72449311232781	0\\
3.72459311482787	0\\
3.72469311732793	0\\
3.724793119828	0\\
3.72489312232806	0\\
3.72499312482812	0\\
3.72509312732818	0\\
3.72519312982825	0\\
3.72529313232831	0\\
3.72539313482837	0\\
3.72549313732843	0\\
3.7255931398285	0\\
3.72569314232856	0\\
3.72579314482862	0\\
3.72589314732868	0\\
3.72599314982875	0\\
3.72609315232881	0\\
3.72619315482887	0\\
3.72629315732893	0\\
3.726393159829	0\\
3.72649316232906	0\\
3.72659316482912	0\\
3.72669316732918	0\\
3.72679316982925	0\\
3.72689317232931	0\\
3.72699317482937	0\\
3.72709317732943	0\\
3.7271931798295	0\\
3.72729318232956	0\\
3.72739318482962	0\\
3.72749318732968	0\\
3.72759318982975	0\\
3.72769319232981	0\\
3.72779319482987	0\\
3.72789319732993	0\\
3.72799319983	0\\
3.72809320233006	0\\
3.72819320483012	0\\
3.72829320733018	0\\
3.72839320983025	0\\
3.72849321233031	0\\
3.72859321483037	0\\
3.72869321733043	0\\
3.7287932198305	0\\
3.72889322233056	0\\
3.72899322483062	0\\
3.72909322733068	0\\
3.72919322983075	0\\
3.72929323233081	0\\
3.72939323483087	0\\
3.72949323733093	0\\
3.729593239831	0\\
3.72969324233106	0\\
3.72979324483112	0\\
3.72989324733118	0\\
3.72999324983125	0\\
3.73009325233131	0\\
3.73019325483137	0\\
3.73029325733143	0\\
3.7303932598315	0\\
3.73049326233156	0\\
3.73059326483162	0\\
3.73069326733168	0\\
3.73079326983175	0\\
3.73089327233181	0\\
3.73099327483187	0\\
3.73109327733193	0\\
3.731193279832	0\\
3.73129328233206	0\\
3.73139328483212	0\\
3.73149328733218	0\\
3.73159328983225	0\\
3.73169329233231	0\\
3.73179329483237	0\\
3.73189329733243	0\\
3.7319932998325	0\\
3.73209330233256	0\\
3.73219330483262	0\\
3.73229330733268	0\\
3.73239330983275	0\\
3.73249331233281	0\\
3.73259331483287	0\\
3.73269331733293	0\\
3.732793319833	0\\
3.73289332233306	0\\
3.73299332483312	0\\
3.73309332733318	0\\
3.73319332983325	0\\
3.73329333233331	0\\
3.73339333483337	0\\
3.73349333733343	0\\
3.7335933398335	0\\
3.73369334233356	0\\
3.73379334483362	0\\
3.73389334733368	0\\
3.73399334983375	0\\
3.73409335233381	0\\
3.73419335483387	0\\
3.73429335733393	0\\
3.734393359834	0\\
3.73449336233406	0\\
3.73459336483412	0\\
3.73469336733418	0\\
3.73479336983425	0\\
3.73489337233431	0\\
3.73499337483437	0\\
3.73509337733443	0\\
3.7351933798345	0\\
3.73529338233456	0\\
3.73539338483462	0\\
3.73549338733468	0\\
3.73559338983475	0\\
3.73569339233481	0\\
3.73579339483487	0\\
3.73589339733493	0\\
3.735993399835	0\\
3.73609340233506	0\\
3.73619340483512	0\\
3.73629340733518	0\\
3.73639340983525	0\\
3.73649341233531	0\\
3.73659341483537	0\\
3.73669341733543	0\\
3.7367934198355	0\\
3.73689342233556	0\\
3.73699342483562	0\\
3.73709342733568	0\\
3.73719342983575	0\\
3.73729343233581	0\\
3.73739343483587	0\\
3.73749343733593	0\\
3.737593439836	0\\
3.73769344233606	0\\
3.73779344483612	0\\
3.73789344733618	0\\
3.73799344983625	0\\
3.73809345233631	0\\
3.73819345483637	0\\
3.73829345733643	0\\
3.7383934598365	0\\
3.73849346233656	0\\
3.73859346483662	0\\
3.73869346733668	0\\
3.73879346983675	0\\
3.73889347233681	0\\
3.73899347483687	0\\
3.73909347733693	0\\
3.739193479837	0\\
3.73929348233706	0\\
3.73939348483712	0\\
3.73949348733718	0\\
3.73959348983725	0\\
3.73969349233731	0\\
3.73979349483737	0\\
3.73989349733743	0\\
3.7399934998375	0\\
3.74009350233756	0\\
3.74019350483762	0\\
3.74029350733768	0\\
3.74039350983775	0\\
3.74049351233781	0\\
3.74059351483787	0\\
3.74069351733793	0\\
3.740793519838	0\\
3.74089352233806	0\\
3.74099352483812	0\\
3.74109352733818	0\\
3.74119352983825	0\\
3.74129353233831	0\\
3.74139353483837	0\\
3.74149353733843	0\\
3.7415935398385	0\\
3.74169354233856	0\\
3.74179354483862	0\\
3.74189354733868	0\\
3.74199354983875	0\\
3.74209355233881	0\\
3.74219355483887	0\\
3.74229355733893	0\\
3.742393559839	0\\
3.74249356233906	0\\
3.74259356483912	0\\
3.74269356733918	0\\
3.74279356983925	0\\
3.74289357233931	0\\
3.74299357483937	0\\
3.74309357733943	0\\
3.7431935798395	0\\
3.74329358233956	0\\
3.74339358483962	0\\
3.74349358733968	0\\
3.74359358983975	0\\
3.74369359233981	0\\
3.74379359483987	0\\
3.74389359733993	0\\
3.74399359984	0\\
3.74409360234006	0\\
3.74419360484012	0\\
3.74429360734018	0\\
3.74439360984025	0\\
3.74449361234031	0\\
3.74459361484037	0\\
3.74469361734043	0\\
3.7447936198405	0\\
3.74489362234056	0\\
3.74499362484062	0\\
3.74509362734068	0\\
3.74519362984075	0\\
3.74529363234081	0\\
3.74539363484087	0\\
3.74549363734093	0\\
3.745593639841	0\\
3.74569364234106	0\\
3.74579364484112	0\\
3.74589364734118	0\\
3.74599364984125	0\\
3.74609365234131	0\\
3.74619365484137	0\\
3.74629365734143	0\\
3.7463936598415	0\\
3.74649366234156	0\\
3.74659366484162	0\\
3.74669366734168	0\\
3.74679366984175	0\\
3.74689367234181	0\\
3.74699367484187	0\\
3.74709367734193	0\\
3.747193679842	0\\
3.74729368234206	0\\
3.74739368484212	0\\
3.74749368734218	0\\
3.74759368984225	0\\
3.74769369234231	0\\
3.74779369484237	0\\
3.74789369734243	0\\
3.7479936998425	0\\
3.74809370234256	0\\
3.74819370484262	0\\
3.74829370734268	0\\
3.74839370984275	0\\
3.74849371234281	0\\
3.74859371484287	0\\
3.74869371734293	0\\
3.748793719843	0\\
3.74889372234306	0\\
3.74899372484312	0\\
3.74909372734318	0\\
3.74919372984325	0\\
3.74929373234331	0\\
3.74939373484337	0\\
3.74949373734343	0\\
3.7495937398435	0\\
3.74969374234356	0\\
3.74979374484362	0\\
3.74989374734368	0\\
3.74999374984375	0\\
3.75009375234381	0\\
3.75019375484387	0\\
3.75029375734393	0\\
3.750393759844	0\\
3.75049376234406	0\\
3.75059376484412	0\\
3.75069376734418	0\\
3.75079376984425	0\\
3.75089377234431	0\\
3.75099377484437	0\\
3.75109377734443	0\\
3.7511937798445	0\\
3.75129378234456	0\\
3.75139378484462	0\\
3.75149378734468	0\\
3.75159378984475	0\\
3.75169379234481	0\\
3.75179379484487	0\\
3.75189379734493	0\\
3.751993799845	0\\
3.75209380234506	0\\
3.75219380484512	0\\
3.75229380734518	0\\
3.75239380984525	0\\
3.75249381234531	0\\
3.75259381484537	0\\
3.75269381734543	0\\
3.7527938198455	0\\
3.75289382234556	0\\
3.75299382484562	0\\
3.75309382734568	0\\
3.75319382984575	0\\
3.75329383234581	0\\
3.75339383484587	0\\
3.75349383734593	0\\
3.753593839846	0\\
3.75369384234606	0\\
3.75379384484612	0\\
3.75389384734618	0\\
3.75399384984625	0\\
3.75409385234631	0\\
3.75419385484637	0\\
3.75429385734643	0\\
3.7543938598465	0\\
3.75449386234656	0\\
3.75459386484662	0\\
3.75469386734668	0\\
3.75479386984675	0\\
3.75489387234681	0\\
3.75499387484687	0\\
3.75509387734693	0\\
3.755193879847	0\\
3.75529388234706	0\\
3.75539388484712	0\\
3.75549388734718	0\\
3.75559388984725	0\\
3.75569389234731	0\\
3.75579389484737	0\\
3.75589389734743	0\\
3.7559938998475	0\\
3.75609390234756	0\\
3.75619390484762	0\\
3.75629390734768	0\\
3.75639390984775	0\\
3.75649391234781	0\\
3.75659391484787	0\\
3.75669391734793	0\\
3.756793919848	0\\
3.75689392234806	0\\
3.75699392484812	0\\
3.75709392734818	0\\
3.75719392984825	0\\
3.75729393234831	0\\
3.75739393484837	0\\
3.75749393734843	0\\
3.7575939398485	0\\
3.75769394234856	0\\
3.75779394484862	0\\
3.75789394734868	0\\
3.75799394984875	0\\
3.75809395234881	0\\
3.75819395484887	0\\
3.75829395734893	0\\
3.758393959849	0\\
3.75849396234906	0\\
3.75859396484912	0\\
3.75869396734918	0\\
3.75879396984925	0\\
3.75889397234931	0\\
3.75899397484937	0\\
3.75909397734943	0\\
3.7591939798495	0\\
3.75929398234956	0\\
3.75939398484962	0\\
3.75949398734968	0\\
3.75959398984975	0\\
3.75969399234981	0\\
3.75979399484987	0\\
3.75989399734993	0\\
3.75999399985	0\\
3.76009400235006	0\\
3.76019400485012	0\\
3.76029400735018	0\\
3.76039400985025	0\\
3.76049401235031	0\\
3.76059401485037	0\\
3.76069401735043	0\\
3.7607940198505	0\\
3.76089402235056	0\\
3.76099402485062	0\\
3.76109402735068	0\\
3.76119402985075	0\\
3.76129403235081	0\\
3.76139403485087	0\\
3.76149403735093	0\\
3.761594039851	0\\
3.76169404235106	0\\
3.76179404485112	0\\
3.76189404735118	0\\
3.76199404985125	0\\
3.76209405235131	0\\
3.76219405485137	0\\
3.76229405735143	0\\
3.7623940598515	0\\
3.76249406235156	0\\
3.76259406485162	0\\
3.76269406735168	0\\
3.76279406985175	0\\
3.76289407235181	0\\
3.76299407485187	0\\
3.76309407735193	0\\
3.763194079852	0\\
3.76329408235206	0\\
3.76339408485212	0\\
3.76349408735218	0\\
3.76359408985225	0\\
3.76369409235231	0\\
3.76379409485237	0\\
3.76389409735243	0\\
3.7639940998525	0\\
3.76409410235256	0\\
3.76419410485262	0\\
3.76429410735268	0\\
3.76439410985275	0\\
3.76449411235281	0\\
3.76459411485287	0\\
3.76469411735293	0\\
3.764794119853	0\\
3.76489412235306	0\\
3.76499412485312	0\\
3.76509412735318	0\\
3.76519412985325	0\\
3.76529413235331	0\\
3.76539413485337	0\\
3.76549413735343	0\\
3.7655941398535	0\\
3.76569414235356	0\\
3.76579414485362	0\\
3.76589414735368	0\\
3.76599414985375	0\\
3.76609415235381	0\\
3.76619415485387	0\\
3.76629415735393	0\\
3.766394159854	0\\
3.76649416235406	0\\
3.76659416485412	0\\
3.76669416735418	0\\
3.76679416985425	0\\
3.76689417235431	0\\
3.76699417485437	0\\
3.76709417735443	0\\
3.7671941798545	0\\
3.76729418235456	0\\
3.76739418485462	0\\
3.76749418735468	0\\
3.76759418985475	0\\
3.76769419235481	0\\
3.76779419485487	0\\
3.76789419735493	0\\
3.767994199855	0\\
3.76809420235506	0\\
3.76819420485512	0\\
3.76829420735518	0\\
3.76839420985525	0\\
3.76849421235531	0\\
3.76859421485537	0\\
3.76869421735543	0\\
3.7687942198555	0\\
3.76889422235556	0\\
3.76899422485562	0\\
3.76909422735568	0\\
3.76919422985575	0\\
3.76929423235581	0\\
3.76939423485587	0\\
3.76949423735593	0\\
3.769594239856	0\\
3.76969424235606	0\\
3.76979424485612	0\\
3.76989424735618	0\\
3.76999424985625	0\\
3.77009425235631	0\\
3.77019425485637	0\\
3.77029425735643	0\\
3.7703942598565	0\\
3.77049426235656	0\\
3.77059426485662	0\\
3.77069426735668	0\\
3.77079426985675	0\\
3.77089427235681	0\\
3.77099427485687	0\\
3.77109427735693	0\\
3.771194279857	0\\
3.77129428235706	0\\
3.77139428485712	0\\
3.77149428735718	0\\
3.77159428985725	0\\
3.77169429235731	0\\
3.77179429485737	0\\
3.77189429735743	0\\
3.7719942998575	0\\
3.77209430235756	0\\
3.77219430485762	0\\
3.77229430735768	0\\
3.77239430985775	0\\
3.77249431235781	0\\
3.77259431485787	0\\
3.77269431735793	0\\
3.772794319858	0\\
3.77289432235806	0\\
3.77299432485812	0\\
3.77309432735818	0\\
3.77319432985825	0\\
3.77329433235831	0\\
3.77339433485837	0\\
3.77349433735843	0\\
3.7735943398585	0\\
3.77369434235856	0\\
3.77379434485862	0\\
3.77389434735868	0\\
3.77399434985875	0\\
3.77409435235881	0\\
3.77419435485887	0\\
3.77429435735893	0\\
3.774394359859	0\\
3.77449436235906	0\\
3.77459436485912	0\\
3.77469436735918	0\\
3.77479436985925	0\\
3.77489437235931	0\\
3.77499437485937	0\\
3.77509437735943	0\\
3.7751943798595	0\\
3.77529438235956	0\\
3.77539438485962	0\\
3.77549438735968	0\\
3.77559438985975	0\\
3.77569439235981	0\\
3.77579439485987	0\\
3.77589439735993	0\\
3.77599439986	0\\
3.77609440236006	0\\
3.77619440486012	0\\
3.77629440736018	0\\
3.77639440986025	0\\
3.77649441236031	0\\
3.77659441486037	0\\
3.77669441736043	0\\
3.7767944198605	0\\
3.77689442236056	0\\
3.77699442486062	0\\
3.77709442736068	0\\
3.77719442986075	0\\
3.77729443236081	0\\
3.77739443486087	0\\
3.77749443736093	0\\
3.777594439861	0\\
3.77769444236106	0\\
3.77779444486112	0\\
3.77789444736118	0\\
3.77799444986125	0\\
3.77809445236131	0\\
3.77819445486137	0\\
3.77829445736143	0\\
3.7783944598615	0\\
3.77849446236156	0\\
3.77859446486162	0\\
3.77869446736168	0\\
3.77879446986175	0\\
3.77889447236181	0\\
3.77899447486187	0\\
3.77909447736193	0\\
3.779194479862	0\\
3.77929448236206	0\\
3.77939448486212	0\\
3.77949448736218	0\\
3.77959448986225	0\\
3.77969449236231	0\\
3.77979449486237	0\\
3.77989449736243	0\\
3.7799944998625	0\\
3.78009450236256	0\\
3.78019450486262	0\\
3.78029450736268	0\\
3.78039450986275	0\\
3.78049451236281	0\\
3.78059451486287	0\\
3.78069451736293	0\\
3.780794519863	0\\
3.78089452236306	0\\
3.78099452486312	0\\
3.78109452736318	0\\
3.78119452986325	0\\
3.78129453236331	0\\
3.78139453486337	0\\
3.78149453736343	0\\
3.7815945398635	0\\
3.78169454236356	0\\
3.78179454486362	0\\
3.78189454736368	0\\
3.78199454986375	0\\
3.78209455236381	0\\
3.78219455486387	0\\
3.78229455736393	0\\
3.782394559864	0\\
3.78249456236406	0\\
3.78259456486412	0\\
3.78269456736418	0\\
3.78279456986425	0\\
3.78289457236431	0\\
3.78299457486437	0\\
3.78309457736443	0\\
3.7831945798645	0\\
3.78329458236456	0\\
3.78339458486462	0\\
3.78349458736468	0\\
3.78359458986475	0\\
3.78369459236481	0\\
3.78379459486487	0\\
3.78389459736493	0\\
3.783994599865	0\\
3.78409460236506	0\\
3.78419460486512	0\\
3.78429460736518	0\\
3.78439460986525	0\\
3.78449461236531	0\\
3.78459461486537	0\\
3.78469461736543	0\\
3.7847946198655	0\\
3.78489462236556	0\\
3.78499462486562	0\\
3.78509462736568	0\\
3.78519462986575	0\\
3.78529463236581	0\\
3.78539463486587	0\\
3.78549463736593	0\\
3.785594639866	0\\
3.78569464236606	0\\
3.78579464486612	0\\
3.78589464736618	0\\
3.78599464986625	0\\
3.78609465236631	0\\
3.78619465486637	0\\
3.78629465736643	0\\
3.7863946598665	0\\
3.78649466236656	0\\
3.78659466486662	0\\
3.78669466736668	0\\
3.78679466986675	0\\
3.78689467236681	0\\
3.78699467486687	0\\
3.78709467736693	0\\
3.787194679867	0\\
3.78729468236706	0\\
3.78739468486712	0\\
3.78749468736718	0\\
3.78759468986725	0\\
3.78769469236731	0\\
3.78779469486737	0\\
3.78789469736743	0\\
3.7879946998675	0\\
3.78809470236756	0\\
3.78819470486762	0\\
3.78829470736768	0\\
3.78839470986775	0\\
3.78849471236781	0\\
3.78859471486787	0\\
3.78869471736793	0\\
3.788794719868	0\\
3.78889472236806	0\\
3.78899472486812	0\\
3.78909472736818	0\\
3.78919472986825	0\\
3.78929473236831	0\\
3.78939473486837	0\\
3.78949473736843	0\\
3.7895947398685	0\\
3.78969474236856	0\\
3.78979474486862	0\\
3.78989474736868	0\\
3.78999474986875	0\\
3.79009475236881	0\\
3.79019475486887	0\\
3.79029475736893	0\\
3.790394759869	0\\
3.79049476236906	0\\
3.79059476486912	0\\
3.79069476736918	0\\
3.79079476986925	0\\
3.79089477236931	0\\
3.79099477486937	0\\
3.79109477736943	0\\
3.7911947798695	0\\
3.79129478236956	0\\
3.79139478486962	0\\
3.79149478736968	0\\
3.79159478986975	0\\
3.79169479236981	0\\
3.79179479486987	0\\
3.79189479736993	0\\
3.79199479987	0\\
3.79209480237006	0\\
3.79219480487012	0\\
3.79229480737018	0\\
3.79239480987025	0\\
3.79249481237031	0\\
3.79259481487037	0\\
3.79269481737043	0\\
3.7927948198705	0\\
3.79289482237056	0\\
3.79299482487062	0\\
3.79309482737068	0\\
3.79319482987075	0\\
3.79329483237081	0\\
3.79339483487087	0\\
3.79349483737093	0\\
3.793594839871	0\\
3.79369484237106	0\\
3.79379484487112	0\\
3.79389484737118	0\\
3.79399484987125	0\\
3.79409485237131	0\\
3.79419485487137	0\\
3.79429485737143	0\\
3.7943948598715	0\\
3.79449486237156	0\\
3.79459486487162	0\\
3.79469486737168	0\\
3.79479486987175	0\\
3.79489487237181	0\\
3.79499487487187	0\\
3.79509487737193	0\\
3.795194879872	0\\
3.79529488237206	0\\
3.79539488487212	0\\
3.79549488737218	0\\
3.79559488987225	0\\
3.79569489237231	0\\
3.79579489487237	0\\
3.79589489737243	0\\
3.7959948998725	0\\
3.79609490237256	0\\
3.79619490487262	0\\
3.79629490737268	0\\
3.79639490987275	0\\
3.79649491237281	0\\
3.79659491487287	0\\
3.79669491737293	0\\
3.796794919873	0\\
3.79689492237306	0\\
3.79699492487312	0\\
3.79709492737318	0\\
3.79719492987325	0\\
3.79729493237331	0\\
3.79739493487337	0\\
3.79749493737343	0\\
3.7975949398735	0\\
3.79769494237356	0\\
3.79779494487362	0\\
3.79789494737368	0\\
3.79799494987375	0\\
3.79809495237381	0\\
3.79819495487387	0\\
3.79829495737393	0\\
3.798394959874	0\\
3.79849496237406	0\\
3.79859496487412	0\\
3.79869496737418	0\\
3.79879496987425	0\\
3.79889497237431	0\\
3.79899497487437	0\\
3.79909497737443	0\\
3.7991949798745	0\\
3.79929498237456	0\\
3.79939498487462	0\\
3.79949498737468	0\\
3.79959498987475	0\\
3.79969499237481	0\\
3.79979499487487	0\\
3.79989499737493	0\\
3.799994999875	0\\
3.80009500237506	0\\
3.80019500487512	0\\
3.80029500737518	0\\
3.80039500987525	0\\
3.80049501237531	0\\
3.80059501487537	0\\
3.80069501737543	0\\
3.8007950198755	0\\
3.80089502237556	0\\
3.80099502487562	0\\
3.80109502737568	0\\
3.80119502987575	0\\
3.80129503237581	0\\
3.80139503487587	0\\
3.80149503737593	0\\
3.801595039876	0\\
3.80169504237606	0\\
3.80179504487612	0\\
3.80189504737618	0\\
3.80199504987625	0\\
3.80209505237631	0\\
3.80219505487637	0\\
3.80229505737643	0\\
3.8023950598765	0\\
3.80249506237656	0\\
3.80259506487662	0\\
3.80269506737668	0\\
3.80279506987675	0\\
3.80289507237681	0\\
3.80299507487687	0\\
3.80309507737693	0\\
3.803195079877	0\\
3.80329508237706	0\\
3.80339508487712	0\\
3.80349508737718	0\\
3.80359508987725	0\\
3.80369509237731	0\\
3.80379509487737	0\\
3.80389509737743	0\\
3.8039950998775	0\\
3.80409510237756	0\\
3.80419510487762	0\\
3.80429510737768	0\\
3.80439510987775	0\\
3.80449511237781	0\\
3.80459511487787	0\\
3.80469511737793	0\\
3.804795119878	0\\
3.80489512237806	0\\
3.80499512487812	0\\
3.80509512737818	0\\
3.80519512987825	0\\
3.80529513237831	0\\
3.80539513487837	0\\
3.80549513737843	0\\
3.8055951398785	0\\
3.80569514237856	0\\
3.80579514487862	0\\
3.80589514737868	0\\
3.80599514987875	0\\
3.80609515237881	0\\
3.80619515487887	0\\
3.80629515737893	0\\
3.806395159879	0\\
3.80649516237906	0\\
3.80659516487912	0\\
3.80669516737918	0\\
3.80679516987925	0\\
3.80689517237931	0\\
3.80699517487937	0\\
3.80709517737943	0\\
3.8071951798795	0\\
3.80729518237956	0\\
3.80739518487962	0\\
3.80749518737968	0\\
3.80759518987975	0\\
3.80769519237981	0\\
3.80779519487987	0\\
3.80789519737993	0\\
3.80799519988	0\\
3.80809520238006	0\\
3.80819520488012	0\\
3.80829520738018	0\\
3.80839520988025	0\\
3.80849521238031	0\\
3.80859521488037	0\\
3.80869521738043	0\\
3.8087952198805	0\\
3.80889522238056	0\\
3.80899522488062	0\\
3.80909522738068	0\\
3.80919522988075	0\\
3.80929523238081	0\\
3.80939523488087	0\\
3.80949523738093	0\\
3.809595239881	0\\
3.80969524238106	0\\
3.80979524488112	0\\
3.80989524738118	0\\
3.80999524988125	0\\
3.81009525238131	0\\
3.81019525488137	0\\
3.81029525738143	0\\
3.8103952598815	0\\
3.81049526238156	0\\
3.81059526488162	0\\
3.81069526738168	0\\
3.81079526988175	0\\
3.81089527238181	0\\
3.81099527488187	0\\
3.81109527738193	0\\
3.811195279882	0\\
3.81129528238206	0\\
3.81139528488212	0\\
3.81149528738218	0\\
3.81159528988225	0\\
3.81169529238231	0\\
3.81179529488237	0\\
3.81189529738243	0\\
3.8119952998825	0\\
3.81209530238256	0\\
3.81219530488262	0\\
3.81229530738268	0\\
3.81239530988275	0\\
3.81249531238281	0\\
3.81259531488287	0\\
3.81269531738293	0\\
3.812795319883	0\\
3.81289532238306	0\\
3.81299532488312	0\\
3.81309532738318	0\\
3.81319532988325	0\\
3.81329533238331	0\\
3.81339533488337	0\\
3.81349533738343	0\\
3.8135953398835	0\\
3.81369534238356	0\\
3.81379534488362	0\\
3.81389534738368	0\\
3.81399534988375	0\\
3.81409535238381	0\\
3.81419535488387	0\\
3.81429535738393	0\\
3.814395359884	0\\
3.81449536238406	0\\
3.81459536488412	0\\
3.81469536738418	0\\
3.81479536988425	0\\
3.81489537238431	0\\
3.81499537488437	0\\
3.81509537738443	0\\
3.8151953798845	0\\
3.81529538238456	0\\
3.81539538488462	0\\
3.81549538738468	0\\
3.81559538988475	0\\
3.81569539238481	0\\
3.81579539488487	0\\
3.81589539738493	0\\
3.815995399885	0\\
3.81609540238506	0\\
3.81619540488512	0\\
3.81629540738518	0\\
3.81639540988525	0\\
3.81649541238531	0\\
3.81659541488537	0\\
3.81669541738543	0\\
3.8167954198855	0\\
3.81689542238556	0\\
3.81699542488562	0\\
3.81709542738568	0\\
3.81719542988575	0\\
3.81729543238581	0\\
3.81739543488587	0\\
3.81749543738593	0\\
3.817595439886	0\\
3.81769544238606	0\\
3.81779544488612	0\\
3.81789544738618	0\\
3.81799544988625	0\\
3.81809545238631	0\\
3.81819545488637	0\\
3.81829545738643	0\\
3.8183954598865	0\\
3.81849546238656	0\\
3.81859546488662	0\\
3.81869546738668	0\\
3.81879546988675	0\\
3.81889547238681	0\\
3.81899547488687	0\\
3.81909547738693	0\\
3.819195479887	0\\
3.81929548238706	0\\
3.81939548488712	0\\
3.81949548738718	0\\
3.81959548988725	0\\
3.81969549238731	0\\
3.81979549488737	0\\
3.81989549738743	0\\
3.8199954998875	0\\
3.82009550238756	0\\
3.82019550488762	0\\
3.82029550738768	0\\
3.82039550988775	0\\
3.82049551238781	0\\
3.82059551488787	0\\
3.82069551738793	0\\
3.820795519888	0\\
3.82089552238806	0\\
3.82099552488812	0\\
3.82109552738818	0\\
3.82119552988825	0\\
3.82129553238831	0\\
3.82139553488837	0\\
3.82149553738843	0\\
3.8215955398885	0\\
3.82169554238856	0\\
3.82179554488862	0\\
3.82189554738868	0\\
3.82199554988875	0\\
3.82209555238881	0\\
3.82219555488887	0\\
3.82229555738893	0\\
3.822395559889	0\\
3.82249556238906	0\\
3.82259556488912	0\\
3.82269556738918	0\\
3.82279556988925	0\\
3.82289557238931	0\\
3.82299557488937	0\\
3.82309557738943	0\\
3.8231955798895	0\\
3.82329558238956	0\\
3.82339558488962	0\\
3.82349558738968	0\\
3.82359558988975	0\\
3.82369559238981	0\\
3.82379559488987	0\\
3.82389559738993	0\\
3.82399559989	0\\
3.82409560239006	0\\
3.82419560489012	0\\
3.82429560739018	0\\
3.82439560989025	0\\
3.82449561239031	0\\
3.82459561489037	0\\
3.82469561739043	0\\
3.8247956198905	0\\
3.82489562239056	0\\
3.82499562489062	0\\
3.82509562739068	0\\
3.82519562989075	0\\
3.82529563239081	0\\
3.82539563489087	0\\
3.82549563739093	0\\
3.825595639891	0\\
3.82569564239106	0\\
3.82579564489112	0\\
3.82589564739118	0\\
3.82599564989125	0\\
3.82609565239131	0\\
3.82619565489137	0\\
3.82629565739143	0\\
3.8263956598915	0\\
3.82649566239156	0\\
3.82659566489162	0\\
3.82669566739168	0\\
3.82679566989175	0\\
3.82689567239181	0\\
3.82699567489187	0\\
3.82709567739193	0\\
3.827195679892	0\\
3.82729568239206	0\\
3.82739568489212	0\\
3.82749568739218	0\\
3.82759568989225	0\\
3.82769569239231	0\\
3.82779569489237	0\\
3.82789569739243	0\\
3.8279956998925	0\\
3.82809570239256	0\\
3.82819570489262	0\\
3.82829570739268	0\\
3.82839570989275	0\\
3.82849571239281	0\\
3.82859571489287	0\\
3.82869571739294	0\\
3.828795719893	0\\
3.82889572239306	0\\
3.82899572489312	0\\
3.82909572739318	0\\
3.82919572989325	0\\
3.82929573239331	0\\
3.82939573489337	0\\
3.82949573739343	0\\
3.8295957398935	0\\
3.82969574239356	0\\
3.82979574489362	0\\
3.82989574739368	0\\
3.82999574989375	0\\
3.83009575239381	0\\
3.83019575489387	0\\
3.83029575739393	0\\
3.830395759894	0\\
3.83049576239406	0\\
3.83059576489412	0\\
3.83069576739419	0\\
3.83079576989425	0\\
3.83089577239431	0\\
3.83099577489437	0\\
3.83109577739443	0\\
3.8311957798945	0\\
3.83129578239456	0\\
3.83139578489462	0\\
3.83149578739468	0\\
3.83159578989475	0\\
3.83169579239481	0\\
3.83179579489487	0\\
3.83189579739493	0\\
3.831995799895	0\\
3.83209580239506	0\\
3.83219580489512	0\\
3.83229580739518	0\\
3.83239580989525	0\\
3.83249581239531	0\\
3.83259581489537	0\\
3.83269581739544	0\\
3.8327958198955	0\\
3.83289582239556	0\\
3.83299582489562	0\\
3.83309582739568	0\\
3.83319582989575	0\\
3.83329583239581	0\\
3.83339583489587	0\\
3.83349583739593	0\\
3.833595839896	0\\
3.83369584239606	0\\
3.83379584489612	0\\
3.83389584739618	0\\
3.83399584989625	0\\
3.83409585239631	0\\
3.83419585489637	0\\
3.83429585739643	0\\
3.8343958598965	0\\
3.83449586239656	0\\
3.83459586489662	0\\
3.83469586739669	0\\
3.83479586989675	0\\
3.83489587239681	0\\
3.83499587489687	0\\
3.83509587739694	0\\
3.835195879897	0\\
3.83529588239706	0\\
3.83539588489712	0\\
3.83549588739718	0\\
3.83559588989725	0\\
3.83569589239731	0\\
3.83579589489737	0\\
3.83589589739743	0\\
3.8359958998975	0\\
3.83609590239756	0\\
3.83619590489762	0\\
3.83629590739768	0\\
3.83639590989775	0\\
3.83649591239781	0\\
3.83659591489787	0\\
3.83669591739794	0\\
3.836795919898	0\\
3.83689592239806	0\\
3.83699592489812	0\\
3.83709592739819	0\\
3.83719592989825	0\\
3.83729593239831	0\\
3.83739593489837	0\\
3.83749593739843	0\\
3.8375959398985	0\\
3.83769594239856	0\\
3.83779594489862	0\\
3.83789594739868	0\\
3.83799594989875	0\\
3.83809595239881	0\\
3.83819595489887	0\\
3.83829595739893	0\\
3.838395959899	0\\
3.83849596239906	0\\
3.83859596489912	0\\
3.83869596739919	0\\
3.83879596989925	0\\
3.83889597239931	0\\
3.83899597489937	0\\
3.83909597739944	0\\
3.8391959798995	0\\
3.83929598239956	0\\
3.83939598489962	0\\
3.83949598739969	0\\
3.83959598989975	0\\
3.83969599239981	0\\
3.83979599489987	0\\
3.83989599739993	0\\
3.8399959999	0\\
3.84009600240006	0\\
3.84019600490012	0\\
3.84029600740018	0\\
3.84039600990025	0\\
3.84049601240031	0\\
3.84059601490037	0\\
3.84069601740044	0\\
3.8407960199005	0\\
3.84089602240056	0\\
3.84099602490062	0\\
3.84109602740069	0\\
3.84119602990075	0\\
3.84129603240081	0\\
3.84139603490087	0\\
3.84149603740094	0\\
3.841596039901	0\\
3.84169604240106	0\\
3.84179604490112	0\\
3.84189604740118	0\\
3.84199604990125	0\\
3.84209605240131	0\\
3.84219605490137	0\\
3.84229605740143	0\\
3.8423960599015	0\\
3.84249606240156	0\\
3.84259606490162	0\\
3.84269606740169	0\\
3.84279606990175	0\\
3.84289607240181	0\\
3.84299607490187	0\\
3.84309607740194	0\\
3.843196079902	0\\
3.84329608240206	0\\
3.84339608490212	0\\
3.84349608740219	0\\
3.84359608990225	0\\
3.84369609240231	0\\
3.84379609490237	0\\
3.84389609740243	0\\
3.8439960999025	0\\
3.84409610240256	0\\
3.84419610490262	0\\
3.84429610740268	0\\
3.84439610990275	0\\
3.84449611240281	0\\
3.84459611490287	0\\
3.84469611740294	0\\
3.844796119903	0\\
3.84489612240306	0\\
3.84499612490312	0\\
3.84509612740319	0\\
3.84519612990325	0\\
3.84529613240331	0\\
3.84539613490337	0\\
3.84549613740344	0\\
3.8455961399035	0\\
3.84569614240356	0\\
3.84579614490362	0\\
3.84589614740369	0\\
3.84599614990375	0\\
3.84609615240381	0\\
3.84619615490387	0\\
3.84629615740393	0\\
3.846396159904	0\\
3.84649616240406	0\\
3.84659616490412	0\\
3.84669616740419	0\\
3.84679616990425	0\\
3.84689617240431	0\\
3.84699617490437	0\\
3.84709617740444	0\\
3.8471961799045	0\\
3.84729618240456	0\\
3.84739618490462	0\\
3.84749618740469	0\\
3.84759618990475	0\\
3.84769619240481	0\\
3.84779619490487	0\\
3.84789619740494	0\\
3.847996199905	0\\
3.84809620240506	0\\
3.84819620490512	0\\
3.84829620740518	0\\
3.84839620990525	0\\
3.84849621240531	0\\
3.84859621490537	0\\
3.84869621740544	0\\
3.8487962199055	0\\
3.84889622240556	0\\
3.84899622490562	0\\
3.84909622740569	0\\
3.84919622990575	0\\
3.84929623240581	0\\
3.84939623490587	0\\
3.84949623740594	0\\
3.849596239906	0\\
3.84969624240606	0\\
3.84979624490612	0\\
3.84989624740619	0\\
3.84999624990625	0\\
3.85009625240631	0\\
3.85019625490637	0\\
3.85029625740643	0\\
3.8503962599065	0\\
3.85049626240656	0\\
3.85059626490662	0\\
3.85069626740669	0\\
3.85079626990675	0\\
3.85089627240681	0\\
3.85099627490687	0\\
3.85109627740694	0\\
3.851196279907	0\\
3.85129628240706	0\\
3.85139628490712	0\\
3.85149628740719	0\\
3.85159628990725	0\\
3.85169629240731	0\\
3.85179629490737	0\\
3.85189629740744	0\\
3.8519962999075	0\\
3.85209630240756	0\\
3.85219630490762	0\\
3.85229630740769	0\\
3.85239630990775	0\\
3.85249631240781	0\\
3.85259631490787	0\\
3.85269631740794	0\\
3.852796319908	0\\
3.85289632240806	0\\
3.85299632490812	0\\
3.85309632740819	0\\
3.85319632990825	0\\
3.85329633240831	0\\
3.85339633490837	0\\
3.85349633740844	0\\
3.8535963399085	0\\
3.85369634240856	0\\
3.85379634490862	0\\
3.85389634740869	0\\
3.85399634990875	0\\
3.85409635240881	0\\
3.85419635490887	0\\
3.85429635740894	0\\
3.854396359909	0\\
3.85449636240906	0\\
3.85459636490912	0\\
3.85469636740919	0\\
3.85479636990925	0\\
3.85489637240931	0\\
3.85499637490937	0\\
3.85509637740944	0\\
3.8551963799095	0\\
3.85529638240956	0\\
3.85539638490962	0\\
3.85549638740969	0\\
3.85559638990975	0\\
3.85569639240981	0\\
3.85579639490987	0\\
3.85589639740994	0\\
3.85599639991	0\\
3.85609640241006	0\\
3.85619640491012	0\\
3.85629640741019	0\\
3.85639640991025	0\\
3.85649641241031	0\\
3.85659641491037	0\\
3.85669641741044	0\\
3.8567964199105	0\\
3.85689642241056	0\\
3.85699642491062	0\\
3.85709642741069	0\\
3.85719642991075	0\\
3.85729643241081	0\\
3.85739643491087	0\\
3.85749643741094	0\\
3.857596439911	0\\
3.85769644241106	0\\
3.85779644491112	0\\
3.85789644741119	0\\
3.85799644991125	0\\
3.85809645241131	0\\
3.85819645491137	0\\
3.85829645741144	0\\
3.8583964599115	0\\
3.85849646241156	0\\
3.85859646491162	0\\
3.85869646741169	0\\
3.85879646991175	0\\
3.85889647241181	0\\
3.85899647491187	0\\
3.85909647741194	0\\
3.859196479912	0\\
3.85929648241206	0\\
3.85939648491212	0\\
3.85949648741219	0\\
3.85959648991225	0\\
3.85969649241231	0\\
3.85979649491237	0\\
3.85989649741244	0\\
3.8599964999125	0\\
3.86009650241256	0\\
3.86019650491262	0\\
3.86029650741269	0\\
3.86039650991275	0\\
3.86049651241281	0\\
3.86059651491287	0\\
3.86069651741294	0\\
3.860796519913	0\\
3.86089652241306	0\\
3.86099652491312	0\\
3.86109652741319	0\\
3.86119652991325	0\\
3.86129653241331	0\\
3.86139653491337	0\\
3.86149653741344	0\\
3.8615965399135	0\\
3.86169654241356	0\\
3.86179654491362	0\\
3.86189654741369	0\\
3.86199654991375	0\\
3.86209655241381	0\\
3.86219655491387	0\\
3.86229655741394	0\\
3.862396559914	0\\
3.86249656241406	0\\
3.86259656491412	0\\
3.86269656741419	0\\
3.86279656991425	0\\
3.86289657241431	0\\
3.86299657491437	0\\
3.86309657741444	0\\
3.8631965799145	0\\
3.86329658241456	0\\
3.86339658491462	0\\
3.86349658741469	0\\
3.86359658991475	0\\
3.86369659241481	0\\
3.86379659491487	0\\
3.86389659741494	0\\
3.863996599915	0\\
3.86409660241506	0\\
3.86419660491512	0\\
3.86429660741519	0\\
3.86439660991525	0\\
3.86449661241531	0\\
3.86459661491537	0\\
3.86469661741544	0\\
3.8647966199155	0\\
3.86489662241556	0\\
3.86499662491562	0\\
3.86509662741569	0\\
3.86519662991575	0\\
3.86529663241581	0\\
3.86539663491587	0\\
3.86549663741594	0\\
3.865596639916	0\\
3.86569664241606	0\\
3.86579664491612	0\\
3.86589664741619	0\\
3.86599664991625	0\\
3.86609665241631	0\\
3.86619665491637	0\\
3.86629665741644	0\\
3.8663966599165	0\\
3.86649666241656	0\\
3.86659666491662	0\\
3.86669666741669	0\\
3.86679666991675	0\\
3.86689667241681	0\\
3.86699667491687	0\\
3.86709667741694	0\\
3.867196679917	0\\
3.86729668241706	0\\
3.86739668491712	0\\
3.86749668741719	0\\
3.86759668991725	0\\
3.86769669241731	0\\
3.86779669491737	0\\
3.86789669741744	0\\
3.8679966999175	0\\
3.86809670241756	0\\
3.86819670491762	0\\
3.86829670741769	0\\
3.86839670991775	0\\
3.86849671241781	0\\
3.86859671491787	0\\
3.86869671741794	0\\
3.868796719918	0\\
3.86889672241806	0\\
3.86899672491812	0\\
3.86909672741819	0\\
3.86919672991825	0\\
3.86929673241831	0\\
3.86939673491837	0\\
3.86949673741844	0\\
3.8695967399185	0\\
3.86969674241856	0\\
3.86979674491862	0\\
3.86989674741869	0\\
3.86999674991875	0\\
3.87009675241881	0\\
3.87019675491887	0\\
3.87029675741894	0\\
3.870396759919	0\\
3.87049676241906	0\\
3.87059676491912	0\\
3.87069676741919	0\\
3.87079676991925	0\\
3.87089677241931	0\\
3.87099677491937	0\\
3.87109677741944	0\\
3.8711967799195	0\\
3.87129678241956	0\\
3.87139678491962	0\\
3.87149678741969	0\\
3.87159678991975	0\\
3.87169679241981	0\\
3.87179679491987	0\\
3.87189679741994	0\\
3.87199679992	0\\
3.87209680242006	0\\
3.87219680492012	0\\
3.87229680742019	0\\
3.87239680992025	0\\
3.87249681242031	0\\
3.87259681492037	0\\
3.87269681742044	0\\
3.8727968199205	0\\
3.87289682242056	0\\
3.87299682492062	0\\
3.87309682742069	0\\
3.87319682992075	0\\
3.87329683242081	0\\
3.87339683492087	0\\
3.87349683742094	0\\
3.873596839921	0\\
3.87369684242106	0\\
3.87379684492112	0\\
3.87389684742119	0\\
3.87399684992125	0\\
3.87409685242131	0\\
3.87419685492137	0\\
3.87429685742144	0\\
3.8743968599215	0\\
3.87449686242156	0\\
3.87459686492162	0\\
3.87469686742169	0\\
3.87479686992175	0\\
3.87489687242181	0\\
3.87499687492187	0\\
3.87509687742194	0\\
3.875196879922	0\\
3.87529688242206	0\\
3.87539688492212	0\\
3.87549688742219	0\\
3.87559688992225	0\\
3.87569689242231	0\\
3.87579689492237	0\\
3.87589689742244	0\\
3.8759968999225	0\\
3.87609690242256	0\\
3.87619690492262	0\\
3.87629690742269	0\\
3.87639690992275	0\\
3.87649691242281	0\\
3.87659691492287	0\\
3.87669691742294	0\\
3.876796919923	0\\
3.87689692242306	0\\
3.87699692492312	0\\
3.87709692742319	0\\
3.87719692992325	0\\
3.87729693242331	0\\
3.87739693492337	0\\
3.87749693742344	0\\
3.8775969399235	0\\
3.87769694242356	0\\
3.87779694492362	0\\
3.87789694742369	0\\
3.87799694992375	0\\
3.87809695242381	0\\
3.87819695492387	0\\
3.87829695742394	0\\
3.878396959924	0\\
3.87849696242406	0\\
3.87859696492412	0\\
3.87869696742419	0\\
3.87879696992425	0\\
3.87889697242431	0\\
3.87899697492437	0\\
3.87909697742444	0\\
3.8791969799245	0\\
3.87929698242456	0\\
3.87939698492462	0\\
3.87949698742469	0\\
3.87959698992475	0\\
3.87969699242481	0\\
3.87979699492487	0\\
3.87989699742494	0\\
3.879996999925	0\\
3.88009700242506	0\\
3.88019700492512	0\\
3.88029700742519	0\\
3.88039700992525	0\\
3.88049701242531	0\\
3.88059701492537	0\\
3.88069701742544	0\\
3.8807970199255	0\\
3.88089702242556	0\\
3.88099702492562	0\\
3.88109702742569	0\\
3.88119702992575	0\\
3.88129703242581	0\\
3.88139703492587	0\\
3.88149703742594	0\\
3.881597039926	0\\
3.88169704242606	0\\
3.88179704492612	0\\
3.88189704742619	0\\
3.88199704992625	0\\
3.88209705242631	0\\
3.88219705492637	0\\
3.88229705742644	0\\
3.8823970599265	0\\
3.88249706242656	0\\
3.88259706492662	0\\
3.88269706742669	0\\
3.88279706992675	0\\
3.88289707242681	0\\
3.88299707492687	0\\
3.88309707742694	0\\
3.883197079927	0\\
3.88329708242706	0\\
3.88339708492712	0\\
3.88349708742719	0\\
3.88359708992725	0\\
3.88369709242731	0\\
3.88379709492737	0\\
3.88389709742744	0\\
3.8839970999275	0\\
3.88409710242756	0\\
3.88419710492762	0\\
3.88429710742769	0\\
3.88439710992775	0\\
3.88449711242781	0\\
3.88459711492787	0\\
3.88469711742794	0\\
3.884797119928	0\\
3.88489712242806	0\\
3.88499712492812	0\\
3.88509712742819	0\\
3.88519712992825	0\\
3.88529713242831	0\\
3.88539713492837	0\\
3.88549713742844	0\\
3.8855971399285	0\\
3.88569714242856	0\\
3.88579714492862	0\\
3.88589714742869	0\\
3.88599714992875	0\\
3.88609715242881	0\\
3.88619715492887	0\\
3.88629715742894	0\\
3.886397159929	0\\
3.88649716242906	0\\
3.88659716492912	0\\
3.88669716742919	0\\
3.88679716992925	0\\
3.88689717242931	0\\
3.88699717492937	0\\
3.88709717742944	0\\
3.8871971799295	0\\
3.88729718242956	0\\
3.88739718492962	0\\
3.88749718742969	0\\
3.88759718992975	0\\
3.88769719242981	0\\
3.88779719492987	0\\
3.88789719742994	0\\
3.88799719993	0\\
3.88809720243006	0\\
3.88819720493012	0\\
3.88829720743019	0\\
3.88839720993025	0\\
3.88849721243031	0\\
3.88859721493037	0\\
3.88869721743044	0\\
3.8887972199305	0\\
3.88889722243056	0\\
3.88899722493062	0\\
3.88909722743069	0\\
3.88919722993075	0\\
3.88929723243081	0\\
3.88939723493087	0\\
3.88949723743094	0\\
3.889597239931	0\\
3.88969724243106	0\\
3.88979724493112	0\\
3.88989724743119	0\\
3.88999724993125	0\\
3.89009725243131	0\\
3.89019725493137	0\\
3.89029725743144	0\\
3.8903972599315	0\\
3.89049726243156	0\\
3.89059726493162	0\\
3.89069726743169	0\\
3.89079726993175	0\\
3.89089727243181	0\\
3.89099727493187	0\\
3.89109727743194	0\\
3.891197279932	0\\
3.89129728243206	0\\
3.89139728493212	0\\
3.89149728743219	0\\
3.89159728993225	0\\
3.89169729243231	0\\
3.89179729493237	0\\
3.89189729743244	0\\
3.8919972999325	0\\
3.89209730243256	0\\
3.89219730493262	0\\
3.89229730743269	0\\
3.89239730993275	0\\
3.89249731243281	0\\
3.89259731493287	0\\
3.89269731743294	0\\
3.892797319933	0\\
3.89289732243306	0\\
3.89299732493312	0\\
3.89309732743319	0\\
3.89319732993325	0\\
3.89329733243331	0\\
3.89339733493337	0\\
3.89349733743344	0\\
3.8935973399335	0\\
3.89369734243356	0\\
3.89379734493362	0\\
3.89389734743369	0\\
3.89399734993375	0\\
3.89409735243381	0\\
3.89419735493387	0\\
3.89429735743394	0\\
3.894397359934	0\\
3.89449736243406	0\\
3.89459736493412	0\\
3.89469736743419	0\\
3.89479736993425	0\\
3.89489737243431	0\\
3.89499737493437	0\\
3.89509737743444	0\\
3.8951973799345	0\\
3.89529738243456	0\\
3.89539738493462	0\\
3.89549738743469	0\\
3.89559738993475	0\\
3.89569739243481	0\\
3.89579739493487	0\\
3.89589739743494	0\\
3.895997399935	0\\
3.89609740243506	0\\
3.89619740493512	0\\
3.89629740743519	0\\
3.89639740993525	0\\
3.89649741243531	0\\
3.89659741493537	0\\
3.89669741743544	0\\
3.8967974199355	0\\
3.89689742243556	0\\
3.89699742493562	0\\
3.89709742743569	0\\
3.89719742993575	0\\
3.89729743243581	0\\
3.89739743493587	0\\
3.89749743743594	0\\
3.897597439936	0\\
3.89769744243606	0\\
3.89779744493612	0\\
3.89789744743619	0\\
3.89799744993625	0\\
3.89809745243631	0\\
3.89819745493637	0\\
3.89829745743644	0\\
3.8983974599365	0\\
3.89849746243656	0\\
3.89859746493662	0\\
3.89869746743669	0\\
3.89879746993675	0\\
3.89889747243681	0\\
3.89899747493687	0\\
3.89909747743694	0\\
3.899197479937	0\\
3.89929748243706	0\\
3.89939748493712	0\\
3.89949748743719	0\\
3.89959748993725	0\\
3.89969749243731	0\\
3.89979749493737	0\\
3.89989749743744	0\\
3.8999974999375	0\\
3.90009750243756	0\\
3.90019750493762	0\\
3.90029750743769	0\\
3.90039750993775	0\\
3.90049751243781	0\\
3.90059751493787	0\\
3.90069751743794	0\\
3.900797519938	0\\
3.90089752243806	0\\
3.90099752493812	0\\
3.90109752743819	0\\
3.90119752993825	0\\
3.90129753243831	0\\
3.90139753493837	0\\
3.90149753743844	0\\
3.9015975399385	0\\
3.90169754243856	0\\
3.90179754493862	0\\
3.90189754743869	0\\
3.90199754993875	0\\
3.90209755243881	0\\
3.90219755493887	0\\
3.90229755743894	0\\
3.902397559939	0\\
3.90249756243906	0\\
3.90259756493912	0\\
3.90269756743919	0\\
3.90279756993925	0\\
3.90289757243931	0\\
3.90299757493937	0\\
3.90309757743944	0\\
3.9031975799395	0\\
3.90329758243956	0\\
3.90339758493962	0\\
3.90349758743969	0\\
3.90359758993975	0\\
3.90369759243981	0\\
3.90379759493987	0\\
3.90389759743994	0\\
3.90399759994	0\\
3.90409760244006	0\\
3.90419760494012	0\\
3.90429760744019	0\\
3.90439760994025	0\\
3.90449761244031	0\\
3.90459761494037	0\\
3.90469761744044	0\\
3.9047976199405	0\\
3.90489762244056	0\\
3.90499762494062	0\\
3.90509762744069	0\\
3.90519762994075	0\\
3.90529763244081	0\\
3.90539763494087	0\\
3.90549763744094	0\\
3.905597639941	0\\
3.90569764244106	0\\
3.90579764494112	0\\
3.90589764744119	0\\
3.90599764994125	0\\
3.90609765244131	0\\
3.90619765494137	0\\
3.90629765744144	0\\
3.9063976599415	0\\
3.90649766244156	0\\
3.90659766494162	0\\
3.90669766744169	0\\
3.90679766994175	0\\
3.90689767244181	0\\
3.90699767494187	0\\
3.90709767744194	0\\
3.907197679942	0\\
3.90729768244206	0\\
3.90739768494212	0\\
3.90749768744219	0\\
3.90759768994225	0\\
3.90769769244231	0\\
3.90779769494237	0\\
3.90789769744244	0\\
3.9079976999425	0\\
3.90809770244256	0\\
3.90819770494262	0\\
3.90829770744269	0\\
3.90839770994275	0\\
3.90849771244281	0\\
3.90859771494287	0\\
3.90869771744294	0\\
3.908797719943	0\\
3.90889772244306	0\\
3.90899772494312	0\\
3.90909772744319	0\\
3.90919772994325	0\\
3.90929773244331	0\\
3.90939773494337	0\\
3.90949773744344	0\\
3.9095977399435	0\\
3.90969774244356	0\\
3.90979774494362	0\\
3.90989774744369	0\\
3.90999774994375	0\\
3.91009775244381	0\\
3.91019775494387	0\\
3.91029775744394	0\\
3.910397759944	0\\
3.91049776244406	0\\
3.91059776494412	0\\
3.91069776744419	0\\
3.91079776994425	0\\
3.91089777244431	0\\
3.91099777494437	0\\
3.91109777744444	0\\
3.9111977799445	0\\
3.91129778244456	0\\
3.91139778494462	0\\
3.91149778744469	0\\
3.91159778994475	0\\
3.91169779244481	0\\
3.91179779494487	0\\
3.91189779744494	0\\
3.911997799945	0\\
3.91209780244506	0\\
3.91219780494512	0\\
3.91229780744519	0\\
3.91239780994525	0\\
3.91249781244531	0\\
3.91259781494537	0\\
3.91269781744544	0\\
3.9127978199455	0\\
3.91289782244556	0\\
3.91299782494562	0\\
3.91309782744569	0\\
3.91319782994575	0\\
3.91329783244581	0\\
3.91339783494587	0\\
3.91349783744594	0\\
3.913597839946	0\\
3.91369784244606	0\\
3.91379784494612	0\\
3.91389784744619	0\\
3.91399784994625	0\\
3.91409785244631	0\\
3.91419785494637	0\\
3.91429785744644	0\\
3.9143978599465	0\\
3.91449786244656	0\\
3.91459786494662	0\\
3.91469786744669	0\\
3.91479786994675	0\\
3.91489787244681	0\\
3.91499787494687	0\\
3.91509787744694	0\\
3.915197879947	0\\
3.91529788244706	0\\
3.91539788494712	0\\
3.91549788744719	0\\
3.91559788994725	0\\
3.91569789244731	0\\
3.91579789494737	0\\
3.91589789744744	0\\
3.9159978999475	0\\
3.91609790244756	0\\
3.91619790494762	0\\
3.91629790744769	0\\
3.91639790994775	0\\
3.91649791244781	0\\
3.91659791494787	0\\
3.91669791744794	0\\
3.916797919948	0\\
3.91689792244806	0\\
3.91699792494812	0\\
3.91709792744819	0\\
3.91719792994825	0\\
3.91729793244831	0\\
3.91739793494837	0\\
3.91749793744844	0\\
3.9175979399485	0\\
3.91769794244856	0\\
3.91779794494862	0\\
3.91789794744869	0\\
3.91799794994875	0\\
3.91809795244881	0\\
3.91819795494887	0\\
3.91829795744894	0\\
3.918397959949	0\\
3.91849796244906	0\\
3.91859796494912	0\\
3.91869796744919	0\\
3.91879796994925	0\\
3.91889797244931	0\\
3.91899797494937	0\\
3.91909797744944	0\\
3.9191979799495	0\\
3.91929798244956	0\\
3.91939798494962	0\\
3.91949798744969	0\\
3.91959798994975	0\\
3.91969799244981	0\\
3.91979799494987	0\\
3.91989799744994	0\\
3.91999799995	0\\
3.92009800245006	0\\
3.92019800495012	0\\
3.92029800745019	0\\
3.92039800995025	0\\
3.92049801245031	0\\
3.92059801495037	0\\
3.92069801745044	0\\
3.9207980199505	0\\
3.92089802245056	0\\
3.92099802495062	0\\
3.92109802745069	0\\
3.92119802995075	0\\
3.92129803245081	0\\
3.92139803495087	0\\
3.92149803745094	0\\
3.921598039951	0\\
3.92169804245106	0\\
3.92179804495112	0\\
3.92189804745119	0\\
3.92199804995125	0\\
3.92209805245131	0\\
3.92219805495137	0\\
3.92229805745144	0\\
3.9223980599515	0\\
3.92249806245156	0\\
3.92259806495162	0\\
3.92269806745169	0\\
3.92279806995175	0\\
3.92289807245181	0\\
3.92299807495187	0\\
3.92309807745194	0\\
3.923198079952	0\\
3.92329808245206	0\\
3.92339808495212	0\\
3.92349808745219	0\\
3.92359808995225	0\\
3.92369809245231	0\\
3.92379809495237	0\\
3.92389809745244	0\\
3.9239980999525	0\\
3.92409810245256	0\\
3.92419810495262	0\\
3.92429810745269	0\\
3.92439810995275	0\\
3.92449811245281	0\\
3.92459811495287	0\\
3.92469811745294	0\\
3.924798119953	0\\
3.92489812245306	0\\
3.92499812495312	0\\
3.92509812745319	0\\
3.92519812995325	0\\
3.92529813245331	0\\
3.92539813495337	0\\
3.92549813745344	0\\
3.9255981399535	0\\
3.92569814245356	0\\
3.92579814495362	0\\
3.92589814745369	0\\
3.92599814995375	0\\
3.92609815245381	0\\
3.92619815495387	0\\
3.92629815745394	0\\
3.926398159954	0\\
3.92649816245406	0\\
3.92659816495412	0\\
3.92669816745419	0\\
3.92679816995425	0\\
3.92689817245431	0\\
3.92699817495437	0\\
3.92709817745444	0\\
3.9271981799545	0\\
3.92729818245456	0\\
3.92739818495462	0\\
3.92749818745469	0\\
3.92759818995475	0\\
3.92769819245481	0\\
3.92779819495487	0\\
3.92789819745494	0\\
3.927998199955	0\\
3.92809820245506	0\\
3.92819820495512	0\\
3.92829820745519	0\\
3.92839820995525	0\\
3.92849821245531	0\\
3.92859821495537	0\\
3.92869821745544	0\\
3.9287982199555	0\\
3.92889822245556	0\\
3.92899822495562	0\\
3.92909822745569	0\\
3.92919822995575	0\\
3.92929823245581	0\\
3.92939823495587	0\\
3.92949823745594	0\\
3.929598239956	0\\
3.92969824245606	0\\
3.92979824495612	0\\
3.92989824745619	0\\
3.92999824995625	0\\
3.93009825245631	0\\
3.93019825495637	0\\
3.93029825745644	0\\
3.9303982599565	0\\
3.93049826245656	0\\
3.93059826495662	0\\
3.93069826745669	0\\
3.93079826995675	0\\
3.93089827245681	0\\
3.93099827495687	0\\
3.93109827745694	0\\
3.931198279957	0\\
3.93129828245706	0\\
3.93139828495712	0\\
3.93149828745719	0\\
3.93159828995725	0\\
3.93169829245731	0\\
3.93179829495737	0\\
3.93189829745744	0\\
3.9319982999575	0\\
3.93209830245756	0\\
3.93219830495762	0\\
3.93229830745769	0\\
3.93239830995775	0\\
3.93249831245781	0\\
3.93259831495787	0\\
3.93269831745794	0\\
3.932798319958	0\\
3.93289832245806	0\\
3.93299832495812	0\\
3.93309832745819	0\\
3.93319832995825	0\\
3.93329833245831	0\\
3.93339833495837	0\\
3.93349833745844	0\\
3.9335983399585	0\\
3.93369834245856	0\\
3.93379834495862	0\\
3.93389834745869	0\\
3.93399834995875	0\\
3.93409835245881	0\\
3.93419835495887	0\\
3.93429835745894	0\\
3.934398359959	0\\
3.93449836245906	0\\
3.93459836495912	0\\
3.93469836745919	0\\
3.93479836995925	0\\
3.93489837245931	0\\
3.93499837495937	0\\
3.93509837745944	0\\
3.9351983799595	0\\
3.93529838245956	0\\
3.93539838495962	0\\
3.93549838745969	0\\
3.93559838995975	0\\
3.93569839245981	0\\
3.93579839495987	0\\
3.93589839745994	0\\
3.93599839996	0\\
3.93609840246006	0\\
3.93619840496012	0\\
3.93629840746019	0\\
3.93639840996025	0\\
3.93649841246031	0\\
3.93659841496037	0\\
3.93669841746044	0\\
3.9367984199605	0\\
3.93689842246056	0\\
3.93699842496062	0\\
3.93709842746069	0\\
3.93719842996075	0\\
3.93729843246081	0\\
3.93739843496087	0\\
3.93749843746094	0\\
3.937598439961	0\\
3.93769844246106	0\\
3.93779844496112	0\\
3.93789844746119	0\\
3.93799844996125	0\\
3.93809845246131	0\\
3.93819845496137	0\\
3.93829845746144	0\\
3.9383984599615	0\\
3.93849846246156	0\\
3.93859846496162	0\\
3.93869846746169	0\\
3.93879846996175	0\\
3.93889847246181	0\\
3.93899847496187	0\\
3.93909847746194	0\\
3.939198479962	0\\
3.93929848246206	0\\
3.93939848496212	0\\
3.93949848746219	0\\
3.93959848996225	0\\
3.93969849246231	0\\
3.93979849496237	0\\
3.93989849746244	0\\
3.9399984999625	0\\
3.94009850246256	0\\
3.94019850496262	0\\
3.94029850746269	0\\
3.94039850996275	0\\
3.94049851246281	0\\
3.94059851496287	0\\
3.94069851746294	0\\
3.940798519963	0\\
3.94089852246306	0\\
3.94099852496312	0\\
3.94109852746319	0\\
3.94119852996325	0\\
3.94129853246331	0\\
3.94139853496337	0\\
3.94149853746344	0\\
3.9415985399635	0\\
3.94169854246356	0\\
3.94179854496362	0\\
3.94189854746369	0\\
3.94199854996375	0\\
3.94209855246381	0\\
3.94219855496387	0\\
3.94229855746394	0\\
3.942398559964	0\\
3.94249856246406	0\\
3.94259856496412	0\\
3.94269856746419	0\\
3.94279856996425	0\\
3.94289857246431	0\\
3.94299857496437	0\\
3.94309857746444	0\\
3.9431985799645	0\\
3.94329858246456	0\\
3.94339858496462	0\\
3.94349858746469	0\\
3.94359858996475	0\\
3.94369859246481	0\\
3.94379859496487	0\\
3.94389859746494	0\\
3.943998599965	0\\
3.94409860246506	0\\
3.94419860496512	0\\
3.94429860746519	0\\
3.94439860996525	0\\
3.94449861246531	0\\
3.94459861496537	0\\
3.94469861746544	0\\
3.9447986199655	0\\
3.94489862246556	0\\
3.94499862496562	0\\
3.94509862746569	0\\
3.94519862996575	0\\
3.94529863246581	0\\
3.94539863496587	0\\
3.94549863746594	0\\
3.945598639966	0\\
3.94569864246606	0\\
3.94579864496612	0\\
3.94589864746619	0\\
3.94599864996625	0\\
3.94609865246631	0\\
3.94619865496637	0\\
3.94629865746644	0\\
3.9463986599665	0\\
3.94649866246656	0\\
3.94659866496662	0\\
3.94669866746669	0\\
3.94679866996675	0\\
3.94689867246681	0\\
3.94699867496687	0\\
3.94709867746694	0\\
3.947198679967	0\\
3.94729868246706	0\\
3.94739868496712	0\\
3.94749868746719	0\\
3.94759868996725	0\\
3.94769869246731	0\\
3.94779869496737	0\\
3.94789869746744	0\\
3.9479986999675	0\\
3.94809870246756	0\\
3.94819870496762	0\\
3.94829870746769	0\\
3.94839870996775	0\\
3.94849871246781	0\\
3.94859871496787	0\\
3.94869871746794	0\\
3.948798719968	0\\
3.94889872246806	0\\
3.94899872496812	0\\
3.94909872746819	0\\
3.94919872996825	0\\
3.94929873246831	0\\
3.94939873496837	0\\
3.94949873746844	0\\
3.9495987399685	0\\
3.94969874246856	0\\
3.94979874496862	0\\
3.94989874746869	0\\
3.94999874996875	0\\
3.95009875246881	0\\
3.95019875496887	0\\
3.95029875746894	0\\
3.950398759969	0\\
3.95049876246906	0\\
3.95059876496912	0\\
3.95069876746919	0\\
3.95079876996925	0\\
3.95089877246931	0\\
3.95099877496937	0\\
3.95109877746944	0\\
3.9511987799695	0\\
3.95129878246956	0\\
3.95139878496962	0\\
3.95149878746969	0\\
3.95159878996975	0\\
3.95169879246981	0\\
3.95179879496987	0\\
3.95189879746994	0\\
3.95199879997	0\\
3.95209880247006	0\\
3.95219880497012	0\\
3.95229880747019	0\\
3.95239880997025	0\\
3.95249881247031	0\\
3.95259881497037	0\\
3.95269881747044	0\\
3.9527988199705	0\\
3.95289882247056	0\\
3.95299882497062	0\\
3.95309882747069	0\\
3.95319882997075	0\\
3.95329883247081	0\\
3.95339883497087	0\\
3.95349883747094	0\\
3.953598839971	0\\
3.95369884247106	0\\
3.95379884497112	0\\
3.95389884747119	0\\
3.95399884997125	0\\
3.95409885247131	0\\
3.95419885497137	0\\
3.95429885747144	0\\
3.9543988599715	0\\
3.95449886247156	0\\
3.95459886497162	0\\
3.95469886747169	0\\
3.95479886997175	0\\
3.95489887247181	0\\
3.95499887497187	0\\
3.95509887747194	0\\
3.955198879972	0\\
3.95529888247206	0\\
3.95539888497212	0\\
3.95549888747219	0\\
3.95559888997225	0\\
3.95569889247231	0\\
3.95579889497237	0\\
3.95589889747244	0\\
3.9559988999725	0\\
3.95609890247256	0\\
3.95619890497262	0\\
3.95629890747269	0\\
3.95639890997275	0\\
3.95649891247281	0\\
3.95659891497287	0\\
3.95669891747294	0\\
3.956798919973	0\\
3.95689892247306	0\\
3.95699892497312	0\\
3.95709892747319	0\\
3.95719892997325	0\\
3.95729893247331	0\\
3.95739893497337	0\\
3.95749893747344	0\\
3.9575989399735	0\\
3.95769894247356	0\\
3.95779894497362	0\\
3.95789894747369	0\\
3.95799894997375	0\\
3.95809895247381	0\\
3.95819895497387	0\\
3.95829895747394	0\\
3.958398959974	0\\
3.95849896247406	0\\
3.95859896497412	0\\
3.95869896747419	0\\
3.95879896997425	0\\
3.95889897247431	0\\
3.95899897497437	0\\
3.95909897747444	0\\
3.9591989799745	0\\
3.95929898247456	0\\
3.95939898497462	0\\
3.95949898747469	0\\
3.95959898997475	0\\
3.95969899247481	0\\
3.95979899497487	0\\
3.95989899747494	0\\
3.959998999975	0\\
3.96009900247506	0\\
3.96019900497512	0\\
3.96029900747519	0\\
3.96039900997525	0\\
3.96049901247531	0\\
3.96059901497537	0\\
3.96069901747544	0\\
3.9607990199755	0\\
3.96089902247556	0\\
3.96099902497562	0\\
3.96109902747569	0\\
3.96119902997575	0\\
3.96129903247581	0\\
3.96139903497587	0\\
3.96149903747594	0\\
3.961599039976	0\\
3.96169904247606	0\\
3.96179904497612	0\\
3.96189904747619	0\\
3.96199904997625	0\\
3.96209905247631	0\\
3.96219905497637	0\\
3.96229905747644	0\\
3.9623990599765	0\\
3.96249906247656	0\\
3.96259906497662	0\\
3.96269906747669	0\\
3.96279906997675	0\\
3.96289907247681	0\\
3.96299907497687	0\\
3.96309907747694	0\\
3.963199079977	0\\
3.96329908247706	0\\
3.96339908497712	0\\
3.96349908747719	0\\
3.96359908997725	0\\
3.96369909247731	0\\
3.96379909497737	0\\
3.96389909747744	0\\
3.9639990999775	0\\
3.96409910247756	0\\
3.96419910497762	0\\
3.96429910747769	0\\
3.96439910997775	0\\
3.96449911247781	0\\
3.96459911497787	0\\
3.96469911747794	0\\
3.964799119978	0\\
3.96489912247806	0\\
3.96499912497812	0\\
3.96509912747819	0\\
3.96519912997825	0\\
3.96529913247831	0\\
3.96539913497837	0\\
3.96549913747844	0\\
3.9655991399785	0\\
3.96569914247856	0\\
3.96579914497862	0\\
3.96589914747869	0\\
3.96599914997875	0\\
3.96609915247881	0\\
3.96619915497887	0\\
3.96629915747894	0\\
3.966399159979	0\\
3.96649916247906	0\\
3.96659916497912	0\\
3.96669916747919	0\\
3.96679916997925	0\\
3.96689917247931	0\\
3.96699917497937	0\\
3.96709917747944	0\\
3.9671991799795	0\\
3.96729918247956	0\\
3.96739918497962	0\\
3.96749918747969	0\\
3.96759918997975	0\\
3.96769919247981	0\\
3.96779919497987	0\\
3.96789919747994	0\\
3.96799919998	0\\
3.96809920248006	0\\
3.96819920498012	0\\
3.96829920748019	0\\
3.96839920998025	0\\
3.96849921248031	0\\
3.96859921498037	0\\
3.96869921748044	0\\
3.9687992199805	0\\
3.96889922248056	0\\
3.96899922498062	0\\
3.96909922748069	0\\
3.96919922998075	0\\
3.96929923248081	0\\
3.96939923498087	0\\
3.96949923748094	0\\
3.969599239981	0\\
3.96969924248106	0\\
3.96979924498112	0\\
3.96989924748119	0\\
3.96999924998125	0\\
3.97009925248131	0\\
3.97019925498137	0\\
3.97029925748144	0\\
3.9703992599815	0\\
3.97049926248156	0\\
3.97059926498162	0\\
3.97069926748169	0\\
3.97079926998175	0\\
3.97089927248181	0\\
3.97099927498187	0\\
3.97109927748194	0\\
3.971199279982	0\\
3.97129928248206	0\\
3.97139928498212	0\\
3.97149928748219	0\\
3.97159928998225	0\\
3.97169929248231	0\\
3.97179929498237	0\\
3.97189929748244	0\\
3.9719992999825	0\\
3.97209930248256	0\\
3.97219930498262	0\\
3.97229930748269	0\\
3.97239930998275	0\\
3.97249931248281	0\\
3.97259931498287	0\\
3.97269931748294	0\\
3.972799319983	0\\
3.97289932248306	0\\
3.97299932498312	0\\
3.97309932748319	0\\
3.97319932998325	0\\
3.97329933248331	0\\
3.97339933498337	0\\
3.97349933748344	0\\
3.9735993399835	0\\
3.97369934248356	0\\
3.97379934498362	0\\
3.97389934748369	0\\
3.97399934998375	0\\
3.97409935248381	0\\
3.97419935498387	0\\
3.97429935748394	0\\
3.974399359984	0\\
3.97449936248406	0\\
3.97459936498412	0\\
3.97469936748419	0\\
3.97479936998425	0\\
3.97489937248431	0\\
3.97499937498437	0\\
3.97509937748444	0\\
3.9751993799845	0\\
3.97529938248456	0\\
3.97539938498462	0\\
3.97549938748469	0\\
3.97559938998475	0\\
3.97569939248481	0\\
3.97579939498487	0\\
3.97589939748494	0\\
3.975999399985	0\\
3.97609940248506	0\\
3.97619940498512	0\\
3.97629940748519	0\\
3.97639940998525	0\\
3.97649941248531	0\\
3.97659941498537	0\\
3.97669941748544	0\\
3.9767994199855	0\\
3.97689942248556	0\\
3.97699942498562	0\\
3.97709942748569	0\\
3.97719942998575	0\\
3.97729943248581	0\\
3.97739943498587	0\\
3.97749943748594	0\\
3.977599439986	0\\
3.97769944248606	0\\
3.97779944498612	0\\
3.97789944748619	0\\
3.97799944998625	0\\
3.97809945248631	0\\
3.97819945498637	0\\
3.97829945748644	0\\
3.9783994599865	0\\
3.97849946248656	0\\
3.97859946498662	0\\
3.97869946748669	0\\
3.97879946998675	0\\
3.97889947248681	0\\
3.97899947498687	0\\
3.97909947748694	0\\
3.979199479987	0\\
3.97929948248706	0\\
3.97939948498712	0\\
3.97949948748719	0\\
3.97959948998725	0\\
3.97969949248731	0\\
3.97979949498737	0\\
3.97989949748744	0\\
3.9799994999875	0\\
3.98009950248756	0\\
3.98019950498762	0\\
3.98029950748769	0\\
3.98039950998775	0\\
3.98049951248781	0\\
3.98059951498787	0\\
3.98069951748794	0\\
3.980799519988	0\\
3.98089952248806	0\\
3.98099952498812	0\\
3.98109952748819	0\\
3.98119952998825	0\\
3.98129953248831	0\\
3.98139953498837	0\\
3.98149953748844	0\\
3.9815995399885	0\\
3.98169954248856	0\\
3.98179954498862	0\\
3.98189954748869	0\\
3.98199954998875	0\\
3.98209955248881	0\\
3.98219955498887	0\\
3.98229955748894	0\\
3.982399559989	0\\
3.98249956248906	0\\
3.98259956498912	0\\
3.98269956748919	0\\
3.98279956998925	0\\
3.98289957248931	0\\
3.98299957498937	0\\
3.98309957748944	0\\
3.9831995799895	0\\
3.98329958248956	0\\
3.98339958498962	0\\
3.98349958748969	0\\
3.98359958998975	0\\
3.98369959248981	0\\
3.98379959498987	0\\
3.98389959748994	0\\
3.98399959999	0\\
3.98409960249006	0\\
3.98419960499012	0\\
3.98429960749019	0\\
3.98439960999025	0\\
3.98449961249031	0\\
3.98459961499037	0\\
3.98469961749044	0\\
3.9847996199905	0\\
3.98489962249056	0\\
3.98499962499062	0\\
3.98509962749069	0\\
3.98519962999075	0\\
3.98529963249081	0\\
3.98539963499087	0\\
3.98549963749094	0\\
3.985599639991	0\\
3.98569964249106	0\\
3.98579964499112	0\\
3.98589964749119	0\\
3.98599964999125	0\\
3.98609965249131	0\\
3.98619965499137	0\\
3.98629965749144	0\\
3.9863996599915	0\\
3.98649966249156	0\\
3.98659966499162	0\\
3.98669966749169	0\\
3.98679966999175	0\\
3.98689967249181	0\\
3.98699967499187	0\\
3.98709967749194	0\\
3.987199679992	0\\
3.98729968249206	0\\
3.98739968499212	0\\
3.98749968749219	0\\
3.98759968999225	0\\
3.98769969249231	0\\
3.98779969499237	0\\
3.98789969749244	0\\
3.9879996999925	0\\
3.98809970249256	0\\
3.98819970499262	0\\
3.98829970749269	0\\
3.98839970999275	0\\
3.98849971249281	0\\
3.98859971499287	0\\
3.98869971749294	0\\
3.988799719993	0\\
3.98889972249306	0\\
3.98899972499312	0\\
3.98909972749319	0\\
3.98919972999325	0\\
3.98929973249331	0\\
3.98939973499337	0\\
3.98949973749344	0\\
3.9895997399935	0\\
3.98969974249356	0\\
3.98979974499362	0\\
3.98989974749369	0\\
3.98999974999375	0\\
3.99009975249381	0\\
3.99019975499388	0\\
3.99029975749394	0\\
3.990399759994	0\\
3.99049976249406	0\\
3.99059976499412	0\\
3.99069976749419	0\\
3.99079976999425	0\\
3.99089977249431	0\\
3.99099977499437	0\\
3.99109977749444	0\\
3.9911997799945	0\\
3.99129978249456	0\\
3.99139978499462	0\\
3.99149978749469	0\\
3.99159978999475	0\\
3.99169979249481	0\\
3.99179979499487	0\\
3.99189979749494	0\\
3.991999799995	0\\
3.99209980249506	0\\
3.99219980499513	0\\
3.99229980749519	0\\
3.99239980999525	0\\
3.99249981249531	0\\
3.99259981499537	0\\
3.99269981749544	0\\
3.9927998199955	0\\
3.99289982249556	0\\
3.99299982499562	0\\
3.99309982749569	0\\
3.99319982999575	0\\
3.99329983249581	0\\
3.99339983499587	0\\
3.99349983749594	0\\
3.993599839996	0\\
3.99369984249606	0\\
3.99379984499612	0\\
3.99389984749619	0\\
3.99399984999625	0\\
3.99409985249631	0\\
3.99419985499638	0\\
3.99429985749644	0\\
3.9943998599965	0\\
3.99449986249656	0\\
3.99459986499663	0\\
3.99469986749669	0\\
3.99479986999675	0\\
3.99489987249681	0\\
3.99499987499687	0\\
3.99509987749694	0\\
3.995199879997	0\\
3.99529988249706	0\\
3.99539988499712	0\\
3.99549988749719	0\\
3.99559988999725	0\\
3.99569989249731	0\\
3.99579989499737	0\\
3.99589989749744	0\\
3.9959998999975	0\\
3.99609990249756	0\\
3.99619990499763	0\\
3.99629990749769	0\\
3.99639990999775	0\\
3.99649991249781	0\\
3.99659991499788	0\\
3.99669991749794	0\\
3.996799919998	0\\
3.99689992249806	0\\
3.99699992499812	0\\
3.99709992749819	0\\
3.99719992999825	0\\
3.99729993249831	0\\
3.99739993499837	0\\
3.99749993749844	0\\
3.9975999399985	0\\
3.99769994249856	0\\
3.99779994499862	0\\
3.99789994749869	0\\
3.99799994999875	0\\
3.99809995249881	0\\
3.99819995499888	0\\
3.99829995749894	0\\
3.998399959999	0\\
3.99849996249906	0\\
3.99859996499913	0\\
3.99869996749919	0\\
3.99879996999925	0\\
3.99889997249931	0\\
3.99899997499937	0\\
3.99909997749944	0\\
3.9991999799995	0\\
3.99929998249956	0\\
3.99939998499962	0\\
3.99949998749969	0\\
3.99959998999975	0\\
3.99969999249981	0\\
3.99979999499987	0\\
3.99989999749994	0\\
4	0\\
};
\addlegendentry{c2};

\addplot [color=mycolor3,solid,forget plot]
  table[row sep=crcr]{%
0	-350453646096.334\\
0.000100002500062502	-350382599329.738\\
0.000200005000125003	-350309833689.756\\
0.000300007500187505	-350234776218.594\\
0.000400010000250006	-350157426916.251\\
0.000500012500312508	-350078358740.523\\
0.000600015000375009	-349996425775.82\\
0.000700017500437511	-349912773937.73\\
0.000800020000500012	-349826257310.666\\
0.000900022500562514	-349736875894.625\\
0.00100002500062502	-349645202647.404\\
0.00110002750068752	-349550664611.208\\
0.00120003000075002	-349453261786.036\\
0.00130003250081252	-349352994171.888\\
0.00140003500087502	-349249288810.969\\
0.00150003750093752	-349142145703.279\\
0.00160004000100002	-349032137806.614\\
0.00170004250106253	-348918119205.383\\
0.00180004500112503	-348800662857.382\\
0.00190004750118753	-348679768762.609\\
0.00200005000125003	-348554291005.475\\
0.00210005250131253	-348424802543.776\\
0.00220005500137503	-348291303377.51\\
0.00230005750143754	-348152647591.089\\
0.00240006000150004	-348009981100.101\\
0.00250006250156254	-347861585031.162\\
0.00260006500162504	-347708605299.862\\
0.00270006750168754	-347549895990.611\\
0.00280007000175004	-347386030061.203\\
0.00290007250181255	-347215288638.255\\
0.00300007500187505	-347039390595.149\\
0.00310007750193755	-346856044100.708\\
0.00320008000200005	-346666395070.519\\
0.00330008250206255	-346469297588.994\\
0.00340008500212505	-346264751656.133\\
0.00350008750218755	-346052184314.139\\
0.00360009000225006	-345831595563.014\\
0.00370009250231256	-345601839487.166\\
0.00380009500237506	-345362916086.597\\
0.00390009750243756	-345114825361.305\\
0.00400010000250006	-344855848437.906\\
0.00410010250256256	-344585985316.399\\
0.00420010500262507	-344305235996.785\\
0.00430010750268757	-344012454563.473\\
0.00440011000275007	-343706495100.873\\
0.00450011250281257	-343387357608.985\\
0.00460011500287507	-343053896172.219\\
0.00470011750293757	-342705537832.78\\
0.00480012000300007	-342340563717.281\\
0.00490012250306258	-341958973825.724\\
0.00500012500312508	-341559622242.518\\
0.00510012750318758	-341140790094.278\\
0.00520013000325008	-340701904423.207\\
0.00530013250331258	-340240673398.127\\
0.00540013500337508	-339757097019.037\\
0.00550013750343759	-339248310496.96\\
0.00560014000350009	-338713167916.308\\
0.00570014250356259	-338149950403.695\\
0.00580014500362509	-337556939085.734\\
0.00590014750368759	-336931269173.451\\
0.00600015000375009	-336271221793.461\\
0.0061001525038126	-335573932156.786\\
0.0062001550038751	-334836535474.453\\
0.0063001575039376	-334056166957.485\\
0.0064001600040001	-333229388859.111\\
0.0065001625040626	-332353336390.356\\
0.0066001650041251	-331422852931.064\\
0.0067001675041876	-330433927776.668\\
0.00680017000425011	-329382550222.603\\
0.00690017250431261	-328262417733.122\\
0.00700017500437511	-327067800730.274\\
0.00710017750443761	-325792396678.313\\
0.00720018000450011	-324428757125.902\\
0.00730018250456261	-322969433621.703\\
0.00740018500462512	-321405258840.996\\
0.00750018750468762	-319725919543.468\\
0.00760019000475012	-317921102488.806\\
0.00770019250481262	-315978202605.517\\
0.00780019500487512	-313884041864.314\\
0.00790019750493762	-311622004489.138\\
0.00800020000500012	-309176047661.724\\
0.00810020250506263	-306526690817.039\\
0.00820020500512513	-303651588601.073\\
0.00830020750518763	-300526676786.429\\
0.00840021000525013	-297124453398.942\\
0.00850021250531263	-293413978717.675\\
0.00860021500537513	-289359729359.329\\
0.00870021750543764	-284923890109.427\\
0.00880022000550014	-280060624344.356\\
0.00890022250556264	-274721803609.327\\
0.00900022500562514	-268851278040.417\\
0.00910022750568764	-262387741153.546\\
0.00920023000575014	-255262438013.299\\
0.00930023250581265	-247401457064.104\\
0.00940023500587515	-238724584214.643\\
0.00950023750593765	-229145875795.646\\
0.00960024000600015	-218577669264.458\\
0.00970024250606265	-206935739825.195\\
0.00980024500612515	-194141019302.128\\
0.00990024750618766	-180133920084.565\\
0.0100002500062502	-164885794282.748\\
0.0101002525063127	-148416122461.713\\
0.0102002550063752	-130818296742.065\\
0.0103002575064377	-112286549816.353\\
0.0104002600065002	-93148040585.5984\\
0.0105002625065627	-73886345428.8904\\
0.0106002650066252	-55158532345.6853\\
0.0107002675066877	-37784160166.1382\\
0.0108002700067502	-22695717701.8246\\
0.0109002725068127	-10848497484.566\\
0.0110002750068752	-3092717316.13511\\
0.0111002775069377	-34908771.4712742\\
0.0112002800070002	-1925298619.8222\\
0.0113002825070627	-8608461688.72257\\
0.0114002850071252	-19554132815.3428\\
0.0115002875071877	-33957661531.357\\
0.0116002900072502	-50876245784.8776\\
0.0117002925073127	-69361697720.7423\\
0.0118002950073752	-88562659351.1665\\
0.0119002975074377	-107780809715.445\\
0.0120003000075002	-126491892431.032\\
0.0121003025075627	-144333225213.611\\
0.0122003050076252	-161082500438.67\\
0.0123003075076877	-176625126547.184\\
0.0124003100077502	-190926153113.649\\
0.0125003125078127	-204003914787.51\\
0.0126003150078752	-215913415517.099\\
0.0127003175079377	-226728566857.989\\
0.0128003200080002	-236534166563.858\\
0.0129003225080627	-245417877177.361\\
0.0130003250081252	-253465069409.974\\
0.0131003275081877	-260758249184.194\\
0.0132003300082502	-267372473971.184\\
0.0133003325083127	-273378217579.745\\
0.0134003350083752	-278837932409.547\\
0.0135003375084377	-283809487197.897\\
0.0136003400085002	-288342729272.972\\
0.0137003425085627	-292483495258.383\\
0.0138003450086252	-296272465157.583\\
0.0139003475086877	-299744589396.076\\
0.0140003500087502	-302933099525.979\\
0.0141003525088127	-305864924563.663\\
0.0142003550088752	-308566420567.705\\
0.0143003575089377	-311058786976.524\\
0.0144003600090002	-313362650270.745\\
0.0145003625090627	-315495199184.222\\
0.0146003650091252	-317473049493.013\\
0.0147003675091877	-319309379226.408\\
0.0148003700092502	-321017366413.693\\
0.0149003725093127	-322607897252.976\\
0.0150003750093752	-324091284984.57\\
0.0151003775094377	-325476696933.196\\
0.0152003800095002	-326772154507.987\\
0.0153003825095627	-327985679118.074\\
0.0154003850096252	-329123000341.409\\
0.0155003875096877	-330190420713.737\\
0.0156003900097502	-331193669813.011\\
0.0157003925098127	-332137331301.592\\
0.0158003950098752	-333025988841.84\\
0.0159003975099377	-333864226096.116\\
0.0160004000100002	-334654907853.397\\
0.0161004025100628	-335402617776.042\\
0.0162004050101253	-336109074737.439\\
0.0163004075101878	-336777716484.356\\
0.0164004100102503	-337411407805.771\\
0.0165004125103128	-338011867575.068\\
0.0166004150103753	-338581960581.223\\
0.0167004175104378	-339123978655.417\\
0.0168004200105003	-339638494755.445\\
0.0169004225105628	-340128373670.281\\
0.0170004250106253	-340594188357.723\\
0.0171004275106878	-341038230648.949\\
0.0172004300107503	-341462219417.346\\
0.0173004325108128	-341866154662.913\\
0.0174004350108753	-342251755259.036\\
0.0175004375109378	-342620167121.305\\
0.0176004400110003	-342972536165.311\\
0.0177004425110628	-343309435348.848\\
0.0178004450111253	-343632010587.506\\
0.0179004475111878	-343940834839.082\\
0.0180004500112503	-344237054019.164\\
0.0181004525113128	-344520668127.754\\
0.0182004550113753	-344792823080.441\\
0.0183004575114378	-345054091835.021\\
0.0184004600115003	-345305047349.288\\
0.0185004625115628	-345546262581.038\\
0.0186004650116253	-345777737530.271\\
0.0187004675116878	-346000618112.577\\
0.0188004700117503	-346214904327.956\\
0.0189004725118128	-346421169134.203\\
0.0190004750118753	-346619985489.114\\
0.0191004775119378	-346811926350.482\\
0.0192004800120003	-346996418760.515\\
0.0193004825120628	-347174035677.005\\
0.0194004850121253	-347345923015.544\\
0.0195004875121878	-347511507818.337\\
0.0196004900122503	-347671363043.179\\
0.0197004925123128	-347826061647.864\\
0.0198004950123753	-347975030674.598\\
0.0199004975124378	-348119416038.971\\
0.0200005000125003	-348258644783.188\\
0.0201005025125628	-348393862822.839\\
0.0202005050126253	-348523924242.333\\
0.0203005075126878	-348650547915.057\\
0.0204005100127503	-348772587925.42\\
0.0205005125128128	-348890617231.217\\
0.0206005150128753	-349005208790.243\\
0.0207005175129378	-349116362602.499\\
0.0208005200130003	-349224078667.983\\
0.0209005225130628	-349328356986.697\\
0.0210005250131253	-349429770516.435\\
0.0211005275131878	-349527746299.403\\
0.0212005300132503	-349622857293.394\\
0.0213005325133128	-349715676456.205\\
0.0214005350133753	-349805057872.246\\
0.0215005375134378	-349892147457.106\\
0.0216005400135003	-349976945210.785\\
0.0217005425135628	-350058878175.489\\
0.0218005450136253	-350139092266.807\\
0.0219005475136878	-350216441569.15\\
0.0220005500137503	-350292071998.107\\
0.0221005525138128	-350365410595.884\\
0.0222005550138753	-350437030320.275\\
0.0223005575139378	-350506358213.486\\
0.0224005600140003	-350573967233.311\\
0.0225005625140629	-350639857379.751\\
0.0226005650141254	-350703455695.011\\
0.0227005675141879	-350765908094.68\\
0.0228005700142504	-350826641620.964\\
0.0229005725143129	-350885656273.863\\
0.0230005750143754	-350942952053.376\\
0.0231005775144379	-350999101917.298\\
0.0232005800145004	-351053532907.836\\
0.0233005825145629	-351106817982.783\\
0.0234005850146254	-351158957142.14\\
0.0235005875146879	-351209377428.112\\
0.0236005900147504	-351258651798.493\\
0.0237005925148129	-351306780253.284\\
0.0238005950148754	-351353762792.484\\
0.0239005975149379	-351399599416.095\\
0.0240006000150004	-351444290124.115\\
0.0241006025150629	-351487834916.545\\
0.0242006050151254	-351530233793.385\\
0.0243006075151879	-351571486754.634\\
0.0244006100152504	-351612166758.088\\
0.0245006125153129	-351651700845.953\\
0.0246006150153754	-351690089018.226\\
0.0247006175154379	-351727904232.705\\
0.0248006200155004	-351764573531.593\\
0.0249006225155629	-351800669872.687\\
0.0250006250156254	-351835620298.19\\
0.0251006275156879	-351869997765.897\\
0.0252006300157504	-351903802275.81\\
0.0253006325158129	-351936460870.133\\
0.0254006350158754	-351968546506.66\\
0.0255006375159379	-351999486227.597\\
0.0256006400160004	-352029852990.739\\
0.0257006425160629	-352060219753.881\\
0.0258006450161254	-352089440601.432\\
0.0259006475161879	-352117515533.394\\
0.0260006500162504	-352145590465.355\\
0.0261006525163129	-352173092439.522\\
0.0262006550163754	-352199448498.098\\
0.0263006575164379	-352225804556.674\\
0.0264006600165004	-352251014699.659\\
0.0265006625165629	-352276224842.645\\
0.0266006650166254	-352300862027.836\\
0.0267006675166879	-352324353297.436\\
0.0268006700167504	-352347844567.036\\
0.0269006725168129	-352370762878.842\\
0.0270006750168754	-352393108232.852\\
0.0271006775169379	-352414880629.067\\
0.0272006800170004	-352436080067.487\\
0.0273006825170629	-352457279505.906\\
0.0274006850171254	-352477905986.531\\
0.0275006875171879	-352497959509.361\\
0.0276006900172504	-352517440074.395\\
0.0277006925173129	-352536347681.634\\
0.0278006950173754	-352555255288.874\\
0.0279006975174379	-352573589938.318\\
0.0280007000175004	-352591924587.762\\
0.0281007025175629	-352609113321.616\\
0.0282007050176254	-352626875013.265\\
0.0283007075176879	-352643490789.324\\
0.0284007100177504	-352660106565.383\\
0.0285007125178129	-352676149383.646\\
0.0286007150178754	-352692192201.91\\
0.028700717517938	-352707662062.379\\
0.0288007200180005	-352722558965.052\\
0.028900722518063	-352737455867.725\\
0.0290007250181255	-352751779812.604\\
0.029100727518188	-352766103757.482\\
0.0292007300182505	-352779854744.565\\
0.029300732518313	-352793605731.648\\
0.0294007350183755	-352806783760.936\\
0.029500737518438	-352819961790.224\\
0.0296007400185005	-352832566861.717\\
0.029700742518563	-352845171933.21\\
0.0298007450186255	-352857777004.703\\
0.029900747518688	-352869236160.605\\
0.0300007500187505	-352881268274.303\\
0.030100752518813	-352892727430.206\\
0.0302007550188755	-352903613628.313\\
0.030300757518938	-352915072784.216\\
0.0304007600190005	-352925386024.528\\
0.030500762519063	-352936272222.636\\
0.0306007650191255	-352946012505.153\\
0.030700767519188	-352956325745.465\\
0.0308007700192505	-352966066027.983\\
0.030900772519313	-352975806310.5\\
0.0310007750193755	-352984973635.222\\
0.031100777519438	-352994140959.944\\
0.0312007800195005	-353003308284.666\\
0.031300782519563	-353011902651.593\\
0.0314007850196255	-353020497018.52\\
0.031500787519688	-353029091385.447\\
0.0316007900197505	-353037112794.579\\
0.031700792519813	-353045134203.711\\
0.0318007950198755	-353053155612.842\\
0.031900797519938	-353060604064.179\\
0.0320008000200005	-353068052515.516\\
0.032100802520063	-353075500966.853\\
0.0322008050201255	-353082376460.394\\
0.032300807520188	-353089251953.936\\
0.0324008100202505	-353096127447.477\\
0.032500812520313	-353103002941.019\\
0.0326008150203755	-353109305476.765\\
0.032700817520438	-353115608012.512\\
0.0328008200205005	-353121910548.258\\
0.032900822520563	-353127640126.209\\
0.0330008250206255	-353133369704.161\\
0.033100827520688	-353139099282.112\\
0.0332008300207505	-353144828860.063\\
0.033300832520813	-353149985480.22\\
0.0334008350208755	-353155142100.376\\
0.033500837520938	-353160298720.532\\
0.0336008400210005	-353165455340.688\\
0.033700842521063	-353170611960.844\\
0.0338008450211255	-353175195623.205\\
0.033900847521188	-353179779285.566\\
0.0340008500212505	-353184362947.927\\
0.034100852521313	-353188373652.493\\
0.0342008550213755	-353192384357.059\\
0.034300857521438	-353196968019.42\\
0.0344008600215005	-353200978723.986\\
0.034500862521563	-353204416470.757\\
0.0346008650216255	-353208427175.323\\
0.034700867521688	-353211864922.094\\
0.0348008700217505	-353215302668.865\\
0.034900872521813	-353218740415.635\\
0.0350008750218755	-353222178162.406\\
0.0351008775219381	-353225042951.382\\
0.0352008800220006	-353228480698.153\\
0.0353008825220631	-353231345487.128\\
0.0354008850221256	-353234210276.104\\
0.0355008875221881	-353237075065.079\\
0.0356008900222506	-353239366896.26\\
0.0357008925223131	-353242231685.236\\
0.0358008950223756	-353244523516.416\\
0.0359008975224381	-353246815347.597\\
0.0360009000225006	-353249107178.777\\
0.0361009025225631	-353251399009.958\\
0.0362009050226256	-353253117883.343\\
0.0363009075226881	-353255409714.524\\
0.0364009100227506	-353257128587.909\\
0.0365009125228131	-353258847461.294\\
0.0366009150228756	-353260566334.68\\
0.0367009175229381	-353262285208.065\\
0.0368009200230006	-353263431123.656\\
0.0369009225230631	-353265149997.041\\
0.0370009250231256	-353266295912.631\\
0.0371009275231881	-353267441828.221\\
0.0372009300232506	-353268587743.812\\
0.0373009325233131	-353269733659.402\\
0.0374009350233756	-353270879574.992\\
0.0375009375234381	-353272025490.582\\
0.0376009400235006	-353272598448.378\\
0.0377009425235631	-353273744363.968\\
0.0378009450236256	-353274317321.763\\
0.0379009475236881	-353274890279.558\\
0.0380009500237506	-353275463237.353\\
0.0381009525238131	-353276036195.148\\
0.0382009550238756	-353276036195.148\\
0.0383009575239381	-353276609152.943\\
0.0384009600240006	-353276609152.943\\
0.0385009625240631	-353277182110.739\\
0.0386009650241256	-353277182110.739\\
0.0387009675241881	-353277182110.739\\
0.0388009700242506	-353277182110.739\\
0.0389009725243131	-353277182110.739\\
0.0390009750243756	-353276609152.943\\
0.0391009775244381	-353276609152.943\\
0.0392009800245006	-353276036195.148\\
0.0393009825245631	-353276036195.148\\
0.0394009850246256	-353275463237.353\\
0.0395009875246881	-353274890279.558\\
0.0396009900247506	-353274317321.763\\
0.0397009925248131	-353273744363.968\\
0.0398009950248756	-353273171406.173\\
0.0399009975249381	-353272598448.378\\
0.0400010000250006	-353271452532.787\\
0.0401010025250631	-353270879574.992\\
0.0402010050251256	-353269733659.402\\
0.0403010075251881	-353268587743.812\\
0.0404010100252506	-353267441828.221\\
0.0405010125253131	-353266868870.426\\
0.0406010150253756	-353265149997.041\\
0.0407010175254381	-353264004081.451\\
0.0408010200255006	-353262858165.86\\
0.0409010225255631	-353261712250.27\\
0.0410010250256256	-353259993376.885\\
0.0411010275256881	-353258847461.294\\
0.0412010300257506	-353257128587.909\\
0.0413010325258131	-353255982672.319\\
0.0414010350258756	-353254263798.933\\
0.0415010375259382	-353252544925.548\\
0.0416010400260007	-353250826052.163\\
0.0417010425260632	-353249107178.777\\
0.0418010450261257	-353247388305.392\\
0.0419010475261882	-353245096474.211\\
0.0420010500262507	-353243377600.826\\
0.0421010525263132	-353241658727.441\\
0.0422010550263757	-353239366896.26\\
0.0423010575264382	-353237075065.079\\
0.0424010600265007	-353235356191.694\\
0.0425010625265632	-353233064360.514\\
0.0426010650266257	-353230772529.333\\
0.0427010675266882	-353228480698.153\\
0.0428010700267507	-353226188866.972\\
0.0429010725268132	-353223897035.792\\
0.0430010750268757	-353221605204.611\\
0.0431010775269382	-353219313373.43\\
0.0432010800270007	-353216448584.455\\
0.0433010825270632	-353214156753.274\\
0.0434010850271257	-353211291964.299\\
0.0435010875271882	-353209000133.118\\
0.0436010900272507	-353206135344.142\\
0.0437010925273132	-353203270555.167\\
0.0438010950273757	-353200978723.986\\
0.0439010975274382	-353198113935.011\\
0.0440011000275007	-353195249146.035\\
0.0441011025275632	-353192384357.059\\
0.0442011050276257	-353188946610.289\\
0.0443011075276882	-353186081821.313\\
0.0444011100277507	-353183217032.337\\
0.0445011125278132	-353180352243.362\\
0.0446011150278757	-353176914496.591\\
0.0447011175279382	-353174049707.615\\
0.0448011200280007	-353170611960.844\\
0.0449011225280632	-353167747171.869\\
0.0450011250281257	-353164309425.098\\
0.0451011275281882	-353160871678.327\\
0.0452011300282507	-353157433931.556\\
0.0453011325283132	-353153996184.786\\
0.0454011350283757	-353150558438.015\\
0.0455011375284382	-353147120691.244\\
0.0456011400285007	-353143682944.473\\
0.0457011425285632	-353140245197.702\\
0.0458011450286257	-353136807450.932\\
0.0459011475286882	-353133369704.161\\
0.0460011500287507	-353129358999.595\\
0.0461011525288132	-353125921252.824\\
0.0462011550288757	-353121910548.258\\
0.0463011575289382	-353118472801.487\\
0.0464011600290007	-353114462096.922\\
0.0465011625290632	-353110451392.356\\
0.0466011650291257	-353107013645.585\\
0.0467011675291882	-353103002941.019\\
0.0468011700292507	-353098992236.453\\
0.0469011725293132	-353094981531.887\\
0.0470011750293757	-353090970827.321\\
0.0471011775294382	-353086960122.755\\
0.0472011800295007	-353082949418.189\\
0.0473011825295632	-353078365755.828\\
0.0474011850296257	-353074355051.262\\
0.0475011875296882	-353070344346.696\\
0.0476011900297507	-353065760684.335\\
0.0477011925298132	-353061749979.769\\
0.0478011950298757	-353057739275.204\\
0.0479011975299382	-353053155612.842\\
0.0480012000300008	-353048571950.481\\
0.0481012025300633	-353044561245.916\\
0.0482012050301258	-353039977583.555\\
0.0483012075301883	-353035393921.193\\
0.0484012100302508	-353030810258.832\\
0.0485012125303133	-353026226596.471\\
0.0486012150303758	-353021642934.11\\
0.0487012175304383	-353017059271.749\\
0.0488012200305008	-353012475609.388\\
0.0489012225305633	-353007891947.027\\
0.0490012250306258	-353003308284.666\\
0.0491012275306883	-352998724622.305\\
0.0492012300307508	-352993568002.149\\
0.0493012325308133	-352988984339.788\\
0.0494012350308758	-352984400677.427\\
0.0495012375309383	-352979244057.271\\
0.0496012400310008	-352974660394.91\\
0.0497012425310633	-352969503774.753\\
0.0498012450311258	-352964920112.392\\
0.0499012475311883	-352959763492.236\\
0.0500012500312508	-352954606872.08\\
0.0501012525313133	-352949450251.924\\
0.0502012550313758	-352944293631.768\\
0.0503012575314383	-352939709969.407\\
0.0504012600315008	-352934553349.25\\
0.0505012625315633	-352929396729.094\\
0.0506012650316258	-352924240108.938\\
0.0507012675316883	-352919083488.782\\
0.0508012700317508	-352913353910.831\\
0.0509012725318133	-352908197290.674\\
0.0510012750318758	-352903040670.518\\
0.0511012775319383	-352897884050.362\\
0.0512012800320008	-352892154472.411\\
0.0513012825320633	-352886997852.255\\
0.0514012850321258	-352881268274.303\\
0.0515012875321883	-352876111654.147\\
0.0516012900322508	-352870382076.196\\
0.0517012925323133	-352865225456.04\\
0.0518012950323758	-352859495878.088\\
0.0519012975324383	-352853766300.137\\
0.0520013000325008	-352848609679.981\\
0.0521013025325633	-352842880102.029\\
0.0522013050326258	-352837150524.078\\
0.0523013075326883	-352831420946.127\\
0.0524013100327508	-352825691368.176\\
0.0525013125328133	-352819961790.224\\
0.0526013150328758	-352814232212.273\\
0.0527013175329383	-352808502634.322\\
0.0528013200330008	-352802773056.37\\
0.0529013225330633	-352797043478.419\\
0.0530013250331258	-352791313900.468\\
0.0531013275331883	-352785011364.721\\
0.0532013300332508	-352779281786.77\\
0.0533013325333133	-352773552208.819\\
0.0534013350333758	-352767249673.072\\
0.0535013375334383	-352761520095.121\\
0.0536013400335008	-352755217559.374\\
0.0537013425335633	-352749487981.423\\
0.0538013450336258	-352743185445.677\\
0.0539013475336883	-352736882909.93\\
0.0540013500337508	-352731153331.979\\
0.0541013525338133	-352724850796.233\\
0.0542013550338758	-352718548260.486\\
0.0543013575339383	-352712245724.74\\
0.0544013600340008	-352705943188.993\\
0.0545013625340634	-352700213611.042\\
0.0546013650341259	-352693911075.295\\
0.0547013675341884	-352687608539.549\\
0.0548013700342509	-352681306003.803\\
0.0549013725343134	-352674430510.261\\
0.0550013750343759	-352668127974.515\\
0.0551013775344384	-352661825438.768\\
0.0552013800345009	-352655522903.022\\
0.0553013825345634	-352649220367.275\\
0.0554013850346259	-352642344873.734\\
0.0555013875346884	-352636042337.987\\
0.0556013900347509	-352629739802.241\\
0.0557013925348134	-352622864308.699\\
0.0558013950348759	-352616561772.953\\
0.0559013975349384	-352609686279.411\\
0.0560014000350009	-352602810785.87\\
0.0561014025350634	-352596508250.123\\
0.0562014050351259	-352589632756.582\\
0.0563014075351884	-352583330220.835\\
0.0564014100352509	-352576454727.294\\
0.0565014125353134	-352569579233.752\\
0.0566014150353759	-352562703740.211\\
0.0567014175354384	-352555828246.669\\
0.0568014200355009	-352548952753.127\\
0.0569014225355634	-352542077259.586\\
0.0570014250356259	-352535201766.044\\
0.0571014275356884	-352528326272.503\\
0.0572014300357509	-352521450778.961\\
0.0573014325358134	-352514575285.419\\
0.0574014350358759	-352507699791.878\\
0.0575014375359384	-352500824298.336\\
0.0576014400360009	-352493948804.795\\
0.0577014425360634	-352486500353.458\\
0.0578014450361259	-352479624859.917\\
0.0579014475361884	-352472749366.375\\
0.0580014500362509	-352465300915.038\\
0.0581014525363134	-352458425421.497\\
0.0582014550363759	-352450976970.16\\
0.0583014575364384	-352444101476.618\\
0.0584014600365009	-352436653025.282\\
0.0585014625365634	-352429777531.74\\
0.0586014650366259	-352422329080.403\\
0.0587014675366884	-352414880629.067\\
0.0588014700367509	-352407432177.73\\
0.0589014725368134	-352400556684.188\\
0.0590014750368759	-352393108232.852\\
0.0591014775369384	-352385659781.515\\
0.0592014800370009	-352378211330.178\\
0.0593014825370634	-352370762878.842\\
0.0594014850371259	-352363314427.505\\
0.0595014875371884	-352355865976.168\\
0.0596014900372509	-352348417524.832\\
0.0597014925373134	-352340969073.495\\
0.0598014950373759	-352333520622.158\\
0.0599014975374384	-352326072170.821\\
0.0600015000375009	-352318623719.485\\
0.0601015025375634	-352310602310.353\\
0.0602015050376259	-352303153859.016\\
0.0603015075376884	-352295705407.68\\
0.0604015100377509	-352287683998.548\\
0.0605015125378134	-352280235547.211\\
0.0606015150378759	-352272787095.874\\
0.0607015175379384	-352264765686.742\\
0.0608015200380009	-352257317235.406\\
0.0609015225380635	-352249295826.274\\
0.061001525038126	-352241274417.142\\
0.0611015275381885	-352233825965.805\\
0.061201530038251	-352225804556.674\\
0.0613015325383135	-352217783147.542\\
0.061401535038376	-352210334696.205\\
0.0615015375384385	-352202313287.073\\
0.061601540038501	-352194291877.941\\
0.0617015425385635	-352186270468.81\\
0.061801545038626	-352178249059.678\\
0.0619015475386885	-352170227650.546\\
0.062001550038751	-352162206241.414\\
0.0621015525388135	-352154184832.282\\
0.062201555038876	-352146163423.15\\
0.0623015575389385	-352138142014.019\\
0.062401560039001	-352130120604.887\\
0.0625015625390635	-352122099195.755\\
0.062601565039126	-352114077786.623\\
0.0627015675391885	-352106056377.491\\
0.062801570039251	-352097462010.564\\
0.0629015725393135	-352089440601.432\\
0.063001575039376	-352081419192.301\\
0.0631015775394385	-352072824825.374\\
0.063201580039501	-352064803416.242\\
0.0633015825395635	-352056209049.315\\
0.063401585039626	-352048187640.183\\
0.0635015875396885	-352039593273.256\\
0.063601590039751	-352031571864.124\\
0.0637015925398135	-352022977497.197\\
0.063801595039876	-352014956088.065\\
0.0639015975399385	-352006361721.138\\
0.064001600040001	-351997767354.211\\
0.0641016025400635	-351989172987.285\\
0.064201605040126	-351981151578.153\\
0.0643016075401885	-351972557211.226\\
0.064401610040251	-351963962844.299\\
0.0645016125403135	-351955368477.372\\
0.064601615040376	-351946774110.445\\
0.0647016175404385	-351938179743.518\\
0.064801620040501	-351929585376.591\\
0.0649016225405635	-351920991009.664\\
0.065001625040626	-351912396642.737\\
0.0651016275406885	-351903802275.81\\
0.065201630040751	-351895207908.883\\
0.0653016325408135	-351886613541.956\\
0.065401635040876	-351877446217.234\\
0.0655016375409385	-351868851850.307\\
0.065601640041001	-351860257483.38\\
0.0657016425410635	-351851663116.453\\
0.065801645041126	-351842495791.731\\
0.0659016475411885	-351833901424.804\\
0.066001650041251	-351824734100.082\\
0.0661016525413135	-351816139733.155\\
0.066201655041376	-351806972408.433\\
0.0663016575414385	-351798378041.506\\
0.066401660041501	-351789210716.784\\
0.0665016625415635	-351780616349.857\\
0.066601665041626	-351771449025.135\\
0.0667016675416885	-351762281700.413\\
0.066801670041751	-351753687333.486\\
0.0669016725418135	-351744520008.764\\
0.067001675041876	-351735352684.042\\
0.0671016775419385	-351726185359.32\\
0.0672016800420011	-351717018034.597\\
0.0673016825420635	-351708423667.67\\
0.0674016850421261	-351699256342.948\\
0.0675016875421885	-351690089018.226\\
0.0676016900422511	-351680921693.504\\
0.0677016925423136	-351671754368.782\\
0.0678016950423761	-351662587044.06\\
0.0679016975424386	-351653419719.338\\
0.0680017000425011	-351643679436.821\\
0.0681017025425636	-351634512112.099\\
0.0682017050426261	-351625344787.376\\
0.0683017075426886	-351616177462.654\\
0.0684017100427511	-351607010137.932\\
0.0685017125428136	-351597269855.415\\
0.0686017150428761	-351588102530.693\\
0.0687017175429386	-351578935205.971\\
0.0688017200430011	-351569194923.454\\
0.0689017225430636	-351560027598.732\\
0.0690017250431261	-351550287316.214\\
0.0691017275431886	-351541119991.492\\
0.0692017300432511	-351531379708.975\\
0.0693017325433136	-351522212384.253\\
0.0694017350433761	-351512472101.736\\
0.0695017375434386	-351502731819.219\\
0.0696017400435011	-351493564494.496\\
0.0697017425435636	-351483824211.979\\
0.0698017450436261	-351474083929.462\\
0.0699017475436886	-351464343646.945\\
0.0700017500437511	-351455176322.223\\
0.0701017525438136	-351445436039.705\\
0.0702017550438761	-351435695757.188\\
0.0703017575439386	-351425955474.671\\
0.0704017600440011	-351416215192.154\\
0.0705017625440636	-351406474909.636\\
0.0706017650441261	-351396734627.119\\
0.0707017675441886	-351386994344.602\\
0.0708017700442511	-351377254062.085\\
0.0709017725443136	-351367513779.568\\
0.0710017750443761	-351357773497.05\\
0.0711017775444386	-351348033214.533\\
0.0712017800445011	-351337719974.221\\
0.0713017825445636	-351327979691.704\\
0.0714017850446261	-351318239409.186\\
0.0715017875446886	-351307926168.874\\
0.0716017900447511	-351298185886.357\\
0.0717017925448136	-351288445603.84\\
0.0718017950448761	-351278132363.527\\
0.0719017975449386	-351268392081.01\\
0.0720018000450011	-351258078840.698\\
0.0721018025450636	-351248338558.18\\
0.0722018050451261	-351238025317.868\\
0.0723018075451886	-351228285035.351\\
0.0724018100452511	-351217971795.038\\
0.0725018125453136	-351207658554.726\\
0.0726018150453761	-351197918272.209\\
0.0727018175454386	-351187605031.897\\
0.0728018200455011	-351177291791.584\\
0.0729018225455636	-351167551509.067\\
0.0730018250456261	-351157238268.755\\
0.0731018275456886	-351146925028.442\\
0.0732018300457511	-351136611788.13\\
0.0733018325458136	-351126298547.818\\
0.0734018350458761	-351115985307.505\\
0.0735018375459386	-351105672067.193\\
0.0736018400460011	-351095358826.88\\
0.0737018425460637	-351085045586.568\\
0.0738018450461261	-351074732346.256\\
0.0739018475461887	-351064419105.943\\
0.0740018500462511	-351054105865.631\\
0.0741018525463137	-351043792625.319\\
0.0742018550463762	-351032906427.211\\
0.0743018575464387	-351022593186.899\\
0.0744018600465012	-351012279946.586\\
0.0745018625465637	-351001966706.274\\
0.0746018650466262	-350991080508.167\\
0.0747018675466887	-350980767267.854\\
0.0748018700467512	-350969881069.747\\
0.0749018725468137	-350959567829.434\\
0.0750018750468762	-350949254589.122\\
0.0751018775469387	-350938368391.015\\
0.0752018800470012	-350928055150.702\\
0.0753018825470637	-350917168952.595\\
0.0754018850471262	-350906282754.487\\
0.0755018875471887	-350895969514.175\\
0.0756018900472512	-350885083316.067\\
0.0757018925473137	-350874197117.96\\
0.0758018950473762	-350863883877.648\\
0.0759018975474387	-350852997679.54\\
0.0760019000475012	-350842111481.433\\
0.0761019025475637	-350831225283.325\\
0.0762019050476262	-350820339085.218\\
0.0763019075476887	-350810025844.905\\
0.0764019100477512	-350799139646.798\\
0.0765019125478137	-350788253448.69\\
0.0766019150478762	-350777367250.583\\
0.0767019175479387	-350766481052.475\\
0.0768019200480012	-350755594854.368\\
0.0769019225480637	-350744135698.465\\
0.0770019250481262	-350733249500.358\\
0.0771019275481887	-350722363302.25\\
0.0772019300482512	-350711477104.143\\
0.0773019325483137	-350700590906.035\\
0.0774019350483762	-350689704707.928\\
0.0775019375484387	-350678245552.025\\
0.0776019400485012	-350667359353.918\\
0.0777019425485637	-350656473155.81\\
0.0778019450486262	-350645013999.908\\
0.0779019475486887	-350634127801.8\\
0.0780019500487512	-350622668645.898\\
0.0781019525488137	-350611782447.79\\
0.0782019550488762	-350600896249.683\\
0.0783019575489387	-350589437093.78\\
0.0784019600490012	-350577977937.877\\
0.0785019625490637	-350567091739.77\\
0.0786019650491262	-350555632583.867\\
0.0787019675491887	-350544746385.76\\
0.0788019700492512	-350533287229.857\\
0.0789019725493137	-350521828073.955\\
0.0790019750493762	-350510368918.052\\
0.0791019775494387	-350499482719.944\\
0.0792019800495012	-350488023564.042\\
0.0793019825495637	-350476564408.139\\
0.0794019850496262	-350465105252.237\\
0.0795019875496887	-350453646096.334\\
0.0796019900497512	-350442186940.431\\
0.0797019925498137	-350430727784.529\\
0.0798019950498762	-350419268628.626\\
0.0799019975499387	-350407809472.723\\
0.0800020000500013	-350396350316.821\\
0.0801020025500638	-350384891160.918\\
0.0802020050501263	-350373432005.016\\
0.0803020075501888	-350361972849.113\\
0.0804020100502513	-350350513693.21\\
0.0805020125503138	-350338481579.513\\
0.0806020150503763	-350327022423.61\\
0.0807020175504388	-350315563267.707\\
0.0808020200505013	-350303531154.01\\
0.0809020225505638	-350292071998.107\\
0.0810020250506263	-350280612842.204\\
0.0811020275506888	-350268580728.507\\
0.0812020300507513	-350257121572.604\\
0.0813020325508138	-350245089458.906\\
0.0814020350508763	-350233630303.004\\
0.0815020375509388	-350221598189.306\\
0.0816020400510013	-350210139033.403\\
0.0817020425510638	-350198106919.706\\
0.0818020450511263	-350186647763.803\\
0.0819020475511888	-350174615650.105\\
0.0820020500512513	-350162583536.407\\
0.0821020525513138	-350150551422.71\\
0.0822020550513763	-350139092266.807\\
0.0823020575514388	-350127060153.109\\
0.0824020600515013	-350115028039.412\\
0.0825020625515638	-350102995925.714\\
0.0826020650516263	-350090963812.016\\
0.0827020675516888	-350078931698.318\\
0.0828020700517513	-350067472542.416\\
0.0829020725518138	-350055440428.718\\
0.0830020750518763	-350043408315.02\\
0.0831020775519388	-350031376201.323\\
0.0832020800520013	-350018771129.83\\
0.0833020825520638	-350006739016.132\\
0.0834020850521263	-349994706902.434\\
0.0835020875521888	-349982674788.736\\
0.0836020900522513	-349970642675.039\\
0.0837020925523138	-349958610561.341\\
0.0838020950523763	-349946005489.848\\
0.0839020975524388	-349933973376.15\\
0.0840021000525013	-349921941262.453\\
0.0841021025525638	-349909336190.96\\
0.0842021050526263	-349897304077.262\\
0.0843021075526888	-349885271963.564\\
0.0844021100527513	-349872666892.071\\
0.0845021125528138	-349860634778.374\\
0.0846021150528763	-349848029706.881\\
0.0847021175529388	-349835997593.183\\
0.0848021200530013	-349823392521.69\\
0.0849021225530638	-349811360407.992\\
0.0850021250531263	-349798755336.499\\
0.0851021275531888	-349786150265.007\\
0.0852021300532513	-349774118151.309\\
0.0853021325533138	-349761513079.816\\
0.0854021350533763	-349748908008.323\\
0.0855021375534388	-349736302936.83\\
0.0856021400535013	-349724270823.132\\
0.0857021425535638	-349711665751.64\\
0.0858021450536263	-349699060680.147\\
0.0859021475536888	-349686455608.654\\
0.0860021500537514	-349673850537.161\\
0.0861021525538138	-349661245465.668\\
0.0862021550538764	-349648640394.175\\
0.0863021575539388	-349636035322.682\\
0.0864021600540014	-349623430251.189\\
0.0865021625540639	-349610825179.697\\
0.0866021650541264	-349598220108.204\\
0.0867021675541889	-349585615036.711\\
0.0868021700542514	-349573009965.218\\
0.0869021725543139	-349559831935.93\\
0.0870021750543764	-349547226864.437\\
0.0871021775544389	-349534621792.944\\
0.0872021800545014	-349522016721.451\\
0.0873021825545639	-349508838692.163\\
0.0874021850546264	-349496233620.67\\
0.0875021875546889	-349483055591.382\\
0.0876021900547514	-349470450519.889\\
0.0877021925548139	-349457845448.397\\
0.0878021950548764	-349444667419.109\\
0.0879021975549389	-349432062347.616\\
0.0880022000550014	-349418884318.328\\
0.0881022025550639	-349405706289.04\\
0.0882022050551264	-349393101217.547\\
0.0883022075551889	-349379923188.259\\
0.0884022100552514	-349367318116.766\\
0.0885022125553139	-349354140087.478\\
0.0886022150553764	-349340962058.19\\
0.0887022175554389	-349327784028.902\\
0.0888022200555014	-349315178957.409\\
0.0889022225555639	-349302000928.121\\
0.0890022250556264	-349288822898.833\\
0.0891022275556889	-349275644869.545\\
0.0892022300557514	-349262466840.257\\
0.0893022325558139	-349249288810.969\\
0.0894022350558764	-349236110781.681\\
0.0895022375559389	-349222932752.393\\
0.0896022400560014	-349209754723.105\\
0.0897022425560639	-349196576693.817\\
0.0898022450561264	-349183398664.529\\
0.0899022475561889	-349170220635.241\\
0.0900022500562514	-349157042605.953\\
0.0901022525563139	-349143864576.665\\
0.0902022550563764	-349130686547.377\\
0.0903022575564389	-349116935560.294\\
0.0904022600565014	-349103757531.006\\
0.0905022625565639	-349090579501.718\\
0.0906022650566264	-349076828514.635\\
0.0907022675566889	-349063650485.347\\
0.0908022700567514	-349050472456.059\\
0.0909022725568139	-349036721468.975\\
0.0910022750568764	-349023543439.687\\
0.0911022775569389	-349009792452.604\\
0.0912022800570014	-348996614423.316\\
0.0913022825570639	-348982863436.233\\
0.0914022850571264	-348969685406.945\\
0.0915022875571889	-348955934419.862\\
0.0916022900572514	-348942756390.574\\
0.0917022925573139	-348929005403.491\\
0.0918022950573764	-348915254416.408\\
0.0919022975574389	-348901503429.325\\
0.0920023000575014	-348888325400.037\\
0.0921023025575639	-348874574412.953\\
0.0922023050576264	-348860823425.87\\
0.0923023075576889	-348847072438.787\\
0.0924023100577514	-348833321451.704\\
0.092502312557814	-348820143422.416\\
0.0926023150578764	-348806392435.333\\
0.092702317557939	-348792641448.25\\
0.0928023200580014	-348778890461.167\\
0.092902322558064	-348765139474.083\\
0.0930023250581265	-348751388487\\
0.093102327558189	-348737637499.917\\
0.0932023300582515	-348723313555.039\\
0.093302332558314	-348709562567.956\\
0.0934023350583765	-348695811580.873\\
0.093502337558439	-348682060593.789\\
0.0936023400585015	-348668309606.706\\
0.093702342558564	-348653985661.828\\
0.0938023450586265	-348640234674.745\\
0.093902347558689	-348626483687.662\\
0.0940023500587515	-348612732700.579\\
0.094102352558814	-348598408755.7\\
0.0942023550588765	-348584657768.617\\
0.094302357558939	-348570333823.739\\
0.0944023600590015	-348556582836.656\\
0.094502362559064	-348542258891.778\\
0.0946023650591265	-348528507904.694\\
0.094702367559189	-348514183959.816\\
0.0948023700592515	-348500432972.733\\
0.094902372559314	-348486109027.855\\
0.0950023750593765	-348471785082.976\\
0.095102377559439	-348458034095.893\\
0.0952023800595015	-348443710151.015\\
0.095302382559564	-348429386206.137\\
0.0954023850596265	-348415635219.054\\
0.095502387559689	-348401311274.175\\
0.0956023900597515	-348386987329.297\\
0.095702392559814	-348372663384.419\\
0.0958023950598765	-348358339439.541\\
0.095902397559939	-348344015494.662\\
0.0960024000600015	-348329691549.784\\
0.096102402560064	-348315367604.906\\
0.0962024050601265	-348301043660.027\\
0.096302407560189	-348286719715.149\\
0.0964024100602515	-348272395770.271\\
0.096502412560314	-348258071825.393\\
0.0966024150603765	-348243747880.514\\
0.096702417560439	-348229423935.636\\
0.0968024200605015	-348215099990.758\\
0.096902422560564	-348200776045.88\\
0.0970024250606265	-348185879143.206\\
0.097102427560689	-348171555198.328\\
0.0972024300607515	-348157231253.45\\
0.097302432560814	-348142907308.571\\
0.0974024350608765	-348128010405.898\\
0.097502437560939	-348113686461.02\\
0.0976024400610015	-348098789558.346\\
0.097702442561064	-348084465613.468\\
0.0978024450611265	-348070141668.59\\
0.097902447561189	-348055244765.916\\
0.0980024500612515	-348040920821.038\\
0.098102452561314	-348026023918.365\\
0.0982024550613765	-348011127015.691\\
0.098302457561439	-347996803070.813\\
0.0984024600615015	-347981906168.14\\
0.098502462561564	-347967009265.466\\
0.0986024650616265	-347952685320.588\\
0.098702467561689	-347937788417.914\\
0.0988024700617515	-347922891515.241\\
0.098902472561814	-347907994612.568\\
0.0990024750618766	-347893670667.689\\
0.099102477561939	-347878773765.016\\
0.0992024800620016	-347863876862.343\\
0.099302482562064	-347848979959.669\\
0.0994024850621266	-347834083056.996\\
0.0995024875621891	-347819186154.322\\
0.0996024900622516	-347804289251.649\\
0.0997024925623141	-347789392348.976\\
0.0998024950623766	-347774495446.302\\
0.0999024975624391	-347759598543.629\\
0.100002500062502	-347744701640.955\\
0.100102502562564	-347729804738.282\\
0.100202505062627	-347714907835.609\\
0.100302507562689	-347699437975.14\\
0.100402510062752	-347684541072.467\\
0.100502512562814	-347669644169.793\\
0.100602515062877	-347654747267.12\\
0.100702517562939	-347639277406.651\\
0.100802520063002	-347624380503.978\\
0.100902522563064	-347609483601.305\\
0.101002525063127	-347594013740.836\\
0.101102527563189	-347579116838.163\\
0.101202530063252	-347563646977.694\\
0.101302532563314	-347548750075.021\\
0.101402535063377	-347533280214.552\\
0.101502537563439	-347518383311.879\\
0.101602540063502	-347502913451.41\\
0.101702542563564	-347488016548.737\\
0.101802545063627	-347472546688.268\\
0.101902547563689	-347457076827.8\\
0.102002550063752	-347442179925.126\\
0.102102552563814	-347426710064.658\\
0.102202555063877	-347411240204.189\\
0.102302557563939	-347395770343.721\\
0.102402560064002	-347380300483.252\\
0.102502562564064	-347365403580.579\\
0.102602565064127	-347349933720.11\\
0.102702567564189	-347334463859.642\\
0.102802570064252	-347318993999.173\\
0.102902572564314	-347303524138.705\\
0.103002575064377	-347288054278.236\\
0.103102577564439	-347272584417.768\\
0.103202580064502	-347257114557.299\\
0.103302582564564	-347241644696.831\\
0.103402585064627	-347226174836.362\\
0.103502587564689	-347210704975.893\\
0.103602590064752	-347194662157.63\\
0.103702592564814	-347179192297.161\\
0.103802595064877	-347163722436.693\\
0.103902597564939	-347148252576.224\\
0.104002600065002	-347132782715.756\\
0.104102602565064	-347116739897.492\\
0.104202605065127	-347101270037.023\\
0.104302607565189	-347085800176.555\\
0.104402610065252	-347069757358.291\\
0.104502612565314	-347054287497.823\\
0.104602615065377	-347038244679.559\\
0.104702617565439	-347022774819.091\\
0.104802620065502	-347006732000.827\\
0.104902622565564	-346991262140.358\\
0.105002625065627	-346975219322.095\\
0.105102627565689	-346959749461.626\\
0.105202630065752	-346943706643.363\\
0.105302632565814	-346927663825.099\\
0.105402635065877	-346912193964.63\\
0.105502637565939	-346896151146.367\\
0.105602640066002	-346880108328.103\\
0.105702642566064	-346864065509.839\\
0.105802645066127	-346848595649.371\\
0.105902647566189	-346832552831.107\\
0.106002650066252	-346816510012.844\\
0.106102652566314	-346800467194.58\\
0.106202655066377	-346784424376.316\\
0.106302657566439	-346768381558.052\\
0.106402660066502	-346752338739.789\\
0.106502662566564	-346736295921.525\\
0.106602665066627	-346720253103.261\\
0.106702667566689	-346704210284.998\\
0.106802670066752	-346688167466.734\\
0.106902672566814	-346672124648.471\\
0.107002675066877	-346656081830.207\\
0.107102677566939	-346640039011.943\\
0.107202680067002	-346623996193.68\\
0.107302682567064	-346607380417.621\\
0.107402685067127	-346591337599.357\\
0.107502687567189	-346575294781.093\\
0.107602690067252	-346559251962.83\\
0.107702692567314	-346542636186.771\\
0.107802695067377	-346526593368.507\\
0.107902697567439	-346509977592.448\\
0.108002700067502	-346493934774.185\\
0.108102702567564	-346477891955.921\\
0.108202705067627	-346461276179.862\\
0.108302707567689	-346445233361.599\\
0.108402710067752	-346428617585.54\\
0.108502712567814	-346412001809.481\\
0.108602715067877	-346395958991.217\\
0.108702717567939	-346379343215.159\\
0.108802720068002	-346363300396.895\\
0.108902722568064	-346346684620.836\\
0.109002725068127	-346330068844.777\\
0.109102727568189	-346313453068.719\\
0.109202730068252	-346297410250.455\\
0.109302732568314	-346280794474.396\\
0.109402735068377	-346264178698.337\\
0.109502737568439	-346247562922.279\\
0.109602740068502	-346230947146.22\\
0.109702742568564	-346214331370.161\\
0.109802745068627	-346197715594.102\\
0.109902747568689	-346181099818.043\\
0.110002750068752	-346164484041.985\\
0.110102752568814	-346147868265.926\\
0.110202755068877	-346131252489.867\\
0.110302757568939	-346114636713.808\\
0.110402760069002	-346098020937.749\\
0.110502762569064	-346081405161.691\\
0.110602765069127	-346064789385.632\\
0.110702767569189	-346048173609.573\\
0.110802770069252	-346031557833.514\\
0.110902772569314	-346014369099.66\\
0.111002775069377	-345997753323.602\\
0.111102777569439	-345981137547.543\\
0.111202780069502	-345963948813.689\\
0.111302782569564	-345947333037.63\\
0.111402785069627	-345930717261.571\\
0.111502787569689	-345913528527.717\\
0.111602790069752	-345896912751.659\\
0.111702792569814	-345879724017.805\\
0.111802795069877	-345863108241.746\\
0.111902797569939	-345845919507.892\\
0.112002800070002	-345829303731.833\\
0.112102802570064	-345812114997.979\\
0.112202805070127	-345794926264.125\\
0.112302807570189	-345778310488.066\\
0.112402810070252	-345761121754.213\\
0.112502812570314	-345743933020.359\\
0.112602815070377	-345727317244.3\\
0.112702817570439	-345710128510.446\\
0.112802820070502	-345692939776.592\\
0.112902822570564	-345675751042.738\\
0.113002825070627	-345658562308.884\\
0.113102827570689	-345641946532.825\\
0.113202830070752	-345624757798.971\\
0.113302832570814	-345607569065.117\\
0.113402835070877	-345590380331.263\\
0.113502837570939	-345573191597.41\\
0.113602840071002	-345556002863.556\\
0.113702842571064	-345538814129.702\\
0.113802845071127	-345521625395.848\\
0.113902847571189	-345504436661.994\\
0.114002850071252	-345486674970.345\\
0.114102852571314	-345469486236.491\\
0.114202855071377	-345452297502.637\\
0.114302857571439	-345435108768.783\\
0.114402860071502	-345417920034.929\\
0.114502862571564	-345400158343.28\\
0.114602865071627	-345382969609.426\\
0.114702867571689	-345365780875.572\\
0.114802870071752	-345348019183.923\\
0.114902872571814	-345330830450.069\\
0.115002875071877	-345313641716.215\\
0.115102877571939	-345295880024.566\\
0.115202880072002	-345278691290.712\\
0.115302882572064	-345260929599.063\\
0.115402885072127	-345243740865.209\\
0.115502887572189	-345225979173.56\\
0.115602890072252	-345208790439.706\\
0.115702892572314	-345191028748.057\\
0.115802895072377	-345173267056.408\\
0.115902897572439	-345156078322.554\\
0.116002900072502	-345138316630.905\\
0.116102902572564	-345120554939.256\\
0.116202905072627	-345102793247.607\\
0.116302907572689	-345085604513.753\\
0.116402910072752	-345067842822.104\\
0.116502912572814	-345050081130.455\\
0.116602915072877	-345032319438.806\\
0.116702917572939	-345014557747.157\\
0.116802920073002	-344996796055.508\\
0.116902922573064	-344979034363.859\\
0.117002925073127	-344961272672.21\\
0.117102927573189	-344943510980.561\\
0.117202930073252	-344925749288.912\\
0.117302932573314	-344907987597.263\\
0.117402935073377	-344890225905.614\\
0.117502937573439	-344872464213.965\\
0.117602940073502	-344854702522.316\\
0.117702942573564	-344836940830.667\\
0.117802945073627	-344819179139.017\\
0.117902947573689	-344800844489.573\\
0.118002950073752	-344783082797.924\\
0.118102952573814	-344765321106.275\\
0.118202955073877	-344747559414.626\\
0.118302957573939	-344729224765.182\\
0.118402960074002	-344711463073.533\\
0.118502962574064	-344693701381.884\\
0.118602965074127	-344675366732.44\\
0.118702967574189	-344657605040.791\\
0.118802970074252	-344639270391.346\\
0.118902972574314	-344621508699.697\\
0.119002975074377	-344603174050.253\\
0.119102977574439	-344585412358.604\\
0.119202980074502	-344567077709.16\\
0.119302982574564	-344548743059.716\\
0.119402985074627	-344530981368.067\\
0.119502987574689	-344512646718.622\\
0.119602990074752	-344494312069.178\\
0.119702992574814	-344476550377.529\\
0.119802995074877	-344458215728.085\\
0.119902997574939	-344439881078.641\\
0.120003000075002	-344421546429.197\\
0.120103002575064	-344403211779.752\\
0.120203005075127	-344385450088.103\\
0.120303007575189	-344367115438.659\\
0.120403010075252	-344348780789.215\\
0.120503012575314	-344330446139.771\\
0.120603015075377	-344312111490.327\\
0.120703017575439	-344293776840.882\\
0.120803020075502	-344275442191.438\\
0.120903022575564	-344257107541.994\\
0.121003025075627	-344238772892.55\\
0.121103027575689	-344220438243.106\\
0.121203030075752	-344201530635.866\\
0.121303032575814	-344183195986.422\\
0.121403035075877	-344164861336.978\\
0.121503037575939	-344146526687.534\\
0.121603040076002	-344128192038.09\\
0.121703042576064	-344109284430.85\\
0.121803045076127	-344090949781.406\\
0.121903047576189	-344072615131.962\\
0.122003050076252	-344053707524.723\\
0.122103052576314	-344035372875.278\\
0.122203055076377	-344016465268.039\\
0.122303057576439	-343998130618.595\\
0.122403060076502	-343979795969.151\\
0.122503062576564	-343960888361.911\\
0.122603065076627	-343942553712.467\\
0.122703067576689	-343923646105.228\\
0.122803070076752	-343904738497.989\\
0.122903072576814	-343886403848.544\\
0.123003075076877	-343867496241.305\\
0.123103077576939	-343848588634.066\\
0.123203080077002	-343830253984.622\\
0.123303082577064	-343811346377.382\\
0.123403085077127	-343792438770.143\\
0.123503087577189	-343773531162.904\\
0.123603090077252	-343755196513.459\\
0.123703092577314	-343736288906.22\\
0.123803095077377	-343717381298.981\\
0.123903097577439	-343698473691.742\\
0.124003100077502	-343679566084.502\\
0.124103102577564	-343660658477.263\\
0.124203105077627	-343641750870.024\\
0.124303107577689	-343622843262.784\\
0.124403110077752	-343603935655.545\\
0.124503112577814	-343585028048.306\\
0.124603115077877	-343566120441.066\\
0.124703117577939	-343547212833.827\\
0.124803120078002	-343528305226.588\\
0.124903122578064	-343509397619.348\\
0.125003125078127	-343490490012.109\\
0.125103127578189	-343471009447.075\\
0.125203130078252	-343452101839.835\\
0.125303132578314	-343433194232.596\\
0.125403135078377	-343413713667.562\\
0.125503137578439	-343394806060.322\\
0.125603140078502	-343375898453.083\\
0.125703142578564	-343356417888.048\\
0.125803145078627	-343337510280.809\\
0.125903147578689	-343318602673.57\\
0.126003150078752	-343299122108.535\\
0.126103152578814	-343280214501.296\\
0.126203155078877	-343260733936.262\\
0.126303157578939	-343241826329.022\\
0.126403160079002	-343222345763.988\\
0.126503162579064	-343202865198.953\\
0.126603165079127	-343183957591.714\\
0.126703167579189	-343164477026.68\\
0.126803170079252	-343144996461.645\\
0.126903172579314	-343126088854.406\\
0.127003175079377	-343106608289.371\\
0.127103177579439	-343087127724.337\\
0.127203180079502	-343067647159.302\\
0.127303182579564	-343048739552.063\\
0.127403185079627	-343029258987.029\\
0.127503187579689	-343009778421.994\\
0.127603190079752	-342990297856.96\\
0.127703192579814	-342970817291.925\\
0.127803195079877	-342951336726.891\\
0.12790319757994	-342931856161.857\\
0.128003200080002	-342912375596.822\\
0.128103202580064	-342892895031.788\\
0.128203205080127	-342873414466.753\\
0.12830320758019	-342853933901.719\\
0.128403210080252	-342834453336.684\\
0.128503212580314	-342814972771.65\\
0.128603215080377	-342794919248.82\\
0.12870321758044	-342775438683.786\\
0.128803220080502	-342755958118.751\\
0.128903222580565	-342736477553.717\\
0.129003225080627	-342716424030.887\\
0.12910322758069	-342696943465.853\\
0.129203230080752	-342677462900.818\\
0.129303232580815	-342657409377.989\\
0.129403235080877	-342637928812.954\\
0.12950323758094	-342618448247.92\\
0.129603240081002	-342598394725.09\\
0.129703242581065	-342578914160.056\\
0.129803245081127	-342558860637.226\\
0.12990324758119	-342539380072.192\\
0.130003250081252	-342519326549.362\\
0.130103252581315	-342499273026.533\\
0.130203255081377	-342479792461.498\\
0.13030325758144	-342459738938.669\\
0.130403260081502	-342440258373.634\\
0.130503262581565	-342420204850.805\\
0.130603265081627	-342400151327.975\\
0.13070326758169	-342380097805.146\\
0.130803270081752	-342360617240.111\\
0.130903272581815	-342340563717.281\\
0.131003275081877	-342320510194.452\\
0.13110327758194	-342300456671.622\\
0.131203280082002	-342280403148.793\\
0.131303282582065	-342260349625.963\\
0.131403285082127	-342240296103.134\\
0.13150328758219	-342220242580.304\\
0.131603290082252	-342200189057.474\\
0.131703292582315	-342180135534.645\\
0.131803295082377	-342160082011.815\\
0.13190329758244	-342140028488.986\\
0.132003300082502	-342119974966.156\\
0.132103302582565	-342099921443.327\\
0.132203305082627	-342079867920.497\\
0.13230330758269	-342059814397.667\\
0.132403310082752	-342039187917.043\\
0.132503312582815	-342019134394.213\\
0.132603315082877	-341999080871.384\\
0.13270331758294	-341979027348.554\\
0.132803320083002	-341958400867.929\\
0.132903322583065	-341938347345.1\\
0.133003325083127	-341918293822.27\\
0.13310332758319	-341897667341.645\\
0.133203330083252	-341877613818.816\\
0.133303332583315	-341856987338.191\\
0.133403335083377	-341836933815.362\\
0.13350333758344	-341816307334.737\\
0.133603340083502	-341796253811.907\\
0.133703342583565	-341775627331.282\\
0.133803345083627	-341755573808.453\\
0.13390334758369	-341734947327.828\\
0.134003350083752	-341714893804.999\\
0.134103352583815	-341694267324.374\\
0.134203355083877	-341673640843.749\\
0.13430335758394	-341653014363.125\\
0.134403360084002	-341632960840.295\\
0.134503362584065	-341612334359.67\\
0.134603365084127	-341591707879.046\\
0.13470336758419	-341571081398.421\\
0.134803370084252	-341550454917.796\\
0.134903372584315	-341530401394.966\\
0.135003375084377	-341509774914.342\\
0.13510337758444	-341489148433.717\\
0.135203380084502	-341468521953.092\\
0.135303382584565	-341447895472.468\\
0.135403385084627	-341427268991.843\\
0.13550338758469	-341406642511.218\\
0.135603390084752	-341386016030.594\\
0.135703392584815	-341365389549.969\\
0.135803395084877	-341344190111.549\\
0.13590339758494	-341323563630.924\\
0.136003400085002	-341302937150.3\\
0.136103402585065	-341282310669.675\\
0.136203405085127	-341261684189.05\\
0.13630340758519	-341240484750.63\\
0.136403410085252	-341219858270.006\\
0.136503412585315	-341199231789.381\\
0.136603415085377	-341178032350.961\\
0.13670341758544	-341157405870.336\\
0.136803420085502	-341136779389.712\\
0.136903422585565	-341115579951.292\\
0.137003425085627	-341094953470.667\\
0.13710342758569	-341073754032.247\\
0.137203430085752	-341053127551.623\\
0.137303432585815	-341031928113.203\\
0.137403435085877	-341011301632.578\\
0.13750343758594	-340990102194.158\\
0.137603440086002	-340969475713.533\\
0.137703442586065	-340948276275.114\\
0.137803445086127	-340927076836.694\\
0.13790344758619	-340906450356.069\\
0.138003450086252	-340885250917.649\\
0.138103452586315	-340864051479.229\\
0.138203455086377	-340842852040.81\\
0.13830345758644	-340822225560.185\\
0.138403460086502	-340801026121.765\\
0.138503462586565	-340779826683.345\\
0.138603465086627	-340758627244.925\\
0.13870346758669	-340737427806.505\\
0.138803470086752	-340716228368.086\\
0.138903472586815	-340695028929.666\\
0.139003475086877	-340673829491.246\\
0.13910347758694	-340652630052.826\\
0.139203480087002	-340631430614.406\\
0.139303482587065	-340610231175.986\\
0.139403485087127	-340589031737.567\\
0.13950348758719	-340567832299.147\\
0.139603490087252	-340546632860.727\\
0.139703492587315	-340525433422.307\\
0.139803495087377	-340504233983.887\\
0.13990349758744	-340482461587.672\\
0.140003500087502	-340461262149.252\\
0.140103502587565	-340440062710.833\\
0.140203505087627	-340418863272.413\\
0.14030350758769	-340397090876.198\\
0.140403510087752	-340375891437.778\\
0.140503512587815	-340354691999.358\\
0.140603515087877	-340332919603.143\\
0.14070351758794	-340311720164.723\\
0.140803520088002	-340289947768.508\\
0.140903522588065	-340268748330.088\\
0.141003525088127	-340246975933.873\\
0.14110352758819	-340225776495.454\\
0.141203530088252	-340204004099.239\\
0.141303532588315	-340182804660.819\\
0.141403535088377	-340161032264.604\\
0.14150353758844	-340139259868.389\\
0.141603540088502	-340118060429.969\\
0.141703542588565	-340096288033.754\\
0.141803545088627	-340074515637.539\\
0.14190354758869	-340053316199.119\\
0.142003550088752	-340031543802.904\\
0.142103552588815	-340009771406.689\\
0.142203555088877	-339987999010.474\\
0.14230355758894	-339966226614.259\\
0.142403560089002	-339944454218.044\\
0.142503562589065	-339923254779.625\\
0.142603565089127	-339901482383.41\\
0.14270356758919	-339879709987.195\\
0.142803570089252	-339857937590.98\\
0.142903572589315	-339836165194.765\\
0.143003575089377	-339814392798.55\\
0.14310357758944	-339792620402.335\\
0.143203580089502	-339770275048.325\\
0.143303582589565	-339748502652.11\\
0.143403585089627	-339726730255.895\\
0.14350358758969	-339704957859.68\\
0.143603590089752	-339683185463.465\\
0.143703592589815	-339661413067.25\\
0.143803595089877	-339639067713.24\\
0.14390359758994	-339617295317.025\\
0.144003600090002	-339595522920.81\\
0.144103602590065	-339573177566.8\\
0.144203605090127	-339551405170.585\\
0.14430360759019	-339529632774.37\\
0.144403610090252	-339507287420.36\\
0.144503612590315	-339485515024.145\\
0.144603615090377	-339463169670.135\\
0.14470361759044	-339441397273.919\\
0.144803620090502	-339419051919.909\\
0.144903622590565	-339397279523.694\\
0.145003625090627	-339374934169.684\\
0.14510362759069	-339353161773.469\\
0.145203630090752	-339330816419.459\\
0.145303632590815	-339309044023.244\\
0.145403635090877	-339286698669.234\\
0.14550363759094	-339264353315.224\\
0.145603640091002	-339242007961.214\\
0.145703642591065	-339220235564.999\\
0.145803645091127	-339197890210.989\\
0.14590364759119	-339175544856.979\\
0.146003650091252	-339153199502.969\\
0.146103652591315	-339131427106.754\\
0.146203655091377	-339109081752.744\\
0.14630365759144	-339086736398.734\\
0.146403660091502	-339064391044.723\\
0.146503662591565	-339042045690.713\\
0.146603665091627	-339019700336.703\\
0.14670366759169	-338997354982.693\\
0.146803670091752	-338975009628.683\\
0.146903672591815	-338952664274.673\\
0.147003675091877	-338930318920.663\\
0.14710367759194	-338907973566.653\\
0.147203680092002	-338885055254.847\\
0.147303682592065	-338862709900.837\\
0.147403685092127	-338840364546.827\\
0.14750368759219	-338818019192.817\\
0.147603690092252	-338795673838.807\\
0.147703692592315	-338772755527.002\\
0.147803695092377	-338750410172.992\\
0.14790369759244	-338728064818.982\\
0.148003700092502	-338705146507.176\\
0.148103702592565	-338682801153.166\\
0.148203705092627	-338660455799.156\\
0.14830370759269	-338637537487.351\\
0.148403710092752	-338615192133.341\\
0.148503712592815	-338592273821.536\\
0.148603715092877	-338569928467.526\\
0.14870371759294	-338547010155.72\\
0.148803720093002	-338524664801.71\\
0.148903722593065	-338501746489.905\\
0.149003725093127	-338479401135.895\\
0.14910372759319	-338456482824.09\\
0.149203730093252	-338433564512.284\\
0.149303732593315	-338411219158.274\\
0.149403735093377	-338388300846.469\\
0.14950373759344	-338365382534.664\\
0.149603740093502	-338342464222.859\\
0.149703742593565	-338320118868.849\\
0.149803745093627	-338297200557.043\\
0.14990374759369	-338274282245.238\\
0.150003750093752	-338251363933.433\\
0.150103752593815	-338228445621.628\\
0.150203755093877	-338205527309.822\\
0.15030375759394	-338182608998.017\\
0.150403760094002	-338159690686.212\\
0.150503762594065	-338137345332.202\\
0.150603765094127	-338113854062.601\\
0.15070376759419	-338090935750.796\\
0.150803770094252	-338068017438.991\\
0.150903772594315	-338045099127.186\\
0.151003775094377	-338022180815.38\\
0.15110377759444	-337999262503.575\\
0.151203780094502	-337976344191.77\\
0.151303782594565	-337953425879.965\\
0.151403785094627	-337929934610.364\\
0.15150378759469	-337907016298.559\\
0.151603790094752	-337884097986.754\\
0.151703792594815	-337861179674.949\\
0.151803795094877	-337837688405.348\\
0.15190379759494	-337814770093.543\\
0.152003800095002	-337791851781.738\\
0.152103802595065	-337768360512.138\\
0.152203805095127	-337745442200.332\\
0.15230380759519	-337721950930.732\\
0.152403810095252	-337699032618.927\\
0.152503812595315	-337675541349.326\\
0.152603815095377	-337652623037.521\\
0.15270381759544	-337629131767.921\\
0.152803820095502	-337606213456.115\\
0.152903822595565	-337582722186.515\\
0.153003825095627	-337559230916.915\\
0.15310382759569	-337536312605.11\\
0.153203830095752	-337512821335.509\\
0.153303832595815	-337489330065.909\\
0.153403835095877	-337466411754.104\\
0.15350383759594	-337442920484.503\\
0.153603840096002	-337419429214.903\\
0.153703842596065	-337395937945.302\\
0.153803845096127	-337372446675.702\\
0.15390384759619	-337349528363.897\\
0.154003850096252	-337326037094.297\\
0.154103852596315	-337302545824.696\\
0.154203855096377	-337279054555.096\\
0.15430385759644	-337255563285.495\\
0.154403860096502	-337232072015.895\\
0.154503862596565	-337208580746.295\\
0.154603865096627	-337185089476.694\\
0.15470386759669	-337161598207.094\\
0.154803870096752	-337138106937.494\\
0.154903872596815	-337114042710.098\\
0.155003875096877	-337090551440.498\\
0.15510387759694	-337067060170.897\\
0.155203880097002	-337043568901.297\\
0.155303882597065	-337020077631.697\\
0.155403885097127	-336996586362.096\\
0.15550388759719	-336972522134.701\\
0.155603890097252	-336949030865.1\\
0.155703892597315	-336925539595.5\\
0.155803895097377	-336901475368.105\\
0.15590389759744	-336877984098.504\\
0.156003900097502	-336854492828.904\\
0.156103902597565	-336830428601.508\\
0.156203905097627	-336806937331.908\\
0.15630390759769	-336782873104.513\\
0.156403910097752	-336759381834.912\\
0.156503912597815	-336735317607.517\\
0.156603915097877	-336711826337.916\\
0.15670391759794	-336687762110.521\\
0.156803920098002	-336663697883.125\\
0.156903922598065	-336640206613.525\\
0.157003925098127	-336616142386.129\\
0.15710392759819	-336592078158.734\\
0.157203930098252	-336568586889.134\\
0.157303932598315	-336544522661.738\\
0.157403935098377	-336520458434.343\\
0.15750393759844	-336496394206.947\\
0.157603940098502	-336472902937.347\\
0.157703942598565	-336448838709.951\\
0.157803945098627	-336424774482.556\\
0.15790394759869	-336400710255.16\\
0.158003950098752	-336376646027.765\\
0.158103952598815	-336352581800.369\\
0.158203955098877	-336328517572.974\\
0.15830395759894	-336304453345.578\\
0.158403960099002	-336280389118.183\\
0.158503962599065	-336256324890.787\\
0.158603965099127	-336232260663.392\\
0.15870396759919	-336208196435.996\\
0.158803970099252	-336184132208.601\\
0.158903972599315	-336160067981.205\\
0.159003975099377	-336136003753.81\\
0.15910397759944	-336111939526.414\\
0.159203980099502	-336087302341.224\\
0.159303982599565	-336063238113.828\\
0.159403985099627	-336039173886.433\\
0.15950398759969	-336015109659.037\\
0.159603990099752	-335990472473.847\\
0.159703992599815	-335966408246.451\\
0.159803995099877	-335942344019.056\\
0.15990399759994	-335917706833.865\\
0.160004000100003	-335893642606.469\\
0.160104002600065	-335869005421.279\\
0.160204005100128	-335844941193.883\\
0.16030400760019	-335820304008.693\\
0.160404010100253	-335796239781.297\\
0.160504012600315	-335771602596.107\\
0.160604015100378	-335747538368.711\\
0.16070401760044	-335722901183.521\\
0.160804020100503	-335698836956.125\\
0.160904022600565	-335674199770.934\\
0.161004025100628	-335649562585.744\\
0.16110402760069	-335625498358.348\\
0.161204030100753	-335600861173.158\\
0.161304032600815	-335576223987.967\\
0.161404035100878	-335551586802.776\\
0.16150403760094	-335527522575.381\\
0.161604040101003	-335502885390.19\\
0.161704042601065	-335478248205\\
0.161804045101128	-335453611019.809\\
0.16190404760119	-335428973834.618\\
0.162004050101253	-335404336649.428\\
0.162104052601315	-335379699464.237\\
0.162204055101378	-335355062279.047\\
0.16230405760144	-335330425093.856\\
0.162404060101503	-335305787908.665\\
0.162504062601565	-335281150723.475\\
0.162604065101628	-335256513538.284\\
0.16270406760169	-335231876353.093\\
0.162804070101753	-335207239167.903\\
0.162904072601815	-335182601982.712\\
0.163004075101878	-335157964797.521\\
0.16310407760194	-335133327612.331\\
0.163204080102003	-335108117469.345\\
0.163304082602065	-335083480284.154\\
0.163404085102128	-335058843098.964\\
0.16350408760219	-335034205913.773\\
0.163604090102253	-335008995770.787\\
0.163704092602315	-334984358585.597\\
0.163804095102378	-334959148442.611\\
0.16390409760244	-334934511257.42\\
0.164004100102503	-334909874072.23\\
0.164104102602565	-334884663929.244\\
0.164204105102628	-334860026744.053\\
0.16430410760269	-334834816601.068\\
0.164404110102753	-334810179415.877\\
0.164504112602815	-334784969272.891\\
0.164604115102878	-334760332087.701\\
0.16470411760294	-334735121944.715\\
0.164804120103003	-334709911801.729\\
0.164904122603065	-334685274616.539\\
0.165004125103128	-334660064473.553\\
0.16510412760319	-334634854330.567\\
0.165204130103253	-334610217145.376\\
0.165304132603315	-334585007002.391\\
0.165404135103378	-334559796859.405\\
0.16550413760344	-334535159674.214\\
0.165604140103503	-334509949531.229\\
0.165704142603565	-334484739388.243\\
0.165804145103628	-334459529245.257\\
0.16590414760369	-334434319102.271\\
0.166004150103753	-334409108959.286\\
0.166104152603815	-334383898816.3\\
0.166204155103878	-334358688673.314\\
0.16630415760394	-334333478530.328\\
0.166404160104003	-334308268387.342\\
0.166504162604065	-334283058244.357\\
0.166604165104128	-334257848101.371\\
0.16670416760419	-334232637958.385\\
0.166804170104253	-334207427815.399\\
0.166904172604315	-334182217672.414\\
0.167004175104378	-334157007529.428\\
0.16710417760444	-334131797386.442\\
0.167204180104503	-334106014285.661\\
0.167304182604565	-334080804142.676\\
0.167404185104628	-334055593999.69\\
0.16750418760469	-334030383856.704\\
0.167604190104753	-334004600755.923\\
0.167704192604815	-333979390612.937\\
0.167804195104878	-333954180469.952\\
0.16790419760494	-333928397369.171\\
0.168004200105003	-333903187226.185\\
0.168104202605065	-333877404125.404\\
0.168204205105128	-333852193982.418\\
0.16830420760519	-333826410881.637\\
0.168404210105253	-333801200738.652\\
0.168504212605315	-333775417637.871\\
0.168604215105378	-333750207494.885\\
0.16870421760544	-333724424394.104\\
0.168804220105503	-333699214251.118\\
0.168904222605565	-333673431150.338\\
0.169004225105628	-333647648049.557\\
0.16910422760569	-333622437906.571\\
0.169204230105753	-333596654805.79\\
0.169304232605815	-333570871705.009\\
0.169404235105878	-333545661562.023\\
0.16950423760594	-333519878461.242\\
0.169604240106003	-333494095360.462\\
0.169704242606065	-333468312259.681\\
0.169804245106128	-333442529158.9\\
0.16990424760619	-333416746058.119\\
0.170004250106253	-333391535915.133\\
0.170104252606315	-333365752814.352\\
0.170204255106378	-333339969713.571\\
0.17030425760644	-333314186612.791\\
0.170404260106503	-333288403512.01\\
0.170504262606565	-333262620411.229\\
0.170604265106628	-333236837310.448\\
0.17070426760669	-333211054209.667\\
0.170804270106753	-333185271108.886\\
0.170904272606815	-333158915050.31\\
0.171004275106878	-333133131949.529\\
0.17110427760694	-333107348848.748\\
0.171204280107003	-333081565747.967\\
0.171304282607065	-333055782647.187\\
0.171404285107128	-333029426588.61\\
0.17150428760719	-333003643487.83\\
0.171604290107253	-332977860387.049\\
0.171704292607315	-332952077286.268\\
0.171804295107378	-332925721227.692\\
0.17190429760744	-332899938126.911\\
0.172004300107503	-332874155026.13\\
0.172104302607565	-332847798967.554\\
0.172204305107628	-332822015866.773\\
0.17230430760769	-332795659808.197\\
0.172404310107753	-332769876707.416\\
0.172504312607815	-332743520648.84\\
0.172604315107878	-332717737548.059\\
0.17270431760794	-332691381489.483\\
0.172804320108003	-332665598388.702\\
0.172904322608065	-332639242330.126\\
0.173004325108128	-332612886271.55\\
0.17310432760819	-332587103170.769\\
0.173204330108253	-332560747112.193\\
0.173304332608315	-332534391053.617\\
0.173404335108378	-332508607952.837\\
0.17350433760844	-332482251894.261\\
0.173604340108503	-332455895835.685\\
0.173704342608565	-332429539777.109\\
0.173804345108628	-332403756676.328\\
0.17390434760869	-332377400617.752\\
0.174004350108753	-332351044559.176\\
0.174104352608815	-332324688500.6\\
0.174204355108878	-332298332442.024\\
0.17430435760894	-332271976383.448\\
0.174404360109003	-332245620324.872\\
0.174504362609065	-332219264266.295\\
0.174604365109128	-332192908207.719\\
0.17470436760919	-332166552149.143\\
0.174804370109253	-332140196090.567\\
0.174904372609315	-332113840031.991\\
0.175004375109378	-332087483973.415\\
0.17510437760944	-332061127914.839\\
0.175204380109503	-332034771856.263\\
0.175304382609565	-332007842839.892\\
0.175404385109628	-331981486781.316\\
0.17550438760969	-331955130722.74\\
0.175604390109753	-331928774664.164\\
0.175704392609815	-331901845647.793\\
0.175804395109878	-331875489589.217\\
0.17590439760994	-331849133530.641\\
0.176004400110003	-331822204514.27\\
0.176104402610065	-331795848455.694\\
0.176204405110128	-331769492397.118\\
0.17630440761019	-331742563380.747\\
0.176404410110253	-331716207322.171\\
0.176504412610315	-331689278305.799\\
0.176604415110378	-331662922247.223\\
0.17670441761044	-331635993230.852\\
0.176804420110503	-331609637172.276\\
0.176904422610565	-331582708155.905\\
0.177004425110628	-331556352097.329\\
0.17710442761069	-331529423080.958\\
0.177204430110753	-331502494064.587\\
0.177304432610815	-331476138006.011\\
0.177404435110878	-331449208989.64\\
0.17750443761094	-331422279973.268\\
0.177604440111003	-331395923914.693\\
0.177704442611065	-331368994898.321\\
0.177804445111128	-331342065881.95\\
0.17790444761119	-331315136865.579\\
0.178004450111253	-331288207849.208\\
0.178104452611315	-331261851790.632\\
0.178204455111378	-331234922774.261\\
0.17830445761144	-331207993757.89\\
0.178404460111503	-331181064741.518\\
0.178504462611565	-331154135725.147\\
0.178604465111628	-331127206708.776\\
0.17870446761169	-331100277692.405\\
0.178804470111753	-331073348676.034\\
0.178904472611815	-331046419659.663\\
0.179004475111878	-331019490643.292\\
0.17910447761194	-330992561626.92\\
0.179204480112003	-330965632610.549\\
0.179304482612065	-330938130636.383\\
0.179404485112128	-330911201620.012\\
0.17950448761219	-330884272603.641\\
0.179604490112253	-330857343587.27\\
0.179704492612315	-330830414570.898\\
0.179804495112378	-330802912596.732\\
0.17990449761244	-330775983580.361\\
0.180004500112503	-330749054563.99\\
0.180104502612565	-330721552589.824\\
0.180204505112628	-330694623573.452\\
0.18030450761269	-330667694557.081\\
0.180404510112753	-330640192582.915\\
0.180504512612815	-330613263566.544\\
0.180604515112878	-330585761592.378\\
0.18070451761294	-330558832576.006\\
0.180804520113003	-330531903559.635\\
0.180904522613065	-330504401585.469\\
0.181004525113128	-330476899611.303\\
0.18110452761319	-330449970594.932\\
0.181204530113253	-330422468620.765\\
0.181304532613315	-330395539604.394\\
0.181404535113378	-330368037630.228\\
0.18150453761344	-330340535656.062\\
0.181604540113503	-330313606639.69\\
0.181704542613565	-330286104665.524\\
0.181804545113628	-330258602691.358\\
0.18190454761369	-330231100717.192\\
0.182004550113753	-330204171700.82\\
0.182104552613815	-330176669726.654\\
0.182204555113878	-330149167752.488\\
0.18230455761394	-330121665778.322\\
0.182404560114003	-330094163804.155\\
0.182504562614065	-330066661829.989\\
0.182604565114128	-330039732813.618\\
0.18270456761419	-330012230839.452\\
0.182804570114253	-329984728865.285\\
0.182904572614315	-329957226891.119\\
0.183004575114378	-329929724916.953\\
0.18310457761444	-329902222942.786\\
0.183204580114503	-329874720968.62\\
0.183304582614565	-329846646036.659\\
0.183404585114628	-329819144062.492\\
0.18350458761469	-329791642088.326\\
0.183604590114753	-329764140114.16\\
0.183704592614815	-329736638139.994\\
0.183804595114878	-329709136165.827\\
0.18390459761494	-329681061233.866\\
0.184004600115003	-329653559259.7\\
0.184104602615065	-329626057285.533\\
0.184204605115128	-329598555311.367\\
0.18430460761519	-329570480379.406\\
0.184404610115253	-329542978405.239\\
0.184504612615315	-329515476431.073\\
0.184604615115378	-329487401499.112\\
0.18470461761544	-329459899524.945\\
0.184804620115503	-329431824592.984\\
0.184904622615565	-329404322618.818\\
0.185004625115628	-329376247686.856\\
0.18510462761569	-329348745712.69\\
0.185204630115753	-329320670780.729\\
0.185304632615815	-329293168806.562\\
0.185404635115878	-329265093874.601\\
0.18550463761594	-329237591900.435\\
0.185604640116003	-329209516968.473\\
0.185704642616065	-329181442036.512\\
0.185804645116128	-329153940062.346\\
0.18590464761619	-329125865130.384\\
0.186004650116253	-329097790198.423\\
0.186104652616315	-329070288224.256\\
0.186204655116378	-329042213292.295\\
0.18630465761644	-329014138360.334\\
0.186404660116503	-328986063428.372\\
0.186504662616565	-328957988496.411\\
0.186604665116628	-328930486522.245\\
0.18670466761669	-328902411590.283\\
0.186804670116753	-328874336658.322\\
0.186904672616815	-328846261726.36\\
0.187004675116878	-328818186794.399\\
0.18710467761694	-328790111862.438\\
0.187204680117003	-328762036930.476\\
0.187304682617065	-328733961998.515\\
0.187404685117128	-328705887066.553\\
0.18750468761719	-328677812134.592\\
0.187604690117253	-328649737202.63\\
0.187704692617315	-328621662270.669\\
0.187804695117378	-328593014380.913\\
0.18790469761744	-328564939448.951\\
0.188004700117503	-328536864516.99\\
0.188104702617565	-328508789585.028\\
0.188204705117628	-328480714653.067\\
0.18830470761769	-328452066763.31\\
0.188404710117753	-328423991831.349\\
0.188504712617815	-328395916899.388\\
0.188604715117878	-328367841967.426\\
0.18870471761794	-328339194077.67\\
0.188804720118003	-328311119145.708\\
0.188904722618065	-328282471255.952\\
0.189004725118128	-328254396323.99\\
0.18910472761819	-328226321392.029\\
0.189204730118253	-328197673502.272\\
0.189304732618315	-328169598570.311\\
0.189404735118378	-328140950680.554\\
0.18950473761844	-328112875748.593\\
0.189604740118503	-328084227858.836\\
0.189704742618565	-328055579969.08\\
0.189804745118628	-328027505037.118\\
0.18990474761869	-327998857147.362\\
0.190004750118753	-327970782215.4\\
0.190104752618815	-327942134325.644\\
0.190204755118878	-327913486435.887\\
0.19030475761894	-327885411503.926\\
0.190404760119003	-327856763614.169\\
0.190504762619065	-327828115724.413\\
0.190604765119128	-327799467834.656\\
0.19070476761919	-327771392902.695\\
0.190804770119253	-327742745012.938\\
0.190904772619315	-327714097123.182\\
0.191004775119378	-327685449233.425\\
0.19110477761944	-327656801343.669\\
0.191204780119503	-327628153453.912\\
0.191304782619565	-327599505564.156\\
0.191404785119628	-327570857674.399\\
0.191504787619691	-327542209784.643\\
0.191604790119753	-327513561894.886\\
0.191704792619815	-327484914005.13\\
0.191804795119878	-327456266115.373\\
0.191904797619941	-327427618225.616\\
0.192004800120003	-327398970335.86\\
0.192104802620065	-327370322446.103\\
0.192204805120128	-327341674556.347\\
0.192304807620191	-327313026666.59\\
0.192404810120253	-327283805819.039\\
0.192504812620316	-327255157929.282\\
0.192604815120378	-327226510039.526\\
0.192704817620441	-327197862149.769\\
0.192804820120503	-327168641302.217\\
0.192904822620566	-327139993412.461\\
0.193004825120628	-327111345522.704\\
0.193104827620691	-327082124675.153\\
0.193204830120753	-327053476785.396\\
0.193304832620816	-327024828895.639\\
0.193404835120878	-326995608048.088\\
0.193504837620941	-326966960158.331\\
0.193604840121003	-326937739310.78\\
0.193704842621066	-326909091421.023\\
0.193804845121128	-326879870573.471\\
0.193904847621191	-326851222683.715\\
0.194004850121253	-326822001836.163\\
0.194104852621316	-326793353946.407\\
0.194204855121378	-326764133098.855\\
0.194304857621441	-326734912251.303\\
0.194404860121503	-326706264361.547\\
0.194504862621566	-326677043513.995\\
0.194604865121628	-326647822666.443\\
0.194704867621691	-326619174776.687\\
0.194804870121753	-326589953929.135\\
0.194904872621816	-326560733081.583\\
0.195004875121878	-326531512234.032\\
0.195104877621941	-326502864344.275\\
0.195204880122003	-326473643496.724\\
0.195304882622066	-326444422649.172\\
0.195404885122128	-326415201801.62\\
0.195504887622191	-326385980954.069\\
0.195604890122253	-326356760106.517\\
0.195704892622316	-326327539258.965\\
0.195804895122378	-326298318411.414\\
0.195904897622441	-326269097563.862\\
0.196004900122503	-326239876716.31\\
0.196104902622566	-326210655868.759\\
0.196204905122628	-326181435021.207\\
0.196304907622691	-326152214173.655\\
0.196404910122753	-326122993326.104\\
0.196504912622816	-326093772478.552\\
0.196604915122878	-326064551631\\
0.196704917622941	-326035330783.449\\
0.196804920123003	-326005536978.102\\
0.196904922623066	-325976316130.55\\
0.197004925123128	-325947095282.998\\
0.197104927623191	-325917874435.447\\
0.197204930123253	-325888080630.1\\
0.197304932623316	-325858859782.548\\
0.197404935123378	-325829638934.997\\
0.197504937623441	-325800418087.445\\
0.197604940123503	-325770624282.098\\
0.197704942623566	-325741403434.546\\
0.197804945123628	-325711609629.2\\
0.197904947623691	-325682388781.648\\
0.198004950123753	-325652594976.301\\
0.198104952623816	-325623374128.749\\
0.198204955123878	-325593580323.403\\
0.198304957623941	-325564359475.851\\
0.198404960124003	-325534565670.504\\
0.198504962624066	-325505344822.953\\
0.198604965124128	-325475551017.606\\
0.198704967624191	-325446330170.054\\
0.198804970124253	-325416536364.707\\
0.198904972624316	-325386742559.36\\
0.199004975124378	-325357521711.809\\
0.199104977624441	-325327727906.462\\
0.199204980124503	-325297934101.115\\
0.199304982624566	-325268713253.563\\
0.199404985124628	-325238919448.217\\
0.199504987624691	-325209125642.87\\
0.199604990124753	-325179331837.523\\
0.199704992624816	-325149538032.176\\
0.199804995124878	-325119744226.829\\
0.199904997624941	-325090523379.278\\
0.200005000125003	-325060729573.931\\
0.200105002625066	-325030935768.584\\
0.200205005125128	-325001141963.237\\
0.200305007625191	-324971348157.891\\
0.200405010125253	-324941554352.544\\
0.200505012625316	-324911760547.197\\
0.200605015125378	-324881966741.85\\
0.200705017625441	-324852172936.503\\
0.200805020125503	-324822379131.157\\
0.200905022625566	-324792585325.81\\
0.201005025125628	-324762218562.668\\
0.201105027625691	-324732424757.321\\
0.201205030125753	-324702630951.974\\
0.201305032625816	-324672837146.627\\
0.201405035125878	-324643043341.281\\
0.201505037625941	-324612676578.139\\
0.201605040126003	-324582882772.792\\
0.201705042626066	-324553088967.445\\
0.201805045126128	-324523295162.098\\
0.201905047626191	-324492928398.956\\
0.202005050126253	-324463134593.61\\
0.202105052626316	-324433340788.263\\
0.202205055126378	-324402974025.121\\
0.202305057626441	-324373180219.774\\
0.202405060126503	-324342813456.632\\
0.202505062626566	-324313019651.285\\
0.202605065126628	-324282652888.143\\
0.202705067626691	-324252859082.797\\
0.202805070126753	-324222492319.655\\
0.202905072626816	-324192698514.308\\
0.203005075126878	-324162331751.166\\
0.203105077626941	-324132537945.819\\
0.203205080127003	-324102171182.677\\
0.203305082627066	-324071804419.535\\
0.203405085127128	-324042010614.188\\
0.203505087627191	-324011643851.046\\
0.203605090127253	-323981277087.905\\
0.203705092627316	-323951483282.558\\
0.203805095127378	-323921116519.416\\
0.203905097627441	-323890749756.274\\
0.204005100127503	-323860382993.132\\
0.204105102627566	-323830589187.785\\
0.204205105127628	-323800222424.643\\
0.204305107627691	-323769855661.501\\
0.204405110127753	-323739488898.359\\
0.204505112627816	-323709122135.217\\
0.204605115127878	-323678755372.075\\
0.204705117627941	-323648388608.934\\
0.204805120128003	-323618021845.792\\
0.204905122628066	-323587655082.65\\
0.205005125128128	-323557288319.508\\
0.205105127628191	-323526921556.366\\
0.205205130128253	-323496554793.224\\
0.205305132628316	-323466188030.082\\
0.205405135128378	-323435821266.94\\
0.205505137628441	-323405454503.798\\
0.205605140128503	-323375087740.656\\
0.205705142628566	-323344720977.514\\
0.205805145128628	-323314354214.372\\
0.205905147628691	-323283414493.435\\
0.206005150128753	-323253047730.293\\
0.206105152628816	-323222680967.151\\
0.206205155128878	-323192314204.009\\
0.206305157628941	-323161374483.072\\
0.206405160129003	-323131007719.93\\
0.206505162629066	-323100640956.788\\
0.206605165129128	-323069701235.851\\
0.206705167629191	-323039334472.709\\
0.206805170129253	-323008967709.568\\
0.206905172629316	-322978027988.63\\
0.207005175129378	-322947661225.489\\
0.207105177629441	-322916721504.551\\
0.207205180129503	-322886354741.41\\
0.207305182629566	-322855415020.472\\
0.207405185129628	-322825048257.331\\
0.207505187629691	-322794108536.393\\
0.207605190129753	-322763741773.252\\
0.207705192629816	-322732802052.314\\
0.207805195129878	-322701862331.377\\
0.207905197629941	-322671495568.235\\
0.208005200130003	-322640555847.298\\
0.208105202630066	-322610189084.156\\
0.208205205130128	-322579249363.219\\
0.208305207630191	-322548309642.282\\
0.208405210130253	-322517369921.345\\
0.208505212630316	-322487003158.203\\
0.208605215130378	-322456063437.266\\
0.208705217630441	-322425123716.329\\
0.208805220130503	-322394183995.392\\
0.208905222630566	-322363244274.455\\
0.209005225130628	-322332304553.518\\
0.209105227630691	-322301937790.376\\
0.209205230130753	-322270998069.439\\
0.209305232630816	-322240058348.502\\
0.209405235130878	-322209118627.565\\
0.209505237630941	-322178178906.628\\
0.209605240131003	-322147239185.691\\
0.209705242631066	-322116299464.754\\
0.209805245131128	-322085359743.817\\
0.209905247631191	-322054420022.88\\
0.210005250131253	-322023480301.943\\
0.210105252631316	-321991967623.21\\
0.210205255131378	-321961027902.273\\
0.210305257631441	-321930088181.336\\
0.210405260131503	-321899148460.399\\
0.210505262631566	-321868208739.462\\
0.210605265131628	-321837269018.525\\
0.210705267631691	-321805756339.793\\
0.210805270131753	-321774816618.856\\
0.210905272631816	-321743876897.919\\
0.211005275131878	-321712937176.982\\
0.211105277631941	-321681424498.249\\
0.211205280132003	-321650484777.312\\
0.211305282632066	-321619545056.375\\
0.211405285132128	-321588032377.643\\
0.211505287632191	-321557092656.706\\
0.211605290132253	-321525579977.974\\
0.211705292632316	-321494640257.037\\
0.211805295132378	-321463127578.305\\
0.211905297632441	-321432187857.367\\
0.212005300132503	-321400675178.635\\
0.212105302632566	-321369735457.698\\
0.212205305132628	-321338222778.966\\
0.212305307632691	-321307283058.029\\
0.212405310132753	-321275770379.297\\
0.212505312632816	-321244830658.36\\
0.212605315132878	-321213317979.628\\
0.212705317632941	-321181805300.895\\
0.212805320133003	-321150865579.958\\
0.212905322633066	-321119352901.226\\
0.213005325133128	-321087840222.494\\
0.213105327633191	-321056327543.762\\
0.213205330133253	-321025387822.825\\
0.213305332633316	-320993875144.092\\
0.213405335133378	-320962362465.36\\
0.213505337633441	-320930849786.628\\
0.213605340133503	-320899337107.896\\
0.213705342633566	-320867824429.164\\
0.213805345133628	-320836884708.227\\
0.213905347633691	-320805372029.494\\
0.214005350133753	-320773859350.762\\
0.214105352633816	-320742346672.03\\
0.214205355133878	-320710833993.298\\
0.214305357633941	-320679321314.566\\
0.214405360134003	-320647808635.833\\
0.214505362634066	-320616295957.101\\
0.214605365134128	-320584783278.369\\
0.214705367634191	-320553270599.637\\
0.214805370134253	-320521757920.905\\
0.214905372634316	-320489672284.377\\
0.215005375134378	-320458159605.645\\
0.215105377634441	-320426646926.913\\
0.215205380134503	-320395134248.181\\
0.215305382634566	-320363621569.448\\
0.215405385134628	-320331535932.921\\
0.215505387634691	-320300023254.189\\
0.215605390134753	-320268510575.457\\
0.215705392634816	-320236997896.725\\
0.215805395134878	-320204912260.197\\
0.215905397634941	-320173399581.465\\
0.216005400135003	-320141886902.733\\
0.216105402635066	-320109801266.206\\
0.216205405135128	-320078288587.473\\
0.216305407635191	-320046202950.946\\
0.216405410135253	-320014690272.214\\
0.216505412635316	-319983177593.482\\
0.216605415135378	-319951091956.954\\
0.216705417635441	-319919579278.222\\
0.216805420135503	-319887493641.695\\
0.216905422635566	-319855980962.963\\
0.217005425135628	-319823895326.435\\
0.217105427635691	-319791809689.908\\
0.217205430135753	-319760297011.176\\
0.217305432635816	-319728211374.648\\
0.217405435135878	-319696125738.121\\
0.217505437635941	-319664613059.389\\
0.217605440136003	-319632527422.862\\
0.217705442636066	-319600441786.334\\
0.217805445136128	-319568929107.602\\
0.217905447636191	-319536843471.075\\
0.218005450136253	-319504757834.547\\
0.218105452636316	-319472672198.02\\
0.218205455136378	-319441159519.288\\
0.218305457636441	-319409073882.761\\
0.218405460136503	-319376988246.233\\
0.218505462636566	-319344902609.706\\
0.218605465136628	-319312816973.179\\
0.218705467636691	-319280731336.651\\
0.218805470136753	-319248645700.124\\
0.218905472636816	-319216560063.597\\
0.219005475136878	-319184474427.069\\
0.219105477636941	-319152388790.542\\
0.219205480137003	-319120303154.015\\
0.219305482637066	-319088217517.487\\
0.219405485137128	-319056131880.96\\
0.219505487637191	-319024046244.433\\
0.219605490137253	-318991960607.905\\
0.219705492637316	-318959874971.378\\
0.219805495137378	-318927789334.851\\
0.219905497637441	-318895703698.323\\
0.220005500137503	-318863045104.001\\
0.220105502637566	-318830959467.474\\
0.220205505137628	-318798873830.946\\
0.220305507637691	-318766788194.419\\
0.220405510137753	-318734129600.096\\
0.220505512637816	-318702043963.569\\
0.220605515137878	-318669958327.042\\
0.220705517637941	-318637872690.514\\
0.220805520138003	-318605214096.192\\
0.220905522638066	-318573128459.665\\
0.221005525138128	-318540469865.342\\
0.221105527638191	-318508384228.815\\
0.221205530138253	-318476298592.288\\
0.221305532638316	-318443639997.965\\
0.221405535138378	-318411554361.438\\
0.221505537638441	-318378895767.115\\
0.221605540138503	-318346810130.588\\
0.221705542638566	-318314151536.266\\
0.221805545138628	-318282065899.738\\
0.221905547638691	-318249407305.416\\
0.222005550138753	-318216748711.093\\
0.222105552638816	-318184663074.566\\
0.222205555138878	-318152004480.244\\
0.222305557638941	-318119345885.921\\
0.222405560139003	-318087260249.394\\
0.222505562639066	-318054601655.071\\
0.222605565139128	-318021943060.749\\
0.222705567639191	-317989857424.221\\
0.222805570139253	-317957198829.899\\
0.222905572639316	-317924540235.577\\
0.223005575139378	-317891881641.254\\
0.223105577639441	-317859223046.932\\
0.223205580139503	-317827137410.404\\
0.223305582639566	-317794478816.082\\
0.223405585139628	-317761820221.759\\
0.223505587639691	-317729161627.437\\
0.223605590139754	-317696503033.115\\
0.223705592639816	-317663844438.792\\
0.223805595139878	-317631185844.47\\
0.223905597639941	-317598527250.147\\
0.224005600140004	-317565868655.825\\
0.224105602640066	-317533210061.502\\
0.224205605140128	-317500551467.18\\
0.224305607640191	-317467892872.857\\
0.224405610140254	-317435234278.535\\
0.224505612640316	-317402575684.212\\
0.224605615140379	-317369917089.89\\
0.224705617640441	-317337258495.568\\
0.224805620140504	-317304026943.45\\
0.224905622640566	-317271368349.127\\
0.225005625140629	-317238709754.805\\
0.225105627640691	-317206051160.483\\
0.225205630140754	-317173392566.16\\
0.225305632640816	-317140161014.042\\
0.225405635140879	-317107502419.72\\
0.225505637640941	-317074843825.398\\
0.225605640141004	-317041612273.28\\
0.225705642641066	-317008953678.958\\
0.225805645141129	-316976295084.635\\
0.225905647641191	-316943063532.518\\
0.226005650141254	-316910404938.195\\
0.226105652641316	-316877173386.077\\
0.226205655141379	-316844514791.755\\
0.226305657641441	-316811856197.433\\
0.226405660141504	-316778624645.315\\
0.226505662641566	-316745966050.992\\
0.226605665141629	-316712734498.875\\
0.226705667641691	-316679502946.757\\
0.226805670141754	-316646844352.435\\
0.226905672641816	-316613612800.317\\
0.227005675141879	-316580954205.995\\
0.227105677641941	-316547722653.877\\
0.227205680142004	-316514491101.76\\
0.227305682642066	-316481832507.437\\
0.227405685142129	-316448600955.32\\
0.227505687642191	-316415369403.202\\
0.227605690142254	-316382710808.88\\
0.227705692642316	-316349479256.762\\
0.227805695142379	-316316247704.644\\
0.227905697642441	-316283016152.527\\
0.228005700142504	-316250357558.204\\
0.228105702642566	-316217126006.087\\
0.228205705142629	-316183894453.969\\
0.228305707642691	-316150662901.852\\
0.228405710142754	-316117431349.734\\
0.228505712642816	-316084199797.616\\
0.228605715142879	-316050968245.499\\
0.228705717642941	-316017736693.381\\
0.228805720143004	-315984505141.264\\
0.228905722643066	-315951273589.146\\
0.229005725143129	-315918042037.028\\
0.229105727643191	-315884810484.911\\
0.229205730143254	-315851578932.793\\
0.229305732643316	-315818347380.676\\
0.229405735143379	-315785115828.558\\
0.229505737643441	-315751884276.44\\
0.229605740143504	-315718652724.323\\
0.229705742643566	-315685421172.205\\
0.229805745143629	-315652189620.088\\
0.229905747643691	-315618385110.175\\
0.230005750143754	-315585153558.057\\
0.230105752643816	-315551922005.94\\
0.230205755143879	-315518690453.822\\
0.230305757643941	-315485458901.705\\
0.230405760144004	-315451654391.792\\
0.230505762644066	-315418422839.674\\
0.230605765144129	-315385191287.557\\
0.230705767644191	-315351386777.644\\
0.230805770144254	-315318155225.526\\
0.230905772644316	-315284923673.409\\
0.231005775144379	-315251119163.496\\
0.231105777644441	-315217887611.379\\
0.231205780144504	-315184083101.466\\
0.231305782644566	-315150851549.348\\
0.231405785144629	-315117047039.436\\
0.231505787644691	-315083815487.318\\
0.231605790144754	-315050010977.405\\
0.231705792644816	-315016779425.288\\
0.231805795144879	-314982974915.375\\
0.231905797644941	-314949743363.257\\
0.232005800145004	-314915938853.345\\
0.232105802645066	-314882134343.432\\
0.232205805145129	-314848902791.314\\
0.232305807645191	-314815098281.402\\
0.232405810145254	-314781293771.489\\
0.232505812645316	-314748062219.371\\
0.232605815145379	-314714257709.459\\
0.232705817645441	-314680453199.546\\
0.232805820145504	-314647221647.428\\
0.232905822645566	-314613417137.516\\
0.233005825145629	-314579612627.603\\
0.233105827645691	-314545808117.69\\
0.233205830145754	-314512003607.777\\
0.233305832645816	-314478772055.66\\
0.233405835145879	-314444967545.747\\
0.233505837645941	-314411163035.834\\
0.233605840146004	-314377358525.922\\
0.233705842646066	-314343554016.009\\
0.233805845146129	-314309749506.096\\
0.233905847646191	-314275944996.184\\
0.234005850146254	-314242140486.271\\
0.234105852646316	-314208335976.358\\
0.234205855146379	-314174531466.445\\
0.234305857646441	-314140726956.533\\
0.234405860146504	-314106922446.62\\
0.234505862646566	-314073117936.707\\
0.234605865146629	-314039313426.794\\
0.234705867646691	-314005508916.882\\
0.234805870146754	-313971131449.174\\
0.234905872646816	-313937326939.261\\
0.235005875146879	-313903522429.348\\
0.235105877646941	-313869717919.436\\
0.235205880147004	-313835913409.523\\
0.235305882647066	-313801535941.815\\
0.235405885147129	-313767731431.902\\
0.235505887647191	-313733926921.99\\
0.235605890147254	-313700122412.077\\
0.235705892647316	-313665744944.369\\
0.235805895147379	-313631940434.456\\
0.235905897647441	-313598135924.544\\
0.236005900147504	-313563758456.836\\
0.236105902647566	-313529953946.923\\
0.236205905147629	-313495576479.215\\
0.236305907647691	-313461771969.303\\
0.236405910147754	-313427967459.39\\
0.236505912647816	-313393589991.682\\
0.236605915147879	-313359785481.769\\
0.236705917647941	-313325408014.061\\
0.236805920148004	-313291603504.149\\
0.236905922648066	-313257226036.441\\
0.237005925148129	-313222848568.733\\
0.237105927648191	-313189044058.82\\
0.237205930148254	-313154666591.112\\
0.237305932648316	-313120862081.2\\
0.237405935148379	-313086484613.492\\
0.237505937648441	-313052107145.784\\
0.237605940148504	-313018302635.871\\
0.237705942648566	-312983925168.164\\
0.237805945148629	-312949547700.456\\
0.237905947648691	-312915170232.748\\
0.238005950148754	-312881365722.835\\
0.238105952648816	-312846988255.127\\
0.238205955148879	-312812610787.419\\
0.238305957648941	-312778233319.712\\
0.238405960149004	-312743855852.004\\
0.238505962649066	-312710051342.091\\
0.238605965149129	-312675673874.383\\
0.238705967649191	-312641296406.675\\
0.238805970149254	-312606918938.967\\
0.238905972649316	-312572541471.26\\
0.239005975149379	-312538164003.552\\
0.239105977649441	-312503786535.844\\
0.239205980149504	-312469409068.136\\
0.239305982649566	-312435031600.428\\
0.239405985149629	-312400654132.72\\
0.239505987649691	-312366276665.012\\
0.239605990149754	-312331899197.305\\
0.239705992649816	-312297521729.597\\
0.239805995149879	-312263144261.889\\
0.239905997649941	-312228766794.181\\
0.240006000150004	-312194389326.473\\
0.240106002650066	-312159438900.97\\
0.240206005150129	-312125061433.262\\
0.240306007650191	-312090683965.555\\
0.240406010150254	-312056306497.847\\
0.240506012650316	-312021929030.139\\
0.240606015150379	-311986978604.636\\
0.240706017650441	-311952601136.928\\
0.240806020150504	-311918223669.22\\
0.240906022650566	-311883273243.717\\
0.241006025150629	-311848895776.009\\
0.241106027650691	-311814518308.302\\
0.241206030150754	-311779567882.799\\
0.241306032650816	-311745190415.091\\
0.241406035150879	-311710812947.383\\
0.241506037650941	-311675862521.88\\
0.241606040151004	-311641485054.172\\
0.241706042651066	-311606534628.669\\
0.241806045151129	-311572157160.961\\
0.241906047651191	-311537206735.458\\
0.242006050151254	-311502829267.75\\
0.242106052651316	-311467878842.247\\
0.242206055151379	-311433501374.539\\
0.242306057651441	-311398550949.036\\
0.242406060151504	-311364173481.329\\
0.242506062651566	-311329223055.826\\
0.242606065151629	-311294272630.323\\
0.242706067651691	-311259895162.615\\
0.242806070151754	-311224944737.112\\
0.242906072651816	-311189994311.609\\
0.243006075151879	-311155616843.901\\
0.243106077651941	-311120666418.398\\
0.243206080152004	-311085715992.895\\
0.243306082652066	-311050765567.392\\
0.243406085152129	-311016388099.684\\
0.243506087652191	-310981437674.181\\
0.243606090152254	-310946487248.678\\
0.243706092652316	-310911536823.175\\
0.243806095152379	-310876586397.672\\
0.243906097652441	-310842208929.964\\
0.244006100152504	-310807258504.461\\
0.244106102652566	-310772308078.958\\
0.244206105152629	-310737357653.456\\
0.244306107652691	-310702407227.953\\
0.244406110152754	-310667456802.45\\
0.244506112652816	-310632506376.947\\
0.244606115152879	-310597555951.444\\
0.244706117652941	-310562605525.941\\
0.244806120153004	-310527655100.438\\
0.244906122653066	-310492704674.935\\
0.245006125153129	-310457754249.432\\
0.245106127653191	-310422803823.929\\
0.245206130153254	-310387853398.426\\
0.245306132653316	-310352330015.128\\
0.245406135153379	-310317379589.625\\
0.245506137653441	-310282429164.122\\
0.245606140153504	-310247478738.619\\
0.245706142653566	-310212528313.116\\
0.245806145153629	-310177577887.613\\
0.245906147653691	-310142054504.315\\
0.246006150153754	-310107104078.812\\
0.246106152653816	-310072153653.309\\
0.246206155153879	-310036630270.011\\
0.246306157653941	-310001679844.508\\
0.246406160154004	-309966729419.005\\
0.246506162654066	-309931206035.706\\
0.246606165154129	-309896255610.203\\
0.246706167654191	-309861305184.701\\
0.246806170154254	-309825781801.402\\
0.246906172654316	-309790831375.899\\
0.247006175154379	-309755307992.601\\
0.247106177654441	-309720357567.098\\
0.247206180154504	-309685407141.595\\
0.247306182654566	-309649883758.297\\
0.247406185154629	-309614933332.794\\
0.247506187654691	-309579409949.496\\
0.247606190154754	-309543886566.198\\
0.247706192654816	-309508936140.695\\
0.247806195154879	-309473412757.397\\
0.247906197654941	-309438462331.894\\
0.248006200155004	-309402938948.596\\
0.248106202655066	-309367415565.298\\
0.248206205155129	-309332465139.795\\
0.248306207655191	-309296941756.497\\
0.248406210155254	-309261418373.199\\
0.248506212655316	-309226467947.696\\
0.248606215155379	-309190944564.397\\
0.248706217655441	-309155421181.099\\
0.248806220155504	-309119897797.801\\
0.248906222655566	-309084947372.298\\
0.249006225155629	-309049423989\\
0.249106227655691	-309013900605.702\\
0.249206230155754	-308978377222.404\\
0.249306232655816	-308942853839.106\\
0.249406235155879	-308907330455.808\\
0.249506237655941	-308872380030.305\\
0.249606240156004	-308836856647.007\\
0.249706242656066	-308801333263.708\\
0.249806245156129	-308765809880.41\\
0.249906247656191	-308730286497.112\\
0.250006250156254	-308694763113.814\\
0.250106252656316	-308659239730.516\\
0.250206255156379	-308623716347.218\\
0.250306257656441	-308588192963.92\\
0.250406260156504	-308552669580.622\\
0.250506262656566	-308517146197.324\\
0.250606265156629	-308481049856.23\\
0.250706267656691	-308445526472.932\\
0.250806270156754	-308410003089.634\\
0.250906272656816	-308374479706.336\\
0.251006275156879	-308338956323.038\\
0.251106277656941	-308303432939.74\\
0.251206280157004	-308267909556.442\\
0.251306282657066	-308231813215.348\\
0.251406285157129	-308196289832.05\\
0.251506287657191	-308160766448.752\\
0.251606290157254	-308125243065.454\\
0.251706292657316	-308089146724.361\\
0.251806295157379	-308053623341.063\\
0.251906297657441	-308018099957.765\\
0.252006300157504	-307982003616.671\\
0.252106302657566	-307946480233.373\\
0.252206305157629	-307910956850.075\\
0.252306307657691	-307874860508.982\\
0.252406310157754	-307839337125.684\\
0.252506312657816	-307803240784.591\\
0.252606315157879	-307767717401.292\\
0.252706317657941	-307731621060.199\\
0.252806320158004	-307696097676.901\\
0.252906322658066	-307660001335.808\\
0.253006325158129	-307624477952.51\\
0.253106327658191	-307588381611.417\\
0.253206330158254	-307552858228.118\\
0.253306332658316	-307516761887.025\\
0.253406335158379	-307481238503.727\\
0.253506337658441	-307445142162.634\\
0.253606340158504	-307409045821.541\\
0.253706342658566	-307373522438.242\\
0.253806345158629	-307337426097.149\\
0.253906347658691	-307301329756.056\\
0.254006350158754	-307265806372.758\\
0.254106352658816	-307229710031.665\\
0.254206355158879	-307193613690.571\\
0.254306357658941	-307157517349.478\\
0.254406360159004	-307121993966.18\\
0.254506362659066	-307085897625.087\\
0.254606365159129	-307049801283.994\\
0.254706367659191	-307013704942.9\\
0.254806370159254	-306977608601.807\\
0.254906372659316	-306942085218.509\\
0.255006375159379	-306905988877.416\\
0.255106377659441	-306869892536.323\\
0.255206380159504	-306833796195.229\\
0.255306382659567	-306797699854.136\\
0.255406385159629	-306761603513.043\\
0.255506387659691	-306725507171.95\\
0.255606390159754	-306689410830.856\\
0.255706392659816	-306653314489.763\\
0.255806395159879	-306617218148.67\\
0.255906397659941	-306581121807.577\\
0.256006400160004	-306545025466.483\\
0.256106402660067	-306508929125.39\\
0.256206405160129	-306472832784.297\\
0.256306407660192	-306436736443.204\\
0.256406410160254	-306400640102.11\\
0.256506412660316	-306363970803.222\\
0.256606415160379	-306327874462.129\\
0.256706417660441	-306291778121.035\\
0.256806420160504	-306255681779.942\\
0.256906422660567	-306219585438.849\\
0.257006425160629	-306182916139.961\\
0.257106427660692	-306146819798.867\\
0.257206430160754	-306110723457.774\\
0.257306432660817	-306074627116.681\\
0.257406435160879	-306037957817.793\\
0.257506437660942	-306001861476.699\\
0.257606440161004	-305965765135.606\\
0.257706442661067	-305929095836.718\\
0.257806445161129	-305892999495.624\\
0.257906447661192	-305856903154.531\\
0.258006450161254	-305820233855.643\\
0.258106452661317	-305784137514.55\\
0.258206455161379	-305747468215.661\\
0.258306457661442	-305711371874.568\\
0.258406460161504	-305674702575.68\\
0.258506462661567	-305638606234.586\\
0.258606465161629	-305601936935.698\\
0.258706467661692	-305565840594.605\\
0.258806470161754	-305529171295.716\\
0.258906472661817	-305493074954.623\\
0.259006475161879	-305456405655.735\\
0.259106477661942	-305420309314.641\\
0.259206480162004	-305383640015.753\\
0.259306482662067	-305346970716.865\\
0.259406485162129	-305310874375.771\\
0.259506487662192	-305274205076.883\\
0.259606490162254	-305237535777.995\\
0.259706492662317	-305201439436.902\\
0.259806495162379	-305164770138.013\\
0.259906497662442	-305128100839.125\\
0.260006500162504	-305092004498.032\\
0.260106502662567	-305055335199.143\\
0.260206505162629	-305018665900.255\\
0.260306507662692	-304981996601.366\\
0.260406510162754	-304945327302.478\\
0.260506512662817	-304909230961.385\\
0.260606515162879	-304872561662.496\\
0.260706517662942	-304835892363.608\\
0.260806520163004	-304799223064.72\\
0.260906522663067	-304762553765.831\\
0.261006525163129	-304725884466.943\\
0.261106527663192	-304689215168.055\\
0.261206530163254	-304652545869.166\\
0.261306532663317	-304615876570.278\\
0.261406535163379	-304579207271.389\\
0.261506537663442	-304542537972.501\\
0.261606540163504	-304505868673.613\\
0.261706542663567	-304469199374.724\\
0.261806545163629	-304432530075.836\\
0.261906547663692	-304395860776.948\\
0.262006550163754	-304359191478.059\\
0.262106552663817	-304322522179.171\\
0.262206555163879	-304285852880.282\\
0.262306557663942	-304249183581.394\\
0.262406560164004	-304212514282.506\\
0.262506562664067	-304175272025.822\\
0.262606565164129	-304138602726.934\\
0.262706567664192	-304101933428.045\\
0.262806570164254	-304065264129.157\\
0.262906572664317	-304028594830.269\\
0.263006575164379	-303991352573.585\\
0.263106577664442	-303954683274.697\\
0.263206580164504	-303918013975.808\\
0.263306582664567	-303880771719.125\\
0.263406585164629	-303844102420.237\\
0.263506587664692	-303807433121.348\\
0.263606590164754	-303770190864.665\\
0.263706592664817	-303733521565.776\\
0.263806595164879	-303696852266.888\\
0.263906597664942	-303659610010.204\\
0.264006600165004	-303622940711.316\\
0.264106602665067	-303585698454.633\\
0.264206605165129	-303549029155.744\\
0.264306607665192	-303511786899.061\\
0.264406610165254	-303475117600.172\\
0.264506612665317	-303437875343.489\\
0.264606615165379	-303401206044.6\\
0.264706617665442	-303363963787.917\\
0.264806620165504	-303327294489.029\\
0.264906622665567	-303290052232.345\\
0.265006625165629	-303253382933.457\\
0.265106627665692	-303216140676.773\\
0.265206630165754	-303178898420.09\\
0.265306632665817	-303142229121.201\\
0.265406635165879	-303104986864.518\\
0.265506637665942	-303067744607.834\\
0.265606640166004	-303031075308.946\\
0.265706642666067	-302993833052.262\\
0.265806645166129	-302956590795.579\\
0.265906647666192	-302919921496.691\\
0.266006650166254	-302882679240.007\\
0.266106652666317	-302845436983.324\\
0.266206655166379	-302808194726.64\\
0.266306657666442	-302770952469.957\\
0.266406660166504	-302734283171.068\\
0.266506662666567	-302697040914.385\\
0.266606665166629	-302659798657.701\\
0.266706667666692	-302622556401.018\\
0.266806670166754	-302585314144.334\\
0.266906672666817	-302548071887.651\\
0.267006675166879	-302510829630.967\\
0.267106677666942	-302473587374.284\\
0.267206680167004	-302436345117.6\\
0.267306682667067	-302399102860.917\\
0.267406685167129	-302361860604.233\\
0.267506687667192	-302324618347.55\\
0.267606690167254	-302287376090.866\\
0.267706692667317	-302250133834.183\\
0.267806695167379	-302212891577.499\\
0.267906697667442	-302175649320.816\\
0.268006700167504	-302138407064.132\\
0.268106702667567	-302101164807.449\\
0.268206705167629	-302063922550.765\\
0.268306707667692	-302026680294.082\\
0.268406710167754	-301989438037.398\\
0.268506712667817	-301951622822.919\\
0.268606715167879	-301914380566.236\\
0.268706717667942	-301877138309.552\\
0.268806720168004	-301839896052.869\\
0.268906722668067	-301802653796.185\\
0.269006725168129	-301764838581.707\\
0.269106727668192	-301727596325.023\\
0.269206730168254	-301690354068.34\\
0.269306732668317	-301652538853.861\\
0.269406735168379	-301615296597.178\\
0.269506737668442	-301578054340.494\\
0.269606740168504	-301540239126.016\\
0.269706742668567	-301502996869.332\\
0.269806745168629	-301465754612.649\\
0.269906747668692	-301427939398.17\\
0.270006750168754	-301390697141.486\\
0.270106752668817	-301352881927.008\\
0.270206755168879	-301315639670.324\\
0.270306757668942	-301277824455.846\\
0.270406760169004	-301240582199.162\\
0.270506762669067	-301202766984.684\\
0.270606765169129	-301165524728\\
0.270706767669192	-301127709513.521\\
0.270806770169254	-301090467256.838\\
0.270906772669317	-301052652042.359\\
0.271006775169379	-301015409785.676\\
0.271106777669442	-300977594571.197\\
0.271206780169504	-300940352314.514\\
0.271306782669567	-300902537100.035\\
0.271406785169629	-300864721885.556\\
0.271506787669692	-300827479628.873\\
0.271606790169754	-300789664414.394\\
0.271706792669817	-300751849199.916\\
0.271806795169879	-300714033985.437\\
0.271906797669942	-300676791728.753\\
0.272006800170004	-300638976514.275\\
0.272106802670067	-300601161299.796\\
0.272206805170129	-300563346085.318\\
0.272306807670192	-300526103828.634\\
0.272406810170254	-300488288614.155\\
0.272506812670317	-300450473399.677\\
0.272606815170379	-300412658185.198\\
0.272706817670442	-300374842970.719\\
0.272806820170504	-300337600714.036\\
0.272906822670567	-300299785499.557\\
0.273006825170629	-300261970285.079\\
0.273106827670692	-300224155070.6\\
0.273206830170754	-300186339856.121\\
0.273306832670817	-300148524641.643\\
0.273406835170879	-300110709427.164\\
0.273506837670942	-300072894212.686\\
0.273606840171004	-300035078998.207\\
0.273706842671067	-299997263783.728\\
0.273806845171129	-299959448569.25\\
0.273906847671192	-299921633354.771\\
0.274006850171254	-299883818140.292\\
0.274106852671317	-299846002925.814\\
0.274206855171379	-299808187711.335\\
0.274306857671442	-299770372496.856\\
0.274406860171504	-299731984324.583\\
0.274506862671567	-299694169110.104\\
0.274606865171629	-299656353895.625\\
0.274706867671692	-299618538681.147\\
0.274806870171754	-299580723466.668\\
0.274906872671817	-299542908252.19\\
0.275006875171879	-299504520079.916\\
0.275106877671942	-299466704865.437\\
0.275206880172004	-299428889650.958\\
0.275306882672067	-299391074436.48\\
0.275406885172129	-299352686264.206\\
0.275506887672192	-299314871049.727\\
0.275606890172254	-299277055835.249\\
0.275706892672317	-299238667662.975\\
0.275806895172379	-299200852448.496\\
0.275906897672442	-299163037234.018\\
0.276006900172504	-299124649061.744\\
0.276106902672567	-299086833847.265\\
0.276206905172629	-299048445674.992\\
0.276306907672692	-299010630460.513\\
0.276406910172754	-298972815246.034\\
0.276506912672817	-298934427073.761\\
0.276606915172879	-298896611859.282\\
0.276706917672942	-298858223687.008\\
0.276806920173004	-298820408472.53\\
0.276906922673067	-298782020300.256\\
0.277006925173129	-298744205085.777\\
0.277106927673192	-298705816913.503\\
0.277206930173254	-298667428741.23\\
0.277306932673317	-298629613526.751\\
0.277406935173379	-298591225354.477\\
0.277506937673442	-298553410139.999\\
0.277606940173504	-298515021967.725\\
0.277706942673567	-298476633795.451\\
0.277806945173629	-298438818580.972\\
0.277906947673692	-298400430408.699\\
0.278006950173754	-298362042236.425\\
0.278106952673817	-298324227021.946\\
0.278206955173879	-298285838849.672\\
0.278306957673942	-298247450677.399\\
0.278406960174004	-298209062505.125\\
0.278506962674067	-298171247290.646\\
0.278606965174129	-298132859118.373\\
0.278706967674192	-298094470946.099\\
0.278806970174254	-298056082773.825\\
0.278906972674317	-298017694601.551\\
0.279006975174379	-297979879387.073\\
0.279106977674442	-297941491214.799\\
0.279206980174504	-297903103042.525\\
0.279306982674567	-297864714870.251\\
0.279406985174629	-297826326697.978\\
0.279506987674692	-297787938525.704\\
0.279606990174754	-297749550353.43\\
0.279706992674817	-297711162181.156\\
0.279806995174879	-297672774008.883\\
0.279906997674942	-297634385836.609\\
0.280007000175004	-297595997664.335\\
0.280107002675067	-297557609492.061\\
0.280207005175129	-297519221319.787\\
0.280307007675192	-297480833147.514\\
0.280407010175254	-297442444975.24\\
0.280507012675317	-297404056802.966\\
0.280607015175379	-297365668630.692\\
0.280707017675442	-297327280458.419\\
0.280807020175504	-297288892286.145\\
0.280907022675567	-297250504113.871\\
0.281007025175629	-297211542983.802\\
0.281107027675692	-297173154811.528\\
0.281207030175754	-297134766639.255\\
0.281307032675817	-297096378466.981\\
0.281407035175879	-297057990294.707\\
0.281507037675942	-297019029164.638\\
0.281607040176004	-296980640992.365\\
0.281707042676067	-296942252820.091\\
0.281807045176129	-296903864647.817\\
0.281907047676192	-296864903517.748\\
0.282007050176254	-296826515345.474\\
0.282107052676317	-296788127173.201\\
0.282207055176379	-296749166043.132\\
0.282307057676442	-296710777870.858\\
0.282407060176504	-296672389698.584\\
0.282507062676567	-296633428568.515\\
0.282607065176629	-296595040396.241\\
0.282707067676692	-296556079266.173\\
0.282807070176754	-296517691093.899\\
0.282907072676817	-296479302921.625\\
0.283007075176879	-296440341791.556\\
0.283107077676942	-296401953619.282\\
0.283207080177004	-296362992489.213\\
0.283307082677067	-296324604316.94\\
0.283407085177129	-296285643186.871\\
0.283507087677192	-296247255014.597\\
0.283607090177254	-296208293884.528\\
0.283707092677317	-296169332754.459\\
0.283807095177379	-296130944582.185\\
0.283907097677442	-296091983452.117\\
0.284007100177504	-296053595279.843\\
0.284107102677567	-296014634149.774\\
0.284207105177629	-295975673019.705\\
0.284307107677692	-295937284847.431\\
0.284407110177754	-295898323717.362\\
0.284507112677817	-295859362587.293\\
0.284607115177879	-295820974415.02\\
0.284707117677942	-295782013284.951\\
0.284807120178004	-295743052154.882\\
0.284907122678067	-295704091024.813\\
0.285007125178129	-295665702852.539\\
0.285107127678192	-295626741722.47\\
0.285207130178254	-295587780592.401\\
0.285307132678317	-295548819462.333\\
0.285407135178379	-295510431290.059\\
0.285507137678442	-295471470159.99\\
0.285607140178504	-295432509029.921\\
0.285707142678567	-295393547899.852\\
0.285807145178629	-295354586769.783\\
0.285907147678692	-295315625639.714\\
0.286007150178754	-295276664509.645\\
0.286107152678817	-295237703379.577\\
0.286207155178879	-295198742249.508\\
0.286307157678942	-295159781119.439\\
0.286407160179004	-295121392947.165\\
0.286507162679067	-295082431817.096\\
0.286607165179129	-295043470687.027\\
0.286707167679192	-295004509556.958\\
0.286807170179255	-294964975469.094\\
0.286907172679317	-294926014339.025\\
0.287007175179379	-294887053208.956\\
0.287107177679442	-294848092078.888\\
0.287207180179504	-294809130948.819\\
0.287307182679567	-294770169818.75\\
0.287407185179629	-294731208688.681\\
0.287507187679692	-294692247558.612\\
0.287607190179755	-294653286428.543\\
0.287707192679817	-294614325298.474\\
0.28780719517988	-294574791210.61\\
0.287907197679942	-294535830080.541\\
0.288007200180004	-294496868950.472\\
0.288107202680067	-294457907820.403\\
0.288207205180129	-294418946690.335\\
0.288307207680192	-294379412602.471\\
0.288407210180255	-294340451472.402\\
0.288507212680317	-294301490342.333\\
0.28860721518038	-294261956254.469\\
0.288707217680442	-294222995124.4\\
0.288807220180505	-294184033994.331\\
0.288907222680567	-294145072864.262\\
0.289007225180629	-294105538776.398\\
0.289107227680692	-294066577646.329\\
0.289207230180755	-294027043558.465\\
0.289307232680817	-293988082428.396\\
0.28940723518088	-293949121298.327\\
0.289507237680942	-293909587210.463\\
0.289607240181005	-293870626080.394\\
0.289707242681067	-293831091992.53\\
0.28980724518113	-293792130862.461\\
0.289907247681192	-293752596774.597\\
0.290007250181255	-293713635644.529\\
0.290107252681317	-293674101556.664\\
0.29020725518138	-293635140426.596\\
0.290307257681442	-293595606338.732\\
0.290407260181505	-293556645208.663\\
0.290507262681567	-293517111120.799\\
0.29060726518163	-293478149990.73\\
0.290707267681692	-293438615902.866\\
0.290807270181755	-293399081815.002\\
0.290907272681817	-293360120684.933\\
0.29100727518188	-293320586597.069\\
0.291107277681942	-293281052509.205\\
0.291207280182005	-293242091379.136\\
0.291307282682067	-293202557291.272\\
0.29140728518213	-293163023203.408\\
0.291507287682192	-293124062073.339\\
0.291607290182255	-293084527985.475\\
0.291707292682317	-293044993897.611\\
0.29180729518238	-293005459809.747\\
0.291907297682442	-292966498679.678\\
0.292007300182505	-292926964591.814\\
0.292107302682567	-292887430503.95\\
0.29220730518263	-292847896416.086\\
0.292307307682692	-292808362328.222\\
0.292407310182755	-292769401198.153\\
0.292507312682817	-292729867110.289\\
0.29260731518288	-292690333022.425\\
0.292707317682942	-292650798934.561\\
0.292807320183005	-292611264846.697\\
0.292907322683067	-292571730758.833\\
0.29300732518313	-292532196670.969\\
0.293107327683192	-292492662583.105\\
0.293207330183255	-292453128495.241\\
0.293307332683317	-292413594407.377\\
0.29340733518338	-292374060319.513\\
0.293507337683442	-292334526231.649\\
0.293607340183505	-292294992143.785\\
0.293707342683567	-292255458055.921\\
0.29380734518363	-292215923968.057\\
0.293907347683692	-292176389880.193\\
0.294007350183755	-292136855792.328\\
0.294107352683817	-292097321704.464\\
0.29420735518388	-292057787616.6\\
0.294307357683942	-292018253528.736\\
0.294407360184005	-291978719440.872\\
0.294507362684067	-291939185353.008\\
0.29460736518413	-291899651265.144\\
0.294707367684192	-291859544219.485\\
0.294807370184255	-291820010131.621\\
0.294907372684317	-291780476043.757\\
0.29500737518438	-291740941955.893\\
0.295107377684442	-291701407868.029\\
0.295207380184505	-291661300822.37\\
0.295307382684567	-291621766734.506\\
0.29540738518463	-291582232646.642\\
0.295507387684692	-291542698558.778\\
0.295607390184755	-291502591513.119\\
0.295707392684817	-291463057425.255\\
0.29580739518488	-291423523337.391\\
0.295907397684942	-291383416291.731\\
0.296007400185005	-291343882203.867\\
0.296107402685067	-291304348116.003\\
0.29620740518513	-291264241070.344\\
0.296307407685192	-291224706982.48\\
0.296407410185255	-291184599936.821\\
0.296507412685317	-291145065848.957\\
0.29660741518538	-291105531761.093\\
0.296707417685442	-291065424715.434\\
0.296807420185505	-291025890627.57\\
0.296907422685567	-290985783581.911\\
0.29700742518563	-290946249494.047\\
0.297107427685692	-290906142448.388\\
0.297207430185755	-290866608360.523\\
0.297307432685817	-290826501314.864\\
0.29740743518588	-290786967227\\
0.297507437685942	-290746860181.341\\
0.297607440186005	-290706753135.682\\
0.297707442686067	-290667219047.818\\
0.29780744518613	-290627112002.159\\
0.297907447686192	-290587577914.295\\
0.298007450186255	-290547470868.636\\
0.298107452686317	-290507363822.976\\
0.29820745518638	-290467829735.112\\
0.298307457686442	-290427722689.453\\
0.298407460186505	-290387615643.794\\
0.298507462686567	-290348081555.93\\
0.29860746518663	-290307974510.271\\
0.298707467686692	-290267867464.612\\
0.298807470186755	-290228333376.748\\
0.298907472686817	-290188226331.089\\
0.29900747518688	-290148119285.429\\
0.299107477686942	-290108012239.77\\
0.299207480187005	-290067905194.111\\
0.299307482687067	-290028371106.247\\
0.29940748518713	-289988264060.588\\
0.299507487687192	-289948157014.929\\
0.299607490187255	-289908049969.27\\
0.299707492687317	-289867942923.61\\
0.29980749518738	-289827835877.951\\
0.299907497687442	-289788301790.087\\
0.300007500187505	-289748194744.428\\
0.300107502687567	-289708087698.769\\
0.30020750518763	-289667980653.11\\
0.300307507687692	-289627873607.451\\
0.300407510187755	-289587766561.791\\
0.300507512687817	-289547659516.132\\
0.30060751518788	-289507552470.473\\
0.300707517687942	-289467445424.814\\
0.300807520188005	-289427338379.155\\
0.300907522688067	-289387231333.496\\
0.30100752518813	-289347124287.836\\
0.301107527688192	-289307017242.177\\
0.301207530188255	-289266910196.518\\
0.301307532688317	-289226803150.859\\
0.30140753518838	-289186696105.2\\
0.301507537688442	-289146589059.541\\
0.301607540188505	-289105909056.086\\
0.301707542688567	-289065802010.427\\
0.30180754518863	-289025694964.768\\
0.301907547688692	-288985587919.109\\
0.302007550188755	-288945480873.45\\
0.302107552688817	-288905373827.791\\
0.30220755518888	-288864693824.336\\
0.302307557688942	-288824586778.677\\
0.302407560189005	-288784479733.018\\
0.302507562689067	-288744372687.359\\
0.30260756518913	-288704265641.7\\
0.302707567689192	-288663585638.245\\
0.302807570189255	-288623478592.586\\
0.302907572689317	-288583371546.927\\
0.30300757518938	-288542691543.473\\
0.303107577689442	-288502584497.814\\
0.303207580189505	-288462477452.155\\
0.303307582689567	-288421797448.7\\
0.30340758518963	-288381690403.041\\
0.303507587689692	-288341583357.382\\
0.303607590189755	-288300903353.928\\
0.303707592689817	-288260796308.268\\
0.30380759518988	-288220689262.609\\
0.303907597689942	-288180009259.155\\
0.304007600190005	-288139902213.496\\
0.304107602690067	-288099222210.042\\
0.30420760519013	-288059115164.382\\
0.304307607690192	-288018435160.928\\
0.304407610190255	-287978328115.269\\
0.304507612690317	-287937648111.815\\
0.30460761519038	-287897541066.156\\
0.304707617690442	-287856861062.701\\
0.304807620190505	-287816754017.042\\
0.304907622690567	-287776074013.588\\
0.30500762519063	-287735966967.929\\
0.305107627690692	-287695286964.474\\
0.305207630190755	-287654606961.02\\
0.305307632690817	-287614499915.361\\
0.30540763519088	-287573819911.907\\
0.305507637690942	-287533712866.247\\
0.305607640191005	-287493032862.793\\
0.305707642691067	-287452352859.339\\
0.30580764519113	-287412245813.68\\
0.305907647691192	-287371565810.225\\
0.306007650191255	-287330885806.771\\
0.306107652691317	-287290778761.112\\
0.30620765519138	-287250098757.658\\
0.306307657691442	-287209418754.203\\
0.306407660191505	-287168738750.749\\
0.306507662691567	-287128631705.09\\
0.30660766519163	-287087951701.636\\
0.306707667691692	-287047271698.181\\
0.306807670191755	-287006591694.727\\
0.306907672691817	-286965911691.273\\
0.30700767519188	-286925804645.614\\
0.307107677691942	-286885124642.159\\
0.307207680192005	-286844444638.705\\
0.307307682692067	-286803764635.251\\
0.30740768519213	-286763084631.797\\
0.307507687692192	-286722404628.342\\
0.307607690192255	-286681724624.888\\
0.307707692692317	-286641044621.434\\
0.30780769519238	-286600364617.979\\
0.307907697692442	-286560257572.32\\
0.308007700192505	-286519577568.866\\
0.308107702692567	-286478897565.412\\
0.30820770519263	-286438217561.957\\
0.308307707692692	-286397537558.503\\
0.308407710192755	-286356857555.049\\
0.308507712692817	-286316177551.594\\
0.30860771519288	-286275497548.14\\
0.308707717692942	-286234244586.891\\
0.308807720193005	-286193564583.436\\
0.308907722693067	-286152884579.982\\
0.30900772519313	-286112204576.528\\
0.309107727693192	-286071524573.074\\
0.309207730193255	-286030844569.619\\
0.309307732693317	-285990164566.165\\
0.30940773519338	-285949484562.711\\
0.309507737693442	-285908804559.256\\
0.309607740193505	-285867551598.007\\
0.309707742693567	-285826871594.553\\
0.30980774519363	-285786191591.098\\
0.309907747693692	-285745511587.644\\
0.310007750193755	-285704831584.19\\
0.310107752693817	-285664151580.736\\
0.31020775519388	-285622898619.486\\
0.310307757693942	-285582218616.032\\
0.310407760194005	-285541538612.578\\
0.310507762694067	-285500285651.328\\
0.31060776519413	-285459605647.874\\
0.310707767694192	-285418925644.42\\
0.310807770194255	-285378245640.965\\
0.310907772694317	-285336992679.716\\
0.31100777519438	-285296312676.262\\
0.311107777694442	-285255632672.807\\
0.311207780194505	-285214379711.558\\
0.311307782694567	-285173699708.104\\
0.31140778519463	-285132446746.854\\
0.311507787694692	-285091766743.4\\
0.311607790194755	-285051086739.946\\
0.311707792694817	-285009833778.696\\
0.31180779519488	-284969153775.242\\
0.311907797694942	-284927900813.992\\
0.312007800195005	-284887220810.538\\
0.312107802695067	-284845967849.289\\
0.31220780519513	-284805287845.834\\
0.312307807695192	-284764034884.585\\
0.312407810195255	-284723354881.131\\
0.312507812695317	-284682101919.881\\
0.31260781519538	-284641421916.427\\
0.312707817695442	-284600168955.178\\
0.312807820195505	-284559488951.723\\
0.312907822695567	-284518235990.474\\
0.31300782519563	-284477555987.02\\
0.313107827695692	-284436303025.77\\
0.313207830195755	-284395050064.521\\
0.313307832695817	-284354370061.067\\
0.31340783519588	-284313117099.817\\
0.313507837695942	-284271864138.568\\
0.313607840196005	-284231184135.113\\
0.313707842696067	-284189931173.864\\
0.31380784519613	-284148678212.615\\
0.313907847696192	-284107998209.16\\
0.314007850196255	-284066745247.911\\
0.314107852696317	-284025492286.661\\
0.31420785519638	-283984239325.412\\
0.314307857696442	-283943559321.958\\
0.314407860196505	-283902306360.708\\
0.314507862696567	-283861053399.459\\
0.31460786519663	-283819800438.209\\
0.314707867696692	-283779120434.755\\
0.314807870196755	-283737867473.506\\
0.314907872696817	-283696614512.256\\
0.31500787519688	-283655361551.007\\
0.315107877696942	-283614108589.758\\
0.315207880197005	-283572855628.508\\
0.315307882697067	-283532175625.054\\
0.31540788519713	-283490922663.804\\
0.315507887697192	-283449669702.555\\
0.315607890197255	-283408416741.306\\
0.315707892697317	-283367163780.056\\
0.31580789519738	-283325910818.807\\
0.315907897697442	-283284657857.557\\
0.316007900197505	-283243404896.308\\
0.316107902697567	-283202151935.058\\
0.31620790519763	-283160898973.809\\
0.316307907697692	-283119646012.56\\
0.316407910197755	-283078393051.31\\
0.316507912697817	-283037140090.061\\
0.31660791519788	-282995887128.811\\
0.316707917697942	-282954634167.562\\
0.316807920198005	-282913381206.312\\
0.316907922698067	-282872128245.063\\
0.31700792519813	-282830875283.814\\
0.317107927698192	-282789622322.564\\
0.317207930198255	-282748369361.315\\
0.317307932698317	-282707116400.065\\
0.31740793519838	-282665290481.021\\
0.317507937698442	-282624037519.771\\
0.317607940198505	-282582784558.522\\
0.317707942698567	-282541531597.273\\
0.31780794519863	-282500278636.023\\
0.317907947698692	-282459025674.774\\
0.318007950198755	-282417199755.729\\
0.318107952698817	-282375946794.48\\
0.31820795519888	-282334693833.23\\
0.318307957698942	-282293440871.981\\
0.318407960199005	-282252187910.732\\
0.318507962699068	-282210361991.687\\
0.31860796519913	-282169109030.438\\
0.318707967699192	-282127856069.188\\
0.318807970199255	-282086030150.144\\
0.318907972699317	-282044777188.894\\
0.31900797519938	-282003524227.645\\
0.319107977699442	-281962271266.395\\
0.319207980199505	-281920445347.351\\
0.319307982699568	-281879192386.101\\
0.31940798519963	-281837366467.057\\
0.319507987699693	-281796113505.807\\
0.319607990199755	-281754860544.558\\
0.319707992699817	-281713034625.513\\
0.31980799519988	-281671781664.264\\
0.319907997699942	-281629955745.219\\
0.320008000200005	-281588702783.97\\
0.320108002700068	-281547449822.721\\
0.32020800520013	-281505623903.676\\
0.320308007700193	-281464370942.427\\
0.320408010200255	-281422545023.382\\
0.320508012700318	-281381292062.133\\
0.32060801520038	-281339466143.088\\
0.320708017700442	-281298213181.839\\
0.320808020200505	-281256387262.794\\
0.320908022700568	-281215134301.545\\
0.32100802520063	-281173308382.5\\
0.321108027700693	-281131482463.456\\
0.321208030200755	-281090229502.206\\
0.321308032700818	-281048403583.162\\
0.32140803520088	-281007150621.912\\
0.321508037700943	-280965324702.868\\
0.321608040201005	-280923498783.823\\
0.321708042701068	-280882245822.574\\
0.32180804520113	-280840419903.529\\
0.321908047701193	-280799166942.28\\
0.322008050201255	-280757341023.235\\
0.322108052701318	-280715515104.191\\
0.32220805520138	-280674262142.941\\
0.322308057701443	-280632436223.897\\
0.322408060201505	-280590610304.852\\
0.322508062701568	-280548784385.808\\
0.32260806520163	-280507531424.558\\
0.322708067701693	-280465705505.514\\
0.322808070201755	-280423879586.469\\
0.322908072701818	-280382053667.424\\
0.32300807520188	-280340800706.175\\
0.323108077701943	-280298974787.131\\
0.323208080202005	-280257148868.086\\
0.323308082702068	-280215322949.041\\
0.32340808520213	-280173497029.997\\
0.323508087702193	-280131671110.952\\
0.323608090202255	-280090418149.703\\
0.323708092702318	-280048592230.658\\
0.32380809520238	-280006766311.614\\
0.323908097702443	-279964940392.569\\
0.324008100202505	-279923114473.525\\
0.324108102702568	-279881288554.48\\
0.32420810520263	-279839462635.436\\
0.324308107702693	-279797636716.391\\
0.324408110202755	-279755810797.347\\
0.324508112702818	-279713984878.302\\
0.32460811520288	-279672158959.257\\
0.324708117702943	-279630333040.213\\
0.324808120203005	-279588507121.168\\
0.324908122703068	-279546681202.124\\
0.32500812520313	-279504855283.079\\
0.325108127703193	-279463029364.035\\
0.325208130203255	-279421203444.99\\
0.325308132703318	-279379377525.946\\
0.32540813520338	-279337551606.901\\
0.325508137703443	-279295725687.857\\
0.325608140203505	-279253899768.812\\
0.325708142703568	-279212073849.767\\
0.32580814520363	-279170247930.723\\
0.325908147703693	-279128422011.678\\
0.326008150203755	-279086596092.634\\
0.326108152703818	-279044770173.589\\
0.32620815520388	-279002371296.75\\
0.326308157703943	-278960545377.705\\
0.326408160204005	-278918719458.66\\
0.326508162704068	-278876893539.616\\
0.32660816520413	-278835067620.571\\
0.326708167704193	-278793241701.527\\
0.326808170204255	-278750842824.687\\
0.326908172704318	-278709016905.643\\
0.32700817520438	-278667190986.598\\
0.327108177704443	-278625365067.553\\
0.327208180204505	-278582966190.714\\
0.327308182704568	-278541140271.669\\
0.32740818520463	-278499314352.625\\
0.327508187704693	-278457488433.58\\
0.327608190204755	-278415089556.74\\
0.327708192704818	-278373263637.696\\
0.32780819520488	-278331437718.651\\
0.327908197704943	-278289038841.812\\
0.328008200205005	-278247212922.767\\
0.328108202705068	-278205387003.723\\
0.32820820520513	-278162988126.883\\
0.328308207705193	-278121162207.838\\
0.328408210205255	-278078763330.999\\
0.328508212705318	-278036937411.954\\
0.32860821520538	-277995111492.91\\
0.328708217705443	-277952712616.07\\
0.328808220205505	-277910886697.025\\
0.328908222705568	-277868487820.186\\
0.32900822520563	-277826661901.141\\
0.329108227705693	-277784263024.301\\
0.329208230205755	-277742437105.257\\
0.329308232705818	-277700038228.417\\
0.32940823520588	-277658212309.373\\
0.329508237705943	-277615813432.533\\
0.329608240206005	-277573987513.488\\
0.329708242706068	-277531588636.649\\
0.32980824520613	-277489762717.604\\
0.329908247706193	-277447363840.764\\
0.330008250206255	-277405537921.72\\
0.330108252706318	-277363139044.88\\
0.33020825520638	-277320740168.041\\
0.330308257706443	-277278914248.996\\
0.330408260206505	-277236515372.156\\
0.330508262706568	-277194689453.112\\
0.33060826520663	-277152290576.272\\
0.330708267706693	-277109891699.432\\
0.330808270206755	-277068065780.388\\
0.330908272706818	-277025666903.548\\
0.33100827520688	-276983268026.708\\
0.331108277706943	-276941442107.664\\
0.331208280207005	-276899043230.824\\
0.331308282707068	-276856644353.985\\
0.33140828520713	-276814818434.94\\
0.331508287707193	-276772419558.1\\
0.331608290207255	-276730020681.261\\
0.331708292707318	-276687621804.421\\
0.33180829520738	-276645795885.376\\
0.331908297707443	-276603397008.537\\
0.332008300207505	-276560998131.697\\
0.332108302707568	-276518599254.857\\
0.33220830520763	-276476200378.018\\
0.332308307707693	-276434374458.973\\
0.332408310207755	-276391975582.133\\
0.332508312707818	-276349576705.294\\
0.33260831520788	-276307177828.454\\
0.332708317707943	-276264778951.614\\
0.332808320208005	-276222380074.775\\
0.332908322708068	-276179981197.935\\
0.33300832520813	-276138155278.891\\
0.333108327708193	-276095756402.051\\
0.333208330208255	-276053357525.211\\
0.333308332708318	-276010958648.371\\
0.33340833520838	-275968559771.532\\
0.333508337708443	-275926160894.692\\
0.333608340208505	-275883762017.852\\
0.333708342708568	-275841363141.013\\
0.33380834520863	-275798964264.173\\
0.333908347708693	-275756565387.333\\
0.334008350208755	-275714166510.494\\
0.334108352708818	-275671767633.654\\
0.33420835520888	-275629368756.814\\
0.334308357708943	-275586969879.975\\
0.334408360209005	-275544571003.135\\
0.334508362709068	-275502172126.295\\
0.33460836520913	-275459773249.456\\
0.334708367709193	-275417374372.616\\
0.334808370209255	-275374975495.776\\
0.334908372709318	-275332576618.937\\
0.33500837520938	-275290177742.097\\
0.335108377709443	-275247205907.462\\
0.335208380209505	-275204807030.622\\
0.335308382709568	-275162408153.783\\
0.33540838520963	-275120009276.943\\
0.335508387709693	-275077610400.103\\
0.335608390209755	-275035211523.264\\
0.335708392709818	-274992812646.424\\
0.33580839520988	-274949840811.789\\
0.335908397709943	-274907441934.95\\
0.336008400210005	-274865043058.11\\
0.336108402710068	-274822644181.27\\
0.33620840521013	-274780245304.43\\
0.336308407710193	-274737273469.796\\
0.336408410210255	-274694874592.956\\
0.336508412710318	-274652475716.116\\
0.33660841521038	-274610076839.277\\
0.336708417710443	-274567105004.642\\
0.336808420210505	-274524706127.802\\
0.336908422710568	-274482307250.962\\
0.33700842521063	-274439335416.328\\
0.337108427710693	-274396936539.488\\
0.337208430210755	-274354537662.648\\
0.337308432710818	-274311565828.013\\
0.33740843521088	-274269166951.174\\
0.337508437710943	-274226768074.334\\
0.337608440211005	-274183796239.699\\
0.337708442711068	-274141397362.86\\
0.33780844521113	-274098998486.02\\
0.337908447711193	-274056026651.385\\
0.338008450211255	-274013627774.545\\
0.338108452711318	-273970655939.911\\
0.33820845521138	-273928257063.071\\
0.338308457711443	-273885858186.231\\
0.338408460211505	-273842886351.596\\
0.338508462711568	-273800487474.757\\
0.33860846521163	-273757515640.122\\
0.338708467711693	-273715116763.282\\
0.338808470211755	-273672144928.647\\
0.338908472711818	-273629746051.808\\
0.33900847521188	-273586774217.173\\
0.339108477711943	-273544375340.333\\
0.339208480212005	-273501403505.698\\
0.339308482712068	-273459004628.859\\
0.33940848521213	-273416032794.224\\
0.339508487712193	-273373060959.589\\
0.339608490212255	-273330662082.749\\
0.339708492712318	-273287690248.115\\
0.33980849521238	-273245291371.275\\
0.339908497712443	-273202319536.64\\
0.340008500212505	-273159920659.801\\
0.340108502712568	-273116948825.166\\
0.34020850521263	-273073976990.531\\
0.340308507712693	-273031578113.691\\
0.340408510212755	-272988606279.056\\
0.340508512712818	-272945634444.422\\
0.34060851521288	-272903235567.582\\
0.340708517712943	-272860263732.947\\
0.340808520213005	-272817291898.312\\
0.340908522713068	-272774893021.473\\
0.34100852521313	-272731921186.838\\
0.341108527713193	-272688949352.203\\
0.341208530213255	-272645977517.568\\
0.341308532713318	-272603578640.728\\
0.34140853521338	-272560606806.094\\
0.341508537713443	-272517634971.459\\
0.341608540213505	-272474663136.824\\
0.341708542713568	-272432264259.984\\
0.34180854521363	-272389292425.35\\
0.341908547713693	-272346320590.715\\
0.342008550213755	-272303348756.08\\
0.342108552713818	-272260376921.445\\
0.34220855521388	-272217978044.605\\
0.342308557713943	-272175006209.971\\
0.342408560214005	-272132034375.336\\
0.342508562714068	-272089062540.701\\
0.34260856521413	-272046090706.066\\
0.342708567714193	-272003118871.431\\
0.342808570214255	-271960147036.797\\
0.342908572714318	-271917748159.957\\
0.34300857521438	-271874776325.322\\
0.343108577714443	-271831804490.687\\
0.343208580214505	-271788832656.052\\
0.343308582714568	-271745860821.418\\
0.34340858521463	-271702888986.783\\
0.343508587714693	-271659917152.148\\
0.343608590214755	-271616945317.513\\
0.343708592714818	-271573973482.878\\
0.34380859521488	-271531001648.244\\
0.343908597714943	-271488029813.609\\
0.344008600215005	-271445057978.974\\
0.344108602715068	-271402086144.339\\
0.34420860521513	-271359114309.704\\
0.344308607715193	-271316142475.07\\
0.344408610215255	-271273170640.435\\
0.344508612715318	-271230198805.8\\
0.34460861521538	-271187226971.165\\
0.344708617715443	-271144255136.53\\
0.344808620215505	-271101283301.895\\
0.344908622715568	-271058311467.261\\
0.34500862521563	-271014766674.831\\
0.345108627715693	-270971794840.196\\
0.345208630215755	-270928823005.561\\
0.345308632715818	-270885851170.926\\
0.34540863521588	-270842879336.291\\
0.345508637715943	-270799907501.657\\
0.345608640216005	-270756935667.022\\
0.345708642716068	-270713963832.387\\
0.34580864521613	-270670419039.957\\
0.345908647716193	-270627447205.322\\
0.346008650216255	-270584475370.687\\
0.346108652716318	-270541503536.053\\
0.34620865521638	-270498531701.418\\
0.346308657716443	-270454986908.988\\
0.346408660216505	-270412015074.353\\
0.346508662716568	-270369043239.718\\
0.34660866521663	-270326071405.083\\
0.346708667716693	-270282526612.654\\
0.346808670216755	-270239554778.019\\
0.346908672716818	-270196582943.384\\
0.34700867521688	-270153611108.749\\
0.347108677716943	-270110066316.319\\
0.347208680217005	-270067094481.684\\
0.347308682717068	-270024122647.049\\
0.34740868521713	-269980577854.62\\
0.347508687717193	-269937606019.985\\
0.347608690217255	-269894634185.35\\
0.347708692717318	-269851089392.92\\
0.34780869521738	-269808117558.285\\
0.347908697717443	-269765145723.65\\
0.348008700217505	-269721600931.22\\
0.348108702717568	-269678629096.586\\
0.34820870521763	-269635084304.156\\
0.348308707717693	-269592112469.521\\
0.348408710217755	-269549140634.886\\
0.348508712717818	-269505595842.456\\
0.34860871521788	-269462624007.821\\
0.348708717717943	-269419079215.391\\
0.348808720218005	-269376107380.757\\
0.348908722718068	-269332562588.327\\
0.34900872521813	-269289590753.692\\
0.349108727718193	-269246045961.262\\
0.349208730218255	-269203074126.627\\
0.349308732718318	-269159529334.197\\
0.34940873521838	-269116557499.562\\
0.349508737718443	-269073012707.132\\
0.349608740218505	-269030040872.498\\
0.349708742718568	-268986496080.068\\
0.34980874521863	-268943524245.433\\
0.349908747718693	-268899979453.003\\
0.350008750218755	-268857007618.368\\
0.350108752718818	-268813462825.938\\
0.35020875521888	-268769918033.508\\
0.350308757718943	-268726946198.873\\
0.350408760219005	-268683401406.443\\
0.350508762719068	-268640429571.809\\
0.35060876521913	-268596884779.379\\
0.350708767719193	-268553339986.949\\
0.350808770219255	-268510368152.314\\
0.350908772719318	-268466823359.884\\
0.351008775219381	-268423851525.249\\
0.351108777719443	-268380306732.819\\
0.351208780219505	-268336761940.389\\
0.351308782719568	-268293790105.754\\
0.35140878521963	-268250245313.324\\
0.351508787719693	-268206700520.895\\
0.351608790219755	-268163155728.465\\
0.351708792719818	-268120183893.83\\
0.351808795219881	-268076639101.4\\
0.351908797719943	-268033094308.97\\
0.352008800220006	-267989549516.54\\
0.352108802720068	-267946577681.905\\
0.35220880522013	-267903032889.475\\
0.352308807720193	-267859488097.045\\
0.352408810220255	-267815943304.615\\
0.352508812720318	-267772971469.98\\
0.352608815220381	-267729426677.551\\
0.352708817720443	-267685881885.121\\
0.352808820220506	-267642337092.691\\
0.352908822720568	-267598792300.261\\
0.353008825220631	-267555820465.626\\
0.353108827720693	-267512275673.196\\
0.353208830220755	-267468730880.766\\
0.353308832720818	-267425186088.336\\
0.353408835220881	-267381641295.906\\
0.353508837720943	-267338096503.476\\
0.353608840221006	-267294551711.046\\
0.353708842721068	-267251006918.616\\
0.353808845221131	-267208035083.982\\
0.353908847721193	-267164490291.552\\
0.354008850221256	-267120945499.122\\
0.354108852721318	-267077400706.692\\
0.354208855221381	-267033855914.262\\
0.354308857721443	-266990311121.832\\
0.354408860221506	-266946766329.402\\
0.354508862721568	-266903221536.972\\
0.354608865221631	-266859676744.542\\
0.354708867721693	-266816131952.112\\
0.354808870221756	-266772587159.682\\
0.354908872721818	-266729042367.252\\
0.355008875221881	-266685497574.822\\
0.355108877721943	-266641952782.392\\
0.355208880222006	-266598407989.962\\
0.355308882722068	-266554863197.532\\
0.355408885222131	-266511318405.102\\
0.355508887722193	-266467773612.672\\
0.355608890222256	-266424228820.243\\
0.355708892722318	-266380684027.813\\
0.355808895222381	-266337139235.383\\
0.355908897722443	-266293594442.953\\
0.356008900222506	-266250049650.523\\
0.356108902722568	-266205931900.298\\
0.356208905222631	-266162387107.868\\
0.356308907722693	-266118842315.438\\
0.356408910222756	-266075297523.008\\
0.356508912722818	-266031752730.578\\
0.356608915222881	-265988207938.148\\
0.356708917722943	-265944663145.718\\
0.356808920223006	-265900545395.493\\
0.356908922723068	-265857000603.063\\
0.357008925223131	-265813455810.633\\
0.357108927723193	-265769911018.203\\
0.357208930223256	-265726366225.773\\
0.357308932723318	-265682821433.343\\
0.357408935223381	-265638703683.118\\
0.357508937723443	-265595158890.688\\
0.357608940223506	-265551614098.258\\
0.357708942723568	-265508069305.828\\
0.357808945223631	-265463951555.603\\
0.357908947723693	-265420406763.173\\
0.358008950223756	-265376861970.743\\
0.358108952723818	-265333317178.313\\
0.358208955223881	-265289199428.088\\
0.358308957723943	-265245654635.658\\
0.358408960224006	-265202109843.228\\
0.358508962724068	-265157992093.003\\
0.358608965224131	-265114447300.573\\
0.358708967724193	-265070902508.144\\
0.358808970224256	-265027357715.714\\
0.358908972724318	-264983239965.489\\
0.359008975224381	-264939695173.059\\
0.359108977724443	-264895577422.833\\
0.359208980224506	-264852032630.404\\
0.359308982724568	-264808487837.974\\
0.359408985224631	-264764370087.749\\
0.359508987724693	-264720825295.319\\
0.359608990224756	-264677280502.889\\
0.359708992724818	-264633162752.664\\
0.359808995224881	-264589617960.234\\
0.359908997724943	-264545500210.009\\
0.360009000225006	-264501955417.579\\
0.360109002725068	-264457837667.354\\
0.360209005225131	-264414292874.924\\
0.360309007725193	-264370748082.494\\
0.360409010225256	-264326630332.269\\
0.360509012725318	-264283085539.839\\
0.360609015225381	-264238967789.614\\
0.360709017725443	-264195422997.184\\
0.360809020225506	-264151305246.959\\
0.360909022725568	-264107760454.529\\
0.361009025225631	-264063642704.304\\
0.361109027725693	-264020097911.874\\
0.361209030225756	-263975980161.649\\
0.361309032725818	-263932435369.219\\
0.361409035225881	-263888317618.994\\
0.361509037725943	-263844772826.564\\
0.361609040226006	-263800655076.339\\
0.361709042726068	-263756537326.113\\
0.361809045226131	-263712992533.683\\
0.361909047726193	-263668874783.458\\
0.362009050226256	-263625329991.028\\
0.362109052726318	-263581212240.803\\
0.362209055226381	-263537094490.578\\
0.362309057726443	-263493549698.148\\
0.362409060226506	-263449431947.923\\
0.362509062726568	-263405887155.493\\
0.362609065226631	-263361769405.268\\
0.362709067726693	-263317651655.043\\
0.362809070226756	-263274106862.613\\
0.362909072726818	-263229989112.388\\
0.363009075226881	-263185871362.163\\
0.363109077726943	-263142326569.733\\
0.363209080227006	-263098208819.508\\
0.363309082727068	-263054091069.283\\
0.363409085227131	-263010546276.853\\
0.363509087727193	-262966428526.628\\
0.363609090227256	-262922310776.403\\
0.363709092727318	-262878193026.178\\
0.363809095227381	-262834648233.748\\
0.363909097727443	-262790530483.523\\
0.364009100227506	-262746412733.298\\
0.364109102727568	-262702294983.073\\
0.364209105227631	-262658750190.643\\
0.364309107727693	-262614632440.418\\
0.364409110227756	-262570514690.193\\
0.364509112727818	-262526396939.968\\
0.364609115227881	-262482279189.742\\
0.364709117727943	-262438734397.313\\
0.364809120228006	-262394616647.087\\
0.364909122728068	-262350498896.862\\
0.365009125228131	-262306381146.637\\
0.365109127728193	-262262263396.412\\
0.365209130228256	-262218718603.982\\
0.365309132728318	-262174600853.757\\
0.365409135228381	-262130483103.532\\
0.365509137728443	-262086365353.307\\
0.365609140228506	-262042247603.082\\
0.365709142728568	-261998129852.857\\
0.365809145228631	-261954012102.632\\
0.365909147728693	-261909894352.407\\
0.366009150228756	-261866349559.977\\
0.366109152728818	-261822231809.752\\
0.366209155228881	-261778114059.527\\
0.366309157728943	-261733996309.302\\
0.366409160229006	-261689878559.077\\
0.366509162729068	-261645760808.851\\
0.366609165229131	-261601643058.626\\
0.366709167729193	-261557525308.401\\
0.366809170229256	-261513407558.176\\
0.366909172729318	-261469289807.951\\
0.367009175229381	-261425172057.726\\
0.367109177729443	-261381054307.501\\
0.367209180229506	-261336936557.276\\
0.367309182729568	-261292818807.051\\
0.367409185229631	-261248701056.826\\
0.367509187729693	-261204583306.601\\
0.367609190229756	-261160465556.376\\
0.367709192729818	-261116347806.151\\
0.367809195229881	-261072230055.926\\
0.367909197729943	-261028112305.7\\
0.368009200230006	-260983994555.475\\
0.368109202730068	-260939876805.25\\
0.368209205230131	-260895759055.025\\
0.368309207730193	-260851641304.8\\
0.368409210230256	-260806950596.78\\
0.368509212730318	-260762832846.555\\
0.368609215230381	-260718715096.33\\
0.368709217730443	-260674597346.105\\
0.368809220230506	-260630479595.88\\
0.368909222730568	-260586361845.655\\
0.369009225230631	-260542244095.43\\
0.369109227730693	-260498126345.204\\
0.369209230230756	-260454008594.979\\
0.369309232730818	-260409317886.959\\
0.369409235230881	-260365200136.734\\
0.369509237730943	-260321082386.509\\
0.369609240231006	-260276964636.284\\
0.369709242731068	-260232846886.059\\
0.369809245231131	-260188156178.039\\
0.369909247731193	-260144038427.814\\
0.370009250231256	-260099920677.589\\
0.370109252731318	-260055802927.363\\
0.370209255231381	-260011685177.138\\
0.370309257731443	-259966994469.118\\
0.370409260231506	-259922876718.893\\
0.370509262731568	-259878758968.668\\
0.370609265231631	-259834641218.443\\
0.370709267731693	-259789950510.423\\
0.370809270231756	-259745832760.198\\
0.370909272731818	-259701715009.973\\
0.371009275231881	-259657597259.748\\
0.371109277731943	-259612906551.727\\
0.371209280232006	-259568788801.502\\
0.371309282732068	-259524671051.277\\
0.371409285232131	-259479980343.257\\
0.371509287732193	-259435862593.032\\
0.371609290232256	-259391744842.807\\
0.371709292732318	-259347054134.787\\
0.371809295232381	-259302936384.562\\
0.371909297732443	-259258818634.336\\
0.372009300232506	-259214127926.316\\
0.372109302732568	-259170010176.091\\
0.372209305232631	-259125892425.866\\
0.372309307732693	-259081201717.846\\
0.372409310232756	-259037083967.621\\
0.372509312732818	-258992966217.396\\
0.372609315232881	-258948275509.376\\
0.372709317732943	-258904157759.15\\
0.372809320233006	-258859467051.13\\
0.372909322733068	-258815349300.905\\
0.373009325233131	-258771231550.68\\
0.373109327733193	-258726540842.66\\
0.373209330233256	-258682423092.435\\
0.373309332733318	-258637732384.415\\
0.373409335233381	-258593614634.19\\
0.373509337733443	-258548923926.169\\
0.373609340233506	-258504806175.944\\
0.373709342733568	-258460115467.924\\
0.373809345233631	-258415997717.699\\
0.373909347733693	-258371307009.679\\
0.374009350233756	-258327189259.454\\
0.374109352733818	-258282498551.434\\
0.374209355233881	-258238380801.208\\
0.374309357733943	-258193690093.188\\
0.374409360234006	-258149572342.963\\
0.374509362734068	-258104881634.943\\
0.374609365234131	-258060763884.718\\
0.374709367734193	-258016073176.698\\
0.374809370234256	-257971955426.473\\
0.374909372734318	-257927264718.452\\
0.375009375234381	-257883146968.227\\
0.375109377734443	-257838456260.207\\
0.375209380234506	-257793765552.187\\
0.375309382734568	-257749647801.962\\
0.375409385234631	-257704957093.942\\
0.375509387734693	-257660839343.717\\
0.375609390234756	-257616148635.696\\
0.375709392734818	-257571457927.676\\
0.375809395234881	-257527340177.451\\
0.375909397734943	-257482649469.431\\
0.376009400235006	-257438531719.206\\
0.376109402735068	-257393841011.186\\
0.376209405235131	-257349150303.165\\
0.376309407735193	-257305032552.94\\
0.376409410235256	-257260341844.92\\
0.376509412735318	-257215651136.9\\
0.376609415235381	-257171533386.675\\
0.376709417735443	-257126842678.655\\
0.376809420235506	-257082151970.634\\
0.376909422735568	-257038034220.409\\
0.377009425235631	-256993343512.389\\
0.377109427735693	-256948652804.369\\
0.377209430235756	-256904535054.144\\
0.377309432735818	-256859844346.124\\
0.377409435235881	-256815153638.104\\
0.377509437735943	-256770462930.083\\
0.377609440236006	-256726345179.858\\
0.377709442736068	-256681654471.838\\
0.377809445236131	-256636963763.818\\
0.377909447736193	-256592273055.798\\
0.378009450236256	-256548155305.573\\
0.378109452736318	-256503464597.552\\
0.378209455236381	-256458773889.532\\
0.378309457736443	-256414083181.512\\
0.378409460236506	-256369392473.492\\
0.378509462736568	-256325274723.267\\
0.378609465236631	-256280584015.246\\
0.378709467736693	-256235893307.226\\
0.378809470236756	-256191202599.206\\
0.378909472736818	-256146511891.186\\
0.379009475236881	-256102394140.961\\
0.379109477736943	-256057703432.941\\
0.379209480237006	-256013012724.92\\
0.379309482737068	-255968322016.9\\
0.379409485237131	-255923631308.88\\
0.379509487737193	-255878940600.86\\
0.379609490237256	-255834249892.84\\
0.379709492737318	-255790132142.614\\
0.379809495237381	-255745441434.594\\
0.379909497737443	-255700750726.574\\
0.380009500237506	-255656060018.554\\
0.380109502737568	-255611369310.534\\
0.380209505237631	-255566678602.513\\
0.380309507737693	-255521987894.493\\
0.380409510237756	-255477297186.473\\
0.380509512737818	-255432606478.453\\
0.380609515237881	-255387915770.433\\
0.380709517737943	-255343225062.412\\
0.380809520238006	-255298534354.392\\
0.380909522738068	-255254416604.167\\
0.381009525238131	-255209725896.147\\
0.381109527738193	-255165035188.127\\
0.381209530238256	-255120344480.107\\
0.381309532738318	-255075653772.086\\
0.381409535238381	-255030963064.066\\
0.381509537738443	-254986272356.046\\
0.381609540238506	-254941581648.026\\
0.381709542738568	-254896890940.005\\
0.381809545238631	-254852200231.985\\
0.381909547738693	-254807509523.965\\
0.382009550238756	-254762818815.945\\
0.382109552738818	-254718128107.925\\
0.382209555238881	-254673437399.904\\
0.382309557738943	-254628173734.089\\
0.382409560239006	-254583483026.069\\
0.382509562739068	-254538792318.049\\
0.382609565239131	-254494101610.029\\
0.382709567739193	-254449410902.008\\
0.382809570239256	-254404720193.988\\
0.382909572739318	-254360029485.968\\
0.383009575239381	-254315338777.948\\
0.383109577739443	-254270648069.928\\
0.383209580239506	-254225957361.907\\
0.383309582739568	-254181266653.887\\
0.383409585239631	-254136575945.867\\
0.383509587739694	-254091312280.052\\
0.383609590239756	-254046621572.031\\
0.383709592739818	-254001930864.011\\
0.383809595239881	-253957240155.991\\
0.383909597739943	-253912549447.971\\
0.384009600240006	-253867858739.951\\
0.384109602740069	-253823168031.93\\
0.384209605240131	-253777904366.115\\
0.384309607740194	-253733213658.095\\
0.384409610240256	-253688522950.075\\
0.384509612740319	-253643832242.054\\
0.384609615240381	-253599141534.034\\
0.384709617740443	-253554450826.014\\
0.384809620240506	-253509187160.199\\
0.384909622740569	-253464496452.178\\
0.385009625240631	-253419805744.158\\
0.385109627740694	-253375115036.138\\
0.385209630240756	-253330424328.118\\
0.385309632740819	-253285160662.302\\
0.385409635240881	-253240469954.282\\
0.385509637740944	-253195779246.262\\
0.385609640241006	-253151088538.242\\
0.385709642741069	-253105824872.427\\
0.385809645241131	-253061134164.406\\
0.385909647741194	-253016443456.386\\
0.386009650241256	-252971752748.366\\
0.386109652741319	-252926489082.551\\
0.386209655241381	-252881798374.53\\
0.386309657741444	-252837107666.51\\
0.386409660241506	-252791844000.695\\
0.386509662741569	-252747153292.675\\
0.386609665241631	-252702462584.654\\
0.386709667741694	-252657198918.839\\
0.386809670241756	-252612508210.819\\
0.386909672741819	-252567817502.799\\
0.387009675241881	-252522553836.983\\
0.387109677741944	-252477863128.963\\
0.387209680242006	-252433172420.943\\
0.387309682742069	-252387908755.128\\
0.387409685242131	-252343218047.107\\
0.387509687742194	-252298527339.087\\
0.387609690242256	-252253263673.272\\
0.387709692742319	-252208572965.252\\
0.387809695242381	-252163882257.231\\
0.387909697742444	-252118618591.416\\
0.388009700242506	-252073927883.396\\
0.388109702742569	-252028664217.581\\
0.388209705242631	-251983973509.56\\
0.388309707742694	-251939282801.54\\
0.388409710242756	-251894019135.725\\
0.388509712742819	-251849328427.705\\
0.388609715242881	-251804064761.889\\
0.388709717742944	-251759374053.869\\
0.388809720243006	-251714683345.849\\
0.388909722743069	-251669419680.034\\
0.389009725243131	-251624728972.013\\
0.389109727743194	-251579465306.198\\
0.389209730243256	-251534774598.178\\
0.389309732743319	-251489510932.362\\
0.389409735243381	-251444820224.342\\
0.389509737743444	-251399556558.527\\
0.389609740243506	-251354865850.507\\
0.389709742743569	-251309602184.691\\
0.389809745243631	-251264911476.671\\
0.389909747743694	-251219647810.856\\
0.390009750243756	-251174957102.836\\
0.390109752743819	-251129693437.02\\
0.390209755243881	-251085002729\\
0.390309757743944	-251039739063.185\\
0.390409760244006	-250995048355.165\\
0.390509762744069	-250949784689.349\\
0.390609765244131	-250905093981.329\\
0.390709767744194	-250859830315.514\\
0.390809770244256	-250815139607.494\\
0.390909772744319	-250769875941.678\\
0.391009775244381	-250725185233.658\\
0.391109777744444	-250679921567.843\\
0.391209780244506	-250634657902.027\\
0.391309782744569	-250589967194.007\\
0.391409785244631	-250544703528.192\\
0.391509787744694	-250500012820.172\\
0.391609790244756	-250454749154.356\\
0.391709792744819	-250409485488.541\\
0.391809795244881	-250364794780.521\\
0.391909797744944	-250319531114.705\\
0.392009800245006	-250274840406.685\\
0.392109802745069	-250229576740.87\\
0.392209805245131	-250184313075.054\\
0.392309807745194	-250139622367.034\\
0.392409810245256	-250094358701.219\\
0.392509812745319	-250049667993.199\\
0.392609815245381	-250004404327.383\\
0.392709817745444	-249959140661.568\\
0.392809820245506	-249914449953.548\\
0.392909822745569	-249869186287.733\\
0.393009825245631	-249823922621.917\\
0.393109827745694	-249779231913.897\\
0.393209830245756	-249733968248.082\\
0.393309832745819	-249688704582.266\\
0.393409835245881	-249643440916.451\\
0.393509837745944	-249598750208.431\\
0.393609840246006	-249553486542.615\\
0.393709842746069	-249508222876.8\\
0.393809845246131	-249463532168.78\\
0.393909847746194	-249418268502.965\\
0.394009850246256	-249373004837.149\\
0.394109852746319	-249328314129.129\\
0.394209855246381	-249283050463.314\\
0.394309857746444	-249237786797.498\\
0.394409860246506	-249192523131.683\\
0.394509862746569	-249147832423.663\\
0.394609865246631	-249102568757.848\\
0.394709867746694	-249057305092.032\\
0.394809870246756	-249012041426.217\\
0.394909872746819	-248967350718.197\\
0.395009875246881	-248922087052.381\\
0.395109877746944	-248876823386.566\\
0.395209880247006	-248831559720.751\\
0.395309882747069	-248786296054.935\\
0.395409885247131	-248741605346.915\\
0.395509887747194	-248696341681.1\\
0.395609890247256	-248651078015.284\\
0.395709892747319	-248605814349.469\\
0.395809895247381	-248560550683.654\\
0.395909897747444	-248515859975.634\\
0.396009900247506	-248470596309.818\\
0.396109902747569	-248425332644.003\\
0.396209905247631	-248380068978.188\\
0.396309907747694	-248334805312.372\\
0.396409910247756	-248289541646.557\\
0.396509912747819	-248244850938.537\\
0.396609915247881	-248199587272.721\\
0.396709917747944	-248154323606.906\\
0.396809920248006	-248109059941.091\\
0.396909922748069	-248063796275.275\\
0.397009925248131	-248018532609.46\\
0.397109927748194	-247973268943.645\\
0.397209930248256	-247928005277.829\\
0.397309932748319	-247883314569.809\\
0.397409935248381	-247838050903.994\\
0.397509937748444	-247792787238.178\\
0.397609940248506	-247747523572.363\\
0.397709942748569	-247702259906.548\\
0.397809945248631	-247656996240.732\\
0.397909947748694	-247611732574.917\\
0.398009950248756	-247566468909.102\\
0.398109952748819	-247521205243.286\\
0.398209955248881	-247475941577.471\\
0.398309957748944	-247430677911.656\\
0.398409960249006	-247385414245.84\\
0.398509962749069	-247340150580.025\\
0.398609965249131	-247294886914.21\\
0.398709967749194	-247249623248.394\\
0.398809970249256	-247204932540.374\\
0.398909972749319	-247159668874.559\\
0.399009975249381	-247114405208.744\\
0.399109977749444	-247069141542.928\\
0.399209980249506	-247023877877.113\\
0.399309982749569	-246978614211.298\\
0.399409985249631	-246933350545.482\\
0.399509987749694	-246888086879.667\\
0.399609990249756	-246842823213.852\\
0.399709992749819	-246797559548.036\\
0.399809995249881	-246752295882.221\\
0.399909997749944	-246707032216.406\\
0.400010000250006	-246661768550.59\\
};
\addplot [color=mycolor3,solid,forget plot]
  table[row sep=crcr]{%
0.400010000250006	-246661768550.59\\
0.400110002750069	-246616504884.775\\
0.400210005250131	-246570668261.164\\
0.400310007750194	-246525404595.349\\
0.400410010250256	-246480140929.534\\
0.400510012750319	-246434877263.718\\
0.400610015250381	-246389613597.903\\
0.400710017750444	-246344349932.088\\
0.400810020250506	-246299086266.272\\
0.400910022750569	-246253822600.457\\
0.401010025250631	-246208558934.642\\
0.401110027750694	-246163295268.826\\
0.401210030250756	-246118031603.011\\
0.401310032750819	-246072767937.196\\
0.401410035250881	-246027504271.38\\
0.401510037750944	-245982240605.565\\
0.401610040251006	-245936403981.955\\
0.401710042751069	-245891140316.139\\
0.401810045251131	-245845876650.324\\
0.401910047751194	-245800612984.509\\
0.402010050251256	-245755349318.693\\
0.402110052751319	-245710085652.878\\
0.402210055251381	-245664821987.063\\
0.402310057751444	-245619558321.247\\
0.402410060251506	-245574294655.432\\
0.402510062751569	-245528458031.821\\
0.402610065251631	-245483194366.006\\
0.402710067751694	-245437930700.191\\
0.402810070251756	-245392667034.375\\
0.402910072751819	-245347403368.56\\
0.403010075251881	-245302139702.745\\
0.403110077751944	-245256303079.134\\
0.403210080252006	-245211039413.319\\
0.403310082752069	-245165775747.504\\
0.403410085252131	-245120512081.688\\
0.403510087752194	-245075248415.873\\
0.403610090252256	-245029411792.262\\
0.403710092752319	-244984148126.447\\
0.403810095252381	-244938884460.632\\
0.403910097752444	-244893620794.816\\
0.404010100252506	-244848357129.001\\
0.404110102752569	-244802520505.391\\
0.404210105252631	-244757256839.575\\
0.404310107752694	-244711993173.76\\
0.404410110252756	-244666729507.945\\
0.404510112752819	-244621465842.129\\
0.404610115252881	-244575629218.519\\
0.404710117752944	-244530365552.704\\
0.404810120253006	-244485101886.888\\
0.404910122753069	-244439838221.073\\
0.405010125253131	-244394001597.462\\
0.405110127753194	-244348737931.647\\
0.405210130253256	-244303474265.832\\
0.405310132753319	-244258210600.016\\
0.405410135253381	-244212373976.406\\
0.405510137753444	-244167110310.591\\
0.405610140253506	-244121846644.775\\
0.405710142753569	-244076010021.165\\
0.405810145253631	-244030746355.349\\
0.405910147753694	-243985482689.534\\
0.406010150253756	-243940219023.719\\
0.406110152753819	-243894382400.108\\
0.406210155253881	-243849118734.293\\
0.406310157753944	-243803855068.478\\
0.406410160254006	-243758018444.867\\
0.406510162754069	-243712754779.052\\
0.406610165254131	-243667491113.237\\
0.406710167754194	-243621654489.626\\
0.406810170254256	-243576390823.811\\
0.406910172754319	-243531127157.995\\
0.407010175254381	-243485290534.385\\
0.407110177754444	-243440026868.57\\
0.407210180254506	-243394763202.754\\
0.407310182754569	-243348926579.144\\
0.407410185254631	-243303662913.328\\
0.407510187754694	-243258399247.513\\
0.407610190254756	-243212562623.903\\
0.407710192754819	-243167298958.087\\
0.407810195254881	-243121462334.477\\
0.407910197754944	-243076198668.661\\
0.408010200255006	-243030935002.846\\
0.408110202755069	-242985098379.236\\
0.408210205255131	-242939834713.42\\
0.408310207755194	-242893998089.81\\
0.408410210255256	-242848734423.995\\
0.408510212755319	-242803470758.179\\
0.408610215255381	-242757634134.569\\
0.408710217755444	-242712370468.753\\
0.408810220255506	-242666533845.143\\
0.408910222755569	-242621270179.328\\
0.409010225255631	-242576006513.512\\
0.409110227755694	-242530169889.902\\
0.409210230255756	-242484906224.086\\
0.409310232755819	-242439069600.476\\
0.409410235255881	-242393805934.661\\
0.409510237755944	-242347969311.05\\
0.409610240256006	-242302705645.235\\
0.409710242756069	-242257441979.42\\
0.409810245256131	-242211605355.809\\
0.409910247756194	-242166341689.994\\
0.410010250256256	-242120505066.383\\
0.410110252756319	-242075241400.568\\
0.410210255256381	-242029404776.957\\
0.410310257756444	-241984141111.142\\
0.410410260256506	-241938304487.532\\
0.410510262756569	-241893040821.716\\
0.410610265256631	-241847204198.106\\
0.410710267756694	-241801940532.291\\
0.410810270256756	-241756103908.68\\
0.410910272756819	-241710840242.865\\
0.411010275256881	-241665003619.254\\
0.411110277756944	-241619739953.439\\
0.411210280257006	-241573903329.828\\
0.411310282757069	-241528639664.013\\
0.411410285257131	-241482803040.403\\
0.411510287757194	-241437539374.587\\
0.411610290257256	-241391702750.977\\
0.411710292757319	-241346439085.162\\
0.411810295257381	-241300602461.551\\
0.411910297757444	-241255338795.736\\
0.412010300257506	-241209502172.125\\
0.412110302757569	-241163665548.515\\
0.412210305257631	-241118401882.699\\
0.412310307757694	-241072565259.089\\
0.412410310257756	-241027301593.274\\
0.412510312757819	-240981464969.663\\
0.412610315257881	-240936201303.848\\
0.412710317757944	-240890364680.237\\
0.412810320258006	-240845101014.422\\
0.412910322758069	-240799264390.812\\
0.413010325258131	-240753427767.201\\
0.413110327758194	-240708164101.386\\
0.413210330258256	-240662327477.775\\
0.413310332758319	-240617063811.96\\
0.413410335258381	-240571227188.35\\
0.413510337758444	-240525390564.739\\
0.413610340258506	-240480126898.924\\
0.413710342758569	-240434290275.313\\
0.413810345258631	-240389026609.498\\
0.413910347758694	-240343189985.887\\
0.414010350258756	-240297353362.277\\
0.414110352758819	-240252089696.462\\
0.414210355258881	-240206253072.851\\
0.414310357758944	-240160416449.241\\
0.414410360259006	-240115152783.425\\
0.414510362759069	-240069316159.815\\
0.414610365259131	-240024052494\\
0.414710367759194	-239978215870.389\\
0.414810370259256	-239932379246.779\\
0.414910372759319	-239887115580.963\\
0.415010375259382	-239841278957.353\\
0.415110377759444	-239795442333.742\\
0.415210380259506	-239750178667.927\\
0.415310382759569	-239704342044.317\\
0.415410385259631	-239658505420.706\\
0.415510387759694	-239613241754.891\\
0.415610390259756	-239567405131.28\\
0.415710392759819	-239521568507.67\\
0.415810395259882	-239476304841.855\\
0.415910397759944	-239430468218.244\\
0.416010400260007	-239384631594.634\\
0.416110402760069	-239339367928.818\\
0.416210405260131	-239293531305.208\\
0.416310407760194	-239247694681.597\\
0.416410410260256	-239201858057.987\\
0.416510412760319	-239156594392.172\\
0.416610415260382	-239110757768.561\\
0.416710417760444	-239064921144.951\\
0.416810420260507	-239019657479.135\\
0.416910422760569	-238973820855.525\\
0.417010425260632	-238927984231.914\\
0.417110427760694	-238882147608.304\\
0.417210430260756	-238836883942.489\\
0.417310432760819	-238791047318.878\\
0.417410435260882	-238745210695.268\\
0.417510437760944	-238699374071.657\\
0.417610440261007	-238654110405.842\\
0.417710442761069	-238608273782.231\\
0.417810445261132	-238562437158.621\\
0.417910447761194	-238516600535.01\\
0.418010450261257	-238471336869.195\\
0.418110452761319	-238425500245.585\\
0.418210455261382	-238379663621.974\\
0.418310457761444	-238333826998.364\\
0.418410460261507	-238288563332.548\\
0.418510462761569	-238242726708.938\\
0.418610465261632	-238196890085.327\\
0.418710467761694	-238151053461.717\\
0.418810470261757	-238105216838.106\\
0.418910472761819	-238059953172.291\\
0.419010475261882	-238014116548.681\\
0.419110477761944	-237968279925.07\\
0.419210480262007	-237922443301.46\\
0.419310482762069	-237876606677.849\\
0.419410485262132	-237831343012.034\\
0.419510487762194	-237785506388.423\\
0.419610490262257	-237739669764.813\\
0.419710492762319	-237693833141.203\\
0.419810495262382	-237647996517.592\\
0.419910497762444	-237602732851.777\\
0.420010500262507	-237556896228.166\\
0.420110502762569	-237511059604.556\\
0.420210505262632	-237465222980.945\\
0.420310507762694	-237419386357.335\\
0.420410510262757	-237373549733.724\\
0.420510512762819	-237328286067.909\\
0.420610515262882	-237282449444.299\\
0.420710517762944	-237236612820.688\\
0.420810520263007	-237190776197.078\\
0.420910522763069	-237144939573.467\\
0.421010525263132	-237099102949.857\\
0.421110527763194	-237053839284.041\\
0.421210530263257	-237008002660.431\\
0.421310532763319	-236962166036.82\\
0.421410535263382	-236916329413.21\\
0.421510537763444	-236870492789.6\\
0.421610540263507	-236824656165.989\\
0.421710542763569	-236778819542.379\\
0.421810545263632	-236732982918.768\\
0.421910547763694	-236687719252.953\\
0.422010550263757	-236641882629.342\\
0.422110552763819	-236596046005.732\\
0.422210555263882	-236550209382.121\\
0.422310557763944	-236504372758.511\\
0.422410560264007	-236458536134.9\\
0.422510562764069	-236412699511.29\\
0.422610565264132	-236366862887.68\\
0.422710567764194	-236321026264.069\\
0.422810570264257	-236275189640.459\\
0.422910572764319	-236229925974.643\\
0.423010575264382	-236184089351.033\\
0.423110577764444	-236138252727.422\\
0.423210580264507	-236092416103.812\\
0.423310582764569	-236046579480.201\\
0.423410585264632	-236000742856.591\\
0.423510587764694	-235954906232.98\\
0.423610590264757	-235909069609.37\\
0.423710592764819	-235863232985.76\\
0.423810595264882	-235817396362.149\\
0.423910597764944	-235771559738.539\\
0.424010600265007	-235725723114.928\\
0.424110602765069	-235679886491.318\\
0.424210605265132	-235634049867.707\\
0.424310607765194	-235588213244.097\\
0.424410610265257	-235542376620.486\\
0.424510612765319	-235497112954.671\\
0.424610615265382	-235451276331.06\\
0.424710617765444	-235405439707.45\\
0.424810620265507	-235359603083.84\\
0.424910622765569	-235313766460.229\\
0.425010625265632	-235267929836.619\\
0.425110627765694	-235222093213.008\\
0.425210630265757	-235176256589.398\\
0.425310632765819	-235130419965.787\\
0.425410635265882	-235084583342.177\\
0.425510637765944	-235038746718.566\\
0.425610640266007	-234992910094.956\\
0.425710642766069	-234947073471.345\\
0.425810645266132	-234901236847.735\\
0.425910647766194	-234855400224.124\\
0.426010650266257	-234809563600.514\\
0.426110652766319	-234763726976.904\\
0.426210655266382	-234717890353.293\\
0.426310657766444	-234672053729.683\\
0.426410660266507	-234626217106.072\\
0.426510662766569	-234580380482.462\\
0.426610665266632	-234534543858.851\\
0.426710667766694	-234488707235.241\\
0.426810670266757	-234442870611.63\\
0.426910672766819	-234397033988.02\\
0.427010675266882	-234351197364.409\\
0.427110677766944	-234305360740.799\\
0.427210680267007	-234259524117.188\\
0.427310682767069	-234213114535.783\\
0.427410685267132	-234167277912.172\\
0.427510687767194	-234121441288.562\\
0.427610690267257	-234075604664.951\\
0.427710692767319	-234029768041.341\\
0.427810695267382	-233983931417.73\\
0.427910697767444	-233938094794.12\\
0.428010700267507	-233892258170.51\\
0.428110702767569	-233846421546.899\\
0.428210705267632	-233800584923.289\\
0.428310707767694	-233754748299.678\\
0.428410710267757	-233708911676.068\\
0.428510712767819	-233663075052.457\\
0.428610715267882	-233617238428.847\\
0.428710717767944	-233571401805.236\\
0.428810720268007	-233525565181.626\\
0.428910722768069	-233479728558.015\\
0.429010725268132	-233433318976.61\\
0.429110727768194	-233387482352.999\\
0.429210730268257	-233341645729.389\\
0.429310732768319	-233295809105.778\\
0.429410735268382	-233249972482.168\\
0.429510737768444	-233204135858.557\\
0.429610740268507	-233158299234.947\\
0.429710742768569	-233112462611.336\\
0.429810745268632	-233066625987.726\\
0.429910747768694	-233020789364.116\\
0.430010750268757	-232974952740.505\\
0.430110752768819	-232928543159.099\\
0.430210755268882	-232882706535.489\\
0.430310757768944	-232836869911.879\\
0.430410760269007	-232791033288.268\\
0.430510762769069	-232745196664.658\\
0.430610765269132	-232699360041.047\\
0.430710767769194	-232653523417.437\\
0.430810770269257	-232607686793.826\\
0.430910772769319	-232561277212.421\\
0.431010775269382	-232515440588.81\\
0.431110777769444	-232469603965.2\\
0.431210780269507	-232423767341.589\\
0.431310782769569	-232377930717.979\\
0.431410785269632	-232332094094.368\\
0.431510787769694	-232286257470.758\\
0.431610790269757	-232240420847.147\\
0.431710792769819	-232194011265.742\\
0.431810795269882	-232148174642.131\\
0.431910797769944	-232102338018.521\\
0.432010800270007	-232056501394.91\\
0.432110802770069	-232010664771.3\\
0.432210805270132	-231964828147.689\\
0.432310807770194	-231918418566.284\\
0.432410810270257	-231872581942.673\\
0.432510812770319	-231826745319.063\\
0.432610815270382	-231780908695.452\\
0.432710817770444	-231735072071.842\\
0.432810820270507	-231689235448.232\\
0.432910822770569	-231642825866.826\\
0.433010825270632	-231596989243.215\\
0.433110827770694	-231551152619.605\\
0.433210830270757	-231505315995.995\\
0.433310832770819	-231459479372.384\\
0.433410835270882	-231413642748.774\\
0.433510837770944	-231367233167.368\\
0.433610840271007	-231321396543.758\\
0.433710842771069	-231275559920.147\\
0.433810845271132	-231229723296.537\\
0.433910847771194	-231183886672.926\\
0.434010850271257	-231137477091.521\\
0.434110852771319	-231091640467.91\\
0.434210855271382	-231045803844.3\\
0.434310857771444	-230999967220.689\\
0.434410860271507	-230954130597.079\\
0.434510862771569	-230907721015.673\\
0.434610865271632	-230861884392.063\\
0.434710867771694	-230816047768.452\\
0.434810870271757	-230770211144.842\\
0.434910872771819	-230724374521.231\\
0.435010875271882	-230677964939.826\\
0.435110877771944	-230632128316.215\\
0.435210880272007	-230586291692.605\\
0.435310882772069	-230540455068.994\\
0.435410885272132	-230494045487.589\\
0.435510887772194	-230448208863.978\\
0.435610890272257	-230402372240.368\\
0.435710892772319	-230356535616.757\\
0.435810895272382	-230310698993.147\\
0.435910897772444	-230264289411.741\\
0.436010900272507	-230218452788.131\\
0.436110902772569	-230172616164.52\\
0.436210905272632	-230126779540.91\\
0.436310907772694	-230080369959.504\\
0.436410910272757	-230034533335.894\\
0.436510912772819	-229988696712.283\\
0.436610915272882	-229942860088.673\\
0.436710917772944	-229896450507.267\\
0.436810920273007	-229850613883.657\\
0.436910922773069	-229804777260.046\\
0.437010925273132	-229758940636.436\\
0.437110927773194	-229712531055.03\\
0.437210930273257	-229666694431.42\\
0.437310932773319	-229620857807.809\\
0.437410935273382	-229575021184.199\\
0.437510937773444	-229528611602.793\\
0.437610940273507	-229482774979.183\\
0.437710942773569	-229436938355.572\\
0.437810945273632	-229390528774.167\\
0.437910947773694	-229344692150.556\\
0.438010950273757	-229298855526.946\\
0.438110952773819	-229253018903.335\\
0.438210955273882	-229206609321.93\\
0.438310957773944	-229160772698.319\\
0.438410960274007	-229114936074.709\\
0.438510962774069	-229069099451.098\\
0.438610965274132	-229022689869.693\\
0.438710967774194	-228976853246.082\\
0.438810970274257	-228931016622.472\\
0.438910972774319	-228884607041.066\\
0.439010975274382	-228838770417.456\\
0.439110977774444	-228792933793.845\\
0.439210980274507	-228746524212.44\\
0.439310982774569	-228700687588.829\\
0.439410985274632	-228654850965.219\\
0.439510987774694	-228609014341.608\\
0.439610990274757	-228562604760.203\\
0.439710992774819	-228516768136.592\\
0.439810995274882	-228470931512.982\\
0.439910997774944	-228424521931.576\\
0.440011000275007	-228378685307.966\\
0.440111002775069	-228332848684.355\\
0.440211005275132	-228286439102.95\\
0.440311007775194	-228240602479.339\\
0.440411010275257	-228194765855.729\\
0.440511012775319	-228148929232.118\\
0.440611015275382	-228102519650.713\\
0.440711017775444	-228056683027.102\\
0.440811020275507	-228010846403.492\\
0.440911022775569	-227964436822.086\\
0.441011025275632	-227918600198.476\\
0.441111027775694	-227872763574.865\\
0.441211030275757	-227826353993.46\\
0.441311032775819	-227780517369.849\\
0.441411035275882	-227734680746.239\\
0.441511037775944	-227688271164.833\\
0.441611040276007	-227642434541.223\\
0.441711042776069	-227596597917.612\\
0.441811045276132	-227550188336.206\\
0.441911047776194	-227504351712.596\\
0.442011050276257	-227458515088.986\\
0.442111052776319	-227412105507.58\\
0.442211055276382	-227366268883.969\\
0.442311057776444	-227320432260.359\\
0.442411060276507	-227274022678.953\\
0.442511062776569	-227228186055.343\\
0.442611065276632	-227181776473.937\\
0.442711067776694	-227135939850.327\\
0.442811070276757	-227090103226.716\\
0.442911072776819	-227043693645.311\\
0.443011075276882	-226997857021.7\\
0.443111077776944	-226952020398.09\\
0.443211080277007	-226905610816.684\\
0.443311082777069	-226859774193.074\\
0.443411085277132	-226813937569.463\\
0.443511087777194	-226767527988.058\\
0.443611090277257	-226721691364.447\\
0.443711092777319	-226675854740.837\\
0.443811095277382	-226629445159.431\\
0.443911097777444	-226583608535.821\\
0.444011100277507	-226537198954.415\\
0.444111102777569	-226491362330.805\\
0.444211105277632	-226445525707.194\\
0.444311107777694	-226399116125.789\\
0.444411110277757	-226353279502.178\\
0.444511112777819	-226307442878.568\\
0.444611115277882	-226261033297.162\\
0.444711117777944	-226215196673.552\\
0.444811120278007	-226168787092.146\\
0.444911122778069	-226122950468.536\\
0.445011125278132	-226077113844.925\\
0.445111127778194	-226030704263.52\\
0.445211130278257	-225984867639.909\\
0.445311132778319	-225939031016.299\\
0.445411135278382	-225892621434.893\\
0.445511137778444	-225846784811.283\\
0.445611140278507	-225800375229.877\\
0.445711142778569	-225754538606.266\\
0.445811145278632	-225708701982.656\\
0.445911147778694	-225662292401.25\\
0.446011150278757	-225616455777.64\\
0.446111152778819	-225570046196.234\\
0.446211155278882	-225524209572.624\\
0.446311157778944	-225478372949.013\\
0.446411160279007	-225431963367.608\\
0.446511162779069	-225386126743.997\\
0.446611165279132	-225339717162.592\\
0.446711167779195	-225293880538.981\\
0.446811170279257	-225248043915.371\\
0.446911172779319	-225201634333.965\\
0.447011175279382	-225155797710.355\\
0.447111177779444	-225109388128.949\\
0.447211180279507	-225063551505.339\\
0.447311182779569	-225017714881.728\\
0.447411185279632	-224971305300.323\\
0.447511187779695	-224925468676.712\\
0.447611190279757	-224879059095.307\\
0.44771119277982	-224833222471.696\\
0.447811195279882	-224786812890.29\\
0.447911197779944	-224740976266.68\\
0.448011200280007	-224695139643.07\\
0.448111202780069	-224648730061.664\\
0.448211205280132	-224602893438.053\\
0.448311207780195	-224556483856.648\\
0.448411210280257	-224510647233.037\\
0.44851121278032	-224464810609.427\\
0.448611215280382	-224418401028.021\\
0.448711217780445	-224372564404.411\\
0.448811220280507	-224326154823.005\\
0.448911222780569	-224280318199.395\\
0.449011225280632	-224233908617.989\\
0.449111227780695	-224188071994.379\\
0.449211230280757	-224142235370.768\\
0.44931123278082	-224095825789.363\\
0.449411235280882	-224049989165.752\\
0.449511237780945	-224003579584.347\\
0.449611240281007	-223957742960.736\\
0.44971124278107	-223911333379.331\\
0.449811245281132	-223865496755.72\\
0.449911247781195	-223819087174.315\\
0.450011250281257	-223773250550.704\\
0.45011125278132	-223727413927.094\\
0.450211255281382	-223681004345.688\\
0.450311257781445	-223635167722.078\\
0.450411260281507	-223588758140.672\\
0.45051126278157	-223542921517.061\\
0.450611265281632	-223496511935.656\\
0.450711267781695	-223450675312.045\\
0.450811270281757	-223404838688.435\\
0.45091127278182	-223358429107.029\\
0.451011275281882	-223312592483.419\\
0.451111277781945	-223266182902.013\\
0.451211280282007	-223220346278.403\\
0.45131128278207	-223173936696.997\\
0.451411285282132	-223128100073.387\\
0.451511287782195	-223081690491.981\\
0.451611290282257	-223035853868.371\\
0.45171129278232	-222989444286.965\\
0.451811295282382	-222943607663.355\\
0.451911297782445	-222897771039.744\\
0.452011300282507	-222851361458.339\\
0.45211130278257	-222805524834.728\\
0.452211305282632	-222759115253.323\\
0.452311307782695	-222713278629.712\\
0.452411310282757	-222666869048.306\\
0.45251131278282	-222621032424.696\\
0.452611315282882	-222574622843.29\\
0.452711317782945	-222528786219.68\\
0.452811320283007	-222482376638.274\\
0.45291132278307	-222436540014.664\\
0.453011325283132	-222390703391.053\\
0.453111327783195	-222344293809.648\\
0.453211330283257	-222298457186.037\\
0.45331133278332	-222252047604.632\\
0.453411335283382	-222206210981.021\\
0.453511337783445	-222159801399.616\\
0.453611340283507	-222113964776.005\\
0.45371134278357	-222067555194.6\\
0.453811345283632	-222021718570.989\\
0.453911347783695	-221975308989.584\\
0.454011350283757	-221929472365.973\\
0.45411135278382	-221883062784.567\\
0.454211355283882	-221837226160.957\\
0.454311357783945	-221790816579.551\\
0.454411360284007	-221744979955.941\\
0.45451136278407	-221699143332.33\\
0.454611365284132	-221652733750.925\\
0.454711367784195	-221606897127.314\\
0.454811370284257	-221560487545.909\\
0.45491137278432	-221514650922.298\\
0.455011375284382	-221468241340.893\\
0.455111377784445	-221422404717.282\\
0.455211380284507	-221375995135.877\\
0.45531138278457	-221330158512.266\\
0.455411385284632	-221283748930.861\\
0.455511387784695	-221237912307.25\\
0.455611390284757	-221191502725.845\\
0.45571139278482	-221145666102.234\\
0.455811395284882	-221099256520.829\\
0.455911397784945	-221053419897.218\\
0.456011400285007	-221007010315.812\\
0.45611140278507	-220961173692.202\\
0.456211405285132	-220914764110.796\\
0.456311407785195	-220868927487.186\\
0.456411410285257	-220822517905.78\\
0.45651141278532	-220776681282.17\\
0.456611415285382	-220730271700.764\\
0.456711417785445	-220684435077.154\\
0.456811420285507	-220638598453.543\\
0.45691142278557	-220592188872.138\\
0.457011425285632	-220546352248.527\\
0.457111427785695	-220499942667.122\\
0.457211430285757	-220454106043.511\\
0.45731143278582	-220407696462.106\\
0.457411435285882	-220361859838.495\\
0.457511437785945	-220315450257.09\\
0.457611440286007	-220269613633.479\\
0.45771144278607	-220223204052.073\\
0.457811445286132	-220177367428.463\\
0.457911447786195	-220130957847.057\\
0.458011450286257	-220085121223.447\\
0.45811145278632	-220038711642.041\\
0.458211455286382	-219992875018.431\\
0.458311457786445	-219946465437.025\\
0.458411460286507	-219900628813.415\\
0.45851146278657	-219854219232.009\\
0.458611465286632	-219808382608.399\\
0.458711467786695	-219761973026.993\\
0.458811470286757	-219716136403.383\\
0.45891147278682	-219669726821.977\\
0.459011475286882	-219623890198.367\\
0.459111477786945	-219577480616.961\\
0.459211480287007	-219531643993.351\\
0.45931148278707	-219485234411.945\\
0.459411485287132	-219439397788.335\\
0.459511487787195	-219392988206.929\\
0.459611490287257	-219347151583.318\\
0.45971149278732	-219300742001.913\\
0.459811495287382	-219254905378.302\\
0.459911497787445	-219208495796.897\\
0.460011500287507	-219162659173.286\\
0.46011150278757	-219116249591.881\\
0.460211505287632	-219070412968.27\\
0.460311507787695	-219024003386.865\\
0.460411510287757	-218978166763.254\\
0.46051151278782	-218931757181.849\\
0.460611515287882	-218885920558.238\\
0.460711517787945	-218839510976.833\\
0.460811520288007	-218793674353.222\\
0.46091152278807	-218747264771.816\\
0.461011525288132	-218701428148.206\\
0.461111527788195	-218655018566.8\\
0.461211530288257	-218609181943.19\\
0.46131153278832	-218562772361.784\\
0.461411535288382	-218516935738.174\\
0.461511537788445	-218470526156.768\\
0.461611540288507	-218424689533.158\\
0.46171154278857	-218378279951.752\\
0.461811545288632	-218332443328.142\\
0.461911547788695	-218286033746.736\\
0.462011550288757	-218240197123.126\\
0.46211155278882	-218193787541.72\\
0.462211555288882	-218147950918.11\\
0.462311557788945	-218101541336.704\\
0.462411560289007	-218055704713.094\\
0.46251156278907	-218009295131.688\\
0.462611565289132	-217963458508.078\\
0.462711567789195	-217917621884.467\\
0.462811570289257	-217871212303.061\\
0.46291157278932	-217825375679.451\\
0.463011575289382	-217778966098.045\\
0.463111577789445	-217733129474.435\\
0.463211580289507	-217686719893.029\\
0.46331158278957	-217640883269.419\\
0.463411585289632	-217594473688.013\\
0.463511587789695	-217548637064.403\\
0.463611590289757	-217502227482.997\\
0.46371159278982	-217456390859.387\\
0.463811595289882	-217409981277.981\\
0.463911597789945	-217364144654.371\\
0.464011600290007	-217317735072.965\\
0.46411160279007	-217271898449.355\\
0.464211605290132	-217225488867.949\\
0.464311607790195	-217179652244.339\\
0.464411610290257	-217133242662.933\\
0.46451161279032	-217087406039.322\\
0.464611615290382	-217040996457.917\\
0.464711617790445	-216995159834.306\\
0.464811620290507	-216948750252.901\\
0.46491162279057	-216902913629.29\\
0.465011625290632	-216856504047.885\\
0.465111627790695	-216810667424.274\\
0.465211630290757	-216764257842.869\\
0.46531163279082	-216718421219.258\\
0.465411635290882	-216672011637.853\\
0.465511637790945	-216626175014.242\\
0.465611640291007	-216579765432.837\\
0.46571164279107	-216533928809.226\\
0.465811645291132	-216487519227.821\\
0.465911647791195	-216441682604.21\\
0.466011650291257	-216395273022.804\\
0.46611165279132	-216349436399.194\\
0.466211655291382	-216303026817.788\\
0.466311657791445	-216257190194.178\\
0.466411660291507	-216210780612.772\\
0.46651166279157	-216164943989.162\\
0.466611665291632	-216118534407.756\\
0.466711667791695	-216072697784.146\\
0.466811670291757	-216026288202.74\\
0.46691167279182	-215980451579.13\\
0.467011675291882	-215934041997.724\\
0.467111677791945	-215888205374.114\\
0.467211680292007	-215841795792.708\\
0.46731168279207	-215795959169.098\\
0.467411685292132	-215749549587.692\\
0.467511687792195	-215703712964.082\\
0.467611690292257	-215657303382.676\\
0.46771169279232	-215611466759.065\\
0.467811695292382	-215565057177.66\\
0.467911697792445	-215519220554.049\\
0.468011700292507	-215472810972.644\\
0.46811170279257	-215426974349.033\\
0.468211705292632	-215380564767.628\\
0.468311707792695	-215334728144.017\\
0.468411710292757	-215288318562.612\\
0.46851171279282	-215242481939.001\\
0.468611715292882	-215196072357.596\\
0.468711717792945	-215150235733.985\\
0.468811720293007	-215103826152.58\\
0.46891172279307	-215057989528.969\\
0.469011725293132	-215012152905.359\\
0.469111727793195	-214965743323.953\\
0.469211730293257	-214919906700.343\\
0.46931173279332	-214873497118.937\\
0.469411735293382	-214827660495.327\\
0.469511737793445	-214781250913.921\\
0.469611740293507	-214735414290.31\\
0.46971174279357	-214689004708.905\\
0.469811745293632	-214643168085.294\\
0.469911747793695	-214596758503.889\\
0.470011750293757	-214550921880.278\\
0.47011175279382	-214504512298.873\\
0.470211755293882	-214458675675.262\\
0.470311757793945	-214412266093.857\\
0.470411760294007	-214366429470.246\\
0.47051176279407	-214320019888.841\\
0.470611765294132	-214274183265.23\\
0.470711767794195	-214227773683.825\\
0.470811770294257	-214181937060.214\\
0.47091177279432	-214135527478.808\\
0.471011775294382	-214089690855.198\\
0.471111777794445	-214043281273.792\\
0.471211780294507	-213997444650.182\\
0.47131178279457	-213951608026.572\\
0.471411785294632	-213905198445.166\\
0.471511787794695	-213859361821.555\\
0.471611790294757	-213812952240.15\\
0.47171179279482	-213767115616.539\\
0.471811795294882	-213720706035.134\\
0.471911797794945	-213674869411.523\\
0.472011800295007	-213628459830.118\\
0.47211180279507	-213582623206.507\\
0.472211805295132	-213536213625.102\\
0.472311807795195	-213490377001.491\\
0.472411810295257	-213443967420.086\\
0.47251181279532	-213398130796.475\\
0.472611815295382	-213351721215.07\\
0.472711817795445	-213305884591.459\\
0.472811820295507	-213260047967.849\\
0.47291182279557	-213213638386.443\\
0.473011825295632	-213167801762.833\\
0.473111827795695	-213121392181.427\\
0.473211830295757	-213075555557.816\\
0.47331183279582	-213029145976.411\\
0.473411835295882	-212983309352.8\\
0.473511837795945	-212936899771.395\\
0.473611840296007	-212891063147.784\\
0.47371184279607	-212844653566.379\\
0.473811845296132	-212798816942.768\\
0.473911847796195	-212752980319.158\\
0.474011850296257	-212706570737.752\\
0.47411185279632	-212660734114.142\\
0.474211855296382	-212614324532.736\\
0.474311857796445	-212568487909.126\\
0.474411860296507	-212522078327.72\\
0.47451186279657	-212476241704.11\\
0.474611865296632	-212429832122.704\\
0.474711867796695	-212383995499.094\\
0.474811870296757	-212337585917.688\\
0.47491187279682	-212291749294.077\\
0.475011875296882	-212245912670.467\\
0.475111877796945	-212199503089.061\\
0.475211880297007	-212153666465.451\\
0.47531188279707	-212107256884.045\\
0.475411885297132	-212061420260.435\\
0.475511887797195	-212015010679.029\\
0.475611890297257	-211969174055.419\\
0.47571189279732	-211922764474.013\\
0.475811895297382	-211876927850.403\\
0.475911897797445	-211831091226.792\\
0.476011900297507	-211784681645.387\\
0.47611190279757	-211738845021.776\\
0.476211905297632	-211692435440.371\\
0.476311907797695	-211646598816.76\\
0.476411910297757	-211600189235.355\\
0.47651191279782	-211554352611.744\\
0.476611915297882	-211508515988.134\\
0.476711917797945	-211462106406.728\\
0.476811920298007	-211416269783.118\\
0.47691192279807	-211369860201.712\\
0.477011925298132	-211324023578.102\\
0.477111927798195	-211277613996.696\\
0.477211930298257	-211231777373.085\\
0.47731193279832	-211185940749.475\\
0.477411935298382	-211139531168.069\\
0.477511937798445	-211093694544.459\\
0.477611940298507	-211047284963.053\\
0.47771194279857	-211001448339.443\\
0.477811945298632	-210955038758.037\\
0.477911947798695	-210909202134.427\\
0.478011950298757	-210863365510.816\\
0.47811195279882	-210816955929.411\\
0.478211955298882	-210771119305.8\\
0.478311957798945	-210724709724.395\\
0.478411960299007	-210678873100.784\\
0.47851196279907	-210633036477.174\\
0.478611965299132	-210586626895.768\\
0.478711967799195	-210540790272.158\\
0.478811970299257	-210494380690.752\\
0.47891197279932	-210448544067.142\\
0.479011975299382	-210402707443.531\\
0.479111977799445	-210356297862.126\\
0.479211980299508	-210310461238.515\\
0.47931198279957	-210264051657.109\\
0.479411985299632	-210218215033.499\\
0.479511987799695	-210172378409.889\\
0.479611990299757	-210125968828.483\\
0.47971199279982	-210080132204.873\\
0.479811995299882	-210033722623.467\\
0.479911997799945	-209987885999.856\\
0.480012000300008	-209942049376.246\\
0.48011200280007	-209895639794.84\\
0.480212005300133	-209849803171.23\\
0.480312007800195	-209803393589.824\\
0.480412010300257	-209757556966.214\\
0.48051201280032	-209711720342.603\\
0.480612015300383	-209665310761.198\\
0.480712017800445	-209619474137.587\\
0.480812020300508	-209573064556.182\\
0.48091202280057	-209527227932.571\\
0.481012025300633	-209481391308.961\\
0.481112027800695	-209434981727.555\\
0.481212030300758	-209389145103.945\\
0.48131203280082	-209343308480.334\\
0.481412035300883	-209296898898.929\\
0.481512037800945	-209251062275.318\\
0.481612040301008	-209204652693.913\\
0.48171204280107	-209158816070.302\\
0.481812045301133	-209112979446.692\\
0.481912047801195	-209066569865.286\\
0.482012050301258	-209020733241.676\\
0.48211205280132	-208974896618.065\\
0.482212055301383	-208928487036.66\\
0.482312057801445	-208882650413.049\\
0.482412060301508	-208836240831.643\\
0.48251206280157	-208790404208.033\\
0.482612065301633	-208744567584.423\\
0.482712067801695	-208698158003.017\\
0.482812070301758	-208652321379.406\\
0.48291207280182	-208606484755.796\\
0.483012075301883	-208560075174.39\\
0.483112077801945	-208514238550.78\\
0.483212080302008	-208468401927.169\\
0.48331208280207	-208421992345.764\\
0.483412085302133	-208376155722.153\\
0.483512087802195	-208330319098.543\\
0.483612090302258	-208283909517.137\\
0.48371209280232	-208238072893.527\\
0.483812095302383	-208191663312.121\\
0.483912097802445	-208145826688.511\\
0.484012100302508	-208099990064.9\\
0.48411210280257	-208053580483.495\\
0.484212105302633	-208007743859.884\\
0.484312107802695	-207961907236.274\\
0.484412110302758	-207915497654.868\\
0.48451211280282	-207869661031.258\\
0.484612115302883	-207823824407.647\\
0.484712117802945	-207777414826.242\\
0.484812120303008	-207731578202.631\\
0.48491212280307	-207685741579.021\\
0.485012125303133	-207639331997.615\\
0.485112127803195	-207593495374.005\\
0.485212130303258	-207547658750.394\\
0.48531213280332	-207501249168.989\\
0.485412135303383	-207455412545.378\\
0.485512137803445	-207409575921.768\\
0.485612140303508	-207363739298.157\\
0.48571214280357	-207317329716.752\\
0.485812145303633	-207271493093.141\\
0.485912147803695	-207225656469.531\\
0.486012150303758	-207179246888.125\\
0.48611215280382	-207133410264.515\\
0.486212155303883	-207087573640.904\\
0.486312157803945	-207041164059.499\\
0.486412160304008	-206995327435.888\\
0.48651216280407	-206949490812.278\\
0.486612165304133	-206903081230.872\\
0.486712167804195	-206857244607.262\\
0.486812170304258	-206811407983.651\\
0.48691217280432	-206765571360.041\\
0.487012175304383	-206719161778.635\\
0.487112177804445	-206673325155.025\\
0.487212180304508	-206627488531.414\\
0.48731218280457	-206581078950.009\\
0.487412185304633	-206535242326.398\\
0.487512187804695	-206489405702.788\\
0.487612190304758	-206442996121.382\\
0.48771219280482	-206397159497.772\\
0.487812195304883	-206351322874.161\\
0.487912197804945	-206305486250.551\\
0.488012200305008	-206259076669.145\\
0.48811220280507	-206213240045.535\\
0.488212205305133	-206167403421.924\\
0.488312207805195	-206121566798.314\\
0.488412210305258	-206075157216.908\\
0.48851221280532	-206029320593.298\\
0.488612215305383	-205983483969.687\\
0.488712217805445	-205937074388.282\\
0.488812220305508	-205891237764.671\\
0.48891222280557	-205845401141.061\\
0.489012225305633	-205799564517.45\\
0.489112227805695	-205753154936.045\\
0.489212230305758	-205707318312.434\\
0.48931223280582	-205661481688.824\\
0.489412235305883	-205615645065.213\\
0.489512237805945	-205569235483.808\\
0.489612240306008	-205523398860.197\\
0.48971224280607	-205477562236.587\\
0.489812245306133	-205431725612.976\\
0.489912247806195	-205385316031.571\\
0.490012250306258	-205339479407.96\\
0.49011225280632	-205293642784.35\\
0.490212255306383	-205247806160.739\\
0.490312257806445	-205201396579.334\\
0.490412260306508	-205155559955.723\\
0.49051226280657	-205109723332.113\\
0.490612265306633	-205063886708.502\\
0.490712267806695	-205017477127.097\\
0.490812270306758	-204971640503.486\\
0.49091227280682	-204925803879.876\\
0.491012275306883	-204879967256.265\\
0.491112277806945	-204834130632.655\\
0.491212280307008	-204787721051.249\\
0.49131228280707	-204741884427.639\\
0.491412285307133	-204696047804.028\\
0.491512287807195	-204650211180.418\\
0.491612290307258	-204603801599.012\\
0.49171229280732	-204557964975.402\\
0.491812295307383	-204512128351.791\\
0.491912297807445	-204466291728.181\\
0.492012300307508	-204420455104.57\\
0.49211230280757	-204374045523.165\\
0.492212305307633	-204328208899.554\\
0.492312307807695	-204282372275.944\\
0.492412310307758	-204236535652.333\\
0.49251231280782	-204190699028.723\\
0.492612315307883	-204144289447.317\\
0.492712317807945	-204098452823.707\\
0.492812320308008	-204052616200.096\\
0.49291232280807	-204006779576.486\\
0.493012325308133	-203960942952.875\\
0.493112327808195	-203914533371.47\\
0.493212330308258	-203868696747.859\\
0.49331233280832	-203822860124.249\\
0.493412335308383	-203777023500.638\\
0.493512337808445	-203731186877.028\\
0.493612340308508	-203685350253.417\\
0.49371234280857	-203638940672.012\\
0.493812345308633	-203593104048.401\\
0.493912347808695	-203547267424.791\\
0.494012350308758	-203501430801.18\\
0.49411235280882	-203455594177.57\\
0.494212355308883	-203409757553.959\\
0.494312357808945	-203363347972.554\\
0.494412360309008	-203317511348.943\\
0.49451236280907	-203271674725.333\\
0.494612365309133	-203225838101.722\\
0.494712367809195	-203180001478.112\\
0.494812370309258	-203134164854.502\\
0.49491237280932	-203087755273.096\\
0.495012375309383	-203041918649.485\\
0.495112377809445	-202996082025.875\\
0.495212380309508	-202950245402.265\\
0.49531238280957	-202904408778.654\\
0.495412385309633	-202858572155.044\\
0.495512387809695	-202812735531.433\\
0.495612390309758	-202766898907.823\\
0.49571239280982	-202720489326.417\\
0.495812395309883	-202674652702.807\\
0.495912397809945	-202628816079.196\\
0.496012400310008	-202582979455.586\\
0.49611240281007	-202537142831.975\\
0.496212405310133	-202491306208.365\\
0.496312407810195	-202445469584.754\\
0.496412410310258	-202399632961.144\\
0.49651241281032	-202353223379.738\\
0.496612415310383	-202307386756.128\\
0.496712417810445	-202261550132.517\\
0.496812420310508	-202215713508.907\\
0.49691242281057	-202169876885.296\\
0.497012425310633	-202124040261.686\\
0.497112427810695	-202078203638.075\\
0.497212430310758	-202032367014.465\\
0.49731243281082	-201986530390.854\\
0.497412435310883	-201940693767.244\\
0.497512437810945	-201894284185.838\\
0.497612440311008	-201848447562.228\\
0.49771244281107	-201802610938.618\\
0.497812445311133	-201756774315.007\\
0.497912447811195	-201710937691.397\\
0.498012450311258	-201665101067.786\\
0.49811245281132	-201619264444.176\\
0.498212455311383	-201573427820.565\\
0.498312457811445	-201527591196.955\\
0.498412460311508	-201481754573.344\\
0.49851246281157	-201435917949.734\\
0.498612465311633	-201390081326.123\\
0.498712467811695	-201344244702.513\\
0.498812470311758	-201297835121.107\\
0.49891247281182	-201251998497.497\\
0.499012475311883	-201206161873.886\\
0.499112477811945	-201160325250.276\\
0.499212480312008	-201114488626.665\\
0.49931248281207	-201068652003.055\\
0.499412485312133	-201022815379.444\\
0.499512487812195	-200976978755.834\\
0.499612490312258	-200931142132.224\\
0.49971249281232	-200885305508.613\\
0.499812495312383	-200839468885.003\\
0.499912497812445	-200793632261.392\\
0.500012500312508	-200747795637.782\\
0.50011250281257	-200701959014.171\\
0.500212505312633	-200656122390.561\\
0.500312507812695	-200610285766.95\\
0.500412510312758	-200564449143.34\\
0.50051251281282	-200518612519.729\\
0.500612515312883	-200472775896.119\\
0.500712517812945	-200426939272.508\\
0.500812520313008	-200381102648.898\\
0.50091252281307	-200335266025.287\\
0.501012525313133	-200289429401.677\\
0.501112527813195	-200243592778.067\\
0.501212530313258	-200197756154.456\\
0.50131253281332	-200151919530.846\\
0.501412535313383	-200106082907.235\\
0.501512537813445	-200060246283.625\\
0.501612540313508	-200014409660.014\\
0.50171254281357	-199968573036.404\\
0.501812545313633	-199922736412.793\\
0.501912547813695	-199876899789.183\\
0.502012550313758	-199831063165.572\\
0.50211255281382	-199785226541.962\\
0.502212555313883	-199739389918.351\\
0.502312557813945	-199693553294.741\\
0.502412560314008	-199647716671.13\\
0.50251256281407	-199601880047.52\\
0.502612565314133	-199556043423.91\\
0.502712567814195	-199510206800.299\\
0.502812570314258	-199464370176.689\\
0.50291257281432	-199418533553.078\\
0.503012575314383	-199372696929.468\\
0.503112577814445	-199326860305.857\\
0.503212580314508	-199281023682.247\\
0.50331258281457	-199235187058.636\\
0.503412585314633	-199189350435.026\\
0.503512587814695	-199143513811.415\\
0.503612590314758	-199098250145.6\\
0.50371259281482	-199052413521.99\\
0.503812595314883	-199006576898.379\\
0.503912597814945	-198960740274.769\\
0.504012600315008	-198914903651.158\\
0.50411260281507	-198869067027.548\\
0.504212605315133	-198823230403.937\\
0.504312607815195	-198777393780.327\\
0.504412610315258	-198731557156.716\\
0.50451261281532	-198685720533.106\\
0.504612615315383	-198639883909.495\\
0.504712617815445	-198594047285.885\\
0.504812620315508	-198548210662.274\\
0.50491262281557	-198502946996.459\\
0.505012625315633	-198457110372.849\\
0.505112627815695	-198411273749.238\\
0.505212630315758	-198365437125.628\\
0.50531263281582	-198319600502.017\\
0.505412635315883	-198273763878.407\\
0.505512637815945	-198227927254.796\\
0.505612640316008	-198182090631.186\\
0.50571264281607	-198136254007.575\\
0.505812645316133	-198090990341.76\\
0.505912647816195	-198045153718.15\\
0.506012650316258	-197999317094.539\\
0.50611265281632	-197953480470.929\\
0.506212655316383	-197907643847.318\\
0.506312657816445	-197861807223.708\\
0.506412660316508	-197815970600.097\\
0.50651266281657	-197770706934.282\\
0.506612665316633	-197724870310.671\\
0.506712667816695	-197679033687.061\\
0.506812670316758	-197633197063.451\\
0.50691267281682	-197587360439.84\\
0.507012675316883	-197541523816.23\\
0.507112677816945	-197495687192.619\\
0.507212680317008	-197450423526.804\\
0.50731268281707	-197404586903.193\\
0.507412685317133	-197358750279.583\\
0.507512687817195	-197312913655.972\\
0.507612690317258	-197267077032.362\\
0.50771269281732	-197221240408.751\\
0.507812695317383	-197175976742.936\\
0.507912697817445	-197130140119.326\\
0.508012700317508	-197084303495.715\\
0.50811270281757	-197038466872.105\\
0.508212705317633	-196992630248.494\\
0.508312707817695	-196947366582.679\\
0.508412710317758	-196901529959.068\\
0.50851271281782	-196855693335.458\\
0.508612715317883	-196809856711.848\\
0.508712717817945	-196764020088.237\\
0.508812720318008	-196718756422.422\\
0.50891272281807	-196672919798.811\\
0.509012725318133	-196627083175.201\\
0.509112727818195	-196581246551.59\\
0.509212730318258	-196535409927.98\\
0.50931273281832	-196490146262.165\\
0.509412735318383	-196444309638.554\\
0.509512737818445	-196398473014.944\\
0.509612740318508	-196352636391.333\\
0.50971274281857	-196307372725.518\\
0.509812745318633	-196261536101.907\\
0.509912747818695	-196215699478.297\\
0.510012750318758	-196169862854.686\\
0.510112752818821	-196124599188.871\\
0.510212755318883	-196078762565.261\\
0.510312757818945	-196032925941.65\\
0.510412760319008	-195987089318.04\\
0.51051276281907	-195941825652.224\\
0.510612765319133	-195895989028.614\\
0.510712767819196	-195850152405.003\\
0.510812770319258	-195804315781.393\\
0.51091277281932	-195759052115.578\\
0.511012775319383	-195713215491.967\\
0.511112777819446	-195667378868.357\\
0.511212780319508	-195622115202.541\\
0.51131278281957	-195576278578.931\\
0.511412785319633	-195530441955.32\\
0.511512787819695	-195484605331.71\\
0.511612790319758	-195439341665.895\\
0.511712792819821	-195393505042.284\\
0.511812795319883	-195347668418.674\\
0.511912797819945	-195302404752.858\\
0.512012800320008	-195256568129.248\\
0.512112802820071	-195210731505.637\\
0.512212805320133	-195165467839.822\\
0.512312807820196	-195119631216.212\\
0.512412810320258	-195073794592.601\\
0.51251281282032	-195027957968.991\\
0.512612815320383	-194982694303.175\\
0.512712817820446	-194936857679.565\\
0.512812820320508	-194891021055.954\\
0.51291282282057	-194845757390.139\\
0.513012825320633	-194799920766.529\\
0.513112827820696	-194754084142.918\\
0.513212830320758	-194708820477.103\\
0.513312832820821	-194662983853.492\\
0.513412835320883	-194617720187.677\\
0.513512837820945	-194571883564.067\\
0.513612840321008	-194526046940.456\\
0.513712842821071	-194480783274.641\\
0.513812845321133	-194434946651.03\\
0.513912847821196	-194389110027.42\\
0.514012850321258	-194343846361.604\\
0.514112852821321	-194298009737.994\\
0.514212855321383	-194252173114.384\\
0.514312857821446	-194206909448.568\\
0.514412860321508	-194161072824.958\\
0.51451286282157	-194115809159.142\\
0.514612865321633	-194069972535.532\\
0.514712867821696	-194024135911.921\\
0.514812870321758	-193978872246.106\\
0.514912872821821	-193933035622.496\\
0.515012875321883	-193887771956.68\\
0.515112877821946	-193841935333.07\\
0.515212880322008	-193796098709.459\\
0.515312882822071	-193750835043.644\\
0.515412885322133	-193704998420.034\\
0.515512887822196	-193659734754.218\\
0.515612890322258	-193613898130.608\\
0.515712892822321	-193568634464.792\\
0.515812895322383	-193522797841.182\\
0.515912897822446	-193476961217.572\\
0.516012900322508	-193431697551.756\\
0.516112902822571	-193385860928.146\\
0.516212905322633	-193340597262.33\\
0.516312907822696	-193294760638.72\\
0.516412910322758	-193249496972.905\\
0.516512912822821	-193203660349.294\\
0.516612915322883	-193158396683.479\\
0.516712917822946	-193112560059.868\\
0.516812920323008	-193067296394.053\\
0.516912922823071	-193021459770.443\\
0.517012925323133	-192975623146.832\\
0.517112927823196	-192930359481.017\\
0.517212930323258	-192884522857.406\\
0.517312932823321	-192839259191.591\\
0.517412935323383	-192793422567.98\\
0.517512937823446	-192748158902.165\\
0.517612940323508	-192702322278.555\\
0.517712942823571	-192657058612.739\\
0.517812945323633	-192611221989.129\\
0.517912947823696	-192565958323.314\\
0.518012950323758	-192520121699.703\\
0.518112952823821	-192474858033.888\\
0.518212955323883	-192429021410.277\\
0.518312957823946	-192383757744.462\\
0.518412960324008	-192337921120.851\\
0.518512962824071	-192292657455.036\\
0.518612965324133	-192247393789.221\\
0.518712967824196	-192201557165.61\\
0.518812970324258	-192156293499.795\\
0.518912972824321	-192110456876.185\\
0.519012975324383	-192065193210.369\\
0.519112977824446	-192019356586.759\\
0.519212980324508	-191974092920.943\\
0.519312982824571	-191928256297.333\\
0.519412985324633	-191882992631.518\\
0.519512987824696	-191837156007.907\\
0.519612990324758	-191791892342.092\\
0.519712992824821	-191746628676.276\\
0.519812995324883	-191700792052.666\\
0.519912997824946	-191655528386.851\\
0.520013000325008	-191609691763.24\\
0.520113002825071	-191564428097.425\\
0.520213005325133	-191519164431.609\\
0.520313007825196	-191473327807.999\\
0.520413010325258	-191428064142.184\\
0.520513012825321	-191382227518.573\\
0.520613015325383	-191336963852.758\\
0.520713017825446	-191291700186.943\\
0.520813020325508	-191245863563.332\\
0.520913022825571	-191200599897.517\\
0.521013025325633	-191154763273.906\\
0.521113027825696	-191109499608.091\\
0.521213030325758	-191064235942.276\\
0.521313032825821	-191018399318.665\\
0.521413035325883	-190973135652.85\\
0.521513037825946	-190927299029.239\\
0.521613040326008	-190882035363.424\\
0.521713042826071	-190836771697.609\\
0.521813045326133	-190790935073.998\\
0.521913047826196	-190745671408.183\\
0.522013050326258	-190700407742.368\\
0.522113052826321	-190654571118.757\\
0.522213055326383	-190609307452.942\\
0.522313057826446	-190564043787.126\\
0.522413060326508	-190518207163.516\\
0.522513062826571	-190472943497.701\\
0.522613065326633	-190427679831.885\\
0.522713067826696	-190381843208.275\\
0.522813070326758	-190336579542.459\\
0.522913072826821	-190291315876.644\\
0.523013075326883	-190245479253.034\\
0.523113077826946	-190200215587.218\\
0.523213080327008	-190154951921.403\\
0.523313082827071	-190109688255.588\\
0.523413085327133	-190063851631.977\\
0.523513087827196	-190018587966.162\\
0.523613090327258	-189973324300.347\\
0.523713092827321	-189927487676.736\\
0.523813095327383	-189882224010.921\\
0.523913097827446	-189836960345.105\\
0.524013100327508	-189791696679.29\\
0.524113102827571	-189745860055.68\\
0.524213105327633	-189700596389.864\\
0.524313107827696	-189655332724.049\\
0.524413110327758	-189610069058.234\\
0.524513112827821	-189564232434.623\\
0.524613115327883	-189518968768.808\\
0.524713117827946	-189473705102.992\\
0.524813120328008	-189428441437.177\\
0.524913122828071	-189382604813.567\\
0.525013125328133	-189337341147.751\\
0.525113127828196	-189292077481.936\\
0.525213130328258	-189246813816.121\\
0.525313132828321	-189200977192.51\\
0.525413135328383	-189155713526.695\\
0.525513137828446	-189110449860.88\\
0.525613140328508	-189065186195.064\\
0.525713142828571	-189019922529.249\\
0.525813145328633	-188974085905.638\\
0.525913147828696	-188928822239.823\\
0.526013150328758	-188883558574.008\\
0.526113152828821	-188838294908.192\\
0.526213155328883	-188793031242.377\\
0.526313157828946	-188747767576.562\\
0.526413160329008	-188701930952.951\\
0.526513162829071	-188656667287.136\\
0.526613165329133	-188611403621.321\\
0.526713167829196	-188566139955.505\\
0.526813170329258	-188520876289.69\\
0.526913172829321	-188475612623.875\\
0.527013175329383	-188429776000.264\\
0.527113177829446	-188384512334.449\\
0.527213180329508	-188339248668.633\\
0.527313182829571	-188293985002.818\\
0.527413185329633	-188248721337.003\\
0.527513187829696	-188203457671.187\\
0.527613190329758	-188158194005.372\\
0.527713192829821	-188112357381.762\\
0.527813195329883	-188067093715.946\\
0.527913197829946	-188021830050.131\\
0.528013200330008	-187976566384.316\\
0.528113202830071	-187931302718.5\\
0.528213205330133	-187886039052.685\\
0.528313207830196	-187840775386.87\\
0.528413210330258	-187795511721.054\\
0.528513212830321	-187750248055.239\\
0.528613215330383	-187704984389.424\\
0.528713217830446	-187659147765.813\\
0.528813220330508	-187613884099.998\\
0.528913222830571	-187568620434.182\\
0.529013225330633	-187523356768.367\\
0.529113227830696	-187478093102.552\\
0.529213230330758	-187432829436.736\\
0.529313232830821	-187387565770.921\\
0.529413235330883	-187342302105.106\\
0.529513237830946	-187297038439.29\\
0.529613240331008	-187251774773.475\\
0.529713242831071	-187206511107.66\\
0.529813245331133	-187161247441.844\\
0.529913247831196	-187115983776.029\\
0.530013250331258	-187070720110.214\\
0.530113252831321	-187025456444.398\\
0.530213255331383	-186980192778.583\\
0.530313257831446	-186934929112.768\\
0.530413260331508	-186889665446.952\\
0.530513262831571	-186844401781.137\\
0.530613265331633	-186799138115.322\\
0.530713267831696	-186753874449.506\\
0.530813270331758	-186708610783.691\\
0.530913272831821	-186663347117.876\\
0.531013275331883	-186618083452.06\\
0.531113277831946	-186572819786.245\\
0.531213280332008	-186527556120.43\\
0.531313282832071	-186482292454.614\\
0.531413285332133	-186437028788.799\\
0.531513287832196	-186391765122.984\\
0.531613290332258	-186346501457.168\\
0.531713292832321	-186301237791.353\\
0.531813295332383	-186255974125.538\\
0.531913297832446	-186210710459.722\\
0.532013300332508	-186165446793.907\\
0.532113302832571	-186120183128.092\\
0.532213305332633	-186074919462.276\\
0.532313307832696	-186029655796.461\\
0.532413310332758	-185984392130.646\\
0.532513312832821	-185939701422.626\\
0.532613315332883	-185894437756.81\\
0.532713317832946	-185849174090.995\\
0.532813320333008	-185803910425.18\\
0.532913322833071	-185758646759.364\\
0.533013325333133	-185713383093.549\\
0.533113327833196	-185668119427.734\\
0.533213330333258	-185622855761.918\\
0.533313332833321	-185577592096.103\\
0.533413335333383	-185532328430.288\\
0.533513337833446	-185487637722.267\\
0.533613340333508	-185442374056.452\\
0.533713342833571	-185397110390.637\\
0.533813345333633	-185351846724.821\\
0.533913347833696	-185306583059.006\\
0.534013350333758	-185261319393.191\\
0.534113352833821	-185216055727.375\\
0.534213355333883	-185171365019.355\\
0.534313357833946	-185126101353.54\\
0.534413360334008	-185080837687.724\\
0.534513362834071	-185035574021.909\\
0.534613365334133	-184990310356.094\\
0.534713367834196	-184945046690.278\\
0.534813370334258	-184900355982.258\\
0.534913372834321	-184855092316.443\\
0.535013375334383	-184809828650.628\\
0.535113377834446	-184764564984.812\\
0.535213380334508	-184719301318.997\\
0.535313382834571	-184674610610.977\\
0.535413385334633	-184629346945.161\\
0.535513387834696	-184584083279.346\\
0.535613390334758	-184538819613.531\\
0.535713392834821	-184493555947.715\\
0.535813395334883	-184448865239.695\\
0.535913397834946	-184403601573.88\\
0.536013400335008	-184358337908.064\\
0.536113402835071	-184313074242.249\\
0.536213405335133	-184268383534.229\\
0.536313407835196	-184223119868.414\\
0.536413410335258	-184177856202.598\\
0.536513412835321	-184132592536.783\\
0.536613415335383	-184087901828.763\\
0.536713417835446	-184042638162.947\\
0.536813420335508	-183997374497.132\\
0.536913422835571	-183952110831.317\\
0.537013425335633	-183907420123.297\\
0.537113427835696	-183862156457.481\\
0.537213430335758	-183816892791.666\\
0.537313432835821	-183772202083.646\\
0.537413435335883	-183726938417.83\\
0.537513437835946	-183681674752.015\\
0.537613440336008	-183636411086.2\\
0.537713442836071	-183591720378.179\\
0.537813445336133	-183546456712.364\\
0.537913447836196	-183501193046.549\\
0.538013450336258	-183456502338.529\\
0.538113452836321	-183411238672.713\\
0.538213455336383	-183365975006.898\\
0.538313457836446	-183321284298.878\\
0.538413460336508	-183276020633.062\\
0.538513462836571	-183230756967.247\\
0.538613465336633	-183186066259.227\\
0.538713467836696	-183140802593.411\\
0.538813470336758	-183096111885.391\\
0.538913472836821	-183050848219.576\\
0.539013475336883	-183005584553.761\\
0.539113477836946	-182960893845.74\\
0.539213480337008	-182915630179.925\\
0.539313482837071	-182870366514.11\\
0.539413485337133	-182825675806.09\\
0.539513487837196	-182780412140.274\\
0.539613490337258	-182735721432.254\\
0.539713492837321	-182690457766.439\\
0.539813495337383	-182645194100.623\\
0.539913497837446	-182600503392.603\\
0.540013500337508	-182555239726.788\\
0.540113502837571	-182510549018.768\\
0.540213505337633	-182465285352.952\\
0.540313507837696	-182420594644.932\\
0.540413510337758	-182375330979.117\\
0.540513512837821	-182330067313.301\\
0.540613515337883	-182285376605.281\\
0.540713517837946	-182240112939.466\\
0.540813520338008	-182195422231.446\\
0.540913522838071	-182150158565.63\\
0.541013525338133	-182105467857.61\\
0.541113527838196	-182060204191.795\\
0.541213530338258	-182015513483.775\\
0.541313532838321	-181970249817.959\\
0.541413535338383	-181925559109.939\\
0.541513537838446	-181880295444.124\\
0.541613540338509	-181835604736.103\\
0.541713542838571	-181790341070.288\\
0.541813545338633	-181745650362.268\\
0.541913547838696	-181700386696.453\\
0.542013550338758	-181655695988.432\\
0.542113552838821	-181610432322.617\\
0.542213555338883	-181565741614.597\\
0.542313557838946	-181520477948.781\\
0.542413560339008	-181475787240.761\\
0.542513562839071	-181430523574.946\\
0.542613565339134	-181385832866.926\\
0.542713567839196	-181340569201.11\\
0.542813570339258	-181295878493.09\\
0.542913572839321	-181251187785.07\\
0.543013575339383	-181205924119.255\\
0.543113577839446	-181161233411.234\\
0.543213580339509	-181115969745.419\\
0.543313582839571	-181071279037.399\\
0.543413585339633	-181026015371.584\\
0.543513587839696	-180981324663.563\\
0.543613590339759	-180936633955.543\\
0.543713592839821	-180891370289.728\\
0.543813595339883	-180846679581.708\\
0.543913597839946	-180801415915.892\\
0.544013600340008	-180756725207.872\\
0.544113602840071	-180712034499.852\\
0.544213605340134	-180666770834.037\\
0.544313607840196	-180622080126.016\\
0.544413610340258	-180577389417.996\\
0.544513612840321	-180532125752.181\\
0.544613615340384	-180487435044.161\\
0.544713617840446	-180442744336.14\\
0.544813620340509	-180397480670.325\\
0.544913622840571	-180352789962.305\\
0.545013625340633	-180308099254.285\\
0.545113627840696	-180262835588.469\\
0.545213630340759	-180218144880.449\\
0.545313632840821	-180173454172.429\\
0.545413635340883	-180128190506.614\\
0.545513637840946	-180083499798.593\\
0.545613640341009	-180038809090.573\\
0.545713642841071	-179993545424.758\\
0.545813645341134	-179948854716.738\\
0.545913647841196	-179904164008.717\\
0.546013650341258	-179858900342.902\\
0.546113652841321	-179814209634.882\\
0.546213655341384	-179769518926.862\\
0.546313657841446	-179724828218.841\\
0.546413660341509	-179679564553.026\\
0.546513662841571	-179634873845.006\\
0.546613665341634	-179590183136.986\\
0.546713667841696	-179545492428.966\\
0.546813670341759	-179500228763.15\\
0.546913672841821	-179455538055.13\\
0.547013675341883	-179410847347.11\\
0.547113677841946	-179366156639.09\\
0.547213680342009	-179320892973.274\\
0.547313682842071	-179276202265.254\\
0.547413685342134	-179231511557.234\\
0.547513687842196	-179186820849.214\\
0.547613690342259	-179142130141.193\\
0.547713692842321	-179096866475.378\\
0.547813695342384	-179052175767.358\\
0.547913697842446	-179007485059.338\\
0.548013700342509	-178962794351.318\\
0.548113702842571	-178918103643.297\\
0.548213705342634	-178872839977.482\\
0.548313707842696	-178828149269.462\\
0.548413710342759	-178783458561.442\\
0.548513712842821	-178738767853.421\\
0.548613715342884	-178694077145.401\\
0.548713717842946	-178649386437.381\\
0.548813720343009	-178604122771.566\\
0.548913722843071	-178559432063.545\\
0.549013725343134	-178514741355.525\\
0.549113727843196	-178470050647.505\\
0.549213730343259	-178425359939.485\\
0.549313732843321	-178380669231.465\\
0.549413735343384	-178335978523.444\\
0.549513737843446	-178291287815.424\\
0.549613740343509	-178246597107.404\\
0.549713742843571	-178201333441.589\\
0.549813745343634	-178156642733.568\\
0.549913747843696	-178111952025.548\\
0.550013750343759	-178067261317.528\\
0.550113752843821	-178022570609.508\\
0.550213755343884	-177977879901.488\\
0.550313757843946	-177933189193.467\\
0.550413760344009	-177888498485.447\\
0.550513762844071	-177843807777.427\\
0.550613765344134	-177799117069.407\\
0.550713767844196	-177754426361.387\\
0.550813770344259	-177709735653.366\\
0.550913772844321	-177665044945.346\\
0.551013775344384	-177620354237.326\\
0.551113777844446	-177575663529.306\\
0.551213780344509	-177530972821.286\\
0.551313782844571	-177486282113.265\\
0.551413785344634	-177441591405.245\\
0.551513787844696	-177396900697.225\\
0.551613790344759	-177352209989.205\\
0.551713792844821	-177307519281.185\\
0.551813795344884	-177262828573.164\\
0.551913797844946	-177218137865.144\\
0.552013800345009	-177173447157.124\\
0.552113802845071	-177128756449.104\\
0.552213805345134	-177084065741.084\\
0.552313807845196	-177039375033.063\\
0.552413810345259	-176994684325.043\\
0.552513812845321	-176949993617.023\\
0.552613815345384	-176905302909.003\\
0.552713817845446	-176860612200.983\\
0.552813820345509	-176815921492.962\\
0.552913822845571	-176771230784.942\\
0.553013825345634	-176726540076.922\\
0.553113827845696	-176681849368.902\\
0.553213830345759	-176637158660.882\\
0.553313832845821	-176593040910.656\\
0.553413835345884	-176548350202.636\\
0.553513837845946	-176503659494.616\\
0.553613840346009	-176458968786.596\\
0.553713842846071	-176414278078.576\\
0.553813845346134	-176369587370.555\\
0.553913847846196	-176324896662.535\\
0.554013850346259	-176280205954.515\\
0.554113852846321	-176236088204.29\\
0.554213855346384	-176191397496.27\\
0.554313857846446	-176146706788.25\\
0.554413860346509	-176102016080.229\\
0.554513862846571	-176057325372.209\\
0.554613865346634	-176012634664.189\\
0.554713867846696	-175967943956.169\\
0.554813870346759	-175923826205.944\\
0.554913872846821	-175879135497.923\\
0.555013875346884	-175834444789.903\\
0.555113877846946	-175789754081.883\\
0.555213880347009	-175745063373.863\\
0.555313882847071	-175700945623.638\\
0.555413885347134	-175656254915.618\\
0.555513887847196	-175611564207.597\\
0.555613890347259	-175566873499.577\\
0.555713892847321	-175522755749.352\\
0.555813895347384	-175478065041.332\\
0.555913897847446	-175433374333.312\\
0.556013900347509	-175388683625.291\\
0.556113902847571	-175343992917.271\\
0.556213905347634	-175299875167.046\\
0.556313907847696	-175255184459.026\\
0.556413910347759	-175210493751.006\\
0.556513912847821	-175166376000.781\\
0.556613915347884	-175121685292.76\\
0.556713917847946	-175076994584.74\\
0.556813920348009	-175032303876.72\\
0.556913922848071	-174988186126.495\\
0.557013925348134	-174943495418.475\\
0.557113927848196	-174898804710.455\\
0.557213930348259	-174854686960.229\\
0.557313932848321	-174809996252.209\\
0.557413935348384	-174765305544.189\\
0.557513937848446	-174721187793.964\\
0.557613940348509	-174676497085.944\\
0.557713942848571	-174631806377.924\\
0.557813945348634	-174587688627.699\\
0.557913947848696	-174542997919.678\\
0.558013950348759	-174498307211.658\\
0.558113952848821	-174454189461.433\\
0.558213955348884	-174409498753.413\\
0.558313957848946	-174364808045.393\\
0.558413960349009	-174320690295.168\\
0.558513962849071	-174275999587.147\\
0.558613965349134	-174231881836.922\\
0.558713967849196	-174187191128.902\\
0.558813970349259	-174142500420.882\\
0.558913972849321	-174098382670.657\\
0.559013975349384	-174053691962.637\\
0.559113977849446	-174009574212.412\\
0.559213980349509	-173964883504.391\\
0.559313982849571	-173920192796.371\\
0.559413985349634	-173876075046.146\\
0.559513987849696	-173831384338.126\\
0.559613990349759	-173787266587.901\\
0.559713992849821	-173742575879.881\\
0.559813995349884	-173698458129.655\\
0.559913997849946	-173653767421.635\\
0.560014000350009	-173609649671.41\\
0.560114002850071	-173564958963.39\\
0.560214005350134	-173520841213.165\\
0.560314007850196	-173476150505.145\\
0.560414010350259	-173432032754.92\\
0.560514012850321	-173387342046.899\\
0.560614015350384	-173343224296.674\\
0.560714017850446	-173298533588.654\\
0.560814020350509	-173254415838.429\\
0.560914022850571	-173209725130.409\\
0.561014025350634	-173165607380.184\\
0.561114027850696	-173120916672.164\\
0.561214030350759	-173076798921.939\\
0.561314032850821	-173032108213.918\\
0.561414035350884	-172987990463.693\\
0.561514037850946	-172943299755.673\\
0.561614040351009	-172899182005.448\\
0.561714042851071	-172854491297.428\\
0.561814045351134	-172810373547.203\\
0.561914047851196	-172766255796.978\\
0.562014050351259	-172721565088.957\\
0.562114052851321	-172677447338.732\\
0.562214055351384	-172632756630.712\\
0.562314057851446	-172588638880.487\\
0.562414060351509	-172544521130.262\\
0.562514062851571	-172499830422.242\\
0.562614065351634	-172455712672.017\\
0.562714067851696	-172411021963.997\\
0.562814070351759	-172366904213.771\\
0.562914072851821	-172322786463.546\\
0.563014075351884	-172278095755.526\\
0.563114077851946	-172233978005.301\\
0.563214080352009	-172189860255.076\\
0.563314082852071	-172145169547.056\\
0.563414085352134	-172101051796.831\\
0.563514087852196	-172056934046.606\\
0.563614090352259	-172012243338.585\\
0.563714092852321	-171968125588.36\\
0.563814095352384	-171924007838.135\\
0.563914097852446	-171879317130.115\\
0.564014100352509	-171835199379.89\\
0.564114102852571	-171791081629.665\\
0.564214105352634	-171746963879.44\\
0.564314107852696	-171702273171.42\\
0.564414110352759	-171658155421.195\\
0.564514112852821	-171614037670.97\\
0.564614115352884	-171569346962.949\\
0.564714117852946	-171525229212.724\\
0.564814120353009	-171481111462.499\\
0.564914122853071	-171436993712.274\\
0.565014125353134	-171392875962.049\\
0.565114127853196	-171348185254.029\\
0.565214130353259	-171304067503.804\\
0.565314132853321	-171259949753.579\\
0.565414135353384	-171215832003.354\\
0.565514137853446	-171171141295.333\\
0.565614140353509	-171127023545.108\\
0.565714142853571	-171082905794.883\\
0.565814145353634	-171038788044.658\\
0.565914147853696	-170994670294.433\\
0.566014150353759	-170949979586.413\\
0.566114152853821	-170905861836.188\\
0.566214155353884	-170861744085.963\\
0.566314157853946	-170817626335.738\\
0.566414160354009	-170773508585.513\\
0.566514162854071	-170729390835.288\\
0.566614165354134	-170685273085.062\\
0.566714167854196	-170640582377.042\\
0.566814170354259	-170596464626.817\\
0.566914172854321	-170552346876.592\\
0.567014175354384	-170508229126.367\\
0.567114177854446	-170464111376.142\\
0.567214180354509	-170419993625.917\\
0.567314182854571	-170375875875.692\\
0.567414185354634	-170331758125.467\\
0.567514187854696	-170287640375.242\\
0.567614190354759	-170242949667.221\\
0.567714192854821	-170198831916.996\\
0.567814195354884	-170154714166.771\\
0.567914197854946	-170110596416.546\\
0.568014200355009	-170066478666.321\\
0.568114202855071	-170022360916.096\\
0.568214205355134	-169978243165.871\\
0.568314207855196	-169934125415.646\\
0.568414210355259	-169890007665.421\\
0.568514212855321	-169845889915.196\\
0.568614215355384	-169801772164.971\\
0.568714217855446	-169757654414.746\\
0.568814220355509	-169713536664.521\\
0.568914222855571	-169669418914.296\\
0.569014225355634	-169625301164.07\\
0.569114227855696	-169581183413.845\\
0.569214230355759	-169537065663.62\\
0.569314232855821	-169492947913.395\\
0.569414235355884	-169448830163.17\\
0.569514237855946	-169404712412.945\\
0.569614240356009	-169360594662.72\\
0.569714242856071	-169316476912.495\\
0.569814245356134	-169272359162.27\\
0.569914247856196	-169228241412.045\\
0.570014250356259	-169184123661.82\\
0.570114252856321	-169140005911.595\\
0.570214255356384	-169096461119.165\\
0.570314257856446	-169052343368.94\\
0.570414260356509	-169008225618.715\\
0.570514262856571	-168964107868.49\\
0.570614265356634	-168919990118.264\\
0.570714267856696	-168875872368.039\\
0.570814270356759	-168831754617.814\\
0.570914272856821	-168787636867.589\\
0.571014275356884	-168743519117.364\\
0.571114277856946	-168699974324.934\\
0.571214280357009	-168655856574.709\\
0.571314282857071	-168611738824.484\\
0.571414285357134	-168567621074.259\\
0.571514287857196	-168523503324.034\\
0.571614290357259	-168479385573.809\\
0.571714292857321	-168435267823.584\\
0.571814295357384	-168391723031.154\\
0.571914297857446	-168347605280.929\\
0.572014300357509	-168303487530.704\\
0.572114302857571	-168259369780.479\\
0.572214305357634	-168215252030.254\\
0.572314307857696	-168171707237.824\\
0.572414310357759	-168127589487.599\\
0.572514312857821	-168083471737.373\\
0.572614315357884	-168039353987.148\\
0.572714317857946	-167995809194.718\\
0.572814320358009	-167951691444.493\\
0.572914322858071	-167907573694.268\\
0.573014325358134	-167863455944.043\\
0.573114327858196	-167819911151.613\\
0.573214330358259	-167775793401.388\\
0.573314332858321	-167731675651.163\\
0.573414335358384	-167687557900.938\\
0.573514337858446	-167644013108.508\\
0.573614340358509	-167599895358.283\\
0.573714342858571	-167555777608.058\\
0.573814345358634	-167512232815.628\\
0.573914347858696	-167468115065.403\\
0.574014350358759	-167423997315.178\\
0.574114352858822	-167380452522.748\\
0.574214355358884	-167336334772.523\\
0.574314357858946	-167292217022.298\\
0.574414360359009	-167248672229.868\\
0.574514362859071	-167204554479.643\\
0.574614365359134	-167160436729.418\\
0.574714367859196	-167116891936.988\\
0.574814370359259	-167072774186.763\\
0.574914372859321	-167028656436.538\\
0.575014375359384	-166985111644.108\\
0.575114377859447	-166940993893.883\\
0.575214380359509	-166897449101.453\\
0.575314382859571	-166853331351.228\\
0.575414385359634	-166809213601.002\\
0.575514387859696	-166765668808.573\\
0.575614390359759	-166721551058.347\\
0.575714392859822	-166678006265.918\\
0.575814395359884	-166633888515.692\\
0.575914397859946	-166589770765.467\\
0.576014400360009	-166546225973.037\\
0.576114402860072	-166502108222.812\\
0.576214405360134	-166458563430.382\\
0.576314407860196	-166414445680.157\\
0.576414410360259	-166370900887.727\\
0.576514412860321	-166326783137.502\\
0.576614415360384	-166283238345.072\\
0.576714417860447	-166239120594.847\\
0.576814420360509	-166195575802.417\\
0.576914422860571	-166151458052.192\\
0.577014425360634	-166107913259.762\\
0.577114427860697	-166063795509.537\\
0.577214430360759	-166020250717.107\\
0.577314432860822	-165976132966.882\\
0.577414435360884	-165932588174.452\\
0.577514437860946	-165888470424.227\\
0.577614440361009	-165844925631.797\\
0.577714442861072	-165801380839.367\\
0.577814445361134	-165757263089.142\\
0.577914447861197	-165713718296.712\\
0.578014450361259	-165669600546.487\\
0.578114452861322	-165626055754.057\\
0.578214455361384	-165581938003.832\\
0.578314457861447	-165538393211.402\\
0.578414460361509	-165494848418.972\\
0.578514462861571	-165450730668.747\\
0.578614465361634	-165407185876.317\\
0.578714467861697	-165363068126.092\\
0.578814470361759	-165319523333.662\\
0.578914472861822	-165275978541.232\\
0.579014475361884	-165231860791.007\\
0.579114477861947	-165188315998.577\\
0.579214480362009	-165144771206.147\\
0.579314482862072	-165100653455.922\\
0.579414485362134	-165057108663.492\\
0.579514487862197	-165013563871.062\\
0.579614490362259	-164969446120.837\\
0.579714492862322	-164925901328.407\\
0.579814495362384	-164882356535.978\\
0.579914497862447	-164838238785.752\\
0.580014500362509	-164794693993.323\\
0.580114502862572	-164751149200.893\\
0.580214505362634	-164707604408.463\\
0.580314507862697	-164663486658.238\\
0.580414510362759	-164619941865.808\\
0.580514512862822	-164576397073.378\\
0.580614515362884	-164532852280.948\\
0.580714517862947	-164488734530.723\\
0.580814520363009	-164445189738.293\\
0.580914522863072	-164401644945.863\\
0.581014525363134	-164358100153.433\\
0.581114527863197	-164313982403.208\\
0.581214530363259	-164270437610.778\\
0.581314532863322	-164226892818.348\\
0.581414535363384	-164183348025.918\\
0.581514537863447	-164139803233.488\\
0.581614540363509	-164095685483.263\\
0.581714542863572	-164052140690.833\\
0.581814545363634	-164008595898.403\\
0.581914547863697	-163965051105.973\\
0.582014550363759	-163921506313.543\\
0.582114552863822	-163877961521.113\\
0.582214555363884	-163833843770.888\\
0.582314557863947	-163790298978.458\\
0.582414560364009	-163746754186.028\\
0.582514562864072	-163703209393.598\\
0.582614565364134	-163659664601.168\\
0.582714567864197	-163616119808.738\\
0.582814570364259	-163572575016.308\\
0.582914572864322	-163529030223.879\\
0.583014575364384	-163485485431.449\\
0.583114577864447	-163441367681.224\\
0.583214580364509	-163397822888.794\\
0.583314582864572	-163354278096.364\\
0.583414585364634	-163310733303.934\\
0.583514587864697	-163267188511.504\\
0.583614590364759	-163223643719.074\\
0.583714592864822	-163180098926.644\\
0.583814595364884	-163136554134.214\\
0.583914597864947	-163093009341.784\\
0.584014600365009	-163049464549.354\\
0.584114602865072	-163005919756.924\\
0.584214605365134	-162962374964.494\\
0.584314607865197	-162918830172.064\\
0.584414610365259	-162875285379.634\\
0.584514612865322	-162831740587.204\\
0.584614615365384	-162788195794.774\\
0.584714617865447	-162744651002.344\\
0.584814620365509	-162701106209.914\\
0.584914622865572	-162657561417.485\\
0.585014625365634	-162614016625.055\\
0.585114627865697	-162570471832.625\\
0.585214630365759	-162526927040.195\\
0.585314632865822	-162483382247.765\\
0.585414635365884	-162439837455.335\\
0.585514637865947	-162396865620.7\\
0.585614640366009	-162353320828.27\\
0.585714642866072	-162309776035.84\\
0.585814645366134	-162266231243.41\\
0.585914647866197	-162222686450.98\\
0.586014650366259	-162179141658.55\\
0.586114652866322	-162135596866.12\\
0.586214655366384	-162092052073.69\\
0.586314657866447	-162048507281.26\\
0.586414660366509	-162005535446.626\\
0.586514662866572	-161961990654.196\\
0.586614665366634	-161918445861.766\\
0.586714667866697	-161874901069.336\\
0.586814670366759	-161831356276.906\\
0.586914672866822	-161787811484.476\\
0.587014675366884	-161744839649.841\\
0.587114677866947	-161701294857.411\\
0.587214680367009	-161657750064.981\\
0.587314682867072	-161614205272.551\\
0.587414685367134	-161570660480.121\\
0.587514687867197	-161527688645.487\\
0.587614690367259	-161484143853.057\\
0.587714692867322	-161440599060.627\\
0.587814695367384	-161397054268.197\\
0.587914697867447	-161354082433.562\\
0.588014700367509	-161310537641.132\\
0.588114702867572	-161266992848.702\\
0.588214705367634	-161223448056.272\\
0.588314707867697	-161180476221.637\\
0.588414710367759	-161136931429.207\\
0.588514712867822	-161093386636.777\\
0.588614715367884	-161050414802.143\\
0.588714717867947	-161006870009.713\\
0.588814720368009	-160963325217.283\\
0.588914722868072	-160919780424.853\\
0.589014725368134	-160876808590.218\\
0.589114727868197	-160833263797.788\\
0.589214730368259	-160789719005.358\\
0.589314732868322	-160746747170.723\\
0.589414735368384	-160703202378.293\\
0.589514737868447	-160659657585.863\\
0.589614740368509	-160616685751.229\\
0.589714742868572	-160573140958.799\\
0.589814745368634	-160530169124.164\\
0.589914747868697	-160486624331.734\\
0.590014750368759	-160443079539.304\\
0.590114752868822	-160400107704.669\\
0.590214755368884	-160356562912.239\\
0.590314757868947	-160313591077.604\\
0.590414760369009	-160270046285.174\\
0.590514762869072	-160226501492.744\\
0.590614765369134	-160183529658.11\\
0.590714767869197	-160139984865.68\\
0.590814770369259	-160097013031.045\\
0.590914772869322	-160053468238.615\\
0.591014775369384	-160010496403.98\\
0.591114777869447	-159966951611.55\\
0.591214780369509	-159923979776.915\\
0.591314782869572	-159880434984.485\\
0.591414785369634	-159837463149.851\\
0.591514787869697	-159793918357.421\\
0.591614790369759	-159750946522.786\\
0.591714792869822	-159707401730.356\\
0.591814795369884	-159664429895.721\\
0.591914797869947	-159620885103.291\\
0.592014800370009	-159577913268.656\\
0.592114802870072	-159534368476.226\\
0.592214805370134	-159491396641.592\\
0.592314807870197	-159447851849.162\\
0.592414810370259	-159404880014.527\\
0.592514812870322	-159361908179.892\\
0.592614815370384	-159318363387.462\\
0.592714817870447	-159275391552.827\\
0.592814820370509	-159231846760.397\\
0.592914822870572	-159188874925.763\\
0.593014825370634	-159145903091.128\\
0.593114827870697	-159102358298.698\\
0.593214830370759	-159059386464.063\\
0.593314832870822	-159015841671.633\\
0.593414835370884	-158972869836.998\\
0.593514837870947	-158929898002.363\\
0.593614840371009	-158886353209.933\\
0.593714842871072	-158843381375.299\\
0.593814845371134	-158800409540.664\\
0.593914847871197	-158756864748.234\\
0.594014850371259	-158713892913.599\\
0.594114852871322	-158670921078.964\\
0.594214855371384	-158627376286.534\\
0.594314857871447	-158584404451.9\\
0.594414860371509	-158541432617.265\\
0.594514862871572	-158498460782.63\\
0.594614865371634	-158454915990.2\\
0.594714867871697	-158411944155.565\\
0.594814870371759	-158368972320.93\\
0.594914872871822	-158325427528.5\\
0.595014875371884	-158282455693.866\\
0.595114877871947	-158239483859.231\\
0.595214880372009	-158196512024.596\\
0.595314882872072	-158153540189.961\\
0.595414885372134	-158109995397.531\\
0.595514887872197	-158067023562.896\\
0.595614890372259	-158024051728.262\\
0.595714892872322	-157981079893.627\\
0.595814895372384	-157938108058.992\\
0.595914897872447	-157894563266.562\\
0.596014900372509	-157851591431.927\\
0.596114902872572	-157808619597.292\\
0.596214905372634	-157765647762.658\\
0.596314907872697	-157722675928.023\\
0.596414910372759	-157679704093.388\\
0.596514912872822	-157636159300.958\\
0.596614915372884	-157593187466.323\\
0.596714917872947	-157550215631.688\\
0.596814920373009	-157507243797.054\\
0.596914922873072	-157464271962.419\\
0.597014925373134	-157421300127.784\\
0.597114927873197	-157378328293.149\\
0.597214930373259	-157335356458.514\\
0.597314932873322	-157292384623.88\\
0.597414935373384	-157248839831.45\\
0.597514937873447	-157205867996.815\\
0.597614940373509	-157162896162.18\\
0.597714942873572	-157119924327.545\\
0.597814945373634	-157076952492.91\\
0.597914947873697	-157033980658.276\\
0.598014950373759	-156991008823.641\\
0.598114952873822	-156948036989.006\\
0.598214955373884	-156905065154.371\\
0.598314957873947	-156862093319.736\\
0.598414960374009	-156819121485.101\\
0.598514962874072	-156776149650.467\\
0.598614965374134	-156733177815.832\\
0.598714967874197	-156690205981.197\\
0.598814970374259	-156647234146.562\\
0.598914972874322	-156604262311.927\\
0.599014975374384	-156561290477.293\\
0.599114977874447	-156518318642.658\\
0.599214980374509	-156475346808.023\\
0.599314982874572	-156432947931.183\\
0.599414985374634	-156389976096.548\\
0.599514987874697	-156347004261.914\\
0.599614990374759	-156304032427.279\\
0.599714992874822	-156261060592.644\\
0.599814995374884	-156218088758.009\\
0.599914997874947	-156175116923.374\\
0.600015000375009	-156132145088.74\\
0.600115002875072	-156089173254.105\\
0.600215005375134	-156046201419.47\\
0.600315007875197	-156003802542.63\\
0.600415010375259	-155960830707.995\\
0.600515012875322	-155917858873.361\\
0.600615015375384	-155874887038.726\\
0.600715017875447	-155831915204.091\\
0.600815020375509	-155788943369.456\\
0.600915022875572	-155746544492.617\\
0.601015025375634	-155703572657.982\\
0.601115027875697	-155660600823.347\\
0.601215030375759	-155617628988.712\\
0.601315032875822	-155574657154.077\\
0.601415035375884	-155532258277.238\\
0.601515037875947	-155489286442.603\\
0.601615040376009	-155446314607.968\\
0.601715042876072	-155403342773.333\\
0.601815045376134	-155360943896.493\\
0.601915047876197	-155317972061.859\\
0.602015050376259	-155275000227.224\\
0.602115052876322	-155232028392.589\\
0.602215055376384	-155189629515.749\\
0.602315057876447	-155146657681.115\\
0.602415060376509	-155103685846.48\\
0.602515062876572	-155061286969.64\\
0.602615065376634	-155018315135.005\\
0.602715067876697	-154975343300.37\\
0.602815070376759	-154932944423.531\\
0.602915072876822	-154889972588.896\\
0.603015075376884	-154847000754.261\\
0.603115077876947	-154804601877.421\\
0.603215080377009	-154761630042.787\\
0.603315082877072	-154718658208.152\\
0.603415085377134	-154676259331.312\\
0.603515087877197	-154633287496.677\\
0.603615090377259	-154590888619.838\\
0.603715092877322	-154547916785.203\\
0.603815095377384	-154504944950.568\\
0.603915097877447	-154462546073.728\\
0.604015100377509	-154419574239.094\\
0.604115102877572	-154377175362.254\\
0.604215105377634	-154334203527.619\\
0.604315107877697	-154291804650.779\\
0.604415110377759	-154248832816.145\\
0.604515112877822	-154205860981.51\\
0.604615115377884	-154163462104.67\\
0.604715117877947	-154120490270.035\\
0.604815120378009	-154078091393.196\\
0.604915122878072	-154035119558.561\\
0.605015125378134	-153992720681.721\\
0.605115127878197	-153949748847.086\\
0.605215130378259	-153907349970.247\\
0.605315132878322	-153864378135.612\\
0.605415135378384	-153821979258.772\\
0.605515137878447	-153779007424.137\\
0.605615140378509	-153736608547.298\\
0.605715142878572	-153694209670.458\\
0.605815145378634	-153651237835.823\\
0.605915147878697	-153608838958.983\\
0.606015150378759	-153565867124.349\\
0.606115152878822	-153523468247.509\\
0.606215155378884	-153480496412.874\\
0.606315157878947	-153438097536.034\\
0.606415160379009	-153395698659.195\\
0.606515162879072	-153352726824.56\\
0.606615165379135	-153310327947.72\\
0.606715167879197	-153267356113.085\\
0.606815170379259	-153224957236.246\\
0.606915172879322	-153182558359.406\\
0.607015175379384	-153139586524.771\\
0.607115177879447	-153097187647.932\\
0.60721518037951	-153054788771.092\\
0.607315182879572	-153011816936.457\\
0.607415185379634	-152969418059.617\\
0.607515187879697	-152927019182.778\\
0.60761519037976	-152884047348.143\\
0.607715192879822	-152841648471.303\\
0.607815195379884	-152799249594.464\\
0.607915197879947	-152756850717.624\\
0.608015200380009	-152713878882.989\\
0.608115202880072	-152671480006.149\\
0.608215205380135	-152629081129.31\\
0.608315207880197	-152586682252.47\\
0.608415210380259	-152543710417.835\\
0.608515212880322	-152501311540.996\\
0.608615215380385	-152458912664.156\\
0.608715217880447	-152416513787.316\\
0.60881522038051	-152373541952.681\\
0.608915222880572	-152331143075.842\\
0.609015225380634	-152288744199.002\\
0.609115227880697	-152246345322.162\\
0.60921523038076	-152203946445.323\\
0.609315232880822	-152161547568.483\\
0.609415235380884	-152118575733.848\\
0.609515237880947	-152076176857.008\\
0.60961524038101	-152033777980.169\\
0.609715242881072	-151991379103.329\\
0.609815245381135	-151948980226.489\\
0.609915247881197	-151906581349.65\\
0.610015250381259	-151864182472.81\\
0.610115252881322	-151821210638.175\\
0.610215255381385	-151778811761.336\\
0.610315257881447	-151736412884.496\\
0.61041526038151	-151694014007.656\\
0.610515262881572	-151651615130.817\\
0.610615265381635	-151609216253.977\\
0.610715267881697	-151566817377.137\\
0.61081527038176	-151524418500.298\\
0.610915272881822	-151482019623.458\\
0.611015275381884	-151439620746.618\\
0.611115277881947	-151397221869.778\\
0.61121528038201	-151354822992.939\\
0.611315282882072	-151312424116.099\\
0.611415285382135	-151270025239.259\\
0.611515287882197	-151227626362.42\\
0.61161529038226	-151185227485.58\\
0.611715292882322	-151142828608.74\\
0.611815295382385	-151100429731.901\\
0.611915297882447	-151058030855.061\\
0.61201530038251	-151015631978.221\\
0.612115302882572	-150973233101.382\\
0.612215305382635	-150930834224.542\\
0.612315307882697	-150888435347.702\\
0.61241531038276	-150846036470.863\\
0.612515312882822	-150803637594.023\\
0.612615315382885	-150761238717.183\\
0.612715317882947	-150718839840.344\\
0.61281532038301	-150677013921.299\\
0.612915322883072	-150634615044.459\\
0.613015325383135	-150592216167.62\\
0.613115327883197	-150549817290.78\\
0.61321533038326	-150507418413.94\\
0.613315332883322	-150465019537.101\\
0.613415335383385	-150422620660.261\\
0.613515337883447	-150380794741.216\\
0.61361534038351	-150338395864.377\\
0.613715342883572	-150295996987.537\\
0.613815345383635	-150253598110.697\\
0.613915347883697	-150211199233.858\\
0.61401535038376	-150168800357.018\\
0.614115352883822	-150126974437.973\\
0.614215355383885	-150084575561.134\\
0.614315357883947	-150042176684.294\\
0.61441536038401	-149999777807.454\\
0.614515362884072	-149957951888.41\\
0.614615365384135	-149915553011.57\\
0.614715367884197	-149873154134.73\\
0.61481537038426	-149830755257.891\\
0.614915372884322	-149788929338.846\\
0.615015375384385	-149746530462.007\\
0.615115377884447	-149704131585.167\\
0.61521538038451	-149661732708.327\\
0.615315382884572	-149619906789.283\\
0.615415385384635	-149577507912.443\\
0.615515387884697	-149535109035.603\\
0.61561539038476	-149493283116.559\\
0.615715392884822	-149450884239.719\\
0.615815395384885	-149408485362.879\\
0.615915397884947	-149366659443.835\\
0.61601540038501	-149324260566.995\\
0.616115402885072	-149282434647.951\\
0.616215405385135	-149240035771.111\\
0.616315407885197	-149197636894.271\\
0.61641541038526	-149155810975.227\\
0.616515412885322	-149113412098.387\\
0.616615415385385	-149071013221.547\\
0.616715417885447	-149029187302.503\\
0.61681542038551	-148986788425.663\\
0.616915422885572	-148944962506.619\\
0.617015425385635	-148902563629.779\\
0.617115427885697	-148860737710.734\\
0.61721543038576	-148818338833.895\\
0.617315432885822	-148776512914.85\\
0.617415435385885	-148734114038.01\\
0.617515437885947	-148692288118.966\\
0.61761544038601	-148649889242.126\\
0.617715442886072	-148608063323.082\\
0.617815445386135	-148565664446.242\\
0.617915447886197	-148523838527.197\\
0.61801545038626	-148481439650.358\\
0.618115452886322	-148439613731.313\\
0.618215455386385	-148397214854.473\\
0.618315457886447	-148355388935.429\\
0.61841546038651	-148312990058.589\\
0.618515462886572	-148271164139.545\\
0.618615465386635	-148228765262.705\\
0.618715467886697	-148186939343.66\\
0.61881547038676	-148145113424.616\\
0.618915472886822	-148102714547.776\\
0.619015475386885	-148060888628.732\\
0.619115477886947	-148018489751.892\\
0.61921548038701	-147976663832.847\\
0.619315482887072	-147934837913.803\\
0.619415485387135	-147892439036.963\\
0.619515487887197	-147850613117.919\\
0.61961549038726	-147808787198.874\\
0.619715492887322	-147766388322.034\\
0.619815495387385	-147724562402.99\\
0.619915497887447	-147682736483.945\\
0.62001550038751	-147640337607.106\\
0.620115502887572	-147598511688.061\\
0.620215505387635	-147556685769.017\\
0.620315507887697	-147514286892.177\\
0.62041551038776	-147472460973.132\\
0.620515512887822	-147430635054.088\\
0.620615515387885	-147388236177.248\\
0.620715517887947	-147346410258.204\\
0.62081552038801	-147304584339.159\\
0.620915522888072	-147262758420.114\\
0.621015525388135	-147220932501.07\\
0.621115527888197	-147178533624.23\\
0.62121553038826	-147136707705.186\\
0.621315532888322	-147094881786.141\\
0.621415535388385	-147053055867.097\\
0.621515537888447	-147010656990.257\\
0.62161554038851	-146968831071.212\\
0.621715542888572	-146927005152.168\\
0.621815545388635	-146885179233.123\\
0.621915547888697	-146843353314.079\\
0.62201555038876	-146801527395.034\\
0.622115552888822	-146759701475.99\\
0.622215555388885	-146717302599.15\\
0.622315557888947	-146675476680.105\\
0.62241556038901	-146633650761.061\\
0.622515562889072	-146591824842.016\\
0.622615565389135	-146549998922.972\\
0.622715567889197	-146508173003.927\\
0.62281557038926	-146466347084.883\\
0.622915572889322	-146424521165.838\\
0.623015575389385	-146382695246.794\\
0.623115577889447	-146340869327.749\\
0.62321558038951	-146298470450.909\\
0.623315582889572	-146256644531.865\\
0.623415585389635	-146214818612.82\\
0.623515587889697	-146172992693.776\\
0.62361559038976	-146131166774.731\\
0.623715592889822	-146089340855.687\\
0.623815595389885	-146047514936.642\\
0.623915597889947	-146005689017.597\\
0.62401560039001	-145963863098.553\\
0.624115602890072	-145922037179.508\\
0.624215605390135	-145880211260.464\\
0.624315607890197	-145838385341.419\\
0.62441561039026	-145797132380.17\\
0.624515612890322	-145755306461.125\\
0.624615615390385	-145713480542.081\\
0.624715617890447	-145671654623.036\\
0.62481562039051	-145629828703.992\\
0.624915622890572	-145588002784.947\\
0.625015625390635	-145546176865.902\\
0.625115627890697	-145504350946.858\\
0.62521563039076	-145462525027.813\\
0.625315632890822	-145420699108.769\\
0.625415635390885	-145378873189.724\\
0.625515637890947	-145337620228.475\\
0.62561564039101	-145295794309.43\\
0.625715642891072	-145253968390.386\\
0.625815645391135	-145212142471.341\\
0.625915647891197	-145170316552.297\\
0.62601565039126	-145128490633.252\\
0.626115652891322	-145087237672.003\\
0.626215655391385	-145045411752.958\\
0.626315657891447	-145003585833.914\\
0.62641566039151	-144961759914.869\\
0.626515662891572	-144919933995.825\\
0.626615665391635	-144878681034.575\\
0.626715667891697	-144836855115.531\\
0.62681567039176	-144795029196.486\\
0.626915672891822	-144753203277.441\\
0.627015675391885	-144711950316.192\\
0.627115677891947	-144670124397.147\\
0.62721568039201	-144628298478.103\\
0.627315682892072	-144587045516.854\\
0.627415685392135	-144545219597.809\\
0.627515687892197	-144503393678.764\\
0.62761569039226	-144461567759.72\\
0.627715692892322	-144420314798.47\\
0.627815695392385	-144378488879.426\\
0.627915697892447	-144336662960.381\\
0.62801570039251	-144295409999.132\\
0.628115702892572	-144253584080.087\\
0.628215705392635	-144212331118.838\\
0.628315707892697	-144170505199.793\\
0.62841571039276	-144128679280.749\\
0.628515712892822	-144087426319.499\\
0.628615715392885	-144045600400.455\\
0.628715717892947	-144004347439.205\\
0.62881572039301	-143962521520.161\\
0.628915722893072	-143920695601.116\\
0.629015725393135	-143879442639.867\\
0.629115727893197	-143837616720.822\\
0.62921573039326	-143796363759.573\\
0.629315732893322	-143754537840.528\\
0.629415735393385	-143713284879.279\\
0.629515737893447	-143671458960.234\\
0.62961574039351	-143630205998.985\\
0.629715742893572	-143588380079.94\\
0.629815745393635	-143547127118.691\\
0.629915747893697	-143505301199.647\\
0.63001575039376	-143464048238.397\\
0.630115752893822	-143422222319.353\\
0.630215755393885	-143380969358.103\\
0.630315757893947	-143339143439.059\\
0.63041576039401	-143297890477.809\\
0.630515762894072	-143256064558.765\\
0.630615765394135	-143214811597.515\\
0.630715767894197	-143172985678.471\\
0.63081577039426	-143131732717.221\\
0.630915772894322	-143090479755.972\\
0.631015775394385	-143048653836.927\\
0.631115777894447	-143007400875.678\\
0.63121578039451	-142966147914.428\\
0.631315782894572	-142924321995.384\\
0.631415785394635	-142883069034.134\\
0.631515787894697	-142841243115.09\\
0.63161579039476	-142799990153.84\\
0.631715792894822	-142758737192.591\\
0.631815795394885	-142716911273.547\\
0.631915797894947	-142675658312.297\\
0.63201580039501	-142634405351.048\\
0.632115802895072	-142592579432.003\\
0.632215805395135	-142551326470.754\\
0.632315807895197	-142510073509.504\\
0.63241581039526	-142468820548.255\\
0.632515812895322	-142426994629.21\\
0.632615815395385	-142385741667.961\\
0.632715817895447	-142344488706.711\\
0.63281582039551	-142303235745.462\\
0.632915822895572	-142261409826.417\\
0.633015825395635	-142220156865.168\\
0.633115827895697	-142178903903.919\\
0.63321583039576	-142137650942.669\\
0.633315832895822	-142096397981.42\\
0.633415835395885	-142054572062.375\\
0.633515837895947	-142013319101.126\\
0.63361584039601	-141972066139.876\\
0.633715842896072	-141930813178.627\\
0.633815845396135	-141889560217.378\\
0.633915847896197	-141848307256.128\\
0.63401585039626	-141806481337.084\\
0.634115852896322	-141765228375.834\\
0.634215855396385	-141723975414.585\\
0.634315857896447	-141682722453.335\\
0.63441586039651	-141641469492.086\\
0.634515862896572	-141600216530.837\\
0.634615865396635	-141558963569.587\\
0.634715867896697	-141517710608.338\\
0.63481587039676	-141476457647.088\\
0.634915872896822	-141435204685.839\\
0.635015875396885	-141393378766.794\\
0.635115877896947	-141352125805.545\\
0.63521588039701	-141310872844.295\\
0.635315882897072	-141269619883.046\\
0.635415885397135	-141228366921.797\\
0.635515887897197	-141187113960.547\\
0.63561589039726	-141145860999.298\\
0.635715892897322	-141104608038.048\\
0.635815895397385	-141063355076.799\\
0.635915897897447	-141022102115.55\\
0.63601590039751	-140980849154.3\\
0.636115902897572	-140940169150.846\\
0.636215905397635	-140898916189.596\\
0.636315907897697	-140857663228.347\\
0.63641591039776	-140816410267.098\\
0.636515912897823	-140775157305.848\\
0.636615915397885	-140733904344.599\\
0.636715917897947	-140692651383.349\\
0.63681592039801	-140651398422.1\\
0.636915922898072	-140610145460.85\\
0.637015925398135	-140568892499.601\\
0.637115927898197	-140527639538.352\\
0.63721593039826	-140486959534.897\\
0.637315932898322	-140445706573.648\\
0.637415935398385	-140404453612.398\\
0.637515937898448	-140363200651.149\\
0.63761594039851	-140321947689.9\\
0.637715942898572	-140280694728.65\\
0.637815945398635	-140240014725.196\\
0.637915947898697	-140198761763.947\\
0.63801595039876	-140157508802.697\\
0.638115952898823	-140116255841.448\\
0.638215955398885	-140075002880.198\\
0.638315957898947	-140034322876.744\\
0.63841596039901	-139993069915.495\\
0.638515962899073	-139951816954.245\\
0.638615965399135	-139910563992.996\\
0.638715967899197	-139869883989.541\\
0.63881597039926	-139828631028.292\\
0.638915972899322	-139787378067.043\\
0.639015975399385	-139746698063.588\\
0.639115977899448	-139705445102.339\\
0.63921598039951	-139664192141.089\\
0.639315982899572	-139623512137.635\\
0.639415985399635	-139582259176.386\\
0.639515987899698	-139541006215.136\\
0.63961599039976	-139500326211.682\\
0.639715992899823	-139459073250.433\\
0.639815995399885	-139417820289.183\\
0.639915997899947	-139377140285.729\\
0.64001600040001	-139335887324.48\\
0.640116002900073	-139294634363.23\\
0.640216005400135	-139253954359.776\\
0.640316007900197	-139212701398.526\\
0.64041601040026	-139172021395.072\\
0.640516012900323	-139130768433.823\\
0.640616015400385	-139090088430.368\\
0.640716017900448	-139048835469.119\\
0.64081602040051	-139007582507.87\\
0.640916022900572	-138966902504.415\\
0.641016025400635	-138925649543.166\\
0.641116027900698	-138884969539.712\\
0.64121603040076	-138843716578.462\\
0.641316032900823	-138803036575.008\\
0.641416035400885	-138761783613.758\\
0.641516037900948	-138721103610.304\\
0.64161604040101	-138679850649.055\\
0.641716042901073	-138639170645.6\\
0.641816045401135	-138597917684.351\\
0.641916047901197	-138557237680.897\\
0.64201605040126	-138516557677.442\\
0.642116052901323	-138475304716.193\\
0.642216055401385	-138434624712.739\\
0.642316057901448	-138393371751.489\\
0.64241606040151	-138352691748.035\\
0.642516062901573	-138312011744.581\\
0.642616065401635	-138270758783.331\\
0.642716067901698	-138230078779.877\\
0.64281607040176	-138188825818.628\\
0.642916072901823	-138148145815.173\\
0.643016075401885	-138107465811.719\\
0.643116077901948	-138066212850.47\\
0.64321608040201	-138025532847.015\\
0.643316082902073	-137984852843.561\\
0.643416085402135	-137943599882.312\\
0.643516087902198	-137902919878.857\\
0.64361609040226	-137862239875.403\\
0.643716092902323	-137821559871.949\\
0.643816095402385	-137780306910.699\\
0.643916097902448	-137739626907.245\\
0.64401610040251	-137698946903.791\\
0.644116102902573	-137657693942.541\\
0.644216105402635	-137617013939.087\\
0.644316107902698	-137576333935.633\\
0.64441611040276	-137535653932.178\\
0.644516112902823	-137494973928.724\\
0.644616115402885	-137453720967.475\\
0.644716117902948	-137413040964.02\\
0.64481612040301	-137372360960.566\\
0.644916122903073	-137331680957.112\\
0.645016125403135	-137291000953.658\\
0.645116127903198	-137249747992.408\\
0.64521613040326	-137209067988.954\\
0.645316132903323	-137168387985.5\\
0.645416135403385	-137127707982.045\\
0.645516137903448	-137087027978.591\\
0.64561614040351	-137046347975.137\\
0.645716142903573	-137005667971.682\\
0.645816145403635	-136964987968.228\\
0.645916147903698	-136923735006.979\\
0.64601615040376	-136883055003.524\\
0.646116152903823	-136842375000.07\\
0.646216155403885	-136801694996.616\\
0.646316157903948	-136761014993.162\\
0.64641616040401	-136720334989.707\\
0.646516162904073	-136679654986.253\\
0.646616165404135	-136638974982.799\\
0.646716167904198	-136598294979.344\\
0.64681617040426	-136557614975.89\\
0.646916172904323	-136516934972.436\\
0.647016175404385	-136476254968.982\\
0.647116177904448	-136435574965.527\\
0.64721618040451	-136394894962.073\\
0.647316182904573	-136354214958.619\\
0.647416185404635	-136313534955.164\\
0.647516187904698	-136272854951.71\\
0.64761619040476	-136232174948.256\\
0.647716192904823	-136191494944.802\\
0.647816195404885	-136150814941.347\\
0.647916197904948	-136110707895.688\\
0.64801620040501	-136070027892.234\\
0.648116202905073	-136029347888.78\\
0.648216205405135	-135988667885.325\\
0.648316207905198	-135947987881.871\\
0.64841621040526	-135907307878.417\\
0.648516212905323	-135866627874.962\\
0.648616215405385	-135825947871.508\\
0.648716217905448	-135785840825.849\\
0.64881622040551	-135745160822.395\\
0.648916222905573	-135704480818.94\\
0.649016225405635	-135663800815.486\\
0.649116227905698	-135623120812.032\\
0.64921623040576	-135583013766.373\\
0.649316232905823	-135542333762.918\\
0.649416235405885	-135501653759.464\\
0.649516237905948	-135460973756.01\\
0.64961624040601	-135420293752.555\\
0.649716242906073	-135380186706.896\\
0.649816245406135	-135339506703.442\\
0.649916247906198	-135298826699.988\\
0.65001625040626	-135258719654.329\\
0.650116252906323	-135218039650.874\\
0.650216255406385	-135177359647.42\\
0.650316257906448	-135136679643.966\\
0.65041626040651	-135096572598.307\\
0.650516262906573	-135055892594.852\\
0.650616265406635	-135015212591.398\\
0.650716267906698	-134975105545.739\\
0.65081627040676	-134934425542.285\\
0.650916272906823	-134894318496.625\\
0.651016275406885	-134853638493.171\\
0.651116277906948	-134812958489.717\\
0.65121628040701	-134772851444.058\\
0.651316282907073	-134732171440.603\\
0.651416285407135	-134691491437.149\\
0.651516287907198	-134651384391.49\\
0.65161629040726	-134610704388.036\\
0.651716292907323	-134570597342.376\\
0.651816295407385	-134529917338.922\\
0.651916297907448	-134489810293.263\\
0.65201630040751	-134449130289.809\\
0.652116302907573	-134409023244.15\\
0.652216305407635	-134368343240.695\\
0.652316307907698	-134328236195.036\\
0.65241631040776	-134287556191.582\\
0.652516312907823	-134247449145.923\\
0.652616315407885	-134206769142.468\\
0.652716317907948	-134166662096.809\\
0.65281632040801	-134125982093.355\\
0.652916322908073	-134085875047.696\\
0.653016325408135	-134045195044.241\\
0.653116327908198	-134005087998.582\\
0.65321633040826	-133964407995.128\\
0.653316332908323	-133924300949.469\\
0.653416335408385	-133884193903.81\\
0.653516337908448	-133843513900.355\\
0.65361634040851	-133803406854.696\\
0.653716342908573	-133762726851.242\\
0.653816345408635	-133722619805.583\\
0.653916347908698	-133682512759.924\\
0.65401635040876	-133641832756.469\\
0.654116352908823	-133601725710.81\\
0.654216355408885	-133561618665.151\\
0.654316357908948	-133520938661.697\\
0.65441636040901	-133480831616.038\\
0.654516362909073	-133440724570.378\\
0.654616365409135	-133400044566.924\\
0.654716367909198	-133359937521.265\\
0.65481637040926	-133319830475.606\\
0.654916372909323	-133279723429.947\\
0.655016375409385	-133239043426.492\\
0.655116377909448	-133198936380.833\\
0.65521638040951	-133158829335.174\\
0.655316382909573	-133118722289.515\\
0.655416385409635	-133078042286.061\\
0.655516387909698	-133037935240.402\\
0.65561639040976	-132997828194.742\\
0.655716392909823	-132957721149.083\\
0.655816395409885	-132917614103.424\\
0.655916397909948	-132876934099.97\\
0.65601640041001	-132836827054.311\\
0.656116402910073	-132796720008.651\\
0.656216405410135	-132756612962.992\\
0.656316407910198	-132716505917.333\\
0.65641641041026	-132676398871.674\\
0.656516412910323	-132635718868.22\\
0.656616415410385	-132595611822.561\\
0.656716417910448	-132555504776.901\\
0.65681642041051	-132515397731.242\\
0.656916422910573	-132475290685.583\\
0.657016425410635	-132435183639.924\\
0.657116427910698	-132395076594.265\\
0.65721643041076	-132354969548.606\\
0.657316432910823	-132314862502.946\\
0.657416435410885	-132274755457.287\\
0.657516437910948	-132234648411.628\\
0.65761644041101	-132194541365.969\\
0.657716442911073	-132154434320.31\\
0.657816445411135	-132114327274.651\\
0.657916447911198	-132074220228.991\\
0.65801645041126	-132034113183.332\\
0.658116452911323	-131994006137.673\\
0.658216455411385	-131953899092.014\\
0.658316457911448	-131913792046.355\\
0.65841646041151	-131873685000.696\\
0.658516462911573	-131833577955.037\\
0.658616465411635	-131793470909.377\\
0.658716467911698	-131753363863.718\\
0.65881647041176	-131713256818.059\\
0.658916472911823	-131673149772.4\\
0.659016475411885	-131633042726.741\\
0.659116477911948	-131593508638.877\\
0.65921648041201	-131553401593.218\\
0.659316482912073	-131513294547.558\\
0.659416485412135	-131473187501.899\\
0.659516487912198	-131433080456.24\\
0.65961649041226	-131392973410.581\\
0.659716492912323	-131352866364.922\\
0.659816495412385	-131313332277.058\\
0.659916497912448	-131273225231.399\\
0.66001650041251	-131233118185.739\\
0.660116502912573	-131193011140.08\\
0.660216505412635	-131153477052.216\\
0.660316507912698	-131113370006.557\\
0.66041651041276	-131073262960.898\\
0.660516512912823	-131033155915.239\\
0.660616515412885	-130993048869.58\\
0.660716517912948	-130953514781.716\\
0.66081652041301	-130913407736.056\\
0.660916522913073	-130873300690.397\\
0.661016525413135	-130833766602.533\\
0.661116527913198	-130793659556.874\\
0.66121653041326	-130753552511.215\\
0.661316532913323	-130714018423.351\\
0.661416535413385	-130673911377.692\\
0.661516537913448	-130633804332.033\\
0.66161654041351	-130594270244.169\\
0.661716542913573	-130554163198.509\\
0.661816545413635	-130514056152.85\\
0.661916547913698	-130474522064.986\\
0.66201655041376	-130434415019.327\\
0.662116552913823	-130394880931.463\\
0.662216555413885	-130354773885.804\\
0.662316557913948	-130314666840.145\\
0.66241656041401	-130275132752.281\\
0.662516562914073	-130235025706.622\\
0.662616565414135	-130195491618.757\\
0.662716567914198	-130155384573.098\\
0.66281657041426	-130115850485.234\\
0.662916572914323	-130075743439.575\\
0.663016575414385	-130036209351.711\\
0.663116577914448	-129996102306.052\\
0.66321658041451	-129956568218.188\\
0.663316582914573	-129916461172.529\\
0.663416585414635	-129876927084.665\\
0.663516587914698	-129836820039.006\\
0.66361659041476	-129797285951.142\\
0.663716592914823	-129757178905.482\\
0.663816595414885	-129717644817.618\\
0.663916597914948	-129677537771.959\\
0.66401660041501	-129638003684.095\\
0.664116602915073	-129598469596.231\\
0.664216605415135	-129558362550.572\\
0.664316607915198	-129518828462.708\\
0.66441661041526	-129478721417.049\\
0.664516612915323	-129439187329.185\\
0.664616615415385	-129399653241.321\\
0.664716617915448	-129359546195.662\\
0.66481662041551	-129320012107.798\\
0.664916622915573	-129280478019.934\\
0.665016625415635	-129240370974.274\\
0.665116627915698	-129200836886.41\\
0.66521663041576	-129161302798.546\\
0.665316632915823	-129121195752.887\\
0.665416635415885	-129081661665.023\\
0.665516637915948	-129042127577.159\\
0.66561664041601	-129002593489.295\\
0.665716642916073	-128962486443.636\\
0.665816645416135	-128922952355.772\\
0.665916647916198	-128883418267.908\\
0.66601665041626	-128843884180.044\\
0.666116652916323	-128803777134.385\\
0.666216655416385	-128764243046.521\\
0.666316657916448	-128724708958.657\\
0.66641666041651	-128685174870.793\\
0.666516662916573	-128645640782.929\\
0.666616665416635	-128605533737.269\\
0.666716667916698	-128565999649.405\\
0.66681667041676	-128526465561.541\\
0.666916672916823	-128486931473.677\\
0.667016675416885	-128447397385.813\\
0.667116677916948	-128407863297.949\\
0.66721668041701	-128368329210.085\\
0.667316682917073	-128328222164.426\\
0.667416685417135	-128288688076.562\\
0.667516687917198	-128249153988.698\\
0.66761669041726	-128209619900.834\\
0.667716692917323	-128170085812.97\\
0.667816695417385	-128130551725.106\\
0.667916697917448	-128091017637.242\\
0.66801670041751	-128051483549.378\\
0.668116702917573	-128011949461.514\\
0.668216705417635	-127972415373.65\\
0.668316707917698	-127932881285.786\\
0.66841671041776	-127893347197.922\\
0.668516712917823	-127853813110.058\\
0.668616715417885	-127814279022.194\\
0.668716717917948	-127774744934.33\\
0.66881672041801	-127735210846.466\\
0.668916722918073	-127695676758.602\\
0.669016725418136	-127656142670.738\\
0.669116727918198	-127616608582.874\\
0.66921673041826	-127577074495.01\\
0.669316732918323	-127537540407.146\\
0.669416735418385	-127498006319.282\\
0.669516737918448	-127459045189.213\\
0.66961674041851	-127419511101.349\\
0.669716742918573	-127379977013.485\\
0.669816745418635	-127340442925.621\\
0.669916747918698	-127300908837.757\\
0.670016750418761	-127261374749.893\\
0.670116752918823	-127221840662.029\\
0.670216755418885	-127182879531.96\\
0.670316757918948	-127143345444.096\\
0.67041676041901	-127103811356.232\\
0.670516762919073	-127064277268.368\\
0.670616765419136	-127024743180.504\\
0.670716767919198	-126985782050.435\\
0.67081677041926	-126946247962.571\\
0.670916772919323	-126906713874.707\\
0.671016775419386	-126867179786.843\\
0.671116777919448	-126828218656.774\\
0.67121678041951	-126788684568.91\\
0.671316782919573	-126749150481.046\\
0.671416785419635	-126709616393.182\\
0.671516787919698	-126670655263.113\\
0.671616790419761	-126631121175.249\\
0.671716792919823	-126591587087.385\\
0.671816795419885	-126552625957.316\\
0.671916797919948	-126513091869.452\\
0.672016800420011	-126473557781.588\\
0.672116802920073	-126434596651.519\\
0.672216805420136	-126395062563.655\\
0.672316807920198	-126355528475.791\\
0.67241681042026	-126316567345.722\\
0.672516812920323	-126277033257.858\\
0.672616815420386	-126238072127.789\\
0.672716817920448	-126198538039.925\\
0.67281682042051	-126159003952.061\\
0.672916822920573	-126120042821.992\\
0.673016825420636	-126080508734.128\\
0.673116827920698	-126041547604.059\\
0.673216830420761	-126002013516.195\\
0.673316832920823	-125963052386.126\\
0.673416835420885	-125923518298.262\\
0.673516837920948	-125884557168.193\\
0.673616840421011	-125845023080.329\\
0.673716842921073	-125806061950.26\\
0.673816845421136	-125766527862.396\\
0.673916847921198	-125727566732.327\\
0.674016850421261	-125688032644.463\\
0.674116852921323	-125649071514.394\\
0.674216855421386	-125609537426.53\\
0.674316857921448	-125570576296.461\\
0.674416860421511	-125531042208.597\\
0.674516862921573	-125492081078.529\\
0.674616865421636	-125453119948.46\\
0.674716867921698	-125413585860.596\\
0.674816870421761	-125374624730.527\\
0.674916872921823	-125335090642.663\\
0.675016875421886	-125296129512.594\\
0.675116877921948	-125257168382.525\\
0.675216880422011	-125217634294.661\\
0.675316882922073	-125178673164.592\\
0.675416885422136	-125139712034.523\\
0.675516887922198	-125100177946.659\\
0.675616890422261	-125061216816.59\\
0.675716892922323	-125022255686.521\\
0.675816895422386	-124983294556.452\\
0.675916897922448	-124943760468.588\\
0.676016900422511	-124904799338.519\\
0.676116902922573	-124865838208.451\\
0.676216905422636	-124826877078.382\\
0.676316907922698	-124787342990.518\\
0.676416910422761	-124748381860.449\\
0.676516912922823	-124709420730.38\\
0.676616915422886	-124670459600.311\\
0.676716917922948	-124630925512.447\\
0.676816920423011	-124591964382.378\\
0.676916922923073	-124553003252.309\\
0.677016925423136	-124514042122.24\\
0.677116927923198	-124475080992.171\\
0.677216930423261	-124436119862.102\\
0.677316932923323	-124396585774.238\\
0.677416935423386	-124357624644.17\\
0.677516937923448	-124318663514.101\\
0.677616940423511	-124279702384.032\\
0.677716942923573	-124240741253.963\\
0.677816945423636	-124201780123.894\\
0.677916947923698	-124162818993.825\\
0.678016950423761	-124123857863.756\\
0.678116952923823	-124084896733.687\\
0.678216955423886	-124045935603.618\\
0.678316957923948	-124006974473.549\\
0.678416960424011	-123968013343.481\\
0.678516962924073	-123929052213.412\\
0.678616965424136	-123890091083.343\\
0.678716967924198	-123851129953.274\\
0.678816970424261	-123812168823.205\\
0.678916972924323	-123773207693.136\\
0.679016975424386	-123734246563.067\\
0.679116977924448	-123695285432.998\\
0.679216980424511	-123656324302.929\\
0.679316982924573	-123617363172.861\\
0.679416985424636	-123578402042.792\\
0.679516987924698	-123539440912.723\\
0.679616990424761	-123500479782.654\\
0.679716992924823	-123461518652.585\\
0.679816995424886	-123422557522.516\\
0.679916997924948	-123383596392.447\\
0.680017000425011	-123344635262.378\\
0.680117002925073	-123306247090.104\\
0.680217005425136	-123267285960.036\\
0.680317007925198	-123228324829.967\\
0.680417010425261	-123189363699.898\\
0.680517012925323	-123150402569.829\\
0.680617015425386	-123111441439.76\\
0.680717017925448	-123073053267.486\\
0.680817020425511	-123034092137.417\\
0.680917022925573	-122995131007.348\\
0.681017025425636	-122956169877.28\\
0.681117027925698	-122917208747.211\\
0.681217030425761	-122878820574.937\\
0.681317032925823	-122839859444.868\\
0.681417035425886	-122800898314.799\\
0.681517037925948	-122761937184.73\\
0.681617040426011	-122723549012.456\\
0.681717042926073	-122684587882.388\\
0.681817045426136	-122645626752.319\\
0.681917047926198	-122607238580.045\\
0.682017050426261	-122568277449.976\\
0.682117052926323	-122529316319.907\\
0.682217055426386	-122490928147.633\\
0.682317057926448	-122451967017.564\\
0.682417060426511	-122413005887.496\\
0.682517062926573	-122374617715.222\\
0.682617065426636	-122335656585.153\\
0.682717067926698	-122296695455.084\\
0.682817070426761	-122258307282.81\\
0.682917072926823	-122219346152.741\\
0.683017075426886	-122180957980.468\\
0.683117077926948	-122141996850.399\\
0.683217080427011	-122103608678.125\\
0.683317082927073	-122064647548.056\\
0.683417085427136	-122026259375.782\\
0.683517087927198	-121987298245.713\\
0.683617090427261	-121948337115.644\\
0.683717092927323	-121909948943.371\\
0.683817095427386	-121870987813.302\\
0.683917097927448	-121832599641.028\\
0.684017100427511	-121793638510.959\\
0.684117102927573	-121755250338.685\\
0.684217105427636	-121716862166.412\\
0.684317107927698	-121677901036.343\\
0.684417110427761	-121639512864.069\\
0.684517112927823	-121600551734\\
0.684617115427886	-121562163561.726\\
0.684717117927948	-121523202431.657\\
0.684817120428011	-121484814259.384\\
0.684917122928073	-121446426087.11\\
0.685017125428136	-121407464957.041\\
0.685117127928198	-121369076784.767\\
0.685217130428261	-121330688612.493\\
0.685317132928323	-121291727482.424\\
0.685417135428386	-121253339310.151\\
0.685517137928448	-121214378180.082\\
0.685617140428511	-121175990007.808\\
0.685717142928573	-121137601835.534\\
0.685817145428636	-121099213663.261\\
0.685917147928698	-121060252533.192\\
0.686017150428761	-121021864360.918\\
0.686117152928823	-120983476188.644\\
0.686217155428886	-120944515058.575\\
0.686317157928948	-120906126886.301\\
0.686417160429011	-120867738714.028\\
0.686517162929073	-120829350541.754\\
0.686617165429136	-120790962369.48\\
0.686717167929198	-120752001239.411\\
0.686817170429261	-120713613067.137\\
0.686917172929323	-120675224894.864\\
0.687017175429386	-120636836722.59\\
0.687117177929448	-120598448550.316\\
0.687217180429511	-120559487420.247\\
0.687317182929573	-120521099247.974\\
0.687417185429636	-120482711075.7\\
0.687517187929698	-120444322903.426\\
0.687617190429761	-120405934731.152\\
0.687717192929823	-120367546558.878\\
0.687817195429886	-120329158386.605\\
0.687917197929948	-120290770214.331\\
0.688017200430011	-120251809084.262\\
0.688117202930073	-120213420911.988\\
0.688217205430136	-120175032739.715\\
0.688317207930198	-120136644567.441\\
0.688417210430261	-120098256395.167\\
0.688517212930323	-120059868222.893\\
0.688617215430386	-120021480050.619\\
0.688717217930448	-119983091878.346\\
0.688817220430511	-119944703706.072\\
0.688917222930573	-119906315533.798\\
0.689017225430636	-119867927361.524\\
0.689117227930698	-119829539189.251\\
0.689217230430761	-119791151016.977\\
0.689317232930823	-119752762844.703\\
0.689417235430886	-119714374672.429\\
0.689517237930948	-119675986500.156\\
0.689617240431011	-119638171285.677\\
0.689717242931073	-119599783113.403\\
0.689817245431136	-119561394941.129\\
0.689917247931198	-119523006768.856\\
0.690017250431261	-119484618596.582\\
0.690117252931323	-119446230424.308\\
0.690217255431386	-119407842252.034\\
0.690317257931448	-119369454079.761\\
0.690417260431511	-119331638865.282\\
0.690517262931573	-119293250693.008\\
0.690617265431636	-119254862520.734\\
0.690717267931698	-119216474348.461\\
0.690817270431761	-119178086176.187\\
0.690917272931823	-119139698003.913\\
0.691017275431886	-119101882789.434\\
0.691117277931948	-119063494617.161\\
0.691217280432011	-119025106444.887\\
0.691317282932073	-118986718272.613\\
0.691417285432136	-118948903058.135\\
0.691517287932198	-118910514885.861\\
0.691617290432261	-118872126713.587\\
0.691717292932323	-118834311499.108\\
0.691817295432386	-118795923326.835\\
0.691917297932448	-118757535154.561\\
0.692017300432511	-118719146982.287\\
0.692117302932573	-118681331767.808\\
0.692217305432636	-118642943595.535\\
0.692317307932698	-118604555423.261\\
0.692417310432761	-118566740208.782\\
0.692517312932823	-118528352036.509\\
0.692617315432886	-118490536822.03\\
0.692717317932948	-118452148649.756\\
0.692817320433011	-118413760477.482\\
0.692917322933073	-118375945263.004\\
0.693017325433136	-118337557090.73\\
0.693117327933198	-118299741876.251\\
0.693217330433261	-118261353703.978\\
0.693317332933323	-118223538489.499\\
0.693417335433386	-118185150317.225\\
0.693517337933448	-118146762144.951\\
0.693617340433511	-118108946930.473\\
0.693717342933573	-118070558758.199\\
0.693817345433636	-118032743543.72\\
0.693917347933698	-117994355371.447\\
0.694017350433761	-117956540156.968\\
0.694117352933823	-117918151984.694\\
0.694217355433886	-117880336770.216\\
0.694317357933948	-117842521555.737\\
0.694417360434011	-117804133383.463\\
0.694517362934073	-117766318168.985\\
0.694617365434136	-117727929996.711\\
0.694717367934198	-117690114782.232\\
0.694817370434261	-117652299567.754\\
0.694917372934323	-117613911395.48\\
0.695017375434386	-117576096181.001\\
0.695117377934448	-117537708008.727\\
0.695217380434511	-117499892794.249\\
0.695317382934573	-117462077579.77\\
0.695417385434636	-117423689407.496\\
0.695517387934698	-117385874193.018\\
0.695617390434761	-117348058978.539\\
0.695717392934823	-117309670806.265\\
0.695817395434886	-117271855591.787\\
0.695917397934948	-117234040377.308\\
0.696017400435011	-117196225162.829\\
0.696117402935073	-117157836990.556\\
0.696217405435136	-117120021776.077\\
0.696317407935198	-117082206561.598\\
0.696417410435261	-117044391347.12\\
0.696517412935323	-117006003174.846\\
0.696617415435386	-116968187960.367\\
0.696717417935448	-116930372745.889\\
0.696817420435511	-116892557531.41\\
0.696917422935573	-116854742316.931\\
0.697017425435636	-116816354144.658\\
0.697117427935698	-116778538930.179\\
0.697217430435761	-116740723715.7\\
0.697317432935823	-116702908501.222\\
0.697417435435886	-116665093286.743\\
0.697517437935948	-116627278072.264\\
0.697617440436011	-116589462857.786\\
0.697717442936073	-116551074685.512\\
0.697817445436136	-116513259471.033\\
0.697917447936198	-116475444256.555\\
0.698017450436261	-116437629042.076\\
0.698117452936323	-116399813827.598\\
0.698217455436386	-116361998613.119\\
0.698317457936448	-116324183398.64\\
0.698417460436511	-116286368184.162\\
0.698517462936573	-116248552969.683\\
0.698617465436636	-116210737755.204\\
0.698717467936698	-116172922540.726\\
0.698817470436761	-116135107326.247\\
0.698917472936823	-116097292111.768\\
0.699017475436886	-116059476897.29\\
0.699117477936948	-116021661682.811\\
0.699217480437011	-115983846468.333\\
0.699317482937073	-115946031253.854\\
0.699417485437136	-115908216039.375\\
0.699517487937198	-115870400824.897\\
0.699617490437261	-115832585610.418\\
0.699717492937323	-115795343353.735\\
0.699817495437386	-115757528139.256\\
0.699917497937448	-115719712924.777\\
0.700017500437511	-115681897710.299\\
0.700117502937573	-115644082495.82\\
0.700217505437636	-115606267281.341\\
0.700317507937698	-115568452066.863\\
0.700417510437761	-115531209810.179\\
0.700517512937823	-115493394595.701\\
0.700617515437886	-115455579381.222\\
0.700717517937948	-115417764166.743\\
0.700817520438011	-115379948952.265\\
0.700917522938073	-115342706695.581\\
0.701017525438136	-115304891481.103\\
0.701117527938198	-115267076266.624\\
0.701217530438261	-115229261052.145\\
0.701317532938323	-115192018795.462\\
0.701417535438386	-115154203580.983\\
0.701517537938449	-115116388366.504\\
0.701617540438511	-115078573152.026\\
0.701717542938573	-115041330895.342\\
0.701817545438636	-115003515680.864\\
0.701917547938698	-114965700466.385\\
0.702017550438761	-114928458209.702\\
0.702117552938823	-114890642995.223\\
0.702217555438886	-114852827780.744\\
0.702317557938948	-114815585524.061\\
0.702417560439011	-114777770309.582\\
0.702517562939074	-114740528052.899\\
0.702617565439136	-114702712838.42\\
0.702717567939198	-114664897623.941\\
0.702817570439261	-114627655367.258\\
0.702917572939323	-114589840152.779\\
0.703017575439386	-114552597896.096\\
0.703117577939449	-114514782681.617\\
0.703217580439511	-114477540424.934\\
0.703317582939573	-114439725210.455\\
0.703417585439636	-114402482953.771\\
0.703517587939699	-114364667739.293\\
0.703617590439761	-114327425482.609\\
0.703717592939823	-114289610268.131\\
0.703817595439886	-114252368011.447\\
0.703917597939948	-114214552796.969\\
0.704017600440011	-114177310540.285\\
0.704117602940074	-114139495325.806\\
0.704217605440136	-114102253069.123\\
0.704317607940198	-114064437854.644\\
0.704417610440261	-114027195597.961\\
0.704517612940324	-113989953341.277\\
0.704617615440386	-113952138126.799\\
0.704717617940449	-113914895870.115\\
0.704817620440511	-113877080655.637\\
0.704917622940573	-113839838398.953\\
0.705017625440636	-113802596142.27\\
0.705117627940699	-113764780927.791\\
0.705217630440761	-113727538671.107\\
0.705317632940824	-113690296414.424\\
0.705417635440886	-113652481199.945\\
0.705517637940949	-113615238943.262\\
0.705617640441011	-113577996686.578\\
0.705717642941074	-113540181472.1\\
0.705817645441136	-113502939215.416\\
0.705917647941198	-113465696958.733\\
0.706017650441261	-113428454702.049\\
0.706117652941324	-113390639487.57\\
0.706217655441386	-113353397230.887\\
0.706317657941449	-113316154974.203\\
0.706417660441511	-113278912717.52\\
0.706517662941574	-113241670460.836\\
0.706617665441636	-113203855246.358\\
0.706717667941699	-113166612989.674\\
0.706817670441761	-113129370732.991\\
0.706917672941824	-113092128476.307\\
0.707017675441886	-113054886219.624\\
0.707117677941949	-113017643962.94\\
0.707217680442011	-112980401706.257\\
0.707317682942074	-112942586491.778\\
0.707417685442136	-112905344235.095\\
0.707517687942199	-112868101978.411\\
0.707617690442261	-112830859721.728\\
0.707717692942324	-112793617465.044\\
0.707817695442386	-112756375208.361\\
0.707917697942449	-112719132951.677\\
0.708017700442511	-112681890694.994\\
0.708117702942574	-112644648438.31\\
0.708217705442636	-112607406181.627\\
0.708317707942699	-112570163924.943\\
0.708417710442761	-112532921668.26\\
0.708517712942824	-112495679411.576\\
0.708617715442886	-112458437154.893\\
0.708717717942949	-112421194898.209\\
0.708817720443011	-112383952641.526\\
0.708917722943074	-112346710384.842\\
0.709017725443136	-112309468128.159\\
0.709117727943199	-112272225871.475\\
0.709217730443261	-112234983614.792\\
0.709317732943324	-112197741358.108\\
0.709417735443386	-112160499101.425\\
0.709517737943449	-112123256844.741\\
0.709617740443511	-112086587545.853\\
0.709717742943574	-112049345289.169\\
0.709817745443636	-112012103032.486\\
0.709917747943699	-111974860775.802\\
0.710017750443761	-111937618519.119\\
0.710117752943824	-111900376262.435\\
0.710217755443886	-111863706963.547\\
0.710317757943949	-111826464706.863\\
0.710417760444011	-111789222450.18\\
0.710517762944074	-111751980193.496\\
0.710617765444136	-111714737936.813\\
0.710717767944199	-111678068637.924\\
0.710817770444261	-111640826381.241\\
0.710917772944324	-111603584124.557\\
0.711017775444386	-111566341867.874\\
0.711117777944449	-111529672568.986\\
0.711217780444511	-111492430312.302\\
0.711317782944574	-111455188055.619\\
0.711417785444636	-111418518756.73\\
0.711517787944699	-111381276500.047\\
0.711617790444761	-111344034243.363\\
0.711717792944824	-111307364944.475\\
0.711817795444886	-111270122687.791\\
0.711917797944949	-111232880431.108\\
0.712017800445011	-111196211132.219\\
0.712117802945074	-111158968875.536\\
0.712217805445136	-111121726618.852\\
0.712317807945199	-111085057319.964\\
0.712417810445261	-111047815063.281\\
0.712517812945324	-111011145764.392\\
0.712617815445386	-110973903507.709\\
0.712717817945449	-110937234208.82\\
0.712817820445511	-110899991952.137\\
0.712917822945574	-110863322653.248\\
0.713017825445636	-110826080396.565\\
0.713117827945699	-110789411097.676\\
0.713217830445761	-110752168840.993\\
0.713317832945824	-110715499542.105\\
0.713417835445886	-110678257285.421\\
0.713517837945949	-110641587986.533\\
0.713617840446011	-110604345729.849\\
0.713717842946074	-110567676430.961\\
0.713817845446136	-110530434174.277\\
0.713917847946199	-110493764875.389\\
0.714017850446261	-110456522618.705\\
0.714117852946324	-110419853319.817\\
0.714217855446386	-110383184020.929\\
0.714317857946449	-110345941764.245\\
0.714417860446511	-110309272465.357\\
0.714517862946574	-110272030208.673\\
0.714617865446636	-110235360909.785\\
0.714717867946699	-110198691610.897\\
0.714817870446761	-110161449354.213\\
0.714917872946824	-110124780055.325\\
0.715017875446886	-110088110756.436\\
0.715117877946949	-110050868499.753\\
0.715217880447011	-110014199200.865\\
0.715317882947074	-109977529901.976\\
0.715417885447136	-109940860603.088\\
0.715517887947199	-109903618346.404\\
0.715617890447261	-109866949047.516\\
0.715717892947324	-109830279748.628\\
0.715817895447386	-109793610449.739\\
0.715917897947449	-109756368193.056\\
0.716017900447511	-109719698894.167\\
0.716117902947574	-109683029595.279\\
0.716217905447636	-109646360296.391\\
0.716317907947699	-109609690997.502\\
0.716417910447761	-109573021698.614\\
0.716517912947824	-109535779441.93\\
0.716617915447886	-109499110143.042\\
0.716717917947949	-109462440844.154\\
0.716817920448011	-109425771545.265\\
0.716917922948074	-109389102246.377\\
0.717017925448136	-109352432947.488\\
0.717117927948199	-109315763648.6\\
0.717217930448261	-109279094349.712\\
0.717317932948324	-109242425050.823\\
0.717417935448386	-109205182794.14\\
0.717517937948449	-109168513495.251\\
0.717617940448511	-109131844196.363\\
0.717717942948574	-109095174897.475\\
0.717817945448636	-109058505598.586\\
0.717917947948699	-109021836299.698\\
0.718017950448761	-108985167000.81\\
0.718117952948824	-108948497701.921\\
0.718217955448886	-108911828403.033\\
0.718317957948949	-108875159104.144\\
0.718417960449011	-108838489805.256\\
0.718517962949074	-108801820506.368\\
0.718617965449136	-108765724165.274\\
0.718717967949199	-108729054866.386\\
0.718817970449261	-108692385567.498\\
0.718917972949324	-108655716268.609\\
0.719017975449386	-108619046969.721\\
0.719117977949449	-108582377670.833\\
0.719217980449511	-108545708371.944\\
0.719317982949574	-108509039073.056\\
0.719417985449636	-108472369774.167\\
0.719517987949699	-108436273433.074\\
0.719617990449761	-108399604134.186\\
0.719717992949824	-108362934835.297\\
0.719817995449886	-108326265536.409\\
0.719917997949949	-108289596237.521\\
0.720018000450011	-108253499896.427\\
0.720118002950074	-108216830597.539\\
0.720218005450136	-108180161298.651\\
0.720318007950199	-108143491999.762\\
0.720418010450261	-108107395658.669\\
0.720518012950324	-108070726359.781\\
0.720618015450386	-108034057060.892\\
0.720718017950449	-107997387762.004\\
0.720818020450511	-107961291420.911\\
0.720918022950574	-107924622122.022\\
0.721018025450636	-107887952823.134\\
0.721118027950699	-107851856482.041\\
0.721218030450761	-107815187183.152\\
0.721318032950824	-107778517884.264\\
0.721418035450886	-107742421543.171\\
0.721518037950949	-107705752244.282\\
0.721618040451011	-107669082945.394\\
0.721718042951074	-107632986604.301\\
0.721818045451136	-107596317305.412\\
0.721918047951199	-107560220964.319\\
0.722018050451261	-107523551665.431\\
0.722118052951324	-107487455324.338\\
0.722218055451386	-107450786025.449\\
0.722318057951449	-107414116726.561\\
0.722418060451511	-107378020385.468\\
0.722518062951574	-107341351086.579\\
0.722618065451636	-107305254745.486\\
0.722718067951699	-107268585446.598\\
0.722818070451761	-107232489105.504\\
0.722918072951824	-107195819806.616\\
0.723018075451886	-107159723465.523\\
0.723118077951949	-107123054166.634\\
0.723218080452011	-107086957825.541\\
0.723318082952074	-107050861484.448\\
0.723418085452136	-107014192185.56\\
0.723518087952199	-106978095844.466\\
0.723618090452261	-106941426545.578\\
0.723718092952324	-106905330204.485\\
0.723818095452386	-106869233863.391\\
0.723918097952449	-106832564564.503\\
0.724018100452511	-106796468223.41\\
0.724118102952574	-106759798924.521\\
0.724218105452636	-106723702583.428\\
0.724318107952699	-106687606242.335\\
0.724418110452761	-106650936943.447\\
0.724518112952824	-106614840602.353\\
0.724618115452886	-106578744261.26\\
0.724718117952949	-106542647920.167\\
0.724818120453011	-106505978621.278\\
0.724918122953074	-106469882280.185\\
0.725018125453136	-106433785939.092\\
0.725118127953199	-106397689597.999\\
0.725218130453261	-106361020299.11\\
0.725318132953324	-106324923958.017\\
0.725418135453386	-106288827616.924\\
0.725518137953449	-106252731275.831\\
0.725618140453511	-106216061976.942\\
0.725718142953574	-106179965635.849\\
0.725818145453636	-106143869294.756\\
0.725918147953699	-106107772953.663\\
0.726018150453761	-106071676612.569\\
0.726118152953824	-106035580271.476\\
0.726218155453886	-105999483930.383\\
0.726318157953949	-105962814631.494\\
0.726418160454011	-105926718290.401\\
0.726518162954074	-105890621949.308\\
0.726618165454136	-105854525608.215\\
0.726718167954199	-105818429267.121\\
0.726818170454261	-105782332926.028\\
0.726918172954324	-105746236584.935\\
0.727018175454386	-105710140243.842\\
0.727118177954449	-105674043902.749\\
0.727218180454511	-105637947561.655\\
0.727318182954574	-105601851220.562\\
0.727418185454636	-105565754879.469\\
0.727518187954699	-105529658538.376\\
0.727618190454761	-105493562197.282\\
0.727718192954824	-105457465856.189\\
0.727818195454886	-105421369515.096\\
0.727918197954949	-105385273174.003\\
0.728018200455011	-105349176832.909\\
0.728118202955074	-105313080491.816\\
0.728218205455136	-105276984150.723\\
0.728318207955199	-105240887809.63\\
0.728418210455261	-105204791468.536\\
0.728518212955324	-105169268085.238\\
0.728618215455386	-105133171744.145\\
0.728718217955449	-105097075403.052\\
0.728818220455511	-105060979061.959\\
0.728918222955574	-105024882720.865\\
0.729018225455636	-104988786379.772\\
0.729118227955699	-104952690038.679\\
0.729218230455761	-104917166655.381\\
0.729318232955824	-104881070314.287\\
0.729418235455886	-104844973973.194\\
0.729518237955949	-104808877632.101\\
0.729618240456011	-104773354248.803\\
0.729718242956074	-104737257907.71\\
0.729818245456136	-104701161566.616\\
0.729918247956199	-104665065225.523\\
0.730018250456261	-104629541842.225\\
0.730118252956324	-104593445501.132\\
0.730218255456386	-104557349160.039\\
0.730318257956449	-104521252818.945\\
0.730418260456511	-104485729435.647\\
0.730518262956574	-104449633094.554\\
0.730618265456636	-104413536753.461\\
0.730718267956699	-104378013370.163\\
0.730818270456761	-104341917029.069\\
0.730918272956824	-104306393645.771\\
0.731018275456886	-104270297304.678\\
0.731118277956949	-104234200963.585\\
0.731218280457011	-104198677580.287\\
0.731318282957074	-104162581239.193\\
0.731418285457136	-104127057855.895\\
0.731518287957199	-104090961514.802\\
0.731618290457261	-104054865173.709\\
0.731718292957324	-104019341790.411\\
0.731818295457386	-103983245449.317\\
0.731918297957449	-103947722066.019\\
0.732018300457511	-103911625724.926\\
0.732118302957574	-103876102341.628\\
0.732218305457636	-103840006000.535\\
0.732318307957699	-103804482617.237\\
0.732418310457761	-103768386276.143\\
0.732518312957824	-103732862892.845\\
0.732618315457886	-103697339509.547\\
0.732718317957949	-103661243168.454\\
0.732818320458011	-103625719785.156\\
0.732918322958074	-103589623444.063\\
0.733018325458136	-103554100060.764\\
0.733118327958199	-103518003719.671\\
0.733218330458261	-103482480336.373\\
0.733318332958324	-103446956953.075\\
0.733418335458386	-103410860611.982\\
0.733518337958449	-103375337228.684\\
0.733618340458511	-103339813845.386\\
0.733718342958574	-103303717504.292\\
0.733818345458636	-103268194120.994\\
0.733918347958699	-103232670737.696\\
0.734018350458762	-103196574396.603\\
0.734118352958824	-103161051013.305\\
0.734218355458886	-103125527630.007\\
0.734318357958949	-103090004246.708\\
0.734418360459011	-103053907905.615\\
0.734518362959074	-103018384522.317\\
0.734618365459137	-102982861139.019\\
0.734718367959199	-102947337755.721\\
0.734818370459261	-102911241414.628\\
0.734918372959324	-102875718031.33\\
0.735018375459387	-102840194648.031\\
0.735118377959449	-102804671264.733\\
0.735218380459511	-102769147881.435\\
0.735318382959574	-102733624498.137\\
0.735418385459636	-102697528157.044\\
0.735518387959699	-102662004773.746\\
0.735618390459762	-102626481390.448\\
0.735718392959824	-102590958007.15\\
0.735818395459886	-102555434623.851\\
0.735918397959949	-102519911240.553\\
0.736018400460012	-102484387857.255\\
0.736118402960074	-102448864473.957\\
0.736218405460137	-102413341090.659\\
0.736318407960199	-102377817707.361\\
0.736418410460261	-102342294324.063\\
0.736518412960324	-102306770940.765\\
0.736618415460387	-102271247557.467\\
0.736718417960449	-102235724174.168\\
0.736818420460511	-102200200790.87\\
0.736918422960574	-102164677407.572\\
0.737018425460637	-102129154024.274\\
0.737118427960699	-102093630640.976\\
0.737218430460762	-102058107257.678\\
0.737318432960824	-102022583874.38\\
0.737418435460886	-101987060491.082\\
0.737518437960949	-101951537107.784\\
0.737618440461012	-101916013724.485\\
0.737718442961074	-101880490341.187\\
0.737818445461137	-101844966957.889\\
0.737918447961199	-101809443574.591\\
0.738018450461262	-101774493149.088\\
0.738118452961324	-101738969765.79\\
0.738218455461387	-101703446382.492\\
0.738318457961449	-101667922999.194\\
0.738418460461511	-101632399615.896\\
0.738518462961574	-101596876232.598\\
0.738618465461637	-101561925807.095\\
0.738718467961699	-101526402423.796\\
0.738818470461762	-101490879040.498\\
0.738918472961824	-101455355657.2\\
0.739018475461887	-101420405231.697\\
0.739118477961949	-101384881848.399\\
0.739218480462012	-101349358465.101\\
0.739318482962074	-101313835081.803\\
0.739418485462137	-101278884656.3\\
0.739518487962199	-101243361273.002\\
0.739618490462262	-101207837889.704\\
0.739718492962324	-101172887464.201\\
0.739818495462387	-101137364080.903\\
0.739918497962449	-101101840697.605\\
0.740018500462512	-101066890272.102\\
0.740118502962574	-101031366888.803\\
0.740218505462637	-100995843505.505\\
0.740318507962699	-100960893080.002\\
0.740418510462762	-100925369696.704\\
0.740518512962824	-100890419271.201\\
0.740618515462887	-100854895887.903\\
0.740718517962949	-100819372504.605\\
0.740818520463012	-100784422079.102\\
0.740918522963074	-100748898695.804\\
0.741018525463137	-100713948270.301\\
0.741118527963199	-100678424887.003\\
0.741218530463262	-100643474461.5\\
0.741318532963324	-100607951078.202\\
0.741418535463387	-100573000652.699\\
0.741518537963449	-100537477269.401\\
0.741618540463512	-100502526843.898\\
0.741718542963574	-100467003460.6\\
0.741818545463637	-100432053035.097\\
0.741918547963699	-100396529651.799\\
0.742018550463762	-100361579226.296\\
0.742118552963824	-100326628800.793\\
0.742218555463887	-100291105417.494\\
0.742318557963949	-100256154991.991\\
0.742418560464012	-100220631608.693\\
0.742518562964074	-100185681183.19\\
0.742618565464137	-100150730757.687\\
0.742718567964199	-100115207374.389\\
0.742818570464262	-100080256948.886\\
0.742918572964324	-100045306523.383\\
0.743018575464387	-100009783140.085\\
0.743118577964449	-99974832714.5822\\
0.743218580464512	-99939882289.0792\\
0.743318582964574	-99904358905.7811\\
0.743418585464637	-99869408480.2781\\
0.743518587964699	-99834458054.7752\\
0.743618590464762	-99799507629.2722\\
0.743718592964824	-99763984245.9741\\
0.743818595464887	-99729033820.4711\\
0.743918597964949	-99694083394.9681\\
0.744018600465012	-99659132969.4651\\
0.744118602965074	-99623609586.167\\
0.744218605465137	-99588659160.664\\
0.744318607965199	-99553708735.1611\\
0.744418610465262	-99518758309.6581\\
0.744518612965324	-99483807884.1551\\
0.744618615465387	-99448857458.6521\\
0.744718617965449	-99413334075.354\\
0.744818620465512	-99378383649.851\\
0.744918622965574	-99343433224.3481\\
0.745018625465637	-99308482798.8451\\
0.745118627965699	-99273532373.3421\\
0.745218630465762	-99238581947.8391\\
0.745318632965824	-99203631522.3361\\
0.745418635465887	-99168681096.8331\\
0.745518637965949	-99133730671.3302\\
0.745618640466012	-99098780245.8272\\
0.745718642966074	-99063829820.3242\\
0.745818645466137	-99028879394.8212\\
0.745918647966199	-98993928969.3183\\
0.746018650466262	-98958978543.8153\\
0.746118652966324	-98924028118.3123\\
0.746218655466387	-98889077692.8093\\
0.746318657966449	-98854127267.3063\\
0.746418660466512	-98819176841.8033\\
0.746518662966574	-98784226416.3004\\
0.746618665466637	-98749275990.7974\\
0.746718667966699	-98714325565.2944\\
0.746818670466762	-98679375139.7914\\
0.746918672966824	-98644424714.2884\\
0.747018675466887	-98609474288.7855\\
0.747118677966949	-98574523863.2825\\
0.747218680467012	-98540146395.5746\\
0.747318682967074	-98505195970.0717\\
0.747418685467137	-98470245544.5687\\
0.747518687967199	-98435295119.0657\\
0.747618690467262	-98400344693.5627\\
0.747718692967324	-98365394268.0597\\
0.747818695467387	-98331016800.3519\\
0.747918697967449	-98296066374.8489\\
0.748018700467512	-98261115949.3459\\
0.748118702967574	-98226165523.8429\\
0.748218705467637	-98191788056.1351\\
0.748318707967699	-98156837630.6321\\
0.748418710467762	-98121887205.1291\\
0.748518712967824	-98086936779.6262\\
0.748618715467887	-98052559311.9183\\
0.748718717967949	-98017608886.4153\\
0.748818720468012	-97982658460.9123\\
0.748918722968074	-97948280993.2045\\
0.749018725468137	-97913330567.7015\\
0.749118727968199	-97878380142.1985\\
0.749218730468262	-97844002674.4907\\
0.749318732968324	-97809052248.9877\\
0.749418735468387	-97774101823.4847\\
0.749518737968449	-97739724355.7769\\
0.749618740468512	-97704773930.2739\\
0.749718742968574	-97670396462.566\\
0.749818745468637	-97635446037.0631\\
0.749918747968699	-97600495611.5601\\
0.750018750468762	-97566118143.8522\\
0.750118752968824	-97531167718.3493\\
0.750218755468887	-97496790250.6414\\
0.750318757968949	-97461839825.1384\\
0.750418760469012	-97427462357.4306\\
0.750518762969074	-97392511931.9276\\
0.750618765469137	-97358134464.2197\\
0.750718767969199	-97323184038.7168\\
0.750818770469262	-97288806571.0089\\
0.750918772969324	-97253856145.5059\\
0.751018775469387	-97219478677.7981\\
0.751118777969449	-97184528252.2951\\
0.751218780469512	-97150150784.5872\\
0.751318782969574	-97115773316.8794\\
0.751418785469637	-97080822891.3764\\
0.751518787969699	-97046445423.6686\\
0.751618790469762	-97011494998.1656\\
0.751718792969824	-96977117530.4577\\
0.751818795469887	-96942740062.7499\\
0.751918797969949	-96907789637.2469\\
0.752018800470012	-96873412169.5391\\
0.752118802970074	-96839034701.8312\\
0.752218805470137	-96804084276.3282\\
0.752318807970199	-96769706808.6204\\
0.752418810470262	-96735329340.9125\\
0.752518812970324	-96700378915.4096\\
0.752618815470387	-96666001447.7017\\
0.752718817970449	-96631623979.9939\\
0.752818820470512	-96597246512.286\\
0.752918822970574	-96562296086.783\\
0.753018825470637	-96527918619.0752\\
0.753118827970699	-96493541151.3673\\
0.753218830470762	-96459163683.6595\\
0.753318832970824	-96424786215.9516\\
0.753418835470887	-96389835790.4487\\
0.753518837970949	-96355458322.7408\\
0.753618840471012	-96321080855.033\\
0.753718842971074	-96286703387.3251\\
0.753818845471137	-96252325919.6173\\
0.753918847971199	-96217948451.9094\\
0.754018850471262	-96183570984.2016\\
0.754118852971324	-96148620558.6986\\
0.754218855471387	-96114243090.9907\\
0.754318857971449	-96079865623.2829\\
0.754418860471512	-96045488155.575\\
0.754518862971574	-96011110687.8672\\
0.754618865471637	-95976733220.1593\\
0.754718867971699	-95942355752.4515\\
0.754818870471762	-95907978284.7436\\
0.754918872971824	-95873600817.0358\\
0.755018875471887	-95839223349.3279\\
0.755118877971949	-95804845881.6201\\
0.755218880472012	-95770468413.9122\\
0.755318882972074	-95736090946.2044\\
0.755418885472137	-95701713478.4965\\
0.755518887972199	-95667336010.7887\\
0.755618890472262	-95632958543.0808\\
0.755718892972324	-95598581075.373\\
0.755818895472387	-95564203607.6651\\
0.755918897972449	-95529826139.9573\\
0.756018900472512	-95495448672.2494\\
0.756118902972574	-95461644162.3367\\
0.756218905472637	-95427266694.6289\\
0.756318907972699	-95392889226.921\\
0.756418910472762	-95358511759.2132\\
0.756518912972824	-95324134291.5053\\
0.756618915472887	-95289756823.7975\\
0.756718917972949	-95255379356.0896\\
0.756818920473012	-95221574846.1769\\
0.756918922973074	-95187197378.4691\\
0.757018925473137	-95152819910.7612\\
0.757118927973199	-95118442443.0534\\
0.757218930473262	-95084637933.1406\\
0.757318932973324	-95050260465.4328\\
0.757418935473387	-95015882997.7249\\
0.757518937973449	-94981505530.0171\\
0.757618940473512	-94947701020.1044\\
0.757718942973574	-94913323552.3965\\
0.757818945473637	-94878946084.6887\\
0.757918947973699	-94844568616.9808\\
0.758018950473762	-94810764107.0681\\
0.758118952973824	-94776386639.3603\\
0.758218955473887	-94742009171.6524\\
0.758318957973949	-94708204661.7397\\
0.758418960474012	-94673827194.0318\\
0.758518962974074	-94640022684.1191\\
0.758618965474137	-94605645216.4113\\
0.758718967974199	-94571267748.7034\\
0.758818970474262	-94537463238.7907\\
0.758918972974324	-94503085771.0829\\
0.759018975474387	-94469281261.1701\\
0.759118977974449	-94434903793.4623\\
0.759218980474512	-94400526325.7544\\
0.759318982974574	-94366721815.8417\\
0.759418985474637	-94332344348.1339\\
0.759518987974699	-94298539838.2211\\
0.759618990474762	-94264162370.5133\\
0.759718992974824	-94230357860.6006\\
0.759818995474887	-94195980392.8927\\
0.759918997974949	-94162175882.98\\
0.760019000475012	-94127798415.2722\\
0.760119002975074	-94093993905.3595\\
0.760219005475137	-94060189395.4467\\
0.760319007975199	-94025811927.7389\\
0.760419010475262	-93992007417.8262\\
0.760519012975324	-93957629950.1183\\
0.760619015475387	-93923825440.2056\\
0.760719017975449	-93890020930.2929\\
0.760819020475512	-93855643462.585\\
0.760919022975574	-93821838952.6723\\
0.761019025475637	-93788034442.7596\\
0.761119027975699	-93753656975.0517\\
0.761219030475762	-93719852465.139\\
0.761319032975824	-93686047955.2263\\
0.761419035475887	-93651670487.5184\\
0.761519037975949	-93617865977.6057\\
0.761619040476012	-93584061467.693\\
0.761719042976074	-93549683999.9852\\
0.761819045476137	-93515879490.0724\\
0.761919047976199	-93482074980.1597\\
0.762019050476262	-93448270470.247\\
0.762119052976324	-93413893002.5392\\
0.762219055476387	-93380088492.6264\\
0.762319057976449	-93346283982.7137\\
0.762419060476512	-93312479472.801\\
0.762519062976574	-93278674962.8883\\
0.762619065476637	-93244297495.1804\\
0.762719067976699	-93210492985.2677\\
0.762819070476762	-93176688475.355\\
0.762919072976824	-93142883965.4423\\
0.763019075476887	-93109079455.5296\\
0.763119077976949	-93075274945.6168\\
0.763219080477012	-93041470435.7041\\
0.763319082977074	-93007665925.7914\\
0.763419085477137	-92973288458.0836\\
0.763519087977199	-92939483948.1708\\
0.763619090477262	-92905679438.2581\\
0.763719092977324	-92871874928.3454\\
0.763819095477387	-92838070418.4327\\
0.763919097977449	-92804265908.52\\
0.764019100477512	-92770461398.6072\\
0.764119102977574	-92736656888.6945\\
0.764219105477637	-92702852378.7818\\
0.764319107977699	-92669047868.8691\\
0.764419110477762	-92635243358.9564\\
0.764519112977824	-92601438849.0437\\
0.764619115477887	-92567634339.1309\\
0.764719117977949	-92533829829.2182\\
0.764819120478012	-92500025319.3055\\
0.764919122978074	-92466793767.1879\\
0.765019125478137	-92432989257.2752\\
0.765119127978199	-92399184747.3625\\
0.765219130478262	-92365380237.4498\\
0.765319132978324	-92331575727.537\\
0.765419135478387	-92297771217.6243\\
0.76551913797845	-92263966707.7116\\
0.765619140478512	-92230162197.7989\\
0.765719142978574	-92196930645.6813\\
0.765819145478637	-92163126135.7686\\
0.765919147978699	-92129321625.8559\\
0.766019150478762	-92095517115.9431\\
0.766119152978824	-92061712606.0304\\
0.766219155478887	-92028481053.9128\\
0.766319157978949	-91994676544.0001\\
0.766419160479012	-91960872034.0874\\
0.766519162979075	-91927067524.1747\\
0.766619165479137	-91893835972.0571\\
0.766719167979199	-91860031462.1444\\
0.766819170479262	-91826226952.2316\\
0.766919172979324	-91792995400.1141\\
0.767019175479387	-91759190890.2013\\
0.76711917797945	-91725386380.2886\\
0.767219180479512	-91692154828.171\\
0.767319182979574	-91658350318.2583\\
0.767419185479637	-91624545808.3456\\
0.7675191879797	-91591314256.228\\
0.767619190479762	-91557509746.3153\\
0.767719192979824	-91523705236.4026\\
0.767819195479887	-91490473684.285\\
0.767919197979949	-91456669174.3723\\
0.768019200480012	-91423437622.2547\\
0.768119202980075	-91389633112.342\\
0.768219205480137	-91355828602.4292\\
0.768319207980199	-91322597050.3117\\
0.768419210480262	-91288792540.3989\\
0.768519212980325	-91255560988.2813\\
0.768619215480387	-91221756478.3686\\
0.76871921798045	-91188524926.251\\
0.768819220480512	-91154720416.3383\\
0.768919222980574	-91121488864.2207\\
0.769019225480637	-91087684354.308\\
0.7691192279807	-91054452802.1904\\
0.769219230480762	-91021221250.0728\\
0.769319232980824	-90987416740.1601\\
0.769419235480887	-90954185188.0425\\
0.76951923798095	-90920380678.1298\\
0.769619240481012	-90887149126.0122\\
0.769719242981075	-90853917573.8946\\
0.769819245481137	-90820113063.9819\\
0.769919247981199	-90786881511.8643\\
0.770019250481262	-90753077001.9516\\
0.770119252981325	-90719845449.834\\
0.770219255481387	-90686613897.7164\\
0.77031925798145	-90652809387.8037\\
0.770419260481512	-90619577835.6861\\
0.770519262981575	-90586346283.5685\\
0.770619265481637	-90553114731.451\\
0.7707192679817	-90519310221.5382\\
0.770819270481762	-90486078669.4207\\
0.770919272981824	-90452847117.3031\\
0.771019275481887	-90419615565.1855\\
0.77111927798195	-90385811055.2728\\
0.771219280482012	-90352579503.1552\\
0.771319282982075	-90319347951.0376\\
0.771419285482137	-90286116398.92\\
0.7715192879822	-90252884846.8024\\
0.771619290482262	-90219080336.8897\\
0.771719292982325	-90185848784.7721\\
0.771819295482387	-90152617232.6545\\
0.77191929798245	-90119385680.5369\\
0.772019300482512	-90086154128.4193\\
0.772119302982575	-90052922576.3017\\
0.772219305482637	-90019691024.1842\\
0.7723193079827	-89985886514.2714\\
0.772419310482762	-89952654962.1539\\
0.772519312982825	-89919423410.0363\\
0.772619315482887	-89886191857.9187\\
0.77271931798295	-89852960305.8011\\
0.772819320483012	-89819728753.6835\\
0.772919322983075	-89786497201.5659\\
0.773019325483137	-89753265649.4483\\
0.7731193279832	-89720034097.3307\\
0.773219330483262	-89686802545.2132\\
0.773319332983325	-89653570993.0956\\
0.773419335483387	-89620339440.978\\
0.77351933798345	-89587107888.8604\\
0.773619340483512	-89553876336.7428\\
0.773719342983575	-89520644784.6252\\
0.773819345483637	-89487413232.5076\\
0.7739193479837	-89454181680.39\\
0.774019350483762	-89421523086.0676\\
0.774119352983825	-89388291533.95\\
0.774219355483887	-89355059981.8324\\
0.77431935798395	-89321828429.7148\\
0.774419360484012	-89288596877.5972\\
0.774519362984075	-89255365325.4796\\
0.774619365484137	-89222133773.3621\\
0.7747193679842	-89189475179.0396\\
0.774819370484262	-89156243626.922\\
0.774919372984325	-89123012074.8044\\
0.775019375484387	-89089780522.6868\\
0.77511937798445	-89056548970.5692\\
0.775219380484512	-89023890376.2468\\
0.775319382984575	-88990658824.1292\\
0.775419385484637	-88957427272.0116\\
0.7755193879847	-88924195719.894\\
0.775619390484762	-88891537125.5716\\
0.775719392984825	-88858305573.454\\
0.775819395484887	-88825074021.3364\\
0.77591939798495	-88791842469.2188\\
0.776019400485012	-88759183874.8963\\
0.776119402985075	-88725952322.7788\\
0.776219405485137	-88692720770.6612\\
0.7763194079852	-88660062176.3387\\
0.776419410485262	-88626830624.2211\\
0.776519412985325	-88594172029.8987\\
0.776619415485387	-88560940477.7811\\
0.77671941798545	-88527708925.6635\\
0.776819420485512	-88495050331.341\\
0.776919422985575	-88461818779.2234\\
0.777019425485637	-88429160184.901\\
0.7771194279857	-88395928632.7834\\
0.777219430485762	-88362697080.6658\\
0.777319432985825	-88330038486.3434\\
0.777419435485887	-88296806934.2258\\
0.77751943798595	-88264148339.9033\\
0.777619440486012	-88230916787.7857\\
0.777719442986075	-88198258193.4633\\
0.777819445486137	-88165026641.3457\\
0.7779194479862	-88132368047.0232\\
0.778019450486262	-88099136494.9056\\
0.778119452986325	-88066477900.5832\\
0.778219455486387	-88033819306.2607\\
0.77831945798645	-88000587754.1431\\
0.778419460486512	-87967929159.8207\\
0.778519462986575	-87934697607.7031\\
0.778619465486637	-87902039013.3806\\
0.7787194679867	-87869380419.0582\\
0.778819470486762	-87836148866.9406\\
0.778919472986825	-87803490272.6181\\
0.779019475486887	-87770258720.5005\\
0.77911947798695	-87737600126.1781\\
0.779219480487012	-87704941531.8556\\
0.779319482987075	-87671709979.7381\\
0.779419485487137	-87639051385.4156\\
0.7795194879872	-87606392791.0931\\
0.779619490487262	-87573734196.7707\\
0.779719492987325	-87540502644.6531\\
0.779819495487387	-87507844050.3306\\
0.77991949798745	-87475185456.0082\\
0.780019500487512	-87442526861.6857\\
0.780119502987575	-87409295309.5681\\
0.780219505487637	-87376636715.2457\\
0.7803195079877	-87343978120.9232\\
0.780419510487762	-87311319526.6008\\
0.780519512987825	-87278660932.2783\\
0.780619515487887	-87245429380.1607\\
0.78071951798795	-87212770785.8383\\
0.780819520488012	-87180112191.5158\\
0.780919522988075	-87147453597.1933\\
0.781019525488137	-87114795002.8709\\
0.7811195279882	-87082136408.5484\\
0.781219530488262	-87049477814.226\\
0.781319532988325	-87016819219.9035\\
0.781419535488387	-86983587667.7859\\
0.78151953798845	-86950929073.4635\\
0.781619540488512	-86918270479.141\\
0.781719542988575	-86885611884.8186\\
0.781819545488637	-86852953290.4961\\
0.7819195479887	-86820294696.1736\\
0.782019550488762	-86787636101.8512\\
0.782119552988825	-86754977507.5287\\
0.782219555488887	-86722318913.2063\\
0.78231955798895	-86689660318.8838\\
0.782419560489012	-86657001724.5614\\
0.782519562989075	-86624343130.2389\\
0.782619565489137	-86591684535.9164\\
0.7827195679892	-86559025941.594\\
0.782819570489262	-86526940305.0667\\
0.782919572989325	-86494281710.7442\\
0.783019575489387	-86461623116.4218\\
0.78311957798945	-86428964522.0993\\
0.783219580489512	-86396305927.7768\\
0.783319582989575	-86363647333.4544\\
0.783419585489637	-86330988739.1319\\
0.7835195879897	-86298330144.8095\\
0.783619590489762	-86266244508.2821\\
0.783719592989825	-86233585913.9597\\
0.783819595489887	-86200927319.6372\\
0.78391959798995	-86168268725.3148\\
0.784019600490012	-86135610130.9923\\
0.784119602990075	-86103524494.465\\
0.784219605490137	-86070865900.1425\\
0.7843196079902	-86038207305.8201\\
0.784419610490262	-86005548711.4976\\
0.784519612990325	-85973463074.9703\\
0.784619615490387	-85940804480.6478\\
0.78471961799045	-85908145886.3254\\
0.784819620490512	-85875487292.0029\\
0.784919622990575	-85843401655.4756\\
0.785019625490637	-85810743061.1531\\
0.7851196279907	-85778084466.8307\\
0.785219630490762	-85745998830.3033\\
0.785319632990825	-85713340235.9809\\
0.785419635490887	-85680681641.6584\\
0.78551963799095	-85648596005.1311\\
0.785619640491012	-85615937410.8087\\
0.785719642991075	-85583851774.2813\\
0.785819645491137	-85551193179.9589\\
0.7859196479912	-85518534585.6364\\
0.786019650491262	-85486448949.1091\\
0.786119652991325	-85453790354.7866\\
0.786219655491387	-85421704718.2593\\
0.78631965799145	-85389046123.9368\\
0.786419660491512	-85356960487.4095\\
0.786519662991575	-85324301893.0871\\
0.786619665491637	-85292216256.5597\\
0.7867196679917	-85259557662.2373\\
0.786819670491762	-85227472025.71\\
0.786919672991825	-85194813431.3875\\
0.787019675491887	-85162727794.8602\\
0.78711967799195	-85130069200.5377\\
0.787219680492012	-85097983564.0104\\
0.787319682992075	-85065897927.4831\\
0.787419685492137	-85033239333.1606\\
0.7875196879922	-85001153696.6333\\
0.787619690492262	-84968495102.3108\\
0.787719692992325	-84936409465.7835\\
0.787819695492387	-84904323829.2562\\
0.78791969799245	-84871665234.9337\\
0.788019700492512	-84839579598.4064\\
0.788119702992575	-84807493961.8791\\
0.788219705492637	-84774835367.5566\\
0.7883197079927	-84742749731.0293\\
0.788419710492762	-84710664094.502\\
0.788519712992825	-84678005500.1795\\
0.788619715492887	-84645919863.6522\\
0.78871971799295	-84613834227.1248\\
0.788819720493012	-84581748590.5975\\
0.788919722993075	-84549089996.2751\\
0.789019725493137	-84517004359.7477\\
0.7891197279932	-84484918723.2204\\
0.789219730493262	-84452833086.6931\\
0.789319732993325	-84420747450.1658\\
0.789419735493387	-84388088855.8433\\
0.78951973799345	-84356003219.316\\
0.789619740493512	-84323917582.7887\\
0.789719742993575	-84291831946.2613\\
0.789819745493637	-84259746309.734\\
0.7899197479937	-84227660673.2067\\
0.790019750493762	-84195575036.6793\\
0.790119752993825	-84162916442.3569\\
0.790219755493887	-84130830805.8296\\
0.79031975799395	-84098745169.3022\\
0.790419760494012	-84066659532.7749\\
0.790519762994075	-84034573896.2476\\
0.790619765494137	-84002488259.7203\\
0.7907197679942	-83970402623.1929\\
0.790819770494262	-83938316986.6656\\
0.790919772994325	-83906231350.1383\\
0.791019775494387	-83874145713.6109\\
0.79111977799445	-83842060077.0836\\
0.791219780494512	-83809974440.5563\\
0.791319782994575	-83777888804.029\\
0.791419785494637	-83745803167.5016\\
0.7915197879947	-83713717530.9743\\
0.791619790494762	-83681631894.447\\
0.791719792994825	-83649546257.9197\\
0.791819795494887	-83617460621.3923\\
0.79191979799495	-83585947942.6601\\
0.792019800495012	-83553862306.1328\\
0.792119802995075	-83521776669.6055\\
0.792219805495137	-83489691033.0782\\
0.7923198079952	-83457605396.5508\\
0.792419810495262	-83425519760.0235\\
0.792519812995325	-83393434123.4962\\
0.792619815495387	-83361921444.764\\
0.79271981799545	-83329835808.2367\\
0.792819820495512	-83297750171.7094\\
0.792919822995575	-83265664535.182\\
0.793019825495637	-83233578898.6547\\
0.7931198279957	-83202066219.9225\\
0.793219830495762	-83169980583.3952\\
0.793319832995825	-83137894946.8678\\
0.793419835495887	-83105809310.3405\\
0.79351983799595	-83074296631.6083\\
0.793619840496012	-83042210995.081\\
0.793719842996075	-83010125358.5537\\
0.793819845496137	-82978612679.8215\\
0.7939198479962	-82946527043.2941\\
0.794019850496262	-82914441406.7668\\
0.794119852996325	-82882928728.0346\\
0.794219855496387	-82850843091.5073\\
0.79431985799645	-82818757454.98\\
0.794419860496512	-82787244776.2478\\
0.794519862996575	-82755159139.7205\\
0.794619865496637	-82723073503.1931\\
0.7947198679967	-82691560824.4609\\
0.794819870496762	-82659475187.9336\\
0.794919872996825	-82627962509.2014\\
0.795019875496887	-82595876872.6741\\
0.79511987799695	-82564364193.9419\\
0.795219880497012	-82532278557.4146\\
0.795319882997075	-82500765878.6824\\
0.795419885497137	-82468680242.155\\
0.7955198879972	-82437167563.4229\\
0.795619890497262	-82405081926.8955\\
0.795719892997325	-82373569248.1633\\
0.795819895497387	-82341483611.636\\
0.79591989799745	-82309970932.9038\\
0.796019900497512	-82277885296.3765\\
0.796119902997575	-82246372617.6443\\
0.796219905497637	-82214286981.117\\
0.7963199079977	-82182774302.3848\\
0.796419910497762	-82151261623.6526\\
0.796519912997825	-82119175987.1252\\
0.796619915497887	-82087663308.3931\\
0.79671991799795	-82056150629.6608\\
0.796819920498012	-82024064993.1335\\
0.796919922998075	-81992552314.4013\\
0.797019925498137	-81961039635.6691\\
0.7971199279982	-81928953999.1418\\
0.797219930498262	-81897441320.4096\\
0.797319932998325	-81865928641.6774\\
0.797419935498387	-81833843005.1501\\
0.79751993799845	-81802330326.4179\\
0.797619940498512	-81770817647.6857\\
0.797719942998575	-81739304968.9535\\
0.797819945498637	-81707219332.4262\\
0.7979199479987	-81675706653.694\\
0.798019950498763	-81644193974.9618\\
0.798119952998825	-81612681296.2296\\
0.798219955498887	-81580595659.7023\\
0.79831995799895	-81549082980.9701\\
0.798419960499012	-81517570302.2379\\
0.798519962999075	-81486057623.5057\\
0.798619965499137	-81454544944.7735\\
0.7987199679992	-81423032266.0413\\
0.798819970499262	-81391519587.3091\\
0.798919972999325	-81359433950.7818\\
0.799019975499388	-81327921272.0496\\
0.79911997799945	-81296408593.3174\\
0.799219980499512	-81264895914.5852\\
0.799319982999575	-81233383235.853\\
0.799419985499637	-81201870557.1208\\
0.7995199879997	-81170357878.3886\\
0.799619990499763	-81138845199.6564\\
0.799719992999825	-81107332520.9242\\
0.799819995499887	-81075819842.192\\
0.79991999799995	-81044307163.4598\\
0.800020000500013	-81012794484.7276\\
};
\addplot [color=mycolor3,solid,forget plot]
  table[row sep=crcr]{%
0.800020000500013	-81012794484.7276\\
0.800120003000075	-80981281805.9954\\
0.800220005500137	-80949769127.2632\\
0.8003200080002	-80918256448.531\\
0.800420010500262	-80886743769.7988\\
0.800520013000325	-80855231091.0667\\
0.800620015500388	-80823718412.3344\\
0.80072001800045	-80792205733.6022\\
0.800820020500512	-80760693054.8701\\
0.800920023000575	-80729753333.933\\
0.801020025500638	-80698240655.2008\\
0.8011200280007	-80666727976.4686\\
0.801220030500763	-80635215297.7364\\
0.801320033000825	-80603702619.0042\\
0.801420035500887	-80572189940.272\\
0.80152003800095	-80540677261.5398\\
0.801620040501013	-80509737540.6028\\
0.801720043001075	-80478224861.8706\\
0.801820045501138	-80446712183.1384\\
0.8019200480012	-80415199504.4062\\
0.802020050501263	-80384259783.4691\\
0.802120053001325	-80352747104.7369\\
0.802220055501388	-80321234426.0047\\
0.80232005800145	-80289721747.2725\\
0.802420060501512	-80258782026.3354\\
0.802520063001575	-80227269347.6033\\
0.802620065501638	-80195756668.8711\\
0.8027200680017	-80164816947.934\\
0.802820070501763	-80133304269.2018\\
0.802920073001825	-80101791590.4696\\
0.803020075501888	-80070851869.5325\\
0.80312007800195	-80039339190.8004\\
0.803220080502013	-80007826512.0682\\
0.803320083002075	-79976886791.1311\\
0.803420085502138	-79945374112.3989\\
0.8035200880022	-79913861433.6667\\
0.803620090502263	-79882921712.7296\\
0.803720093002325	-79851409033.9974\\
0.803820095502388	-79820469313.0604\\
0.80392009800245	-79788956634.3282\\
0.804020100502513	-79758016913.3911\\
0.804120103002575	-79726504234.6589\\
0.804220105502638	-79695564513.7219\\
0.8043201080027	-79664051834.9897\\
0.804420110502763	-79633112114.0526\\
0.804520113002825	-79601599435.3204\\
0.804620115502888	-79570659714.3833\\
0.80472011800295	-79539147035.6511\\
0.804820120503013	-79508207314.7141\\
0.804920123003075	-79476694635.9819\\
0.805020125503138	-79445754915.0448\\
0.8051201280032	-79414815194.1078\\
0.805220130503263	-79383302515.3756\\
0.805320133003325	-79352362794.4385\\
0.805420135503388	-79320850115.7063\\
0.80552013800345	-79289910394.7692\\
0.805620140503513	-79258970673.8322\\
0.805720143003575	-79227457995.1\\
0.805820145503638	-79196518274.1629\\
0.8059201480037	-79165578553.2258\\
0.806020150503763	-79134065874.4937\\
0.806120153003825	-79103126153.5566\\
0.806220155503888	-79072186432.6195\\
0.80632015800395	-79040673753.8873\\
0.806420160504013	-79009734032.9503\\
0.806520163004075	-78978794312.0132\\
0.806620165504138	-78947854591.0761\\
0.8067201680042	-78916341912.3439\\
0.806820170504263	-78885402191.4069\\
0.806920173004325	-78854462470.4698\\
0.807020175504388	-78823522749.5327\\
0.80712017800445	-78792583028.5957\\
0.807220180504513	-78761643307.6586\\
0.807320183004575	-78730130628.9264\\
0.807420185504638	-78699190907.9893\\
0.8075201880047	-78668251187.0523\\
0.807620190504763	-78637311466.1152\\
0.807720193004825	-78606371745.1782\\
0.807820195504888	-78575432024.2411\\
0.80792019800495	-78544492303.304\\
0.808020200505013	-78513552582.367\\
0.808120203005075	-78482039903.6348\\
0.808220205505138	-78451100182.6977\\
0.8083202080052	-78420160461.7607\\
0.808420210505263	-78389220740.8236\\
0.808520213005325	-78358281019.8865\\
0.808620215505388	-78327341298.9494\\
0.80872021800545	-78296401578.0124\\
0.808820220505513	-78265461857.0753\\
0.808920223005575	-78234522136.1383\\
0.809020225505638	-78203582415.2012\\
0.8091202280057	-78172642694.2641\\
0.809220230505763	-78141702973.3271\\
0.809320233005825	-78110763252.39\\
0.809420235505888	-78080396489.2481\\
0.80952023800595	-78049456768.311\\
0.809620240506013	-78018517047.3739\\
0.809720243006075	-77987577326.4369\\
0.809820245506138	-77956637605.4998\\
0.8099202480062	-77925697884.5627\\
0.810020250506263	-77894758163.6257\\
0.810120253006325	-77863818442.6886\\
0.810220255506388	-77833451679.5467\\
0.81032025800645	-77802511958.6096\\
0.810420260506513	-77771572237.6725\\
0.810520263006575	-77740632516.7355\\
0.810620265506638	-77709692795.7984\\
0.8107202680067	-77679326032.6565\\
0.810820270506763	-77648386311.7194\\
0.810920273006825	-77617446590.7824\\
0.811020275506888	-77586506869.8453\\
0.81112027800695	-77556140106.7034\\
0.811220280507013	-77525200385.7663\\
0.811320283007075	-77494260664.8292\\
0.811420285507138	-77463320943.8922\\
0.8115202880072	-77432954180.7502\\
0.811620290507263	-77402014459.8132\\
0.811720293007325	-77371074738.8761\\
0.811820295507388	-77340707975.7342\\
0.81192029800745	-77309768254.7971\\
0.812020300507513	-77278828533.86\\
0.812120303007575	-77248461770.7181\\
0.812220305507638	-77217522049.7811\\
0.8123203080077	-77187155286.6391\\
0.812420310507763	-77156215565.702\\
0.812520313007825	-77125275844.765\\
0.812620315507888	-77094909081.623\\
0.81272031800795	-77063969360.686\\
0.812820320508013	-77033602597.5441\\
0.812920323008075	-77002662876.607\\
0.813020325508138	-76972296113.4651\\
0.8131203280082	-76941356392.528\\
0.813220330508263	-76910989629.3861\\
0.813320333008325	-76880049908.449\\
0.813420335508388	-76849683145.3071\\
0.81352033800845	-76818743424.37\\
0.813620340508513	-76788376661.2281\\
0.813720343008575	-76757436940.291\\
0.813820345508638	-76727070177.1491\\
0.8139203480087	-76696703414.0071\\
0.814020350508763	-76665763693.0701\\
0.814120353008825	-76635396929.9281\\
0.814220355508888	-76604457208.9911\\
0.81432035800895	-76574090445.8491\\
0.814420360509013	-76543723682.7072\\
0.814520363009075	-76512783961.7701\\
0.814620365509138	-76482417198.6282\\
0.8147203680092	-76452050435.4863\\
0.814820370509263	-76421110714.5492\\
0.814920373009325	-76390743951.4073\\
0.815020375509388	-76360377188.2653\\
0.81512037800945	-76330010425.1234\\
0.815220380509513	-76299070704.1863\\
0.815320383009575	-76268703941.0444\\
0.815420385509638	-76238337177.9025\\
0.8155203880097	-76207970414.7605\\
0.815620390509763	-76177030693.8235\\
0.815720393009825	-76146663930.6815\\
0.815820395509888	-76116297167.5396\\
0.81592039800995	-76085930404.3977\\
0.816020400510013	-76055563641.2557\\
0.816120403010075	-76024623920.3187\\
0.816220405510138	-75994257157.1767\\
0.8163204080102	-75963890394.0348\\
0.816420410510263	-75933523630.8929\\
0.816520413010325	-75903156867.7509\\
0.816620415510388	-75872790104.609\\
0.81672041801045	-75842423341.4671\\
0.816820420510513	-75812056578.3251\\
0.816920423010575	-75781116857.3881\\
0.817020425510638	-75750750094.2461\\
0.8171204280107	-75720383331.1042\\
0.817220430510763	-75690016567.9623\\
0.817320433010825	-75659649804.8203\\
0.817420435510888	-75629283041.6784\\
0.81752043801095	-75598916278.5365\\
0.817620440511013	-75568549515.3945\\
0.817720443011075	-75538182752.2526\\
0.817820445511138	-75507815989.1107\\
0.8179204480112	-75477449225.9687\\
0.818020450511263	-75447082462.8268\\
0.818120453011325	-75416715699.6849\\
0.818220455511388	-75386348936.5429\\
0.81832045801145	-75356555131.1961\\
0.818420460511513	-75326188368.0542\\
0.818520463011575	-75295821604.9123\\
0.818620465511638	-75265454841.7703\\
0.8187204680117	-75235088078.6284\\
0.818820470511763	-75204721315.4865\\
0.818920473011825	-75174354552.3445\\
0.819020475511888	-75143987789.2026\\
0.81912047801195	-75114193983.8558\\
0.819220480512013	-75083827220.7139\\
0.819320483012075	-75053460457.5719\\
0.819420485512138	-75023093694.43\\
0.8195204880122	-74992726931.2881\\
0.819620490512263	-74962933125.9413\\
0.819720493012325	-74932566362.7993\\
0.819820495512388	-74902199599.6574\\
0.81992049801245	-74871832836.5155\\
0.820020500512513	-74842039031.1687\\
0.820120503012575	-74811672268.0267\\
0.820220505512638	-74781305504.8848\\
0.8203205080127	-74751511699.538\\
0.820420510512763	-74721144936.3961\\
0.820520513012825	-74690778173.2541\\
0.820620515512888	-74660984367.9073\\
0.82072051801295	-74630617604.7654\\
0.820820520513013	-74600250841.6234\\
0.820920523013075	-74570457036.2766\\
0.821020525513138	-74540090273.1347\\
0.8211205280132	-74510296467.7879\\
0.821220530513263	-74479929704.646\\
0.821320533013325	-74449562941.504\\
0.821420535513388	-74419769136.1572\\
0.82152053801345	-74389402373.0153\\
0.821620540513513	-74359608567.6685\\
0.821720543013575	-74329241804.5266\\
0.821820545513638	-74299447999.1798\\
0.8219205480137	-74269081236.0378\\
0.822020550513763	-74239287430.691\\
0.822120553013825	-74208920667.5491\\
0.822220555513888	-74179126862.2023\\
0.82232055801395	-74148760099.0604\\
0.822420560514013	-74118966293.7136\\
0.822520563014075	-74088599530.5716\\
0.822620565514138	-74058805725.2248\\
0.8227205680142	-74028438962.0829\\
0.822820570514263	-73998645156.7361\\
0.822920573014325	-73968851351.3893\\
0.823020575514388	-73938484588.2473\\
0.82312057801445	-73908690782.9005\\
0.823220580514513	-73878896977.5537\\
0.823320583014575	-73848530214.4118\\
0.823420585514638	-73818736409.065\\
0.8235205880147	-73788942603.7182\\
0.823620590514763	-73758575840.5763\\
0.823720593014825	-73728782035.2295\\
0.823820595514888	-73698988229.8827\\
0.82392059801495	-73668621466.7407\\
0.824020600515013	-73638827661.3939\\
0.824120603015075	-73609033856.0471\\
0.824220605515138	-73579240050.7003\\
0.8243206080152	-73548873287.5584\\
0.824420610515263	-73519079482.2116\\
0.824520613015325	-73489285676.8648\\
0.824620615515388	-73459491871.518\\
0.82472061801545	-73429698066.1712\\
0.824820620515513	-73399331303.0292\\
0.824920623015575	-73369537497.6824\\
0.825020625515638	-73339743692.3356\\
0.8251206280157	-73309949886.9888\\
0.825220630515763	-73280156081.642\\
0.825320633015825	-73250362276.2952\\
0.825420635515888	-73220568470.9484\\
0.82552063801595	-73190201707.8065\\
0.825620640516013	-73160407902.4597\\
0.825720643016075	-73130614097.1129\\
0.825820645516138	-73100820291.7661\\
0.8259206480162	-73071026486.4193\\
0.826020650516263	-73041232681.0725\\
0.826120653016325	-73011438875.7257\\
0.826220655516388	-72981645070.3789\\
0.82632065801645	-72951851265.0321\\
0.826420660516513	-72922057459.6853\\
0.826520663016575	-72892263654.3385\\
0.826620665516638	-72862469848.9917\\
0.8267206680167	-72832676043.6449\\
0.826820670516763	-72802882238.298\\
0.826920673016825	-72773088432.9512\\
0.827020675516888	-72743294627.6044\\
0.82712067801695	-72713500822.2576\\
0.827220680517013	-72683707016.9108\\
0.827320683017075	-72654486169.3592\\
0.827420685517138	-72624692364.0124\\
0.8275206880172	-72594898558.6656\\
0.827620690517263	-72565104753.3188\\
0.827720693017325	-72535310947.972\\
0.827820695517388	-72505517142.6252\\
0.82792069801745	-72475723337.2784\\
0.828020700517513	-72446502489.7267\\
0.828120703017575	-72416708684.3799\\
0.828220705517638	-72386914879.0331\\
0.8283207080177	-72357121073.6863\\
0.828420710517763	-72327327268.3395\\
0.828520713017825	-72298106420.7878\\
0.828620715517888	-72268312615.441\\
0.82872071801795	-72238518810.0942\\
0.828820720518013	-72208725004.7474\\
0.828920723018075	-72179504157.1957\\
0.829020725518138	-72149710351.8489\\
0.8291207280182	-72119916546.5021\\
0.829220730518263	-72090695698.9504\\
0.829320733018325	-72060901893.6036\\
0.829420735518388	-72031108088.2568\\
0.82952073801845	-72001887240.7052\\
0.829620740518513	-71972093435.3584\\
0.829720743018575	-71942299630.0116\\
0.829820745518638	-71913078782.4599\\
0.8299207480187	-71883284977.1131\\
0.830020750518763	-71854064129.5614\\
0.830120753018825	-71824270324.2146\\
0.830220755518888	-71795049476.6629\\
0.83032075801895	-71765255671.3161\\
0.830420760519013	-71735461865.9693\\
0.830520763019076	-71706241018.4177\\
0.830620765519138	-71676447213.0709\\
0.8307207680192	-71647226365.5192\\
0.830820770519263	-71617432560.1724\\
0.830920773019325	-71588211712.6207\\
0.831020775519388	-71558417907.2739\\
0.831120778019451	-71529197059.7222\\
0.831220780519513	-71499976212.1706\\
0.831320783019575	-71470182406.8238\\
0.831420785519638	-71440961559.2721\\
0.831520788019701	-71411167753.9253\\
0.831620790519763	-71381946906.3736\\
0.831720793019825	-71352153101.0268\\
0.831820795519888	-71322932253.4751\\
0.83192079801995	-71293711405.9235\\
0.832020800520013	-71263917600.5767\\
0.832120803020076	-71234696753.025\\
0.832220805520138	-71205475905.4733\\
0.8323208080202	-71175682100.1265\\
0.832420810520263	-71146461252.5748\\
0.832520813020326	-71117240405.0232\\
0.832620815520388	-71087446599.6764\\
0.832720818020451	-71058225752.1247\\
0.832820820520513	-71029004904.573\\
0.832920823020575	-70999784057.0213\\
0.833020825520638	-70969990251.6745\\
0.833120828020701	-70940769404.1229\\
0.833220830520763	-70911548556.5712\\
0.833320833020825	-70882327709.0195\\
0.833420835520888	-70852533903.6727\\
0.833520838020951	-70823313056.1211\\
0.833620840521013	-70794092208.5694\\
0.833720843021076	-70764871361.0177\\
0.833820845521138	-70735650513.466\\
0.8339208480212	-70706429665.9144\\
0.834020850521263	-70676635860.5676\\
0.834120853021326	-70647415013.0159\\
0.834220855521388	-70618194165.4642\\
0.834320858021451	-70588973317.9126\\
0.834420860521513	-70559752470.3609\\
0.834520863021576	-70530531622.8092\\
0.834620865521638	-70501310775.2575\\
0.834720868021701	-70472089927.7059\\
0.834820870521763	-70442869080.1542\\
0.834920873021825	-70413648232.6025\\
0.835020875521888	-70384427385.0508\\
0.835120878021951	-70355206537.4992\\
0.835220880522013	-70325985689.9475\\
0.835320883022076	-70296764842.3958\\
0.835420885522138	-70267543994.8442\\
0.835520888022201	-70238323147.2925\\
0.835620890522263	-70209102299.7408\\
0.835720893022326	-70179881452.1891\\
0.835820895522388	-70150660604.6375\\
0.835920898022451	-70121439757.0858\\
0.836020900522513	-70092218909.5341\\
0.836120903022576	-70062998061.9825\\
0.836220905522638	-70033777214.4308\\
0.836320908022701	-70005129324.6742\\
0.836420910522763	-69975908477.1226\\
0.836520913022826	-69946687629.5709\\
0.836620915522888	-69917466782.0192\\
0.836720918022951	-69888245934.4676\\
0.836820920523013	-69859025086.9159\\
0.836920923023076	-69830377197.1593\\
0.837020925523138	-69801156349.6077\\
0.837120928023201	-69771935502.056\\
0.837220930523263	-69742714654.5043\\
0.837320933023326	-69713493806.9527\\
0.837420935523388	-69684845917.1961\\
0.837520938023451	-69655625069.6444\\
0.837620940523513	-69626404222.0928\\
0.837720943023576	-69597756332.3362\\
0.837820945523638	-69568535484.7846\\
0.837920948023701	-69539314637.2329\\
0.838020950523763	-69510093789.6812\\
0.838120953023826	-69481445899.9247\\
0.838220955523888	-69452225052.373\\
0.838320958023951	-69423004204.8213\\
0.838420960524013	-69394356315.0648\\
0.838520963024076	-69365135467.5131\\
0.838620965524138	-69336487577.7566\\
0.838720968024201	-69307266730.2049\\
0.838820970524263	-69278045882.6532\\
0.838920973024326	-69249397992.8967\\
0.839020975524388	-69220177145.345\\
0.839120978024451	-69191529255.5885\\
0.839220980524513	-69162308408.0368\\
0.839320983024576	-69133660518.2803\\
0.839420985524638	-69104439670.7286\\
0.839520988024701	-69075791780.972\\
0.839620990524763	-69046570933.4204\\
0.839720993024826	-69017923043.6638\\
0.839820995524888	-68988702196.1122\\
0.839920998024951	-68960054306.3556\\
0.840021000525013	-68930833458.804\\
0.840121003025076	-68902185569.0474\\
0.840221005525138	-68872964721.4957\\
0.840321008025201	-68844316831.7392\\
0.840421010525263	-68815668941.9827\\
0.840521013025326	-68786448094.431\\
0.840621015525388	-68757800204.6744\\
0.840721018025451	-68728579357.1228\\
0.840821020525513	-68699931467.3662\\
0.840921023025576	-68671283577.6097\\
0.841021025525638	-68642062730.058\\
0.841121028025701	-68613414840.3015\\
0.841221030525763	-68584766950.5449\\
0.841321033025826	-68555546102.9933\\
0.841421035525888	-68526898213.2367\\
0.841521038025951	-68498250323.4802\\
0.841621040526013	-68469602433.7236\\
0.841721043026076	-68440381586.172\\
0.841821045526138	-68411733696.4154\\
0.841921048026201	-68383085806.6589\\
0.842021050526263	-68354437916.9023\\
0.842121053026326	-68325217069.3507\\
0.842221055526388	-68296569179.5941\\
0.842321058026451	-68267921289.8376\\
0.842421060526513	-68239273400.081\\
0.842521063026576	-68210625510.3245\\
0.842621065526638	-68181977620.568\\
0.842721068026701	-68152756773.0163\\
0.842821070526763	-68124108883.2598\\
0.842921073026826	-68095460993.5032\\
0.843021075526888	-68066813103.7467\\
0.843121078026951	-68038165213.9901\\
0.843221080527013	-68009517324.2336\\
0.843321083027076	-67980869434.477\\
0.843421085527138	-67952221544.7205\\
0.843521088027201	-67923573654.964\\
0.843621090527263	-67894925765.2074\\
0.843721093027326	-67866277875.4509\\
0.843821095527388	-67837629985.6943\\
0.843921098027451	-67808982095.9378\\
0.844021100527513	-67780334206.1813\\
0.844121103027576	-67751686316.4247\\
0.844221105527638	-67723038426.6682\\
0.844321108027701	-67694390536.9116\\
0.844421110527763	-67665742647.1551\\
0.844521113027826	-67637094757.3986\\
0.844621115527888	-67608446867.642\\
0.844721118027951	-67579798977.8855\\
0.844821120528013	-67551151088.1289\\
0.844921123028076	-67522503198.3724\\
0.845021125528138	-67493855308.6158\\
0.845121128028201	-67465207418.8593\\
0.845221130528263	-67436559529.1028\\
0.845321133028326	-67408484597.1413\\
0.845421135528388	-67379836707.3848\\
0.845521138028451	-67351188817.6283\\
0.845621140528513	-67322540927.8717\\
0.845721143028576	-67293893038.1152\\
0.845821145528638	-67265818106.1538\\
0.845921148028701	-67237170216.3972\\
0.846021150528763	-67208522326.6407\\
0.846121153028826	-67179874436.8842\\
0.846221155528888	-67151226547.1276\\
0.846321158028951	-67123151615.1662\\
0.846421160529013	-67094503725.4097\\
0.846521163029076	-67065855835.6531\\
0.846621165529138	-67037780903.6917\\
0.846721168029201	-67009133013.9352\\
0.846821170529263	-66980485124.1786\\
0.846921173029326	-66952410192.2172\\
0.847021175529388	-66923762302.4607\\
0.847121178029451	-66895114412.7041\\
0.847221180529513	-66867039480.7427\\
0.847321183029576	-66838391590.9862\\
0.847421185529638	-66809743701.2296\\
0.847521188029701	-66781668769.2682\\
0.847621190529763	-66753020879.5117\\
0.847721193029826	-66724945947.5503\\
0.847821195529888	-66696298057.7937\\
0.847921198029951	-66668223125.8323\\
0.848021200530013	-66639575236.0758\\
0.848121203030076	-66610927346.3192\\
0.848221205530138	-66582852414.3578\\
0.848321208030201	-66554204524.6013\\
0.848421210530263	-66526129592.6399\\
0.848521213030326	-66497481702.8833\\
0.848621215530388	-66469406770.9219\\
0.848721218030451	-66441331838.9605\\
0.848821220530513	-66412683949.204\\
0.848921223030576	-66384609017.2426\\
0.849021225530638	-66355961127.486\\
0.849121228030701	-66327886195.5246\\
0.849221230530763	-66299238305.7681\\
0.849321233030826	-66271163373.8067\\
0.849421235530888	-66243088441.8453\\
0.849521238030951	-66214440552.0887\\
0.849621240531013	-66186365620.1273\\
0.849721243031076	-66158290688.1659\\
0.849821245531138	-66129642798.4094\\
0.849921248031201	-66101567866.4479\\
0.850021250531263	-66073492934.4865\\
0.850121253031326	-66044845044.73\\
0.850221255531388	-66016770112.7686\\
0.850321258031451	-65988695180.8072\\
0.850421260531513	-65960047291.0506\\
0.850521263031576	-65931972359.0892\\
0.850621265531638	-65903897427.1278\\
0.850721268031701	-65875822495.1664\\
0.850821270531763	-65847174605.4099\\
0.850921273031826	-65819099673.4484\\
0.851021275531888	-65791024741.487\\
0.851121278031951	-65762949809.5256\\
0.851221280532013	-65734874877.5642\\
0.851321283032076	-65706226987.8077\\
0.851421285532138	-65678152055.8463\\
0.851521288032201	-65650077123.8849\\
0.851621290532263	-65622002191.9234\\
0.851721293032326	-65593927259.962\\
0.851821295532388	-65565852328.0006\\
0.851921298032451	-65537777396.0392\\
0.852021300532513	-65509702464.0778\\
0.852121303032576	-65481627532.1164\\
0.852221305532638	-65452979642.3599\\
0.852321308032701	-65424904710.3984\\
0.852421310532763	-65396829778.437\\
0.852521313032826	-65368754846.4756\\
0.852621315532888	-65340679914.5142\\
0.852721318032951	-65312604982.5528\\
0.852821320533013	-65284530050.5914\\
0.852921323033076	-65256455118.63\\
0.853021325533138	-65228380186.6686\\
0.853121328033201	-65200305254.7072\\
0.853221330533263	-65172230322.7458\\
0.853321333033326	-65144728348.5795\\
0.853421335533388	-65116653416.6181\\
0.853521338033451	-65088578484.6566\\
0.853621340533513	-65060503552.6952\\
0.853721343033576	-65032428620.7338\\
0.853821345533638	-65004353688.7724\\
0.853921348033701	-64976278756.811\\
0.854021350533763	-64948203824.8496\\
0.854121353033826	-64920128892.8882\\
0.854221355533888	-64892626918.7219\\
0.854321358033951	-64864551986.7605\\
0.854421360534013	-64836477054.7991\\
0.854521363034076	-64808402122.8377\\
0.854621365534138	-64780327190.8763\\
0.854721368034201	-64752825216.71\\
0.854821370534263	-64724750284.7486\\
0.854921373034326	-64696675352.7872\\
0.855021375534388	-64668600420.8258\\
0.855121378034451	-64641098446.6595\\
0.855221380534513	-64613023514.6981\\
0.855321383034576	-64584948582.7367\\
0.855421385534638	-64556873650.7752\\
0.855521388034701	-64529371676.609\\
0.855621390534763	-64501296744.6476\\
0.855721393034826	-64473221812.6861\\
0.855821395534888	-64445719838.5199\\
0.855921398034951	-64417644906.5585\\
0.856021400535013	-64389569974.597\\
0.856121403035076	-64362068000.4308\\
0.856221405535138	-64333993068.4694\\
0.856321408035201	-64306491094.3031\\
0.856421410535263	-64278416162.3417\\
0.856521413035326	-64250341230.3803\\
0.856621415535388	-64222839256.214\\
0.856721418035451	-64194764324.2526\\
0.856821420535513	-64167262350.0863\\
0.856921423035576	-64139187418.1249\\
0.857021425535638	-64111685443.9586\\
0.857121428035701	-64083610511.9972\\
0.857221430535763	-64056108537.8309\\
0.857321433035826	-64028033605.8695\\
0.857421435535888	-64000531631.7032\\
0.857521438035951	-63972456699.7418\\
0.857621440536013	-63944954725.5755\\
0.857721443036076	-63916879793.6141\\
0.857821445536138	-63889377819.4478\\
0.857921448036201	-63861875845.2816\\
0.858021450536263	-63833800913.3201\\
0.858121453036326	-63806298939.1539\\
0.858221455536388	-63778224007.1925\\
0.858321458036451	-63750722033.0262\\
0.858421460536513	-63723220058.8599\\
0.858521463036576	-63695145126.8985\\
0.858621465536638	-63667643152.7322\\
0.858721468036701	-63640141178.5659\\
0.858821470536763	-63612066246.6045\\
0.858921473036826	-63584564272.4382\\
0.859021475536888	-63557062298.272\\
0.859121478036951	-63528987366.3105\\
0.859221480537013	-63501485392.1443\\
0.859321483037076	-63473983417.978\\
0.859421485537138	-63446481443.8117\\
0.859521488037201	-63418406511.8503\\
0.859621490537263	-63390904537.684\\
0.859721493037326	-63363402563.5177\\
0.859821495537388	-63335900589.3515\\
0.859921498037451	-63308398615.1852\\
0.860021500537513	-63280323683.2238\\
0.860121503037576	-63252821709.0575\\
0.860221505537638	-63225319734.8912\\
0.860321508037701	-63197817760.7249\\
0.860421510537763	-63170315786.5587\\
0.860521513037826	-63142813812.3924\\
0.860621515537888	-63115311838.2261\\
0.860721518037951	-63087809864.0598\\
0.860821520538013	-63059734932.0984\\
0.860921523038076	-63032232957.9321\\
0.861021525538138	-63004730983.7658\\
0.861121528038201	-62977229009.5996\\
0.861221530538263	-62949727035.4333\\
0.861321533038326	-62922225061.267\\
0.861421535538388	-62894723087.1007\\
0.861521538038451	-62867221112.9344\\
0.861621540538513	-62839719138.7682\\
0.861721543038576	-62812217164.6019\\
0.861821545538638	-62784715190.4356\\
0.861921548038701	-62757213216.2693\\
0.862021550538764	-62729711242.1031\\
0.862121553038826	-62702209267.9368\\
0.862221555538888	-62674707293.7705\\
0.862321558038951	-62647205319.6042\\
0.862421560539013	-62620276303.2331\\
0.862521563039076	-62592774329.0668\\
0.862621565539138	-62565272354.9005\\
0.862721568039201	-62537770380.7342\\
0.862821570539263	-62510268406.5679\\
0.862921573039326	-62482766432.4017\\
0.863021575539389	-62455264458.2354\\
0.863121578039451	-62428335441.8642\\
0.863221580539513	-62400833467.698\\
0.863321583039576	-62373331493.5317\\
0.863421585539638	-62345829519.3654\\
0.863521588039701	-62318327545.1991\\
0.863621590539764	-62291398528.828\\
0.863721593039826	-62263896554.6617\\
0.863821595539888	-62236394580.4954\\
0.863921598039951	-62208892606.3291\\
0.864021600540014	-62181963589.958\\
0.864121603040076	-62154461615.7917\\
0.864221605540138	-62126959641.6254\\
0.864321608040201	-62099457667.4591\\
0.864421610540263	-62072528651.088\\
0.864521613040326	-62045026676.9217\\
0.864621615540389	-62017524702.7554\\
0.864721618040451	-61990595686.3843\\
0.864821620540513	-61963093712.218\\
0.864921623040576	-61936164695.8469\\
0.865021625540639	-61908662721.6806\\
0.865121628040701	-61881160747.5143\\
0.865221630540764	-61854231731.1432\\
0.865321633040826	-61826729756.9769\\
0.865421635540888	-61799800740.6057\\
0.865521638040951	-61772298766.4394\\
0.865621640541014	-61745369750.0683\\
0.865721643041076	-61717867775.902\\
0.865821645541138	-61690365801.7357\\
0.865921648041201	-61663436785.3646\\
0.866021650541264	-61635934811.1983\\
0.866121653041326	-61609005794.8272\\
0.866221655541389	-61582076778.456\\
0.866321658041451	-61554574804.2897\\
0.866421660541513	-61527645787.9186\\
0.866521663041576	-61500143813.7523\\
0.866621665541639	-61473214797.3812\\
0.866721668041701	-61445712823.2149\\
0.866821670541764	-61418783806.8437\\
0.866921673041826	-61391854790.4726\\
0.867021675541889	-61364352816.3063\\
0.867121678041951	-61337423799.9352\\
0.867221680542014	-61309921825.7689\\
0.867321683042076	-61282992809.3977\\
0.867421685542138	-61256063793.0266\\
0.867521688042201	-61228561818.8603\\
0.867621690542264	-61201632802.4891\\
0.867721693042326	-61174703786.118\\
0.867821695542389	-61147201811.9517\\
0.867921698042451	-61120272795.5806\\
0.868021700542514	-61093343779.2094\\
0.868121703042576	-61066414762.8383\\
0.868221705542639	-61038912788.672\\
0.868321708042701	-61011983772.3008\\
0.868421710542764	-60985054755.9297\\
0.868521713042826	-60958125739.5585\\
0.868621715542889	-60931196723.1874\\
0.868721718042951	-60903694749.0211\\
0.868821720543014	-60876765732.65\\
0.868921723043076	-60849836716.2788\\
0.869021725543139	-60822907699.9077\\
0.869121728043201	-60795978683.5365\\
0.869221730543264	-60769049667.1654\\
0.869321733043326	-60741547692.9991\\
0.869421735543389	-60714618676.6279\\
0.869521738043451	-60687689660.2568\\
0.869621740543514	-60660760643.8857\\
0.869721743043576	-60633831627.5145\\
0.869821745543639	-60606902611.1433\\
0.869921748043701	-60579973594.7722\\
0.870021750543764	-60553044578.4011\\
0.870121753043826	-60526115562.0299\\
0.870221755543889	-60499186545.6588\\
0.870321758043951	-60472257529.2876\\
0.870421760544014	-60445328512.9165\\
0.870521763044076	-60418399496.5453\\
0.870621765544139	-60391470480.1742\\
0.870721768044201	-60364541463.803\\
0.870821770544264	-60337612447.4319\\
0.870921773044326	-60310683431.0607\\
0.871021775544389	-60283754414.6896\\
0.871121778044451	-60256825398.3184\\
0.871221780544514	-60229896381.9473\\
0.871321783044576	-60202967365.5761\\
0.871421785544639	-60176038349.205\\
0.871521788044701	-60149682290.629\\
0.871621790544764	-60122753274.2578\\
0.871721793044826	-60095824257.8867\\
0.871821795544889	-60068895241.5155\\
0.871921798044951	-60041966225.1444\\
0.872021800545014	-60015037208.7732\\
0.872121803045076	-59988681150.1972\\
0.872221805545139	-59961752133.826\\
0.872321808045201	-59934823117.4549\\
0.872421810545264	-59907894101.0837\\
0.872521813045326	-59880965084.7126\\
0.872621815545389	-59854609026.1366\\
0.872721818045451	-59827680009.7654\\
0.872821820545514	-59800750993.3943\\
0.872921823045576	-59774394934.8183\\
0.873021825545639	-59747465918.4471\\
0.873121828045701	-59720536902.076\\
0.873221830545764	-59693607885.7048\\
0.873321833045826	-59667251827.1288\\
0.873421835545889	-59640322810.7577\\
0.873521838045951	-59613393794.3865\\
0.873621840546014	-59587037735.8105\\
0.873721843046076	-59560108719.4393\\
0.873821845546139	-59533752660.8633\\
0.873921848046201	-59506823644.4922\\
0.874021850546264	-59479894628.121\\
0.874121853046326	-59453538569.545\\
0.874221855546389	-59426609553.1739\\
0.874321858046451	-59400253494.5978\\
0.874421860546514	-59373324478.2267\\
0.874521863046576	-59346968419.6507\\
0.874621865546639	-59320039403.2795\\
0.874721868046701	-59293683344.7035\\
0.874821870546764	-59266754328.3324\\
0.874921873046826	-59240398269.7563\\
0.875021875546889	-59213469253.3852\\
0.875121878046951	-59187113194.8092\\
0.875221880547014	-59160184178.438\\
0.875321883047076	-59133828119.862\\
0.875421885547139	-59106899103.4909\\
0.875521888047201	-59080543044.9148\\
0.875621890547264	-59054186986.3388\\
0.875721893047326	-59027257969.9677\\
0.875821895547389	-59000901911.3917\\
0.875921898047451	-58973972895.0205\\
0.876021900547514	-58947616836.4445\\
0.876121903047576	-58921260777.8685\\
0.876221905547639	-58894331761.4973\\
0.876321908047701	-58867975702.9213\\
0.876421910547764	-58841619644.3453\\
0.876521913047826	-58814690627.9741\\
0.876621915547889	-58788334569.3981\\
0.876721918047951	-58761978510.8221\\
0.876821920548014	-58735049494.451\\
0.876921923048076	-58708693435.8749\\
0.877021925548139	-58682337377.2989\\
0.877121928048201	-58655981318.7229\\
0.877221930548264	-58629052302.3517\\
0.877321933048326	-58602696243.7757\\
0.877421935548389	-58576340185.1997\\
0.877521938048451	-58549984126.6237\\
0.877621940548514	-58523628068.0477\\
0.877721943048576	-58497272009.4717\\
0.877821945548639	-58470342993.1005\\
0.877921948048701	-58443986934.5245\\
0.878021950548764	-58417630875.9485\\
0.878121953048826	-58391274817.3725\\
0.878221955548889	-58364918758.7964\\
0.878321958048951	-58338562700.2204\\
0.878421960549014	-58312206641.6444\\
0.878521963049076	-58285850583.0684\\
0.878621965549139	-58258921566.6972\\
0.878721968049201	-58232565508.1212\\
0.878821970549264	-58206209449.5452\\
0.878921973049326	-58179853390.9692\\
0.879021975549389	-58153497332.3932\\
0.879121978049451	-58127141273.8171\\
0.879221980549514	-58100785215.2411\\
0.879321983049576	-58074429156.6651\\
0.879421985549639	-58048073098.0891\\
0.879521988049701	-58021717039.5131\\
0.879621990549764	-57995360980.9371\\
0.879721993049826	-57969004922.361\\
0.879821995549889	-57943221821.5802\\
0.879921998049951	-57916865763.0041\\
0.880022000550014	-57890509704.4281\\
0.880122003050076	-57864153645.8521\\
0.880222005550139	-57837797587.2761\\
0.880322008050201	-57811441528.7001\\
0.880422010550264	-57785085470.124\\
0.880522013050326	-57758729411.548\\
0.880622015550389	-57732373352.972\\
0.880722018050451	-57706590252.1911\\
0.880822020550514	-57680234193.6151\\
0.880922023050576	-57653878135.0391\\
0.881022025550639	-57627522076.4631\\
0.881122028050701	-57601166017.8871\\
0.881222030550764	-57575382917.1062\\
0.881322033050826	-57549026858.5301\\
0.881422035550889	-57522670799.9541\\
0.881522038050951	-57496314741.3781\\
0.881622040551014	-57470531640.5972\\
0.881722043051076	-57444175582.0212\\
0.881822045551139	-57417819523.4452\\
0.881922048051201	-57392036422.6643\\
0.882022050551264	-57365680364.0883\\
0.882122053051326	-57339324305.5123\\
0.882222055551389	-57312968246.9362\\
0.882322058051451	-57287070554.5963\\
0.882422060551514	-57260943679.1384\\
0.882522063051576	-57234816803.6804\\
0.882622065551639	-57208632632.4429\\
0.882722068051701	-57182505756.985\\
0.882822070551764	-57156378881.527\\
0.882922073051826	-57130309301.8485\\
0.883022075551889	-57104182426.3906\\
0.883122078051951	-57078055550.9326\\
0.883222080552014	-57051985971.2542\\
0.883322083052076	-57025859095.7962\\
0.883422085552139	-56999789516.1177\\
0.883522088052201	-56973719936.4393\\
0.883622090552264	-56947650356.7608\\
0.883722093052326	-56921580777.0824\\
0.883822095552389	-56895511197.4039\\
0.883922098052451	-56869441617.7255\\
0.884022100552514	-56843429333.8265\\
0.884122103052576	-56817359754.1481\\
0.884222105552639	-56791347470.2491\\
0.884322108052701	-56765335186.3502\\
0.884422110552764	-56739322902.4513\\
0.884522113052826	-56713310618.5523\\
0.884622115552889	-56687298334.6534\\
0.884722118052951	-56661286050.7544\\
0.884822120553014	-56635273766.8555\\
0.884922123053076	-56609318778.7361\\
0.885022125553139	-56583306494.8371\\
0.885122128053201	-56557351506.7177\\
0.885222130553264	-56531396518.5983\\
0.885322133053326	-56505384234.6994\\
0.885422135553389	-56479429246.5799\\
0.885522138053451	-56453531554.24\\
0.885622140553514	-56427576566.1206\\
0.885722143053576	-56401621578.0012\\
0.885822145553639	-56375666589.8817\\
0.885922148053701	-56349768897.5418\\
0.886022150553764	-56323871205.2019\\
0.886122153053826	-56297916217.0825\\
0.886222155553889	-56272018524.7426\\
0.886322158053951	-56246120832.4027\\
0.886422160554014	-56220280435.8423\\
0.886522163054076	-56194382743.5023\\
0.886622165554139	-56168485051.1624\\
0.886722168054201	-56142644654.602\\
0.886822170554264	-56116746962.2621\\
0.886922173054326	-56090906565.7017\\
0.887022175554389	-56065066169.1413\\
0.887122178054451	-56039168476.8014\\
0.887222180554514	-56013328080.241\\
0.887322183054576	-55987544979.4601\\
0.887422185554639	-55961704582.8997\\
0.887522188054701	-55935864186.3393\\
0.887622190554764	-55910081085.5584\\
0.887722193054826	-55884240688.998\\
0.887822195554889	-55858457588.2171\\
0.887922198054951	-55832674487.4363\\
0.888022200555014	-55806891386.6554\\
0.888122203055076	-55781108285.8745\\
0.888222205555139	-55755325185.0936\\
0.888322208055201	-55729542084.3127\\
0.888422210555264	-55703816279.3113\\
0.888522213055326	-55678033178.5304\\
0.888622215555389	-55652307373.5291\\
0.888722218055451	-55626524272.7482\\
0.888822220555514	-55600798467.7468\\
0.888922223055576	-55575072662.7454\\
0.889022225555639	-55549346857.7441\\
0.889122228055701	-55523621052.7427\\
0.889222230555764	-55497952543.5208\\
0.889322233055826	-55472226738.5195\\
0.889422235555889	-55446500933.5181\\
0.889522238055951	-55420832424.2962\\
0.889622240556014	-55395163915.0744\\
0.889722243056076	-55369495405.8525\\
0.889822245556139	-55343826896.6306\\
0.889922248056201	-55318158387.4088\\
0.890022250556264	-55292489878.1869\\
0.890122253056326	-55266821368.965\\
0.890222255556389	-55241210155.5227\\
0.890322258056451	-55215541646.3008\\
0.890422260556514	-55189930432.8585\\
0.890522263056576	-55164261923.6366\\
0.890622265556639	-55138650710.1943\\
0.890722268056701	-55113039496.7519\\
0.890822270556764	-55087428283.3096\\
0.890922273056826	-55061874365.6468\\
0.891022275556889	-55036263152.2044\\
0.891122278056951	-55010651938.7621\\
0.891222280557014	-54985098021.0992\\
0.891322283057076	-54959486807.6569\\
0.891422285557139	-54933932889.994\\
0.891522288057201	-54908378972.3312\\
0.891622290557264	-54882825054.6684\\
0.891722293057326	-54857271137.0055\\
0.891822295557389	-54831717219.3427\\
0.891922298057451	-54806220597.4594\\
0.892022300557514	-54780666679.7965\\
0.892122303057576	-54755170057.9132\\
0.892222305557639	-54729616140.2504\\
0.892322308057701	-54704119518.3671\\
0.892422310557764	-54678622896.4837\\
0.892522313057826	-54653126274.6004\\
0.892622315557889	-54627629652.7171\\
0.892722318057951	-54602133030.8338\\
0.892822320558014	-54576693704.73\\
0.892922323058077	-54551197082.8467\\
0.893022325558139	-54525757756.7428\\
0.893122328058201	-54500261134.8595\\
0.893222330558264	-54474821808.7557\\
0.893322333058326	-54449382482.6519\\
0.893422335558389	-54423943156.5481\\
0.893522338058451	-54398503830.4443\\
0.893622340558514	-54373064504.3405\\
0.893722343058576	-54347682474.0162\\
0.893822345558639	-54322243147.9124\\
0.893922348058702	-54296861117.5881\\
0.894022350558764	-54271421791.4843\\
0.894122353058826	-54246039761.16\\
0.894222355558889	-54220657730.8357\\
0.894322358058951	-54195275700.5114\\
0.894422360559014	-54169893670.1871\\
0.894522363059077	-54144568935.6423\\
0.894622365559139	-54119186905.318\\
0.894722368059201	-54093804874.9937\\
0.894822370559264	-54068480140.4489\\
0.894922373059327	-54043155405.9042\\
0.895022375559389	-54017830671.3594\\
0.895122378059451	-53992448641.0351\\
0.895222380559514	-53967181202.2698\\
0.895322383059576	-53941856467.725\\
0.895422385559639	-53916531733.1802\\
0.895522388059702	-53891206998.6355\\
0.895622390559764	-53865939559.8702\\
0.895722393059826	-53840614825.3254\\
0.895822395559889	-53815347386.5601\\
0.895922398059952	-53790079947.7949\\
0.896022400560014	-53764812509.0296\\
0.896122403060077	-53739545070.2643\\
0.896222405560139	-53714277631.4991\\
0.896322408060201	-53689010192.7338\\
0.896422410560264	-53663800049.748\\
0.896522413060327	-53638532610.9828\\
0.896622415560389	-53613322467.997\\
0.896722418060451	-53588112325.0113\\
0.896822420560514	-53562844886.246\\
0.896922423060577	-53537634743.2602\\
0.897022425560639	-53512424600.2745\\
0.897122428060702	-53487271753.0682\\
0.897222430560764	-53462061610.0825\\
0.897322433060826	-53436851467.0967\\
0.897422435560889	-53411698619.8905\\
0.897522438060952	-53386488476.9047\\
0.897622440561014	-53361335629.6985\\
0.897722443061077	-53336182782.4922\\
0.897822445561139	-53311029935.286\\
0.897922448061202	-53285877088.0797\\
0.898022450561264	-53260724240.8735\\
0.898122453061327	-53235628689.4468\\
0.898222455561389	-53210475842.2405\\
0.898322458061452	-53185380290.8138\\
0.898422460561514	-53160227443.6076\\
0.898522463061577	-53135131892.1808\\
0.898622465561639	-53110036340.7541\\
0.898722468061702	-53084940789.3274\\
0.898822470561764	-53059845237.9006\\
0.898922473061827	-53034749686.4739\\
0.899022475561889	-53009654135.0472\\
0.899122478061952	-52984615879.4\\
0.899222480562014	-52959520327.9732\\
0.899322483062077	-52934482072.326\\
0.899422485562139	-52909443816.6788\\
0.899522488062202	-52884405561.0316\\
0.899622490562264	-52859367305.3844\\
0.899722493062327	-52834329049.7371\\
0.899822495562389	-52809290794.0899\\
0.899922498062452	-52784252538.4427\\
0.900022500562514	-52759271578.575\\
0.900122503062577	-52734233322.9278\\
0.900222505562639	-52709252363.0601\\
0.900322508062702	-52684271403.1924\\
0.900422510562764	-52659290443.3247\\
0.900522513062827	-52634309483.457\\
0.900622515562889	-52609328523.5893\\
0.900722518062952	-52584347563.7216\\
0.900822520563014	-52559366603.8539\\
0.900922523063077	-52534442939.7657\\
0.901022525563139	-52509461979.898\\
0.901122528063202	-52484538315.8098\\
0.901222530563264	-52459614651.7216\\
0.901322533063327	-52434690987.6334\\
0.901422535563389	-52409767323.5452\\
0.901522538063452	-52384843659.457\\
0.901622540563514	-52359919995.3688\\
0.901722543063577	-52334996331.2806\\
0.901822545563639	-52310129962.972\\
0.901922548063702	-52285206298.8838\\
0.902022550563764	-52260339930.5751\\
0.902122553063827	-52235473562.2664\\
0.902222555563889	-52210607193.9577\\
0.902322558063952	-52185740825.649\\
0.902422560564014	-52160874457.3404\\
0.902522563064077	-52136008089.0317\\
0.902622565564139	-52111199016.5025\\
0.902722568064202	-52086332648.1939\\
0.902822570564264	-52061523575.6647\\
0.902922573064327	-52036657207.356\\
0.903022575564389	-52011848134.8268\\
0.903122578064452	-51987039062.2977\\
0.903222580564514	-51962229989.7685\\
0.903322583064577	-51937420917.2393\\
0.903422585564639	-51912669140.4897\\
0.903522588064702	-51887860067.9605\\
0.903622590564764	-51863050995.4314\\
0.903722593064827	-51838299218.6817\\
0.903822595564889	-51813547441.9321\\
0.903922598064952	-51788795665.1824\\
0.904022600565014	-51764043888.4328\\
0.904122603065077	-51739292111.6831\\
0.904222605565139	-51714540334.9335\\
0.904322608065202	-51689788558.1838\\
0.904422610565264	-51665036781.4342\\
0.904522613065327	-51640342300.464\\
0.904622615565389	-51615590523.7144\\
0.904722618065452	-51590896042.7442\\
0.904822620565514	-51566201561.7741\\
0.904922623065577	-51541507080.804\\
0.905022625565639	-51516812599.8338\\
0.905122628065702	-51492118118.8637\\
0.905222630565764	-51467423637.8935\\
0.905322633065827	-51442786452.7029\\
0.905422635565889	-51418091971.7328\\
0.905522638065952	-51393454786.5421\\
0.905622640566014	-51368817601.3515\\
0.905722643066077	-51344180416.1609\\
0.905822645566139	-51319543230.9703\\
0.905922648066202	-51294906045.7796\\
0.906022650566264	-51270268860.589\\
0.906122653066327	-51245631675.3984\\
0.906222655566389	-51220994490.2078\\
0.906322658066452	-51196414600.7967\\
0.906422660566514	-51171834711.3855\\
0.906522663066577	-51147197526.1949\\
0.906622665566639	-51122617636.7838\\
0.906722668066702	-51098037747.3727\\
0.906822670566764	-51073457857.9616\\
0.906922673066827	-51048877968.5505\\
0.907022675566889	-51024355374.9189\\
0.907122678066952	-50999775485.5078\\
0.907222680567014	-50975252891.8762\\
0.907322683067077	-50950673002.465\\
0.907422685567139	-50926150408.8335\\
0.907522688067202	-50901627815.2019\\
0.907622690567264	-50877105221.5703\\
0.907722693067327	-50852582627.9387\\
0.907822695567389	-50828060034.3071\\
0.907922698067452	-50803537440.6755\\
0.908022700567514	-50779072142.8234\\
0.908122703067577	-50754549549.1918\\
0.908222705567639	-50730084251.3397\\
0.908322708067702	-50705618953.4876\\
0.908422710567764	-50681153655.6355\\
0.908522713067827	-50656688357.7834\\
0.908622715567889	-50632223059.9313\\
0.908722718067952	-50607757762.0792\\
0.908822720568014	-50583292464.2272\\
0.908922723068077	-50558884462.1546\\
0.909022725568139	-50534419164.3025\\
0.909122728068202	-50510011162.2299\\
0.909222730568264	-50485603160.1574\\
0.909322733068327	-50461195158.0848\\
0.909422735568389	-50436787156.0122\\
0.909522738068452	-50412379153.9396\\
0.909622740568514	-50387971151.8671\\
0.909722743068577	-50363563149.7945\\
0.909822745568639	-50339212443.5014\\
0.909922748068702	-50314804441.4289\\
0.910022750568764	-50290453735.1358\\
0.910122753068827	-50266103028.8427\\
0.910222755568889	-50241695026.7702\\
0.910322758068952	-50217344320.4771\\
0.910422760569014	-50193050909.9636\\
0.910522763069077	-50168700203.6705\\
0.910622765569139	-50144349497.3774\\
0.910722768069202	-50120056086.8639\\
0.910822770569264	-50095705380.5708\\
0.910922773069327	-50071411970.0573\\
0.911022775569389	-50047118559.5437\\
0.911122778069452	-50022825149.0302\\
0.911222780569514	-49998531738.5166\\
0.911322783069577	-49974238328.0031\\
0.911422785569639	-49949944917.4896\\
0.911522788069702	-49925651506.976\\
0.911622790569764	-49901415392.242\\
0.911722793069827	-49877121981.7284\\
0.911822795569889	-49852885866.9944\\
0.911922798069952	-49828649752.2604\\
0.912022800570014	-49804413637.5263\\
0.912122803070077	-49780177522.7923\\
0.912222805570139	-49755941408.0583\\
0.912322808070202	-49731705293.3242\\
0.912422810570264	-49707469178.5902\\
0.912522813070327	-49683290359.6357\\
0.912622815570389	-49659054244.9016\\
0.912722818070452	-49634875425.9471\\
0.912822820570514	-49610696606.9926\\
0.912922823070577	-49586517788.0381\\
0.913022825570639	-49562338969.0835\\
0.913122828070702	-49538160150.129\\
0.913222830570764	-49513981331.1745\\
0.913322833070827	-49489802512.22\\
0.913422835570889	-49465680989.045\\
0.913522838070952	-49441559465.87\\
0.913622840571014	-49417380646.9155\\
0.913722843071077	-49393259123.7404\\
0.913822845571139	-49369137600.5654\\
0.913922848071202	-49345016077.3904\\
0.914022850571264	-49320894554.2154\\
0.914122853071327	-49296773031.0404\\
0.914222855571389	-49272708803.6449\\
0.914322858071452	-49248587280.4699\\
0.914422860571514	-49224523053.0744\\
0.914522863071577	-49200401529.8994\\
0.914622865571639	-49176337302.5039\\
0.914722868071702	-49152273075.1084\\
0.914822870571764	-49128208847.7129\\
0.914922873071827	-49104144620.3174\\
0.915022875571889	-49080137688.7014\\
0.915122878071952	-49056073461.306\\
0.915222880572014	-49032009233.9105\\
0.915322883072077	-49008002302.2945\\
0.915422885572139	-48983995370.6785\\
0.915522888072202	-48959988439.0625\\
0.915622890572264	-48935924211.667\\
0.915722893072327	-48911974575.8306\\
0.915822895572389	-48887967644.2146\\
0.915922898072452	-48863960712.5986\\
0.916022900572514	-48839953780.9826\\
0.916122903072577	-48816004145.1461\\
0.916222905572639	-48791997213.5302\\
0.916322908072702	-48768047577.6937\\
0.916422910572764	-48744097941.8572\\
0.916522913072827	-48720148306.0208\\
0.916622915572889	-48696198670.1843\\
0.916722918072952	-48672249034.3478\\
0.916822920573014	-48648299398.5113\\
0.916922923073077	-48624407058.4544\\
0.917022925573139	-48600457422.6179\\
0.917122928073202	-48576565082.561\\
0.917222930573264	-48552615446.7245\\
0.917322933073327	-48528723106.6675\\
0.917422935573389	-48504830766.6106\\
0.917522938073452	-48480938426.5536\\
0.917622940573514	-48457046086.4967\\
0.917722943073577	-48433211042.2192\\
0.917822945573639	-48409318702.1623\\
0.917922948073702	-48385483657.8848\\
0.918022950573764	-48361591317.8279\\
0.918122953073827	-48337756273.5504\\
0.918222955573889	-48313921229.273\\
0.918322958073952	-48290086184.9956\\
0.918422960574014	-48266251140.7181\\
0.918522963074077	-48242416096.4407\\
0.918622965574139	-48218581052.1632\\
0.918722968074202	-48194803303.6653\\
0.918822970574264	-48170968259.3879\\
0.918922973074327	-48147190510.8899\\
0.919022975574389	-48123355466.6125\\
0.919122978074452	-48099577718.1146\\
0.919222980574514	-48075799969.6166\\
0.919322983074577	-48052022221.1187\\
0.919422985574639	-48028244472.6208\\
0.919522988074702	-48004524019.9024\\
0.919622990574764	-47980746271.4044\\
0.919722993074827	-47957025818.686\\
0.919822995574889	-47933248070.1881\\
0.919922998074952	-47909527617.4697\\
0.920023000575014	-47885807164.7513\\
0.920123003075077	-47862086712.0328\\
0.920223005575139	-47838366259.3144\\
0.920323008075202	-47814645806.596\\
0.920423010575264	-47790925353.8776\\
0.920523013075327	-47767262196.9387\\
0.920623015575389	-47743541744.2203\\
0.920723018075452	-47719878587.2814\\
0.920823020575514	-47696158134.5629\\
0.920923023075577	-47672494977.624\\
0.921023025575639	-47648831820.6851\\
0.921123028075702	-47625168663.7462\\
0.921223030575764	-47601562802.5868\\
0.921323033075827	-47577899645.6479\\
0.921423035575889	-47554236488.709\\
0.921523038075952	-47530630627.5497\\
0.921623040576014	-47506967470.6107\\
0.921723043076077	-47483361609.4514\\
0.921823045576139	-47459755748.292\\
0.921923048076202	-47436149887.1326\\
0.922023050576264	-47412544025.9732\\
0.922123053076327	-47388938164.8138\\
0.922223055576389	-47365332303.6544\\
0.922323058076452	-47341783738.2745\\
0.922423060576514	-47318177877.1151\\
0.922523063076577	-47294629311.7353\\
0.922623065576639	-47271080746.3554\\
0.922723068076702	-47247532180.9755\\
0.922823070576764	-47223926319.8161\\
0.922923073076827	-47200435050.2158\\
0.923023075576889	-47176886484.8359\\
0.923123078076952	-47153337919.456\\
0.923223080577014	-47129789354.0761\\
0.923323083077077	-47106298084.4758\\
0.923423085577139	-47082806814.8754\\
0.923523088077202	-47059258249.4955\\
0.923623090577264	-47035766979.8952\\
0.923723093077327	-47012275710.2948\\
0.923823095577389	-46988784440.6944\\
0.923923098077452	-46965293171.0941\\
0.924023100577514	-46941859197.2732\\
0.924123103077577	-46918367927.6729\\
0.924223105577639	-46894933953.852\\
0.924323108077702	-46871442684.2516\\
0.924423110577764	-46848008710.4308\\
0.924523113077827	-46824574736.6099\\
0.924623115577889	-46801140762.7891\\
0.924723118077952	-46777706788.9682\\
0.924823120578014	-46754272815.1474\\
0.924923123078077	-46730838841.3265\\
0.925023125578139	-46707462163.2852\\
0.925123128078202	-46684028189.4643\\
0.925223130578264	-46660651511.423\\
0.925323133078327	-46637274833.3817\\
0.92542313557839	-46613840859.5608\\
0.925523138078452	-46590464181.5195\\
0.925623140578514	-46567144799.2577\\
0.925723143078577	-46543768121.2163\\
0.925823145578639	-46520391443.175\\
0.925923148078702	-46497014765.1336\\
0.926023150578764	-46473695382.8718\\
0.926123153078827	-46450376000.61\\
0.926223155578889	-46426999322.5687\\
0.926323158078952	-46403679940.3068\\
0.926423160579015	-46380360558.045\\
0.926523163079077	-46357041175.7832\\
0.926623165579139	-46333721793.5214\\
0.926723168079202	-46310459707.039\\
0.926823170579264	-46287140324.7772\\
0.926923173079327	-46263878238.2949\\
0.92702317557939	-46240558856.0331\\
0.927123178079452	-46217296769.5508\\
0.927223180579514	-46194034683.0685\\
0.927323183079577	-46170772596.5862\\
0.92742318557964	-46147510510.1038\\
0.927523188079702	-46124248423.6215\\
0.927623190579765	-46100986337.1392\\
0.927723193079827	-46077781546.4364\\
0.927823195579889	-46054519459.9541\\
0.927923198079952	-46031314669.2513\\
0.928023200580015	-46008109878.5485\\
0.928123203080077	-45984905087.8457\\
0.928223205580139	-45961643001.3634\\
0.928323208080202	-45938495506.4401\\
0.928423210580265	-45915290715.7373\\
0.928523213080327	-45892085925.0345\\
0.92862321558039	-45868881134.3317\\
0.928723218080452	-45845733639.4084\\
0.928823220580514	-45822586144.4852\\
0.928923223080577	-45799381353.7824\\
0.92902322558064	-45776233858.8591\\
0.929123228080702	-45753086363.9358\\
0.929223230580765	-45729938869.0125\\
0.929323233080827	-45706791374.0892\\
0.92942323558089	-45683701174.9454\\
0.929523238080952	-45660553680.0222\\
0.929623240581015	-45637463480.8784\\
0.929723243081077	-45614315985.9551\\
0.929823245581139	-45591225786.8113\\
0.929923248081202	-45568135587.6676\\
0.930023250581265	-45545045388.5238\\
0.930123253081327	-45521955189.38\\
0.93022325558139	-45498864990.2362\\
0.930323258081452	-45475774791.0925\\
0.930423260581515	-45452741887.7282\\
0.930523263081577	-45429651688.5844\\
0.93062326558164	-45406618785.2202\\
0.930723268081702	-45383585881.8559\\
0.930823270581765	-45360552978.4917\\
0.930923273081827	-45337520075.1274\\
0.93102327558189	-45314487171.7631\\
0.931123278081952	-45291454268.3989\\
0.931223280582015	-45268421365.0346\\
0.931323283082077	-45245388461.6704\\
0.93142328558214	-45222412854.0856\\
0.931523288082202	-45199437246.5009\\
0.931623290582265	-45176404343.1366\\
0.931723293082327	-45153428735.5519\\
0.93182329558239	-45130453127.9671\\
0.931923298082452	-45107477520.3824\\
0.932023300582515	-45084501912.7976\\
0.932123303082577	-45061583600.9924\\
0.93222330558264	-45038607993.4076\\
0.932323308082702	-45015689681.6024\\
0.932423310582765	-44992714074.0177\\
0.932523313082827	-44969795762.2124\\
0.93262331558289	-44946877450.4072\\
0.932723318082952	-44923959138.602\\
0.932823320583015	-44901040826.7967\\
0.932923323083077	-44878122514.9915\\
0.93302332558314	-44855204203.1863\\
0.933123328083202	-44832343187.1606\\
0.933223330583265	-44809424875.3553\\
0.933323333083327	-44786563859.3296\\
0.93342333558339	-44763702843.3039\\
0.933523338083452	-44740784531.4986\\
0.933623340583515	-44717923515.4729\\
0.933723343083577	-44695119795.2267\\
0.93382334558364	-44672258779.201\\
0.933923348083702	-44649397763.1753\\
0.934023350583765	-44626536747.1496\\
0.934123353083827	-44603733026.9034\\
0.93422335558389	-44580929306.6571\\
0.934323358083952	-44558068290.6314\\
0.934423360584015	-44535264570.3852\\
0.934523363084077	-44512460850.139\\
0.93462336558414	-44489657129.8928\\
0.934723368084202	-44466853409.6466\\
0.934823370584265	-44444106985.1799\\
0.934923373084327	-44421303264.9337\\
0.93502337558439	-44398499544.6875\\
0.935123378084452	-44375753120.2208\\
0.935223380584515	-44353006695.7541\\
0.935323383084577	-44330260271.2874\\
0.93542338558464	-44307513846.8207\\
0.935523388084702	-44284767422.354\\
0.935623390584765	-44262020997.8873\\
0.935723393084827	-44239274573.4206\\
0.93582339558489	-44216585444.7335\\
0.935923398084952	-44193839020.2668\\
0.936023400585015	-44171149891.5796\\
0.936123403085077	-44148403467.1129\\
0.93622340558514	-44125714338.4257\\
0.936323408085202	-44103025209.7385\\
0.936423410585265	-44080336081.0513\\
0.936523413085327	-44057646952.3642\\
0.93662341558539	-44035015119.4565\\
0.936723418085452	-44012325990.7693\\
0.936823420585515	-43989694157.8616\\
0.936923423085577	-43967005029.1745\\
0.93702342558564	-43944373196.2668\\
0.937123428085702	-43921741363.3591\\
0.937223430585765	-43899109530.4515\\
0.937323433085827	-43876477697.5438\\
0.93742343558589	-43853845864.6361\\
0.937523438085952	-43831214031.7285\\
0.937623440586015	-43808582198.8208\\
0.937723443086077	-43786007661.6926\\
0.93782344558614	-43763433124.5645\\
0.937923448086202	-43740801291.6568\\
0.938023450586265	-43718226754.5287\\
0.938123453086327	-43695652217.4005\\
0.93822345558639	-43673077680.2724\\
0.938323458086452	-43650503143.1442\\
0.938423460586515	-43627985901.7956\\
0.938523463086577	-43605411364.6674\\
0.93862346558664	-43582836827.5393\\
0.938723468086702	-43560319586.1906\\
0.938823470586765	-43537802344.842\\
0.938923473086827	-43515285103.4933\\
0.93902347558689	-43492710566.3652\\
0.939123478086952	-43470250620.796\\
0.939223480587015	-43447733379.4474\\
0.939323483087077	-43425216138.0988\\
0.93942348558714	-43402698896.7501\\
0.939523488087202	-43380238951.181\\
0.939623490587265	-43357721709.8324\\
0.939723493087327	-43335261764.2632\\
0.93982349558739	-43312801818.6941\\
0.939923498087452	-43290341873.125\\
0.940023500587515	-43267881927.5558\\
0.940123503087577	-43245421981.9867\\
0.94022350558764	-43222962036.4176\\
0.940323508087702	-43200559386.628\\
0.940423510587765	-43178099441.0588\\
0.940523513087827	-43155696791.2692\\
0.94062351558789	-43133236845.7001\\
0.940723518087952	-43110834195.9105\\
0.940823520588015	-43088431546.1209\\
0.940923523088077	-43066028896.3313\\
0.94102352558814	-43043626246.5416\\
0.941123528088202	-43021280892.5315\\
0.941223530588265	-42998878242.7419\\
0.941323533088327	-42976475592.9523\\
0.94142353558839	-42954130238.9422\\
0.941523538088452	-42931784884.9321\\
0.941623540588515	-42909382235.1425\\
0.941723543088577	-42887036881.1324\\
0.94182354558864	-42864691527.1223\\
0.941923548088702	-42842403468.8917\\
0.942023550588765	-42820058114.8816\\
0.942123553088827	-42797712760.8715\\
0.94222355558889	-42775424702.6409\\
0.942323558088952	-42753079348.6308\\
0.942423560589015	-42730791290.4002\\
0.942523563089077	-42708503232.1696\\
0.94262356558914	-42686215173.939\\
0.942723568089202	-42663927115.7084\\
0.942823570589265	-42641639057.4779\\
0.942923573089327	-42619350999.2473\\
0.94302357558939	-42597062941.0167\\
0.943123578089452	-42574832178.5656\\
0.943223580589515	-42552544120.335\\
0.943323583089577	-42530313357.8839\\
0.94342358558964	-42508082595.4329\\
0.943523588089702	-42485851832.9818\\
0.943623590589765	-42463621070.5307\\
0.943723593089827	-42441390308.0796\\
0.94382359558989	-42419159545.6285\\
0.943923598089952	-42396928783.1775\\
0.944023600590015	-42374755316.5059\\
0.944123603090077	-42352524554.0548\\
0.94422360559014	-42330351087.3833\\
0.944323608090202	-42308177620.7117\\
0.944423610590265	-42286004154.0401\\
0.944523613090327	-42263830687.3686\\
0.94462361559039	-42241657220.697\\
0.944723618090452	-42219483754.0255\\
0.944823620590515	-42197310287.3539\\
0.944923623090577	-42175194116.4618\\
0.94502362559064	-42153020649.7903\\
0.945123628090702	-42130904478.8982\\
0.945223630590765	-42108788308.0062\\
0.945323633090827	-42086672137.1141\\
0.94542363559089	-42064555966.2221\\
0.945523638090952	-42042439795.33\\
0.945623640591015	-42020323624.438\\
0.945723643091077	-41998207453.5459\\
0.94582364559114	-41976148578.4334\\
0.945923648091202	-41954032407.5414\\
0.946023650591265	-41931973532.4288\\
0.946123653091327	-41909914657.3163\\
0.94622365559139	-41887855782.2037\\
0.946323658091452	-41865796907.0912\\
0.946423660591515	-41843738031.9787\\
0.946523663091577	-41821679156.8661\\
0.94662366559164	-41799620281.7536\\
0.946723668091702	-41777618702.4206\\
0.946823670591765	-41755559827.308\\
0.946923673091827	-41733558247.975\\
0.94702367559189	-41711556668.642\\
0.947123678091952	-41689497793.5294\\
0.947223680592015	-41667496214.1964\\
0.947323683092077	-41645494634.8634\\
0.94742368559214	-41623550351.3099\\
0.947523688092202	-41601548771.9769\\
0.947623690592265	-41579547192.6438\\
0.947723693092327	-41557602909.0903\\
0.94782369559239	-41535601329.7573\\
0.947923698092452	-41513657046.2038\\
0.948023700592515	-41491712762.6503\\
0.948123703092577	-41469768479.0968\\
0.94822370559264	-41447824195.5433\\
0.948323708092702	-41425879911.9898\\
0.948423710592765	-41403992924.2158\\
0.948523713092827	-41382048640.6622\\
0.94862371559289	-41360161652.8883\\
0.948723718092952	-41338217369.3347\\
0.948823720593015	-41316330381.5607\\
0.948923723093077	-41294443393.7867\\
0.94902372559314	-41272556406.0127\\
0.949123728093202	-41250669418.2387\\
0.949223730593265	-41228782430.4648\\
0.949323733093327	-41206895442.6908\\
0.94942373559339	-41185065750.6963\\
0.949523738093452	-41163178762.9223\\
0.949623740593515	-41141349070.9278\\
0.949723743093577	-41119462083.1538\\
0.94982374559364	-41097632391.1593\\
0.949923748093702	-41075802699.1648\\
0.950023750593765	-41053973007.1703\\
0.950123753093827	-41032143315.1758\\
0.95022375559389	-41010370918.9609\\
0.950323758093952	-40988541226.9664\\
0.950423760594015	-40966768830.7514\\
0.950523763094077	-40944939138.7569\\
0.95062376559414	-40923166742.542\\
0.950723768094202	-40901394346.327\\
0.950823770594265	-40879621950.112\\
0.950923773094327	-40857849553.8971\\
0.95102377559439	-40836077157.6821\\
0.951123778094452	-40814304761.4671\\
0.951223780594515	-40792532365.2521\\
0.951323783094577	-40770817264.8167\\
0.95142378559464	-40749044868.6017\\
0.951523788094702	-40727329768.1663\\
0.951623790594765	-40705614667.7308\\
0.951723793094827	-40683899567.2953\\
0.95182379559489	-40662184466.8599\\
0.951923798094952	-40640469366.4244\\
0.952023800595015	-40618754265.989\\
0.952123803095077	-40597096461.333\\
0.95222380559514	-40575381360.8976\\
0.952323808095202	-40553723556.2416\\
0.952423810595265	-40532008455.8062\\
0.952523813095327	-40510350651.1502\\
0.95262381559539	-40488692846.4943\\
0.952723818095452	-40467035041.8383\\
0.952823820595515	-40445377237.1824\\
0.952923823095577	-40423719432.5264\\
0.95302382559564	-40402118923.65\\
0.953123828095702	-40380461118.9941\\
0.953223830595765	-40358860610.1176\\
0.953323833095827	-40337202805.4617\\
0.95342383559589	-40315602296.5852\\
0.953523838095952	-40294001787.7088\\
0.953623840596015	-40272401278.8324\\
0.953723843096077	-40250800769.9559\\
0.95382384559614	-40229200261.0795\\
0.953923848096202	-40207657047.9826\\
0.954023850596265	-40186056539.1062\\
0.954123853096327	-40164513326.0092\\
0.95422385559639	-40142912817.1328\\
0.954323858096452	-40121369604.0359\\
0.954423860596515	-40099826390.939\\
0.954523863096577	-40078283177.8421\\
0.95462386559664	-40056739964.7451\\
0.954723868096702	-40035196751.6482\\
0.954823870596765	-40013710834.3308\\
0.954923873096827	-39992167621.2339\\
0.95502387559689	-39970681703.9165\\
0.955123878096952	-39949138490.8196\\
0.955223880597015	-39927652573.5022\\
0.955323883097077	-39906166656.1848\\
0.95542388559714	-39884680738.8674\\
0.955523888097202	-39863194821.5499\\
0.955623890597265	-39841708904.2325\\
0.955723893097327	-39820280282.6946\\
0.95582389559739	-39798794365.3772\\
0.955923898097452	-39777308448.0598\\
0.956023900597515	-39755879826.5219\\
0.956123903097577	-39734451204.9841\\
0.95622390559764	-39713022583.4462\\
0.956323908097702	-39691593961.9083\\
0.956423910597765	-39670165340.3704\\
0.956523913097827	-39648736718.8325\\
0.95662391559789	-39627308097.2946\\
0.956723918097952	-39605936771.5362\\
0.956823920598015	-39584508149.9983\\
0.956923923098078	-39563136824.2399\\
0.95702392559814	-39541708202.702\\
0.957123928098202	-39520336876.9437\\
0.957223930598265	-39498965551.1853\\
0.957323933098327	-39477594225.4269\\
0.95742393559839	-39456222899.6685\\
0.957523938098452	-39434908869.6897\\
0.957623940598515	-39413537543.9313\\
0.957723943098577	-39392223513.9524\\
0.95782394559864	-39370852188.194\\
0.957923948098703	-39349538158.2152\\
0.958023950598765	-39328224128.2363\\
0.958123953098827	-39306910098.2574\\
0.95822395559889	-39285596068.2786\\
0.958323958098952	-39264282038.2997\\
0.958423960599015	-39242968008.3208\\
0.958523963099078	-39221653978.342\\
0.95862396559914	-39200397244.1426\\
0.958723968099202	-39179083214.1637\\
0.958823970599265	-39157826479.9644\\
0.958923973099328	-39136569745.765\\
0.95902397559939	-39115313011.5657\\
0.959123978099452	-39094056277.3663\\
0.959223980599515	-39072799543.167\\
0.959323983099577	-39051542808.9676\\
0.95942398559964	-39030343370.5478\\
0.959523988099703	-39009086636.3484\\
0.959623990599765	-38987887197.9286\\
0.959723993099827	-38966630463.7292\\
0.95982399559989	-38945431025.3094\\
0.959923998099953	-38924231586.8896\\
0.960024000600015	-38903032148.4697\\
0.960124003100078	-38881832710.0499\\
0.96022400560014	-38860633271.63\\
0.960324008100202	-38839491128.9897\\
0.960424010600265	-38818291690.5699\\
0.960524013100328	-38797149547.9295\\
0.96062401560039	-38775950109.5097\\
0.960724018100452	-38754807966.8694\\
0.960824020600515	-38733665824.229\\
0.960924023100578	-38712523681.5887\\
0.96102402560064	-38691381538.9484\\
0.961124028100703	-38670239396.3081\\
0.961224030600765	-38649154549.4472\\
0.961324033100827	-38628012406.8069\\
0.96142403560089	-38606927559.9461\\
0.961524038100953	-38585785417.3058\\
0.961624040601015	-38564700570.445\\
0.961724043101078	-38543615723.5842\\
0.96182404560114	-38522530876.7233\\
0.961924048101203	-38501446029.8625\\
0.962024050601265	-38480361183.0017\\
0.962124053101328	-38459276336.1409\\
0.96222405560139	-38438248785.0596\\
0.962324058101452	-38417163938.1988\\
0.962424060601515	-38396136387.1175\\
0.962524063101578	-38375108836.0362\\
0.96262406560164	-38354023989.1754\\
0.962724068101703	-38332996438.0941\\
0.962824070601765	-38311968887.0128\\
0.962924073101828	-38290998631.711\\
0.96302407560189	-38269971080.6297\\
0.963124078101953	-38248943529.5484\\
0.963224080602015	-38227973274.2466\\
0.963324083102078	-38206945723.1653\\
0.96342408560214	-38185975467.8635\\
0.963524088102203	-38165005212.5617\\
0.963624090602265	-38144034957.2599\\
0.963724093102328	-38123064701.9581\\
0.96382409560239	-38102094446.6563\\
0.963924098102453	-38081124191.3545\\
0.964024100602515	-38060211231.8323\\
0.964124103102578	-38039240976.5305\\
0.96422410560264	-38018328017.0082\\
0.964324108102703	-37997357761.7064\\
0.964424110602765	-37976444802.1842\\
0.964524113102828	-37955531842.6619\\
0.96462411560289	-37934618883.1396\\
0.964724118102953	-37913705923.6173\\
0.964824120603015	-37892850259.8746\\
0.964924123103078	-37871937300.3523\\
0.96502412560314	-37851024340.83\\
0.965124128103203	-37830168677.0872\\
0.965224130603265	-37809313013.3445\\
0.965324133103328	-37788400053.8222\\
0.96542413560339	-37767544390.0795\\
0.965524138103453	-37746688726.3367\\
0.965624140603515	-37725833062.5939\\
0.965724143103578	-37705034694.6307\\
0.96582414560364	-37684179030.8879\\
0.965924148103703	-37663323367.1451\\
0.966024150603765	-37642524999.1819\\
0.966124153103828	-37621726631.2187\\
0.96622415560389	-37600870967.4759\\
0.966324158103953	-37580072599.5126\\
0.966424160604015	-37559274231.5494\\
0.966524163104078	-37538475863.5861\\
0.96662416560414	-37517677495.6229\\
0.966724168104203	-37496936423.4392\\
0.966824170604265	-37476138055.4759\\
0.966924173104328	-37455396983.2922\\
0.96702417560439	-37434598615.3289\\
0.967124178104453	-37413857543.1452\\
0.967224180604515	-37393116470.9615\\
0.967324183104578	-37372375398.7777\\
0.96742418560464	-37351634326.594\\
0.967524188104703	-37330893254.4102\\
0.967624190604765	-37310152182.2265\\
0.967724193104828	-37289468405.8223\\
0.96782419560489	-37268727333.6386\\
0.967924198104953	-37248043557.2343\\
0.968024200605015	-37227359780.8301\\
0.968124203105078	-37206618708.6464\\
0.96822420560514	-37185934932.2421\\
0.968324208105203	-37165251155.8379\\
0.968424210605265	-37144624675.2132\\
0.968524213105328	-37123940898.809\\
0.96862421560539	-37103257122.4048\\
0.968724218105453	-37082630641.7801\\
0.968824220605515	-37061946865.3758\\
0.968924223105578	-37041320384.7511\\
0.96902422560564	-37020693904.1264\\
0.969124228105703	-37000067423.5017\\
0.969224230605765	-36979440942.877\\
0.969324233105828	-36958814462.2523\\
0.96942423560589	-36938187981.6276\\
0.969524238105953	-36917618796.7824\\
0.969624240606015	-36896992316.1577\\
0.969724243106078	-36876423131.3125\\
0.96982424560614	-36855796650.6878\\
0.969924248106203	-36835227465.8426\\
0.970024250606265	-36814658280.9974\\
0.970124253106328	-36794089096.1522\\
0.97022425560639	-36773519911.307\\
0.970324258106453	-36752950726.4618\\
0.970424260606515	-36732438837.3961\\
0.970524263106578	-36711869652.5509\\
0.97062426560664	-36691357763.4852\\
0.970724268106703	-36670788578.64\\
0.970824270606765	-36650276689.5743\\
0.970924273106828	-36629764800.5087\\
0.97102427560689	-36609252911.443\\
0.971124278106953	-36588741022.3773\\
0.971224280607015	-36568229133.3116\\
0.971324283107078	-36547717244.2459\\
0.97142428560714	-36527262650.9598\\
0.971524288107203	-36506750761.8941\\
0.971624290607265	-36486296168.6079\\
0.971724293107328	-36465841575.3217\\
0.97182429560739	-36445386982.0356\\
0.971924298107453	-36424932388.7494\\
0.972024300607515	-36404477795.4632\\
0.972124303107578	-36384023202.177\\
0.97222430560764	-36363568608.8909\\
0.972324308107703	-36343114015.6047\\
0.972424310607765	-36322716718.0981\\
0.972524313107828	-36302319420.5914\\
0.97262431560789	-36281864827.3052\\
0.972724318107953	-36261467529.7986\\
0.972824320608015	-36241070232.2919\\
0.972924323108078	-36220672934.7853\\
0.97302432560814	-36200275637.2786\\
0.973124328108203	-36179878339.7719\\
0.973224330608265	-36159538338.0448\\
0.973324333108328	-36139141040.5381\\
0.97342433560839	-36118801038.811\\
0.973524338108453	-36098461037.0838\\
0.973624340608515	-36078063739.5772\\
0.973724343108578	-36057723737.85\\
0.97382434560864	-36037383736.1229\\
0.973924348108703	-36017043734.3958\\
0.974024350608765	-35996761028.4481\\
0.974124353108828	-35976421026.721\\
0.97422435560889	-35956081024.9938\\
0.974324358108953	-35935798319.0462\\
0.974424360609015	-35915515613.0986\\
0.974524363109078	-35895175611.3714\\
0.97462436560914	-35874892905.4238\\
0.974724368109203	-35854610199.4762\\
0.974824370609265	-35834327493.5285\\
0.974924373109328	-35814102083.3604\\
0.97502437560939	-35793819377.4128\\
0.975124378109453	-35773536671.4652\\
0.975224380609515	-35753311261.297\\
0.975324383109578	-35733028555.3494\\
0.97542438560964	-35712803145.1813\\
0.975524388109703	-35692577735.0132\\
0.975624390609765	-35672352324.8451\\
0.975724393109828	-35652126914.6769\\
0.97582439560989	-35631901504.5088\\
0.975924398109953	-35611733390.1202\\
0.976024400610015	-35591507979.9521\\
0.976124403110078	-35571282569.784\\
0.97622440561014	-35551114455.3954\\
0.976324408110203	-35530946341.0068\\
0.976424410610265	-35510778226.6182\\
0.976524413110328	-35490610112.2296\\
0.97662441561039	-35470441997.841\\
0.976724418110453	-35450273883.4523\\
0.976824420610515	-35430105769.0637\\
0.976924423110578	-35409937654.6751\\
0.97702442561064	-35389826836.066\\
0.977124428110703	-35369658721.6774\\
0.977224430610765	-35349547903.0684\\
0.977324433110828	-35329437084.4593\\
0.97742443561089	-35309326265.8502\\
0.977524438110953	-35289215447.2411\\
0.977624440611015	-35269104628.632\\
0.977724443111078	-35248993810.0229\\
0.97782444561114	-35228940287.1933\\
0.977924448111203	-35208829468.5842\\
0.978024450611265	-35188775945.7546\\
0.978124453111328	-35168665127.1455\\
0.97822445561139	-35148611604.316\\
0.978324458111453	-35128558081.4864\\
0.978424460611515	-35108504558.6568\\
0.978524463111578	-35088451035.8272\\
0.97862446561164	-35068397512.9977\\
0.978724468111703	-35048401285.9476\\
0.978824470611765	-35028347763.118\\
0.978924473111828	-35008351536.0679\\
0.97902447561189	-34988298013.2384\\
0.979124478111953	-34968301786.1883\\
0.979224480612015	-34948305559.1382\\
0.979324483112078	-34928309332.0882\\
0.97942448561214	-34908313105.0381\\
0.979524488112203	-34888316877.988\\
0.979624490612265	-34868320650.938\\
0.979724493112328	-34848381719.6674\\
0.97982449561239	-34828385492.6174\\
0.979924498112453	-34808446561.3468\\
0.980024500612515	-34788450334.2967\\
0.980124503112578	-34768511403.0262\\
0.98022450561264	-34748572471.7556\\
0.980324508112703	-34728633540.4851\\
0.980424510612765	-34708694609.2145\\
0.980524513112828	-34688812973.7235\\
0.98062451561289	-34668874042.4529\\
0.980724518112953	-34648992406.9619\\
0.980824520613015	-34629053475.6913\\
0.980924523113078	-34609171840.2003\\
0.98102452561314	-34589290204.7093\\
0.981124528113203	-34569351273.4387\\
0.981224530613265	-34549469637.9477\\
0.981324533113328	-34529645298.2361\\
0.98142453561339	-34509763662.7451\\
0.981524538113453	-34489882027.2541\\
0.981624540613515	-34470057687.5425\\
0.981724543113578	-34450176052.0515\\
0.98182454561364	-34430351712.34\\
0.981924548113703	-34410527372.6284\\
0.982024550613765	-34390645737.1374\\
0.982124553113828	-34370821397.4259\\
0.98222455561389	-34351054353.4939\\
0.982324558113953	-34331230013.7823\\
0.982424560614015	-34311405674.0708\\
0.982524563114078	-34291581334.3593\\
0.98262456561414	-34271814290.4273\\
0.982724568114203	-34252047246.4953\\
0.982824570614265	-34232222906.7837\\
0.982924573114328	-34212455862.8517\\
0.98302457561439	-34192688818.9197\\
0.983124578114453	-34172921774.9877\\
0.983224580614515	-34153154731.0557\\
0.983324583114578	-34133444982.9032\\
0.98342458561464	-34113677938.9712\\
0.983524588114703	-34093910895.0392\\
0.983624590614765	-34074201146.8867\\
0.983724593114828	-34054491398.7341\\
0.98382459561489	-34034781650.5816\\
0.983924598114953	-34015071902.4291\\
0.984024600615015	-33995362154.2766\\
0.984124603115078	-33975652406.1241\\
0.98422460561514	-33955942657.9716\\
0.984324608115203	-33936232909.8191\\
0.984424610615265	-33916580457.4462\\
0.984524613115328	-33896870709.2937\\
0.98462461561539	-33877218256.9207\\
0.984724618115453	-33857565804.5477\\
0.984824620615515	-33837913352.1747\\
0.984924623115578	-33818260899.8017\\
0.98502462561564	-33798608447.4287\\
0.985124628115703	-33778955995.0557\\
0.985224630615765	-33759303542.6827\\
0.985324633115828	-33739708386.0893\\
0.98542463561589	-33720055933.7163\\
0.985524638115953	-33700460777.1228\\
0.985624640616015	-33680865620.5293\\
0.985724643116078	-33661213168.1564\\
0.98582464561614	-33641618011.5629\\
0.985924648116203	-33622080150.7489\\
0.986024650616265	-33602484994.1554\\
0.986124653116328	-33582889837.562\\
0.98622465561639	-33563294680.9685\\
0.986324658116453	-33543756820.1545\\
0.986424660616515	-33524161663.5611\\
0.986524663116578	-33504623802.7471\\
0.98662466561664	-33485085941.9331\\
0.986724668116703	-33465548081.1192\\
0.986824670616765	-33446010220.3052\\
0.986924673116828	-33426472359.4913\\
0.98702467561689	-33406934498.6773\\
0.987124678116953	-33387453933.6428\\
0.987224680617015	-33367916072.8289\\
0.987324683117078	-33348435507.7944\\
0.98742468561714	-33328897646.9805\\
0.987524688117203	-33309417081.946\\
0.987624690617265	-33289936516.9116\\
0.987724693117328	-33270455951.8771\\
0.98782469561739	-33250975386.8427\\
0.987924698117453	-33231552117.5877\\
0.988024700617515	-33212071552.5533\\
0.988124703117578	-33192590987.5189\\
0.98822470561764	-33173167718.2639\\
0.988324708117703	-33153687153.2295\\
0.988424710617765	-33134263883.9745\\
0.988524713117828	-33114840614.7196\\
0.98862471561789	-33095417345.4647\\
0.988724718117953	-33075994076.2097\\
0.988824720618015	-33056570806.9548\\
0.988924723118078	-33037204833.4794\\
0.98902472561814	-33017781564.2244\\
0.989124728118203	-32998415590.749\\
0.989224730618265	-32978992321.4941\\
0.989324733118328	-32959626348.0187\\
0.989424735618391	-32940260374.5432\\
0.989524738118453	-32920894401.0678\\
0.989624740618515	-32901528427.5924\\
0.989724743118578	-32882162454.117\\
0.98982474561864	-32862796480.6415\\
0.989924748118703	-32843430507.1661\\
0.990024750618765	-32824121829.4702\\
0.990124753118828	-32804755855.9948\\
0.99022475561889	-32785447178.2989\\
0.990324758118953	-32766138500.603\\
0.990424760619016	-32746829822.9071\\
0.990524763119078	-32727521145.2112\\
0.99062476561914	-32708212467.5153\\
0.990724768119203	-32688903789.8193\\
0.990824770619265	-32669652407.903\\
0.990924773119328	-32650343730.207\\
0.991024775619391	-32631092348.2906\\
0.991124778119453	-32611783670.5947\\
0.991224780619515	-32592532288.6783\\
0.991324783119578	-32573280906.7619\\
0.991424785619641	-32554029524.8456\\
0.991524788119703	-32534778142.9292\\
0.991624790619765	-32515526761.0128\\
0.991724793119828	-32496275379.0964\\
0.99182479561989	-32477081292.9595\\
0.991924798119953	-32457829911.0431\\
0.992024800620016	-32438635824.9062\\
0.992124803120078	-32419441738.7693\\
0.99222480562014	-32400190356.8529\\
0.992324808120203	-32380996270.716\\
0.992424810620266	-32361802184.5792\\
0.992524813120328	-32342665394.2218\\
0.992624815620391	-32323471308.0849\\
0.992724818120453	-32304277221.948\\
0.992824820620515	-32285140431.5907\\
0.992924823120578	-32265946345.4538\\
0.993024825620641	-32246809555.0964\\
0.993124828120703	-32227672764.739\\
0.993224830620765	-32208535974.3817\\
0.993324833120828	-32189399184.0243\\
0.993424835620891	-32170262393.6669\\
0.993524838120953	-32151125603.3096\\
0.993624840621016	-32131988812.9522\\
0.993724843121078	-32112909318.3743\\
0.99382484562114	-32093772528.017\\
0.993924848121203	-32074693033.4391\\
0.994024850621266	-32055613538.8612\\
0.994124853121328	-32036476748.5039\\
0.994224855621391	-32017397253.926\\
0.994324858121453	-31998375055.1277\\
0.994424860621516	-31979295560.5498\\
0.994524863121578	-31960216065.972\\
0.994624865621641	-31941136571.3941\\
0.994724868121703	-31922114372.5958\\
0.994824870621766	-31903092173.7974\\
0.994924873121828	-31884012679.2196\\
0.995024875621891	-31864990480.4212\\
0.995124878121953	-31845968281.6229\\
0.995224880622016	-31826946082.8245\\
0.995324883122078	-31807923884.0262\\
0.995424885622141	-31788901685.2278\\
0.995524888122203	-31769936782.209\\
0.995624890622266	-31750914583.4107\\
0.995724893122328	-31731949680.3918\\
0.995824895622391	-31712927481.5935\\
0.995924898122453	-31693962578.5747\\
0.996024900622516	-31674997675.5558\\
0.996124903122578	-31656032772.537\\
0.996224905622641	-31637067869.5182\\
0.996324908122703	-31618102966.4993\\
0.996424910622766	-31599195359.26\\
0.996524913122828	-31580230456.2412\\
0.996624915622891	-31561322849.0019\\
0.996724918122953	-31542357945.9831\\
0.996824920623016	-31523450338.7437\\
0.996924923123078	-31504542731.5044\\
0.997024925623141	-31485635124.2651\\
0.997124928123203	-31466727517.0258\\
0.997224930623266	-31447819909.7865\\
0.997324933123328	-31428912302.5471\\
0.997424935623391	-31410061991.0873\\
0.997524938123453	-31391154383.848\\
0.997624940623516	-31372304072.3882\\
0.997724943123578	-31353453760.9284\\
0.997824945623641	-31334546153.6891\\
0.997924948123703	-31315695842.2293\\
0.998024950623766	-31296845530.7695\\
0.998124953123828	-31278052515.0892\\
0.998224955623891	-31259202203.6294\\
0.998324958123953	-31240351892.1696\\
0.998424960624016	-31221558876.4893\\
0.998524963124078	-31202708565.0295\\
0.998624965624141	-31183915549.3492\\
0.998724968124203	-31165122533.6689\\
0.998824970624266	-31146272222.2091\\
0.998924973124328	-31127479206.5288\\
0.999024975624391	-31108743486.628\\
0.999124978124453	-31089950470.9478\\
0.999224980624516	-31071157455.2675\\
0.999324983124578	-31052364439.5872\\
0.999424985624641	-31033628719.6864\\
0.999524988124703	-31014892999.7856\\
0.999624990624766	-30996099984.1053\\
0.999724993124828	-30977364264.2045\\
0.999824995624891	-30958628544.3038\\
0.999924998124953	-30939892824.403\\
1.00002500062502	-30921157104.5022\\
1.00012500312508	-30902478680.381\\
1.00022500562514	-30883742960.4802\\
1.0003250081252	-30865007240.5794\\
1.00042501062527	-30846328816.4581\\
1.00052501312533	-30827650392.3369\\
1.00062501562539	-30808914672.4361\\
1.00072501812545	-30790236248.3148\\
1.00082502062552	-30771557824.1936\\
1.00092502312558	-30752879400.0723\\
1.00102502562564	-30734258271.7305\\
1.0011250281257	-30715579847.6093\\
1.00122503062577	-30696901423.488\\
1.00132503312583	-30678280295.1463\\
1.00142503562589	-30659659166.8045\\
1.00152503812595	-30640980742.6832\\
1.00162504062602	-30622359614.3415\\
1.00172504312608	-30603738485.9997\\
1.00182504562614	-30585117357.658\\
1.0019250481262	-30566496229.3162\\
1.00202505062627	-30547932396.754\\
1.00212505312633	-30529311268.4122\\
1.00222505562639	-30510747435.85\\
1.00232505812645	-30492126307.5083\\
1.00242506062652	-30473562474.946\\
1.00252506312658	-30454998642.3838\\
1.00262506562664	-30436434809.8215\\
1.0027250681267	-30417870977.2593\\
1.00282507062677	-30399307144.6971\\
1.00292507312683	-30380743312.1348\\
1.00302507562689	-30362179479.5726\\
1.00312507812695	-30343672942.7899\\
1.00322508062702	-30325109110.2276\\
1.00332508312708	-30306602573.4449\\
1.00342508562714	-30288096036.6622\\
1.0035250881272	-30269589499.8794\\
1.00362509062727	-30251082963.0967\\
1.00372509312733	-30232576426.314\\
1.00382509562739	-30214069889.5313\\
1.00392509812745	-30195563352.7485\\
1.00402510062752	-30177114111.7453\\
1.00412510312758	-30158607574.9626\\
1.00422510562764	-30140158333.9594\\
1.0043251081277	-30121709092.9562\\
1.00442511062777	-30103259851.953\\
1.00452511312783	-30084753315.1702\\
1.00462511562789	-30066304074.167\\
1.00472511812795	-30047912128.9433\\
1.00482512062802	-30029462887.9401\\
1.00492512312808	-30011013646.9369\\
1.00502512562814	-29992621701.7132\\
1.0051251281282	-29974172460.71\\
1.00522513062827	-29955780515.4863\\
1.00532513312833	-29937388570.2626\\
1.00542513562839	-29918996625.0389\\
1.00552513812845	-29900604679.8152\\
1.00562514062852	-29882212734.5915\\
1.00572514312858	-29863820789.3678\\
1.00582514562864	-29845428844.1441\\
1.0059251481287	-29827094194.6999\\
1.00602515062877	-29808702249.4762\\
1.00612515312883	-29790367600.032\\
1.00622515562889	-29772032950.5878\\
1.00632515812895	-29753698301.1437\\
1.00642516062902	-29735306355.92\\
1.00652516312908	-29717029002.2553\\
1.00662516562914	-29698694352.8111\\
1.0067251681292	-29680359703.3669\\
1.00682517062927	-29662025053.9227\\
1.00692517312933	-29643747700.258\\
1.00702517562939	-29625413050.8139\\
1.00712517812945	-29607135697.1492\\
1.00722518062952	-29588858343.4845\\
1.00732518312958	-29570580989.8198\\
1.00742518562964	-29552303636.1552\\
1.0075251881297	-29534026282.4905\\
1.00762519062977	-29515748928.8258\\
1.00772519312983	-29497528870.9407\\
1.00782519562989	-29479251517.276\\
1.00792519812995	-29461031459.3908\\
1.00802520063002	-29442754105.7262\\
1.00812520313008	-29424534047.841\\
1.00822520563014	-29406313989.9558\\
1.0083252081302	-29388093932.0707\\
1.00842521063027	-29369873874.1855\\
1.00852521313033	-29351653816.3004\\
1.00862521563039	-29333433758.4152\\
1.00872521813045	-29315270996.3095\\
1.00882522063052	-29297050938.4244\\
1.00892522313058	-29278888176.3187\\
1.00902522563064	-29260725414.2131\\
1.0091252281307	-29242505356.3279\\
1.00922523063077	-29224342594.2223\\
1.00932523313083	-29206179832.1166\\
1.00942523563089	-29188017070.011\\
1.00952523813095	-29169911603.6849\\
1.00962524063102	-29151748841.5792\\
1.00972524313108	-29133643375.2531\\
1.00982524563114	-29115480613.1474\\
1.0099252481312	-29097375146.8213\\
1.01002525063127	-29079269680.4952\\
1.01012525313133	-29061106918.3895\\
1.01022525563139	-29043001452.0634\\
1.01032525813145	-29024895985.7372\\
1.01042526063152	-29006847815.1906\\
1.01052526313158	-28988742348.8645\\
1.01062526563164	-28970636882.5384\\
1.0107252681317	-28952588711.9917\\
1.01082527063177	-28934483245.6656\\
1.01092527313183	-28916435075.119\\
1.01102527563189	-28898386904.5724\\
1.01112527813195	-28880338734.0257\\
1.01122528063202	-28862290563.4791\\
1.01132528313208	-28844242392.9325\\
1.01142528563214	-28826194222.3859\\
1.0115252881322	-28808203347.6188\\
1.01162529063227	-28790155177.0721\\
1.01172529313233	-28772164302.305\\
1.01182529563239	-28754116131.7584\\
1.01192529813245	-28736125256.9913\\
1.01202530063252	-28718134382.2242\\
1.01212530313258	-28700143507.4571\\
1.01222530563264	-28682152632.69\\
1.0123253081327	-28664161757.9229\\
1.01242531063277	-28646228178.9353\\
1.01252531313283	-28628237304.1682\\
1.01262531563289	-28610303725.1806\\
1.01272531813295	-28592312850.4135\\
1.01282532063302	-28574379271.4259\\
1.01292532313308	-28556445692.4383\\
1.01302532563314	-28538512113.4507\\
1.0131253281332	-28520578534.4631\\
1.01322533063327	-28502644955.4755\\
1.01332533313333	-28484711376.4879\\
1.01342533563339	-28466835093.2798\\
1.01352533813345	-28448901514.2922\\
1.01362534063352	-28431025231.0841\\
1.01372534313358	-28413091652.0965\\
1.01382534563364	-28395215368.8885\\
1.0139253481337	-28377339085.6804\\
1.01402535063377	-28359462802.4723\\
1.01412535313383	-28341586519.2642\\
1.01422535563389	-28323710236.0561\\
1.01432535813395	-28305891248.6276\\
1.01442536063402	-28288014965.4195\\
1.01452536313408	-28270195977.9909\\
1.01462536563414	-28252319694.7828\\
1.0147253681342	-28234500707.3543\\
1.01482537063427	-28216681719.9257\\
1.01492537313433	-28198862732.4971\\
1.01502537563439	-28181043745.0686\\
1.01512537813445	-28163224757.64\\
1.01522538063452	-28145405770.2114\\
1.01532538313458	-28127586782.7829\\
1.01542538563464	-28109825091.1338\\
1.0155253881347	-28092063399.4847\\
1.01562539063477	-28074244412.0562\\
1.01572539313483	-28056482720.4071\\
1.01582539563489	-28038721028.7581\\
1.01592539813495	-28020959337.109\\
1.01602540063502	-28003197645.46\\
1.01612540313508	-27985435953.8109\\
1.01622540563514	-27967731557.9414\\
1.0163254081352	-27949969866.2923\\
1.01642541063527	-27932208174.6432\\
1.01652541313533	-27914503778.7737\\
1.01662541563539	-27896799382.9042\\
1.01672541813545	-27879094987.0346\\
1.01682542063552	-27861390591.1651\\
1.01692542313558	-27843686195.2955\\
1.01702542563564	-27825981799.426\\
1.0171254281357	-27808277403.5565\\
1.01722543063577	-27790573007.6869\\
1.01732543313583	-27772925907.5969\\
1.01742543563589	-27755221511.7273\\
1.01752543813595	-27737574411.6373\\
1.01762544063602	-27719927311.5473\\
1.01772544313608	-27702280211.4573\\
1.01782544563614	-27684633111.3672\\
1.0179254481362	-27666986011.2772\\
1.01802545063627	-27649338911.1872\\
1.01812545313633	-27631691811.0971\\
1.01822545563639	-27614102006.7866\\
1.01832545813645	-27596454906.6966\\
1.01842546063652	-27578865102.3861\\
1.01852546313658	-27561275298.0756\\
1.01862546563664	-27543685493.765\\
1.0187254681367	-27526095689.4545\\
1.01882547063677	-27508505885.144\\
1.01892547313683	-27490916080.8335\\
1.01902547563689	-27473326276.523\\
1.01912547813695	-27455736472.2125\\
1.01922548063702	-27438203963.6815\\
1.01932548313708	-27420614159.3709\\
1.01942548563714	-27403081650.8399\\
1.0195254881372	-27385549142.3089\\
1.01962549063727	-27368016633.7779\\
1.01972549313733	-27350484125.2469\\
1.01982549563739	-27332951616.7159\\
1.01992549813745	-27315419108.1849\\
1.02002550063752	-27297886599.6539\\
1.02012550313758	-27280411386.9024\\
1.02022550563764	-27262878878.3714\\
1.0203255081377	-27245403665.6199\\
1.02042551063777	-27227928452.8684\\
1.02052551313783	-27210395944.3374\\
1.02062551563789	-27192920731.5859\\
1.02072551813795	-27175445518.8345\\
1.02082552063802	-27158027601.8625\\
1.02092552313808	-27140552389.111\\
1.02102552563814	-27123077176.3595\\
1.0211255281382	-27105659259.3875\\
1.02122553063827	-27088184046.636\\
1.02132553313833	-27070766129.6641\\
1.02142553563839	-27053348212.6921\\
1.02152553813845	-27035930295.7201\\
1.02162554063852	-27018512378.7481\\
1.02172554313858	-27001094461.7761\\
1.02182554563864	-26983676544.8042\\
1.0219255481387	-26966258627.8322\\
1.02202555063877	-26948898006.6397\\
1.02212555313883	-26931480089.6678\\
1.02222555563889	-26914119468.4753\\
1.02232555813895	-26896701551.5033\\
1.02242556063902	-26879340930.3108\\
1.02252556313908	-26861980309.1184\\
1.02262556563914	-26844619687.9259\\
1.0227255681392	-26827259066.7335\\
1.02282557063927	-26809955741.3205\\
1.02292557313933	-26792595120.128\\
1.02302557563939	-26775234498.9356\\
1.02312557813945	-26757931173.5226\\
1.02322558063952	-26740627848.1097\\
1.02332558313958	-26723267226.9172\\
1.02342558563964	-26705963901.5043\\
1.0235255881397	-26688660576.0913\\
1.02362559063977	-26671357250.6784\\
1.02372559313983	-26654111221.0449\\
1.02382559563989	-26636807895.632\\
1.02392559813995	-26619504570.219\\
1.02402560064002	-26602258540.5856\\
1.02412560314008	-26584955215.1726\\
1.02422560564014	-26567709185.5392\\
1.0243256081402	-26550463155.9058\\
1.02442561064027	-26533217126.2723\\
1.02452561314033	-26515971096.6389\\
1.02462561564039	-26498725067.0054\\
1.02472561814045	-26481479037.372\\
1.02482562064052	-26464290303.5181\\
1.02492562314058	-26447044273.8846\\
1.02502562564064	-26429855540.0307\\
1.0251256281407	-26412609510.3973\\
1.02522563064077	-26395420776.5434\\
1.02532563314083	-26378232042.6894\\
1.02542563564089	-26361043308.8355\\
1.02552563814095	-26343854574.9816\\
1.02562564064102	-26326665841.1277\\
1.02572564314108	-26309477107.2737\\
1.02582564564114	-26292345669.1993\\
1.0259256481412	-26275156935.3454\\
1.02602565064127	-26258025497.271\\
1.02612565314133	-26240894059.1966\\
1.02622565564139	-26223705325.3426\\
1.02632565814145	-26206573887.2682\\
1.02642566064152	-26189442449.1938\\
1.02652566314158	-26172311011.1194\\
1.02662566564164	-26155236868.8245\\
1.0267256681417	-26138105430.7501\\
1.02682567064177	-26120973992.6757\\
1.02692567314183	-26103899850.3808\\
1.02702567564189	-26086825708.0859\\
1.02712567814195	-26069694270.0115\\
1.02722568064202	-26052620127.7166\\
1.02732568314208	-26035545985.4217\\
1.02742568564214	-26018471843.1268\\
1.0275256881422	-26001397700.8319\\
1.02762569064227	-25984380854.3165\\
1.02772569314233	-25967306712.0216\\
1.02782569564239	-25950289865.5062\\
1.02792569814245	-25933215723.2113\\
1.02802570064252	-25916198876.6959\\
1.02812570314258	-25899182030.1805\\
1.02822570564264	-25882165183.6652\\
1.0283257081427	-25865148337.1498\\
1.02842571064277	-25848131490.6344\\
1.02852571314283	-25831114644.119\\
1.02862571564289	-25814097797.6036\\
1.02872571814295	-25797138246.8678\\
1.02882572064302	-25780121400.3524\\
1.02892572314308	-25763161849.6165\\
1.02902572564314	-25746145003.1011\\
1.0291257281432	-25729185452.3652\\
1.02922573064327	-25712225901.6294\\
1.02932573314333	-25695266350.8935\\
1.02942573564339	-25678306800.1576\\
1.02952573814345	-25661404545.2013\\
1.02962574064352	-25644444994.4654\\
1.02972574314358	-25627542739.509\\
1.02982574564364	-25610583188.7732\\
1.0299257481437	-25593680933.8168\\
1.03002575064377	-25576778678.8604\\
1.03012575314383	-25559819128.1246\\
1.03022575564389	-25542916873.1682\\
1.03032575814395	-25526014618.2118\\
1.03042576064402	-25509169659.035\\
1.03052576314408	-25492267404.0786\\
1.03062576564414	-25475365149.1223\\
1.0307257681442	-25458520189.9454\\
1.03082577064427	-25441617934.9891\\
1.03092577314433	-25424772975.8122\\
1.03102577564439	-25407928016.6354\\
1.03112577814445	-25391083057.4585\\
1.03122578064452	-25374238098.2817\\
1.03132578314458	-25357393139.1048\\
1.03142578564464	-25340548179.928\\
1.0315257881447	-25323760516.5307\\
1.03162579064477	-25306915557.3538\\
1.03172579314483	-25290127893.9565\\
1.03182579564489	-25273282934.7796\\
1.03192579814495	-25256495271.3823\\
1.03202580064502	-25239707607.985\\
1.03212580314508	-25222919944.5876\\
1.03222580564514	-25206132281.1903\\
1.0323258081452	-25189344617.793\\
1.03242581064527	-25172556954.3956\\
1.03252581314533	-25155826586.7778\\
1.03262581564539	-25139038923.3805\\
1.03272581814545	-25122308555.7627\\
1.03282582064552	-25105578188.1448\\
1.03292582314558	-25088790524.7475\\
1.03302582564564	-25072060157.1297\\
1.0331258281457	-25055329789.5119\\
1.03322583064577	-25038599421.8941\\
1.03332583314583	-25021926350.0557\\
1.03342583564589	-25005195982.4379\\
1.03352583814595	-24988465614.8201\\
1.03362584064602	-24971792542.9818\\
1.03372584314608	-24955119471.1435\\
1.03382584564614	-24938389103.5257\\
1.0339258481462	-24921716031.6874\\
1.03402585064627	-24905042959.8491\\
1.03412585314633	-24888369888.0108\\
1.03422585564639	-24871696816.1724\\
1.03432585814645	-24855023744.3341\\
1.03442586064652	-24838407968.2753\\
1.03452586314658	-24821734896.437\\
1.03462586564664	-24805119120.3782\\
1.0347258681467	-24788503344.3195\\
1.03482587064677	-24771830272.4811\\
1.03492587314683	-24755214496.4223\\
1.03502587564689	-24738598720.3636\\
1.03512587814695	-24721982944.3048\\
1.03522588064702	-24705367168.246\\
1.03532588314708	-24688808687.9667\\
1.03542588564714	-24672192911.9079\\
1.0355258881472	-24655634431.6286\\
1.03562589064727	-24639018655.5698\\
1.03572589314733	-24622460175.2905\\
1.03582589564739	-24605901695.0113\\
1.03592589814745	-24589343214.732\\
1.03602590064752	-24572784734.4527\\
1.03612590314758	-24556226254.1734\\
1.03622590564764	-24539667773.8941\\
1.0363259081477	-24523109293.6149\\
1.03642591064777	-24506608109.1151\\
1.03652591314783	-24490049628.8358\\
1.03662591564789	-24473548444.336\\
1.03672591814795	-24457047259.8363\\
1.03682592064802	-24440546075.3365\\
1.03692592314808	-24424044890.8367\\
1.03702592564814	-24407543706.337\\
1.0371259281482	-24391042521.8372\\
1.03722593064827	-24374541337.3374\\
1.03732593314833	-24358040152.8377\\
1.03742593564839	-24341596264.1174\\
1.03752593814845	-24325095079.6176\\
1.03762594064852	-24308651190.8974\\
1.03772594314858	-24292207302.1771\\
1.03782594564864	-24275763413.4569\\
1.0379259481487	-24259319524.7366\\
1.03802595064877	-24242875636.0164\\
1.03812595314883	-24226431747.2961\\
1.03822595564889	-24209987858.5759\\
1.03832595814895	-24193601265.6351\\
1.03842596064902	-24177157376.9149\\
1.03852596314908	-24160770783.9741\\
1.03862596564914	-24144384191.0334\\
1.0387259681492	-24127940302.3131\\
1.03882597064927	-24111553709.3724\\
1.03892597314933	-24095167116.4316\\
1.03902597564939	-24078837819.2704\\
1.03912597814945	-24062451226.3297\\
1.03922598064952	-24046064633.3889\\
1.03932598314958	-24029735336.2277\\
1.03942598564964	-24013348743.287\\
1.0395259881497	-23997019446.1257\\
1.03962599064977	-23980690148.9645\\
1.03972599314983	-23964303556.0238\\
1.03982599564989	-23947974258.8625\\
1.03992599814995	-23931644961.7013\\
1.04002600065002	-23915372960.3196\\
1.04012600315008	-23899043663.1584\\
1.04022600565014	-23882714365.9971\\
1.0403260081502	-23866442364.6154\\
1.04042601065027	-23850113067.4542\\
1.04052601315033	-23833841066.0725\\
1.04062601565039	-23817569064.6908\\
1.04072601815045	-23801297063.309\\
1.04082602065052	-23785025061.9273\\
1.04092602315058	-23768753060.5456\\
1.04102602565064	-23752481059.1639\\
1.0411260281507	-23736209057.7822\\
1.04122603065077	-23719994352.18\\
1.04132603315083	-23703722350.7983\\
1.04142603565089	-23687507645.1961\\
1.04152603815095	-23671292939.5939\\
1.04162604065102	-23655020938.2121\\
1.04172604315108	-23638806232.6099\\
1.04182604565114	-23622591527.0077\\
1.0419260481512	-23606376821.4055\\
1.04202605065127	-23590219411.5828\\
1.04212605315133	-23574004705.9806\\
1.04222605565139	-23557847296.158\\
1.04232605815145	-23541632590.5558\\
1.04242606065152	-23525475180.7331\\
1.04252606315158	-23509260475.1309\\
1.04262606565164	-23493103065.3082\\
1.0427260681517	-23476945655.4855\\
1.04282607065177	-23460788245.6628\\
1.04292607315183	-23444688131.6196\\
1.04302607565189	-23428530721.7969\\
1.04312607815195	-23412373311.9742\\
1.04322608065202	-23396273197.9311\\
1.04332608315208	-23380115788.1084\\
1.04342608565214	-23364015674.0652\\
1.0435260881522	-23347915560.022\\
1.04362609065227	-23331815445.9788\\
1.04372609315233	-23315715331.9357\\
1.04382609565239	-23299615217.8925\\
1.04392609815245	-23283515103.8493\\
1.04402610065252	-23267414989.8061\\
1.04412610315258	-23251372171.5425\\
1.04422610565264	-23235272057.4993\\
1.0443261081527	-23219229239.2356\\
1.04442611065277	-23203129125.1925\\
1.04452611315283	-23187086306.9288\\
1.04462611565289	-23171043488.6651\\
1.04472611815295	-23155000670.4015\\
1.04482612065302	-23138957852.1378\\
1.04492612315308	-23122972329.6537\\
1.04502612565314	-23106929511.39\\
1.0451261281532	-23090886693.1263\\
1.04522613065327	-23074901170.6422\\
1.04532613315333	-23058915648.158\\
1.04542613565339	-23042872829.8944\\
1.04552613815345	-23026887307.4102\\
1.04562614065352	-23010901784.9261\\
1.04572614315358	-22994916262.4419\\
1.04582614565364	-22978930739.9578\\
1.0459261481537	-22963002513.2531\\
1.04602615065377	-22947016990.769\\
1.04612615315383	-22931031468.2848\\
1.04622615565389	-22915103241.5802\\
1.04632615815395	-22899175014.8756\\
1.04642616065402	-22883189492.3914\\
1.04652616315408	-22867261265.6868\\
1.04662616565414	-22851333038.9821\\
1.0467261681542	-22835404812.2775\\
1.04682617065427	-22819533881.3524\\
1.04692617315433	-22803605654.6477\\
1.04702617565439	-22787677427.9431\\
1.04712617815445	-22771806497.018\\
1.04722618065452	-22755878270.3133\\
1.04732618315458	-22740007339.3882\\
1.04742618565464	-22724136408.4631\\
1.0475261881547	-22708265477.538\\
1.04762619065477	-22692394546.6128\\
1.04772619315483	-22676523615.6877\\
1.04782619565489	-22660652684.7626\\
1.04792619815495	-22644839049.617\\
1.04802620065502	-22628968118.6919\\
1.04812620315508	-22613154483.5463\\
1.04822620565514	-22597283552.6211\\
1.0483262081552	-22581469917.4755\\
1.04842621065527	-22565656282.3299\\
1.04852621315533	-22549842647.1843\\
1.04862621565539	-22534029012.0387\\
1.04872621815545	-22518215376.8931\\
1.04882622065552	-22502401741.7475\\
1.04892622315558	-22486645402.3814\\
1.04902622565564	-22470831767.2358\\
1.0491262281557	-22455075427.8697\\
1.04922623065577	-22439261792.724\\
1.04932623315583	-22423505453.358\\
1.04942623565589	-22407749113.9919\\
1.04952623815595	-22391992774.6258\\
1.04962624065602	-22376236435.2597\\
1.04972624315608	-22360480095.8936\\
1.04982624565614	-22344723756.5275\\
1.0499262481562	-22329024712.9409\\
1.05002625065627	-22313268373.5748\\
1.05012625315633	-22297569329.9882\\
1.05022625565639	-22281812990.6221\\
1.05032625815645	-22266113947.0355\\
1.05042626065652	-22250414903.4489\\
1.05052626315658	-22234715859.8623\\
1.05062626565664	-22219016816.2758\\
1.0507262681567	-22203375068.4687\\
1.05082627065677	-22187676024.8821\\
1.05092627315683	-22171976981.2955\\
1.05102627565689	-22156335233.4884\\
1.05112627815695	-22140636189.9019\\
1.05122628065702	-22124994442.0948\\
1.05132628315708	-22109352694.2877\\
1.05142628565714	-22093710946.4806\\
1.0515262881572	-22078069198.6736\\
1.05162629065727	-22062427450.8665\\
1.05172629315733	-22046785703.0594\\
1.05182629565739	-22031201251.0319\\
1.05192629815745	-22015559503.2248\\
1.05202630065752	-21999975051.1972\\
1.05212630315758	-21984333303.3902\\
1.05222630565764	-21968748851.3626\\
1.0523263081577	-21953164399.3351\\
1.05242631065777	-21937579947.3075\\
1.05252631315783	-21921995495.2799\\
1.05262631565789	-21906411043.2524\\
1.05272631815795	-21890826591.2248\\
1.05282632065802	-21875299434.9768\\
1.05292632315808	-21859714982.9492\\
1.05302632565814	-21844187826.7012\\
1.0531263281582	-21828603374.6736\\
1.05322633065827	-21813076218.4256\\
1.05332633315833	-21797549062.1775\\
1.05342633565839	-21782021905.9295\\
1.05352633815845	-21766494749.6814\\
1.05362634065852	-21750967593.4334\\
1.05372634315858	-21735440437.1853\\
1.05382634565864	-21719970576.7168\\
1.0539263481587	-21704443420.4688\\
1.05402635065877	-21688973560.0002\\
1.05412635315883	-21673503699.5317\\
1.05422635565889	-21657976543.2837\\
1.05432635815895	-21642506682.8151\\
1.05442636065902	-21627036822.3466\\
1.05452636315908	-21611566961.8781\\
1.05462636565914	-21596154397.189\\
1.0547263681592	-21580684536.7205\\
1.05482637065927	-21565214676.252\\
1.05492637315933	-21549802111.563\\
1.05502637565939	-21534389546.8739\\
1.05512637815945	-21518919686.4054\\
1.05522638065952	-21503507121.7164\\
1.05532638315958	-21488094557.0274\\
1.05542638565964	-21472681992.3383\\
1.0555263881597	-21457269427.6493\\
1.05562639065977	-21441856862.9603\\
1.05572639315983	-21426501594.0508\\
1.05582639565989	-21411089029.3618\\
1.05592639815995	-21395733760.4523\\
1.05602640066002	-21380321195.7633\\
1.05612640316008	-21364965926.8538\\
1.05622640566014	-21349610657.9442\\
1.0563264081602	-21334255389.0347\\
1.05642641066027	-21318900120.1252\\
1.05652641316033	-21303544851.2157\\
1.05662641566039	-21288189582.3062\\
1.05672641816045	-21272891609.1762\\
1.05682642066052	-21257536340.2667\\
1.05692642316058	-21242238367.1367\\
1.05702642566064	-21226883098.2272\\
1.0571264281607	-21211585125.0972\\
1.05722643066077	-21196287151.9672\\
1.05732643316083	-21180989178.8372\\
1.05742643566089	-21165691205.7073\\
1.05752643816095	-21150393232.5773\\
1.05762644066102	-21135152555.2268\\
1.05772644316108	-21119854582.0968\\
1.05782644566114	-21104556608.9668\\
1.0579264481612	-21089315931.6163\\
1.05802645066127	-21074075254.2658\\
1.05812645316133	-21058777281.1358\\
1.05822645566139	-21043536603.7854\\
1.05832645816145	-21028295926.4349\\
1.05842646066152	-21013055249.0844\\
1.05852646316158	-20997871867.5134\\
1.05862646566164	-20982631190.163\\
1.0587264681617	-20967390512.8125\\
1.05882647066177	-20952207131.2415\\
1.05892647316183	-20936966453.891\\
1.05902647566189	-20921783072.3201\\
1.05912647816195	-20906599690.7491\\
1.05922648066202	-20891416309.1781\\
1.05932648316208	-20876232927.6072\\
1.05942648566214	-20861049546.0362\\
1.0595264881622	-20845866164.4652\\
1.05962649066227	-20830740078.6738\\
1.05972649316233	-20815556697.1028\\
1.05982649566239	-20800430611.3114\\
1.05992649816245	-20785247229.7404\\
1.06002650066252	-20770121143.9489\\
1.06012650316258	-20754995058.1575\\
1.06022650566264	-20739868972.366\\
1.0603265081627	-20724742886.5746\\
1.06042651066277	-20709616800.7831\\
1.06052651316283	-20694490714.9917\\
1.06062651566289	-20679364629.2002\\
1.06072651816295	-20664295839.1883\\
1.06082652066302	-20649169753.3968\\
1.06092652316308	-20634100963.3849\\
1.06102652566314	-20619032173.3729\\
1.0611265281632	-20603963383.361\\
1.06122653066327	-20588894593.3491\\
1.06132653316333	-20573825803.3371\\
1.06142653566339	-20558757013.3252\\
1.06152653816345	-20543688223.3132\\
1.06162654066352	-20528676729.0808\\
1.06172654316358	-20513607939.0689\\
1.06182654566364	-20498596444.8364\\
1.0619265481637	-20483527654.8245\\
1.06202655066377	-20468516160.5921\\
1.06212655316383	-20453504666.3596\\
1.06222655566389	-20438493172.1272\\
1.06232655816395	-20423481677.8948\\
1.06242656066402	-20408470183.6624\\
1.06252656316408	-20393458689.4299\\
1.06262656566414	-20378504490.977\\
1.0627265681642	-20363492996.7446\\
1.06282657066427	-20348538798.2917\\
1.06292657316433	-20333584599.8388\\
1.06302657566439	-20318573105.6063\\
1.06312657816445	-20303618907.1534\\
1.06322658066452	-20288664708.7005\\
1.06332658316458	-20273710510.2476\\
1.06342658566464	-20258813607.5742\\
1.0635265881647	-20243859409.1213\\
1.06362659066477	-20228905210.6684\\
1.06372659316483	-20214008307.995\\
1.06382659566489	-20199054109.542\\
1.06392659816495	-20184157206.8686\\
1.06402660066502	-20169260304.1952\\
1.06412660316508	-20154363401.5218\\
1.06422660566514	-20139466498.8484\\
1.0643266081652	-20124569596.175\\
1.06442661066527	-20109672693.5016\\
1.06452661316533	-20094775790.8282\\
1.06462661566539	-20079936183.9343\\
1.06472661816545	-20065039281.2609\\
1.06482662066552	-20050199674.3671\\
1.06492662316558	-20035360067.4732\\
1.06502662566564	-20020520460.5793\\
1.0651266281657	-20005623557.9059\\
1.06522663066577	-19990783951.012\\
1.06532663316583	-19976001639.8976\\
1.06542663566589	-19961162033.0037\\
1.06552663816595	-19946322426.1098\\
1.06562664066602	-19931540114.9955\\
1.06572664316608	-19916700508.1016\\
1.06582664566614	-19901918196.9872\\
1.0659266481662	-19887135885.8728\\
1.06602665066627	-19872296278.9789\\
1.06612665316633	-19857513967.8646\\
1.06622665566639	-19842731656.7502\\
1.06632665816645	-19828006641.4153\\
1.06642666066652	-19813224330.3009\\
1.06652666316658	-19798442019.1866\\
1.06662666566664	-19783717003.8517\\
1.0667266681667	-19768934692.7373\\
1.06682667066677	-19754209677.4025\\
1.06692667316683	-19739484662.0676\\
1.06702667566689	-19724702350.9532\\
1.06712667816695	-19709977335.6184\\
1.06722668066702	-19695252320.2835\\
1.06732668316708	-19680584600.7282\\
1.06742668566714	-19665859585.3933\\
1.0675266881672	-19651134570.0584\\
1.06762669066727	-19636466850.5031\\
1.06772669316733	-19621741835.1682\\
1.06782669566739	-19607074115.6129\\
1.06792669816745	-19592406396.0575\\
1.06802670066752	-19577738676.5022\\
1.06812670316758	-19563070956.9468\\
1.06822670566764	-19548403237.3915\\
1.0683267081677	-19533735517.8361\\
1.06842671066777	-19519067798.2808\\
1.06852671316783	-19504400078.7254\\
1.06862671566789	-19489789654.9496\\
1.06872671816795	-19475179231.1738\\
1.06882672066802	-19460511511.6184\\
1.06892672316808	-19445901087.8426\\
1.06902672566814	-19431290664.0667\\
1.0691267281682	-19416680240.2909\\
1.06922673066827	-19402069816.5151\\
1.06932673316833	-19387459392.7392\\
1.06942673566839	-19372848968.9634\\
1.06952673816845	-19358295840.9671\\
1.06962674066852	-19343685417.1912\\
1.06972674316858	-19329132289.1949\\
1.06982674566864	-19314579161.1986\\
1.0699267481687	-19299968737.4228\\
1.07002675066877	-19285415609.4264\\
1.07012675316883	-19270862481.4301\\
1.07022675566889	-19256309353.4338\\
1.07032675816895	-19241813521.217\\
1.07042676066902	-19227260393.2207\\
1.07052676316908	-19212707265.2243\\
1.07062676566914	-19198211433.0075\\
1.0707267681692	-19183715600.7907\\
1.07082677066927	-19169162472.7944\\
1.07092677316933	-19154666640.5776\\
1.07102677566939	-19140170808.3608\\
1.07112677816945	-19125674976.144\\
1.07122678066952	-19111179143.9271\\
1.07132678316958	-19096683311.7103\\
1.07142678566964	-19082244775.273\\
1.0715267881697	-19067748943.0562\\
1.07162679066977	-19053310406.6189\\
1.07172679316983	-19038814574.4021\\
1.07182679566989	-19024376037.9648\\
1.07192679816995	-19009937501.5275\\
1.07202680067002	-18995498965.0902\\
1.07212680317008	-18981060428.6529\\
1.07222680567014	-18966621892.2156\\
1.0723268081702	-18952183355.7783\\
1.07242681067027	-18937744819.341\\
1.07252681317033	-18923363578.6833\\
1.07262681567039	-18908925042.246\\
1.07272681817045	-18894543801.5882\\
1.07282682067052	-18880162560.9304\\
1.07292682317058	-18865781320.2726\\
1.07302682567064	-18851400079.6148\\
1.0731268281707	-18837018838.9571\\
1.07322683067077	-18822637598.2993\\
1.07332683317083	-18808256357.6415\\
1.07342683567089	-18793875116.9837\\
1.07352683817095	-18779551172.1054\\
1.07362684067102	-18765169931.4476\\
1.07372684317108	-18750845986.5694\\
1.07382684567114	-18736522041.6911\\
1.0739268481712	-18722198096.8128\\
1.07402685067127	-18707874151.9346\\
1.07412685317133	-18693550207.0563\\
1.07422685567139	-18679226262.178\\
1.07432685817145	-18664902317.2998\\
1.07442686067152	-18650578372.4215\\
1.07452686317158	-18636311723.3227\\
1.07462686567164	-18621987778.4445\\
1.0747268681717	-18607721129.3457\\
1.07482687067177	-18593454480.2469\\
1.07492687317183	-18579187831.1482\\
1.07502687567189	-18564921182.0494\\
1.07512687817195	-18550654532.9507\\
1.07522688067202	-18536387883.8519\\
1.07532688317208	-18522121234.7531\\
1.07542688567214	-18507854585.6544\\
1.0755268881722	-18493645232.3351\\
1.07562689067227	-18479378583.2364\\
1.07572689317233	-18465169229.9171\\
1.07582689567239	-18450959876.5979\\
1.07592689817245	-18436750523.2787\\
1.07602690067252	-18422541169.9594\\
1.07612690317258	-18408331816.6402\\
1.07622690567264	-18394122463.3209\\
1.0763269081727	-18379913110.0017\\
1.07642691067277	-18365761052.4619\\
1.07652691317283	-18351551699.1427\\
1.07662691567289	-18337399641.603\\
1.07672691817295	-18323190288.2837\\
1.07682692067302	-18309038230.744\\
1.07692692317308	-18294886173.2043\\
1.07702692567314	-18280734115.6645\\
1.0771269281732	-18266582058.1248\\
1.07722693067327	-18252430000.5851\\
1.07732693317333	-18238335238.8249\\
1.07742693567339	-18224183181.2851\\
1.07752693817345	-18210031123.7454\\
1.07762694067352	-18195936361.9852\\
1.07772694317358	-18181841600.225\\
1.07782694567364	-18167689542.6852\\
1.0779269481737	-18153594780.925\\
1.07802695067377	-18139500019.1648\\
1.07812695317383	-18125405257.4046\\
1.07822695567389	-18111367791.4239\\
1.07832695817395	-18097273029.6636\\
1.07842696067402	-18083178267.9034\\
1.07852696317408	-18069140801.9227\\
1.07862696567414	-18055046040.1625\\
1.0787269681742	-18041008574.1818\\
1.07882697067427	-18026971108.2011\\
1.07892697317433	-18012933642.2204\\
1.07902697567439	-17998896176.2397\\
1.07912697817445	-17984858710.259\\
1.07922698067452	-17970821244.2783\\
1.07932698317458	-17956783778.2976\\
1.07942698567464	-17942803608.0964\\
1.0795269881747	-17928766142.1157\\
1.07962699067477	-17914785971.9145\\
1.07972699317483	-17900805801.7133\\
1.07982699567489	-17886768335.7326\\
1.07992699817495	-17872788165.5314\\
1.08002700067502	-17858807995.3302\\
1.08012700317508	-17844827825.129\\
1.08022700567514	-17830904950.7073\\
1.0803270081752	-17816924780.5061\\
1.08042701067527	-17802944610.3049\\
1.08052701317533	-17789021735.8833\\
1.08062701567539	-17775041565.6821\\
1.08072701817545	-17761118691.2604\\
1.08082702067552	-17747195816.8387\\
1.08092702317558	-17733272942.417\\
1.08102702567564	-17719350067.9954\\
1.0811270281757	-17705427193.5737\\
1.08122703067577	-17691504319.152\\
1.08132703317583	-17677581444.7303\\
1.08142703567589	-17663715866.0882\\
1.08152703817595	-17649792991.6665\\
1.08162704067602	-17635927413.0243\\
1.08172704317608	-17622061834.3821\\
1.08182704567614	-17608138959.9605\\
1.0819270481762	-17594273381.3183\\
1.08202705067627	-17580407802.6761\\
1.08212705317633	-17566542224.034\\
1.08222705567639	-17552733941.1713\\
1.08232705817645	-17538868362.5291\\
1.08242706067652	-17525002783.887\\
1.08252706317658	-17511194501.0243\\
1.08262706567664	-17497386218.1617\\
1.0827270681767	-17483520639.5195\\
1.08282707067677	-17469712356.6569\\
1.08292707317683	-17455904073.7942\\
1.08302707567689	-17442095790.9315\\
1.08312707817695	-17428287508.0689\\
1.08322708067702	-17414479225.2062\\
1.08332708317708	-17400728238.1231\\
1.08342708567714	-17386919955.2604\\
1.0835270881772	-17373168968.1773\\
1.08362709067727	-17359360685.3147\\
1.08372709317733	-17345609698.2315\\
1.08382709567739	-17331858711.1484\\
1.08392709817745	-17318107724.0652\\
1.08402710067752	-17304356736.9821\\
1.08412710317758	-17290605749.899\\
1.08422710567764	-17276854762.8158\\
1.0843271081777	-17263103775.7327\\
1.08442711067777	-17249410084.4291\\
1.08452711317783	-17235659097.3459\\
1.08462711567789	-17221965406.0423\\
1.08472711817795	-17208271714.7387\\
1.08482712067802	-17194520727.6555\\
1.08492712317808	-17180827036.3519\\
1.08502712567814	-17167133345.0483\\
1.0851271281782	-17153496949.5242\\
1.08522713067827	-17139803258.2205\\
1.08532713317833	-17126109566.9169\\
1.08542713567839	-17112415875.6133\\
1.08552713817845	-17098779480.0892\\
1.08562714067852	-17085143084.565\\
1.08572714317858	-17071449393.2614\\
1.08582714567864	-17057812997.7373\\
1.0859271481787	-17044176602.2132\\
1.08602715067877	-17030540206.6891\\
1.08612715317883	-17016903811.165\\
1.08622715567889	-17003267415.6409\\
1.08632715817895	-16989688315.8962\\
1.08642716067902	-16976051920.3721\\
1.08652716317908	-16962472820.6275\\
1.08662716567914	-16948836425.1034\\
1.0867271681792	-16935257325.3588\\
1.08682717067927	-16921678225.6142\\
1.08692717317933	-16908099125.8696\\
1.08702717567939	-16894520026.125\\
1.08712717817945	-16880940926.3804\\
1.08722718067952	-16867361826.6358\\
1.08732718317958	-16853840022.6707\\
1.08742718567964	-16840260922.9261\\
1.0875271881797	-16826739118.961\\
1.08762719067977	-16813160019.2164\\
1.08772719317983	-16799638215.2514\\
1.08782719567989	-16786116411.2863\\
1.08792719817995	-16772594607.3212\\
1.08802720068002	-16759072803.3561\\
1.08812720318008	-16745550999.391\\
1.08822720568014	-16732029195.4259\\
1.0883272081802	-16718507391.4608\\
1.08842721068027	-16705042883.2753\\
1.08852721318033	-16691521079.3102\\
1.08862721568039	-16678056571.1246\\
1.08872721818045	-16664592062.939\\
1.08882722068052	-16651070258.9739\\
1.08892722318058	-16637605750.7884\\
1.08902722568064	-16624141242.6028\\
1.0891272281807	-16610734030.1967\\
1.08922723068077	-16597269522.0111\\
1.08932723318083	-16583805013.8256\\
1.08942723568089	-16570397801.4195\\
1.08952723818095	-16556933293.2339\\
1.08962724068102	-16543526080.8279\\
1.08972724318108	-16530061572.6423\\
1.08982724568114	-16516654360.2362\\
1.0899272481812	-16503247147.8302\\
1.09002725068127	-16489839935.4241\\
1.09012725318133	-16476432723.0181\\
1.09022725568139	-16463082806.3915\\
1.09032725818145	-16449675593.9854\\
1.09042726068152	-16436268381.5794\\
1.09052726318158	-16422918464.9528\\
1.09062726568164	-16409511252.5468\\
1.0907272681817	-16396161335.9202\\
1.09082727068177	-16382811419.2937\\
1.09092727318183	-16369461502.6671\\
1.09102727568189	-16356111586.0406\\
1.09112727818195	-16342761669.414\\
1.09122728068202	-16329411752.7875\\
1.09132728318208	-16316119131.9405\\
1.09142728568214	-16302769215.3139\\
1.0915272881822	-16289476594.4669\\
1.09162729068227	-16276126677.8403\\
1.09172729318233	-16262834056.9933\\
1.09182729568239	-16249541436.1463\\
1.09192729818245	-16236248815.2992\\
1.09202730068252	-16222956194.4522\\
1.09212730318258	-16209663573.6051\\
1.09222730568264	-16196370952.7581\\
1.0923273081827	-16183078331.9111\\
1.09242731068277	-16169843006.8436\\
1.09252731318283	-16156550385.9965\\
1.09262731568289	-16143315060.929\\
1.09272731818295	-16130079735.8615\\
1.09282732068302	-16116787115.0144\\
1.09292732318308	-16103551789.9469\\
1.09302732568314	-16090316464.8794\\
1.0931273281832	-16077138435.5914\\
1.09322733068327	-16063903110.5239\\
1.09332733318333	-16050667785.4563\\
1.09342733568339	-16037489756.1683\\
1.09352733818345	-16024254431.1008\\
1.09362734068352	-16011076401.8128\\
1.09372734318358	-15997841076.7453\\
1.09382734568364	-15984663047.4573\\
1.0939273481837	-15971485018.1693\\
1.09402735068377	-15958306988.8813\\
1.09412735318383	-15945128959.5932\\
1.09422735568389	-15931950930.3052\\
1.09432735818395	-15918830196.7967\\
1.09442736068402	-15905652167.5087\\
1.09452736318408	-15892531434.0002\\
1.09462736568414	-15879353404.7122\\
1.0947273681842	-15866232671.2037\\
1.09482737068427	-15853111937.6952\\
1.09492737318433	-15839991204.1867\\
1.09502737568439	-15826870470.6782\\
1.09512737818445	-15813749737.1697\\
1.09522738068452	-15800629003.6613\\
1.09532738318458	-15787508270.1528\\
1.09542738568464	-15774444832.4238\\
1.0955273881847	-15761324098.9153\\
1.09562739068477	-15748260661.1863\\
1.09572739318483	-15735197223.4573\\
1.09582739568489	-15722076489.9488\\
1.09592739818495	-15709013052.2198\\
1.09602740068502	-15695949614.4909\\
1.09612740318508	-15682886176.7619\\
1.09622740568514	-15669880034.8124\\
1.0963274081852	-15656816597.0834\\
1.09642741068527	-15643753159.3544\\
1.09652741318533	-15630747017.405\\
1.09662741568539	-15617740875.4555\\
1.09672741818545	-15604677437.7265\\
1.09682742068552	-15591671295.777\\
1.09692742318558	-15578665153.8276\\
1.09702742568564	-15565659011.8781\\
1.0971274281857	-15552652869.9286\\
1.09722743068577	-15539646727.9792\\
1.09732743318583	-15526697881.8092\\
1.09742743568589	-15513691739.8597\\
1.09752743818595	-15500685597.9103\\
1.09762744068602	-15487736751.7403\\
1.09772744318608	-15474787905.5704\\
1.09782744568614	-15461839059.4004\\
1.0979274481862	-15448832917.4509\\
1.09802745068627	-15435884071.281\\
1.09812745318633	-15422992520.8905\\
1.09822745568639	-15410043674.7206\\
1.09832745818645	-15397094828.5506\\
1.09842746068652	-15384145982.3807\\
1.09852746318658	-15371254431.9902\\
1.09862746568664	-15358305585.8203\\
1.0987274681867	-15345414035.4298\\
1.09882747068677	-15332522485.0394\\
1.09892747318683	-15319630934.6489\\
1.09902747568689	-15306739384.2585\\
1.09912747818695	-15293847833.868\\
1.09922748068702	-15280956283.4776\\
1.09932748318708	-15268064733.0872\\
1.09942748568714	-15255230478.4762\\
1.0995274881872	-15242338928.0858\\
1.09962749068727	-15229504673.4748\\
1.09972749318733	-15216670418.8639\\
1.09982749568739	-15203778868.4735\\
1.09992749818745	-15190944613.8625\\
1.10002750068752	-15178110359.2516\\
1.10012750318758	-15165276104.6407\\
1.10022750568764	-15152441850.0298\\
1.1003275081877	-15139664891.1983\\
1.10042751068777	-15126830636.5874\\
1.10052751318783	-15114053677.756\\
1.10062751568789	-15101219423.1451\\
1.10072751818795	-15088442464.3136\\
1.10082752068802	-15075665505.4822\\
1.10092752318808	-15062831250.8713\\
1.10102752568814	-15050054292.0399\\
1.1011275281882	-15037334628.988\\
1.10122753068827	-15024557670.1566\\
1.10132753318833	-15011780711.3251\\
1.10142753568839	-14999003752.4937\\
1.10152753818845	-14986284089.4418\\
1.10162754068852	-14973507130.6104\\
1.10172754318858	-14960787467.5585\\
1.10182754568864	-14948067804.5066\\
1.1019275481887	-14935348141.4547\\
1.10202755068877	-14922628478.4028\\
1.10212755318883	-14909908815.3509\\
1.10222755568889	-14897189152.299\\
1.10232755818895	-14884469489.2471\\
1.10242756068902	-14871749826.1952\\
1.10252756318908	-14859087458.9228\\
1.10262756568914	-14846425091.6504\\
1.1027275681892	-14833705428.5985\\
1.10282757068927	-14821043061.3261\\
1.10292757318933	-14808380694.0537\\
1.10302757568939	-14795718326.7813\\
1.10312757818945	-14783055959.5089\\
1.10322758068952	-14770393592.2365\\
1.10332758318958	-14757731224.9641\\
1.10342758568964	-14745126153.4712\\
1.1035275881897	-14732463786.1989\\
1.10362759068977	-14719858714.706\\
1.10372759318983	-14707196347.4336\\
1.10382759568989	-14694591275.9407\\
1.10392759818995	-14681986204.4478\\
1.10402760069002	-14669381132.955\\
1.10412760319008	-14656776061.4621\\
1.10422760569014	-14644170989.9692\\
1.1043276081902	-14631565918.4763\\
1.10442761069027	-14618960846.9834\\
1.10452761319033	-14606413071.2701\\
1.10462761569039	-14593807999.7772\\
1.10472761819045	-14581260224.0638\\
1.10482762069052	-14568712448.3505\\
1.10492762319058	-14556164672.6371\\
1.10502762569064	-14543616896.9237\\
1.1051276281907	-14531069121.2104\\
1.10522763069077	-14518521345.497\\
1.10532763319083	-14505973569.7836\\
1.10542763569089	-14493425794.0703\\
1.10552763819095	-14480935314.1364\\
1.10562764069102	-14468387538.4231\\
1.10572764319108	-14455897058.4892\\
1.10582764569114	-14443406578.5554\\
1.1059276481912	-14430858802.842\\
1.10602765069127	-14418368322.9081\\
1.10612765319133	-14405877842.9743\\
1.10622765569139	-14393387363.0404\\
1.10632765819145	-14380954178.8861\\
1.10642766069152	-14368463698.9522\\
1.10652766319158	-14355973219.0184\\
1.10662766569164	-14343540034.8641\\
1.1067276681917	-14331106850.7097\\
1.10682767069177	-14318616370.7759\\
1.10692767319183	-14306183186.6215\\
1.10702767569189	-14293750002.4672\\
1.10712767819195	-14281316818.3128\\
1.10722768069202	-14268883634.1585\\
1.10732768319208	-14256450450.0042\\
1.10742768569214	-14244074561.6293\\
1.1075276881922	-14231641377.475\\
1.10762769069227	-14219265489.1002\\
1.10772769319233	-14206832304.9458\\
1.10782769569239	-14194456416.571\\
1.10792769819245	-14182080528.1962\\
1.10802770069252	-14169704639.8214\\
1.10812770319258	-14157328751.4465\\
1.10822770569264	-14144952863.0717\\
1.1083277081927	-14132576974.6969\\
1.10842771069277	-14120201086.3221\\
1.10852771319283	-14107825197.9472\\
1.10862771569289	-14095506605.3519\\
1.10872771819295	-14083188012.7566\\
1.10882772069302	-14070812124.3818\\
1.10892772319308	-14058493531.7865\\
1.10902772569314	-14046174939.1912\\
1.1091277281932	-14033856346.5958\\
1.10922773069327	-14021537754.0005\\
1.10932773319333	-14009219161.4052\\
1.10942773569339	-13996900568.8099\\
1.10952773819345	-13984639271.9941\\
1.10962774069352	-13972320679.3988\\
1.10972774319358	-13960059382.583\\
1.10982774569364	-13947740789.9877\\
1.1099277481937	-13935479493.1719\\
1.11002775069377	-13923218196.3561\\
1.11012775319383	-13910956899.5403\\
1.11022775569389	-13898695602.7245\\
1.11032775819395	-13886434305.9087\\
1.11042776069402	-13874230304.8724\\
1.11052776319408	-13861969008.0566\\
1.11062776569414	-13849707711.2408\\
1.1107277681942	-13837503710.2045\\
1.11082777069427	-13825299709.1682\\
1.11092777319433	-13813038412.3524\\
1.11102777569439	-13800834411.3161\\
1.11112777819445	-13788630410.2799\\
1.11122778069452	-13776426409.2436\\
1.11132778319458	-13764222408.2073\\
1.11142778569464	-13752075702.9505\\
1.1115277881947	-13739871701.9142\\
1.11162779069477	-13727667700.8779\\
1.11172779319483	-13715520995.6212\\
1.11182779569489	-13703374290.3644\\
1.11192779819495	-13691170289.3281\\
1.11202780069502	-13679023584.0713\\
1.11212780319508	-13666876878.8146\\
1.11222780569514	-13654730173.5578\\
1.1123278081952	-13642583468.301\\
1.11242781069527	-13630436763.0442\\
1.11252781319533	-13618347353.567\\
1.11262781569539	-13606200648.3102\\
1.11272781819545	-13594111238.8329\\
1.11282782069552	-13581964533.5762\\
1.11292782319558	-13569875124.0989\\
1.11302782569564	-13557785714.6216\\
1.1131278281957	-13545696305.1444\\
1.11322783069577	-13533606895.6671\\
1.11332783319583	-13521517486.1899\\
1.11342783569589	-13509428076.7126\\
1.11352783819595	-13497338667.2353\\
1.11362784069602	-13485306553.5376\\
1.11372784319608	-13473217144.0603\\
1.11382784569614	-13461185030.3626\\
1.1139278481962	-13449152916.6648\\
1.11402785069627	-13437063507.1876\\
1.11412785319633	-13425031393.4898\\
1.11422785569639	-13412999279.7921\\
1.11432785819645	-13400967166.0943\\
1.11442786069652	-13388992348.1761\\
1.11452786319658	-13376960234.4784\\
1.11462786569664	-13364928120.7806\\
1.1147278681967	-13352953302.8624\\
1.11482787069677	-13340921189.1646\\
1.11492787319683	-13328946371.2464\\
1.11502787569689	-13316971553.3282\\
1.11512787819695	-13304996735.4099\\
1.11522788069702	-13293021917.4917\\
1.11532788319708	-13281047099.5735\\
1.11542788569714	-13269072281.6552\\
1.1155278881972	-13257097463.737\\
1.11562789069727	-13245179941.5983\\
1.11572789319733	-13233205123.68\\
1.11582789569739	-13221287601.5413\\
1.11592789819745	-13209312783.6231\\
1.11602790069752	-13197395261.4844\\
1.11612790319758	-13185477739.3456\\
1.11622790569764	-13173560217.2069\\
1.1163279081977	-13161642695.0682\\
1.11642791069777	-13149725172.9295\\
1.11652791319783	-13137807650.7908\\
1.11662791569789	-13125947424.4315\\
1.11672791819796	-13114029902.2928\\
1.11682792069802	-13102169675.9336\\
1.11692792319808	-13090252153.7949\\
1.11702792569814	-13078391927.4357\\
1.1171279281982	-13066531701.0765\\
1.11722793069827	-13054671474.7173\\
1.11732793319833	-13042811248.3581\\
1.11742793569839	-13030951021.9989\\
1.11752793819845	-13019090795.6396\\
1.11762794069852	-13007287865.0599\\
1.11772794319858	-12995427638.7007\\
1.11782794569864	-12983624708.121\\
1.11792794819871	-12971764481.7618\\
1.11802795069877	-12959961551.1821\\
1.11812795319883	-12948158620.6024\\
1.11822795569889	-12936355690.0228\\
1.11832795819895	-12924552759.4431\\
1.11842796069902	-12912749828.8634\\
1.11852796319908	-12900946898.2837\\
1.11862796569914	-12889143967.704\\
1.11872796819921	-12877398332.9038\\
1.11882797069927	-12865595402.3241\\
1.11892797319933	-12853849767.5239\\
1.11902797569939	-12842104132.7237\\
1.11912797819945	-12830301202.144\\
1.11922798069952	-12818555567.3439\\
1.11932798319958	-12806809932.5437\\
1.11942798569964	-12795064297.7435\\
1.1195279881997	-12783375958.7228\\
1.11962799069977	-12771630323.9226\\
1.11972799319983	-12759884689.1225\\
1.11982799569989	-12748196350.1018\\
1.11992799819996	-12736450715.3016\\
1.12002800070002	-12724762376.2809\\
1.12012800320008	-12713074037.2603\\
1.12022800570014	-12701385698.2396\\
1.1203280082002	-12689697359.2189\\
1.12042801070027	-12678009020.1983\\
1.12052801320033	-12666320681.1776\\
1.12062801570039	-12654632342.1569\\
1.12072801820046	-12643001298.9158\\
1.12082802070052	-12631312959.8951\\
1.12092802320058	-12619681916.6539\\
1.12102802570064	-12607993577.6333\\
1.12112802820071	-12596362534.3921\\
1.12122803070077	-12584731491.151\\
1.12132803320083	-12573100447.9098\\
1.12142803570089	-12561469404.6687\\
1.12152803820095	-12549838361.4275\\
1.12162804070102	-12538207318.1863\\
1.12172804320108	-12526633570.7247\\
1.12182804570114	-12515002527.4835\\
1.12192804820121	-12503428780.0219\\
1.12202805070127	-12491797736.7807\\
1.12212805320133	-12480223989.3191\\
1.12222805570139	-12468650241.8575\\
1.12232805820145	-12457076494.3958\\
1.12242806070152	-12445502746.9342\\
1.12252806320158	-12433928999.4725\\
1.12262806570164	-12422355252.0109\\
1.12272806820171	-12410781504.5492\\
1.12282807070177	-12399265052.8671\\
1.12292807320183	-12387691305.4055\\
1.12302807570189	-12376174853.7233\\
1.12312807820196	-12364658402.0412\\
1.12322808070202	-12353141950.3591\\
1.12332808320208	-12341568202.8974\\
1.12342808570214	-12330051751.2153\\
1.1235280882022	-12318592595.3127\\
1.12362809070227	-12307076143.6306\\
1.12372809320233	-12295559691.9484\\
1.12382809570239	-12284043240.2663\\
1.12392809820246	-12272584084.3637\\
1.12402810070252	-12261124928.4611\\
1.12412810320258	-12249608476.7789\\
1.12422810570264	-12238149320.8763\\
1.12432810820271	-12226690164.9737\\
1.12442811070277	-12215231009.0711\\
1.12452811320283	-12203771853.1685\\
1.12462811570289	-12192312697.2659\\
1.12472811820296	-12180853541.3632\\
1.12482812070302	-12169451681.2401\\
1.12492812320308	-12157992525.3375\\
1.12502812570314	-12146590665.2144\\
1.12512812820321	-12135188805.0913\\
1.12522813070327	-12123729649.1887\\
1.12532813320333	-12112327789.0656\\
1.12542813570339	-12100925928.9425\\
1.12552813820345	-12089524068.8194\\
1.12562814070352	-12078122208.6963\\
1.12572814320358	-12066777644.3527\\
1.12582814570364	-12055375784.2296\\
1.12592814820371	-12043973924.1065\\
1.12602815070377	-12032629359.7629\\
1.12612815320383	-12021284795.4193\\
1.12622815570389	-12009882935.2962\\
1.12632815820396	-11998538370.9526\\
1.12642816070402	-11987193806.609\\
1.12652816320408	-11975849242.2654\\
1.12662816570414	-11964504677.9219\\
1.12672816820421	-11953160113.5783\\
1.12682817070427	-11941872845.0142\\
1.12692817320433	-11930528280.6706\\
1.12702817570439	-11919241012.1065\\
1.12712817820446	-11907896447.7629\\
1.12722818070452	-11896609179.1988\\
1.12732818320458	-11885321910.6348\\
1.12742818570464	-11874034642.0707\\
1.12752818820471	-11862747373.5066\\
1.12762819070477	-11851460104.9425\\
1.12772819320483	-11840172836.3785\\
1.12782819570489	-11828885567.8144\\
1.12792819820496	-11817655595.0298\\
1.12802820070502	-11806368326.4657\\
1.12812820320508	-11795138353.6812\\
1.12822820570514	-11783851085.1171\\
1.12832820820521	-11772621112.3325\\
1.12842821070527	-11761391139.548\\
1.12852821320533	-11750161166.7634\\
1.12862821570539	-11738931193.9788\\
1.12872821820546	-11727701221.1943\\
1.12882822070552	-11716528544.1892\\
1.12892822320558	-11705298571.4047\\
1.12902822570564	-11694068598.6201\\
1.12912822820571	-11682895921.6151\\
1.12922823070577	-11671723244.61\\
1.12932823320583	-11660493271.8254\\
1.12942823570589	-11649320594.8204\\
1.12952823820596	-11638147917.8153\\
1.12962824070602	-11626975240.8103\\
1.12972824320608	-11615802563.8052\\
1.12982824570614	-11604687182.5797\\
1.12992824820621	-11593514505.5746\\
1.13002825070627	-11582341828.5696\\
1.13012825320633	-11571226447.3441\\
1.13022825570639	-11560111066.1185\\
1.13032825820646	-11548938389.1135\\
1.13042826070652	-11537823007.8879\\
1.13052826320658	-11526707626.6624\\
1.13062826570664	-11515592245.4369\\
1.13072826820671	-11504476864.2113\\
1.13082827070677	-11493361482.9858\\
1.13092827320683	-11482303397.5397\\
1.13102827570689	-11471188016.3142\\
1.13112827820696	-11460129930.8682\\
1.13122828070702	-11449014549.6426\\
1.13132828320708	-11437956464.1966\\
1.13142828570714	-11426898378.7506\\
1.13152828820721	-11415840293.3046\\
1.13162829070727	-11404782207.8585\\
1.13172829320733	-11393724122.4125\\
1.13182829570739	-11382666036.9665\\
1.13192829820746	-11371607951.5205\\
1.13202830070752	-11360549866.0744\\
1.13212830320758	-11349549076.4079\\
1.13222830570764	-11338490990.9619\\
1.13232830820771	-11327490201.2954\\
1.13242831070777	-11316489411.6289\\
1.13252831320783	-11305488621.9624\\
1.13262831570789	-11294487832.2959\\
1.13272831820796	-11283487042.6294\\
1.13282832070802	-11272486252.9628\\
1.13292832320808	-11261485463.2963\\
1.13302832570814	-11250541969.4093\\
1.13312832820821	-11239541179.7428\\
1.13322833070827	-11228597685.8558\\
1.13332833320833	-11217596896.1893\\
1.13342833570839	-11206653402.3023\\
1.13352833820846	-11195709908.4153\\
1.13362834070852	-11184766414.5283\\
1.13372834320858	-11173822920.6413\\
1.13382834570864	-11162879426.7543\\
1.13392834820871	-11151935932.8673\\
1.13402835070877	-11140992438.9803\\
1.13412835320883	-11130106240.8728\\
1.13422835570889	-11119162746.9858\\
1.13432835820896	-11108276548.8783\\
1.13442836070902	-11097390350.7709\\
1.13452836320908	-11086446856.8839\\
1.13462836570914	-11075560658.7764\\
1.13472836820921	-11064674460.6689\\
1.13482837070927	-11053788262.5614\\
1.13492837320933	-11042959360.2334\\
1.13502837570939	-11032073162.1259\\
1.13512837820946	-11021186964.0185\\
1.13522838070952	-11010358061.6905\\
1.13532838320958	-10999471863.583\\
1.13542838570964	-10988642961.255\\
1.13552838820971	-10977814058.9271\\
1.13562839070977	-10966985156.5991\\
1.13572839320983	-10956156254.2711\\
1.13582839570989	-10945327351.9431\\
1.13592839820996	-10934498449.6152\\
1.13602840071002	-10923669547.2872\\
1.13612840321008	-10912840644.9592\\
1.13622840571014	-10902069038.4108\\
1.13632840821021	-10891240136.0828\\
1.13642841071027	-10880468529.5343\\
1.13652841321033	-10869696922.9859\\
1.13662841571039	-10858925316.4374\\
1.13672841821046	-10848096414.1094\\
1.13682842071052	-10837324807.561\\
1.13692842321058	-10826610496.792\\
1.13702842571064	-10815838890.2436\\
1.13712842821071	-10805067283.6951\\
1.13722843071077	-10794352972.9262\\
1.13732843321083	-10783581366.3777\\
1.13742843571089	-10772867055.6088\\
1.13752843821096	-10762095449.0603\\
1.13762844071102	-10751381138.2914\\
1.13772844321108	-10740666827.5224\\
1.13782844571114	-10729952516.7535\\
1.13792844821121	-10719238205.9845\\
1.13802845071127	-10708523895.2156\\
1.13812845321133	-10697866880.2261\\
1.13822845571139	-10687152569.4572\\
1.13832845821146	-10676438258.6882\\
1.13842846071152	-10665781243.6988\\
1.13852846321158	-10655124228.7094\\
1.13862846571164	-10644409917.9404\\
1.13872846821171	-10633752902.951\\
1.13882847071177	-10623095887.9616\\
1.13892847321183	-10612438872.9721\\
1.13902847571189	-10601839153.7622\\
1.13912847821196	-10591182138.7728\\
1.13922848071202	-10580525123.7833\\
1.13932848321208	-10569925404.5734\\
1.13942848571214	-10559268389.584\\
1.13952848821221	-10548668670.3741\\
1.13962849071227	-10538011655.3846\\
1.13972849321233	-10527411936.1747\\
1.13982849571239	-10516812216.9648\\
1.13992849821246	-10506212497.7549\\
1.14002850071252	-10495612778.545\\
1.14012850321258	-10485070355.1146\\
1.14022850571264	-10474470635.9046\\
1.14032850821271	-10463870916.6947\\
1.14042851071277	-10453328493.2643\\
1.14052851321283	-10442728774.0544\\
1.14062851571289	-10432186350.624\\
1.14072851821296	-10421643927.1936\\
1.14082852071302	-10411101503.7632\\
1.14092852321308	-10400559080.3328\\
1.14102852571314	-10390016656.9023\\
1.14112852821321	-10379474233.4719\\
1.14122853071327	-10368931810.0415\\
1.14132853321333	-10358446682.3906\\
1.14142853571339	-10347904258.9602\\
1.14152853821346	-10337419131.3093\\
1.14162854071352	-10326876707.8789\\
1.14172854321358	-10316391580.228\\
1.14182854571364	-10305906452.5771\\
1.14192854821371	-10295421324.9263\\
1.14202855071377	-10284936197.2754\\
1.14212855321383	-10274451069.6245\\
1.14222855571389	-10263965941.9736\\
1.14232855821396	-10253538110.1022\\
1.14242856071402	-10243052982.4513\\
1.14252856321408	-10232625150.5799\\
1.14262856571414	-10222140022.929\\
1.14272856821421	-10211712191.0576\\
1.14282857071427	-10201284359.1863\\
1.14292857321433	-10190856527.3149\\
1.14302857571439	-10180428695.4435\\
1.14312857821446	-10170000863.5721\\
1.14322858071452	-10159573031.7007\\
1.14332858321458	-10149145199.8293\\
1.14342858571464	-10138774663.7375\\
1.14352858821471	-10128346831.8661\\
1.14362859071477	-10117976295.7742\\
1.14372859321483	-10107605759.6824\\
1.14382859571489	-10097177927.811\\
1.14392859821496	-10086807391.7191\\
1.14402860071502	-10076436855.6272\\
1.14412860321508	-10066066319.5354\\
1.14422860571514	-10055695783.4435\\
1.14432860821521	-10045382543.1312\\
1.14442861071527	-10035012007.0393\\
1.14452861321533	-10024698766.7269\\
1.14462861571539	-10014328230.6351\\
1.14472861821546	-10004014990.3227\\
1.14482862071552	-9993701750.01036\\
1.14492862321558	-9983331213.91849\\
1.14502862571564	-9973017973.60613\\
1.14512862821571	-9962704733.29378\\
1.14522863071577	-9952391492.98143\\
1.14532863321583	-9942135548.44858\\
1.14542863571589	-9931822308.13623\\
1.14552863821596	-9921509067.82387\\
1.14562864071602	-9911253123.29103\\
1.14572864321608	-9900997178.75819\\
1.14582864571614	-9890683938.44584\\
1.14592864821621	-9880427993.91299\\
1.14602865071627	-9870172049.38015\\
1.14612865321633	-9859916104.84731\\
1.14622865571639	-9849660160.31447\\
1.14632865821646	-9839404215.78163\\
1.14642866071652	-9829205567.0283\\
1.14652866321658	-9818949622.49546\\
1.14662866571664	-9808693677.96262\\
1.14672866821671	-9798495029.20929\\
1.14682867071677	-9788296380.45596\\
1.14692867321683	-9778040435.92312\\
1.14702867571689	-9767841787.16979\\
1.14712867821696	-9757643138.41646\\
1.14722868071702	-9747444489.66313\\
1.14732868321708	-9737245840.9098\\
1.14742868571714	-9727104487.93599\\
1.14752868821721	-9716905839.18266\\
1.14762869071727	-9706707190.42933\\
1.14772869321733	-9696565837.45551\\
1.14782869571739	-9686424484.4817\\
1.14792869821746	-9676225835.72837\\
1.14802870071752	-9666084482.75455\\
1.14812870321758	-9655943129.78074\\
1.14822870571764	-9645801776.80692\\
1.14832870821771	-9635660423.83311\\
1.14842871071777	-9625519070.85929\\
1.14852871321783	-9615435013.66499\\
1.14862871571789	-9605293660.69117\\
1.14872871821796	-9595209603.49687\\
1.14882872071802	-9585068250.52305\\
1.14892872321808	-9574984193.32875\\
1.14902872571814	-9564900136.13445\\
1.14912872821821	-9554816078.94015\\
1.14922873071827	-9544732021.74585\\
1.14932873321833	-9534647964.55154\\
1.14942873571839	-9524563907.35724\\
1.14952873821846	-9514479850.16294\\
1.14962874071852	-9504395792.96864\\
1.14972874321858	-9494369031.55385\\
1.14982874571864	-9484284974.35954\\
1.14992874821871	-9474258212.94475\\
1.15002875071877	-9464231451.52996\\
1.15012875321883	-9454204690.11518\\
1.15022875571889	-9444177928.70039\\
1.15032875821896	-9434151167.2856\\
1.15042876071902	-9424124405.87081\\
1.15052876321908	-9414097644.45602\\
1.15062876571914	-9404070883.04123\\
1.15072876821921	-9394101417.40595\\
1.15082877071927	-9384074655.99116\\
1.15092877321933	-9374105190.35589\\
1.15102877571939	-9364078428.9411\\
1.15112877821946	-9354108963.30582\\
1.15122878071952	-9344139497.67054\\
1.15132878321958	-9334170032.03527\\
1.15142878571964	-9324200566.39999\\
1.15152878821971	-9314231100.76472\\
1.15162879071977	-9304318930.90895\\
1.15172879321983	-9294349465.27368\\
1.15182879571989	-9284437295.41791\\
1.15192879821996	-9274467829.78263\\
1.15202880072002	-9264555659.92687\\
1.15212880322008	-9254643490.07111\\
1.15222880572014	-9244674024.43583\\
1.15232880822021	-9234761854.58007\\
1.15242881072027	-9224849684.72431\\
1.15252881322033	-9214994810.64806\\
1.15262881572039	-9205082640.79229\\
1.15272881822046	-9195170470.93653\\
1.15282882072052	-9185315596.86028\\
1.15292882322058	-9175403427.00452\\
1.15302882572064	-9165548552.92827\\
1.15312882822071	-9155636383.0725\\
1.15322883072077	-9145781508.99625\\
1.15332883322083	-9135926634.92\\
1.15342883572089	-9126071760.84375\\
1.15352883822096	-9116216886.7675\\
1.15362884072102	-9106362012.69125\\
1.15372884322108	-9096564434.39451\\
1.15382884572114	-9086709560.31826\\
1.15392884822121	-9076911982.02153\\
1.15402885072127	-9067057107.94528\\
1.15412885322133	-9057259529.64854\\
1.15422885572139	-9047461951.3518\\
1.15432885822146	-9037607077.27555\\
1.15442886072152	-9027809498.97882\\
1.15452886322158	-9018011920.68208\\
1.15462886572164	-9008271638.16486\\
1.15472886822171	-8998474059.86812\\
1.15482887072177	-8988676481.57138\\
1.15492887322183	-8978936199.05416\\
1.15502887572189	-8969138620.75742\\
1.15512887822196	-8959398338.2402\\
1.15522888072202	-8949600759.94346\\
1.15532888322208	-8939860477.42624\\
1.15542888572214	-8930120194.90901\\
1.15552888822221	-8920379912.39179\\
1.15562889072227	-8910639629.87456\\
1.15572889322233	-8900899347.35734\\
1.15582889572239	-8891216360.61963\\
1.15592889822246	-8881476078.1024\\
1.15602890072252	-8871793091.36469\\
1.15612890322258	-8862052808.84747\\
1.15622890572264	-8852369822.10976\\
1.15632890822271	-8842686835.37205\\
1.15642891072277	-8833003848.63434\\
1.15652891322283	-8823320861.89663\\
1.15662891572289	-8813637875.15891\\
1.15672891822296	-8803954888.4212\\
1.15682892072302	-8794271901.68349\\
1.15692892322308	-8784588914.94578\\
1.15702892572314	-8774963223.98758\\
1.15712892822321	-8765280237.24987\\
1.15722893072327	-8755654546.29168\\
1.15732893322333	-8746028855.33348\\
1.15742893572339	-8736403164.37528\\
1.15752893822346	-8726720177.63757\\
1.15762894072352	-8717094486.67937\\
1.15772894322358	-8707526091.50069\\
1.15782894572364	-8697900400.54249\\
1.15792894822371	-8688274709.58429\\
1.15802895072377	-8678649018.62609\\
1.15812895322383	-8669080623.44741\\
1.15822895572389	-8659512228.26872\\
1.15832895822396	-8649886537.31053\\
1.15842896072402	-8640318142.13184\\
1.15852896322408	-8630749746.95316\\
1.15862896572414	-8621181351.77447\\
1.15872896822421	-8611612956.59579\\
1.15882897072427	-8602044561.4171\\
1.15892897322433	-8592476166.23842\\
1.15902897572439	-8582965066.83924\\
1.15912897822446	-8573396671.66056\\
1.15922898072452	-8563885572.26139\\
1.15932898322458	-8554317177.0827\\
1.15942898572464	-8544806077.68353\\
1.15952898822471	-8535294978.28436\\
1.15962899072477	-8525783878.88519\\
1.15972899322483	-8516272779.48602\\
1.15982899572489	-8506761680.08685\\
1.15992899822496	-8497250580.68767\\
1.16002900072502	-8487739481.2885\\
1.16012900322508	-8478285677.66884\\
1.16022900572514	-8468774578.26967\\
1.16032900822521	-8459320774.65001\\
1.16042901072527	-8449866971.03035\\
1.16052901322533	-8440355871.63118\\
1.16062901572539	-8430902068.01152\\
1.16072901822546	-8421448264.39187\\
1.16082902072552	-8411994460.77221\\
1.16092902322558	-8402540657.15255\\
1.16102902572564	-8393144149.3124\\
1.16112902822571	-8383690345.69275\\
1.16122903072577	-8374293837.8526\\
1.16132903322583	-8364840034.23294\\
1.16142903572589	-8355443526.3928\\
1.16152903822596	-8345989722.77314\\
1.16162904072602	-8336593214.93299\\
1.16172904322608	-8327196707.09285\\
1.16182904572614	-8317800199.2527\\
1.16192904822621	-8308403691.41255\\
1.16202905072627	-8299064479.35192\\
1.16212905322633	-8289667971.51178\\
1.16222905572639	-8280271463.67163\\
1.16232905822646	-8270932251.611\\
1.16242906072652	-8261535743.77085\\
1.16252906322658	-8252196531.71022\\
1.16262906572664	-8242857319.64959\\
1.16272906822671	-8233518107.58896\\
1.16282907072677	-8224178895.52832\\
1.16292907322683	-8214839683.46769\\
1.16302907572689	-8205500471.40706\\
1.16312907822696	-8196161259.34643\\
1.16322908072702	-8186879343.06531\\
1.16332908322708	-8177540131.00467\\
1.16342908572714	-8168258214.72356\\
1.16352908822721	-8158919002.66292\\
1.16362909072727	-8149637086.3818\\
1.16372909322733	-8140355170.10068\\
1.16382909572739	-8131073253.81956\\
1.16392909822746	-8121791337.53845\\
1.16402910072752	-8112509421.25733\\
1.16412910322758	-8103227504.97621\\
1.16422910572764	-8093945588.69509\\
1.16432910822771	-8084720968.19348\\
1.16442911072777	-8075439051.91236\\
1.16452911322783	-8066214431.41076\\
1.16462911572789	-8056932515.12964\\
1.16472911822796	-8047707894.62803\\
1.16482912072802	-8038483274.12642\\
1.16492912322808	-8029258653.62482\\
1.16502912572814	-8020034033.12321\\
1.16512912822821	-8010809412.6216\\
1.16522913072827	-8001642087.89951\\
1.16532913322833	-7992417467.39791\\
1.16542913572839	-7983192846.8963\\
1.16552913822846	-7974025522.17421\\
1.16562914072852	-7964858197.45211\\
1.16572914322858	-7955633576.95051\\
1.16582914572864	-7946466252.22841\\
1.16592914822871	-7937298927.50632\\
1.16602915072877	-7928131602.78423\\
1.16612915322883	-7918964278.06213\\
1.16622915572889	-7909796953.34004\\
1.16632915822896	-7900686924.39746\\
1.16642916072902	-7891519599.67537\\
1.16652916322908	-7882409570.73279\\
1.16662916572914	-7873242246.01069\\
1.16672916822921	-7864132217.06811\\
1.16682917072927	-7855022188.12553\\
1.16692917322933	-7845854863.40344\\
1.16702917572939	-7836744834.46086\\
1.16712917822946	-7827634805.51828\\
1.16722918072952	-7818582072.35521\\
1.16732918322958	-7809472043.41263\\
1.16742918572964	-7800362014.47005\\
1.16752918822971	-7791309281.30699\\
1.16762919072977	-7782199252.36441\\
1.16772919322983	-7773146519.20134\\
1.16782919572989	-7764093786.03827\\
1.16792919822996	-7754983757.09569\\
1.16802920073002	-7745931023.93263\\
1.16812920323008	-7736878290.76956\\
1.16822920573014	-7727825557.60649\\
1.16832920823021	-7718830120.22294\\
1.16842921073027	-7709777387.05987\\
1.16852921323033	-7700724653.8968\\
1.16862921573039	-7691729216.51325\\
1.16872921823046	-7682676483.35018\\
1.16882922073052	-7673681045.96663\\
1.16892922323058	-7664685608.58307\\
1.16902922573064	-7655690171.19952\\
1.16912922823071	-7646694733.81597\\
1.16922923073077	-7637699296.43241\\
1.16932923323083	-7628703859.04886\\
1.16942923573089	-7619708421.66531\\
1.16952923823096	-7610712984.28175\\
1.16962924073102	-7601774842.67771\\
1.16972924323108	-7592779405.29416\\
1.16982924573114	-7583841263.69012\\
1.16992924823121	-7574903122.08607\\
1.17002925073127	-7565907684.70252\\
1.17012925323133	-7556969543.09848\\
1.17022925573139	-7548031401.49444\\
1.17032925823146	-7539093259.8904\\
1.17042926073152	-7530212414.06587\\
1.17052926323158	-7521274272.46183\\
1.17062926573164	-7512336130.85779\\
1.17072926823171	-7503455285.03326\\
1.17082927073177	-7494517143.42922\\
1.17092927323183	-7485636297.60469\\
1.17102927573189	-7476755451.78016\\
1.17112927823196	-7467874605.95564\\
1.17122928073202	-7458993760.13111\\
1.17132928323208	-7450112914.30658\\
1.17142928573214	-7441232068.48205\\
1.17152928823221	-7432351222.65753\\
1.17162929073227	-7423470376.833\\
1.17172929323233	-7414646826.78798\\
1.17182929573239	-7405765980.96346\\
1.17192929823246	-7396942430.91844\\
1.17202930073252	-7388118880.87343\\
1.17212930323258	-7379238035.0489\\
1.17222930573264	-7370414485.00388\\
1.17232930823271	-7361590934.95887\\
1.17242931073277	-7352767384.91385\\
1.17252931323283	-7343943834.86884\\
1.17262931573289	-7335177580.60334\\
1.17272931823296	-7326354030.55832\\
1.17282932073302	-7317587776.29282\\
1.17292932323308	-7308764226.24781\\
1.17302932573314	-7299997971.98231\\
1.17312932823321	-7291231717.7168\\
1.17322933073327	-7282408167.67179\\
1.17332933323333	-7273641913.40629\\
1.17342933573339	-7264875659.14079\\
1.17352933823346	-7256109404.87528\\
1.17362934073352	-7247400446.3893\\
1.17372934323358	-7238634192.12379\\
1.17382934573364	-7229867937.85829\\
1.17392934823371	-7221158979.3723\\
1.17402935073377	-7212392725.1068\\
1.17412935323383	-7203683766.62081\\
1.17422935573389	-7194974808.13483\\
1.17432935823396	-7186265849.64884\\
1.17442936073402	-7177556891.16285\\
1.17452936323408	-7168847932.67686\\
1.17462936573414	-7160138974.19087\\
1.17472936823421	-7151430015.70488\\
1.17482937073427	-7142778352.99841\\
1.17492937323433	-7134069394.51242\\
1.17502937573439	-7125417731.80594\\
1.17512937823446	-7116708773.31996\\
1.17522938073452	-7108057110.61348\\
1.17532938323458	-7099405447.907\\
1.17542938573464	-7090753785.20053\\
1.17552938823471	-7082102122.49405\\
1.17562939073477	-7073450459.78758\\
1.17572939323483	-7064798797.0811\\
1.17582939573489	-7056147134.37463\\
1.17592939823496	-7047552767.44767\\
1.17602940073502	-7038901104.74119\\
1.17612940323508	-7030306737.81423\\
1.17622940573514	-7021655075.10775\\
1.17632940823521	-7013060708.18079\\
1.17642941073527	-7004466341.25383\\
1.17652941323533	-6995871974.32687\\
1.17662941573539	-6987277607.3999\\
1.17672941823546	-6978683240.47294\\
1.17682942073552	-6970146169.32549\\
1.17692942323558	-6961551802.39853\\
1.17702942573564	-6952957435.47157\\
1.17712942823571	-6944420364.32412\\
1.17722943073577	-6935825997.39715\\
1.17732943323583	-6927288926.24971\\
1.17742943573589	-6918751855.10226\\
1.17752943823596	-6910214783.95481\\
1.17762944073602	-6901677712.80736\\
1.17772944323608	-6893140641.65991\\
1.17782944573614	-6884603570.51246\\
1.17792944823621	-6876066499.36501\\
1.17802945073627	-6867586723.99707\\
1.17812945323633	-6859049652.84962\\
1.17822945573639	-6850569877.48169\\
1.17832945823646	-6842032806.33424\\
1.17842946073652	-6833553030.9663\\
1.17852946323658	-6825073255.59837\\
1.17862946573664	-6816593480.23043\\
1.17872946823671	-6808113704.86249\\
1.17882947073677	-6799633929.49456\\
1.17892947323683	-6791154154.12662\\
1.17902947573689	-6782731674.5382\\
1.17912947823696	-6774251899.17026\\
1.17922948073702	-6765829419.58184\\
1.17932948323708	-6757349644.2139\\
1.17942948573714	-6748927164.62548\\
1.17952948823721	-6740504685.03706\\
1.17962949073727	-6732082205.44863\\
1.17972949323733	-6723659725.86021\\
1.17982949573739	-6715237246.27179\\
1.17992949823746	-6706814766.68336\\
1.18002950073752	-6698392287.09494\\
1.18012950323758	-6689969807.50652\\
1.18022950573764	-6681604623.69761\\
1.18032950823771	-6673182144.10919\\
1.18042951073777	-6664816960.30027\\
1.18052951323783	-6656451776.49137\\
1.18062951573789	-6648086592.68246\\
1.18072951823796	-6639664113.09403\\
1.18082952073802	-6631356225.06464\\
1.18092952323808	-6622991041.25572\\
1.18102952573814	-6614625857.44681\\
1.18112952823821	-6606260673.63791\\
1.18122953073827	-6597895489.82899\\
1.18132953323833	-6589587601.7996\\
1.18142953573839	-6581279713.7702\\
1.18152953823846	-6572914529.96129\\
1.18162954073852	-6564606641.93189\\
1.18172954323858	-6556298753.9025\\
1.18182954573864	-6547990865.8731\\
1.18192954823871	-6539682977.8437\\
1.18202955073877	-6531375089.81431\\
1.18212955323883	-6523067201.78491\\
1.18222955573889	-6514816609.53503\\
1.18232955823896	-6506508721.50563\\
1.18242956073902	-6498258129.25574\\
1.18252956323908	-6489950241.22635\\
1.18262956573914	-6481699648.97646\\
1.18272956823921	-6473449056.72658\\
1.18282957073927	-6465198464.4767\\
1.18292957323933	-6456947872.22681\\
1.18302957573939	-6448697279.97693\\
1.18312957823946	-6440446687.72705\\
1.18322958073952	-6432196095.47716\\
1.18332958323958	-6423945503.22728\\
1.18342958573964	-6415752206.75691\\
1.18352958823971	-6407501614.50702\\
1.18362959073977	-6399308318.03665\\
1.18372959323983	-6391115021.56628\\
1.18382959573989	-6382921725.09591\\
1.18392959823996	-6374728428.62554\\
1.18402960074002	-6366535132.15517\\
1.18412960324008	-6358341835.6848\\
1.18422960574014	-6350148539.21443\\
1.18432960824021	-6341955242.74406\\
1.18442961074027	-6333819242.0532\\
1.18452961324033	-6325625945.58283\\
1.18462961574039	-6317489944.89197\\
1.18472961824046	-6309296648.4216\\
1.18482962074052	-6301160647.73074\\
1.18492962324058	-6293024647.03988\\
1.18502962574064	-6284888646.34903\\
1.18512962824071	-6276752645.65817\\
1.18522963074077	-6268616644.96731\\
1.18532963324083	-6260480644.27645\\
1.18542963574089	-6252401939.36511\\
1.18552963824096	-6244265938.67425\\
1.18562964074102	-6236187233.76291\\
1.18572964324108	-6228051233.07205\\
1.18582964574114	-6219972528.1607\\
1.18592964824121	-6211893823.24936\\
1.18602965074127	-6203815118.33801\\
1.18612965324133	-6195736413.42667\\
1.18622965574139	-6187657708.51533\\
1.18632965824146	-6179579003.60398\\
1.18642966074152	-6171500298.69264\\
1.18652966324158	-6163478889.5608\\
1.18662966574164	-6155400184.64946\\
1.18672966824171	-6147378775.51763\\
1.18682967074177	-6139300070.60628\\
1.18692967324183	-6131278661.47445\\
1.18702967574189	-6123257252.34262\\
1.18712967824196	-6115235843.21079\\
1.18722968074202	-6107214434.07896\\
1.18732968324208	-6099193024.94713\\
1.18742968574214	-6091171615.8153\\
1.18752968824221	-6083207502.46298\\
1.18762969074227	-6075186093.33115\\
1.18772969324233	-6067221979.97883\\
1.18782969574239	-6059200570.847\\
1.18792969824246	-6051236457.49468\\
1.18802970074252	-6043272344.14236\\
1.18812970324258	-6035250935.01053\\
1.18822970574264	-6027286821.65821\\
1.18832970824271	-6019322708.30589\\
1.18842971074277	-6011415890.73308\\
1.18852971324283	-6003451777.38077\\
1.18862971574289	-5995487664.02845\\
1.18872971824296	-5987580846.45564\\
1.18882972074302	-5979616733.10332\\
1.18892972324308	-5971709915.53052\\
1.18902972574314	-5963803097.95771\\
1.18912972824321	-5955838984.60539\\
1.18922973074327	-5947932167.03259\\
1.18932973324333	-5940025349.45978\\
1.18942973574339	-5932118531.88698\\
1.18952973824346	-5924269010.09369\\
1.18962974074352	-5916362192.52088\\
1.18972974324358	-5908455374.94808\\
1.18982974574364	-5900605853.15478\\
1.18992974824371	-5892699035.58198\\
1.19002975074377	-5884849513.78869\\
1.19012975324383	-5876999991.99539\\
1.19022975574389	-5869093174.42259\\
1.19032975824396	-5861243652.6293\\
1.19042976074402	-5853394130.836\\
1.19052976324408	-5845601904.82222\\
1.19062976574414	-5837752383.02893\\
1.19072976824421	-5829902861.23564\\
1.19082977074427	-5822053339.44235\\
1.19092977324433	-5814261113.42857\\
1.19102977574439	-5806468887.41479\\
1.19112977824446	-5798619365.6215\\
1.19122978074452	-5790827139.60772\\
1.19132978324458	-5783034913.59394\\
1.19142978574464	-5775242687.58016\\
1.19152978824471	-5767450461.56638\\
1.19162979074477	-5759658235.5526\\
1.19172979324483	-5751866009.53882\\
1.19182979574489	-5744131079.30455\\
1.19192979824496	-5736338853.29078\\
1.19202980074502	-5728581004.7447\\
1.19212980324508	-5720828885.77658\\
1.19222980574514	-5713076766.80847\\
1.19232980824521	-5705330377.4183\\
1.19242981074527	-5697589717.60608\\
1.19252981324533	-5689854787.37181\\
1.19262981574539	-5682125586.7155\\
1.19272981824546	-5674402115.63713\\
1.19282982074552	-5666678644.55877\\
1.19292982324558	-5658966632.63631\\
1.19302982574564	-5651260350.2918\\
1.19312982824571	-5643554067.94729\\
1.19322983074577	-5635859244.75868\\
1.19332983324583	-5628164421.57008\\
1.19342983574589	-5620481057.53737\\
1.19352983824596	-5612797693.50467\\
1.19362984074602	-5605120059.04992\\
1.19372984324608	-5597448154.17311\\
1.19382984574614	-5589781978.87426\\
1.19392984824621	-5582127262.73131\\
1.19402985074627	-5574472546.58837\\
1.19412985324633	-5566817830.44542\\
1.19422985574639	-5559174573.45837\\
1.19432985824646	-5551537046.04928\\
1.19442986074652	-5543905248.21814\\
1.19452986324658	-5536279179.96495\\
1.19462986574664	-5528653111.71175\\
1.19472986824671	-5521038502.61447\\
1.19482987074677	-5513423893.51718\\
1.19492987324683	-5505820743.57579\\
1.19502987574689	-5498217593.63441\\
1.19512987824696	-5490620173.27097\\
1.19522988074702	-5483028482.48549\\
1.19532988324708	-5475448250.85591\\
1.19542988574714	-5467868019.22633\\
1.19552988824721	-5460293517.1747\\
1.19562989074727	-5452724744.70102\\
1.19572989324733	-5445155972.22734\\
1.19582989574739	-5437598658.90956\\
1.19592989824746	-5430047075.16974\\
1.19602990074752	-5422501221.00787\\
1.19612990324758	-5414955366.84599\\
1.19622990574764	-5407420971.84002\\
1.19632990824771	-5399886576.83405\\
1.19642991074777	-5392363640.98399\\
1.19652991324783	-5384840705.13392\\
1.19662991574789	-5377323498.8618\\
1.19672991824796	-5369817751.74559\\
1.19682992074802	-5362312004.62937\\
1.19692992324808	-5354811987.09111\\
1.19702992574814	-5347317699.1308\\
1.19712992824821	-5339829140.74844\\
1.19722993074827	-5332346311.94403\\
1.19732993324833	-5324869212.71758\\
1.19742993574839	-5317392113.49112\\
1.19752993824846	-5309926473.42056\\
1.19762994074852	-5302466562.92796\\
1.19772994324858	-5295006652.43536\\
1.19782994574864	-5287558201.09866\\
1.19792994824871	-5280109749.76196\\
1.19802995074877	-5272667028.00321\\
1.19812995324883	-5265235765.40036\\
1.19822995574889	-5257804502.79751\\
1.19832995824896	-5250378969.77262\\
1.19842996074902	-5242959166.32567\\
1.19852996324908	-5235545092.45668\\
1.19862996574914	-5228136748.16564\\
1.19872996824921	-5220734133.45255\\
1.19882997074927	-5213337248.31741\\
1.19892997324933	-5205946092.76022\\
1.19902997574939	-5198554937.20303\\
1.19912997824946	-5191175240.80175\\
1.19922998074952	-5183801273.97842\\
1.19932998324958	-5176427307.15508\\
1.19942998574964	-5169059069.9097\\
1.19952998824971	-5161702291.82022\\
1.19962999074977	-5154345513.73074\\
1.19972999324983	-5146994465.21921\\
1.19982999574989	-5139649146.28563\\
1.19992999824996	-5132309556.93001\\
1.20003000075002	-5124981426.73028\\
};
\addplot [color=mycolor3,solid,forget plot]
  table[row sep=crcr]{%
1.20003000075002	-5124981426.73028\\
1.20013000325008	-5117647566.95261\\
1.20023000575014	-5110325166.33084\\
1.20033000825021	-5103008495.28702\\
1.20043001075027	-5095697553.82115\\
1.20053001325033	-5088392341.93323\\
1.20063001575039	-5081087130.04531\\
1.20073001825046	-5073793377.3133\\
1.20083002075052	-5066499624.58128\\
1.20093002325058	-5059217331.00517\\
1.20103002575064	-5051935037.42906\\
1.20113002825071	-5044658473.43089\\
1.20123003075077	-5037387639.01068\\
1.20133003325083	-5030122534.16843\\
1.20143003575089	-5022863158.90412\\
1.20153003825096	-5015609513.21776\\
1.20163004075102	-5008361597.10936\\
1.20173004325108	-5001119410.5789\\
1.20183004575114	-4993882953.6264\\
1.20193004825121	-4986652226.25185\\
1.20203005075127	-4979421498.8773\\
1.20213005325133	-4972202230.65865\\
1.20223005575139	-4964982962.44\\
1.20233005825146	-4957775153.37726\\
1.20243006075152	-4950567344.31451\\
1.20253006325158	-4943370994.40767\\
1.20263006575164	-4936174644.50082\\
1.20273006825171	-4928984024.17193\\
1.20283007075177	-4921799133.42099\\
1.20293007325183	-4914619972.248\\
1.20303007575189	-4907446540.65297\\
1.20313007825196	-4900278838.63588\\
1.20323008075202	-4893116866.19674\\
1.20333008325208	-4885954893.75761\\
1.20343008575214	-4878804380.47438\\
1.20353008825221	-4871659596.76909\\
1.20363009075227	-4864514813.06381\\
1.20373009325233	-4857381488.51443\\
1.20383009575239	-4850248163.96506\\
1.20393009825246	-4843126298.57158\\
1.20403010075252	-4836004433.1781\\
1.20413010325258	-4828888297.36258\\
1.20423010575264	-4821777891.125\\
1.20433010825271	-4814673214.46538\\
1.20443011075277	-4807574267.38371\\
1.20453011325283	-4800481049.87999\\
1.20463011575289	-4793393561.95422\\
1.20473011825296	-4786311803.60641\\
1.20483012075302	-4779230045.25859\\
1.20493012325308	-4772159746.06668\\
1.20503012575314	-4765095176.45271\\
1.20513012825321	-4758030606.83875\\
1.20523013075327	-4750971766.80274\\
1.20533013325333	-4743924385.92263\\
1.20543013575339	-4736877005.04252\\
1.20553013825346	-4729835353.74036\\
1.20563014075352	-4722799432.01616\\
1.20573014325358	-4715769239.8699\\
1.20583014575364	-4708744777.3016\\
1.20593014825371	-4701726044.31124\\
1.20603015075377	-4694713040.89884\\
1.20613015325383	-4687705767.06439\\
1.20623015575389	-4680704222.80789\\
1.20633015825396	-4673702678.55139\\
1.20643016075402	-4666712593.4508\\
1.20653016325408	-4659722508.3502\\
1.20663016575414	-4652743882.40551\\
1.20673016825421	-4645765256.46082\\
1.20683017075427	-4638798089.67202\\
1.20693017325433	-4631830922.88323\\
1.20703017575439	-4624869485.67239\\
1.20713017825446	-4617913778.03951\\
1.20723018075452	-4610963799.98457\\
1.20733018325458	-4604019551.50758\\
1.20743018575464	-4597081032.60855\\
1.20753018825471	-4590148243.28747\\
1.20763019075477	-4583215453.96638\\
1.20773019325483	-4576294123.8012\\
1.20783019575489	-4569378523.21397\\
1.20793019825496	-4562462922.62675\\
1.20803020075502	-4555558781.19542\\
1.20813020325508	-4548654639.76409\\
1.20823020575514	-4541756227.91072\\
1.20833020825521	-4534863545.63529\\
1.20843021075527	-4527982322.51577\\
1.20853021325533	-4521101099.39625\\
1.20863021575539	-4514225605.85468\\
1.20873021825546	-4507355841.89106\\
1.20883022075552	-4500486077.92744\\
1.20893022325558	-4493627773.11973\\
1.20903022575564	-4486775197.88996\\
1.20913022825571	-4479928352.23815\\
1.20923023075577	-4473081506.58634\\
1.20933023325583	-4466246120.09043\\
1.20943023575589	-4459410733.59452\\
1.20953023825596	-4452581076.67656\\
1.20963024075602	-4445762878.9145\\
1.20973024325608	-4438944681.15244\\
1.20983024575614	-4432132212.96834\\
1.20993024825621	-4425325474.36218\\
1.21003025075627	-4418524465.33398\\
1.21013025325633	-4411729185.88373\\
1.21023025575639	-4404939636.01143\\
1.21033025825646	-4398155815.71708\\
1.21043026075652	-4391371995.42273\\
1.21053026325658	-4384599634.28428\\
1.21063026575664	-4377833002.72379\\
1.21073026825671	-4371066371.16329\\
1.21083027075677	-4364305469.18075\\
1.21093027325683	-4357556026.35411\\
1.21103027575689	-4350806583.52747\\
1.21113027825696	-4344062870.27878\\
1.21123028075702	-4337324886.60804\\
1.21133028325708	-4330592632.51525\\
1.21143028575714	-4323866108.00042\\
1.21153028825721	-4317145313.06353\\
1.21163029075727	-4310430247.7046\\
1.21173029325733	-4303720911.92362\\
1.21183029575739	-4297017305.72059\\
1.21193029825746	-4290313699.51756\\
1.21203030075752	-4283621552.47043\\
1.21213030325758	-4276929405.4233\\
1.21223030575764	-4270248717.53207\\
1.21233030825771	-4263568029.64085\\
1.21243031075777	-4256893071.32757\\
1.21253031325783	-4250223842.59225\\
1.21263031575789	-4243560343.43488\\
1.21273031825796	-4236908303.43341\\
1.21283032075802	-4230250533.85399\\
1.21293032325808	-4223604223.43047\\
1.21303032575814	-4216963642.58491\\
1.21313032825821	-4210328791.31729\\
1.21323033075827	-4203699669.62763\\
1.21333033325833	-4197070547.93797\\
1.21343033575839	-4190452885.4042\\
1.21353033825846	-4183835222.87044\\
1.21363034075852	-4177223289.91463\\
1.21373034325858	-4170622816.11473\\
1.21383034575864	-4164022342.31482\\
1.21393034825871	-4157427598.09286\\
1.21403035075877	-4150838583.44886\\
1.21413035325883	-4144255298.38281\\
1.21423035575889	-4137677742.8947\\
1.21433035825896	-4131105916.98455\\
1.21443036075902	-4124539820.65235\\
1.21453036325908	-4117973724.32016\\
1.21463036575914	-4111419087.14386\\
1.21473036825921	-4104864449.96756\\
1.21483037075927	-4098321271.94717\\
1.21493037325933	-4091778093.92677\\
1.21503037575939	-4085246375.06228\\
1.21513037825946	-4078714656.19779\\
1.21523038075952	-4072188666.91125\\
1.21533038325958	-4065668407.20266\\
1.21543038575964	-4059153877.07202\\
1.21553038825971	-4052645076.51934\\
1.21563039075977	-4046142005.5446\\
1.21573039325983	-4039644664.14782\\
1.21583039575989	-4033153052.32899\\
1.21593039825996	-4026661440.51016\\
1.21603040076002	-4020181287.84723\\
1.21613040326008	-4013701135.1843\\
1.21623040576014	-4007232441.67727\\
1.21633040826021	-4000763748.17024\\
1.21643041076027	-3994300784.24117\\
1.21653041326033	-3987849279.46799\\
1.21663041576039	-3981397774.69482\\
1.21673041826046	-3974951999.4996\\
1.21683042076052	-3968511953.88233\\
1.21693042326058	-3962077637.84301\\
1.21703042576064	-3955643321.80369\\
1.21713042826071	-3949220464.92027\\
1.21723043076077	-3942803337.61481\\
1.21733043326083	-3936386210.30934\\
1.21743043576089	-3929980542.15978\\
1.21753043826096	-3923574874.01022\\
1.21763044076102	-3917180665.01656\\
1.21773044326108	-3910786456.0229\\
1.21783044576114	-3904397976.60719\\
1.21793044826121	-3898015226.76943\\
1.21803045076127	-3891638206.50963\\
1.21813045326133	-3885266915.82777\\
1.21823045576139	-3878901354.72387\\
1.21833045826146	-3872541523.19792\\
1.21843046076152	-3866187421.24992\\
1.21853046326158	-3859839048.87987\\
1.21863046576164	-3853490676.50982\\
1.21873046826171	-3847153763.29567\\
1.21883047076177	-3840816850.08152\\
1.21893047326183	-3834491396.02328\\
1.21903047576189	-3828165941.96503\\
1.21913047826196	-3821846217.48474\\
1.21923048076202	-3815532222.5824\\
1.21933048326208	-3809223957.25801\\
1.21943048576214	-3802921421.51157\\
1.21953048826221	-3796624615.34308\\
1.21963049076227	-3790333538.75255\\
1.21973049326233	-3784048191.73996\\
1.21983049576239	-3777768574.30533\\
1.21993049826246	-3771488956.87069\\
1.22003050076252	-3765220798.59196\\
1.22013050326258	-3758952640.31323\\
1.22023050576264	-3752695941.1904\\
1.22033050826271	-3746439242.06757\\
1.22043051076277	-3740188272.5227\\
1.22053051326283	-3733943032.55577\\
1.22063051576289	-3727703522.1668\\
1.22073051826296	-3721469741.35577\\
1.22083052076302	-3715241690.1227\\
1.22093052326308	-3709019368.46758\\
1.22103052576314	-3702802776.39041\\
1.22113052826321	-3696591913.89119\\
1.22123053076327	-3690381051.39197\\
1.22133053326333	-3684181648.04866\\
1.22143053576339	-3677982244.70534\\
1.22153053826346	-3671794300.51793\\
1.22163054076352	-3665606356.33052\\
1.22173054326358	-3659424141.72106\\
1.22183054576364	-3653247656.68954\\
1.22193054826371	-3647076901.23599\\
1.22203055076377	-3640911875.36038\\
1.22213055326383	-3634752579.06272\\
1.22223055576389	-3628599012.34302\\
1.22233055826396	-3622451175.20126\\
1.22243056076402	-3616303338.05951\\
1.22253056326408	-3610166960.07366\\
1.22263056576414	-3604036311.66576\\
1.22273056826421	-3597905663.25786\\
1.22283057076427	-3591780744.42791\\
1.22293057326433	-3585667284.75386\\
1.22303057576439	-3579553825.07982\\
1.22313057826446	-3573446094.98372\\
1.22323058076452	-3567344094.46558\\
1.22333058326458	-3561247823.52539\\
1.22343058576464	-3555157282.16315\\
1.22353058826471	-3549072470.37886\\
1.22363059076477	-3542993388.17252\\
1.22373059326483	-3536914305.96618\\
1.22383059576489	-3530846682.91575\\
1.22393059826496	-3524779059.86531\\
1.22403060076502	-3518722895.97078\\
1.22413060326508	-3512666732.07625\\
1.22423060576514	-3506622027.33762\\
1.22433060826521	-3500577322.59899\\
1.22443061076527	-3494538347.43831\\
1.22453061326533	-3488505101.85558\\
1.22463061576539	-3482477585.8508\\
1.22473061826546	-3476455799.42398\\
1.22483062076552	-3470439742.5751\\
1.22493062326558	-3464423685.72623\\
1.22503062576564	-3458419088.03326\\
1.22513062826571	-3452414490.34029\\
1.22523063076577	-3446421351.80322\\
1.22533063326583	-3440428213.26615\\
1.22543063576589	-3434446533.88499\\
1.22553063826596	-3428464854.50382\\
1.22563064076602	-3422488904.70061\\
1.22573064326608	-3416518684.47534\\
1.22583064576614	-3410554193.82803\\
1.22593064826621	-3404595432.75867\\
1.22603065076627	-3398642401.26726\\
1.22613065326633	-3392695099.3538\\
1.22623065576639	-3386753527.0183\\
1.22633065826646	-3380811954.68279\\
1.22643066076652	-3374881841.50319\\
1.22653066326658	-3368951728.32358\\
1.22663066576664	-3363033074.29988\\
1.22673066826671	-3357114420.27618\\
1.22683067076677	-3351201495.83043\\
1.22693067326683	-3345294300.96263\\
1.22703067576689	-3339392835.67278\\
1.22713067826696	-3333497099.96089\\
1.22723068076702	-3327607093.82694\\
1.22733068326708	-3321722817.27095\\
1.22743068576714	-3315844270.29291\\
1.22753068826721	-3309971452.89281\\
1.22763069076727	-3304098635.49272\\
1.22773069326733	-3298237277.24854\\
1.22783069576739	-3292375919.00435\\
1.22793069826746	-3286520290.33811\\
1.22803070076752	-3280676120.82778\\
1.22813070326758	-3274831951.31744\\
1.22823070576764	-3268993511.38506\\
1.22833070826771	-3263160801.03063\\
1.22843071076777	-3257333820.25415\\
1.22853071326783	-3251512569.05562\\
1.22863071576789	-3245697047.43504\\
1.22873071826796	-3239881525.81446\\
1.22883072076802	-3234077463.34979\\
1.22893072326808	-3228279130.46306\\
1.22903072576814	-3222480797.57634\\
1.22913072826821	-3216688194.26757\\
1.22923073076827	-3210907050.1147\\
1.22933073326833	-3205125905.96183\\
1.22943073576839	-3199350491.38691\\
1.22953073826846	-3193580806.38994\\
1.22963074076852	-3187816850.97092\\
1.22973074326858	-3182058625.12986\\
1.22983074576864	-3176306128.86674\\
1.22993074826871	-3170559362.18158\\
1.23003075076877	-3164812595.49642\\
1.23013075326883	-3159077287.96716\\
1.23023075576889	-3153341980.4379\\
1.23033075826896	-3147618132.06454\\
1.23043076076902	-3141894283.69119\\
1.23053076326908	-3136176164.89578\\
1.23063076576914	-3130469505.25628\\
1.23073076826921	-3124762845.61678\\
1.23083077076927	-3119061915.55522\\
1.23093077326933	-3113366715.07162\\
1.23103077576939	-3107677244.16597\\
1.23113077826946	-3101987773.26033\\
1.23123078076952	-3096309761.51058\\
1.23133078326958	-3090637479.33878\\
1.23143078576964	-3084965197.16699\\
1.23153078826971	-3079304374.1511\\
1.23163079076977	-3073643551.1352\\
1.23173079326983	-3067988457.69726\\
1.23183079576989	-3062339093.83727\\
1.23193079826996	-3056701189.13319\\
1.23203080077002	-3051063284.4291\\
1.23213080327008	-3045431109.30296\\
1.23223080577014	-3039798934.17683\\
1.23233080827021	-3034178218.20659\\
1.23243081077027	-3028563231.81431\\
1.23253081327033	-3022953974.99998\\
1.23263081577039	-3017344718.18565\\
1.23273081827046	-3011746920.52722\\
1.23283082077052	-3006149122.86879\\
1.23293082327058	-3000557054.78832\\
1.23303082577064	-2994976445.86374\\
1.23313082827071	-2989395836.93917\\
1.23323083077077	-2983820957.59254\\
1.23333083327083	-2978251807.82387\\
1.23343083577089	-2972688387.63315\\
1.23353083827096	-2967130697.02038\\
1.23363084077102	-2961573006.40761\\
1.23373084327108	-2956026774.95075\\
1.23383084577114	-2950480543.49388\\
1.23393084827121	-2944945771.19292\\
1.23403085077127	-2939410998.89195\\
1.23413085327133	-2933887685.74689\\
1.23423085577139	-2928364372.60183\\
1.23433085827146	-2922846789.03472\\
1.23443086077152	-2917334935.04556\\
1.23453086327158	-2911828810.63436\\
1.23463086577164	-2906328415.8011\\
1.23473086827171	-2900833750.5458\\
1.23483087077177	-2895344814.86844\\
1.23493087327183	-2889855879.19109\\
1.23503087577189	-2884378402.66964\\
1.23513087827196	-2878900926.14819\\
1.23523088077202	-2873434908.78264\\
1.23533088327208	-2867968891.41709\\
1.23543088577214	-2862508603.6295\\
1.23553088827221	-2857059774.9978\\
1.23563089077227	-2851610946.36611\\
1.23573089327233	-2846167847.31236\\
1.23583089577239	-2840730477.83657\\
1.23593089827246	-2835298837.93873\\
1.23603090077252	-2829867198.04089\\
1.23613090327258	-2824447017.29895\\
1.23623090577264	-2819032566.13497\\
1.23633090827271	-2813618114.97098\\
1.23643091077277	-2808209393.38495\\
1.23653091327283	-2802812130.95481\\
1.23663091577289	-2797414868.52468\\
1.23673091827296	-2792023335.6725\\
1.23683092077302	-2786637532.39827\\
1.23693092327308	-2781257458.70199\\
1.23703092577314	-2775883114.58367\\
1.23713092827321	-2770514500.04329\\
1.23723093077327	-2765151615.08087\\
1.23733093327333	-2759794459.69639\\
1.23743093577339	-2754437304.31192\\
1.23753093827346	-2749091608.08335\\
1.23763094077352	-2743745911.85478\\
1.23773094327358	-2738405945.20416\\
1.23783094577364	-2733077437.70944\\
1.23793094827371	-2727748930.21473\\
1.23803095077377	-2722426152.29796\\
1.23813095327383	-2717109103.95915\\
1.23823095577389	-2711797785.19828\\
1.23833095827396	-2706492196.01537\\
1.23843096077402	-2701186606.83246\\
1.23853096327408	-2695892476.80545\\
1.23863096577414	-2690604076.35639\\
1.23873096827421	-2685315675.90734\\
1.23883097077427	-2680038734.61418\\
1.23893097327433	-2674761793.32103\\
1.23903097577439	-2669490581.60582\\
1.23913097827446	-2664225099.46857\\
1.23923098077452	-2658971076.48722\\
1.23933098327458	-2653717053.50587\\
1.23943098577464	-2648463030.52452\\
1.23953098827471	-2643220466.69908\\
1.23963099077477	-2637983632.45158\\
1.23973099327483	-2632752527.78204\\
1.23983099577489	-2627521423.11249\\
1.23993099827496	-2622301777.59885\\
1.24003100077502	-2617082132.08521\\
1.24013100327508	-2611873945.72747\\
1.24023100577514	-2606665759.36973\\
1.24033100827521	-2601463302.58994\\
1.24043101077527	-2596266575.3881\\
1.24053101327533	-2591075577.76422\\
1.24063101577539	-2585890309.71829\\
1.24073101827546	-2580710771.2503\\
1.24083102077552	-2575531232.78232\\
1.24093102327558	-2570363153.47024\\
1.24103102577564	-2565200803.73611\\
1.24113102827571	-2560038454.00198\\
1.24123103077577	-2554887563.42376\\
1.24133103327583	-2549736672.84553\\
1.24143103577589	-2544591511.84526\\
1.24153103827596	-2539452080.42293\\
1.24163104077602	-2534318378.57856\\
1.24173104327608	-2529190406.31214\\
1.24183104577614	-2524068163.62367\\
1.24193104827621	-2518951650.51315\\
1.24203105077627	-2513840866.98058\\
1.24213105327633	-2508730083.44802\\
1.24223105577639	-2503630759.07135\\
1.24233105827646	-2498531434.69469\\
1.24243106077652	-2493443569.47393\\
1.24253106327658	-2488355704.25317\\
1.24263106577664	-2483273568.61035\\
1.24273106827671	-2478197162.5455\\
1.24283107077677	-2473126486.05859\\
1.24293107327683	-2468061539.14963\\
1.24303107577689	-2463002321.81863\\
1.24313107827696	-2457948834.06557\\
1.24323108077702	-2452901075.89047\\
1.24333108327708	-2447853317.71537\\
1.24343108577714	-2442817018.69617\\
1.24353108827721	-2437780719.67697\\
1.24363109077727	-2432755879.81367\\
1.24373109327733	-2427731039.95037\\
1.24383109577739	-2422711929.66503\\
1.24393109827746	-2417698548.95763\\
1.24403110077752	-2412690897.82819\\
1.24413110327758	-2407688976.2767\\
1.24423110577764	-2402692784.30316\\
1.24433110827771	-2397702321.90757\\
1.24443111077777	-2392711859.51198\\
1.24453111327783	-2387732856.27229\\
1.24463111577789	-2382753853.0326\\
1.24473111827796	-2377786308.94882\\
1.24483112077802	-2372818764.86503\\
1.24493112327808	-2367856950.3592\\
1.24503112577814	-2362906595.00927\\
1.24513112827821	-2357956239.65934\\
1.24523113077827	-2353011613.88736\\
1.24533113327833	-2348066988.11538\\
1.24543113577839	-2343133821.49931\\
1.24553113827846	-2338206384.46118\\
1.24563114077852	-2333284677.00101\\
1.24573114327858	-2328362969.54083\\
1.24583114577864	-2323452721.23656\\
1.24593114827871	-2318542472.93229\\
1.24603115077877	-2313637954.20597\\
1.24613115327883	-2308744894.63555\\
1.24623115577889	-2303851835.06514\\
1.24633115827896	-2298964505.07267\\
1.24643116077902	-2294082904.65816\\
1.24653116327908	-2289201304.24364\\
1.24663116577914	-2284331162.98503\\
1.24673116827921	-2279466751.30437\\
1.24683117077927	-2274608069.20166\\
1.24693117327933	-2269749387.09895\\
1.24703117577939	-2264896434.57419\\
1.24713117827946	-2260054941.20534\\
1.24723118077952	-2255213447.83648\\
1.24733118327958	-2250377684.04558\\
1.24743118577964	-2245547649.83262\\
1.24753118827971	-2240723345.19762\\
1.24763119077977	-2235904770.14057\\
1.24773119327983	-2231091924.66147\\
1.24783119577989	-2226284808.76033\\
1.24793119827996	-2221477692.85918\\
1.24803120078002	-2216682036.11393\\
1.24813120328008	-2211886379.36869\\
1.24823120578014	-2207102181.77935\\
1.24833120828021	-2202317984.19\\
1.24843121078027	-2197539516.17861\\
1.24853121328033	-2192766777.74517\\
1.24863121578039	-2187999768.88969\\
1.24873121828046	-2183238489.61215\\
1.24883122078052	-2178482939.91256\\
1.24893122328058	-2173733119.79093\\
1.24903122578064	-2168989029.24724\\
1.24913122828071	-2164244938.70356\\
1.24923123078077	-2159512307.31578\\
1.24933123328083	-2154779675.928\\
1.24943123578089	-2150058503.69612\\
1.24953123828096	-2145337331.46424\\
1.24963124078102	-2140621888.81032\\
1.24973124328108	-2135912175.73434\\
1.24983124578114	-2131208192.23632\\
1.24993124828121	-2126509938.31625\\
1.25003125078127	-2121817413.97412\\
1.25013125328133	-2117124889.632\\
1.25023125578139	-2112443824.44578\\
1.25033125828146	-2107768488.83752\\
1.25043126078152	-2103093153.22925\\
1.25053126328158	-2098423547.19893\\
1.25063126578164	-2093765400.32452\\
1.25073126828171	-2089107253.4501\\
1.25083127078177	-2084454836.15364\\
1.25093127328183	-2079808148.43513\\
1.25103127578189	-2075167190.29457\\
1.25113127828196	-2070531961.73196\\
1.25123128078202	-2065902462.74731\\
1.25133128328208	-2061272963.76265\\
1.25143128578214	-2056654923.9339\\
1.25153128828221	-2052036884.10514\\
1.25163129078227	-2047430303.43229\\
1.25173129328233	-2042823722.75944\\
1.25183129578239	-2038222871.66454\\
1.25193129828246	-2033627750.14759\\
1.25203130078252	-2029038358.20859\\
1.25213130328258	-2024454695.84754\\
1.25223130578264	-2019876763.06445\\
1.25233130828271	-2015304559.8593\\
1.25243131078277	-2010738086.23211\\
1.25253131328283	-2006171612.60492\\
1.25263131578289	-2001616598.13363\\
1.25273131828296	-1997061583.66234\\
1.25283132078302	-1992518028.34695\\
1.25293132328308	-1987974473.03156\\
1.25303132578314	-1983436647.29413\\
1.25313132828321	-1978904551.13464\\
1.25323133078327	-1974378184.55311\\
1.25333133328333	-1969857547.54953\\
1.25343133578339	-1965342640.1239\\
1.25353133828346	-1960833462.27622\\
1.25363134078352	-1956324284.42854\\
1.25373134328358	-1951826565.73676\\
1.25383134578364	-1947328847.04498\\
1.25393134828371	-1942842587.50911\\
1.25403135078377	-1938356327.97323\\
1.25413135328383	-1933875798.01531\\
1.25423135578389	-1929400997.63534\\
1.25433135828396	-1924931926.83332\\
1.25443136078402	-1920468585.60925\\
1.25453136328408	-1916010973.96313\\
1.25463136578414	-1911559091.89497\\
1.25473136828421	-1907107209.8268\\
1.25483137078427	-1902666786.91454\\
1.25493137328433	-1898226364.00227\\
1.25503137578439	-1893797400.24591\\
1.25513137828446	-1889368436.48955\\
1.25523138078452	-1884945202.31114\\
1.25533138328458	-1880527697.71068\\
1.25543138578464	-1876115922.68817\\
1.25553138828471	-1871709877.24362\\
1.25563139078477	-1867309561.37701\\
1.25573139328483	-1862914975.08836\\
1.25583139578489	-1858526118.37766\\
1.25593139828496	-1854137261.66695\\
1.25603140078502	-1849759864.11215\\
1.25613140328508	-1845382466.55736\\
1.25623140578514	-1841010798.58051\\
1.25633140828521	-1836650589.75956\\
1.25643141078527	-1832290380.93862\\
1.25653141328533	-1827935901.69562\\
1.25663141578539	-1823587152.03058\\
1.25673141828546	-1819244131.94349\\
1.25683142078552	-1814901111.8564\\
1.25693142328558	-1810569550.92521\\
1.25703142578564	-1806243719.57197\\
1.25713142828571	-1801917888.21873\\
1.25723143078577	-1797603516.0214\\
1.25733143328583	-1793289143.82406\\
1.25743143578589	-1788980501.20468\\
1.25753143828596	-1784677588.16324\\
1.25763144078602	-1780380404.69976\\
1.25773144328608	-1776088950.81423\\
1.25783144578614	-1771803226.50666\\
1.25793144828621	-1767523231.77703\\
1.25803145078627	-1763248966.62535\\
1.25813145328633	-1758974701.47368\\
1.25823145578639	-1754711895.4779\\
1.25833145828646	-1750449089.48213\\
1.25843146078652	-1746197742.64226\\
1.25853146328658	-1741946395.80239\\
1.25863146578664	-1737700778.54047\\
1.25873146828671	-1733460890.8565\\
1.25883147078677	-1729226732.75048\\
1.25893147328683	-1724998304.22242\\
1.25903147578689	-1720775605.2723\\
1.25913147828696	-1716552906.32219\\
1.25923148078702	-1712341666.52798\\
1.25933148328708	-1708130426.73377\\
1.25943148578714	-1703930646.09546\\
1.25953148828721	-1699730865.45715\\
1.25963149078727	-1695536814.39679\\
1.25973149328733	-1691354222.49234\\
1.25983149578739	-1687171630.58788\\
1.25993149828746	-1682994768.26138\\
1.26003150078752	-1678823635.51283\\
1.26013150328758	-1674652502.76427\\
1.26023150578764	-1670492829.17162\\
1.26033150828771	-1666338885.15692\\
1.26043151078777	-1662184941.14223\\
1.26053151328783	-1658042456.28343\\
1.26063151578789	-1653899971.42463\\
1.26073151828796	-1649763216.14379\\
1.26083152078802	-1645632190.4409\\
1.26093152328808	-1641506894.31595\\
1.26103152578814	-1637387327.76896\\
1.26113152828821	-1633273490.79992\\
1.26123153078827	-1629165383.40884\\
1.26133153328833	-1625063005.5957\\
1.26143153578839	-1620960627.78256\\
1.26153153828846	-1616869709.12533\\
1.26163154078852	-1612778790.4681\\
1.26173154328858	-1608699330.96676\\
1.26183154578864	-1604619871.46543\\
1.26193154828871	-1600546141.54205\\
1.26203155078877	-1596478141.19662\\
1.26213155328883	-1592415870.42915\\
1.26223155578889	-1588359329.23962\\
1.26233155828896	-1584308517.62804\\
1.26243156078902	-1580263435.59442\\
1.26253156328908	-1576218353.5608\\
1.26263156578914	-1572184730.68308\\
1.26273156828921	-1568151107.80536\\
1.26283157078927	-1564123214.50559\\
1.26293157328933	-1560106780.36172\\
1.26303157578939	-1556090346.21785\\
1.26313157828946	-1552079641.65194\\
1.26323158078952	-1548074666.66397\\
1.26333158328958	-1544075421.25396\\
1.26343158578964	-1540081905.4219\\
1.26353158828971	-1536088389.58983\\
1.26363159078977	-1532106332.91368\\
1.26373159328983	-1528124276.23752\\
1.26383159578989	-1524153678.71726\\
1.26393159828996	-1520183081.197\\
1.26403160079002	-1516218213.2547\\
1.26413160329008	-1512264804.46829\\
1.26423160579014	-1508311395.68189\\
1.26433160829021	-1504363716.47344\\
1.26443161079027	-1500416037.26499\\
1.26453161329033	-1496479817.21244\\
1.26463161579039	-1492549326.73784\\
1.26473161829046	-1488618836.26325\\
1.26483162079052	-1484699804.94455\\
1.26493162329058	-1480780773.62586\\
1.26503162579064	-1476873201.46306\\
1.26513162829071	-1472965629.30027\\
1.26523163079077	-1469063786.71543\\
1.26533163329083	-1465167673.70854\\
1.26543163579089	-1461277290.2796\\
1.26553163829096	-1457392636.42862\\
1.26563164079102	-1453513712.15558\\
1.26573164329108	-1449634787.88254\\
1.26583164579114	-1445767322.76541\\
1.26593164829121	-1441899857.64828\\
1.26603165079127	-1438043851.68705\\
1.26613165329133	-1434187845.72582\\
1.26623165579139	-1430337569.34254\\
1.26633165829146	-1426493022.53721\\
1.26643166079152	-1422654205.30983\\
1.26653166329158	-1418821117.66041\\
1.26663166579164	-1414993759.58893\\
1.26673166829171	-1411172131.09541\\
1.26683167079177	-1407356232.17984\\
1.26693167329183	-1403540333.26427\\
1.26703167579189	-1399735893.5046\\
1.26713167829196	-1395931453.74493\\
1.26723168079202	-1392132743.56322\\
1.26733168329208	-1388345492.5374\\
1.26743168579214	-1384558241.51159\\
1.26753168829221	-1380776720.06372\\
1.26763169079227	-1377000928.19381\\
1.26773169329233	-1373225136.3239\\
1.26783169579239	-1369460803.60989\\
1.26793169829246	-1365702200.47383\\
1.26803170079252	-1361943597.33777\\
1.26813170329258	-1358196453.35762\\
1.26823170579264	-1354449309.37746\\
1.26833170829271	-1350707894.97526\\
1.26843171079277	-1346977939.72896\\
1.26853171329283	-1343247984.48265\\
1.26863171579289	-1339523758.8143\\
1.26873171829296	-1335805262.7239\\
1.26883172079302	-1332086766.6335\\
1.26893172329308	-1328379729.69901\\
1.26903172579314	-1324678422.34246\\
1.26913172829321	-1320977114.98592\\
1.26923173079327	-1317287266.78528\\
1.26933173329333	-1313597418.58463\\
1.26943173579339	-1309913299.96194\\
1.26953173829346	-1306234910.9172\\
1.26963174079352	-1302562251.45041\\
1.26973174329358	-1298895321.56158\\
1.26983174579364	-1295234121.25069\\
1.26993174829371	-1291578650.51776\\
1.27003175079377	-1287928909.36277\\
1.27013175329383	-1284279168.20779\\
1.27023175579389	-1280640886.20871\\
1.27033175829396	-1277002604.20963\\
1.27043176079402	-1273370051.7885\\
1.27053176329408	-1269748958.52327\\
1.27063176579414	-1266127865.25804\\
1.27073176829421	-1262512501.57077\\
1.27083177079427	-1258902867.46144\\
1.27093177329433	-1255298962.93007\\
1.27103177579439	-1251695058.3987\\
1.27113177829446	-1248102613.02323\\
1.27123178079452	-1244515897.22571\\
1.27133178329458	-1240929181.42819\\
1.27143178579464	-1237348195.20862\\
1.27153178829471	-1233778668.14496\\
1.27163179079477	-1230209141.08129\\
1.27173179329483	-1226645343.59558\\
1.27183179579489	-1223087275.68782\\
1.27193179829496	-1219534937.35801\\
1.27203180079502	-1215988328.60615\\
1.27213180329508	-1212441719.85429\\
1.27223180579514	-1208906570.25833\\
1.27233180829521	-1205371420.66237\\
1.27243181079527	-1201847730.22232\\
1.27253181329533	-1198324039.78226\\
1.27263181579539	-1194811808.49811\\
1.27273181829546	-1191299577.21396\\
1.27283182079552	-1187793075.50776\\
1.27293182329558	-1184292303.37951\\
1.27303182579565	-1180797260.82921\\
1.27313182829571	-1177302218.27891\\
1.27323183079577	-1173818634.88452\\
1.27333183329583	-1170340781.06807\\
1.27343183579589	-1166862927.25163\\
1.27353183829596	-1163390803.01314\\
1.27363184079602	-1159930137.93055\\
1.27373184329608	-1156469472.84796\\
1.27383184579614	-1153014537.34332\\
1.27393184829621	-1149565331.41663\\
1.27403185079627	-1146121855.06789\\
1.27413185329633	-1142684108.29711\\
1.27423185579639	-1139252091.10427\\
1.27433185829646	-1135820073.91144\\
1.27443186079652	-1132399515.87451\\
1.27453186329658	-1128978957.83758\\
1.27463186579664	-1125569858.95655\\
1.27473186829671	-1122160760.07552\\
1.27483187079677	-1118757390.77245\\
1.27493187329683	-1115359751.04732\\
1.2750318757969	-1111967840.90015\\
1.27513187829696	-1108581660.33092\\
1.27523188079702	-1105201209.33965\\
1.27533188329708	-1101826487.92633\\
1.27543188579714	-1098451766.51301\\
1.27553188829721	-1095088504.25559\\
1.27563189079727	-1091725241.99817\\
1.27573189329733	-1088367709.31871\\
1.27583189579739	-1085021635.79514\\
1.27593189829746	-1081675562.27158\\
1.27603190079752	-1078335218.32597\\
1.27613190329758	-1075000603.9583\\
1.27623190579765	-1071671719.16859\\
1.27633190829771	-1068342834.37888\\
1.27643191079777	-1065025408.74508\\
1.27653191329783	-1061713712.68922\\
1.27663191579789	-1058402016.63336\\
1.27673191829796	-1055096050.15546\\
1.27683192079802	-1051801542.83346\\
1.27693192329808	-1048507035.51146\\
1.27703192579815	-1045218257.7674\\
1.27713192829821	-1041935209.6013\\
1.27723193079827	-1038657891.01316\\
1.27733193329833	-1035386302.00296\\
1.27743193579839	-1032114712.99276\\
1.27753193829846	-1028854583.13847\\
1.27763194079852	-1025594453.28417\\
1.27773194329858	-1022345782.58578\\
1.27783194579864	-1019097111.88739\\
1.27793194829871	-1015854170.76695\\
1.27803195079877	-1012622688.80241\\
1.27813195329883	-1009391206.83787\\
1.2782319557989	-1006165454.45129\\
1.27833195829896	-1002939702.0647\\
1.27843196079902	-999725408.834017\\
1.27853196329908	-996516845.181284\\
1.27863196579914	-993308281.528552\\
1.27873196829921	-990111177.031722\\
1.27883197079927	-986914072.534892\\
1.27893197329933	-983722697.616013\\
1.2790319757994	-980542781.853037\\
1.27913197829946	-977362866.090061\\
1.27923198079952	-974188679.905036\\
1.27933198329958	-971020223.297963\\
1.27943198579965	-967851766.690889\\
1.27953198829971	-964694769.239718\\
1.27963199079977	-961543501.366499\\
1.27973199329983	-958392233.493279\\
1.27983199579989	-955252424.775962\\
1.27993199829996	-952112616.058645\\
1.28003200080002	-948978536.91928\\
1.28013200330008	-945850187.357866\\
1.28023200580015	-942727567.374403\\
1.28033200830021	-939610676.968891\\
1.28043201080027	-936499516.141331\\
1.28053201330033	-933394084.891721\\
1.28063201580039	-930294383.220064\\
1.28073201830046	-927194681.548406\\
1.28083202080052	-924106439.032651\\
1.28093202330058	-921018196.516896\\
1.28103202580065	-917935683.579092\\
1.28113202830071	-914858900.219239\\
1.28123203080077	-911787846.437338\\
1.28133203330083	-908722522.233388\\
1.2814320358009	-905662927.60739\\
1.28153203830096	-902609062.559342\\
1.28163204080102	-899560927.089246\\
1.28173204330108	-896512791.61915\\
1.28183204580114	-893476115.304957\\
1.28193204830121	-890439438.990764\\
1.28203205080127	-887414221.832473\\
1.28213205330133	-884389004.674182\\
1.2822320558014	-881369517.093843\\
1.28233205830146	-878355759.091455\\
1.28243206080152	-875347730.667018\\
1.28253206330158	-872345431.820532\\
1.28263206580165	-869343132.974047\\
1.28273206830171	-866352293.283464\\
1.28283207080177	-863367183.170832\\
1.28293207330183	-860382073.058201\\
1.2830320758019	-857402692.52352\\
1.28313207830196	-854434771.144743\\
1.28323208080202	-851466849.765965\\
1.28333208330208	-848504657.965139\\
1.28343208580215	-845548195.742264\\
1.28353208830221	-842597463.09734\\
1.28363209080227	-839646730.452416\\
1.28373209330233	-836707456.963395\\
1.28383209580239	-833773913.052325\\
1.28393209830246	-830840369.141255\\
1.28403210080252	-827918284.386088\\
1.28413210330258	-824996199.630921\\
1.28423210580265	-822079844.453705\\
1.28433210830271	-819169218.854441\\
1.28443211080277	-816264322.833127\\
1.28453211330283	-813365156.389765\\
1.2846321158029	-810471719.524355\\
1.28473211830296	-807584012.236895\\
1.28483212080302	-804696304.949436\\
1.28493212330308	-801820056.817879\\
1.28503212580315	-798943808.686323\\
1.28513212830321	-796073290.132717\\
1.28523213080327	-793214230.735014\\
1.28533213330333	-790355171.337312\\
1.2854321358034	-787501841.51756\\
1.28553213830346	-784654241.27576\\
1.28563214080352	-781812370.611911\\
1.28573214330358	-778970499.948062\\
1.28583214580365	-776140088.440116\\
1.28593214830371	-773309676.93217\\
1.28603215080377	-770490724.580126\\
1.28613215330383	-767671772.228082\\
1.2862321558039	-764864279.031941\\
1.28633215830396	-762056785.8358\\
1.28643216080402	-759255022.21761\\
1.28653216330408	-756458988.177372\\
1.28663216580415	-753668683.715085\\
1.28673216830421	-750878379.252798\\
1.28683217080427	-748099533.946413\\
1.28693217330433	-745326418.21798\\
1.2870321758044	-742553302.489547\\
1.28713217830446	-739785916.339065\\
1.28723218080452	-737029989.344486\\
1.28733218330458	-734274062.349907\\
1.28743218580465	-731523864.933279\\
1.28753218830471	-728779397.094602\\
1.28763219080477	-726040658.833877\\
1.28773219330483	-723307650.151103\\
1.2878321958049	-720574641.468328\\
1.28793219830496	-717853091.941457\\
1.28803220080502	-715131542.414586\\
1.28813220330508	-712421452.043617\\
1.28823220580515	-709711361.672648\\
1.28833220830521	-707007000.879631\\
1.28843221080527	-704314099.242516\\
1.28853221330533	-701621197.605401\\
1.2886322158054	-698928295.968286\\
1.28873221830546	-696246853.487074\\
1.28883222080552	-693571140.583813\\
1.28893222330558	-690901157.258503\\
1.28903222580565	-688231173.933194\\
1.28913222830571	-685572649.763787\\
1.28923223080577	-682914125.59438\\
1.28933223330583	-680261331.002924\\
1.2894322358059	-677614265.989419\\
1.28953223830596	-674972930.553866\\
1.28963224080602	-672337324.696265\\
1.28973224330608	-669707448.416614\\
1.28983224580615	-667083301.714915\\
1.28993224830621	-664464884.591167\\
1.29003225080627	-661846467.467419\\
1.29013225330633	-659239509.499574\\
1.2902322558064	-656632551.531729\\
1.29033225830646	-654031323.141835\\
1.29043226080652	-651435824.329892\\
1.29053226330658	-648846055.095901\\
1.29063226580665	-646262015.439861\\
1.29073226830671	-643683705.361772\\
1.29083227080677	-641111124.861635\\
1.29093227330683	-638544273.939449\\
1.2910322758069	-635977423.017262\\
1.29113227830696	-633422031.250979\\
1.29123228080702	-630866639.484696\\
1.29133228330708	-628316977.296363\\
1.29143228580715	-625778774.263934\\
1.29153228830721	-623240571.231504\\
1.29163229080727	-620708097.777026\\
1.29173229330733	-618181353.900499\\
1.2918322958074	-615654610.023972\\
1.29193229830746	-613139325.303348\\
1.29203230080752	-610629770.160675\\
1.29213230330758	-608120215.018002\\
1.29223230580765	-605622119.031232\\
1.29233230830771	-603124023.044461\\
1.29243231080777	-600631656.635642\\
1.29253231330783	-598145019.804774\\
1.2926323158079	-595664112.551858\\
1.29273231830796	-593188934.876893\\
1.29283232080802	-590719486.779879\\
1.29293232330808	-588255768.260816\\
1.29303232580815	-585792049.741754\\
1.29313232830821	-583339790.378594\\
1.29323233080827	-580887531.015434\\
1.29333233330833	-578446730.808177\\
1.2934323358084	-576005930.600919\\
1.29353233830846	-573570859.971613\\
1.29363234080852	-571140945.962463\\
1.29373234330858	-568716761.531265\\
1.29383234580865	-566297733.720223\\
1.29393234830871	-563884435.487131\\
1.29403235080877	-561475720.916402\\
1.29413235330883	-559072162.965828\\
1.2942323558089	-556674334.593205\\
1.29433235830896	-554281089.882944\\
1.29443236080902	-551893574.750634\\
1.29453236330908	-549510643.280685\\
1.29463236580915	-547132868.430892\\
1.29473236830921	-544760823.15905\\
1.29483237080927	-542393361.549569\\
1.29493237330933	-540031629.51804\\
1.2950323758094	-537674481.148872\\
1.29513237830946	-535323062.357655\\
1.29523238080952	-532976227.228799\\
1.29533238330958	-530635121.677895\\
1.29543238580965	-528298599.789351\\
1.29553238830971	-525967807.478759\\
1.29563239080977	-523642171.788323\\
1.29573239330983	-521321119.760248\\
1.2958323958099	-519005797.310124\\
1.29593239830996	-516695058.522362\\
1.29603240081002	-514390049.31255\\
1.29613240331008	-512090196.722895\\
1.29623240581015	-509794927.795601\\
1.29633240831021	-507505388.446258\\
1.29643241081027	-505221005.717072\\
1.29653241331033	-502941206.650246\\
1.2966324158104	-500667137.161372\\
1.29673241831046	-498398224.292654\\
1.29683242081052	-496134468.044092\\
1.29693242331058	-493875295.457891\\
1.29703242581065	-491621852.449642\\
1.29713242831071	-489373566.061548\\
1.29723243081077	-487130436.293611\\
1.29733243331083	-484891890.188035\\
1.2974324358109	-482659073.66041\\
1.29753243831096	-480431413.752942\\
1.29763244081102	-478208910.465629\\
1.29773244331108	-475991563.798473\\
1.29783244581115	-473779373.751473\\
1.29793244831121	-471571767.366834\\
1.29803245081127	-469369890.560146\\
1.29813245331133	-467173170.373614\\
1.2982324558114	-464981606.807239\\
1.29833245831146	-462795199.86102\\
1.29843246081152	-460613949.534957\\
1.29853246331158	-458437855.82905\\
1.29863246581165	-456266918.743299\\
1.29873246831171	-454101138.277705\\
1.29883247081177	-451939941.474471\\
1.29893247331183	-449784474.249189\\
1.2990324758119	-447634163.644063\\
1.29913247831196	-445489009.659093\\
1.29923248081202	-443349012.29428\\
1.29933248331208	-441214171.549622\\
1.29943248581215	-439084487.425121\\
1.29953248831221	-436959959.920776\\
1.29963249081227	-434840589.036587\\
1.29973249331233	-432726374.772554\\
1.2998324958124	-430617317.128678\\
1.29993249831246	-428513416.104957\\
1.30003250081252	-426414671.701393\\
1.30013250331258	-424321083.917985\\
1.30023250581265	-422232652.754733\\
1.30033250831271	-420149378.211638\\
1.30043251081277	-418071260.288698\\
1.30053251331283	-415998871.94371\\
1.3006325158129	-413931067.261083\\
1.30073251831296	-411868419.198612\\
1.30083252081302	-409810927.756297\\
1.30093252331308	-407758592.934138\\
1.30103252581315	-405711414.732136\\
1.30113252831321	-403669393.15029\\
1.30123253081327	-401632528.1886\\
1.30133253331333	-399600819.847066\\
1.3014325358134	-397574268.125688\\
1.30153253831346	-395553445.982262\\
1.30163254081352	-393537207.501196\\
1.30173254331358	-391526125.640287\\
1.30183254581365	-389520200.399534\\
1.30193254831371	-387519431.778937\\
1.30203255081377	-385523819.778497\\
1.30213255331383	-383533364.398212\\
1.3022325558139	-381548065.638084\\
1.30233255831396	-379568496.455907\\
1.30243256081402	-377593510.936091\\
1.30253256331408	-375623682.036431\\
1.30263256581415	-373659009.756927\\
1.30273256831421	-371699494.09758\\
1.30283257081427	-369745135.058389\\
1.30293257331433	-367796505.597149\\
1.3030325758144	-365852459.79827\\
1.30313257831446	-363913570.619547\\
1.30323258081452	-361979838.060981\\
1.30333258331458	-360051262.12257\\
1.30343258581465	-358127842.804316\\
1.30353258831471	-356210153.064013\\
1.30363259081477	-354297046.986072\\
1.30373259331483	-352389097.528286\\
1.3038325958149	-350486304.690656\\
1.30393259831496	-348588668.473183\\
1.30403260081502	-346696761.833661\\
1.30413260331508	-344809438.8565\\
1.30423260581515	-342927272.499495\\
1.30433260831521	-341050262.762647\\
1.30443261081527	-339178409.645954\\
1.30453261331533	-337312286.107213\\
1.3046326158154	-335450746.230833\\
1.30473261831546	-333594362.974609\\
1.30483262081552	-331743136.338542\\
1.30493262331558	-329897066.32263\\
1.30503262581565	-328056725.88467\\
1.30513262831571	-326220969.109071\\
1.30523263081577	-324390368.953628\\
1.30533263331583	-322564925.418341\\
1.3054326358159	-320744638.50321\\
1.30553263831596	-318930081.166031\\
1.30563264081602	-317120107.491213\\
1.30573264331608	-315315290.436551\\
1.30583264581615	-313515630.002045\\
1.30593264831621	-311721126.187695\\
1.30603265081627	-309932351.951297\\
1.30613265331633	-308148161.377259\\
1.3062326558164	-306369127.423378\\
1.30633265831646	-304595250.089653\\
1.30643266081652	-302826529.376084\\
1.30653266331658	-301063538.240467\\
1.30663266581665	-299305130.76721\\
1.30673266831671	-297551879.91411\\
1.30683267081677	-295803785.681166\\
1.30693267331683	-294060848.068378\\
1.3070326758169	-292323640.033541\\
1.30713267831696	-290591015.661066\\
1.30723268081702	-288863547.908746\\
1.30733268331708	-287141236.776583\\
1.30743268581715	-285424082.264576\\
1.30753268831721	-283712084.372725\\
1.30763269081727	-282005816.058825\\
1.30773269331733	-280304131.407287\\
1.3078326958174	-278607603.375904\\
1.30793269831746	-276916231.964678\\
1.30803270081752	-275230017.173608\\
1.30813270331758	-273548959.002694\\
1.30823270581765	-271873057.451937\\
1.30833270831771	-270202885.47913\\
1.30843271081777	-268537297.168685\\
1.30853271331783	-266876865.478396\\
1.3086327158179	-265221590.408263\\
1.30873271831796	-263571471.958286\\
1.30883272081802	-261926510.128466\\
1.30893272331808	-260286704.918801\\
1.30903272581815	-258652056.329293\\
1.30913272831821	-257023137.317736\\
1.30923273081827	-255398801.96854\\
1.30933273331833	-253779623.2395\\
1.3094327358184	-252165601.130617\\
1.30953273831846	-250556735.64189\\
1.30963274081852	-248953026.773318\\
1.30973274331858	-247354474.524903\\
1.30983274581865	-245761078.896645\\
1.30993274831871	-244172839.888542\\
1.31003275081877	-242589757.500595\\
1.31013275331883	-241011831.732805\\
1.3102327558189	-239439062.585171\\
1.31033275831896	-237871450.057693\\
1.31043276081902	-236308994.150371\\
1.31053276331908	-234751694.863206\\
1.31063276581915	-233199552.196196\\
1.31073276831921	-231652566.149343\\
1.31083277081927	-230110736.722646\\
1.31093277331933	-228574063.916105\\
1.3110327758194	-227042547.729721\\
1.31113277831946	-225516188.163492\\
1.31123278081952	-223994985.21742\\
1.31133278331958	-222478938.891504\\
1.31143278581965	-220968049.185744\\
1.31153278831971	-219462316.10014\\
1.31163279081977	-217961739.634692\\
1.31173279331983	-216466319.789401\\
1.3118327958199	-214976056.564265\\
1.31193279831996	-213490949.959286\\
1.31203280082002	-212010999.974463\\
1.31213280332008	-210536206.609797\\
1.31223280582015	-209066569.865286\\
1.31233280832021	-207602089.740932\\
1.31243281082027	-206142766.236733\\
1.31253281332033	-204688026.394896\\
1.3126328158204	-203239016.13101\\
1.31273281832046	-201795162.487281\\
1.31283282082052	-200356465.463707\\
1.31293282332058	-198922925.06029\\
1.31303282582065	-197494541.277029\\
1.31313282832071	-196070741.156129\\
1.31323283082077	-194652670.61318\\
1.31333283332083	-193239756.690387\\
1.3134328358209	-191831999.387751\\
1.31353283832096	-190429398.705271\\
1.31363284082102	-189031381.685151\\
1.31373284332108	-187639094.242984\\
1.31383284582115	-186251963.420972\\
1.31393284832121	-184869989.219116\\
1.31403285082127	-183492598.679622\\
1.31413285332133	-182120937.718079\\
1.3142328558214	-180754433.376692\\
1.31433285832146	-179393085.655461\\
1.31443286082152	-178036321.596591\\
1.31453286332158	-176685287.115672\\
1.31463286582165	-175339409.25491\\
1.31473286832171	-173998115.056509\\
1.31483287082177	-172662550.436059\\
1.31493287332183	-171332142.435765\\
1.3150328758219	-170006318.097832\\
1.31513287832196	-168686223.337851\\
1.31523288082202	-167370712.240231\\
1.31533288332208	-166060930.720562\\
1.31543288582215	-164756305.821049\\
1.31553288832221	-163456264.583897\\
1.31563289082227	-162161952.924696\\
1.31573289332233	-160872224.927857\\
1.3158328958224	-159588226.508969\\
1.31593289832246	-158308811.752442\\
1.31603290082252	-157035126.573866\\
1.31613290332258	-155766025.057651\\
1.31623290582265	-154502653.119388\\
1.31633290832271	-153243864.843485\\
1.31643291082277	-151990806.145534\\
1.31653291332283	-150742331.109944\\
1.3166329158229	-149499585.652305\\
1.31673291832296	-148261423.857027\\
1.31683292082302	-147028991.639701\\
1.31693292332308	-145801143.084736\\
1.31703292582315	-144578451.149927\\
1.31713292832321	-143361488.793069\\
1.31723293082327	-142149110.098572\\
1.31733293332333	-140942460.982026\\
1.3174329358234	-139740395.527842\\
1.31753293832346	-138543486.693814\\
1.31763294082352	-137352307.437737\\
1.31773294332358	-136165711.844021\\
1.31783294582365	-134984272.870461\\
1.31793294832371	-133807990.517057\\
1.31803295082377	-132637437.741605\\
1.31813295332383	-131471468.628514\\
1.3182329558239	-130310656.135579\\
1.31833295832396	-129155000.2628\\
1.31843296082402	-128005073.967972\\
1.31853296332408	-126859731.335506\\
1.31863296582415	-125719545.323196\\
1.31873296832421	-124584515.931041\\
1.31883297082427	-123454643.159043\\
1.31893297332433	-122329927.007202\\
1.3190329758244	-121210940.433311\\
1.31913297832446	-120096537.521782\\
1.31923298082452	-118987291.230408\\
1.31933298332458	-117883201.559191\\
1.31943298582465	-116784268.50813\\
1.31953298832471	-115690492.077226\\
1.31963299082477	-114601872.266477\\
1.31973299332483	-113518409.075885\\
1.3198329958249	-112440102.505448\\
1.31993299832496	-111366952.555168\\
1.32003300082502	-110298959.225045\\
1.32013300332508	-109236122.515077\\
1.32023300582515	-108178442.425265\\
1.32033300832521	-107125918.95561\\
1.32043301082527	-106078552.106111\\
1.32053301332533	-105036341.876768\\
1.3206330158254	-103999288.267581\\
1.32073301832546	-102966818.320755\\
1.32083302082552	-101940077.951881\\
1.32093302332558	-100918494.203163\\
1.32103302582565	-99902067.0746006\\
1.32113302832571	-98890796.5661947\\
1.32123303082577	-97884682.677945\\
1.32133303332583	-96883152.4520563\\
1.3214330358259	-95887351.8041189\\
1.32153303832596	-94896707.7763377\\
1.32163304082602	-93911220.3687127\\
1.32173304332608	-92930316.6234487\\
1.32183304582615	-91955142.4561361\\
1.32193304832621	-90985124.9089796\\
1.32203305082627	-90019691.0241842\\
1.32213305332633	-89059986.71734\\
1.3222330558264	-88105439.0306521\\
1.32233305832646	-87155475.0063252\\
1.32243306082652	-86211240.5599496\\
1.32253306332658	-85271589.775935\\
1.32263306582665	-84337668.5698718\\
1.32273306832671	-83408903.9839647\\
1.32283307082677	-82484723.0604187\\
1.32293307332683	-81566271.714824\\
1.3230330758269	-80652404.0315903\\
1.32313307832696	-79744265.926308\\
1.32323308082702	-78840711.4833867\\
1.32333308332708	-77942313.6606215\\
1.32343308582715	-77049645.4158077\\
1.32353308832721	-76161560.8333549\\
1.32363309082727	-75279205.8288535\\
1.32373309332733	-74401434.4867131\\
1.3238330958274	-73528819.7647288\\
1.32393309832746	-72661934.6206959\\
1.32403310082752	-71799633.139024\\
1.32413310332758	-70942488.2775083\\
1.32423310582765	-70091072.9939439\\
1.32433310832771	-69244241.3727405\\
1.32443311082777	-68402566.3716933\\
1.32453311332783	-67566047.9908023\\
1.3246331158279	-66735259.1878626\\
1.32473311832796	-65909054.047284\\
1.32483312082802	-65088005.5268615\\
1.32493312332808	-64272113.6265952\\
1.32503312582815	-63461378.3464851\\
1.32513312832821	-62655799.6865312\\
1.32523313082827	-61855950.6045285\\
1.32533313332833	-61060685.184887\\
1.3254331358284	-60270576.3854016\\
1.32553313832846	-59485624.2060723\\
1.32563314082852	-58705828.6468993\\
1.32573314332858	-57931189.7078824\\
1.32583314582865	-57161592.7974627\\
1.32593314832871	-56397209.8029786\\
1.32603315082877	-55638040.7244303\\
1.32613315332883	-54883970.9702586\\
1.3262331558289	-54135057.8362431\\
1.32633315832896	-53391301.3223838\\
1.32643316082902	-52652701.4286807\\
1.32653316332908	-51919258.1551337\\
1.32663316582915	-51190914.2059634\\
1.32673316832921	-50467726.8769493\\
1.32683317082927	-49749696.1680913\\
1.32693317332933	-49036822.0793896\\
1.3270331758294	-48329104.610844\\
1.32713317832946	-47626486.466675\\
1.32723318082952	-46929024.9426623\\
1.32733318332958	-46236720.0388057\\
1.32743318582965	-45549571.7551053\\
1.32753318832971	-44867580.0915611\\
1.32763319082977	-44190687.7523935\\
1.32773319332983	-43519009.3291617\\
1.3278331958299	-42852430.2303065\\
1.32793319832996	-42190950.4558279\\
1.32803320083002	-41534684.5972851\\
1.32813320333008	-40883575.3588984\\
1.32823320583015	-40237565.4448884\\
1.32833320833021	-39596712.1510346\\
1.32843321083027	-38961015.477337\\
1.32853321333033	-38330418.128016\\
1.3286332158304	-37705034.6946307\\
1.32873321833046	-37084750.585622\\
1.32883322083052	-36469623.0967696\\
1.32893322333058	-35859652.2280733\\
1.32903322583065	-35254837.9795332\\
1.32913322833071	-34655123.0553698\\
1.32923323083077	-34060564.7513625\\
1.32933323333083	-33471163.0675115\\
1.3294332358309	-32886918.0038166\\
1.32953323833096	-32307829.5602778\\
1.32963324083102	-31733840.4411158\\
1.32973324333108	-31165007.9421099\\
1.32983324583115	-30601332.0632602\\
1.32993324833121	-30042812.8045667\\
1.33003325083127	-29489450.1660293\\
1.33013325333133	-28941186.8518686\\
1.3302332558314	-28398080.1578641\\
1.33033325833146	-27860130.0840158\\
1.33043326083152	-27327336.6303236\\
1.33053326333158	-26799699.7967877\\
1.33063326583165	-26277162.2876284\\
1.33073326833171	-25759781.3986252\\
1.33083327083177	-25247557.1297783\\
1.33093327333183	-24740489.4810875\\
1.3310332758319	-24238521.1567734\\
1.33113327833196	-23741766.7483949\\
1.33123328083202	-23250111.6643932\\
1.33133328333208	-22763555.9047681\\
1.33143328583215	-22282214.0610787\\
1.33153328833221	-21806028.8375455\\
1.33163329083227	-21334942.9383889\\
1.33173329333233	-20869013.6593885\\
1.3318332958324	-20408241.0005443\\
1.33193329833246	-19952567.6660768\\
1.33203330083252	-19502108.2475449\\
1.33213330333258	-19056748.1533897\\
1.33223330583265	-18616544.6793907\\
1.33233330833271	-18181440.5297684\\
1.33243331083277	-17751550.2960817\\
1.33253331333283	-17326759.3867717\\
1.3326333158329	-16907125.0976179\\
1.33273331833296	-16492647.4286203\\
1.33283332083302	-16083326.3797788\\
1.33293332333308	-15679104.655314\\
1.33303332583315	-15280039.5510054\\
1.33313332833321	-14886131.0668529\\
1.33323333083327	-14497379.2028567\\
1.33333333333333	-14113783.9590166\\
1.3334333358334	-13735288.0395532\\
1.33353333833346	-13361948.7402459\\
1.33363334083352	-12993766.0610949\\
1.33373334333358	-12630740.0021\\
1.33383334583365	-12272813.2674817\\
1.33393334833371	-11920100.4487992\\
1.33403335083377	-11572486.9544933\\
1.33413335333383	-11229972.7845641\\
1.3342333558339	-10892672.5305706\\
1.33433335833396	-10560471.6009538\\
1.33443336083402	-10233427.2914931\\
1.33453336333408	-9911539.6021886\\
1.33463336583415	-9594808.53304028\\
1.33473336833421	-9283234.08404814\\
1.33483337083427	-8976758.95943266\\
1.33493337333433	-8675440.45497336\\
1.3350333758344	-8379278.57067024\\
1.33513337833446	-8088216.01074378\\
1.33523338083452	-7802367.36675301\\
1.33533338333458	-7521618.04713891\\
1.33543338583465	-7246025.34768098\\
1.33553338833471	-6975531.97259972\\
1.33563339083477	-6710252.51345415\\
1.33573339333483	-6450072.37868524\\
1.3358333958349	-6195048.86407251\\
1.33593339833496	-5945181.96961596\\
1.33603340083502	-5700448.77700378\\
1.33613340333508	-5460860.74539188\\
1.33623340583515	-5226412.14520229\\
1.33633340833521	-4997108.70601299\\
1.33643341083527	-4772950.42782396\\
1.33653341333533	-4553937.3106352\\
1.3366334158354	-4340063.62486876\\
1.33673341833546	-4131340.82968056\\
1.33683342083552	-3927757.46591467\\
1.33693342333558	-3729313.53357111\\
1.33703342583565	-3536020.49180578\\
1.33713342833571	-3347866.88146277\\
1.33723343083577	-3164858.43212003\\
1.33733343333583	-2986989.41419962\\
1.3374334358359	-2814271.28685743\\
1.33753343833596	-2646692.59093757\\
1.33763344083602	-2484259.05601798\\
1.33773344333608	-2326964.95252071\\
1.33783344583615	-2174816.01002372\\
1.33793344833621	-2027812.22852701\\
1.33803345083627	-1885953.60803057\\
1.33813345333633	-1749240.1485344\\
1.3382334558364	-1617666.12046056\\
1.33833345833646	-1491237.25338699\\
1.33843346083652	-1369953.5473137\\
1.33853346333658	-1253809.27266273\\
1.33863346583665	-1142810.15901204\\
1.33873346833671	-1036956.20636162\\
1.33883347083677	-936247.414711473\\
1.33893347333683	-840678.054483652\\
1.3390334758369	-750253.855256105\\
1.33913347833696	-664974.817028833\\
1.33923348083702	-584840.939801837\\
1.33933348333708	-509845.348081573\\
1.33943348583715	-439995.490319379\\
1.33953348833721	-375289.64764187\\
1.33963349083727	-315727.24709125\\
1.33973349333733	-261308.288667519\\
1.3398334958374	-212033.345328473\\
1.33993349833746	-167901.844116317\\
1.34003350083752	-128913.78503105\\
1.34013350333758	-95069.1680726721\\
1.34023350583765	-66367.9932411838\\
1.34033350833771	-42810.6616070414\\
1.34043351083777	-24396.7720997885\\
1.34053351333783	-11126.439310984\\
1.3406335158379	-2999.70334767357\\
1.34073351833796	-16.4929912032973\\
1.34083352083802	-2176.83855104054\\
1.34093352333808	-9480.73263602973\\
1.34103352583815	-21928.1834394674\\
1.34113352833821	-39519.1909613534\\
1.34123353083827	-62253.5833143493\\
1.34133353333833	-90131.9907520298\\
1.3414335358384	-123153.8403166\\
1.34153353833846	-161318.559050264\\
1.34163354083852	-204627.865826408\\
1.34173354333858	-253080.041771646\\
1.34183354583865	-306675.659843773\\
1.34193354833871	-365415.293000585\\
1.34203355083877	-429298.368284287\\
1.34213355333883	-498324.885694877\\
1.3422335558389	-572495.418190153\\
1.34233355833896	-651808.246896727\\
1.34243356083902	-736267.955476962\\
1.34253356333908	-825867.09547952\\
1.34263356583915	-920611.396482353\\
1.34273356833921	-1020500.85848546\\
1.34283357083927	-1125529.75191089\\
1.34293357333933	-1235709.53591455\\
1.3430335758394	-1351023.02176258\\
1.34313357833946	-1471487.39818883\\
1.34323358083952	-1597096.93561536\\
1.34333358333958	-1727845.90446422\\
1.34343358583965	-1863740.03431335\\
1.34353358833971	-2004773.5955848\\
1.34363359083977	-2150958.04743448\\
1.34373359333983	-2302281.93070648\\
1.3438335958399	-2458745.2454008\\
1.34393359833996	-2620359.45067336\\
1.34403360084002	-2787113.08736823\\
1.34413360334008	-2959011.88506338\\
1.34423360584015	-3136055.8437588\\
1.34433360834021	-3318239.23387655\\
1.34443361084027	-3505573.51457253\\
1.34453361334033	-3698047.22669082\\
1.3446336158404	-3895660.37023145\\
1.34473361834046	-4098424.40435029\\
1.34483362084052	-4306327.86989146\\
1.34493362334058	-4519376.49643291\\
1.34503362584065	-4737570.28397463\\
1.34513362834071	-4960903.50293867\\
1.34523363084077	-5189387.61248094\\
1.34533363334083	-5423011.15344553\\
1.3454336358409	-5661774.12583245\\
1.34553363834096	-5905705.17753145\\
1.34563364084102	-6154769.93107482\\
1.34573364334108	-6408934.00899485\\
1.34583364584115	-6668254.70707106\\
1.34593364834121	-6932789.32108296\\
1.34603365084127	-7202423.25947153\\
1.34613365334133	-7477156.52223676\\
1.3462336558414	-7757103.70093768\\
1.34633365834146	-8042150.20401526\\
1.34643366084152	-8332353.32724902\\
1.34653366334158	-8627713.07063896\\
1.34663366584165	-8928172.13840557\\
1.34673366834171	-9233845.12210786\\
1.34683367084177	-9544617.43018682\\
1.34693367334183	-9860546.35842196\\
1.3470336758419	-10181631.9068133\\
1.34713367834196	-10507816.7795812\\
1.34723368084202	-10839158.2725054\\
1.34733368334208	-11175656.3855857\\
1.34743368584215	-11517311.1188222\\
1.34753368834221	-11864122.4722149\\
1.34763369084227	-12216033.1499843\\
1.34773369334233	-12573100.4479098\\
1.3478336958424	-12935324.3659915\\
1.34793369834246	-13302704.9042294\\
1.34803370084252	-13675242.0626235\\
1.34813370334258	-14052878.5453942\\
1.34823370584265	-14435671.6483211\\
1.34833370834271	-14823621.3714042\\
1.34843371084277	-15216727.7146434\\
1.34853371334283	-15614933.3822594\\
1.3486337158429	-16018295.6700315\\
1.34873371834296	-16426814.5779597\\
1.34883372084302	-16840490.1060442\\
1.34893372334308	-17259264.9585053\\
1.34903372584315	-17683253.7269021\\
1.34913372834321	-18112341.8196756\\
1.34923373084327	-18546586.5326052\\
1.34933373334333	-18985987.8656911\\
1.3494337358434	-19430488.5231536\\
1.34953373834346	-19880145.8007722\\
1.34963374084352	-20334959.6985471\\
1.34973374334358	-20794930.2164781\\
1.34983374584365	-21260057.3545653\\
1.34993374834371	-21730283.8170292\\
1.35003375084377	-22205666.8996492\\
1.35013375334383	-22686206.6024254\\
1.3502337558439	-23171902.9253578\\
1.35033375834396	-23662755.8684464\\
1.35043376084402	-24158708.1359116\\
1.35053376334408	-24659817.0235331\\
1.35063376584415	-25166082.5313107\\
1.35073376834421	-25677504.6592444\\
1.35083377084427	-26194026.1115549\\
1.35093377334433	-26715761.479801\\
1.3510337758444	-27242596.1724238\\
1.35113377834446	-27774587.4852028\\
1.35123378084452	-28311678.1223584\\
1.35133378334458	-28853982.6754497\\
1.35143378584465	-29401386.5529177\\
1.35153378834471	-29953947.0505419\\
1.35163379084477	-30511664.1683222\\
1.35173379334483	-31074480.6104792\\
1.3518337958449	-31642510.9685719\\
1.35193379834496	-32215640.6510413\\
1.35203380084502	-32793926.9536668\\
1.35213380334508	-33377369.8764485\\
1.35223380584515	-33965969.4193864\\
1.35233380834521	-34559668.286701\\
1.35243381084527	-35158523.7741717\\
1.35253381334533	-35762535.8817986\\
1.3526338158454	-36371704.6095817\\
1.35273381834546	-36986029.957521\\
1.35283382084552	-37605454.6298369\\
1.35293382334558	-38230035.922309\\
1.35303382584565	-38859773.8349373\\
1.35313382834571	-39494668.3677218\\
1.35323383084577	-40134662.2248829\\
1.35333383334583	-40779869.9979798\\
1.3534338358459	-41430177.0954532\\
1.35353383834596	-42085640.8130829\\
1.35363384084602	-42746261.1508687\\
1.35373384334608	-43411980.8130312\\
1.35383384584615	-44082914.3911294\\
1.35393384834621	-44758947.2936043\\
1.35403385084627	-45440136.8162353\\
1.35413385334633	-46126482.9590225\\
1.3542338558464	-46817928.4261864\\
1.35433385834646	-47514587.809286\\
1.35443386084652	-48216346.5167622\\
1.35453386334658	-48923261.8443946\\
1.35463386584665	-49635333.7921832\\
1.35473386834671	-50352505.0643485\\
1.35483387084677	-51074890.2524494\\
1.35493387334683	-51802374.764927\\
1.3550338758469	-52535015.8975608\\
1.35513387834696	-53272813.6503508\\
1.35523388084702	-54015768.0232969\\
1.35533388334708	-54763821.7206197\\
1.35543388584715	-55517032.0380987\\
1.35553388834721	-56275398.9757338\\
1.35563389084727	-57038922.5335252\\
1.35573389334733	-57807430.8241341\\
1.3558338958474	-58581496.8053559\\
1.35593389834746	-59360146.4489387\\
1.35603390084752	-60144525.6704728\\
1.35613390334758	-60933488.5543679\\
1.35623390584765	-61728181.0162144\\
1.35633390834771	-62527457.1404219\\
1.35643391084777	-63332462.8425807\\
1.35653391334783	-64142052.2071005\\
1.3566339158479	-64957371.1495717\\
1.35673391834796	-65777273.7544039\\
1.35683392084802	-66602332.9793923\\
1.35693392334808	-67433121.782332\\
1.35703392584815	-68268494.2476327\\
1.35713392834821	-69109023.3330896\\
1.35723393084827	-69955281.9964979\\
1.35733393334833	-70806124.3222671\\
1.3574339358484	-71662123.2681926\\
1.35753393834846	-72523278.8342742\\
1.35763394084852	-73390163.9783071\\
1.35773394334858	-74261632.7847011\\
1.35783394584865	-75138258.2112513\\
1.35793394834871	-76020040.2579576\\
1.35803395084877	-76906978.9248201\\
1.35813395334883	-77799647.169634\\
1.3582339558489	-78696899.0768088\\
1.35833395834896	-79599307.6041399\\
1.35843396084902	-80506872.7516271\\
1.35853396334908	-81419594.5192705\\
1.35863396584915	-82337472.9070701\\
1.35873396834921	-83260507.9150258\\
1.35883397084927	-84188699.5431378\\
1.35893397334933	-85122047.7914059\\
1.3590339758494	-86060552.6598302\\
1.35913397834946	-87004214.1484106\\
1.35923398084952	-87953032.2571473\\
1.35933398334958	-88907006.9860401\\
1.35943398584965	-89866138.3350891\\
1.35953398834971	-90830426.3042943\\
1.35963399084977	-91799870.8936556\\
1.35973399334983	-92773899.145378\\
1.3598339958499	-93753656.9750517\\
1.35993399834996	-94738571.4248816\\
1.36003400085002	-95728642.4948677\\
1.36013400335008	-96723870.1850099\\
1.36023400585015	-97723681.5375132\\
1.36033400835021	-98729222.4679678\\
1.36043401085027	-99739920.0185786\\
1.36053401335033	-100755774.189346\\
1.3606340158504	-101776212.022474\\
1.36073401835046	-102802379.433553\\
1.36083402085052	-103833703.464788\\
1.36093402335058	-104870184.11618\\
1.36103402585065	-105911248.429933\\
1.36113402835071	-106958042.321637\\
1.36123403085077	-108009419.875702\\
1.36133403335083	-109066527.007718\\
1.3614340358509	-110128790.759891\\
1.36153403835096	-111195638.174424\\
1.36163404085102	-112268215.166909\\
1.36173404335108	-113345375.821755\\
1.36183404585115	-114428266.054552\\
1.36193404835121	-115515739.949711\\
1.36203405085127	-116608943.42282\\
1.36213405335133	-117706730.558291\\
1.3622340558514	-118810247.271713\\
1.36233405835146	-119918347.647496\\
1.36243406085152	-121032177.60123\\
1.36253406335158	-122150591.217326\\
1.36263406585165	-123274734.411372\\
1.36273406835171	-124403461.26778\\
1.36283407085177	-125537917.702139\\
1.36293407335183	-126676957.798859\\
1.3630340758519	-127821154.515735\\
1.36313407835196	-128971080.810563\\
1.36323408085202	-130125590.767752\\
1.36333408335208	-131285257.345096\\
1.36343408585215	-132450653.500392\\
1.36353408835221	-133620633.31805\\
1.36363409085227	-134795769.755863\\
1.36373409335233	-135976062.813832\\
1.3638340958524	-137162085.449753\\
1.36393409835246	-138352691.748035\\
1.36403410085252	-139548454.666473\\
1.36413410335258	-140749374.205067\\
1.36423410585265	-141956023.321613\\
1.36433410835271	-143167256.100519\\
1.36443411085277	-144383645.499582\\
1.36453411335283	-145605191.518801\\
1.3646341158529	-146831894.158176\\
1.36473411835296	-148063753.417707\\
1.36483412085302	-149301342.25519\\
1.36493412335308	-150543514.755034\\
1.36503412585315	-151790843.875033\\
1.36513412835321	-153043329.615189\\
1.36523413085327	-154300971.975501\\
1.36533413335333	-155563770.95597\\
1.3654341358534	-156831726.556594\\
1.36553413835346	-158104838.777375\\
1.36563414085352	-159383107.618312\\
1.36573414335358	-160666533.079405\\
1.36583414585365	-161955115.160654\\
1.36593414835371	-163248853.86206\\
1.36603415085377	-164547749.183621\\
1.36613415335383	-165851801.125339\\
1.3662341558539	-167161009.687213\\
1.36633415835396	-168475374.869243\\
1.36643416085402	-169794896.671429\\
1.36653416335408	-171119575.093772\\
1.36663416585415	-172449410.13627\\
1.36673416835421	-173784401.798925\\
1.36683417085427	-175124550.081736\\
1.36693417335433	-176469854.984703\\
1.3670341758544	-177819743.550032\\
1.36713417835446	-179175361.693311\\
1.36723418085452	-180536136.456747\\
1.36733418335458	-181902067.840339\\
1.36743418585465	-183273155.844087\\
1.36753418835471	-184649400.467991\\
1.36763419085477	-186030228.754256\\
1.36773419335483	-187416786.618473\\
1.3678341958549	-188808501.102846\\
1.36793419835496	-190205372.207375\\
1.36803420085502	-191607399.93206\\
1.36813420335508	-193014011.319106\\
1.36823420585515	-194426352.284103\\
1.36833420835521	-195843849.869257\\
1.36843421085527	-197266504.074567\\
1.36853421335533	-198693741.942238\\
1.3686342158554	-200126709.38786\\
1.36873421835546	-201564833.453638\\
1.36883422085552	-203007541.181778\\
1.36893422335558	-204455978.487868\\
1.36903422585565	-205909572.414115\\
1.36913422835571	-207368322.960518\\
1.36923423085577	-208831657.169282\\
1.36933423335583	-210300720.955998\\
1.3694342358559	-211774941.36287\\
1.36953423835596	-213253745.432102\\
1.36963424085602	-214738279.079286\\
1.36973424335608	-216227396.388831\\
1.36983424585615	-217722243.276327\\
1.36993424835621	-219222246.78398\\
1.37003425085627	-220726833.953993\\
1.37013425335633	-222237150.701958\\
1.3702342558564	-223752051.112284\\
1.37033425835646	-225272681.100561\\
1.37043426085652	-226798467.708995\\
1.37053426335658	-228328837.979789\\
1.37063426585665	-229864937.828535\\
1.37073426835671	-231405621.339642\\
1.37083427085677	-232952034.4287\\
1.37093427335683	-234503031.180119\\
1.3710342758569	-236059757.509489\\
1.37113427835696	-237621067.501221\\
1.37123428085702	-239188107.070904\\
1.37133428335708	-240760303.260743\\
1.37143428585715	-242337083.112943\\
1.37153428835721	-243919592.543094\\
1.37163429085727	-245506685.635606\\
1.37173429335733	-247099508.30607\\
1.3718342958574	-248696914.638895\\
1.37193429835746	-250300050.549671\\
1.37203430085752	-251907770.122808\\
1.37213430335758	-253520646.316101\\
1.37223430585765	-255139252.087346\\
1.37233430835771	-256762441.520951\\
1.37243431085777	-258391360.532508\\
1.37253431335783	-260024863.206426\\
1.3726343158579	-261664095.458296\\
1.37273431835796	-263307911.372526\\
1.37283432085802	-264957456.864708\\
1.37293432335808	-266611586.01925\\
1.37303432585815	-268271444.751744\\
1.37313432835821	-269935887.146599\\
1.37323433085827	-271605486.161611\\
1.37333433335833	-273280814.754573\\
1.3734343358584	-274960727.009897\\
1.37353433835846	-276646368.843172\\
1.37363434085852	-278336594.338807\\
1.37373434335858	-280031976.4546\\
1.37383434585865	-281733088.148343\\
1.37393434835871	-283438783.504447\\
1.37403435085877	-285150208.438503\\
1.37413435335883	-286866217.03492\\
1.3742343558589	-288587382.251493\\
1.37433435835896	-290314277.046017\\
1.37443436085902	-292045755.502903\\
1.37453436335908	-293782963.537739\\
1.37463436585915	-295524755.234937\\
1.37473436835921	-297271703.552291\\
1.37483437085927	-299024381.447596\\
1.37493437335933	-300781643.005262\\
1.3750343758594	-302544061.183085\\
1.37513437835946	-304312208.938859\\
1.37523438085952	-306084940.356993\\
1.37533438335958	-307863401.353079\\
1.37543438585965	-309646446.011526\\
1.37553438835971	-311434647.29013\\
1.37563439085977	-313228578.146684\\
1.37573439335983	-315027092.6656\\
1.3758343958599	-316830763.804672\\
1.37593439835996	-318640164.521695\\
1.37603440086002	-320454148.901079\\
1.37613440336008	-322273289.90062\\
1.37623440586015	-324098160.478111\\
1.37633440836021	-325927614.717964\\
1.37643441086027	-327762225.577973\\
1.37653441336033	-329602566.015933\\
1.3766344158604	-331447490.116254\\
1.37673441836046	-333298143.794527\\
1.37683442086052	-335153381.13516\\
1.37693442336058	-337013775.09595\\
1.37703442586065	-338879898.634691\\
1.37713442836071	-340750605.835793\\
1.37723443086077	-342626469.657052\\
1.37733443336083	-344508063.056261\\
1.3774344358609	-346394240.117832\\
1.37753443836096	-348285573.799559\\
1.37763444086102	-350182637.059237\\
1.37773444336108	-352084283.981276\\
1.37783444586115	-353991660.481267\\
1.37793444836121	-355903620.643618\\
1.37803445086127	-357820737.426126\\
1.37813445336133	-359743583.786585\\
1.3782344558614	-361671013.809405\\
1.37833445836146	-363603600.452381\\
1.37843446086152	-365541916.673309\\
1.37853446336158	-367484816.556598\\
1.37863446586165	-369433446.017838\\
1.37873446836171	-371386659.141439\\
1.37883447086177	-373345028.885196\\
1.37893447336183	-375309128.206904\\
1.3790344758619	-377277811.190974\\
1.37913447836196	-379252223.752995\\
1.37923448086202	-381231219.977376\\
1.37933448336208	-383215372.821914\\
1.37943448586215	-385205255.244404\\
1.37953448836221	-387199721.329254\\
1.37963449086227	-389199916.992056\\
1.37973449336233	-391204696.317219\\
1.3798344958624	-393215205.220333\\
1.37993449836246	-395230297.785808\\
1.38003450086252	-397250546.971439\\
1.38013450336258	-399276525.735022\\
1.38023450586265	-401307088.160965\\
1.38033450836271	-403343380.16486\\
1.38043451086277	-405384255.831116\\
1.38053451336283	-407430861.075324\\
1.3806345158629	-409482049.981892\\
1.38073451836296	-411538968.466412\\
1.38083452086302	-413600470.613292\\
1.38093452336308	-415667702.338124\\
1.38103452586315	-417739517.725317\\
1.38113452836321	-419817062.690462\\
1.38123453086327	-421899191.317967\\
1.38133453336333	-423987049.523424\\
1.3814345358634	-426079491.391242\\
1.38153453836346	-428177662.837011\\
1.38163454086352	-430280417.945141\\
1.38173454336358	-432388902.631222\\
1.38183454586365	-434501970.979665\\
1.38193454836371	-436620768.906058\\
1.38203455086377	-438744150.494813\\
1.38213455336383	-440873261.661519\\
1.3822345558639	-443007529.448382\\
1.38233455836396	-445146380.897605\\
1.38243456086402	-447290961.92478\\
1.38253456336408	-449440126.614316\\
1.38263456586415	-451595020.881802\\
1.38273456836421	-453755071.769446\\
1.38283457086427	-455919706.31945\\
1.38293457336433	-458090070.447406\\
1.3830345758644	-460265591.195517\\
1.38313457836446	-462445695.60599\\
1.38323458086452	-464631529.594414\\
1.38333458336458	-466822520.202994\\
1.38343458586465	-469018094.473936\\
1.38353458836471	-471219398.322828\\
1.38363459086477	-473425858.791877\\
1.38373459336483	-475636902.923287\\
1.3838345958649	-477853676.632648\\
1.38393459836496	-480075606.962165\\
1.38403460086502	-482302693.911839\\
1.38413460336508	-484534364.523874\\
1.38423460586515	-486771764.713859\\
1.38433460836521	-489014321.524001\\
1.38443461086527	-491262034.9543\\
1.38453461336533	-493514332.046959\\
1.3846346158654	-495772358.71757\\
1.38473461836546	-498035542.008336\\
1.38483462086552	-500303881.919259\\
1.38493462336558	-502577378.450338\\
1.38503462586565	-504856031.601574\\
1.38513462836571	-507139841.372965\\
1.38523463086577	-509428234.806718\\
1.38533463336583	-511722357.818421\\
1.3854346358659	-514021637.450281\\
1.38553463836596	-516326073.702298\\
1.38563464086602	-518635666.57447\\
1.38573464336608	-520950416.066798\\
1.38583464586615	-523270322.179283\\
1.38593464836621	-525595384.911924\\
1.38603465086627	-527925604.264721\\
1.38613465336633	-530260980.237674\\
1.3862346558664	-532601512.830784\\
1.38633465836646	-534947202.044049\\
1.38643466086652	-537298047.877471\\
1.38653466336658	-539654050.331049\\
1.38663466586665	-542015209.404783\\
1.38673466836671	-544381525.098673\\
1.38683467086677	-546752997.41272\\
1.38693467336683	-549130199.304718\\
1.3870346758669	-551511984.859076\\
1.38713467836696	-553898927.033592\\
1.38723468086702	-556291025.828263\\
1.38733468336708	-558688281.24309\\
1.38743468586715	-561090693.278074\\
1.38753468836721	-563498834.891008\\
1.38763469086727	-565911560.166304\\
1.38773469336733	-568329442.061756\\
1.3878346958674	-570752480.577365\\
1.38793469836746	-573181248.670924\\
1.38803470086752	-575616319.30023\\
1.38813470336758	-578051389.929536\\
1.38823470586765	-580497919.714745\\
1.38833470836771	-582944449.499954\\
1.38843471086777	-585402438.441065\\
1.38853471336783	-587860427.382176\\
1.3886347158679	-590324145.901238\\
1.38873471836796	-592793593.998252\\
1.38883472086802	-595268771.673218\\
1.38893472336808	-597749678.926134\\
1.38903472586815	-600236315.757002\\
1.38913472836821	-602722952.58787\\
1.38923473086827	-605221048.57464\\
1.38933473336833	-607719144.56141\\
1.3894347358684	-610228699.704083\\
1.38953473836846	-612738254.846756\\
1.38963474086852	-615253539.567381\\
1.38973474336858	-617774553.865956\\
1.38983474586865	-620301297.742483\\
1.38993474836871	-622833771.196961\\
1.39003475086877	-625371974.229391\\
1.39013475336883	-627910177.26182\\
1.3902347558689	-630459839.450153\\
1.39033475836896	-633009501.638485\\
1.39043476086902	-635570622.98272\\
1.39053476336908	-638131744.326954\\
1.39063476586915	-640698595.249141\\
1.39073476836921	-643271175.749278\\
1.39083477086927	-645849485.827367\\
1.39093477336933	-648433525.483407\\
1.3910347758694	-651023294.717398\\
1.39113477836946	-653618793.529341\\
1.39123478086952	-656214292.341283\\
1.39133478336958	-658821250.309128\\
1.39143478586965	-661428208.276974\\
1.39153478836971	-664040895.82277\\
1.39163479086977	-666665042.524469\\
1.39173479336983	-669289189.226169\\
1.3918347958699	-671919065.505819\\
1.39193479836996	-674554671.363421\\
1.39203480087002	-677196006.798974\\
1.39213480337008	-679837342.234527\\
1.39223480587015	-682490136.825983\\
1.39233480837021	-685142931.417438\\
1.39243481087027	-687807185.164797\\
1.39253481337033	-690471438.912155\\
1.3926348158704	-693141422.237465\\
1.39273481837046	-695817135.140726\\
1.39283482087052	-698498577.621938\\
1.39293482337058	-701185749.681101\\
1.39303482587065	-703878651.318216\\
1.39313482837071	-706577282.533283\\
1.39323483087077	-709281643.3263\\
1.39333483337083	-711986004.119318\\
1.3934348358709	-714701824.068238\\
1.39353483837096	-717417644.017158\\
1.39363484087102	-720139193.544029\\
1.39373484337108	-722866472.648852\\
1.39383484587115	-725599481.331626\\
1.39393484837121	-728338219.592351\\
1.39403485087127	-731082687.431028\\
1.39413485337133	-733832884.847656\\
1.3942348558714	-736588811.842235\\
1.39433485837146	-739344738.836814\\
1.39443486087152	-742112124.987296\\
1.39453486337158	-744879511.137778\\
1.39463486587165	-747652626.866211\\
1.39473486837171	-750437201.750547\\
1.39483487087177	-753221776.634883\\
1.39493487337183	-756012081.09717\\
1.3950348758719	-758808115.137408\\
1.39513487837196	-761604149.177647\\
1.39523488087202	-764411642.373788\\
1.39533488337208	-767224865.14788\\
1.39543488587215	-770038087.921973\\
1.39553488837221	-772862769.851967\\
1.39563489087227	-775687451.781962\\
1.39573489337233	-778517863.289909\\
1.3958348958724	-781354004.375806\\
1.39593489837246	-784195875.039655\\
1.39603490087252	-787043475.281455\\
1.39613490337258	-789896805.101207\\
1.39623490587265	-792755864.49891\\
1.39633490837271	-795614923.896612\\
1.39643491087277	-798485442.450218\\
1.39653491337283	-801355961.003823\\
1.3966349158729	-804237938.713331\\
1.39673491837296	-807119916.422839\\
1.39683492087302	-810007623.710299\\
1.39693492337308	-812901060.575709\\
1.39703492587315	-815800227.019071\\
1.39713492837321	-818705123.040385\\
1.39723493087327	-821615748.639649\\
1.39733493337333	-824526374.238914\\
1.3974349358734	-827448458.994081\\
1.39753493837346	-830370543.749248\\
1.39763494087352	-833304087.660318\\
1.39773494337358	-836237631.571388\\
1.39783494587365	-839176905.060409\\
1.39793494837371	-842121908.127381\\
1.39803495087377	-845072640.772305\\
1.39813495337383	-848029102.99518\\
1.3982349558739	-850991294.796007\\
1.39833495837396	-853959216.174784\\
1.39843496087402	-856927137.553562\\
1.39853496337408	-859906518.088242\\
1.39863496587415	-862885898.622922\\
1.39873496837421	-865876738.313505\\
1.39883497087427	-868867578.004088\\
1.39893497337433	-871864147.272622\\
1.3990349758744	-874866446.119108\\
1.39913497837446	-877874474.543545\\
1.39923498087452	-880888232.545933\\
1.39933498337458	-883901990.548321\\
1.39943498587465	-886927207.706612\\
1.39953498837471	-889958154.442854\\
1.39963499087477	-892989101.179096\\
1.39973499337483	-896025777.493289\\
1.3998349958749	-899073912.963385\\
1.39993499837496	-902122048.433481\\
1.40003500087502	-905175913.481529\\
1.40013500337508	-908235508.107527\\
1.40023500587515	-911300832.311477\\
1.40033500837521	-914371886.093378\\
1.40043501087527	-917442939.875279\\
1.40053501337533	-920525452.813083\\
1.4006350158754	-923607965.750887\\
1.40073501837546	-926701937.844594\\
1.40083502087552	-929795909.9383\\
1.40093502337558	-932895611.609958\\
1.40103502587565	-936001042.859567\\
1.40113502837571	-939112203.687127\\
1.40123503087577	-942229094.092639\\
1.40133503337583	-945351714.076102\\
1.4014350358759	-948480063.637516\\
1.40153503837596	-951608413.19893\\
1.40163504087602	-954748221.916247\\
1.40173504337608	-957888030.633564\\
1.40183504587615	-961039298.506784\\
1.40193504837621	-964190566.380003\\
1.40203505087627	-967347563.831174\\
1.40213505337633	-970510290.860296\\
1.4022350558764	-973678747.46737\\
1.40233505837646	-976852933.652394\\
1.40243506087652	-980032849.41537\\
1.40253506337658	-983218494.756298\\
1.40263506587665	-986404140.097225\\
1.40273506837671	-989601244.594055\\
1.40283507087677	-992798349.090885\\
1.40293507337683	-996001183.165667\\
1.4030350758769	-999209746.818399\\
1.40313507837696	-1002429769.62703\\
1.40323508087702	-1005649792.43567\\
1.40333508337708	-1008869815.2443\\
1.40343508587715	-1012101297.20884\\
1.40353508837721	-1015338508.75133\\
1.40363509087727	-1018581449.87177\\
1.40373509337733	-1021824390.99221\\
1.4038350958774	-1025078791.26856\\
1.40393509837746	-1028333191.5449\\
1.40403510087752	-1031593321.39919\\
1.40413510337758	-1034859180.83144\\
1.40423510587765	-1038130769.84164\\
1.40433510837771	-1041408088.42978\\
1.40443511087777	-1044691136.59588\\
1.40453511337783	-1047979914.33993\\
1.4046351158779	-1051274421.66194\\
1.40473511837796	-1054568928.98394\\
1.40483512087802	-1057874895.46184\\
1.40493512337808	-1061180861.93975\\
1.40503512587815	-1064492557.99561\\
1.40513512837821	-1067815713.20736\\
1.40523513087827	-1071138868.41912\\
1.40533513337833	-1074467753.20883\\
1.4054351358784	-1077802367.57649\\
1.40553513837846	-1081136981.94416\\
1.40563514087852	-1084483055.46772\\
1.40573514337858	-1087834858.56923\\
1.40583514587865	-1091186661.67075\\
1.40593514837871	-1094549923.92817\\
1.40603515087877	-1097913186.18559\\
1.40613515337883	-1101282178.02096\\
1.4062351558789	-1104656899.43428\\
1.40633515837896	-1108043080.0035\\
1.40643516087902	-1111429260.57272\\
1.40653516337908	-1114815441.14195\\
1.40663516587915	-1118213080.86707\\
1.40673516837921	-1121616450.17015\\
1.40683517087927	-1125019819.47323\\
1.40693517337933	-1128434647.9322\\
1.4070351758794	-1131849476.39118\\
1.40713517837946	-1135275764.00607\\
1.40723518087952	-1138702051.62095\\
1.40733518337958	-1142134068.81378\\
1.40743518587965	-1145571815.58457\\
1.40753518837971	-1149015291.9333\\
1.40763519087977	-1152464497.85999\\
1.40773519337983	-1155913703.78668\\
1.4078351958799	-1159374368.86927\\
1.40793519837996	-1162840763.52981\\
1.40803520088002	-1166307158.19035\\
1.40813520338008	-1169785012.0068\\
1.40823520588015	-1173262865.82324\\
1.40833520838021	-1176746449.21764\\
1.40843521088027	-1180235762.18998\\
1.40853521338033	-1183730804.74028\\
1.4086352158804	-1187231576.86853\\
1.40873521838046	-1190738078.57473\\
1.40883522088052	-1194244580.28093\\
1.40893522338058	-1197762541.14303\\
1.40903522588065	-1201286231.58309\\
1.40913522838071	-1204809922.02314\\
1.40923523088077	-1208339342.04115\\
1.40933523338083	-1211880221.21506\\
1.4094352358809	-1215421100.38897\\
1.40953523838096	-1218967709.14083\\
1.40963524088102	-1222520047.47064\\
1.40973524338108	-1226078115.3784\\
1.40983524588115	-1229636183.28616\\
1.40993524838121	-1233205710.34983\\
1.41003525088127	-1236780966.99144\\
1.41013525338133	-1240356223.63306\\
1.4102352558814	-1243937209.85263\\
1.41033525838146	-1247529655.2281\\
1.41043526088152	-1251122100.60357\\
1.41053526338158	-1254720275.55699\\
1.41063526588165	-1258324180.08836\\
1.41073526838171	-1261933814.19769\\
1.41083527088177	-1265549177.88496\\
1.41093527338183	-1269164541.57224\\
1.4110352758819	-1272791364.41542\\
1.41113527838196	-1276423916.83655\\
1.41123528088202	-1280056469.25767\\
1.41133528338208	-1283694751.25676\\
1.41143528588215	-1287344492.41174\\
1.41153528838221	-1290994233.56672\\
1.41163529088227	-1294649704.29966\\
1.41173529338233	-1298310904.61054\\
1.4118352958824	-1301977834.49938\\
1.41193529838246	-1305650493.96617\\
1.41203530088252	-1309323153.43296\\
1.41213530338258	-1313007272.05565\\
1.41223530588265	-1316697120.25629\\
1.41233530838271	-1320386968.45693\\
1.41243531088277	-1324082546.23553\\
1.41253531338283	-1327789583.17002\\
1.4126353158829	-1331496620.10452\\
1.41273531838296	-1335209386.61697\\
1.41283532088302	-1338927882.70737\\
1.41293532338308	-1342652108.37572\\
1.41303532588315	-1346376334.04407\\
1.41313532838321	-1350112018.86832\\
1.41323533088327	-1353853433.27052\\
1.41333533338333	-1357594847.67273\\
1.4134353358834	-1361347721.23084\\
1.41353533838346	-1365100594.78894\\
1.41363534088352	-1368859197.925\\
1.41373534338358	-1372623530.63901\\
1.41383534588365	-1376393592.93097\\
1.41393534838371	-1380169384.80088\\
1.41403535088377	-1383950906.24875\\
1.41413535338383	-1387738157.27456\\
1.4142353558839	-1391525408.30038\\
1.41433535838396	-1395324118.48209\\
1.41443536088402	-1399122828.66381\\
1.41453536338408	-1402932998.00143\\
1.41463536588415	-1406743167.33905\\
1.41473536838421	-1410559066.25462\\
1.41483537088427	-1414380694.74815\\
1.41493537338433	-1418208052.81962\\
1.4150353758844	-1422041140.46904\\
1.41513537838446	-1425879957.69642\\
1.41523538088452	-1429724504.50175\\
1.41533538338458	-1433574780.88503\\
1.41543538588465	-1437425057.26831\\
1.41553538838471	-1441286792.80749\\
1.41563539088477	-1445148528.34667\\
1.41573539338483	-1449015993.4638\\
1.4158353958849	-1452889188.15889\\
1.41593539838496	-1456773842.00987\\
1.41603540088502	-1460652766.28291\\
1.41613540338508	-1464543149.71185\\
1.41623540588515	-1468439262.71874\\
1.41633540838521	-1472341105.30358\\
1.41643541088527	-1476248677.46637\\
1.41653541338533	-1480156249.62916\\
1.4166354158854	-1484075280.94786\\
1.41673541838546	-1487994312.26655\\
1.41683542088552	-1491919073.1632\\
1.41693542338558	-1495849563.6378\\
1.41703542588565	-1499785783.69035\\
1.41713542838571	-1503727733.32085\\
1.41723543088577	-1507675412.5293\\
1.41733543338583	-1511628821.3157\\
1.4174354358859	-1515587959.68005\\
1.41753543838596	-1519547098.04441\\
1.41763544088602	-1523517695.56466\\
1.41773544338608	-1527488293.08492\\
1.41783544588615	-1531470349.76108\\
1.41793544838621	-1535452406.43724\\
1.41803545088627	-1539440192.69135\\
1.41813545338633	-1543433708.52341\\
1.4182354558864	-1547432953.93342\\
1.41833545838646	-1551437928.92139\\
1.41843546088652	-1555448633.4873\\
1.41853546338658	-1559465067.63117\\
1.41863546588665	-1563481501.77504\\
1.41873546838671	-1567509395.07481\\
1.41883547088677	-1571537288.37458\\
1.41893547338683	-1575570911.2523\\
1.4190354758869	-1579615993.28592\\
1.41913547838696	-1583661075.31955\\
1.41923548088702	-1587711886.93112\\
1.41933548338708	-1591768428.12065\\
1.41943548588715	-1595830698.88813\\
1.41953548838721	-1599898699.23355\\
1.41963549088727	-1603966699.57898\\
1.41973549338733	-1608046159.08031\\
1.4198354958874	-1612125618.58165\\
1.41993549838746	-1616216537.23888\\
1.42003550088752	-1620307455.89611\\
1.42013550338758	-1624404104.1313\\
1.42023550588765	-1628506481.94444\\
1.42033550838771	-1632620318.91348\\
1.42043551088777	-1636734155.88251\\
1.42053551338783	-1640847992.85155\\
1.4206355158879	-1644973288.9765\\
1.42073551838796	-1649104314.67939\\
1.42083552088802	-1653235340.38228\\
1.42093552338808	-1657377825.24108\\
1.42103552588815	-1661520310.09987\\
1.42113552838821	-1665674254.11457\\
1.42123553088827	-1669828198.12927\\
1.42133553338833	-1673987871.72192\\
1.4214355358884	-1678153274.89252\\
1.42153553838846	-1682324407.64107\\
1.42163554088852	-1686501269.96758\\
1.42173554338858	-1690683861.87203\\
1.42183554588865	-1694872183.35444\\
1.42193554838871	-1699060504.83685\\
1.42203555088877	-1703260285.47515\\
1.42213555338883	-1707460066.11346\\
1.4222355558889	-1711665576.32972\\
1.42233555838896	-1715882545.70189\\
1.42243556088902	-1720099515.07405\\
1.42253556338908	-1724322214.02416\\
1.42263556588915	-1728550642.55223\\
1.42273556838921	-1732784800.65825\\
1.42283557088927	-1737018958.76426\\
1.42293557338933	-1741264576.02618\\
1.4230355758894	-1745515922.86605\\
1.42313557838946	-1749767269.70592\\
1.42323558088952	-1754030075.7017\\
1.42333558338958	-1758292881.69747\\
1.42343558588965	-1762561417.27119\\
1.42353558838971	-1766835682.42287\\
1.42363559088977	-1771121406.73045\\
1.42373559338983	-1775407131.03803\\
1.4238355958899	-1779692855.34561\\
1.42393559838996	-1783990038.80909\\
1.42403560089002	-1788292951.85052\\
1.42413560339008	-1792601594.4699\\
1.42423560589015	-1796910237.08929\\
1.42433560839021	-1801230338.86457\\
1.42443561089027	-1805550440.63986\\
1.42453561339033	-1809876271.9931\\
1.4246356158904	-1814207832.92429\\
1.42473561839046	-1818545123.43343\\
1.42483562089052	-1822888143.52052\\
1.42493562339058	-1827236893.18556\\
1.42503562589065	-1831591372.42856\\
1.42513562839071	-1835951581.2495\\
1.42523563089077	-1840311790.07045\\
1.42533563339083	-1844683458.0473\\
1.4254356358909	-1849055126.02414\\
1.42553563839096	-1853438253.15689\\
1.42563564089102	-1857821380.28965\\
1.42573564339108	-1862210237.00035\\
1.42583564589115	-1866604823.289\\
1.42593564839121	-1871005139.15561\\
1.42603565089127	-1875411184.60016\\
1.42613565339133	-1879822959.62267\\
1.4262356558914	-1884240464.22313\\
1.42633565839146	-1888657968.82359\\
1.42643566089152	-1893086932.57995\\
1.42653566339158	-1897515896.33631\\
1.42663566589165	-1901956319.24857\\
1.42673566839171	-1906396742.16084\\
1.42683567089177	-1910842894.65105\\
1.42693567339183	-1915294776.71922\\
1.4270356758919	-1919758117.94329\\
1.42713567839196	-1924215729.58941\\
1.42723568089202	-1928684800.39143\\
1.42733568339208	-1933159600.7714\\
1.42743568589215	-1937640130.72932\\
1.42753568839221	-1942120660.68724\\
1.42763569089227	-1946612649.80107\\
1.42773569339233	-1951104638.9149\\
1.4278356958924	-1955602357.60667\\
1.42793569839246	-1960111535.45435\\
1.42803570089252	-1964620713.30203\\
1.42813570339258	-1969135620.72766\\
1.42823570589265	-1973656257.73124\\
1.42833570839271	-1978182624.31278\\
1.42843571089277	-1982708990.89431\\
1.42853571339283	-1987246816.63175\\
1.4286357158929	-1991790371.94713\\
1.42873571839296	-1996333927.26252\\
1.42883572089302	-2000888941.73381\\
1.42893572339308	-2005443956.2051\\
1.42903572589315	-2010004700.25434\\
1.42913572839321	-2014571173.88154\\
1.42923573089327	-2019149106.66463\\
1.42933573339333	-2023727039.44773\\
1.4294357358934	-2028304972.23082\\
1.42953573839346	-2032894364.16982\\
1.42963574089352	-2037489485.68677\\
1.42973574339358	-2042090336.78167\\
1.42983574589365	-2046691187.87657\\
1.42993574839371	-2051303498.12737\\
1.43003575089377	-2055915808.37818\\
1.43013575339383	-2060533848.20693\\
1.4302357558939	-2065157617.61364\\
1.43033575839396	-2069787116.59829\\
1.43043576089402	-2074422345.1609\\
1.43053576339408	-2079063303.30146\\
1.43063576589415	-2083709991.01997\\
1.43073576839421	-2088362408.31643\\
1.43083577089427	-2093020555.19085\\
1.43093577339433	-2097678702.06526\\
1.4310357758944	-2102348308.09558\\
1.43113577839446	-2107017914.12589\\
1.43123578089452	-2111693249.73416\\
1.43133578339458	-2116380044.49833\\
1.43143578589465	-2121066839.2625\\
1.43153578839471	-2125759363.60462\\
1.43163579089477	-2130457617.5247\\
1.43173579339483	-2135161601.02272\\
1.4318357958949	-2139865584.52074\\
1.43193579839496	-2144581027.17467\\
1.43203580089502	-2149302199.40655\\
1.43213580339508	-2154023371.63843\\
1.43223580589515	-2158756003.02621\\
1.43233580839521	-2163488634.41399\\
1.43243581089527	-2168226995.37972\\
1.43253581339533	-2172971085.9234\\
1.4326358158954	-2177726635.62299\\
1.43273581839546	-2182482185.32258\\
1.43283582089552	-2187237735.02216\\
1.43293582339558	-2192004743.87765\\
1.43303582589565	-2196777482.31109\\
1.43313582839571	-2201555950.32248\\
1.43323583089577	-2206334418.33387\\
1.43333583339583	-2211124345.50117\\
1.4334358358959	-2215914272.66846\\
1.43353583839596	-2220709929.4137\\
1.43363584089602	-2225517045.31485\\
1.43373584339608	-2230324161.216\\
1.43383584589615	-2235137006.6951\\
1.43393584839621	-2239955581.75215\\
1.43403585089627	-2244779886.38715\\
1.43413585339633	-2249604191.02215\\
1.4342358558964	-2254439954.81306\\
1.43433585839646	-2259281448.18191\\
1.43443586089652	-2264122941.55077\\
1.43453586339659	-2268975894.07552\\
1.43463586589665	-2273828846.60028\\
1.43473586839671	-2278687528.70299\\
1.43483587089677	-2283551940.38365\\
1.43493587339683	-2288422081.64226\\
1.4350358758969	-2293297952.47883\\
1.43513587839696	-2298179552.89334\\
1.43523588089702	-2303066882.88581\\
1.43533588339708	-2307959942.45623\\
1.43543588589715	-2312858731.60459\\
1.43553588839721	-2317757520.75296\\
1.43563589089727	-2322667769.05723\\
1.43573589339733	-2327578017.36151\\
1.4358358958974	-2332493995.24373\\
1.43593589839746	-2337415702.7039\\
1.43603590089752	-2342348869.31998\\
1.43613590339758	-2347282035.93605\\
1.43623590589765	-2352220932.13008\\
1.43633590839771	-2357159828.32411\\
1.43643591089777	-2362110183.67404\\
1.43653591339784	-2367066268.60192\\
1.4366359158979	-2372028083.10775\\
1.43673591839796	-2376989897.61359\\
1.43683592089802	-2381963171.27532\\
1.43693592339808	-2386936444.93706\\
1.43703592589815	-2391915448.17674\\
1.43713592839821	-2396900180.99438\\
1.43723593089827	-2401890643.38997\\
1.43733593339833	-2406886835.36351\\
1.4374359358984	-2411888756.91501\\
1.43753593839846	-2416896408.04445\\
1.43763594089852	-2421909788.75184\\
1.43773594339859	-2426928899.03719\\
1.43783594589865	-2431948009.32254\\
1.43793594839871	-2436978578.76378\\
1.43803595089877	-2442009148.20503\\
1.43813595339883	-2447051176.80218\\
1.4382359558989	-2452093205.39934\\
1.43833595839896	-2457140963.57444\\
1.43843596089902	-2462194451.32749\\
1.43853596339909	-2467253668.6585\\
1.43863596589915	-2472318615.56745\\
1.43873596839921	-2477389292.05436\\
1.43883597089927	-2482459968.54127\\
1.43893597339933	-2487542104.18408\\
1.4390359758994	-2492629969.40484\\
1.43913597839946	-2497717834.6256\\
1.43923598089952	-2502811429.42432\\
1.43933598339958	-2507916483.37893\\
1.43943598589965	-2513021537.33355\\
1.43953598839971	-2518132320.86611\\
1.43963599089977	-2523248833.97663\\
1.43973599339984	-2528371076.6651\\
1.4398359958999	-2533499048.93152\\
1.43993599839996	-2538632750.7759\\
1.44003600090002	-2543772182.19822\\
1.44013600340008	-2548911613.62054\\
1.44023600590015	-2554062504.19877\\
1.44033600840021	-2559213394.77699\\
1.44043601090027	-2564375744.51112\\
1.44053601340034	-2569538094.24525\\
1.4406360159004	-2574706173.55733\\
1.44073601840046	-2579879982.44736\\
1.44083602090052	-2585059520.91535\\
1.44093602340059	-2590244788.96128\\
1.44103602590065	-2595435786.58517\\
1.44113602840071	-2600632513.787\\
1.44123603090077	-2605829240.98884\\
1.44133603340083	-2611037427.34658\\
1.4414360359009	-2616251343.28227\\
1.44153603840096	-2621465259.21796\\
1.44163604090102	-2626684904.7316\\
1.44173604340109	-2631916009.40114\\
1.44183604590115	-2637147114.07069\\
1.44193604840121	-2642383948.31818\\
1.44203605090127	-2647626512.14363\\
1.44213605340133	-2652874805.54703\\
1.4422360559014	-2658128828.52838\\
1.44233605840146	-2663388581.08768\\
1.44243606090152	-2668648333.64698\\
1.44253606340159	-2673919545.36219\\
1.44263606590165	-2679190757.07739\\
1.44273606840171	-2684473427.94849\\
1.44283607090177	-2689756098.8196\\
1.44293607340184	-2695044499.26866\\
1.4430360759019	-2700344358.87362\\
1.44313607840196	-2705644218.47858\\
1.44323608090202	-2710949807.66149\\
1.44333608340208	-2716261126.42235\\
1.44343608590215	-2721572445.18322\\
1.44353608840221	-2726895223.09998\\
1.44363609090227	-2732223730.5947\\
1.44373609340234	-2737552238.08941\\
1.4438360959024	-2742892204.74003\\
1.44393609840246	-2748232171.39065\\
1.44403610090252	-2753583597.19717\\
1.44413610340259	-2758935023.0037\\
1.44423610590265	-2764292178.38817\\
1.44433610840271	-2769655063.35059\\
1.44443611090277	-2775023677.89097\\
1.44453611340284	-2780398022.0093\\
1.4446361159029	-2785778095.70558\\
1.44473611840296	-2791163898.97981\\
1.44483612090302	-2796549702.25403\\
1.44493612340309	-2801946964.68417\\
1.44503612590315	-2807349956.69225\\
1.44513612840321	-2812752948.70034\\
1.44523613090327	-2818161670.28637\\
1.44533613340333	-2823581851.02831\\
1.4454361359034	-2829002031.77025\\
1.44553613840346	-2834427942.09013\\
1.44563614090352	-2839859581.98797\\
1.44573614340359	-2845296951.46377\\
1.44583614590365	-2850740050.51751\\
1.44593614840371	-2856183149.57125\\
1.44603615090377	-2861637707.7809\\
1.44613615340384	-2867097995.56849\\
1.4462361559039	-2872558283.35609\\
1.44633615840396	-2878030030.29959\\
1.44643616090402	-2883501777.24309\\
1.44653616340409	-2888979253.76454\\
1.44663616590415	-2894462459.86394\\
1.44673616840421	-2899957125.11925\\
1.44683617090427	-2905451790.37455\\
1.44693617340434	-2910952185.20781\\
1.4470361759044	-2916452580.04106\\
1.44713617840446	-2921964434.03022\\
1.44723618090452	-2927482017.59733\\
1.44733618340459	-2932999601.16444\\
1.44743618590465	-2938528643.88745\\
1.44753618840471	-2944057686.61047\\
1.44763619090477	-2949598188.48938\\
1.44773619340484	-2955138690.3683\\
1.4478361959049	-2960684921.82516\\
1.44793619840496	-2966236882.85998\\
1.44803620090502	-2971794573.47275\\
1.44813620340509	-2977357993.66347\\
1.44823620590515	-2982927143.43214\\
1.44833620840521	-2988502022.77876\\
1.44843621090527	-2994082631.70334\\
1.44853621340534	-2999663240.62791\\
1.4486362159054	-3005255308.70839\\
1.44873621840546	-3010847376.78887\\
1.44883622090552	-3016450904.02525\\
1.44893622340559	-3022054431.26162\\
1.44903622590565	-3027663688.07596\\
1.44913622840571	-3033278674.46824\\
1.44923623090577	-3038899390.43847\\
1.44933623340584	-3044525835.98666\\
1.4494362359059	-3050158011.11279\\
1.44953623840596	-3055795915.81688\\
1.44963624090602	-3061439550.09892\\
1.44973624340609	-3067083184.38096\\
1.44983624590615	-3072738277.8189\\
1.44993624840621	-3078399100.83479\\
1.45003625090627	-3084059923.85068\\
1.45013625340634	-3089726476.44453\\
1.4502362559064	-3095398758.61632\\
1.45033625840646	-3101082499.94402\\
1.45043626090652	-3106766241.27172\\
1.45053626340659	-3112455712.17737\\
1.45063626590665	-3118150912.66097\\
1.45073626840671	-3123846113.14457\\
1.45083627090677	-3129552772.78407\\
1.45093627340684	-3135265162.00152\\
1.4510362759069	-3140977551.21898\\
1.45113627840696	-3146701399.59234\\
1.45123628090702	-3152425247.96569\\
1.45133628340709	-3158160555.49495\\
1.45143628590715	-3163895863.02421\\
1.45153628840721	-3169636900.13142\\
1.45163629090727	-3175383666.81658\\
1.45173629340734	-3181136163.0797\\
1.4518362959074	-3186894388.92076\\
1.45193629840746	-3192658344.33978\\
1.45203630090752	-3198428029.33675\\
1.45213630340759	-3204203443.91166\\
1.45223630590765	-3209978858.48658\\
1.45233630840771	-3215765732.2174\\
1.45243631090777	-3221552605.94823\\
1.45253631340784	-3227350938.83495\\
1.4526363159079	-3233149271.72167\\
1.45273631840796	-3238953334.18635\\
1.45283632090802	-3244763126.22898\\
1.45293632340809	-3250578647.84955\\
1.45303632590815	-3256399899.04808\\
1.45313632840821	-3262226879.82456\\
1.45323633090827	-3268059590.17899\\
1.45333633340834	-3273898030.11138\\
1.4534363359084	-3279736470.04376\\
1.45353633840846	-3285586369.13205\\
1.45363634090852	-3291441997.79828\\
1.45373634340859	-3297297626.46452\\
1.45383634590865	-3303158984.70871\\
1.45393634840871	-3309031802.1088\\
1.45403635090877	-3314904619.50889\\
1.45413635340884	-3320783166.48693\\
1.4542363559089	-3326667443.04293\\
1.45433635840896	-3332557449.17687\\
1.45443636090902	-3338453184.88877\\
1.45453636340909	-3344348920.60066\\
1.45463636590915	-3350256115.46846\\
1.45473636840921	-3356169039.91421\\
1.45483637090927	-3362081964.35996\\
1.45493637340934	-3368006347.96162\\
1.4550363759094	-3373930731.56327\\
1.45513637840946	-3379866574.32082\\
1.45523638090952	-3385802417.07838\\
1.45533638340959	-3391743989.41389\\
1.45543638590965	-3397691291.32734\\
1.45553638840971	-3403644322.81875\\
1.45563639090977	-3409603083.88811\\
1.45573639340984	-3415567574.53543\\
1.4558363959099	-3421532065.18274\\
1.45593639840996	-3427508014.98595\\
1.45603640091002	-3433489694.36712\\
1.45613640341009	-3439471373.74828\\
1.45623640591015	-3445464512.28535\\
1.45633640841021	-3451457650.82242\\
1.45643641091027	-3457456518.93744\\
1.45653641341034	-3463461116.63041\\
1.4566364159104	-3469477173.47928\\
1.45673641841046	-3475493230.32816\\
1.45683642091052	-3481515016.75498\\
1.45693642341059	-3487542532.75976\\
1.45703642591065	-3493570048.76454\\
1.45713642841071	-3499609023.92521\\
1.45723643091077	-3505653728.66384\\
1.45733643341084	-3511698433.40247\\
1.4574364359109	-3517754597.29701\\
1.45753643841096	-3523810761.19154\\
1.45763644091102	-3529878384.24198\\
1.45773644341109	-3535946007.29241\\
1.45783644591115	-3542019359.9208\\
1.45793644841121	-3548098442.12714\\
1.45803645091127	-3554183253.91143\\
1.45813645341134	-3560273795.27367\\
1.4582364559114	-3566370066.21386\\
1.45833645841146	-3572472066.732\\
1.45843646091152	-3578574067.25014\\
1.45853646341159	-3584687526.92419\\
1.45863646591165	-3590806716.17619\\
1.45873646841171	-3596925905.42818\\
1.45883647091177	-3603050824.25813\\
1.45893647341184	-3609187202.24398\\
1.4590364759119	-3615323580.22984\\
1.45913647841196	-3621465687.79364\\
1.45923648091202	-3627613524.93539\\
1.45933648341209	-3633767091.6551\\
1.45943648591215	-3639926387.95275\\
1.45953648841221	-3646091413.82836\\
1.45963649091227	-3652262169.28192\\
1.45973649341234	-3658438654.31343\\
1.4598364959124	-3664615139.34494\\
1.45993649841246	-3670803083.53235\\
1.46003650091252	-3676991027.71977\\
1.46013650341259	-3683190431.06308\\
1.46023650591265	-3689389834.4064\\
1.46033650841271	-3695594967.32766\\
1.46043651091277	-3701805829.82688\\
1.46053651341284	-3708022421.90405\\
1.4606365159129	-3714244743.55917\\
1.46073651841296	-3720472794.79224\\
1.46083652091302	-3726706575.60327\\
1.46093652341309	-3732946085.99224\\
1.46103652591315	-3739191325.95917\\
1.46113652841321	-3745436565.92609\\
1.46123653091327	-3751693265.04892\\
1.46133653341334	-3757949964.17175\\
1.4614365359134	-3764218122.45048\\
1.46153653841346	-3770486280.72921\\
1.46163654091352	-3776760168.5859\\
1.46173654341359	-3783039786.02053\\
1.46183654591365	-3789325133.03311\\
1.46193654841371	-3795616209.62365\\
1.46203655091377	-3801913015.79214\\
1.46213655341384	-3808215551.53858\\
1.4622365559139	-3814523816.86297\\
1.46233655841396	-3820837811.76531\\
1.46243656091402	-3827151806.66765\\
1.46253656341409	-3833477260.7259\\
1.46263656591415	-3839802714.78414\\
1.46273656841421	-3846139627.99829\\
1.46283657091427	-3852476541.21243\\
1.46293657341434	-3858819184.00453\\
1.4630365759144	-3865173285.95253\\
1.46313657841446	-3871527387.90053\\
1.46323658091452	-3877887219.42649\\
1.46333658341459	-3884252780.53039\\
1.46343658591465	-3890618341.63429\\
1.46353658841471	-3896995361.8941\\
1.46363659091477	-3903378111.73186\\
1.46373659341484	-3909766591.14757\\
1.4638365959149	-3916155070.56327\\
1.46393659841496	-3922555009.13489\\
1.46403660091502	-3928954947.7065\\
1.46413660341509	-3935360615.85606\\
1.46423660591515	-3941777743.16152\\
1.46433660841521	-3948194870.46699\\
1.46443661091527	-3954617727.35041\\
1.46453661341534	-3961046313.81177\\
1.4646366159154	-3967480629.85109\\
1.46473661841546	-3973920675.46836\\
1.46483662091552	-3980366450.66359\\
1.46493662341559	-3986812225.85881\\
1.46503662591565	-3993269460.20993\\
1.46513662841571	-3999732424.13901\\
1.46523663091577	-4006195388.06808\\
1.46533663341584	-4012664081.57511\\
1.4654366359159	-4019144234.23804\\
1.46553663841596	-4025624386.90097\\
1.46563664091602	-4032110269.14185\\
1.46573664341609	-4038601880.96068\\
1.46583664591615	-4045104951.93542\\
1.46593664841621	-4051602293.3322\\
1.46603665091627	-4058111093.88489\\
1.46613665341634	-4064625624.01552\\
1.4662366559164	-4071145883.72411\\
1.46633665841646	-4077671873.01065\\
1.46643666091652	-4084197862.29719\\
1.46653666341659	-4090735310.73964\\
1.46663666591665	-4097272759.18208\\
1.46673666841671	-4103821666.78042\\
1.46683667091677	-4110370574.37877\\
1.46693667341684	-4116925211.55507\\
1.4670366759169	-4123485578.30931\\
1.46713667841696	-4130051674.64151\\
1.46723668091702	-4136623500.55166\\
1.46733668341709	-4143201056.03977\\
1.46743668591715	-4149784341.10582\\
1.46753668841721	-4156373355.74982\\
1.46763669091727	-4162968099.97178\\
1.46773669341734	-4169562844.19373\\
1.4678366959174	-4176169047.57159\\
1.46793669841746	-4182775250.94945\\
1.46803670091752	-4189392913.48321\\
1.46813670341759	-4196010576.01697\\
1.46823670591765	-4202633968.12869\\
1.46833670841771	-4209268819.3963\\
1.46843671091777	-4215903670.66392\\
1.46853671341784	-4222544251.50948\\
1.4686367159179	-4229190561.933\\
1.46873671841796	-4235842601.93447\\
1.46883672091802	-4242500371.51389\\
1.46893672341809	-4249158141.09331\\
1.46903672591815	-4255827369.82863\\
1.46913672841821	-4262502328.14191\\
1.46923673091827	-4269177286.45518\\
1.46933673341834	-4275863703.92436\\
1.4694367359184	-4282550121.39353\\
1.46953673841846	-4289242268.44066\\
1.46963674091852	-4295940145.06574\\
1.46973674341859	-4302649480.84672\\
1.46983674591865	-4309358816.6277\\
1.46993674841871	-4316073881.98664\\
1.47003675091877	-4322794676.92352\\
1.47013675341884	-4329521201.43836\\
1.4702367559189	-4336247725.95319\\
1.47033675841896	-4342985709.62393\\
1.47043676091902	-4349729422.87262\\
1.47053676341909	-4356473136.12131\\
1.47063676591915	-4363228308.5259\\
1.47073676841921	-4369983480.9305\\
1.47083677091927	-4376750112.49099\\
1.47093677341934	-4383516744.05149\\
1.4710367759194	-4390289105.18993\\
1.47113677841946	-4397067195.90633\\
1.47123678091952	-4403851016.20068\\
1.47133678341959	-4410640566.07298\\
1.47143678591965	-4417435845.52323\\
1.47153678841971	-4424236854.55143\\
1.47163679091977	-4431043593.15759\\
1.47173679341984	-4437856061.34169\\
1.4718367959199	-4444668529.5258\\
1.47193679841996	-4451492456.86581\\
1.47203680092002	-4458316384.20582\\
1.47213680342009	-4465151770.70173\\
1.47223680592015	-4471987157.19764\\
1.47233680842021	-4478828273.2715\\
1.47243681092027	-4485680848.50126\\
1.47253681342034	-4492533423.73103\\
1.4726368159204	-4499391728.53874\\
1.47273681842046	-4506255762.92441\\
1.47283682092052	-4513125526.88803\\
1.47293682342059	-4520001020.4296\\
1.47303682592065	-4526876513.97117\\
1.47313682842071	-4533763466.66864\\
1.47323683092077	-4540656148.94407\\
1.47333683342084	-4547548831.21949\\
1.4734368359209	-4554452972.65082\\
1.47353683842096	-4561357114.08214\\
1.47363684092102	-4568266985.09142\\
1.47373684342109	-4575188315.2566\\
1.47383684592115	-4582109645.42178\\
1.47393684842121	-4589036705.16491\\
1.47403685092127	-4595969494.486\\
1.47413685342134	-4602908013.38503\\
1.4742368559214	-4609852261.86202\\
1.47433685842146	-4616802239.91695\\
1.47443686092152	-4623757947.54984\\
1.47453686342159	-4630713655.18273\\
1.47463686592165	-4637680821.97152\\
1.47473686842171	-4644653718.33826\\
1.47483687092177	-4651626614.705\\
1.47493687342184	-4658605240.6497\\
1.4750368759219	-4665595325.75029\\
1.47513687842196	-4672585410.85089\\
1.47523688092202	-4679581225.52944\\
1.47533688342209	-4686582769.78594\\
1.47543688592215	-4693595773.19834\\
1.47553688842221	-4700608776.61074\\
1.47563689092227	-4707621780.02314\\
1.47573689342234	-4714646242.59144\\
1.4758368959224	-4721676434.7377\\
1.47593689842246	-4728712356.46191\\
1.47603690092252	-4735748278.18611\\
1.47613690342259	-4742795659.06622\\
1.47623690592265	-4749843039.94633\\
1.47633690842271	-4756901879.98234\\
1.47643691092277	-4763960720.01835\\
1.47653691342284	-4771031019.21027\\
1.4766369159229	-4778101318.40218\\
1.47673691842296	-4785177347.17205\\
1.47683692092302	-4792259105.51987\\
1.47693692342309	-4799346593.44563\\
1.47703692592315	-4806439810.94935\\
1.47713692842321	-4813538758.03102\\
1.47723693092327	-4820643434.69065\\
1.47733693342334	-4827748111.35027\\
1.4774369359234	-4834864247.16579\\
1.47753693842346	-4841986112.55927\\
1.47763694092352	-4849107977.95274\\
1.47773694342359	-4856241302.50212\\
1.47783694592365	-4863374627.0515\\
1.47793694842371	-4870513681.17883\\
1.47803695092377	-4877664194.46206\\
1.47813695342384	-4884814707.7453\\
1.4782369559239	-4891970950.60648\\
1.47833695842396	-4899132923.04562\\
1.47843696092402	-4906300625.0627\\
1.47853696342409	-4913474056.65774\\
1.47863696592415	-4920653217.83073\\
1.47873696842421	-4927832379.00372\\
1.47883697092427	-4935022999.33261\\
1.47893697342434	-4942219349.23946\\
1.4790369759244	-4949415699.1463\\
1.47913697842446	-4956623508.20904\\
1.47923698092452	-4963831317.27179\\
1.47933698342459	-4971044855.91249\\
1.47943698592465	-4978269853.70909\\
1.47953698842471	-4985494851.50569\\
1.47963699092477	-4992725578.88024\\
1.47973699342484	-4999962035.83274\\
1.4798369959249	-5007204222.36319\\
1.47993699842496	-5014452138.4716\\
1.48003700092502	-5021705784.15795\\
1.48013700342509	-5028965159.42226\\
1.48023700592515	-5036224534.68657\\
1.48033700842521	-5043495369.10678\\
1.48043701092527	-5050771933.10494\\
1.48053701342534	-5058048497.1031\\
1.4806370159254	-5065336520.25717\\
1.48073701842546	-5072624543.41123\\
1.48083702092552	-5079918296.14325\\
1.48093702342559	-5087223508.03116\\
1.48103702592565	-5094528719.91908\\
1.48113702842571	-5101839661.38495\\
1.48123703092577	-5109156332.42877\\
1.48133703342584	-5116478733.05054\\
1.4814370359259	-5123806863.25027\\
1.48153703842596	-5131140723.02794\\
1.48163704092602	-5138474582.80562\\
1.48173704342609	-5145819901.73919\\
1.48183704592615	-5153170950.25072\\
1.48193704842621	-5160521998.76225\\
1.48203705092627	-5167884506.42968\\
1.48213705342634	-5175247014.09711\\
1.4822370559264	-5182620980.92045\\
1.48233705842646	-5189994947.74378\\
1.48243706092652	-5197374644.14506\\
1.48253706342659	-5204760070.1243\\
1.48263706592665	-5212151225.68149\\
1.48273706842671	-5219548110.81663\\
1.48283707092677	-5226950725.52972\\
1.48293707342684	-5234359069.82076\\
1.4830370759269	-5241773143.68975\\
1.48313707842696	-5249192947.1367\\
1.48323708092702	-5256618480.16159\\
1.48333708342709	-5264044013.18649\\
1.48343708592715	-5271481005.36728\\
1.48353708842721	-5278917997.54808\\
1.48363709092727	-5286366448.88478\\
1.48373709342734	-5293814900.22149\\
1.4838370959274	-5301269081.13614\\
1.48393709842746	-5308734721.20669\\
1.48403710092752	-5316200361.27725\\
1.48413710342759	-5323671730.92575\\
1.48423710592765	-5331148830.15221\\
1.48433710842771	-5338631658.95662\\
1.48443711092777	-5346120217.33898\\
1.48453711342784	-5353614505.29929\\
1.4846371159279	-5361108793.2596\\
1.48473711842796	-5368614540.37581\\
1.48483712092802	-5376126017.06998\\
1.48493712342809	-5383637493.76414\\
1.48503712592815	-5391160429.61421\\
1.48513712842821	-5398683365.46428\\
1.48523713092827	-5406217760.47025\\
1.48533713342834	-5413752155.47622\\
1.4854371359284	-5421292280.06014\\
1.48553713842846	-5428838134.22201\\
1.48563714092852	-5436389717.96184\\
1.48573714342859	-5443947031.27961\\
1.48583714592865	-5451510074.17534\\
1.48593714842871	-5459078846.64902\\
1.48603715092877	-5466653348.70065\\
1.48613715342884	-5474233580.33023\\
1.4862371559289	-5481819541.53776\\
1.48633715842896	-5489405502.74529\\
1.48643716092902	-5497002923.10873\\
1.48653716342909	-5504600343.47216\\
1.48663716592915	-5512209222.9915\\
1.48673716842921	-5519818102.51084\\
1.48683717092927	-5527432711.60813\\
1.48693717342934	-5535058779.86132\\
1.4870371759294	-5542684848.11451\\
1.48713717842946	-5550316645.94565\\
1.48723718092952	-5557954173.35475\\
1.48733718342959	-5565597430.34179\\
1.48743718592965	-5573246416.90679\\
1.48753718842971	-5580901133.04974\\
1.48763719092977	-5588561578.77063\\
1.48773719342984	-5596222024.49153\\
1.4878371959299	-5603893929.36833\\
1.48793719842996	-5611571563.82309\\
1.48803720093002	-5619249198.27784\\
1.48813720343009	-5626938291.8885\\
1.48823720593015	-5634627385.49915\\
1.48833720843021	-5642322208.68776\\
1.48843721093027	-5650028491.03227\\
1.48853721343034	-5657734773.37678\\
1.4886372159304	-5665446785.29924\\
1.48873721843046	-5673164526.79965\\
1.48883722093052	-5680887997.87801\\
1.48893722343059	-5688617198.53433\\
1.48903722593065	-5696352128.7686\\
1.48913722843071	-5704092788.58081\\
1.48923723093077	-5711833448.39303\\
1.48933723343084	-5719585567.36115\\
1.4894372359309	-5727343415.90722\\
1.48953723843096	-5735078346.14149\\
1.48963724093102	-5742870572.15527\\
1.48973724343109	-5750662798.16905\\
1.48983724593115	-5758397728.40331\\
1.48993724843121	-5766189954.41709\\
1.49003725093127	-5773982180.43087\\
1.49013725343134	-5781774406.44465\\
1.4902372559314	-5789566632.45843\\
1.49033725843146	-5797358858.47221\\
1.49043726093152	-5805208380.2655\\
1.49053726343159	-5813000606.27928\\
1.49063726593165	-5820850128.07257\\
1.49073726843171	-5828642354.08635\\
1.49083727093177	-5836491875.87964\\
1.49093727343184	-5844341397.67294\\
1.4910372759319	-5852190919.46623\\
1.49113727843196	-5859983145.48001\\
1.49123728093202	-5867889963.05281\\
1.49133728343209	-5875739484.84611\\
1.49143728593215	-5883589006.6394\\
1.49153728843221	-5891438528.43269\\
1.49163729093227	-5899345346.0055\\
1.49173729343234	-5907194867.79879\\
1.4918372959324	-5915101685.37159\\
1.49193729843246	-5922951207.16489\\
1.49203730093252	-5930858024.73769\\
1.49213730343259	-5938764842.3105\\
1.49223730593265	-5946671659.8833\\
1.49233730843271	-5954578477.45611\\
1.49243731093277	-5962485295.02891\\
1.49253731343284	-5970449408.38123\\
1.4926373159329	-5978356225.95404\\
1.49273731843296	-5986320339.30635\\
1.49283732093302	-5994227156.87916\\
1.49293732343309	-6002191270.23148\\
1.49303732593315	-6010098087.80428\\
1.49313732843321	-6018062201.1566\\
1.49323733093327	-6026026314.50892\\
1.49333733343334	-6033990427.86124\\
1.4934373359334	-6041954541.21356\\
1.49353733843346	-6049975950.34539\\
1.49363734093352	-6057940063.69771\\
1.49373734343359	-6065904177.05003\\
1.49383734593365	-6073925586.18186\\
1.49393734843371	-6081889699.53418\\
1.49403735093377	-6089911108.66601\\
1.49413735343384	-6097932517.79784\\
1.4942373559339	-6105953926.92967\\
1.49433735843396	-6113975336.0615\\
1.49443736093402	-6121996745.19333\\
1.49453736343409	-6130018154.32516\\
1.49463736593415	-6138039563.457\\
1.49473736843421	-6146060972.58883\\
1.49483737093427	-6154139677.50017\\
1.49493737343434	-6162161086.632\\
1.4950373759344	-6170239791.54335\\
1.49513737843446	-6178318496.45469\\
1.49523738093452	-6186339905.58652\\
1.49533738343459	-6194418610.49787\\
1.49543738593465	-6202497315.40921\\
1.49553738843471	-6210576020.32056\\
1.49563739093477	-6218712021.01142\\
1.49573739343484	-6226790725.92276\\
1.4958373959349	-6234869430.83411\\
1.49593739843496	-6243005431.52496\\
1.49603740093502	-6251084136.43631\\
1.49613740343509	-6259220137.12716\\
1.49623740593515	-6267356137.81802\\
1.49633740843521	-6275434842.72937\\
1.49643741093527	-6283570843.42023\\
1.49653741343534	-6291706844.11108\\
1.4966374159354	-6299842844.80194\\
1.49673741843546	-6308036141.27231\\
1.49683742093552	-6316172141.96317\\
1.49693742343559	-6324308142.65403\\
1.49703742593565	-6332501439.1244\\
1.49713742843571	-6340637439.81526\\
1.49723743093577	-6348830736.28563\\
1.49733743343584	-6357024032.756\\
1.4974374359359	-6365217329.22637\\
1.49753743843596	-6373410625.69674\\
1.49763744093602	-6381603922.16711\\
1.49773744343609	-6389797218.63748\\
1.49783744593615	-6397990515.10785\\
1.49793744843621	-6406183811.57822\\
1.49803745093627	-6414434403.8281\\
1.49813745343634	-6422627700.29848\\
1.4982374559364	-6430878292.54836\\
1.49833745843646	-6439128884.79824\\
1.49843746093652	-6447379477.04813\\
1.49853746343659	-6455572773.5185\\
1.49863746593665	-6463823365.76838\\
1.49873746843671	-6472131253.79778\\
1.49883747093677	-6480381846.04766\\
1.49893747343684	-6488632438.29755\\
1.4990374759369	-6496883030.54743\\
1.49913747843696	-6505190918.57683\\
1.49923748093702	-6513441510.82671\\
1.49933748343709	-6521749398.85611\\
1.49943748593715	-6530057286.88551\\
1.49953748843721	-6538365174.9149\\
1.49963749093727	-6546673062.9443\\
1.49973749343734	-6554980950.9737\\
1.4998374959374	-6563288839.00309\\
1.49993749843746	-6571596727.03249\\
1.50003750093752	-6579904615.06189\\
1.50013750343759	-6588269798.8708\\
1.50023750593765	-6596577686.90019\\
1.50033750843771	-6604942870.7091\\
1.50043751093777	-6613250758.7385\\
1.50053751343784	-6621615942.54741\\
1.5006375159379	-6629981126.35632\\
1.50073751843796	-6638346310.16523\\
1.50083752093802	-6646711493.97414\\
1.50093752343809	-6655076677.78305\\
1.50103752593815	-6663499157.37147\\
1.50113752843821	-6671864341.18038\\
1.50123753093827	-6680229524.98929\\
1.50133753343834	-6688652004.57772\\
1.5014375359384	-6697074484.16614\\
1.50153753843846	-6705439667.97505\\
1.50163754093852	-6713862147.56347\\
1.50173754343859	-6722284627.1519\\
1.50183754593865	-6730707106.74032\\
1.50193754843871	-6739129586.32874\\
1.50203755093877	-6747552065.91717\\
1.50213755343884	-6756031841.2851\\
1.5022375559389	-6764454320.87353\\
1.50233755843896	-6772934096.24146\\
1.50243756093902	-6781356575.82988\\
1.50253756343909	-6789836351.19782\\
1.50263756593915	-6798316126.56576\\
1.50273756843921	-6806738606.15418\\
1.50283757093927	-6815218381.52212\\
1.50293757343934	-6823698156.89005\\
1.5030375759394	-6832235228.0375\\
1.50313757843946	-6840715003.40544\\
1.50323758093952	-6849194778.77337\\
1.50333758343959	-6857674554.14131\\
1.50343758593965	-6866211625.28876\\
1.50353758843971	-6874748696.43621\\
1.50363759093977	-6883228471.80414\\
1.50373759343984	-6891765542.95159\\
1.5038375959399	-6900302614.09904\\
1.50393759843996	-6908839685.24649\\
1.50403760094002	-6917376756.39394\\
1.50413760344009	-6925913827.54139\\
1.50423760594015	-6934508194.46835\\
1.50433760844021	-6943045265.6158\\
1.50443761094027	-6951582336.76325\\
1.50453761344034	-6960176703.69021\\
1.5046376159404	-6968771070.61718\\
1.50473761844046	-6977308141.76463\\
1.50483762094052	-6985902508.69159\\
1.50493762344059	-6994496875.61855\\
1.50503762594065	-7003091242.54551\\
1.50513762844071	-7011685609.47248\\
1.50523763094077	-7020279976.39944\\
1.50533763344084	-7028931639.10591\\
1.5054376359409	-7037526006.03288\\
1.50553763844096	-7046177668.73935\\
1.50563764094102	-7054772035.66631\\
1.50573764344109	-7063423698.37279\\
1.50583764594115	-7072075361.07926\\
1.50593764844121	-7080727023.78574\\
1.50603765094127	-7089378686.49222\\
1.50613765344134	-7098030349.19869\\
1.5062376559414	-7106682011.90517\\
1.50633765844146	-7115333674.61164\\
1.50643766094152	-7123985337.31812\\
1.50653766344159	-7132694295.80411\\
1.50663766594165	-7141345958.51058\\
1.50673766844171	-7150054916.99657\\
1.50683767094177	-7158763875.48256\\
1.50693767344184	-7167472833.96855\\
1.5070376759419	-7176124496.67502\\
1.50713767844196	-7184833455.16101\\
1.50723768094202	-7193599709.42651\\
1.50733768344209	-7202308667.9125\\
1.50743768594215	-7211017626.39849\\
1.50753768844221	-7219726584.88448\\
1.50763769094227	-7228492839.14998\\
1.50773769344234	-7237259093.41548\\
1.5078376959424	-7245968051.90147\\
1.50793769844246	-7254734306.16697\\
1.50803770094252	-7263500560.43247\\
1.50813770344259	-7272266814.69797\\
1.50823770594265	-7281033068.96348\\
1.50833770844271	-7289799323.22898\\
1.50843771094277	-7298565577.49448\\
1.50853771344284	-7307389127.53949\\
1.5086377159429	-7316155381.80499\\
1.50873771844296	-7324978931.85001\\
1.50883772094302	-7333745186.11551\\
1.50893772344309	-7342568736.16053\\
1.50903772594315	-7351392286.20554\\
1.50913772844321	-7360215836.25056\\
1.50923773094327	-7369039386.29557\\
1.50933773344334	-7377862936.34058\\
1.5094377359434	-7386686486.3856\\
1.50953773844346	-7395510036.43061\\
1.50963774094352	-7404333586.47563\\
1.50973774344359	-7413214432.30016\\
1.50983774594365	-7422095278.12468\\
1.50993774844371	-7430918828.1697\\
1.51003775094377	-7439799673.99423\\
1.51013775344384	-7448680519.81875\\
1.5102377559439	-7457561365.64328\\
1.51033775844396	-7466442211.46781\\
1.51043776094402	-7475323057.29234\\
1.51053776344409	-7484203903.11687\\
1.51063776594415	-7493084748.94139\\
1.51073776844421	-7502022890.54543\\
1.51083777094427	-7510903736.36996\\
1.51093777344434	-7519841877.974\\
1.5110377759444	-7528780019.57804\\
1.51113777844446	-7537660865.40257\\
1.51123778094452	-7546599007.00661\\
1.51133778344459	-7555537148.61065\\
1.51143778594465	-7564475290.21469\\
1.51153778844471	-7573470727.59825\\
1.51163779094477	-7582408869.20229\\
1.51173779344484	-7591347010.80633\\
1.5118377959449	-7600342448.18988\\
1.51193779844496	-7609280589.79392\\
1.51203780094502	-7618276027.17748\\
1.51213780344509	-7627271464.56103\\
1.51223780594515	-7636266901.94459\\
1.51233780844521	-7645205043.54863\\
1.51243781094527	-7654200480.93218\\
1.51253781344534	-7663253214.09525\\
1.5126378159454	-7672248651.4788\\
1.51273781844546	-7681244088.86236\\
1.51283782094552	-7690296822.02542\\
1.51293782344559	-7699292259.40898\\
1.51303782594565	-7708344992.57204\\
1.51313782844571	-7717340429.9556\\
1.51323783094577	-7726393163.11866\\
1.51333783344584	-7735445896.28173\\
1.5134378359459	-7744498629.4448\\
1.51353783844596	-7753551362.60787\\
1.51363784094602	-7762604095.77093\\
1.51373784344609	-7771656828.934\\
1.51383784594615	-7780766857.87658\\
1.51393784844621	-7789819591.03965\\
1.51403785094627	-7798929619.98223\\
1.51413785344634	-7807982353.14529\\
1.5142378559464	-7817092382.08787\\
1.51433785844646	-7826202411.03045\\
1.51443786094652	-7835312439.97303\\
1.51453786344659	-7844422468.91561\\
1.51463786594665	-7853532497.85819\\
1.51473786844671	-7862642526.80077\\
1.51483787094677	-7871809851.52287\\
1.51493787344684	-7880919880.46545\\
1.5150378759469	-7890029909.40803\\
1.51513787844696	-7899197234.13012\\
1.51523788094702	-7908364558.85221\\
1.51533788344709	-7917531883.57431\\
1.51543788594715	-7926641912.51689\\
1.51553788844721	-7935809237.23898\\
1.51563789094727	-7944976561.96107\\
1.51573789344734	-7954201182.46268\\
1.5158378959474	-7963368507.18477\\
1.51593789844746	-7972535831.90687\\
1.51603790094752	-7981760452.40847\\
1.51613790344759	-7990927777.13057\\
1.51623790594765	-8000152397.63217\\
1.51633790844771	-8009377018.13378\\
1.51643791094777	-8018544342.85587\\
1.51653791344784	-8027768963.35748\\
1.5166379159479	-8036993583.85908\\
1.51673791844796	-8046218204.36069\\
1.51683792094802	-8055500120.64181\\
1.51693792344809	-8064724741.14342\\
1.51703792594815	-8073949361.64502\\
1.51713792844821	-8083231277.92614\\
1.51723793094827	-8092455898.42775\\
1.51733793344834	-8101737814.70887\\
1.5174379359484	-8111019730.98999\\
1.51753793844846	-8120301647.27111\\
1.51763794094852	-8129583563.55223\\
1.51773794344859	-8138865479.83334\\
1.51783794594865	-8148147396.11446\\
1.51793794844871	-8157429312.39558\\
1.51803795094877	-8166768524.45621\\
1.51813795344884	-8176050440.73733\\
1.5182379559489	-8185389652.79797\\
1.51833795844896	-8194671569.07909\\
1.51843796094902	-8204010781.13972\\
1.51853796344909	-8213349993.20035\\
1.51863796594915	-8222689205.26098\\
1.51873796844921	-8232028417.32162\\
1.51883797094927	-8241367629.38225\\
1.51893797344934	-8250706841.44288\\
1.5190379759494	-8260046053.50351\\
1.51913797844946	-8269442561.34366\\
1.51923798094952	-8278781773.40429\\
1.51933798344959	-8288178281.24444\\
1.51943798594965	-8297517493.30507\\
1.51953798844971	-8306914001.14521\\
1.51963799094977	-8316310508.98536\\
1.51973799344984	-8325707016.82551\\
1.5198379959499	-8335103524.66565\\
1.51993799844996	-8344500032.5058\\
1.52003800095002	-8353896540.34594\\
1.52013800345009	-8363350343.9656\\
1.52023800595015	-8372746851.80575\\
1.52033800845021	-8382200655.4254\\
1.52043801095027	-8391597163.26555\\
1.52053801345034	-8401050966.88521\\
1.5206380159504	-8410504770.50487\\
1.52073801845046	-8419958574.12453\\
1.52083802095052	-8429412377.74418\\
1.52093802345059	-8438866181.36384\\
1.52103802595065	-8448319984.9835\\
1.52113802845071	-8457773788.60316\\
1.52123803095077	-8467284888.00233\\
1.52133803345084	-8476738691.62199\\
1.5214380359509	-8486249791.02116\\
1.52153803845096	-8495760890.42033\\
1.52163804095102	-8505214694.03999\\
1.52173804345109	-8514725793.43916\\
1.52183804595115	-8524236892.83834\\
1.52193804845121	-8533747992.23751\\
1.52203805095127	-8543259091.63668\\
1.52213805345134	-8552827486.81536\\
1.5222380559514	-8562338586.21454\\
1.52233805845146	-8571906981.39322\\
1.52243806095152	-8581418080.79239\\
1.52253806345159	-8590986475.97108\\
1.52263806595165	-8600497575.37025\\
1.52273806845171	-8610065970.54893\\
1.52283807095177	-8619634365.72762\\
1.52293807345184	-8629202760.9063\\
1.5230380759519	-8638771156.08499\\
1.52313807845196	-8648339551.26367\\
1.52323808095202	-8657965242.22187\\
1.52333808345209	-8667533637.40055\\
1.52343808595215	-8677159328.35875\\
1.52353808845221	-8686727723.53744\\
1.52363809095227	-8696353414.49563\\
1.52373809345234	-8705979105.45383\\
1.5238380959524	-8715604796.41203\\
1.52393809845246	-8725230487.37023\\
1.52403810095252	-8734856178.32843\\
1.52413810345259	-8744481869.28662\\
1.52423810595265	-8754107560.24482\\
1.52433810845271	-8763733251.20302\\
1.52443811095277	-8773416237.94073\\
1.52453811345284	-8783041928.89893\\
1.5246381159529	-8792724915.63664\\
1.52473811845296	-8802407902.37435\\
1.52483812095302	-8812090889.11206\\
1.52493812345309	-8821773875.84977\\
1.52503812595315	-8831456862.58748\\
1.52513812845321	-8841139849.32519\\
1.52523813095327	-8850822836.0629\\
1.52533813345334	-8860505822.80062\\
1.5254381359534	-8870246105.31784\\
1.52553813845346	-8879929092.05555\\
1.52563814095352	-8889669374.57277\\
1.52573814345359	-8899352361.31049\\
1.52583814595365	-8909092643.82771\\
1.52593814845371	-8918832926.34493\\
1.52603815095377	-8928573208.86216\\
1.52613815345384	-8938313491.37938\\
1.5262381559539	-8948053773.89661\\
1.52633815845396	-8957794056.41383\\
1.52643816095402	-8967591634.71057\\
1.52653816345409	-8977331917.22779\\
1.52663816595415	-8987129495.52453\\
1.52673816845421	-8996869778.04175\\
1.52683817095427	-9006667356.33849\\
1.52693817345434	-9016464934.63523\\
1.5270381759544	-9026262512.93196\\
1.52713817845446	-9036060091.2287\\
1.52723818095452	-9045857669.52544\\
1.52733818345459	-9055655247.82217\\
1.52743818595465	-9065510121.89842\\
1.52753818845471	-9075307700.19516\\
1.52763819095477	-9085162574.27141\\
1.52773819345484	-9094960152.56815\\
1.5278381959549	-9104815026.6444\\
1.52793819845496	-9114669900.72065\\
1.52803820095502	-9124524774.7969\\
1.52813820345509	-9134379648.87315\\
1.52823820595515	-9144234522.9494\\
1.52833820845521	-9154089397.02565\\
1.52843821095527	-9163944271.1019\\
1.52853821345534	-9173799145.17815\\
1.5286382159554	-9183711315.03391\\
1.52873821845546	-9193566189.11016\\
1.52883822095552	-9203478358.96593\\
1.52893822345559	-9213390528.82169\\
1.52903822595565	-9223302698.67745\\
1.52913822845571	-9233214868.53322\\
1.52923823095577	-9243127038.38898\\
1.52933823345584	-9253039208.24474\\
1.5294382359559	-9262951378.10051\\
1.52953823845596	-9272863547.95627\\
1.52963824095602	-9282833013.59155\\
1.52973824345609	-9292745183.44731\\
1.52983824595615	-9302714649.08258\\
1.52993824845621	-9312684114.71786\\
1.53003825095627	-9322596284.57363\\
1.53013825345634	-9332565750.2089\\
1.5302382559564	-9342535215.84418\\
1.53033825845646	-9352504681.47945\\
1.53043826095652	-9362531442.89424\\
1.53053826345659	-9372500908.52952\\
1.53063826595665	-9382470374.16479\\
1.53073826845671	-9392497135.57959\\
1.53083827095677	-9402466601.21486\\
1.53093827345684	-9412493362.62965\\
1.5310382759569	-9422520124.04444\\
1.53113827845696	-9432546885.45923\\
1.53123828095702	-9442573646.87402\\
1.53133828345709	-9452600408.28881\\
1.53143828595715	-9462627169.7036\\
1.53153828845721	-9472653931.11839\\
1.53163829095727	-9482680692.53318\\
1.53173829345734	-9492764749.72748\\
1.5318382959574	-9502791511.14227\\
1.53193829845746	-9512875568.33657\\
1.53203830095752	-9522959625.53087\\
1.53213830345759	-9532986386.94566\\
1.53223830595765	-9543070444.13997\\
1.53233830845771	-9553154501.33427\\
1.53243831095777	-9563295854.30808\\
1.53253831345784	-9573379911.50239\\
1.5326383159579	-9583463968.69669\\
1.53273831845796	-9593548025.89099\\
1.53283832095802	-9603689378.86481\\
1.53293832345809	-9613773436.05911\\
1.53303832595815	-9623914789.03292\\
1.53313832845821	-9634056142.00674\\
1.53323833095827	-9644197494.98056\\
1.53333833345834	-9654338847.95437\\
1.5334383359584	-9664480200.92819\\
1.53353833845846	-9674621553.902\\
1.53363834095852	-9684762906.87582\\
1.53373834345859	-9694961555.62915\\
1.53383834595865	-9705102908.60296\\
1.53393834845871	-9715301557.35629\\
1.53403835095877	-9725442910.33011\\
1.53413835345884	-9735641559.08344\\
1.5342383559589	-9745840207.83676\\
1.53433835845896	-9756038856.59009\\
1.53443836095902	-9766237505.34342\\
1.53453836345909	-9776436154.09675\\
1.53463836595915	-9786634802.85008\\
1.53473836845921	-9796833451.60341\\
1.53483837095927	-9807089396.13625\\
1.53493837345934	-9817288044.88958\\
1.5350383759594	-9827543989.42242\\
1.53513837845946	-9837799933.95526\\
1.53523838095952	-9847998582.70859\\
1.53533838345959	-9858254527.24143\\
1.53543838595965	-9868510471.77427\\
1.53553838845971	-9878766416.30712\\
1.53563839095977	-9889079656.61947\\
1.53573839345984	-9899335601.15231\\
1.5358383959599	-9909591545.68515\\
1.53593839845996	-9919904785.99751\\
1.53603840096002	-9930160730.53035\\
1.53613840346009	-9940473970.8427\\
1.53623840596015	-9950787211.15506\\
1.53633840846021	-9961100451.46741\\
1.53643841096027	-9971413691.77977\\
1.53653841346034	-9981726932.09212\\
1.5366384159604	-9992040172.40448\\
1.53673841846046	-10002353412.7168\\
1.53683842096052	-10012666653.0292\\
1.53693842346059	-10023037189.1211\\
1.53703842596065	-10033350429.4334\\
1.53713842846071	-10043720965.5253\\
1.53723843096077	-10054091501.6171\\
1.53733843346084	-10064404741.9295\\
1.5374384359609	-10074775278.0214\\
1.53753843846096	-10085145814.1132\\
1.53763844096102	-10095516350.2051\\
1.53773844346109	-10105944182.0765\\
1.53783844596115	-10116314718.1684\\
1.53793844846121	-10126685254.2602\\
1.53803845096127	-10137113086.1316\\
1.53813845346134	-10147483622.2235\\
1.5382384559614	-10157911454.0949\\
1.53833845846146	-10168339285.9662\\
1.53843846096152	-10178767117.8376\\
1.53853846346159	-10189194949.709\\
1.53863846596165	-10199622781.5804\\
1.53873846846171	-10210050613.4518\\
1.53883847096177	-10220478445.3231\\
1.53893847346184	-10230963572.974\\
1.5390384759619	-10241391404.8454\\
1.53913847846196	-10251876532.4963\\
1.53923848096202	-10262304364.3677\\
1.53933848346209	-10272789492.0186\\
1.53943848596215	-10283274619.6695\\
1.53953848846221	-10293759747.3204\\
1.53963849096227	-10304244874.9713\\
1.53973849346234	-10314730002.6222\\
1.5398384959624	-10325215130.2731\\
1.53993849846246	-10335700257.9239\\
1.54003850096252	-10346242681.3544\\
1.54013850346259	-10356727809.0052\\
1.54023850596265	-10367270232.4357\\
1.54033850846271	-10377812655.8661\\
1.54043851096277	-10388297783.517\\
1.54053851346284	-10398840206.9474\\
1.5406385159629	-10409382630.3778\\
1.54073851846296	-10419925053.8082\\
1.54083852096302	-10430524773.0181\\
1.54093852346309	-10441067196.4485\\
1.54103852596315	-10451609619.8789\\
1.54113852846321	-10462209339.0888\\
1.54123853096327	-10472751762.5192\\
1.54133853346334	-10483351481.7292\\
1.5414385359634	-10493951200.9391\\
1.54153853846346	-10504550920.149\\
1.54163854096352	-10515150639.3589\\
1.54173854346359	-10525750358.5688\\
1.54183854596365	-10536350077.7788\\
1.54193854846371	-10546949796.9887\\
1.54203855096377	-10557549516.1986\\
1.54213855346384	-10568206531.188\\
1.5422385559639	-10578806250.398\\
1.54233855846396	-10589463265.3874\\
1.54243856096402	-10600120280.3768\\
1.54253856346409	-10610777295.3663\\
1.54263856596415	-10621434310.3557\\
1.54273856846421	-10632091325.3451\\
1.54283857096427	-10642748340.3346\\
1.54293857346434	-10653405355.324\\
1.5430385759644	-10664062370.3134\\
1.54313857846446	-10674776681.0824\\
1.54323858096452	-10685433696.0718\\
1.54333858346459	-10696148006.8407\\
1.54343858596465	-10706805021.8302\\
1.54353858846471	-10717519332.5991\\
1.54363859096477	-10728233643.3681\\
1.54373859346484	-10738947954.137\\
1.5438385959649	-10749662264.906\\
1.54393859846496	-10760376575.6749\\
1.54403860096502	-10771148182.2234\\
1.54413860346509	-10781862492.9923\\
1.54423860596515	-10792634099.5408\\
1.54433860846521	-10803348410.3097\\
1.54443861096527	-10814120016.8582\\
1.54453861346534	-10824891623.4066\\
1.5446386159654	-10835605934.1756\\
1.54473861846546	-10846377540.724\\
1.54483862096552	-10857149147.2725\\
1.54493862346559	-10867978049.6005\\
1.54503862596565	-10878749656.1489\\
1.54513862846571	-10889521262.6974\\
1.54523863096577	-10900350165.0254\\
1.54533863346584	-10911121771.5738\\
1.5454386359659	-10921950673.9018\\
1.54553863846596	-10932779576.2298\\
1.54563864096602	-10943551182.7782\\
1.54573864346609	-10954380085.1062\\
1.54583864596615	-10965208987.4342\\
1.54593864846621	-10976037889.7622\\
1.54603865096627	-10986924087.8696\\
1.54613865346634	-10997752990.1976\\
1.5462386559664	-11008581892.5256\\
1.54633865846646	-11019468090.6331\\
1.54643866096652	-11030296992.961\\
1.54653866346659	-11041183191.0685\\
1.54663866596665	-11052069389.176\\
1.54673866846671	-11062955587.2835\\
1.54683867096677	-11073841785.391\\
1.54693867346684	-11084727983.4985\\
1.5470386759669	-11095614181.606\\
1.54713867846696	-11106500379.7134\\
1.54723868096702	-11117443873.6004\\
1.54733868346709	-11128330071.7079\\
1.54743868596715	-11139273565.5949\\
1.54753868846721	-11150159763.7024\\
1.54763869096727	-11161103257.5894\\
1.54773869346734	-11172046751.4764\\
1.5478386959674	-11182990245.3634\\
1.54793869846746	-11193933739.2504\\
1.54803870096752	-11204877233.1374\\
1.54813870346759	-11215878022.8039\\
1.54823870596765	-11226821516.6909\\
1.54833870846771	-11237765010.5779\\
1.54843871096777	-11248765800.2444\\
1.54853871346784	-11259766589.9109\\
1.5486387159679	-11270710083.7979\\
1.54873871846796	-11281710873.4644\\
1.54883872096802	-11292711663.131\\
1.54893872346809	-11303712452.7975\\
1.54903872596815	-11314713242.464\\
1.54913872846821	-11325714032.1305\\
1.54923873096827	-11336772117.5765\\
1.54933873346834	-11347772907.243\\
1.5494387359684	-11358830992.6891\\
1.54953873846846	-11369831782.3556\\
1.54963874096852	-11380889867.8016\\
1.54973874346859	-11391947953.2476\\
1.54983874596865	-11403006038.6936\\
1.54993874846871	-11414064124.1397\\
1.55003875096877	-11425122209.5857\\
1.55013875346884	-11436180295.0317\\
1.5502387559689	-11447238380.4777\\
1.55033875846896	-11458353761.7033\\
1.55043876096902	-11469411847.1493\\
1.55053876346909	-11480527228.3748\\
1.55063876596915	-11491585313.8209\\
1.55073876846921	-11502700695.0464\\
1.55083877096927	-11513816076.2719\\
1.55093877346934	-11524931457.4975\\
1.5510387759694	-11536046838.723\\
1.55113877846946	-11547162219.9486\\
1.55123878096952	-11558277601.1741\\
1.55133878346959	-11569450278.1791\\
1.55143878596965	-11580565659.4047\\
1.55153878846971	-11591738336.4097\\
1.55163879096977	-11602853717.6353\\
1.55173879346984	-11614026394.6403\\
1.5518387959699	-11625199071.6454\\
1.55193879846996	-11636371748.6504\\
1.55203880097002	-11647544425.6555\\
1.55213880347009	-11658717102.6605\\
1.55223880597015	-11669889779.6656\\
1.55233880847021	-11681119752.4501\\
1.55243881097027	-11692292429.4552\\
1.55253881347034	-11703522402.2398\\
1.5526388159704	-11714695079.2448\\
1.55273881847046	-11725925052.0294\\
1.55283882097052	-11737155024.8139\\
1.55293882347059	-11748384997.5985\\
1.55303882597065	-11759614970.3831\\
1.55313882847071	-11770844943.1676\\
1.55323883097077	-11782074915.9522\\
1.55333883347084	-11793304888.7368\\
1.5534388359709	-11804592157.3008\\
1.55353883847096	-11815822130.0854\\
1.55363884097102	-11827109398.6495\\
1.55373884347109	-11838396667.2136\\
1.55383884597115	-11849626639.9981\\
1.55393884847121	-11860913908.5622\\
1.55403885097127	-11872201177.1263\\
1.55413885347134	-11883488445.6904\\
1.5542388559714	-11894775714.2544\\
1.55433885847146	-11906120278.598\\
1.55443886097152	-11917407547.1621\\
1.55453886347159	-11928752111.5057\\
1.55463886597165	-11940039380.0698\\
1.55473886847171	-11951383944.4134\\
1.55483887097177	-11962728508.7569\\
1.55493887347184	-11974015777.321\\
1.5550388759719	-11985360341.6646\\
1.55513887847196	-11996704906.0082\\
1.55523888097202	-12008106766.1313\\
1.55533888347209	-12019451330.4749\\
1.55543888597215	-12030795894.8185\\
1.55553888847221	-12042197754.9416\\
1.55563889097227	-12053542319.2852\\
1.55573889347234	-12064944179.4083\\
1.5558388959724	-12076288743.7519\\
1.55593889847246	-12087690603.875\\
1.55603890097252	-12099092463.9981\\
1.55613890347259	-12110494324.1212\\
1.55623890597265	-12121896184.2443\\
1.55633890847271	-12133355340.1469\\
1.55643891097277	-12144757200.27\\
1.55653891347284	-12156159060.3931\\
1.5566389159729	-12167618216.2957\\
1.55673891847296	-12179020076.4188\\
1.55683892097302	-12190479232.3214\\
1.55693892347309	-12201938388.2241\\
1.55703892597315	-12213397544.1267\\
1.55713892847321	-12224856700.0293\\
1.55723893097327	-12236315855.9319\\
1.55733893347334	-12247775011.8345\\
1.5574389359734	-12259291463.5167\\
1.55753893847346	-12270750619.4193\\
1.55763894097352	-12282209775.3219\\
1.55773894347359	-12293726227.004\\
1.55783894597365	-12305242678.6862\\
1.55793894847371	-12316701834.5888\\
1.55803895097377	-12328218286.2709\\
1.55813895347384	-12339734737.953\\
1.5582389559739	-12351251189.6352\\
1.55833895847396	-12362824937.0968\\
1.55843896097402	-12374341388.7789\\
1.55853896347409	-12385857840.4611\\
1.55863896597415	-12397431587.9227\\
1.55873896847421	-12408948039.6048\\
1.55883897097427	-12420521787.0665\\
1.55893897347434	-12432095534.5281\\
1.5590389759744	-12443669281.9898\\
1.55913897847446	-12455185733.6719\\
1.55923898097452	-12466816776.913\\
1.55933898347459	-12478390524.3747\\
1.55943898597465	-12489964271.8363\\
1.55953898847471	-12501538019.298\\
1.55963899097477	-12513169062.5391\\
1.55973899347484	-12524742810.0008\\
1.5598389959749	-12536373853.2419\\
1.55993899847496	-12548004896.4831\\
1.56003900097502	-12559578643.9447\\
1.56013900347509	-12571209687.1859\\
1.56023900597515	-12582840730.427\\
1.56033900847521	-12594529069.4477\\
1.56043901097527	-12606160112.6889\\
1.56053901347534	-12617791155.93\\
1.5606390159754	-12629479494.9507\\
1.56073901847546	-12641110538.1918\\
1.56083902097552	-12652798877.2125\\
1.56093902347559	-12664429920.4537\\
1.56103902597565	-12676118259.4743\\
1.56113902847571	-12687806598.495\\
1.56123903097577	-12699494937.5157\\
1.56133903347584	-12711183276.5363\\
1.5614390359759	-12722871615.557\\
1.56153903847596	-12734617250.3572\\
1.56163904097602	-12746305589.3779\\
1.56173904347609	-12758051224.178\\
1.56183904597615	-12769739563.1987\\
1.56193904847621	-12781485197.9989\\
1.56203905097627	-12793230832.7991\\
1.56213905347634	-12804976467.5993\\
1.5622390559764	-12816722102.3994\\
1.56233905847646	-12828467737.1996\\
1.56243906097652	-12840213371.9998\\
1.56253906347659	-12851959006.8\\
1.56263906597665	-12863704641.6002\\
1.56273906847671	-12875507572.1799\\
1.56283907097677	-12887310502.7596\\
1.56293907347684	-12899056137.5597\\
1.5630390759769	-12910859068.1394\\
1.56313907847696	-12922661998.7191\\
1.56323908097702	-12934464929.2988\\
1.56333908347709	-12946267859.8785\\
1.56343908597715	-12958070790.4582\\
1.56353908847721	-12969873721.0379\\
1.56363909097727	-12981733947.3971\\
1.56373909347734	-12993536877.9768\\
1.5638390959774	-13005397104.336\\
1.56393909847746	-13017200034.9157\\
1.56403910097752	-13029060261.2749\\
1.56413910347759	-13040920487.6341\\
1.56423910597765	-13052780713.9933\\
1.56433910847771	-13064640940.3525\\
1.56443911097777	-13076501166.7118\\
1.56453911347784	-13088361393.071\\
1.5646391159779	-13100278915.2097\\
1.56473911847796	-13112139141.5689\\
1.56483912097802	-13124056663.7076\\
1.56493912347809	-13135916890.0668\\
1.56503912597815	-13147834412.2055\\
1.56513912847821	-13159751934.3443\\
1.56523913097827	-13171669456.483\\
1.56533913347834	-13183586978.6217\\
1.5654391359784	-13195504500.7604\\
1.56553913847846	-13207422022.8991\\
1.56563914097852	-13219339545.0379\\
1.56573914347859	-13231314362.9561\\
1.56583914597865	-13243231885.0948\\
1.56593914847871	-13255206703.0131\\
1.56603915097877	-13267181520.9313\\
1.56613915347884	-13279099043.07\\
1.5662391559789	-13291073860.9882\\
1.56633915847896	-13303048678.9065\\
1.56643916097902	-13315023496.8247\\
1.56653916347909	-13327055610.5225\\
1.56663916597915	-13339030428.4407\\
1.56673916847921	-13351005246.3589\\
1.56683917097927	-13363037360.0567\\
1.56693917347934	-13375012177.9749\\
1.5670391759794	-13387044291.6727\\
1.56713917847946	-13399076405.3704\\
1.56723918097952	-13411108519.0682\\
1.56733918347959	-13423140632.7659\\
1.56743918597965	-13435172746.4636\\
1.56753918847971	-13447204860.1614\\
1.56763919097977	-13459236973.8591\\
1.56773919347984	-13471326383.3364\\
1.5678391959799	-13483358497.0341\\
1.56793919847996	-13495447906.5114\\
1.56803920098002	-13507480020.2092\\
1.56813920348009	-13519569429.6864\\
1.56823920598015	-13531658839.1637\\
1.56833920848021	-13543748248.6409\\
1.56843921098027	-13555837658.1182\\
1.56853921348034	-13567927067.5955\\
1.5686392159804	-13580073772.8522\\
1.56873921848046	-13592163182.3295\\
1.56883922098052	-13604252591.8068\\
1.56893922348059	-13616399297.0635\\
1.56903922598065	-13628546002.3203\\
1.56913922848071	-13640635411.7976\\
1.56923923098077	-13652782117.0543\\
1.56933923348084	-13664928822.3111\\
1.5694392359809	-13677075527.5679\\
1.56953923848096	-13689222232.8247\\
1.56963924098102	-13701426233.8609\\
1.56973924348109	-13713572939.1177\\
1.56983924598115	-13725719644.3745\\
1.56993924848121	-13737923645.4108\\
1.57003925098127	-13750127646.4471\\
1.57013925348134	-13762274351.7038\\
1.5702392559814	-13774478352.7401\\
1.57033925848146	-13786682353.7764\\
1.57043926098152	-13798886354.8127\\
1.57053926348159	-13811090355.849\\
1.57063926598165	-13823294356.8853\\
1.57073926848171	-13835555653.7011\\
1.57083927098177	-13847759654.7374\\
1.57093927348184	-13860020951.5532\\
1.5710392759819	-13872224952.5894\\
1.57113927848196	-13884486249.4052\\
1.57123928098202	-13896747546.221\\
1.57133928348209	-13909008843.0368\\
1.57143928598215	-13921270139.8526\\
1.57153928848221	-13933531436.6684\\
1.57163929098227	-13945792733.4842\\
1.57173929348234	-13958111326.0795\\
1.5718392959824	-13970372622.8953\\
1.57193929848246	-13982633919.7111\\
1.57203930098252	-13994952512.3065\\
1.57213930348259	-14007271104.9018\\
1.57223930598265	-14019589697.4971\\
1.57233930848271	-14031850994.3129\\
1.57243931098277	-14044169586.9082\\
1.57253931348284	-14056545475.283\\
1.5726393159829	-14068864067.8783\\
1.57273931848296	-14081182660.4737\\
1.57283932098302	-14093501253.069\\
1.57293932348309	-14105877141.4438\\
1.57303932598315	-14118253029.8186\\
1.57313932848321	-14130571622.4139\\
1.57323933098327	-14142947510.7888\\
1.57333933348334	-14155323399.1636\\
1.5734393359834	-14167699287.5384\\
1.57353933848346	-14180075175.9132\\
1.57363934098352	-14192451064.2881\\
1.57373934348359	-14204884248.4424\\
1.57383934598365	-14217260136.8172\\
1.57393934848371	-14229636025.192\\
1.57403935098377	-14242069209.3464\\
1.57413935348384	-14254502393.5007\\
1.5742393559839	-14266878281.8756\\
1.57433935848396	-14279311466.0299\\
1.57443936098402	-14291744650.1842\\
1.57453936348409	-14304177834.3386\\
1.57463936598415	-14316668314.2724\\
1.57473936848421	-14329101498.4268\\
1.57483937098427	-14341534682.5811\\
1.57493937348434	-14354025162.5149\\
1.5750393759844	-14366458346.6693\\
1.57513937848446	-14378948826.6031\\
1.57523938098452	-14391439306.537\\
1.57533938348459	-14403929786.4708\\
1.57543938598465	-14416362970.6252\\
1.57553938848471	-14428910746.3385\\
1.57563939098477	-14441401226.2724\\
1.57573939348484	-14453891706.2063\\
1.5758393959849	-14466382186.1401\\
1.57593939848496	-14478929961.8535\\
1.57603940098502	-14491420441.7873\\
1.57613940348509	-14503968217.5007\\
1.57623940598515	-14516515993.2141\\
1.57633940848521	-14529063768.9274\\
1.57643941098527	-14541611544.6408\\
1.57653941348534	-14554159320.3541\\
1.5766394159854	-14566707096.0675\\
1.57673941848546	-14579254871.7809\\
1.57683942098552	-14591802647.4942\\
1.57693942348559	-14604407718.9871\\
1.57703942598565	-14616955494.7005\\
1.57713942848571	-14629560566.1934\\
1.57723943098577	-14642165637.6862\\
1.57733943348584	-14654770709.1791\\
1.5774394359859	-14667375780.672\\
1.57753943848596	-14679980852.1649\\
1.57763944098602	-14692585923.6578\\
1.57773944348609	-14705190995.1506\\
1.57783944598615	-14717796066.6435\\
1.57793944848621	-14730458433.9159\\
1.57803945098627	-14743063505.4088\\
1.57813945348634	-14755725872.6812\\
1.5782394559864	-14768388239.9536\\
1.57833945848646	-14780993311.4464\\
1.57843946098652	-14793655678.7188\\
1.57853946348659	-14806318045.9912\\
1.57863946598665	-14819037709.0431\\
1.57873946848671	-14831700076.3155\\
1.57883947098677	-14844362443.5879\\
1.57893947348684	-14857082106.6398\\
1.5790394759869	-14869744473.9122\\
1.57913947848696	-14882464136.9641\\
1.57923948098702	-14895126504.2365\\
1.57933948348709	-14907846167.2884\\
1.57943948598715	-14920565830.3403\\
1.57953948848721	-14933285493.3922\\
1.57963949098727	-14946005156.4441\\
1.57973949348734	-14958724819.496\\
1.5798394959874	-14971501778.3274\\
1.57993949848746	-14984221441.3793\\
1.58003950098752	-14996998400.2108\\
1.58013950348759	-15009718063.2627\\
1.58023950598765	-15022495022.0941\\
1.58033950848771	-15035271980.9255\\
1.58043951098777	-15048048939.7569\\
1.58053951348784	-15060825898.5883\\
1.5806395159879	-15073602857.4198\\
1.58073951848796	-15086379816.2512\\
1.58083952098802	-15099156775.0826\\
1.58093952348809	-15111991029.6935\\
1.58103952598815	-15124767988.5249\\
1.58113952848821	-15137602243.1359\\
1.58123953098827	-15150436497.7468\\
1.58133953348834	-15163213456.5782\\
1.5814395359884	-15176047711.1891\\
1.58153953848846	-15188881965.8001\\
1.58163954098852	-15201716220.411\\
1.58173954348859	-15214607770.8014\\
1.58183954598865	-15227442025.4124\\
1.58193954848871	-15240276280.0233\\
1.58203955098877	-15253167830.4137\\
1.58213955348884	-15266002085.0247\\
1.5822395559889	-15278893635.4151\\
1.58233955848896	-15291785185.8056\\
1.58243956098902	-15304676736.196\\
1.58253956348909	-15317568286.5865\\
1.58263956598915	-15330459836.9769\\
1.58273956848921	-15343351387.3673\\
1.58283957098927	-15356242937.7578\\
1.58293957348934	-15369191783.9277\\
1.5830395759894	-15382083334.3182\\
1.58313957848946	-15395032180.4881\\
1.58323958098952	-15407981026.6581\\
1.58333958348959	-15420872577.0485\\
1.58343958598965	-15433821423.2185\\
1.58353958848971	-15446770269.3885\\
1.58363959098977	-15459719115.5584\\
1.58373959348984	-15472725257.5079\\
1.5838395959899	-15485674103.6778\\
1.58393959848996	-15498622949.8478\\
1.58403960099002	-15511629091.7973\\
1.58413960349009	-15524577937.9672\\
1.58423960599015	-15537584079.9167\\
1.58433960849021	-15550590221.8662\\
1.58443961099027	-15563596363.8156\\
1.58453961349034	-15576602505.7651\\
1.5846396159904	-15589608647.7146\\
1.58473961849046	-15602614789.664\\
1.58483962099052	-15615620931.6135\\
1.58493962349059	-15628684369.3425\\
1.58503962599065	-15641690511.292\\
1.58513962849071	-15654753949.0209\\
1.58523963099077	-15667760090.9704\\
1.58533963349084	-15680823528.6994\\
1.5854396359909	-15693886966.4284\\
1.58553963849096	-15706950404.1574\\
1.58563964099102	-15720013841.8863\\
1.58573964349109	-15733077279.6153\\
1.58583964599115	-15746140717.3443\\
1.58593964849121	-15759261450.8528\\
1.58603965099127	-15772324888.5818\\
1.58613965349134	-15785445622.0903\\
1.5862396559914	-15798566355.5988\\
1.58633965849146	-15811629793.3278\\
1.58643966099152	-15824750526.8363\\
1.58653966349159	-15837871260.3448\\
1.58663966599165	-15850991993.8533\\
1.58673966849171	-15864112727.3617\\
1.58683967099177	-15877290756.6498\\
1.58693967349184	-15890411490.1583\\
1.5870396759919	-15903532223.6667\\
1.58713967849196	-15916710252.9548\\
1.58723968099202	-15929888282.2428\\
1.58733968349209	-15943009015.7513\\
1.58743968599215	-15956187045.0393\\
1.58753968849221	-15969365074.3273\\
1.58763969099227	-15982543103.6153\\
1.58773969349234	-15995778428.6828\\
1.5878396959924	-16008956457.9708\\
1.58793969849246	-16022134487.2588\\
1.58803970099252	-16035369812.3263\\
1.58813970349259	-16048547841.6144\\
1.58823970599265	-16061783166.6819\\
1.58833970849271	-16075018491.7494\\
1.58843971099277	-16088196521.0374\\
1.58853971349284	-16101431846.1049\\
1.5886397159929	-16114724466.952\\
1.58873971849296	-16127959792.0195\\
1.58883972099302	-16141195117.087\\
1.58893972349309	-16154430442.1545\\
1.58903972599315	-16167723063.0016\\
1.58913972849321	-16180958388.0691\\
1.58923973099327	-16194251008.9161\\
1.58933973349334	-16207543629.7632\\
1.5894397359934	-16220836250.6102\\
1.58953973849346	-16234128871.4572\\
1.58963974099352	-16247421492.3043\\
1.58973974349359	-16260714113.1513\\
1.58983974599365	-16274006733.9983\\
1.58993974849371	-16287299354.8454\\
1.59003975099377	-16300649271.4719\\
1.59013975349384	-16313941892.319\\
1.5902397559939	-16327291808.9455\\
1.59033975849396	-16340641725.5721\\
1.59043976099402	-16353991642.1986\\
1.59053976349409	-16367341558.8251\\
1.59063976599415	-16380691475.4517\\
1.59073976849421	-16394041392.0782\\
1.59083977099427	-16407391308.7048\\
1.59093977349434	-16420798521.1109\\
1.5910397759944	-16434148437.7374\\
1.59113977849446	-16447555650.1435\\
1.59123978099452	-16460905566.77\\
1.59133978349459	-16474312779.1761\\
1.59143978599465	-16487719991.5821\\
1.59153978849471	-16501127203.9882\\
1.59163979099477	-16514534416.3943\\
1.59173979349484	-16527941628.8003\\
1.5918397959949	-16541348841.2064\\
1.59193979849496	-16554813349.392\\
1.59203980099502	-16568220561.798\\
1.59213980349509	-16581685069.9836\\
1.59223980599515	-16595092282.3897\\
1.59233980849521	-16608556790.5752\\
1.59243981099527	-16622021298.7608\\
1.59253981349534	-16635485806.9464\\
1.5926398159954	-16648950315.1319\\
1.59273981849546	-16662414823.3175\\
1.59283982099552	-16675879331.5031\\
1.59293982349559	-16689401135.4682\\
1.59303982599565	-16702865643.6538\\
1.59313982849571	-16716387447.6188\\
1.59323983099577	-16729851955.8044\\
1.59333983349584	-16743373759.7695\\
1.5934398359959	-16756895563.7346\\
1.59353983849596	-16770417367.6997\\
1.59363984099602	-16783939171.6648\\
1.59373984349609	-16797460975.6299\\
1.59383984599615	-16810982779.5949\\
1.59393984849621	-16824561879.3395\\
1.59403985099627	-16838083683.3046\\
1.59413985349634	-16851662783.0492\\
1.5942398559964	-16865184587.0143\\
1.59433985849646	-16878763686.7589\\
1.59443986099652	-16892342786.5035\\
1.59453986349659	-16905921886.2481\\
1.59463986599665	-16919500985.9927\\
1.59473986849671	-16933080085.7373\\
1.59483987099677	-16946659185.4819\\
1.59493987349684	-16960295581.006\\
1.5950398759969	-16973874680.7506\\
1.59513987849696	-16987511076.2748\\
1.59523988099702	-17001090176.0194\\
1.59533988349709	-17014726571.5435\\
1.59543988599715	-17028362967.0676\\
1.59553988849721	-17041999362.5917\\
1.59563989099727	-17055635758.1158\\
1.59573989349734	-17069272153.6399\\
1.5958398959974	-17082965844.9435\\
1.59593989849746	-17096602240.4677\\
1.59603990099753	-17110238635.9918\\
1.59613990349759	-17123932327.2954\\
1.59623990599765	-17137626018.599\\
1.59633990849771	-17151262414.1231\\
1.59643991099777	-17164956105.4268\\
1.59653991349784	-17178649796.7304\\
1.5966399159979	-17192343488.034\\
1.59673991849796	-17206037179.3376\\
1.59683992099802	-17219788166.4208\\
1.59693992349809	-17233481857.7244\\
1.59703992599815	-17247175549.028\\
1.59713992849821	-17260926536.1112\\
1.59723993099827	-17274677523.1943\\
1.59733993349834	-17288371214.4979\\
1.5974399359984	-17302122201.5811\\
1.59753993849846	-17315873188.6642\\
1.59763994099852	-17329624175.7474\\
1.59773994349859	-17343432458.61\\
1.59783994599865	-17357183445.6932\\
1.59793994849871	-17370934432.7763\\
1.59803995099878	-17384742715.639\\
1.59813995349884	-17398493702.7221\\
1.5982399559989	-17412301985.5847\\
1.59833995849896	-17426110268.4474\\
1.59843996099902	-17439861255.5305\\
1.59853996349909	-17453669538.3932\\
1.59863996599915	-17467535117.0354\\
1.59873996849921	-17481343399.898\\
1.59883997099927	-17495151682.7607\\
1.59893997349934	-17508959965.6233\\
1.5990399759994	-17522825544.2655\\
1.59913997849946	-17536633827.1281\\
1.59923998099953	-17550499405.7703\\
1.59933998349959	-17564364984.4125\\
1.59943998599965	-17578230563.0546\\
1.59953998849971	-17592096141.6968\\
1.59963999099977	-17605961720.339\\
1.59973999349984	-17619827298.9811\\
1.5998399959999	-17633692877.6233\\
1.59993999849996	-17647615752.045\\
1.60004000100003	-17661481330.6871\\
};
\addplot [color=mycolor3,solid,forget plot]
  table[row sep=crcr]{%
1.60004000100003	-17661481330.6871\\
1.60014000350009	-17675404205.1088\\
1.60024000600015	-17689269783.751\\
1.60034000850021	-17703192658.1727\\
1.60044001100027	-17717115532.5943\\
1.60054001350034	-17731038407.016\\
1.6006400160004	-17744961281.4377\\
1.60074001850046	-17758884155.8594\\
1.60084002100052	-17772807030.2811\\
1.60094002350059	-17786787200.4823\\
1.60104002600065	-17800710074.9039\\
1.60114002850071	-17814690245.1051\\
1.60124003100078	-17828670415.3063\\
1.60134003350084	-17842593289.728\\
1.6014400360009	-17856573459.9292\\
1.60154003850096	-17870553630.1304\\
1.60164004100102	-17884533800.3316\\
1.60174004350109	-17898513970.5328\\
1.60184004600115	-17912551436.5135\\
1.60194004850121	-17926531606.7147\\
1.60204005100128	-17940569072.6954\\
1.60214005350134	-17954549242.8966\\
1.6022400560014	-17968586708.8773\\
1.60234005850146	-17982624174.858\\
1.60244006100153	-17996661640.8387\\
1.60254006350159	-18010699106.8194\\
1.60264006600165	-18024736572.8001\\
1.60274006850171	-18038774038.7808\\
1.60284007100177	-18052811504.7615\\
1.60294007350184	-18066906266.5217\\
1.6030400760019	-18080943732.5024\\
1.60314007850196	-18095038494.2626\\
1.60324008100203	-18109075960.2433\\
1.60334008350209	-18123170722.0036\\
1.60344008600215	-18137265483.7638\\
1.60354008850221	-18151360245.524\\
1.60364009100228	-18165455007.2842\\
1.60374009350234	-18179549769.0444\\
1.6038400960024	-18193701826.5842\\
1.60394009850246	-18207796588.3444\\
1.60404010100253	-18221891350.1046\\
1.60414010350259	-18236043407.6443\\
1.60424010600265	-18250195465.1841\\
1.60434010850271	-18264347522.7238\\
1.60444011100278	-18278442284.484\\
1.60454011350284	-18292594342.0237\\
1.6046401160029	-18306803695.343\\
1.60474011850296	-18320955752.8827\\
1.60484012100302	-18335107810.4224\\
1.60494012350309	-18349317163.7417\\
1.60504012600315	-18363469221.2814\\
1.60514012850321	-18377678574.6007\\
1.60524013100328	-18391830632.1404\\
1.60534013350334	-18406039985.4596\\
1.6054401360034	-18420249338.7789\\
1.60554013850346	-18434458692.0981\\
1.60564014100353	-18448668045.4174\\
1.60574014350359	-18462877398.7366\\
1.60584014600365	-18477144047.8354\\
1.60594014850371	-18491353401.1546\\
1.60604015100378	-18505620050.2534\\
1.60614015350384	-18519829403.5726\\
1.6062401560039	-18534096052.6714\\
1.60634015850396	-18548362701.7701\\
1.60644016100403	-18562629350.8689\\
1.60654016350409	-18576895999.9677\\
1.60664016600415	-18591162649.0664\\
1.60674016850421	-18605429298.1652\\
1.60684017100428	-18619695947.2639\\
1.60694017350434	-18634019892.1422\\
1.6070401760044	-18648286541.241\\
1.60714017850446	-18662610486.1192\\
1.60724018100453	-18676934430.9975\\
1.60734018350459	-18691258375.8758\\
1.60744018600465	-18705582320.754\\
1.60754018850471	-18719906265.6323\\
1.60764019100478	-18734230210.5106\\
1.60774019350484	-18748554155.3889\\
1.6078401960049	-18762878100.2671\\
1.60794019850496	-18777259340.9249\\
1.60804020100503	-18791583285.8032\\
1.60814020350509	-18805964526.461\\
1.60824020600515	-18820345767.1187\\
1.60834020850521	-18834669711.997\\
1.60844021100528	-18849050952.6548\\
1.60854021350534	-18863432193.3126\\
1.6086402160054	-18877870729.7499\\
1.60874021850546	-18892251970.4077\\
1.60884022100553	-18906633211.0654\\
1.60894022350559	-18921071747.5027\\
1.60904022600565	-18935452988.1605\\
1.60914022850571	-18949891524.5978\\
1.60924023100578	-18964330061.0351\\
1.60934023350584	-18978711301.6929\\
1.6094402360059	-18993149838.1302\\
1.60954023850596	-19007588374.5675\\
1.60964024100603	-19022084206.7843\\
1.60974024350609	-19036522743.2216\\
1.60984024600615	-19050961279.6589\\
1.60994024850621	-19065457111.8757\\
1.61004025100628	-19079895648.313\\
1.61014025350634	-19094391480.5298\\
1.6102402560064	-19108887312.7466\\
1.61034025850646	-19123383144.9634\\
1.61044026100653	-19137878977.1802\\
1.61054026350659	-19152374809.3971\\
1.61064026600665	-19166870641.6139\\
1.61074026850671	-19181366473.8307\\
1.61084027100678	-19195862306.0475\\
1.61094027350684	-19210415434.0438\\
1.6110402760069	-19224911266.2606\\
1.61114027850696	-19239464394.2569\\
1.61124028100703	-19254017522.2533\\
1.61134028350709	-19268570650.2496\\
1.61144028600715	-19283123778.2459\\
1.61154028850721	-19297676906.2422\\
1.61164029100728	-19312230034.2386\\
1.61174029350734	-19326783162.2349\\
1.6118402960074	-19341393586.0107\\
1.61194029850746	-19355946714.007\\
1.61204030100753	-19370557137.7829\\
1.61214030350759	-19385110265.7792\\
1.61224030600765	-19399720689.555\\
1.61234030850771	-19414331113.3309\\
1.61244031100778	-19428941537.1067\\
1.61254031350784	-19443551960.8825\\
1.6126403160079	-19458162384.6584\\
1.61274031850796	-19472830104.2137\\
1.61284032100803	-19487440527.9896\\
1.61294032350809	-19502108247.5449\\
1.61304032600815	-19516718671.3207\\
1.61314032850821	-19531386390.8761\\
1.61324033100828	-19546054110.4314\\
1.61334033350834	-19560721829.9868\\
1.6134403360084	-19575389549.5421\\
1.61354033850846	-19590057269.0975\\
1.61364034100853	-19604724988.6528\\
1.61374034350859	-19619392708.2082\\
1.61384034600865	-19634117723.543\\
1.61394034850871	-19648785443.0984\\
1.61404035100878	-19663510458.4333\\
1.61414035350884	-19678235473.7681\\
1.6142403560089	-19692903193.3235\\
1.61434035850896	-19707628208.6583\\
1.61444036100903	-19722353223.9932\\
1.61454036350909	-19737078239.3281\\
1.61464036600915	-19751860550.4424\\
1.61474036850921	-19766585565.7773\\
1.61484037100928	-19781310581.1122\\
1.61494037350934	-19796092892.2265\\
1.6150403760094	-19810875203.3409\\
1.61514037850946	-19825600218.6758\\
1.61524038100953	-19840382529.7901\\
1.61534038350959	-19855164840.9045\\
1.61544038600965	-19869947152.0189\\
1.61554038850971	-19884729463.1333\\
1.61564039100978	-19899511774.2476\\
1.61574039350984	-19914351381.1415\\
1.6158403960099	-19929133692.2559\\
1.61594039850996	-19943973299.1498\\
1.61604040101003	-19958755610.2642\\
1.61614040351009	-19973595217.1581\\
1.61624040601015	-19988434824.052\\
1.61634040851021	-20003274430.9458\\
1.61644041101028	-20018114037.8397\\
1.61654041351034	-20032953644.7336\\
1.6166404160104	-20047793251.6275\\
1.61674041851046	-20062690154.3009\\
1.61684042101053	-20077529761.1948\\
1.61694042351059	-20092426663.8682\\
1.61704042601065	-20107323566.5416\\
1.61714042851071	-20122163173.4355\\
1.61724043101078	-20137060076.1089\\
1.61734043351084	-20151956978.7823\\
1.6174404360109	-20166853881.4557\\
1.61754043851096	-20181750784.1291\\
1.61764044101103	-20196704982.582\\
1.61774044351109	-20211601885.2554\\
1.61784044601115	-20226556083.7083\\
1.61794044851121	-20241452986.3817\\
1.61804045101128	-20256407184.8346\\
1.61814045351134	-20271361383.2876\\
1.6182404560114	-20286258285.961\\
1.61834045851146	-20301212484.4139\\
1.61844046101153	-20316223978.6463\\
1.61854046351159	-20331178177.0992\\
1.61864046601165	-20346132375.5521\\
1.61874046851171	-20361086574.005\\
1.61884047101178	-20376098068.2375\\
1.61894047351184	-20391109562.4699\\
1.6190404760119	-20406063760.9228\\
1.61914047851196	-20421075255.1552\\
1.61924048101203	-20436086749.3877\\
1.61934048351209	-20451098243.6201\\
1.61944048601215	-20466109737.8525\\
1.61954048851221	-20481121232.0849\\
1.61964049101228	-20496190022.0969\\
1.61974049351234	-20511201516.3293\\
1.6198404960124	-20526270306.3413\\
1.61994049851246	-20541281800.5737\\
1.62004050101253	-20556350590.5856\\
1.62014050351259	-20571419380.5976\\
1.62024050601265	-20586488170.6095\\
1.62034050851271	-20601556960.6214\\
1.62044051101278	-20616625750.6334\\
1.62054051351284	-20631694540.6453\\
1.6206405160129	-20646763330.6573\\
1.62074051851296	-20661889416.4487\\
1.62084052101303	-20676958206.4607\\
1.62094052351309	-20692084292.2521\\
1.62104052601315	-20707210378.0436\\
1.62114052851321	-20722336463.835\\
1.62124053101328	-20737405253.847\\
1.62134053351334	-20752588635.4179\\
1.6214405360134	-20767714721.2094\\
1.62154053851346	-20782840807.0008\\
1.62164054101353	-20797966892.7923\\
1.62174054351359	-20813150274.3633\\
1.62184054601365	-20828276360.1547\\
1.62194054851371	-20843459741.7257\\
1.62204055101378	-20858643123.2966\\
1.62214055351384	-20873826504.8676\\
1.6222405560139	-20889009886.4386\\
1.62234055851396	-20904193268.0095\\
1.62244056101403	-20919376649.5805\\
1.62254056351409	-20934560031.1515\\
1.62264056601415	-20949743412.7224\\
1.62274056851421	-20964984090.0729\\
1.62284057101428	-20980167471.6439\\
1.62294057351434	-20995408148.9944\\
1.6230405760144	-21010648826.3449\\
1.62314057851446	-21025889503.6953\\
1.62324058101453	-21041130181.0458\\
1.62334058351459	-21056370858.3963\\
1.62344058601465	-21071611535.7468\\
1.62354058851471	-21086852213.0973\\
1.62364059101478	-21102150186.2272\\
1.62374059351484	-21117390863.5777\\
1.6238405960149	-21132688836.7077\\
1.62394059851496	-21147986809.8377\\
1.62404060101503	-21163227487.1882\\
1.62414060351509	-21178525460.3182\\
1.62424060601515	-21193823433.4482\\
1.62434060851521	-21209121406.5782\\
1.62444061101528	-21224476675.4877\\
1.62454061351534	-21239774648.6177\\
1.6246406160154	-21255072621.7477\\
1.62474061851546	-21270427890.6572\\
1.62484062101553	-21285783159.5667\\
1.62494062351559	-21301081132.6967\\
1.62504062601565	-21316436401.6062\\
1.62514062851571	-21331791670.5157\\
1.62524063101578	-21347146939.4252\\
1.62534063351584	-21362502208.3347\\
1.6254406360159	-21377857477.2442\\
1.62554063851596	-21393270041.9332\\
1.62564064101603	-21408625310.8427\\
1.62574064351609	-21424037875.5317\\
1.62584064601615	-21439393144.4412\\
1.62594064851621	-21454805709.1303\\
1.62604065101628	-21470218273.8193\\
1.62614065351634	-21485630838.5083\\
1.6262406560164	-21501043403.1973\\
1.62634065851646	-21516455967.8863\\
1.62644066101653	-21531868532.5754\\
1.62654066351659	-21547338393.0439\\
1.62664066601665	-21562750957.7329\\
1.62674066851671	-21578220818.2014\\
1.62684067101678	-21593690678.67\\
1.62694067351684	-21609103243.359\\
1.6270406760169	-21624573103.8275\\
1.62714067851696	-21640042964.2961\\
1.62724068101703	-21655512824.7646\\
1.62734068351709	-21670982685.2331\\
1.62744068601715	-21686509841.4812\\
1.62754068851721	-21701979701.9497\\
1.62764069101728	-21717506858.1978\\
1.62774069351734	-21732976718.6663\\
1.6278406960174	-21748503874.9143\\
1.62794069851746	-21764031031.1624\\
1.62804070101753	-21779558187.4104\\
1.62814070351759	-21795085343.6585\\
1.62824070601765	-21810612499.9065\\
1.62834070851771	-21826139656.1546\\
1.62844071101778	-21841666812.4026\\
1.62854071351784	-21857251264.4302\\
1.6286407160179	-21872778420.6782\\
1.62874071851796	-21888362872.7058\\
1.62884072101803	-21903890028.9538\\
1.62894072351809	-21919474480.9814\\
1.62904072601815	-21935058933.0089\\
1.62914072851821	-21950643385.0365\\
1.62924073101828	-21966227837.064\\
1.62934073351834	-21981869584.8711\\
1.6294407360184	-21997454036.8987\\
1.62954073851846	-22013038488.9262\\
1.62964074101853	-22028680236.7333\\
1.62974074351859	-22044321984.5404\\
1.62984074601865	-22059906436.5679\\
1.62994074851871	-22075548184.375\\
1.63004075101878	-22091189932.1821\\
1.63014075351884	-22106831679.9891\\
1.6302407560189	-22122473427.7962\\
1.63034075851896	-22138172471.3828\\
1.63044076101903	-22153814219.1899\\
1.63054076351909	-22169455966.9969\\
1.63064076601915	-22185155010.5835\\
1.63074076851921	-22200854054.1701\\
1.63084077101928	-22216495801.9772\\
1.63094077351934	-22232194845.5638\\
1.6310407760194	-22247893889.1504\\
1.63114077851946	-22263592932.7369\\
1.63124078101953	-22279349272.103\\
1.63134078351959	-22295048315.6896\\
1.63144078601965	-22310747359.2762\\
1.63154078851971	-22326503698.6423\\
1.63164079101978	-22342202742.2289\\
1.63174079351984	-22357959081.595\\
1.6318407960199	-22373715420.9611\\
1.63194079851996	-22389471760.3272\\
1.63204080102003	-22405228099.6933\\
1.63214080352009	-22420984439.0594\\
1.63224080602015	-22436740778.4255\\
1.63234080852021	-22452497117.7916\\
1.63244081102028	-22468310752.9372\\
1.63254081352034	-22484067092.3033\\
1.6326408160204	-22499880727.4489\\
1.63274081852046	-22515694362.5945\\
1.63284082102053	-22531507997.7401\\
1.63294082352059	-22547321632.8857\\
1.63304082602065	-22563135268.0313\\
1.63314082852071	-22578948903.1769\\
1.63324083102078	-22594762538.3226\\
1.63334083352084	-22610576173.4682\\
1.6334408360209	-22626447104.3933\\
1.63354083852096	-22642260739.5389\\
1.63364084102103	-22658131670.464\\
1.63374084352109	-22674002601.3891\\
1.63384084602115	-22689873532.3143\\
1.63394084852121	-22705744463.2394\\
1.63404085102128	-22721615394.1645\\
1.63414085352134	-22737486325.0896\\
1.6342408560214	-22753357256.0148\\
1.63434085852146	-22769228186.9399\\
1.63444086102153	-22785156413.6445\\
1.63454086352159	-22801084640.3492\\
1.63464086602165	-22816955571.2743\\
1.63474086852171	-22832883797.9789\\
1.63484087102178	-22848812024.6836\\
1.63494087352184	-22864740251.3882\\
1.6350408760219	-22880668478.0928\\
1.63514087852196	-22896596704.7975\\
1.63524088102203	-22912524931.5021\\
1.63534088352209	-22928510453.9863\\
1.63544088602215	-22944438680.6909\\
1.63554088852221	-22960424203.175\\
1.63564089102228	-22976409725.6592\\
1.63574089352234	-22992395248.1433\\
1.6358408960224	-23008323474.848\\
1.63594089852246	-23024366293.1116\\
1.63604090102253	-23040351815.5958\\
1.63614090352259	-23056337338.0799\\
1.63624090602265	-23072322860.5641\\
1.63634090852271	-23088365678.8278\\
1.63644091102278	-23104351201.3119\\
1.63654091352284	-23120394019.5756\\
1.6366409160229	-23136436837.8392\\
1.63674091852296	-23152422360.3234\\
1.63684092102303	-23168465178.587\\
1.63694092352309	-23184507996.8507\\
1.63704092602315	-23200608110.8939\\
1.63714092852321	-23216650929.1575\\
1.63724093102328	-23232693747.4212\\
1.63734093352334	-23248793861.4644\\
1.6374409360234	-23264836679.7281\\
1.63754093852346	-23280936793.7712\\
1.63764094102353	-23297036907.8144\\
1.63774094352359	-23313137021.8576\\
1.63784094602365	-23329237135.9008\\
1.63794094852371	-23345337249.9439\\
1.63804095102378	-23361437363.9871\\
1.63814095352384	-23377537478.0303\\
1.6382409560239	-23393694887.853\\
1.63834095852396	-23409795001.8961\\
1.63844096102403	-23425952411.7188\\
1.63854096352409	-23442052525.762\\
1.63864096602415	-23458209935.5847\\
1.63874096852421	-23474367345.4074\\
1.63884097102428	-23490524755.2301\\
1.63894097352434	-23506682165.0528\\
1.6390409760244	-23522896870.655\\
1.63914097852446	-23539054280.4777\\
1.63924098102453	-23555211690.3004\\
1.63934098352459	-23571426395.9026\\
1.63944098602465	-23587583805.7252\\
1.63954098852471	-23603798511.3274\\
1.63964099102478	-23620013216.9296\\
1.63974099352484	-23636227922.5318\\
1.6398409960249	-23652442628.1341\\
1.63994099852496	-23668657333.7363\\
1.64004100102503	-23684929335.118\\
1.64014100352509	-23701144040.7202\\
1.64024100602515	-23717358746.3224\\
1.64034100852521	-23733630747.7041\\
1.64044101102528	-23749902749.0858\\
1.64054101352534	-23766117454.688\\
1.6406410160254	-23782389456.0697\\
1.64074101852546	-23798661457.4514\\
1.64084102102553	-23814933458.8332\\
1.64094102352559	-23831262755.9944\\
1.64104102602565	-23847534757.3761\\
1.64114102852571	-23863806758.7578\\
1.64124103102578	-23880136055.919\\
1.64134103352584	-23896408057.3008\\
1.6414410360259	-23912737354.462\\
1.64154103852596	-23929066651.6232\\
1.64164104102603	-23945395948.7844\\
1.64174104352609	-23961725245.9457\\
1.64184104602615	-23978054543.1069\\
1.64194104852621	-23994383840.2681\\
1.64204105102628	-24010713137.4294\\
1.64214105352634	-24027099730.3701\\
1.6422410560264	-24043429027.5313\\
1.64234105852646	-24059815620.4721\\
1.64244106102653	-24076202213.4128\\
1.64254106352659	-24092588806.3535\\
1.64264106602665	-24108975399.2943\\
1.64274106852671	-24125361992.235\\
1.64284107102678	-24141748585.1758\\
1.64294107352684	-24158135178.1165\\
1.6430410760269	-24174579066.8368\\
1.64314107852696	-24190965659.7775\\
1.64324108102703	-24207409548.4978\\
1.64334108352709	-24223796141.4385\\
1.64344108602715	-24240240030.1588\\
1.64354108852721	-24256683918.879\\
1.64364109102728	-24273127807.5993\\
1.64374109352734	-24289571696.3195\\
1.6438410960274	-24306015585.0398\\
1.64394109852746	-24322459473.76\\
1.64404110102753	-24338960658.2598\\
1.64414110352759	-24355404546.9801\\
1.64424110602765	-24371905731.4798\\
1.64434110852771	-24388406915.9796\\
1.64444111102778	-24404908100.4794\\
1.64454111352784	-24421351989.1996\\
1.6446411160279	-24437910469.4789\\
1.64474111852796	-24454411653.9787\\
1.64484112102803	-24470912838.4784\\
1.64494112352809	-24487414022.9782\\
1.64504112602815	-24503972503.2575\\
1.64514112852821	-24520473687.7573\\
1.64524113102828	-24537032168.0365\\
1.64534113352834	-24553590648.3158\\
1.6454411360284	-24570091832.8156\\
1.64554113852846	-24586650313.0949\\
1.64564114102853	-24603266089.1537\\
1.64574114352859	-24619824569.4329\\
1.64584114602865	-24636383049.7122\\
1.64594114852871	-24652941529.9915\\
1.64604115102878	-24669557306.0503\\
1.64614115352884	-24686115786.3296\\
1.6462411560289	-24702731562.3884\\
1.64634115852896	-24719347338.4472\\
1.64644116102903	-24735963114.506\\
1.64654116352909	-24752578890.5647\\
1.64664116602915	-24769194666.6235\\
1.64674116852921	-24785810442.6823\\
1.64684117102928	-24802426218.7411\\
1.64694117352934	-24819099290.5794\\
1.6470411760294	-24835715066.6382\\
1.64714117852946	-24852388138.4765\\
1.64724118102953	-24869061210.3148\\
1.64734118352959	-24885734282.1531\\
1.64744118602965	-24902350058.2119\\
1.64754118852971	-24919080425.8298\\
1.64764119102978	-24935753497.6681\\
1.64774119352984	-24952426569.5064\\
1.6478411960299	-24969099641.3447\\
1.64794119852996	-24985830008.9625\\
1.64804120103003	-25002503080.8008\\
1.64814120353009	-25019233448.4186\\
1.64824120603015	-25035963816.0365\\
1.64834120853021	-25052694183.6543\\
1.64844121103028	-25069424551.2721\\
1.64854121353034	-25086154918.8899\\
1.6486412160304	-25102885286.5077\\
1.64874121853046	-25119615654.1256\\
1.64884122103053	-25136403317.5229\\
1.64894122353059	-25153133685.1407\\
1.64904122603065	-25169921348.538\\
1.64914122853071	-25186651716.1559\\
1.64924123103078	-25203439379.5532\\
1.64934123353084	-25220227042.9505\\
1.6494412360309	-25237014706.3479\\
1.64954123853096	-25253802369.7452\\
1.64964124103103	-25270590033.1425\\
1.64974124353109	-25287434992.3194\\
1.64984124603115	-25304222655.7167\\
1.64994124853121	-25321067614.8936\\
1.65004125103128	-25337855278.2909\\
1.65014125353134	-25354700237.4677\\
1.6502412560314	-25371545196.6446\\
1.65034125853146	-25388390155.8214\\
1.65044126103153	-25405235114.9983\\
1.65054126353159	-25422080074.1751\\
1.65064126603165	-25438925033.352\\
1.65074126853171	-25455827288.3083\\
1.65084127103178	-25472672247.4852\\
1.65094127353184	-25489574502.4415\\
1.6510412760319	-25506476757.3979\\
1.65114127853196	-25523321716.5747\\
1.65124128103203	-25540223971.5311\\
1.65134128353209	-25557126226.4874\\
1.65144128603215	-25574028481.4438\\
1.65154128853221	-25590988032.1797\\
1.65164129103228	-25607890287.136\\
1.65174129353234	-25624792542.0924\\
1.6518412960324	-25641752092.8283\\
1.65194129853246	-25658654347.7846\\
1.65204130103253	-25675613898.5205\\
1.65214130353259	-25692573449.2564\\
1.65224130603265	-25709532999.9922\\
1.65234130853271	-25726492550.7281\\
1.65244131103278	-25743452101.464\\
1.65254131353284	-25760411652.1999\\
1.6526413160329	-25777428498.7153\\
1.65274131853296	-25794388049.4511\\
1.65284132103303	-25811404895.9665\\
1.65294132353309	-25828364446.7024\\
1.65304132603315	-25845381293.2178\\
1.65314132853321	-25862398139.7332\\
1.65324133103328	-25879414986.2485\\
1.65334133353334	-25896431832.7639\\
1.6534413360334	-25913448679.2793\\
1.65354133853346	-25930522821.5742\\
1.65364134103353	-25947539668.0896\\
1.65374134353359	-25964556514.605\\
1.65384134603365	-25981630656.8999\\
1.65394134853371	-25998704799.1948\\
1.65404135103378	-26015778941.4897\\
1.65414135353384	-26032853083.7846\\
1.6542413560339	-26049927226.0795\\
1.65434135853396	-26067001368.3744\\
1.65444136103403	-26084075510.6693\\
1.65454136353409	-26101149652.9642\\
1.65464136603415	-26118281091.0386\\
1.65474136853421	-26135355233.3335\\
1.65484137103428	-26152486671.4079\\
1.65494137353434	-26169618109.4823\\
1.6550413760344	-26186749547.5567\\
1.65514137853446	-26203823689.8516\\
1.65524138103453	-26221012423.7055\\
1.65534138353459	-26238143861.7799\\
1.65544138603465	-26255275299.8544\\
1.65554138853471	-26272406737.9288\\
1.65564139103478	-26289595471.7827\\
1.65574139353484	-26306726909.8571\\
1.6558413960349	-26323915643.711\\
1.65594139853496	-26341104377.565\\
1.65604140103503	-26358293111.4189\\
1.65614140353509	-26375481845.2728\\
1.65624140603515	-26392670579.1267\\
1.65634140853521	-26409859312.9807\\
1.65644141103528	-26427105342.6141\\
1.65654141353534	-26444294076.468\\
1.6566414160354	-26461540106.1015\\
1.65674141853546	-26478728839.9554\\
1.65684142103553	-26495974869.5888\\
1.65694142353559	-26513220899.2223\\
1.65704142603565	-26530466928.8557\\
1.65714142853571	-26547712958.4891\\
1.65724143103578	-26564958988.1226\\
1.65734143353584	-26582205017.756\\
1.6574414360359	-26599508343.169\\
1.65754143853596	-26616754372.8024\\
1.65764144103603	-26634057698.2153\\
1.65774144353609	-26651303727.8488\\
1.65784144603615	-26668607053.2617\\
1.65794144853621	-26685910378.6747\\
1.65804145103628	-26703213704.0876\\
1.65814145353634	-26720517029.5006\\
1.6582414560364	-26737820354.9135\\
1.65834145853646	-26755180976.106\\
1.65844146103653	-26772484301.519\\
1.65854146353659	-26789844922.7114\\
1.65864146603665	-26807148248.1244\\
1.65874146853671	-26824508869.3168\\
1.65884147103678	-26841869490.5093\\
1.65894147353684	-26859230111.7018\\
1.6590414760369	-26876590732.8942\\
1.65914147853696	-26893951354.0867\\
1.65924148103703	-26911311975.2791\\
1.65934148353709	-26928729892.2511\\
1.65944148603715	-26946090513.4436\\
1.65954148853721	-26963508430.4156\\
1.65964149103728	-26980869051.608\\
1.65974149353734	-26998286968.58\\
1.6598414960374	-27015704885.552\\
1.65994149853746	-27033122802.524\\
1.66004150103753	-27050540719.4959\\
1.66014150353759	-27067958636.4679\\
1.66024150603765	-27085433849.2194\\
1.66034150853771	-27102851766.1914\\
1.66044151103778	-27120326978.9429\\
1.66054151353784	-27137744895.9149\\
1.6606415160379	-27155220108.6663\\
1.66074151853796	-27172695321.4178\\
1.66084152103803	-27190170534.1693\\
1.66094152353809	-27207645746.9208\\
1.66104152603815	-27225120959.6723\\
1.66114152853821	-27242596172.4238\\
1.66124153103828	-27260071385.1753\\
1.66134153353834	-27277603893.7063\\
1.6614415360384	-27295079106.4578\\
1.66154153853846	-27312611614.9888\\
1.66164154103853	-27330144123.5198\\
1.66174154353859	-27347676632.0508\\
1.66184154603865	-27365209140.5818\\
1.66194154853871	-27382741649.1128\\
1.66204155103878	-27400274157.6438\\
1.66214155353884	-27417806666.1748\\
1.6622415560389	-27435396470.4853\\
1.66234155853896	-27452928979.0163\\
1.66244156103903	-27470518783.3268\\
1.66254156353909	-27488108587.6373\\
1.66264156603915	-27505641096.1684\\
1.66274156853921	-27523230900.4789\\
1.66284157103928	-27540820704.7894\\
1.66294157353934	-27558467804.8794\\
1.6630415760394	-27576057609.1899\\
1.66314157853946	-27593647413.5004\\
1.66324158103953	-27611294513.5905\\
1.66334158353959	-27628884317.901\\
1.66344158603965	-27646531417.991\\
1.66354158853971	-27664178518.0811\\
1.66364159103978	-27681825618.1711\\
1.66374159353984	-27699472718.2611\\
1.6638415960399	-27717119818.3511\\
1.66394159853996	-27734766918.4412\\
1.66404160104003	-27752414018.5312\\
1.66414160354009	-27770118414.4007\\
1.66424160604015	-27787765514.4908\\
1.66434160854021	-27805469910.3603\\
1.66444161104028	-27823117010.4503\\
1.66454161354034	-27840821406.3199\\
1.6646416160404	-27858525802.1894\\
1.66474161854046	-27876230198.059\\
1.66484162104053	-27893934593.9285\\
1.66494162354059	-27911696285.5776\\
1.66504162604065	-27929400681.4471\\
1.66514162854071	-27947105077.3167\\
1.66524163104078	-27964866768.9657\\
1.66534163354084	-27982628460.6148\\
1.6654416360409	-28000332856.4843\\
1.66554163854096	-28018094548.1334\\
1.66564164104103	-28035856239.7824\\
1.66574164354109	-28053617931.4315\\
1.66584164604115	-28071436918.86\\
1.66594164854121	-28089198610.5091\\
1.66604165104128	-28106960302.1581\\
1.66614165354134	-28124779289.5867\\
1.6662416560414	-28142540981.2358\\
1.66634165854146	-28160359968.6643\\
1.66644166104153	-28178178956.0929\\
1.66654166354159	-28195997943.5215\\
1.66664166604165	-28213816930.9501\\
1.66674166854171	-28231635918.3786\\
1.66684167104178	-28249454905.8072\\
1.66694167354184	-28267331189.0153\\
1.6670416760419	-28285150176.4438\\
1.66714167854196	-28303026459.6519\\
1.66724168104203	-28320845447.0805\\
1.66734168354209	-28338721730.2886\\
1.66744168604215	-28356598013.4967\\
1.66754168854221	-28374474296.7047\\
1.66764169104228	-28392350579.9128\\
1.66774169354234	-28410226863.1209\\
1.6678416960424	-28428160442.1085\\
1.66794169854246	-28446036725.3166\\
1.66804170104253	-28463970304.3042\\
1.66814170354259	-28481846587.5122\\
1.66824170604265	-28499780166.4998\\
1.66834170854271	-28517713745.4874\\
1.66844171104278	-28535647324.475\\
1.66854171354284	-28553580903.4626\\
1.6686417160429	-28571514482.4502\\
1.66874171854296	-28589448061.4378\\
1.66884172104303	-28607381640.4254\\
1.66894172354309	-28625372515.1925\\
1.66904172604315	-28643363389.9596\\
1.66914172854321	-28661296968.9472\\
1.66924173104328	-28679287843.7143\\
1.66934173354334	-28697278718.4814\\
1.6694417360434	-28715269593.2485\\
1.66954173854346	-28733260468.0157\\
1.66964174104353	-28751251342.7828\\
1.66974174354359	-28769242217.5499\\
1.66984174604365	-28787290388.0965\\
1.66994174854371	-28805281262.8636\\
1.67004175104378	-28823329433.4102\\
1.67014175354384	-28841377603.9568\\
1.6702417560439	-28859425774.5035\\
1.67034175854396	-28877473945.0501\\
1.67044176104403	-28895522115.5967\\
1.67054176354409	-28913570286.1433\\
1.67064176604415	-28931618456.6899\\
1.67074176854421	-28949666627.2366\\
1.67084177104428	-28967772093.5627\\
1.67094177354434	-28985820264.1093\\
1.6710417760444	-29003925730.4355\\
1.67114177854446	-29022031196.7616\\
1.67124178104453	-29040136663.0877\\
1.67134178354459	-29058242129.4139\\
1.67144178604465	-29076347595.74\\
1.67154178854471	-29094453062.0661\\
1.67164179104478	-29112558528.3923\\
1.67174179354484	-29130721290.4979\\
1.6718417960449	-29148826756.824\\
1.67194179854496	-29166989518.9297\\
1.67204180104503	-29185152281.0353\\
1.67214180354509	-29203315043.141\\
1.67224180604515	-29221477805.2466\\
1.67234180854521	-29239640567.3523\\
1.67244181104528	-29257803329.4579\\
1.67254181354534	-29275966091.5636\\
1.6726418160454	-29294128853.6692\\
1.67274181854546	-29312348911.5544\\
1.67284182104553	-29330568969.4395\\
1.67294182354559	-29348731731.5452\\
1.67304182604565	-29366951789.4303\\
1.67314182854571	-29385171847.3155\\
1.67324183104578	-29403391905.2007\\
1.67334183354584	-29421611963.0858\\
1.6734418360459	-29439832020.971\\
1.67354183854596	-29458109374.6357\\
1.67364184104603	-29476329432.5208\\
1.67374184354609	-29494606786.1855\\
1.67384184604615	-29512826844.0707\\
1.67394184854621	-29531104197.7353\\
1.67404185104628	-29549381551.4\\
1.67414185354634	-29567658905.0647\\
1.6742418560464	-29585936258.7293\\
1.67434185854646	-29604213612.394\\
1.67444186104653	-29622490966.0587\\
1.67454186354659	-29640825615.5029\\
1.67464186604665	-29659102969.1676\\
1.67474186854671	-29677437618.6117\\
1.67484187104678	-29695772268.0559\\
1.67494187354684	-29714049621.7206\\
1.6750418760469	-29732384271.1648\\
1.67514187854696	-29750718920.609\\
1.67524188104703	-29769053570.0532\\
1.67534188354709	-29787445515.2769\\
1.67544188604715	-29805780164.721\\
1.67554188854721	-29824114814.1652\\
1.67564189104728	-29842506759.3889\\
1.67574189354734	-29860898704.6126\\
1.6758418960474	-29879233354.0568\\
1.67594189854746	-29897625299.2805\\
1.67604190104753	-29916017244.5042\\
1.67614190354759	-29934409189.7279\\
1.67624190604765	-29952858430.7311\\
1.67634190854771	-29971250375.9548\\
1.67644191104778	-29989642321.1785\\
1.67654191354784	-30008091562.1817\\
1.6766419160479	-30026483507.4054\\
1.67674191854796	-30044932748.4086\\
1.67684192104803	-30063381989.4119\\
1.67694192354809	-30081831230.4151\\
1.67704192604815	-30100280471.4183\\
1.67714192854821	-30118729712.4215\\
1.67724193104828	-30137178953.4247\\
1.67734193354834	-30155685490.2074\\
1.6774419360484	-30174134731.2106\\
1.67754193854846	-30192641267.9934\\
1.67764194104853	-30211090508.9966\\
1.67774194354859	-30229597045.7793\\
1.67784194604865	-30248103582.562\\
1.67794194854871	-30266610119.3448\\
1.67804195104878	-30285116656.1275\\
1.67814195354884	-30303623192.9102\\
1.6782419560489	-30322187025.4725\\
1.67834195854896	-30340693562.2552\\
1.67844196104903	-30359257394.8174\\
1.67854196354909	-30377763931.6001\\
1.67864196604915	-30396327764.1624\\
1.67874196854921	-30414891596.7246\\
1.67884197104928	-30433455429.2869\\
1.67894197354934	-30452019261.8491\\
1.6790419760494	-30470583094.4113\\
1.67914197854946	-30489146926.9736\\
1.67924198104953	-30507768055.3153\\
1.67934198354959	-30526331887.8776\\
1.67944198604965	-30544953016.2193\\
1.67954198854971	-30563574144.5611\\
1.67964199104978	-30582137977.1233\\
1.67974199354984	-30600759105.4651\\
1.6798419960499	-30619380233.8068\\
1.67994199854996	-30638001362.1486\\
1.68004200105003	-30656679786.2698\\
1.68014200355009	-30675300914.6116\\
1.68024200605015	-30693922042.9533\\
1.68034200855021	-30712600467.0746\\
1.68044201105028	-30731278891.1959\\
1.68054201355034	-30749900019.5376\\
1.6806420160504	-30768578443.6589\\
1.68074201855046	-30787256867.7801\\
1.68084202105053	-30805935291.9014\\
1.68094202355059	-30824613716.0227\\
1.68104202605065	-30843349435.9235\\
1.68114202855071	-30862027860.0447\\
1.68124203105078	-30880763579.9455\\
1.68134203355084	-30899442004.0668\\
1.6814420360509	-30918177723.9675\\
1.68154203855096	-30936913443.8683\\
1.68164204105103	-30955649163.7691\\
1.68174204355109	-30974384883.6699\\
1.68184204605115	-30993120603.5706\\
1.68194204855121	-31011856323.4714\\
1.68204205105128	-31030649339.1517\\
1.68214205355134	-31049385059.0525\\
1.6822420560514	-31068178074.7328\\
1.68234205855146	-31086913794.6336\\
1.68244206105153	-31105706810.3139\\
1.68254206355159	-31124499825.9941\\
1.68264206605165	-31143292841.6744\\
1.68274206855171	-31162085857.3547\\
1.68284207105178	-31180878873.035\\
1.68294207355184	-31199729184.4948\\
1.6830420760519	-31218522200.1751\\
1.68314207855196	-31237372511.6349\\
1.68324208105203	-31256165527.3152\\
1.68334208355209	-31275015838.775\\
1.68344208605215	-31293866150.2348\\
1.68354208855221	-31312716461.6946\\
1.68364209105228	-31331566773.1544\\
1.68374209355234	-31350417084.6142\\
1.6838420960524	-31369267396.074\\
1.68394209855246	-31388175003.3134\\
1.68404210105253	-31407025314.7732\\
1.68414210355259	-31425932922.0125\\
1.68424210605265	-31444783233.4723\\
1.68434210855271	-31463690840.7116\\
1.68444211105278	-31482598447.9509\\
1.68454211355284	-31501506055.1902\\
1.6846421160529	-31520413662.4295\\
1.68474211855296	-31539378565.4484\\
1.68484212105303	-31558286172.6877\\
1.68494212355309	-31577193779.927\\
1.68504212605315	-31596158682.9458\\
1.68514212855321	-31615123585.9647\\
1.68524213105328	-31634031193.204\\
1.68534213355334	-31652996096.2228\\
1.6854421360534	-31671960999.2416\\
1.68554213855346	-31690925902.2605\\
1.68564214105353	-31709890805.2793\\
1.68574214355359	-31728913004.0777\\
1.68584214605365	-31747877907.0965\\
1.68594214855371	-31766900105.8948\\
1.68604215105378	-31785865008.9137\\
1.68614215355384	-31804887207.712\\
1.6862421560539	-31823909406.5103\\
1.68634215855396	-31842931605.3087\\
1.68644216105403	-31861953804.107\\
1.68654216355409	-31880976002.9054\\
1.68664216605415	-31899998201.7037\\
1.68674216855421	-31919077696.2816\\
1.68684217105428	-31938099895.0799\\
1.68694217355434	-31957179389.6578\\
1.6870421760544	-31976201588.4561\\
1.68714217855446	-31995281083.034\\
1.68724218105453	-32014360577.6118\\
1.68734218355459	-32033440072.1897\\
1.68744218605465	-32052519566.7675\\
1.68754218855471	-32071656357.1249\\
1.68764219105478	-32090735851.7028\\
1.68774219355484	-32109815346.2806\\
1.6878421960549	-32128952136.638\\
1.68794219855496	-32148088926.9954\\
1.68804220105503	-32167168421.5732\\
1.68814220355509	-32186305211.9306\\
1.68824220605515	-32205442002.288\\
1.68834220855521	-32224578792.6453\\
1.68844221105528	-32243715583.0027\\
1.68854221355534	-32262909669.1396\\
1.6886422160554	-32282046459.4969\\
1.68874221855546	-32301240545.6338\\
1.68884222105553	-32320377335.9912\\
1.68894222355559	-32339571422.1281\\
1.68904222605565	-32358765508.265\\
1.68914222855571	-32377959594.4018\\
1.68924223105578	-32397153680.5387\\
1.68934223355584	-32416347766.6756\\
1.6894422360559	-32435541852.8125\\
1.68954223855596	-32454793234.7289\\
1.68964224105603	-32473987320.8658\\
1.68974224355609	-32493238702.7822\\
1.68984224605615	-32512432788.9191\\
1.68994224855621	-32531684170.8354\\
1.69004225105628	-32550935552.7518\\
1.69014225355634	-32570186934.6682\\
1.6902422560564	-32589438316.5846\\
1.69034225855646	-32608689698.501\\
1.69044226105653	-32627998376.1969\\
1.69054226355659	-32647249758.1133\\
1.69064226605665	-32666558435.8092\\
1.69074226855671	-32685809817.7256\\
1.69084227105678	-32705118495.4215\\
1.69094227355684	-32724427173.1175\\
1.6910422760569	-32743735850.8134\\
1.69114227855696	-32763044528.5093\\
1.69124228105703	-32782353206.2052\\
1.69134228355709	-32801719179.6806\\
1.69144228605715	-32821027857.3765\\
1.69154228855721	-32840336535.0724\\
1.69164229105728	-32859702508.5478\\
1.69174229355734	-32879068482.0233\\
1.6918422960574	-32898434455.4987\\
1.69194229855746	-32917800428.9741\\
1.69204230105753	-32937166402.4495\\
1.69214230355759	-32956532375.925\\
1.69224230605765	-32975898349.4004\\
1.69234230855771	-32995264322.8758\\
1.69244231105778	-33014687592.1307\\
1.69254231355784	-33034053565.6062\\
1.6926423160579	-33053476834.8611\\
1.69274231855796	-33072900104.116\\
1.69284232105803	-33092323373.371\\
1.69294232355809	-33111746642.6259\\
1.69304232605815	-33131169911.8808\\
1.69314232855821	-33150593181.1358\\
1.69324233105828	-33170073746.1702\\
1.69334233355834	-33189497015.4251\\
1.6934423360584	-33208977580.4596\\
1.69354233855846	-33228400849.7145\\
1.69364234105853	-33247881414.749\\
1.69374234355859	-33267361979.7834\\
1.69384234605865	-33286842544.8179\\
1.69394234855871	-33306323109.8523\\
1.69404235105878	-33325803674.8868\\
1.69414235355884	-33345284239.9212\\
1.6942423560589	-33364822100.7352\\
1.69434235855896	-33384302665.7696\\
1.69444236105903	-33403840526.5836\\
1.69454236355909	-33423378387.3975\\
1.69464236605915	-33442858952.432\\
1.69474236855921	-33462396813.246\\
1.69484237105928	-33481934674.0599\\
1.69494237355934	-33501529830.6534\\
1.6950423760594	-33521067691.4674\\
1.69514237855946	-33540605552.2813\\
1.69524238105953	-33560200708.8748\\
1.69534238355959	-33579738569.6887\\
1.69544238605965	-33599333726.2822\\
1.69554238855971	-33618928882.8757\\
1.69564239105978	-33638524039.4692\\
1.69574239355984	-33658119196.0626\\
1.6958423960599	-33677714352.6561\\
1.69594239855996	-33697309509.2496\\
1.69604240106003	-33716904665.8431\\
1.69614240356009	-33736557118.2161\\
1.69624240606015	-33756152274.8095\\
1.69634240856021	-33775804727.1825\\
1.69644241106028	-33795457179.5555\\
1.69654241356034	-33815109631.9285\\
1.6966424160604	-33834762084.3015\\
1.69674241856046	-33854414536.6745\\
1.69684242106053	-33874066989.0475\\
1.69694242356059	-33893719441.4204\\
1.69704242606065	-33913429189.5729\\
1.69714242856071	-33933081641.9459\\
1.69724243106078	-33952791390.0984\\
1.69734243356084	-33972501138.2509\\
1.6974424360609	-33992210886.4034\\
1.69754243856096	-34011863338.7764\\
1.69764244106103	-34031630382.7084\\
1.69774244356109	-34051340130.8609\\
1.69784244606115	-34071049879.0134\\
1.69794244856121	-34090759627.1659\\
1.69804245106128	-34110526671.0979\\
1.69814245356134	-34130236419.2504\\
1.6982424560614	-34150003463.1825\\
1.69834245856146	-34169770507.1145\\
1.69844246106153	-34189537551.0465\\
1.69854246356159	-34209304594.9785\\
1.69864246606165	-34229071638.9105\\
1.69874246856171	-34248838682.8425\\
1.69884247106178	-34268663022.5541\\
1.69894247356184	-34288430066.4861\\
1.6990424760619	-34308254406.1976\\
1.69914247856196	-34328021450.1296\\
1.69924248106203	-34347845789.8411\\
1.69934248356209	-34367670129.5527\\
1.69944248606215	-34387494469.2642\\
1.69954248856221	-34407318808.9757\\
1.69964249106228	-34427143148.6872\\
1.69974249356234	-34447024784.1783\\
1.6998424960624	-34466849123.8898\\
1.69994249856246	-34486730759.3808\\
1.70004250106253	-34506555099.0924\\
1.70014250356259	-34526436734.5834\\
1.70024250606265	-34546318370.0744\\
1.70034250856271	-34566200005.5655\\
1.70044251106278	-34586081641.0565\\
1.70054251356284	-34605963276.5476\\
1.7006425160629	-34625902207.8181\\
1.70074251856296	-34645783843.3092\\
1.70084252106303	-34665665478.8002\\
1.70094252356309	-34685604410.0707\\
1.70104252606315	-34705543341.3413\\
1.70114252856321	-34725482272.6119\\
1.70124253106328	-34745421203.8824\\
1.70134253356334	-34765360135.153\\
1.7014425360634	-34785299066.4235\\
1.70154253856346	-34805237997.6941\\
1.70164254106353	-34825176928.9646\\
1.70174254356359	-34845173156.0147\\
1.70184254606365	-34865112087.2852\\
1.70194254856371	-34885108314.3353\\
1.70204255106378	-34905104541.3854\\
1.70214255356384	-34925100768.4354\\
1.7022425560639	-34945096995.4855\\
1.70234255856396	-34965093222.5356\\
1.70244256106403	-34985089449.5856\\
1.70254256356409	-35005142972.4152\\
1.70264256606415	-35025139199.4653\\
1.70274256856421	-35045192722.2949\\
1.70284257106428	-35065188949.3449\\
1.70294257356434	-35085242472.1745\\
1.7030425760644	-35105295995.0041\\
1.70314257856446	-35125349517.8337\\
1.70324258106453	-35145403040.6632\\
1.70334258356459	-35165456563.4928\\
1.70344258606465	-35185567382.1019\\
1.70354258856471	-35205620904.9315\\
1.70364259106478	-35225731723.5406\\
1.70374259356484	-35245785246.3702\\
1.7038425960649	-35265896064.9792\\
1.70394259856496	-35286006883.5883\\
1.70404260106503	-35306117702.1974\\
1.70414260356509	-35326228520.8065\\
1.70424260606515	-35346339339.4156\\
1.70434260856521	-35366450158.0247\\
1.70444261106528	-35386618272.4133\\
1.70454261356534	-35406729091.0224\\
1.7046426160654	-35426897205.411\\
1.70474261856546	-35447065319.7996\\
1.70484262106553	-35467176138.4087\\
1.70494262356559	-35487344252.7973\\
1.70504262606565	-35507512367.1859\\
1.70514262856571	-35527680481.5745\\
1.70524263106578	-35547905891.7426\\
1.70534263356584	-35568074006.1312\\
1.7054426360659	-35588299416.2994\\
1.70554263856596	-35608467530.688\\
1.70564264106603	-35628692940.8561\\
1.70574264356609	-35648918351.0242\\
1.70584264606615	-35669143761.1923\\
1.70594264856621	-35689369171.3604\\
1.70604265106628	-35709594581.5286\\
1.70614265356634	-35729819991.6967\\
1.7062426560664	-35750045401.8648\\
1.70634265856646	-35770328107.8124\\
1.70644266106653	-35790553517.9805\\
1.70654266356659	-35810836223.9282\\
1.70664266606665	-35831118929.8758\\
1.70674266856671	-35851401635.8234\\
1.70684267106678	-35871684341.7711\\
1.70694267356684	-35891967047.7187\\
1.7070426760669	-35912249753.6663\\
1.70714267856696	-35932532459.614\\
1.70724268106703	-35952872461.3411\\
1.70734268356709	-35973155167.2887\\
1.70744268606715	-35993495169.0159\\
1.70754268856721	-36013835170.743\\
1.70764269106728	-36034117876.6907\\
1.70774269356734	-36054457878.4178\\
1.7078426960674	-36074855175.9245\\
1.70794269856746	-36095195177.6516\\
1.70804270106753	-36115535179.3787\\
1.70814270356759	-36135875181.1059\\
1.70824270606765	-36156272478.6125\\
1.70834270856771	-36176669776.1192\\
1.70844271106778	-36197009777.8464\\
1.70854271356784	-36217407075.353\\
1.7086427160679	-36237804372.8597\\
1.70874271856796	-36258201670.3663\\
1.70884272106803	-36278598967.873\\
1.70894272356809	-36299053561.1591\\
1.70904272606815	-36319450858.6658\\
1.70914272856821	-36339848156.1725\\
1.70924273106828	-36360302749.4586\\
1.70934273356834	-36380757342.7448\\
1.7094427360684	-36401154640.2515\\
1.70954273856846	-36421609233.5376\\
1.70964274106853	-36442063826.8238\\
1.70974274356859	-36462575715.8895\\
1.70984274606865	-36483030309.1757\\
1.70994274856871	-36503484902.4618\\
1.71004275106878	-36523996791.5275\\
1.71014275356884	-36544451384.8137\\
1.7102427560689	-36564963273.8794\\
1.71034275856896	-36585475162.945\\
1.71044276106903	-36605987052.0107\\
1.71054276356909	-36626498941.0764\\
1.71064276606915	-36647010830.1421\\
1.71074276856921	-36667522719.2078\\
1.71084277106928	-36688034608.2735\\
1.71094277356934	-36708603793.1187\\
1.7110427760694	-36729115682.1843\\
1.71114277856946	-36749684867.0295\\
1.71124278106953	-36770254051.8747\\
1.71134278356959	-36790765940.9404\\
1.71144278606965	-36811335125.7856\\
1.71154278856971	-36831961606.4103\\
1.71164279106978	-36852530791.2555\\
1.71174279356984	-36873099976.1007\\
1.7118427960699	-36893669160.9459\\
1.71194279856996	-36914295641.5706\\
1.71204280107003	-36934922122.1953\\
1.71214280357009	-36955491307.0405\\
1.71224280607015	-36976117787.6652\\
1.71234280857021	-36996744268.29\\
1.71244281107028	-37017370748.9147\\
1.71254281357034	-37037997229.5394\\
1.7126428160704	-37058681005.9436\\
1.71274281857046	-37079307486.5683\\
1.71284282107053	-37099991262.9725\\
1.71294282357059	-37120617743.5972\\
1.71304282607065	-37141301520.0015\\
1.71314282857071	-37161985296.4057\\
1.71324283107078	-37182669072.8099\\
1.71334283357084	-37203352849.2141\\
1.7134428360709	-37224036625.6183\\
1.71354283857096	-37244720402.0226\\
1.71364284107103	-37265461474.2063\\
1.71374284357109	-37286145250.6105\\
1.71384284607115	-37306886322.7943\\
1.71394284857121	-37327570099.1985\\
1.71404285107128	-37348311171.3822\\
1.71414285357134	-37369052243.566\\
1.7142428560714	-37389793315.7497\\
1.71434285857146	-37410534387.9334\\
1.71444286107153	-37431275460.1172\\
1.71454286357159	-37452073828.0804\\
1.71464286607165	-37472814900.2642\\
1.71474286857171	-37493613268.2274\\
1.71484287107178	-37514354340.4111\\
1.71494287357184	-37535152708.3744\\
1.7150428760719	-37555951076.3376\\
1.71514287857196	-37576749444.3009\\
1.71524288107203	-37597547812.2641\\
1.71534288357209	-37618346180.2274\\
1.71544288607215	-37639201843.9701\\
1.71554288857221	-37660000211.9334\\
1.71564289107228	-37680855875.6762\\
1.71574289357234	-37701654243.6394\\
1.7158428960724	-37722509907.3822\\
1.71594289857246	-37743365571.1249\\
1.71604290107253	-37764221234.8677\\
1.71614290357259	-37785076898.6105\\
1.71624290607265	-37805932562.3532\\
1.71634290857271	-37826845521.8755\\
1.71644291107278	-37847701185.6182\\
1.71654291357284	-37868614145.1405\\
1.7166429160729	-37889469808.8833\\
1.71674291857296	-37910382768.4056\\
1.71684292107303	-37931295727.9278\\
1.71694292357309	-37952208687.4501\\
1.71704292607315	-37973121646.9724\\
1.71714292857321	-37994034606.4947\\
1.71724293107328	-38014947566.0169\\
1.71734293357334	-38035917821.3187\\
1.7174429360734	-38056830780.841\\
1.71754293857346	-38077801036.1428\\
1.71764294107353	-38098771291.4446\\
1.71774294357359	-38119684250.9669\\
1.71784294607365	-38140654506.2686\\
1.71794294857371	-38161624761.5704\\
1.71804295107378	-38182595016.8722\\
1.71814295357384	-38203622567.9535\\
1.7182429560739	-38224592823.2553\\
1.71834295857396	-38245620374.3366\\
1.71844296107403	-38266590629.6384\\
1.71854296357409	-38287618180.7197\\
1.71864296607415	-38308645731.801\\
1.71874296857421	-38329673282.8823\\
1.71884297107428	-38350700833.9636\\
1.71894297357434	-38371728385.0449\\
1.7190429760744	-38392755936.1262\\
1.71914297857446	-38413783487.2075\\
1.71924298107453	-38434868334.0683\\
1.71934298357459	-38455895885.1496\\
1.71944298607465	-38476980732.0104\\
1.71954298857471	-38498065578.8712\\
1.71964299107478	-38519150425.7321\\
1.71974299357484	-38540235272.5929\\
1.7198429960749	-38561320119.4537\\
1.71994299857496	-38582404966.3145\\
1.72004300107503	-38603547108.9548\\
1.72014300357509	-38624631955.8157\\
1.72024300607515	-38645774098.456\\
1.72034300857521	-38666858945.3168\\
1.72044301107528	-38688001087.9571\\
1.72054301357534	-38709143230.5974\\
1.7206430160754	-38730285373.2378\\
1.72074301857546	-38751427515.8781\\
1.72084302107553	-38772569658.5184\\
1.72094302357559	-38793769096.9383\\
1.72104302607565	-38814911239.5786\\
1.72114302857571	-38836110677.9984\\
1.72124303107578	-38857252820.6388\\
1.72134303357584	-38878452259.0586\\
1.7214430360759	-38899651697.4784\\
1.72154303857596	-38920851135.8983\\
1.72164304107603	-38942050574.3181\\
1.72174304357609	-38963250012.738\\
1.72184304607615	-38984449451.1578\\
1.72194304857621	-39005706185.3572\\
1.72204305107628	-39026905623.777\\
1.72214305357634	-39048162357.9763\\
1.7222430560764	-39069419092.1757\\
1.72234305857646	-39090675826.3751\\
1.72244306107653	-39111932560.5744\\
1.72254306357659	-39133189294.7738\\
1.72264306607665	-39154446028.9731\\
1.72274306857671	-39175702763.1725\\
1.72284307107678	-39196959497.3718\\
1.72294307357684	-39218273527.3507\\
1.7230430760769	-39239587557.3296\\
1.72314307857696	-39260844291.5289\\
1.72324308107703	-39282158321.5078\\
1.72334308357709	-39303472351.4866\\
1.72344308607715	-39324786381.4655\\
1.72354308857721	-39346100411.4444\\
1.72364309107728	-39367471737.2028\\
1.72374309357734	-39388785767.1816\\
1.7238430960774	-39410099797.1605\\
1.72394309857746	-39431471122.9189\\
1.72404310107753	-39452842448.6773\\
1.72414310357759	-39474213774.4356\\
1.72424310607765	-39495527804.4145\\
1.72434310857771	-39516956425.9524\\
1.72444311107778	-39538327751.7108\\
1.72454311357784	-39559699077.4692\\
1.7246431160779	-39581070403.2275\\
1.72474311857796	-39602499024.7654\\
1.72484312107803	-39623870350.5238\\
1.72494312357809	-39645298972.0617\\
1.72504312607815	-39666727593.5996\\
1.72514312857821	-39688156215.1375\\
1.72524313107828	-39709584836.6754\\
1.72534313357834	-39731013458.2133\\
1.7254431360784	-39752442079.7512\\
1.72554313857846	-39773927997.0686\\
1.72564314107853	-39795356618.6065\\
1.72574314357859	-39816842535.9239\\
1.72584314607865	-39838271157.4618\\
1.72594314857871	-39859757074.7792\\
1.72604315107878	-39881242992.0966\\
1.72614315357884	-39902728909.414\\
1.7262431560789	-39924214826.7314\\
1.72634315857896	-39945700744.0488\\
1.72644316107903	-39967243957.1457\\
1.72654316357909	-39988729874.4631\\
1.72664316607915	-40010273087.56\\
1.72674316857921	-40031759004.8774\\
1.72684317107928	-40053302217.9744\\
1.72694317357934	-40074845431.0713\\
1.7270431760794	-40096388644.1682\\
1.72714317857946	-40117931857.2651\\
1.72724318107953	-40139475070.362\\
1.72734318357959	-40161075579.2385\\
1.72744318607965	-40182618792.3354\\
1.72754318857971	-40204219301.2118\\
1.72764319107978	-40225762514.3087\\
1.72774319357984	-40247363023.1852\\
1.7278431960799	-40268963532.0616\\
1.72794319857996	-40290564040.938\\
1.72804320108003	-40312164549.8145\\
1.72814320358009	-40333765058.6909\\
1.72824320608015	-40355365567.5673\\
1.72834320858021	-40377023372.2233\\
1.72844321108028	-40398623881.0997\\
1.72854321358034	-40420281685.7556\\
1.7286432160804	-40441939490.4116\\
1.72874321858046	-40463597295.0675\\
1.72884322108053	-40485255099.7235\\
1.72894322358059	-40506912904.3794\\
1.72904322608065	-40528570709.0354\\
1.72914322858071	-40550228513.6913\\
1.72924323108078	-40571943614.1268\\
1.72934323358084	-40593601418.7827\\
1.7294432360809	-40615316519.2182\\
1.72954323858096	-40636974323.8741\\
1.72964324108103	-40658689424.3096\\
1.72974324358109	-40680404524.745\\
1.72984324608115	-40702119625.1805\\
1.72994324858121	-40723834725.616\\
1.73004325108128	-40745607121.8309\\
1.73014325358134	-40767322222.2664\\
1.7302432560814	-40789094618.4814\\
1.73034325858146	-40810809718.9168\\
1.73044326108153	-40832582115.1318\\
1.73054326358159	-40854354511.3468\\
1.73064326608165	-40876126907.5617\\
1.73074326858171	-40897899303.7767\\
1.73084327108178	-40919671699.9917\\
1.73094327358184	-40941444096.2066\\
1.7310432760819	-40963273788.2011\\
1.73114327858196	-40985046184.4161\\
1.73124328108203	-41006875876.4106\\
1.73134328358209	-41028648272.6256\\
1.73144328608215	-41050477964.62\\
1.73154328858221	-41072307656.6145\\
1.73164329108228	-41094137348.609\\
1.73174329358234	-41115967040.6035\\
1.7318432960824	-41137854028.3775\\
1.73194329858246	-41159683720.372\\
1.73204330108253	-41181570708.146\\
1.73214330358259	-41203400400.1405\\
1.73224330608265	-41225287387.9145\\
1.73234330858271	-41247174375.6885\\
1.73244331108278	-41269061363.4624\\
1.73254331358284	-41290948351.2364\\
1.7326433160829	-41312835339.0104\\
1.73274331858296	-41334722326.7844\\
1.73284332108303	-41356609314.5584\\
1.73294332358309	-41378553598.1119\\
1.73304332608315	-41400440585.8859\\
1.73314332858321	-41422384869.4395\\
1.73324333108328	-41444329152.993\\
1.73334333358334	-41466273436.5465\\
1.7334433360834	-41488217720.1\\
1.73354333858346	-41510162003.6535\\
1.73364334108353	-41532106287.207\\
1.73374334358359	-41554107866.54\\
1.73384334608365	-41576052150.0935\\
1.73394334858371	-41598053729.4266\\
1.73404335108378	-41619998012.9801\\
1.73414335358384	-41641999592.3131\\
1.7342433560839	-41664001171.6461\\
1.73434335858396	-41686002750.9791\\
1.73444336108403	-41708004330.3122\\
1.73454336358409	-41730005909.6452\\
1.73464336608415	-41752064784.7577\\
1.73474336858421	-41774066364.0908\\
1.73484337108428	-41796125239.2033\\
1.73494337358434	-41818126818.5363\\
1.7350433760844	-41840185693.6489\\
1.73514337858446	-41862244568.7614\\
1.73524338108453	-41884303443.8739\\
1.73534338358459	-41906362318.9865\\
1.73544338608465	-41928421194.099\\
1.73554338858471	-41950537364.9911\\
1.73564339108478	-41972596240.1036\\
1.73574339358484	-41994712410.9956\\
1.7358433960849	-42016771286.1082\\
1.73594339858496	-42038887457.0002\\
1.73604340108503	-42061003627.8923\\
1.73614340358509	-42083119798.7843\\
1.73624340608515	-42105235969.6764\\
1.73634340858521	-42127352140.5684\\
1.73644341108528	-42149525607.24\\
1.73654341358534	-42171641778.132\\
1.7366434160854	-42193815244.8036\\
1.73674341858546	-42215931415.6956\\
1.73684342108553	-42238104882.3672\\
1.73694342358559	-42260278349.0388\\
1.73704342608565	-42282451815.7103\\
1.73714342858571	-42304625282.3819\\
1.73724343108578	-42326798749.0535\\
1.73734343358584	-42348972215.725\\
1.7374434360859	-42371202978.1761\\
1.73754343858596	-42393376444.8477\\
1.73764344108603	-42415607207.2987\\
1.73774344358609	-42437837969.7498\\
1.73784344608615	-42460068732.2009\\
1.73794344858621	-42482299494.652\\
1.73804345108628	-42504530257.103\\
1.73814345358634	-42526761019.5541\\
1.7382434560864	-42548991782.0052\\
1.73834345858646	-42571279840.2358\\
1.73844346108653	-42593510602.6869\\
1.73854346358659	-42615798660.9174\\
1.73864346608665	-42638029423.3685\\
1.73874346858671	-42660317481.5991\\
1.73884347108678	-42682605539.8297\\
1.73894347358684	-42704893598.0603\\
1.7390434760869	-42727238952.0704\\
1.73914347858696	-42749527010.301\\
1.73924348108703	-42771815068.5316\\
1.73934348358709	-42794160422.5417\\
1.73944348608715	-42816448480.7723\\
1.73954348858721	-42838793834.7824\\
1.73964349108728	-42861139188.7925\\
1.73974349358734	-42883484542.8026\\
1.7398434960874	-42905829896.8127\\
1.73994349858746	-42928175250.8228\\
1.74004350108753	-42950577900.6124\\
1.74014350358759	-42972923254.6225\\
1.74024350608765	-42995268608.6326\\
1.74034350858771	-43017671258.4222\\
1.74044351108778	-43040073908.2118\\
1.74054351358784	-43062476558.0014\\
1.7406435160879	-43084879207.7911\\
1.74074351858796	-43107281857.5807\\
1.74084352108803	-43129684507.3703\\
1.74094352358809	-43152087157.1599\\
1.74104352608815	-43174489806.9495\\
1.74114352858821	-43196949752.5186\\
1.74124353108828	-43219409698.0878\\
1.74134353358834	-43241812347.8774\\
1.7414435360884	-43264272293.4465\\
1.74154353858846	-43286732239.0156\\
1.74164354108853	-43309192184.5848\\
1.74174354358859	-43331652130.1539\\
1.74184354608865	-43354169371.5025\\
1.74194354858871	-43376629317.0717\\
1.74204355108878	-43399146558.4203\\
1.74214355358884	-43421606503.9894\\
1.7422435560889	-43444123745.3381\\
1.74234355858896	-43466640986.6867\\
1.74244356108903	-43489158228.0354\\
1.74254356358909	-43511675469.384\\
1.74264356608915	-43534192710.7326\\
1.74274356858921	-43556709952.0813\\
1.74284357108928	-43579227193.4299\\
1.74294357358934	-43601801730.5581\\
1.7430435760894	-43624376267.6862\\
1.74314357858946	-43646893509.0349\\
1.74324358108953	-43669468046.163\\
1.74334358358959	-43692042583.2912\\
1.74344358608965	-43714617120.4193\\
1.74354358858971	-43737191657.5475\\
1.74364359108978	-43759823490.4552\\
1.74374359358984	-43782398027.5833\\
1.7438435960899	-43804972564.7115\\
1.74394359858996	-43827604397.6191\\
1.74404360109003	-43850236230.5268\\
1.74414360359009	-43872810767.655\\
1.74424360609015	-43895442600.5626\\
1.74434360859021	-43918074433.4703\\
1.74444361109028	-43940763562.1575\\
1.74454361359034	-43963395395.0651\\
1.7446436160904	-43986027227.9728\\
1.74474361859046	-44008716356.66\\
1.74484362109053	-44031348189.5677\\
1.74494362359059	-44054037318.2548\\
1.74504362609065	-44076726446.942\\
1.74514362859071	-44099415575.6292\\
1.74524363109078	-44122104704.3164\\
1.74534363359084	-44144793833.0036\\
1.7454436360909	-44167482961.6907\\
1.74554363859096	-44190229386.1574\\
1.74564364109103	-44212918514.8446\\
1.74574364359109	-44235664939.3113\\
1.74584364609115	-44258354067.9985\\
1.74594364859121	-44281100492.4652\\
1.74604365109128	-44303846916.9319\\
1.74614365359134	-44326593341.3986\\
1.7462436560914	-44349339765.8653\\
1.74634365859146	-44372143486.1115\\
1.74644366109153	-44394889910.5782\\
1.74654366359159	-44417636335.0449\\
1.74664366609165	-44440440055.2911\\
1.74674366859171	-44463243775.5373\\
1.74684367109178	-44485990200.004\\
1.74694367359184	-44508793920.2502\\
1.7470436760919	-44531597640.4964\\
1.74714367859196	-44554458656.5221\\
1.74724368109203	-44577262376.7683\\
1.74734368359209	-44600066097.0145\\
1.74744368609215	-44622927113.0402\\
1.74754368859221	-44645730833.2864\\
1.74764369109228	-44668591849.3122\\
1.74774369359234	-44691452865.3379\\
1.7478436960924	-44714313881.3636\\
1.74794369859246	-44737174897.3893\\
1.74804370109253	-44760035913.415\\
1.74814370359259	-44782896929.4408\\
1.74824370609265	-44805757945.4665\\
1.74834370859271	-44828676257.2717\\
1.74844371109278	-44851537273.2974\\
1.74854371359284	-44874455585.1027\\
1.7486437160929	-44897373896.9079\\
1.74874371859296	-44920292208.7131\\
1.74884372109303	-44943210520.5184\\
1.74894372359309	-44966128832.3236\\
1.74904372609315	-44989047144.1288\\
1.74914372859321	-45012022751.7136\\
1.74924373109328	-45034941063.5188\\
1.74934373359334	-45057916671.1036\\
1.7494437360934	-45080834982.9088\\
1.74954373859346	-45103810590.4935\\
1.74964374109353	-45126786198.0783\\
1.74974374359359	-45149761805.663\\
1.74984374609365	-45172737413.2478\\
1.74994374859371	-45195713020.8325\\
1.75004375109378	-45218745924.1968\\
1.75014375359384	-45241721531.7815\\
1.7502437560939	-45264754435.1458\\
1.75034375859396	-45287787338.51\\
1.75044376109403	-45310762946.0948\\
1.75054376359409	-45333795849.459\\
1.75064376609415	-45356828752.8233\\
1.75074376859421	-45379861656.1876\\
1.75084377109428	-45402951855.3313\\
1.75094377359434	-45425984758.6956\\
1.7510437760944	-45449017662.0599\\
1.75114377859446	-45472107861.2036\\
1.75124378109453	-45495198060.3474\\
1.75134378359459	-45518230963.7117\\
1.75144378609465	-45541321162.8554\\
1.75154378859471	-45564411361.9992\\
1.75164379109478	-45587558856.9225\\
1.75174379359484	-45610649056.0663\\
1.7518437960949	-45633739255.21\\
1.75194379859496	-45656886750.1333\\
1.75204380109503	-45679976949.2771\\
1.75214380359509	-45703124444.2004\\
1.75224380609515	-45726271939.1237\\
1.75234380859521	-45749419434.0469\\
1.75244381109528	-45772566928.9702\\
1.75254381359534	-45795714423.8935\\
1.7526438160954	-45818861918.8168\\
1.75274381859546	-45842009413.7401\\
1.75284382109553	-45865214204.4429\\
1.75294382359559	-45888361699.3662\\
1.75304382609565	-45911566490.069\\
1.75314382859571	-45934771280.7718\\
1.75324383109578	-45957976071.4746\\
1.75334383359584	-45981180862.1774\\
1.7534438360959	-46004385652.8802\\
1.75354383859596	-46027590443.583\\
1.75364384109603	-46050795234.2858\\
1.75374384359609	-46074057320.7681\\
1.75384384609615	-46097262111.4709\\
1.75394384859621	-46120524197.9532\\
1.75404385109628	-46143786284.4355\\
1.75414385359634	-46167048370.9178\\
1.7542438560964	-46190310457.4001\\
1.75434385859647	-46213572543.8824\\
1.75444386109653	-46236834630.3647\\
1.75454386359659	-46260154012.6266\\
1.75464386609665	-46283416099.1089\\
1.75474386859671	-46306735481.3707\\
1.75484387109678	-46329997567.853\\
1.75494387359684	-46353316950.1148\\
1.7550438760969	-46376636332.3767\\
1.75514387859696	-46399955714.6385\\
1.75524388109703	-46423275096.9003\\
1.75534388359709	-46446594479.1621\\
1.75544388609715	-46469971157.2035\\
1.75554388859721	-46493290539.4653\\
1.75564389109728	-46516667217.5066\\
1.75574389359734	-46539986599.7685\\
1.7558438960974	-46563363277.8098\\
1.75594389859746	-46586739955.8511\\
1.75604390109753	-46610116633.8925\\
1.75614390359759	-46633493311.9338\\
1.75624390609765	-46656927285.7547\\
1.75634390859772	-46680303963.796\\
1.75644391109778	-46703680641.8373\\
1.75654391359784	-46727114615.6582\\
1.7566439160979	-46750548589.479\\
1.75674391859796	-46773925267.5204\\
1.75684392109803	-46797359241.3412\\
1.75694392359809	-46820793215.1621\\
1.75704392609815	-46844284484.7624\\
1.75714392859821	-46867718458.5833\\
1.75724393109828	-46891152432.4041\\
1.75734393359834	-46914643702.0045\\
1.7574439360984	-46938077675.8254\\
1.75754393859847	-46961568945.4257\\
1.75764394109853	-46985060215.0261\\
1.75774394359859	-47008551484.6264\\
1.75784394609865	-47032042754.2268\\
1.75794394859871	-47055534023.8272\\
1.75804395109878	-47079025293.4275\\
1.75814395359884	-47102516563.0279\\
1.7582439560989	-47126065128.4078\\
1.75834395859897	-47149556398.0081\\
1.75844396109903	-47173104963.388\\
1.75854396359909	-47196653528.7679\\
1.75864396609915	-47220202094.1478\\
1.75874396859922	-47243750659.5276\\
1.75884397109928	-47267299224.9075\\
1.75894397359934	-47290847790.2874\\
1.7590439760994	-47314453651.4468\\
1.75914397859946	-47338002216.8267\\
1.75924398109953	-47361608077.9861\\
1.75934398359959	-47385156643.3659\\
1.75944398609965	-47408762504.5253\\
1.75954398859972	-47432368365.6847\\
1.75964399109978	-47455974226.8441\\
1.75974399359984	-47479580088.0035\\
1.7598439960999	-47503185949.1629\\
1.75994399859996	-47526849106.1018\\
1.76004400110003	-47550454967.2612\\
1.76014400360009	-47574118124.2001\\
1.76024400610015	-47597723985.3595\\
1.76034400860022	-47621387142.2984\\
1.76044401110028	-47645050299.2373\\
1.76054401360034	-47668713456.1762\\
1.7606440161004	-47692376613.1151\\
1.76074401860047	-47716097065.8335\\
1.76084402110053	-47739760222.7724\\
1.76094402360059	-47763423379.7113\\
1.76104402610065	-47787143832.4297\\
1.76114402860071	-47810864285.1481\\
1.76124403110078	-47834584737.8666\\
1.76134403360084	-47858305190.585\\
1.7614440361009	-47882025643.3034\\
1.76154403860097	-47905746096.0218\\
1.76164404110103	-47929466548.7402\\
1.76174404360109	-47953187001.4586\\
1.76184404610115	-47976964749.9566\\
1.76194404860122	-48000685202.675\\
1.76204405110128	-48024462951.1729\\
1.76214405360134	-48048240699.6708\\
1.7622440561014	-48072018448.1688\\
1.76234405860147	-48095796196.6667\\
1.76244406110153	-48119573945.1646\\
1.76254406360159	-48143351693.6626\\
1.76264406610165	-48167186737.94\\
1.76274406860172	-48190964486.4379\\
1.76284407110178	-48214799530.7154\\
1.76294407360184	-48238577279.2133\\
1.7630440761019	-48262412323.4907\\
1.76314407860196	-48286247367.7682\\
1.76324408110203	-48310082412.0456\\
1.76334408360209	-48333917456.3231\\
1.76344408610215	-48357809796.38\\
1.76354408860222	-48381644840.6575\\
1.76364409110228	-48405479884.9349\\
1.76374409360234	-48429372224.9919\\
1.7638440961024	-48453264565.0488\\
1.76394409860247	-48477156905.1058\\
1.76404410110253	-48501049245.1627\\
1.76414410360259	-48524941585.2197\\
1.76424410610265	-48548833925.2766\\
1.76434410860272	-48572726265.3336\\
1.76444411110278	-48596618605.3905\\
1.76454411360284	-48620568241.227\\
1.7646441161029	-48644460581.284\\
1.76474411860297	-48668410217.1204\\
1.76484412110303	-48692359852.9569\\
1.76494412360309	-48716309488.7934\\
1.76504412610315	-48740259124.6298\\
1.76514412860322	-48764208760.4663\\
1.76524413110328	-48788158396.3028\\
1.76534413360334	-48812165327.9188\\
1.7654441361034	-48836114963.7552\\
1.76554413860347	-48860121895.3712\\
1.76564414110353	-48884128826.9872\\
1.76574414360359	-48908135758.6032\\
1.76584414610365	-48932085394.4396\\
1.76594414860372	-48956149621.8351\\
1.76604415110378	-48980156553.4511\\
1.76614415360384	-49004163485.0671\\
1.7662441561039	-49028170416.6831\\
1.76634415860397	-49052234644.0786\\
1.76644416110403	-49076241575.6946\\
1.76654416360409	-49100305803.0901\\
1.76664416610415	-49124370030.4855\\
1.76674416860422	-49148434257.881\\
1.76684417110428	-49172498485.2765\\
1.76694417360434	-49196562712.672\\
1.7670441761044	-49220684235.847\\
1.76714417860447	-49244748463.2425\\
1.76724418110453	-49268812690.638\\
1.76734418360459	-49292934213.813\\
1.76744418610465	-49317055736.988\\
1.76754418860472	-49341177260.1631\\
1.76764419110478	-49365298783.3381\\
1.76774419360484	-49389420306.5131\\
1.7678441961049	-49413541829.6881\\
1.76794419860497	-49437663352.8631\\
1.76804420110503	-49461842171.8176\\
1.76814420360509	-49485963694.9926\\
1.76824420610515	-49510142513.9471\\
1.76834420860522	-49534264037.1221\\
1.76844421110528	-49558442856.0767\\
1.76854421360534	-49582621675.0312\\
1.7686442161054	-49606800493.9857\\
1.76874421860547	-49630979312.9402\\
1.76884422110553	-49655215427.6743\\
1.76894422360559	-49679394246.6288\\
1.76904422610565	-49703630361.3628\\
1.76914422860572	-49727809180.3173\\
1.76924423110578	-49752045295.0514\\
1.76934423360584	-49776281409.7854\\
1.7694442361059	-49800517524.5194\\
1.76954423860597	-49824753639.2535\\
1.76964424110603	-49848989753.9875\\
1.76974424360609	-49873225868.7215\\
1.76984424610615	-49897519279.2351\\
1.76994424860622	-49921755393.9691\\
1.77004425110628	-49946048804.4827\\
1.77014425360634	-49970342214.9962\\
1.7702442561064	-49994635625.5098\\
1.77034425860647	-50018929036.0233\\
1.77044426110653	-50043222446.5368\\
1.77054426360659	-50067515857.0504\\
1.77064426610665	-50091809267.5639\\
1.77074426860672	-50116159973.857\\
1.77084427110678	-50140453384.3706\\
1.77094427360684	-50164804090.6636\\
1.7710442761069	-50189154796.9567\\
1.77114427860697	-50213505503.2497\\
1.77124428110703	-50237798913.7633\\
1.77134428360709	-50262206915.8359\\
1.77144428610715	-50286557622.1289\\
1.77154428860722	-50310908328.422\\
1.77164429110728	-50335316330.4945\\
1.77174429360734	-50359667036.7876\\
1.7718442961074	-50384075038.8602\\
1.77194429860747	-50408483040.9327\\
1.77204430110753	-50432833747.2258\\
1.77214430360759	-50457241749.2984\\
1.77224430610765	-50481707047.1505\\
1.77234430860772	-50506115049.223\\
1.77244431110778	-50530523051.2956\\
1.77254431360784	-50554931053.3682\\
1.7726443161079	-50579396351.2203\\
1.77274431860797	-50603861649.0724\\
1.77284432110803	-50628269651.1449\\
1.77294432360809	-50652734948.997\\
1.77304432610815	-50677200246.8491\\
1.77314432860822	-50701665544.7012\\
1.77324433110828	-50726188138.3328\\
1.77334433360834	-50750653436.1849\\
1.7734443361084	-50775118734.037\\
1.77354433860847	-50799641327.6686\\
1.77364434110853	-50824163921.3002\\
1.77374434360859	-50848686514.9318\\
1.77384434610865	-50873151812.7838\\
1.77394434860872	-50897674406.4154\\
1.77404435110878	-50922254295.8266\\
1.77414435360884	-50946776889.4582\\
1.7742443561089	-50971299483.0898\\
1.77434435860897	-50995879372.5009\\
1.77444436110903	-51020401966.1325\\
1.77454436360909	-51044981855.5436\\
1.77464436610915	-51069561744.9547\\
1.77474436860922	-51094141634.3658\\
1.77484437110928	-51118721523.7769\\
1.77494437360934	-51143301413.188\\
1.7750443761094	-51167881302.5991\\
1.77514437860947	-51192461192.0103\\
1.77524438110953	-51217098377.2009\\
1.77534438360959	-51241678266.612\\
1.77544438610965	-51266315451.8026\\
1.77554438860972	-51290952636.9932\\
1.77564439110978	-51315589822.1839\\
1.77574439360984	-51340227007.3745\\
1.7758443961099	-51364864192.5651\\
1.77594439860997	-51389501377.7557\\
1.77604440111003	-51414195858.7259\\
1.77614440361009	-51438833043.9165\\
1.77624440611015	-51463527524.8867\\
1.77634440861022	-51488164710.0773\\
1.77644441111028	-51512859191.0474\\
1.77654441361034	-51537553672.0175\\
1.7766444161104	-51562248152.9877\\
1.77674441861047	-51586942633.9578\\
1.77684442111053	-51611637114.928\\
1.77694442361059	-51636388891.6776\\
1.77704442611065	-51661083372.6478\\
1.77714442861072	-51685835149.3974\\
1.77724443111078	-51710586926.1471\\
1.77734443361084	-51735281407.1172\\
1.7774444361109	-51760033183.8669\\
1.77754443861097	-51784784960.6165\\
1.77764444111103	-51809594033.1457\\
1.77774444361109	-51834345809.8953\\
1.77784444611115	-51859097586.645\\
1.77794444861122	-51883906659.1741\\
1.77804445111128	-51908658435.9238\\
1.77814445361134	-51933467508.4529\\
1.7782444561114	-51958276580.9821\\
1.77834445861147	-51983085653.5113\\
1.77844446111153	-52007894726.0404\\
1.77854446361159	-52032703798.5696\\
1.77864446611165	-52057512871.0988\\
1.77874446861172	-52082379239.4074\\
1.77884447111178	-52107188311.9366\\
1.77894447361184	-52132054680.2453\\
1.7790444761119	-52156921048.554\\
1.77914447861197	-52181787416.8626\\
1.77924448111203	-52206596489.3918\\
1.77934448361209	-52231520153.48\\
1.77944448611215	-52256386521.7887\\
1.77954448861222	-52281252890.0974\\
1.77964449111228	-52306119258.406\\
1.77974449361234	-52331042922.4942\\
1.7798444961124	-52355966586.5824\\
1.77994449861247	-52380832954.8911\\
1.78004450111253	-52405756618.9793\\
1.78014450361259	-52430680283.0675\\
1.78024450611265	-52455603947.1557\\
1.78034450861272	-52480527611.2439\\
1.78044451111278	-52505508571.1116\\
1.78054451361284	-52530432235.1998\\
1.7806445161129	-52555413195.0675\\
1.78074451861297	-52580336859.1556\\
1.78084452111303	-52605317819.0234\\
1.78094452361309	-52630298778.8911\\
1.78104452611315	-52655279738.7588\\
1.78114452861322	-52680260698.6265\\
1.78124453111328	-52705241658.4942\\
1.78134453361334	-52730279914.1414\\
1.7814445361134	-52755260874.0091\\
1.78154453861347	-52780241833.8768\\
1.78164454111353	-52805280089.524\\
1.78174454361359	-52830318345.1712\\
1.78184454611365	-52855356600.8184\\
1.78194454861372	-52880394856.4657\\
1.78204455111378	-52905433112.1129\\
1.78214455361384	-52930471367.7601\\
1.7822445561139	-52955509623.4073\\
1.78234455861397	-52980605174.834\\
1.78244456111403	-53005643430.4813\\
1.78254456361409	-53030738981.908\\
1.78264456611415	-53055834533.3347\\
1.78274456861422	-53080930084.7615\\
1.78284457111428	-53106025636.1882\\
1.78294457361434	-53131121187.6149\\
1.7830445761144	-53156216739.0416\\
1.78314457861447	-53181312290.4684\\
1.78324458111453	-53206465137.6746\\
1.78334458361459	-53231560689.1013\\
1.78344458611465	-53256713536.3076\\
1.78354458861472	-53281866383.5138\\
1.78364459111478	-53307019230.7201\\
1.78374459361484	-53332172077.9263\\
1.7838445961149	-53357324925.1326\\
1.78394459861497	-53382477772.3388\\
1.78404460111503	-53407687915.3246\\
1.78414460361509	-53432840762.5308\\
1.78424460611515	-53458050905.5166\\
1.78434460861522	-53483203752.7228\\
1.78444461111528	-53508413895.7086\\
1.78454461361534	-53533624038.6943\\
1.7846446161154	-53558834181.6801\\
1.78474461861547	-53584044324.6658\\
1.78484462111553	-53609254467.6516\\
1.78494462361559	-53634521906.4168\\
1.78504462611565	-53659732049.4026\\
1.78514462861572	-53684999488.1679\\
1.78524463111578	-53710266926.9331\\
1.78534463361584	-53735477069.9189\\
1.7854446361159	-53760744508.6842\\
1.78554463861597	-53786011947.4494\\
1.78564464111603	-53811336681.9942\\
1.78574464361609	-53836604120.7595\\
1.78584464611615	-53861871559.5248\\
1.78594464861622	-53887196294.0695\\
1.78604465111628	-53912463732.8348\\
1.78614465361634	-53937788467.3796\\
1.7862446561164	-53963113201.9244\\
1.78634465861647	-53988437936.4692\\
1.78644466111653	-54013762671.0139\\
1.78654466361659	-54039087405.5587\\
1.78664466611665	-54064412140.1035\\
1.78674466861672	-54089794170.4278\\
1.78684467111678	-54115118904.9726\\
1.78694467361684	-54140500935.2969\\
1.7870446761169	-54165882965.6212\\
1.78714467861697	-54191207700.166\\
1.78724468111703	-54216589730.4902\\
1.78734468361709	-54241971760.8145\\
1.78744468611715	-54267411086.9184\\
1.78754468861722	-54292793117.2427\\
1.78764469111728	-54318175147.5669\\
1.78774469361734	-54343614473.6708\\
1.7878446961174	-54368996503.995\\
1.78794469861747	-54394435830.0989\\
1.78804470111753	-54419875156.2027\\
1.78814470361759	-54445314482.3065\\
1.78824470611765	-54470753808.4103\\
1.78834470861772	-54496193134.5141\\
1.78844471111778	-54521689756.3974\\
1.78854471361784	-54547129082.5012\\
1.7886447161179	-54572625704.3845\\
1.78874471861797	-54598065030.4884\\
1.78884472111803	-54623561652.3717\\
1.78894472361809	-54649058274.255\\
1.78904472611815	-54674554896.1383\\
1.78914472861822	-54700051518.0216\\
1.78924473111828	-54725548139.905\\
1.78934473361834	-54751044761.7883\\
1.7894447361184	-54776598679.4511\\
1.78954473861847	-54802095301.3344\\
1.78964474111853	-54827649218.9973\\
1.78974474361859	-54853203136.6601\\
1.78984474611865	-54878757054.3229\\
1.78994474861872	-54904310971.9858\\
1.79004475111878	-54929864889.6486\\
1.79014475361884	-54955418807.3114\\
1.7902447561189	-54980972724.9743\\
1.79034475861897	-55006583938.4166\\
1.79044476111903	-55032137856.0795\\
1.79054476361909	-55057749069.5218\\
1.79064476611915	-55083360282.9642\\
1.79074476861922	-55108971496.4065\\
1.79084477111928	-55134582709.8489\\
1.79094477361934	-55160193923.2912\\
1.7910447761194	-55185805136.7336\\
1.79114477861947	-55211416350.1759\\
1.79124478111953	-55237084859.3978\\
1.79134478361959	-55262753368.6196\\
1.79144478611965	-55288364582.062\\
1.79154478861972	-55314033091.2838\\
1.79164479111978	-55339701600.5057\\
1.79174479361984	-55365370109.7276\\
1.7918447961199	-55391038618.9494\\
1.79194479861997	-55416707128.1713\\
1.79204480112003	-55442432933.1726\\
1.79214480362009	-55468101442.3945\\
1.79224480612015	-55493827247.3959\\
1.79234480862022	-55519553052.3973\\
1.79244481112028	-55545221561.6191\\
1.79254481362034	-55570947366.6205\\
1.7926448161204	-55596673171.6219\\
1.79274481862047	-55622398976.6232\\
1.79284482112053	-55648182077.4041\\
1.79294482362059	-55673907882.4055\\
1.79304482612065	-55699690983.1864\\
1.79314482862072	-55725416788.1878\\
1.79324483112078	-55751199888.9687\\
1.79334483362084	-55776982989.7495\\
1.7934448361209	-55802766090.5304\\
1.79354483862097	-55828549191.3113\\
1.79364484112103	-55854332292.0922\\
1.79374484362109	-55880115392.8731\\
1.79384484612115	-55905955789.4335\\
1.79394484862122	-55931738890.2144\\
1.79404485112128	-55957579286.7748\\
1.79414485362134	-55983419683.3352\\
1.7942448561214	-56009202784.1161\\
1.79434485862147	-56035043180.6765\\
1.79444486112153	-56060883577.2369\\
1.79454486362159	-56086781269.5768\\
1.79464486612165	-56112621666.1372\\
1.79474486862172	-56138462062.6976\\
1.79484487112178	-56164359755.0375\\
1.79494487362184	-56190200151.5979\\
1.7950448761219	-56216097843.9378\\
1.79514487862197	-56241995536.2777\\
1.79524488112203	-56267893228.6176\\
1.79534488362209	-56293790920.9575\\
1.79544488612215	-56319688613.2975\\
1.79554488862222	-56345643601.4169\\
1.79564489112228	-56371541293.7568\\
1.79574489362234	-56397496281.8762\\
1.7958448961224	-56423393974.2161\\
1.79594489862247	-56449348962.3356\\
1.79604490112253	-56475303950.455\\
1.79614490362259	-56501258938.5744\\
1.79624490612265	-56527213926.6938\\
1.79634490862272	-56553168914.8133\\
1.79644491112278	-56579181198.7122\\
1.79654491362284	-56605136186.8316\\
1.7966449161229	-56631148470.7306\\
1.79674491862297	-56657103458.85\\
1.79684492112303	-56683115742.7489\\
1.79694492362309	-56709128026.6479\\
1.79704492612315	-56735140310.5468\\
1.79714492862322	-56761152594.4458\\
1.79724493112328	-56787164878.3447\\
1.79734493362334	-56813234458.0231\\
1.7974449361234	-56839246741.9221\\
1.79754493862347	-56865316321.6005\\
1.79764494112353	-56891328605.4995\\
1.79774494362359	-56917398185.1779\\
1.79784494612365	-56943467764.8564\\
1.79794494862372	-56969537344.5348\\
1.79804495112378	-56995606924.2133\\
1.79814495362384	-57021733799.6712\\
1.7982449561239	-57047803379.3497\\
1.79834495862397	-57073872959.0282\\
1.79844496112403	-57099999834.4861\\
1.79854496362409	-57126126709.9441\\
1.79864496612415	-57152253585.4021\\
1.79874496862422	-57178323165.0805\\
1.79884497112428	-57204507336.318\\
1.79894497362434	-57230634211.7759\\
1.7990449761244	-57256761087.2339\\
1.79914497862447	-57282887962.6919\\
1.79924498112453	-57308957542.3703\\
1.79934498362459	-57335313600.9464\\
1.79944498612465	-57361669659.5224\\
1.79954498862472	-57387452760.3033\\
1.79964499112478	-57413808818.8793\\
1.79974499362484	-57440164877.4553\\
1.7998449961249	-57465947978.2362\\
1.79994499862497	-57492304036.8122\\
1.80004500112503	-57518660095.3882\\
1.80014500362509	-57545016153.9642\\
1.80024500612515	-57570799254.7451\\
1.80034500862522	-57597155313.3211\\
1.80044501112528	-57623511371.8972\\
1.80054501362534	-57649867430.4732\\
1.8006450161254	-57675650531.2541\\
1.80074501862547	-57702006589.8301\\
1.80084502112553	-57728362648.4061\\
1.80094502362559	-57754718706.9821\\
1.80104502612565	-57781074765.5581\\
1.80114502862572	-57807430824.1341\\
1.80124503112578	-57833786882.7102\\
1.80134503362584	-57859569983.4911\\
1.8014450361259	-57885926042.0671\\
1.80154503862597	-57912282100.6431\\
1.80164504112603	-57938638159.2191\\
1.80174504362609	-57964994217.7951\\
1.80184504612615	-57991350276.3711\\
1.80194504862622	-58017706334.9472\\
1.80204505112628	-58044062393.5232\\
1.80214505362634	-58070418452.0992\\
1.8022450561264	-58096774510.6752\\
1.80234505862647	-58123130569.2512\\
1.80244506112653	-58149486627.8272\\
1.80254506362659	-58175842686.4033\\
1.80264506612665	-58202198744.9793\\
1.80274506862672	-58228554803.5553\\
1.80284507112678	-58254910862.1313\\
1.80294507362684	-58281266920.7073\\
1.8030450761269	-58307622979.2834\\
1.80314507862697	-58333979037.8594\\
1.80324508112703	-58360335096.4354\\
1.80334508362709	-58387264112.8065\\
1.80344508612715	-58413620171.3826\\
1.80354508862722	-58439976229.9586\\
1.80364509112728	-58466332288.5346\\
1.80374509362734	-58492688347.1106\\
1.8038450961274	-58519044405.6866\\
1.80394509862747	-58545400464.2626\\
1.80404510112753	-58572329480.6338\\
1.80414510362759	-58598685539.2098\\
1.80424510612765	-58625041597.7858\\
1.80434510862772	-58651397656.3619\\
1.80444511112778	-58678326672.733\\
1.80454511362784	-58704682731.309\\
1.8046451161279	-58731038789.885\\
1.80474511862797	-58757394848.4611\\
1.80484512112803	-58784323864.8322\\
1.80494512362809	-58810679923.4082\\
1.80504512612815	-58837035981.9842\\
1.80514512862822	-58863964998.3554\\
1.80524513112828	-58890321056.9314\\
1.80534513362834	-58916677115.5074\\
1.8054451361284	-58943606131.8786\\
1.80554513862847	-58969962190.4546\\
1.80564514112853	-58996318249.0306\\
1.80574514362859	-59023247265.4018\\
1.80584514612865	-59049603323.9778\\
1.80594514862872	-59075959382.5538\\
1.80604515112878	-59102888398.9249\\
1.80614515362884	-59129244457.501\\
1.8062451561289	-59156173473.8721\\
1.80634515862897	-59182529532.4481\\
1.80644516112903	-59209458548.8193\\
1.80654516362909	-59235814607.3953\\
1.80664516612915	-59262743623.7664\\
1.80674516862922	-59289099682.3425\\
1.80684517112928	-59316028698.7136\\
1.80694517362934	-59342384757.2896\\
1.8070451761294	-59369313773.6608\\
1.80714517862947	-59395669832.2368\\
1.80724518112953	-59422598848.6079\\
1.80734518362959	-59448954907.184\\
1.80744518612965	-59475883923.5551\\
1.80754518862972	-59502812939.9263\\
1.80764519112978	-59529168998.5023\\
1.80774519362984	-59556098014.8734\\
1.8078451961299	-59582454073.4494\\
1.80794519862997	-59609383089.8206\\
1.80804520113003	-59636312106.1917\\
1.80814520363009	-59662668164.7678\\
1.80824520613015	-59689597181.1389\\
1.80834520863022	-59716526197.51\\
1.80844521113028	-59742882256.0861\\
1.80854521363034	-59769811272.4572\\
1.8086452161304	-59796740288.8284\\
1.80874521863047	-59823669305.1995\\
1.80884522113053	-59850025363.7755\\
1.80894522363059	-59876954380.1467\\
1.80904522613065	-59903883396.5178\\
1.80914522863072	-59930812412.889\\
1.80924523113078	-59957168471.465\\
1.80934523363084	-59984097487.8361\\
1.8094452361309	-60011026504.2073\\
1.80954523863097	-60037955520.5784\\
1.80964524113103	-60064884536.9496\\
1.80974524363109	-60091240595.5256\\
1.80984524613115	-60118169611.8968\\
1.80994524863122	-60145098628.2679\\
1.81004525113128	-60172027644.6391\\
1.81014525363134	-60198956661.0102\\
1.8102452561314	-60225885677.3814\\
1.81034525863147	-60252814693.7525\\
1.81044526113153	-60279743710.1236\\
1.81054526363159	-60306099768.6997\\
1.81064526613165	-60333028785.0708\\
1.81074526863172	-60359957801.442\\
1.81084527113178	-60386886817.8131\\
1.81094527363184	-60413815834.1843\\
1.8110452761319	-60440744850.5554\\
1.81114527863197	-60467673866.9266\\
1.81124528113203	-60494602883.2977\\
1.81134528363209	-60521531899.6689\\
1.81144528613215	-60548460916.04\\
1.81154528863222	-60575389932.4112\\
1.81164529113228	-60602318948.7823\\
1.81174529363234	-60629820922.9486\\
1.8118452961324	-60656749939.3197\\
1.81194529863247	-60683678955.6909\\
1.81204530113253	-60710607972.062\\
1.81214530363259	-60737536988.4332\\
1.81224530613265	-60764466004.8043\\
1.81234530863272	-60791395021.1755\\
1.81244531113278	-60818324037.5466\\
1.81254531363284	-60845253053.9178\\
1.8126453161329	-60872755028.0841\\
1.81274531863297	-60899684044.4552\\
1.81284532113303	-60926613060.8263\\
1.81294532363309	-60953542077.1975\\
1.81304532613315	-60980471093.5686\\
1.81314532863322	-61007973067.7349\\
1.81324533113328	-61034902084.1061\\
1.81334533363334	-61061831100.4772\\
1.8134453361334	-61088760116.8484\\
1.81354533863347	-61116262091.0147\\
1.81364534113353	-61143191107.3858\\
1.81374534363359	-61170120123.757\\
1.81384534613365	-61197622097.9232\\
1.81394534863372	-61224551114.2944\\
1.81404535113378	-61251480130.6655\\
1.81414535363384	-61278409147.0367\\
1.8142453561339	-61305911121.203\\
1.81434535863397	-61332840137.5741\\
1.81444536113403	-61360342111.7404\\
1.81454536363409	-61387271128.1115\\
1.81464536613415	-61414200144.4827\\
1.81474536863422	-61441702118.649\\
1.81484537113428	-61468631135.0201\\
1.81494537363434	-61496133109.1864\\
1.8150453761344	-61523062125.5575\\
1.81514537863447	-61549991141.9287\\
1.81524538113453	-61577493116.095\\
1.81534538363459	-61604422132.4661\\
1.81544538613465	-61631924106.6324\\
1.81554538863472	-61658853123.0035\\
1.81564539113478	-61686355097.1698\\
1.81574539363484	-61713284113.541\\
1.8158453961349	-61740786087.7073\\
1.81594539863497	-61767715104.0784\\
1.81604540113503	-61795217078.2447\\
1.81614540363509	-61822719052.411\\
1.81624540613515	-61849648068.7821\\
1.81634540863522	-61877150042.9484\\
1.81644541113528	-61904079059.3195\\
1.81654541363534	-61931581033.4858\\
1.8166454161354	-61959083007.6521\\
1.81674541863547	-61986012024.0232\\
1.81684542113553	-62013513998.1895\\
1.81694542363559	-62041015972.3558\\
1.81704542613565	-62067944988.727\\
1.81714542863572	-62095446962.8932\\
1.81724543113578	-62122948937.0595\\
1.81734543363584	-62149877953.4307\\
1.8174454361359	-62177379927.5969\\
1.81754543863597	-62204881901.7632\\
1.81764544113603	-62231810918.1344\\
1.81774544363609	-62259312892.3006\\
1.81784544613615	-62286814866.4669\\
1.81794544863622	-62314316840.6332\\
1.81804545113628	-62341245857.0043\\
1.81814545363634	-62368747831.1706\\
1.8182454561364	-62396249805.3369\\
1.81834545863647	-62423751779.5032\\
1.81844546113653	-62451253753.6695\\
1.81854546363659	-62478755727.8357\\
1.81864546613665	-62505684744.2069\\
1.81874546863672	-62533186718.3732\\
1.81884547113678	-62560688692.5395\\
1.81894547363684	-62588190666.7057\\
1.8190454761369	-62615692640.872\\
1.81914547863697	-62643194615.0383\\
1.81924548113703	-62670696589.2046\\
1.81934548363709	-62698198563.3709\\
1.81944548613715	-62725700537.5371\\
1.81954548863722	-62753202511.7034\\
1.81964549113728	-62780131528.0746\\
1.81974549363734	-62807633502.2408\\
1.8198454961374	-62835135476.4071\\
1.81994549863747	-62862637450.5734\\
1.82004550113753	-62890139424.7397\\
1.82014550363759	-62917641398.906\\
1.82024550613765	-62945143373.0722\\
1.82034550863772	-62973218305.0337\\
1.82044551113778	-63000720279.1999\\
1.82054551363784	-63028222253.3662\\
1.8206455161379	-63055724227.5325\\
1.82074551863797	-63083226201.6988\\
1.82084552113803	-63110728175.8651\\
1.82094552363809	-63138230150.0313\\
1.82104552613815	-63165732124.1976\\
1.82114552863822	-63193234098.3639\\
1.82124553113828	-63220736072.5302\\
1.82134553363834	-63248811004.4916\\
1.8214455361384	-63276312978.6579\\
1.82154553863847	-63303814952.8241\\
1.82164554113853	-63331316926.9904\\
1.82174554363859	-63358818901.1567\\
1.82184554613865	-63386893833.1181\\
1.82194554863872	-63414395807.2844\\
1.82204555113878	-63441897781.4507\\
1.82214555363884	-63469399755.6169\\
1.8222455561389	-63496901729.7832\\
1.82234555863897	-63524976661.7446\\
1.82244556113903	-63552478635.9109\\
1.82254556363909	-63579980610.0772\\
1.82264556613915	-63608055542.0386\\
1.82274556863922	-63635557516.2049\\
1.82284557113928	-63663059490.3712\\
1.82294557363934	-63691134422.3326\\
1.8230455761394	-63718636396.4988\\
1.82314557863947	-63746138370.6651\\
1.82324558113953	-63774213302.6265\\
1.82334558363959	-63801715276.7928\\
1.82344558613965	-63829217250.9591\\
1.82354558863972	-63857292182.9205\\
1.82364559113978	-63884794157.0868\\
1.82374559363984	-63912869089.0482\\
1.8238455961399	-63940371063.2145\\
1.82394559863997	-63968445995.1759\\
1.82404560114003	-63995947969.3422\\
1.82414560364009	-64023449943.5084\\
1.82424560614015	-64051524875.4699\\
1.82434560864022	-64079026849.6361\\
1.82444561114028	-64107101781.5975\\
1.82454561364034	-64134603755.7638\\
1.8246456161404	-64162678687.7252\\
1.82474561864047	-64190753619.6867\\
1.82484562114053	-64218255593.8529\\
1.82494562364059	-64246330525.8143\\
1.82504562614065	-64273832499.9806\\
1.82514562864072	-64301907431.942\\
1.82524563114078	-64329409406.1083\\
1.82534563364084	-64357484338.0697\\
1.8254456361409	-64385559270.0311\\
1.82554563864097	-64413061244.1974\\
1.82564564114103	-64441136176.1588\\
1.82574564364109	-64469211108.1202\\
1.82584564614115	-64496713082.2865\\
1.82594564864122	-64524788014.2479\\
1.82604565114128	-64552862946.2093\\
1.82614565364134	-64580364920.3756\\
1.8262456561414	-64608439852.337\\
1.82634565864147	-64636514784.2984\\
1.82644566114153	-64664016758.4647\\
1.82654566364159	-64692091690.4261\\
1.82664566614165	-64720166622.3875\\
1.82674566864172	-64748241554.3489\\
1.82684567114178	-64776316486.3103\\
1.82694567364184	-64803818460.4766\\
1.8270456761419	-64831893392.438\\
1.82714567864197	-64859968324.3995\\
1.82724568114203	-64888043256.3609\\
1.82734568364209	-64916118188.3223\\
1.82744568614215	-64943620162.4885\\
1.82754568864222	-64971695094.45\\
1.82764569114228	-64999770026.4114\\
1.82774569364234	-65027844958.3728\\
1.8278456961424	-65055919890.3342\\
1.82794569864247	-65083994822.2956\\
1.82804570114253	-65112069754.257\\
1.82814570364259	-65140144686.2184\\
1.82824570614265	-65168219618.1798\\
1.82834570864272	-65196294550.1412\\
1.82844571114278	-65224369482.1027\\
1.82854571364284	-65252444414.0641\\
1.8286457161429	-65279946388.2303\\
1.82874571864297	-65308021320.1918\\
1.82884572114303	-65336096252.1532\\
1.82894572364309	-65364744141.9097\\
1.82904572614315	-65392819073.8711\\
1.82914572864322	-65420894005.8325\\
1.82924573114328	-65448968937.7939\\
1.82934573364334	-65477043869.7553\\
1.8294457361434	-65505118801.7168\\
1.82954573864347	-65533193733.6782\\
1.82964574114353	-65561268665.6396\\
1.82974574364359	-65589343597.601\\
1.82984574614365	-65617418529.5624\\
1.82994574864372	-65645493461.5238\\
1.83004575114378	-65673568393.4852\\
1.83014575364384	-65702216283.2418\\
1.8302457561439	-65730291215.2032\\
1.83034575864397	-65758366147.1646\\
1.83044576114403	-65786441079.126\\
1.83054576364409	-65814516011.0874\\
1.83064576614415	-65843163900.8439\\
1.83074576864422	-65871238832.8054\\
1.83084577114428	-65899313764.7668\\
1.83094577364434	-65927388696.7282\\
1.8310457761444	-65956036586.4847\\
1.83114577864447	-65984111518.4461\\
1.83124578114453	-66012186450.4075\\
1.83134578364459	-66040261382.3689\\
1.83144578614465	-66068909272.1255\\
1.83154578864472	-66096984204.0869\\
1.83164579114478	-66125059136.0483\\
1.83174579364484	-66153707025.8048\\
1.8318457961449	-66181781957.7663\\
1.83194579864497	-66209856889.7277\\
1.83204580114503	-66238504779.4842\\
1.83214580364509	-66266579711.4456\\
1.83224580614515	-66294654643.407\\
1.83234580864522	-66323302533.1636\\
1.83244581114528	-66351377465.125\\
1.83254581364534	-66380025354.8815\\
1.8326458161454	-66408100286.8429\\
1.83274581864547	-66436748176.5995\\
1.83284582114553	-66464823108.5609\\
1.83294582364559	-66493470998.3174\\
1.83304582614565	-66521545930.2788\\
1.83314582864572	-66550193820.0354\\
1.83324583114578	-66578268751.9968\\
1.83334583364584	-66606916641.7533\\
1.8334458361459	-66634991573.7147\\
1.83354583864597	-66663639463.4713\\
1.83364584114603	-66691714395.4327\\
1.83374584364609	-66720362285.1892\\
1.83384584614615	-66748437217.1506\\
1.83394584864622	-66777085106.9072\\
1.83404585114628	-66805160038.8686\\
1.83414585364634	-66833807928.6251\\
1.8342458561464	-66862455818.3817\\
1.83434585864647	-66890530750.3431\\
1.83444586114653	-66919178640.0996\\
1.83454586364659	-66947826529.8562\\
1.83464586614665	-66975901461.8176\\
1.83474586864672	-67004549351.5741\\
1.83484587114678	-67033197241.3307\\
1.83494587364684	-67061272173.2921\\
1.8350458761469	-67089920063.0486\\
1.83514587864697	-67118567952.8052\\
1.83524588114703	-67147215842.5617\\
1.83534588364709	-67175290774.5231\\
1.83544588614715	-67203938664.2796\\
1.83554588864722	-67232586554.0362\\
1.83564589114728	-67261234443.7927\\
1.83574589364734	-67289309375.7541\\
1.8358458961474	-67317957265.5107\\
1.83594589864747	-67346605155.2672\\
1.83604590114753	-67375253045.0238\\
1.83614590364759	-67403900934.7803\\
1.83624590614765	-67431975866.7417\\
1.83634590864772	-67460623756.4983\\
1.83644591114778	-67489271646.2548\\
1.83654591364784	-67517919536.0113\\
1.8366459161479	-67546567425.7679\\
1.83674591864797	-67575215315.5244\\
1.83684592114803	-67603863205.281\\
1.83694592364809	-67632511095.0375\\
1.83704592614815	-67661158984.794\\
1.83714592864822	-67689806874.5506\\
1.83724593114828	-67718454764.3071\\
1.83734593364834	-67747102654.0637\\
1.8374459361484	-67775750543.8202\\
1.83754593864847	-67804398433.5768\\
1.83764594114853	-67833046323.3333\\
1.83774594364859	-67861694213.0898\\
1.83784594614865	-67890342102.8464\\
1.83794594864872	-67918989992.6029\\
1.83804595114878	-67947637882.3595\\
1.83814595364884	-67976285772.116\\
1.8382459561489	-68004933661.8725\\
1.83834595864897	-68033581551.6291\\
1.83844596114903	-68062229441.3856\\
1.83854596364909	-68090877331.1422\\
1.83864596614915	-68119525220.8987\\
1.83874596864922	-68148173110.6552\\
1.83884597114928	-68177393958.2069\\
1.83894597364934	-68206041847.9635\\
1.8390459761494	-68234689737.72\\
1.83914597864947	-68263337627.4765\\
1.83924598114953	-68291985517.2331\\
1.83934598364959	-68320633406.9896\\
1.83944598614965	-68349854254.5413\\
1.83954598864972	-68378502144.2978\\
1.83964599114978	-68407150034.0544\\
1.83974599364984	-68435797923.8109\\
1.8398459961499	-68465018771.3626\\
1.83994599864997	-68493666661.1191\\
1.84004600115003	-68522314550.8757\\
1.84014600365009	-68550962440.6322\\
1.84024600615015	-68580183288.1839\\
1.84034600865022	-68608831177.9404\\
1.84044601115028	-68637479067.697\\
1.84054601365034	-68666699915.2486\\
1.8406460161504	-68695347805.0052\\
1.84074601865047	-68723995694.7617\\
1.84084602115053	-68753216542.3134\\
1.84094602365059	-68781864432.0699\\
1.84104602615065	-68811085279.6216\\
1.84114602865072	-68839733169.3781\\
1.84124603115078	-68868381059.1347\\
1.84134603365084	-68897601906.6864\\
1.8414460361509	-68926249796.4429\\
1.84154603865097	-68955470643.9946\\
1.84164604115103	-68984118533.7511\\
1.84174604365109	-69013339381.3028\\
1.84184604615115	-69041987271.0593\\
1.84194604865122	-69071208118.611\\
1.84204605115128	-69099856008.3675\\
1.84214605365134	-69129076855.9192\\
1.8422460561514	-69157724745.6758\\
1.84234605865147	-69186945593.2274\\
1.84244606115153	-69215593482.984\\
1.84254606365159	-69244814330.5356\\
1.84264606615165	-69273462220.2922\\
1.84274606865172	-69302683067.8439\\
1.84284607115178	-69331903915.3955\\
1.84294607365184	-69360551805.1521\\
1.8430460761519	-69389772652.7037\\
1.84314607865197	-69418420542.4603\\
1.84324608115203	-69447641390.0119\\
1.84334608365209	-69476862237.5636\\
1.84344608615215	-69505510127.3202\\
1.84354608865222	-69534730974.8718\\
1.84364609115228	-69563951822.4235\\
1.84374609365234	-69592599712.1801\\
1.8438460961524	-69621820559.7317\\
1.84394609865247	-69651041407.2834\\
1.84404610115253	-69680262254.8351\\
1.84414610365259	-69708910144.5916\\
1.84424610615265	-69738130992.1433\\
1.84434610865272	-69767351839.6949\\
1.84444611115278	-69796572687.2466\\
1.84454611365284	-69825220577.0032\\
1.8446461161529	-69854441424.5548\\
1.84474611865297	-69883662272.1065\\
1.84484612115303	-69912883119.6582\\
1.84494612365309	-69942103967.2099\\
1.84504612615315	-69971324814.7615\\
1.84514612865322	-69999972704.5181\\
1.84524613115328	-70029193552.0697\\
1.84534613365334	-70058414399.6214\\
1.8454461361534	-70087635247.1731\\
1.84554613865347	-70116856094.7248\\
1.84564614115353	-70146076942.2764\\
1.84574614365359	-70175297789.8281\\
1.84584614615365	-70204518637.3798\\
1.84594614865372	-70233739484.9314\\
1.84604615115378	-70262960332.4831\\
1.84614615365384	-70292181180.0348\\
1.8462461561539	-70321402027.5865\\
1.84634615865397	-70350622875.1381\\
1.84644616115403	-70379843722.6898\\
1.84654616365409	-70409064570.2415\\
1.84664616615415	-70438285417.7932\\
1.84674616865422	-70467506265.3448\\
1.84684617115428	-70496727112.8965\\
1.84694617365434	-70525947960.4482\\
1.8470461761544	-70555168807.9998\\
1.84714617865447	-70584389655.5515\\
1.84724618115453	-70613610503.1032\\
1.84734618365459	-70642831350.6548\\
1.84744618615465	-70672052198.2065\\
1.84754618865472	-70701273045.7582\\
1.84764619115478	-70731066851.105\\
1.84774619365484	-70760287698.6567\\
1.8478461961549	-70789508546.2083\\
1.84794619865497	-70818729393.76\\
1.84804620115503	-70847950241.3117\\
1.84814620365509	-70877171088.8634\\
1.84824620615515	-70906964894.2102\\
1.84834620865522	-70936185741.7618\\
1.84844621115528	-70965406589.3135\\
1.84854621365534	-70994627436.8652\\
1.8486462161554	-71024421242.212\\
1.84874621865547	-71053642089.7637\\
1.84884622115553	-71082862937.3153\\
1.84894622365559	-71112083784.867\\
1.84904622615565	-71141877590.2138\\
1.84914622865572	-71171098437.7655\\
1.84924623115578	-71200319285.3171\\
1.84934623365584	-71230113090.6639\\
1.8494462361559	-71259333938.2156\\
1.84954623865597	-71288554785.7673\\
1.84964624115603	-71318348591.1141\\
1.84974624365609	-71347569438.6658\\
1.84984624615615	-71377363244.0126\\
1.84994624865622	-71406584091.5642\\
1.85004625115628	-71435804939.1159\\
1.85014625365634	-71465598744.4627\\
1.8502462561564	-71494819592.0144\\
1.85034625865647	-71524613397.3612\\
1.85044626115653	-71553834244.9129\\
1.85054626365659	-71583628050.2597\\
1.85064626615665	-71612848897.8113\\
1.85074626865672	-71642642703.1581\\
1.85084627115678	-71671863550.7098\\
1.85094627365684	-71701657356.0566\\
1.8510462761569	-71730878203.6083\\
1.85114627865697	-71760672008.9551\\
1.85124628115703	-71789892856.5068\\
1.85134628365709	-71819686661.8536\\
1.85144628615715	-71848907509.4052\\
1.85154628865722	-71878701314.752\\
1.85164629115728	-71908495120.0988\\
1.85174629365734	-71937715967.6505\\
1.8518462961574	-71967509772.9973\\
1.85194629865747	-71996730620.549\\
1.85204630115753	-72026524425.8958\\
1.85214630365759	-72056318231.2426\\
1.85224630615765	-72085539078.7943\\
1.85234630865772	-72115332884.1411\\
1.85244631115778	-72145126689.4879\\
1.85254631365784	-72174920494.8347\\
1.8526463161579	-72204141342.3863\\
1.85274631865797	-72233935147.7331\\
1.85284632115803	-72263728953.08\\
1.85294632365809	-72292949800.6316\\
1.85304632615815	-72322743605.9784\\
1.85314632865822	-72352537411.3252\\
1.85324633115828	-72382331216.672\\
1.85334633365834	-72412125022.0188\\
1.8534463361584	-72441345869.5705\\
1.85354633865847	-72471139674.9173\\
1.85364634115853	-72500933480.2641\\
1.85374634365859	-72530727285.6109\\
1.85384634615865	-72560521090.9577\\
1.85394634865872	-72590314896.3045\\
1.85404635115878	-72619535743.8562\\
1.85414635365884	-72649329549.203\\
1.8542463561589	-72679123354.5498\\
1.85434635865897	-72708917159.8966\\
1.85444636115903	-72738710965.2434\\
1.85454636365909	-72768504770.5902\\
1.85464636615915	-72798298575.937\\
1.85474636865922	-72828092381.2838\\
1.85484637115928	-72857886186.6306\\
1.85494637365934	-72887679991.9774\\
1.8550463761594	-72917473797.3242\\
1.85514637865947	-72947267602.671\\
1.85524638115953	-72977061408.0178\\
1.85534638365959	-73006855213.3646\\
1.85544638615965	-73036649018.7114\\
1.85554638865972	-73066442824.0582\\
1.85564639115978	-73096236629.405\\
1.85574639365984	-73126030434.7518\\
1.8558463961599	-73155824240.0986\\
1.85594639865997	-73185618045.4454\\
1.85604640116003	-73215411850.7923\\
1.85614640366009	-73245205656.1391\\
1.85624640616015	-73274999461.4859\\
1.85634640866022	-73305366224.6278\\
1.85644641116028	-73335160029.9746\\
1.85654641366034	-73364953835.3214\\
1.8566464161604	-73394747640.6682\\
1.85674641866047	-73424541446.015\\
1.85684642116053	-73454335251.3618\\
1.85694642366059	-73484702014.5037\\
1.85704642616065	-73514495819.8505\\
1.85714642866072	-73544289625.1973\\
1.85724643116078	-73574083430.5441\\
1.85734643366084	-73603877235.8909\\
1.8574464361609	-73634243999.0329\\
1.85754643866097	-73664037804.3797\\
1.85764644116103	-73693831609.7265\\
1.85774644366109	-73724198372.8684\\
1.85784644616115	-73753992178.2152\\
1.85794644866122	-73783785983.562\\
1.85804645116128	-73813579788.9088\\
1.85814645366134	-73843946552.0508\\
1.8582464561614	-73873740357.3976\\
1.85834645866147	-73904107120.5395\\
1.85844646116153	-73933900925.8863\\
1.85854646366159	-73963694731.2331\\
1.85864646616165	-73994061494.375\\
1.85874646866172	-74023855299.7218\\
1.85884647116178	-74054222062.8638\\
1.85894647366184	-74084015868.2106\\
1.8590464761619	-74113809673.5574\\
1.85914647866197	-74144176436.6993\\
1.85924648116203	-74173970242.0461\\
1.85934648366209	-74204337005.188\\
1.85944648616215	-74234130810.5349\\
1.85954648866222	-74264497573.6768\\
1.85964649116228	-74294291379.0236\\
1.85974649366234	-74324658142.1655\\
1.8598464961624	-74354451947.5123\\
1.85994649866247	-74384818710.6543\\
1.86004650116253	-74414612516.0011\\
1.86014650366259	-74444979279.143\\
1.86024650616265	-74475346042.2849\\
1.86034650866272	-74505139847.6317\\
1.86044651116278	-74535506610.7737\\
1.86054651366284	-74565300416.1205\\
1.8606465161629	-74595667179.2624\\
1.86074651866297	-74626033942.4043\\
1.86084652116303	-74655827747.7511\\
1.86094652366309	-74686194510.8931\\
1.86104652616315	-74716561274.035\\
1.86114652866322	-74746355079.3818\\
1.86124653116328	-74776721842.5237\\
1.86134653366334	-74807088605.6657\\
1.8614465361634	-74836882411.0125\\
1.86154653866347	-74867249174.1544\\
1.86164654116353	-74897615937.2963\\
1.86174654366359	-74927409742.6431\\
1.86184654616365	-74957776505.7851\\
1.86194654866372	-74988143268.927\\
1.86204655116378	-75018510032.069\\
1.86214655366384	-75048876795.2109\\
1.8622465561639	-75078670600.5577\\
1.86234655866397	-75109037363.6996\\
1.86244656116403	-75139404126.8416\\
1.86254656366409	-75169770889.9835\\
1.86264656616415	-75200137653.1254\\
1.86274656866422	-75230504416.2673\\
1.86284657116428	-75260298221.6142\\
1.86294657366434	-75290664984.7561\\
1.8630465761644	-75321031747.898\\
1.86314657866447	-75351398511.04\\
1.86324658116453	-75381765274.1819\\
1.86334658366459	-75412132037.3238\\
1.86344658616465	-75442498800.4658\\
1.86354658866472	-75472865563.6077\\
1.86364659116478	-75503232326.7496\\
1.86374659366484	-75533599089.8916\\
1.8638465961649	-75563965853.0335\\
1.86394659866497	-75594332616.1754\\
1.86404660116503	-75624699379.3174\\
1.86414660366509	-75655066142.4593\\
1.86424660616515	-75685432905.6012\\
1.86434660866522	-75715799668.7432\\
1.86444661116528	-75746166431.8851\\
1.86454661366534	-75776533195.027\\
1.8646466161654	-75806899958.169\\
1.86474661866547	-75837266721.3109\\
1.86484662116553	-75867633484.4528\\
1.86494662366559	-75898000247.5948\\
1.86504662616565	-75928367010.7367\\
1.86514662866572	-75959306731.6738\\
1.86524663116578	-75989673494.8157\\
1.86534663366584	-76020040257.9576\\
1.8654466361659	-76050407021.0996\\
1.86554663866597	-76080773784.2415\\
1.86564664116603	-76111140547.3834\\
1.86574664366609	-76142080268.3205\\
1.86584664616615	-76172447031.4624\\
1.86594664866622	-76202813794.6044\\
1.86604665116628	-76233180557.7463\\
1.86614665366634	-76263547320.8882\\
1.8662466561664	-76294487041.8253\\
1.86634665866647	-76324853804.9672\\
1.86644666116653	-76355220568.1092\\
1.86654666366659	-76386160289.0462\\
1.86664666616665	-76416527052.1882\\
1.86674666866672	-76446893815.3301\\
1.86684667116678	-76477833536.2672\\
1.86694667366684	-76508200299.4091\\
1.8670466761669	-76538567062.551\\
1.86714667866697	-76569506783.4881\\
1.86724668116703	-76599873546.63\\
1.86734668366709	-76630240309.772\\
1.86744668616715	-76661180030.709\\
1.86754668866722	-76691546793.851\\
1.86764669116728	-76721913556.9929\\
1.86774669366734	-76752853277.9299\\
1.8678466961674	-76783220041.0719\\
1.86794669866747	-76814159762.0089\\
1.86804670116753	-76844526525.1509\\
1.86814670366759	-76875466246.088\\
1.86824670616765	-76905833009.2299\\
1.86834670866772	-76936772730.1669\\
1.86844671116778	-76967139493.3089\\
1.86854671366784	-76998079214.2459\\
1.8686467161679	-77028445977.3879\\
1.86874671866797	-77059385698.3249\\
1.86884672116803	-77089752461.4669\\
1.86894672366809	-77120692182.4039\\
1.86904672616815	-77151058945.5459\\
1.86914672866822	-77181998666.4829\\
1.86924673116828	-77212938387.42\\
1.86934673366834	-77243305150.5619\\
1.8694467361684	-77274244871.499\\
1.86954673866847	-77305184592.4361\\
1.86964674116853	-77335551355.578\\
1.86974674366859	-77366491076.5151\\
1.86984674616865	-77396857839.657\\
1.86994674866872	-77427797560.5941\\
1.87004675116878	-77458737281.5311\\
1.87014675366884	-77489677002.4682\\
1.8702467561689	-77520043765.6101\\
1.87034675866897	-77550983486.5472\\
1.87044676116903	-77581923207.4843\\
1.87054676366909	-77612289970.6262\\
1.87064676616915	-77643229691.5632\\
1.87074676866922	-77674169412.5003\\
1.87084677116928	-77705109133.4374\\
1.87094677366934	-77736048854.3744\\
1.8710467761694	-77766415617.5164\\
1.87114677866947	-77797355338.4534\\
1.87124678116953	-77828295059.3905\\
1.87134678366959	-77859234780.3276\\
1.87144678616965	-77890174501.2646\\
1.87154678866972	-77920541264.4066\\
1.87164679116978	-77951480985.3436\\
1.87174679366984	-77982420706.2807\\
1.8718467961699	-78013360427.2178\\
1.87194679866997	-78044300148.1548\\
1.87204680117003	-78075239869.0919\\
1.87214680367009	-78106179590.029\\
1.87224680617015	-78137119310.966\\
1.87234680867022	-78168059031.9031\\
1.87244681117028	-78198998752.8401\\
1.87254681367034	-78229938473.7772\\
1.8726468161704	-78260878194.7143\\
1.87274681867047	-78291817915.6513\\
1.87284682117053	-78322757636.5884\\
1.87294682367059	-78353697357.5255\\
1.87304682617065	-78384637078.4625\\
1.87314682867072	-78415576799.3996\\
1.87324683117078	-78446516520.3367\\
1.87334683367084	-78477456241.2737\\
1.8734468361709	-78508395962.2108\\
1.87354683867097	-78539335683.1479\\
1.87364684117103	-78570275404.0849\\
1.87374684367109	-78601215125.022\\
1.87384684617115	-78632154845.959\\
1.87394684867122	-78663094566.8961\\
1.87404685117128	-78694607245.6283\\
1.87414685367134	-78725546966.5654\\
1.8742468561714	-78756486687.5024\\
1.87434685867147	-78787426408.4395\\
1.87444686117153	-78818366129.3766\\
1.87454686367159	-78849305850.3136\\
1.87464686617165	-78880818529.0458\\
1.87474686867172	-78911758249.9829\\
1.87484687117178	-78942697970.92\\
1.87494687367184	-78973637691.857\\
1.8750468761719	-79005150370.5892\\
1.87514687867197	-79036090091.5263\\
1.87524688117203	-79067029812.4633\\
1.87534688367209	-79097969533.4004\\
1.87544688617215	-79129482212.1326\\
1.87554688867222	-79160421933.0697\\
1.87564689117228	-79191361654.0067\\
1.87574689367234	-79222874332.7389\\
1.8758468961724	-79253814053.676\\
1.87594689867247	-79284753774.6131\\
1.87604690117253	-79316266453.3452\\
1.87614690367259	-79347206174.2823\\
1.87624690617265	-79378145895.2194\\
1.87634690867272	-79409658573.9516\\
1.87644691117278	-79440598294.8886\\
1.87654691367284	-79472110973.6208\\
1.8766469161729	-79503050694.5579\\
1.87674691867297	-79534563373.2901\\
1.87684692117303	-79565503094.2272\\
1.87694692367309	-79596442815.1642\\
1.87704692617315	-79627955493.8964\\
1.87714692867322	-79658895214.8335\\
1.87724693117328	-79690407893.5657\\
1.87734693367334	-79721347614.5027\\
1.8774469361734	-79752860293.2349\\
1.87754693867347	-79784372971.9671\\
1.87764694117353	-79815312692.9042\\
1.87774694367359	-79846825371.6364\\
1.87784694617365	-79877765092.5735\\
1.87794694867372	-79909277771.3056\\
1.87804695117378	-79940217492.2427\\
1.87814695367384	-79971730170.9749\\
1.8782469561739	-80003242849.7071\\
1.87834695867397	-80034182570.6442\\
1.87844696117403	-80065695249.3764\\
1.87854696367409	-80096634970.3134\\
1.87864696617415	-80128147649.0456\\
1.87874696867422	-80159660327.7778\\
1.87884697117428	-80190600048.7149\\
1.87894697367434	-80222112727.4471\\
1.8790469761744	-80253625406.1793\\
1.87914697867447	-80285138084.9115\\
1.87924698117453	-80316077805.8485\\
1.87934698367459	-80347590484.5807\\
1.87944698617465	-80379103163.3129\\
1.87954698867472	-80410615842.0451\\
1.87964699117478	-80441555562.9822\\
1.87974699367484	-80473068241.7144\\
1.8798469961749	-80504580920.4466\\
1.87994699867497	-80536093599.1788\\
1.88004700117503	-80567606277.911\\
1.88014700367509	-80598545998.848\\
1.88024700617515	-80630058677.5802\\
1.88034700867522	-80661571356.3124\\
1.88044701117528	-80693084035.0446\\
1.88054701367534	-80724596713.7768\\
1.8806470161754	-80756109392.509\\
1.88074701867547	-80787622071.2412\\
1.88084702117553	-80818561792.1783\\
1.88094702367559	-80850074470.9105\\
1.88104702617565	-80881587149.6427\\
1.88114702867572	-80913099828.3749\\
1.88124703117578	-80944612507.1071\\
1.88134703367584	-80976125185.8392\\
1.8814470361759	-81007637864.5714\\
1.88154703867597	-81039150543.3036\\
1.88164704117603	-81070663222.0358\\
1.88174704367609	-81102175900.768\\
1.88184704617615	-81133688579.5002\\
1.88194704867622	-81165201258.2324\\
1.88204705117628	-81196713936.9646\\
1.88214705367634	-81228226615.6968\\
1.8822470561764	-81259739294.429\\
1.88234705867647	-81291251973.1612\\
1.88244706117653	-81323337609.6885\\
1.88254706367659	-81354850288.4207\\
1.88264706617665	-81386362967.1529\\
1.88274706867672	-81417875645.8851\\
1.88284707117678	-81449388324.6173\\
1.88294707367684	-81480901003.3495\\
1.8830470761769	-81512413682.0817\\
1.88314707867697	-81543926360.8139\\
1.88324708117703	-81576011997.3412\\
1.88334708367709	-81607524676.0734\\
1.88344708617715	-81639037354.8056\\
1.88354708867722	-81670550033.5378\\
1.88364709117728	-81702062712.27\\
1.88374709367734	-81734148348.7973\\
1.8838470961774	-81765661027.5295\\
1.88394709867747	-81797173706.2617\\
1.88404710117753	-81828686384.9939\\
1.88414710367759	-81860772021.5212\\
1.88424710617765	-81892284700.2534\\
1.88434710867772	-81923797378.9856\\
1.88444711117778	-81955883015.513\\
1.88454711367784	-81987395694.2451\\
1.8846471161779	-82018908372.9773\\
1.88474711867797	-82050994009.5047\\
1.88484712117803	-82082506688.2369\\
1.88494712367809	-82114019366.9691\\
1.88504712617815	-82146105003.4964\\
1.88514712867822	-82177617682.2286\\
1.88524713117828	-82209703318.7559\\
1.88534713367834	-82241215997.4881\\
1.8854471361784	-82272728676.2203\\
1.88554713867847	-82304814312.7476\\
1.88564714117853	-82336326991.4798\\
1.88574714367859	-82368412628.0071\\
1.88584714617865	-82399925306.7393\\
1.88594714867872	-82432010943.2667\\
1.88604715117878	-82463523621.9989\\
1.88614715367884	-82495609258.5262\\
1.8862471561789	-82527121937.2584\\
1.88634715867897	-82559207573.7857\\
1.88644716117903	-82590720252.5179\\
1.88654716367909	-82622805889.0452\\
1.88664716617915	-82654318567.7774\\
1.88674716867922	-82686404204.3047\\
1.88684717117928	-82718489840.8321\\
1.88694717367934	-82750002519.5643\\
1.8870471761794	-82782088156.0916\\
1.88714717867947	-82813600834.8238\\
1.88724718117953	-82845686471.3511\\
1.88734718367959	-82877772107.8784\\
1.88744718617965	-82909284786.6106\\
1.88754718867972	-82941370423.138\\
1.88764719117978	-82973456059.6653\\
1.88774719367984	-83004968738.3975\\
1.8878471961799	-83037054374.9248\\
1.88794719867997	-83069140011.4521\\
1.88804720118003	-83100652690.1843\\
1.88814720368009	-83132738326.7117\\
1.88824720618015	-83164823963.239\\
1.88834720868022	-83196909599.7663\\
1.88844721118028	-83228422278.4985\\
1.88854721368034	-83260507915.0258\\
1.8886472161804	-83292593551.5532\\
1.88874721868047	-83324679188.0805\\
1.88884722118053	-83356764824.6078\\
1.88894722368059	-83388277503.34\\
1.88904722618065	-83420363139.8673\\
1.88914722868072	-83452448776.3947\\
1.88924723118078	-83484534412.922\\
1.88934723368084	-83516620049.4493\\
1.8894472361809	-83548705685.9766\\
1.88954723868097	-83580791322.504\\
1.88964724118103	-83612304001.2362\\
1.88974724368109	-83644389637.7635\\
1.88984724618115	-83676475274.2908\\
1.88994724868122	-83708560910.8181\\
1.89004725118128	-83740646547.3455\\
1.89014725368134	-83772732183.8728\\
1.8902472561814	-83804817820.4001\\
1.89034725868147	-83836903456.9274\\
1.89044726118153	-83868989093.4548\\
1.89054726368159	-83901074729.9821\\
1.89064726618165	-83933160366.5094\\
1.89074726868172	-83965246003.0368\\
1.89084727118178	-83997331639.5641\\
1.89094727368184	-84029417276.0914\\
1.8910472761819	-84061502912.6187\\
1.89114727868197	-84093588549.1461\\
1.89124728118203	-84125674185.6734\\
1.89134728368209	-84157759822.2007\\
1.89144728618215	-84190418416.5232\\
1.89154728868222	-84222504053.0505\\
1.89164729118228	-84254589689.5778\\
1.89174729368234	-84286675326.1051\\
1.8918472961824	-84318760962.6325\\
1.89194729868247	-84350846599.1598\\
1.89204730118253	-84382932235.6871\\
1.89214730368259	-84415590830.0096\\
1.89224730618265	-84447676466.5369\\
1.89234730868272	-84479762103.0642\\
1.89244731118278	-84511847739.5916\\
1.89254731368284	-84543933376.1189\\
1.8926473161829	-84576591970.4413\\
1.89274731868297	-84608677606.9687\\
1.89284732118303	-84640763243.496\\
1.89294732368309	-84672848880.0233\\
1.89304732618315	-84705507474.3458\\
1.89314732868322	-84737593110.8731\\
1.89324733118328	-84769678747.4004\\
1.89334733368334	-84802337341.7229\\
1.8934473361834	-84834422978.2502\\
1.89354733868347	-84866508614.7775\\
1.89364734118353	-84899167209.1\\
1.89374734368359	-84931252845.6273\\
1.89384734618365	-84963338482.1546\\
1.89394734868372	-84995997076.4771\\
1.89404735118378	-85028082713.0044\\
1.89414735368384	-85060741307.3269\\
1.8942473561839	-85092826943.8542\\
1.89434735868397	-85124912580.3815\\
1.89444736118403	-85157571174.704\\
1.89454736368409	-85189656811.2313\\
1.89464736618415	-85222315405.5538\\
1.89474736868422	-85254401042.0811\\
1.89484737118428	-85287059636.4036\\
1.89494737368434	-85319145272.9309\\
1.8950473761844	-85351803867.2533\\
1.89514737868447	-85383889503.7807\\
1.89524738118453	-85416548098.1031\\
1.89534738368459	-85448633734.6304\\
1.89544738618465	-85481292328.9529\\
1.89554738868472	-85513377965.4802\\
1.89564739118478	-85546036559.8027\\
1.89574739368484	-85578695154.1252\\
1.8958473961849	-85610780790.6525\\
1.89594739868497	-85643439384.9749\\
1.89604740118503	-85675525021.5023\\
1.89614740368509	-85708183615.8247\\
1.89624740618515	-85740842210.1472\\
1.89634740868522	-85772927846.6745\\
1.89644741118528	-85805586440.997\\
1.89654741368534	-85838245035.3194\\
1.8966474161854	-85870330671.8467\\
1.89674741868547	-85902989266.1692\\
1.89684742118553	-85935647860.4917\\
1.89694742368559	-85967733497.019\\
1.89704742618565	-86000392091.3414\\
1.89714742868572	-86033050685.6639\\
1.89724743118578	-86065709279.9863\\
1.89734743368584	-86097794916.5137\\
1.8974474361859	-86130453510.8361\\
1.89754743868597	-86163112105.1586\\
1.89764744118603	-86195770699.481\\
1.89774744368609	-86228429293.8035\\
1.89784744618615	-86260514930.3308\\
1.89794744868622	-86293173524.6533\\
1.89804745118628	-86325832118.9757\\
1.89814745368634	-86358490713.2982\\
1.8982474561864	-86391149307.6207\\
1.89834745868647	-86423807901.9431\\
1.89844746118653	-86456466496.2656\\
1.89854746368659	-86489125090.588\\
1.89864746618665	-86521210727.1154\\
1.89874746868672	-86553869321.4378\\
1.89884747118678	-86586527915.7603\\
1.89894747368684	-86619186510.0827\\
1.8990474761869	-86651845104.4052\\
1.89914747868697	-86684503698.7276\\
1.89924748118703	-86717162293.0501\\
1.89934748368709	-86749820887.3726\\
1.89944748618715	-86782479481.695\\
1.89954748868722	-86815138076.0175\\
1.89964749118728	-86847796670.3399\\
1.89974749368734	-86880455264.6624\\
1.8998474961874	-86913113858.9848\\
1.89994749868747	-86945772453.3073\\
1.90004750118753	-86978431047.6297\\
1.90014750368759	-87011089641.9522\\
1.90024750618765	-87043748236.2747\\
1.90034750868772	-87076979788.3923\\
1.90044751118778	-87109638382.7147\\
1.90054751368784	-87142296977.0372\\
1.9006475161879	-87174955571.3596\\
1.90074751868797	-87207614165.6821\\
1.90084752118803	-87240272760.0045\\
1.90094752368809	-87272931354.327\\
1.90104752618815	-87306162906.4446\\
1.90114752868822	-87338821500.767\\
1.90124753118828	-87371480095.0895\\
1.90134753368834	-87404138689.412\\
1.9014475361884	-87436797283.7344\\
1.90154753868847	-87470028835.852\\
1.90164754118853	-87502687430.1745\\
1.90174754368859	-87535346024.4969\\
1.90184754618865	-87568004618.8194\\
1.90194754868872	-87601236170.937\\
1.90204755118878	-87633894765.2594\\
1.90214755368884	-87666553359.5819\\
1.9022475561889	-87699784911.6995\\
1.90234755868897	-87732443506.0219\\
1.90244756118903	-87765102100.3444\\
1.90254756368909	-87798333652.462\\
1.90264756618915	-87830992246.7844\\
1.90274756868922	-87863650841.1069\\
1.90284757118928	-87896882393.2245\\
1.90294757368934	-87929540987.5469\\
1.9030475761894	-87962772539.6645\\
1.90314757868947	-87995431133.987\\
1.90324758118953	-88028089728.3094\\
1.90334758368959	-88061321280.427\\
1.90344758618965	-88093979874.7495\\
1.90354758868972	-88127211426.8671\\
1.90364759118978	-88159870021.1895\\
1.90374759368984	-88193101573.3071\\
1.9038475961899	-88225760167.6295\\
1.90394759868997	-88258991719.7471\\
1.90404760119003	-88291650314.0696\\
1.90414760369009	-88324881866.1872\\
1.90424760619015	-88357540460.5096\\
1.90434760869022	-88390772012.6272\\
1.90444761119028	-88423430606.9497\\
1.90454761369034	-88456662159.0673\\
1.9046476161904	-88489320753.3897\\
1.90474761869047	-88522552305.5073\\
1.90484762119053	-88555783857.6249\\
1.90494762369059	-88588442451.9474\\
1.90504762619065	-88621674004.065\\
1.90514762869072	-88654905556.1825\\
1.90524763119078	-88687564150.505\\
1.90534763369084	-88720795702.6226\\
1.9054476361909	-88754027254.7402\\
1.90554763869097	-88786685849.0626\\
1.90564764119103	-88819917401.1802\\
1.90574764369109	-88853148953.2978\\
1.90584764619115	-88885807547.6203\\
1.90594764869122	-88919039099.7379\\
1.90604765119128	-88952270651.8554\\
1.90614765369134	-88985502203.973\\
1.9062476561914	-89018160798.2955\\
1.90634765869147	-89051392350.4131\\
1.90644766119153	-89084623902.5307\\
1.90654766369159	-89117855454.6482\\
1.90664766619165	-89150514048.9707\\
1.90674766869172	-89183745601.0883\\
1.90684767119178	-89216977153.2059\\
1.90694767369184	-89250208705.3235\\
1.9070476761919	-89283440257.4411\\
1.90714767869197	-89316671809.5586\\
1.90724768119203	-89349903361.6762\\
1.90734768369209	-89382561955.9987\\
1.90744768619215	-89415793508.1163\\
1.90754768869222	-89449025060.2339\\
1.90764769119228	-89482256612.3515\\
1.90774769369234	-89515488164.469\\
1.9078476961924	-89548719716.5866\\
1.90794769869247	-89581951268.7042\\
1.90804770119253	-89615182820.8218\\
1.90814770369259	-89648414372.9394\\
1.90824770619265	-89681645925.057\\
1.90834770869272	-89714877477.1746\\
1.90844771119278	-89748109029.2921\\
1.90854771369284	-89781340581.4097\\
1.9086477161929	-89814572133.5273\\
1.90874771869297	-89847803685.6449\\
1.90884772119303	-89881035237.7625\\
1.90894772369309	-89914266789.8801\\
1.90904772619315	-89947498341.9977\\
1.90914772869322	-89980729894.1153\\
1.90924773119328	-90013961446.2328\\
1.90934773369334	-90047192998.3504\\
1.9094477361934	-90080424550.468\\
1.90954773869347	-90114229060.3808\\
1.90964774119353	-90147460612.4983\\
1.90974774369359	-90180692164.6159\\
1.90984774619365	-90213923716.7335\\
1.90994774869372	-90247155268.8511\\
1.91004775119378	-90280386820.9687\\
1.91014775369384	-90314191330.8814\\
1.9102477561939	-90347422882.999\\
1.91034775869397	-90380654435.1166\\
1.91044776119403	-90413885987.2342\\
1.91054776369409	-90447117539.3518\\
1.91064776619415	-90480922049.2645\\
1.91074776869422	-90514153601.3821\\
1.91084777119428	-90547385153.4996\\
1.91094777369434	-90581189663.4124\\
1.9110477761944	-90614421215.53\\
1.91114777869447	-90647652767.6475\\
1.91124778119453	-90680884319.7651\\
1.91134778369459	-90714688829.6779\\
1.91144778619465	-90747920381.7954\\
1.91154778869472	-90781151933.913\\
1.91164779119478	-90814956443.8257\\
1.91174779369484	-90848187995.9433\\
1.9118477961949	-90881992505.856\\
1.91194779869497	-90915224057.9736\\
1.91204780119503	-90948455610.0912\\
1.91214780369509	-90982260120.0039\\
1.91224780619515	-91015491672.1215\\
1.91234780869522	-91049296182.0343\\
1.91244781119528	-91082527734.1518\\
1.91254781369534	-91116332244.0646\\
1.9126478161954	-91149563796.1821\\
1.91274781869547	-91183368306.0949\\
1.91284782119553	-91216599858.2124\\
1.91294782369559	-91250404368.1252\\
1.91304782619565	-91283635920.2428\\
1.91314782869572	-91317440430.1555\\
1.91324783119578	-91350671982.2731\\
1.91334783369584	-91384476492.1858\\
1.9134478361959	-91417708044.3034\\
1.91354783869597	-91451512554.2161\\
1.91364784119603	-91484744106.3337\\
1.91374784369609	-91518548616.2464\\
1.91384784619616	-91552353126.1591\\
1.91394784869622	-91585584678.2767\\
1.91404785119628	-91619389188.1894\\
1.91414785369634	-91652620740.307\\
1.9142478561964	-91686425250.2197\\
1.91434785869647	-91720229760.1324\\
1.91444786119653	-91753461312.25\\
1.91454786369659	-91787265822.1628\\
1.91464786619665	-91821070332.0755\\
1.91474786869672	-91854874841.9882\\
1.91484787119678	-91888106394.1058\\
1.91494787369684	-91921910904.0185\\
1.9150478761969	-91955715413.9312\\
1.91514787869697	-91988946966.0488\\
1.91524788119703	-92022751475.9615\\
1.91534788369709	-92056555985.8742\\
1.91544788619715	-92090360495.787\\
1.91554788869722	-92124165005.6997\\
1.91564789119728	-92157396557.8173\\
1.91574789369734	-92191201067.73\\
1.91584789619741	-92225005577.6427\\
1.91594789869747	-92258810087.5554\\
1.91604790119753	-92292614597.4681\\
1.91614790369759	-92326419107.3809\\
1.91624790619765	-92359650659.4984\\
1.91634790869772	-92393455169.4112\\
1.91644791119778	-92427259679.3239\\
1.91654791369784	-92461064189.2366\\
1.9166479161979	-92494868699.1493\\
1.91674791869797	-92528673209.062\\
1.91684792119803	-92562477718.9748\\
1.91694792369809	-92596282228.8875\\
1.91704792619816	-92630086738.8002\\
1.91714792869822	-92663891248.7129\\
1.91724793119828	-92697695758.6256\\
1.91734793369834	-92731500268.5383\\
1.9174479361984	-92765304778.4511\\
1.91754793869847	-92799109288.3638\\
1.91764794119853	-92832913798.2765\\
1.91774794369859	-92866718308.1892\\
1.91784794619866	-92900522818.1019\\
1.91794794869872	-92934327328.0147\\
1.91804795119878	-92968131837.9274\\
1.91814795369884	-93001936347.8401\\
1.9182479561989	-93035740857.7528\\
1.91834795869897	-93069545367.6655\\
1.91844796119903	-93103349877.5782\\
1.91854796369909	-93137727345.2861\\
1.91864796619915	-93171531855.1988\\
1.91874796869922	-93205336365.1115\\
1.91884797119928	-93239140875.0243\\
1.91894797369934	-93272945384.937\\
1.91904797619941	-93306749894.8497\\
1.91914797869947	-93341127362.5575\\
1.91924798119953	-93374931872.4703\\
1.91934798369959	-93408736382.383\\
1.91944798619965	-93442540892.2957\\
1.91954798869972	-93476345402.2084\\
1.91964799119978	-93510722869.9163\\
1.91974799369984	-93544527379.829\\
1.91984799619991	-93578331889.7417\\
1.91994799869997	-93612136399.6544\\
1.92004800120003	-93646513867.3623\\
1.92014800370009	-93680318377.275\\
1.92024800620016	-93714122887.1877\\
1.92034800870022	-93748500354.8956\\
1.92044801120028	-93782304864.8083\\
1.92054801370034	-93816109374.721\\
1.9206480162004	-93850486842.4288\\
1.92074801870047	-93884291352.3416\\
1.92084802120053	-93918095862.2543\\
1.92094802370059	-93952473329.9621\\
1.92104802620066	-93986277839.8748\\
1.92114802870072	-94020655307.5827\\
1.92124803120078	-94054459817.4954\\
1.92134803370084	-94088837285.2033\\
1.9214480362009	-94122641795.116\\
1.92154803870097	-94156446305.0287\\
1.92164804120103	-94190823772.7366\\
1.92174804370109	-94224628282.6493\\
1.92184804620116	-94259005750.3571\\
1.92194804870122	-94292810260.2698\\
1.92204805120128	-94327187727.9777\\
1.92214805370134	-94360992237.8904\\
1.92224805620141	-94395369705.5983\\
1.92234805870147	-94429174215.511\\
1.92244806120153	-94463551683.2188\\
1.92254806370159	-94497929150.9267\\
1.92264806620165	-94531733660.8394\\
1.92274806870172	-94566111128.5472\\
1.92284807120178	-94599915638.46\\
1.92294807370184	-94634293106.1678\\
1.92304807620191	-94668670573.8757\\
1.92314807870197	-94702475083.7884\\
1.92324808120203	-94736852551.4962\\
1.92334808370209	-94771230019.2041\\
1.92344808620216	-94805034529.1168\\
1.92354808870222	-94839411996.8246\\
1.92364809120228	-94873789464.5325\\
1.92374809370234	-94907593974.4452\\
1.92384809620241	-94941971442.1531\\
1.92394809870247	-94976348909.8609\\
1.92404810120253	-95010153419.7736\\
1.92414810370259	-95044530887.4815\\
1.92424810620266	-95078908355.1893\\
1.92434810870272	-95113285822.8972\\
1.92444811120278	-95147090332.8099\\
1.92454811370284	-95181467800.5177\\
1.9246481162029	-95215845268.2256\\
1.92474811870297	-95250222735.9334\\
1.92484812120303	-95284600203.6413\\
1.92494812370309	-95318977671.3492\\
1.92504812620316	-95352782181.2619\\
1.92514812870322	-95387159648.9697\\
1.92524813120328	-95421537116.6776\\
1.92534813370334	-95455914584.3854\\
1.92544813620341	-95490292052.0933\\
1.92554813870347	-95524669519.8011\\
1.92564814120353	-95559046987.509\\
1.92574814370359	-95593424455.2168\\
1.92584814620366	-95627801922.9247\\
1.92594814870372	-95661606432.8374\\
1.92604815120378	-95695983900.5452\\
1.92614815370384	-95730361368.2531\\
1.92624815620391	-95764738835.9609\\
1.92634815870397	-95799116303.6688\\
1.92644816120403	-95833493771.3766\\
1.92654816370409	-95867871239.0845\\
1.92664816620416	-95902248706.7923\\
1.92674816870422	-95936626174.5002\\
1.92684817120428	-95971003642.208\\
1.92694817370434	-96005381109.9159\\
1.92704817620441	-96040331535.4189\\
1.92714817870447	-96074709003.1267\\
1.92724818120453	-96109086470.8345\\
1.92734818370459	-96143463938.5424\\
1.92744818620466	-96177841406.2502\\
1.92754818870472	-96212218873.9581\\
1.92764819120478	-96246596341.666\\
1.92774819370484	-96280973809.3738\\
1.92784819620491	-96315351277.0816\\
1.92794819870497	-96350301702.5846\\
1.92804820120503	-96384679170.2925\\
1.92814820370509	-96419056638.0003\\
1.92824820620516	-96453434105.7082\\
1.92834820870522	-96487811573.416\\
1.92844821120528	-96522761998.919\\
1.92854821370534	-96557139466.6269\\
1.92864821620541	-96591516934.3347\\
1.92874821870547	-96625894402.0426\\
1.92884822120553	-96660844827.5455\\
1.92894822370559	-96695222295.2534\\
1.92904822620566	-96729599762.9612\\
1.92914822870572	-96763977230.6691\\
1.92924823120578	-96798927656.1721\\
1.92934823370584	-96833305123.8799\\
1.92944823620591	-96867682591.5878\\
1.92954823870597	-96902633017.0907\\
1.92964824120603	-96937010484.7986\\
1.92974824370609	-96971387952.5064\\
1.92984824620616	-97006338378.0094\\
1.92994824870622	-97040715845.7173\\
1.93004825120628	-97075666271.2202\\
1.93014825370634	-97110043738.9281\\
1.93024825620641	-97144421206.6359\\
1.93034825870647	-97179371632.1389\\
1.93044826120653	-97213749099.8468\\
1.93054826370659	-97248699525.3498\\
1.93064826620666	-97283076993.0576\\
1.93074826870672	-97318027418.5606\\
1.93084827120678	-97352404886.2684\\
1.93094827370684	-97387355311.7714\\
1.93104827620691	-97421732779.4793\\
1.93114827870697	-97456683204.9823\\
1.93124828120703	-97491060672.6901\\
1.93134828370709	-97526011098.1931\\
1.93144828620716	-97560388565.9009\\
1.93154828870722	-97595338991.4039\\
1.93164829120728	-97629716459.1118\\
1.93174829370734	-97664666884.6147\\
1.93184829620741	-97699044352.3226\\
1.93194829870747	-97733994777.8256\\
1.93204830120753	-97768945203.3286\\
1.93214830370759	-97803322671.0364\\
1.93224830620766	-97838273096.5394\\
1.93234830870772	-97873223522.0424\\
1.93244831120778	-97907600989.7502\\
1.93254831370784	-97942551415.2532\\
1.93264831620791	-97977501840.7562\\
1.93274831870797	-98011879308.464\\
1.93284832120803	-98046829733.967\\
1.93294832370809	-98081780159.47\\
1.93304832620816	-98116157627.1778\\
1.93314832870822	-98151108052.6808\\
1.93324833120828	-98186058478.1838\\
1.93334833370834	-98221008903.6868\\
1.93344833620841	-98255386371.3946\\
1.93354833870847	-98290336796.8976\\
1.93364834120853	-98325287222.4006\\
1.93374834370859	-98360237647.9035\\
1.93384834620866	-98395188073.4065\\
1.93394834870872	-98429565541.1144\\
1.93404835120878	-98464515966.6174\\
1.93414835370884	-98499466392.1203\\
1.93424835620891	-98534416817.6233\\
1.93434835870897	-98569367243.1263\\
1.93444836120903	-98604317668.6293\\
1.93454836370909	-98638695136.3371\\
1.93464836620916	-98673645561.8401\\
1.93474836870922	-98708595987.3431\\
1.93484837120928	-98743546412.8461\\
1.93494837370934	-98778496838.3491\\
1.93504837620941	-98813447263.852\\
1.93514837870947	-98848397689.355\\
1.93524838120953	-98883348114.858\\
1.93534838370959	-98918298540.361\\
1.93544838620966	-98953248965.864\\
1.93554838870972	-98988199391.3669\\
1.93564839120978	-99023149816.8699\\
1.93574839370984	-99058100242.3729\\
1.93584839620991	-99093050667.8759\\
1.93594839870997	-99128001093.3789\\
1.93604840121003	-99162951518.8818\\
1.93614840371009	-99197901944.3848\\
1.93624840621016	-99232852369.8878\\
1.93634840871022	-99267802795.3908\\
1.93644841121028	-99302753220.8938\\
1.93654841371034	-99337703646.3967\\
1.93664841621041	-99373227029.6949\\
1.93674841871047	-99408177455.1978\\
1.93684842121053	-99443127880.7008\\
1.93694842371059	-99478078306.2038\\
1.93704842621066	-99513028731.7068\\
1.93714842871072	-99547979157.2097\\
1.93724843121078	-99582929582.7127\\
1.93734843371084	-99618452966.0108\\
1.93744843621091	-99653403391.5138\\
1.93754843871097	-99688353817.0168\\
1.93764844121103	-99723304242.5198\\
1.93774844371109	-99758254668.0228\\
1.93784844621116	-99793778051.3209\\
1.93794844871122	-99828728476.8239\\
1.93804845121128	-99863678902.3268\\
1.93814845371134	-99899202285.6249\\
1.93824845621141	-99934152711.1279\\
1.93834845871147	-99969103136.6309\\
1.93844846121153	-100004053562.134\\
1.93854846371159	-100039576945.432\\
1.93864846621166	-100074527370.935\\
1.93874846871172	-100109477796.438\\
1.93884847121178	-100145001179.736\\
1.93894847371184	-100179951605.239\\
1.93904847621191	-100214902030.742\\
1.93914847871197	-100250425414.04\\
1.93924848121203	-100285375839.543\\
1.93934848371209	-100320899222.841\\
1.93944848621216	-100355849648.344\\
1.93954848871222	-100391373031.642\\
1.93964849121228	-100426323457.145\\
1.93974849371234	-100461273882.648\\
1.93984849621241	-100496797265.946\\
1.93994849871247	-100531747691.449\\
1.94004850121253	-100567271074.747\\
1.94014850371259	-100602221500.25\\
1.94024850621266	-100637744883.549\\
1.94034850871272	-100672695309.052\\
1.94044851121278	-100708218692.35\\
1.94054851371284	-100743169117.853\\
1.94064851621291	-100778692501.151\\
1.94074851871297	-100814215884.449\\
1.94084852121303	-100849166309.952\\
1.94094852371309	-100884689693.25\\
1.94104852621316	-100919640118.753\\
1.94114852871322	-100955163502.051\\
1.94124853121328	-100990113927.554\\
1.94134853371334	-101025637310.852\\
1.94144853621341	-101061160694.15\\
1.94154853871347	-101096111119.653\\
1.94164854121353	-101131634502.951\\
1.94174854371359	-101167157886.249\\
1.94184854621366	-101202108311.752\\
1.94194854871372	-101237631695.051\\
1.94204855121378	-101273155078.349\\
1.94214855371384	-101308105503.852\\
1.94224855621391	-101343628887.15\\
1.94234855871397	-101379152270.448\\
1.94244856121403	-101414675653.746\\
1.94254856371409	-101449626079.249\\
1.94264856621416	-101485149462.547\\
1.94274856871422	-101520672845.845\\
1.94284857121428	-101556196229.143\\
1.94294857371434	-101591719612.441\\
1.94304857621441	-101626670037.944\\
1.94314857871447	-101662193421.242\\
1.94324858121453	-101697716804.541\\
1.94334858371459	-101733240187.839\\
1.94344858621466	-101768763571.137\\
1.94354858871472	-101804286954.435\\
1.94364859121478	-101839237379.938\\
1.94374859371484	-101874760763.236\\
1.94384859621491	-101910284146.534\\
1.94394859871497	-101945807529.832\\
1.94404860121503	-101981330913.13\\
1.94414860371509	-102016854296.428\\
1.94424860621516	-102052377679.727\\
1.94434860871522	-102087901063.025\\
1.94444861121528	-102123424446.323\\
1.94454861371534	-102158947829.621\\
1.94464861621541	-102194471212.919\\
1.94474861871547	-102229994596.217\\
1.94484862121553	-102265517979.515\\
1.94494862371559	-102301041362.813\\
1.94504862621566	-102336564746.111\\
1.94514862871572	-102372088129.41\\
1.94524863121578	-102407611512.708\\
1.94534863371584	-102443134896.006\\
1.94544863621591	-102478658279.304\\
1.94554863871597	-102514181662.602\\
1.94564864121603	-102549705045.9\\
1.94574864371609	-102585228429.198\\
1.94584864621616	-102620751812.496\\
1.94594864871622	-102656275195.794\\
1.94604865121628	-102692371536.888\\
1.94614865371634	-102727894920.186\\
1.94624865621641	-102763418303.484\\
1.94634865871647	-102798941686.782\\
1.94644866121653	-102834465070.08\\
1.94654866371659	-102869988453.378\\
1.94664866621666	-102906084794.472\\
1.94674866871672	-102941608177.77\\
1.94684867121678	-102977131561.068\\
1.94694867371684	-103012654944.366\\
1.94704867621691	-103048178327.664\\
1.94714867871697	-103084274668.757\\
1.94724868121703	-103119798052.055\\
1.94734868371709	-103155321435.353\\
1.94744868621716	-103190844818.652\\
1.94754868871722	-103226941159.745\\
1.94764869121728	-103262464543.043\\
1.94774869371734	-103297987926.341\\
1.94784869621741	-103334084267.434\\
1.94794869871747	-103369607650.732\\
1.94804870121753	-103405131034.03\\
1.94814870371759	-103441227375.124\\
1.94824870621766	-103476750758.422\\
1.94834870871772	-103512274141.72\\
1.94844871121778	-103548370482.813\\
1.94854871371784	-103583893866.111\\
1.94864871621791	-103619990207.205\\
1.94874871871797	-103655513590.503\\
1.94884872121803	-103691609931.596\\
1.94894872371809	-103727133314.894\\
1.94904872621816	-103762656698.192\\
1.94914872871822	-103798753039.285\\
1.94924873121828	-103834276422.583\\
1.94934873371834	-103870372763.677\\
1.94944873621841	-103905896146.975\\
1.94954873871847	-103941992488.068\\
1.94964874121853	-103977515871.366\\
1.94974874371859	-104013612212.459\\
1.94984874621866	-104049135595.757\\
1.94994874871872	-104085231936.851\\
1.95004875121878	-104121328277.944\\
1.95014875371884	-104156851661.242\\
1.95024875621891	-104192948002.335\\
1.95034875871897	-104228471385.633\\
1.95044876121903	-104264567726.727\\
1.95054876371909	-104300664067.82\\
1.95064876621916	-104336187451.118\\
1.95074876871922	-104372283792.211\\
1.95084877121928	-104407807175.509\\
1.95094877371934	-104443903516.603\\
1.95104877621941	-104479999857.696\\
1.95114877871947	-104515523240.994\\
1.95124878121953	-104551619582.087\\
1.95134878371959	-104587715923.18\\
1.95144878621966	-104623812264.274\\
1.95154878871972	-104659335647.572\\
1.95164879121978	-104695431988.665\\
1.95174879371984	-104731528329.758\\
1.95184879621991	-104767051713.056\\
1.95194879871997	-104803148054.15\\
1.95204880122003	-104839244395.243\\
1.95214880372009	-104875340736.336\\
1.95224880622016	-104911437077.429\\
1.95234880872022	-104946960460.727\\
1.95244881122028	-104983056801.821\\
1.95254881372034	-105019153142.914\\
1.95264881622041	-105055249484.007\\
1.95274881872047	-105091345825.1\\
1.95284882122053	-105127442166.194\\
1.95294882372059	-105162965549.492\\
1.95304882622066	-105199061890.585\\
1.95314882872072	-105235158231.678\\
1.95324883122078	-105271254572.772\\
1.95334883372084	-105307350913.865\\
1.95344883622091	-105343447254.958\\
1.95354883872097	-105379543596.051\\
1.95364884122103	-105415639937.145\\
1.95374884372109	-105451736278.238\\
1.95384884622116	-105487832619.331\\
1.95394884872122	-105523928960.424\\
1.95404885122128	-105560025301.517\\
1.95414885372134	-105596121642.611\\
1.95424885622141	-105632217983.704\\
1.95434885872147	-105668314324.797\\
1.95444886122153	-105704410665.89\\
1.95454886372159	-105740507006.984\\
1.95464886622166	-105776603348.077\\
1.95474886872172	-105812699689.17\\
1.95484887122178	-105848796030.263\\
1.95494887372184	-105884892371.357\\
1.95504887622191	-105920988712.45\\
1.95514887872197	-105957085053.543\\
1.95524888122203	-105993181394.636\\
1.95534888372209	-106029850693.525\\
1.95544888622216	-106065947034.618\\
1.95554888872222	-106102043375.711\\
1.95564889122228	-106138139716.804\\
1.95574889372234	-106174236057.898\\
1.95584889622241	-106210332398.991\\
1.95594889872247	-106247001697.879\\
1.95604890122253	-106283098038.973\\
1.95614890372259	-106319194380.066\\
1.95624890622266	-106355290721.159\\
1.95634890872272	-106391387062.252\\
1.95644891122278	-106428056361.141\\
1.95654891372284	-106464152702.234\\
1.95664891622291	-106500249043.327\\
1.95674891872297	-106536345384.42\\
1.95684892122303	-106573014683.309\\
1.95694892372309	-106609111024.402\\
1.95704892622316	-106645207365.495\\
1.95714892872322	-106681876664.384\\
1.95724893122328	-106717973005.477\\
1.95734893372334	-106754069346.57\\
1.95744893622341	-106790738645.458\\
1.95754893872347	-106826834986.552\\
1.95764894122353	-106862931327.645\\
1.95774894372359	-106899600626.533\\
1.95784894622366	-106935696967.627\\
1.95794894872372	-106972366266.515\\
1.95804895122378	-107008462607.608\\
1.95814895372384	-107044558948.701\\
1.95824895622391	-107081228247.59\\
1.95834895872397	-107117324588.683\\
1.95844896122403	-107153993887.571\\
1.95854896372409	-107190090228.665\\
1.95864896622416	-107226759527.553\\
1.95874896872422	-107262855868.646\\
1.95884897122428	-107299525167.535\\
1.95894897372434	-107335621508.628\\
1.95904897622441	-107372290807.516\\
1.95914897872447	-107408387148.61\\
1.95924898122453	-107445056447.498\\
1.95934898372459	-107481152788.591\\
1.95944898622466	-107517822087.48\\
1.95954898872472	-107553918428.573\\
1.95964899122478	-107590587727.461\\
1.95974899372484	-107627257026.349\\
1.95984899622491	-107663353367.443\\
1.95994899872497	-107700022666.331\\
1.96004900122503	-107736691965.219\\
1.96014900372509	-107772788306.313\\
1.96024900622516	-107809457605.201\\
1.96034900872522	-107845553946.294\\
1.96044901122528	-107882223245.183\\
1.96054901372534	-107918892544.071\\
1.96064901622541	-107954988885.164\\
1.96074901872547	-107991658184.053\\
1.96084902122553	-108028327482.941\\
1.96094902372559	-108064996781.829\\
1.96104902622566	-108101093122.923\\
1.96114902872572	-108137762421.811\\
1.96124903122578	-108174431720.699\\
1.96134903372584	-108211101019.588\\
1.96144903622591	-108247197360.681\\
1.96154903872597	-108283866659.569\\
1.96164904122603	-108320535958.458\\
1.96174904372609	-108357205257.346\\
1.96184904622616	-108393874556.235\\
1.96194904872622	-108429970897.328\\
1.96204905122628	-108466640196.216\\
1.96214905372634	-108503309495.105\\
1.96224905622641	-108539978793.993\\
1.96234905872647	-108576648092.881\\
1.96244906122653	-108613317391.77\\
1.96254906372659	-108649986690.658\\
1.96264906622666	-108686083031.751\\
1.96274906872672	-108722752330.64\\
1.96284907122678	-108759421629.528\\
1.96294907372684	-108796090928.416\\
1.96304907622691	-108832760227.305\\
1.96314907872697	-108869429526.193\\
1.96324908122703	-108906098825.081\\
1.96334908372709	-108942768123.97\\
1.96344908622716	-108979437422.858\\
1.96354908872722	-109016106721.747\\
1.96364909122728	-109052776020.635\\
1.96374909372734	-109089445319.523\\
1.96384909622741	-109126114618.412\\
1.96394909872747	-109162783917.3\\
1.96404910122753	-109199453216.188\\
1.96414910372759	-109236122515.077\\
1.96424910622766	-109272791813.965\\
1.96434910872772	-109309461112.854\\
1.96444911122778	-109346130411.742\\
1.96454911372784	-109383372668.425\\
1.96464911622791	-109420041967.314\\
1.96474911872797	-109456711266.202\\
1.96484912122803	-109493380565.091\\
1.96494912372809	-109530049863.979\\
1.96504912622816	-109566719162.867\\
1.96514912872822	-109603388461.756\\
1.96524913122828	-109640630718.439\\
1.96534913372834	-109677300017.328\\
1.96544913622841	-109713969316.216\\
1.96554913872847	-109750638615.104\\
1.96564914122853	-109787307913.993\\
1.96574914372859	-109824550170.676\\
1.96584914622866	-109861219469.565\\
1.96594914872872	-109897888768.453\\
1.96604915122878	-109934558067.341\\
1.96614915372884	-109971800324.025\\
1.96624915622891	-110008469622.913\\
1.96634915872897	-110045138921.802\\
1.96644916122903	-110081808220.69\\
1.96654916372909	-110119050477.373\\
1.96664916622916	-110155719776.262\\
1.96674916872922	-110192389075.15\\
1.96684917122928	-110229631331.834\\
1.96694917372934	-110266300630.722\\
1.96704917622941	-110303542887.406\\
1.96714917872947	-110340212186.294\\
1.96724918122953	-110376881485.182\\
1.96734918372959	-110414123741.866\\
1.96744918622966	-110450793040.754\\
1.96754918872972	-110487462339.643\\
1.96764919122978	-110524704596.326\\
1.96774919372984	-110561373895.214\\
1.96784919622991	-110598616151.898\\
1.96794919872997	-110635285450.786\\
1.96804920123003	-110672527707.47\\
1.96814920373009	-110709197006.358\\
1.96824920623016	-110746439263.042\\
1.96834920873022	-110783108561.93\\
1.96844921123028	-110820350818.614\\
1.96854921373034	-110857020117.502\\
1.96864921623041	-110894262374.185\\
1.96874921873047	-110930931673.074\\
1.96884922123053	-110968173929.757\\
1.96894922373059	-111004843228.646\\
1.96904922623066	-111042085485.329\\
1.96914922873072	-111079327742.013\\
1.96924923123078	-111115997040.901\\
1.96934923373084	-111153239297.585\\
1.96944923623091	-111189908596.473\\
1.96954923873097	-111227150853.156\\
1.96964924123103	-111264393109.84\\
1.96974924373109	-111301062408.728\\
1.96984924623116	-111338304665.412\\
1.96994924873122	-111375546922.095\\
1.97004925123128	-111412216220.984\\
1.97014925373134	-111449458477.667\\
1.97024925623141	-111486700734.351\\
1.97034925873147	-111523370033.239\\
1.97044926123153	-111560612289.923\\
1.97054926373159	-111597854546.606\\
1.97064926623166	-111635096803.29\\
1.97074926873172	-111671766102.178\\
1.97084927123178	-111709008358.861\\
1.97094927373184	-111746250615.545\\
1.97104927623191	-111783492872.228\\
1.97114927873197	-111820162171.117\\
1.97124928123203	-111857404427.8\\
1.97134928373209	-111894646684.484\\
1.97144928623216	-111931888941.167\\
1.97154928873222	-111969131197.851\\
1.97164929123228	-112006373454.534\\
1.97174929373234	-112043042753.423\\
1.97184929623241	-112080285010.106\\
1.97194929873247	-112117527266.79\\
1.97204930123253	-112154769523.473\\
1.97214930373259	-112192011780.157\\
1.97224930623266	-112229254036.84\\
1.97234930873272	-112266496293.524\\
1.97244931123278	-112303738550.207\\
1.97254931373284	-112340980806.891\\
1.97264931623291	-112378223063.574\\
1.97274931873297	-112414892362.463\\
1.97284932123303	-112452134619.146\\
1.97294932373309	-112489376875.83\\
1.97304932623316	-112526619132.513\\
1.97314932873322	-112563861389.197\\
1.97324933123328	-112601103645.88\\
1.97334933373334	-112638345902.564\\
1.97344933623341	-112675588159.247\\
1.97354933873347	-112713403373.726\\
1.97364934123353	-112750645630.409\\
1.97374934373359	-112787887887.093\\
1.97384934623366	-112825130143.776\\
1.97394934873372	-112862372400.46\\
1.97404935123378	-112899614657.143\\
1.97414935373384	-112936856913.827\\
1.97424935623391	-112974099170.51\\
1.97434935873397	-113011341427.194\\
1.97444936123403	-113048583683.877\\
1.97454936373409	-113085825940.561\\
1.97464936623416	-113123641155.039\\
1.97474936873422	-113160883411.723\\
1.97484937123428	-113198125668.406\\
1.97494937373434	-113235367925.09\\
1.97504937623441	-113272610181.773\\
1.97514937873447	-113310425396.252\\
1.97524938123453	-113347667652.936\\
1.97534938373459	-113384909909.619\\
1.97544938623466	-113422152166.303\\
1.97554938873472	-113459967380.781\\
1.97564939123478	-113497209637.465\\
1.97574939373484	-113534451894.148\\
1.97584939623491	-113571694150.832\\
1.97594939873497	-113609509365.31\\
1.97604940123503	-113646751621.994\\
1.97614940373509	-113683993878.677\\
1.97624940623516	-113721809093.156\\
1.97634940873522	-113759051349.84\\
1.97644941123528	-113796293606.523\\
1.97654941373534	-113834108821.002\\
1.97664941623541	-113871351077.685\\
1.97674941873547	-113908593334.369\\
1.97684942123553	-113946408548.847\\
1.97694942373559	-113983650805.531\\
1.97704942623566	-114020893062.214\\
1.97714942873572	-114058708276.693\\
1.97724943123578	-114095950533.376\\
1.97734943373584	-114133765747.855\\
1.97744943623591	-114171008004.539\\
1.97754943873597	-114208823219.017\\
1.97764944123603	-114246065475.701\\
1.97774944373609	-114283880690.179\\
1.97784944623616	-114321122946.863\\
1.97794944873622	-114358938161.342\\
1.97804945123628	-114396180418.025\\
1.97814945373634	-114433995632.504\\
1.97824945623641	-114471237889.187\\
1.97834945873647	-114509053103.666\\
1.97844946123653	-114546295360.349\\
1.97854946373659	-114584110574.828\\
1.97864946623666	-114621352831.511\\
1.97874946873672	-114659168045.99\\
1.97884947123678	-114696410302.674\\
1.97894947373684	-114734225517.152\\
1.97904947623691	-114772040731.631\\
1.97914947873697	-114809282988.314\\
1.97924948123703	-114847098202.793\\
1.97934948373709	-114884913417.272\\
1.97944948623716	-114922155673.955\\
1.97954948873722	-114959970888.434\\
1.97964949123728	-114997786102.912\\
1.97974949373734	-115035028359.596\\
1.97984949623741	-115072843574.075\\
1.97994949873747	-115110658788.553\\
1.98004950123753	-115147901045.237\\
1.98014950373759	-115185716259.715\\
1.98024950623766	-115223531474.194\\
1.98034950873772	-115261346688.673\\
1.98044951123778	-115298588945.356\\
1.98054951373784	-115336404159.835\\
1.98064951623791	-115374219374.313\\
1.98074951873797	-115412034588.792\\
1.98084952123803	-115449276845.475\\
1.98094952373809	-115487092059.954\\
1.98104952623816	-115524907274.433\\
1.98114952873822	-115562722488.911\\
1.98124953123828	-115600537703.39\\
1.98134953373834	-115638352917.869\\
1.98144953623841	-115675595174.552\\
1.98154953873847	-115713410389.031\\
1.98164954123853	-115751225603.509\\
1.98174954373859	-115789040817.988\\
1.98184954623866	-115826856032.467\\
1.98194954873872	-115864671246.945\\
1.98204955123878	-115902486461.424\\
1.98214955373884	-115940301675.903\\
1.98224955623891	-115978116890.381\\
1.98234955873897	-116015932104.86\\
1.98244956123903	-116053174361.543\\
1.98254956373909	-116090989576.022\\
1.98264956623916	-116128804790.501\\
1.98274956873922	-116166620004.979\\
1.98284957123928	-116204435219.458\\
1.98294957373934	-116242250433.937\\
1.98304957623941	-116280065648.415\\
1.98314957873947	-116317880862.894\\
1.98324958123953	-116356269035.168\\
1.98334958373959	-116394084249.646\\
1.98344958623966	-116431899464.125\\
1.98354958873972	-116469714678.603\\
1.98364959123978	-116507529893.082\\
1.98374959373984	-116545345107.561\\
1.98384959623991	-116583160322.039\\
1.98394959873997	-116620975536.518\\
1.98404960124003	-116658790750.997\\
1.98414960374009	-116696605965.475\\
1.98424960624016	-116734421179.954\\
1.98434960874022	-116772809352.228\\
1.98444961124028	-116810624566.706\\
1.98454961374034	-116848439781.185\\
1.98464961624041	-116886254995.664\\
1.98474961874047	-116924070210.142\\
1.98484962124053	-116962458382.416\\
1.98494962374059	-117000273596.895\\
1.98504962624066	-117038088811.373\\
1.98514962874072	-117075904025.852\\
1.98524963124078	-117113719240.331\\
1.98534963374084	-117152107412.604\\
1.98544963624091	-117189922627.083\\
1.98554963874097	-117227737841.562\\
1.98564964124103	-117265553056.04\\
1.98574964374109	-117303941228.314\\
1.98584964624116	-117341756442.793\\
1.98594964874122	-117379571657.271\\
1.98604965124128	-117417959829.545\\
1.98614965374134	-117455775044.024\\
1.98624965624141	-117493590258.502\\
1.98634965874147	-117531978430.776\\
1.98644966124153	-117569793645.255\\
1.98654966374159	-117608181817.528\\
1.98664966624166	-117645997032.007\\
1.98674966874172	-117683812246.486\\
1.98684967124178	-117722200418.759\\
1.98694967374184	-117760015633.238\\
1.98704967624191	-117798403805.512\\
1.98714967874197	-117836219019.99\\
1.98724968124203	-117874034234.469\\
1.98734968374209	-117912422406.743\\
1.98744968624216	-117950237621.222\\
1.98754968874222	-117988625793.495\\
1.98764969124228	-118026441007.974\\
1.98774969374234	-118064829180.248\\
1.98784969624241	-118102644394.726\\
1.98794969874247	-118141032567\\
1.98804970124253	-118178847781.479\\
1.98814970374259	-118217235953.753\\
1.98824970624266	-118255051168.231\\
1.98834970874272	-118293439340.505\\
1.98844971124278	-118331827512.779\\
1.98854971374284	-118369642727.257\\
1.98864971624291	-118408030899.531\\
1.98874971874297	-118445846114.01\\
1.98884972124303	-118484234286.283\\
1.98894972374309	-118522622458.557\\
1.98904972624316	-118560437673.036\\
1.98914972874322	-118598825845.31\\
1.98924973124328	-118636641059.788\\
1.98934973374334	-118675029232.062\\
1.98944973624341	-118713417404.336\\
1.98954973874347	-118751232618.814\\
1.98964974124353	-118789620791.088\\
1.98974974374359	-118828008963.362\\
1.98984974624366	-118866397135.636\\
1.98994974874372	-118904212350.114\\
1.99004975124378	-118942600522.388\\
1.99014975374384	-118980988694.662\\
1.99024975624391	-119018803909.141\\
1.99034975874397	-119057192081.414\\
1.99044976124403	-119095580253.688\\
1.99054976374409	-119133968425.962\\
1.99064976624416	-119172356598.236\\
1.99074976874422	-119210171812.714\\
1.99084977124428	-119248559984.988\\
1.99094977374434	-119286948157.262\\
1.99104977624441	-119325336329.536\\
1.99114977874447	-119363724501.809\\
1.99124978124453	-119402112674.083\\
1.99134978374459	-119439927888.562\\
1.99144978624466	-119478316060.835\\
1.99154978874472	-119516704233.109\\
1.99164979124478	-119555092405.383\\
1.99174979374484	-119593480577.657\\
1.99184979624491	-119631868749.93\\
1.99194979874497	-119670256922.204\\
1.99204980124503	-119708645094.478\\
1.99214980374509	-119747033266.752\\
1.99224980624516	-119785421439.026\\
1.99234980874522	-119823809611.299\\
1.99244981124528	-119861624825.778\\
1.99254981374534	-119900012998.052\\
1.99264981624541	-119938401170.325\\
1.99274981874547	-119976789342.599\\
1.99284982124553	-120015177514.873\\
1.99294982374559	-120053565687.147\\
1.99304982624566	-120092526817.216\\
1.99314982874572	-120130914989.489\\
1.99324983124578	-120169303161.763\\
1.99334983374584	-120207691334.037\\
1.99344983624591	-120246079506.311\\
1.99354983874597	-120284467678.585\\
1.99364984124603	-120322855850.858\\
1.99374984374609	-120361244023.132\\
1.99384984624616	-120399632195.406\\
1.99394984874622	-120438020367.68\\
1.99404985124628	-120476408539.953\\
1.99414985374634	-120514796712.227\\
1.99424985624641	-120553757842.296\\
1.99434985874647	-120592146014.57\\
1.99444986124653	-120630534186.844\\
1.99454986374659	-120668922359.117\\
1.99464986624666	-120707310531.391\\
1.99474986874672	-120746271661.46\\
1.99484987124678	-120784659833.734\\
1.99494987374684	-120823048006.007\\
1.99504987624691	-120861436178.281\\
1.99514987874697	-120899824350.555\\
1.99524988124703	-120938785480.624\\
1.99534988374709	-120977173652.898\\
1.99544988624716	-121015561825.171\\
1.99554988874722	-121053949997.445\\
1.99564989124728	-121092911127.514\\
1.99574989374734	-121131299299.788\\
1.99584989624741	-121169687472.062\\
1.99594989874747	-121208648602.131\\
1.99604990124753	-121247036774.404\\
1.99614990374759	-121285424946.678\\
1.99624990624766	-121324386076.747\\
1.99634990874772	-121362774249.021\\
1.99644991124778	-121401162421.294\\
1.99654991374784	-121440123551.363\\
1.99664991624791	-121478511723.637\\
1.99674991874797	-121517472853.706\\
1.99684992124803	-121555861025.98\\
1.99694992374809	-121594249198.254\\
1.99704992624816	-121633210328.322\\
1.99714992874822	-121671598500.596\\
1.99724993124828	-121710559630.665\\
1.99734993374834	-121748947802.939\\
1.99744993624841	-121787908933.008\\
1.99754993874847	-121826297105.282\\
1.99764994124853	-121865258235.35\\
1.99774994374859	-121903646407.624\\
1.99784994624866	-121942607537.693\\
1.99794994874872	-121980995709.967\\
1.99804995124878	-122019956840.036\\
1.99814995374884	-122058345012.31\\
1.99824995624891	-122097306142.378\\
1.99834995874897	-122135694314.652\\
1.99844996124903	-122174655444.721\\
1.99854996374909	-122213043616.995\\
1.99864996624916	-122252004747.064\\
1.99874996874922	-122290965877.133\\
1.99884997124928	-122329354049.406\\
1.99894997374934	-122368315179.475\\
1.99904997624941	-122407276309.544\\
1.99914997874947	-122445664481.818\\
1.99924998124953	-122484625611.887\\
1.99934998374959	-122523013784.161\\
1.99944998624966	-122561974914.23\\
1.99954998874972	-122600936044.298\\
1.99964999124978	-122639324216.572\\
1.99974999374984	-122678285346.641\\
1.99984999624991	-122717246476.71\\
1.99994999874997	-122756207606.779\\
2.00005000125003	-122794595779.053\\
};
\addplot [color=mycolor3,solid,forget plot]
  table[row sep=crcr]{%
2.00005000125003	-122794595779.053\\
2.00015000375009	-122833556909.122\\
2.00025000625016	-122872518039.19\\
2.00035000875022	-122911479169.259\\
2.00045001125028	-122949867341.533\\
2.00055001375034	-122988828471.602\\
2.00065001625041	-123027789601.671\\
2.00075001875047	-123066750731.74\\
2.00085002125053	-123105138904.014\\
2.00095002375059	-123144100034.082\\
2.00105002625066	-123183061164.151\\
2.00115002875072	-123222022294.22\\
2.00125003125078	-123260983424.289\\
2.00135003375084	-123299944554.358\\
2.00145003625091	-123338905684.427\\
2.00155003875097	-123377293856.701\\
2.00165004125103	-123416254986.77\\
2.00175004375109	-123455216116.838\\
2.00185004625116	-123494177246.907\\
2.00195004875122	-123533138376.976\\
2.00205005125128	-123572099507.045\\
2.00215005375134	-123611060637.114\\
2.00225005625141	-123650021767.183\\
2.00235005875147	-123688982897.252\\
2.00245006125153	-123727944027.321\\
2.00255006375159	-123766905157.39\\
2.00265006625166	-123805866287.459\\
2.00275006875172	-123844827417.527\\
2.00285007125178	-123883788547.596\\
2.00295007375184	-123922749677.665\\
2.00305007625191	-123961710807.734\\
2.00315007875197	-124000671937.803\\
2.00325008125203	-124039633067.872\\
2.00335008375209	-124078594197.941\\
2.00345008625216	-124117555328.01\\
2.00355008875222	-124156516458.079\\
2.00365009125228	-124195477588.148\\
2.00375009375234	-124234438718.216\\
2.00385009625241	-124273399848.285\\
2.00395009875247	-124312360978.354\\
2.00405010125253	-124351322108.423\\
2.00415010375259	-124390856196.287\\
2.00425010625266	-124429817326.356\\
2.00435010875272	-124468778456.425\\
2.00445011125278	-124507739586.494\\
2.00455011375284	-124546700716.563\\
2.00465011625291	-124585661846.632\\
2.00475011875297	-124625195934.496\\
2.00485012125303	-124664157064.565\\
2.00495012375309	-124703118194.633\\
2.00505012625316	-124742079324.702\\
2.00515012875322	-124781040454.771\\
2.00525013125328	-124820574542.635\\
2.00535013375334	-124859535672.704\\
2.00545013625341	-124898496802.773\\
2.00555013875347	-124937457932.842\\
2.00565014125353	-124976992020.706\\
2.00575014375359	-125015953150.775\\
2.00585014625366	-125054914280.844\\
2.00595014875372	-125094448368.708\\
2.00605015125378	-125133409498.777\\
2.00615015375384	-125172370628.846\\
2.00625015625391	-125211904716.71\\
2.00635015875397	-125250865846.778\\
2.00645016125403	-125289826976.847\\
2.00655016375409	-125329361064.711\\
2.00665016625416	-125368322194.78\\
2.00675016875422	-125407283324.849\\
2.00685017125428	-125446817412.713\\
2.00695017375434	-125485778542.782\\
2.00705017625441	-125525312630.646\\
2.00715017875447	-125564273760.715\\
2.00725018125453	-125603234890.784\\
2.00735018375459	-125642768978.648\\
2.00745018625466	-125681730108.717\\
2.00755018875472	-125721264196.581\\
2.00765019125478	-125760225326.65\\
2.00775019375484	-125799759414.514\\
2.00785019625491	-125838720544.583\\
2.00795019875497	-125878254632.447\\
2.00805020125503	-125917215762.516\\
2.00815020375509	-125956749850.38\\
2.00825020625516	-125995710980.449\\
2.00835020875522	-126035245068.313\\
2.00845021125528	-126074206198.381\\
2.00855021375534	-126113740286.245\\
2.00865021625541	-126152701416.314\\
2.00875021875547	-126192235504.178\\
2.00885022125553	-126231769592.042\\
2.00895022375559	-126270730722.111\\
2.00905022625566	-126310264809.975\\
2.00915022875572	-126349225940.044\\
2.00925023125578	-126388760027.908\\
2.00935023375584	-126428294115.772\\
2.00945023625591	-126467255245.841\\
2.00955023875597	-126506789333.705\\
2.00965024125603	-126546323421.569\\
2.00975024375609	-126585284551.638\\
2.00985024625616	-126624818639.502\\
2.00995024875622	-126664352727.366\\
2.01005025125628	-126703313857.435\\
2.01015025375634	-126742847945.299\\
2.01025025625641	-126782382033.163\\
2.01035025875647	-126821916121.027\\
2.01045026125653	-126860877251.096\\
2.01055026375659	-126900411338.96\\
2.01065026625666	-126939945426.824\\
2.01075026875672	-126979479514.688\\
2.01085027125678	-127018440644.757\\
2.01095027375684	-127057974732.621\\
2.01105027625691	-127097508820.485\\
2.01115027875697	-127137042908.349\\
2.01125028125703	-127176576996.213\\
2.01135028375709	-127215538126.282\\
2.01145028625716	-127255072214.146\\
2.01155028875722	-127294606302.01\\
2.01165029125728	-127334140389.874\\
2.01175029375734	-127373674477.738\\
2.01185029625741	-127413208565.602\\
2.01195029875747	-127452742653.466\\
2.01205030125753	-127491703783.535\\
2.01215030375759	-127531237871.399\\
2.01225030625766	-127570771959.263\\
2.01235030875772	-127610306047.127\\
2.01245031125778	-127649840134.991\\
2.01255031375784	-127689374222.855\\
2.01265031625791	-127728908310.719\\
2.01275031875797	-127768442398.583\\
2.01285032125803	-127807976486.447\\
2.01295032375809	-127847510574.311\\
2.01305032625816	-127887044662.175\\
2.01315032875822	-127926578750.039\\
2.01325033125828	-127966112837.903\\
2.01335033375834	-128005646925.767\\
2.01345033625841	-128045181013.632\\
2.01355033875847	-128084715101.496\\
2.01365034125853	-128124249189.36\\
2.01375034375859	-128163783277.224\\
2.01385034625866	-128203317365.088\\
2.01395034875872	-128242851452.952\\
2.01405035125878	-128282385540.816\\
2.01415035375884	-128321919628.68\\
2.01425035625891	-128361453716.544\\
2.01435035875897	-128401560762.203\\
2.01445036125903	-128441094850.067\\
2.01455036375909	-128480628937.931\\
2.01465036625916	-128520163025.795\\
2.01475036875922	-128559697113.659\\
2.01485037125928	-128599231201.523\\
2.01495037375934	-128638765289.387\\
2.01505037625941	-128678872335.046\\
2.01515037875947	-128718406422.91\\
2.01525038125953	-128757940510.774\\
2.01535038375959	-128797474598.638\\
2.01545038625966	-128837008686.502\\
2.01555038875972	-128877115732.161\\
2.01565039125978	-128916649820.025\\
2.01575039375984	-128956183907.89\\
2.01585039625991	-128995717995.754\\
2.01595039875997	-129035825041.413\\
2.01605040126003	-129075359129.277\\
2.01615040376009	-129114893217.141\\
2.01625040626016	-129155000262.8\\
2.01635040876022	-129194534350.664\\
2.01645041126028	-129234068438.528\\
2.01655041376034	-129274175484.187\\
2.01665041626041	-129313709572.051\\
2.01675041876047	-129353243659.915\\
2.01685042126053	-129393350705.574\\
2.01695042376059	-129432884793.438\\
2.01705042626066	-129472418881.302\\
2.01715042876072	-129512525926.962\\
2.01725043126078	-129552060014.826\\
2.01735043376084	-129591594102.69\\
2.01745043626091	-129631701148.349\\
2.01755043876097	-129671235236.213\\
2.01765044126103	-129711342281.872\\
2.01775044376109	-129750876369.736\\
2.01785044626116	-129790983415.395\\
2.01795044876122	-129830517503.259\\
2.01805045126128	-129870051591.123\\
2.01815045376134	-129910158636.782\\
2.01825045626141	-129949692724.646\\
2.01835045876147	-129989799770.306\\
2.01845046126153	-130029333858.17\\
2.01855046376159	-130069440903.829\\
2.01865046626166	-130108974991.693\\
2.01875046876172	-130149082037.352\\
2.01885047126178	-130189189083.011\\
2.01895047376184	-130228723170.875\\
2.01905047626191	-130268830216.534\\
2.01915047876197	-130308364304.398\\
2.01925048126203	-130348471350.057\\
2.01935048376209	-130388005437.921\\
2.01945048626216	-130428112483.581\\
2.01955048876222	-130468219529.24\\
2.01965049126228	-130507753617.104\\
2.01975049376234	-130547860662.763\\
2.01985049626241	-130587394750.627\\
2.01995049876247	-130627501796.286\\
2.02005050126253	-130667608841.945\\
2.02015050376259	-130707142929.809\\
2.02025050626266	-130747249975.468\\
2.02035050876272	-130787357021.128\\
2.02045051126278	-130826891108.992\\
2.02055051376284	-130866998154.651\\
2.02065051626291	-130907105200.31\\
2.02075051876297	-130947212245.969\\
2.02085052126303	-130986746333.833\\
2.02095052376309	-131026853379.492\\
2.02105052626316	-131066960425.151\\
2.02115052876322	-131107067470.811\\
2.02125053126328	-131146601558.675\\
2.02135053376334	-131186708604.334\\
2.02145053626341	-131226815649.993\\
2.02155053876347	-131266922695.652\\
2.02165054126353	-131307029741.311\\
2.02175054376359	-131346563829.175\\
2.02185054626366	-131386670874.834\\
2.02195054876372	-131426777920.494\\
2.02205055126378	-131466884966.153\\
2.02215055376384	-131506992011.812\\
2.02225055626391	-131547099057.471\\
2.02235055876397	-131586633145.335\\
2.02245056126403	-131626740190.994\\
2.02255056376409	-131666847236.653\\
2.02265056626416	-131706954282.313\\
2.02275056876422	-131747061327.972\\
2.02285057126428	-131787168373.631\\
2.02295057376434	-131827275419.29\\
2.02305057626441	-131867382464.949\\
2.02315057876447	-131907489510.608\\
2.02325058126453	-131947596556.268\\
2.02335058376459	-131987703601.927\\
2.02345058626466	-132027810647.586\\
2.02355058876472	-132067917693.245\\
2.02365059126478	-132108024738.904\\
2.02375059376484	-132148131784.563\\
2.02385059626491	-132188238830.222\\
2.02395059876497	-132228345875.882\\
2.02405060126503	-132268452921.541\\
2.02415060376509	-132308559967.2\\
2.02425060626516	-132348667012.859\\
2.02435060876522	-132388774058.518\\
2.02445061126528	-132428881104.177\\
2.02455061376534	-132468988149.837\\
2.02465061626541	-132509095195.496\\
2.02475061876547	-132549202241.155\\
2.02485062126553	-132589309286.814\\
2.02495062376559	-132629416332.473\\
2.02505062626566	-132669523378.132\\
2.02515062876572	-132710203381.587\\
2.02525063126578	-132750310427.246\\
2.02535063376584	-132790417472.905\\
2.02545063626591	-132830524518.564\\
2.02555063876597	-132870631564.223\\
2.02565064126603	-132910738609.882\\
2.02575064376609	-132950845655.542\\
2.02585064626616	-132991525658.996\\
2.02595064876622	-133031632704.655\\
2.02605065126628	-133071739750.314\\
2.02615065376634	-133111846795.973\\
2.02625065626641	-133152526799.428\\
2.02635065876647	-133192633845.087\\
2.02645066126653	-133232740890.746\\
2.02655066376659	-133272847936.405\\
2.02665066626666	-133313527939.859\\
2.02675066876672	-133353634985.519\\
2.02685067126678	-133393742031.178\\
2.02695067376684	-133433849076.837\\
2.02705067626691	-133474529080.291\\
2.02715067876697	-133514636125.95\\
2.02725068126703	-133554743171.609\\
2.02735068376709	-133595423175.064\\
2.02745068626716	-133635530220.723\\
2.02755068876722	-133675637266.382\\
2.02765069126728	-133716317269.836\\
2.02775069376734	-133756424315.496\\
2.02785069626741	-133797104318.95\\
2.02795069876747	-133837211364.609\\
2.02805070126753	-133877318410.268\\
2.02815070376759	-133917998413.722\\
2.02825070626766	-133958105459.382\\
2.02835070876772	-133998785462.836\\
2.02845071126778	-134038892508.495\\
2.02855071376784	-134078999554.154\\
2.02865071626791	-134119679557.608\\
2.02875071876797	-134159786603.268\\
2.02885072126803	-134200466606.722\\
2.02895072376809	-134240573652.381\\
2.02905072626816	-134281253655.835\\
2.02915072876822	-134321360701.495\\
2.02925073126828	-134362040704.949\\
2.02935073376834	-134402147750.608\\
2.02945073626841	-134442827754.062\\
2.02955073876847	-134482934799.721\\
2.02965074126853	-134523614803.176\\
2.02975074376859	-134563721848.835\\
2.02985074626866	-134604401852.289\\
2.02995074876872	-134645081855.743\\
2.03005075126878	-134685188901.403\\
2.03015075376884	-134725868904.857\\
2.03025075626891	-134765975950.516\\
2.03035075876897	-134806655953.97\\
2.03045076126903	-134847335957.425\\
2.03055076376909	-134887443003.084\\
2.03065076626916	-134928123006.538\\
2.03075076876922	-134968230052.197\\
2.03085077126928	-135008910055.652\\
2.03095077376934	-135049590059.106\\
2.03105077626941	-135089697104.765\\
2.03115077876947	-135130377108.219\\
2.03125078126953	-135171057111.674\\
2.03135078376959	-135211737115.128\\
2.03145078626966	-135251844160.787\\
2.03155078876972	-135292524164.241\\
2.03165079126978	-135333204167.696\\
2.03175079376984	-135373311213.355\\
2.03185079626991	-135413991216.809\\
2.03195079876997	-135454671220.263\\
2.03205080127003	-135495351223.718\\
2.03215080377009	-135535458269.377\\
2.03225080627016	-135576138272.831\\
2.03235080877022	-135616818276.285\\
2.03245081127028	-135657498279.74\\
2.03255081377034	-135698178283.194\\
2.03265081627041	-135738285328.853\\
2.03275081877047	-135778965332.307\\
2.03285082127053	-135819645335.762\\
2.03295082377059	-135860325339.216\\
2.03305082627066	-135901005342.67\\
2.03315082877072	-135941685346.125\\
2.03325083127078	-135982365349.579\\
2.03335083377084	-136022472395.238\\
2.03345083627091	-136063152398.692\\
2.03355083877097	-136103832402.147\\
2.03365084127103	-136144512405.601\\
2.03375084377109	-136185192409.055\\
2.03385084627116	-136225872412.509\\
2.03395084877122	-136266552415.964\\
2.03405085127128	-136307232419.418\\
2.03415085377134	-136347912422.872\\
2.03425085627141	-136388592426.327\\
2.03435085877147	-136429272429.781\\
2.03445086127153	-136469952433.235\\
2.03455086377159	-136510632436.689\\
2.03465086627166	-136551312440.144\\
2.03475086877172	-136591992443.598\\
2.03485087127178	-136632672447.052\\
2.03495087377184	-136673352450.507\\
2.03505087627191	-136714032453.961\\
2.03515087877197	-136754712457.415\\
2.03525088127203	-136795392460.869\\
2.03535088377209	-136836072464.324\\
2.03545088627216	-136876752467.778\\
2.03555088877222	-136917432471.232\\
2.03565089127228	-136958112474.687\\
2.03575089377234	-136998792478.141\\
2.03585089627241	-137039472481.595\\
2.03595089877247	-137080725442.845\\
2.03605090127253	-137121405446.299\\
2.03615090377259	-137162085449.753\\
2.03625090627266	-137202765453.207\\
2.03635090877272	-137243445456.662\\
2.03645091127278	-137284125460.116\\
2.03655091377284	-137324805463.57\\
2.03665091627291	-137366058424.82\\
2.03675091877297	-137406738428.274\\
2.03685092127303	-137447418431.728\\
2.03695092377309	-137488098435.183\\
2.03705092627316	-137528778438.637\\
2.03715092877322	-137570031399.886\\
2.03725093127328	-137610711403.341\\
2.03735093377334	-137651391406.795\\
2.03745093627341	-137692071410.249\\
2.03755093877347	-137733324371.499\\
2.03765094127353	-137774004374.953\\
2.03775094377359	-137814684378.407\\
2.03785094627366	-137855364381.861\\
2.03795094877372	-137896617343.111\\
2.03805095127378	-137937297346.565\\
2.03815095377384	-137977977350.019\\
2.03825095627391	-138019230311.269\\
2.03835095877397	-138059910314.723\\
2.03845096127403	-138100590318.177\\
2.03855096377409	-138141843279.427\\
2.03865096627416	-138182523282.881\\
2.03875096877422	-138223203286.335\\
2.03885097127428	-138264456247.585\\
2.03895097377434	-138305136251.039\\
2.03905097627441	-138346389212.289\\
2.03915097877447	-138387069215.743\\
2.03925098127453	-138427749219.197\\
2.03935098377459	-138469002180.447\\
2.03945098627466	-138509682183.901\\
2.03955098877472	-138550935145.15\\
2.03965099127478	-138591615148.605\\
2.03975099377484	-138632868109.854\\
2.03985099627491	-138673548113.308\\
2.03995099877497	-138714228116.763\\
2.04005100127503	-138755481078.012\\
2.04015100377509	-138796161081.466\\
2.04025100627516	-138837414042.716\\
2.04035100877522	-138878094046.17\\
2.04045101127528	-138919347007.419\\
2.04055101377534	-138960027010.874\\
2.04065101627541	-139001279972.123\\
2.04075101877547	-139042532933.373\\
2.04085102127553	-139083212936.827\\
2.04095102377559	-139124465898.076\\
2.04105102627566	-139165145901.531\\
2.04115102877572	-139206398862.78\\
2.04125103127578	-139247078866.234\\
2.04135103377584	-139288331827.484\\
2.04145103627591	-139329584788.733\\
2.04155103877597	-139370264792.187\\
2.04165104127603	-139411517753.437\\
2.04175104377609	-139452197756.891\\
2.04185104627616	-139493450718.14\\
2.04195104877622	-139534703679.39\\
2.04205105127628	-139575383682.844\\
2.04215105377634	-139616636644.094\\
2.04225105627641	-139657889605.343\\
2.04235105877647	-139698569608.797\\
2.04245106127653	-139739822570.047\\
2.04255106377659	-139781075531.296\\
2.04265106627666	-139822328492.546\\
2.04275106877672	-139863008496\\
2.04285107127678	-139904261457.249\\
2.04295107377684	-139945514418.499\\
2.04305107627691	-139986194421.953\\
2.04315107877697	-140027447383.202\\
2.04325108127703	-140068700344.452\\
2.04335108377709	-140109953305.701\\
2.04345108627716	-140151206266.951\\
2.04355108877722	-140191886270.405\\
2.04365109127728	-140233139231.654\\
2.04375109377734	-140274392192.904\\
2.04385109627741	-140315645154.153\\
2.04395109877747	-140356898115.403\\
2.04405110127753	-140397578118.857\\
2.04415110377759	-140438831080.106\\
2.04425110627766	-140480084041.356\\
2.04435110877772	-140521337002.605\\
2.04445111127778	-140562589963.855\\
2.04455111377784	-140603842925.104\\
2.04465111627791	-140645095886.353\\
2.04475111877797	-140685775889.808\\
2.04485112127803	-140727028851.057\\
2.04495112377809	-140768281812.307\\
2.04505112627816	-140809534773.556\\
2.04515112877822	-140850787734.805\\
2.04525113127828	-140892040696.055\\
2.04535113377834	-140933293657.304\\
2.04545113627841	-140974546618.554\\
2.04555113877847	-141015799579.803\\
2.04565114127853	-141057052541.052\\
2.04575114377859	-141098305502.302\\
2.04585114627866	-141139558463.551\\
2.04595114877872	-141180811424.801\\
2.04605115127878	-141222064386.05\\
2.04615115377884	-141263317347.3\\
2.04625115627891	-141304570308.549\\
2.04635115877897	-141345823269.798\\
2.04645116127903	-141387076231.048\\
2.04655116377909	-141428329192.297\\
2.04665116627916	-141469582153.547\\
2.04675116877922	-141510835114.796\\
2.04685117127928	-141552088076.046\\
2.04695117377934	-141593341037.295\\
2.04705117627941	-141634593998.544\\
2.04715117877947	-141675846959.794\\
2.04725118127953	-141717672878.838\\
2.04735118377959	-141758925840.088\\
2.04745118627966	-141800178801.337\\
2.04755118877972	-141841431762.587\\
2.04765119127978	-141882684723.836\\
2.04775119377984	-141923937685.085\\
2.04785119627991	-141965190646.335\\
2.04795119877997	-142007016565.379\\
2.04805120128003	-142048269526.629\\
2.04815120378009	-142089522487.878\\
2.04825120628016	-142130775449.128\\
2.04835120878022	-142172028410.377\\
2.04845121128028	-142213854329.422\\
2.04855121378034	-142255107290.671\\
2.04865121628041	-142296360251.92\\
2.04875121878047	-142337613213.17\\
2.04885122128053	-142379439132.214\\
2.04895122378059	-142420692093.464\\
2.04905122628066	-142461945054.713\\
2.04915122878072	-142503198015.963\\
2.04925123128078	-142545023935.007\\
2.04935123378084	-142586276896.257\\
2.04945123628091	-142627529857.506\\
2.04955123878097	-142669355776.551\\
2.04965124128103	-142710608737.8\\
2.04975124378109	-142751861699.049\\
2.04985124628116	-142793687618.094\\
2.04995124878122	-142834940579.343\\
2.05005125128128	-142876193540.593\\
2.05015125378134	-142918019459.637\\
2.05025125628141	-142959272420.887\\
2.05035125878147	-143000525382.136\\
2.05045126128153	-143042351301.181\\
2.05055126378159	-143083604262.43\\
2.05065126628166	-143124857223.68\\
2.05075126878172	-143166683142.724\\
2.05085127128178	-143207936103.974\\
2.05095127378184	-143249762023.018\\
2.05105127628191	-143291014984.268\\
2.05115127878197	-143332840903.312\\
2.05125128128203	-143374093864.562\\
2.05135128378209	-143415919783.606\\
2.05145128628216	-143457172744.856\\
2.05155128878222	-143498425706.105\\
2.05165129128228	-143540251625.149\\
2.05175129378234	-143581504586.399\\
2.05185129628241	-143623330505.443\\
2.05195129878247	-143664583466.693\\
2.05205130128253	-143706409385.737\\
2.05215130378259	-143748235304.782\\
2.05225130628266	-143789488266.031\\
2.05235130878272	-143831314185.076\\
2.05245131128278	-143872567146.325\\
2.05255131378284	-143914393065.37\\
2.05265131628291	-143955646026.619\\
2.05275131878297	-143997471945.664\\
2.05285132128303	-144038724906.913\\
2.05295132378309	-144080550825.958\\
2.05305132628316	-144122376745.002\\
2.05315132878322	-144163629706.252\\
2.05325133128328	-144205455625.296\\
2.05335133378334	-144247281544.341\\
2.05345133628341	-144288534505.59\\
2.05355133878347	-144330360424.635\\
2.05365134128353	-144372186343.679\\
2.05375134378359	-144413439304.929\\
2.05385134628366	-144455265223.973\\
2.05395134878372	-144497091143.018\\
2.05405135128378	-144538344104.267\\
2.05415135378384	-144580170023.312\\
2.05425135628391	-144621995942.356\\
2.05435135878397	-144663248903.606\\
2.05445136128403	-144705074822.65\\
2.05455136378409	-144746900741.695\\
2.05465136628416	-144788726660.74\\
2.05475136878422	-144829979621.989\\
2.05485137128428	-144871805541.034\\
2.05495137378434	-144913631460.078\\
2.05505137628441	-144955457379.123\\
2.05515137878447	-144996710340.372\\
2.05525138128453	-145038536259.417\\
2.05535138378459	-145080362178.461\\
2.05545138628466	-145122188097.506\\
2.05555138878472	-145164014016.55\\
2.05565139128478	-145205266977.8\\
2.05575139378484	-145247092896.844\\
2.05585139628491	-145288918815.889\\
2.05595139878497	-145330744734.933\\
2.05605140128503	-145372570653.978\\
2.05615140378509	-145414396573.022\\
2.05625140628516	-145456222492.067\\
2.05635140878522	-145497475453.316\\
2.05645141128528	-145539301372.361\\
2.05655141378534	-145581127291.405\\
2.05665141628541	-145622953210.45\\
2.05675141878547	-145664779129.495\\
2.05685142128553	-145706605048.539\\
2.05695142378559	-145748430967.584\\
2.05705142628566	-145790256886.628\\
2.05715142878572	-145832082805.673\\
2.05725143128578	-145873908724.717\\
2.05735143378584	-145915734643.762\\
2.05745143628591	-145957560562.806\\
2.05755143878597	-145999386481.851\\
2.05765144128603	-146041212400.896\\
2.05775144378609	-146083038319.94\\
2.05785144628616	-146124864238.985\\
2.05795144878622	-146166690158.029\\
2.05805145128628	-146208516077.074\\
2.05815145378634	-146250341996.118\\
2.05825145628641	-146292167915.163\\
2.05835145878647	-146333993834.207\\
2.05845146128653	-146375819753.252\\
2.05855146378659	-146417645672.296\\
2.05865146628666	-146459471591.341\\
2.05875146878672	-146501297510.386\\
2.05885147128678	-146543123429.43\\
2.05895147378684	-146584949348.475\\
2.05905147628691	-146626775267.519\\
2.05915147878697	-146669174144.359\\
2.05925148128703	-146711000063.403\\
2.05935148378709	-146752825982.448\\
2.05945148628716	-146794651901.493\\
2.05955148878722	-146836477820.537\\
2.05965149128728	-146878303739.582\\
2.05975149378734	-146920129658.626\\
2.05985149628741	-146962528535.466\\
2.05995149878747	-147004354454.51\\
2.06005150128753	-147046180373.555\\
2.06015150378759	-147088006292.6\\
2.06025150628766	-147129832211.644\\
2.06035150878772	-147172231088.484\\
2.06045151128778	-147214057007.528\\
2.06055151378784	-147255882926.573\\
2.06065151628791	-147297708845.617\\
2.06075151878797	-147340107722.457\\
2.06085152128803	-147381933641.502\\
2.06095152378809	-147423759560.546\\
2.06105152628816	-147465585479.591\\
2.06115152878822	-147507984356.43\\
2.06125153128828	-147549810275.475\\
2.06135153378834	-147591636194.52\\
2.06145153628841	-147633462113.564\\
2.06155153878847	-147675860990.404\\
2.06165154128853	-147717686909.448\\
2.06175154378859	-147759512828.493\\
2.06185154628866	-147801911705.333\\
2.06195154878872	-147843737624.377\\
2.06205155128878	-147885563543.422\\
2.06215155378884	-147927962420.261\\
2.06225155628891	-147969788339.306\\
2.06235155878897	-148012187216.146\\
2.06245156128903	-148054013135.19\\
2.06255156378909	-148095839054.235\\
2.06265156628916	-148138237931.074\\
2.06275156878922	-148180063850.119\\
2.06285157128928	-148222462726.959\\
2.06295157378934	-148264288646.003\\
2.06305157628941	-148306114565.048\\
2.06315157878947	-148348513441.887\\
2.06325158128953	-148390339360.932\\
2.06335158378959	-148432738237.772\\
2.06345158628966	-148474564156.816\\
2.06355158878972	-148516963033.656\\
2.06365159128978	-148558788952.7\\
2.06375159378984	-148601187829.54\\
2.06385159628991	-148643013748.585\\
2.06395159878997	-148685412625.424\\
2.06405160129003	-148727238544.469\\
2.06415160379009	-148769637421.309\\
2.06425160629016	-148811463340.353\\
2.06435160879022	-148853862217.193\\
2.06445161129028	-148895688136.237\\
2.06455161379034	-148938087013.077\\
2.06465161629041	-148980485889.917\\
2.06475161879047	-149022311808.961\\
2.06485162129053	-149064710685.801\\
2.06495162379059	-149106536604.845\\
2.06505162629066	-149148935481.685\\
2.06515162879072	-149190761400.73\\
2.06525163129078	-149233160277.569\\
2.06535163379084	-149275559154.409\\
2.06545163629091	-149317385073.454\\
2.06555163879097	-149359783950.293\\
2.06565164129103	-149402182827.133\\
2.06575164379109	-149444008746.177\\
2.06585164629116	-149486407623.017\\
2.06595164879122	-149528806499.857\\
2.06605165129128	-149570632418.901\\
2.06615165379134	-149613031295.741\\
2.06625165629141	-149655430172.581\\
2.06635165879147	-149697256091.625\\
2.06645166129153	-149739654968.465\\
2.06655166379159	-149782053845.305\\
2.06665166629166	-149824452722.144\\
2.06675166879172	-149866278641.189\\
2.06685167129178	-149908677518.029\\
2.06695167379184	-149951076394.868\\
2.06705167629191	-149993475271.708\\
2.06715167879197	-150035301190.753\\
2.06725168129203	-150077700067.592\\
2.06735168379209	-150120098944.432\\
2.06745168629216	-150162497821.272\\
2.06755168879222	-150204323740.316\\
2.06765169129228	-150246722617.156\\
2.06775169379234	-150289121493.995\\
2.06785169629241	-150331520370.835\\
2.06795169879247	-150373919247.675\\
2.06805170129253	-150416318124.514\\
2.06815170379259	-150458144043.559\\
2.06825170629266	-150500542920.399\\
2.06835170879272	-150542941797.238\\
2.06845171129278	-150585340674.078\\
2.06855171379284	-150627739550.918\\
2.06865171629291	-150670138427.757\\
2.06875171879297	-150712537304.597\\
2.06885172129303	-150754936181.437\\
2.06895172379309	-150796762100.481\\
2.06905172629316	-150839160977.321\\
2.06915172879322	-150881559854.161\\
2.06925173129328	-150923958731\\
2.06935173379334	-150966357607.84\\
2.06945173629341	-151008756484.68\\
2.06955173879347	-151051155361.519\\
2.06965174129353	-151093554238.359\\
2.06975174379359	-151135953115.199\\
2.06985174629366	-151178351992.038\\
2.06995174879372	-151220750868.878\\
2.07005175129378	-151263149745.718\\
2.07015175379384	-151305548622.558\\
2.07025175629391	-151347947499.397\\
2.07035175879397	-151390346376.237\\
2.07045176129403	-151432745253.077\\
2.07055176379409	-151475144129.916\\
2.07065176629416	-151517543006.756\\
2.07075176879422	-151559941883.596\\
2.07085177129428	-151602340760.435\\
2.07095177379435	-151644739637.275\\
2.07105177629441	-151687138514.115\\
2.07115177879447	-151729537390.954\\
2.07125178129453	-151772509225.589\\
2.07135178379459	-151814908102.429\\
2.07145178629466	-151857306979.269\\
2.07155178879472	-151899705856.108\\
2.07165179129478	-151942104732.948\\
2.07175179379484	-151984503609.788\\
2.07185179629491	-152026902486.627\\
2.07195179879497	-152069301363.467\\
2.07205180129503	-152112273198.102\\
2.07215180379509	-152154672074.941\\
2.07225180629516	-152197070951.781\\
2.07235180879522	-152239469828.621\\
2.07245181129528	-152281868705.46\\
2.07255181379534	-152324267582.3\\
2.07265181629541	-152367239416.935\\
2.07275181879547	-152409638293.775\\
2.07285182129553	-152452037170.614\\
2.0729518237956	-152494436047.454\\
2.07305182629566	-152536834924.294\\
2.07315182879572	-152579806758.928\\
2.07325183129578	-152622205635.768\\
2.07335183379584	-152664604512.608\\
2.07345183629591	-152707003389.448\\
2.07355183879597	-152749975224.082\\
2.07365184129603	-152792374100.922\\
2.07375184379609	-152834772977.762\\
2.07385184629616	-152877744812.397\\
2.07395184879622	-152920143689.236\\
2.07405185129628	-152962542566.076\\
2.07415185379634	-153005514400.711\\
2.07425185629641	-153047913277.55\\
2.07435185879647	-153090312154.39\\
2.07445186129653	-153133283989.025\\
2.07455186379659	-153175682865.865\\
2.07465186629666	-153218081742.704\\
2.07475186879672	-153261053577.339\\
2.07485187129678	-153303452454.179\\
2.07495187379685	-153345851331.018\\
2.07505187629691	-153388823165.653\\
2.07515187879697	-153431222042.493\\
2.07525188129703	-153474193877.128\\
2.0753518837971	-153516592753.967\\
2.07545188629716	-153558991630.807\\
2.07555188879722	-153601963465.442\\
2.07565189129728	-153644362342.282\\
2.07575189379734	-153687334176.916\\
2.07585189629741	-153729733053.756\\
2.07595189879747	-153772704888.391\\
2.07605190129753	-153815103765.231\\
2.07615190379759	-153857502642.07\\
2.07625190629766	-153900474476.705\\
2.07635190879772	-153942873353.545\\
2.07645191129778	-153985845188.18\\
2.07655191379784	-154028244065.019\\
2.07665191629791	-154071215899.654\\
2.07675191879797	-154113614776.494\\
2.07685192129803	-154156586611.129\\
2.0769519237981	-154199558445.763\\
2.07705192629816	-154241957322.603\\
2.07715192879822	-154284929157.238\\
2.07725193129828	-154327328034.077\\
2.07735193379835	-154370299868.712\\
2.07745193629841	-154412698745.552\\
2.07755193879847	-154455670580.187\\
2.07765194129853	-154498069457.026\\
2.07775194379859	-154541041291.661\\
2.07785194629866	-154584013126.296\\
2.07795194879872	-154626412003.136\\
2.07805195129878	-154669383837.771\\
2.07815195379884	-154711782714.61\\
2.07825195629891	-154754754549.245\\
2.07835195879897	-154797726383.88\\
2.07845196129903	-154840125260.72\\
2.07855196379909	-154883097095.354\\
2.07865196629916	-154926068929.989\\
2.07875196879922	-154968467806.829\\
2.07885197129928	-155011439641.464\\
2.07895197379935	-155054411476.099\\
2.07905197629941	-155096810352.938\\
2.07915197879947	-155139782187.573\\
2.07925198129953	-155182754022.208\\
2.0793519837996	-155225725856.843\\
2.07945198629966	-155268124733.682\\
2.07955198879972	-155311096568.317\\
2.07965199129978	-155354068402.952\\
2.07975199379984	-155396467279.792\\
2.07985199629991	-155439439114.426\\
2.07995199879997	-155482410949.061\\
2.08005200130003	-155525382783.696\\
2.08015200380009	-155568354618.331\\
2.08025200630016	-155610753495.171\\
2.08035200880022	-155653725329.805\\
2.08045201130028	-155696697164.44\\
2.08055201380034	-155739668999.075\\
2.08065201630041	-155782067875.915\\
2.08075201880047	-155825039710.549\\
2.08085202130053	-155868011545.184\\
2.0809520238006	-155910983379.819\\
2.08105202630066	-155953955214.454\\
2.08115202880072	-155996927049.089\\
2.08125203130078	-156039898883.724\\
2.08135203380085	-156082297760.563\\
2.08145203630091	-156125269595.198\\
2.08155203880097	-156168241429.833\\
2.08165204130103	-156211213264.468\\
2.0817520438011	-156254185099.102\\
2.08185204630116	-156297156933.737\\
2.08195204880122	-156340128768.372\\
2.08205205130128	-156383100603.007\\
2.08215205380134	-156426072437.642\\
2.08225205630141	-156468471314.481\\
2.08235205880147	-156511443149.116\\
2.08245206130153	-156554414983.751\\
2.08255206380159	-156597386818.386\\
2.08265206630166	-156640358653.021\\
2.08275206880172	-156683330487.655\\
2.08285207130178	-156726302322.29\\
2.08295207380185	-156769274156.925\\
2.08305207630191	-156812245991.56\\
2.08315207880197	-156855217826.195\\
2.08325208130203	-156898189660.829\\
2.0833520838021	-156941161495.464\\
2.08345208630216	-156984133330.099\\
2.08355208880222	-157027105164.734\\
2.08365209130228	-157070076999.369\\
2.08375209380235	-157113048834.004\\
2.08385209630241	-157156020668.638\\
2.08395209880247	-157198992503.273\\
2.08405210130253	-157241964337.908\\
2.08415210380259	-157285509130.338\\
2.08425210630266	-157328480964.973\\
2.08435210880272	-157371452799.608\\
2.08445211130278	-157414424634.242\\
2.08455211380284	-157457396468.877\\
2.08465211630291	-157500368303.512\\
2.08475211880297	-157543340138.147\\
2.08485212130303	-157586311972.782\\
2.0849521238031	-157629283807.416\\
2.08505212630316	-157672828599.846\\
2.08515212880322	-157715800434.481\\
2.08525213130328	-157758772269.116\\
2.08535213380335	-157801744103.751\\
2.08545213630341	-157844715938.386\\
2.08555213880347	-157887687773.02\\
2.08565214130353	-157930659607.655\\
2.0857521438036	-157974204400.085\\
2.08585214630366	-158017176234.72\\
2.08595214880372	-158060148069.355\\
2.08605215130378	-158103119903.99\\
2.08615215380384	-158146091738.624\\
2.08625215630391	-158189636531.054\\
2.08635215880397	-158232608365.689\\
2.08645216130403	-158275580200.324\\
2.08655216380409	-158318552034.959\\
2.08665216630416	-158362096827.389\\
2.08675216880422	-158405068662.024\\
2.08685217130428	-158448040496.658\\
2.08695217380435	-158491012331.293\\
2.08705217630441	-158534557123.723\\
2.08715217880447	-158577528958.358\\
2.08725218130453	-158620500792.993\\
2.0873521838046	-158664045585.423\\
2.08745218630466	-158707017420.058\\
2.08755218880472	-158749989254.692\\
2.08765219130478	-158793534047.122\\
2.08775219380485	-158836505881.757\\
2.08785219630491	-158879477716.392\\
2.08795219880497	-158923022508.822\\
2.08805220130503	-158965994343.457\\
2.0881522038051	-159008966178.091\\
2.08825220630516	-159052510970.521\\
2.08835220880522	-159095482805.156\\
2.08845221130528	-159138454639.791\\
2.08855221380534	-159181999432.221\\
2.08865221630541	-159224971266.856\\
2.08875221880547	-159268516059.286\\
2.08885222130553	-159311487893.921\\
2.0889522238056	-159355032686.35\\
2.08905222630566	-159398004520.985\\
2.08915222880572	-159440976355.62\\
2.08925223130578	-159484521148.05\\
2.08935223380585	-159527492982.685\\
2.08945223630591	-159571037775.115\\
2.08955223880597	-159614009609.75\\
2.08965224130603	-159657554402.18\\
2.0897522438061	-159700526236.814\\
2.08985224630616	-159744071029.244\\
2.08995224880622	-159787042863.879\\
2.09005225130628	-159830587656.309\\
2.09015225380635	-159873559490.944\\
2.09025225630641	-159917104283.374\\
2.09035225880647	-159960076118.009\\
2.09045226130653	-160003620910.439\\
2.09055226380659	-160046592745.073\\
2.09065226630666	-160090137537.503\\
2.09075226880672	-160133109372.138\\
2.09085227130678	-160176654164.568\\
2.09095227380685	-160219625999.203\\
2.09105227630691	-160263170791.633\\
2.09115227880697	-160306715584.063\\
2.09125228130703	-160349687418.698\\
2.0913522838071	-160393232211.128\\
2.09145228630716	-160436204045.762\\
2.09155228880722	-160479748838.192\\
2.09165229130728	-160523293630.622\\
2.09175229380735	-160566265465.257\\
2.09185229630741	-160609810257.687\\
2.09195229880747	-160652782092.322\\
2.09205230130753	-160696326884.752\\
2.0921523038076	-160739871677.182\\
2.09225230630766	-160782843511.816\\
2.09235230880772	-160826388304.246\\
2.09245231130778	-160869933096.676\\
2.09255231380785	-160912904931.311\\
2.09265231630791	-160956449723.741\\
2.09275231880797	-160999994516.171\\
2.09285232130803	-161042966350.806\\
2.0929523238081	-161086511143.236\\
2.09305232630816	-161130055935.666\\
2.09315232880822	-161173600728.096\\
2.09325233130828	-161216572562.731\\
2.09335233380835	-161260117355.16\\
2.09345233630841	-161303662147.59\\
2.09355233880847	-161346633982.225\\
2.09365234130853	-161390178774.655\\
2.0937523438086	-161433723567.085\\
2.09385234630866	-161477268359.515\\
2.09395234880872	-161520813151.945\\
2.09405235130878	-161563784986.58\\
2.09415235380885	-161607329779.01\\
2.09425235630891	-161650874571.44\\
2.09435235880897	-161694419363.87\\
2.09445236130903	-161737391198.504\\
2.0945523638091	-161780935990.934\\
2.09465236630916	-161824480783.364\\
2.09475236880922	-161868025575.794\\
2.09485237130928	-161911570368.224\\
2.09495237380935	-161955115160.654\\
2.09505237630941	-161998086995.289\\
2.09515237880947	-162041631787.719\\
2.09525238130953	-162085176580.149\\
2.0953523838096	-162128721372.579\\
2.09545238630966	-162172266165.009\\
2.09555238880972	-162215810957.439\\
2.09565239130978	-162259355749.869\\
2.09575239380985	-162302900542.299\\
2.09585239630991	-162345872376.933\\
2.09595239880997	-162389417169.363\\
2.09605240131003	-162432961961.793\\
2.0961524038101	-162476506754.223\\
2.09625240631016	-162520051546.653\\
2.09635240881022	-162563596339.083\\
2.09645241131028	-162607141131.513\\
2.09655241381035	-162650685923.943\\
2.09665241631041	-162694230716.373\\
2.09675241881047	-162737775508.803\\
2.09685242131053	-162781320301.233\\
2.0969524238106	-162824865093.663\\
2.09705242631066	-162868409886.093\\
2.09715242881072	-162911954678.523\\
2.09725243131078	-162955499470.953\\
2.09735243381085	-162999044263.383\\
2.09745243631091	-163042589055.812\\
2.09755243881097	-163086133848.242\\
2.09765244131103	-163129678640.672\\
2.0977524438111	-163173223433.102\\
2.09785244631116	-163216768225.532\\
2.09795244881122	-163260313017.962\\
2.09805245131128	-163303857810.392\\
2.09815245381135	-163347402602.822\\
2.09825245631141	-163390947395.252\\
2.09835245881147	-163434492187.682\\
2.09845246131153	-163478036980.112\\
2.0985524638116	-163521581772.542\\
2.09865246631166	-163565699522.767\\
2.09875246881172	-163609244315.197\\
2.09885247131178	-163652789107.627\\
2.09895247381185	-163696333900.057\\
2.09905247631191	-163739878692.487\\
2.09915247881197	-163783423484.917\\
2.09925248131203	-163826968277.347\\
2.0993524838121	-163870513069.776\\
2.09945248631216	-163914630820.002\\
2.09955248881222	-163958175612.432\\
2.09965249131228	-164001720404.861\\
2.09975249381235	-164045265197.291\\
2.09985249631241	-164088809989.721\\
2.09995249881247	-164132354782.151\\
2.10005250131253	-164176472532.376\\
2.1001525038126	-164220017324.806\\
2.10025250631266	-164263562117.236\\
2.10035250881272	-164307106909.666\\
2.10045251131278	-164350651702.096\\
2.10055251381285	-164394769452.321\\
2.10065251631291	-164438314244.751\\
2.10075251881297	-164481859037.181\\
2.10085252131303	-164525403829.611\\
2.1009525238131	-164569521579.836\\
2.10105252631316	-164613066372.266\\
2.10115252881322	-164656611164.696\\
2.10125253131328	-164700155957.126\\
2.10135253381335	-164744273707.351\\
2.10145253631341	-164787818499.781\\
2.10155253881347	-164831363292.211\\
2.10165254131353	-164875481042.436\\
2.1017525438136	-164919025834.866\\
2.10185254631366	-164962570627.296\\
2.10195254881372	-165006115419.726\\
2.10205255131378	-165050233169.951\\
2.10215255381385	-165093777962.381\\
2.10225255631391	-165137322754.811\\
2.10235255881397	-165181440505.036\\
2.10245256131403	-165224985297.466\\
2.1025525638141	-165269103047.691\\
2.10265256631416	-165312647840.121\\
2.10275256881422	-165356192632.551\\
2.10285257131428	-165400310382.776\\
2.10295257381435	-165443855175.206\\
2.10305257631441	-165487399967.636\\
2.10315257881447	-165531517717.861\\
2.10325258131453	-165575062510.291\\
2.1033525838146	-165619180260.516\\
2.10345258631466	-165662725052.946\\
2.10355258881472	-165706269845.376\\
2.10365259131478	-165750387595.601\\
2.10375259381485	-165793932388.031\\
2.10385259631491	-165838050138.256\\
2.10395259881497	-165881594930.686\\
2.10405260131503	-165925712680.911\\
2.1041526038151	-165969257473.341\\
2.10425260631516	-166013375223.566\\
2.10435260881522	-166056920015.996\\
2.10445261131528	-166101037766.221\\
2.10455261381535	-166144582558.651\\
2.10465261631541	-166188127351.081\\
2.10475261881547	-166232245101.306\\
2.10485262131553	-166276362851.531\\
2.1049526238156	-166319907643.961\\
2.10505262631566	-166364025394.186\\
2.10515262881572	-166407570186.616\\
2.10525263131578	-166451687936.841\\
2.10535263381585	-166495232729.271\\
2.10545263631591	-166539350479.496\\
2.10555263881597	-166582895271.926\\
2.10565264131603	-166627013022.151\\
2.1057526438161	-166670557814.581\\
2.10585264631616	-166714675564.806\\
2.10595264881622	-166758220357.236\\
2.10605265131628	-166802338107.461\\
2.10615265381635	-166846455857.686\\
2.10625265631641	-166890000650.116\\
2.10635265881647	-166934118400.341\\
2.10645266131653	-166977663192.771\\
2.1065526638166	-167021780942.996\\
2.10665266631666	-167065898693.221\\
2.10675266881672	-167109443485.651\\
2.10685267131678	-167153561235.876\\
2.10695267381685	-167197678986.101\\
2.10705267631691	-167241223778.531\\
2.10715267881697	-167285341528.756\\
2.10725268131703	-167329459278.981\\
2.1073526838171	-167373004071.411\\
2.10745268631716	-167417121821.636\\
2.10755268881722	-167461239571.861\\
2.10765269131728	-167504784364.291\\
2.10775269381735	-167548902114.516\\
2.10785269631741	-167593019864.741\\
2.10795269881747	-167636564657.171\\
2.10805270131753	-167680682407.396\\
2.1081527038176	-167724800157.622\\
2.10825270631766	-167768917907.847\\
2.10835270881772	-167812462700.277\\
2.10845271131778	-167856580450.502\\
2.10855271381785	-167900698200.727\\
2.10865271631791	-167944815950.952\\
2.10875271881797	-167988360743.382\\
2.10885272131803	-168032478493.607\\
2.1089527238181	-168076596243.832\\
2.10905272631816	-168120713994.057\\
2.10915272881822	-168164258786.487\\
2.10925273131828	-168208376536.712\\
2.10935273381835	-168252494286.937\\
2.10945273631841	-168296612037.162\\
2.10955273881847	-168340729787.387\\
2.10965274131853	-168384274579.817\\
2.1097527438186	-168428392330.042\\
2.10985274631866	-168472510080.267\\
2.10995274881872	-168516627830.492\\
2.11005275131878	-168560745580.717\\
2.11015275381885	-168604863330.942\\
2.11025275631891	-168648408123.372\\
2.11035275881897	-168692525873.598\\
2.11045276131903	-168736643623.823\\
2.1105527638191	-168780761374.048\\
2.11065276631916	-168824879124.273\\
2.11075276881922	-168868996874.498\\
2.11085277131928	-168913114624.723\\
2.11095277381935	-168956659417.153\\
2.11105277631941	-169000777167.378\\
2.11115277881947	-169044894917.603\\
2.11125278131953	-169089012667.828\\
2.1113527838196	-169133130418.053\\
2.11145278631966	-169177248168.278\\
2.11155278881972	-169221365918.503\\
2.11165279131978	-169265483668.728\\
2.11175279381985	-169309601418.953\\
2.11185279631991	-169353719169.178\\
2.11195279881997	-169397836919.404\\
2.11205280132003	-169441954669.629\\
2.1121528038201	-169486072419.854\\
2.11225280632016	-169530190170.079\\
2.11235280882022	-169574307920.304\\
2.11245281132028	-169618425670.529\\
2.11255281382035	-169662543420.754\\
2.11265281632041	-169706661170.979\\
2.11275281882047	-169750778921.204\\
2.11285282132053	-169794896671.429\\
2.1129528238206	-169839014421.654\\
2.11305282632066	-169883132171.879\\
2.11315282882072	-169927249922.104\\
2.11325283132078	-169971367672.329\\
2.11335283382085	-170015485422.555\\
2.11345283632091	-170059603172.78\\
2.11355283882097	-170103720923.005\\
2.11365284132103	-170147838673.23\\
2.1137528438211	-170191956423.455\\
2.11385284632116	-170236074173.68\\
2.11395284882122	-170280191923.905\\
2.11405285132128	-170324309674.13\\
2.11415285382135	-170368427424.355\\
2.11425285632141	-170412545174.58\\
2.11435285882147	-170457235882.6\\
2.11445286132153	-170501353632.826\\
2.1145528638216	-170545471383.051\\
2.11465286632166	-170589589133.276\\
2.11475286882172	-170633706883.501\\
2.11485287132178	-170677824633.726\\
2.11495287382185	-170721942383.951\\
2.11505287632191	-170766060134.176\\
2.11515287882197	-170810750842.196\\
2.11525288132203	-170854868592.421\\
2.1153528838221	-170898986342.646\\
2.11545288632216	-170943104092.871\\
2.11555288882222	-170987221843.096\\
2.11565289132228	-171031339593.322\\
2.11575289382235	-171076030301.342\\
2.11585289632241	-171120148051.567\\
2.11595289882247	-171164265801.792\\
2.11605290132253	-171208383552.017\\
2.1161529038226	-171252501302.242\\
2.11625290632266	-171297192010.262\\
2.11635290882272	-171341309760.487\\
2.11645291132278	-171385427510.712\\
2.11655291382285	-171429545260.937\\
2.11665291632291	-171474235968.958\\
2.11675291882297	-171518353719.183\\
2.11685292132303	-171562471469.408\\
2.1169529238231	-171606589219.633\\
2.11705292632316	-171651279927.653\\
2.11715292882322	-171695397677.878\\
2.11725293132328	-171739515428.103\\
2.11735293382335	-171783633178.328\\
2.11745293632341	-171828323886.348\\
2.11755293882347	-171872441636.574\\
2.11765294132353	-171916559386.799\\
2.1177529438236	-171961250094.819\\
2.11785294632366	-172005367845.044\\
2.11795294882372	-172049485595.269\\
2.11805295132378	-172094176303.289\\
2.11815295382385	-172138294053.514\\
2.11825295632391	-172182411803.739\\
2.11835295882397	-172227102511.76\\
2.11845296132403	-172271220261.985\\
2.1185529638241	-172315338012.21\\
2.11865296632416	-172360028720.23\\
2.11875296882422	-172404146470.455\\
2.11885297132428	-172448264220.68\\
2.11895297382435	-172492954928.7\\
2.11905297632441	-172537072678.925\\
2.11915297882447	-172581763386.946\\
2.11925298132453	-172625881137.171\\
2.1193529838246	-172669998887.396\\
2.11945298632466	-172714689595.416\\
2.11955298882472	-172758807345.641\\
2.11965299132478	-172803498053.661\\
2.11975299382485	-172847615803.886\\
2.11985299632491	-172892306511.906\\
2.11995299882497	-172936424262.132\\
2.12005300132503	-172980542012.357\\
2.1201530038251	-173025232720.377\\
2.12025300632516	-173069350470.602\\
2.12035300882522	-173114041178.622\\
2.12045301132528	-173158158928.847\\
2.12055301382535	-173202849636.867\\
2.12065301632541	-173246967387.092\\
2.12075301882547	-173291658095.113\\
2.12085302132553	-173335775845.338\\
2.1209530238256	-173380466553.358\\
2.12105302632566	-173424584303.583\\
2.12115302882572	-173469275011.603\\
2.12125303132578	-173513392761.828\\
2.12135303382585	-173558083469.848\\
2.12145303632591	-173602201220.074\\
2.12155303882597	-173646891928.094\\
2.12165304132603	-173691009678.319\\
2.1217530438261	-173735700386.339\\
2.12185304632616	-173779818136.564\\
2.12195304882622	-173824508844.584\\
2.12205305132628	-173868626594.809\\
2.12215305382635	-173913317302.83\\
2.12225305632641	-173958008010.85\\
2.12235305882647	-174002125761.075\\
2.12245306132653	-174046816469.095\\
2.1225530638266	-174090934219.32\\
2.12265306632666	-174135624927.34\\
2.12275306882672	-174180315635.361\\
2.12285307132678	-174224433385.586\\
2.12295307382685	-174269124093.606\\
2.12305307632691	-174313241843.831\\
2.12315307882697	-174357932551.851\\
2.12325308132703	-174402623259.871\\
2.1233530838271	-174446741010.096\\
2.12345308632716	-174491431718.117\\
2.12355308882722	-174535549468.342\\
2.12365309132728	-174580240176.362\\
2.12375309382735	-174624930884.382\\
2.12385309632741	-174669048634.607\\
2.12395309882747	-174713739342.627\\
2.12405310132753	-174758430050.648\\
2.1241531038276	-174802547800.873\\
2.12425310632766	-174847238508.893\\
2.12435310882772	-174891929216.913\\
2.12445311132778	-174936619924.933\\
2.12455311382785	-174980737675.158\\
2.12465311632791	-175025428383.178\\
2.12475311882797	-175070119091.199\\
2.12485312132803	-175114236841.424\\
2.1249531238281	-175158927549.444\\
2.12505312632816	-175203618257.464\\
2.12515312882822	-175248308965.484\\
2.12525313132828	-175292426715.709\\
2.12535313382835	-175337117423.73\\
2.12545313632841	-175381808131.75\\
2.12555313882847	-175426498839.77\\
2.12565314132853	-175470616589.995\\
2.1257531438286	-175515307298.015\\
2.12585314632866	-175559998006.036\\
2.12595314882872	-175604688714.056\\
2.12605315132878	-175648806464.281\\
2.12615315382885	-175693497172.301\\
2.12625315632891	-175738187880.321\\
2.12635315882897	-175782878588.341\\
2.12645316132903	-175827569296.362\\
2.1265531638291	-175871687046.587\\
2.12665316632916	-175916377754.607\\
2.12675316882922	-175961068462.627\\
2.12685317132928	-176005759170.647\\
2.12695317382935	-176050449878.668\\
2.12705317632941	-176095140586.688\\
2.12715317882947	-176139258336.913\\
2.12725318132953	-176183949044.933\\
2.1273531838296	-176228639752.953\\
2.12745318632966	-176273330460.973\\
2.12755318882972	-176318021168.994\\
2.12765319132978	-176362711877.014\\
2.12775319382985	-176407402585.034\\
2.12785319632991	-176451520335.259\\
2.12795319882997	-176496211043.279\\
2.12805320133003	-176540901751.3\\
2.1281532038301	-176585592459.32\\
2.12825320633016	-176630283167.34\\
2.12835320883022	-176674973875.36\\
2.12845321133028	-176719664583.38\\
2.12855321383035	-176764355291.401\\
2.12865321633041	-176809045999.421\\
2.12875321883047	-176853736707.441\\
2.12885322133053	-176898427415.461\\
2.1289532238306	-176942545165.686\\
2.12905322633066	-176987235873.706\\
2.12915322883072	-177031926581.727\\
2.12925323133078	-177076617289.747\\
2.12935323383085	-177121307997.767\\
2.12945323633091	-177165998705.787\\
2.12955323883097	-177210689413.807\\
2.12965324133103	-177255380121.828\\
2.1297532438311	-177300070829.848\\
2.12985324633116	-177344761537.868\\
2.12995324883122	-177389452245.888\\
2.13005325133128	-177434142953.908\\
2.13015325383135	-177478833661.929\\
2.13025325633141	-177523524369.949\\
2.13035325883147	-177568215077.969\\
2.13045326133153	-177612905785.989\\
2.1305532638316	-177657596494.009\\
2.13065326633166	-177702287202.03\\
2.13075326883172	-177746977910.05\\
2.13085327133178	-177791668618.07\\
2.13095327383185	-177836359326.09\\
2.13105327633191	-177881050034.111\\
2.13115327883197	-177926313699.926\\
2.13125328133203	-177971004407.946\\
2.1313532838321	-178015695115.966\\
2.13145328633216	-178060385823.986\\
2.13155328883222	-178105076532.007\\
2.13165329133228	-178149767240.027\\
2.13175329383235	-178194457948.047\\
2.13185329633241	-178239148656.067\\
2.13195329883247	-178283839364.087\\
2.13205330133253	-178328530072.108\\
2.1321533038326	-178373220780.128\\
2.13225330633266	-178417911488.148\\
2.13235330883272	-178463175153.963\\
2.13245331133278	-178507865861.984\\
2.13255331383285	-178552556570.004\\
2.13265331633291	-178597247278.024\\
2.13275331883297	-178641937986.044\\
2.13285332133303	-178686628694.064\\
2.1329533238331	-178731319402.085\\
2.13305332633316	-178776583067.9\\
2.13315332883322	-178821273775.92\\
2.13325333133328	-178865964483.94\\
2.13335333383335	-178910655191.961\\
2.13345333633341	-178955345899.981\\
2.13355333883347	-179000036608.001\\
2.13365334133353	-179045300273.816\\
2.1337533438336	-179089990981.837\\
2.13385334633366	-179134681689.857\\
2.13395334883372	-179179372397.877\\
2.13405335133378	-179224063105.897\\
2.13415335383385	-179269326771.712\\
2.13425335633391	-179314017479.733\\
2.13435335883397	-179358708187.753\\
2.13445336133403	-179403398895.773\\
2.1345533638341	-179448662561.588\\
2.13465336633416	-179493353269.609\\
2.13475336883422	-179538043977.629\\
2.13485337133428	-179582734685.649\\
2.13495337383435	-179627998351.464\\
2.13505337633441	-179672689059.485\\
2.13515337883447	-179717379767.505\\
2.13525338133453	-179762070475.525\\
2.1353533838346	-179807334141.34\\
2.13545338633466	-179852024849.361\\
2.13555338883472	-179896715557.381\\
2.13565339133478	-179941406265.401\\
2.13575339383485	-179986669931.216\\
2.13585339633491	-180031360639.236\\
2.13595339883497	-180076051347.257\\
2.13605340133503	-180121315013.072\\
2.1361534038351	-180166005721.092\\
2.13625340633516	-180210696429.112\\
2.13635340883522	-180255960094.928\\
2.13645341133528	-180300650802.948\\
2.13655341383535	-180345341510.968\\
2.13665341633541	-180390605176.784\\
2.13675341883547	-180435295884.804\\
2.13685342133553	-180479986592.824\\
2.1369534238356	-180525250258.639\\
2.13705342633566	-180569940966.659\\
2.13715342883572	-180614631674.68\\
2.13725343133578	-180659895340.495\\
2.13735343383585	-180704586048.515\\
2.13745343633591	-180749849714.331\\
2.13755343883597	-180794540422.351\\
2.13765344133603	-180839231130.371\\
2.1377534438361	-180884494796.186\\
2.13785344633616	-180929185504.206\\
2.13795344883622	-180974449170.022\\
2.13805345133628	-181019139878.042\\
2.13815345383635	-181063830586.062\\
2.13825345633641	-181109094251.878\\
2.13835345883647	-181153784959.898\\
2.13845346133653	-181199048625.713\\
2.1385534638366	-181243739333.733\\
2.13865346633666	-181288430041.754\\
2.13875346883672	-181333693707.569\\
2.13885347133678	-181378384415.589\\
2.13895347383685	-181423648081.404\\
2.13905347633691	-181468338789.425\\
2.13915347883697	-181513602455.24\\
2.13925348133703	-181558293163.26\\
2.1393534838371	-181603556829.075\\
2.13945348633716	-181648247537.096\\
2.13955348883722	-181693511202.911\\
2.13965349133728	-181738201910.931\\
2.13975349383735	-181783465576.747\\
2.13985349633741	-181828156284.767\\
2.13995349883747	-181873419950.582\\
2.14005350133753	-181918110658.602\\
2.1401535038376	-181963374324.418\\
2.14025350633766	-182008065032.438\\
2.14035350883772	-182053328698.253\\
2.14045351133778	-182098019406.273\\
2.14055351383785	-182143283072.089\\
2.14065351633791	-182187973780.109\\
2.14075351883797	-182233237445.924\\
2.14085352133803	-182277928153.944\\
2.1409535238381	-182323191819.76\\
2.14105352633816	-182367882527.78\\
2.14115352883822	-182413146193.595\\
2.14125353133828	-182458409859.411\\
2.14135353383835	-182503100567.431\\
2.14145353633841	-182548364233.246\\
2.14155353883847	-182593054941.266\\
2.14165354133853	-182638318607.082\\
2.1417535438386	-182683009315.102\\
2.14185354633866	-182728272980.917\\
2.14195354883872	-182773536646.733\\
2.14205355133878	-182818227354.753\\
2.14215355383885	-182863491020.568\\
2.14225355633891	-182908181728.588\\
2.14235355883897	-182953445394.404\\
2.14245356133903	-182998709060.219\\
2.1425535638391	-183043399768.239\\
2.14265356633916	-183088663434.055\\
2.14275356883922	-183133927099.87\\
2.14285357133928	-183178617807.89\\
2.14295357383935	-183223881473.705\\
2.14305357633941	-183269145139.521\\
2.14315357883947	-183313835847.541\\
2.14325358133953	-183359099513.356\\
2.1433535838396	-183404363179.172\\
2.14345358633966	-183449053887.192\\
2.14355358883972	-183494317553.007\\
2.14365359133978	-183539581218.823\\
2.14375359383985	-183584271926.843\\
2.14385359633991	-183629535592.658\\
2.14395359883997	-183674799258.473\\
2.14405360134003	-183719489966.494\\
2.1441536038401	-183764753632.309\\
2.14425360634016	-183810017298.124\\
2.14435360884022	-183854708006.144\\
2.14445361134028	-183899971671.96\\
2.14455361384035	-183945235337.775\\
2.14465361634041	-183990499003.59\\
2.14475361884047	-184035189711.611\\
2.14485362134053	-184080453377.426\\
2.1449536238406	-184125717043.241\\
2.14505362634066	-184170980709.057\\
2.14515362884072	-184215671417.077\\
2.14525363134078	-184260935082.892\\
2.14535363384085	-184306198748.708\\
2.14545363634091	-184351462414.523\\
2.14555363884097	-184396153122.543\\
2.14565364134103	-184441416788.358\\
2.1457536438411	-184486680454.174\\
2.14585364634116	-184531944119.989\\
2.14595364884122	-184576634828.009\\
2.14605365134128	-184621898493.825\\
2.14615365384135	-184667162159.64\\
2.14625365634141	-184712425825.455\\
2.14635365884147	-184757689491.271\\
2.14645366134153	-184802380199.291\\
2.1465536638416	-184847643865.106\\
2.14665366634166	-184892907530.922\\
2.14675366884172	-184938171196.737\\
2.14685367134178	-184983434862.552\\
2.14695367384185	-185028125570.572\\
2.14705367634191	-185073389236.388\\
2.14715367884197	-185118652902.203\\
2.14725368134203	-185163916568.018\\
2.1473536838421	-185209180233.834\\
2.14745368634216	-185254443899.649\\
2.14755368884222	-185299134607.669\\
2.14765369134228	-185344398273.485\\
2.14775369384235	-185389661939.3\\
2.14785369634241	-185434925605.115\\
2.14795369884247	-185480189270.931\\
2.14805370134253	-185525452936.746\\
2.1481537038426	-185570716602.561\\
2.14825370634266	-185615980268.377\\
2.14835370884272	-185660670976.397\\
2.14845371134278	-185705934642.212\\
2.14855371384285	-185751198308.027\\
2.14865371634291	-185796461973.843\\
2.14875371884297	-185841725639.658\\
2.14885372134303	-185886989305.474\\
2.1489537238431	-185932252971.289\\
2.14905372634316	-185977516637.104\\
2.14915372884322	-186022780302.919\\
2.14925373134328	-186068043968.735\\
2.14935373384335	-186113307634.55\\
2.14945373634341	-186158571300.366\\
2.14955373884347	-186203262008.386\\
2.14965374134353	-186248525674.201\\
2.1497537438436	-186293789340.016\\
2.14985374634366	-186339053005.832\\
2.14995374884372	-186384316671.647\\
2.15005375134378	-186429580337.462\\
2.15015375384385	-186474844003.278\\
2.15025375634391	-186520107669.093\\
2.15035375884397	-186565371334.908\\
2.15045376134403	-186610635000.724\\
2.1505537638441	-186655898666.539\\
2.15065376634416	-186701162332.354\\
2.15075376884422	-186746425998.17\\
2.15085377134428	-186791689663.985\\
2.15095377384435	-186836953329.8\\
2.15105377634441	-186882216995.616\\
2.15115377884447	-186927480661.431\\
2.15125378134453	-186972744327.246\\
2.1513537838446	-187018007993.062\\
2.15145378634466	-187063271658.877\\
2.15155378884472	-187108535324.692\\
2.15165379134478	-187153798990.508\\
2.15175379384485	-187199062656.323\\
2.15185379634491	-187244326322.138\\
2.15195379884497	-187289589987.954\\
2.15205380134503	-187334853653.769\\
2.1521538038451	-187380117319.584\\
2.15225380634516	-187425380985.4\\
2.15235380884522	-187471217609.01\\
2.15245381134528	-187516481274.826\\
2.15255381384535	-187561744940.641\\
2.15265381634541	-187607008606.456\\
2.15275381884547	-187652272272.272\\
2.15285382134553	-187697535938.087\\
2.1529538238456	-187742799603.902\\
2.15305382634566	-187788063269.718\\
2.15315382884572	-187833326935.533\\
2.15325383134578	-187878590601.348\\
2.15335383384585	-187923854267.164\\
2.15345383634591	-187969117932.979\\
2.15355383884597	-188014954556.589\\
2.15365384134603	-188060218222.405\\
2.1537538438461	-188105481888.22\\
2.15385384634616	-188150745554.035\\
2.15395384884622	-188196009219.851\\
2.15405385134628	-188241272885.666\\
2.15415385384635	-188286536551.481\\
2.15425385634641	-188331800217.297\\
2.15435385884647	-188377636840.907\\
2.15445386134653	-188422900506.723\\
2.1545538638466	-188468164172.538\\
2.15465386634666	-188513427838.353\\
2.15475386884672	-188558691504.169\\
2.15485387134678	-188603955169.984\\
2.15495387384685	-188649791793.594\\
2.15505387634691	-188695055459.41\\
2.15515387884697	-188740319125.225\\
2.15525388134703	-188785582791.04\\
2.1553538838471	-188830846456.856\\
2.15545388634716	-188876110122.671\\
2.15555388884722	-188921946746.281\\
2.15565389134728	-188967210412.097\\
2.15575389384735	-189012474077.912\\
2.15585389634741	-189057737743.727\\
2.15595389884747	-189103001409.543\\
2.15605390134753	-189148838033.153\\
2.1561539038476	-189194101698.969\\
2.15625390634766	-189239365364.784\\
2.15635390884772	-189284629030.599\\
2.15645391134778	-189329892696.415\\
2.15655391384785	-189375729320.025\\
2.15665391634791	-189420992985.84\\
2.15675391884797	-189466256651.656\\
2.15685392134803	-189511520317.471\\
2.1569539238481	-189557356941.082\\
2.15705392634816	-189602620606.897\\
2.15715392884822	-189647884272.712\\
2.15725393134828	-189693147938.528\\
2.15735393384835	-189738984562.138\\
2.15745393634841	-189784248227.953\\
2.15755393884847	-189829511893.769\\
2.15765394134853	-189874775559.584\\
2.1577539438486	-189920612183.194\\
2.15785394634866	-189965875849.01\\
2.15795394884872	-190011139514.825\\
2.15805395134878	-190056976138.436\\
2.15815395384885	-190102239804.251\\
2.15825395634891	-190147503470.066\\
2.15835395884897	-190193340093.677\\
2.15845396134903	-190238603759.492\\
2.1585539638491	-190283867425.307\\
2.15865396634916	-190329131091.123\\
2.15875396884922	-190374967714.733\\
2.15885397134928	-190420231380.549\\
2.15895397384935	-190465495046.364\\
2.15905397634941	-190511331669.974\\
2.15915397884947	-190556595335.79\\
2.15925398134953	-190601859001.605\\
2.1593539838496	-190647695625.216\\
2.15945398634966	-190692959291.031\\
2.15955398884972	-190738222956.846\\
2.15965399134978	-190784059580.457\\
2.15975399384985	-190829323246.272\\
2.15985399634991	-190874586912.087\\
2.15995399884997	-190920423535.698\\
2.16005400135003	-190965687201.513\\
2.1601540038501	-191011523825.124\\
2.16025400635016	-191056787490.939\\
2.16035400885022	-191102051156.754\\
2.16045401135028	-191147887780.365\\
2.16055401385035	-191193151446.18\\
2.16065401635041	-191238415111.995\\
2.16075401885047	-191284251735.606\\
2.16085402135053	-191329515401.421\\
2.1609540238506	-191375352025.032\\
2.16105402635066	-191420615690.847\\
2.16115402885072	-191465879356.662\\
2.16125403135078	-191511715980.273\\
2.16135403385085	-191556979646.088\\
2.16145403635091	-191602816269.699\\
2.16155403885097	-191648079935.514\\
2.16165404135103	-191693343601.329\\
2.1617540438511	-191739180224.94\\
2.16185404635116	-191784443890.755\\
2.16195404885122	-191830280514.366\\
2.16205405135128	-191875544180.181\\
2.16215405385135	-191921380803.791\\
2.16225405635141	-191966644469.607\\
2.16235405885147	-192012481093.217\\
2.16245406135153	-192057744759.032\\
2.1625540638516	-192103008424.848\\
2.16265406635166	-192148845048.458\\
2.16275406885172	-192194108714.274\\
2.16285407135178	-192239945337.884\\
2.16295407385185	-192285209003.699\\
2.16305407635191	-192331045627.31\\
2.16315407885197	-192376309293.125\\
2.16325408135203	-192422145916.736\\
2.1633540838521	-192467409582.551\\
2.16345408635216	-192513246206.161\\
2.16355408885222	-192558509871.977\\
2.16365409135228	-192604346495.587\\
2.16375409385235	-192649610161.403\\
2.16385409635241	-192695446785.013\\
2.16395409885247	-192740710450.828\\
2.16405410135253	-192786547074.439\\
2.1641541038526	-192831810740.254\\
2.16425410635266	-192877647363.865\\
2.16435410885272	-192922911029.68\\
2.16445411135278	-192968747653.29\\
2.16455411385285	-193014011319.106\\
2.16465411635291	-193059847942.716\\
2.16475411885297	-193105111608.532\\
2.16485412135303	-193150948232.142\\
2.1649541238531	-193196211897.957\\
2.16505412635316	-193242048521.568\\
2.16515412885322	-193287312187.383\\
2.16525413135328	-193333148810.994\\
2.16535413385335	-193378985434.604\\
2.16545413635341	-193424249100.419\\
2.16555413885347	-193470085724.03\\
2.16565414135353	-193515349389.845\\
2.1657541438536	-193561186013.456\\
2.16585414635366	-193606449679.271\\
2.16595414885372	-193652286302.882\\
2.16605415135378	-193698122926.492\\
2.16615415385385	-193743386592.307\\
2.16625415635391	-193789223215.918\\
2.16635415885397	-193834486881.733\\
2.16645416135403	-193880323505.344\\
2.1665541638541	-193925587171.159\\
2.16665416635416	-193971423794.769\\
2.16675416885422	-194017260418.38\\
2.16685417135428	-194062524084.195\\
2.16695417385435	-194108360707.806\\
2.16705417635441	-194154197331.416\\
2.16715417885447	-194199460997.231\\
2.16725418135453	-194245297620.842\\
2.1673541838546	-194290561286.657\\
2.16745418635466	-194336397910.268\\
2.16755418885472	-194382234533.878\\
2.16765419135478	-194427498199.694\\
2.16775419385485	-194473334823.304\\
2.16785419635491	-194519171446.914\\
2.16795419885497	-194564435112.73\\
2.16805420135503	-194610271736.34\\
2.1681542038551	-194655535402.156\\
2.16825420635516	-194701372025.766\\
2.16835420885522	-194747208649.377\\
2.16845421135528	-194792472315.192\\
2.16855421385535	-194838308938.802\\
2.16865421635541	-194884145562.413\\
2.16875421885547	-194929409228.228\\
2.16885422135553	-194975245851.839\\
2.1689542238556	-195021082475.449\\
2.16905422635566	-195066346141.264\\
2.16915422885572	-195112182764.875\\
2.16925423135578	-195158019388.485\\
2.16935423385585	-195203283054.301\\
2.16945423635591	-195249119677.911\\
2.16955423885597	-195294956301.522\\
2.16965424135603	-195340792925.132\\
2.1697542438561	-195386056590.947\\
2.16985424635616	-195431893214.558\\
2.16995424885622	-195477729838.168\\
2.17005425135628	-195522993503.984\\
2.17015425385635	-195568830127.594\\
2.17025425635641	-195614666751.205\\
2.17035425885647	-195659930417.02\\
2.17045426135653	-195705767040.63\\
2.1705542638566	-195751603664.241\\
2.17065426635666	-195797440287.851\\
2.17075426885672	-195842703953.667\\
2.17085427135678	-195888540577.277\\
2.17095427385685	-195934377200.888\\
2.17105427635691	-195980213824.498\\
2.17115427885697	-196025477490.313\\
2.17125428135703	-196071314113.924\\
2.1713542838571	-196117150737.534\\
2.17145428635716	-196162987361.145\\
2.17155428885722	-196208251026.96\\
2.17165429135728	-196254087650.571\\
2.17175429385735	-196299924274.181\\
2.17185429635741	-196345760897.792\\
2.17195429885747	-196391024563.607\\
2.17205430135753	-196436861187.217\\
2.1721543038576	-196482697810.828\\
2.17225430635766	-196528534434.438\\
2.17235430885772	-196573798100.254\\
2.17245431135778	-196619634723.864\\
2.17255431385785	-196665471347.475\\
2.17265431635791	-196711307971.085\\
2.17275431885797	-196757144594.695\\
2.17285432135803	-196802408260.511\\
2.1729543238581	-196848244884.121\\
2.17305432635816	-196894081507.732\\
2.17315432885822	-196939918131.342\\
2.17325433135828	-196985754754.953\\
2.17335433385835	-197031018420.768\\
2.17345433635841	-197076855044.378\\
2.17355433885847	-197122691667.989\\
2.17365434135853	-197168528291.599\\
2.1737543438586	-197214364915.21\\
2.17385434635866	-197259628581.025\\
2.17395434885872	-197305465204.636\\
2.17405435135878	-197351301828.246\\
2.17415435385885	-197397138451.857\\
2.17425435635891	-197442975075.467\\
2.17435435885897	-197488811699.078\\
2.17445436135903	-197534075364.893\\
2.1745543638591	-197579911988.503\\
2.17465436635916	-197625748612.114\\
2.17475436885922	-197671585235.724\\
2.17485437135928	-197717421859.335\\
2.17495437385935	-197763258482.945\\
2.17505437635941	-197809095106.556\\
2.17515437885947	-197854358772.371\\
2.17525438135953	-197900195395.981\\
2.1753543838596	-197946032019.592\\
2.17545438635966	-197991868643.202\\
2.17555438885972	-198037705266.813\\
2.17565439135978	-198083541890.423\\
2.17575439385985	-198129378514.034\\
2.17585439635991	-198175215137.644\\
2.17595439885997	-198220478803.46\\
2.17605440136003	-198266315427.07\\
2.1761544038601	-198312152050.681\\
2.17625440636016	-198357988674.291\\
2.17635440886022	-198403825297.901\\
2.17645441136028	-198449661921.512\\
2.17655441386035	-198495498545.122\\
2.17665441636041	-198541335168.733\\
2.17675441886047	-198587171792.343\\
2.17685442136053	-198633008415.954\\
2.1769544238606	-198678272081.769\\
2.17705442636066	-198724108705.38\\
2.17715442886072	-198769945328.99\\
2.17725443136078	-198815781952.601\\
2.17735443386085	-198861618576.211\\
2.17745443636091	-198907455199.821\\
2.17755443886097	-198953291823.432\\
2.17765444136103	-198999128447.042\\
2.1777544438611	-199044965070.653\\
2.17785444636116	-199090801694.263\\
2.17795444886122	-199136638317.874\\
2.17805445136128	-199182474941.484\\
2.17815445386135	-199228311565.095\\
2.17825445636141	-199274148188.705\\
2.17835445886147	-199319411854.521\\
2.17845446136153	-199365248478.131\\
2.1785544638616	-199411085101.741\\
2.17865446636166	-199456921725.352\\
2.17875446886172	-199502758348.962\\
2.17885447136178	-199548594972.573\\
2.17895447386185	-199594431596.183\\
2.17905447636191	-199640268219.794\\
2.17915447886197	-199686104843.404\\
2.17925448136203	-199731941467.015\\
2.1793544838621	-199777778090.625\\
2.17945448636216	-199823614714.236\\
2.17955448886222	-199869451337.846\\
2.17965449136228	-199915287961.457\\
2.17975449386235	-199961124585.067\\
2.17985449636241	-200006961208.678\\
2.17995449886247	-200052797832.288\\
2.18005450136253	-200098634455.898\\
2.1801545038626	-200144471079.509\\
2.18025450636266	-200190307703.119\\
2.18035450886272	-200236144326.73\\
2.18045451136278	-200281980950.34\\
2.18055451386285	-200327817573.951\\
2.18065451636291	-200373654197.561\\
2.18075451886297	-200419490821.172\\
2.18085452136303	-200465327444.782\\
2.1809545238631	-200511164068.393\\
2.18105452636316	-200557000692.003\\
2.18115452886322	-200602837315.614\\
2.18125453136328	-200648673939.224\\
2.18135453386335	-200694510562.835\\
2.18145453636341	-200740347186.445\\
2.18155453886347	-200786183810.055\\
2.18165454136353	-200832020433.666\\
2.1817545438636	-200877857057.276\\
2.18185454636366	-200923693680.887\\
2.18195454886372	-200969530304.497\\
2.18205455136378	-201015366928.108\\
2.18215455386385	-201061203551.718\\
2.18225455636391	-201107040175.329\\
2.18235455886397	-201152876798.939\\
2.18245456136403	-201199286380.345\\
2.1825545638641	-201245123003.955\\
2.18265456636416	-201290959627.566\\
2.18275456886422	-201336796251.176\\
2.18285457136428	-201382632874.787\\
2.18295457386435	-201428469498.397\\
2.18305457636441	-201474306122.008\\
2.18315457886447	-201520142745.618\\
2.18325458136453	-201565979369.228\\
2.1833545838646	-201611815992.839\\
2.18345458636466	-201657652616.449\\
2.18355458886472	-201703489240.06\\
2.18365459136478	-201749325863.67\\
2.18375459386485	-201795162487.281\\
2.18385459636491	-201841572068.686\\
2.18395459886497	-201887408692.297\\
2.18405460136503	-201933245315.907\\
2.1841546038651	-201979081939.518\\
2.18425460636516	-202024918563.128\\
2.18435460886522	-202070755186.739\\
2.18445461136528	-202116591810.349\\
2.18455461386535	-202162428433.96\\
2.18465461636541	-202208265057.57\\
2.18475461886547	-202254101681.181\\
2.18485462136553	-202299938304.791\\
2.1849546238656	-202346347886.197\\
2.18505462636566	-202392184509.807\\
2.18515462886572	-202438021133.418\\
2.18525463136578	-202483857757.028\\
2.18535463386585	-202529694380.639\\
2.18545463636591	-202575531004.249\\
2.18555463886597	-202621367627.859\\
2.18565464136603	-202667204251.47\\
2.1857546438661	-202713613832.876\\
2.18585464636616	-202759450456.486\\
2.18595464886622	-202805287080.096\\
2.18605465136628	-202851123703.707\\
2.18615465386635	-202896960327.317\\
2.18625465636641	-202942796950.928\\
2.18635465886647	-202988633574.538\\
2.18645466136653	-203035043155.944\\
2.1865546638666	-203080879779.554\\
2.18665466636666	-203126716403.165\\
2.18675466886672	-203172553026.775\\
2.18685467136678	-203218389650.386\\
2.18695467386685	-203264226273.996\\
2.18705467636691	-203310062897.607\\
2.18715467886697	-203356472479.012\\
2.18725468136703	-203402309102.623\\
2.1873546838671	-203448145726.233\\
2.18745468636716	-203493982349.844\\
2.18755468886722	-203539818973.454\\
2.18765469136728	-203585655597.065\\
2.18775469386735	-203632065178.47\\
2.18785469636741	-203677901802.081\\
2.18795469886747	-203723738425.691\\
2.18805470136753	-203769575049.302\\
2.1881547038676	-203815411672.912\\
2.18825470636766	-203861248296.523\\
2.18835470886772	-203907657877.928\\
2.18845471136778	-203953494501.539\\
2.18855471386785	-203999331125.149\\
2.18865471636791	-204045167748.76\\
2.18875471886797	-204091004372.37\\
2.18885472136803	-204137413953.776\\
2.1889547238681	-204183250577.386\\
2.18905472636816	-204229087200.997\\
2.18915472886822	-204274923824.607\\
2.18925473136828	-204320760448.217\\
2.18935473386835	-204367170029.623\\
2.18945473636841	-204413006653.234\\
2.18955473886847	-204458843276.844\\
2.18965474136853	-204504679900.454\\
2.1897547438686	-204550516524.065\\
2.18985474636866	-204596926105.471\\
2.18995474886872	-204642762729.081\\
2.19005475136878	-204688599352.691\\
2.19015475386885	-204734435976.302\\
2.19025475636891	-204780272599.912\\
2.19035475886897	-204826682181.318\\
2.19045476136903	-204872518804.928\\
2.1905547638691	-204918355428.539\\
2.19065476636916	-204964192052.149\\
2.19075476886922	-205010601633.555\\
2.19085477136928	-205056438257.165\\
2.19095477386935	-205102274880.776\\
2.19105477636941	-205148111504.386\\
2.19115477886947	-205194521085.792\\
2.19125478136953	-205240357709.402\\
2.1913547838696	-205286194333.013\\
2.19145478636966	-205332030956.623\\
2.19155478886972	-205378440538.029\\
2.19165479136978	-205424277161.639\\
2.19175479386985	-205470113785.25\\
2.19185479636991	-205515950408.86\\
2.19195479886997	-205562359990.266\\
2.19205480137003	-205608196613.876\\
2.1921548038701	-205654033237.487\\
2.19225480637016	-205699869861.097\\
2.19235480887022	-205746279442.503\\
2.19245481137028	-205792116066.113\\
2.19255481387035	-205837952689.724\\
2.19265481637041	-205883789313.334\\
2.19275481887047	-205930198894.74\\
2.19285482137053	-205976035518.35\\
2.1929548238706	-206021872141.961\\
2.19305482637066	-206067708765.571\\
2.19315482887072	-206114118346.977\\
2.19325483137078	-206159954970.587\\
2.19335483387085	-206205791594.198\\
2.19345483637091	-206251628217.808\\
2.19355483887097	-206298037799.214\\
2.19365484137103	-206343874422.824\\
2.1937548438711	-206389711046.435\\
2.19385484637116	-206436120627.84\\
2.19395484887122	-206481957251.451\\
2.19405485137128	-206527793875.061\\
2.19415485387135	-206573630498.672\\
2.19425485637141	-206620040080.077\\
2.19435485887147	-206665876703.688\\
2.19445486137153	-206711713327.298\\
2.1945548638716	-206758122908.704\\
2.19465486637166	-206803959532.314\\
2.19475486887172	-206849796155.925\\
2.19485487137178	-206896205737.331\\
2.19495487387185	-206942042360.941\\
2.19505487637191	-206987878984.551\\
2.19515487887197	-207033715608.162\\
2.19525488137203	-207080125189.568\\
2.1953548838721	-207125961813.178\\
2.19545488637216	-207171798436.788\\
2.19555488887222	-207218208018.194\\
2.19565489137228	-207264044641.805\\
2.19575489387235	-207309881265.415\\
2.19585489637241	-207356290846.821\\
2.19595489887247	-207402127470.431\\
2.19605490137253	-207447964094.042\\
2.1961549038726	-207494373675.447\\
2.19625490637266	-207540210299.058\\
2.19635490887272	-207586046922.668\\
2.19645491137278	-207632456504.074\\
2.19655491387285	-207678293127.684\\
2.19665491637291	-207724129751.295\\
2.19675491887297	-207770539332.7\\
2.19685492137303	-207816375956.311\\
2.1969549238731	-207862212579.921\\
2.19705492637316	-207908622161.327\\
2.19715492887322	-207954458784.937\\
2.19725493137328	-208000295408.548\\
2.19735493387335	-208046704989.953\\
2.19745493637341	-208092541613.564\\
2.19755493887347	-208138378237.174\\
2.19765494137353	-208184787818.58\\
2.1977549438736	-208230624442.19\\
2.19785494637366	-208276461065.801\\
2.19795494887372	-208322870647.206\\
2.19805495137378	-208368707270.817\\
2.19815495387385	-208414543894.427\\
2.19825495637391	-208460953475.833\\
2.19835495887397	-208506790099.443\\
2.19845496137403	-208552626723.054\\
2.1985549638741	-208599036304.459\\
2.19865496637416	-208644872928.07\\
2.19875496887422	-208690709551.68\\
2.19885497137428	-208737119133.086\\
2.19895497387435	-208782955756.696\\
2.19905497637441	-208829365338.102\\
2.19915497887447	-208875201961.712\\
2.19925498137453	-208921038585.323\\
2.1993549838746	-208967448166.728\\
2.19945498637466	-209013284790.339\\
2.19955498887472	-209059121413.949\\
2.19965499137478	-209105530995.355\\
2.19975499387485	-209151367618.965\\
2.19985499637491	-209197777200.371\\
2.19995499887497	-209243613823.982\\
2.20005500137503	-209289450447.592\\
2.2001550038751	-209335860028.998\\
2.20025500637516	-209381696652.608\\
2.20035500887522	-209427533276.219\\
2.20045501137528	-209473942857.624\\
2.20055501387535	-209519779481.235\\
2.20065501637541	-209566189062.64\\
2.20075501887547	-209612025686.251\\
2.20085502137553	-209657862309.861\\
2.2009550238756	-209704271891.267\\
2.20105502637566	-209750108514.877\\
2.20115502887572	-209796518096.283\\
2.20125503137578	-209842354719.893\\
2.20135503387585	-209888191343.504\\
2.20145503637591	-209934600924.909\\
2.20155503887597	-209980437548.52\\
2.20165504137603	-210026274172.13\\
2.2017550438761	-210072683753.536\\
2.20185504637616	-210118520377.146\\
2.20195504887622	-210164929958.552\\
2.20205505137628	-210210766582.162\\
2.20215505387635	-210256603205.773\\
2.20225505637641	-210303012787.178\\
2.20235505887647	-210348849410.789\\
2.20245506137653	-210395258992.194\\
2.2025550638766	-210441095615.805\\
2.20265506637666	-210486932239.415\\
2.20275506887672	-210533341820.821\\
2.20285507137678	-210579178444.431\\
2.20295507387685	-210625588025.837\\
2.20305507637691	-210671424649.448\\
2.20315507887697	-210717834230.853\\
2.20325508137703	-210763670854.464\\
2.2033550838771	-210809507478.074\\
2.20345508637716	-210855917059.48\\
2.20355508887722	-210901753683.09\\
2.20365509137728	-210948163264.496\\
2.20375509387735	-210993999888.106\\
2.20385509637741	-211039836511.717\\
2.20395509887747	-211086246093.122\\
2.20405510137753	-211132082716.733\\
2.2041551038776	-211178492298.138\\
2.20425510637766	-211224328921.749\\
2.20435510887772	-211270738503.154\\
2.20445511137778	-211316575126.765\\
2.20455511387785	-211362411750.375\\
2.20465511637791	-211408821331.781\\
2.20475511887797	-211454657955.391\\
2.20485512137803	-211501067536.797\\
2.2049551238781	-211546904160.407\\
2.20505512637816	-211593313741.813\\
2.20515512887822	-211639150365.423\\
2.20525513137828	-211684986989.034\\
2.20535513387835	-211731396570.44\\
2.20545513637841	-211777233194.05\\
2.20555513887847	-211823642775.456\\
2.20565514137853	-211869479399.066\\
2.2057551438786	-211915888980.472\\
2.20585514637866	-211961725604.082\\
2.20595514887872	-212007562227.693\\
2.20605515137878	-212053971809.098\\
2.20615515387885	-212099808432.709\\
2.20625515637891	-212146218014.114\\
2.20635515887897	-212192054637.725\\
2.20645516137903	-212238464219.13\\
2.2065551638791	-212284300842.741\\
2.20665516637916	-212330710424.146\\
2.20675516887922	-212376547047.757\\
2.20685517137928	-212422383671.367\\
2.20695517387935	-212468793252.773\\
2.20705517637941	-212514629876.383\\
2.20715517887947	-212561039457.789\\
2.20725518137953	-212606876081.399\\
2.2073551838796	-212653285662.805\\
2.20745518637966	-212699122286.415\\
2.20755518887972	-212745531867.821\\
2.20765519137978	-212791368491.432\\
2.20775519387985	-212837778072.837\\
2.20785519637991	-212883614696.448\\
2.20795519887997	-212929451320.058\\
2.20805520138003	-212975860901.464\\
2.2081552038801	-213021697525.074\\
2.20825520638016	-213068107106.48\\
2.20835520888022	-213113943730.09\\
2.20845521138028	-213160353311.496\\
2.20855521388035	-213206189935.106\\
2.20865521638041	-213252599516.512\\
2.20875521888047	-213298436140.122\\
2.20885522138053	-213344845721.528\\
2.2089552238806	-213390682345.138\\
2.20905522638066	-213437091926.544\\
2.20915522888072	-213482928550.154\\
2.20925523138078	-213528765173.765\\
2.20935523388085	-213575174755.171\\
2.20945523638091	-213621011378.781\\
2.20955523888097	-213667420960.187\\
2.20965524138103	-213713257583.797\\
2.2097552438811	-213759667165.203\\
2.20985524638116	-213805503788.813\\
2.20995524888122	-213851913370.219\\
2.21005525138128	-213897749993.829\\
2.21015525388135	-213944159575.235\\
2.21025525638141	-213989996198.845\\
2.21035525888147	-214036405780.251\\
2.21045526138153	-214082242403.861\\
2.2105552638816	-214128651985.267\\
2.21065526638166	-214174488608.877\\
2.21075526888172	-214220898190.283\\
2.21085527138178	-214266734813.893\\
2.21095527388185	-214312571437.504\\
2.21105527638191	-214358981018.91\\
2.21115527888197	-214404817642.52\\
2.21125528138203	-214451227223.926\\
2.2113552838821	-214497063847.536\\
2.21145528638216	-214543473428.942\\
2.21155528888222	-214589310052.552\\
2.21165529138228	-214635719633.958\\
2.21175529388235	-214681556257.568\\
2.21185529638241	-214727965838.974\\
2.21195529888247	-214773802462.584\\
2.21205530138253	-214820212043.99\\
2.2121553038826	-214866048667.6\\
2.21225530638266	-214912458249.006\\
2.21235530888272	-214958294872.616\\
2.21245531138278	-215004704454.022\\
2.21255531388285	-215050541077.632\\
2.21265531638291	-215096950659.038\\
2.21275531888297	-215142787282.648\\
2.21285532138303	-215189196864.054\\
2.2129553238831	-215235033487.665\\
2.21305532638316	-215281443069.07\\
2.21315532888322	-215327279692.681\\
2.21325533138328	-215373689274.086\\
2.21335533388335	-215419525897.697\\
2.21345533638341	-215465362521.307\\
2.21355533888347	-215511772102.713\\
2.21365534138353	-215557608726.323\\
2.2137553438836	-215604018307.729\\
2.21385534638366	-215649854931.339\\
2.21395534888372	-215696264512.745\\
2.21405535138378	-215742101136.355\\
2.21415535388385	-215788510717.761\\
2.21425535638391	-215834347341.371\\
2.21435535888397	-215880756922.777\\
2.21445536138403	-215926593546.387\\
2.2145553638841	-215973003127.793\\
2.21465536638416	-216018839751.404\\
2.21475536888422	-216065249332.809\\
2.21485537138428	-216111085956.42\\
2.21495537388435	-216157495537.825\\
2.21505537638441	-216203332161.436\\
2.21515537888447	-216249741742.841\\
2.21525538138453	-216295578366.452\\
2.2153553838846	-216341987947.857\\
2.21545538638466	-216387824571.468\\
2.21555538888472	-216434234152.873\\
2.21565539138478	-216480070776.484\\
2.21575539388485	-216526480357.889\\
2.21585539638491	-216572316981.5\\
2.21595539888497	-216618726562.905\\
2.21605540138503	-216664563186.516\\
2.2161554038851	-216710972767.922\\
2.21625540638516	-216756809391.532\\
2.21635540888522	-216803218972.938\\
2.21645541138528	-216849055596.548\\
2.21655541388535	-216895465177.954\\
2.21665541638541	-216941301801.564\\
2.21675541888547	-216987711382.97\\
2.21685542138553	-217033548006.58\\
2.2169554238856	-217079957587.986\\
2.21705542638566	-217125794211.596\\
2.21715542888572	-217172203793.002\\
2.21725543138578	-217218040416.612\\
2.21735543388585	-217264449998.018\\
2.21745543638591	-217310286621.628\\
2.21755543888597	-217356696203.034\\
2.21765544138603	-217402532826.644\\
2.2177554438861	-217448942408.05\\
2.21785544638616	-217494779031.66\\
2.21795544888622	-217541188613.066\\
2.21805545138628	-217587025236.677\\
2.21815545388635	-217633434818.082\\
2.21825545638641	-217679271441.693\\
2.21835545888647	-217725681023.098\\
2.21845546138653	-217771517646.709\\
2.2185554638866	-217817927228.114\\
2.21865546638666	-217863763851.725\\
2.21875546888672	-217910173433.13\\
2.21885547138678	-217956010056.741\\
2.21895547388685	-218002419638.146\\
2.21905547638691	-218048256261.757\\
2.21915547888697	-218094665843.162\\
2.21925548138703	-218140502466.773\\
2.2193554838871	-218186912048.179\\
2.21945548638716	-218232748671.789\\
2.21955548888722	-218279158253.195\\
2.21965549138728	-218324994876.805\\
2.21975549388735	-218371404458.211\\
2.21985549638741	-218417241081.821\\
2.21995549888747	-218463650663.227\\
2.22005550138753	-218509487286.837\\
2.2201555038876	-218555896868.243\\
2.22025550638766	-218601733491.853\\
2.22035550888772	-218648143073.259\\
2.22045551138778	-218693979696.869\\
2.22055551388785	-218740389278.275\\
2.22065551638791	-218786225901.885\\
2.22075551888797	-218832635483.291\\
2.22085552138803	-218878472106.901\\
2.2209555238881	-218924881688.307\\
2.22105552638816	-218970718311.918\\
2.22115552888822	-219017127893.323\\
2.22125553138828	-219062964516.934\\
2.22135553388835	-219109374098.339\\
2.22145553638841	-219155210721.95\\
2.22155553888847	-219201047345.56\\
2.22165554138853	-219247456926.966\\
2.2217555438886	-219293293550.576\\
2.22185554638866	-219339703131.982\\
2.22195554888872	-219385539755.592\\
2.22205555138878	-219431949336.998\\
2.22215555388885	-219477785960.608\\
2.22225555638891	-219524195542.014\\
2.22235555888897	-219570032165.624\\
2.22245556138903	-219616441747.03\\
2.2225555638891	-219662278370.64\\
2.22265556638916	-219708687952.046\\
2.22275556888922	-219754524575.656\\
2.22285557138928	-219800934157.062\\
2.22295557388935	-219846770780.673\\
2.22305557638941	-219893180362.078\\
2.22315557888947	-219939016985.689\\
2.22325558138953	-219985426567.094\\
2.2233555838896	-220031263190.705\\
2.22345558638966	-220077672772.11\\
2.22355558888972	-220123509395.721\\
2.22365559138978	-220169918977.126\\
2.22375559388985	-220215755600.737\\
2.22385559638991	-220262165182.142\\
2.22395559888997	-220308001805.753\\
2.22405560139003	-220354411387.158\\
2.2241556038901	-220400248010.769\\
2.22425560639016	-220446657592.174\\
2.22435560889022	-220492494215.785\\
2.22445561139028	-220538903797.191\\
2.22455561389035	-220584740420.801\\
2.22465561639041	-220631150002.207\\
2.22475561889047	-220676986625.817\\
2.22485562139053	-220723396207.223\\
2.2249556238906	-220769232830.833\\
2.22505562639066	-220815642412.239\\
2.22515562889072	-220861479035.849\\
2.22525563139078	-220907888617.255\\
2.22535563389085	-220953725240.865\\
2.22545563639091	-221000134822.271\\
2.22555563889097	-221045971445.881\\
2.22565564139103	-221091808069.492\\
2.2257556438911	-221138217650.897\\
2.22585564639116	-221184054274.508\\
2.22595564889122	-221230463855.913\\
2.22605565139128	-221276300479.524\\
2.22615565389135	-221322710060.93\\
2.22625565639141	-221368546684.54\\
2.22635565889147	-221414956265.946\\
2.22645566139153	-221460792889.556\\
2.2265556638916	-221507202470.962\\
2.22665566639166	-221553039094.572\\
2.22675566889172	-221599448675.978\\
2.22685567139178	-221645285299.588\\
2.22695567389185	-221691694880.994\\
2.22705567639191	-221737531504.604\\
2.22715567889197	-221783941086.01\\
2.22725568139203	-221829777709.62\\
2.2273556838921	-221876187291.026\\
2.22745568639216	-221922023914.636\\
2.22755568889222	-221967860538.247\\
2.22765569139228	-222014270119.652\\
2.22775569389235	-222060106743.263\\
2.22785569639241	-222106516324.668\\
2.22795569889247	-222152352948.279\\
2.22805570139253	-222198762529.685\\
2.2281557038926	-222244599153.295\\
2.22825570639266	-222291008734.701\\
2.22835570889272	-222336845358.311\\
2.22845571139278	-222383254939.717\\
2.22855571389285	-222429091563.327\\
2.22865571639291	-222475501144.733\\
2.22875571889297	-222521337768.343\\
2.22885572139303	-222567747349.749\\
2.2289557238931	-222613583973.359\\
2.22905572639316	-222659420596.97\\
2.22915572889322	-222705830178.375\\
2.22925573139328	-222751666801.986\\
2.22935573389335	-222798076383.391\\
2.22945573639341	-222843913007.002\\
2.22955573889347	-222890322588.407\\
2.22965574139353	-222936159212.018\\
2.2297557438936	-222982568793.424\\
2.22985574639366	-223028405417.034\\
2.22995574889372	-223074814998.44\\
2.23005575139378	-223120651622.05\\
2.23015575389385	-223166488245.661\\
2.23025575639391	-223212897827.066\\
2.23035575889397	-223258734450.677\\
2.23045576139404	-223305144032.082\\
2.2305557638941	-223350980655.693\\
2.23065576639416	-223397390237.098\\
2.23075576889422	-223443226860.709\\
2.23085577139428	-223489636442.114\\
2.23095577389435	-223535473065.725\\
2.23105577639441	-223581309689.335\\
2.23115577889447	-223627719270.741\\
2.23125578139453	-223673555894.351\\
2.2313557838946	-223719965475.757\\
2.23145578639466	-223765802099.367\\
2.23155578889472	-223812211680.773\\
2.23165579139478	-223858048304.383\\
2.23175579389485	-223904457885.789\\
2.23185579639491	-223950294509.399\\
2.23195579889497	-223996131133.01\\
2.23205580139503	-224042540714.416\\
2.2321558038951	-224088377338.026\\
2.23225580639516	-224134786919.432\\
2.23235580889522	-224180623543.042\\
2.23245581139529	-224227033124.448\\
2.23255581389535	-224272869748.058\\
2.23265581639541	-224318706371.669\\
2.23275581889547	-224365115953.074\\
2.23285582139553	-224410952576.685\\
2.2329558238956	-224457362158.09\\
2.23305582639566	-224503198781.701\\
2.23315582889572	-224549608363.106\\
2.23325583139578	-224595444986.717\\
2.23335583389585	-224641281610.327\\
2.23345583639591	-224687691191.733\\
2.23355583889597	-224733527815.343\\
2.23365584139603	-224779937396.749\\
2.2337558438961	-224825774020.359\\
2.23385584639616	-224871610643.97\\
2.23395584889622	-224918020225.375\\
2.23405585139628	-224963856848.986\\
2.23415585389635	-225010266430.392\\
2.23425585639641	-225056103054.002\\
2.23435585889647	-225101939677.612\\
2.23445586139654	-225148349259.018\\
2.2345558638966	-225194185882.629\\
2.23465586639666	-225240595464.034\\
2.23475586889672	-225286432087.645\\
2.23485587139678	-225332841669.05\\
2.23495587389685	-225378678292.661\\
2.23505587639691	-225424514916.271\\
2.23515587889697	-225470924497.677\\
2.23525588139703	-225516761121.287\\
2.2353558838971	-225563170702.693\\
2.23545588639716	-225609007326.303\\
2.23555588889722	-225654843949.914\\
2.23565589139728	-225701253531.319\\
2.23575589389735	-225747090154.93\\
2.23585589639741	-225792926778.54\\
2.23595589889747	-225839336359.946\\
2.23605590139753	-225885172983.556\\
2.2361559038976	-225931582564.962\\
2.23625590639766	-225977419188.572\\
2.23635590889772	-226023255812.183\\
2.23645591139779	-226069665393.588\\
2.23655591389785	-226115502017.199\\
2.23665591639791	-226161911598.604\\
2.23675591889797	-226207748222.215\\
2.23685592139804	-226253584845.825\\
2.2369559238981	-226299994427.231\\
2.23705592639816	-226345831050.841\\
2.23715592889822	-226391667674.452\\
2.23725593139828	-226438077255.858\\
2.23735593389835	-226483913879.468\\
2.23745593639841	-226530323460.874\\
2.23755593889847	-226576160084.484\\
2.23765594139853	-226621996708.095\\
2.2377559438986	-226668406289.5\\
2.23785594639866	-226714242913.111\\
2.23795594889872	-226760079536.721\\
2.23805595139878	-226806489118.127\\
2.23815595389885	-226852325741.737\\
2.23825595639891	-226898162365.348\\
2.23835595889897	-226944571946.753\\
2.23845596139904	-226990408570.364\\
2.2385559638991	-227036818151.769\\
2.23865596639916	-227082654775.38\\
2.23875596889922	-227128491398.99\\
2.23885597139929	-227174900980.396\\
2.23895597389935	-227220737604.006\\
2.23905597639941	-227266574227.617\\
2.23915597889947	-227312983809.022\\
2.23925598139953	-227358820432.633\\
2.2393559838996	-227404657056.243\\
2.23945598639966	-227451066637.649\\
2.23955598889972	-227496903261.259\\
2.23965599139978	-227542739884.87\\
2.23975599389985	-227589149466.275\\
2.23985599639991	-227634986089.886\\
2.23995599889997	-227680822713.496\\
2.24005600140003	-227727232294.902\\
2.2401560039001	-227773068918.512\\
2.24025600640016	-227818905542.123\\
2.24035600890022	-227865315123.528\\
2.24045601140029	-227911151747.139\\
2.24055601390035	-227956988370.749\\
2.24065601640041	-228003397952.155\\
2.24075601890047	-228049234575.765\\
2.24085602140054	-228095071199.376\\
2.2409560239006	-228141480780.781\\
2.24105602640066	-228187317404.392\\
2.24115602890072	-228233154028.002\\
2.24125603140078	-228279563609.408\\
2.24135603390085	-228325400233.018\\
2.24145603640091	-228371236856.629\\
2.24155603890097	-228417646438.035\\
2.24165604140103	-228463483061.645\\
2.2417560439011	-228509319685.255\\
2.24185604640116	-228555156308.866\\
2.24195604890122	-228601565890.272\\
2.24205605140128	-228647402513.882\\
2.24215605390135	-228693239137.492\\
2.24225605640141	-228739648718.898\\
2.24235605890147	-228785485342.509\\
2.24245606140154	-228831321966.119\\
2.2425560639016	-228877731547.525\\
2.24265606640166	-228923568171.135\\
2.24275606890172	-228969404794.745\\
2.24285607140179	-229015241418.356\\
2.24295607390185	-229061650999.762\\
2.24305607640191	-229107487623.372\\
2.24315607890197	-229153324246.982\\
2.24325608140204	-229199733828.388\\
2.2433560839021	-229245570451.999\\
2.24345608640216	-229291407075.609\\
2.24355608890222	-229337243699.219\\
2.24365609140228	-229383653280.625\\
2.24375609390235	-229429489904.236\\
2.24385609640241	-229475326527.846\\
2.24395609890247	-229521163151.456\\
2.24405610140253	-229567572732.862\\
2.2441561039026	-229613409356.473\\
2.24425610640266	-229659245980.083\\
2.24435610890272	-229705082603.693\\
2.24445611140279	-229751492185.099\\
2.24455611390285	-229797328808.71\\
2.24465611640291	-229843165432.32\\
2.24475611890297	-229889575013.726\\
2.24485612140304	-229935411637.336\\
2.2449561239031	-229981248260.947\\
2.24505612640316	-230027084884.557\\
2.24515612890322	-230072921508.167\\
2.24525613140329	-230119331089.573\\
2.24535613390335	-230165167713.184\\
2.24545613640341	-230211004336.794\\
2.24555613890347	-230256840960.404\\
2.24565614140353	-230303250541.81\\
2.2457561439036	-230349087165.421\\
2.24585614640366	-230394923789.031\\
2.24595614890372	-230440760412.641\\
2.24605615140378	-230487169994.047\\
2.24615615390385	-230533006617.658\\
2.24625615640391	-230578843241.268\\
2.24635615890397	-230624679864.878\\
2.24645616140404	-230670516488.489\\
2.2465561639041	-230716926069.895\\
2.24665616640416	-230762762693.505\\
2.24675616890422	-230808599317.115\\
2.24685617140429	-230854435940.726\\
2.24695617390435	-230900845522.132\\
2.24705617640441	-230946682145.742\\
2.24715617890447	-230992518769.352\\
2.24725618140454	-231038355392.963\\
2.2473561839046	-231084192016.573\\
2.24745618640466	-231130601597.979\\
2.24755618890472	-231176438221.589\\
2.24765619140479	-231222274845.2\\
2.24775619390485	-231268111468.81\\
2.24785619640491	-231313948092.421\\
2.24795619890497	-231360357673.826\\
2.24805620140503	-231406194297.437\\
2.2481562039051	-231452030921.047\\
2.24825620640516	-231497867544.658\\
2.24835620890522	-231543704168.268\\
2.24845621140529	-231589540791.879\\
2.24855621390535	-231635950373.284\\
2.24865621640541	-231681786996.895\\
2.24875621890547	-231727623620.505\\
2.24885622140554	-231773460244.116\\
2.2489562239056	-231819296867.726\\
2.24905622640566	-231865133491.337\\
2.24915622890572	-231911543072.742\\
2.24925623140579	-231957379696.353\\
2.24935623390585	-232003216319.963\\
2.24945623640591	-232049052943.574\\
2.24955623890597	-232094889567.184\\
2.24965624140604	-232140726190.795\\
2.2497562439061	-232187135772.2\\
2.24985624640616	-232232972395.811\\
2.24995624890622	-232278809019.421\\
2.25005625140628	-232324645643.032\\
2.25015625390635	-232370482266.642\\
2.25025625640641	-232416318890.253\\
2.25035625890647	-232462155513.863\\
2.25045626140654	-232508565095.269\\
2.2505562639066	-232554401718.879\\
2.25065626640666	-232600238342.49\\
2.25075626890672	-232646074966.1\\
2.25085627140679	-232691911589.71\\
2.25095627390685	-232737748213.321\\
2.25105627640691	-232783584836.931\\
2.25115627890697	-232829421460.542\\
2.25125628140704	-232875258084.152\\
2.2513562839071	-232921667665.558\\
2.25145628640716	-232967504289.168\\
2.25155628890722	-233013340912.779\\
2.25165629140729	-233059177536.389\\
2.25175629390735	-233105014160\\
2.25185629640741	-233150850783.61\\
2.25195629890747	-233196687407.221\\
2.25205630140753	-233242524030.831\\
2.2521563039076	-233288360654.442\\
2.25225630640766	-233334197278.052\\
2.25235630890772	-233380606859.458\\
2.25245631140779	-233426443483.068\\
2.25255631390785	-233472280106.679\\
2.25265631640791	-233518116730.289\\
2.25275631890797	-233563953353.9\\
2.25285632140804	-233609789977.51\\
2.2529563239081	-233655626601.12\\
2.25305632640816	-233701463224.731\\
2.25315632890822	-233747299848.341\\
2.25325633140829	-233793136471.952\\
2.25335633390835	-233838973095.562\\
2.25345633640841	-233884809719.173\\
2.25355633890847	-233930646342.783\\
2.25365634140854	-233976482966.394\\
2.2537563439086	-234022892547.799\\
2.25385634640866	-234068729171.41\\
2.25395634890872	-234114565795.02\\
2.25405635140879	-234160402418.631\\
2.25415635390885	-234206239042.241\\
2.25425635640891	-234252075665.852\\
2.25435635890897	-234297912289.462\\
2.25445636140904	-234343748913.073\\
2.2545563639091	-234389585536.683\\
2.25465636640916	-234435422160.294\\
2.25475636890922	-234481258783.904\\
2.25485637140929	-234527095407.514\\
2.25495637390935	-234572932031.125\\
2.25505637640941	-234618768654.735\\
2.25515637890947	-234664605278.346\\
2.25525638140954	-234710441901.956\\
2.2553563839096	-234756278525.567\\
2.25545638640966	-234802115149.177\\
2.25555638890972	-234847951772.788\\
2.25565639140979	-234893788396.398\\
2.25575639390985	-234939625020.009\\
2.25585639640991	-234985461643.619\\
2.25595639890997	-235031298267.23\\
2.25605640141004	-235077134890.84\\
2.2561564039101	-235122971514.451\\
2.25625640641016	-235168808138.061\\
2.25635640891022	-235214644761.671\\
2.25645641141029	-235260481385.282\\
2.25655641391035	-235306318008.892\\
2.25665641641041	-235352154632.503\\
2.25675641891047	-235397991256.113\\
2.25685642141054	-235443827879.724\\
2.2569564239106	-235489664503.334\\
2.25705642641066	-235535501126.945\\
2.25715642891072	-235581337750.555\\
2.25725643141079	-235627174374.166\\
2.25735643391085	-235673010997.776\\
2.25745643641091	-235718274663.591\\
2.25755643891097	-235764111287.202\\
2.25765644141104	-235809947910.812\\
2.2577564439111	-235855784534.423\\
2.25785644641116	-235901621158.033\\
2.25795644891122	-235947457781.644\\
2.25805645141129	-235993294405.254\\
2.25815645391135	-236039131028.865\\
2.25825645641141	-236084967652.475\\
2.25835645891147	-236130804276.086\\
2.25845646141154	-236176640899.696\\
2.2585564639116	-236222477523.307\\
2.25865646641166	-236268314146.917\\
2.25875646891172	-236314150770.528\\
2.25885647141179	-236359414436.343\\
2.25895647391185	-236405251059.953\\
2.25905647641191	-236451087683.564\\
2.25915647891197	-236496924307.174\\
2.25925648141204	-236542760930.785\\
2.2593564839121	-236588597554.395\\
2.25945648641216	-236634434178.006\\
2.25955648891222	-236680270801.616\\
2.25965649141229	-236726107425.227\\
2.25975649391235	-236771944048.837\\
2.25985649641241	-236817207714.652\\
2.25995649891247	-236863044338.263\\
2.26005650141254	-236908880961.873\\
2.2601565039126	-236954717585.484\\
2.26025650641266	-237000554209.094\\
2.26035650891272	-237046390832.705\\
2.26045651141279	-237092227456.315\\
2.26055651391285	-237137491122.131\\
2.26065651641291	-237183327745.741\\
2.26075651891297	-237229164369.351\\
2.26085652141304	-237275000992.962\\
2.2609565239131	-237320837616.572\\
2.26105652641316	-237366674240.183\\
2.26115652891322	-237412510863.793\\
2.26125653141329	-237457774529.609\\
2.26135653391335	-237503611153.219\\
2.26145653641341	-237549447776.83\\
2.26155653891347	-237595284400.44\\
2.26165654141354	-237641121024.051\\
2.2617565439136	-237686384689.866\\
2.26185654641366	-237732221313.476\\
2.26195654891372	-237778057937.087\\
2.26205655141379	-237823894560.697\\
2.26215655391385	-237869731184.308\\
2.26225655641391	-237915567807.918\\
2.26235655891397	-237960831473.733\\
2.26245656141404	-238006668097.344\\
2.2625565639141	-238052504720.954\\
2.26265656641416	-238098341344.565\\
2.26275656891422	-238143605010.38\\
2.26285657141429	-238189441633.991\\
2.26295657391435	-238235278257.601\\
2.26305657641441	-238281114881.212\\
2.26315657891447	-238326951504.822\\
2.26325658141454	-238372215170.637\\
2.2633565839146	-238418051794.248\\
2.26345658641466	-238463888417.858\\
2.26355658891472	-238509725041.469\\
2.26365659141479	-238554988707.284\\
2.26375659391485	-238600825330.895\\
2.26385659641491	-238646661954.505\\
2.26395659891497	-238692498578.116\\
2.26405660141504	-238737762243.931\\
2.2641566039151	-238783598867.541\\
2.26425660641516	-238829435491.152\\
2.26435660891522	-238875272114.762\\
2.26445661141529	-238920535780.578\\
2.26455661391535	-238966372404.188\\
2.26465661641541	-239012209027.799\\
2.26475661891547	-239058045651.409\\
2.26485662141554	-239103309317.224\\
2.2649566239156	-239149145940.835\\
2.26505662641566	-239194982564.445\\
2.26515662891572	-239240246230.261\\
2.26525663141579	-239286082853.871\\
2.26535663391585	-239331919477.482\\
2.26545663641591	-239377183143.297\\
2.26555663891597	-239423019766.907\\
2.26565664141604	-239468856390.518\\
2.2657566439161	-239514693014.128\\
2.26585664641616	-239559956679.944\\
2.26595664891622	-239605793303.554\\
2.26605665141629	-239651629927.165\\
2.26615665391635	-239696893592.98\\
2.26625665641641	-239742730216.59\\
2.26635665891647	-239788566840.201\\
2.26645666141654	-239833830506.016\\
2.2665566639166	-239879667129.627\\
2.26665666641666	-239925503753.237\\
2.26675666891672	-239970767419.052\\
2.26685667141679	-240016604042.663\\
2.26695667391685	-240061867708.478\\
2.26705667641691	-240107704332.089\\
2.26715667891697	-240153540955.699\\
2.26725668141704	-240198804621.514\\
2.2673566839171	-240244641245.125\\
2.26745668641716	-240290477868.735\\
2.26755668891722	-240335741534.551\\
2.26765669141729	-240381578158.161\\
2.26775669391735	-240426841823.977\\
2.26785669641741	-240472678447.587\\
2.26795669891747	-240518515071.198\\
2.26805670141754	-240563778737.013\\
2.2681567039176	-240609615360.623\\
2.26825670641766	-240654879026.439\\
2.26835670891772	-240700715650.049\\
2.26845671141779	-240746552273.66\\
2.26855671391785	-240791815939.475\\
2.26865671641791	-240837652563.085\\
2.26875671891797	-240882916228.901\\
2.26885672141804	-240928752852.511\\
2.2689567239181	-240974016518.327\\
2.26905672641816	-241019853141.937\\
2.26915672891822	-241065689765.547\\
2.26925673141829	-241110953431.363\\
2.26935673391835	-241156790054.973\\
2.26945673641841	-241202053720.789\\
2.26955673891847	-241247890344.399\\
2.26965674141854	-241293154010.214\\
2.2697567439186	-241338990633.825\\
2.26985674641866	-241384254299.64\\
2.26995674891872	-241430090923.251\\
2.27005675141879	-241475354589.066\\
2.27015675391885	-241521191212.676\\
2.27025675641891	-241567027836.287\\
2.27035675891897	-241612291502.102\\
2.27045676141904	-241658128125.713\\
2.2705567639191	-241703391791.528\\
2.27065676641916	-241749228415.139\\
2.27075676891922	-241794492080.954\\
2.27085677141929	-241840328704.564\\
2.27095677391935	-241885592370.38\\
2.27105677641941	-241931428993.99\\
2.27115677891947	-241976692659.805\\
2.27125678141954	-242021956325.621\\
2.2713567839196	-242067792949.231\\
2.27145678641966	-242113056615.047\\
2.27155678891972	-242158893238.657\\
2.27165679141979	-242204156904.472\\
2.27175679391985	-242249993528.083\\
2.27185679641991	-242295257193.898\\
2.27195679891997	-242341093817.509\\
2.27205680142004	-242386357483.324\\
2.2721568039201	-242432194106.934\\
2.27225680642016	-242477457772.75\\
2.27235680892022	-242522721438.565\\
2.27245681142029	-242568558062.176\\
2.27255681392035	-242613821727.991\\
2.27265681642041	-242659658351.601\\
2.27275681892047	-242704922017.417\\
2.27285682142054	-242750758641.027\\
2.2729568239206	-242796022306.843\\
2.27305682642066	-242841285972.658\\
2.27315682892072	-242887122596.268\\
2.27325683142079	-242932386262.084\\
2.27335683392085	-242978222885.694\\
2.27345683642091	-243023486551.509\\
2.27355683892097	-243068750217.325\\
2.27365684142104	-243114586840.935\\
2.2737568439211	-243159850506.751\\
2.27385684642116	-243205114172.566\\
2.27395684892122	-243250950796.176\\
2.27405685142129	-243296214461.992\\
2.27415685392135	-243342051085.602\\
2.27425685642141	-243387314751.418\\
2.27435685892147	-243432578417.233\\
2.27445686142154	-243478415040.843\\
2.2745568639216	-243523678706.659\\
2.27465686642166	-243568942372.474\\
2.27475686892172	-243614778996.084\\
2.27485687142179	-243660042661.9\\
2.27495687392185	-243705306327.715\\
2.27505687642191	-243751142951.326\\
2.27515687892197	-243796406617.141\\
2.27525688142204	-243841670282.956\\
2.2753568839221	-243887506906.567\\
2.27545688642216	-243932770572.382\\
2.27555688892222	-243978034238.197\\
2.27565689142229	-244023297904.013\\
2.27575689392235	-244069134527.623\\
2.27585689642241	-244114398193.439\\
2.27595689892247	-244159661859.254\\
2.27605690142254	-244205498482.864\\
2.2761569039226	-244250762148.68\\
2.27625690642266	-244296025814.495\\
2.27635690892272	-244341289480.31\\
2.27645691142279	-244387126103.921\\
2.27655691392285	-244432389769.736\\
2.27665691642291	-244477653435.551\\
2.27675691892297	-244522917101.367\\
2.27685692142304	-244568753724.977\\
2.2769569239231	-244614017390.793\\
2.27705692642316	-244659281056.608\\
2.27715692892322	-244704544722.423\\
2.27725693142329	-244750381346.034\\
2.27735693392335	-244795645011.849\\
2.27745693642341	-244840908677.664\\
2.27755693892347	-244886172343.48\\
2.27765694142354	-244931436009.295\\
2.2777569439236	-244977272632.906\\
2.27785694642366	-245022536298.721\\
2.27795694892372	-245067799964.536\\
2.27805695142379	-245113063630.352\\
2.27815695392385	-245158327296.167\\
2.27825695642391	-245204163919.777\\
2.27835695892397	-245249427585.593\\
2.27845696142404	-245294691251.408\\
2.2785569639241	-245339954917.223\\
2.27865696642416	-245385218583.039\\
2.27875696892422	-245430482248.854\\
2.27885697142429	-245476318872.465\\
2.27895697392435	-245521582538.28\\
2.27905697642441	-245566846204.095\\
2.27915697892447	-245612109869.911\\
2.27925698142454	-245657373535.726\\
2.2793569839246	-245702637201.541\\
2.27945698642466	-245747900867.357\\
2.27955698892472	-245793737490.967\\
2.27965699142479	-245839001156.782\\
2.27975699392485	-245884264822.598\\
2.27985699642491	-245929528488.413\\
2.27995699892497	-245974792154.228\\
2.28005700142504	-246020055820.044\\
2.2801570039251	-246065319485.859\\
2.28025700642516	-246110583151.674\\
2.28035700892522	-246155846817.49\\
2.28045701142529	-246201110483.305\\
2.28055701392535	-246246374149.12\\
2.28065701642541	-246292210772.731\\
2.28075701892547	-246337474438.546\\
2.28085702142554	-246382738104.361\\
2.2809570239256	-246428001770.177\\
2.28105702642566	-246473265435.992\\
2.28115702892572	-246518529101.807\\
2.28125703142579	-246563792767.623\\
2.28135703392585	-246609056433.438\\
2.28145703642591	-246654320099.253\\
2.28155703892597	-246699583765.069\\
2.28165704142604	-246744847430.884\\
2.2817570439261	-246790111096.699\\
2.28185704642616	-246835374762.515\\
2.28195704892622	-246880638428.33\\
2.28205705142629	-246925902094.146\\
2.28215705392635	-246971165759.961\\
2.28225705642641	-247016429425.776\\
2.28235705892647	-247061693091.591\\
2.28245706142654	-247106956757.407\\
2.2825570639266	-247152220423.222\\
2.28265706642666	-247197484089.038\\
2.28275706892672	-247242747754.853\\
2.28285707142679	-247288011420.668\\
2.28295707392685	-247333275086.484\\
2.28305707642691	-247378538752.299\\
2.28315707892697	-247423802418.114\\
2.28325708142704	-247469066083.93\\
2.2833570839271	-247514329749.745\\
2.28345708642716	-247559020457.765\\
2.28355708892722	-247604284123.58\\
2.28365709142729	-247649547789.396\\
2.28375709392735	-247694811455.211\\
2.28385709642741	-247740075121.026\\
2.28395709892747	-247785338786.842\\
2.28405710142754	-247830602452.657\\
2.2841571039276	-247875866118.472\\
2.28425710642766	-247921129784.288\\
2.28435710892772	-247966393450.103\\
2.28445711142779	-248011657115.918\\
2.28455711392785	-248056347823.939\\
2.28465711642791	-248101611489.754\\
2.28475711892797	-248146875155.569\\
2.28485712142804	-248192138821.385\\
2.2849571239281	-248237402487.2\\
2.28505712642816	-248282666153.015\\
2.28515712892822	-248327929818.831\\
2.28525713142829	-248372620526.851\\
2.28535713392835	-248417884192.666\\
2.28545713642841	-248463147858.482\\
2.28555713892847	-248508411524.297\\
2.28565714142854	-248553675190.112\\
2.2857571439286	-248598938855.927\\
2.28585714642866	-248643629563.948\\
2.28595714892872	-248688893229.763\\
2.28605715142879	-248734156895.578\\
2.28615715392885	-248779420561.394\\
2.28625715642891	-248824684227.209\\
2.28635715892897	-248869374935.229\\
2.28645716142904	-248914638601.045\\
2.2865571639291	-248959902266.86\\
2.28665716642916	-249005165932.675\\
2.28675716892922	-249049856640.695\\
2.28685717142929	-249095120306.511\\
2.28695717392935	-249140383972.326\\
2.28705717642941	-249185647638.141\\
2.28715717892947	-249230338346.162\\
2.28725718142954	-249275602011.977\\
2.2873571839296	-249320865677.792\\
2.28745718642966	-249366129343.608\\
2.28755718892972	-249410820051.628\\
2.28765719142979	-249456083717.443\\
2.28775719392985	-249501347383.259\\
2.28785719642991	-249546611049.074\\
2.28795719892997	-249591301757.094\\
2.28805720143004	-249636565422.909\\
2.2881572039301	-249681829088.725\\
2.28825720643016	-249726519796.745\\
2.28835720893022	-249771783462.56\\
2.28845721143029	-249817047128.376\\
2.28855721393035	-249861737836.396\\
2.28865721643041	-249907001502.211\\
2.28875721893047	-249952265168.026\\
2.28885722143054	-249996955876.047\\
2.2889572239306	-250042219541.862\\
2.28905722643066	-250087483207.677\\
2.28915722893072	-250132173915.698\\
2.28925723143079	-250177437581.513\\
2.28935723393085	-250222128289.533\\
2.28945723643091	-250267391955.348\\
2.28955723893097	-250312655621.164\\
2.28965724143104	-250357346329.184\\
2.2897572439311	-250402609994.999\\
2.28985724643116	-250447873660.815\\
2.28995724893122	-250492564368.835\\
2.29005725143129	-250537828034.65\\
2.29015725393135	-250582518742.67\\
2.29025725643141	-250627782408.486\\
2.29035725893147	-250672473116.506\\
2.29045726143154	-250717736782.321\\
2.2905572639316	-250763000448.137\\
2.29065726643166	-250807691156.157\\
2.29075726893172	-250852954821.972\\
2.29085727143179	-250897645529.992\\
2.29095727393185	-250942909195.808\\
2.29105727643191	-250987599903.828\\
2.29115727893197	-251032863569.643\\
2.29125728143204	-251077554277.663\\
2.2913572839321	-251122817943.479\\
2.29145728643216	-251167508651.499\\
2.29155728893222	-251212772317.314\\
2.29165729143229	-251257463025.335\\
2.29175729393235	-251302726691.15\\
2.29185729643241	-251347417399.17\\
2.29195729893247	-251392681064.985\\
2.29205730143254	-251437371773.006\\
2.2921573039326	-251482635438.821\\
2.29225730643266	-251527326146.841\\
2.29235730893272	-251572589812.656\\
2.29245731143279	-251617280520.677\\
2.29255731393285	-251662544186.492\\
2.29265731643291	-251707234894.512\\
2.29275731893297	-251751925602.532\\
2.29285732143304	-251797189268.348\\
2.2929573239331	-251841879976.368\\
2.29305732643316	-251887143642.183\\
2.29315732893322	-251931834350.203\\
2.29325733143329	-251977098016.019\\
2.29335733393335	-252021788724.039\\
2.29345733643341	-252066479432.059\\
2.29355733893347	-252111743097.875\\
2.29365734143354	-252156433805.895\\
2.2937573439336	-252201124513.915\\
2.29385734643366	-252246388179.73\\
2.29395734893372	-252291078887.751\\
2.29405735143379	-252336342553.566\\
2.29415735393385	-252381033261.586\\
2.29425735643391	-252425723969.606\\
2.29435735893397	-252470987635.422\\
2.29445736143404	-252515678343.442\\
2.2945573639341	-252560369051.462\\
2.29465736643416	-252605632717.277\\
2.29475736893422	-252650323425.298\\
2.29485737143429	-252695014133.318\\
2.29495737393435	-252740277799.133\\
2.29505737643441	-252784968507.153\\
2.29515737893447	-252829659215.173\\
2.29525738143454	-252874349923.194\\
2.2953573839346	-252919613589.009\\
2.29545738643466	-252964304297.029\\
2.29555738893472	-253008995005.049\\
2.29565739143479	-253054258670.865\\
2.29575739393485	-253098949378.885\\
2.29585739643491	-253143640086.905\\
2.29595739893497	-253188330794.925\\
2.29605740143504	-253233594460.741\\
2.2961574039351	-253278285168.761\\
2.29625740643516	-253322975876.781\\
2.29635740893522	-253367666584.801\\
2.29645741143529	-253412357292.822\\
2.29655741393535	-253457620958.637\\
2.29665741643541	-253502311666.657\\
2.29675741893547	-253547002374.677\\
2.29685742143554	-253591693082.697\\
2.2969574239356	-253636383790.718\\
2.29705742643566	-253681647456.533\\
2.29715742893572	-253726338164.553\\
2.29725743143579	-253771028872.573\\
2.29735743393585	-253815719580.594\\
2.29745743643591	-253860410288.614\\
2.29755743893597	-253905100996.634\\
2.29765744143604	-253950364662.449\\
2.2977574439361	-253995055370.47\\
2.29785744643616	-254039746078.49\\
2.29795744893622	-254084436786.51\\
2.29805745143629	-254129127494.53\\
2.29815745393635	-254173818202.55\\
2.29825745643641	-254218508910.571\\
2.29835745893647	-254263199618.591\\
2.29845746143654	-254308463284.406\\
2.2985574639366	-254353153992.426\\
2.29865746643666	-254397844700.447\\
2.29875746893672	-254442535408.467\\
2.29885747143679	-254487226116.487\\
2.29895747393685	-254531916824.507\\
2.29905747643691	-254576607532.527\\
2.29915747893697	-254621298240.548\\
2.29925748143704	-254665988948.568\\
2.2993574839371	-254710679656.588\\
2.29945748643716	-254755370364.608\\
2.29955748893722	-254800061072.628\\
2.29965749143729	-254844751780.649\\
2.29975749393735	-254889442488.669\\
2.29985749643741	-254934133196.689\\
2.29995749893747	-254978823904.709\\
2.30005750143754	-255023514612.729\\
2.3001575039376	-255068205320.75\\
2.30025750643766	-255112896028.77\\
2.30035750893772	-255157586736.79\\
2.30045751143779	-255202277444.81\\
2.30055751393785	-255246968152.83\\
2.30065751643791	-255291658860.851\\
2.30075751893797	-255336349568.871\\
2.30085752143804	-255381040276.891\\
2.3009575239381	-255425730984.911\\
2.30105752643816	-255470421692.931\\
2.30115752893822	-255515112400.952\\
2.30125753143829	-255559803108.972\\
2.30135753393835	-255604493816.992\\
2.30145753643841	-255648611567.217\\
2.30155753893847	-255693302275.237\\
2.30165754143854	-255737992983.258\\
2.3017575439386	-255782683691.278\\
2.30185754643866	-255827374399.298\\
2.30195754893872	-255872065107.318\\
2.30205755143879	-255916755815.338\\
2.30215755393885	-255961446523.359\\
2.30225755643891	-256005564273.584\\
2.30235755893897	-256050254981.604\\
2.30245756143904	-256094945689.624\\
2.3025575639391	-256139636397.644\\
2.30265756643916	-256184327105.664\\
2.30275756893922	-256229017813.685\\
2.30285757143929	-256273135563.91\\
2.30295757393935	-256317826271.93\\
2.30305757643941	-256362516979.95\\
2.30315757893947	-256407207687.97\\
2.30325758143954	-256451898395.991\\
2.3033575839396	-256496016146.216\\
2.30345758643966	-256540706854.236\\
2.30355758893972	-256585397562.256\\
2.30365759143979	-256630088270.276\\
2.30375759393985	-256674206020.501\\
2.30385759643991	-256718896728.522\\
2.30395759893997	-256763587436.542\\
2.30405760144004	-256808278144.562\\
2.3041576039401	-256852395894.787\\
2.30425760644016	-256897086602.807\\
2.30435760894022	-256941777310.827\\
2.30445761144029	-256986468018.848\\
2.30455761394035	-257030585769.073\\
2.30465761644041	-257075276477.093\\
2.30475761894047	-257119967185.113\\
2.30485762144054	-257164084935.338\\
2.3049576239406	-257208775643.358\\
2.30505762644066	-257253466351.379\\
2.30515762894072	-257297584101.604\\
2.30525763144079	-257342274809.624\\
2.30535763394085	-257386965517.644\\
2.30545763644091	-257431083267.869\\
2.30555763894097	-257475773975.889\\
2.30565764144104	-257519891726.114\\
2.3057576439411	-257564582434.135\\
2.30585764644116	-257609273142.155\\
2.30595764894122	-257653390892.38\\
2.30605765144129	-257698081600.4\\
2.30615765394135	-257742199350.625\\
2.30625765644141	-257786890058.645\\
2.30635765894147	-257831580766.666\\
2.30645766144154	-257875698516.891\\
2.3065576639416	-257920389224.911\\
2.30665766644166	-257964506975.136\\
2.30675766894172	-258009197683.156\\
2.30685767144179	-258053315433.381\\
2.30695767394185	-258098006141.401\\
2.30705767644191	-258142123891.626\\
2.30715767894197	-258186814599.647\\
2.30725768144204	-258231505307.667\\
2.3073576839421	-258275623057.892\\
2.30745768644216	-258320313765.912\\
2.30755768894222	-258364431516.137\\
2.30765769144229	-258409122224.157\\
2.30775769394235	-258453239974.383\\
2.30785769644241	-258497357724.608\\
2.30795769894247	-258542048432.628\\
2.30805770144254	-258586166182.853\\
2.3081577039426	-258630856890.873\\
2.30825770644266	-258674974641.098\\
2.30835770894272	-258719665349.118\\
2.30845771144279	-258763783099.343\\
2.30855771394285	-258808473807.364\\
2.30865771644291	-258852591557.589\\
2.30875771894297	-258896709307.814\\
2.30885772144304	-258941400015.834\\
2.3089577239431	-258985517766.059\\
2.30905772644316	-259030208474.079\\
2.30915772894322	-259074326224.304\\
2.30925773144329	-259118443974.529\\
2.30935773394335	-259163134682.55\\
2.30945773644341	-259207252432.775\\
2.30955773894347	-259251370183\\
2.30965774144354	-259296060891.02\\
2.3097577439436	-259340178641.245\\
2.30985774644366	-259384296391.47\\
2.30995774894372	-259428987099.49\\
2.31005775144379	-259473104849.715\\
2.31015775394385	-259517222599.94\\
2.31025775644391	-259561913307.961\\
2.31035775894397	-259606031058.186\\
2.31045776144404	-259650148808.411\\
2.3105577639441	-259694839516.431\\
2.31065776644416	-259738957266.656\\
2.31075776894422	-259783075016.881\\
2.31085777144429	-259827192767.106\\
2.31095777394435	-259871883475.126\\
2.31105777644441	-259916001225.352\\
2.31115777894447	-259960118975.577\\
2.31125778144454	-260004236725.802\\
2.3113577839446	-260048927433.822\\
2.31145778644466	-260093045184.047\\
2.31155778894472	-260137162934.272\\
2.31165779144479	-260181280684.497\\
2.31175779394485	-260225398434.722\\
2.31185779644491	-260270089142.742\\
2.31195779894497	-260314206892.967\\
2.31205780144504	-260358324643.193\\
2.3121578039451	-260402442393.418\\
2.31225780644516	-260446560143.643\\
2.31235780894522	-260490677893.868\\
2.31245781144529	-260534795644.093\\
2.31255781394535	-260579486352.113\\
2.31265781644541	-260623604102.338\\
2.31275781894547	-260667721852.563\\
2.31285782144554	-260711839602.788\\
2.3129578239456	-260755957353.013\\
2.31305782644566	-260800075103.238\\
2.31315782894572	-260844192853.463\\
2.31325783144579	-260888310603.689\\
2.31335783394585	-260932428353.914\\
2.31345783644591	-260976546104.139\\
2.31355783894597	-261021236812.159\\
2.31365784144604	-261065354562.384\\
2.3137578439461	-261109472312.609\\
2.31385784644616	-261153590062.834\\
2.31395784894622	-261197707813.059\\
2.31405785144629	-261241825563.284\\
2.31415785394635	-261285943313.509\\
2.31425785644641	-261330061063.734\\
2.31435785894647	-261374178813.959\\
2.31445786144654	-261418296564.185\\
2.3145578639466	-261462414314.41\\
2.31465786644666	-261506532064.635\\
2.31475786894672	-261550649814.86\\
2.31485787144679	-261594767565.085\\
2.31495787394685	-261638885315.31\\
2.31505787644691	-261683003065.535\\
2.31515787894697	-261726547857.965\\
2.31525788144704	-261770665608.19\\
2.3153578839471	-261814783358.415\\
2.31545788644716	-261858901108.64\\
2.31555788894722	-261903018858.865\\
2.31565789144729	-261947136609.09\\
2.31575789394735	-261991254359.315\\
2.31585789644741	-262035372109.54\\
2.31595789894747	-262079489859.766\\
2.31605790144754	-262123607609.991\\
2.3161579039476	-262167152402.421\\
2.31625790644766	-262211270152.646\\
2.31635790894772	-262255387902.871\\
2.31645791144779	-262299505653.096\\
2.31655791394785	-262343623403.321\\
2.31665791644791	-262387741153.546\\
2.31675791894797	-262431285945.976\\
2.31685792144804	-262475403696.201\\
2.3169579239481	-262519521446.426\\
2.31705792644816	-262563639196.651\\
2.31715792894822	-262607756946.876\\
2.31725793144829	-262651301739.306\\
2.31735793394835	-262695419489.531\\
2.31745793644841	-262739537239.756\\
2.31755793894847	-262783654989.981\\
2.31765794144854	-262827199782.411\\
2.3177579439486	-262871317532.636\\
2.31785794644866	-262915435282.861\\
2.31795794894872	-262959553033.086\\
2.31805795144879	-263003097825.516\\
2.31815795394885	-263047215575.741\\
2.31825795644891	-263091333325.967\\
2.31835795894897	-263134878118.396\\
2.31845796144904	-263178995868.622\\
2.3185579639491	-263223113618.847\\
2.31865796644916	-263266658411.277\\
2.31875796894922	-263310776161.502\\
2.31885797144929	-263354893911.727\\
2.31895797394935	-263398438704.157\\
2.31905797644941	-263442556454.382\\
2.31915797894947	-263486674204.607\\
2.31925798144954	-263530218997.037\\
2.3193579839496	-263574336747.262\\
2.31945798644966	-263617881539.692\\
2.31955798894972	-263661999289.917\\
2.31965799144979	-263706117040.142\\
2.31975799394985	-263749661832.572\\
2.31985799644991	-263793779582.797\\
2.31995799894997	-263837324375.227\\
2.32005800145004	-263881442125.452\\
2.3201580039501	-263924986917.882\\
2.32025800645016	-263969104668.107\\
2.32035800895022	-264012649460.537\\
2.32045801145029	-264056767210.762\\
2.32055801395035	-264100312003.192\\
2.32065801645041	-264144429753.417\\
2.32075801895047	-264187974545.847\\
2.32085802145054	-264232092296.072\\
2.3209580239506	-264275637088.502\\
2.32105802645066	-264319754838.727\\
2.32115802895072	-264363299631.157\\
2.32125803145079	-264407417381.382\\
2.32135803395085	-264450962173.812\\
2.32145803645091	-264495079924.037\\
2.32155803895097	-264538624716.467\\
2.32165804145104	-264582742466.692\\
2.3217580439511	-264626287259.122\\
2.32185804645116	-264669832051.552\\
2.32195804895122	-264713949801.777\\
2.32205805145129	-264757494594.207\\
2.32215805395135	-264801612344.432\\
2.32225805645141	-264845157136.862\\
2.32235805895147	-264888701929.292\\
2.32245806145154	-264932819679.517\\
2.3225580639516	-264976364471.947\\
2.32265806645166	-265019909264.377\\
2.32275806895172	-265064027014.602\\
2.32285807145179	-265107571807.032\\
2.32295807395185	-265151116599.462\\
2.32305807645191	-265195234349.687\\
2.32315807895197	-265238779142.117\\
2.32325808145204	-265282323934.547\\
2.3233580839521	-265325868726.977\\
2.32345808645216	-265369986477.202\\
2.32355808895222	-265413531269.632\\
2.32365809145229	-265457076062.062\\
2.32375809395235	-265500620854.492\\
2.32385809645241	-265544738604.717\\
2.32395809895247	-265588283397.147\\
2.32405810145254	-265631828189.577\\
2.3241581039526	-265675372982.007\\
2.32425810645266	-265719490732.232\\
2.32435810895272	-265763035524.662\\
2.32445811145279	-265806580317.092\\
2.32455811395285	-265850125109.521\\
2.32465811645291	-265893669901.951\\
2.32475811895297	-265937214694.381\\
2.32485812145304	-265981332444.606\\
2.3249581239531	-266024877237.036\\
2.32505812645316	-266068422029.466\\
2.32515812895322	-266111966821.896\\
2.32525813145329	-266155511614.326\\
2.32535813395335	-266199056406.756\\
2.32545813645341	-266242601199.186\\
2.32555813895347	-266286145991.616\\
2.32565814145354	-266330263741.841\\
2.3257581439536	-266373808534.271\\
2.32585814645366	-266417353326.701\\
2.32595814895372	-266460898119.131\\
2.32605815145379	-266504442911.561\\
2.32615815395385	-266547987703.991\\
2.32625815645391	-266591532496.421\\
2.32635815895397	-266635077288.851\\
2.32645816145404	-266678622081.281\\
2.3265581639541	-266722166873.711\\
2.32665816645416	-266765711666.141\\
2.32675816895422	-266809256458.57\\
2.32685817145429	-266852801251\\
2.32695817395435	-266896346043.43\\
2.32705817645441	-266939890835.86\\
2.32715817895447	-266983435628.29\\
2.32725818145454	-267026980420.72\\
2.3273581839546	-267070525213.15\\
2.32745818645466	-267114070005.58\\
2.32755818895472	-267157614798.01\\
2.32765819145479	-267200586632.645\\
2.32775819395485	-267244131425.075\\
2.32785819645491	-267287676217.505\\
2.32795819895497	-267331221009.935\\
2.32805820145504	-267374765802.365\\
2.3281582039551	-267418310594.795\\
2.32825820645516	-267461855387.224\\
2.32835820895522	-267505400179.654\\
2.32845821145529	-267548372014.289\\
2.32855821395535	-267591916806.719\\
2.32865821645541	-267635461599.149\\
2.32875821895547	-267679006391.579\\
2.32885822145554	-267722551184.009\\
2.3289582239556	-267766095976.439\\
2.32905822645566	-267809067811.074\\
2.32915822895572	-267852612603.504\\
2.32925823145579	-267896157395.934\\
2.32935823395585	-267939702188.364\\
2.32945823645591	-267982674022.998\\
2.32955823895597	-268026218815.428\\
2.32965824145604	-268069763607.858\\
2.3297582439561	-268113308400.288\\
2.32985824645616	-268156280234.923\\
2.32995824895622	-268199825027.353\\
2.33005825145629	-268243369819.783\\
2.33015825395635	-268286341654.418\\
2.33025825645641	-268329886446.848\\
2.33035825895647	-268373431239.278\\
2.33045826145654	-268416403073.912\\
2.3305582639566	-268459947866.342\\
2.33065826645666	-268503492658.772\\
2.33075826895672	-268546464493.407\\
2.33085827145679	-268590009285.837\\
2.33095827395685	-268633554078.267\\
2.33105827645691	-268676525912.902\\
2.33115827895697	-268720070705.332\\
2.33125828145704	-268763042539.967\\
2.3313582839571	-268806587332.397\\
2.33145828645716	-268850132124.826\\
2.33155828895722	-268893103959.461\\
2.33165829145729	-268936648751.891\\
2.33175829395735	-268979620586.526\\
2.33185829645741	-269023165378.956\\
2.33195829895747	-269066137213.591\\
2.33205830145754	-269109682006.021\\
2.3321583039576	-269152653840.656\\
2.33225830645766	-269196198633.085\\
2.33235830895772	-269239170467.72\\
2.33245831145779	-269282715260.15\\
2.33255831395785	-269325687094.785\\
2.33265831645791	-269369231887.215\\
2.33275831895797	-269412203721.85\\
2.33285832145804	-269455748514.28\\
2.3329583239581	-269498720348.915\\
2.33305832645816	-269541692183.549\\
2.33315832895822	-269585236975.979\\
2.33325833145829	-269628208810.614\\
2.33335833395835	-269671753603.044\\
2.33345833645841	-269714725437.679\\
2.33355833895847	-269757697272.314\\
2.33365834145854	-269801242064.744\\
2.3337583439586	-269844213899.378\\
2.33385834645866	-269887758691.808\\
2.33395834895872	-269930730526.443\\
2.33405835145879	-269973702361.078\\
2.33415835395885	-270017247153.508\\
2.33425835645891	-270060218988.143\\
2.33435835895897	-270103190822.778\\
2.33445836145904	-270146735615.207\\
2.3345583639591	-270189707449.842\\
2.33465836645916	-270232679284.477\\
2.33475836895922	-270275651119.112\\
2.33485837145929	-270319195911.542\\
2.33495837395935	-270362167746.177\\
2.33505837645941	-270405139580.811\\
2.33515837895947	-270448111415.446\\
2.33525838145954	-270491656207.876\\
2.3353583839596	-270534628042.511\\
2.33545838645966	-270577599877.146\\
2.33555838895972	-270620571711.781\\
2.33565839145979	-270663543546.415\\
2.33575839395985	-270707088338.845\\
2.33585839645991	-270750060173.48\\
2.33595839895997	-270793032008.115\\
2.33605840146004	-270836003842.75\\
2.3361584039601	-270878975677.385\\
2.33625840646016	-270921947512.02\\
2.33635840896022	-270964919346.654\\
2.33645841146029	-271007891181.289\\
2.33655841396035	-271051435973.719\\
2.33665841646041	-271094407808.354\\
2.33675841896047	-271137379642.989\\
2.33685842146054	-271180351477.624\\
2.3369584239606	-271223323312.258\\
2.33705842646066	-271266295146.893\\
2.33715842896072	-271309266981.528\\
2.33725843146079	-271352238816.163\\
2.33735843396085	-271395210650.798\\
2.33745843646091	-271438182485.432\\
2.33755843896097	-271481154320.067\\
2.33765844146104	-271524126154.702\\
2.3377584439611	-271567097989.337\\
2.33785844646116	-271610069823.972\\
2.33795844896122	-271653041658.606\\
2.33805845146129	-271696013493.241\\
2.33815845396135	-271738985327.876\\
2.33825845646141	-271781957162.511\\
2.33835845896147	-271824928997.146\\
2.33845846146154	-271867900831.78\\
2.3385584639616	-271910872666.415\\
2.33865846646166	-271953271543.255\\
2.33875846896172	-271996243377.89\\
2.33885847146179	-272039215212.525\\
2.33895847396185	-272082187047.159\\
2.33905847646191	-272125158881.794\\
2.33915847896197	-272168130716.429\\
2.33925848146204	-272211102551.064\\
2.3393584839621	-272253501427.904\\
2.33945848646216	-272296473262.538\\
2.33955848896222	-272339445097.173\\
2.33965849146229	-272382416931.808\\
2.33975849396235	-272425388766.443\\
2.33985849646241	-272467787643.282\\
2.33995849896247	-272510759477.917\\
2.34005850146254	-272553731312.552\\
2.3401585039626	-272596703147.187\\
2.34025850646266	-272639102024.027\\
2.34035850896272	-272682073858.661\\
2.34045851146279	-272725045693.296\\
2.34055851396285	-272768017527.931\\
2.34065851646291	-272810416404.771\\
2.34075851896297	-272853388239.406\\
2.34085852146304	-272896360074.04\\
2.3409585239631	-272938758950.88\\
2.34105852646316	-272981730785.515\\
2.34115852896322	-273024702620.15\\
2.34125853146329	-273067101496.989\\
2.34135853396335	-273110073331.624\\
2.34145853646341	-273153045166.259\\
2.34155853896347	-273195444043.099\\
2.34165854146354	-273238415877.733\\
2.3417585439636	-273280814754.573\\
2.34185854646366	-273323786589.208\\
2.34195854896372	-273366758423.843\\
2.34205855146379	-273409157300.682\\
2.34215855396385	-273452129135.317\\
2.34225855646391	-273494528012.157\\
2.34235855896397	-273537499846.792\\
2.34245856146404	-273579898723.631\\
2.3425585639641	-273622870558.266\\
2.34265856646416	-273665269435.106\\
2.34275856896422	-273708241269.741\\
2.34285857146429	-273750640146.58\\
2.34295857396435	-273793611981.215\\
2.34305857646441	-273836010858.055\\
2.34315857896447	-273878982692.69\\
2.34325858146454	-273921381569.529\\
2.3433585839646	-273963780446.369\\
2.34345858646466	-274006752281.004\\
2.34355858896472	-274049151157.844\\
2.34365859146479	-274092122992.478\\
2.34375859396485	-274134521869.318\\
2.34385859646491	-274176920746.158\\
2.34395859896497	-274219892580.793\\
2.34405860146504	-274262291457.632\\
2.3441586039651	-274305263292.267\\
2.34425860646516	-274347662169.107\\
2.34435860896522	-274390061045.946\\
2.34445861146529	-274433032880.581\\
2.34455861396535	-274475431757.421\\
2.34465861646541	-274517830634.261\\
2.34475861896547	-274560229511.1\\
2.34485862146554	-274603201345.735\\
2.3449586239656	-274645600222.575\\
2.34505862646566	-274687999099.414\\
2.34515862896572	-274730397976.254\\
2.34525863146579	-274773369810.889\\
2.34535863396585	-274815768687.729\\
2.34545863646591	-274858167564.568\\
2.34555863896597	-274900566441.408\\
2.34565864146604	-274943538276.043\\
2.3457586439661	-274985937152.882\\
2.34585864646616	-275028336029.722\\
2.34595864896622	-275070734906.562\\
2.34605865146629	-275113133783.401\\
2.34615865396635	-275155532660.241\\
2.34625865646641	-275198504494.876\\
2.34635865896647	-275240903371.716\\
2.34645866146654	-275283302248.555\\
2.3465586639666	-275325701125.395\\
2.34665866646666	-275368100002.235\\
2.34675866896672	-275410498879.074\\
2.34685867146679	-275452897755.914\\
2.34695867396685	-275495296632.754\\
2.34705867646691	-275537695509.593\\
2.34715867896697	-275580094386.433\\
2.34725868146704	-275622493263.273\\
2.3473586839671	-275664892140.112\\
2.34745868646716	-275707291016.952\\
2.34755868896722	-275749689893.792\\
2.34765869146729	-275792088770.632\\
2.34775869396735	-275834487647.471\\
2.34785869646741	-275876886524.311\\
2.34795869896747	-275919285401.151\\
2.34805870146754	-275961684277.99\\
2.3481587039676	-276004083154.83\\
2.34825870646766	-276046482031.67\\
2.34835870896772	-276088880908.509\\
2.34845871146779	-276131279785.349\\
2.34855871396785	-276173678662.189\\
2.34865871646791	-276216077539.028\\
2.34875871896797	-276257903458.073\\
2.34885872146804	-276300302334.913\\
2.3489587239681	-276342701211.752\\
2.34905872646816	-276385100088.592\\
2.34915872896822	-276427498965.432\\
2.34925873146829	-276469897842.271\\
2.34935873396835	-276511723761.316\\
2.34945873646841	-276554122638.156\\
2.34955873896847	-276596521514.995\\
2.34965874146854	-276638920391.835\\
2.3497587439686	-276681319268.675\\
2.34985874646866	-276723145187.719\\
2.34995874896872	-276765544064.559\\
2.35005875146879	-276807942941.398\\
2.35015875396885	-276849768860.443\\
2.35025875646891	-276892167737.283\\
2.35035875896897	-276934566614.122\\
2.35045876146904	-276976965490.962\\
2.3505587639691	-277018791410.007\\
2.35065876646916	-277061190286.846\\
2.35075876896922	-277103589163.686\\
2.35085877146929	-277145415082.731\\
2.35095877396935	-277187813959.57\\
2.35105877646941	-277229639878.615\\
2.35115877896947	-277272038755.454\\
2.35125878146954	-277314437632.294\\
2.3513587839696	-277356263551.339\\
2.35145878646966	-277398662428.178\\
2.35155878896972	-277440488347.223\\
2.35165879146979	-277482887224.063\\
2.35175879396985	-277525286100.902\\
2.35185879646991	-277567112019.947\\
2.35195879896997	-277609510896.786\\
2.35205880147004	-277651336815.831\\
2.3521588039701	-277693735692.671\\
2.35225880647016	-277735561611.715\\
2.35235880897022	-277777960488.555\\
2.35245881147029	-277819786407.599\\
2.35255881397035	-277862185284.439\\
2.35265881647041	-277904011203.484\\
2.35275881897047	-277945837122.528\\
2.35285882147054	-277988235999.368\\
2.3529588239706	-278030061918.412\\
2.35305882647066	-278072460795.252\\
2.35315882897072	-278114286714.297\\
2.35325883147079	-278156112633.341\\
2.35335883397085	-278198511510.181\\
2.35345883647091	-278240337429.226\\
2.35355883897097	-278282736306.065\\
2.35365884147104	-278324562225.11\\
2.3537588439711	-278366388144.154\\
2.35385884647116	-278408214063.199\\
2.35395884897122	-278450612940.039\\
2.35405885147129	-278492438859.083\\
2.35415885397135	-278534264778.128\\
2.35425885647141	-278576663654.967\\
2.35435885897147	-278618489574.012\\
2.35445886147154	-278660315493.056\\
2.3545588639716	-278702141412.101\\
2.35465886647166	-278744540288.941\\
2.35475886897172	-278786366207.985\\
2.35485887147179	-278828192127.03\\
2.35495887397185	-278870018046.074\\
2.35505887647191	-278911843965.119\\
2.35515887897197	-278954242841.958\\
2.35525888147204	-278996068761.003\\
2.3553588839721	-279037894680.048\\
2.35545888647216	-279079720599.092\\
2.35555888897222	-279121546518.137\\
2.35565889147229	-279163372437.181\\
2.35575889397235	-279205198356.226\\
2.35585889647241	-279247024275.27\\
2.35595889897247	-279289423152.11\\
2.35605890147254	-279331249071.155\\
2.3561589039726	-279373074990.199\\
2.35625890647266	-279414900909.244\\
2.35635890897272	-279456726828.288\\
2.35645891147279	-279498552747.333\\
2.35655891397285	-279540378666.377\\
2.35665891647291	-279582204585.422\\
2.35675891897297	-279624030504.466\\
2.35685892147304	-279665856423.511\\
2.3569589239731	-279707682342.556\\
2.35705892647316	-279749508261.6\\
2.35715892897322	-279791334180.645\\
2.35725893147329	-279833160099.689\\
2.35735893397335	-279874986018.734\\
2.35745893647341	-279916238979.983\\
2.35755893897347	-279958064899.028\\
2.35765894147354	-279999890818.072\\
2.3577589439736	-280041716737.117\\
2.35785894647366	-280083542656.161\\
2.35795894897372	-280125368575.206\\
2.35805895147379	-280167194494.25\\
2.35815895397385	-280208447455.5\\
2.35825895647391	-280250273374.544\\
2.35835895897397	-280292099293.589\\
2.35845896147404	-280333925212.634\\
2.3585589639741	-280375751131.678\\
2.35865896647416	-280417004092.927\\
2.35875896897422	-280458830011.972\\
2.35885897147429	-280500655931.017\\
2.35895897397435	-280542481850.061\\
2.35905897647441	-280583734811.311\\
2.35915897897447	-280625560730.355\\
2.35925898147454	-280667386649.4\\
2.3593589839746	-280709212568.444\\
2.35945898647466	-280750465529.694\\
2.35955898897472	-280792291448.738\\
2.35965899147479	-280834117367.783\\
2.35975899397485	-280875370329.032\\
2.35985899647491	-280917196248.077\\
2.35995899897497	-280958449209.326\\
2.36005900147504	-281000275128.371\\
2.3601590039751	-281042101047.415\\
2.36025900647516	-281083354008.665\\
2.36035900897522	-281125179927.709\\
2.36045901147529	-281166432888.959\\
2.36055901397535	-281208258808.003\\
2.36065901647541	-281250084727.048\\
2.36075901897547	-281291337688.297\\
2.36085902147554	-281333163607.342\\
2.3609590239756	-281374416568.591\\
2.36105902647566	-281416242487.636\\
2.36115902897572	-281457495448.885\\
2.36125903147579	-281499321367.93\\
2.36135903397585	-281540574329.179\\
2.36145903647591	-281581827290.428\\
2.36155903897597	-281623653209.473\\
2.36165904147604	-281664906170.722\\
2.3617590439761	-281706732089.767\\
2.36185904647616	-281747985051.016\\
2.36195904897622	-281789810970.061\\
2.36205905147629	-281831063931.31\\
2.36215905397635	-281872316892.56\\
2.36225905647641	-281914142811.604\\
2.36235905897647	-281955395772.854\\
2.36245906147654	-281996648734.103\\
2.3625590639766	-282038474653.148\\
2.36265906647666	-282079727614.397\\
2.36275906897672	-282120980575.647\\
2.36285907147679	-282162233536.896\\
2.36295907397685	-282204059455.941\\
2.36305907647691	-282245312417.19\\
2.36315907897697	-282286565378.439\\
2.36325908147704	-282327818339.689\\
2.3633590839771	-282369644258.733\\
2.36345908647716	-282410897219.983\\
2.36355908897722	-282452150181.232\\
2.36365909147729	-282493403142.482\\
2.36375909397735	-282534656103.731\\
2.36385909647741	-282576482022.776\\
2.36395909897747	-282617734984.025\\
2.36405910147754	-282658987945.274\\
2.3641591039776	-282700240906.524\\
2.36425910647766	-282741493867.773\\
2.36435910897772	-282782746829.023\\
2.36445911147779	-282823999790.272\\
2.36455911397785	-282865252751.521\\
2.36465911647791	-282907078670.566\\
2.36475911897797	-282948331631.815\\
2.36485912147804	-282989584593.065\\
2.3649591239781	-283030837554.314\\
2.36505912647816	-283072090515.564\\
2.36515912897822	-283113343476.813\\
2.36525913147829	-283154596438.063\\
2.36535913397835	-283195849399.312\\
2.36545913647841	-283237102360.561\\
2.36555913897847	-283278355321.811\\
2.36565914147854	-283319608283.06\\
2.3657591439786	-283360288286.515\\
2.36585914647866	-283401541247.764\\
2.36595914897872	-283442794209.013\\
2.36605915147879	-283484047170.263\\
2.36615915397885	-283525300131.512\\
2.36625915647891	-283566553092.762\\
2.36635915897897	-283607806054.011\\
2.36645916147904	-283649059015.26\\
2.3665591639791	-283690311976.51\\
2.36665916647916	-283730991979.964\\
2.36675916897922	-283772244941.214\\
2.36685917147929	-283813497902.463\\
2.36695917397935	-283854750863.712\\
2.36705917647941	-283895430867.167\\
2.36715917897947	-283936683828.416\\
2.36725918147954	-283977936789.666\\
2.3673591839796	-284019189750.915\\
2.36745918647966	-284059869754.369\\
2.36755918897972	-284101122715.619\\
2.36765919147979	-284142375676.868\\
2.36775919397985	-284183628638.118\\
2.36785919647991	-284224308641.572\\
2.36795919897997	-284265561602.821\\
2.36805920148004	-284306814564.071\\
2.3681592039801	-284347494567.525\\
2.36825920648016	-284388747528.774\\
2.36835920898022	-284429427532.229\\
2.36845921148029	-284470680493.478\\
2.36855921398035	-284511933454.727\\
2.36865921648041	-284552613458.182\\
2.36875921898047	-284593866419.431\\
2.36885922148054	-284634546422.885\\
2.3689592239806	-284675799384.135\\
2.36905922648066	-284716479387.589\\
2.36915922898072	-284757732348.839\\
2.36925923148079	-284798412352.293\\
2.36935923398085	-284839665313.542\\
2.36945923648091	-284880345316.997\\
2.36955923898097	-284921598278.246\\
2.36965924148104	-284962278281.7\\
2.3697592439811	-285003531242.95\\
2.36985924648116	-285044211246.404\\
2.36995924898122	-285085464207.653\\
2.37005925148129	-285126144211.108\\
2.37015925398135	-285166824214.562\\
2.37025925648141	-285208077175.811\\
2.37035925898147	-285248757179.266\\
2.37045926148154	-285289437182.72\\
2.3705592639816	-285330690143.969\\
2.37065926648166	-285371370147.424\\
2.37075926898172	-285412050150.878\\
2.37085927148179	-285453303112.127\\
2.37095927398185	-285493983115.582\\
2.37105927648191	-285534663119.036\\
2.37115927898197	-285575916080.285\\
2.37125928148204	-285616596083.74\\
2.3713592839821	-285657276087.194\\
2.37145928648216	-285697956090.648\\
2.37155928898222	-285739209051.898\\
2.37165929148229	-285779889055.352\\
2.37175929398235	-285820569058.806\\
2.37185929648241	-285861249062.261\\
2.37195929898247	-285901929065.715\\
2.37205930148254	-285942609069.169\\
2.3721593039826	-285983862030.419\\
2.37225930648266	-286024542033.873\\
2.37235930898272	-286065222037.327\\
2.37245931148279	-286105902040.781\\
2.37255931398285	-286146582044.236\\
2.37265931648291	-286187262047.69\\
2.37275931898297	-286227942051.144\\
2.37285932148304	-286268622054.599\\
2.3729593239831	-286309302058.053\\
2.37305932648316	-286349982061.507\\
2.37315932898322	-286390662064.961\\
2.37325933148329	-286431342068.416\\
2.37335933398335	-286472022071.87\\
2.37345933648341	-286512702075.324\\
2.37355933898347	-286553382078.779\\
2.37365934148354	-286594062082.233\\
2.3737593439836	-286634742085.687\\
2.37385934648366	-286675422089.141\\
2.37395934898372	-286716102092.596\\
2.37405935148379	-286756782096.05\\
2.37415935398385	-286797462099.504\\
2.37425935648391	-286838142102.959\\
2.37435935898397	-286878249148.618\\
2.37445936148404	-286918929152.072\\
2.3745593639841	-286959609155.526\\
2.37465936648416	-287000289158.981\\
2.37475936898422	-287040969162.435\\
2.37485937148429	-287081076208.094\\
2.37495937398435	-287121756211.548\\
2.37505937648441	-287162436215.003\\
2.37515937898447	-287203116218.457\\
2.37525938148454	-287243223264.116\\
2.3753593839846	-287283903267.57\\
2.37545938648466	-287324583271.025\\
2.37555938898472	-287365263274.479\\
2.37565939148479	-287405370320.138\\
2.37575939398485	-287446050323.592\\
2.37585939648491	-287486730327.047\\
2.37595939898497	-287526837372.706\\
2.37605940148504	-287567517376.16\\
2.3761594039851	-287608197379.614\\
2.37625940648516	-287648304425.274\\
2.37635940898522	-287688984428.728\\
2.37645941148529	-287729091474.387\\
2.37655941398535	-287769771477.841\\
2.37665941648541	-287809878523.5\\
2.37675941898547	-287850558526.955\\
2.37685942148554	-287891238530.409\\
2.3769594239856	-287931345576.068\\
2.37705942648566	-287972025579.523\\
2.37715942898572	-288012132625.182\\
2.37725943148579	-288052812628.636\\
2.37735943398585	-288092919674.295\\
2.37745943648591	-288133026719.954\\
2.37755943898597	-288173706723.409\\
2.37765944148604	-288213813769.068\\
2.3777594439861	-288254493772.522\\
2.37785944648616	-288294600818.181\\
2.37795944898622	-288334707863.84\\
2.37805945148629	-288375387867.295\\
2.37815945398635	-288415494912.954\\
2.37825945648641	-288456174916.408\\
2.37835945898647	-288496281962.067\\
2.37845946148654	-288536389007.726\\
2.3785594639866	-288577069011.181\\
2.37865946648666	-288617176056.84\\
2.37875946898672	-288657283102.499\\
2.37885947148679	-288697390148.158\\
2.37895947398685	-288738070151.612\\
2.37905947648691	-288778177197.272\\
2.37915947898697	-288818284242.931\\
2.37925948148704	-288858391288.59\\
2.3793594839871	-288898498334.249\\
2.37945948648716	-288939178337.703\\
2.37955948898722	-288979285383.363\\
2.37965949148729	-289019392429.022\\
2.37975949398735	-289059499474.681\\
2.37985949648741	-289099606520.34\\
2.37995949898747	-289139713565.999\\
2.38005950148754	-289179820611.658\\
2.3801595039876	-289220500615.113\\
2.38025950648766	-289260607660.772\\
2.38035950898772	-289300714706.431\\
2.38045951148779	-289340821752.09\\
2.38055951398785	-289380928797.749\\
2.38065951648791	-289421035843.408\\
2.38075951898797	-289461142889.068\\
2.38085952148804	-289501249934.727\\
2.3809595239881	-289541356980.386\\
2.38105952648816	-289581464026.045\\
2.38115952898822	-289621571071.704\\
2.38125953148829	-289661678117.363\\
2.38135953398835	-289701785163.023\\
2.38145953648841	-289741892208.682\\
2.38155953898847	-289781426296.546\\
2.38165954148854	-289821533342.205\\
2.3817595439886	-289861640387.864\\
2.38185954648866	-289901747433.523\\
2.38195954898872	-289941854479.182\\
2.38205955148879	-289981961524.841\\
2.38215955398885	-290022068570.501\\
2.38225955648891	-290061602658.365\\
2.38235955898897	-290101709704.024\\
2.38245956148904	-290141816749.683\\
2.3825595639891	-290181923795.342\\
2.38265956648916	-290221457883.206\\
2.38275956898922	-290261564928.865\\
2.38285957148929	-290301671974.524\\
2.38295957398935	-290341779020.184\\
2.38305957648941	-290381313108.048\\
2.38315957898947	-290421420153.707\\
2.38325958148954	-290461527199.366\\
2.3833595839896	-290501061287.23\\
2.38345958648966	-290541168332.889\\
2.38355958898972	-290581275378.548\\
2.38365959148979	-290620809466.412\\
2.38375959398985	-290660916512.071\\
2.38385959648991	-290700450599.936\\
2.38395959898997	-290740557645.595\\
2.38405960149004	-290780091733.459\\
2.3841596039901	-290820198779.118\\
2.38425960649016	-290860305824.777\\
2.38435960899022	-290899839912.641\\
2.38445961149029	-290939946958.3\\
2.38455961399035	-290979481046.164\\
2.38465961649041	-291019588091.823\\
2.38475961899047	-291059122179.687\\
2.38485962149054	-291098656267.551\\
2.3849596239906	-291138763313.211\\
2.38505962649066	-291178297401.075\\
2.38515962899072	-291218404446.734\\
2.38525963149079	-291257938534.598\\
2.38535963399085	-291297472622.462\\
2.38545963649091	-291337579668.121\\
2.38555963899097	-291377113755.985\\
2.38565964149104	-291417220801.644\\
2.3857596439911	-291456754889.508\\
2.38585964649116	-291496288977.372\\
2.38595964899122	-291535823065.236\\
2.38605965149129	-291575930110.895\\
2.38615965399135	-291615464198.759\\
2.38625965649141	-291654998286.623\\
2.38635965899147	-291695105332.283\\
2.38645966149154	-291734639420.147\\
2.3865596639916	-291774173508.011\\
2.38665966649166	-291813707595.875\\
2.38675966899172	-291853241683.739\\
2.38685967149179	-291893348729.398\\
2.38695967399185	-291932882817.262\\
2.38705967649191	-291972416905.126\\
2.38715967899197	-292011950992.99\\
2.38725968149204	-292051485080.854\\
2.3873596839921	-292091019168.718\\
2.38745968649216	-292130553256.582\\
2.38755968899222	-292170087344.446\\
2.38765969149229	-292209621432.31\\
2.38775969399235	-292249155520.174\\
2.38785969649241	-292288689608.038\\
2.38795969899247	-292328223695.902\\
2.38805970149254	-292367757783.766\\
2.3881597039926	-292407291871.63\\
2.38825970649266	-292446825959.494\\
2.38835970899272	-292486360047.358\\
2.38845971149279	-292525894135.222\\
2.38855971399285	-292565428223.086\\
2.38865971649291	-292604962310.95\\
2.38875971899297	-292644496398.814\\
2.38885972149304	-292684030486.678\\
2.3889597239931	-292723564574.542\\
2.38905972649316	-292763098662.406\\
2.38915972899322	-292802059792.475\\
2.38925973149329	-292841593880.339\\
2.38935973399335	-292881127968.203\\
2.38945973649341	-292920662056.067\\
2.38955973899347	-292960196143.931\\
2.38965974149354	-292999157274\\
2.3897597439936	-293038691361.864\\
2.38985974649366	-293078225449.728\\
2.38995974899372	-293117759537.592\\
2.39005975149379	-293156720667.661\\
2.39015975399385	-293196254755.525\\
2.39025975649391	-293235788843.389\\
2.39035975899397	-293274749973.458\\
2.39045976149404	-293314284061.322\\
2.3905597639941	-293353818149.186\\
2.39065976649416	-293392779279.255\\
2.39075976899422	-293432313367.119\\
2.39085977149429	-293471847454.983\\
2.39095977399435	-293510808585.052\\
2.39105977649441	-293550342672.916\\
2.39115977899447	-293589303802.985\\
2.39125978149454	-293628837890.849\\
2.3913597839946	-293667799020.918\\
2.39145978649466	-293707333108.782\\
2.39155978899472	-293746294238.851\\
2.39165979149479	-293785828326.715\\
2.39175979399485	-293824789456.784\\
2.39185979649491	-293864323544.648\\
2.39195979899498	-293903284674.717\\
2.39205980149504	-293942818762.581\\
2.3921598039951	-293981779892.65\\
2.39225980649516	-294020741022.719\\
2.39235980899522	-294060275110.583\\
2.39245981149529	-294099236240.652\\
2.39255981399535	-294138770328.516\\
2.39265981649541	-294177731458.584\\
2.39275981899547	-294216692588.653\\
2.39285982149554	-294256226676.517\\
2.3929598239956	-294295187806.586\\
2.39305982649566	-294334148936.655\\
2.39315982899572	-294373110066.724\\
2.39325983149579	-294412644154.588\\
2.39335983399585	-294451605284.657\\
2.39345983649591	-294490566414.726\\
2.39355983899597	-294529527544.795\\
2.39365984149604	-294569061632.659\\
2.3937598439961	-294608022762.728\\
2.39385984649616	-294646983892.797\\
2.39395984899623	-294685945022.866\\
2.39405985149629	-294724906152.934\\
2.39415985399635	-294763867283.003\\
2.39425985649641	-294802828413.072\\
2.39435985899647	-294841789543.141\\
2.39445986149654	-294881323631.005\\
2.3945598639966	-294920284761.074\\
2.39465986649666	-294959245891.143\\
2.39475986899672	-294998207021.212\\
2.39485987149679	-295037168151.281\\
2.39495987399685	-295076129281.35\\
2.39505987649691	-295115090411.419\\
2.39515987899697	-295154051541.487\\
2.39525988149704	-295193012671.556\\
2.3953598839971	-295231973801.625\\
2.39545988649716	-295270361973.899\\
2.39555988899722	-295309323103.968\\
2.39565989149729	-295348284234.037\\
2.39575989399735	-295387245364.106\\
2.39585989649741	-295426206494.175\\
2.39595989899748	-295465167624.243\\
2.39605990149754	-295504128754.312\\
2.3961599039976	-295543089884.381\\
2.39625990649766	-295581478056.655\\
2.39635990899772	-295620439186.724\\
2.39645991149779	-295659400316.793\\
2.39655991399785	-295698361446.862\\
2.39665991649791	-295736749619.135\\
2.39675991899797	-295775710749.204\\
2.39685992149804	-295814671879.273\\
2.3969599239981	-295853633009.342\\
2.39705992649816	-295892021181.616\\
2.39715992899822	-295930982311.685\\
2.39725993149829	-295969943441.754\\
2.39735993399835	-296008331614.027\\
2.39745993649841	-296047292744.096\\
2.39755993899847	-296085680916.37\\
2.39765994149854	-296124642046.439\\
2.3977599439986	-296163603176.508\\
2.39785994649866	-296201991348.782\\
2.39795994899873	-296240952478.851\\
2.39805995149879	-296279340651.124\\
2.39815995399885	-296318301781.193\\
2.39825995649891	-296356689953.467\\
2.39835995899898	-296395651083.536\\
2.39845996149904	-296434039255.81\\
2.3985599639991	-296473000385.879\\
2.39865996649916	-296511388558.152\\
2.39875996899922	-296550349688.221\\
2.39885997149929	-296588737860.495\\
2.39895997399935	-296627126032.769\\
2.39905997649941	-296666087162.838\\
2.39915997899947	-296704475335.111\\
2.39925998149954	-296743436465.18\\
2.3993599839996	-296781824637.454\\
2.39945998649966	-296820212809.728\\
2.39955998899972	-296859173939.797\\
2.39965999149979	-296897562112.07\\
2.39975999399985	-296935950284.344\\
2.39985999649991	-296974338456.618\\
2.39995999899998	-297013299586.687\\
2.40006000150004	-297051687758.961\\
};
\addplot [color=mycolor3,solid,forget plot]
  table[row sep=crcr]{%
2.40006000150004	-297051687758.961\\
2.4001600040001	-297090075931.234\\
2.40026000650016	-297128464103.508\\
2.40036000900023	-297166852275.782\\
2.40046001150029	-297205813405.851\\
2.40056001400035	-297244201578.125\\
2.40066001650041	-297282589750.398\\
2.40076001900047	-297320977922.672\\
2.40086002150054	-297359366094.946\\
2.4009600240006	-297397754267.22\\
2.40106002650066	-297436142439.493\\
2.40116002900072	-297474530611.767\\
2.40126003150079	-297512918784.041\\
2.40136003400085	-297551306956.315\\
2.40146003650091	-297590268086.384\\
2.40156003900097	-297628656258.657\\
2.40166004150104	-297667044430.931\\
2.4017600440011	-297704859645.41\\
2.40186004650116	-297743247817.684\\
2.40196004900123	-297781635989.957\\
2.40206005150129	-297820024162.231\\
2.40216005400135	-297858412334.505\\
2.40226005650141	-297896800506.779\\
2.40236005900148	-297935188679.052\\
2.40246006150154	-297973576851.326\\
2.4025600640016	-298011965023.6\\
2.40266006650166	-298050353195.874\\
2.40276006900173	-298088168410.352\\
2.40286007150179	-298126556582.626\\
2.40296007400185	-298164944754.9\\
2.40306007650191	-298203332927.174\\
2.40316007900197	-298241148141.652\\
2.40326008150204	-298279536313.926\\
2.4033600840021	-298317924486.2\\
2.40346008650216	-298356312658.474\\
2.40356008900222	-298394127872.952\\
2.40366009150229	-298432516045.226\\
2.40376009400235	-298470904217.5\\
2.40386009650241	-298508719431.978\\
2.40396009900248	-298547107604.252\\
2.40406010150254	-298585495776.526\\
2.4041601040026	-298623310991.005\\
2.40426010650266	-298661699163.278\\
2.40436010900273	-298699514377.757\\
2.40446011150279	-298737902550.031\\
2.40456011400285	-298775717764.509\\
2.40466011650291	-298814105936.783\\
2.40476011900298	-298851921151.262\\
2.40486012150304	-298890309323.536\\
2.4049601240031	-298928124538.014\\
2.40506012650316	-298966512710.288\\
2.40516012900322	-299004327924.767\\
2.40526013150329	-299042716097.04\\
2.40536013400335	-299080531311.519\\
2.40546013650341	-299118919483.793\\
2.40556013900347	-299156734698.271\\
2.40566014150354	-299194549912.75\\
2.4057601440036	-299232938085.024\\
2.40586014650366	-299270753299.502\\
2.40596014900373	-299308568513.981\\
2.40606015150379	-299346956686.255\\
2.40616015400385	-299384771900.733\\
2.40626015650391	-299422587115.212\\
2.40636015900398	-299460402329.691\\
2.40646016150404	-299498790501.964\\
2.4065601640041	-299536605716.443\\
2.40666016650416	-299574420930.922\\
2.40676016900423	-299612236145.4\\
2.40686017150429	-299650624317.674\\
2.40696017400435	-299688439532.153\\
2.40706017650441	-299726254746.631\\
2.40716017900447	-299764069961.11\\
2.40726018150454	-299801885175.589\\
2.4073601840046	-299839700390.067\\
2.40746018650466	-299877515604.546\\
2.40756018900472	-299915330819.025\\
2.40766019150479	-299953146033.503\\
2.40776019400485	-299990961247.982\\
2.40786019650491	-300028776462.46\\
2.40796019900498	-300066591676.939\\
2.40806020150504	-300104406891.418\\
2.4081602040051	-300142222105.896\\
2.40826020650516	-300180037320.375\\
2.40836020900523	-300217852534.854\\
2.40846021150529	-300255667749.332\\
2.40856021400535	-300293482963.811\\
2.40866021650541	-300331298178.29\\
2.40876021900548	-300369113392.768\\
2.40886022150554	-300406928607.247\\
2.4089602240056	-300444743821.725\\
2.40906022650566	-300481986078.409\\
2.40916022900573	-300519801292.888\\
2.40926023150579	-300557616507.366\\
2.40936023400585	-300595431721.845\\
2.40946023650591	-300633246936.323\\
2.40956023900597	-300670489193.007\\
2.40966024150604	-300708304407.486\\
2.4097602440061	-300746119621.964\\
2.40986024650616	-300783361878.648\\
2.40996024900623	-300821177093.126\\
2.41006025150629	-300858992307.605\\
2.41016025400635	-300896234564.289\\
2.41026025650641	-300934049778.767\\
2.41036025900648	-300971864993.246\\
2.41046026150654	-301009107249.929\\
2.4105602640066	-301046922464.408\\
2.41066026650666	-301084164721.091\\
2.41076026900673	-301121979935.57\\
2.41086027150679	-301159222192.254\\
2.41096027400685	-301197037406.732\\
2.41106027650691	-301234279663.416\\
2.41116027900698	-301272094877.894\\
2.41126028150704	-301309337134.578\\
2.4113602840071	-301347152349.056\\
2.41146028650716	-301384394605.74\\
2.41156028900722	-301422209820.219\\
2.41166029150729	-301459452076.902\\
2.41176029400735	-301497267291.381\\
2.41186029650741	-301534509548.064\\
2.41196029900748	-301571751804.748\\
2.41206030150754	-301609567019.226\\
2.4121603040076	-301646809275.91\\
2.41226030650766	-301684051532.593\\
2.41236030900773	-301721866747.072\\
2.41246031150779	-301759109003.756\\
2.41256031400785	-301796351260.439\\
2.41266031650791	-301833593517.123\\
2.41276031900798	-301871408731.601\\
2.41286032150804	-301908650988.285\\
2.4129603240081	-301945893244.968\\
2.41306032650816	-301983135501.652\\
2.41316032900823	-302020377758.335\\
2.41326033150829	-302057620015.019\\
2.41336033400835	-302095435229.497\\
2.41346033650841	-302132677486.181\\
2.41356033900847	-302169919742.864\\
2.41366034150854	-302207161999.548\\
2.4137603440086	-302244404256.231\\
2.41386034650866	-302281646512.915\\
2.41396034900873	-302318888769.598\\
2.41406035150879	-302356131026.282\\
2.41416035400885	-302393373282.965\\
2.41426035650891	-302430615539.649\\
2.41436035900898	-302467857796.332\\
2.41446036150904	-302505100053.016\\
2.4145603640091	-302542342309.699\\
2.41466036650916	-302579584566.383\\
2.41476036900923	-302616826823.066\\
2.41486037150929	-302653496121.955\\
2.41496037400935	-302690738378.638\\
2.41506037650941	-302727980635.322\\
2.41516037900948	-302765222892.005\\
2.41526038150954	-302802465148.689\\
2.4153603840096	-302839707405.372\\
2.41546038650966	-302876376704.261\\
2.41556038900973	-302913618960.944\\
2.41566039150979	-302950861217.628\\
2.41576039400985	-302988103474.311\\
2.41586039650991	-303024772773.2\\
2.41596039900998	-303062015029.883\\
2.41606040151004	-303099257286.567\\
2.4161604040101	-303135926585.455\\
2.41626040651016	-303173168842.138\\
2.41636040901023	-303210411098.822\\
2.41646041151029	-303247080397.71\\
2.41656041401035	-303284322654.394\\
2.41666041651041	-303320991953.282\\
2.41676041901048	-303358234209.966\\
2.41686042151054	-303395476466.649\\
2.4169604240106	-303432145765.538\\
2.41706042651066	-303469388022.221\\
2.41716042901073	-303506057321.109\\
2.41726043151079	-303543299577.793\\
2.41736043401085	-303579968876.681\\
2.41746043651091	-303616638175.57\\
2.41756043901098	-303653880432.253\\
2.41766044151104	-303690549731.142\\
2.4177604440111	-303727791987.825\\
2.41786044651116	-303764461286.713\\
2.41796044901123	-303801130585.602\\
2.41806045151129	-303838372842.285\\
2.41816045401135	-303875042141.174\\
2.41826045651141	-303911711440.062\\
2.41836045901148	-303948953696.746\\
2.41846046151154	-303985622995.634\\
2.4185604640116	-304022292294.522\\
2.41866046651166	-304059534551.206\\
2.41876046901173	-304096203850.094\\
2.41886047151179	-304132873148.983\\
2.41896047401185	-304169542447.871\\
2.41906047651191	-304206211746.759\\
2.41916047901198	-304242881045.648\\
2.41926048151204	-304280123302.331\\
2.4193604840121	-304316792601.22\\
2.41946048651216	-304353461900.108\\
2.41956048901223	-304390131198.996\\
2.41966049151229	-304426800497.885\\
2.41976049401235	-304463469796.773\\
2.41986049651241	-304500139095.661\\
2.41996049901248	-304536808394.55\\
2.42006050151254	-304573477693.438\\
2.4201605040126	-304610146992.327\\
2.42026050651266	-304646816291.215\\
2.42036050901273	-304683485590.103\\
2.42046051151279	-304720154888.992\\
2.42056051401285	-304756824187.88\\
2.42066051651291	-304793493486.768\\
2.42076051901298	-304830162785.657\\
2.42086052151304	-304866832084.545\\
2.4209605240131	-304902928425.638\\
2.42106052651316	-304939597724.527\\
2.42116052901323	-304976267023.415\\
2.42126053151329	-305012936322.303\\
2.42136053401335	-305049605621.192\\
2.42146053651341	-305085701962.285\\
2.42156053901348	-305122371261.173\\
2.42166054151354	-305159040560.062\\
2.4217605440136	-305195709858.95\\
2.42186054651366	-305231806200.043\\
2.42196054901373	-305268475498.932\\
2.42206055151379	-305305144797.82\\
2.42216055401385	-305341241138.913\\
2.42226055651391	-305377910437.802\\
2.42236055901398	-305414579736.69\\
2.42246056151404	-305450676077.783\\
2.4225605640141	-305487345376.672\\
2.42266056651416	-305523441717.765\\
2.42276056901423	-305560111016.653\\
2.42286057151429	-305596207357.747\\
2.42296057401435	-305632876656.635\\
2.42306057651441	-305668972997.728\\
2.42316057901448	-305705642296.617\\
2.42326058151454	-305741738637.71\\
2.4233605840146	-305778407936.598\\
2.42346058651466	-305814504277.692\\
2.42356058901473	-305851173576.58\\
2.42366059151479	-305887269917.673\\
2.42376059401485	-305923366258.766\\
2.42386059651491	-305960035557.655\\
2.42396059901498	-305996131898.748\\
2.42406060151504	-306032228239.841\\
2.4241606040151	-306068897538.73\\
2.42426060651516	-306104993879.823\\
2.42436060901523	-306141090220.916\\
2.42446061151529	-306177186562.009\\
2.42456061401535	-306213855860.898\\
2.42466061651541	-306249952201.991\\
2.42476061901548	-306286048543.084\\
2.42486062151554	-306322144884.177\\
2.4249606240156	-306358241225.271\\
2.42506062651566	-306394337566.364\\
2.42516062901573	-306431006865.252\\
2.42526063151579	-306467103206.346\\
2.42536063401585	-306503199547.439\\
2.42546063651591	-306539295888.532\\
2.42556063901598	-306575392229.625\\
2.42566064151604	-306611488570.719\\
2.4257606440161	-306647584911.812\\
2.42586064651616	-306683681252.905\\
2.42596064901623	-306719777593.998\\
2.42606065151629	-306755873935.091\\
2.42616065401635	-306791970276.185\\
2.42626065651641	-306828066617.278\\
2.42636065901648	-306864162958.371\\
2.42646066151654	-306900259299.464\\
2.4265606640166	-306936355640.558\\
2.42666066651666	-306971879023.856\\
2.42676066901673	-307007975364.949\\
2.42686067151679	-307044071706.042\\
2.42696067401685	-307080168047.135\\
2.42706067651691	-307116264388.229\\
2.42716067901698	-307151787771.527\\
2.42726068151704	-307187884112.62\\
2.4273606840171	-307223980453.713\\
2.42746068651716	-307260076794.807\\
2.42756068901723	-307295600178.105\\
2.42766069151729	-307331696519.198\\
2.42776069401735	-307367792860.291\\
2.42786069651741	-307403316243.589\\
2.42796069901748	-307439412584.682\\
2.42806070151754	-307475508925.776\\
2.4281607040176	-307511032309.074\\
2.42826070651766	-307547128650.167\\
2.42836070901773	-307582652033.465\\
2.42846071151779	-307618748374.558\\
2.42856071401785	-307654271757.857\\
2.42866071651791	-307690368098.95\\
2.42876071901798	-307725891482.248\\
2.42886072151804	-307761987823.341\\
2.4289607240181	-307797511206.639\\
2.42906072651816	-307833607547.733\\
2.42916072901823	-307869130931.031\\
2.42926073151829	-307905227272.124\\
2.42936073401835	-307940750655.422\\
2.42946073651841	-307976274038.72\\
2.42956073901848	-308012370379.813\\
2.42966074151854	-308047893763.111\\
2.4297607440186	-308083417146.41\\
2.42986074651866	-308119513487.503\\
2.42996074901873	-308155036870.801\\
2.43006075151879	-308190560254.099\\
2.43016075401885	-308226083637.397\\
2.43026075651891	-308262179978.49\\
2.43036075901898	-308297703361.789\\
2.43046076151904	-308333226745.087\\
2.4305607640191	-308368750128.385\\
2.43066076651916	-308404273511.683\\
2.43076076901923	-308439796894.981\\
2.43086077151929	-308475893236.074\\
2.43096077401935	-308511416619.372\\
2.43106077651941	-308546940002.67\\
2.43116077901948	-308582463385.969\\
2.43126078151954	-308617986769.267\\
2.4313607840196	-308653510152.565\\
2.43146078651966	-308689033535.863\\
2.43156078901973	-308724556919.161\\
2.43166079151979	-308760080302.459\\
2.43176079401985	-308795603685.757\\
2.43186079651991	-308831127069.055\\
2.43196079901998	-308866650452.353\\
2.43206080152004	-308902173835.651\\
2.4321608040201	-308937124261.154\\
2.43226080652016	-308972647644.453\\
2.43236080902023	-309008171027.751\\
2.43246081152029	-309043694411.049\\
2.43256081402035	-309079217794.347\\
2.43266081652041	-309114741177.645\\
2.43276081902048	-309149691603.148\\
2.43286082152054	-309185214986.446\\
2.4329608240206	-309220738369.744\\
2.43306082652066	-309256261753.042\\
2.43316082902073	-309291212178.545\\
2.43326083152079	-309326735561.843\\
2.43336083402085	-309362258945.142\\
2.43346083652091	-309397209370.645\\
2.43356083902098	-309432732753.943\\
2.43366084152104	-309467683179.446\\
2.4337608440211	-309503206562.744\\
2.43386084652116	-309538729946.042\\
2.43396084902123	-309573680371.545\\
2.43406085152129	-309609203754.843\\
2.43416085402135	-309644154180.346\\
2.43426085652141	-309679677563.644\\
2.43436085902148	-309714627989.147\\
2.43446086152154	-309750151372.445\\
2.4345608640216	-309785101797.948\\
2.43466086652166	-309820052223.451\\
2.43476086902173	-309855575606.749\\
2.43486087152179	-309890526032.252\\
2.43496087402185	-309926049415.55\\
2.43506087652191	-309960999841.053\\
2.43516087902198	-309995950266.556\\
2.43526088152204	-310031473649.854\\
2.4353608840221	-310066424075.357\\
2.43546088652216	-310101374500.86\\
2.43556088902223	-310136324926.363\\
2.43566089152229	-310171848309.661\\
2.43576089402235	-310206798735.164\\
2.43586089652241	-310241749160.667\\
2.43596089902248	-310276699586.17\\
2.43606090152254	-310312222969.468\\
2.4361609040226	-310347173394.971\\
2.43626090652266	-310382123820.474\\
2.43636090902273	-310417074245.977\\
2.43646091152279	-310452024671.48\\
2.43656091402285	-310486975096.983\\
2.43666091652291	-310521925522.486\\
2.43676091902298	-310556875947.989\\
2.43686092152304	-310591826373.492\\
2.4369609240231	-310626776798.995\\
2.43706092652316	-310661727224.498\\
2.43716092902323	-310696677650.001\\
2.43726093152329	-310731628075.504\\
2.43736093402335	-310766578501.007\\
2.43746093652341	-310801528926.51\\
2.43756093902348	-310836479352.013\\
2.43766094152354	-310871429777.516\\
2.4377609440236	-310906380203.019\\
2.43786094652366	-310940757670.727\\
2.43796094902373	-310975708096.23\\
2.43806095152379	-311010658521.733\\
2.43816095402385	-311045608947.236\\
2.43826095652391	-311080559372.739\\
2.43836095902398	-311114936840.447\\
2.43846096152404	-311149887265.95\\
2.4385609640241	-311184837691.453\\
2.43866096652416	-311219215159.161\\
2.43876096902423	-311254165584.664\\
2.43886097152429	-311289116010.167\\
2.43896097402435	-311323493477.874\\
2.43906097652441	-311358443903.377\\
2.43916097902448	-311393394328.88\\
2.43926098152454	-311427771796.588\\
2.4393609840246	-311462722222.091\\
2.43946098652466	-311497099689.799\\
2.43956098902473	-311532050115.302\\
2.43966099152479	-311566427583.01\\
2.43976099402485	-311601378008.513\\
2.43986099652491	-311635755476.221\\
2.43996099902498	-311670705901.724\\
2.44006100152504	-311705083369.432\\
2.4401610040251	-311739460837.139\\
2.44026100652516	-311774411262.642\\
2.44036100902523	-311808788730.35\\
2.44046101152529	-311843166198.058\\
2.44056101402535	-311878116623.561\\
2.44066101652541	-311912494091.269\\
2.44076101902548	-311946871558.977\\
2.44086102152554	-311981821984.48\\
2.4409610240256	-312016199452.188\\
2.44106102652566	-312050576919.895\\
2.44116102902573	-312084954387.603\\
2.44126103152579	-312119904813.106\\
2.44136103402585	-312154282280.814\\
2.44146103652591	-312188659748.522\\
2.44156103902598	-312223037216.23\\
2.44166104152604	-312257414683.938\\
2.4417610440261	-312291792151.645\\
2.44186104652616	-312326169619.353\\
2.44196104902623	-312360547087.061\\
2.44206105152629	-312394924554.769\\
2.44216105402635	-312429874980.272\\
2.44226105652641	-312464252447.98\\
2.44236105902648	-312498629915.688\\
2.44246106152654	-312532434425.6\\
2.4425610640266	-312566811893.308\\
2.44266106652666	-312601189361.016\\
2.44276106902673	-312635566828.724\\
2.44286107152679	-312669944296.432\\
2.44296107402685	-312704321764.14\\
2.44306107652691	-312738699231.848\\
2.44316107902698	-312773076699.555\\
2.44326108152704	-312807454167.263\\
2.4433610840271	-312841258677.176\\
2.44346108652716	-312875636144.884\\
2.44356108902723	-312910013612.592\\
2.44366109152729	-312944391080.299\\
2.44376109402735	-312978195590.212\\
2.44386109652741	-313012573057.92\\
2.44396109902748	-313046950525.628\\
2.44406110152754	-313080755035.541\\
2.4441611040276	-313115132503.248\\
2.44426110652766	-313149509970.956\\
2.44436110902773	-313183314480.869\\
2.44446111152779	-313217691948.577\\
2.44456111402785	-313251496458.49\\
2.44466111652791	-313285873926.197\\
2.44476111902798	-313320251393.905\\
2.44486112152804	-313354055903.818\\
2.4449611240281	-313388433371.526\\
2.44506112652816	-313422237881.439\\
2.44516112902823	-313456042391.351\\
2.44526113152829	-313490419859.059\\
2.44536113402835	-313524224368.972\\
2.44546113652841	-313558601836.68\\
2.44556113902848	-313592406346.592\\
2.44566114152854	-313626210856.505\\
2.4457611440286	-313660588324.213\\
2.44586114652866	-313694392834.126\\
2.44596114902873	-313728197344.038\\
2.44606115152879	-313762574811.746\\
2.44616115402885	-313796379321.659\\
2.44626115652891	-313830183831.572\\
2.44636115902898	-313863988341.484\\
2.44646116152904	-313898365809.192\\
2.4465611640291	-313932170319.105\\
2.44666116652916	-313965974829.018\\
2.44676116902923	-313999779338.93\\
2.44686117152929	-314033583848.843\\
2.44696117402935	-314067388358.756\\
2.44706117652941	-314101192868.669\\
2.44716117902948	-314134997378.581\\
2.44726118152954	-314169374846.289\\
2.4473611840296	-314203179356.202\\
2.44746118652966	-314236983866.115\\
2.44756118902973	-314270788376.027\\
2.44766119152979	-314304592885.94\\
2.44776119402985	-314337824438.058\\
2.44786119652991	-314371628947.97\\
2.44796119902998	-314405433457.883\\
2.44806120153004	-314439237967.796\\
2.4481612040301	-314473042477.708\\
2.44826120653016	-314506846987.621\\
2.44836120903023	-314540651497.534\\
2.44846121153029	-314574456007.447\\
2.44856121403035	-314607687559.564\\
2.44866121653041	-314641492069.477\\
2.44876121903048	-314675296579.39\\
2.44886122153054	-314709101089.302\\
2.4489612240306	-314742332641.42\\
2.44906122653066	-314776137151.333\\
2.44916122903073	-314809941661.245\\
2.44926123153079	-314843173213.363\\
2.44936123403085	-314876977723.276\\
2.44946123653091	-314910782233.188\\
2.44956123903098	-314944013785.306\\
2.44966124153104	-314977818295.219\\
2.4497612440311	-315011049847.336\\
2.44986124653116	-315044854357.249\\
2.44996124903123	-315078085909.367\\
2.45006125153129	-315111890419.279\\
2.45016125403135	-315145121971.397\\
2.45026125653141	-315178926481.31\\
2.45036125903148	-315212158033.427\\
2.45046126153154	-315245962543.34\\
2.4505612640316	-315279194095.458\\
2.45066126653166	-315312998605.37\\
2.45076126903173	-315346230157.488\\
2.45086127153179	-315379461709.605\\
2.45096127403185	-315413266219.518\\
2.45106127653191	-315446497771.636\\
2.45116127903198	-315479729323.753\\
2.45126128153204	-315512960875.871\\
2.4513612840321	-315546765385.784\\
2.45146128653216	-315579996937.901\\
2.45156128903223	-315613228490.019\\
2.45166129153229	-315646460042.136\\
2.45176129403235	-315680264552.049\\
2.45186129653241	-315713496104.167\\
2.45196129903248	-315746727656.284\\
2.45206130153254	-315779959208.402\\
2.4521613040326	-315813190760.52\\
2.45226130653266	-315846422312.637\\
2.45236130903273	-315879653864.755\\
2.45246131153279	-315912885416.872\\
2.45256131403285	-315946116968.99\\
2.45266131653291	-315979348521.107\\
2.45276131903298	-316012580073.225\\
2.45286132153304	-316045811625.343\\
2.4529613240331	-316079043177.46\\
2.45306132653316	-316112274729.578\\
2.45316132903323	-316145506281.695\\
2.45326133153329	-316178737833.813\\
2.45336133403335	-316211396428.135\\
2.45346133653341	-316244627980.253\\
2.45356133903348	-316277859532.371\\
2.45366134153354	-316311091084.488\\
2.4537613440336	-316344322636.606\\
2.45386134653366	-316376981230.928\\
2.45396134903373	-316410212783.046\\
2.45406135153379	-316443444335.163\\
2.45416135403385	-316476675887.281\\
2.45426135653391	-316509334481.603\\
2.45436135903398	-316542566033.721\\
2.45446136153404	-316575797585.839\\
2.4545613640341	-316608456180.161\\
2.45466136653416	-316641687732.279\\
2.45476136903423	-316674346326.601\\
2.45486137153429	-316707577878.719\\
2.45496137403435	-316740236473.041\\
2.45506137653441	-316773468025.159\\
2.45516137903448	-316806126619.481\\
2.45526138153454	-316839358171.599\\
2.4553613840346	-316872016765.921\\
2.45546138653466	-316905248318.039\\
2.45556138903473	-316937906912.361\\
2.45566139153479	-316971138464.479\\
2.45576139403485	-317003797058.801\\
2.45586139653491	-317036455653.124\\
2.45596139903498	-317069687205.241\\
2.45606140153504	-317102345799.564\\
2.4561614040351	-317135004393.886\\
2.45626140653516	-317167662988.209\\
2.45636140903523	-317200894540.326\\
2.45646141153529	-317233553134.649\\
2.45656141403535	-317266211728.971\\
2.45666141653541	-317298870323.294\\
2.45676141903548	-317331528917.616\\
2.45686142153554	-317364760469.734\\
2.4569614240356	-317397419064.056\\
2.45706142653566	-317430077658.379\\
2.45716142903573	-317462736252.701\\
2.45726143153579	-317495394847.024\\
2.45736143403585	-317528053441.346\\
2.45746143653591	-317560712035.669\\
2.45756143903598	-317593370629.991\\
2.45766144153604	-317626029224.313\\
2.4577614440361	-317658687818.636\\
2.45786144653616	-317691346412.958\\
2.45796144903623	-317724005007.281\\
2.45806145153629	-317756663601.603\\
2.45816145403635	-317789322195.926\\
2.45826145653641	-317821980790.248\\
2.45836145903648	-317854066426.776\\
2.45846146153654	-317886725021.098\\
2.4585614640366	-317919383615.42\\
2.45866146653666	-317952042209.743\\
2.45876146903673	-317984700804.065\\
2.45886147153679	-318016786440.593\\
2.45896147403685	-318049445034.915\\
2.45906147653691	-318082103629.238\\
2.45916147903698	-318114189265.765\\
2.45926148153704	-318146847860.087\\
2.4593614840371	-318179506454.41\\
2.45946148653716	-318211592090.937\\
2.45956148903723	-318244250685.26\\
2.45966149153729	-318276336321.787\\
2.45976149403735	-318308994916.109\\
2.45986149653741	-318341653510.432\\
2.45996149903748	-318373739146.959\\
2.46006150153754	-318406397741.282\\
2.4601615040376	-318438483377.809\\
2.46026150653766	-318471141972.131\\
2.46036150903773	-318503227608.659\\
2.46046151153779	-318535313245.186\\
2.46056151403785	-318567971839.508\\
2.46066151653791	-318600057476.036\\
2.46076151903798	-318632716070.358\\
2.46086152153804	-318664801706.886\\
2.4609615240381	-318696887343.413\\
2.46106152653816	-318728972979.94\\
2.46116152903823	-318761631574.263\\
2.46126153153829	-318793717210.79\\
2.46136153403835	-318825802847.317\\
2.46146153653841	-318857888483.845\\
2.46156153903848	-318890547078.167\\
2.46166154153854	-318922632714.694\\
2.4617615440386	-318954718351.222\\
2.46186154653866	-318986803987.749\\
2.46196154903873	-319018889624.276\\
2.46206155153879	-319050975260.804\\
2.46216155403885	-319083060897.331\\
2.46226155653891	-319115146533.858\\
2.46236155903898	-319147232170.386\\
2.46246156153904	-319179317806.913\\
2.4625615640391	-319211403443.44\\
2.46266156653916	-319243489079.968\\
2.46276156903923	-319275574716.495\\
2.46286157153929	-319307660353.022\\
2.46296157403935	-319339745989.55\\
2.46306157653941	-319371831626.077\\
2.46316157903948	-319403917262.604\\
2.46326158153954	-319436002899.132\\
2.4633615840396	-319467515577.864\\
2.46346158653966	-319499601214.391\\
2.46356158903973	-319531686850.919\\
2.46366159153979	-319563772487.446\\
2.46376159403985	-319595285166.178\\
2.46386159653991	-319627370802.705\\
2.46396159903998	-319659456439.233\\
2.46406160154004	-319691542075.76\\
2.4641616040401	-319723054754.492\\
2.46426160654016	-319755140391.02\\
2.46436160904023	-319786653069.752\\
2.46446161154029	-319818738706.279\\
2.46456161404035	-319850824342.806\\
2.46466161654041	-319882337021.539\\
2.46476161904048	-319914422658.066\\
2.46486162154054	-319945935336.798\\
2.4649616240406	-319978020973.325\\
2.46506162654066	-320009533652.058\\
2.46516162904073	-320041619288.585\\
2.46526163154079	-320073131967.317\\
2.46536163404085	-320104644646.049\\
2.46546163654091	-320136730282.577\\
2.46556163904098	-320168242961.309\\
2.46566164154104	-320199755640.041\\
2.4657616440411	-320231841276.568\\
2.46586164654116	-320263353955.301\\
2.46596164904123	-320294866634.033\\
2.46606165154129	-320326952270.56\\
2.46616165404135	-320358464949.292\\
2.46626165654141	-320389977628.025\\
2.46636165904148	-320421490306.757\\
2.46646166154154	-320453002985.489\\
2.4665616640416	-320485088622.016\\
2.46666166654166	-320516601300.748\\
2.46676166904173	-320548113979.481\\
2.46686167154179	-320579626658.213\\
2.46696167404185	-320611139336.945\\
2.46706167654191	-320642652015.677\\
2.46716167904198	-320674164694.409\\
2.46726168154204	-320705677373.142\\
2.4673616840421	-320737190051.874\\
2.46746168654216	-320768702730.606\\
2.46756168904223	-320800215409.338\\
2.46766169154229	-320831728088.07\\
2.46776169404235	-320863240766.803\\
2.46786169654241	-320894753445.535\\
2.46796169904248	-320925693166.472\\
2.46806170154254	-320957205845.204\\
2.4681617040426	-320988718523.936\\
2.46826170654266	-321020231202.668\\
2.46836170904273	-321051743881.401\\
2.46846171154279	-321082683602.338\\
2.46856171404285	-321114196281.07\\
2.46866171654291	-321145708959.802\\
2.46876171904298	-321176648680.739\\
2.46886172154304	-321208161359.471\\
2.4689617240431	-321239674038.204\\
2.46906172654316	-321270613759.141\\
2.46916172904323	-321302126437.873\\
2.46926173154329	-321333639116.605\\
2.46936173404335	-321364578837.542\\
2.46946173654341	-321396091516.274\\
2.46956173904348	-321427031237.211\\
2.46966174154354	-321458543915.943\\
2.4697617440436	-321489483636.881\\
2.46986174654366	-321520996315.613\\
2.46996174904373	-321551936036.55\\
2.47006175154379	-321582875757.487\\
2.47016175404385	-321614388436.219\\
2.47026175654391	-321645328157.156\\
2.47036175904398	-321676840835.888\\
2.47046176154404	-321707780556.825\\
2.4705617640441	-321738720277.762\\
2.47066176654416	-321769659998.7\\
2.47076176904423	-321801172677.432\\
2.47086177154429	-321832112398.369\\
2.47096177404435	-321863052119.306\\
2.47106177654441	-321893991840.243\\
2.47116177904448	-321925504518.975\\
2.47126178154454	-321956444239.912\\
2.4713617840446	-321987383960.849\\
2.47146178654466	-322018323681.786\\
2.47156178904473	-322049263402.723\\
2.47166179154479	-322080203123.66\\
2.47176179404485	-322111142844.598\\
2.47186179654491	-322142082565.535\\
2.47196179904498	-322173022286.472\\
2.47206180154504	-322203962007.409\\
2.4721618040451	-322234901728.346\\
2.47226180654516	-322265841449.283\\
2.47236180904523	-322296781170.22\\
2.47246181154529	-322327720891.157\\
2.47256181404535	-322358660612.094\\
2.47266181654541	-322389600333.031\\
2.47276181904548	-322419967096.173\\
2.47286182154554	-322450906817.11\\
2.4729618240456	-322481846538.047\\
2.47306182654566	-322512786258.984\\
2.47316182904573	-322543153022.126\\
2.47326183154579	-322574092743.063\\
2.47336183404585	-322605032464\\
2.47346183654591	-322635972184.937\\
2.47356183904598	-322666338948.079\\
2.47366184154604	-322697278669.016\\
2.4737618440461	-322727645432.158\\
2.47386184654616	-322758585153.095\\
2.47396184904623	-322789524874.032\\
2.47406185154629	-322819891637.174\\
2.47416185404635	-322850831358.111\\
2.47426185654641	-322881198121.253\\
2.47436185904648	-322912137842.19\\
2.47446186154654	-322942504605.332\\
2.4745618640466	-322973444326.269\\
2.47466186654666	-323003811089.411\\
2.47476186904673	-323034177852.553\\
2.47486187154679	-323065117573.49\\
2.47496187404685	-323095484336.632\\
2.47506187654691	-323126424057.569\\
2.47516187904698	-323156790820.711\\
2.47526188154704	-323187157583.853\\
2.4753618840471	-323217524346.995\\
2.47546188654716	-323248464067.932\\
2.47556188904723	-323278830831.074\\
2.47566189154729	-323309197594.216\\
2.47576189404735	-323339564357.358\\
2.47586189654741	-323369931120.5\\
2.47596189904748	-323400870841.437\\
2.47606190154754	-323431237604.579\\
2.4761619040476	-323461604367.721\\
2.47626190654766	-323491971130.863\\
2.47636190904773	-323522337894.005\\
2.47646191154779	-323552704657.147\\
2.47656191404785	-323583071420.289\\
2.47666191654791	-323613438183.431\\
2.47676191904798	-323643804946.572\\
2.47686192154804	-323674171709.714\\
2.4769619240481	-323704538472.856\\
2.47706192654816	-323734905235.998\\
2.47716192904823	-323764699041.345\\
2.47726193154829	-323795065804.487\\
2.47736193404835	-323825432567.629\\
2.47746193654841	-323855799330.771\\
2.47756193904848	-323886166093.913\\
2.47766194154854	-323916532857.055\\
2.4777619440486	-323946326662.402\\
2.47786194654866	-323976693425.543\\
2.47796194904873	-324007060188.685\\
2.47806195154879	-324036853994.032\\
2.47816195404885	-324067220757.174\\
2.47826195654891	-324097587520.316\\
2.47836195904898	-324127381325.663\\
2.47846196154904	-324157748088.805\\
2.4785619640491	-324187541894.152\\
2.47866196654916	-324217908657.294\\
2.47876196904923	-324248275420.435\\
2.47886197154929	-324278069225.782\\
2.47896197404935	-324308435988.924\\
2.47906197654941	-324338229794.271\\
2.47916197904948	-324368023599.618\\
2.47926198154954	-324398390362.76\\
2.4793619840496	-324428184168.107\\
2.47946198654966	-324458550931.248\\
2.47956198904973	-324488344736.595\\
2.47966199154979	-324518138541.942\\
2.47976199404985	-324548505305.084\\
2.47986199654991	-324578299110.431\\
2.47996199904998	-324608092915.778\\
2.48006200155004	-324637886721.124\\
2.4801620040501	-324668253484.266\\
2.48026200655016	-324698047289.613\\
2.48036200905023	-324727841094.96\\
2.48046201155029	-324757634900.307\\
2.48056201405035	-324787428705.654\\
2.48066201655041	-324817222511\\
2.48076201905048	-324847016316.347\\
2.48086202155054	-324876810121.694\\
2.4809620240506	-324906603927.041\\
2.48106202655066	-324936397732.388\\
2.48116202905073	-324966191537.734\\
2.48126203155079	-324995985343.081\\
2.48136203405085	-325025779148.428\\
2.48146203655091	-325055572953.775\\
2.48156203905098	-325085366759.122\\
2.48166204155104	-325115160564.468\\
2.4817620440511	-325144954369.815\\
2.48186204655116	-325174748175.162\\
2.48196204905123	-325204541980.509\\
2.48206205155129	-325233762828.06\\
2.48216205405135	-325263556633.407\\
2.48226205655141	-325293350438.754\\
2.48236205905148	-325323144244.101\\
2.48246206155154	-325352365091.653\\
2.4825620640516	-325382158896.999\\
2.48266206655166	-325411952702.346\\
2.48276206905173	-325441173549.898\\
2.48286207155179	-325470967355.245\\
2.48296207405185	-325500761160.591\\
2.48306207655191	-325529982008.143\\
2.48316207905198	-325559775813.49\\
2.48326208155204	-325588996661.042\\
2.4833620840521	-325618790466.388\\
2.48346208655216	-325648011313.94\\
2.48356208905223	-325677805119.287\\
2.48366209155229	-325707025966.839\\
2.48376209405235	-325736819772.185\\
2.48386209655241	-325766040619.737\\
2.48396209905248	-325795261467.289\\
2.48406210155254	-325825055272.635\\
2.4841621040526	-325854276120.187\\
2.48426210655266	-325883496967.739\\
2.48436210905273	-325913290773.086\\
2.48446211155279	-325942511620.637\\
2.48456211405285	-325971732468.189\\
2.48466211655291	-326000953315.741\\
2.48476211905298	-326030174163.292\\
2.48486212155304	-326059967968.639\\
2.4849621240531	-326089188816.191\\
2.48506212655316	-326118409663.742\\
2.48516212905323	-326147630511.294\\
2.48526213155329	-326176851358.846\\
2.48536213405335	-326206072206.398\\
2.48546213655341	-326235293053.949\\
2.48556213905348	-326264513901.501\\
2.48566214155354	-326293734749.053\\
2.4857621440536	-326322955596.604\\
2.48586214655366	-326352176444.156\\
2.48596214905373	-326381397291.708\\
2.48606215155379	-326410618139.259\\
2.48616215405385	-326439838986.811\\
2.48626215655391	-326469059834.363\\
2.48636215905398	-326497707724.119\\
2.48646216155404	-326526928571.671\\
2.4865621640541	-326556149419.222\\
2.48666216655416	-326585370266.774\\
2.48676216905423	-326614591114.326\\
2.48686217155429	-326643239004.082\\
2.48696217405435	-326672459851.634\\
2.48706217655441	-326701680699.186\\
2.48716217905448	-326730328588.942\\
2.48726218155454	-326759549436.494\\
2.4873621840546	-326788770284.046\\
2.48746218655466	-326817418173.802\\
2.48756218905473	-326846639021.354\\
2.48766219155479	-326875286911.11\\
2.48776219405485	-326904507758.662\\
2.48786219655491	-326933155648.419\\
2.48796219905498	-326962376495.97\\
2.48806220155504	-326991024385.727\\
2.4881622040551	-327020245233.278\\
2.48826220655516	-327048893123.035\\
2.48836220905523	-327077541012.792\\
2.48846221155529	-327106761860.343\\
2.48856221405535	-327135409750.1\\
2.48866221655541	-327164057639.856\\
2.48876221905548	-327193278487.408\\
2.48886222155554	-327221926377.164\\
2.4889622240556	-327250574266.921\\
2.48906222655566	-327279222156.678\\
2.48916222905573	-327308443004.229\\
2.48926223155579	-327337090893.986\\
2.48936223405585	-327365738783.742\\
2.48946223655591	-327394386673.499\\
2.48956223905598	-327423034563.255\\
2.48966224155604	-327451682453.012\\
2.4897622440561	-327480330342.768\\
2.48986224655616	-327508978232.525\\
2.48996224905623	-327537626122.282\\
2.49006225155629	-327566274012.038\\
2.49016225405635	-327594921901.795\\
2.49026225655641	-327623569791.551\\
2.49036225905648	-327652217681.308\\
2.49046226155654	-327680865571.064\\
2.4905622640566	-327709513460.821\\
2.49066226655666	-327738161350.577\\
2.49076226905673	-327766809240.334\\
2.49086227155679	-327794884172.295\\
2.49096227405685	-327823532062.052\\
2.49106227655691	-327852179951.808\\
2.49116227905698	-327880827841.565\\
2.49126228155704	-327908902773.526\\
2.4913622840571	-327937550663.283\\
2.49146228655716	-327966198553.039\\
2.49156228905723	-327994273485.001\\
2.49166229155729	-328022921374.757\\
2.49176229405735	-328051569264.514\\
2.49186229655741	-328079644196.475\\
2.49196229905748	-328108292086.232\\
2.49206230155754	-328136367018.193\\
2.4921623040576	-328165014907.95\\
2.49226230655766	-328193089839.911\\
2.49236230905773	-328221737729.668\\
2.49246231155779	-328249812661.629\\
2.49256231405785	-328278460551.386\\
2.49266231655791	-328306535483.347\\
2.49276231905798	-328334610415.309\\
2.49286232155804	-328363258305.065\\
2.4929623240581	-328391333237.026\\
2.49306232655816	-328419408168.988\\
2.49316232905823	-328448056058.744\\
2.49326233155829	-328476130990.706\\
2.49336233405835	-328504205922.667\\
2.49346233655841	-328532280854.629\\
2.49356233905848	-328560355786.59\\
2.49366234155854	-328589003676.347\\
2.4937623440586	-328617078608.308\\
2.49386234655866	-328645153540.269\\
2.49396234905873	-328673228472.231\\
2.49406235155879	-328701303404.192\\
2.49416235405885	-328729378336.154\\
2.49426235655891	-328757453268.115\\
2.49436235905898	-328785528200.076\\
2.49446236155904	-328813603132.038\\
2.4945623640591	-328841678063.999\\
2.49466236655916	-328869752995.961\\
2.49476236905923	-328897827927.922\\
2.49486237155929	-328925902859.884\\
2.49496237405935	-328953977791.845\\
2.49506237655941	-328981479766.011\\
2.49516237905948	-329009554697.973\\
2.49526238155954	-329037629629.934\\
2.4953623840596	-329065704561.895\\
2.49546238655966	-329093779493.857\\
2.49556238905973	-329121281468.023\\
2.49566239155979	-329149356399.985\\
2.49576239405985	-329177431331.946\\
2.49586239655991	-329204933306.112\\
2.49596239905998	-329233008238.074\\
2.49606240156004	-329260510212.24\\
2.4961624040601	-329288585144.201\\
2.49626240656016	-329316660076.163\\
2.49636240906023	-329344162050.329\\
2.49646241156029	-329372236982.29\\
2.49656241406035	-329399738956.457\\
2.49666241656041	-329427813888.418\\
2.49676241906048	-329455315862.584\\
2.49686242156054	-329482817836.751\\
2.4969624240606	-329510892768.712\\
2.49706242656066	-329538394742.878\\
2.49716242906073	-329566469674.84\\
2.49726243156079	-329593971649.006\\
2.49736243406085	-329621473623.172\\
2.49746243656091	-329648975597.339\\
2.49756243906098	-329677050529.3\\
2.49766244156104	-329704552503.466\\
2.4977624440611	-329732054477.633\\
2.49786244656116	-329759556451.799\\
2.49796244906123	-329787631383.76\\
2.49806245156129	-329815133357.927\\
2.49816245406135	-329842635332.093\\
2.49826245656141	-329870137306.259\\
2.49836245906148	-329897639280.425\\
2.49846246156154	-329925141254.592\\
2.4985624640616	-329952643228.758\\
2.49866246656166	-329980145202.924\\
2.49876246906173	-330007647177.091\\
2.49886247156179	-330035149151.257\\
2.49896247406185	-330062651125.423\\
2.49906247656191	-330090153099.589\\
2.49916247906198	-330117655073.756\\
2.49926248156204	-330144584090.127\\
2.4993624840621	-330172086064.293\\
2.49946248656216	-330199588038.459\\
2.49956248906223	-330227090012.626\\
2.49966249156229	-330254591986.792\\
2.49976249406235	-330281521003.163\\
2.49986249656241	-330309022977.329\\
2.49996249906248	-330336524951.496\\
2.50006250156254	-330363453967.867\\
2.5001625040626	-330390955942.033\\
2.50026250656266	-330418457916.199\\
2.50036250906273	-330445386932.57\\
2.50046251156279	-330472888906.737\\
2.50056251406285	-330499817923.108\\
2.50066251656291	-330527319897.274\\
2.50076251906298	-330554248913.645\\
2.50086252156304	-330581750887.812\\
2.5009625240631	-330608679904.183\\
2.50106252656316	-330636181878.349\\
2.50116252906323	-330663110894.72\\
2.50126253156329	-330690612868.886\\
2.50136253406335	-330717541885.258\\
2.50146253656341	-330744470901.629\\
2.50156253906348	-330771972875.795\\
2.50166254156354	-330798901892.166\\
2.5017625440636	-330825830908.537\\
2.50186254656366	-330852759924.909\\
2.50196254906373	-330880261899.075\\
2.50206255156379	-330907190915.446\\
2.50216255406385	-330934119931.817\\
2.50226255656391	-330961048948.188\\
2.50236255906398	-330987977964.559\\
2.50246256156404	-331014906980.93\\
2.5025625640641	-331041835997.302\\
2.50266256656416	-331068765013.673\\
2.50276256906423	-331095694030.044\\
2.50286257156429	-331122623046.415\\
2.50296257406435	-331149552062.786\\
2.50306257656441	-331176481079.157\\
2.50316257906448	-331203410095.529\\
2.50326258156454	-331230339111.9\\
2.5033625840646	-331257268128.271\\
2.50346258656466	-331284197144.642\\
2.50356258906473	-331311126161.013\\
2.50366259156479	-331338055177.384\\
2.50376259406485	-331364411235.96\\
2.50386259656491	-331391340252.331\\
2.50396259906498	-331418269268.703\\
2.50406260156504	-331445198285.074\\
2.5041626040651	-331471554343.65\\
2.50426260656516	-331498483360.021\\
2.50436260906523	-331525412376.392\\
2.50446261156529	-331551768434.968\\
2.50456261406535	-331578697451.339\\
2.50466261656541	-331605053509.915\\
2.50476261906548	-331631982526.286\\
2.50486262156554	-331658338584.862\\
2.5049626240656	-331685267601.234\\
2.50506262656566	-331711623659.81\\
2.50516262906573	-331738552676.181\\
2.50526263156579	-331764908734.757\\
2.50536263406585	-331791837751.128\\
2.50546263656591	-331818193809.704\\
2.50556263906598	-331844549868.28\\
2.50566264156604	-331871478884.651\\
2.5057626440661	-331897834943.227\\
2.50586264656616	-331924191001.803\\
2.50596264906623	-331951120018.174\\
2.50606265156629	-331977476076.75\\
2.50616265406635	-332003832135.326\\
2.50626265656641	-332030188193.902\\
2.50636265906648	-332056544252.478\\
2.50646266156654	-332082900311.054\\
2.5065626640666	-332109829327.426\\
2.50666266656666	-332136185386.002\\
2.50676266906673	-332162541444.578\\
2.50686267156679	-332188897503.154\\
2.50696267406685	-332215253561.73\\
2.50706267656691	-332241609620.306\\
2.50716267906698	-332267965678.882\\
2.50726268156704	-332294321737.458\\
2.5073626840671	-332320677796.034\\
2.50746268656716	-332346460896.815\\
2.50756268906723	-332372816955.391\\
2.50766269156729	-332399173013.967\\
2.50776269406735	-332425529072.543\\
2.50786269656741	-332451885131.119\\
2.50796269906748	-332478241189.695\\
2.50806270156754	-332504024290.476\\
2.5081627040676	-332530380349.052\\
2.50826270656766	-332556736407.628\\
2.50836270906773	-332582519508.408\\
2.50846271156779	-332608875566.984\\
2.50856271406785	-332635231625.56\\
2.50866271656791	-332661014726.341\\
2.50876271906798	-332687370784.917\\
2.50886272156804	-332713153885.698\\
2.5089627240681	-332739509944.274\\
2.50906272656816	-332765866002.85\\
2.50916272906823	-332791649103.631\\
2.50926273156829	-332817432204.412\\
2.50936273406835	-332843788262.988\\
2.50946273656841	-332869571363.769\\
2.50956273906848	-332895927422.345\\
2.50966274156854	-332921710523.126\\
2.5097627440686	-332947493623.907\\
2.50986274656866	-332973849682.483\\
2.50996274906873	-332999632783.264\\
2.51006275156879	-333025415884.045\\
2.51016275406885	-333051771942.621\\
2.51026275656891	-333077555043.401\\
2.51036275906898	-333103338144.182\\
2.51046276156904	-333129121244.963\\
2.5105627640691	-333154904345.744\\
2.51066276656916	-333180687446.525\\
2.51076276906923	-333206470547.306\\
2.51086277156929	-333232253648.087\\
2.51096277406935	-333258609706.663\\
2.51106277656941	-333284392807.444\\
2.51116277906948	-333310175908.225\\
2.51126278156954	-333335959009.005\\
2.5113627840696	-333361169151.991\\
2.51146278656966	-333386952252.772\\
2.51156278906973	-333412735353.553\\
2.51166279156979	-333438518454.334\\
2.51176279406985	-333464301555.115\\
2.51186279656991	-333490084655.896\\
2.51196279906998	-333515867756.677\\
2.51206280157004	-333541077899.662\\
2.5121628040701	-333566861000.443\\
2.51226280657016	-333592644101.224\\
2.51236280907023	-333618427202.005\\
2.51246281157029	-333643637344.991\\
2.51256281407035	-333669420445.772\\
2.51266281657041	-333695203546.552\\
2.51276281907048	-333720413689.538\\
2.51286282157054	-333746196790.319\\
2.5129628240706	-333771406933.305\\
2.51306282657066	-333797190034.086\\
2.51316282907073	-333822400177.072\\
2.51326283157079	-333848183277.852\\
2.51336283407085	-333873393420.838\\
2.51346283657091	-333899176521.619\\
2.51356283907098	-333924386664.605\\
2.51366284157104	-333950169765.386\\
2.5137628440711	-333975379908.371\\
2.51386284657116	-334000590051.357\\
2.51396284907123	-334026373152.138\\
2.51406285157129	-334051583295.124\\
2.51416285407135	-334076793438.11\\
2.51426285657141	-334102003581.095\\
2.51436285907148	-334127213724.081\\
2.51446286157154	-334152996824.862\\
2.5145628640716	-334178206967.848\\
2.51466286657166	-334203417110.834\\
2.51476286907173	-334228627253.819\\
2.51486287157179	-334253837396.805\\
2.51496287407185	-334279047539.791\\
2.51506287657191	-334304257682.777\\
2.51516287907198	-334329467825.762\\
2.51526288157204	-334354677968.748\\
2.5153628840721	-334379888111.734\\
2.51546288657216	-334405098254.72\\
2.51556288907223	-334430308397.705\\
2.51566289157229	-334455518540.691\\
2.51576289407235	-334480728683.677\\
2.51586289657241	-334505938826.663\\
2.51596289907248	-334530576011.853\\
2.51606290157254	-334555786154.839\\
2.5161629040726	-334580996297.825\\
2.51626290657266	-334606206440.81\\
2.51636290907273	-334631416583.796\\
2.51646291157279	-334656053768.987\\
2.51656291407285	-334681263911.973\\
2.51666291657291	-334706474054.958\\
2.51676291907298	-334731111240.149\\
2.51686292157304	-334756321383.135\\
2.5169629240731	-334780958568.325\\
2.51706292657316	-334806168711.311\\
2.51716292907323	-334830805896.502\\
2.51726293157329	-334856016039.488\\
2.51736293407335	-334880653224.678\\
2.51746293657341	-334905863367.664\\
2.51756293907348	-334930500552.855\\
2.51766294157354	-334955710695.84\\
2.5177629440736	-334980347881.031\\
2.51786294657366	-335004985066.222\\
2.51796294907373	-335030195209.207\\
2.51806295157379	-335054832394.398\\
2.51816295407385	-335079469579.589\\
2.51826295657391	-335104106764.779\\
2.51836295907398	-335129316907.765\\
2.51846296157404	-335153954092.956\\
2.5185629640741	-335178591278.146\\
2.51866296657416	-335203228463.337\\
2.51876296907423	-335227865648.527\\
2.51886297157429	-335252502833.718\\
2.51896297407435	-335277140018.909\\
2.51906297657441	-335301777204.099\\
2.51916297907448	-335326414389.29\\
2.51926298157454	-335351051574.481\\
2.5193629840746	-335375688759.671\\
2.51946298657466	-335400325944.862\\
2.51956298907473	-335424963130.052\\
2.51966299157479	-335449600315.243\\
2.51976299407485	-335474237500.434\\
2.51986299657491	-335498874685.624\\
2.51996299907498	-335523511870.815\\
2.52006300157504	-335547576098.21\\
2.5201630040751	-335572213283.401\\
2.52026300657516	-335596850468.592\\
2.52036300907523	-335621487653.782\\
2.52046301157529	-335645551881.178\\
2.52056301407535	-335670189066.368\\
2.52066301657541	-335694826251.559\\
2.52076301907548	-335718890478.955\\
2.52086302157554	-335743527664.145\\
2.5209630240756	-335767591891.541\\
2.52106302657566	-335792229076.731\\
2.52116302907573	-335816866261.922\\
2.52126303157579	-335840930489.317\\
2.52136303407585	-335865567674.508\\
2.52146303657591	-335889631901.904\\
2.52156303907598	-335913696129.299\\
2.52166304157604	-335938333314.49\\
2.5217630440761	-335962397541.885\\
2.52186304657616	-335986461769.281\\
2.52196304907623	-336011098954.471\\
2.52206305157629	-336035163181.867\\
2.52216305407635	-336059227409.262\\
2.52226305657641	-336083864594.453\\
2.52236305907648	-336107928821.848\\
2.52246306157654	-336131993049.244\\
2.5225630640766	-336156057276.639\\
2.52266306657666	-336180121504.035\\
2.52276306907673	-336204185731.43\\
2.52286307157679	-336228249958.826\\
2.52296307407685	-336252887144.016\\
2.52306307657691	-336276951371.412\\
2.52316307907698	-336301015598.807\\
2.52326308157704	-336325079826.203\\
2.5233630840771	-336349144053.598\\
2.52346308657716	-336372635323.199\\
2.52356308907723	-336396699550.594\\
2.52366309157729	-336420763777.99\\
2.52376309407735	-336444828005.385\\
2.52386309657741	-336468892232.781\\
2.52396309907748	-336492956460.176\\
2.52406310157754	-336517020687.572\\
2.5241631040776	-336540511957.172\\
2.52426310657766	-336564576184.568\\
2.52436310907773	-336588640411.963\\
2.52446311157779	-336612131681.564\\
2.52456311407785	-336636195908.959\\
2.52466311657791	-336660260136.354\\
2.52476311907798	-336683751405.955\\
2.52486312157804	-336707815633.35\\
2.5249631240781	-336731306902.951\\
2.52506312657816	-336755371130.346\\
2.52516312907823	-336779435357.742\\
2.52526313157829	-336802926627.342\\
2.52536313407835	-336826417896.942\\
2.52546313657841	-336850482124.338\\
2.52556313907848	-336873973393.938\\
2.52566314157854	-336898037621.334\\
2.5257631440786	-336921528890.934\\
2.52586314657866	-336945020160.535\\
2.52596314907873	-336969084387.93\\
2.52606315157879	-336992575657.53\\
2.52616315407885	-337016066927.131\\
2.52626315657891	-337039558196.731\\
2.52636315907898	-337063622424.127\\
2.52646316157904	-337087113693.727\\
2.5265631640791	-337110604963.327\\
2.52666316657916	-337134096232.928\\
2.52676316907923	-337157587502.528\\
2.52686317157929	-337181078772.128\\
2.52696317407935	-337204570041.729\\
2.52706317657941	-337228061311.329\\
2.52716317907948	-337251552580.93\\
2.52726318157954	-337275043850.53\\
2.5273631840796	-337298535120.13\\
2.52746318657966	-337322026389.731\\
2.52756318907973	-337345517659.331\\
2.52766319157979	-337369008928.931\\
2.52776319407985	-337392500198.532\\
2.52786319657991	-337415991468.132\\
2.52796319907998	-337438909779.937\\
2.52806320158004	-337462401049.538\\
2.5281632040801	-337485892319.138\\
2.52826320658016	-337509383588.738\\
2.52836320908023	-337532301900.544\\
2.52846321158029	-337555793170.144\\
2.52856321408035	-337579284439.744\\
2.52866321658041	-337602202751.55\\
2.52876321908048	-337625694021.15\\
2.52886322158054	-337648612332.955\\
2.5289632240806	-337672103602.556\\
2.52906322658066	-337695021914.361\\
2.52916322908073	-337718513183.961\\
2.52926323158079	-337741431495.766\\
2.52936323408085	-337764922765.367\\
2.52946323658091	-337787841077.172\\
2.52956323908098	-337810759388.977\\
2.52966324158104	-337834250658.578\\
2.5297632440811	-337857168970.383\\
2.52986324658116	-337880087282.188\\
2.52996324908123	-337903578551.788\\
2.53006325158129	-337926496863.594\\
2.53016325408135	-337949415175.399\\
2.53026325658141	-337972333487.204\\
2.53036325908148	-337995824756.804\\
2.53046326158154	-338018743068.61\\
2.5305632640816	-338041661380.415\\
2.53066326658166	-338064579692.22\\
2.53076326908173	-338087498004.025\\
2.53086327158179	-338110416315.831\\
2.53096327408185	-338133334627.636\\
2.53106327658191	-338156252939.441\\
2.53116327908198	-338179171251.246\\
2.53126328158204	-338202089563.052\\
2.5313632840821	-338225007874.857\\
2.53146328658216	-338247926186.662\\
2.53156328908223	-338270844498.467\\
2.53166329158229	-338293762810.272\\
2.53176329408235	-338316108164.283\\
2.53186329658241	-338339026476.088\\
2.53196329908248	-338361944787.893\\
2.53206330158254	-338384863099.698\\
2.5321633040826	-338407208453.708\\
2.53226330658266	-338430126765.514\\
2.53236330908273	-338453045077.319\\
2.53246331158279	-338475390431.329\\
2.53256331408285	-338498308743.134\\
2.53266331658291	-338520654097.144\\
2.53276331908298	-338543572408.95\\
2.53286332158304	-338566490720.755\\
2.5329633240831	-338588836074.765\\
2.53306332658316	-338611754386.57\\
2.53316332908323	-338634099740.58\\
2.53326333158329	-338656445094.59\\
2.53336333408335	-338679363406.396\\
2.53346333658341	-338701708760.406\\
2.53356333908348	-338724627072.211\\
2.53366334158354	-338746972426.221\\
2.5337633440836	-338769317780.231\\
2.53386334658366	-338791663134.241\\
2.53396334908373	-338814581446.046\\
2.53406335158379	-338836926800.057\\
2.53416335408385	-338859272154.067\\
2.53426335658391	-338881617508.077\\
2.53436335908398	-338903962862.087\\
2.53446336158404	-338926308216.097\\
2.5345633640841	-338949226527.902\\
2.53466336658416	-338971571881.912\\
2.53476336908423	-338993917235.922\\
2.53486337158429	-339016262589.932\\
2.53496337408435	-339038607943.943\\
2.53506337658441	-339060953297.953\\
2.53516337908448	-339083298651.963\\
2.53526338158454	-339105071048.178\\
2.5353633840846	-339127416402.188\\
2.53546338658466	-339149761756.198\\
2.53556338908473	-339172107110.208\\
2.53566339158479	-339194452464.218\\
2.53576339408485	-339216797818.228\\
2.53586339658491	-339238570214.443\\
2.53596339908498	-339260915568.453\\
2.53606340158504	-339283260922.463\\
2.5361634040851	-339305033318.678\\
2.53626340658516	-339327378672.688\\
2.53636340908523	-339349724026.699\\
2.53646341158529	-339371496422.914\\
2.53656341408535	-339393841776.924\\
2.53666341658541	-339415614173.139\\
2.53676341908548	-339437959527.149\\
2.53686342158554	-339459731923.364\\
2.5369634240856	-339482077277.374\\
2.53706342658566	-339503849673.589\\
2.53716342908573	-339526195027.599\\
2.53726343158579	-339547967423.814\\
2.53736343408585	-339569739820.029\\
2.53746343658591	-339592085174.039\\
2.53756343908598	-339613857570.254\\
2.53766344158604	-339635629966.469\\
2.5377634440861	-339657975320.479\\
2.53786344658616	-339679747716.694\\
2.53796344908623	-339701520112.909\\
2.53806345158629	-339723292509.124\\
2.53816345408635	-339745064905.339\\
2.53826345658641	-339766837301.554\\
2.53836345908648	-339788609697.769\\
2.53846346158654	-339810955051.779\\
2.5385634640866	-339832727447.994\\
2.53866346658666	-339854499844.209\\
2.53876346908673	-339876272240.424\\
2.53886347158679	-339898044636.639\\
2.53896347408685	-339919244075.059\\
2.53906347658691	-339941016471.274\\
2.53916347908698	-339962788867.489\\
2.53926348158704	-339984561263.704\\
2.5393634840871	-340006333659.919\\
2.53946348658716	-340028106056.133\\
2.53956348908723	-340049305494.553\\
2.53966349158729	-340071077890.768\\
2.53976349408735	-340092850286.983\\
2.53986349658741	-340114622683.198\\
2.53996349908748	-340135822121.618\\
2.54006350158754	-340157594517.833\\
2.5401635040876	-340179366914.048\\
2.54026350658766	-340200566352.468\\
2.54036350908773	-340222338748.683\\
2.54046351158779	-340243538187.103\\
2.54056351408785	-340265310583.318\\
2.54066351658791	-340286510021.737\\
2.54076351908798	-340308282417.952\\
2.54086352158804	-340329481856.372\\
2.5409635240881	-340351254252.587\\
2.54106352658816	-340372453691.007\\
2.54116352908823	-340393653129.427\\
2.54126353158829	-340415425525.642\\
2.54136353408835	-340436624964.062\\
2.54146353658841	-340457824402.482\\
2.54156353908848	-340479023840.901\\
2.54166354158854	-340500796237.116\\
2.5417635440886	-340521995675.536\\
2.54186354658866	-340543195113.956\\
2.54196354908873	-340564394552.376\\
2.54206355158879	-340585593990.796\\
2.54216355408885	-340606793429.216\\
2.54226355658891	-340627992867.635\\
2.54236355908898	-340649192306.055\\
2.54246356158904	-340670391744.475\\
2.5425635640891	-340691591182.895\\
2.54266356658916	-340712790621.315\\
2.54276356908923	-340733990059.735\\
2.54286357158929	-340755189498.154\\
2.54296357408935	-340776388936.574\\
2.54306357658941	-340797588374.994\\
2.54316357908948	-340818787813.414\\
2.54326358158954	-340839987251.834\\
2.5433635840896	-340860613732.459\\
2.54346358658966	-340881813170.878\\
2.54356358908973	-340903012609.298\\
2.54366359158979	-340923639089.923\\
2.54376359408985	-340944838528.343\\
2.54386359658991	-340966037966.763\\
2.54396359908998	-340986664447.387\\
2.54406360159004	-341007863885.807\\
2.5441636040901	-341029063324.227\\
2.54426360659016	-341049689804.852\\
2.54436360909023	-341070889243.272\\
2.54446361159029	-341091515723.896\\
2.54456361409035	-341112142204.521\\
2.54466361659041	-341133341642.941\\
2.54476361909048	-341153968123.566\\
2.54486362159054	-341175167561.985\\
2.5449636240906	-341195794042.61\\
2.54506362659066	-341216420523.235\\
2.54516362909073	-341237619961.655\\
2.54526363159079	-341258246442.279\\
2.54536363409085	-341278872922.904\\
2.54546363659091	-341299499403.529\\
2.54556363909098	-341320125884.154\\
2.54566364159104	-341341325322.573\\
2.5457636440911	-341361951803.198\\
2.54586364659116	-341382578283.823\\
2.54596364909123	-341403204764.447\\
2.54606365159129	-341423831245.072\\
2.54616365409135	-341444457725.697\\
2.54626365659141	-341465084206.322\\
2.54636365909148	-341485710686.946\\
2.54646366159154	-341506337167.571\\
2.5465636640916	-341526963648.196\\
2.54666366659166	-341547590128.82\\
2.54676366909173	-341568216609.445\\
2.54686367159179	-341588270132.275\\
2.54696367409185	-341608896612.899\\
2.54706367659192	-341629523093.524\\
2.54716367909198	-341650149574.149\\
2.54726368159204	-341670203096.978\\
2.5473636840921	-341690829577.603\\
2.54746368659216	-341711456058.228\\
2.54756368909223	-341731509581.057\\
2.54766369159229	-341752136061.682\\
2.54776369409235	-341772762542.307\\
2.54786369659241	-341792816065.136\\
2.54796369909248	-341813442545.761\\
2.54806370159254	-341833496068.591\\
2.5481637040926	-341854122549.215\\
2.54826370659266	-341874176072.045\\
2.54836370909273	-341894802552.67\\
2.54846371159279	-341914856075.499\\
2.54856371409285	-341934909598.329\\
2.54866371659291	-341955536078.954\\
2.54876371909298	-341975589601.783\\
2.54886372159304	-341995643124.613\\
2.5489637240931	-342016269605.237\\
2.54906372659317	-342036323128.067\\
2.54916372909323	-342056376650.897\\
2.54926373159329	-342076430173.726\\
2.54936373409335	-342096483696.556\\
2.54946373659341	-342117110177.18\\
2.54956373909348	-342137163700.01\\
2.54966374159354	-342157217222.84\\
2.5497637440936	-342177270745.669\\
2.54986374659366	-342197324268.499\\
2.54996374909373	-342217377791.328\\
2.55006375159379	-342237431314.158\\
2.55016375409385	-342257484836.988\\
2.55026375659391	-342277538359.817\\
2.55036375909398	-342297591882.647\\
2.55046376159404	-342317072447.681\\
2.5505637640941	-342337125970.511\\
2.55066376659416	-342357179493.34\\
2.55076376909423	-342377233016.17\\
2.55086377159429	-342397286538.999\\
2.55096377409435	-342416767104.034\\
2.55106377659442	-342436820626.863\\
2.55116377909448	-342456874149.693\\
2.55126378159454	-342476354714.727\\
2.5513637840946	-342496408237.557\\
2.55146378659466	-342515888802.591\\
2.55156378909473	-342535942325.421\\
2.55166379159479	-342555995848.251\\
2.55176379409485	-342575476413.285\\
2.55186379659491	-342595529936.115\\
2.55196379909498	-342615010501.149\\
2.55206380159504	-342634491066.184\\
2.5521638040951	-342654544589.013\\
2.55226380659516	-342674025154.048\\
2.55236380909523	-342694078676.877\\
2.55246381159529	-342713559241.912\\
2.55256381409535	-342733039806.946\\
2.55266381659541	-342752520371.981\\
2.55276381909548	-342772573894.81\\
2.55286382159554	-342792054459.845\\
2.5529638240956	-342811535024.879\\
2.55306382659567	-342831015589.913\\
2.55316382909573	-342850496154.948\\
2.55326383159579	-342869976719.982\\
2.55336383409585	-342890030242.812\\
2.55346383659592	-342909510807.846\\
2.55356383909598	-342928991372.881\\
2.55366384159604	-342948471937.915\\
2.5537638440961	-342967952502.95\\
2.55386384659616	-342986860110.189\\
2.55396384909623	-343006340675.224\\
2.55406385159629	-343025821240.258\\
2.55416385409635	-343045301805.292\\
2.55426385659641	-343064782370.327\\
2.55436385909648	-343084262935.361\\
2.55446386159654	-343103743500.396\\
2.5545638640966	-343122651107.635\\
2.55466386659666	-343142131672.669\\
2.55476386909673	-343161612237.704\\
2.55486387159679	-343180519844.943\\
2.55496387409685	-343200000409.978\\
2.55506387659692	-343219480975.012\\
2.55516387909698	-343238388582.251\\
2.55526388159704	-343257869147.286\\
2.5553638840971	-343276776754.525\\
2.55546388659717	-343296257319.56\\
2.55556388909723	-343315164926.799\\
2.55566389159729	-343334645491.833\\
2.55576389409735	-343353553099.073\\
2.55586389659741	-343372460706.312\\
2.55596389909748	-343391941271.347\\
2.55606390159754	-343410848878.586\\
2.5561639040976	-343429756485.825\\
2.55626390659766	-343449237050.86\\
2.55636390909773	-343468144658.099\\
2.55646391159779	-343487052265.338\\
2.55656391409785	-343505959872.578\\
2.55666391659791	-343525440437.612\\
2.55676391909798	-343544348044.851\\
2.55686392159804	-343563255652.091\\
2.5569639240981	-343582163259.33\\
2.55706392659817	-343601070866.569\\
2.55716392909823	-343619978473.809\\
2.55726393159829	-343638886081.048\\
2.55736393409835	-343657793688.287\\
2.55746393659842	-343676701295.527\\
2.55756393909848	-343695608902.766\\
2.55766394159854	-343714516510.005\\
2.5577639440986	-343733424117.245\\
2.55786394659867	-343751758766.689\\
2.55796394909873	-343770666373.928\\
2.55806395159879	-343789573981.167\\
2.55816395409885	-343808481588.407\\
2.55826395659891	-343827389195.646\\
2.55836395909898	-343845723845.09\\
2.55846396159904	-343864631452.329\\
2.5585639640991	-343883539059.569\\
2.55866396659916	-343901873709.013\\
2.55876396909923	-343920781316.252\\
2.55886397159929	-343939115965.696\\
2.55896397409935	-343958023572.936\\
2.55906397659942	-343976358222.38\\
2.55916397909948	-343995265829.619\\
2.55926398159954	-344013600479.063\\
2.5593639840996	-344032508086.303\\
2.55946398659967	-344050842735.747\\
2.55956398909973	-344069177385.191\\
2.55966399159979	-344088084992.43\\
2.55976399409985	-344106419641.875\\
2.55986399659992	-344124754291.319\\
2.55996399909998	-344143661898.558\\
2.56006400160004	-344161996548.002\\
2.5601640041001	-344180331197.447\\
2.56026400660016	-344198665846.891\\
2.56036400910023	-344217000496.335\\
2.56046401160029	-344235908103.574\\
2.56056401410035	-344254242753.018\\
2.56066401660041	-344272577402.463\\
2.56076401910048	-344290912051.907\\
2.56086402160054	-344309246701.351\\
2.5609640241006	-344327581350.795\\
2.56106402660067	-344345916000.239\\
2.56116402910073	-344364250649.684\\
2.56126403160079	-344382012341.333\\
2.56136403410085	-344400346990.777\\
2.56146403660092	-344418681640.221\\
2.56156403910098	-344437016289.665\\
2.56166404160104	-344455350939.109\\
2.5617640441011	-344473685588.554\\
2.56186404660117	-344491447280.203\\
2.56196404910123	-344509781929.647\\
2.56206405160129	-344528116579.091\\
2.56216405410135	-344545878270.74\\
2.56226405660141	-344564212920.184\\
2.56236405910148	-344582547569.628\\
2.56246406160154	-344600309261.277\\
2.5625640641016	-344618643910.722\\
2.56266406660166	-344636405602.371\\
2.56276406910173	-344654740251.815\\
2.56286407160179	-344672501943.464\\
2.56296407410185	-344690263635.113\\
2.56306407660192	-344708598284.557\\
2.56316407910198	-344726359976.206\\
2.56326408160204	-344744694625.65\\
2.5633640841021	-344762456317.299\\
2.56346408660217	-344780218008.949\\
2.56356408910223	-344797979700.598\\
2.56366409160229	-344816314350.042\\
2.56376409410235	-344834076041.691\\
2.56386409660242	-344851837733.34\\
2.56396409910248	-344869599424.989\\
2.56406410160254	-344887361116.638\\
2.5641641041026	-344905122808.287\\
2.56426410660267	-344922884499.936\\
2.56436410910273	-344940646191.585\\
2.56446411160279	-344958407883.234\\
2.56456411410285	-344976169574.883\\
2.56466411660291	-344993931266.532\\
2.56476411910298	-345011692958.181\\
2.56486412160304	-345029454649.83\\
2.5649641241031	-345047216341.479\\
2.56506412660317	-345064978033.129\\
2.56516412910323	-345082739724.778\\
2.56526413160329	-345100501416.427\\
2.56536413410335	-345117690150.281\\
2.56546413660342	-345135451841.93\\
2.56556413910348	-345153213533.579\\
2.56566414160354	-345170402267.433\\
2.5657641441036	-345188163959.082\\
2.56586414660367	-345205925650.731\\
2.56596414910373	-345223114384.585\\
2.56606415160379	-345240876076.234\\
2.56616415410385	-345258064810.088\\
2.56626415660392	-345275826501.737\\
2.56636415910398	-345293015235.591\\
2.56646416160404	-345310776927.24\\
2.5665641641041	-345327965661.094\\
2.56666416660416	-345345154394.948\\
2.56676416910423	-345362916086.597\\
2.56686417160429	-345380104820.451\\
2.56696417410435	-345397293554.304\\
2.56706417660442	-345415055245.953\\
2.56716417910448	-345432243979.807\\
2.56726418160454	-345449432713.661\\
2.5673641841046	-345466621447.515\\
2.56746418660467	-345484383139.164\\
2.56756418910473	-345501571873.018\\
2.56766419160479	-345518760606.872\\
2.56776419410485	-345535949340.726\\
2.56786419660492	-345553138074.58\\
2.56796419910498	-345570326808.434\\
2.56806420160504	-345587515542.288\\
2.5681642041051	-345604704276.142\\
2.56826420660517	-345621893009.996\\
2.56836420910523	-345639081743.85\\
2.56846421160529	-345656270477.704\\
2.56856421410535	-345672886253.762\\
2.56866421660541	-345690074987.616\\
2.56876421910548	-345707263721.47\\
2.56886422160554	-345724452455.324\\
2.5689642241056	-345741641189.178\\
2.56906422660567	-345758256965.237\\
2.56916422910573	-345775445699.091\\
2.56926423160579	-345792634432.945\\
2.56936423410585	-345809250209.003\\
2.56946423660592	-345826438942.857\\
2.56956423910598	-345843054718.916\\
2.56966424160604	-345860243452.77\\
2.5697642441061	-345877432186.624\\
2.56986424660617	-345894047962.683\\
2.56996424910623	-345910663738.742\\
2.57006425160629	-345927852472.596\\
2.57016425410635	-345944468248.654\\
2.57026425660642	-345961656982.508\\
2.57036425910648	-345978272758.567\\
2.57046426160654	-345994888534.626\\
2.5705642641066	-346012077268.48\\
2.57066426660667	-346028693044.539\\
2.57076426910673	-346045308820.597\\
2.57086427160679	-346061924596.656\\
2.57096427410685	-346078540372.715\\
2.57106427660692	-346095729106.569\\
2.57116427910698	-346112344882.628\\
2.57126428160704	-346128960658.686\\
2.5713642841071	-346145576434.745\\
2.57146428660717	-346162192210.804\\
2.57156428910723	-346178807986.863\\
2.57166429160729	-346195423762.922\\
2.57176429410735	-346212039538.98\\
2.57186429660742	-346228655315.039\\
2.57196429910748	-346245271091.098\\
2.57206430160754	-346261313909.362\\
2.5721643041076	-346277929685.421\\
2.57226430660767	-346294545461.479\\
2.57236430910773	-346311161237.538\\
2.57246431160779	-346327777013.597\\
2.57256431410785	-346343819831.861\\
2.57266431660792	-346360435607.919\\
2.57276431910798	-346377051383.978\\
2.57286432160804	-346393094202.242\\
2.5729643241081	-346409709978.301\\
2.57306432660817	-346425752796.564\\
2.57316432910823	-346442368572.623\\
2.57326433160829	-346458411390.887\\
2.57336433410835	-346475027166.945\\
2.57346433660842	-346491069985.209\\
2.57356433910848	-346507685761.268\\
2.57366434160854	-346523728579.532\\
2.5737643441086	-346540344355.59\\
2.57386434660867	-346556387173.854\\
2.57396434910873	-346572429992.118\\
2.57406435160879	-346589045768.177\\
2.57416435410885	-346605088586.44\\
2.57426435660892	-346621131404.704\\
2.57436435910898	-346637174222.968\\
2.57446436160904	-346653217041.231\\
2.5745643641091	-346669832817.29\\
2.57466436660917	-346685875635.554\\
2.57476436910923	-346701918453.817\\
2.57486437160929	-346717961272.081\\
2.57496437410935	-346734004090.345\\
2.57506437660942	-346750046908.608\\
2.57516437910948	-346766089726.872\\
2.57526438160954	-346782132545.136\\
2.5753643841096	-346798175363.399\\
2.57546438660967	-346814218181.663\\
2.57556438910973	-346829688042.131\\
2.57566439160979	-346845730860.395\\
2.57576439410985	-346861773678.659\\
2.57586439660992	-346877816496.922\\
2.57596439910998	-346893859315.186\\
2.57606440161004	-346909329175.655\\
2.5761644041101	-346925371993.918\\
2.57626440661017	-346941414812.182\\
2.57636440911023	-346956884672.651\\
2.57646441161029	-346972927490.914\\
2.57656441411035	-346988397351.383\\
2.57666441661042	-347004440169.646\\
2.57676441911048	-347019910030.115\\
2.57686442161054	-347035952848.379\\
2.5769644241106	-347051422708.847\\
2.57706442661067	-347067465527.111\\
2.57716442911073	-347082935387.579\\
2.57726443161079	-347098978205.843\\
2.57736443411085	-347114448066.312\\
2.57746443661092	-347129917926.78\\
2.57756443911098	-347145387787.249\\
2.57766444161104	-347161430605.512\\
2.5777644441111	-347176900465.981\\
2.57786444661117	-347192370326.449\\
2.57796444911123	-347207840186.918\\
2.57806445161129	-347223310047.386\\
2.57816445411135	-347239352865.65\\
2.57826445661142	-347254822726.119\\
2.57836445911148	-347270292586.587\\
2.57846446161154	-347285762447.056\\
2.5785644641116	-347301232307.524\\
2.57866446661167	-347316702167.993\\
2.57876446911173	-347332172028.461\\
2.57886447161179	-347347068931.135\\
2.57896447411185	-347362538791.603\\
2.57906447661192	-347378008652.072\\
2.57916447911198	-347393478512.54\\
2.57926448161204	-347408948373.009\\
2.5793644841121	-347424418233.477\\
2.57946448661217	-347439315136.151\\
2.57956448911223	-347454784996.619\\
2.57966449161229	-347470254857.088\\
2.57976449411235	-347485151759.761\\
2.57986449661242	-347500621620.23\\
2.57996449911248	-347515518522.903\\
2.58006450161254	-347530988383.372\\
2.5801645041126	-347546458243.84\\
2.58026450661267	-347561355146.514\\
2.58036450911273	-347576252049.187\\
2.58046451161279	-347591721909.655\\
2.58056451411285	-347606618812.329\\
2.58066451661292	-347622088672.797\\
2.58076451911298	-347636985575.471\\
2.58086452161304	-347651882478.144\\
2.5809645241131	-347667352338.613\\
2.58106452661317	-347682249241.286\\
2.58116452911323	-347697146143.96\\
2.58126453161329	-347712043046.633\\
2.58136453411335	-347727512907.102\\
2.58146453661342	-347742409809.775\\
2.58156453911348	-347757306712.448\\
2.58166454161354	-347772203615.122\\
2.5817645441136	-347787100517.795\\
2.58186454661367	-347801997420.469\\
2.58196454911373	-347816894323.142\\
2.58206455161379	-347831791225.815\\
2.58216455411385	-347846688128.489\\
2.58226455661392	-347861585031.162\\
2.58236455911398	-347876481933.836\\
2.58246456161404	-347891378836.509\\
2.5825645641141	-347905702781.387\\
2.58266456661417	-347920599684.061\\
2.58276456911423	-347935496586.734\\
2.58286457161429	-347950393489.407\\
2.58296457411435	-347964717434.286\\
2.58306457661442	-347979614336.959\\
2.58316457911448	-347994511239.632\\
2.58326458161454	-348008835184.511\\
2.5833645841146	-348023732087.184\\
2.58346458661467	-348038056032.062\\
2.58356458911473	-348052952934.736\\
2.58366459161479	-348067276879.614\\
2.58376459411485	-348082173782.287\\
2.58386459661492	-348096497727.166\\
2.58396459911498	-348111394629.839\\
2.58406460161504	-348125718574.717\\
2.5841646041151	-348140615477.391\\
2.58426460661517	-348154939422.269\\
2.58436460911523	-348169263367.147\\
2.58446461161529	-348183587312.026\\
2.58456461411535	-348198484214.699\\
2.58466461661542	-348212808159.577\\
2.58476461911548	-348227132104.456\\
2.58486462161554	-348241456049.334\\
2.5849646241156	-348255779994.212\\
2.58506462661567	-348270103939.09\\
2.58516462911573	-348284427883.969\\
2.58526463161579	-348298751828.847\\
2.58536463411585	-348313075773.725\\
2.58546463661592	-348327399718.603\\
2.58556463911598	-348341723663.482\\
2.58566464161604	-348356047608.36\\
2.5857646441161	-348370371553.238\\
2.58586464661617	-348384695498.117\\
2.58596464911623	-348399019442.995\\
2.58606465161629	-348413343387.873\\
2.58616465411635	-348427094374.956\\
2.58626465661642	-348441418319.834\\
2.58636465911648	-348455742264.713\\
2.58646466161654	-348470066209.591\\
2.5865646641166	-348483817196.674\\
2.58666466661667	-348498141141.552\\
2.58676466911673	-348511892128.636\\
2.58686467161679	-348526216073.514\\
2.58696467411685	-348539967060.597\\
2.58706467661692	-348554291005.475\\
2.58716467911698	-348568041992.558\\
2.58726468161704	-348582365937.437\\
2.5873646841171	-348596116924.52\\
2.58746468661717	-348610440869.398\\
2.58756468911723	-348624191856.481\\
2.58766469161729	-348637942843.564\\
2.58776469411735	-348652266788.443\\
2.58786469661742	-348666017775.526\\
2.58796469911748	-348679768762.609\\
2.58806470161754	-348693519749.692\\
2.5881647041176	-348707270736.775\\
2.58826470661767	-348721594681.653\\
2.58836470911773	-348735345668.737\\
2.58846471161779	-348749096655.82\\
2.58856471411785	-348762847642.903\\
2.58866471661792	-348776598629.986\\
2.58876471911798	-348790349617.069\\
2.58886472161804	-348804100604.152\\
2.5889647241181	-348817851591.235\\
2.58906472661817	-348831602578.319\\
2.58916472911823	-348844780607.607\\
2.58926473161829	-348858531594.69\\
2.58936473411835	-348872282581.773\\
2.58946473661842	-348886033568.856\\
2.58956473911848	-348899784555.939\\
2.58966474161854	-348912962585.227\\
2.5897647441186	-348926713572.31\\
2.58986474661867	-348940464559.393\\
2.58996474911873	-348953642588.681\\
2.59006475161879	-348967393575.765\\
2.59016475411885	-348981144562.848\\
2.59026475661892	-348994322592.136\\
2.59036475911898	-349008073579.219\\
2.59046476161904	-349021251608.507\\
2.5905647641191	-349035002595.59\\
2.59066476661917	-349048180624.878\\
2.59076476911923	-349061358654.166\\
2.59086477161929	-349075109641.249\\
2.59096477411935	-349088287670.537\\
2.59106477661942	-349101465699.825\\
2.59116477911948	-349115216686.908\\
2.59126478161954	-349128394716.196\\
2.5913647841196	-349141572745.484\\
2.59146478661967	-349154750774.772\\
2.59156478911973	-349167928804.06\\
2.59166479161979	-349181106833.348\\
2.59176479411985	-349194857820.432\\
2.59186479661992	-349208035849.72\\
2.59196479911998	-349221213879.008\\
2.59206480162004	-349234391908.296\\
2.5921648041201	-349247569937.584\\
2.59226480662017	-349260747966.872\\
2.59236480912023	-349273353038.364\\
2.59246481162029	-349286531067.652\\
2.59256481412035	-349299709096.94\\
2.59266481662042	-349312887126.228\\
2.59276481912048	-349326065155.516\\
2.59286482162054	-349339243184.805\\
2.5929648241206	-349351848256.297\\
2.59306482662067	-349365026285.585\\
2.59316482912073	-349378204314.873\\
2.59326483162079	-349390809386.366\\
2.59336483412085	-349403987415.654\\
2.59346483662092	-349416592487.147\\
2.59356483912098	-349429770516.435\\
2.59366484162104	-349442948545.723\\
2.5937648441211	-349455553617.216\\
2.59386484662117	-349468158688.709\\
2.59396484912123	-349481336717.997\\
2.59406485162129	-349493941789.49\\
2.59416485412135	-349507119818.778\\
2.59426485662142	-349519724890.271\\
2.59436485912148	-349532329961.764\\
2.59446486162154	-349545507991.052\\
2.5945648641216	-349558113062.544\\
2.59466486662167	-349570718134.037\\
2.59476486912173	-349583323205.53\\
2.59486487162179	-349595928277.023\\
2.59496487412185	-349608533348.516\\
2.59506487662192	-349621138420.009\\
2.59516487912198	-349634316449.297\\
2.59526488162204	-349646921520.79\\
2.5953648841221	-349659526592.283\\
2.59546488662217	-349672131663.776\\
2.59556488912223	-349684163777.473\\
2.59566489162229	-349696768848.966\\
2.59576489412235	-349709373920.459\\
2.59586489662242	-349721978991.952\\
2.59596489912248	-349734584063.445\\
2.59606490162254	-349747189134.938\\
2.5961649041226	-349759221248.635\\
2.59626490662267	-349771826320.128\\
2.59636490912273	-349784431391.621\\
2.59646491162279	-349797036463.114\\
2.59656491412285	-349809068576.812\\
2.59666491662292	-349821673648.305\\
2.59676491912298	-349833705762.002\\
2.59686492162304	-349846310833.495\\
2.5969649241231	-349858342947.193\\
2.59706492662317	-349870948018.686\\
2.59716492912323	-349882980132.384\\
2.59726493162329	-349895585203.877\\
2.59736493412335	-349907617317.574\\
2.59746493662342	-349919649431.272\\
2.59756493912348	-349932254502.765\\
2.59766494162354	-349944286616.463\\
2.5977649441236	-349956318730.16\\
2.59786494662367	-349968350843.858\\
2.59796494912373	-349980955915.351\\
2.59806495162379	-349992988029.049\\
2.59816495412385	-350005020142.747\\
2.59826495662392	-350017052256.444\\
2.59836495912398	-350029084370.142\\
2.59846496162404	-350041116483.84\\
2.5985649641241	-350053148597.537\\
2.59866496662417	-350065180711.235\\
2.59876496912423	-350077212824.933\\
2.59886497162429	-350089244938.631\\
2.59896497412435	-350101277052.328\\
2.59906497662442	-350113309166.026\\
2.59916497912448	-350125341279.724\\
2.59926498162454	-350136800435.627\\
2.5993649841246	-350148832549.324\\
2.59946498662467	-350160864663.022\\
2.59956498912473	-350172896776.72\\
2.59966499162479	-350184355932.622\\
2.59976499412485	-350196388046.32\\
2.59986499662492	-350208420160.018\\
2.59996499912498	-350219879315.921\\
2.60006500162504	-350231911429.618\\
2.6001650041251	-350243370585.521\\
2.60026500662517	-350255402699.219\\
2.60036500912523	-350266861855.121\\
2.60046501162529	-350278321011.024\\
2.60056501412535	-350290353124.722\\
2.60066501662542	-350301812280.624\\
2.60076501912548	-350313844394.322\\
2.60086502162554	-350325303550.225\\
2.6009650241256	-350336762706.127\\
2.60106502662567	-350348221862.03\\
2.60116502912573	-350360253975.728\\
2.60126503162579	-350371713131.63\\
2.60136503412585	-350383172287.533\\
2.60146503662592	-350394631443.435\\
2.60156503912598	-350406090599.338\\
2.60166504162604	-350417549755.241\\
2.6017650441261	-350429008911.143\\
2.60186504662617	-350440468067.046\\
2.60196504912623	-350451927222.949\\
2.60206505162629	-350463386378.851\\
2.60216505412635	-350474845534.754\\
2.60226505662642	-350486304690.656\\
2.60236505912648	-350497190888.764\\
2.60246506162654	-350508650044.667\\
2.6025650641266	-350520109200.569\\
2.60266506662667	-350531568356.472\\
2.60276506912673	-350542454554.579\\
2.60286507162679	-350553913710.482\\
2.60296507412685	-350565372866.384\\
2.60306507662692	-350576259064.492\\
2.60316507912698	-350587718220.395\\
2.60326508162704	-350598604418.502\\
2.6033650841271	-350610063574.405\\
2.60346508662717	-350620949772.512\\
2.60356508912723	-350632408928.415\\
2.60366509162729	-350643295126.522\\
2.60376509412735	-350654754282.425\\
2.60386509662742	-350665640480.532\\
2.60396509912748	-350676526678.64\\
2.60406510162754	-350687985834.542\\
2.6041651041276	-350698872032.65\\
2.60426510662767	-350709758230.757\\
2.60436510912773	-350720644428.865\\
2.60446511162779	-350731530626.972\\
2.60456511412785	-350742416825.08\\
2.60466511662792	-350753875980.982\\
2.60476511912798	-350764762179.09\\
2.60486512162804	-350775648377.197\\
2.6049651241281	-350786534575.305\\
2.60506512662817	-350797420773.412\\
2.60516512912823	-350807734013.725\\
2.60526513162829	-350818620211.832\\
2.60536513412835	-350829506409.94\\
2.60546513662842	-350840392608.047\\
2.60556513912848	-350851278806.155\\
2.60566514162854	-350862165004.262\\
2.6057651441286	-350872478244.575\\
2.60586514662867	-350883364442.682\\
2.60596514912873	-350894250640.79\\
2.60606515162879	-350904563881.102\\
2.60616515412885	-350915450079.209\\
2.60626515662892	-350926336277.317\\
2.60636515912898	-350936649517.629\\
2.60646516162904	-350947535715.737\\
2.6065651641291	-350957848956.049\\
2.60666516662917	-350968735154.157\\
2.60676516912923	-350979048394.469\\
2.60686517162929	-350989361634.781\\
2.60696517412935	-351000247832.889\\
2.60706517662942	-351010561073.201\\
2.60716517912948	-351020874313.513\\
2.60726518162954	-351031187553.826\\
2.6073651841296	-351042073751.933\\
2.60746518662967	-351052386992.246\\
2.60756518912973	-351062700232.558\\
2.60766519162979	-351073013472.87\\
2.60776519412985	-351083326713.183\\
2.60786519662992	-351093639953.495\\
2.60796519912998	-351103953193.807\\
2.60806520163004	-351114266434.12\\
2.6081652041301	-351124579674.432\\
2.60826520663017	-351134892914.745\\
2.60836520913023	-351145206155.057\\
2.60846521163029	-351155519395.369\\
2.60856521413035	-351165832635.682\\
2.60866521663042	-351175572918.199\\
2.60876521913048	-351185886158.511\\
2.60886522163054	-351196199398.823\\
2.6089652241306	-351206512639.136\\
2.60906522663067	-351216252921.653\\
2.60916522913073	-351226566161.965\\
2.60926523163079	-351236306444.483\\
2.60936523413085	-351246619684.795\\
2.60946523663092	-351256932925.107\\
2.60956523913098	-351266673207.625\\
2.60966524163104	-351276986447.937\\
2.6097652441311	-351286726730.454\\
2.60986524663117	-351296467012.971\\
2.60996524913123	-351306780253.284\\
2.61006525163129	-351316520535.801\\
2.61016525413135	-351326260818.318\\
2.61026525663142	-351336574058.631\\
2.61036525913148	-351346314341.148\\
2.61046526163154	-351356054623.665\\
2.6105652641316	-351365794906.182\\
2.61066526663167	-351375535188.699\\
2.61076526913173	-351385275471.217\\
2.61086527163179	-351395015753.734\\
2.61096527413185	-351404756036.251\\
2.61106527663192	-351414496318.768\\
2.61116527913198	-351424236601.286\\
2.61126528163204	-351433976883.803\\
2.6113652841321	-351443717166.32\\
2.61146528663217	-351453457448.837\\
2.61156528913223	-351463197731.354\\
2.61166529163229	-351472938013.872\\
2.61176529413235	-351482105338.594\\
2.61186529663242	-351491845621.111\\
2.61196529913248	-351501585903.628\\
2.61206530163254	-351510753228.35\\
2.6121653041326	-351520493510.868\\
2.61226530663267	-351530233793.385\\
2.61236530913273	-351539401118.107\\
2.61246531163279	-351549141400.624\\
2.61256531413285	-351558308725.346\\
2.61266531663292	-351568049007.863\\
2.61276531913298	-351577216332.586\\
2.61286532163304	-351586383657.308\\
2.6129653241331	-351596123939.825\\
2.61306532663317	-351605291264.547\\
2.61316532913323	-351614458589.269\\
2.61326533163329	-351623625913.991\\
2.61336533413335	-351633366196.508\\
2.61346533663342	-351642533521.23\\
2.61356533913348	-351651700845.953\\
2.61366534163354	-351660868170.675\\
2.6137653441336	-351670035495.397\\
2.61386534663367	-351679202820.119\\
2.61396534913373	-351688370144.841\\
2.61406535163379	-351697537469.563\\
2.61416535413385	-351706704794.285\\
2.61426535663392	-351715872119.007\\
2.61436535913398	-351725039443.729\\
2.61446536163404	-351734206768.451\\
2.6145653641341	-351742801135.378\\
2.61466536663417	-351751968460.1\\
2.61476536913423	-351761135784.823\\
2.61486537163429	-351770303109.545\\
2.61496537413435	-351778897476.472\\
2.61506537663442	-351788064801.194\\
2.61516537913448	-351796659168.121\\
2.61526538163454	-351805826492.843\\
2.6153653841346	-351814420859.77\\
2.61546538663467	-351823588184.492\\
2.61556538913473	-351832182551.419\\
2.61566539163479	-351841349876.141\\
2.61576539413485	-351849944243.068\\
2.61586539663492	-351858538609.995\\
2.61596539913498	-351867705934.717\\
2.61606540163504	-351876300301.644\\
2.6161654041351	-351884894668.571\\
2.61626540663517	-351893489035.498\\
2.61636540913523	-351902656360.22\\
2.61646541163529	-351911250727.147\\
2.61656541413535	-351919845094.074\\
2.61666541663542	-351928439461.001\\
2.61676541913548	-351937033827.928\\
2.61686542163554	-351945628194.855\\
2.6169654241356	-351954222561.782\\
2.61706542663567	-351962816928.709\\
2.61716542913573	-351970838337.84\\
2.61726543163579	-351979432704.767\\
2.61736543413585	-351988027071.694\\
2.61746543663592	-351996621438.621\\
2.61756543913598	-352005215805.548\\
2.61766544163604	-352013237214.68\\
2.6177654441361	-352021831581.607\\
2.61786544663617	-352029852990.739\\
2.61796544913623	-352038447357.666\\
2.61806545163629	-352047041724.593\\
2.61816545413635	-352055063133.725\\
2.61826545663642	-352063657500.652\\
2.61836545913648	-352071678909.783\\
2.61846546163654	-352079700318.915\\
2.6185654641366	-352088294685.842\\
2.61866546663667	-352096316094.974\\
2.61876546913673	-352104337504.106\\
2.61886547163679	-352112358913.238\\
2.61896547413685	-352120953280.165\\
2.61906547663692	-352128974689.296\\
2.61916547913698	-352136996098.428\\
2.61926548163704	-352145017507.56\\
2.6193654841371	-352153038916.692\\
2.61946548663717	-352161060325.824\\
2.61956548913723	-352169081734.956\\
2.61966549163729	-352177103144.087\\
2.61976549413735	-352185124553.219\\
2.61986549663742	-352193145962.351\\
2.61996549913748	-352201167371.483\\
2.62006550163754	-352208615822.82\\
2.6201655041376	-352216637231.951\\
2.62026550663767	-352224658641.083\\
2.62036550913773	-352232680050.215\\
2.62046551163779	-352240128501.552\\
2.62056551413785	-352248149910.684\\
2.62066551663792	-352255598362.02\\
2.62076551913798	-352263619771.152\\
2.62086552163804	-352271068222.489\\
2.6209655241381	-352279089631.621\\
2.62106552663817	-352286538082.957\\
2.62116552913823	-352294559492.089\\
2.62126553163829	-352302007943.426\\
2.62136553413835	-352309456394.763\\
2.62146553663842	-352316904846.099\\
2.62156553913848	-352324926255.231\\
2.62166554163854	-352332374706.568\\
2.6217655441386	-352339823157.905\\
2.62186554663867	-352347271609.241\\
2.62196554913873	-352354720060.578\\
2.62206555163879	-352362168511.915\\
2.62216555413885	-352369616963.251\\
2.62226555663892	-352377065414.588\\
2.62236555913898	-352384513865.925\\
2.62246556163904	-352391962317.262\\
2.6225655641391	-352399410768.598\\
2.62266556663917	-352406286262.14\\
2.62276556913923	-352413734713.477\\
2.62286557163929	-352421183164.813\\
2.62296557413935	-352428631616.15\\
2.62306557663942	-352435507109.691\\
2.62316557913948	-352442955561.028\\
2.62326558163954	-352449831054.57\\
2.6233655841396	-352457279505.906\\
2.62346558663967	-352464154999.448\\
2.62356558913973	-352471603450.785\\
2.62366559163979	-352478478944.326\\
2.62376559413985	-352485354437.868\\
2.62386559663992	-352492802889.205\\
2.62396559913998	-352499678382.746\\
2.62406560164004	-352506553876.288\\
2.6241656041401	-352513429369.829\\
2.62426560664017	-352520304863.371\\
2.62436560914023	-352527180356.912\\
2.62446561164029	-352534628808.249\\
2.62456561414035	-352541504301.791\\
2.62466561664042	-352547806837.537\\
2.62476561914048	-352554682331.079\\
2.62486562164054	-352561557824.62\\
2.6249656241406	-352568433318.162\\
2.62506562664067	-352575308811.703\\
2.62516562914073	-352582184305.245\\
2.62526563164079	-352588486840.991\\
2.62536563414085	-352595362334.533\\
2.62546563664092	-352602237828.075\\
2.62556563914098	-352608540363.821\\
2.62566564164104	-352615415857.363\\
2.6257656441411	-352621718393.109\\
2.62586564664117	-352628593886.651\\
2.62596564914123	-352634896422.397\\
2.62606565164129	-352641198958.143\\
2.62616565414135	-352648074451.685\\
2.62626565664142	-352654376987.431\\
2.62636565914148	-352660679523.178\\
2.62646566164154	-352667555016.719\\
2.6265656641416	-352673857552.466\\
2.62666566664167	-352680160088.212\\
2.62676566914173	-352686462623.959\\
2.62686567164179	-352692765159.705\\
2.62696567414185	-352699067695.452\\
2.62706567664192	-352705370231.198\\
2.62716567914198	-352711672766.945\\
2.62726568164204	-352717402344.896\\
2.6273656841421	-352723704880.642\\
2.62746568664217	-352730007416.389\\
2.62756568914223	-352736309952.135\\
2.62766569164229	-352742039530.086\\
2.62776569414235	-352748342065.833\\
2.62786569664242	-352754071643.784\\
2.62796569914248	-352760374179.531\\
2.62806570164254	-352766103757.482\\
2.6281657041426	-352772406293.228\\
2.62826570664267	-352778135871.18\\
2.62836570914273	-352784438406.926\\
2.62846571164279	-352790167984.877\\
2.62856571414285	-352795897562.829\\
2.62866571664292	-352801627140.78\\
2.62876571914298	-352807356718.731\\
2.62886572164304	-352813659254.478\\
2.6289657241431	-352819388832.429\\
2.62906572664317	-352825118410.38\\
2.62916572914323	-352830847988.332\\
2.62926573164329	-352836004608.488\\
2.62936573414335	-352841734186.439\\
2.62946573664342	-352847463764.391\\
2.62956573914348	-352853193342.342\\
2.62966574164354	-352858922920.293\\
2.6297657441436	-352864079540.449\\
2.62986574664367	-352869809118.401\\
2.62996574914373	-352874965738.557\\
2.63006575164379	-352880695316.508\\
2.63016575414385	-352885851936.664\\
2.63026575664392	-352891581514.616\\
2.63036575914398	-352896738134.772\\
2.63046576164404	-352901894754.928\\
2.6305657641441	-352907624332.879\\
2.63066576664417	-352912780953.035\\
2.63076576914423	-352917937573.192\\
2.63086577164429	-352923094193.348\\
2.63096577414435	-352928250813.504\\
2.63106577664442	-352933407433.66\\
2.63116577914448	-352938564053.816\\
2.63126578164454	-352943720673.972\\
2.6313657841446	-352948877294.129\\
2.63146578664467	-352954033914.285\\
2.63156578914473	-352958617576.646\\
2.63166579164479	-352963774196.802\\
2.63176579414485	-352968930816.958\\
2.63186579664492	-352973514479.319\\
2.63196579914498	-352978671099.475\\
2.63206580164504	-352983254761.837\\
2.6321658041451	-352988411381.993\\
2.63226580664517	-352992995044.354\\
2.63236580914523	-352997578706.715\\
2.63246581164529	-353002735326.871\\
2.63256581414535	-353007318989.232\\
2.63266581664542	-353011902651.593\\
2.63276581914548	-353016486313.954\\
2.63286582164554	-353021069976.315\\
2.6329658241456	-353025653638.676\\
2.63306582664567	-353030237301.037\\
2.63316582914573	-353034820963.398\\
2.63326583164579	-353039404625.759\\
2.63336583414585	-353043415330.325\\
2.63346583664592	-353047998992.686\\
2.63356583914598	-353052582655.047\\
2.63366584164604	-353056593359.613\\
2.6337658441461	-353061177021.974\\
2.63386584664617	-353065187726.54\\
2.63396584914623	-353069771388.901\\
2.63406585164629	-353073782093.467\\
2.63416585414635	-353077792798.033\\
2.63426585664642	-353081803502.599\\
2.63436585914648	-353086387164.96\\
2.63446586164654	-353090397869.526\\
2.6345658641466	-353094408574.092\\
2.63466586664667	-353098419278.658\\
2.63476586914673	-353102429983.224\\
2.63486587164679	-353105867729.995\\
2.63496587414685	-353109878434.56\\
2.63506587664692	-353113889139.126\\
2.63516587914698	-353117899843.692\\
2.63526588164704	-353121337590.463\\
2.6353658841471	-353125348295.029\\
2.63546588664717	-353128786041.8\\
2.63556588914723	-353132223788.571\\
2.63566589164729	-353136234493.136\\
2.63576589414735	-353139672239.907\\
2.63586589664742	-353143109986.678\\
2.63596589914748	-353146547733.449\\
2.63606590164754	-353149985480.22\\
2.6361659041476	-353153423226.99\\
2.63626590664767	-353156860973.761\\
2.63636590914773	-353160298720.532\\
2.63646591164779	-353163736467.303\\
2.63656591414785	-353167174214.074\\
2.63666591664792	-353170039003.049\\
2.63676591914798	-353173476749.82\\
2.63686592164804	-353176341538.796\\
2.6369659241481	-353179779285.566\\
2.63706592664817	-353182644074.542\\
2.63716592914823	-353185508863.518\\
2.63726593164829	-353188946610.289\\
2.63736593414835	-353191811399.264\\
2.63746593664842	-353194676188.24\\
2.63756593914848	-353197540977.215\\
2.63766594164854	-353200405766.191\\
2.6377659441486	-353203270555.167\\
2.63786594664867	-353205562386.347\\
2.63796594914873	-353208427175.323\\
2.63806595164879	-353211291964.299\\
2.63816595414885	-353213583795.479\\
2.63826595664892	-353216448584.455\\
2.63836595914898	-353218740415.635\\
2.63846596164904	-353221032246.816\\
2.6385659641491	-353223324077.996\\
2.63866596664917	-353226188866.972\\
2.63876596914923	-353228480698.153\\
2.63886597164929	-353230772529.333\\
2.63896597414935	-353232491402.718\\
2.63906597664942	-353234783233.899\\
2.63916597914948	-353237075065.079\\
2.63926598164954	-353239366896.26\\
2.6393659841496	-353241085769.645\\
2.63946598664967	-353242804643.031\\
2.63956598914973	-353245096474.211\\
2.63966599164979	-353246815347.597\\
2.63976599414985	-353248534220.982\\
2.63986599664992	-353250253094.367\\
2.63996599914998	-353251971967.753\\
2.64006600165004	-353253690841.138\\
2.6401660041501	-353255409714.524\\
2.64026600665017	-353257128587.909\\
2.64036600915023	-353258274503.499\\
2.64046601165029	-353259993376.885\\
2.64056601415035	-353261139292.475\\
2.64066601665042	-353262858165.86\\
2.64076601915048	-353264004081.451\\
2.64086602165054	-353265149997.041\\
2.6409660241506	-353266295912.631\\
2.64106602665067	-353267441828.221\\
2.64116602915073	-353268587743.812\\
2.64126603165079	-353269733659.402\\
2.64136603415085	-353270306617.197\\
2.64146603665092	-353271452532.787\\
2.64156603915098	-353272025490.582\\
2.64166604165104	-353273171406.173\\
2.6417660441511	-353273744363.968\\
2.64186604665117	-353274317321.763\\
2.64196604915123	-353274890279.558\\
2.64206605165129	-353275463237.353\\
2.64216605415135	-353276036195.148\\
2.64226605665142	-353276036195.148\\
2.64236605915148	-353276609152.943\\
2.64246606165154	-353276609152.943\\
2.6425660641516	-353277182110.739\\
2.64266606665167	-353277182110.739\\
2.64276606915173	-353277182110.739\\
2.64286607165179	-353277182110.739\\
2.64296607415185	-353277182110.739\\
2.64306607665192	-353276609152.943\\
2.64316607915198	-353276609152.943\\
2.64326608165204	-353276036195.148\\
2.6433660841521	-353276036195.148\\
2.64346608665217	-353275463237.353\\
2.64356608915223	-353274890279.558\\
2.64366609165229	-353274317321.763\\
2.64376609415235	-353273744363.968\\
2.64386609665242	-353272598448.378\\
2.64396609915248	-353272025490.582\\
2.64406610165254	-353270879574.992\\
2.6441661041526	-353270306617.197\\
2.64426610665267	-353269160701.607\\
2.64436610915273	-353268014786.017\\
2.64446611165279	-353266868870.426\\
2.64456611415285	-353265149997.041\\
2.64466611665292	-353264004081.451\\
2.64476611915298	-353262285208.065\\
2.64486612165304	-353260566334.68\\
2.6449661241531	-353258847461.294\\
2.64506612665317	-353257128587.909\\
2.64516612915323	-353255409714.524\\
2.64526613165329	-353253690841.138\\
2.64536613415335	-353251399009.958\\
2.64546613665342	-353249107178.777\\
2.64556613915348	-353247388305.392\\
2.64566614165354	-353244523516.416\\
2.6457661441536	-353242231685.236\\
2.64586614665367	-353239939854.055\\
2.64596614915373	-353237075065.079\\
2.64606615165379	-353234783233.899\\
2.64616615415385	-353231918444.923\\
2.64626615665392	-353228480698.153\\
2.64636615915398	-353225615909.177\\
2.64646616165404	-353222751120.201\\
2.6465661641541	-353219313373.43\\
2.64666616665417	-353215875626.66\\
2.64676616915423	-353212437879.889\\
2.64686617165429	-353209000133.118\\
2.64696617415435	-353204989428.552\\
2.64706617665442	-353201551681.781\\
2.64716617915448	-353197540977.215\\
2.64726618165454	-353193530272.65\\
2.6473661841546	-353188946610.289\\
2.64746618665467	-353184935905.723\\
2.64756618915473	-353180352243.362\\
2.64766619165479	-353175768581\\
2.64776619415485	-353171184918.639\\
2.64786619665492	-353166028298.483\\
2.64796619915498	-353161444636.122\\
2.64806620165504	-353156288015.966\\
2.6481662041551	-353151131395.81\\
2.64826620665517	-353145401817.859\\
2.64836620915523	-353140245197.702\\
2.64846621165529	-353134515619.751\\
2.64856621415535	-353128786041.8\\
2.64866621665542	-353122483506.053\\
2.64876621915548	-353116753928.102\\
2.64886622165554	-353110451392.356\\
2.6489662241556	-353103575898.814\\
2.64906622665567	-353097273363.068\\
2.64916622915573	-353090397869.526\\
2.64926623165579	-353083522375.984\\
2.64936623415585	-353076646882.443\\
2.64946623665592	-353069198431.106\\
2.64956623915598	-353061749979.769\\
2.64966624165604	-353054301528.433\\
2.6497662441561	-353046280119.301\\
2.64986624665617	-353038258710.169\\
2.64996624915623	-353030237301.037\\
2.65006625165629	-353021642934.11\\
2.65016625415635	-353013621524.978\\
2.65026625665642	-353004454200.256\\
2.65036625915648	-352995859833.329\\
2.65046626165654	-352986692508.607\\
2.6505662641566	-352976952226.09\\
2.65066626665667	-352967784901.368\\
2.65076626915673	-352958044618.851\\
2.65086627165679	-352947731378.538\\
2.65096627415685	-352937418138.226\\
2.65106627665692	-352927104897.914\\
2.65116627915698	-352916791657.601\\
2.65126628165704	-352905332501.699\\
2.6513662841571	-352894446303.591\\
2.65146628665717	-352882987147.689\\
2.65156628915723	-352871527991.786\\
2.65166629165729	-352859495878.088\\
2.65176629415735	-352847463764.391\\
2.65186629665742	-352834858692.898\\
2.65196629915748	-352822253621.405\\
2.65206630165754	-352809075592.117\\
2.6521663041576	-352795897562.829\\
2.65226630665767	-352782146575.746\\
2.65236630915773	-352768395588.662\\
2.65246631165779	-352754071643.784\\
2.65256631415785	-352739747698.906\\
2.65266631665792	-352724850796.233\\
2.65276631915798	-352709953893.559\\
2.65286632165804	-352694484033.091\\
2.6529663241581	-352679014172.622\\
2.65306632665817	-352662398396.563\\
2.65316632915823	-352646355578.3\\
2.65326633165829	-352629166844.446\\
2.65336633415835	-352611978110.592\\
2.65346633665842	-352594789376.738\\
2.65356633915848	-352576454727.294\\
2.65366634165854	-352558120077.849\\
2.6537663441586	-352539785428.405\\
2.65386634665867	-352520304863.371\\
2.65396634915873	-352500824298.336\\
2.65406635165879	-352480770775.507\\
2.65416635415885	-352460717252.677\\
2.65426635665892	-352439517814.257\\
2.65436635915898	-352418318375.838\\
2.65446636165904	-352396545979.623\\
2.6545663641591	-352374200625.612\\
2.65466636665917	-352351282313.807\\
2.65476636915923	-352328364002.002\\
2.65486637165929	-352304299774.607\\
2.65496637415935	-352280235547.211\\
2.65506637665942	-352255025404.225\\
2.65516637915948	-352229815261.24\\
2.65526638165954	-352204032160.459\\
2.6553663841596	-352177103144.087\\
2.65546638665967	-352150174127.716\\
2.65556638915973	-352122099195.755\\
2.65566639165979	-352094024263.794\\
2.65576639415985	-352064803416.242\\
2.65586639665992	-352035009610.895\\
2.65596639915998	-352004642847.753\\
2.65606640166004	-351973130169.021\\
2.6561664041601	-351941617490.289\\
2.65626640666017	-351908958895.966\\
2.65636640916023	-351875727343.849\\
2.65646641166029	-351841349876.141\\
2.65656641416035	-351806399450.638\\
2.65666641666042	-351770303109.545\\
2.65676641916048	-351733633810.656\\
2.65686642166054	-351696391553.973\\
2.6569664241606	-351658003381.699\\
2.65706642666067	-351618469293.835\\
2.65716642916073	-351578362248.176\\
2.65726643166079	-351537109286.926\\
2.65736643416085	-351494710410.087\\
2.65746643666092	-351451165617.657\\
2.65756643916098	-351406474909.636\\
2.65766644166104	-351361211243.821\\
2.6577664441611	-351314228704.62\\
2.65786644666117	-351266673207.625\\
2.65796644916123	-351217398837.243\\
2.65806645166129	-351166978551.272\\
2.65816645416135	-351115412349.71\\
2.65826645666142	-351062127274.763\\
2.65836645916148	-351008269242.021\\
2.65846646166154	-350952119378.098\\
2.6585664641616	-350894823598.585\\
2.65866646666167	-350835808945.686\\
2.65876646916173	-350775648377.197\\
2.65886647166179	-350713768935.323\\
2.65896647416185	-350650170620.064\\
2.65906647666192	-350584280473.624\\
2.65916647916198	-350517244411.593\\
2.65926648166204	-350447916518.383\\
2.6593664841621	-350376869751.786\\
2.65946648666217	-350304104111.805\\
2.65956648916223	-350229046640.643\\
2.65966649166229	-350151697338.3\\
2.65976649416235	-350072056204.777\\
2.65986649666242	-349990123240.073\\
2.65996649916248	-349905898444.189\\
2.66006650166254	-349819381817.124\\
2.6601665041626	-349730000401.084\\
2.66026650666267	-349637754196.068\\
2.66036650916273	-349543216159.871\\
2.66046651166279	-349445813334.699\\
2.66056651416285	-349344972762.756\\
2.66066651666292	-349241267401.837\\
2.66076651916298	-349134124294.148\\
2.66086652166304	-349023543439.687\\
2.6609665241631	-348909524838.456\\
2.66106652666317	-348791495532.659\\
2.66116652916323	-348670028480.092\\
2.66126653166329	-348544550722.958\\
2.66136653416335	-348415062261.258\\
2.66146653666342	-348280417179.403\\
2.66156653916348	-348141761392.981\\
2.66166654166354	-347998521944.198\\
2.6617665441636	-347850125875.259\\
2.66186654666367	-347696573186.164\\
2.66196654916373	-347537290919.118\\
2.66206655166379	-347372852031.916\\
2.66216655416385	-347202110608.967\\
2.66226655666392	-347025066650.271\\
2.66236655916398	-346841720155.829\\
2.66246656166404	-346651498167.846\\
2.6625665641641	-346453827728.526\\
2.66266656666417	-346248708837.869\\
2.66276656916423	-346035568538.08\\
2.66286657166429	-345813833871.365\\
2.66296657416435	-345583504837.722\\
2.66306657666442	-345344008479.357\\
2.66316657916448	-345094771838.475\\
2.66326658166454	-344835221957.281\\
2.6633665841646	-344564785877.979\\
2.66346658666467	-344282890642.775\\
2.66356658916473	-343988963293.873\\
2.66366659166479	-343682430873.478\\
2.66376659416485	-343362147466\\
2.66386659666492	-343027540113.643\\
2.66396659916498	-342677462900.818\\
2.66406660166504	-342311915827.525\\
2.6641666041651	-341929180020.378\\
2.66426660666517	-341528109563.786\\
2.66436660916523	-341107558542.16\\
2.66446661166529	-340666953997.704\\
2.66456661416535	-340204004099.239\\
2.66466661666542	-339718708846.763\\
2.66476661916548	-339207630493.506\\
2.66486662166554	-338670769039.469\\
2.6649666241656	-338105259695.674\\
2.66506662666567	-337509383588.738\\
2.66516662916573	-336881421845.275\\
2.66526663166579	-336218509676.309\\
2.66536663416585	-335517782292.864\\
2.66546663666592	-334777520821.555\\
2.66556663916598	-333993714557.816\\
2.66566664166604	-333163498712.671\\
2.6657666441661	-332282862581.555\\
2.66586664666617	-331348368417.697\\
2.66596664916623	-330354859600.94\\
2.66606665166629	-329298325426.719\\
2.66616665416635	-328172463359.286\\
2.66626665666642	-326971543820.692\\
2.66636665916648	-325689837232.985\\
2.66646666166654	-324319322187.032\\
2.6665666641666	-322851977273.702\\
2.66666666666667	-321279208126.068\\
2.66676666916673	-319590701503.817\\
2.66686667166679	-317775571208.843\\
2.66696667416685	-315821785127.446\\
2.66706667666692	-313715019314.75\\
2.66716667916698	-311439230952.491\\
2.66726668166704	-308978377222.404\\
2.6673666841671	-306311831643.865\\
2.66746668666717	-303418394778.454\\
2.66756668916723	-300272856483.186\\
2.66766669166729	-296848287741.689\\
2.66776669416735	-293112029959.641\\
2.66786669666742	-289030278627.129\\
2.66796669916748	-284562926698.494\\
2.66806670166754	-279664710507.921\\
2.6681667041676	-274286355685.028\\
2.66826670666767	-268372285323.687\\
2.66836670916773	-261860047024.23\\
2.66846671166779	-254680312893.446\\
2.66856671416785	-246759171375.762\\
2.66866671666792	-238015262464.271\\
2.66876671916798	-228363215447.497\\
2.66886672166804	-217714794824.991\\
2.6689667241681	-205985775800.868\\
2.66906672666817	-193099382030.58\\
2.66916672916823	-178996598861.23\\
2.66926673166829	-163652216149.832\\
2.66936673416835	-147090298123.78\\
2.66946673666842	-129411685355.019\\
2.66956673916848	-110819777860.818\\
2.66966674166854	-91652047782.5119\\
2.6697667441686	-72405822486.2724\\
2.66986674666867	-53751978254.4187\\
2.66996674916873	-36520501748.9772\\
2.67006675166879	-21649611359.4747\\
2.67016675416885	-10091391054.0802\\
2.67026675666892	-2682095653.0987\\
2.67036675916898	-3998070.84653313\\
2.67046676166904	-2276664987.68618\\
2.6705667641691	-9313428959.85153\\
2.67066676666917	-20558585125.9866\\
2.67076676916923	-35191640734.7303\\
2.67086677166929	-52264866297.1566\\
2.67096677416935	-70834772212.0237\\
2.67106677666942	-90059798069.8433\\
2.67116677916948	-109255030122.316\\
2.67126678166954	-127909390016.186\\
2.6713667841696	-145672800538.626\\
2.67146678666967	-162331548432.055\\
2.67156678916973	-177778490588.782\\
2.67166679166979	-191982687287.87\\
2.67176679416985	-204968202756.715\\
2.67186679666992	-216790040943.65\\
2.67196679916998	-227523832277.63\\
2.67206680167004	-237254947470.132\\
2.6721668041701	-246070476106.015\\
2.67226680667017	-254056361854.549\\
2.67236680917023	-261293964722.641\\
2.67246681167029	-267858915139.25\\
2.67256681417035	-273819968039.791\\
2.67266681667042	-279240148781.729\\
2.67276681917048	-284175607228.986\\
2.67286682167054	-288677336625.329\\
2.6729668241706	-292789454720.982\\
2.67306682667067	-296552068561.607\\
2.67316682917073	-300001847446.089\\
2.67326683167079	-303169158137.573\\
2.67336683417085	-306082075568.018\\
2.67346683667092	-308766382838.206\\
2.67356683917098	-311243852344.351\\
2.67366684167104	-313533964651.489\\
2.6737668441711	-315653908493.473\\
2.67386684667117	-317620299646.362\\
2.67396684917123	-319446316139.444\\
2.67406685167129	-321144563044.212\\
2.67416685417135	-322726499516.568\\
2.67426685667142	-324201865839.03\\
2.67436685917148	-325580402294.115\\
2.67446686167154	-326868984375.364\\
2.6745668641716	-328076206449.705\\
2.67466686667167	-329207798095.088\\
2.67476686917173	-330270634805.056\\
2.67486687167179	-331268727284.173\\
2.67496687417185	-332208378068.188\\
2.67506687667192	-333093024903.87\\
2.67516687917198	-333927251453.581\\
2.67526688167204	-334714495464.09\\
2.6753668841721	-335458767639.965\\
2.67546688667217	-336162359812.386\\
2.67556688917223	-336828136770.328\\
2.67566689167229	-337458963302.767\\
2.67576689417235	-338057131240.883\\
2.67586689667242	-338624932415.858\\
2.67596689917248	-339164658658.871\\
2.67606690167254	-339677455885.513\\
2.6761669041726	-340165042969.17\\
2.67626690667267	-340629711741.021\\
2.67636690917273	-341072035158.862\\
2.67646691167279	-341493732096.078\\
2.67656691417285	-341896521426.055\\
2.67666691667292	-342280976106.588\\
2.67676691917298	-342648242053.267\\
2.67686692167304	-342998892223.887\\
2.6769669241731	-343335218449.629\\
2.67706692667317	-343656647772.697\\
2.67716692917323	-343964326108.682\\
2.67726693167329	-344259399373.175\\
2.67736693417335	-344541867566.174\\
2.67746693667342	-344813449561.066\\
2.67756693917348	-345074145357.851\\
2.67766694167354	-345323954956.528\\
2.6777669441736	-345564597230.483\\
2.67786694667367	-345795499221.92\\
2.67796694917373	-346017806846.431\\
2.67806695167379	-346231520104.015\\
2.67816695417385	-346437211952.467\\
2.67826695667392	-346635455349.582\\
2.67836695917398	-346826250295.361\\
2.67846696167404	-347010169747.598\\
2.6785669641741	-347187786664.088\\
2.67866696667417	-347359101044.832\\
2.67876696917423	-347524112889.83\\
2.67886697167429	-347683968114.672\\
2.67896697417435	-347837520803.767\\
2.67906697667442	-347986489830.501\\
2.67916697917448	-348130302237.078\\
2.67926698167454	-348269530981.295\\
2.6793669841746	-348404176063.151\\
2.67946698667467	-348534237482.646\\
2.67956698917473	-348659715239.779\\
2.67966699167479	-348781755250.142\\
2.67976699417485	-348899784555.939\\
2.67986699667492	-349014376114.965\\
2.67996699917498	-349124956969.426\\
2.68006700167504	-349232100077.115\\
2.6801670041751	-349336378395.829\\
2.68026700667517	-349437218967.772\\
2.68036700917523	-349535194750.739\\
2.68046701167529	-349630305744.731\\
2.68056701417535	-349722551949.747\\
2.68066701667542	-349811933365.787\\
2.68076701917548	-349899022950.647\\
2.68086702167554	-349983247746.532\\
2.6809670241756	-350065180711.235\\
2.68106702667567	-350145394802.554\\
2.68116702917573	-350222744104.896\\
2.68126703167579	-350297801576.058\\
2.68136703417585	-350371140173.835\\
2.68146703667592	-350442186940.431\\
2.68156703917598	-350511514833.642\\
2.68166704167604	-350579123853.468\\
2.6817670441761	-350644441042.112\\
2.68186704667617	-350708612315.167\\
2.68196704917623	-350770491757.041\\
2.68206705167629	-350831225283.325\\
2.68216705417635	-350890239936.224\\
2.68226705667642	-350947535715.737\\
2.68236705917648	-351003685579.66\\
2.68246706167654	-351058116570.197\\
2.6825670641766	-351110828687.349\\
2.68266706667667	-351162967846.706\\
2.68276706917673	-351213388132.677\\
2.68286707167679	-351262662503.059\\
2.68296707417685	-351310790957.85\\
2.68306707667692	-351357200539.255\\
2.68316707917698	-351403037162.866\\
2.68326708167704	-351447727870.886\\
2.6833670841771	-351491272663.316\\
2.68346708667717	-351533671540.156\\
2.68356708917723	-351574924501.405\\
2.68366709167729	-351615031547.064\\
2.68376709417735	-351654565634.928\\
2.68386709667742	-351692953807.202\\
2.68396709917748	-351730769021.681\\
2.68406710167754	-351767438320.569\\
2.6841671041776	-351803534661.662\\
2.68426710667767	-351838485087.165\\
2.68436710917773	-351872862554.873\\
2.68446711167779	-351906094106.991\\
2.68456711417785	-351938752701.313\\
2.68466711667792	-351970838337.84\\
2.68476711917798	-352001778058.777\\
2.68486712167804	-352032717779.714\\
2.6849671241781	-352062511585.061\\
2.68506712667817	-352091732432.613\\
2.68516712917823	-352119807364.574\\
2.68526713167829	-352147882296.536\\
2.68536713417835	-352174811312.907\\
2.68546713667842	-352201740329.278\\
2.68556713917848	-352227523430.059\\
2.68566714167854	-352253306530.84\\
2.6857671441786	-352277943716.03\\
2.68586714667867	-352302580901.221\\
2.68596714917873	-352326072170.821\\
2.68606715167879	-352349563440.422\\
2.68616715417885	-352372481752.227\\
2.68626715667892	-352394827106.237\\
2.68636715917898	-352416599502.452\\
2.68646716167904	-352437798940.872\\
2.6865671641791	-352458998379.292\\
2.68666716667917	-352479051902.121\\
2.68676716917923	-352499105424.951\\
2.68686717167929	-352519158947.781\\
2.68696717417935	-352538066555.02\\
2.68706717667942	-352556974162.259\\
2.68716717917948	-352575308811.703\\
2.68726718167954	-352593070503.352\\
2.6873671841796	-352610832195.001\\
2.68746718667967	-352628020928.855\\
2.68756718917973	-352644636704.914\\
2.68766719167979	-352661252480.973\\
2.68776719417985	-352677295299.237\\
2.68786719667992	-352693338117.5\\
2.68796719917998	-352708807977.969\\
2.68806720168004	-352723704880.642\\
2.6881672041801	-352738601783.316\\
2.68826720668017	-352752925728.194\\
2.68836720918023	-352767249673.072\\
2.68846721168029	-352781000660.155\\
2.68856721418035	-352794751647.238\\
2.68866721668042	-352807929676.526\\
2.68876721918048	-352821107705.815\\
2.68886722168054	-352833712777.307\\
2.6889672241806	-352846317848.8\\
2.68906722668067	-352858349962.498\\
2.68916722918073	-352870382076.196\\
2.68926723168079	-352881841232.098\\
2.68936723418085	-352893300388.001\\
2.68946723668092	-352904759543.904\\
2.68956723918098	-352915645742.011\\
2.68966724168104	-352926531940.119\\
2.6897672441811	-352936845180.431\\
2.68986724668117	-352947158420.743\\
2.68996724918123	-352956898703.26\\
2.69006725168129	-352966638985.778\\
2.69016725418135	-352976379268.295\\
2.69026725668142	-352985546593.017\\
2.69036725918148	-352994713917.739\\
2.69046726168154	-353003881242.461\\
2.6905672641816	-353012475609.388\\
2.69066726668167	-353021069976.315\\
2.69076726918173	-353029664343.242\\
2.69086727168179	-353037685752.374\\
2.69096727418185	-353045707161.506\\
2.69106727668192	-353053728570.638\\
2.69116727918198	-353061177021.974\\
2.69126728168204	-353068625473.311\\
2.6913672841821	-353076073924.648\\
2.69146728668217	-353082949418.189\\
2.69156728918223	-353089824911.731\\
2.69166729168229	-353096700405.272\\
2.69176729418235	-353103575898.814\\
2.69186729668242	-353109878434.56\\
2.69196729918248	-353116180970.307\\
2.69206730168254	-353121910548.258\\
2.6921673041826	-353128213084.005\\
2.69226730668267	-353133942661.956\\
2.69236730918273	-353139672239.907\\
2.69246731168279	-353145401817.859\\
2.69256731418285	-353150558438.015\\
2.69266731668292	-353155715058.171\\
2.69276731918298	-353160871678.327\\
2.69286732168304	-353166028298.483\\
2.6929673241831	-353170611960.844\\
2.69306732668317	-353175195623.205\\
2.69316732918323	-353179779285.566\\
2.69326733168329	-353184362947.927\\
2.69336733418335	-353188946610.289\\
2.69346733668342	-353192957314.854\\
2.69356733918348	-353196968019.42\\
2.69366734168354	-353200978723.986\\
2.6937673441836	-353204989428.552\\
2.69386734668367	-353208427175.323\\
2.69396734918373	-353212437879.889\\
2.69406735168379	-353215875626.66\\
2.69416735418385	-353219313373.43\\
2.69426735668392	-353222178162.406\\
2.69436735918398	-353225615909.177\\
2.69446736168404	-353228480698.153\\
2.6945673641841	-353231345487.128\\
2.69466736668417	-353234210276.104\\
2.69476736918423	-353237075065.079\\
2.69486737168429	-353239939854.055\\
2.69496737418435	-353242231685.236\\
2.69506737668442	-353244523516.416\\
2.69516737918448	-353246815347.597\\
2.69526738168454	-353249107178.777\\
2.6953673841846	-353251399009.958\\
2.69546738668467	-353253117883.343\\
2.69556738918473	-353255409714.524\\
2.69566739168479	-353257128587.909\\
2.69576739418485	-353258847461.294\\
2.69586739668492	-353260566334.68\\
2.69596739918498	-353262285208.065\\
2.69606740168504	-353264004081.451\\
2.6961674041851	-353265149997.041\\
2.69626740668517	-353266295912.631\\
2.69636740918523	-353268014786.017\\
2.69646741168529	-353269160701.607\\
2.69656741418535	-353269733659.402\\
2.69666741668542	-353270879574.992\\
2.69676741918548	-353272025490.582\\
2.69686742168554	-353272598448.378\\
2.6969674241856	-353273744363.968\\
2.69706742668567	-353274317321.763\\
2.69716742918573	-353274890279.558\\
2.69726743168579	-353275463237.353\\
2.69736743418585	-353276036195.148\\
2.69746743668592	-353276036195.148\\
2.69756743918598	-353276609152.943\\
2.69766744168604	-353276609152.943\\
2.6977674441861	-353277182110.739\\
2.69786744668617	-353277182110.739\\
2.69796744918623	-353277182110.739\\
2.69806745168629	-353277182110.739\\
2.69816745418635	-353277182110.739\\
2.69826745668642	-353276609152.943\\
2.69836745918648	-353276609152.943\\
2.69846746168654	-353276036195.148\\
2.6985674641866	-353276036195.148\\
2.69866746668667	-353275463237.353\\
2.69876746918673	-353274890279.558\\
2.69886747168679	-353274317321.763\\
2.69896747418685	-353273744363.968\\
2.69906747668692	-353273171406.173\\
2.69916747918698	-353272025490.582\\
2.69926748168704	-353271452532.787\\
2.6993674841871	-353270879574.992\\
2.69946748668717	-353269733659.402\\
2.69956748918723	-353268587743.812\\
2.69966749168729	-353267441828.221\\
2.69976749418735	-353266295912.631\\
2.69986749668742	-353265149997.041\\
2.69996749918748	-353264004081.451\\
2.70006750168754	-353262858165.86\\
2.7001675041876	-353261712250.27\\
2.70026750668767	-353259993376.885\\
2.70036750918773	-353258847461.294\\
2.70046751168779	-353257128587.909\\
2.70056751418785	-353255409714.524\\
2.70066751668792	-353254263798.933\\
2.70076751918798	-353252544925.548\\
2.70086752168804	-353250826052.163\\
2.7009675241881	-353249107178.777\\
2.70106752668817	-353246815347.597\\
2.70116752918823	-353245096474.211\\
2.70126753168829	-353243377600.826\\
2.70136753418835	-353241085769.645\\
2.70146753668842	-353239366896.26\\
2.70156753918848	-353237075065.079\\
2.70166754168854	-353235356191.694\\
2.7017675441886	-353233064360.514\\
2.70186754668867	-353230772529.333\\
2.70196754918873	-353228480698.153\\
2.70206755168879	-353226188866.972\\
2.70216755418885	-353223897035.792\\
2.70226755668892	-353221605204.611\\
2.70236755918898	-353218740415.635\\
2.70246756168904	-353216448584.455\\
2.7025675641891	-353214156753.274\\
2.70266756668917	-353211291964.299\\
2.70276756918923	-353208427175.323\\
2.70286757168929	-353206135344.142\\
2.70296757418935	-353203270555.167\\
2.70306757668942	-353200405766.191\\
2.70316757918948	-353197540977.215\\
2.70326758168954	-353194676188.24\\
2.7033675841896	-353191811399.264\\
2.70346758668967	-353188946610.289\\
2.70356758918973	-353186081821.313\\
2.70366759168979	-353183217032.337\\
2.70376759418985	-353179779285.566\\
2.70386759668992	-353176914496.591\\
2.70396759918998	-353173476749.82\\
2.70406760169004	-353170611960.844\\
2.7041676041901	-353167174214.074\\
2.70426760669017	-353163736467.303\\
2.70436760919023	-353160871678.327\\
2.70446761169029	-353157433931.556\\
2.70456761419035	-353153996184.786\\
2.70466761669042	-353150558438.015\\
2.70476761919048	-353147120691.244\\
2.70486762169054	-353143682944.473\\
2.7049676241906	-353140245197.702\\
2.70506762669067	-353136234493.136\\
2.70516762919073	-353132796746.366\\
2.70526763169079	-353129358999.595\\
2.70536763419085	-353125348295.029\\
2.70546763669092	-353121910548.258\\
2.70556763919098	-353117899843.692\\
2.70566764169104	-353113889139.126\\
2.7057676441911	-353110451392.356\\
2.70586764669117	-353106440687.79\\
2.70596764919123	-353102429983.224\\
2.70606765169129	-353098419278.658\\
2.70616765419135	-353094408574.092\\
2.70626765669142	-353090397869.526\\
2.70636765919148	-353086387164.96\\
2.70646766169154	-353082376460.394\\
2.7065676641916	-353078365755.828\\
2.70666766669167	-353074355051.262\\
2.70676766919173	-353069771388.901\\
2.70686767169179	-353065760684.335\\
2.70696767419185	-353061177021.974\\
2.70706767669192	-353057166317.408\\
2.70716767919198	-353052582655.047\\
2.70726768169204	-353048571950.481\\
2.7073676841921	-353043988288.12\\
2.70746768669217	-353039404625.759\\
2.70756768919223	-353035393921.193\\
2.70766769169229	-353030810258.832\\
2.70776769419235	-353026226596.471\\
2.70786769669242	-353021642934.11\\
2.70796769919248	-353017059271.749\\
2.70806770169254	-353012475609.388\\
2.7081677041926	-353007891947.027\\
2.70826770669267	-353002735326.871\\
2.70836770919273	-352998151664.51\\
2.70846771169279	-352993568002.149\\
2.70856771419286	-352988411381.993\\
2.70866771669292	-352983827719.632\\
2.70876771919298	-352979244057.271\\
2.70886772169304	-352974087437.114\\
2.7089677241931	-352969503774.753\\
2.70906772669317	-352964347154.597\\
2.70916772919323	-352959190534.441\\
2.70926773169329	-352954033914.285\\
2.70936773419335	-352949450251.924\\
2.70946773669342	-352944293631.768\\
2.70956773919348	-352939137011.611\\
2.70966774169354	-352933980391.455\\
2.7097677441936	-352928823771.299\\
2.70986774669367	-352923667151.143\\
2.70996774919373	-352918510530.987\\
2.71006775169379	-352913353910.831\\
2.71016775419385	-352907624332.879\\
2.71026775669392	-352902467712.723\\
2.71036775919398	-352897311092.567\\
2.71046776169404	-352892154472.411\\
2.71056776419411	-352886424894.459\\
2.71066776669417	-352881268274.303\\
2.71076776919423	-352875538696.352\\
2.71086777169429	-352870382076.196\\
2.71096777419435	-352864652498.244\\
2.71106777669442	-352858922920.293\\
2.71116777919448	-352853766300.137\\
2.71126778169454	-352848036722.186\\
2.7113677841946	-352842307144.234\\
2.71146778669467	-352836577566.283\\
2.71156778919473	-352830847988.332\\
2.71166779169479	-352825118410.38\\
2.71176779419485	-352819388832.429\\
2.71186779669492	-352813659254.478\\
2.71196779919498	-352807929676.526\\
2.71206780169504	-352802200098.575\\
2.7121678041951	-352796470520.624\\
2.71226780669517	-352790740942.673\\
2.71236780919523	-352784438406.926\\
2.71246781169529	-352778708828.975\\
2.71256781419536	-352772979251.023\\
2.71266781669542	-352766676715.277\\
2.71276781919548	-352760947137.326\\
2.71286782169554	-352754644601.579\\
2.71296782419561	-352748915023.628\\
2.71306782669567	-352742612487.882\\
2.71316782919573	-352736882909.93\\
2.71326783169579	-352730580374.184\\
2.71336783419585	-352724277838.437\\
2.71346783669592	-352717975302.691\\
2.71356783919598	-352712245724.74\\
2.71366784169604	-352705943188.993\\
2.7137678441961	-352699640653.247\\
2.71386784669617	-352693338117.5\\
2.71396784919623	-352687035581.754\\
2.71406785169629	-352680733046.007\\
2.71416785419635	-352674430510.261\\
2.71426785669642	-352667555016.719\\
2.71436785919648	-352661252480.973\\
2.71446786169654	-352654949945.227\\
2.71456786419661	-352648647409.48\\
2.71466786669667	-352641771915.939\\
2.71476786919673	-352635469380.192\\
2.71486787169679	-352629166844.446\\
2.71496787419686	-352622291350.904\\
2.71506787669692	-352615988815.158\\
2.71516787919698	-352609113321.616\\
2.71526788169704	-352602810785.87\\
2.7153678841971	-352595935292.328\\
2.71546788669717	-352589059798.786\\
2.71556788919723	-352582757263.04\\
2.71566789169729	-352575881769.499\\
2.71576789419735	-352569006275.957\\
2.71586789669742	-352562130782.415\\
2.71596789919748	-352555255288.874\\
2.71606790169754	-352548379795.332\\
2.7161679041976	-352542077259.586\\
2.71626790669767	-352535201766.044\\
2.71636790919773	-352527753314.708\\
2.71646791169779	-352520877821.166\\
2.71656791419786	-352514002327.624\\
2.71666791669792	-352507126834.083\\
2.71676791919798	-352500251340.541\\
2.71686792169804	-352493375847\\
2.71696792419811	-352485927395.663\\
2.71706792669817	-352479051902.121\\
2.71716792919823	-352472176408.58\\
2.71726793169829	-352464727957.243\\
2.71736793419835	-352457852463.702\\
2.71746793669842	-352450404012.365\\
2.71756793919848	-352443528518.823\\
2.71766794169854	-352436080067.487\\
2.7177679441986	-352429204573.945\\
2.71786794669867	-352421756122.608\\
2.71796794919873	-352414307671.272\\
2.71806795169879	-352406859219.935\\
2.71816795419885	-352399983726.393\\
2.71826795669892	-352392535275.057\\
2.71836795919898	-352385086823.72\\
2.71846796169904	-352377638372.383\\
2.71856796419911	-352370189921.047\\
2.71866796669917	-352362741469.71\\
2.71876796919923	-352355293018.373\\
2.71886797169929	-352347844567.036\\
2.71896797419936	-352340396115.7\\
2.71906797669942	-352332947664.363\\
2.71916797919948	-352325499213.026\\
2.71926798169954	-352318050761.69\\
2.71936798419961	-352310029352.558\\
2.71946798669967	-352302580901.221\\
2.71956798919973	-352295132449.884\\
2.71966799169979	-352287111040.753\\
2.71976799419985	-352279662589.416\\
2.71986799669992	-352272214138.079\\
2.71996799919998	-352264192728.947\\
2.72006800170004	-352256744277.611\\
2.7201680042001	-352248722868.479\\
2.72026800670017	-352240701459.347\\
2.72036800920023	-352233253008.01\\
2.72046801170029	-352225231598.878\\
2.72056801420036	-352217210189.747\\
2.72066801670042	-352209761738.41\\
2.72076801920048	-352201740329.278\\
2.72086802170054	-352193718920.146\\
2.72096802420061	-352185697511.014\\
2.72106802670067	-352177676101.883\\
2.72116802920073	-352169654692.751\\
2.72126803170079	-352161633283.619\\
2.72136803420086	-352153611874.487\\
2.72146803670092	-352145590465.355\\
2.72156803920098	-352137569056.223\\
2.72166804170104	-352129547647.092\\
2.7217680442011	-352121526237.96\\
2.72186804670117	-352113504828.828\\
2.72196804920123	-352104910461.901\\
2.72206805170129	-352096889052.769\\
2.72216805420135	-352088867643.637\\
2.72226805670142	-352080846234.505\\
2.72236805920148	-352072251867.578\\
2.72246806170154	-352064230458.447\\
2.72256806420161	-352055636091.52\\
2.72266806670167	-352047614682.388\\
2.72276806920173	-352039020315.461\\
2.72286807170179	-352030998906.329\\
2.72296807420186	-352022404539.402\\
2.72306807670192	-352013810172.475\\
2.72316807920198	-352005788763.343\\
2.72326808170204	-351997194396.416\\
2.72336808420211	-351988600029.489\\
2.72346808670217	-351980578620.358\\
2.72356808920223	-351971984253.431\\
2.72366809170229	-351963389886.504\\
2.72376809420235	-351954795519.577\\
2.72386809670242	-351946201152.65\\
2.72396809920248	-351937606785.723\\
2.72406810170254	-351929012418.796\\
2.7241681042026	-351920418051.869\\
2.72426810670267	-351911823684.942\\
2.72436810920273	-351903229318.015\\
2.72446811170279	-351894634951.088\\
2.72456811420286	-351885467626.366\\
2.72466811670292	-351876873259.439\\
2.72476811920298	-351868278892.512\\
2.72486812170304	-351859684525.585\\
2.72496812420311	-351850517200.863\\
2.72506812670317	-351841922833.936\\
2.72516812920323	-351833328467.009\\
2.72526813170329	-351824161142.287\\
2.72536813420336	-351815566775.36\\
2.72546813670342	-351806399450.638\\
2.72556813920348	-351797805083.711\\
2.72566814170354	-351788637758.989\\
2.72576814420361	-351780043392.062\\
2.72586814670367	-351770876067.34\\
2.72596814920373	-351761708742.618\\
2.72606815170379	-351752541417.896\\
2.72616815420385	-351743947050.969\\
2.72626815670392	-351734779726.246\\
2.72636815920398	-351725612401.524\\
2.72646816170404	-351716445076.802\\
2.72656816420411	-351707277752.08\\
2.72666816670417	-351698110427.358\\
2.72676816920423	-351689516060.431\\
2.72686817170429	-351680348735.709\\
2.72696817420436	-351671181410.987\\
2.72706817670442	-351661441128.47\\
2.72716817920448	-351652273803.748\\
2.72726818170454	-351643106479.026\\
2.72736818420461	-351633939154.303\\
2.72746818670467	-351624771829.581\\
2.72756818920473	-351615604504.859\\
2.72766819170479	-351605864222.342\\
2.72776819420486	-351596696897.62\\
2.72786819670492	-351587529572.898\\
2.72796819920498	-351577789290.381\\
2.72806820170504	-351568621965.659\\
2.7281682042051	-351558881683.141\\
2.72826820670517	-351549714358.419\\
2.72836820920523	-351540547033.697\\
2.72846821170529	-351530806751.18\\
2.72856821420536	-351521066468.663\\
2.72866821670542	-351511899143.941\\
2.72876821920548	-351502158861.423\\
2.72886822170554	-351492418578.906\\
2.72896822420561	-351483251254.184\\
2.72906822670567	-351473510971.667\\
2.72916822920573	-351463770689.15\\
2.72926823170579	-351454030406.632\\
2.72936823420586	-351444863081.91\\
2.72946823670592	-351435122799.393\\
2.72956823920598	-351425382516.876\\
2.72966824170604	-351415642234.359\\
2.72976824420611	-351405901951.841\\
2.72986824670617	-351396161669.324\\
2.72996824920623	-351386421386.807\\
2.73006825170629	-351376681104.29\\
2.73016825420636	-351366940821.772\\
2.73026825670642	-351356627581.46\\
2.73036825920648	-351346887298.943\\
2.73046826170654	-351337147016.426\\
2.73056826420661	-351327406733.908\\
2.73066826670667	-351317093493.596\\
2.73076826920673	-351307353211.079\\
2.73086827170679	-351297612928.562\\
2.73096827420686	-351287299688.249\\
2.73106827670692	-351277559405.732\\
2.73116827920698	-351267819123.215\\
2.73126828170704	-351257505882.902\\
2.73136828420711	-351247765600.385\\
2.73146828670717	-351237452360.073\\
2.73156828920723	-351227139119.761\\
2.73166829170729	-351217398837.243\\
2.73176829420736	-351207085596.931\\
2.73186829670742	-351196772356.619\\
2.73196829920748	-351187032074.101\\
2.73206830170754	-351176718833.789\\
2.73216830420761	-351166405593.477\\
2.73226830670767	-351156092353.164\\
2.73236830920773	-351146352070.647\\
2.73246831170779	-351136038830.335\\
2.73256831420786	-351125725590.022\\
2.73266831670792	-351115412349.71\\
2.73276831920798	-351105099109.398\\
2.73286832170804	-351094785869.085\\
2.73296832420811	-351084472628.773\\
2.73306832670817	-351074159388.461\\
2.73316832920823	-351063846148.148\\
2.73326833170829	-351052959950.041\\
2.73336833420836	-351042646709.728\\
2.73346833670842	-351032333469.416\\
2.73356833920848	-351022020229.104\\
2.73366834170854	-351011134030.996\\
2.73376834420861	-351000820790.684\\
2.73386834670867	-350990507550.372\\
2.73396834920873	-350979621352.264\\
2.73406835170879	-350969308111.952\\
2.73416835420886	-350958994871.639\\
2.73426835670892	-350948108673.532\\
2.73436835920898	-350937795433.219\\
2.73446836170904	-350926909235.112\\
2.73456836420911	-350916595994.8\\
2.73466836670917	-350905709796.692\\
2.73476836920923	-350894823598.585\\
2.73486837170929	-350884510358.272\\
2.73496837420936	-350873624160.165\\
2.73506837670942	-350862737962.057\\
2.73516837920948	-350851851763.95\\
2.73526838170954	-350841538523.638\\
2.73536838420961	-350830652325.53\\
2.73546838670967	-350819766127.423\\
2.73556838920973	-350808879929.315\\
2.73566839170979	-350797993731.208\\
2.73576839420986	-350787107533.1\\
2.73586839670992	-350776221334.993\\
2.73596839920998	-350765335136.885\\
2.73606840171004	-350754448938.778\\
2.73616840421011	-350743562740.67\\
2.73626840671017	-350732676542.563\\
2.73636840921023	-350721790344.455\\
2.73646841171029	-350710904146.348\\
2.73656841421036	-350699444990.445\\
2.73666841671042	-350688558792.338\\
2.73676841921048	-350677672594.23\\
2.73686842171054	-350666786396.123\\
2.73696842421061	-350655327240.22\\
2.73706842671067	-350644441042.112\\
2.73716842921073	-350632981886.21\\
2.73726843171079	-350622095688.102\\
2.73736843421086	-350611209489.995\\
2.73746843671092	-350599750334.092\\
2.73756843921098	-350588864135.985\\
2.73766844171104	-350577404980.082\\
2.73776844421111	-350565945824.18\\
2.73786844671117	-350555059626.072\\
2.73796844921123	-350543600470.169\\
2.73806845171129	-350532141314.267\\
2.73816845421136	-350521255116.159\\
2.73826845671142	-350509795960.257\\
2.73836845921148	-350498336804.354\\
2.73846846171154	-350486877648.452\\
2.73856846421161	-350475418492.549\\
2.73866846671167	-350464532294.441\\
2.73876846921173	-350453073138.539\\
2.73886847171179	-350441613982.636\\
2.73896847421186	-350430154826.734\\
2.73906847671192	-350418695670.831\\
2.73916847921198	-350407236514.928\\
2.73926848171204	-350395777359.026\\
2.73936848421211	-350384318203.123\\
2.73946848671217	-350372286089.425\\
2.73956848921223	-350360826933.523\\
2.73966849171229	-350349367777.62\\
2.73976849421236	-350337908621.718\\
2.73986849671242	-350326449465.815\\
2.73996849921248	-350314417352.117\\
2.74006850171254	-350302958196.215\\
2.74016850421261	-350291499040.312\\
2.74026850671267	-350279466926.614\\
2.74036850921273	-350268007770.712\\
2.74046851171279	-350255975657.014\\
2.74056851421286	-350244516501.111\\
2.74066851671292	-350232484387.413\\
2.74076851921298	-350221025231.511\\
2.74086852171304	-350208993117.813\\
2.74096852421311	-350197533961.91\\
2.74106852671317	-350185501848.213\\
2.74116852921323	-350173469734.515\\
2.74126853171329	-350162010578.612\\
2.74136853421336	-350149978464.915\\
2.74146853671342	-350137946351.217\\
2.74156853921348	-350125914237.519\\
2.74166854171354	-350114455081.617\\
2.74176854421361	-350102422967.919\\
2.74186854671367	-350090390854.221\\
2.74196854921373	-350078358740.523\\
2.74206855171379	-350066326626.826\\
2.74216855421386	-350054294513.128\\
2.74226855671392	-350042262399.43\\
2.74236855921398	-350030230285.732\\
2.74246856171404	-350018198172.035\\
2.74256856421411	-350006166058.337\\
2.74266856671417	-349994133944.639\\
2.74276856921423	-349981528873.146\\
2.74286857171429	-349969496759.448\\
2.74296857421436	-349957464645.751\\
2.74306857671442	-349945432532.053\\
2.74316857921448	-349932827460.56\\
2.74326858171454	-349920795346.862\\
2.74336858421461	-349908763233.165\\
2.74346858671467	-349896158161.672\\
2.74356858921473	-349884126047.974\\
2.74366859171479	-349872093934.276\\
2.74376859421486	-349859488862.783\\
2.74386859671492	-349847456749.086\\
2.74396859921498	-349834851677.593\\
2.74406860171504	-349822819563.895\\
2.74416860421511	-349810214492.402\\
2.74426860671517	-349797609420.909\\
2.74436860921523	-349785577307.211\\
2.74446861171529	-349772972235.719\\
2.74456861421536	-349760367164.226\\
2.74466861671542	-349748335050.528\\
2.74476861921548	-349735729979.035\\
2.74486862171554	-349723124907.542\\
2.74496862421561	-349710519836.049\\
2.74506862671567	-349697914764.556\\
2.74516862921573	-349685309693.064\\
2.74526863171579	-349672704621.571\\
2.74536863421586	-349660099550.078\\
2.74546863671592	-349647494478.585\\
2.74556863921598	-349634889407.092\\
2.74566864171604	-349622284335.599\\
2.74576864421611	-349609679264.106\\
2.74586864671617	-349597074192.613\\
2.74596864921623	-349584469121.12\\
2.74606865171629	-349571864049.628\\
2.74616865421636	-349559258978.135\\
2.74626865671642	-349546080948.847\\
2.74636865921648	-349533475877.354\\
2.74646866171654	-349520870805.861\\
2.74656866421661	-349507692776.573\\
2.74666866671667	-349495087705.08\\
2.74676866921673	-349482482633.587\\
2.74686867171679	-349469304604.299\\
2.74696867421686	-349456699532.806\\
2.74706867671692	-349443521503.518\\
2.74716867921698	-349430916432.025\\
2.74726868171704	-349417738402.737\\
2.74736868421711	-349405133331.245\\
2.74746868671717	-349391955301.957\\
2.74756868921723	-349379350230.464\\
2.74766869171729	-349366172201.176\\
2.74776869421736	-349352994171.888\\
2.74786869671742	-349339816142.6\\
2.74796869921748	-349327211071.107\\
2.74806870171754	-349314033041.819\\
2.74816870421761	-349300855012.531\\
2.74826870671767	-349287676983.243\\
2.74836870921773	-349274498953.955\\
2.74846871171779	-349261320924.667\\
2.74856871421786	-349248715853.174\\
2.74866871671792	-349235537823.886\\
2.74876871921798	-349222359794.598\\
2.74886872171804	-349209181765.31\\
2.74896872421811	-349195430778.227\\
2.74906872671817	-349182252748.939\\
2.74916872921823	-349169074719.651\\
2.74926873171829	-349155896690.363\\
2.74936873421836	-349142718661.075\\
2.74946873671842	-349129540631.787\\
2.74956873921848	-349116362602.499\\
2.74966874171854	-349102611615.415\\
2.74976874421861	-349089433586.127\\
2.74986874671867	-349076255556.839\\
2.74996874921873	-349062504569.756\\
2.75006875171879	-349049326540.468\\
2.75016875421886	-349035575553.385\\
2.75026875671892	-349022397524.097\\
2.75036875921898	-349009219494.809\\
2.75046876171904	-348995468507.726\\
2.75056876421911	-348981717520.643\\
2.75066876671917	-348968539491.355\\
2.75076876921923	-348954788504.272\\
2.75086877171929	-348941610474.984\\
2.75096877421936	-348927859487.901\\
2.75106877671942	-348914108500.817\\
2.75116877921948	-348900930471.529\\
2.75126878171954	-348887179484.446\\
2.75136878421961	-348873428497.363\\
2.75146878671967	-348859677510.28\\
2.75156878921973	-348845926523.197\\
2.75166879171979	-348832748493.909\\
2.75176879421986	-348818997506.826\\
2.75186879671992	-348805246519.743\\
2.75196879921998	-348791495532.659\\
2.75206880172004	-348777744545.576\\
2.75216880422011	-348763993558.493\\
2.75226880672017	-348750242571.41\\
2.75236880922023	-348736491584.327\\
2.75246881172029	-348722740597.244\\
2.75256881422036	-348708416652.365\\
2.75266881672042	-348694665665.282\\
2.75276881922048	-348680914678.199\\
2.75286882172054	-348667163691.116\\
2.75296882422061	-348653412704.033\\
2.75306882672067	-348639088759.155\\
2.75316882922073	-348625337772.071\\
2.75326883172079	-348611586784.988\\
2.75336883422086	-348597262840.11\\
2.75346883672092	-348583511853.027\\
2.75356883922098	-348569187908.149\\
2.75366884172104	-348555436921.066\\
2.75376884422111	-348541112976.187\\
2.75386884672117	-348527361989.104\\
2.75396884922123	-348513038044.226\\
2.75406885172129	-348499287057.143\\
2.75416885422136	-348484963112.264\\
2.75426885672142	-348471212125.181\\
2.75436885922148	-348456888180.303\\
2.75446886172154	-348442564235.425\\
2.75456886422161	-348428240290.547\\
2.75466886672167	-348414489303.463\\
2.75476886922173	-348400165358.585\\
2.75486887172179	-348385841413.707\\
2.75496887422186	-348371517468.829\\
2.75506887672192	-348357193523.95\\
2.75516887922198	-348342869579.072\\
2.75526888172204	-348329118591.989\\
2.75536888422211	-348314794647.111\\
2.75546888672217	-348300470702.232\\
2.75556888922223	-348286146757.354\\
2.75566889172229	-348271249854.681\\
2.75576889422236	-348256925909.802\\
2.75586889672242	-348242601964.924\\
2.75596889922248	-348228278020.046\\
2.75606890172254	-348213954075.168\\
2.75616890422261	-348199630130.289\\
2.75626890672267	-348185306185.411\\
2.75636890922273	-348170409282.738\\
2.75646891172279	-348156085337.859\\
2.75656891422286	-348141761392.981\\
2.75666891672292	-348126864490.308\\
2.75676891922298	-348112540545.429\\
2.75686892172304	-348098216600.551\\
2.75696892422311	-348083319697.878\\
2.75706892672317	-348068995752.999\\
2.75716892922323	-348054098850.326\\
2.75726893172329	-348039774905.448\\
2.75736893422336	-348024878002.774\\
2.75746893672342	-348009981100.101\\
2.75756893922348	-347995657155.223\\
2.75766894172354	-347980760252.549\\
2.75776894422361	-347966436307.671\\
2.75786894672367	-347951539404.998\\
2.75796894922373	-347936642502.324\\
2.75806895172379	-347921745599.651\\
2.75816895422386	-347907421654.773\\
2.75826895672392	-347892524752.099\\
2.75836895922398	-347877627849.426\\
2.75846896172404	-347862730946.752\\
2.75856896422411	-347847834044.079\\
2.75866896672417	-347832937141.406\\
2.75876896922423	-347818040238.732\\
2.75886897172429	-347803143336.059\\
2.75896897422436	-347788246433.385\\
2.75906897672442	-347773349530.712\\
2.75916897922448	-347758452628.039\\
2.75926898172454	-347743555725.365\\
2.75936898422461	-347728658822.692\\
2.75946898672467	-347713761920.018\\
2.75956898922473	-347698292059.55\\
2.75966899172479	-347683395156.876\\
2.75976899422486	-347668498254.203\\
2.75986899672492	-347653601351.53\\
2.75996899922498	-347638131491.061\\
2.76006900172504	-347623234588.388\\
2.76016900422511	-347608337685.714\\
2.76026900672517	-347592867825.246\\
2.76036900922523	-347577970922.572\\
2.76046901172529	-347562501062.104\\
2.76056901422536	-347547604159.43\\
2.76066901672542	-347532134298.962\\
2.76076901922548	-347517237396.288\\
2.76086902172554	-347501767535.82\\
2.76096902422561	-347486297675.351\\
2.76106902672567	-347471400772.678\\
2.76116902922573	-347455930912.209\\
2.76126903172579	-347440461051.741\\
2.76136903422586	-347425564149.068\\
2.76146903672592	-347410094288.599\\
2.76156903922598	-347394624428.13\\
2.76166904172604	-347379154567.662\\
2.76176904422611	-347364257664.989\\
2.76186904672617	-347348787804.52\\
2.76196904922623	-347333317944.051\\
2.76206905172629	-347317848083.583\\
2.76216905422636	-347302378223.114\\
2.76226905672642	-347286908362.646\\
2.76236905922648	-347271438502.177\\
2.76246906172654	-347255968641.709\\
2.76256906422661	-347240498781.24\\
2.76266906672667	-347225028920.772\\
2.76276906922673	-347209559060.303\\
2.76286907172679	-347193516242.04\\
2.76296907422686	-347178046381.571\\
2.76306907672692	-347162576521.102\\
2.76316907922698	-347147106660.634\\
2.76326908172704	-347131063842.37\\
2.76336908422711	-347115593981.902\\
2.76346908672717	-347100124121.433\\
2.76356908922723	-347084081303.17\\
2.76366909172729	-347068611442.701\\
2.76376909422736	-347053141582.232\\
2.76386909672742	-347037098763.969\\
2.76396909922748	-347021628903.5\\
2.76406910172754	-347005586085.237\\
2.76416910422761	-346990116224.768\\
2.76426910672767	-346974073406.504\\
2.76436910922773	-346958603546.036\\
2.76446911172779	-346942560727.772\\
2.76456911422786	-346926517909.509\\
2.76466911672792	-346911048049.04\\
2.76476911922798	-346895005230.776\\
2.76486912172804	-346878962412.513\\
2.76496912422811	-346862919594.249\\
2.76506912672817	-346847449733.781\\
2.76516912922823	-346831406915.517\\
2.76526913172829	-346815364097.253\\
2.76536913422836	-346799321278.99\\
2.76546913672842	-346783278460.726\\
2.76556913922848	-346767235642.462\\
2.76566914172854	-346751192824.199\\
2.76576914422861	-346735150005.935\\
2.76586914672867	-346719107187.671\\
2.76596914922873	-346703064369.408\\
2.76606915172879	-346687021551.144\\
2.76616915422886	-346670978732.88\\
2.76626915672892	-346654935914.617\\
2.76636915922898	-346638893096.353\\
2.76646916172904	-346622277320.294\\
2.76656916422911	-346606234502.03\\
2.76666916672917	-346590191683.767\\
2.76676916922923	-346574148865.503\\
2.76686917172929	-346557533089.444\\
2.76696917422936	-346541490271.181\\
2.76706917672942	-346525447452.917\\
2.76716917922948	-346508831676.858\\
2.76726918172954	-346492788858.595\\
2.76736918422961	-346476173082.536\\
2.76746918672967	-346460130264.272\\
2.76756918922973	-346443514488.213\\
2.76766919172979	-346427471669.95\\
2.76776919422986	-346410855893.891\\
2.76786919672992	-346394813075.627\\
2.76796919922998	-346378197299.568\\
2.76806920173004	-346361581523.51\\
2.76816920423011	-346345538705.246\\
2.76826920673017	-346328922929.187\\
2.76836920923023	-346312307153.128\\
2.76846921173029	-346295691377.07\\
2.76856921423036	-346279648558.806\\
2.76866921673042	-346263032782.747\\
2.76876921923048	-346246417006.688\\
2.76886922173054	-346229801230.63\\
2.76896922423061	-346213185454.571\\
2.76906922673067	-346196569678.512\\
2.76916922923073	-346179953902.453\\
2.76926923173079	-346163338126.394\\
2.76936923423086	-346146722350.336\\
2.76946923673092	-346130106574.277\\
2.76956923923098	-346113490798.218\\
2.76966924173104	-346096875022.159\\
2.76976924423111	-346080259246.1\\
2.76986924673117	-346063643470.042\\
2.76996924923123	-346046454736.188\\
2.77006925173129	-346029838960.129\\
2.77016925423136	-346013223184.07\\
2.77026925673142	-345996607408.011\\
2.77036925923148	-345979418674.157\\
2.77046926173154	-345962802898.099\\
2.77056926423161	-345946187122.04\\
2.77066926673167	-345928998388.186\\
2.77076926923173	-345912382612.127\\
2.77086927173179	-345895193878.273\\
2.77096927423186	-345878578102.214\\
2.77106927673192	-345861389368.36\\
2.77116927923198	-345844773592.302\\
2.77126928173204	-345827584858.448\\
2.77136928423211	-345810969082.389\\
2.77146928673217	-345793780348.535\\
2.77156928923223	-345776591614.681\\
2.77166929173229	-345759975838.622\\
2.77176929423236	-345742787104.768\\
2.77186929673242	-345725598370.914\\
2.77196929923248	-345708982594.856\\
2.77206930173254	-345691793861.002\\
2.77216930423261	-345674605127.148\\
2.77226930673267	-345657416393.294\\
2.77236930923273	-345640227659.44\\
2.77246931173279	-345623038925.586\\
2.77256931423286	-345605850191.732\\
2.77266931673292	-345588661457.878\\
2.77276931923298	-345571472724.024\\
2.77286932173304	-345554283990.17\\
2.77296932423311	-345537095256.316\\
2.77306932673317	-345519906522.462\\
2.77316932923323	-345502717788.609\\
2.77326933173329	-345485529054.755\\
2.77336933423336	-345468340320.901\\
2.77346933673342	-345451151587.047\\
2.77356933923348	-345433962853.193\\
2.77366934173354	-345416201161.544\\
2.77376934423361	-345399012427.69\\
2.77386934673367	-345381823693.836\\
2.77396934923373	-345364062002.187\\
2.77406935173379	-345346873268.333\\
2.77416935423386	-345329684534.479\\
2.77426935673392	-345311922842.83\\
2.77436935923398	-345294734108.976\\
2.77446936173404	-345276972417.327\\
2.77456936423411	-345259783683.473\\
2.77466936673417	-345242021991.824\\
2.77476936923423	-345224833257.97\\
2.77486937173429	-345207071566.321\\
2.77496937423436	-345189882832.467\\
2.77506937673442	-345172121140.818\\
2.77516937923448	-345154359449.169\\
2.77526938173454	-345137170715.315\\
2.77536938423461	-345119409023.666\\
2.77546938673467	-345101647332.017\\
2.77556938923473	-345083885640.368\\
2.77566939173479	-345066696906.514\\
2.77576939423486	-345048935214.865\\
2.77586939673492	-345031173523.216\\
2.77596939923498	-345013411831.567\\
2.77606940173504	-344995650139.918\\
2.77616940423511	-344977888448.269\\
2.77626940673517	-344960126756.62\\
2.77636940923523	-344942365064.971\\
2.77646941173529	-344924603373.321\\
2.77656941423536	-344906841681.672\\
2.77666941673542	-344889079990.023\\
2.77676941923548	-344871318298.374\\
2.77686942173554	-344853556606.725\\
2.77696942423561	-344835221957.281\\
2.77706942673567	-344817460265.632\\
2.77716942923573	-344799698573.983\\
2.77726943173579	-344781936882.334\\
2.77736943423586	-344764175190.685\\
2.77746943673592	-344745840541.241\\
2.77756943923598	-344728078849.592\\
2.77766944173604	-344709744200.147\\
2.77776944423611	-344691982508.498\\
2.77786944673617	-344674220816.849\\
2.77796944923623	-344655886167.405\\
2.77806945173629	-344638124475.756\\
2.77816945423636	-344619789826.312\\
2.77826945673642	-344602028134.663\\
2.77836945923648	-344583693485.219\\
2.77846946173654	-344565931793.57\\
2.77856946423661	-344547597144.125\\
2.77866946673667	-344529262494.681\\
2.77876946923673	-344511500803.032\\
2.77886947173679	-344493166153.588\\
2.77896947423686	-344474831504.144\\
2.77906947673692	-344456496854.7\\
2.77916947923698	-344438735163.051\\
2.77926948173704	-344420400513.606\\
2.77936948423711	-344402065864.162\\
2.77946948673717	-344383731214.718\\
2.77956948923723	-344365396565.274\\
2.77966949173729	-344347061915.83\\
2.77976949423736	-344328727266.385\\
2.77986949673742	-344310392616.941\\
2.77996949923748	-344292057967.497\\
2.78006950173754	-344273723318.053\\
2.78016950423761	-344255388668.609\\
2.78026950673767	-344237054019.164\\
2.78036950923773	-344218719369.72\\
2.78046951173779	-344200384720.276\\
2.78056951423786	-344182050070.832\\
2.78066951673792	-344163715421.388\\
2.78076951923798	-344144807814.148\\
2.78086952173804	-344126473164.704\\
2.78096952423811	-344108138515.26\\
2.78106952673817	-344089230908.021\\
2.78116952923823	-344070896258.577\\
2.78126953173829	-344052561609.132\\
2.78136953423836	-344033654001.893\\
2.78146953673842	-344015319352.449\\
2.78156953923848	-343996984703.005\\
2.78166954173854	-343978077095.765\\
2.78176954423861	-343959742446.321\\
2.78186954673867	-343940834839.082\\
2.78196954923873	-343921927231.843\\
2.78206955173879	-343903592582.398\\
2.78216955423886	-343884684975.159\\
2.78226955673892	-343866350325.715\\
2.78236955923898	-343847442718.476\\
2.78246956173904	-343828535111.236\\
2.78256956423911	-343810200461.792\\
2.78266956673917	-343791292854.553\\
2.78276956923923	-343772385247.313\\
2.78286957173929	-343753477640.074\\
2.78296957423936	-343734570032.835\\
2.78306957673942	-343715662425.595\\
2.78316957923948	-343697327776.151\\
2.78326958173954	-343678420168.912\\
2.78336958423961	-343659512561.673\\
2.78346958673967	-343640604954.433\\
2.78356958923973	-343621697347.194\\
2.78366959173979	-343602789739.955\\
2.78376959423986	-343583882132.715\\
2.78386959673992	-343564974525.476\\
2.78396959923998	-343545493960.442\\
2.78406960174004	-343526586353.202\\
2.78416960424011	-343507678745.963\\
2.78426960674017	-343488771138.724\\
2.78436960924023	-343469863531.484\\
2.78446961174029	-343450382966.45\\
2.78456961424036	-343431475359.211\\
2.78466961674042	-343412567751.971\\
2.78476961924048	-343393660144.732\\
2.78486962174054	-343374179579.698\\
2.78496962424061	-343355271972.458\\
2.78506962674067	-343335791407.424\\
2.78516962924073	-343316883800.184\\
2.78526963174079	-343297976192.945\\
2.78536963424086	-343278495627.911\\
2.78546963674092	-343259015062.876\\
2.78556963924098	-343240107455.637\\
2.78566964174104	-343220626890.602\\
2.78576964424111	-343201719283.363\\
2.78586964674117	-343182238718.329\\
2.78596964924123	-343162758153.294\\
2.78606965174129	-343143850546.055\\
2.78616965424136	-343124369981.02\\
2.78626965674142	-343104889415.986\\
2.78636965924148	-343085981808.747\\
2.78646966174154	-343066501243.712\\
2.78656966424161	-343047020678.678\\
2.78666966674167	-343027540113.643\\
2.78676966924173	-343008059548.609\\
2.78686967174179	-342988578983.574\\
2.78696967424186	-342969098418.54\\
2.78706967674192	-342949617853.506\\
2.78716967924198	-342930137288.471\\
2.78726968174204	-342910656723.437\\
2.78736968424211	-342891176158.402\\
2.78746968674217	-342871695593.368\\
2.78756968924223	-342852215028.333\\
2.78766969174229	-342832734463.299\\
2.78776969424236	-342813253898.264\\
2.78786969674242	-342793773333.23\\
2.78796969924248	-342774292768.195\\
2.78806970174254	-342754239245.366\\
2.78816970424261	-342734758680.331\\
2.78826970674267	-342715278115.297\\
2.78836970924273	-342695224592.467\\
2.78846971174279	-342675744027.433\\
2.78856971424286	-342656263462.399\\
2.78866971674292	-342636209939.569\\
2.78876971924298	-342616729374.535\\
2.78886972174304	-342596675851.705\\
2.78896972424311	-342577195286.671\\
2.78906972674317	-342557141763.841\\
2.78916972924323	-342537661198.807\\
2.78926973174329	-342517607675.977\\
2.78936973424336	-342498127110.942\\
2.78946973674342	-342478073588.113\\
2.78956973924348	-342458593023.078\\
2.78966974174354	-342438539500.249\\
2.78976974424361	-342418485977.419\\
2.78986974674367	-342398432454.59\\
2.78996974924373	-342378951889.555\\
2.79006975174379	-342358898366.726\\
2.79016975424386	-342338844843.896\\
2.79026975674392	-342318791321.067\\
2.79036975924398	-342298737798.237\\
2.79046976174404	-342279257233.203\\
2.79056976424411	-342259203710.373\\
2.79066976674417	-342239150187.543\\
2.79076976924423	-342219096664.714\\
2.79086977174429	-342199043141.884\\
2.79096977424436	-342178989619.055\\
2.79106977674442	-342158936096.225\\
2.79116977924448	-342138882573.395\\
2.79126978174454	-342118256092.771\\
2.79136978424461	-342098202569.941\\
2.79146978674467	-342078149047.112\\
2.79156978924473	-342058095524.282\\
2.79166979174479	-342038042001.452\\
2.79176979424486	-342017988478.623\\
2.79186979674492	-341997361997.998\\
2.79196979924498	-341977308475.169\\
2.79206980174504	-341957254952.339\\
2.79216980424511	-341936628471.714\\
2.79226980674517	-341916574948.885\\
2.79236980924523	-341896521426.055\\
2.79246981174529	-341875894945.43\\
2.79256981424536	-341855841422.601\\
2.79266981674542	-341835214941.976\\
2.79276981924548	-341815161419.147\\
2.79286982174554	-341794534938.522\\
2.79296982424561	-341774481415.692\\
2.79306982674567	-341753854935.068\\
2.79316982924573	-341733228454.443\\
2.79326983174579	-341713174931.613\\
2.79336983424586	-341692548450.989\\
2.79346983674592	-341671921970.364\\
2.79356983924598	-341651868447.534\\
2.79366984174604	-341631241966.91\\
2.79376984424611	-341610615486.285\\
2.79386984674617	-341589989005.66\\
2.79396984924623	-341569935482.831\\
2.79406985174629	-341549309002.206\\
2.79416985424636	-341528682521.581\\
2.79426985674642	-341508056040.956\\
2.79436985924648	-341487429560.332\\
2.79446986174654	-341466803079.707\\
2.79456986424661	-341446176599.082\\
2.79466986674667	-341425550118.458\\
2.79476986924673	-341404923637.833\\
2.79486987174679	-341384297157.208\\
2.79496987424686	-341363670676.583\\
2.79506987674692	-341343044195.959\\
2.79516987924698	-341321844757.539\\
2.79526988174704	-341301218276.914\\
2.79536988424711	-341280591796.289\\
2.79546988674717	-341259965315.665\\
2.79556988924723	-341239338835.04\\
2.79566989174729	-341218139396.62\\
2.79576989424736	-341197512915.995\\
2.79586989674742	-341176886435.371\\
2.79596989924748	-341155686996.951\\
2.79606990174754	-341135060516.326\\
2.79616990424761	-341113861077.906\\
2.79626990674767	-341093234597.282\\
2.79636990924773	-341072608116.657\\
2.79646991174779	-341051408678.237\\
2.79656991424786	-341030782197.612\\
2.79666991674792	-341009582759.193\\
2.79676991924798	-340988383320.773\\
2.79686992174804	-340967756840.148\\
2.79696992424811	-340946557401.728\\
2.79706992674817	-340925930921.104\\
2.79716992924823	-340904731482.684\\
2.79726993174829	-340883532044.264\\
2.79736993424836	-340862332605.844\\
2.79746993674842	-340841706125.219\\
2.79756993924848	-340820506686.799\\
2.79766994174854	-340799307248.38\\
2.79776994424861	-340778107809.96\\
2.79786994674867	-340756908371.54\\
2.79796994924873	-340735708933.12\\
2.79806995174879	-340714509494.7\\
2.79816995424886	-340693883014.076\\
2.79826995674892	-340672683575.656\\
2.79836995924898	-340651484137.236\\
2.79846996174904	-340630284698.816\\
2.79856996424911	-340608512302.601\\
2.79866996674917	-340587312864.181\\
2.79876996924923	-340566113425.761\\
2.79886997174929	-340544913987.341\\
2.79896997424936	-340523714548.922\\
2.79906997674942	-340502515110.502\\
2.79916997924948	-340481315672.082\\
2.79926998174954	-340459543275.867\\
2.79936998424961	-340438343837.447\\
2.79946998674967	-340417144399.027\\
2.79956998924973	-340395372002.812\\
2.79966999174979	-340374172564.393\\
2.79976999424986	-340352973125.973\\
2.79986999674992	-340331200729.758\\
2.79996999924998	-340310001291.338\\
2.80007000175004	-340288228895.123\\
};
\addplot [color=mycolor3,solid,forget plot]
  table[row sep=crcr]{%
2.80007000175004	-340288228895.123\\
2.80017000425011	-340267029456.703\\
2.80027000675017	-340245257060.488\\
2.80037000925023	-340224057622.068\\
2.80047001175029	-340202285225.853\\
2.80057001425036	-340181085787.433\\
2.80067001675042	-340159313391.218\\
2.80077001925048	-340137540995.003\\
2.80087002175054	-340116341556.584\\
2.80097002425061	-340094569160.369\\
2.80107002675067	-340072796764.154\\
2.80117002925073	-340051597325.734\\
2.80127003175079	-340029824929.519\\
2.80137003425086	-340008052533.304\\
2.80147003675092	-339986280137.089\\
2.80157003925098	-339964507740.874\\
2.80167004175104	-339943308302.454\\
2.80177004425111	-339921535906.239\\
2.80187004675117	-339899763510.024\\
2.80197004925123	-339877991113.809\\
2.80207005175129	-339856218717.594\\
2.80217005425136	-339834446321.379\\
2.80227005675142	-339812673925.164\\
2.80237005925148	-339790901528.949\\
2.80247006175154	-339769129132.734\\
2.80257006425161	-339746783778.724\\
2.80267006675167	-339725011382.509\\
2.80277006925173	-339703238986.294\\
2.80287007175179	-339681466590.079\\
2.80297007425186	-339659694193.864\\
2.80307007675192	-339637348839.854\\
2.80317007925198	-339615576443.639\\
2.80327008175204	-339593804047.424\\
2.80337008425211	-339572031651.209\\
2.80347008675217	-339549686297.199\\
2.80357008925223	-339527913900.984\\
2.80367009175229	-339505568546.974\\
2.80377009425236	-339483796150.759\\
2.80387009675242	-339462023754.544\\
2.80397009925248	-339439678400.534\\
2.80407010175254	-339417906004.319\\
2.80417010425261	-339395560650.309\\
2.80427010675267	-339373215296.299\\
2.80437010925273	-339351442900.084\\
2.80447011175279	-339329097546.074\\
2.80457011425286	-339307325149.859\\
2.80467011675292	-339284979795.849\\
2.80477011925298	-339262634441.839\\
2.80487012175304	-339240862045.624\\
2.80497012425311	-339218516691.614\\
2.80507012675317	-339196171337.604\\
2.80517012925323	-339173825983.593\\
2.80527013175329	-339151480629.583\\
2.80537013425336	-339129708233.368\\
2.80547013675342	-339107362879.358\\
2.80557013925348	-339085017525.348\\
2.80567014175354	-339062672171.338\\
2.80577014425361	-339040326817.328\\
2.80587014675367	-339017981463.318\\
2.80597014925373	-338995636109.308\\
2.80607015175379	-338973290755.298\\
2.80617015425386	-338950945401.288\\
2.80627015675392	-338928600047.277\\
2.80637015925398	-338906254693.267\\
2.80647016175404	-338883336381.462\\
2.80657016425411	-338860991027.452\\
2.80667016675417	-338838645673.442\\
2.80677016925423	-338816300319.432\\
2.80687017175429	-338793954965.422\\
2.80697017425436	-338771036653.616\\
2.80707017675442	-338748691299.606\\
2.80717017925448	-338726345945.596\\
2.80727018175454	-338703427633.791\\
2.80737018425461	-338681082279.781\\
2.80747018675467	-338658736925.771\\
2.80757018925473	-338635818613.966\\
2.80767019175479	-338613473259.956\\
2.80777019425486	-338590554948.15\\
2.80787019675492	-338568209594.14\\
2.80797019925498	-338545291282.335\\
2.80807020175504	-338522945928.325\\
2.80817020425511	-338500027616.52\\
2.80827020675517	-338477682262.509\\
2.80837020925523	-338454763950.704\\
2.80847021175529	-338431845638.899\\
2.80857021425536	-338409500284.889\\
2.80867021675542	-338386581973.084\\
2.80877021925548	-338363663661.278\\
2.80887022175554	-338340745349.473\\
2.80897022425561	-338318399995.463\\
2.80907022675567	-338295481683.658\\
2.80917022925573	-338272563371.853\\
2.80927023175579	-338249645060.047\\
2.80937023425586	-338226726748.242\\
2.80947023675592	-338203808436.437\\
2.80957023925598	-338180890124.632\\
2.80967024175604	-338157971812.826\\
2.80977024425611	-338135053501.021\\
2.80987024675617	-338112135189.216\\
2.80997024925623	-338089216877.411\\
2.81007025175629	-338066298565.606\\
2.81017025425636	-338043380253.8\\
2.81027025675642	-338020461941.995\\
2.81037025925648	-337997543630.19\\
2.81047026175654	-337974625318.385\\
2.81057026425661	-337951707006.579\\
2.81067026675667	-337928215736.979\\
2.81077026925673	-337905297425.174\\
2.81087027175679	-337882379113.369\\
2.81097027425686	-337859460801.563\\
2.81107027675692	-337835969531.963\\
2.81117027925698	-337813051220.158\\
2.81127028175704	-337789559950.557\\
2.81137028425711	-337766641638.752\\
2.81147028675717	-337743723326.947\\
2.81157028925723	-337720232057.347\\
2.81167029175729	-337697313745.541\\
2.81177029425736	-337673822475.941\\
2.81187029675742	-337650904164.136\\
2.81197029925748	-337627412894.535\\
2.81207030175754	-337604494582.73\\
2.81217030425761	-337581003313.13\\
2.81227030675767	-337557512043.529\\
2.81237030925773	-337534593731.724\\
2.81247031175779	-337511102462.124\\
2.81257031425786	-337487611192.523\\
2.81267031675792	-337464119922.923\\
2.81277031925798	-337441201611.118\\
2.81287032175804	-337417710341.517\\
2.81297032425811	-337394219071.917\\
2.81307032675817	-337370727802.317\\
2.81317032925823	-337347236532.716\\
2.81327033175829	-337324318220.911\\
2.81337033425836	-337300826951.311\\
2.81347033675842	-337277335681.71\\
2.81357033925848	-337253844412.11\\
2.81367034175854	-337230353142.51\\
2.81377034425861	-337206861872.909\\
2.81387034675867	-337183370603.309\\
2.81397034925873	-337159879333.709\\
2.81407035175879	-337135815106.313\\
2.81417035425886	-337112323836.713\\
2.81427035675892	-337088832567.112\\
2.81437035925898	-337065341297.512\\
2.81447036175904	-337041850027.912\\
2.81457036425911	-337018358758.311\\
2.81467036675917	-336994294530.916\\
2.81477036925923	-336970803261.315\\
2.81487037175929	-336947311991.715\\
2.81497037425936	-336923247764.32\\
2.81507037675942	-336899756494.719\\
2.81517037925948	-336876265225.119\\
2.81527038175954	-336852200997.723\\
2.81537038425961	-336828709728.123\\
2.81547038675967	-336804645500.727\\
2.81557038925973	-336781154231.127\\
2.81567039175979	-336757090003.732\\
2.81577039425986	-336733598734.131\\
2.81587039675992	-336709534506.736\\
2.81597039925998	-336686043237.135\\
2.81607040176004	-336661979009.74\\
2.81617040426011	-336638487740.14\\
2.81627040676017	-336614423512.744\\
2.81637040926023	-336590359285.349\\
2.81647041176029	-336566868015.748\\
2.81657041426036	-336542803788.353\\
2.81667041676042	-336518739560.957\\
2.81677041926048	-336494675333.562\\
2.81687042176054	-336470611106.166\\
2.81697042426061	-336447119836.566\\
2.81707042676067	-336423055609.17\\
2.81717042926073	-336398991381.775\\
2.81727043176079	-336374927154.379\\
2.81737043426086	-336350862926.984\\
2.81747043676092	-336326798699.588\\
2.81757043926098	-336302734472.193\\
2.81767044176104	-336278670244.797\\
2.81777044426111	-336254606017.402\\
2.81787044676117	-336230541790.006\\
2.81797044926123	-336206477562.611\\
2.81807045176129	-336182413335.215\\
2.81817045426136	-336158349107.82\\
2.81827045676142	-336133711922.629\\
2.81837045926148	-336109647695.234\\
2.81847046176154	-336085583467.838\\
2.81857046426161	-336061519240.443\\
2.81867046676167	-336037455013.047\\
2.81877046926173	-336012817827.857\\
2.81887047176179	-335988753600.461\\
2.81897047426186	-335964689373.066\\
2.81907047676192	-335940052187.875\\
2.81917047926198	-335915987960.48\\
2.81927048176204	-335891350775.289\\
2.81937048426211	-335867286547.893\\
2.81947048676217	-335843222320.498\\
2.81957048926223	-335818585135.307\\
2.81967049176229	-335794520907.912\\
2.81977049426236	-335769883722.721\\
2.81987049676242	-335745246537.531\\
2.81997049926248	-335721182310.135\\
2.82007050176254	-335696545124.944\\
2.82017050426261	-335672480897.549\\
2.82027050676267	-335647843712.358\\
2.82037050926273	-335623206527.168\\
2.82047051176279	-335599142299.772\\
2.82057051426286	-335574505114.582\\
2.82067051676292	-335549867929.391\\
2.82077051926298	-335525230744.2\\
2.82087052176304	-335500593559.01\\
2.82097052426311	-335476529331.614\\
2.82107052676317	-335451892146.424\\
2.82117052926323	-335427254961.233\\
2.82127053176329	-335402617776.042\\
2.82137053426336	-335377980590.852\\
2.82147053676342	-335353343405.661\\
2.82157053926348	-335328706220.47\\
2.82167054176354	-335304069035.28\\
2.82177054426361	-335279431850.089\\
2.82187054676367	-335254794664.899\\
2.82197054926373	-335230157479.708\\
2.82207055176379	-335205520294.517\\
2.82217055426386	-335180883109.327\\
2.82227055676392	-335155672966.341\\
2.82237055926398	-335131035781.15\\
2.82247056176404	-335106398595.96\\
2.82257056426411	-335081761410.769\\
2.82267056676417	-335056551267.783\\
2.82277056926423	-335031914082.593\\
2.82287057176429	-335007276897.402\\
2.82297057426436	-334982639712.211\\
2.82307057676442	-334957429569.226\\
2.82317057926448	-334932792384.035\\
2.82327058176454	-334907582241.049\\
2.82337058426461	-334882945055.859\\
2.82347058676467	-334857734912.873\\
2.82357058926473	-334833097727.682\\
2.82367059176479	-334807887584.697\\
2.82377059426486	-334783250399.506\\
2.82387059676492	-334758040256.52\\
2.82397059926498	-334733403071.33\\
2.82407060176504	-334708192928.344\\
2.82417060426511	-334682982785.358\\
2.82427060676517	-334658345600.167\\
2.82437060926523	-334633135457.182\\
2.82447061176529	-334607925314.196\\
2.82457061426536	-334583288129.005\\
2.82467061676542	-334558077986.02\\
2.82477061926548	-334532867843.034\\
2.82487062176554	-334507657700.048\\
2.82497062426561	-334482447557.062\\
2.82507062676567	-334457810371.872\\
2.82517062926573	-334432600228.886\\
2.82527063176579	-334407390085.9\\
2.82537063426586	-334382179942.914\\
2.82547063676592	-334356969799.929\\
2.82557063926598	-334331759656.943\\
2.82567064176604	-334306549513.957\\
2.82577064426611	-334281339370.971\\
2.82587064676617	-334256129227.986\\
2.82597064926623	-334230919085\\
2.82607065176629	-334205708942.014\\
2.82617065426636	-334180498799.028\\
2.82627065676642	-334154715698.247\\
2.82637065926648	-334129505555.262\\
2.82647066176654	-334104295412.276\\
2.82657066426661	-334079085269.29\\
2.82667066676667	-334053302168.509\\
2.82677066926673	-334028092025.523\\
2.82687067176679	-334002881882.538\\
2.82697067426686	-333977671739.552\\
2.82707067676692	-333951888638.771\\
2.82717067926698	-333926678495.785\\
2.82727068176704	-333900895395.004\\
2.82737068426711	-333875685252.019\\
2.82747068676717	-333850475109.033\\
2.82757068926723	-333824692008.252\\
2.82767069176729	-333799481865.266\\
2.82777069426736	-333773698764.485\\
2.82787069676742	-333747915663.705\\
2.82797069926748	-333722705520.719\\
2.82807070176754	-333696922419.938\\
2.82817070426761	-333671712276.952\\
2.82827070676767	-333645929176.171\\
2.82837070926773	-333620146075.39\\
2.82847071176779	-333594935932.405\\
2.82857071426786	-333569152831.624\\
2.82867071676792	-333543369730.843\\
2.82877071926798	-333517586630.062\\
2.82887072176804	-333492376487.076\\
2.82897072426811	-333466593386.295\\
2.82907072676817	-333440810285.514\\
2.82917072926823	-333415027184.734\\
2.82927073176829	-333389244083.953\\
2.82937073426836	-333363460983.172\\
2.82947073676842	-333337677882.391\\
2.82957073926848	-333311894781.61\\
2.82967074176854	-333286111680.829\\
2.82977074426861	-333260328580.048\\
2.82987074676867	-333234545479.267\\
2.82997074926873	-333208762378.486\\
2.83007075176879	-333182979277.706\\
2.83017075426886	-333157196176.925\\
2.83027075676892	-333131413076.144\\
2.83037075926898	-333105629975.363\\
2.83047076176904	-333079273916.787\\
2.83057076426911	-333053490816.006\\
2.83067076676917	-333027707715.225\\
2.83077076926923	-333001924614.444\\
2.83087077176929	-332975568555.868\\
2.83097077426936	-332949785455.087\\
2.83107077676942	-332924002354.306\\
2.83117077926948	-332897646295.73\\
2.83127078176954	-332871863194.95\\
2.83137078426961	-332846080094.169\\
2.83147078676967	-332819724035.593\\
2.83157078926973	-332793940934.812\\
2.83167079176979	-332767584876.236\\
2.83177079426986	-332741801775.455\\
2.83187079676992	-332715445716.879\\
2.83197079926998	-332689662616.098\\
2.83207080177004	-332663306557.522\\
2.83217080427011	-332637523456.741\\
2.83227080677017	-332611167398.165\\
2.83237080927023	-332584811339.589\\
2.83247081177029	-332559028238.808\\
2.83257081427036	-332532672180.232\\
2.83267081677042	-332506316121.656\\
2.83277081927048	-332479960063.08\\
2.83287082177054	-332454176962.299\\
2.83297082427061	-332427820903.723\\
2.83307082677067	-332401464845.147\\
2.83317082927073	-332375108786.571\\
2.83327083177079	-332348752727.995\\
2.83337083427086	-332322396669.419\\
2.83347083677092	-332296040610.843\\
2.83357083927098	-332270257510.062\\
2.83367084177104	-332243901451.486\\
2.83377084427111	-332217545392.91\\
2.83387084677117	-332191189334.334\\
2.83397084927123	-332164833275.758\\
2.83407085177129	-332137904259.387\\
2.83417085427136	-332111548200.811\\
2.83427085677142	-332085192142.235\\
2.83437085927148	-332058836083.659\\
2.83447086177154	-332032480025.083\\
2.83457086427161	-332006123966.507\\
2.83467086677167	-331979767907.931\\
2.83477086927173	-331952838891.56\\
2.83487087177179	-331926482832.984\\
2.83497087427186	-331900126774.408\\
2.83507087677192	-331873770715.832\\
2.83517087927198	-331846841699.46\\
2.83527088177204	-331820485640.884\\
2.83537088427211	-331793556624.513\\
2.83547088677217	-331767200565.937\\
2.83557088927223	-331740844507.361\\
2.83567089177229	-331713915490.99\\
2.83577089427236	-331687559432.414\\
2.83587089677242	-331660630416.043\\
2.83597089927248	-331634274357.467\\
2.83607090177254	-331607345341.096\\
2.83617090427261	-331580989282.52\\
2.83627090677267	-331554060266.149\\
2.83637090927273	-331527131249.777\\
2.83647091177279	-331500775191.201\\
2.83657091427286	-331473846174.83\\
2.83667091677292	-331446917158.459\\
2.83677091927298	-331420561099.883\\
2.83687092177304	-331393632083.512\\
2.83697092427311	-331366703067.141\\
2.83707092677317	-331339774050.77\\
2.83717092927323	-331313417992.194\\
2.83727093177329	-331286488975.823\\
2.83737093427336	-331259559959.451\\
2.83747093677342	-331232630943.08\\
2.83757093927348	-331205701926.709\\
2.83767094177354	-331178772910.338\\
2.83777094427361	-331151843893.967\\
2.83787094677367	-331124914877.596\\
2.83797094927373	-331097985861.224\\
2.83807095177379	-331071056844.853\\
2.83817095427386	-331044127828.482\\
2.83827095677392	-331017198812.111\\
2.83837095927398	-330990269795.74\\
2.83847096177404	-330963340779.369\\
2.83857096427411	-330936411762.998\\
2.83867096677417	-330909482746.626\\
2.83877096927423	-330881980772.46\\
2.83887097177429	-330855051756.089\\
2.83897097427436	-330828122739.718\\
2.83907097677442	-330801193723.347\\
2.83917097927448	-330773691749.18\\
2.83927098177454	-330746762732.809\\
2.83937098427461	-330719833716.438\\
2.83947098677467	-330692331742.272\\
2.83957098927473	-330665402725.901\\
2.83967099177479	-330638473709.53\\
2.83977099427486	-330610971735.363\\
2.83987099677492	-330584042718.992\\
2.83997099927498	-330556540744.826\\
2.84007100177504	-330529611728.455\\
2.84017100427511	-330502109754.288\\
2.84027100677517	-330475180737.917\\
2.84037100927523	-330447678763.751\\
2.84047101177529	-330420749747.38\\
2.84057101427536	-330393247773.214\\
2.84067101677542	-330365745799.047\\
2.84077101927548	-330338816782.676\\
2.84087102177554	-330311314808.51\\
2.84097102427561	-330283812834.344\\
2.84107102677567	-330256883817.972\\
2.84117102927573	-330229381843.806\\
2.84127103177579	-330201879869.64\\
2.84137103427586	-330174377895.474\\
2.84147103677592	-330146875921.307\\
2.84157103927598	-330119946904.936\\
2.84167104177604	-330092444930.77\\
2.84177104427611	-330064942956.604\\
2.84187104677617	-330037440982.437\\
2.84197104927623	-330009939008.271\\
2.84207105177629	-329982437034.105\\
2.84217105427636	-329954935059.938\\
2.84227105677642	-329927433085.772\\
2.84237105927648	-329899931111.606\\
2.84247106177654	-329872429137.44\\
2.84257106427661	-329844927163.273\\
2.84267106677667	-329817425189.107\\
2.84277106927673	-329789923214.941\\
2.84287107177679	-329761848282.979\\
2.84297107427686	-329734346308.813\\
2.84307107677692	-329706844334.647\\
2.84317107927698	-329679342360.481\\
2.84327108177704	-329651267428.519\\
2.84337108427711	-329623765454.353\\
2.84347108677717	-329596263480.187\\
2.84357108927723	-329568761506.02\\
2.84367109177729	-329540686574.059\\
2.84377109427736	-329513184599.893\\
2.84387109677742	-329485109667.931\\
2.84397109927748	-329457607693.765\\
2.84407110177754	-329430105719.599\\
2.84417110427761	-329402030787.637\\
2.84427110677767	-329374528813.471\\
2.84437110927773	-329346453881.51\\
2.84447111177779	-329318951907.343\\
2.84457111427786	-329290876975.382\\
2.84467111677792	-329263375001.216\\
2.84477111927798	-329235300069.254\\
2.84487112177804	-329207225137.293\\
2.84497112427811	-329179723163.126\\
2.84507112677817	-329151648231.165\\
2.84517112927823	-329123573299.204\\
2.84527113177829	-329096071325.037\\
2.84537113427836	-329067996393.076\\
2.84547113677842	-329039921461.115\\
2.84557113927848	-329011846529.153\\
2.84567114177854	-328983771597.192\\
2.84577114427861	-328956269623.025\\
2.84587114677867	-328928194691.064\\
2.84597114927873	-328900119759.103\\
2.84607115177879	-328872044827.141\\
2.84617115427886	-328843969895.18\\
2.84627115677892	-328815894963.218\\
2.84637115927898	-328787820031.257\\
2.84647116177904	-328759745099.296\\
2.84657116427911	-328731670167.334\\
2.84667116677917	-328703595235.373\\
2.84677116927923	-328675520303.411\\
2.84687117177929	-328647445371.45\\
2.84697117427936	-328619370439.489\\
2.84707117677942	-328591295507.527\\
2.84717117927948	-328563220575.566\\
2.84727118177954	-328534572685.809\\
2.84737118427961	-328506497753.848\\
2.84747118677967	-328478422821.886\\
2.84757118927973	-328450347889.925\\
2.84767119177979	-328421700000.168\\
2.84777119427986	-328393625068.207\\
2.84787119677992	-328365550136.246\\
2.84797119927998	-328336902246.489\\
2.84807120178004	-328308827314.528\\
2.84817120428011	-328280752382.566\\
2.84827120678017	-328252104492.81\\
2.84837120928023	-328224029560.848\\
2.84847121178029	-328195381671.092\\
2.84857121428036	-328167306739.13\\
2.84867121678042	-328138658849.374\\
2.84877121928048	-328110583917.412\\
2.84887122178054	-328081936027.656\\
2.84897122428061	-328053861095.694\\
2.84907122678067	-328025213205.938\\
2.84917122928073	-327996565316.181\\
2.84927123178079	-327968490384.22\\
2.84937123428086	-327939842494.463\\
2.84947123678092	-327911194604.707\\
2.84957123928098	-327883119672.745\\
2.84967124178104	-327854471782.989\\
2.84977124428111	-327825823893.232\\
2.84987124678117	-327797176003.476\\
2.84997124928123	-327769101071.514\\
2.85007125178129	-327740453181.758\\
2.85017125428136	-327711805292.001\\
2.85027125678142	-327683157402.245\\
2.85037125928148	-327654509512.488\\
2.85047126178154	-327625861622.732\\
2.85057126428161	-327597213732.975\\
2.85067126678167	-327568565843.219\\
2.85077126928173	-327539917953.462\\
2.85087127178179	-327511270063.706\\
2.85097127428186	-327482622173.949\\
2.85107127678192	-327453974284.192\\
2.85117127928198	-327425326394.436\\
2.85127128178204	-327396678504.679\\
2.85137128428211	-327368030614.923\\
2.85147128678217	-327339382725.166\\
2.85157128928223	-327310734835.41\\
2.85167129178229	-327281513987.858\\
2.85177129428236	-327252866098.102\\
2.85187129678242	-327224218208.345\\
2.85197129928248	-327195570318.588\\
2.85207130178254	-327166349471.037\\
2.85217130428261	-327137701581.28\\
2.85227130678267	-327109053691.524\\
2.85237130928273	-327080405801.767\\
2.85247131178279	-327051184954.215\\
2.85257131428286	-327022537064.459\\
2.85267131678292	-326993316216.907\\
2.85277131928298	-326964668327.151\\
2.85287132178304	-326935447479.599\\
2.85297132428311	-326906799589.843\\
2.85307132678317	-326877578742.291\\
2.85317132928323	-326848930852.534\\
2.85327133178329	-326819710004.983\\
2.85337133428336	-326791062115.226\\
2.85347133678342	-326761841267.674\\
2.85357133928348	-326733193377.918\\
2.85367134178354	-326703972530.366\\
2.85377134428361	-326674751682.815\\
2.85387134678367	-326645530835.263\\
2.85397134928373	-326616882945.506\\
2.85407135178379	-326587662097.955\\
2.85417135428386	-326558441250.403\\
2.85427135678392	-326529220402.851\\
2.85437135928398	-326500572513.095\\
2.85447136178404	-326471351665.543\\
2.85457136428411	-326442130817.991\\
2.85467136678417	-326412909970.44\\
2.85477136928423	-326383689122.888\\
2.85487137178429	-326354468275.336\\
2.85497137428436	-326325247427.785\\
2.85507137678442	-326296026580.233\\
2.85517137928448	-326266805732.681\\
2.85527138178454	-326237584885.13\\
2.85537138428461	-326208364037.578\\
2.85547138678467	-326179143190.026\\
2.85557138928473	-326149922342.475\\
2.85567139178479	-326120701494.923\\
2.85577139428486	-326091480647.371\\
2.85587139678492	-326062259799.82\\
2.85597139928498	-326033038952.268\\
2.85607140178504	-326003818104.716\\
2.85617140428511	-325974024299.37\\
2.85627140678517	-325944803451.818\\
2.85637140928523	-325915582604.266\\
2.85647141178529	-325886361756.714\\
2.85657141428536	-325856567951.368\\
2.85667141678542	-325827347103.816\\
2.85677141928548	-325798126256.264\\
2.85687142178554	-325768332450.918\\
2.85697142428561	-325739111603.366\\
2.85707142678567	-325709317798.019\\
2.85717142928573	-325680096950.467\\
2.85727143178579	-325650876102.916\\
2.85737143428586	-325621082297.569\\
2.85747143678592	-325591861450.017\\
2.85757143928598	-325562067644.67\\
2.85767144178604	-325532273839.324\\
2.85777144428611	-325503052991.772\\
2.85787144678617	-325473259186.425\\
2.85797144928623	-325444038338.874\\
2.85807145178629	-325414244533.527\\
2.85817145428636	-325384450728.18\\
2.85827145678642	-325355229880.628\\
2.85837145928648	-325325436075.281\\
2.85847146178654	-325295642269.935\\
2.85857146428661	-325266421422.383\\
2.85867146678667	-325236627617.036\\
2.85877146928673	-325206833811.689\\
2.85887147178679	-325177040006.343\\
2.85897147428686	-325147246200.996\\
2.85907147678692	-325117452395.649\\
2.85917147928698	-325088231548.097\\
2.85927148178704	-325058437742.75\\
2.85937148428711	-325028643937.404\\
2.85947148678717	-324998850132.057\\
2.85957148928723	-324969056326.71\\
2.85967149178729	-324939262521.363\\
2.85977149428736	-324909468716.016\\
2.85987149678742	-324879674910.67\\
2.85997149928748	-324849881105.323\\
2.86007150178754	-324820087299.976\\
2.86017150428761	-324790293494.629\\
2.86027150678767	-324759926731.487\\
2.86037150928773	-324730132926.141\\
2.86047151178779	-324700339120.794\\
2.86057151428786	-324670545315.447\\
2.86067151678792	-324640751510.1\\
2.86077151928798	-324610384746.958\\
2.86087152178804	-324580590941.611\\
2.86097152428811	-324550797136.265\\
2.86107152678817	-324521003330.918\\
2.86117152928823	-324490636567.776\\
2.86127153178829	-324460842762.429\\
2.86137153428836	-324431048957.082\\
2.86147153678842	-324400682193.94\\
2.86157153928848	-324370888388.594\\
2.86167154178854	-324340521625.452\\
2.86177154428861	-324310727820.105\\
2.86187154678867	-324280361056.963\\
2.86197154928873	-324250567251.616\\
2.86207155178879	-324220200488.474\\
2.86217155428886	-324190406683.127\\
2.86227155678892	-324160039919.985\\
2.86237155928898	-324130246114.639\\
2.86247156178904	-324099879351.497\\
2.86257156428911	-324069512588.355\\
2.86267156678917	-324039718783.008\\
2.86277156928923	-324009352019.866\\
2.86287157178929	-323978985256.724\\
2.86297157428936	-323949191451.377\\
2.86307157678942	-323918824688.235\\
2.86317157928948	-323888457925.093\\
2.86327158178954	-323858091161.951\\
2.86337158428961	-323828297356.605\\
2.86347158678967	-323797930593.463\\
2.86357158928973	-323767563830.321\\
2.86367159178979	-323737197067.179\\
2.86377159428986	-323706830304.037\\
2.86387159678992	-323676463540.895\\
2.86397159928998	-323646096777.753\\
2.86407160179004	-323615730014.611\\
2.86417160429011	-323585363251.469\\
2.86427160679017	-323554996488.327\\
2.86437160929023	-323524629725.185\\
2.86447161179029	-323494262962.043\\
2.86457161429036	-323463896198.901\\
2.86467161679042	-323433529435.759\\
2.86477161929048	-323403162672.617\\
2.86487162179054	-323372795909.476\\
2.86497162429061	-323342429146.334\\
2.86507162679067	-323311489425.397\\
2.86517162929073	-323281122662.255\\
2.86527163179079	-323250755899.113\\
2.86537163429086	-323220389135.971\\
2.86547163679092	-323190022372.829\\
2.86557163929098	-323159082651.892\\
2.86567164179104	-323128715888.75\\
2.86577164429111	-323098349125.608\\
2.86587164679117	-323067409404.671\\
2.86597164929123	-323037042641.529\\
2.86607165179129	-323006102920.592\\
2.86617165429136	-322975736157.45\\
2.86627165679142	-322945369394.308\\
2.86637165929148	-322914429673.371\\
2.86647166179154	-322884062910.229\\
2.86657166429161	-322853123189.292\\
2.86667166679167	-322822756426.15\\
2.86677166929173	-322791816705.213\\
2.86687167179179	-322761449942.071\\
2.86697167429186	-322730510221.134\\
2.86707167679192	-322699570500.197\\
2.86717167929198	-322669203737.055\\
2.86727168179204	-322638264016.118\\
2.86737168429211	-322607324295.181\\
2.86747168679217	-322576957532.039\\
2.86757168929223	-322546017811.102\\
2.86767169179229	-322515078090.165\\
2.86777169429236	-322484138369.228\\
2.86787169679242	-322453771606.086\\
2.86797169929248	-322422831885.149\\
2.86807170179255	-322391892164.212\\
2.86817170429261	-322360952443.275\\
2.86827170679267	-322330012722.337\\
2.86837170929273	-322299073001.4\\
2.86847171179279	-322268133280.463\\
2.86857171429286	-322237766517.321\\
2.86867171679292	-322206826796.384\\
2.86877171929298	-322175887075.447\\
2.86887172179304	-322144947354.51\\
2.86897172429311	-322114007633.573\\
2.86907172679317	-322083067912.636\\
2.86917172929323	-322051555233.904\\
2.86927173179329	-322020615512.967\\
2.86937173429336	-321989675792.03\\
2.86947173679342	-321958736071.093\\
2.86957173929348	-321927796350.156\\
2.86967174179354	-321896856629.219\\
2.86977174429361	-321865916908.281\\
2.86987174679367	-321834404229.549\\
2.86997174929373	-321803464508.612\\
2.8700717517938	-321772524787.675\\
2.87017175429386	-321741585066.738\\
2.87027175679392	-321710072388.006\\
2.87037175929398	-321679132667.069\\
2.87047176179404	-321648192946.132\\
2.87057176429411	-321616680267.4\\
2.87067176679417	-321585740546.463\\
2.87077176929423	-321554800825.526\\
2.87087177179429	-321523288146.793\\
2.87097177429436	-321492348425.856\\
2.87107177679442	-321460835747.124\\
2.87117177929448	-321429896026.187\\
2.87127178179454	-321398383347.455\\
2.87137178429461	-321367443626.518\\
2.87147178679467	-321335930947.786\\
2.87157178929473	-321304991226.848\\
2.87167179179479	-321273478548.116\\
2.87177179429486	-321241965869.384\\
2.87187179679492	-321211026148.447\\
2.87197179929498	-321179513469.715\\
2.87207180179505	-321148000790.983\\
2.87217180429511	-321117061070.046\\
2.87227180679517	-321085548391.313\\
2.87237180929523	-321054035712.581\\
2.87247181179529	-321022523033.849\\
2.87257181429536	-320991583312.912\\
2.87267181679542	-320960070634.18\\
2.87277181929548	-320928557955.448\\
2.87287182179554	-320897045276.715\\
2.87297182429561	-320865532597.983\\
2.87307182679567	-320834019919.251\\
2.87317182929573	-320802507240.519\\
2.87327183179579	-320771567519.582\\
2.87337183429586	-320740054840.849\\
2.87347183679592	-320708542162.117\\
2.87357183929598	-320677029483.385\\
2.87367184179604	-320645516804.653\\
2.87377184429611	-320614004125.921\\
2.87387184679617	-320581918489.393\\
2.87397184929623	-320550405810.661\\
2.8740718517963	-320518893131.929\\
2.87417185429636	-320487380453.197\\
2.87427185679642	-320455867774.465\\
2.87437185929648	-320424355095.732\\
2.87447186179655	-320392842417\\
2.87457186429661	-320360756780.473\\
2.87467186679667	-320329244101.741\\
2.87477186929673	-320297731423.008\\
2.87487187179679	-320266218744.276\\
2.87497187429686	-320234133107.749\\
2.87507187679692	-320202620429.017\\
2.87517187929698	-320171107750.285\\
2.87527188179704	-320139022113.757\\
2.87537188429711	-320107509435.025\\
2.87547188679717	-320075996756.293\\
2.87557188929723	-320043911119.766\\
2.87567189179729	-320012398441.033\\
2.87577189429736	-319980312804.506\\
2.87587189679742	-319948800125.774\\
2.87597189929748	-319916714489.246\\
2.87607190179755	-319885201810.514\\
2.87617190429761	-319853116173.987\\
2.87627190679767	-319821603495.255\\
2.87637190929773	-319789517858.727\\
2.8764719117978	-319757432222.2\\
2.87657191429786	-319725919543.468\\
2.87667191679792	-319693833906.941\\
2.87677191929798	-319662321228.208\\
2.87687192179804	-319630235591.681\\
2.87697192429811	-319598149955.154\\
2.87707192679817	-319566064318.626\\
2.87717192929823	-319534551639.894\\
2.87727193179829	-319502466003.367\\
2.87737193429836	-319470380366.84\\
2.87747193679842	-319438294730.312\\
2.87757193929848	-319406209093.785\\
2.87767194179854	-319374696415.053\\
2.87777194429861	-319342610778.525\\
2.87787194679867	-319310525141.998\\
2.87797194929873	-319278439505.471\\
2.8780719517988	-319246353868.943\\
2.87817195429886	-319214268232.416\\
2.87827195679892	-319182182595.889\\
2.87837195929898	-319150096959.361\\
2.87847196179905	-319118011322.834\\
2.87857196429911	-319085925686.307\\
2.87867196679917	-319053840049.779\\
2.87877196929923	-319021754413.252\\
2.87887197179929	-318989668776.725\\
2.87897197429936	-318957583140.197\\
2.87907197679942	-318924924545.875\\
2.87917197929948	-318892838909.348\\
2.87927198179954	-318860753272.82\\
2.87937198429961	-318828667636.293\\
2.87947198679967	-318796581999.766\\
2.87957198929973	-318763923405.443\\
2.87967199179979	-318731837768.916\\
2.87977199429986	-318699752132.389\\
2.87987199679992	-318667666495.861\\
2.87997199929998	-318635007901.539\\
2.88007200180005	-318602922265.011\\
2.88017200430011	-318570263670.689\\
2.88027200680017	-318538178034.162\\
2.88037200930023	-318506092397.634\\
2.8804720118003	-318473433803.312\\
2.88057201430036	-318441348166.785\\
2.88067201680042	-318408689572.462\\
2.88077201930048	-318376603935.935\\
2.88087202180055	-318343945341.612\\
2.88097202430061	-318311859705.085\\
2.88107202680067	-318279201110.763\\
2.88117202930073	-318247115474.235\\
2.88127203180079	-318214456879.913\\
2.88137203430086	-318181798285.59\\
2.88147203680092	-318149712649.063\\
2.88157203930098	-318117054054.741\\
2.88167204180104	-318084395460.418\\
2.88177204430111	-318052309823.891\\
2.88187204680117	-318019651229.568\\
2.88197204930123	-317986992635.246\\
2.8820720518013	-317954906998.719\\
2.88217205430136	-317922248404.396\\
2.88227205680142	-317889589810.074\\
2.88237205930148	-317856931215.751\\
2.88247206180155	-317824272621.429\\
2.88257206430161	-317791614027.106\\
2.88267206680167	-317759528390.579\\
2.88277206930173	-317726869796.256\\
2.8828720718018	-317694211201.934\\
2.88297207430186	-317661552607.612\\
2.88307207680192	-317628894013.289\\
2.88317207930198	-317596235418.967\\
2.88327208180204	-317563576824.644\\
2.88337208430211	-317530918230.322\\
2.88347208680217	-317498259635.999\\
2.88357208930223	-317465601041.677\\
2.88367209180229	-317432942447.354\\
2.88377209430236	-317399710895.237\\
2.88387209680242	-317367052300.914\\
2.88397209930248	-317334393706.592\\
2.88407210180255	-317301735112.269\\
2.88417210430261	-317269076517.947\\
2.88427210680267	-317236417923.624\\
2.88437210930273	-317203186371.507\\
2.8844721118028	-317170527777.184\\
2.88457211430286	-317137869182.862\\
2.88467211680292	-317105210588.539\\
2.88477211930298	-317071979036.422\\
2.88487212180305	-317039320442.099\\
2.88497212430311	-317006661847.777\\
2.88507212680317	-316973430295.659\\
2.88517212930323	-316940771701.337\\
2.8852721318033	-316907540149.219\\
2.88537213430336	-316874881554.897\\
2.88547213680342	-316842222960.574\\
2.88557213930348	-316808991408.457\\
2.88567214180354	-316776332814.134\\
2.88577214430361	-316743101262.017\\
2.88587214680367	-316710442667.694\\
2.88597214930373	-316677211115.577\\
2.8860721518038	-316643979563.459\\
2.88617215430386	-316611320969.137\\
2.88627215680392	-316578089417.019\\
2.88637215930398	-316545430822.697\\
2.88647216180405	-316512199270.579\\
2.88657216430411	-316478967718.462\\
2.88667216680417	-316446309124.139\\
2.88677216930423	-316413077572.021\\
2.8868721718043	-316379846019.904\\
2.88697217430436	-316346614467.786\\
2.88707217680442	-316313955873.464\\
2.88717217930448	-316280724321.346\\
2.88727218180455	-316247492769.229\\
2.88737218430461	-316214261217.111\\
2.88747218680467	-316181029664.993\\
2.88757218930473	-316148371070.671\\
2.88767219180479	-316115139518.553\\
2.88777219430486	-316081907966.436\\
2.88787219680492	-316048676414.318\\
2.88797219930498	-316015444862.201\\
2.88807220180505	-315982213310.083\\
2.88817220430511	-315948981757.966\\
2.88827220680517	-315915750205.848\\
2.88837220930523	-315882518653.73\\
2.8884722118053	-315849287101.613\\
2.88857221430536	-315816055549.495\\
2.88867221680542	-315782823997.378\\
2.88877221930548	-315749592445.26\\
2.88887222180555	-315715787935.347\\
2.88897222430561	-315682556383.23\\
2.88907222680567	-315649324831.112\\
2.88917222930573	-315616093278.995\\
2.8892722318058	-315582861726.877\\
2.88937223430586	-315549630174.759\\
2.88947223680592	-315515825664.847\\
2.88957223930598	-315482594112.729\\
2.88967224180604	-315449362560.611\\
2.88977224430611	-315415558050.699\\
2.88987224680617	-315382326498.581\\
2.88997224930623	-315349094946.464\\
2.8900722518063	-315315290436.551\\
2.89017225430636	-315282058884.433\\
2.89027225680642	-315248827332.316\\
2.89037225930648	-315215022822.403\\
2.89047226180655	-315181791270.285\\
2.89057226430661	-315147986760.373\\
2.89067226680667	-315114755208.255\\
2.89077226930673	-315080950698.342\\
2.8908722718068	-315047719146.225\\
2.89097227430686	-315013914636.312\\
2.89107227680692	-314980683084.194\\
2.89117227930698	-314946878574.282\\
2.89127228180705	-314913647022.164\\
2.89137228430711	-314879842512.251\\
2.89147228680717	-314846038002.339\\
2.89157228930723	-314812806450.221\\
2.8916722918073	-314779001940.308\\
2.89177229430736	-314745197430.396\\
2.89187229680742	-314711965878.278\\
2.89197229930748	-314678161368.365\\
2.89207230180755	-314644356858.453\\
2.89217230430761	-314610552348.54\\
2.89227230680767	-314576747838.627\\
2.89237230930773	-314543516286.51\\
2.8924723118078	-314509711776.597\\
2.89257231430786	-314475907266.684\\
2.89267231680792	-314442102756.771\\
2.89277231930798	-314408298246.859\\
2.89287232180805	-314374493736.946\\
2.89297232430811	-314340689227.033\\
2.89307232680817	-314306884717.121\\
2.89317232930823	-314273080207.208\\
2.8932723318083	-314239275697.295\\
2.89337233430836	-314205471187.382\\
2.89347233680842	-314171666677.47\\
2.89357233930848	-314137862167.557\\
2.89367234180855	-314104057657.644\\
2.89377234430861	-314070253147.732\\
2.89387234680867	-314036448637.819\\
2.89397234930873	-314002644127.906\\
2.8940723518088	-313968839617.993\\
2.89417235430886	-313935035108.081\\
2.89427235680892	-313901230598.168\\
2.89437235930898	-313866853130.46\\
2.89447236180905	-313833048620.547\\
2.89457236430911	-313799244110.635\\
2.89467236680917	-313765439600.722\\
2.89477236930923	-313731062133.014\\
2.8948723718093	-313697257623.101\\
2.89497237430936	-313663453113.189\\
2.89507237680942	-313629075645.481\\
2.89517237930948	-313595271135.568\\
2.89527238180955	-313561466625.655\\
2.89537238430961	-313527089157.948\\
2.89547238680967	-313493284648.035\\
2.89557238930973	-313458907180.327\\
2.8956723918098	-313425102670.414\\
2.89577239430986	-313390725202.706\\
2.89587239680992	-313356920692.794\\
2.89597239930998	-313322543225.086\\
2.89607240181005	-313288738715.173\\
2.89617240431011	-313254361247.465\\
2.89627240681017	-313220556737.553\\
2.89637240931023	-313186179269.845\\
2.8964724118103	-313152374759.932\\
2.89657241431036	-313117997292.224\\
2.89667241681042	-313083619824.516\\
2.89677241931048	-313049815314.604\\
2.89687242181055	-313015437846.896\\
2.89697242431061	-312981060379.188\\
2.89707242681067	-312947255869.275\\
2.89717242931073	-312912878401.567\\
2.8972724318108	-312878500933.859\\
2.89737243431086	-312844123466.152\\
2.89747243681092	-312810318956.239\\
2.89757243931098	-312775941488.531\\
2.89767244181105	-312741564020.823\\
2.89777244431111	-312707186553.115\\
2.89787244681117	-312672809085.407\\
2.89797244931123	-312638431617.7\\
2.8980724518113	-312604054149.992\\
2.89817245431136	-312569676682.284\\
2.89827245681142	-312535299214.576\\
2.89837245931148	-312500921746.868\\
2.89847246181155	-312466544279.16\\
2.89857246431161	-312432166811.453\\
2.89867246681167	-312397789343.745\\
2.89877246931173	-312363411876.037\\
2.8988724718118	-312329034408.329\\
2.89897247431186	-312294656940.621\\
2.89907247681192	-312260279472.913\\
2.89917247931198	-312225902005.205\\
2.89927248181205	-312191524537.498\\
2.89937248431211	-312157147069.79\\
2.89947248681217	-312122769602.082\\
2.89957248931223	-312087819176.579\\
2.8996724918123	-312053441708.871\\
2.89977249431236	-312019064241.163\\
2.89987249681242	-311984686773.455\\
2.89997249931248	-311949736347.952\\
2.90007250181255	-311915358880.245\\
2.90017250431261	-311880981412.537\\
2.90027250681267	-311846030987.034\\
2.90037250931273	-311811653519.326\\
2.9004725118128	-311777276051.618\\
2.90057251431286	-311742325626.115\\
2.90067251681292	-311707948158.407\\
2.90077251931298	-311673570690.699\\
2.90087252181305	-311638620265.196\\
2.90097252431311	-311604242797.488\\
2.90107252681317	-311569292371.985\\
2.90117252931323	-311534914904.278\\
2.9012725318133	-311499964478.775\\
2.90137253431336	-311465587011.067\\
2.90147253681342	-311430636585.564\\
2.90157253931348	-311396259117.856\\
2.90167254181355	-311361308692.353\\
2.90177254431361	-311326358266.85\\
2.90187254681367	-311291980799.142\\
2.90197254931373	-311257030373.639\\
2.9020725518138	-311222079948.136\\
2.90217255431386	-311187702480.428\\
2.90227255681392	-311152752054.925\\
2.90237255931398	-311117801629.422\\
2.90247256181405	-311083424161.715\\
2.90257256431411	-311048473736.212\\
2.90267256681417	-311013523310.709\\
2.90277256931423	-310978572885.206\\
2.9028725718143	-310943622459.703\\
2.90297257431436	-310909244991.995\\
2.90307257681442	-310874294566.492\\
2.90317257931448	-310839344140.989\\
2.90327258181455	-310804393715.486\\
2.90337258431461	-310769443289.983\\
2.90347258681467	-310734492864.48\\
2.90357258931473	-310699542438.977\\
2.9036725918148	-310664592013.474\\
2.90377259431486	-310629641587.971\\
2.90387259681492	-310594691162.468\\
2.90397259931498	-310559740736.965\\
2.90407260181505	-310524790311.462\\
2.90417260431511	-310489839885.959\\
2.90427260681517	-310454889460.456\\
2.90437260931523	-310419939034.953\\
2.9044726118153	-310384988609.45\\
2.90457261431536	-310350038183.947\\
2.90467261681542	-310315087758.444\\
2.90477261931548	-310279564375.146\\
2.90487262181555	-310244613949.643\\
2.90497262431561	-310209663524.14\\
2.90507262681567	-310174713098.637\\
2.90517262931573	-310139762673.134\\
2.9052726318158	-310104239289.836\\
2.90537263431586	-310069288864.333\\
2.90547263681592	-310034338438.83\\
2.90557263931598	-309998815055.532\\
2.90567264181605	-309963864630.029\\
2.90577264431611	-309928914204.526\\
2.90587264681617	-309893390821.228\\
2.90597264931623	-309858440395.725\\
2.9060726518163	-309823489970.222\\
2.90617265431636	-309787966586.924\\
2.90627265681642	-309753016161.421\\
2.90637265931648	-309717492778.123\\
2.90647266181655	-309682542352.62\\
2.90657266431661	-309647018969.322\\
2.90667266681667	-309612068543.819\\
2.90677266931673	-309576545160.521\\
2.9068726718168	-309541594735.018\\
2.90697267431686	-309506071351.719\\
2.90707267681692	-309471120926.216\\
2.90717267931698	-309435597542.918\\
2.90727268181705	-309400074159.62\\
2.90737268431711	-309365123734.117\\
2.90747268681717	-309329600350.819\\
2.90757268931723	-309294076967.521\\
2.9076726918173	-309259126542.018\\
2.90777269431736	-309223603158.72\\
2.90787269681742	-309188079775.422\\
2.90797269931748	-309152556392.124\\
2.90807270181755	-309117605966.621\\
2.90817270431761	-309082082583.323\\
2.90827270681767	-309046559200.024\\
2.90837270931773	-309011035816.726\\
2.9084727118178	-308975512433.428\\
2.90857271431786	-308940562007.925\\
2.90867271681792	-308905038624.627\\
2.90877271931798	-308869515241.329\\
2.90887272181805	-308833991858.031\\
2.90897272431811	-308798468474.733\\
2.90907272681817	-308762945091.435\\
2.90917272931823	-308727421708.137\\
2.9092727318183	-308691898324.839\\
2.90937273431836	-308656374941.54\\
2.90947273681842	-308620851558.242\\
2.90957273931848	-308585328174.944\\
2.90967274181855	-308549804791.646\\
2.90977274431861	-308514281408.348\\
2.90987274681867	-308478758025.05\\
2.90997274931873	-308443234641.752\\
2.9100727518188	-308407138300.659\\
2.91017275431886	-308371614917.36\\
2.91027275681892	-308336091534.062\\
2.91037275931898	-308300568150.764\\
2.91047276181905	-308265044767.466\\
2.91057276431911	-308229521384.168\\
2.91067276681917	-308193425043.075\\
2.91077276931923	-308157901659.777\\
2.9108727718193	-308122378276.478\\
2.91097277431936	-308086281935.385\\
2.91107277681942	-308050758552.087\\
2.91117277931948	-308015235168.789\\
2.91127278181955	-307979138827.696\\
2.91137278431961	-307943615444.398\\
2.91147278681967	-307908092061.1\\
2.91157278931973	-307871995720.006\\
2.9116727918198	-307836472336.708\\
2.91177279431986	-307800375995.615\\
2.91187279681992	-307764852612.317\\
2.91197279931998	-307729329229.019\\
2.91207280182005	-307693232887.925\\
2.91217280432011	-307657709504.627\\
2.91227280682017	-307621613163.534\\
2.91237280932023	-307586089780.236\\
2.9124728118203	-307549993439.143\\
2.91257281432036	-307513897098.049\\
2.91267281682042	-307478373714.751\\
2.91277281932048	-307442277373.658\\
2.91287282182055	-307406753990.36\\
2.91297282432061	-307370657649.267\\
2.91307282682067	-307334561308.174\\
2.91317282932073	-307299037924.875\\
2.9132728318208	-307262941583.782\\
2.91337283432086	-307226845242.689\\
2.91347283682092	-307190748901.596\\
2.91357283932098	-307155225518.298\\
2.91367284182105	-307119129177.204\\
2.91377284432111	-307083032836.111\\
2.91387284682117	-307046936495.018\\
2.91397284932123	-307010840153.925\\
2.9140728518213	-306975316770.627\\
2.91417285432136	-306939220429.533\\
2.91427285682142	-306903124088.44\\
2.91437285932148	-306867027747.347\\
2.91447286182155	-306830931406.254\\
2.91457286432161	-306794835065.16\\
2.91467286682167	-306758738724.067\\
2.91477286932173	-306722642382.974\\
2.9148728718218	-306686546041.881\\
2.91497287432186	-306650449700.787\\
2.91507287682192	-306614353359.694\\
2.91517287932198	-306578257018.601\\
2.91527288182205	-306542160677.508\\
2.91537288432211	-306506064336.414\\
2.91547288682217	-306469967995.321\\
2.91557288932223	-306433871654.228\\
2.9156728918223	-306397775313.135\\
2.91577289432236	-306361678972.041\\
2.91587289682242	-306325009673.153\\
2.91597289932248	-306288913332.06\\
2.91607290182255	-306252816990.967\\
2.91617290432261	-306216720649.873\\
2.91627290682267	-306180624308.78\\
2.91637290932273	-306143955009.892\\
2.9164729118228	-306107858668.798\\
2.91657291432286	-306071762327.705\\
2.91667291682292	-306035665986.612\\
2.91677291932298	-305998996687.724\\
2.91687292182305	-305962900346.63\\
2.91697292432311	-305926804005.537\\
2.91707292682317	-305890134706.649\\
2.91717292932323	-305854038365.556\\
2.9172729318233	-305817369066.667\\
2.91737293432336	-305781272725.574\\
2.91747293682342	-305745176384.481\\
2.91757293932348	-305708507085.592\\
2.91767294182355	-305672410744.499\\
2.91777294432361	-305635741445.611\\
2.91787294682367	-305599645104.517\\
2.91797294932373	-305562975805.629\\
2.9180729518238	-305526879464.536\\
2.91817295432386	-305490210165.647\\
2.91827295682392	-305453540866.759\\
2.91837295932398	-305417444525.666\\
2.91847296182405	-305380775226.777\\
2.91857296432411	-305344678885.684\\
2.91867296682417	-305308009586.796\\
2.91877296932423	-305271340287.907\\
2.9188729718243	-305235243946.814\\
2.91897297432436	-305198574647.926\\
2.91907297682442	-305161905349.037\\
2.91917297932448	-305125236050.149\\
2.91927298182455	-305089139709.056\\
2.91937298432461	-305052470410.168\\
2.91947298682467	-305015801111.279\\
2.91957298932473	-304979131812.391\\
2.9196729918248	-304943035471.298\\
2.91977299432486	-304906366172.409\\
2.91987299682492	-304869696873.521\\
2.91997299932498	-304833027574.632\\
2.92007300182505	-304796358275.744\\
2.92017300432511	-304759688976.856\\
2.92027300682517	-304723019677.967\\
2.92037300932523	-304686350379.079\\
2.9204730118253	-304649681080.191\\
2.92057301432536	-304613011781.302\\
2.92067301682542	-304576342482.414\\
2.92077301932548	-304539673183.525\\
2.92087302182555	-304503003884.637\\
2.92097302432561	-304466334585.749\\
2.92107302682567	-304429665286.86\\
2.92117302932573	-304392995987.972\\
2.9212730318258	-304356326689.084\\
2.92137303432586	-304319657390.195\\
2.92147303682592	-304282988091.307\\
2.92157303932598	-304246318792.418\\
2.92167304182605	-304209649493.53\\
2.92177304432611	-304172407236.847\\
2.92187304682617	-304135737937.958\\
2.92197304932623	-304099068639.07\\
2.9220730518263	-304062399340.181\\
2.92217305432636	-304025730041.293\\
2.92227305682642	-303988487784.61\\
2.92237305932648	-303951818485.721\\
2.92247306182655	-303915149186.833\\
2.92257306432661	-303877906930.149\\
2.92267306682667	-303841237631.261\\
2.92277306932673	-303804568332.373\\
2.9228730718268	-303767326075.689\\
2.92297307432686	-303730656776.801\\
2.92307307682692	-303693987477.912\\
2.92317307932698	-303656745221.229\\
2.92327308182705	-303620075922.34\\
2.92337308432711	-303582833665.657\\
2.92347308682717	-303546164366.769\\
2.92357308932723	-303508922110.085\\
2.9236730918273	-303472252811.197\\
2.92377309432736	-303435010554.513\\
2.92387309682742	-303398341255.625\\
2.92397309932748	-303361098998.941\\
2.92407310182755	-303324429700.053\\
2.92417310432761	-303287187443.369\\
2.92427310682767	-303250518144.481\\
2.92437310932773	-303213275887.798\\
2.9244731118278	-303176033631.114\\
2.92457311432786	-303139364332.226\\
2.92467311682792	-303102122075.542\\
2.92477311932798	-303064879818.859\\
2.92487312182805	-303028210519.97\\
2.92497312432811	-302990968263.287\\
2.92507312682817	-302953726006.603\\
2.92517312932823	-302917056707.715\\
2.9252731318283	-302879814451.031\\
2.92537313432836	-302842572194.348\\
2.92547313682842	-302805329937.664\\
2.92557313932848	-302768087680.981\\
2.92567314182855	-302731418382.093\\
2.92577314432861	-302694176125.409\\
2.92587314682867	-302656933868.726\\
2.92597314932873	-302619691612.042\\
2.9260731518288	-302582449355.359\\
2.92617315432886	-302545207098.675\\
2.92627315682892	-302507964841.992\\
2.92637315932898	-302470722585.308\\
2.92647316182905	-302433480328.625\\
2.92657316432911	-302396238071.941\\
2.92667316682917	-302358995815.258\\
2.92677316932923	-302321753558.574\\
2.9268731718293	-302284511301.891\\
2.92697317432936	-302247269045.207\\
2.92707317682942	-302210026788.523\\
2.92717317932948	-302172784531.84\\
2.92727318182955	-302135542275.156\\
2.92737318432961	-302098300018.473\\
2.92747318682967	-302061057761.789\\
2.92757318932973	-302023815505.106\\
2.9276731918298	-301986573248.422\\
2.92777319432986	-301948758033.944\\
2.92787319682992	-301911515777.26\\
2.92797319932998	-301874273520.577\\
2.92807320183005	-301837031263.893\\
2.92817320433011	-301799216049.415\\
2.92827320683017	-301761973792.731\\
2.92837320933023	-301724731536.048\\
2.9284732118303	-301687489279.364\\
2.92857321433036	-301649674064.886\\
2.92867321683042	-301612431808.202\\
2.92877321933048	-301575189551.519\\
2.92887322183055	-301537374337.04\\
2.92897322433061	-301500132080.356\\
2.92907322683067	-301462889823.673\\
2.92917322933073	-301425074609.194\\
2.9292732318308	-301387832352.511\\
2.92937323433086	-301350017138.032\\
2.92947323683092	-301312774881.349\\
2.92957323933098	-301274959666.87\\
2.92967324183105	-301237717410.186\\
2.92977324433111	-301199902195.708\\
2.92987324683117	-301162659939.024\\
2.92997324933123	-301124844724.546\\
2.9300732518313	-301087602467.862\\
2.93017325433136	-301049787253.384\\
2.93027325683142	-301012544996.7\\
2.93037325933148	-300974729782.221\\
2.93047326183155	-300936914567.743\\
2.93057326433161	-300899672311.059\\
2.93067326683167	-300861857096.581\\
2.93077326933173	-300824614839.897\\
2.9308732718318	-300786799625.419\\
2.93097327433186	-300748984410.94\\
2.93107327683192	-300711169196.461\\
2.93117327933198	-300673926939.778\\
2.93127328183205	-300636111725.299\\
2.93137328433211	-300598296510.82\\
2.93147328683217	-300560481296.342\\
2.93157328933223	-300523239039.658\\
2.9316732918323	-300485423825.18\\
2.93177329433236	-300447608610.701\\
2.93187329683242	-300409793396.222\\
2.93197329933248	-300371978181.744\\
2.93207330183255	-300334162967.265\\
2.93217330433261	-300296920710.582\\
2.93227330683267	-300259105496.103\\
2.93237330933273	-300221290281.624\\
2.9324733118328	-300183475067.146\\
2.93257331433286	-300145659852.667\\
2.93267331683292	-300107844638.189\\
2.93277331933298	-300070029423.71\\
2.93287332183305	-300032214209.231\\
2.93297332433311	-299994398994.753\\
2.93307332683317	-299956583780.274\\
2.93317332933323	-299918768565.795\\
2.9332733318333	-299880953351.317\\
2.93337333433336	-299843138136.838\\
2.93347333683342	-299805322922.359\\
2.93357333933348	-299766934750.086\\
2.93367334183355	-299729119535.607\\
2.93377334433361	-299691304321.128\\
2.93387334683367	-299653489106.65\\
2.93397334933373	-299615673892.171\\
2.9340733518338	-299577858677.693\\
2.93417335433386	-299539470505.419\\
2.93427335683392	-299501655290.94\\
2.93437335933398	-299463840076.461\\
2.93447336183405	-299426024861.983\\
2.93457336433411	-299387636689.709\\
2.93467336683417	-299349821475.23\\
2.93477336933423	-299312006260.752\\
2.9348733718343	-299274191046.273\\
2.93497337433436	-299235802873.999\\
2.93507337683442	-299197987659.521\\
2.93517337933448	-299159599487.247\\
2.93527338183455	-299121784272.768\\
2.93537338433461	-299083969058.29\\
2.93547338683467	-299045580886.016\\
2.93557338933473	-299007765671.537\\
2.9356733918348	-298969377499.264\\
2.93577339433486	-298931562284.785\\
2.93587339683492	-298893174112.511\\
2.93597339933498	-298855358898.033\\
2.93607340183505	-298816970725.759\\
2.93617340433511	-298779155511.28\\
2.93627340683517	-298740767339.006\\
2.93637340933523	-298702952124.528\\
2.9364734118353	-298664563952.254\\
2.93657341433536	-298626748737.775\\
2.93667341683542	-298588360565.502\\
2.93677341933548	-298550545351.023\\
2.93687342183555	-298512157178.749\\
2.93697342433561	-298473769006.475\\
2.93707342683567	-298435953791.997\\
2.93717342933573	-298397565619.723\\
2.9372734318358	-298359177447.449\\
2.93737343433586	-298320789275.175\\
2.93747343683592	-298282974060.697\\
2.93757343933598	-298244585888.423\\
2.93767344183605	-298206197716.149\\
2.93777344433611	-298167809543.876\\
2.93787344683617	-298129994329.397\\
2.93797344933623	-298091606157.123\\
2.9380734518363	-298053217984.849\\
2.93817345433636	-298014829812.576\\
2.93827345683642	-297976441640.302\\
2.93837345933648	-297938626425.823\\
2.93847346183655	-297900238253.549\\
2.93857346433661	-297861850081.276\\
2.93867346683667	-297823461909.002\\
2.93877346933673	-297785073736.728\\
2.9388734718368	-297746685564.454\\
2.93897347433686	-297708297392.181\\
2.93907347683692	-297669909219.907\\
2.93917347933698	-297631521047.633\\
2.93927348183705	-297593132875.359\\
2.93937348433711	-297554744703.086\\
2.93947348683717	-297516356530.812\\
2.93957348933723	-297477968358.538\\
2.9396734918373	-297439580186.264\\
2.93977349433736	-297401192013.99\\
2.93987349683742	-297362803841.717\\
2.93997349933748	-297324415669.443\\
2.94007350183755	-297285454539.374\\
2.94017350433761	-297247066367.1\\
2.94027350683767	-297208678194.827\\
2.94037350933773	-297170290022.553\\
2.9404735118378	-297131901850.279\\
2.94057351433786	-297093513678.005\\
2.94067351683792	-297054552547.936\\
2.94077351933798	-297016164375.663\\
2.94087352183805	-296977776203.389\\
2.94097352433811	-296939388031.115\\
2.94107352683817	-296900426901.046\\
2.94117352933823	-296862038728.772\\
2.9412735318383	-296823650556.499\\
2.94137353433836	-296784689426.43\\
2.94147353683842	-296746301254.156\\
2.94157353933848	-296707913081.882\\
2.94167354183855	-296668951951.813\\
2.94177354433861	-296630563779.54\\
2.94187354683867	-296592175607.266\\
2.94197354933873	-296553214477.197\\
2.9420735518388	-296514826304.923\\
2.94217355433886	-296475865174.854\\
2.94227355683892	-296437477002.58\\
2.94237355933898	-296398515872.512\\
2.94247356183905	-296360127700.238\\
2.94257356433911	-296321166570.169\\
2.94267356683917	-296282778397.895\\
2.94277356933923	-296243817267.826\\
2.9428735718393	-296205429095.552\\
2.94297357433936	-296166467965.484\\
2.94307357683942	-296128079793.21\\
2.94317357933948	-296089118663.141\\
2.94327358183955	-296050157533.072\\
2.94337358433961	-296011769360.798\\
2.94347358683967	-295972808230.729\\
2.94357358933973	-295934420058.456\\
2.9436735918398	-295895458928.387\\
2.94377359433986	-295856497798.318\\
2.94387359683992	-295817536668.249\\
2.94397359933998	-295779148495.975\\
2.94407360184005	-295740187365.906\\
2.94417360434011	-295701226235.837\\
2.94427360684017	-295662265105.768\\
2.94437360934023	-295623876933.495\\
2.9444736118403	-295584915803.426\\
2.94457361434036	-295545954673.357\\
2.94467361684042	-295506993543.288\\
2.94477361934048	-295468032413.219\\
2.94487362184055	-295429644240.945\\
2.94497362434061	-295390683110.876\\
2.94507362684067	-295351721980.808\\
2.94517362934073	-295312760850.739\\
2.9452736318408	-295273799720.67\\
2.94537363434086	-295234838590.601\\
2.94547363684092	-295195877460.532\\
2.94557363934098	-295156916330.463\\
2.94567364184105	-295117955200.394\\
2.94577364434111	-295078994070.325\\
2.94587364684117	-295040032940.256\\
2.94597364934123	-295001071810.188\\
2.9460736518413	-294962110680.119\\
2.94617365434136	-294923149550.05\\
2.94627365684142	-294884188419.981\\
2.94637365934148	-294845227289.912\\
2.94647366184155	-294806266159.843\\
2.94657366434161	-294767305029.774\\
2.94667366684167	-294728343899.705\\
2.94677366934173	-294689382769.636\\
2.9468736718418	-294649848681.772\\
2.94697367434186	-294610887551.703\\
2.94707367684192	-294571926421.635\\
2.94717367934198	-294532965291.566\\
2.94727368184205	-294494004161.497\\
2.94737368434211	-294455043031.428\\
2.94747368684217	-294415508943.564\\
2.94757368934223	-294376547813.495\\
2.9476736918423	-294337586683.426\\
2.94777369434236	-294298625553.357\\
2.94787369684242	-294259091465.493\\
2.94797369934248	-294220130335.424\\
2.94807370184255	-294181169205.355\\
2.94817370434261	-294141635117.491\\
2.94827370684267	-294102673987.422\\
2.94837370934273	-294063712857.353\\
2.9484737118428	-294024178769.489\\
2.94857371434286	-293985217639.421\\
2.94867371684292	-293945683551.557\\
2.94877371934298	-293906722421.488\\
2.94887372184305	-293867188333.624\\
2.94897372434311	-293828227203.555\\
2.94907372684317	-293789266073.486\\
2.94917372934323	-293749731985.622\\
2.9492737318433	-293710770855.553\\
2.94937373434336	-293671236767.689\\
2.94947373684342	-293632275637.62\\
2.94957373934348	-293592741549.756\\
2.94967374184355	-293553207461.892\\
2.94977374434361	-293514246331.823\\
2.94987374684367	-293474712243.959\\
2.94997374934373	-293435751113.89\\
2.9500737518438	-293396217026.026\\
2.95017375434386	-293356682938.162\\
2.95027375684392	-293317721808.093\\
2.95037375934398	-293278187720.229\\
2.95047376184405	-293238653632.365\\
2.95057376434411	-293199692502.296\\
2.95067376684417	-293160158414.432\\
2.95077376934423	-293120624326.568\\
2.9508737718443	-293081663196.499\\
2.95097377434436	-293042129108.635\\
2.95107377684442	-293002595020.771\\
2.95117377934448	-292963060932.907\\
2.95127378184455	-292924099802.838\\
2.95137378434461	-292884565714.974\\
2.95147378684467	-292845031627.11\\
2.95157378934473	-292805497539.246\\
2.9516737918448	-292765963451.382\\
2.95177379434486	-292726429363.518\\
2.95187379684492	-292687468233.449\\
2.95197379934498	-292647934145.585\\
2.95207380184505	-292608400057.721\\
2.95217380434511	-292568865969.857\\
2.95227380684517	-292529331881.993\\
2.95237380934523	-292489797794.129\\
2.9524738118453	-292450263706.265\\
2.95257381434536	-292410729618.401\\
2.95267381684542	-292371195530.537\\
2.95277381934548	-292331661442.673\\
2.95287382184555	-292292127354.809\\
2.95297382434561	-292252593266.945\\
2.95307382684567	-292213059179.081\\
2.95317382934573	-292173525091.217\\
2.9532738318458	-292133991003.353\\
2.95337383434586	-292094456915.489\\
2.95347383684592	-292054922827.625\\
2.95357383934598	-292015388739.761\\
2.95367384184605	-291975854651.897\\
2.95377384434611	-291935747606.238\\
2.95387384684617	-291896213518.374\\
2.95397384934623	-291856679430.51\\
2.9540738518463	-291817145342.646\\
2.95417385434636	-291777611254.781\\
2.95427385684642	-291737504209.122\\
2.95437385934648	-291697970121.258\\
2.95447386184655	-291658436033.394\\
2.95457386434661	-291618901945.53\\
2.95467386684667	-291578794899.871\\
2.95477386934673	-291539260812.007\\
2.9548738718468	-291499726724.143\\
2.95497387434686	-291460192636.279\\
2.95507387684692	-291420085590.62\\
2.95517387934698	-291380551502.756\\
2.95527388184705	-291341017414.892\\
2.95537388434711	-291300910369.233\\
2.95547388684717	-291261376281.369\\
2.95557388934723	-291221269235.709\\
2.9556738918473	-291181735147.845\\
2.95577389434736	-291142201059.981\\
2.95587389684742	-291102094014.322\\
2.95597389934748	-291062559926.458\\
2.95607390184755	-291022452880.799\\
2.95617390434761	-290982918792.935\\
2.95627390684767	-290942811747.276\\
2.95637390934773	-290903277659.412\\
2.9564739118478	-290863170613.753\\
2.95657391434786	-290823636525.889\\
2.95667391684792	-290783529480.229\\
2.95677391934798	-290743995392.365\\
2.95687392184805	-290703888346.706\\
2.95697392434811	-290664354258.842\\
2.95707392684817	-290624247213.183\\
2.95717392934823	-290584140167.524\\
2.9572739318483	-290544606079.66\\
2.95737393434836	-290504499034.001\\
2.95747393684842	-290464391988.342\\
2.95757393934848	-290424857900.478\\
2.95767394184855	-290384750854.818\\
2.95777394434861	-290344643809.159\\
2.95787394684867	-290305109721.295\\
2.95797394934873	-290265002675.636\\
2.9580739518488	-290224895629.977\\
2.95817395434886	-290184788584.318\\
2.95827395684892	-290145254496.454\\
2.95837395934898	-290105147450.795\\
2.95847396184905	-290065040405.135\\
2.95857396434911	-290024933359.476\\
2.95867396684917	-289984826313.817\\
2.95877396934923	-289945292225.953\\
2.9588739718493	-289905185180.294\\
2.95897397434936	-289865078134.635\\
2.95907397684942	-289824971088.976\\
2.95917397934948	-289784864043.316\\
2.95927398184955	-289744756997.657\\
2.95937398434961	-289704649951.998\\
2.95947398684967	-289664542906.339\\
2.95957398934973	-289625008818.475\\
2.9596739918498	-289584901772.816\\
2.95977399434986	-289544794727.157\\
2.95987399684992	-289504687681.497\\
2.95997399934998	-289464580635.838\\
2.96007400185005	-289424473590.179\\
2.96017400435011	-289384366544.52\\
2.96027400685017	-289344259498.861\\
2.96037400935023	-289304152453.202\\
2.9604740118503	-289263472449.747\\
2.96057401435036	-289223365404.088\\
2.96067401685042	-289183258358.429\\
2.96077401935048	-289143151312.77\\
2.96087402185055	-289103044267.111\\
2.96097402435061	-289062937221.452\\
2.96107402685067	-289022830175.792\\
2.96117402935073	-288982723130.133\\
2.9612740318508	-288942043126.679\\
2.96137403435086	-288901936081.02\\
2.96147403685092	-288861829035.361\\
2.96157403935098	-288821721989.702\\
2.96167404185105	-288781614944.042\\
2.96177404435111	-288740934940.588\\
2.96187404685117	-288700827894.929\\
2.96197404935123	-288660720849.27\\
2.9620740518513	-288620613803.611\\
2.96217405435136	-288579933800.156\\
2.96227405685142	-288539826754.497\\
2.96237405935148	-288499719708.838\\
2.96247406185155	-288459039705.384\\
2.96257406435161	-288418932659.725\\
2.96267406685167	-288378825614.065\\
2.96277406935173	-288338145610.611\\
2.9628740718518	-288298038564.952\\
2.96297407435186	-288257931519.293\\
2.96307407685192	-288217251515.839\\
2.96317407935198	-288177144470.179\\
2.96327408185205	-288136464466.725\\
2.96337408435211	-288096357421.066\\
2.96347408685217	-288055677417.612\\
2.96357408935223	-288015570371.952\\
2.9636740918523	-287974890368.498\\
2.96377409435236	-287934783322.839\\
2.96387409685242	-287894103319.385\\
2.96397409935248	-287853996273.726\\
2.96407410185255	-287813316270.271\\
2.96417410435261	-287773209224.612\\
2.96427410685267	-287732529221.158\\
2.96437410935273	-287692422175.499\\
2.9644741118528	-287651742172.044\\
2.96457411435286	-287611062168.59\\
2.96467411685292	-287570955122.931\\
2.96477411935298	-287530275119.477\\
2.96487412185305	-287489595116.022\\
2.96497412435311	-287449488070.363\\
2.96507412685317	-287408808066.909\\
2.96517412935323	-287368128063.455\\
2.9652741318533	-287328021017.795\\
2.96537413435336	-287287341014.341\\
2.96547413685342	-287246661010.887\\
2.96557413935348	-287206553965.228\\
2.96567414185355	-287165873961.773\\
2.96577414435361	-287125193958.319\\
2.96587414685367	-287084513954.865\\
2.96597414935373	-287044406909.206\\
2.9660741518538	-287003726905.751\\
2.96617415435386	-286963046902.297\\
2.96627415685392	-286922366898.843\\
2.96637415935398	-286881686895.389\\
2.96647416185405	-286841006891.934\\
2.96657416435411	-286800899846.275\\
2.96667416685417	-286760219842.821\\
2.96677416935423	-286719539839.367\\
2.9668741718543	-286678859835.912\\
2.96697417435436	-286638179832.458\\
2.96707417685442	-286597499829.004\\
2.96717417935448	-286556819825.549\\
2.96727418185455	-286516139822.095\\
2.96737418435461	-286475459818.641\\
2.96747418685467	-286434779815.187\\
2.96757418935473	-286394099811.732\\
2.9676741918548	-286353419808.278\\
2.96777419435486	-286312739804.824\\
2.96787419685492	-286272059801.369\\
2.96797419935498	-286231379797.915\\
2.96807420185505	-286190699794.461\\
2.96817420435511	-286150019791.007\\
2.96827420685517	-286109339787.552\\
2.96837420935523	-286068659784.098\\
2.9684742118553	-286027979780.644\\
2.96857421435536	-285986726819.394\\
2.96867421685542	-285946046815.94\\
2.96877421935548	-285905366812.486\\
2.96887422185555	-285864686809.031\\
2.96897422435561	-285824006805.577\\
2.96907422685567	-285783326802.123\\
2.96917422935573	-285742073840.873\\
2.9692742318558	-285701393837.419\\
2.96937423435586	-285660713833.965\\
2.96947423685592	-285620033830.51\\
2.96957423935598	-285579353827.056\\
2.96967424185605	-285538100865.807\\
2.96977424435611	-285497420862.353\\
2.96987424685617	-285456740858.898\\
2.96997424935623	-285415487897.649\\
2.9700742518563	-285374807894.195\\
2.97017425435636	-285334127890.74\\
2.97027425685642	-285292874929.491\\
2.97037425935648	-285252194926.036\\
2.97047426185655	-285211514922.582\\
2.97057426435661	-285170261961.333\\
2.97067426685667	-285129581957.879\\
2.97077426935673	-285088901954.424\\
2.9708742718568	-285047648993.175\\
2.97097427435686	-285006968989.721\\
2.97107427685692	-284965716028.471\\
2.97117427935698	-284925036025.017\\
2.97127428185705	-284883783063.767\\
2.97137428435711	-284843103060.313\\
2.97147428685717	-284801850099.064\\
2.97157428935723	-284761170095.609\\
2.9716742918573	-284719917134.36\\
2.97177429435736	-284679237130.906\\
2.97187429685742	-284637984169.656\\
2.97197429935748	-284597304166.202\\
2.97207430185755	-284556051204.953\\
2.97217430435761	-284515371201.498\\
2.97227430685767	-284474118240.249\\
2.97237430935773	-284432865278.999\\
2.9724743118578	-284392185275.545\\
2.97257431435786	-284350932314.296\\
2.97267431685792	-284310252310.841\\
2.97277431935798	-284268999349.592\\
2.97287432185805	-284227746388.343\\
2.97297432435811	-284187066384.888\\
2.97307432685817	-284145813423.639\\
2.97317432935823	-284104560462.389\\
2.9732743318583	-284063307501.14\\
2.97337433435836	-284022627497.686\\
2.97347433685842	-283981374536.436\\
2.97357433935848	-283940121575.187\\
2.97367434185855	-283898868613.938\\
2.97377434435861	-283858188610.483\\
2.97387434685867	-283816935649.234\\
2.97397434935873	-283775682687.984\\
2.9740743518588	-283734429726.735\\
2.97417435435886	-283693176765.486\\
2.97427435685892	-283652496762.031\\
2.97437435935898	-283611243800.782\\
2.97447436185905	-283569990839.532\\
2.97457436435911	-283528737878.283\\
2.97467436685917	-283487484917.034\\
2.97477436935923	-283446231955.784\\
2.9748743718593	-283404978994.535\\
2.97497437435936	-283363726033.285\\
2.97507437685942	-283322473072.036\\
2.97517437935948	-283281793068.582\\
2.97527438185955	-283240540107.332\\
2.97537438435961	-283199287146.083\\
2.97547438685967	-283158034184.833\\
2.97557438935973	-283116781223.584\\
2.9756743918598	-283075528262.335\\
2.97577439435986	-283034275301.085\\
2.97587439685992	-282993022339.836\\
2.97597439935998	-282951769378.586\\
2.97607440186005	-282909943459.542\\
2.97617440436011	-282868690498.292\\
2.97627440686017	-282827437537.043\\
2.97637440936023	-282786184575.793\\
2.9764744118603	-282744931614.544\\
2.97657441436036	-282703678653.295\\
2.97667441686042	-282662425692.045\\
2.97677441936048	-282621172730.796\\
2.97687442186055	-282579919769.546\\
2.97697442436061	-282538093850.502\\
2.97707442686067	-282496840889.252\\
2.97717442936073	-282455587928.003\\
2.9772744318608	-282414334966.754\\
2.97737443436086	-282373082005.504\\
2.97747443686092	-282331256086.46\\
2.97757443936098	-282290003125.21\\
2.97767444186105	-282248750163.961\\
2.97777444436111	-282207497202.711\\
2.97787444686117	-282165671283.667\\
2.97797444936123	-282124418322.417\\
2.9780744518613	-282083165361.168\\
2.97817445436136	-282041912399.919\\
2.97827445686142	-282000086480.874\\
2.97837445936148	-281958833519.625\\
2.97847446186155	-281917580558.375\\
2.97857446436161	-281875754639.331\\
2.97867446686167	-281834501678.081\\
2.97877446936173	-281793248716.832\\
2.9788744718618	-281751422797.787\\
2.97897447436186	-281710169836.538\\
2.97907447686192	-281668343917.493\\
2.97917447936198	-281627090956.244\\
2.97927448186205	-281585265037.199\\
2.97937448436211	-281544012075.95\\
2.97947448686217	-281502759114.7\\
2.97957448936223	-281460933195.656\\
2.9796744918623	-281419680234.406\\
2.97977449436236	-281377854315.362\\
2.97987449686242	-281336601354.112\\
2.97997449936248	-281294775435.068\\
2.98007450186255	-281253522473.818\\
2.98017450436261	-281211696554.774\\
2.98027450686267	-281169870635.729\\
2.98037450936273	-281128617674.48\\
2.9804745118628	-281086791755.435\\
2.98057451436286	-281045538794.186\\
2.98067451686292	-281003712875.141\\
2.98077451936298	-280962459913.892\\
2.98087452186305	-280920633994.847\\
2.98097452436311	-280878808075.803\\
2.98107452686317	-280837555114.554\\
2.98117452936323	-280795729195.509\\
2.9812745318633	-280753903276.464\\
2.98137453436336	-280712650315.215\\
2.98147453686342	-280670824396.17\\
2.98157453936348	-280628998477.126\\
2.98167454186355	-280587172558.081\\
2.98177454436361	-280545919596.832\\
2.98187454686367	-280504093677.787\\
2.98197454936373	-280462267758.743\\
2.9820745518638	-280421014797.493\\
2.98217455436386	-280379188878.449\\
2.98227455686392	-280337362959.404\\
2.98237455936398	-280295537040.36\\
2.98247456186405	-280253711121.315\\
2.98257456436411	-280212458160.066\\
2.98267456686417	-280170632241.021\\
2.98277456936423	-280128806321.977\\
2.9828745718643	-280086980402.932\\
2.98297457436436	-280045154483.888\\
2.98307457686442	-280003328564.843\\
2.98317457936448	-279961502645.799\\
2.98327458186455	-279919676726.754\\
2.98337458436461	-279878423765.505\\
2.98347458686467	-279836597846.46\\
2.98357458936473	-279794771927.415\\
2.9836745918648	-279752946008.371\\
2.98377459436486	-279711120089.326\\
2.98387459686492	-279669294170.282\\
2.98397459936498	-279627468251.237\\
2.98407460186505	-279585642332.193\\
2.98417460436511	-279543816413.148\\
2.98427460686517	-279501990494.104\\
2.98437460936523	-279460164575.059\\
2.9844746118653	-279418338656.014\\
2.98457461436536	-279376512736.97\\
2.98467461686542	-279334686817.925\\
2.98477461936548	-279292860898.881\\
2.98487462186555	-279251034979.836\\
2.98497462436561	-279208636102.997\\
2.98507462686567	-279166810183.952\\
2.98517462936573	-279124984264.907\\
2.9852746318658	-279083158345.863\\
2.98537463436586	-279041332426.818\\
2.98547463686592	-278999506507.774\\
2.98557463936598	-278957680588.729\\
2.98567464186605	-278915281711.89\\
2.98577464436611	-278873455792.845\\
2.98587464686617	-278831629873.801\\
2.98597464936623	-278789803954.756\\
2.9860746518663	-278747978035.711\\
2.98617465436636	-278705579158.872\\
2.98627465686642	-278663753239.827\\
2.98637465936648	-278621927320.783\\
2.98647466186655	-278580101401.738\\
2.98657466436661	-278537702524.898\\
2.98667466686667	-278495876605.854\\
2.98677466936673	-278454050686.809\\
2.9868746718668	-278412224767.765\\
2.98697467436686	-278369825890.925\\
2.98707467686692	-278327999971.881\\
2.98717467936698	-278286174052.836\\
2.98727468186705	-278243775175.996\\
2.98737468436711	-278201949256.952\\
2.98747468686717	-278160123337.907\\
2.98757468936723	-278117724461.068\\
2.9876746918673	-278075898542.023\\
2.98777469436736	-278033499665.183\\
2.98787469686742	-277991673746.139\\
2.98797469936748	-277949274869.299\\
2.98807470186755	-277907448950.255\\
2.98817470436761	-277865623031.21\\
2.98827470686767	-277823224154.37\\
2.98837470936773	-277781398235.326\\
2.9884747118678	-277738999358.486\\
2.98857471436786	-277697173439.441\\
2.98867471686792	-277654774562.602\\
2.98877471936798	-277612948643.557\\
2.98887472186805	-277570549766.718\\
2.98897472436811	-277528723847.673\\
2.98907472686817	-277486324970.833\\
2.98917472936823	-277444499051.789\\
2.9892747318683	-277402100174.949\\
2.98937473436836	-277359701298.109\\
2.98947473686842	-277317875379.065\\
2.98957473936848	-277275476502.225\\
2.98967474186855	-277233650583.181\\
2.98977474436861	-277191251706.341\\
2.98987474686867	-277148852829.501\\
2.98997474936873	-277107026910.457\\
2.9900747518688	-277064628033.617\\
2.99017475436886	-277022229156.777\\
2.99027475686892	-276980403237.733\\
2.99037475936898	-276938004360.893\\
2.99047476186905	-276895605484.053\\
2.99057476436911	-276853779565.009\\
2.99067476686917	-276811380688.169\\
2.99077476936923	-276768981811.33\\
2.9908747718693	-276726582934.49\\
2.99097477436936	-276684757015.445\\
2.99107477686942	-276642358138.606\\
2.99117477936948	-276599959261.766\\
2.99127478186955	-276557560384.926\\
2.99137478436961	-276515734465.882\\
2.99147478686967	-276473335589.042\\
2.99157478936973	-276430936712.202\\
2.9916747918698	-276388537835.363\\
2.99177479436986	-276346138958.523\\
2.99187479686992	-276303740081.683\\
2.99197479936998	-276261914162.639\\
2.99207480187005	-276219515285.799\\
2.99217480437011	-276177116408.959\\
2.99227480687017	-276134717532.12\\
2.99237480937023	-276092318655.28\\
2.9924748118703	-276049919778.44\\
2.99257481437036	-276007520901.601\\
2.99267481687042	-275965122024.761\\
2.99277481937048	-275922723147.921\\
2.99287482187055	-275880324271.082\\
2.99297482437061	-275837925394.242\\
2.99307482687067	-275795526517.402\\
2.99317482937073	-275753127640.563\\
2.9932748318708	-275710728763.723\\
2.99337483437086	-275668329886.883\\
2.99347483687092	-275625931010.044\\
2.99357483937098	-275583532133.204\\
2.99367484187105	-275541133256.364\\
2.99377484437111	-275498734379.525\\
2.99387484687117	-275456335502.685\\
2.99397484937123	-275413936625.845\\
2.9940748518713	-275371537749.005\\
2.99417485437136	-275329138872.166\\
2.99427485687142	-275286739995.326\\
2.99437485937148	-275244341118.486\\
2.99447486187155	-275201942241.647\\
2.99457486437161	-275159543364.807\\
2.99467486687167	-275116571530.172\\
2.99477486937173	-275074172653.333\\
2.9948748718718	-275031773776.493\\
2.99497487437186	-274989374899.653\\
2.99507487687192	-274946976022.814\\
2.99517487937198	-274904577145.974\\
2.99527488187205	-274861605311.339\\
2.99537488437211	-274819206434.499\\
2.99547488687217	-274776807557.66\\
2.99557488937223	-274734408680.82\\
2.9956748918723	-274691436846.185\\
2.99577489437236	-274649037969.346\\
2.99587489687242	-274606639092.506\\
2.99597489937248	-274564240215.666\\
2.99607490187255	-274521268381.031\\
2.99617490437261	-274478869504.192\\
2.99627490687267	-274436470627.352\\
2.99637490937273	-274393498792.717\\
2.9964749118728	-274351099915.877\\
2.99657491437286	-274308701039.038\\
2.99667491687292	-274265729204.403\\
2.99677491937298	-274223330327.563\\
2.99687492187305	-274180931450.724\\
2.99697492437311	-274137959616.089\\
2.99707492687317	-274095560739.249\\
2.99717492937323	-274052588904.614\\
2.9972749318733	-274010190027.775\\
2.99737493437336	-273967791150.935\\
2.99747493687342	-273924819316.3\\
2.99757493937348	-273882420439.46\\
2.99767494187355	-273839448604.826\\
2.99777494437361	-273797049727.986\\
2.99787494687367	-273754077893.351\\
2.99797494937373	-273711679016.511\\
2.9980749518738	-273668707181.877\\
2.99817495437386	-273626308305.037\\
2.99827495687392	-273583336470.402\\
2.99837495937398	-273540937593.562\\
2.99847496187405	-273497965758.928\\
2.99857496437411	-273455566882.088\\
2.99867496687417	-273412595047.453\\
2.99877496937423	-273370196170.614\\
2.9988749718743	-273327224335.979\\
2.99897497437436	-273284825459.139\\
2.99907497687442	-273241853624.504\\
2.99917497937448	-273198881789.869\\
2.99927498187455	-273156482913.03\\
2.99937498437461	-273113511078.395\\
2.99947498687467	-273071112201.555\\
2.99957498937473	-273028140366.92\\
2.9996749918748	-272985168532.286\\
2.99977499437486	-272942769655.446\\
2.99987499687492	-272899797820.811\\
2.99997499937498	-272856825986.176\\
3.00007500187505	-272814427109.337\\
3.00017500437511	-272771455274.702\\
3.00027500687517	-272728483440.067\\
3.00037500937523	-272685511605.432\\
3.0004750118753	-272643112728.592\\
3.00057501437536	-272600140893.958\\
3.00067501687542	-272557169059.323\\
3.00077501937548	-272514197224.688\\
3.00087502187555	-272471798347.848\\
3.00097502437561	-272428826513.214\\
3.00107502687567	-272385854678.579\\
3.00117502937573	-272342882843.944\\
3.0012750318758	-272299911009.309\\
3.00137503437586	-272257512132.469\\
3.00147503687592	-272214540297.835\\
3.00157503937598	-272171568463.2\\
3.00167504187605	-272128596628.565\\
3.00177504437611	-272085624793.93\\
3.00187504687617	-272042652959.295\\
3.00197504937623	-272000254082.456\\
3.0020750518763	-271957282247.821\\
3.00217505437636	-271914310413.186\\
3.00227505687642	-271871338578.551\\
3.00237505937648	-271828366743.916\\
3.00247506187655	-271785394909.282\\
3.00257506437661	-271742423074.647\\
3.00267506687667	-271699451240.012\\
3.00277506937673	-271656479405.377\\
3.0028750718768	-271613507570.742\\
3.00297507437686	-271570535736.108\\
3.00307507687692	-271527563901.473\\
3.00317507937698	-271484592066.838\\
3.00327508187705	-271441620232.203\\
3.00337508437711	-271398648397.568\\
3.00347508687717	-271355676562.934\\
3.00357508937723	-271312704728.299\\
3.0036750918773	-271269732893.664\\
3.00377509437736	-271226761059.029\\
3.00387509687742	-271183789224.394\\
3.00397509937748	-271140817389.759\\
3.00407510187755	-271097845555.125\\
3.00417510437761	-271054873720.49\\
3.00427510687767	-271011901885.855\\
3.00437510937773	-270968930051.22\\
3.0044751118778	-270925385258.79\\
3.00457511437786	-270882413424.155\\
3.00467511687792	-270839441589.521\\
3.00477511937798	-270796469754.886\\
3.00487512187805	-270753497920.251\\
3.00497512437811	-270710526085.616\\
3.00507512687817	-270667554250.981\\
3.00517512937823	-270624009458.551\\
3.0052751318783	-270581037623.917\\
3.00537513437836	-270538065789.282\\
3.00547513687842	-270495093954.647\\
3.00557513937848	-270452122120.012\\
3.00567514187855	-270408577327.582\\
3.00577514437861	-270365605492.947\\
3.00587514687867	-270322633658.313\\
3.00597514937873	-270279661823.678\\
3.0060751518788	-270236117031.248\\
3.00617515437886	-270193145196.613\\
3.00627515687892	-270150173361.978\\
3.00637515937898	-270106628569.548\\
3.00647516187905	-270063656734.914\\
3.00657516437911	-270020684900.279\\
3.00667516687917	-269977140107.849\\
3.00677516937923	-269934168273.214\\
3.0068751718793	-269891196438.579\\
3.00697517437936	-269847651646.149\\
3.00707517687942	-269804679811.514\\
3.00717517937948	-269761707976.88\\
3.00727518187955	-269718163184.45\\
3.00737518437961	-269675191349.815\\
3.00747518687967	-269632219515.18\\
3.00757518937973	-269588674722.75\\
3.0076751918798	-269545702888.115\\
3.00777519437986	-269502158095.685\\
3.00787519687992	-269459186261.051\\
3.00797519937998	-269415641468.621\\
3.00807520188005	-269372669633.986\\
3.00817520438011	-269329697799.351\\
3.00827520688017	-269286153006.921\\
3.00837520938023	-269243181172.286\\
3.0084752118803	-269199636379.856\\
3.00857521438036	-269156664545.221\\
3.00867521688042	-269113119752.792\\
3.00877521938048	-269070147918.157\\
3.00887522188055	-269026603125.727\\
3.00897522438061	-268983058333.297\\
3.00907522688067	-268940086498.662\\
3.00917522938073	-268896541706.232\\
3.0092752318808	-268853569871.597\\
3.00937523438086	-268810025079.167\\
3.00947523688092	-268767053244.532\\
3.00957523938098	-268723508452.103\\
3.00967524188105	-268679963659.673\\
3.00977524438111	-268636991825.038\\
3.00987524688117	-268593447032.608\\
3.00997524938123	-268550475197.973\\
3.0100752518813	-268506930405.543\\
3.01017525438136	-268463385613.113\\
3.01027525688142	-268420413778.478\\
3.01037525938148	-268376868986.048\\
3.01047526188155	-268333324193.618\\
3.01057526438161	-268290352358.984\\
3.01067526688167	-268246807566.554\\
3.01077526938173	-268203262774.124\\
3.0108752718818	-268160290939.489\\
3.01097527438186	-268116746147.059\\
3.01107527688192	-268073201354.629\\
3.01117527938198	-268029656562.199\\
3.01127528188205	-267986684727.564\\
3.01137528438211	-267943139935.134\\
3.01147528688217	-267899595142.704\\
3.01157528938223	-267856050350.274\\
3.0116752918823	-267813078515.64\\
3.01177529438236	-267769533723.21\\
3.01187529688242	-267725988930.78\\
3.01197529938248	-267682444138.35\\
3.01207530188255	-267638899345.92\\
3.01217530438261	-267595927511.285\\
3.01227530688267	-267552382718.855\\
3.01237530938273	-267508837926.425\\
3.0124753118828	-267465293133.995\\
3.01257531438286	-267421748341.565\\
3.01267531688292	-267378203549.135\\
3.01277531938298	-267334658756.705\\
3.01287532188305	-267291686922.071\\
3.01297532438311	-267248142129.641\\
3.01307532688317	-267204597337.211\\
3.01317532938323	-267161052544.781\\
3.0132753318833	-267117507752.351\\
3.01337533438336	-267073962959.921\\
3.01347533688342	-267030418167.491\\
3.01357533938348	-266986873375.061\\
3.01367534188355	-266943328582.631\\
3.01377534438361	-266899783790.201\\
3.01387534688367	-266856238997.771\\
3.01397534938373	-266812694205.341\\
3.0140753518838	-266769149412.911\\
3.01417535438386	-266725604620.481\\
3.01427535688392	-266682059828.051\\
3.01437535938398	-266638515035.621\\
3.01447536188405	-266594970243.192\\
3.01457536438411	-266551425450.762\\
3.01467536688417	-266507880658.332\\
3.01477536938423	-266464335865.902\\
3.0148753718843	-266420791073.472\\
3.01497537438436	-266377246281.042\\
3.01507537688442	-266333701488.612\\
3.01517537938448	-266290156696.182\\
3.01527538188455	-266246611903.752\\
3.01537538438461	-266203067111.322\\
3.01547538688467	-266158949361.097\\
3.01557538938473	-266115404568.667\\
3.0156753918848	-266071859776.237\\
3.01577539438486	-266028314983.807\\
3.01587539688492	-265984770191.377\\
3.01597539938498	-265941225398.947\\
3.01607540188505	-265897680606.517\\
3.01617540438511	-265853562856.292\\
3.01627540688517	-265810018063.862\\
3.01637540938523	-265766473271.432\\
3.0164754118853	-265722928479.002\\
3.01657541438536	-265679383686.572\\
3.01667541688542	-265635265936.347\\
3.01677541938548	-265591721143.917\\
3.01687542188555	-265548176351.488\\
3.01697542438561	-265504631559.058\\
3.01707542688567	-265461086766.628\\
3.01717542938573	-265416969016.403\\
3.0172754318858	-265373424223.973\\
3.01737543438586	-265329879431.543\\
3.01747543688592	-265285761681.318\\
3.01757543938598	-265242216888.888\\
3.01767544188605	-265198672096.458\\
3.01777544438611	-265155127304.028\\
3.01787544688617	-265111009553.803\\
3.01797544938623	-265067464761.373\\
3.0180754518863	-265023919968.943\\
3.01817545438636	-264979802218.718\\
3.01827545688642	-264936257426.288\\
3.01837545938648	-264892712633.858\\
3.01847546188655	-264848594883.633\\
3.01857546438661	-264805050091.203\\
3.01867546688667	-264760932340.978\\
3.01877546938673	-264717387548.548\\
3.0188754718868	-264673842756.118\\
3.01897547438686	-264629725005.893\\
3.01907547688692	-264586180213.463\\
3.01917547938698	-264542062463.238\\
3.01927548188705	-264498517670.808\\
3.01937548438711	-264454972878.378\\
3.01947548688717	-264410855128.153\\
3.01957548938723	-264367310335.723\\
3.0196754918873	-264323192585.498\\
3.01977549438736	-264279647793.068\\
3.01987549688742	-264235530042.843\\
3.01997549938748	-264191985250.413\\
3.02007550188755	-264147867500.188\\
3.02017550438761	-264104322707.758\\
3.02027550688767	-264060204957.533\\
3.02037550938773	-264016660165.103\\
3.0204755118878	-263972542414.878\\
3.02057551438786	-263928997622.448\\
3.02067551688792	-263884879872.223\\
3.02077551938798	-263841335079.793\\
3.02087552188805	-263797217329.568\\
3.02097552438811	-263753099579.343\\
3.02107552688817	-263709554786.913\\
3.02117552938823	-263665437036.688\\
3.0212755318883	-263621892244.258\\
3.02137553438836	-263577774494.033\\
3.02147553688842	-263534229701.603\\
3.02157553938848	-263490111951.378\\
3.02167554188855	-263445994201.153\\
3.02177554438861	-263402449408.723\\
3.02187554688867	-263358331658.497\\
3.02197554938873	-263314213908.272\\
3.0220755518888	-263270669115.842\\
3.02217555438886	-263226551365.617\\
3.02227555688892	-263182433615.392\\
3.02237555938898	-263138888822.962\\
3.02247556188905	-263094771072.737\\
3.02257556438911	-263050653322.512\\
3.02267556688917	-263007108530.082\\
3.02277556938923	-262962990779.857\\
3.0228755718893	-262918873029.632\\
3.02297557438936	-262874755279.407\\
3.02307557688942	-262831210486.977\\
3.02317557938948	-262787092736.752\\
3.02327558188955	-262742974986.527\\
3.02337558438961	-262698857236.302\\
3.02347558688967	-262655312443.872\\
3.02357558938973	-262611194693.647\\
3.0236755918898	-262567076943.422\\
3.02377559438986	-262522959193.197\\
3.02387559688992	-262479414400.767\\
3.02397559938998	-262435296650.542\\
3.02407560189005	-262391178900.317\\
3.02417560439011	-262347061150.092\\
3.02427560689017	-262302943399.867\\
3.02437560939023	-262258825649.641\\
3.0244756118903	-262215280857.212\\
3.02457561439036	-262171163106.986\\
3.02467561689042	-262127045356.761\\
3.02477561939048	-262082927606.536\\
3.02487562189055	-262038809856.311\\
3.02497562439061	-261994692106.086\\
3.02507562689067	-261950574355.861\\
3.02517562939073	-261906456605.636\\
3.0252756318908	-261862911813.206\\
3.02537563439086	-261818794062.981\\
3.02547563689092	-261774676312.756\\
3.02557563939098	-261730558562.531\\
3.02567564189105	-261686440812.306\\
3.02577564439111	-261642323062.081\\
3.02587564689117	-261598205311.856\\
3.02597564939123	-261554087561.631\\
3.0260756518913	-261509969811.405\\
3.02617565439136	-261465852061.18\\
3.02627565689142	-261421734310.955\\
3.02637565939148	-261377616560.73\\
3.02647566189155	-261333498810.505\\
3.02657566439161	-261289381060.28\\
3.02667566689167	-261245263310.055\\
3.02677566939173	-261201145559.83\\
3.0268756718918	-261157027809.605\\
3.02697567439186	-261112910059.38\\
3.02707567689192	-261068792309.155\\
3.02717567939198	-261024674558.93\\
3.02727568189205	-260980556808.705\\
3.02737568439211	-260936439058.48\\
3.02747568689217	-260892321308.254\\
3.02757568939223	-260848203558.029\\
3.0276756918923	-260804085807.804\\
3.02777569439236	-260759395099.784\\
3.02787569689242	-260715277349.559\\
3.02797569939248	-260671159599.334\\
3.02807570189255	-260627041849.109\\
3.02817570439261	-260582924098.884\\
3.02827570689267	-260538806348.659\\
3.02837570939273	-260494688598.434\\
3.0284757118928	-260450570848.209\\
3.02857571439286	-260405880140.188\\
3.02867571689292	-260361762389.963\\
3.02877571939298	-260317644639.738\\
3.02887572189305	-260273526889.513\\
3.02897572439311	-260229409139.288\\
3.02907572689317	-260185291389.063\\
3.02917572939323	-260140600681.043\\
3.0292757318933	-260096482930.818\\
3.02937573439336	-260052365180.593\\
3.02947573689342	-260008247430.368\\
3.02957573939349	-259963556722.347\\
3.02967574189355	-259919438972.122\\
3.02977574439361	-259875321221.897\\
3.02987574689367	-259831203471.672\\
3.02997574939373	-259786512763.652\\
3.0300757518938	-259742395013.427\\
3.03017575439386	-259698277263.202\\
3.03027575689392	-259654159512.977\\
3.03037575939398	-259609468804.957\\
3.03047576189405	-259565351054.731\\
3.03057576439411	-259521233304.506\\
3.03067576689417	-259476542596.486\\
3.03077576939423	-259432424846.261\\
3.0308757718943	-259388307096.036\\
3.03097577439436	-259344189345.811\\
3.03107577689442	-259299498637.791\\
3.03117577939448	-259255380887.566\\
3.03127578189455	-259210690179.545\\
3.03137578439461	-259166572429.32\\
3.03147578689467	-259122454679.095\\
3.03157578939474	-259077763971.075\\
3.0316757918948	-259033646220.85\\
3.03177579439486	-258989528470.625\\
3.03187579689492	-258944837762.605\\
3.03197579939498	-258900720012.38\\
3.03207580189505	-258856029304.359\\
3.03217580439511	-258811911554.134\\
3.03227580689517	-258767793803.909\\
3.03237580939523	-258723103095.889\\
3.0324758118953	-258678985345.664\\
3.03257581439536	-258634294637.644\\
3.03267581689542	-258590176887.419\\
3.03277581939548	-258545486179.399\\
3.03287582189555	-258501368429.174\\
3.03297582439561	-258456677721.153\\
3.03307582689567	-258412559970.928\\
3.03317582939573	-258367869262.908\\
3.0332758318958	-258323751512.683\\
3.03337583439586	-258279060804.663\\
3.03347583689592	-258234943054.438\\
3.03357583939599	-258190252346.417\\
3.03367584189605	-258146134596.192\\
3.03377584439611	-258101443888.172\\
3.03387584689617	-258057326137.947\\
3.03397584939623	-258012635429.927\\
3.0340758518963	-257968517679.702\\
3.03417585439636	-257923826971.682\\
3.03427585689642	-257879709221.457\\
3.03437585939648	-257835018513.436\\
3.03447586189655	-257790327805.416\\
3.03457586439661	-257746210055.191\\
3.03467586689667	-257701519347.171\\
3.03477586939673	-257657401596.946\\
3.0348758718968	-257612710888.926\\
3.03497587439686	-257568593138.701\\
3.03507587689692	-257523902430.68\\
3.03517587939698	-257479211722.66\\
3.03527588189705	-257435093972.435\\
3.03537588439711	-257390403264.415\\
3.03547588689717	-257345712556.395\\
3.03557588939724	-257301594806.17\\
3.0356758918973	-257256904098.149\\
3.03577589439736	-257212213390.129\\
3.03587589689742	-257168095639.904\\
3.03597589939749	-257123404931.884\\
3.03607590189755	-257078714223.864\\
3.03617590439761	-257034596473.639\\
3.03627590689767	-256989905765.618\\
3.03637590939773	-256945215057.598\\
3.0364759118978	-256901097307.373\\
3.03657591439786	-256856406599.353\\
3.03667591689792	-256811715891.333\\
3.03677591939798	-256767025183.312\\
3.03687592189805	-256722907433.087\\
3.03697592439811	-256678216725.067\\
3.03707592689817	-256633526017.047\\
3.03717592939823	-256588835309.027\\
3.0372759318983	-256544717558.802\\
3.03737593439836	-256500026850.782\\
3.03747593689842	-256455336142.761\\
3.03757593939849	-256410645434.741\\
3.03767594189855	-256365954726.721\\
3.03777594439861	-256321836976.496\\
3.03787594689867	-256277146268.476\\
3.03797594939874	-256232455560.455\\
3.0380759518988	-256187764852.435\\
3.03817595439886	-256143074144.415\\
3.03827595689892	-256098956394.19\\
3.03837595939898	-256054265686.17\\
3.03847596189905	-256009574978.15\\
3.03857596439911	-255964884270.129\\
3.03867596689917	-255920193562.109\\
3.03877596939923	-255875502854.089\\
3.0388759718993	-255830812146.069\\
3.03897597439936	-255786694395.844\\
3.03907597689942	-255742003687.823\\
3.03917597939948	-255697312979.803\\
3.03927598189955	-255652622271.783\\
3.03937598439961	-255607931563.763\\
3.03947598689967	-255563240855.743\\
3.03957598939974	-255518550147.722\\
3.0396759918998	-255473859439.702\\
3.03977599439986	-255429168731.682\\
3.03987599689992	-255384478023.662\\
3.03997599939999	-255339787315.642\\
3.04007600190005	-255295096607.621\\
3.04017600440011	-255250978857.396\\
3.04027600690017	-255206288149.376\\
3.04037600940024	-255161597441.356\\
3.0404760119003	-255116906733.336\\
3.04057601440036	-255072216025.316\\
3.04067601690042	-255027525317.295\\
3.04077601940048	-254982834609.275\\
3.04087602190055	-254938143901.255\\
3.04097602440061	-254893453193.235\\
3.04107602690067	-254848762485.215\\
3.04117602940073	-254804071777.194\\
3.0412760319008	-254759381069.174\\
3.04137603440086	-254714690361.154\\
3.04147603690092	-254669999653.134\\
3.04157603940099	-254624735987.318\\
3.04167604190105	-254580045279.298\\
3.04177604440111	-254535354571.278\\
3.04187604690117	-254490663863.258\\
3.04197604940124	-254445973155.238\\
3.0420760519013	-254401282447.217\\
3.04217605440136	-254356591739.197\\
3.04227605690142	-254311901031.177\\
3.04237605940149	-254267210323.157\\
3.04247606190155	-254222519615.137\\
3.04257606440161	-254177828907.116\\
3.04267606690167	-254133138199.096\\
3.04277606940173	-254087874533.281\\
3.0428760719018	-254043183825.261\\
3.04297607440186	-253998493117.24\\
3.04307607690192	-253953802409.22\\
3.04317607940198	-253909111701.2\\
3.04327608190205	-253864420993.18\\
3.04337608440211	-253819730285.16\\
3.04347608690217	-253774466619.344\\
3.04357608940224	-253729775911.324\\
3.0436760919023	-253685085203.304\\
3.04377609440236	-253640394495.284\\
3.04387609690242	-253595703787.263\\
3.04397609940249	-253551013079.243\\
3.04407610190255	-253505749413.428\\
3.04417610440261	-253461058705.408\\
3.04427610690267	-253416367997.387\\
3.04437610940274	-253371677289.367\\
3.0444761119028	-253326413623.552\\
3.04457611440286	-253281722915.532\\
3.04467611690292	-253237032207.512\\
3.04477611940298	-253192341499.491\\
3.04487612190305	-253147650791.471\\
3.04497612440311	-253102387125.656\\
3.04507612690317	-253057696417.636\\
3.04517612940323	-253013005709.615\\
3.0452761319033	-252967742043.8\\
3.04537613440336	-252923051335.78\\
3.04547613690342	-252878360627.76\\
3.04557613940349	-252833669919.739\\
3.04567614190355	-252788406253.924\\
3.04577614440361	-252743715545.904\\
3.04587614690367	-252699024837.884\\
3.04597614940374	-252653761172.068\\
3.0460761519038	-252609070464.048\\
3.04617615440386	-252564379756.028\\
3.04627615690392	-252519116090.213\\
3.04637615940399	-252474425382.192\\
3.04647616190405	-252429734674.172\\
3.04657616440411	-252384471008.357\\
3.04667616690417	-252339780300.337\\
3.04677616940424	-252295089592.316\\
3.0468761719043	-252249825926.501\\
3.04697617440436	-252205135218.481\\
3.04707617690442	-252160444510.461\\
3.04717617940448	-252115180844.645\\
3.04727618190455	-252070490136.625\\
3.04737618440461	-252025226470.81\\
3.04747618690467	-251980535762.79\\
3.04757618940474	-251935845054.769\\
3.0476761919048	-251890581388.954\\
3.04777619440486	-251845890680.934\\
3.04787619690492	-251800627015.119\\
3.04797619940499	-251755936307.098\\
3.04807620190505	-251710672641.283\\
3.04817620440511	-251665981933.263\\
3.04827620690517	-251621291225.243\\
3.04837620940524	-251576027559.427\\
3.0484762119053	-251531336851.407\\
3.04857621440536	-251486073185.592\\
3.04867621690542	-251441382477.572\\
3.04877621940549	-251396118811.756\\
3.04887622190555	-251351428103.736\\
3.04897622440561	-251306164437.921\\
3.04907622690567	-251261473729.9\\
3.04917622940573	-251216210064.085\\
3.0492762319058	-251171519356.065\\
3.04937623440586	-251126255690.25\\
3.04947623690592	-251081564982.229\\
3.04957623940599	-251036301316.414\\
3.04967624190605	-250991610608.394\\
3.04977624440611	-250946346942.578\\
3.04987624690617	-250901656234.558\\
3.04997624940624	-250856392568.743\\
3.0500762519063	-250811701860.723\\
3.05017625440636	-250766438194.907\\
3.05027625690642	-250721174529.092\\
3.05037625940649	-250676483821.072\\
3.05047626190655	-250631220155.257\\
3.05057626440661	-250586529447.236\\
3.05067626690667	-250541265781.421\\
3.05077626940674	-250496575073.401\\
3.0508762719068	-250451311407.585\\
3.05097627440686	-250406047741.77\\
3.05107627690692	-250361357033.75\\
3.05117627940698	-250316093367.935\\
3.05127628190705	-250271402659.914\\
3.05137628440711	-250226138994.099\\
3.05147628690717	-250180875328.284\\
3.05157628940724	-250136184620.263\\
3.0516762919073	-250090920954.448\\
3.05177629440736	-250045657288.633\\
3.05187629690742	-250000966580.613\\
3.05197629940749	-249955702914.797\\
3.05207630190755	-249910439248.982\\
3.05217630440761	-249865748540.962\\
3.05227630690767	-249820484875.146\\
3.05237630940774	-249775221209.331\\
3.0524763119078	-249730530501.311\\
3.05257631440786	-249685266835.496\\
3.05267631690792	-249640003169.68\\
3.05277631940799	-249595312461.66\\
3.05287632190805	-249550048795.845\\
3.05297632440811	-249504785130.029\\
3.05307632690817	-249460094422.009\\
3.05317632940824	-249414830756.194\\
3.0532763319083	-249369567090.378\\
3.05337633440836	-249324303424.563\\
3.05347633690842	-249279612716.543\\
3.05357633940849	-249234349050.728\\
3.05367634190855	-249189085384.912\\
3.05377634440861	-249143821719.097\\
3.05387634690867	-249099131011.077\\
3.05397634940874	-249053867345.261\\
3.0540763519088	-249008603679.446\\
3.05417635440886	-248963340013.631\\
3.05427635690892	-248918649305.611\\
3.05437635940899	-248873385639.795\\
3.05447636190905	-248828121973.98\\
3.05457636440911	-248782858308.164\\
3.05467636690917	-248738167600.144\\
3.05477636940924	-248692903934.329\\
3.0548763719093	-248647640268.514\\
3.05497637440936	-248602376602.698\\
3.05507637690942	-248557112936.883\\
3.05517637940949	-248511849271.068\\
3.05527638190955	-248467158563.047\\
3.05537638440961	-248421894897.232\\
3.05547638690967	-248376631231.417\\
3.05557638940974	-248331367565.601\\
3.0556763919098	-248286103899.786\\
3.05577639440986	-248240840233.971\\
3.05587639690992	-248196149525.951\\
3.05597639940999	-248150885860.135\\
3.05607640191005	-248105622194.32\\
3.05617640441011	-248060358528.505\\
3.05627640691017	-248015094862.689\\
3.05637640941024	-247969831196.874\\
3.0564764119103	-247924567531.059\\
3.05657641441036	-247879303865.243\\
3.05667641691042	-247834613157.223\\
3.05677641941049	-247789349491.408\\
3.05687642191055	-247744085825.592\\
3.05697642441061	-247698822159.777\\
3.05707642691067	-247653558493.962\\
3.05717642941074	-247608294828.146\\
3.0572764319108	-247563031162.331\\
3.05737643441086	-247517767496.516\\
3.05747643691092	-247472503830.7\\
3.05757643941099	-247427240164.885\\
3.05767644191105	-247381976499.07\\
3.05777644441111	-247336712833.254\\
3.05787644691117	-247291449167.439\\
3.05797644941124	-247246185501.624\\
3.0580764519113	-247200921835.808\\
3.05817645441136	-247155658169.993\\
3.05827645691142	-247110394504.178\\
3.05837645941149	-247065130838.362\\
3.05847646191155	-247020440130.342\\
3.05857646441161	-246975176464.527\\
3.05867646691167	-246929912798.711\\
3.05877646941174	-246884649132.896\\
3.0588764719118	-246839385467.081\\
3.05897647441186	-246793548843.47\\
3.05907647691192	-246748285177.655\\
3.05917647941199	-246703021511.84\\
3.05927648191205	-246657757846.024\\
3.05937648441211	-246612494180.209\\
3.05947648691217	-246567230514.394\\
3.05957648941224	-246521966848.578\\
3.0596764919123	-246476703182.763\\
3.05977649441236	-246431439516.948\\
3.05987649691242	-246386175851.132\\
3.05997649941249	-246340912185.317\\
3.06007650191255	-246295648519.502\\
3.06017650441261	-246250384853.686\\
3.06027650691267	-246205121187.871\\
3.06037650941274	-246159857522.056\\
3.0604765119128	-246114593856.24\\
3.06057651441286	-246069330190.425\\
3.06067651691292	-246024066524.61\\
3.06077651941299	-245978229900.999\\
3.06087652191305	-245932966235.184\\
3.06097652441311	-245887702569.368\\
3.06107652691317	-245842438903.553\\
3.06117652941324	-245797175237.738\\
3.0612765319133	-245751911571.922\\
3.06137653441336	-245706647906.107\\
3.06147653691342	-245661384240.292\\
3.06157653941349	-245616120574.476\\
3.06167654191355	-245570283950.866\\
3.06177654441361	-245525020285.051\\
3.06187654691367	-245479756619.235\\
3.06197654941374	-245434492953.42\\
3.0620765519138	-245389229287.605\\
3.06217655441386	-245343965621.789\\
3.06227655691392	-245298128998.179\\
3.06237655941399	-245252865332.363\\
3.06247656191405	-245207601666.548\\
3.06257656441411	-245162338000.733\\
3.06267656691417	-245117074334.917\\
3.06277656941424	-245071810669.102\\
3.0628765719143	-245025974045.492\\
3.06297657441436	-244980710379.676\\
3.06307657691442	-244935446713.861\\
3.06317657941449	-244890183048.046\\
3.06327658191455	-244844919382.23\\
3.06337658441461	-244799082758.62\\
3.06347658691467	-244753819092.805\\
3.06357658941474	-244708555426.989\\
3.0636765919148	-244663291761.174\\
3.06377659441486	-244617455137.563\\
3.06387659691492	-244572191471.748\\
3.06397659941499	-244526927805.933\\
3.06407660191505	-244481664140.117\\
3.06417660441511	-244435827516.507\\
3.06427660691517	-244390563850.692\\
3.06437660941524	-244345300184.876\\
3.0644766119153	-244300036519.061\\
3.06457661441536	-244254199895.45\\
3.06467661691542	-244208936229.635\\
3.06477661941549	-244163672563.82\\
3.06487662191555	-244118408898.004\\
3.06497662441561	-244072572274.394\\
3.06507662691567	-244027308608.579\\
3.06517662941574	-243982044942.763\\
3.0652766319158	-243936208319.153\\
3.06537663441586	-243890944653.338\\
3.06547663691592	-243845680987.522\\
3.06557663941599	-243799844363.912\\
3.06567664191605	-243754580698.096\\
3.06577664441611	-243709317032.281\\
3.06587664691617	-243664053366.466\\
3.06597664941624	-243618216742.855\\
3.0660766519163	-243572953077.04\\
3.06617665441636	-243527689411.225\\
3.06627665691642	-243481852787.614\\
3.06637665941649	-243436589121.799\\
3.06647666191655	-243390752498.188\\
3.06657666441661	-243345488832.373\\
3.06667666691667	-243300225166.558\\
3.06677666941674	-243254388542.947\\
3.0668766719168	-243209124877.132\\
3.06697667441686	-243163861211.316\\
3.06707667691692	-243118024587.706\\
3.06717667941699	-243072760921.891\\
3.06727668191705	-243027497256.075\\
3.06737668441711	-242981660632.465\\
3.06747668691717	-242936396966.65\\
3.06757668941724	-242890560343.039\\
3.0676766919173	-242845296677.224\\
3.06777669441736	-242800033011.408\\
3.06787669691742	-242754196387.798\\
3.06797669941749	-242708932721.983\\
3.06807670191755	-242663096098.372\\
3.06817670441761	-242617832432.557\\
3.06827670691767	-242571995808.946\\
3.06837670941774	-242526732143.131\\
3.0684767119178	-242481468477.316\\
3.06857671441786	-242435631853.705\\
3.06867671691792	-242390368187.89\\
3.06877671941799	-242344531564.279\\
3.06887672191805	-242299267898.464\\
3.06897672441811	-242253431274.854\\
3.06907672691817	-242208167609.038\\
3.06917672941824	-242162330985.428\\
3.0692767319183	-242117067319.612\\
3.06937673441836	-242071803653.797\\
3.06947673691842	-242025967030.187\\
3.06957673941849	-241980703364.371\\
3.06967674191855	-241934866740.761\\
3.06977674441861	-241889603074.946\\
3.06987674691867	-241843766451.335\\
3.06997674941874	-241798502785.52\\
3.0700767519188	-241752666161.909\\
3.07017675441886	-241707402496.094\\
3.07027675691892	-241661565872.483\\
3.07037675941899	-241616302206.668\\
3.07047676191905	-241570465583.058\\
3.07057676441911	-241525201917.242\\
3.07067676691917	-241479365293.632\\
3.07077676941924	-241434101627.817\\
3.0708767719193	-241388265004.206\\
3.07097677441936	-241342428380.596\\
3.07107677691942	-241297164714.78\\
3.07117677941949	-241251328091.17\\
3.07127678191955	-241206064425.354\\
3.07137678441961	-241160227801.744\\
3.07147678691967	-241114964135.929\\
3.07157678941974	-241069127512.318\\
3.0716767919198	-241023863846.503\\
3.07177679441986	-240978027222.892\\
3.07187679691992	-240932763557.077\\
3.07197679941999	-240886926933.467\\
3.07207680192005	-240841090309.856\\
3.07217680442011	-240795826644.041\\
3.07227680692017	-240749990020.43\\
3.07237680942024	-240704726354.615\\
3.0724768119203	-240658889731.005\\
3.07257681442036	-240613053107.394\\
3.07267681692042	-240567789441.579\\
3.07277681942049	-240521952817.968\\
3.07287682192055	-240476689152.153\\
3.07297682442061	-240430852528.542\\
3.07307682692067	-240385015904.932\\
3.07317682942074	-240339752239.117\\
3.0732768319208	-240293915615.506\\
3.07337683442086	-240248651949.691\\
3.07347683692092	-240202815326.08\\
3.07357683942099	-240156978702.47\\
3.07367684192105	-240111715036.655\\
3.07377684442111	-240065878413.044\\
3.07387684692117	-240020041789.434\\
3.07397684942124	-239974778123.618\\
3.0740768519213	-239928941500.008\\
3.07417685442136	-239883104876.397\\
3.07427685692142	-239837841210.582\\
3.07437685942149	-239792004586.972\\
3.07447686192155	-239746167963.361\\
3.07457686442161	-239700904297.546\\
3.07467686692167	-239655067673.935\\
3.07477686942174	-239609231050.325\\
3.0748768719218	-239563967384.51\\
3.07497687442186	-239518130760.899\\
3.07507687692192	-239472294137.289\\
3.07517687942199	-239427030471.473\\
3.07527688192205	-239381193847.863\\
3.07537688442211	-239335357224.252\\
3.07547688692217	-239290093558.437\\
3.07557688942224	-239244256934.827\\
3.0756768919223	-239198420311.216\\
3.07577689442236	-239153156645.401\\
3.07587689692242	-239107320021.79\\
3.07597689942249	-239061483398.18\\
3.07607690192255	-239015646774.569\\
3.07617690442261	-238970383108.754\\
3.07627690692267	-238924546485.144\\
3.07637690942274	-238878709861.533\\
3.0764769119228	-238832873237.923\\
3.07657691442286	-238787609572.107\\
3.07667691692292	-238741772948.497\\
3.07677691942299	-238695936324.886\\
3.07687692192305	-238650672659.071\\
3.07697692442311	-238604836035.461\\
3.07707692692317	-238558999411.85\\
3.07717692942324	-238513162788.24\\
3.0772769319233	-238467899122.424\\
3.07737693442336	-238422062498.814\\
3.07747693692342	-238376225875.203\\
3.07757693942349	-238330389251.593\\
3.07767694192355	-238284552627.982\\
3.07777694442361	-238239288962.167\\
3.07787694692367	-238193452338.557\\
3.07797694942374	-238147615714.946\\
3.0780769519238	-238101779091.336\\
3.07817695442386	-238056515425.52\\
3.07827695692392	-238010678801.91\\
3.07837695942399	-237964842178.299\\
3.07847696192405	-237919005554.689\\
3.07857696442411	-237873168931.078\\
3.07867696692417	-237827905265.263\\
3.07877696942424	-237782068641.653\\
3.0788769719243	-237736232018.042\\
3.07897697442436	-237690395394.432\\
3.07907697692442	-237644558770.821\\
3.07917697942449	-237598722147.211\\
3.07927698192455	-237553458481.395\\
3.07937698442461	-237507621857.785\\
3.07947698692467	-237461785234.175\\
3.07957698942474	-237415948610.564\\
3.0796769919248	-237370111986.954\\
3.07977699442486	-237324275363.343\\
3.07987699692492	-237279011697.528\\
3.07997699942499	-237233175073.917\\
3.08007700192505	-237187338450.307\\
3.08017700442511	-237141501826.696\\
3.08027700692517	-237095665203.086\\
3.08037700942524	-237049828579.475\\
3.0804770119253	-237003991955.865\\
3.08057701442536	-236958728290.05\\
3.08067701692542	-236912891666.439\\
3.08077701942549	-236867055042.829\\
3.08087702192555	-236821218419.218\\
3.08097702442561	-236775381795.608\\
3.08107702692567	-236729545171.997\\
3.08117702942574	-236683708548.387\\
3.0812770319258	-236637871924.776\\
3.08137703442586	-236592608258.961\\
3.08147703692592	-236546771635.351\\
3.08157703942599	-236500935011.74\\
3.08167704192605	-236455098388.13\\
3.08177704442611	-236409261764.519\\
3.08187704692617	-236363425140.909\\
3.08197704942624	-236317588517.298\\
3.0820770519263	-236271751893.688\\
3.08217705442636	-236225915270.077\\
3.08227705692642	-236180078646.467\\
3.08237705942649	-236134814980.652\\
3.08247706192655	-236088978357.041\\
3.08257706442661	-236043141733.431\\
3.08267706692667	-235997305109.82\\
3.08277706942674	-235951468486.21\\
3.0828770719268	-235905631862.599\\
3.08297707442686	-235859795238.989\\
3.08307707692692	-235813958615.378\\
3.08317707942699	-235768121991.768\\
3.08327708192705	-235722285368.157\\
3.08337708442711	-235676448744.547\\
3.08347708692717	-235630612120.936\\
3.08357708942724	-235584775497.326\\
3.0836770919273	-235538938873.716\\
3.08377709442736	-235493102250.105\\
3.08387709692742	-235447265626.495\\
3.08397709942749	-235401429002.884\\
3.08407710192755	-235355592379.274\\
3.08417710442761	-235309755755.663\\
3.08427710692767	-235263919132.053\\
3.08437710942774	-235218655466.237\\
3.0844771119278	-235172818842.627\\
3.08457711442786	-235126982219.016\\
3.08467711692792	-235081145595.406\\
3.08477711942799	-235035308971.796\\
3.08487712192805	-234989472348.185\\
3.08497712442811	-234943635724.575\\
3.08507712692817	-234897799100.964\\
3.08517712942824	-234851962477.354\\
3.0852771319283	-234806125853.743\\
3.08537713442836	-234760289230.133\\
3.08547713692842	-234714452606.522\\
3.08557713942849	-234668615982.912\\
3.08567714192855	-234622779359.301\\
3.08577714442861	-234576942735.691\\
3.08587714692867	-234531106112.08\\
3.08597714942874	-234484696530.675\\
3.0860771519288	-234438859907.064\\
3.08617715442886	-234393023283.454\\
3.08627715692892	-234347186659.843\\
3.08637715942899	-234301350036.233\\
3.08647716192905	-234255513412.622\\
3.08657716442911	-234209676789.012\\
3.08667716692917	-234163840165.402\\
3.08677716942924	-234118003541.791\\
3.0868771719293	-234072166918.181\\
3.08697717442936	-234026330294.57\\
3.08707717692942	-233980493670.96\\
3.08717717942949	-233934657047.349\\
3.08727718192955	-233888820423.739\\
3.08737718442961	-233842983800.128\\
3.08747718692967	-233797147176.518\\
3.08757718942974	-233751310552.907\\
3.0876771919298	-233705473929.297\\
3.08777719442986	-233659637305.686\\
3.08787719692992	-233613800682.076\\
3.08797719942999	-233567964058.465\\
3.08807720193005	-233521554477.06\\
3.08817720443011	-233475717853.449\\
3.08827720693017	-233429881229.839\\
3.08837720943024	-233384044606.228\\
3.0884772119303	-233338207982.618\\
3.08857721443036	-233292371359.008\\
3.08867721693042	-233246534735.397\\
3.08877721943049	-233200698111.787\\
3.08887722193055	-233154861488.176\\
3.08897722443061	-233109024864.566\\
3.08907722693067	-233063188240.955\\
3.08917722943074	-233016778659.55\\
3.0892772319308	-232970942035.939\\
3.08937723443086	-232925105412.329\\
3.08947723693092	-232879268788.718\\
3.08957723943099	-232833432165.108\\
3.08967724193105	-232787595541.497\\
3.08977724443111	-232741758917.887\\
3.08987724693117	-232695922294.276\\
3.08997724943124	-232650085670.666\\
3.0900772519313	-232603676089.26\\
3.09017725443136	-232557839465.65\\
3.09027725693142	-232512002842.039\\
3.09037725943149	-232466166218.429\\
3.09047726193155	-232420329594.818\\
3.09057726443161	-232374492971.208\\
3.09067726693167	-232328656347.598\\
3.09077726943174	-232282246766.192\\
3.0908772719318	-232236410142.581\\
3.09097727443186	-232190573518.971\\
3.09107727693192	-232144736895.361\\
3.09117727943199	-232098900271.75\\
3.09127728193205	-232053063648.14\\
3.09137728443211	-232007227024.529\\
3.09147728693217	-231960817443.124\\
3.09157728943224	-231914980819.513\\
3.0916772919323	-231869144195.903\\
3.09177729443236	-231823307572.292\\
3.09187729693242	-231777470948.682\\
3.09197729943249	-231731634325.071\\
3.09207730193255	-231685224743.666\\
3.09217730443261	-231639388120.055\\
3.09227730693267	-231593551496.445\\
3.09237730943274	-231547714872.834\\
3.0924773119328	-231501878249.224\\
3.09257731443286	-231456041625.613\\
3.09267731693292	-231409632044.208\\
3.09277731943299	-231363795420.597\\
3.09287732193305	-231317958796.987\\
3.09297732443311	-231272122173.376\\
3.09307732693317	-231226285549.766\\
3.09317732943324	-231179875968.36\\
3.0932773319333	-231134039344.75\\
3.09337733443336	-231088202721.139\\
3.09347733693342	-231042366097.529\\
3.09357733943349	-230996529473.918\\
3.09367734193355	-230950119892.513\\
3.09377734443361	-230904283268.902\\
3.09387734693367	-230858446645.292\\
3.09397734943374	-230812610021.681\\
3.0940773519338	-230766773398.071\\
3.09417735443386	-230720363816.665\\
3.09427735693392	-230674527193.055\\
3.09437735943399	-230628690569.444\\
3.09447736193405	-230582853945.834\\
3.09457736443411	-230536444364.428\\
3.09467736693417	-230490607740.818\\
3.09477736943424	-230444771117.207\\
3.0948773719343	-230398934493.597\\
3.09497737443436	-230353097869.986\\
3.09507737693442	-230306688288.581\\
3.09517737943449	-230260851664.97\\
3.09527738193455	-230215015041.36\\
3.09537738443461	-230169178417.749\\
3.09547738693467	-230122768836.344\\
3.09557738943474	-230076932212.733\\
3.0956773919348	-230031095589.123\\
3.09577739443486	-229985258965.512\\
3.09587739693492	-229938849384.107\\
3.09597739943499	-229893012760.496\\
3.09607740193505	-229847176136.886\\
3.09617740443511	-229801339513.275\\
3.09627740693517	-229754929931.87\\
3.09637740943524	-229709093308.259\\
3.0964774119353	-229663256684.649\\
3.09657741443536	-229617420061.038\\
3.09667741693542	-229571010479.633\\
3.09677741943549	-229525173856.022\\
3.09687742193555	-229479337232.412\\
3.09697742443561	-229433500608.801\\
3.09707742693567	-229387091027.396\\
3.09717742943574	-229341254403.785\\
3.0972774319358	-229295417780.175\\
3.09737743443586	-229249008198.769\\
3.09747743693592	-229203171575.159\\
3.09757743943599	-229157334951.548\\
3.09767744193605	-229111498327.938\\
3.09777744443611	-229065088746.532\\
3.09787744693617	-229019252122.922\\
3.09797744943624	-228973415499.311\\
3.0980774519363	-228927005917.906\\
3.09817745443636	-228881169294.295\\
3.09827745693642	-228835332670.685\\
3.09837745943649	-228789496047.074\\
3.09847746193655	-228743086465.669\\
3.09857746443661	-228697249842.058\\
3.09867746693667	-228651413218.448\\
3.09877746943674	-228605003637.042\\
3.0988774719368	-228559167013.432\\
3.09897747443686	-228513330389.821\\
3.09907747693692	-228466920808.416\\
3.09917747943699	-228421084184.805\\
3.09927748193705	-228375247561.195\\
3.09937748443711	-228329410937.584\\
3.09947748693717	-228283001356.179\\
3.09957748943724	-228237164732.568\\
3.0996774919373	-228191328108.958\\
3.09977749443736	-228144918527.552\\
3.09987749693742	-228099081903.942\\
3.09997749943749	-228053245280.331\\
3.10007750193755	-228006835698.926\\
3.10017750443761	-227960999075.315\\
3.10027750693767	-227915162451.705\\
3.10037750943774	-227868752870.299\\
3.1004775119378	-227822916246.689\\
3.10057751443786	-227777079623.078\\
3.10067751693792	-227730670041.673\\
3.10077751943799	-227684833418.062\\
3.10087752193805	-227638996794.452\\
3.10097752443811	-227592587213.046\\
3.10107752693817	-227546750589.436\\
3.10117752943824	-227500913965.825\\
3.1012775319383	-227454504384.42\\
3.10137753443836	-227408667760.809\\
3.10147753693842	-227362831137.199\\
3.10157753943849	-227316421555.793\\
3.10167754193855	-227270584932.183\\
3.10177754443861	-227224748308.572\\
3.10187754693867	-227178338727.167\\
3.10197754943874	-227132502103.556\\
3.1020775519388	-227086665479.946\\
3.10217755443886	-227040255898.54\\
3.10227755693892	-226994419274.93\\
3.10237755943899	-226948582651.319\\
3.10247756193905	-226902173069.913\\
3.10257756443911	-226856336446.303\\
3.10267756693917	-226809926864.897\\
3.10277756943924	-226764090241.287\\
3.1028775719393	-226718253617.676\\
3.10297757443936	-226671844036.271\\
3.10307757693942	-226626007412.66\\
3.10317757943949	-226580170789.05\\
3.10327758193955	-226533761207.644\\
3.10337758443961	-226487924584.034\\
3.10347758693967	-226441515002.628\\
3.10357758943974	-226395678379.018\\
3.1036775919398	-226349841755.407\\
3.10377759443986	-226303432174.002\\
3.10387759693992	-226257595550.391\\
3.10397759943999	-226211758926.781\\
3.10407760194005	-226165349345.375\\
3.10417760444011	-226119512721.765\\
3.10427760694017	-226073103140.359\\
3.10437760944024	-226027266516.749\\
3.1044776119403	-225981429893.138\\
3.10457761444036	-225935020311.733\\
3.10467761694042	-225889183688.122\\
3.10477761944049	-225843347064.512\\
3.10487762194055	-225796937483.106\\
3.10497762444061	-225751100859.496\\
3.10507762694067	-225704691278.09\\
3.10517762944074	-225658854654.48\\
3.1052776319408	-225613018030.869\\
3.10537763444086	-225566608449.464\\
3.10547763694092	-225520771825.853\\
3.10557763944099	-225474362244.447\\
3.10567764194105	-225428525620.837\\
3.10577764444111	-225382688997.227\\
3.10587764694117	-225336279415.821\\
3.10597764944124	-225290442792.21\\
3.1060776519413	-225244033210.805\\
3.10617765444136	-225198196587.194\\
3.10627765694142	-225152359963.584\\
3.10637765944149	-225105950382.178\\
3.10647766194155	-225060113758.568\\
3.10657766444161	-225013704177.162\\
3.10667766694167	-224967867553.552\\
3.10677766944174	-224921457972.146\\
3.1068776719418	-224875621348.536\\
3.10697767444186	-224829784724.925\\
3.10707767694192	-224783375143.52\\
3.10717767944199	-224737538519.909\\
3.10727768194205	-224691128938.504\\
3.10737768444211	-224645292314.893\\
3.10747768694217	-224599455691.283\\
3.10757768944224	-224553046109.877\\
3.1076776919423	-224507209486.267\\
3.10777769444236	-224460799904.861\\
3.10787769694242	-224414963281.251\\
3.10797769944249	-224368553699.845\\
3.10807770194255	-224322717076.235\\
3.10817770444261	-224276880452.624\\
3.10827770694267	-224230470871.218\\
3.10837770944274	-224184634247.608\\
3.1084777119428	-224138224666.202\\
3.10857771444286	-224092388042.592\\
3.10867771694292	-224045978461.186\\
3.10877771944299	-224000141837.576\\
3.10887772194305	-223954305213.965\\
3.10897772444311	-223907895632.56\\
3.10907772694317	-223862059008.949\\
3.10917772944324	-223815649427.544\\
3.1092777319433	-223769812803.933\\
3.10937773444336	-223723403222.528\\
3.10947773694342	-223677566598.917\\
3.10957773944349	-223631729975.307\\
3.10967774194355	-223585320393.901\\
3.10977774444361	-223539483770.291\\
3.10987774694367	-223493074188.885\\
3.10997774944374	-223447237565.275\\
3.1100777519438	-223400827983.869\\
3.11017775444386	-223354991360.259\\
3.11027775694392	-223308581778.853\\
3.11037775944399	-223262745155.242\\
3.11047776194405	-223216908531.632\\
3.11057776444411	-223170498950.226\\
3.11067776694417	-223124662326.616\\
3.11077776944424	-223078252745.21\\
3.1108777719443	-223032416121.6\\
3.11097777444436	-222986006540.194\\
3.11107777694442	-222940169916.584\\
3.11117777944449	-222893760335.178\\
3.11127778194455	-222847923711.568\\
3.11137778444461	-222801514130.162\\
3.11147778694467	-222755677506.552\\
3.11157778944474	-222709840882.941\\
3.1116777919448	-222663431301.536\\
3.11177779444486	-222617594677.925\\
3.11187779694492	-222571185096.52\\
3.11197779944499	-222525348472.909\\
3.11207780194505	-222478938891.504\\
3.11217780444511	-222433102267.893\\
3.11227780694517	-222386692686.487\\
3.11237780944524	-222340856062.877\\
3.1124778119453	-222294446481.471\\
3.11257781444536	-222248609857.861\\
3.11267781694542	-222202200276.455\\
3.11277781944549	-222156363652.845\\
3.11287782194555	-222110527029.234\\
3.11297782444561	-222064117447.829\\
3.11307782694567	-222018280824.218\\
3.11317782944574	-221971871242.813\\
3.1132778319458	-221926034619.202\\
3.11337783444586	-221879625037.797\\
3.11347783694592	-221833788414.186\\
3.11357783944599	-221787378832.781\\
3.11367784194605	-221741542209.17\\
3.11377784444611	-221695132627.765\\
3.11387784694617	-221649296004.154\\
3.11397784944624	-221602886422.749\\
3.1140778519463	-221557049799.138\\
3.11417785444636	-221510640217.732\\
3.11427785694642	-221464803594.122\\
3.11437785944649	-221418394012.716\\
3.11447786194655	-221372557389.106\\
3.11457786444661	-221326720765.495\\
3.11467786694667	-221280311184.09\\
3.11477786944674	-221234474560.479\\
3.1148778719468	-221188064979.074\\
3.11497787444686	-221142228355.463\\
3.11507787694692	-221095818774.058\\
3.11517787944699	-221049982150.447\\
3.11527788194705	-221003572569.042\\
3.11537788444711	-220957735945.431\\
3.11547788694717	-220911326364.026\\
3.11557788944724	-220865489740.415\\
3.1156778919473	-220819080159.01\\
3.11577789444736	-220773243535.399\\
3.11587789694742	-220726833953.993\\
3.11597789944749	-220680997330.383\\
3.11607790194755	-220634587748.977\\
3.11617790444761	-220588751125.367\\
3.11627790694767	-220542341543.961\\
3.11637790944774	-220496504920.351\\
3.1164779119478	-220450095338.945\\
3.11657791444786	-220404258715.335\\
3.11667791694792	-220357849133.929\\
3.11677791944799	-220312012510.319\\
3.11687792194805	-220265602928.913\\
3.11697792444811	-220219766305.303\\
3.11707792694817	-220173356723.897\\
3.11717792944824	-220127520100.287\\
3.1172779319483	-220081110518.881\\
3.11737793444836	-220035273895.271\\
3.11747793694842	-219989437271.66\\
3.11757793944849	-219943027690.255\\
3.11767794194855	-219897191066.644\\
3.11777794444861	-219850781485.238\\
3.11787794694867	-219804944861.628\\
3.11797794944874	-219758535280.222\\
3.1180779519488	-219712698656.612\\
3.11817795444886	-219666289075.206\\
3.11827795694892	-219620452451.596\\
3.11837795944899	-219574042870.19\\
3.11847796194905	-219528206246.58\\
3.11857796444911	-219481796665.174\\
3.11867796694917	-219435960041.564\\
3.11877796944924	-219389550460.158\\
3.1188779719493	-219343713836.548\\
3.11897797444936	-219297304255.142\\
3.11907797694942	-219251467631.532\\
3.11917797944949	-219205058050.126\\
3.11927798194955	-219159221426.516\\
3.11937798444961	-219112811845.11\\
3.11947798694967	-219066975221.499\\
3.11957798944974	-219020565640.094\\
3.1196779919498	-218974729016.483\\
3.11977799444986	-218928319435.078\\
3.11987799694992	-218882482811.467\\
3.11997799944999	-218836073230.062\\
3.12007800195005	-218790236606.451\\
3.12017800445011	-218743827025.046\\
3.12027800695017	-218697990401.435\\
3.12037800945024	-218651580820.03\\
3.1204780119503	-218605744196.419\\
3.12057801445036	-218559334615.014\\
3.12067801695042	-218513497991.403\\
3.12077801945049	-218467088409.997\\
3.12087802195055	-218421251786.387\\
3.12097802445061	-218374842204.981\\
3.12107802695067	-218329005581.371\\
3.12117802945074	-218282595999.965\\
3.1212780319508	-218236759376.355\\
3.12137803445086	-218190349794.949\\
3.12147803695092	-218144513171.339\\
3.12157803945099	-218098103589.933\\
3.12167804195105	-218052266966.323\\
3.12177804445111	-218005857384.917\\
3.12187804695117	-217960020761.307\\
3.12197804945124	-217913611179.901\\
3.1220780519513	-217867774556.291\\
3.12217805445136	-217821364974.885\\
3.12227805695142	-217775528351.275\\
3.12237805945149	-217729118769.869\\
3.12247806195155	-217683282146.259\\
3.12257806445161	-217636872564.853\\
3.12267806695167	-217591035941.242\\
3.12277806945174	-217544626359.837\\
3.1228780719518	-217498789736.226\\
3.12297807445186	-217452380154.821\\
3.12307807695192	-217406543531.21\\
3.12317807945199	-217360133949.805\\
3.12327808195205	-217314297326.194\\
3.12337808445211	-217267887744.789\\
3.12347808695217	-217222051121.178\\
3.12357808945224	-217175641539.773\\
3.1236780919523	-217129804916.162\\
3.12377809445236	-217083395334.757\\
3.12387809695242	-217037558711.146\\
3.12397809945249	-216991149129.741\\
3.12407810195255	-216945312506.13\\
3.12417810445261	-216898902924.724\\
3.12427810695267	-216853066301.114\\
3.12437810945274	-216806656719.708\\
3.1244781119528	-216760820096.098\\
3.12457811445286	-216714410514.692\\
3.12467811695292	-216668573891.082\\
3.12477811945299	-216622164309.676\\
3.12487812195305	-216576327686.066\\
3.12497812445311	-216529918104.66\\
3.12507812695317	-216484081481.05\\
3.12517812945324	-216437671899.644\\
3.1252781319533	-216391835276.034\\
3.12537813445336	-216345425694.628\\
3.12547813695342	-216299589071.018\\
3.12557813945349	-216253179489.612\\
3.12567814195355	-216207342866.002\\
3.12577814445361	-216160933284.596\\
3.12587814695367	-216115096660.985\\
3.12597814945374	-216069260037.375\\
3.1260781519538	-216022850455.969\\
3.12617815445386	-215977013832.359\\
3.12627815695392	-215930604250.953\\
3.12637815945399	-215884767627.343\\
3.12647816195405	-215838358045.937\\
3.12657816445411	-215792521422.327\\
3.12667816695417	-215746111840.921\\
3.12677816945424	-215700275217.311\\
3.1268781719543	-215653865635.905\\
3.12697817445436	-215608029012.295\\
3.12707817695442	-215561619430.889\\
3.12717817945449	-215515782807.279\\
3.12727818195455	-215469373225.873\\
3.12737818445461	-215423536602.263\\
3.12747818695467	-215377127020.857\\
3.12757818945474	-215331290397.247\\
3.1276781919548	-215284880815.841\\
3.12777819445486	-215239044192.23\\
3.12787819695492	-215192634610.825\\
3.12797819945499	-215146797987.214\\
3.12807820195505	-215100388405.809\\
3.12817820445511	-215054551782.198\\
3.12827820695517	-215008142200.793\\
3.12837820945524	-214962305577.182\\
3.1284782119553	-214915895995.777\\
3.12857821445536	-214870059372.166\\
3.12867821695542	-214823649790.761\\
3.12877821945549	-214777813167.15\\
3.12887822195555	-214731403585.745\\
3.12897822445561	-214685566962.134\\
3.12907822695567	-214639157380.728\\
3.12917822945574	-214593320757.118\\
3.1292782319558	-214546911175.712\\
3.12937823445586	-214501074552.102\\
3.12947823695592	-214455237928.491\\
3.12957823945599	-214408828347.086\\
3.12967824195605	-214362991723.475\\
3.12977824445611	-214316582142.07\\
3.12987824695617	-214270745518.459\\
3.12997824945624	-214224335937.054\\
3.1300782519563	-214178499313.443\\
3.13017825445636	-214132089732.038\\
3.13027825695642	-214086253108.427\\
3.13037825945649	-214039843527.022\\
3.13047826195655	-213994006903.411\\
3.13057826445661	-213947597322.006\\
3.13067826695667	-213901760698.395\\
3.13077826945674	-213855351116.99\\
3.1308782719568	-213809514493.379\\
3.13097827445686	-213763104911.973\\
3.13107827695692	-213717268288.363\\
3.13117827945699	-213671431664.753\\
3.13127828195705	-213625022083.347\\
3.13137828445711	-213579185459.736\\
3.13147828695717	-213532775878.331\\
3.13157828945724	-213486939254.72\\
3.1316782919573	-213440529673.315\\
3.13177829445736	-213394693049.704\\
3.13187829695742	-213348283468.299\\
3.13197829945749	-213302446844.688\\
3.13207830195755	-213256037263.283\\
3.13217830445761	-213210200639.672\\
3.13227830695767	-213163791058.267\\
3.13237830945774	-213117954434.656\\
3.1324783119578	-213071544853.251\\
3.13257831445786	-213025708229.64\\
3.13267831695792	-212979871606.03\\
3.13277831945799	-212933462024.624\\
3.13287832195805	-212887625401.014\\
3.13297832445811	-212841215819.608\\
3.13307832695817	-212795379195.997\\
3.13317832945824	-212748969614.592\\
3.1332783319583	-212703132990.981\\
3.13337833445836	-212656723409.576\\
3.13347833695842	-212610886785.965\\
3.13357833945849	-212564477204.56\\
3.13367834195855	-212518640580.949\\
3.13377834445861	-212472803957.339\\
3.13387834695867	-212426394375.933\\
3.13397834945874	-212380557752.323\\
3.1340783519588	-212334148170.917\\
3.13417835445886	-212288311547.307\\
3.13427835695892	-212241901965.901\\
3.13437835945899	-212196065342.291\\
3.13447836195905	-212149655760.885\\
3.13457836445911	-212103819137.275\\
3.13467836695917	-212057982513.664\\
3.13477836945924	-212011572932.259\\
3.1348783719593	-211965736308.648\\
3.13497837445936	-211919326727.242\\
3.13507837695942	-211873490103.632\\
3.13517837945949	-211827080522.226\\
3.13527838195955	-211781243898.616\\
3.13537838445961	-211735407275.005\\
3.13547838695967	-211688997693.6\\
3.13557838945974	-211643161069.989\\
3.1356783919598	-211596751488.584\\
3.13577839445986	-211550914864.973\\
3.13587839695992	-211504505283.568\\
3.13597839945999	-211458668659.957\\
3.13607840196005	-211412832036.347\\
3.13617840446011	-211366422454.941\\
3.13627840696017	-211320585831.331\\
3.13637840946024	-211274176249.925\\
3.1364784119603	-211228339626.315\\
3.13657841446036	-211181930044.909\\
3.13667841696042	-211136093421.299\\
3.13677841946049	-211090256797.688\\
3.13687842196055	-211043847216.283\\
3.13697842446061	-210998010592.672\\
3.13707842696067	-210951601011.267\\
3.13717842946074	-210905764387.656\\
3.1372784319608	-210859354806.25\\
3.13737843446086	-210813518182.64\\
3.13747843696092	-210767681559.03\\
3.13757843946099	-210721271977.624\\
3.13767844196105	-210675435354.013\\
3.13777844446111	-210629025772.608\\
3.13787844696117	-210583189148.997\\
3.13797844946124	-210537352525.387\\
3.1380784519613	-210490942943.981\\
3.13817845446136	-210445106320.371\\
3.13827845696142	-210398696738.965\\
3.13837845946149	-210352860115.355\\
3.13847846196155	-210307023491.744\\
3.13857846446161	-210260613910.339\\
3.13867846696167	-210214777286.728\\
3.13877846946174	-210168367705.323\\
3.1388784719618	-210122531081.712\\
3.13897847446186	-210076694458.102\\
3.13907847696192	-210030284876.696\\
3.13917847946199	-209984448253.086\\
3.13927848196205	-209938038671.68\\
3.13937848446211	-209892202048.07\\
3.13947848696217	-209846365424.459\\
3.13957848946224	-209799955843.054\\
3.1396784919623	-209754119219.443\\
3.13977849446236	-209707709638.037\\
3.13987849696242	-209661873014.427\\
3.13997849946249	-209616036390.817\\
3.14007850196255	-209569626809.411\\
3.14017850446261	-209523790185.8\\
3.14027850696267	-209477380604.395\\
3.14037850946274	-209431543980.784\\
3.1404785119628	-209385707357.174\\
3.14057851446286	-209339297775.768\\
3.14067851696292	-209293461152.158\\
3.14077851946299	-209247624528.547\\
3.14087852196305	-209201214947.142\\
3.14097852446311	-209155378323.531\\
3.14107852696317	-209108968742.126\\
3.14117852946324	-209063132118.515\\
3.1412785319633	-209017295494.905\\
3.14137853446336	-208970885913.499\\
3.14147853696342	-208925049289.889\\
3.14157853946349	-208879212666.278\\
3.14167854196355	-208832803084.873\\
3.14177854446361	-208786966461.262\\
3.14187854696367	-208741129837.652\\
3.14197854946374	-208694720256.246\\
3.1420785519638	-208648883632.636\\
3.14217855446386	-208602474051.23\\
3.14227855696392	-208556637427.62\\
3.14237855946399	-208510800804.009\\
3.14247856196405	-208464391222.604\\
3.14257856446411	-208418554598.993\\
3.14267856696417	-208372717975.383\\
3.14277856946424	-208326308393.977\\
3.1428785719643	-208280471770.367\\
3.14297857446436	-208234635146.756\\
3.14307857696442	-208188225565.351\\
3.14317857946449	-208142388941.74\\
3.14327858196455	-208096552318.13\\
3.14337858446461	-208050142736.724\\
3.14347858696467	-208004306113.114\\
3.14357858946474	-207958469489.503\\
3.1436785919648	-207912059908.097\\
3.14377859446486	-207866223284.487\\
3.14387859696492	-207820386660.877\\
3.14397859946499	-207773977079.471\\
3.14407860196505	-207728140455.86\\
3.14417860446511	-207682303832.25\\
3.14427860696517	-207635894250.844\\
3.14437860946524	-207590057627.234\\
3.1444786119653	-207544221003.624\\
3.14457861446536	-207497811422.218\\
3.14467861696542	-207451974798.607\\
3.14477861946549	-207406138174.997\\
3.14487862196555	-207359728593.591\\
3.14497862446561	-207313891969.981\\
3.14507862696567	-207268055346.37\\
3.14517862946574	-207221645764.965\\
3.1452786319658	-207175809141.354\\
3.14537863446586	-207129972517.744\\
3.14547863696592	-207083562936.338\\
3.14557863946599	-207037726312.728\\
3.14567864196605	-206991889689.117\\
3.14577864446611	-206946053065.507\\
3.14587864696617	-206899643484.101\\
3.14597864946624	-206853806860.491\\
3.1460786519663	-206807970236.88\\
3.14617865446636	-206761560655.475\\
3.14627865696642	-206715724031.864\\
3.14637865946649	-206669887408.254\\
3.14647866196655	-206623477826.848\\
3.14657866446661	-206577641203.238\\
3.14667866696667	-206531804579.627\\
3.14677866946674	-206485967956.017\\
3.1468786719668	-206439558374.611\\
3.14697867446686	-206393721751.001\\
3.14707867696692	-206347885127.39\\
3.14717867946699	-206301475545.985\\
3.14727868196705	-206255638922.374\\
3.14737868446711	-206209802298.764\\
3.14747868696717	-206163965675.153\\
3.14757868946724	-206117556093.748\\
3.1476786919673	-206071719470.137\\
3.14777869446736	-206025882846.527\\
3.14787869696742	-205979473265.121\\
3.14797869946749	-205933636641.511\\
3.14807870196755	-205887800017.9\\
3.14817870446761	-205841963394.29\\
3.14827870696767	-205795553812.884\\
3.14837870946774	-205749717189.274\\
3.1484787119678	-205703880565.663\\
3.14857871446786	-205658043942.053\\
3.14867871696792	-205611634360.647\\
3.14877871946799	-205565797737.037\\
3.14887872196805	-205519961113.426\\
3.14897872446811	-205474124489.816\\
3.14907872696817	-205427714908.41\\
3.14917872946824	-205381878284.8\\
3.1492787319683	-205336041661.189\\
3.14937873446836	-205290205037.579\\
3.14947873696842	-205243795456.173\\
3.14957873946849	-205197958832.563\\
3.14967874196855	-205152122208.952\\
3.14977874446861	-205106285585.342\\
3.14987874696867	-205059876003.936\\
3.14997874946874	-205014039380.326\\
3.1500787519688	-204968202756.715\\
3.15017875446886	-204922366133.105\\
3.15027875696892	-204876529509.494\\
3.15037875946899	-204830119928.089\\
3.15047876196905	-204784283304.478\\
3.15057876446911	-204738446680.868\\
3.15067876696917	-204692610057.257\\
3.15077876946924	-204646200475.852\\
3.1508787719693	-204600363852.241\\
3.15097877446936	-204554527228.631\\
3.15107877696942	-204508690605.02\\
3.15117877946949	-204462853981.41\\
3.15127878196955	-204416444400.004\\
3.15137878446961	-204370607776.394\\
3.15147878696967	-204324771152.783\\
3.15157878946974	-204278934529.173\\
3.1516787919698	-204233097905.562\\
3.15177879446986	-204186688324.157\\
3.15187879696992	-204140851700.546\\
3.15197879946999	-204095015076.936\\
3.15207880197005	-204049178453.325\\
3.15217880447011	-204003341829.715\\
3.15227880697017	-203956932248.309\\
3.15237880947024	-203911095624.699\\
3.1524788119703	-203865259001.088\\
3.15257881447036	-203819422377.478\\
3.15267881697042	-203773585753.868\\
3.15277881947049	-203727749130.257\\
3.15287882197055	-203681339548.851\\
3.15297882447061	-203635502925.241\\
3.15307882697067	-203589666301.631\\
3.15317882947074	-203543829678.02\\
3.1532788319708	-203497993054.41\\
3.15337883447086	-203452156430.799\\
3.15347883697092	-203405746849.394\\
3.15357883947099	-203359910225.783\\
3.15367884197105	-203314073602.173\\
3.15377884447111	-203268236978.562\\
3.15387884697117	-203222400354.952\\
3.15397884947124	-203176563731.341\\
3.1540788519713	-203130154149.936\\
3.15417885447136	-203084317526.325\\
3.15427885697142	-203038480902.715\\
3.15437885947149	-202992644279.104\\
3.15447886197155	-202946807655.494\\
3.15457886447161	-202900971031.883\\
3.15467886697167	-202855134408.273\\
3.15477886947174	-202808724826.867\\
3.1548788719718	-202762888203.257\\
3.15497887447186	-202717051579.646\\
3.15507887697192	-202671214956.036\\
3.15517887947199	-202625378332.425\\
3.15527888197205	-202579541708.815\\
3.15537888447211	-202533705085.204\\
3.15547888697217	-202487868461.594\\
3.15557888947224	-202441458880.188\\
3.1556788919723	-202395622256.578\\
3.15577889447236	-202349785632.967\\
3.15587889697242	-202303949009.357\\
3.15597889947249	-202258112385.747\\
3.15607890197255	-202212275762.136\\
3.15617890447261	-202166439138.526\\
3.15627890697267	-202120602514.915\\
3.15637890947274	-202074765891.305\\
3.1564789119728	-202028929267.694\\
3.15657891447286	-201982519686.289\\
3.15667891697292	-201936683062.678\\
3.15677891947299	-201890846439.068\\
3.15687892197305	-201845009815.457\\
3.15697892447311	-201799173191.847\\
3.15707892697317	-201753336568.236\\
3.15717892947324	-201707499944.626\\
3.1572789319733	-201661663321.015\\
3.15737893447336	-201615826697.405\\
3.15747893697342	-201569990073.794\\
3.15757893947349	-201524153450.184\\
3.15767894197355	-201478316826.573\\
3.15777894447361	-201431907245.168\\
3.15787894697367	-201386070621.557\\
3.15797894947374	-201340233997.947\\
3.1580789519738	-201294397374.336\\
3.15817895447386	-201248560750.726\\
3.15827895697392	-201202724127.116\\
3.15837895947399	-201156887503.505\\
3.15847896197405	-201111050879.895\\
3.15857896447411	-201065214256.284\\
3.15867896697417	-201019377632.674\\
3.15877896947424	-200973541009.063\\
3.1588789719743	-200927704385.453\\
3.15897897447436	-200881867761.842\\
3.15907897697442	-200836031138.232\\
3.15917897947449	-200790194514.621\\
3.15927898197455	-200744357891.011\\
3.15937898447461	-200698521267.4\\
3.15947898697467	-200652684643.79\\
3.15957898947474	-200606848020.179\\
3.1596789919748	-200561011396.569\\
3.15977899447486	-200515174772.959\\
3.15987899697492	-200469338149.348\\
3.15997899947499	-200423501525.738\\
3.16007900197505	-200377664902.127\\
3.16017900447511	-200331828278.517\\
3.16027900697517	-200285991654.906\\
3.16037900947524	-200240155031.296\\
3.1604790119753	-200194318407.685\\
3.16057901447536	-200148481784.075\\
3.16067901697542	-200102645160.464\\
3.16077901947549	-200056808536.854\\
3.16087902197555	-200010971913.243\\
3.16097902447561	-199965135289.633\\
3.16107902697567	-199919298666.022\\
3.16117902947574	-199873462042.412\\
3.1612790319758	-199827625418.802\\
3.16137903447586	-199781788795.191\\
3.16147903697592	-199735952171.581\\
3.16157903947599	-199690115547.97\\
3.16167904197605	-199644278924.36\\
3.16177904447611	-199598442300.749\\
3.16187904697617	-199552605677.139\\
3.16197904947624	-199506769053.528\\
3.1620790519763	-199460932429.918\\
3.16217905447636	-199415095806.307\\
3.16227905697642	-199369259182.697\\
3.16237905947649	-199323422559.086\\
3.16247906197655	-199277585935.476\\
3.16257906447661	-199231749311.866\\
3.16267906697667	-199185912688.255\\
3.16277906947674	-199140076064.645\\
3.1628790719768	-199094239441.034\\
3.16297907447686	-199048402817.424\\
3.16307907697692	-199002566193.813\\
3.16317907947699	-198957302527.998\\
3.16327908197705	-198911465904.387\\
3.16337908447711	-198865629280.777\\
3.16347908697717	-198819792657.166\\
3.16357908947724	-198773956033.556\\
3.1636790919773	-198728119409.946\\
3.16377909447736	-198682282786.335\\
3.16387909697742	-198636446162.725\\
3.16397909947749	-198590609539.114\\
3.16407910197755	-198544772915.504\\
3.16417910447761	-198498936291.893\\
3.16427910697767	-198453672626.078\\
3.16437910947774	-198407836002.467\\
3.1644791119778	-198361999378.857\\
3.16457911447786	-198316162755.246\\
3.16467911697792	-198270326131.636\\
3.16477911947799	-198224489508.026\\
3.16487912197805	-198178652884.415\\
3.16497912447811	-198132816260.805\\
3.16507912697817	-198086979637.194\\
3.16517912947824	-198041715971.379\\
3.1652791319783	-197995879347.768\\
3.16537913447836	-197950042724.158\\
3.16547913697842	-197904206100.547\\
3.16557913947849	-197858369476.937\\
3.16567914197855	-197812532853.326\\
3.16577914447861	-197766696229.716\\
3.16587914697867	-197720859606.106\\
3.16597914947874	-197675595940.29\\
3.1660791519788	-197629759316.68\\
3.16617915447886	-197583922693.069\\
3.16627915697892	-197538086069.459\\
3.16637915947899	-197492249445.848\\
3.16647916197905	-197446412822.238\\
3.16657916447911	-197401149156.423\\
3.16667916697917	-197355312532.812\\
3.16677916947924	-197309475909.202\\
3.1668791719793	-197263639285.591\\
3.16697917447936	-197217802661.981\\
3.16707917697942	-197171966038.37\\
3.16717917947949	-197126702372.555\\
3.16727918197955	-197080865748.944\\
3.16737918447961	-197035029125.334\\
3.16747918697967	-196989192501.723\\
3.16757918947974	-196943355878.113\\
3.1676791919798	-196898092212.298\\
3.16777919447986	-196852255588.687\\
3.16787919697992	-196806418965.077\\
3.16797919947999	-196760582341.466\\
3.16807920198005	-196714745717.856\\
3.16817920448011	-196669482052.04\\
3.16827920698017	-196623645428.43\\
3.16837920948024	-196577808804.82\\
3.1684792119803	-196531972181.209\\
3.16857921448036	-196486708515.394\\
3.16867921698042	-196440871891.783\\
3.16877921948049	-196395035268.173\\
3.16887922198055	-196349198644.562\\
3.16897922448061	-196303362020.952\\
3.16907922698067	-196258098355.137\\
3.16917922948074	-196212261731.526\\
3.1692792319808	-196166425107.916\\
3.16937923448086	-196120588484.305\\
3.16947923698092	-196075324818.49\\
3.16957923948099	-196029488194.879\\
3.16967924198105	-195983651571.269\\
3.16977924448111	-195937814947.658\\
3.16987924698117	-195892551281.843\\
3.16997924948124	-195846714658.233\\
3.1700792519813	-195800878034.622\\
3.17017925448136	-195755614368.807\\
3.17027925698142	-195709777745.196\\
3.17037925948149	-195663941121.586\\
3.17047926198155	-195618104497.975\\
3.17057926448161	-195572840832.16\\
3.17067926698167	-195527004208.55\\
3.17077926948174	-195481167584.939\\
3.1708792719818	-195435903919.124\\
3.17097927448186	-195390067295.513\\
3.17107927698192	-195344230671.903\\
3.17117927948199	-195298394048.292\\
3.17127928198205	-195253130382.477\\
3.17137928448211	-195207293758.867\\
3.17147928698217	-195161457135.256\\
3.17157928948224	-195116193469.441\\
3.1716792919823	-195070356845.83\\
3.17177929448236	-195024520222.22\\
3.17187929698242	-194979256556.405\\
3.17197929948249	-194933419932.794\\
3.17207930198255	-194887583309.184\\
3.17217930448261	-194842319643.368\\
3.17227930698267	-194796483019.758\\
3.17237930948274	-194750646396.147\\
3.1724793119828	-194705382730.332\\
3.17257931448286	-194659546106.722\\
3.17267931698292	-194613709483.111\\
3.17277931948299	-194568445817.296\\
3.17287932198305	-194522609193.685\\
3.17297932448311	-194476772570.075\\
3.17307932698317	-194431508904.259\\
3.17317932948324	-194385672280.649\\
3.1732793319833	-194340408614.834\\
3.17337933448336	-194294571991.223\\
3.17347933698342	-194248735367.613\\
3.17357933948349	-194203471701.797\\
3.17367934198355	-194157635078.187\\
3.17377934448361	-194111798454.576\\
3.17387934698367	-194066534788.761\\
3.17397934948374	-194020698165.151\\
3.1740793519838	-193975434499.335\\
3.17417935448386	-193929597875.725\\
3.17427935698392	-193883761252.114\\
3.17437935948399	-193838497586.299\\
3.17447936198405	-193792660962.689\\
3.17457936448411	-193747397296.873\\
3.17467936698417	-193701560673.263\\
3.17477936948424	-193656297007.447\\
3.1748793719843	-193610460383.837\\
3.17497937448436	-193564623760.227\\
3.17507937698442	-193519360094.411\\
3.17517937948449	-193473523470.801\\
3.17527938198455	-193428259804.985\\
3.17537938448461	-193382423181.375\\
3.17547938698467	-193337159515.56\\
3.17557938948474	-193291322891.949\\
3.1756793919848	-193246059226.134\\
3.17577939448486	-193200222602.523\\
3.17587939698492	-193154385978.913\\
3.17597939948499	-193109122313.098\\
3.17607940198505	-193063285689.487\\
3.17617940448511	-193018022023.672\\
3.17627940698517	-192972185400.061\\
3.17637940948524	-192926921734.246\\
3.1764794119853	-192881085110.635\\
3.17657941448536	-192835821444.82\\
3.17667941698542	-192789984821.21\\
3.17677941948549	-192744721155.394\\
3.17687942198555	-192698884531.784\\
3.17697942448561	-192653620865.969\\
3.17707942698567	-192607784242.358\\
3.17717942948574	-192562520576.543\\
3.1772794319858	-192516683952.932\\
3.17737943448586	-192471420287.117\\
3.17747943698592	-192425583663.506\\
3.17757943948599	-192380319997.691\\
3.17767944198605	-192334483374.081\\
3.17777944448611	-192289219708.265\\
3.17787944698617	-192243383084.655\\
3.17797944948624	-192198119418.84\\
3.1780794519863	-192152282795.229\\
3.17817945448636	-192107019129.414\\
3.17827945698642	-192061755463.598\\
3.17837945948649	-192015918839.988\\
3.17847946198655	-191970655174.173\\
3.17857946448661	-191924818550.562\\
3.17867946698667	-191879554884.747\\
3.17877946948674	-191833718261.136\\
3.1788794719868	-191788454595.321\\
3.17897947448686	-191742617971.711\\
3.17907947698692	-191697354305.895\\
3.17917947948699	-191652090640.08\\
3.17927948198705	-191606254016.469\\
3.17937948448711	-191560990350.654\\
3.17947948698717	-191515153727.044\\
3.17957948948724	-191469890061.228\\
3.1796794919873	-191424626395.413\\
3.17977949448736	-191378789771.802\\
3.17987949698742	-191333526105.987\\
3.17997949948749	-191287689482.377\\
3.18007950198755	-191242425816.561\\
3.18017950448761	-191197162150.746\\
3.18027950698767	-191151325527.135\\
3.18037950948774	-191106061861.32\\
3.1804795119878	-191060225237.71\\
3.18057951448786	-191014961571.894\\
3.18067951698792	-190969697906.079\\
3.18077951948799	-190923861282.469\\
3.18087952198805	-190878597616.653\\
3.18097952448811	-190833333950.838\\
3.18107952698817	-190787497327.227\\
3.18117952948824	-190742233661.412\\
3.1812795319883	-190696969995.597\\
3.18137953448836	-190651133371.986\\
3.18147953698842	-190605869706.171\\
3.18157953948849	-190560606040.356\\
3.18167954198855	-190514769416.745\\
3.18177954448861	-190469505750.93\\
3.18187954698867	-190424242085.115\\
3.18197954948874	-190378405461.504\\
3.1820795519888	-190333141795.689\\
3.18217955448886	-190287878129.873\\
3.18227955698892	-190242041506.263\\
3.18237955948899	-190196777840.448\\
3.18247956198905	-190151514174.632\\
3.18257956448911	-190105677551.022\\
3.18267956698917	-190060413885.206\\
3.18277956948924	-190015150219.391\\
3.1828795719893	-189969886553.576\\
3.18297957448936	-189924049929.965\\
3.18307957698942	-189878786264.15\\
3.18317957948949	-189833522598.335\\
3.18327958198955	-189788258932.519\\
3.18337958448961	-189742422308.909\\
3.18347958698967	-189697158643.093\\
3.18357958948974	-189651894977.278\\
3.1836795919898	-189606058353.668\\
3.18377959448986	-189560794687.852\\
3.18387959698992	-189515531022.037\\
3.18397959948999	-189470267356.222\\
3.18407960199005	-189424430732.611\\
3.18417960449011	-189379167066.796\\
3.18427960699017	-189333903400.981\\
3.18437960949024	-189288639735.165\\
3.1844796119903	-189243376069.35\\
3.18457961449036	-189197539445.739\\
3.18467961699042	-189152275779.924\\
3.18477961949049	-189107012114.109\\
3.18487962199055	-189061748448.293\\
3.18497962449061	-189016484782.478\\
3.18507962699067	-188970648158.868\\
3.18517962949074	-188925384493.052\\
3.1852796319908	-188880120827.237\\
3.18537963449086	-188834857161.422\\
3.18547963699092	-188789593495.606\\
3.18557963949099	-188743756871.996\\
3.18567964199105	-188698493206.18\\
3.18577964449111	-188653229540.365\\
3.18587964699117	-188607965874.55\\
3.18597964949124	-188562702208.734\\
3.1860796519913	-188517438542.919\\
3.18617965449136	-188471601919.309\\
3.18627965699142	-188426338253.493\\
3.18637965949149	-188381074587.678\\
3.18647966199155	-188335810921.863\\
3.18657966449161	-188290547256.047\\
3.18667966699167	-188245283590.232\\
3.18677966949174	-188200019924.417\\
3.1868796719918	-188154183300.806\\
3.18697967449186	-188108919634.991\\
3.18707967699192	-188063655969.176\\
3.18717967949199	-188018392303.36\\
3.18727968199205	-187973128637.545\\
3.18737968449211	-187927864971.729\\
3.18747968699217	-187882601305.914\\
3.18757968949224	-187837337640.099\\
3.1876796919923	-187792073974.283\\
3.18777969449236	-187746810308.468\\
3.18787969699242	-187700973684.858\\
3.18797969949249	-187655710019.042\\
3.18807970199255	-187610446353.227\\
3.18817970449261	-187565182687.412\\
3.18827970699267	-187519919021.596\\
3.18837970949274	-187474655355.781\\
3.1884797119928	-187429391689.966\\
3.18857971449286	-187384128024.15\\
3.18867971699292	-187338864358.335\\
3.18877971949299	-187293600692.52\\
3.18887972199305	-187248337026.704\\
3.18897972449311	-187203073360.889\\
3.18907972699317	-187157809695.074\\
3.18917972949324	-187112546029.258\\
3.1892797319933	-187067282363.443\\
3.18937973449336	-187022018697.628\\
3.18947973699342	-186976755031.812\\
3.18957973949349	-186931491365.997\\
3.18967974199355	-186886227700.182\\
3.18977974449361	-186840964034.366\\
3.18987974699367	-186795700368.551\\
3.18997974949374	-186750436702.736\\
3.1900797519938	-186705173036.92\\
3.19017975449386	-186659909371.105\\
3.19027975699392	-186614645705.29\\
3.19037975949399	-186569382039.474\\
3.19047976199405	-186524118373.659\\
3.19057976449411	-186478854707.844\\
3.19067976699417	-186433591042.028\\
3.19077976949424	-186388327376.213\\
3.1908797719943	-186343063710.398\\
3.19097977449436	-186297800044.582\\
3.19107977699443	-186252536378.767\\
3.19117977949449	-186207272712.952\\
3.19127978199455	-186162009047.136\\
3.19137978449461	-186116745381.321\\
3.19147978699467	-186071481715.506\\
3.19157978949474	-186026218049.69\\
3.1916797919948	-185980954383.875\\
3.19177979449486	-185935690718.06\\
3.19187979699492	-185891000010.039\\
3.19197979949499	-185845736344.224\\
3.19207980199505	-185800472678.409\\
3.19217980449511	-185755209012.593\\
3.19227980699517	-185709945346.778\\
3.19237980949524	-185664681680.963\\
3.1924798119953	-185619418015.147\\
3.19257981449536	-185574154349.332\\
3.19267981699542	-185528890683.517\\
3.19277981949549	-185483627017.701\\
3.19287982199555	-185438936309.681\\
3.19297982449561	-185393672643.866\\
3.19307982699568	-185348408978.051\\
3.19317982949574	-185303145312.235\\
3.1932798319958	-185257881646.42\\
3.19337983449586	-185212617980.605\\
3.19347983699592	-185167354314.789\\
3.19357983949599	-185122663606.769\\
3.19367984199605	-185077399940.954\\
3.19377984449611	-185032136275.138\\
3.19387984699617	-184986872609.323\\
3.19397984949624	-184941608943.508\\
3.1940798519963	-184896345277.692\\
3.19417985449636	-184851654569.672\\
3.19427985699642	-184806390903.857\\
3.19437985949649	-184761127238.041\\
3.19447986199655	-184715863572.226\\
3.19457986449661	-184670599906.411\\
3.19467986699667	-184625909198.391\\
3.19477986949674	-184580645532.575\\
3.1948798719968	-184535381866.76\\
3.19497987449686	-184490118200.945\\
3.19507987699693	-184445427492.924\\
3.19517987949699	-184400163827.109\\
3.19527988199705	-184354900161.294\\
3.19537988449711	-184309636495.478\\
3.19547988699718	-184264945787.458\\
3.19557988949724	-184219682121.643\\
3.1956798919973	-184174418455.827\\
3.19577989449736	-184129154790.012\\
3.19587989699742	-184084464081.992\\
3.19597989949749	-184039200416.177\\
3.19607990199755	-183993936750.361\\
3.19617990449761	-183948673084.546\\
3.19627990699767	-183903982376.526\\
3.19637990949774	-183858718710.71\\
3.1964799119978	-183813455044.895\\
3.19657991449786	-183768764336.875\\
3.19667991699792	-183723500671.06\\
3.19677991949799	-183678237005.244\\
3.19687992199805	-183632973339.429\\
3.19697992449811	-183588282631.409\\
3.19707992699818	-183543018965.593\\
3.19717992949824	-183497755299.778\\
3.1972799319983	-183453064591.758\\
3.19737993449836	-183407800925.942\\
3.19747993699843	-183362537260.127\\
3.19757993949849	-183317846552.107\\
3.19767994199855	-183272582886.292\\
3.19777994449861	-183227319220.476\\
3.19787994699867	-183182628512.456\\
3.19797994949874	-183137364846.641\\
3.1980799519988	-183092101180.825\\
3.19817995449886	-183047410472.805\\
3.19827995699892	-183002146806.99\\
3.19837995949899	-182957456098.97\\
3.19847996199905	-182912192433.154\\
3.19857996449911	-182866928767.339\\
3.19867996699917	-182822238059.319\\
3.19877996949924	-182776974393.503\\
3.1988799719993	-182732283685.483\\
3.19897997449936	-182687020019.668\\
3.19907997699943	-182641756353.853\\
3.19917997949949	-182597065645.832\\
3.19927998199955	-182551801980.017\\
3.19937998449961	-182507111271.997\\
3.19947998699968	-182461847606.181\\
3.19957998949974	-182417156898.161\\
3.1996799919998	-182371893232.346\\
3.19977999449986	-182326629566.531\\
3.19987999699992	-182281938858.51\\
3.19997999949999	-182236675192.695\\
3.20008000200005	-182191984484.675\\
};
\addplot [color=mycolor3,solid,forget plot]
  table[row sep=crcr]{%
3.20008000200005	-182191984484.675\\
3.20018000450011	-182146720818.859\\
3.20028000700017	-182102030110.839\\
3.20038000950024	-182056766445.024\\
3.2004800120003	-182012075737.004\\
3.20058001450036	-181966812071.188\\
3.20068001700042	-181922121363.168\\
3.20078001950049	-181876857697.353\\
3.20088002200055	-181832166989.333\\
3.20098002450061	-181786903323.517\\
3.20108002700068	-181742212615.497\\
3.20118002950074	-181696948949.682\\
3.2012800320008	-181652258241.662\\
3.20138003450086	-181606994575.846\\
3.20148003700093	-181562303867.826\\
3.20158003950099	-181517040202.011\\
3.20168004200105	-181472349493.991\\
3.20178004450111	-181427085828.175\\
3.20188004700118	-181382395120.155\\
3.20198004950124	-181337131454.34\\
3.2020800520013	-181292440746.319\\
3.20218005450136	-181247750038.299\\
3.20228005700142	-181202486372.484\\
3.20238005950149	-181157795664.464\\
3.20248006200155	-181112531998.648\\
3.20258006450161	-181067841290.628\\
3.20268006700167	-181022577624.813\\
3.20278006950174	-180977886916.793\\
3.2028800720018	-180933196208.772\\
3.20298007450186	-180887932542.957\\
3.20308007700193	-180843241834.937\\
3.20318007950199	-180797978169.122\\
3.20328008200205	-180753287461.101\\
3.20338008450211	-180708596753.081\\
3.20348008700218	-180663333087.266\\
3.20358008950224	-180618642379.246\\
3.2036800920023	-180573951671.225\\
3.20378009450236	-180528688005.41\\
3.20388009700243	-180483997297.39\\
3.20398009950249	-180438733631.574\\
3.20408010200255	-180394042923.554\\
3.20418010450261	-180349352215.534\\
3.20428010700267	-180304088549.719\\
3.20438010950274	-180259397841.699\\
3.2044801120028	-180214707133.678\\
3.20458011450286	-180169443467.863\\
3.20468011700292	-180124752759.843\\
3.20478011950299	-180080062051.823\\
3.20488012200305	-180035371343.802\\
3.20498012450311	-179990107677.987\\
3.20508012700318	-179945416969.967\\
3.20518012950324	-179900726261.947\\
3.2052801320033	-179855462596.131\\
3.20538013450336	-179810771888.111\\
3.20548013700343	-179766081180.091\\
3.20558013950349	-179721390472.071\\
3.20568014200355	-179676126806.255\\
3.20578014450361	-179631436098.235\\
3.20588014700368	-179586745390.215\\
3.20598014950374	-179542054682.195\\
3.2060801520038	-179496791016.379\\
3.20618015450386	-179452100308.359\\
3.20628015700392	-179407409600.339\\
3.20638015950399	-179362718892.319\\
3.20648016200405	-179317455226.503\\
3.20658016450411	-179272764518.483\\
3.20668016700417	-179228073810.463\\
3.20678016950424	-179183383102.443\\
3.2068801720043	-179138119436.628\\
3.20698017450436	-179093428728.607\\
3.20708017700443	-179048738020.587\\
3.20718017950449	-179004047312.567\\
3.20728018200455	-178959356604.547\\
3.20738018450461	-178914665896.527\\
3.20748018700468	-178869402230.711\\
3.20758018950474	-178824711522.691\\
3.2076801920048	-178780020814.671\\
3.20778019450486	-178735330106.651\\
3.20788019700493	-178690639398.63\\
3.20798019950499	-178645948690.61\\
3.20808020200505	-178600685024.795\\
3.20818020450511	-178555994316.775\\
3.20828020700518	-178511303608.754\\
3.20838020950524	-178466612900.734\\
3.2084802120053	-178421922192.714\\
3.20858021450536	-178377231484.694\\
3.20868021700542	-178332540776.674\\
3.20878021950549	-178287850068.653\\
3.20888022200555	-178243159360.633\\
3.20898022450561	-178197895694.818\\
3.20908022700568	-178153204986.798\\
3.20918022950574	-178108514278.777\\
3.2092802320058	-178063823570.757\\
3.20938023450586	-178019132862.737\\
3.20948023700593	-177974442154.717\\
3.20958023950599	-177929751446.697\\
3.20968024200605	-177885060738.676\\
3.20978024450611	-177840370030.656\\
3.20988024700618	-177795679322.636\\
3.20998024950624	-177750988614.616\\
3.2100802520063	-177706297906.596\\
3.21018025450636	-177661607198.575\\
3.21028025700643	-177616916490.555\\
3.21038025950649	-177572225782.535\\
3.21048026200655	-177527535074.515\\
3.21058026450661	-177482844366.495\\
3.21068026700667	-177438153658.474\\
3.21078026950674	-177393462950.454\\
3.2108802720068	-177348772242.434\\
3.21098027450686	-177304081534.414\\
3.21108027700693	-177259390826.394\\
3.21118027950699	-177214700118.373\\
3.21128028200705	-177170009410.353\\
3.21138028450711	-177125318702.333\\
3.21148028700718	-177080627994.313\\
3.21158028950724	-177035937286.293\\
3.2116802920073	-176991246578.272\\
3.21178029450736	-176946555870.252\\
3.21188029700743	-176901865162.232\\
3.21198029950749	-176857174454.212\\
3.21208030200755	-176812483746.192\\
3.21218030450761	-176767793038.171\\
3.21228030700768	-176723102330.151\\
3.21238030950774	-176678411622.131\\
3.2124803120078	-176633720914.111\\
3.21258031450786	-176589603163.886\\
3.21268031700793	-176544912455.865\\
3.21278031950799	-176500221747.845\\
3.21288032200805	-176455531039.825\\
3.21298032450811	-176410840331.805\\
3.21308032700818	-176366149623.785\\
3.21318032950824	-176321458915.764\\
3.2132803320083	-176276768207.744\\
3.21338033450836	-176232077499.724\\
3.21348033700843	-176187959749.499\\
3.21358033950849	-176143269041.479\\
3.21368034200855	-176098578333.459\\
3.21378034450861	-176053887625.438\\
3.21388034700868	-176009196917.418\\
3.21398034950874	-175964506209.398\\
3.2140803520088	-175920388459.173\\
3.21418035450886	-175875697751.153\\
3.21428035700893	-175831007043.132\\
3.21438035950899	-175786316335.112\\
3.21448036200905	-175741625627.092\\
3.21458036450911	-175697507876.867\\
3.21468036700918	-175652817168.847\\
3.21478036950924	-175608126460.827\\
3.2148803720093	-175563435752.806\\
3.21498037450936	-175518745044.786\\
3.21508037700943	-175474627294.561\\
3.21518037950949	-175429936586.541\\
3.21528038200955	-175385245878.521\\
3.21538038450961	-175340555170.5\\
3.21548038700968	-175296437420.275\\
3.21558038950974	-175251746712.255\\
3.2156803920098	-175207056004.235\\
3.21578039450986	-175162938254.01\\
3.21588039700993	-175118247545.99\\
3.21598039950999	-175073556837.969\\
3.21608040201005	-175028866129.949\\
3.21618040451011	-174984748379.724\\
3.21628040701018	-174940057671.704\\
3.21638040951024	-174895366963.684\\
3.2164804120103	-174851249213.459\\
3.21658041451036	-174806558505.439\\
3.21668041701043	-174761867797.418\\
3.21678041951049	-174717750047.193\\
3.21688042201055	-174673059339.173\\
3.21698042451061	-174628368631.153\\
3.21708042701068	-174584250880.928\\
3.21718042951074	-174539560172.908\\
3.2172804320108	-174494869464.887\\
3.21738043451086	-174450751714.662\\
3.21748043701093	-174406061006.642\\
3.21758043951099	-174361370298.622\\
3.21768044201105	-174317252548.397\\
3.21778044451111	-174272561840.377\\
3.21788044701118	-174228444090.152\\
3.21798044951124	-174183753382.131\\
3.2180804520113	-174139062674.111\\
3.21818045451136	-174094944923.886\\
3.21828045701143	-174050254215.866\\
3.21838045951149	-174006136465.641\\
3.21848046201155	-173961445757.621\\
3.21858046451161	-173916755049.6\\
3.21868046701168	-173872637299.375\\
3.21878046951174	-173827946591.355\\
3.2188804720118	-173783828841.13\\
3.21898047451186	-173739138133.11\\
3.21908047701193	-173695020382.885\\
3.21918047951199	-173650329674.865\\
3.21928048201205	-173606211924.639\\
3.21938048451211	-173561521216.619\\
3.21948048701218	-173517403466.394\\
3.21958048951224	-173472712758.374\\
3.2196804920123	-173428595008.149\\
3.21978049451236	-173383904300.129\\
3.21988049701243	-173339786549.904\\
3.21998049951249	-173295095841.883\\
3.22008050201255	-173250978091.658\\
3.22018050451261	-173206287383.638\\
3.22028050701268	-173162169633.413\\
3.22038050951274	-173117478925.393\\
3.2204805120128	-173073361175.168\\
3.22058051451286	-173028670467.148\\
3.22068051701293	-172984552716.922\\
3.22078051951299	-172939862008.902\\
3.22088052201305	-172895744258.677\\
3.22098052451311	-172851053550.657\\
3.22108052701318	-172806935800.432\\
3.22118052951324	-172762818050.207\\
3.2212805320133	-172718127342.187\\
3.22138053451336	-172674009591.962\\
3.22148053701343	-172629318883.941\\
3.22158053951349	-172585201133.716\\
3.22168054201355	-172541083383.491\\
3.22178054451361	-172496392675.471\\
3.22188054701368	-172452274925.246\\
3.22198054951374	-172407584217.226\\
3.2220805520138	-172363466467.001\\
3.22218055451386	-172319348716.776\\
3.22228055701393	-172274658008.755\\
3.22238055951399	-172230540258.53\\
3.22248056201405	-172186422508.305\\
3.22258056451411	-172141731800.285\\
3.22268056701418	-172097614050.06\\
3.22278056951424	-172053496299.835\\
3.2228805720143	-172008805591.815\\
3.22298057451436	-171964687841.59\\
3.22308057701443	-171920570091.365\\
3.22318057951449	-171875879383.344\\
3.22328058201455	-171831761633.119\\
3.22338058451461	-171787643882.894\\
3.22348058701468	-171743526132.669\\
3.22358058951474	-171698835424.649\\
3.2236805920148	-171654717674.424\\
3.22378059451486	-171610599924.199\\
3.22388059701493	-171566482173.974\\
3.22398059951499	-171521791465.953\\
3.22408060201505	-171477673715.728\\
3.22418060451511	-171433555965.503\\
3.22428060701518	-171389438215.278\\
3.22438060951524	-171344747507.258\\
3.2244806120153	-171300629757.033\\
3.22458061451536	-171256512006.808\\
3.22468061701543	-171212394256.583\\
3.22478061951549	-171167703548.563\\
3.22488062201555	-171123585798.338\\
3.22498062451561	-171079468048.113\\
3.22508062701568	-171035350297.887\\
3.22518062951574	-170991232547.662\\
3.2252806320158	-170947114797.437\\
3.22538063451586	-170902424089.417\\
3.22548063701593	-170858306339.192\\
3.22558063951599	-170814188588.967\\
3.22568064201605	-170770070838.742\\
3.22578064451611	-170725953088.517\\
3.22588064701618	-170681835338.292\\
3.22598064951624	-170637144630.272\\
3.2260806520163	-170593026880.046\\
3.22618065451636	-170548909129.821\\
3.22628065701643	-170504791379.596\\
3.22638065951649	-170460673629.371\\
3.22648066201655	-170416555879.146\\
3.22658066451661	-170372438128.921\\
3.22668066701668	-170328320378.696\\
3.22678066951674	-170284202628.471\\
3.2268806720168	-170240084878.246\\
3.22698067451686	-170195394170.226\\
3.22708067701693	-170151276420.001\\
3.22718067951699	-170107158669.776\\
3.22728068201705	-170063040919.55\\
3.22738068451711	-170018923169.325\\
3.22748068701718	-169974805419.1\\
3.22758068951724	-169930687668.875\\
3.2276806920173	-169886569918.65\\
3.22778069451736	-169842452168.425\\
3.22788069701743	-169798334418.2\\
3.22798069951749	-169754216667.975\\
3.22808070201755	-169710098917.75\\
3.22818070451761	-169665981167.525\\
3.22828070701768	-169621863417.3\\
3.22838070951774	-169577745667.075\\
3.2284807120178	-169533627916.85\\
3.22858071451786	-169489510166.624\\
3.22868071701793	-169445392416.399\\
3.22878071951799	-169401274666.174\\
3.22888072201805	-169357156915.949\\
3.22898072451811	-169313039165.724\\
3.22908072701818	-169268921415.499\\
3.22918072951824	-169224803665.274\\
3.2292807320183	-169180685915.049\\
3.22938073451836	-169136568164.824\\
3.22948073701843	-169093023372.394\\
3.22958073951849	-169048905622.169\\
3.22968074201855	-169004787871.944\\
3.22978074451861	-168960670121.719\\
3.22988074701868	-168916552371.494\\
3.22998074951874	-168872434621.269\\
3.2300807520188	-168828316871.044\\
3.23018075451886	-168784199120.818\\
3.23028075701893	-168740081370.593\\
3.23038075951899	-168696536578.163\\
3.23048076201905	-168652418827.938\\
3.23058076451911	-168608301077.713\\
3.23068076701918	-168564183327.488\\
3.23078076951924	-168520065577.263\\
3.2308807720193	-168475947827.038\\
3.23098077451936	-168432403034.608\\
3.23108077701943	-168388285284.383\\
3.23118077951949	-168344167534.158\\
3.23128078201955	-168300049783.933\\
3.23138078451961	-168255932033.708\\
3.23148078701968	-168211814283.483\\
3.23158078951974	-168168269491.053\\
3.2316807920198	-168124151740.828\\
3.23178079451986	-168080033990.603\\
3.23188079701993	-168035916240.378\\
3.23198079951999	-167992371447.948\\
3.23208080202005	-167948253697.723\\
3.23218080452011	-167904135947.497\\
3.23228080702018	-167860018197.272\\
3.23238080952024	-167816473404.842\\
3.2324808120203	-167772355654.617\\
3.23258081452036	-167728237904.392\\
3.23268081702043	-167684120154.167\\
3.23278081952049	-167640575361.737\\
3.23288082202055	-167596457611.512\\
3.23298082452061	-167552339861.287\\
3.23308082702068	-167508795068.857\\
3.23318082952074	-167464677318.632\\
3.2332808320208	-167420559568.407\\
3.23338083452086	-167377014775.977\\
3.23348083702093	-167332897025.752\\
3.23358083952099	-167288779275.527\\
3.23368084202105	-167245234483.097\\
3.23378084452111	-167201116732.872\\
3.23388084702118	-167156998982.647\\
3.23398084952124	-167113454190.217\\
3.2340808520213	-167069336439.992\\
3.23418085452136	-167025218689.767\\
3.23428085702143	-166981673897.337\\
3.23438085952149	-166937556147.112\\
3.23448086202155	-166894011354.682\\
3.23458086452161	-166849893604.457\\
3.23468086702168	-166805775854.232\\
3.23478086952174	-166762231061.802\\
3.2348808720218	-166718113311.577\\
3.23498087452186	-166674568519.147\\
3.23508087702193	-166630450768.922\\
3.23518087952199	-166586905976.492\\
3.23528088202205	-166542788226.267\\
3.23538088452211	-166498670476.042\\
3.23548088702218	-166455125683.612\\
3.23558088952224	-166411007933.387\\
3.2356808920223	-166367463140.957\\
3.23578089452236	-166323345390.732\\
3.23588089702243	-166279800598.302\\
3.23598089952249	-166235682848.077\\
3.23608090202255	-166192138055.647\\
3.23618090452261	-166148020305.422\\
3.23628090702268	-166104475512.992\\
3.23638090952274	-166060357762.767\\
3.2364809120228	-166016812970.337\\
3.23658091452286	-165972695220.111\\
3.23668091702293	-165929150427.682\\
3.23678091952299	-165885605635.252\\
3.23688092202305	-165841487885.027\\
3.23698092452311	-165797943092.597\\
3.23708092702318	-165753825342.372\\
3.23718092952324	-165710280549.942\\
3.2372809320233	-165666162799.716\\
3.23738093452336	-165622618007.287\\
3.23748093702343	-165579073214.857\\
3.23758093952349	-165534955464.632\\
3.23768094202355	-165491410672.202\\
3.23778094452361	-165447292921.977\\
3.23788094702368	-165403748129.547\\
3.23798094952374	-165360203337.117\\
3.2380809520238	-165316085586.892\\
3.23818095452386	-165272540794.462\\
3.23828095702393	-165228423044.237\\
3.23838095952399	-165184878251.807\\
3.23848096202405	-165141333459.377\\
3.23858096452411	-165097215709.152\\
3.23868096702418	-165053670916.722\\
3.23878096952424	-165010126124.292\\
3.2388809720243	-164966008374.067\\
3.23898097452436	-164922463581.637\\
3.23908097702443	-164878918789.207\\
3.23918097952449	-164835373996.777\\
3.23928098202455	-164791256246.552\\
3.23938098452461	-164747711454.122\\
3.23948098702468	-164704166661.692\\
3.23958098952474	-164660048911.467\\
3.2396809920248	-164616504119.037\\
3.23978099452486	-164572959326.607\\
3.23988099702493	-164529414534.177\\
3.23998099952499	-164485296783.952\\
3.24008100202505	-164441751991.522\\
3.24018100452511	-164398207199.092\\
3.24028100702518	-164354662406.662\\
3.24038100952524	-164311117614.232\\
3.2404810120253	-164266999864.007\\
3.24058101452536	-164223455071.577\\
3.24068101702543	-164179910279.147\\
3.24078101952549	-164136365486.717\\
3.24088102202555	-164092820694.287\\
3.24098102452561	-164048702944.062\\
3.24108102702568	-164005158151.632\\
3.24118102952574	-163961613359.202\\
3.2412810320258	-163918068566.772\\
3.24138103452586	-163874523774.342\\
3.24148103702593	-163830978981.912\\
3.24158103952599	-163786861231.687\\
3.24168104202605	-163743316439.257\\
3.24178104452611	-163699771646.828\\
3.24188104702618	-163656226854.398\\
3.24198104952624	-163612682061.968\\
3.2420810520263	-163569137269.538\\
3.24218105452636	-163525592477.108\\
3.24228105702643	-163482047684.678\\
3.24238105952649	-163438502892.248\\
3.24248106202655	-163394958099.818\\
3.24258106452661	-163350840349.593\\
3.24268106702668	-163307295557.163\\
3.24278106952674	-163263750764.733\\
3.2428810720268	-163220205972.303\\
3.24298107452686	-163176661179.873\\
3.24308107702693	-163133116387.443\\
3.24318107952699	-163089571595.013\\
3.24328108202705	-163046026802.583\\
3.24338108452711	-163002482010.153\\
3.24348108702718	-162958937217.723\\
3.24358108952724	-162915392425.293\\
3.2436810920273	-162871847632.863\\
3.24378109452736	-162828302840.434\\
3.24388109702743	-162784758048.004\\
3.24398109952749	-162741213255.574\\
3.24408110202755	-162697668463.144\\
3.24418110452761	-162654123670.714\\
3.24428110702768	-162610578878.284\\
3.24438110952774	-162567034085.854\\
3.2444811120278	-162523489293.424\\
3.24458111452786	-162480517458.789\\
3.24468111702793	-162436972666.359\\
3.24478111952799	-162393427873.929\\
3.24488112202805	-162349883081.499\\
3.24498112452811	-162306338289.069\\
3.24508112702818	-162262793496.639\\
3.24518112952824	-162219248704.209\\
3.2452811320283	-162175703911.78\\
3.24538113452836	-162132159119.35\\
3.24548113702843	-162088614326.92\\
3.24558113952849	-162045642492.285\\
3.24568114202855	-162002097699.855\\
3.24578114452861	-161958552907.425\\
3.24588114702868	-161915008114.995\\
3.24598114952874	-161871463322.565\\
3.2460811520288	-161827918530.135\\
3.24618115452886	-161784946695.5\\
3.24628115702893	-161741401903.07\\
3.24638115952899	-161697857110.64\\
3.24648116202905	-161654312318.21\\
3.24658116452911	-161610767525.781\\
3.24668116702918	-161567795691.146\\
3.24678116952924	-161524250898.716\\
3.2468811720293	-161480706106.286\\
3.24698117452936	-161437161313.856\\
3.24708117702943	-161393616521.426\\
3.24718117952949	-161350644686.791\\
3.24728118202955	-161307099894.361\\
3.24738118452961	-161263555101.931\\
3.24748118702968	-161220583267.296\\
3.24758118952974	-161177038474.866\\
3.2476811920298	-161133493682.437\\
3.24778119452986	-161089948890.007\\
3.24788119702993	-161046977055.372\\
3.24798119952999	-161003432262.942\\
3.24808120203005	-160959887470.512\\
3.24818120453011	-160916915635.877\\
3.24828120703018	-160873370843.447\\
3.24838120953024	-160829826051.017\\
3.2484812120303	-160786854216.382\\
3.24858121453036	-160743309423.952\\
3.24868121703043	-160699764631.523\\
3.24878121953049	-160656792796.888\\
3.24888122203055	-160613248004.458\\
3.24898122453061	-160569703212.028\\
3.24908122703068	-160526731377.393\\
3.24918122953074	-160483186584.963\\
3.2492812320308	-160440214750.328\\
3.24938123453086	-160396669957.898\\
3.24948123703093	-160353125165.468\\
3.24958123953099	-160310153330.834\\
3.24968124203105	-160266608538.404\\
3.24978124453111	-160223636703.769\\
3.24988124703118	-160180091911.339\\
3.24998124953124	-160137120076.704\\
3.2500812520313	-160093575284.274\\
3.25018125453136	-160050030491.844\\
3.25028125703143	-160007058657.209\\
3.25038125953149	-159963513864.779\\
3.25048126203155	-159920542030.145\\
3.25058126453161	-159876997237.715\\
3.25068126703168	-159834025403.08\\
3.25078126953174	-159790480610.65\\
3.2508812720318	-159747508776.015\\
3.25098127453186	-159703963983.585\\
3.25108127703193	-159660992148.95\\
3.25118127953199	-159617447356.52\\
3.25128128203205	-159574475521.886\\
3.25138128453211	-159530930729.456\\
3.25148128703218	-159487958894.821\\
3.25158128953224	-159444987060.186\\
3.2516812920323	-159401442267.756\\
3.25178129453236	-159358470433.121\\
3.25188129703243	-159314925640.691\\
3.25198129953249	-159271953806.057\\
3.25208130203255	-159228981971.422\\
3.25218130453261	-159185437178.992\\
3.25228130703268	-159142465344.357\\
3.25238130953274	-159098920551.927\\
3.2524813120328	-159055948717.292\\
3.25258131453286	-159012976882.657\\
3.25268131703293	-158969432090.227\\
3.25278131953299	-158926460255.593\\
3.25288132203305	-158883488420.958\\
3.25298132453311	-158839943628.528\\
3.25308132703318	-158796971793.893\\
3.25318132953324	-158753999959.258\\
3.2532813320333	-158710455166.828\\
3.25338133453336	-158667483332.193\\
3.25348133703343	-158624511497.559\\
3.25358133953349	-158580966705.129\\
3.25368134203355	-158537994870.494\\
3.25378134453361	-158495023035.859\\
3.25388134703368	-158451478243.429\\
3.25398134953374	-158408506408.794\\
3.2540813520338	-158365534574.16\\
3.25418135453386	-158322562739.525\\
3.25428135703393	-158279017947.095\\
3.25438135953399	-158236046112.46\\
3.25448136203405	-158193074277.825\\
3.25458136453411	-158150102443.19\\
3.25468136703418	-158107130608.556\\
3.25478136953424	-158063585816.126\\
3.2548813720343	-158020613981.491\\
3.25498137453436	-157977642146.856\\
3.25508137703443	-157934670312.221\\
3.25518137953449	-157891698477.586\\
3.25528138203455	-157848153685.156\\
3.25538138453461	-157805181850.522\\
3.25548138703468	-157762210015.887\\
3.25558138953474	-157719238181.252\\
3.2556813920348	-157676266346.617\\
3.25578139453486	-157633294511.982\\
3.25588139703493	-157589749719.552\\
3.25598139953499	-157546777884.918\\
3.25608140203505	-157503806050.283\\
3.25618140453511	-157460834215.648\\
3.25628140703518	-157417862381.013\\
3.25638140953524	-157374890546.378\\
3.2564814120353	-157331918711.744\\
3.25658141453536	-157288946877.109\\
3.25668141703543	-157245975042.474\\
3.25678141953549	-157203003207.839\\
3.25688142203555	-157160031373.204\\
3.25698142453561	-157117059538.569\\
3.25708142703568	-157073514746.14\\
3.25718142953574	-157030542911.505\\
3.2572814320358	-156987571076.87\\
3.25738143453586	-156944599242.235\\
3.25748143703593	-156901627407.6\\
3.25758143953599	-156858655572.965\\
3.25768144203605	-156815683738.331\\
3.25778144453611	-156772711903.696\\
3.25788144703618	-156729740069.061\\
3.25798144953624	-156686768234.426\\
3.2580814520363	-156643796399.791\\
3.25818145453636	-156601397522.952\\
3.25828145703643	-156558425688.317\\
3.25838145953649	-156515453853.682\\
3.25848146203655	-156472482019.047\\
3.25858146453661	-156429510184.412\\
3.25868146703668	-156386538349.778\\
3.25878146953674	-156343566515.143\\
3.2588814720368	-156300594680.508\\
3.25898147453686	-156257622845.873\\
3.25908147703693	-156214651011.238\\
3.25918147953699	-156171679176.604\\
3.25928148203705	-156128707341.969\\
3.25938148453711	-156086308465.129\\
3.25948148703718	-156043336630.494\\
3.25958148953724	-156000364795.859\\
3.2596814920373	-155957392961.225\\
3.25978149453736	-155914421126.59\\
3.25988149703743	-155871449291.955\\
3.25998149953749	-155828477457.32\\
3.26008150203755	-155786078580.481\\
3.26018150453761	-155743106745.846\\
3.26028150703768	-155700134911.211\\
3.26038150953774	-155657163076.576\\
3.2604815120378	-155614191241.941\\
3.26058151453786	-155571792365.102\\
3.26068151703793	-155528820530.467\\
3.26078151953799	-155485848695.832\\
3.26088152203805	-155442876861.197\\
3.26098152453811	-155400477984.358\\
3.26108152703818	-155357506149.723\\
3.26118152953824	-155314534315.088\\
3.2612815320383	-155271562480.453\\
3.26138153453836	-155229163603.613\\
3.26148153703843	-155186191768.979\\
3.26158153953849	-155143219934.344\\
3.26168154203855	-155100821057.504\\
3.26178154453861	-155057849222.869\\
3.26188154703868	-155014877388.234\\
3.26198154953874	-154972478511.395\\
3.2620815520388	-154929506676.76\\
3.26218155453886	-154886534842.125\\
3.26228155703893	-154844135965.285\\
3.26238155953899	-154801164130.651\\
3.26248156203905	-154758192296.016\\
3.26258156453911	-154715793419.176\\
3.26268156703918	-154672821584.541\\
3.26278156953924	-154629849749.907\\
3.2628815720393	-154587450873.067\\
3.26298157453936	-154544479038.432\\
3.26308157703943	-154502080161.592\\
3.26318157953949	-154459108326.958\\
3.26328158203955	-154416136492.323\\
3.26338158453961	-154373737615.483\\
3.26348158703968	-154330765780.848\\
3.26358158953974	-154288366904.009\\
3.2636815920398	-154245395069.374\\
3.26378159453986	-154202996192.534\\
3.26388159703993	-154160024357.899\\
3.26398159953999	-154117625481.06\\
3.26408160204005	-154074653646.425\\
3.26418160454011	-154032254769.585\\
3.26428160704018	-153989282934.95\\
3.26438160954024	-153946884058.111\\
3.2644816120403	-153903912223.476\\
3.26458161454036	-153861513346.636\\
3.26468161704043	-153818541512.001\\
3.26478161954049	-153776142635.162\\
3.26488162204055	-153733170800.527\\
3.26498162454061	-153690771923.687\\
3.26508162704068	-153647800089.052\\
3.26518162954074	-153605401212.213\\
3.2652816320408	-153562429377.578\\
3.26538163454086	-153520030500.738\\
3.26548163704093	-153477631623.898\\
3.26558163954099	-153434659789.264\\
3.26568164204105	-153392260912.424\\
3.26578164454111	-153349289077.789\\
3.26588164704118	-153306890200.949\\
3.26598164954124	-153264491324.11\\
3.2660816520413	-153221519489.475\\
3.26618165454136	-153179120612.635\\
3.26628165704143	-153136721735.796\\
3.26638165954149	-153093749901.161\\
3.26648166204155	-153051351024.321\\
3.26658166454161	-153008952147.481\\
3.26668166704168	-152965980312.847\\
3.26678166954174	-152923581436.007\\
3.2668816720418	-152881182559.167\\
3.26698167454186	-152838210724.533\\
3.26708167704193	-152795811847.693\\
3.26718167954199	-152753412970.853\\
3.26728168204205	-152711014094.013\\
3.26738168454211	-152668042259.379\\
3.26748168704218	-152625643382.539\\
3.26758168954224	-152583244505.699\\
3.2676816920423	-152540845628.86\\
3.26778169454236	-152497873794.225\\
3.26788169704243	-152455474917.385\\
3.26798169954249	-152413076040.545\\
3.26808170204255	-152370677163.706\\
3.26818170454261	-152328278286.866\\
3.26828170704268	-152285306452.231\\
3.26838170954274	-152242907575.392\\
3.2684817120428	-152200508698.552\\
3.26858171454286	-152158109821.712\\
3.26868171704293	-152115710944.873\\
3.26878171954299	-152073312068.033\\
3.26888172204305	-152030340233.398\\
3.26898172454311	-151987941356.558\\
3.26908172704318	-151945542479.719\\
3.26918172954324	-151903143602.879\\
3.2692817320433	-151860744726.039\\
3.26938173454336	-151818345849.2\\
3.26948173704343	-151775946972.36\\
3.26958173954349	-151733548095.52\\
3.26968174204355	-151690576260.885\\
3.26978174454361	-151648177384.046\\
3.26988174704368	-151605778507.206\\
3.26998174954374	-151563379630.366\\
3.2700817520438	-151520980753.527\\
3.27018175454386	-151478581876.687\\
3.27028175704393	-151436182999.847\\
3.27038175954399	-151393784123.008\\
3.27048176204405	-151351385246.168\\
3.27058176454411	-151308986369.328\\
3.27068176704418	-151266587492.489\\
3.27078176954424	-151224188615.649\\
3.2708817720443	-151181789738.809\\
3.27098177454436	-151139390861.97\\
3.27108177704443	-151096991985.13\\
3.27118177954449	-151054593108.29\\
3.27128178204455	-151012194231.451\\
3.27138178454461	-150969795354.611\\
3.27148178704468	-150927396477.771\\
3.27158178954474	-150884997600.932\\
3.2716817920448	-150842598724.092\\
3.27178179454486	-150800772805.047\\
3.27188179704493	-150758373928.208\\
3.27198179954499	-150715975051.368\\
3.27208180204505	-150673576174.528\\
3.27218180454511	-150631177297.689\\
3.27228180704518	-150588778420.849\\
3.27238180954524	-150546379544.009\\
3.2724818120453	-150503980667.17\\
3.27258181454536	-150461581790.33\\
3.27268181704543	-150419755871.285\\
3.27278181954549	-150377356994.446\\
3.27288182204555	-150334958117.606\\
3.27298182454561	-150292559240.766\\
3.27308182704568	-150250160363.927\\
3.27318182954574	-150208334444.882\\
3.2732818320458	-150165935568.042\\
3.27338183454586	-150123536691.203\\
3.27348183704593	-150081137814.363\\
3.27358183954599	-150038738937.523\\
3.27368184204605	-149996913018.479\\
3.27378184454611	-149954514141.639\\
3.27388184704618	-149912115264.799\\
3.27398184954624	-149869716387.96\\
3.2740818520463	-149827890468.915\\
3.27418185454636	-149785491592.075\\
3.27428185704643	-149743092715.236\\
3.27438185954649	-149700693838.396\\
3.27448186204655	-149658867919.352\\
3.27458186454661	-149616469042.512\\
3.27468186704668	-149574070165.672\\
3.27478186954674	-149532244246.628\\
3.2748818720468	-149489845369.788\\
3.27498187454686	-149447446492.948\\
3.27508187704693	-149405620573.904\\
3.27518187954699	-149363221697.064\\
3.27528188204705	-149320822820.224\\
3.27538188454711	-149278996901.18\\
3.27548188704718	-149236598024.34\\
3.27558188954724	-149194772105.296\\
3.2756818920473	-149152373228.456\\
3.27578189454736	-149109974351.616\\
3.27588189704743	-149068148432.572\\
3.27598189954749	-149025749555.732\\
3.27608190204755	-148983923636.687\\
3.27618190454761	-148941524759.848\\
3.27628190704768	-148899698840.803\\
3.27638190954774	-148857299963.964\\
3.2764819120478	-148814901087.124\\
3.27658191454786	-148773075168.079\\
3.27668191704793	-148730676291.24\\
3.27678191954799	-148688850372.195\\
3.27688192204805	-148646451495.355\\
3.27698192454811	-148604625576.311\\
3.27708192704818	-148562226699.471\\
3.27718192954824	-148520400780.427\\
3.2772819320483	-148478001903.587\\
3.27738193454836	-148436175984.542\\
3.27748193704843	-148393777107.703\\
3.27758193954849	-148351951188.658\\
3.27768194204855	-148310125269.614\\
3.27778194454861	-148267726392.774\\
3.27788194704868	-148225900473.729\\
3.27798194954874	-148183501596.89\\
3.2780819520488	-148141675677.845\\
3.27818195454886	-148099276801.005\\
3.27828195704893	-148057450881.961\\
3.27838195954899	-148015624962.916\\
3.27848196204905	-147973226086.077\\
3.27858196454911	-147931400167.032\\
3.27868196704918	-147889574247.988\\
3.27878196954924	-147847175371.148\\
3.2788819720493	-147805349452.103\\
3.27898197454936	-147763523533.059\\
3.27908197704943	-147721124656.219\\
3.27918197954949	-147679298737.175\\
3.27928198204955	-147637472818.13\\
3.27938198454961	-147595073941.29\\
3.27948198704968	-147553248022.246\\
3.27958198954974	-147511422103.201\\
3.2796819920498	-147469023226.362\\
3.27978199454986	-147427197307.317\\
3.27988199704993	-147385371388.272\\
3.27998199954999	-147343545469.228\\
3.28008200205005	-147301146592.388\\
3.28018200455011	-147259320673.344\\
3.28028200705018	-147217494754.299\\
3.28038200955024	-147175668835.255\\
3.2804820120503	-147133269958.415\\
3.28058201455036	-147091444039.37\\
3.28068201705043	-147049618120.326\\
3.28078201955049	-147007792201.281\\
3.28088202205055	-146965966282.237\\
3.28098202455061	-146923567405.397\\
3.28108202705068	-146881741486.352\\
3.28118202955074	-146839915567.308\\
3.2812820320508	-146798089648.263\\
3.28138203455086	-146756263729.219\\
3.28148203705093	-146714437810.174\\
3.28158203955099	-146672611891.13\\
3.28168204205105	-146630213014.29\\
3.28178204455111	-146588387095.245\\
3.28188204705118	-146546561176.201\\
3.28198204955124	-146504735257.156\\
3.2820820520513	-146462909338.112\\
3.28218205455136	-146421083419.067\\
3.28228205705143	-146379257500.023\\
3.28238205955149	-146337431580.978\\
3.28248206205155	-146295605661.934\\
3.28258206455161	-146253779742.889\\
3.28268206705168	-146211953823.845\\
3.28278206955174	-146170127904.8\\
3.2828820720518	-146128301985.755\\
3.28298207455186	-146086476066.711\\
3.28308207705193	-146044650147.666\\
3.28318207955199	-146002824228.622\\
3.28328208205205	-145960998309.577\\
3.28338208455211	-145919172390.533\\
3.28348208705218	-145877346471.488\\
3.28358208955224	-145835520552.444\\
3.2836820920523	-145793694633.399\\
3.28378209455236	-145751868714.354\\
3.28388209705243	-145710042795.31\\
3.28398209955249	-145668216876.265\\
3.28408210205255	-145626390957.221\\
3.28418210455261	-145584565038.176\\
3.28428210705268	-145542739119.132\\
3.28438210955274	-145500913200.087\\
3.2844821120528	-145459660238.838\\
3.28458211455286	-145417834319.793\\
3.28468211705293	-145376008400.749\\
3.28478211955299	-145334182481.704\\
3.28488212205305	-145292356562.66\\
3.28498212455311	-145250530643.615\\
3.28508212705318	-145208704724.57\\
3.28518212955324	-145167451763.321\\
3.2852821320533	-145125625844.276\\
3.28538213455336	-145083799925.232\\
3.28548213705343	-145041974006.187\\
3.28558213955349	-145000148087.143\\
3.28568214205355	-144958895125.893\\
3.28578214455361	-144917069206.849\\
3.28588214705368	-144875243287.804\\
3.28598214955374	-144833417368.76\\
3.2860821520538	-144792164407.51\\
3.28618215455386	-144750338488.466\\
3.28628215705393	-144708512569.421\\
3.28638215955399	-144666686650.377\\
3.28648216205405	-144625433689.127\\
3.28658216455411	-144583607770.083\\
3.28668216705418	-144541781851.038\\
3.28678216955424	-144500528889.789\\
3.2868821720543	-144458702970.744\\
3.28698217455436	-144416877051.7\\
3.28708217705443	-144375624090.45\\
3.28718217955449	-144333798171.406\\
3.28728218205455	-144291972252.361\\
3.28738218455461	-144250719291.112\\
3.28748218705468	-144208893372.067\\
3.28758218955474	-144167067453.023\\
3.2876821920548	-144125814491.773\\
3.28778219455486	-144083988572.729\\
3.28788219705493	-144042162653.684\\
3.28798219955499	-144000909692.435\\
3.28808220205505	-143959083773.39\\
3.28818220455511	-143917830812.141\\
3.28828220705518	-143876004893.096\\
3.28838220955524	-143834751931.847\\
3.2884822120553	-143792926012.802\\
3.28858221455536	-143751673051.553\\
3.28868221705543	-143709847132.508\\
3.28878221955549	-143668021213.464\\
3.28888222205555	-143626768252.214\\
3.28898222455561	-143584942333.17\\
3.28908222705568	-143543689371.92\\
3.28918222955574	-143501863452.876\\
3.2892822320558	-143460610491.626\\
3.28938223455586	-143419357530.377\\
3.28948223705593	-143377531611.332\\
3.28958223955599	-143336278650.083\\
3.28968224205605	-143294452731.038\\
3.28978224455611	-143253199769.789\\
3.28988224705618	-143211373850.744\\
3.28998224955624	-143170120889.495\\
3.2900822520563	-143128294970.45\\
3.29018225455636	-143087042009.201\\
3.29028225705643	-143045789047.952\\
3.29038225955649	-143003963128.907\\
3.29048226205655	-142962710167.658\\
3.29058226455661	-142921457206.408\\
3.29068226705668	-142879631287.364\\
3.29078226955674	-142838378326.114\\
3.2908822720568	-142796552407.07\\
3.29098227455686	-142755299445.82\\
3.29108227705693	-142714046484.571\\
3.29118227955699	-142672793523.321\\
3.29128228205705	-142630967604.277\\
3.29138228455711	-142589714643.027\\
3.29148228705718	-142548461681.778\\
3.29158228955724	-142506635762.733\\
3.2916822920573	-142465382801.484\\
3.29178229455736	-142424129840.235\\
3.29188229705743	-142382876878.985\\
3.29198229955749	-142341050959.941\\
3.29208230205755	-142299797998.691\\
3.29218230455761	-142258545037.442\\
3.29228230705768	-142217292076.192\\
3.29238230955774	-142175466157.148\\
3.2924823120578	-142134213195.898\\
3.29258231455786	-142092960234.649\\
3.29268231705793	-142051707273.4\\
3.29278231955799	-142010454312.15\\
3.29288232205805	-141968628393.106\\
3.29298232455811	-141927375431.856\\
3.29308232705818	-141886122470.607\\
3.29318232955824	-141844869509.357\\
3.2932823320583	-141803616548.108\\
3.29338233455836	-141762363586.859\\
3.29348233705843	-141721110625.609\\
3.29358233955849	-141679284706.565\\
3.29368234205855	-141638031745.315\\
3.29378234455861	-141596778784.066\\
3.29388234705868	-141555525822.816\\
3.29398234955874	-141514272861.567\\
3.2940823520588	-141473019900.317\\
3.29418235455886	-141431766939.068\\
3.29428235705893	-141390513977.819\\
3.29438235955899	-141349261016.569\\
3.29448236205905	-141308008055.32\\
3.29458236455911	-141266755094.07\\
3.29468236705918	-141225502132.821\\
3.29478236955924	-141184249171.572\\
3.2948823720593	-141142996210.322\\
3.29498237455936	-141101743249.073\\
3.29508237705943	-141060490287.823\\
3.29518237955949	-141019237326.574\\
3.29528238205955	-140977984365.324\\
3.29538238455961	-140936731404.075\\
3.29548238705968	-140895478442.826\\
3.29558238955974	-140854225481.576\\
3.2956823920598	-140812972520.327\\
3.29578239455986	-140771719559.077\\
3.29588239705993	-140730466597.828\\
3.29598239955999	-140689213636.579\\
3.29608240206005	-140647960675.329\\
3.29618240456011	-140607280671.875\\
3.29628240706018	-140566027710.625\\
3.29638240956024	-140524774749.376\\
3.2964824120603	-140483521788.127\\
3.29658241456036	-140442268826.877\\
3.29668241706043	-140401015865.628\\
3.29678241956049	-140359762904.378\\
3.29688242206055	-140319082900.924\\
3.29698242456061	-140277829939.675\\
3.29708242706068	-140236576978.425\\
3.29718242956074	-140195324017.176\\
3.2972824320608	-140154071055.926\\
3.29738243456086	-140113391052.472\\
3.29748243706093	-140072138091.223\\
3.29758243956099	-140030885129.973\\
3.29768244206105	-139989632168.724\\
3.29778244456111	-139948952165.27\\
3.29788244706118	-139907699204.02\\
3.29798244956124	-139866446242.771\\
3.2980824520613	-139825193281.521\\
3.29818245456136	-139784513278.067\\
3.29828245706143	-139743260316.818\\
3.29838245956149	-139702007355.568\\
3.29848246206155	-139661327352.114\\
3.29858246456161	-139620074390.864\\
3.29868246706168	-139578821429.615\\
3.29878246956174	-139538141426.161\\
3.2988824720618	-139496888464.911\\
3.29898247456186	-139455635503.662\\
3.29908247706193	-139414955500.208\\
3.29918247956199	-139373702538.958\\
3.29928248206205	-139332449577.709\\
3.29938248456211	-139291769574.254\\
3.29948248706218	-139250516613.005\\
3.29958248956224	-139209836609.551\\
3.2996824920623	-139168583648.301\\
3.29978249456236	-139127903644.847\\
3.29988249706243	-139086650683.598\\
3.29998249956249	-139045397722.348\\
3.30008250206255	-139004717718.894\\
3.30018250456261	-138963464757.644\\
3.30028250706268	-138922784754.19\\
3.30038250956274	-138881531792.941\\
3.3004825120628	-138840851789.486\\
3.30058251456286	-138799598828.237\\
3.30068251706293	-138758918824.783\\
3.30078251956299	-138717665863.533\\
3.30088252206305	-138676985860.079\\
3.30098252456311	-138635732898.83\\
3.30108252706318	-138595052895.375\\
3.30118252956324	-138554372891.921\\
3.3012825320633	-138513119930.672\\
3.30138253456336	-138472439927.217\\
3.30148253706343	-138431186965.968\\
3.30158253956349	-138390506962.514\\
3.30168254206355	-138349254001.264\\
3.30178254456361	-138308573997.81\\
3.30188254706368	-138267893994.356\\
3.30198254956374	-138226641033.106\\
3.3020825520638	-138185961029.652\\
3.30218255456386	-138145281026.198\\
3.30228255706393	-138104028064.948\\
3.30238255956399	-138063348061.494\\
3.30248256206405	-138022668058.04\\
3.30258256456411	-137981415096.79\\
3.30268256706418	-137940735093.336\\
3.30278256956424	-137900055089.882\\
3.3028825720643	-137858802128.632\\
3.30298257456436	-137818122125.178\\
3.30308257706443	-137777442121.724\\
3.30318257956449	-137736189160.474\\
3.30328258206455	-137695509157.02\\
3.30338258456461	-137654829153.566\\
3.30348258706468	-137614149150.111\\
3.30358258956474	-137572896188.862\\
3.3036825920648	-137532216185.408\\
3.30378259456486	-137491536181.953\\
3.30388259706493	-137450856178.499\\
3.30398259956499	-137410176175.045\\
3.30408260206505	-137368923213.795\\
3.30418260456511	-137328243210.341\\
3.30428260706518	-137287563206.887\\
3.30438260956524	-137246883203.433\\
3.3044826120653	-137206203199.978\\
3.30458261456536	-137165523196.524\\
3.30468261706543	-137124843193.07\\
3.30478261956549	-137083590231.82\\
3.30488262206555	-137042910228.366\\
3.30498262456561	-137002230224.912\\
3.30508262706568	-136961550221.457\\
3.30518262956574	-136920870218.003\\
3.3052826320658	-136880190214.549\\
3.30538263456586	-136839510211.095\\
3.30548263706593	-136798830207.64\\
3.30558263956599	-136758150204.186\\
3.30568264206605	-136717470200.732\\
3.30578264456611	-136676790197.277\\
3.30588264706618	-136636110193.823\\
3.30598264956624	-136595430190.369\\
3.3060826520663	-136554750186.914\\
3.30618265456636	-136514070183.46\\
3.30628265706643	-136473390180.006\\
3.30638265956649	-136432710176.552\\
3.30648266206655	-136392030173.097\\
3.30658266456661	-136351350169.643\\
3.30668266706668	-136310670166.189\\
3.30678266956674	-136269990162.734\\
3.3068826720668	-136229310159.28\\
3.30698267456686	-136188630155.826\\
3.30708267706693	-136147950152.372\\
3.30718267956699	-136107270148.917\\
3.30728268206705	-136066590145.463\\
3.30738268456711	-136025910142.009\\
3.30748268706718	-135985230138.554\\
3.30758268956724	-135945123092.895\\
3.3076826920673	-135904443089.441\\
3.30778269456736	-135863763085.987\\
3.30788269706743	-135823083082.532\\
3.30798269956749	-135782403079.078\\
3.30808270206755	-135741723075.624\\
3.30818270456761	-135701616029.965\\
3.30828270706768	-135660936026.51\\
3.30838270956774	-135620256023.056\\
3.3084827120678	-135579576019.602\\
3.30858271456786	-135538896016.148\\
3.30868271706793	-135498788970.488\\
3.30878271956799	-135458108967.034\\
3.30888272206805	-135417428963.58\\
3.30898272456811	-135376748960.126\\
3.30908272706818	-135336641914.466\\
3.30918272956824	-135295961911.012\\
3.3092827320683	-135255281907.558\\
3.30938273456836	-135214601904.104\\
3.30948273706843	-135174494858.444\\
3.30958273956849	-135133814854.99\\
3.30968274206855	-135093134851.536\\
3.30978274456861	-135053027805.877\\
3.30988274706868	-135012347802.422\\
3.30998274956874	-134971667798.968\\
3.3100827520688	-134931560753.309\\
3.31018275456886	-134890880749.855\\
3.31028275706893	-134850200746.4\\
3.31038275956899	-134810093700.741\\
3.31048276206905	-134769413697.287\\
3.31058276456911	-134729306651.628\\
3.31068276706918	-134688626648.173\\
3.31078276956924	-134647946644.719\\
3.3108827720693	-134607839599.06\\
3.31098277456936	-134567159595.606\\
3.31108277706943	-134527052549.947\\
3.31118277956949	-134486372546.492\\
3.31128278206955	-134446265500.833\\
3.31138278456961	-134405585497.379\\
3.31148278706968	-134365478451.72\\
3.31158278956974	-134324798448.265\\
3.3116827920698	-134284691402.606\\
3.31178279456986	-134244011399.152\\
3.31188279706993	-134203904353.493\\
3.31198279956999	-134163224350.038\\
3.31208280207005	-134123117304.379\\
3.31218280457011	-134082437300.925\\
3.31228280707018	-134042330255.266\\
3.31238280957024	-134001650251.812\\
3.3124828120703	-133961543206.152\\
3.31258281457036	-133921436160.493\\
3.31268281707043	-133880756157.039\\
3.31278281957049	-133840649111.38\\
3.31288282207055	-133799969107.925\\
3.31298282457061	-133759862062.266\\
3.31308282707068	-133719755016.607\\
3.31318282957074	-133679075013.153\\
3.3132828320708	-133638967967.494\\
3.31338283457086	-133598860921.835\\
3.31348283707093	-133558180918.38\\
3.31358283957099	-133518073872.721\\
3.31368284207105	-133477966827.062\\
3.31378284457111	-133437286823.608\\
3.31388284707118	-133397179777.949\\
3.31398284957124	-133357072732.289\\
3.3140828520713	-133316392728.835\\
3.31418285457136	-133276285683.176\\
3.31428285707143	-133236178637.517\\
3.31438285957149	-133196071591.858\\
3.31448286207155	-133155391588.403\\
3.31458286457161	-133115284542.744\\
3.31468286707168	-133075177497.085\\
3.31478286957174	-133035070451.426\\
3.3148828720718	-132994390447.972\\
3.31498287457186	-132954283402.312\\
3.31508287707193	-132914176356.653\\
3.31518287957199	-132874069310.994\\
3.31528288207205	-132833962265.335\\
3.31538288457211	-132793855219.676\\
3.31548288707218	-132753175216.221\\
3.31558288957224	-132713068170.562\\
3.3156828920723	-132672961124.903\\
3.31578289457236	-132632854079.244\\
3.31588289707243	-132592747033.585\\
3.31598289957249	-132552639987.926\\
3.31608290207255	-132512532942.267\\
3.31618290457261	-132472425896.607\\
3.31628290707268	-132432318850.948\\
3.31638290957274	-132392211805.289\\
3.3164829120728	-132351531801.835\\
3.31658291457286	-132311424756.176\\
3.31668291707293	-132271317710.516\\
3.31678291957299	-132231210664.857\\
3.31688292207305	-132191103619.198\\
3.31698292457311	-132150996573.539\\
3.31708292707318	-132110889527.88\\
3.31718292957324	-132070782482.221\\
3.3172829320733	-132030675436.562\\
3.31738293457336	-131990568390.902\\
3.31748293707343	-131950461345.243\\
3.31758293957349	-131910354299.584\\
3.31768294207355	-131870820211.72\\
3.31778294457361	-131830713166.061\\
3.31788294707368	-131790606120.402\\
3.31798294957374	-131750499074.743\\
3.3180829520738	-131710392029.083\\
3.31818295457386	-131670284983.424\\
3.31828295707393	-131630177937.765\\
3.31838295957399	-131590070892.106\\
3.31848296207405	-131549963846.447\\
3.31858296457411	-131509856800.788\\
3.31868296707418	-131470322712.924\\
3.31878296957424	-131430215667.264\\
3.3188829720743	-131390108621.605\\
3.31898297457436	-131350001575.946\\
3.31908297707443	-131309894530.287\\
3.31918297957449	-131269787484.628\\
3.31928298207455	-131230253396.764\\
3.31938298457461	-131190146351.105\\
3.31948298707468	-131150039305.445\\
3.31958298957474	-131109932259.786\\
3.3196829920748	-131070398171.922\\
3.31978299457486	-131030291126.263\\
3.31988299707493	-130990184080.604\\
3.31998299957499	-130950077034.945\\
3.32008300207505	-130910542947.081\\
3.32018300457511	-130870435901.422\\
3.32028300707518	-130830328855.762\\
3.32038300957524	-130790794767.898\\
3.3204830120753	-130750687722.239\\
3.32058301457536	-130710580676.58\\
3.32068301707543	-130671046588.716\\
3.32078301957549	-130630939543.057\\
3.32088302207555	-130590832497.398\\
3.32098302457561	-130551298409.534\\
3.32108302707568	-130511191363.875\\
3.32118302957574	-130471084318.215\\
3.3212830320758	-130431550230.351\\
3.32138303457586	-130391443184.692\\
3.32148303707593	-130351909096.828\\
3.32158303957599	-130311802051.169\\
3.32168304207605	-130271695005.51\\
3.32178304457611	-130232160917.646\\
3.32188304707618	-130192053871.987\\
3.32198304957624	-130152519784.123\\
3.3220830520763	-130112412738.464\\
3.32218305457636	-130072878650.599\\
3.32228305707643	-130032771604.94\\
3.32238305957649	-129993237517.076\\
3.32248306207655	-129953130471.417\\
3.32258306457661	-129913596383.553\\
3.32268306707668	-129873489337.894\\
3.32278306957674	-129833955250.03\\
3.3228830720768	-129793848204.371\\
3.32298307457686	-129754314116.507\\
3.32308307707693	-129714207070.848\\
3.32318307957699	-129674672982.984\\
3.32328308207705	-129635138895.12\\
3.32338308457711	-129595031849.46\\
3.32348308707718	-129555497761.596\\
3.32358308957724	-129515390715.937\\
3.3236830920773	-129475856628.073\\
3.32378309457736	-129436322540.209\\
3.32388309707743	-129396215494.55\\
3.32398309957749	-129356681406.686\\
3.32408310207755	-129317147318.822\\
3.32418310457761	-129277040273.163\\
3.32428310707768	-129237506185.299\\
3.32438310957774	-129197972097.435\\
3.3244831120778	-129157865051.776\\
3.32458311457786	-129118330963.912\\
3.32468311707793	-129078796876.048\\
3.32478311957799	-129038689830.388\\
3.32488312207805	-128999155742.524\\
3.32498312457811	-128959621654.66\\
3.32508312707818	-128920087566.796\\
3.32518312957824	-128879980521.137\\
3.3252831320783	-128840446433.273\\
3.32538313457836	-128800912345.409\\
3.32548313707843	-128761378257.545\\
3.32558313957849	-128721844169.681\\
3.32568314207855	-128681737124.022\\
3.32578314457861	-128642203036.158\\
3.32588314707868	-128602668948.294\\
3.32598314957874	-128563134860.43\\
3.3260831520788	-128523600772.566\\
3.32618315457886	-128484066684.702\\
3.32628315707893	-128443959639.043\\
3.32638315957899	-128404425551.179\\
3.32648316207905	-128364891463.314\\
3.32658316457911	-128325357375.45\\
3.32668316707918	-128285823287.586\\
3.32678316957924	-128246289199.722\\
3.3268831720793	-128206755111.858\\
3.32698317457936	-128167221023.994\\
3.32708317707943	-128127686936.13\\
3.32718317957949	-128088152848.266\\
3.32728318207955	-128048618760.402\\
3.32738318457961	-128009084672.538\\
3.32748318707968	-127969550584.674\\
3.32758318957974	-127930016496.81\\
3.3276831920798	-127890482408.946\\
3.32778319457986	-127850948321.082\\
3.32788319707993	-127811414233.218\\
3.32798319957999	-127771880145.354\\
3.32808320208005	-127732346057.49\\
3.32818320458011	-127692811969.626\\
3.32828320708018	-127653277881.762\\
3.32838320958024	-127613743793.898\\
3.3284832120803	-127574209706.034\\
3.32858321458036	-127534675618.17\\
3.32868321708043	-127495141530.306\\
3.32878321958049	-127455607442.442\\
3.32888322208055	-127416073354.578\\
3.32898322458061	-127376539266.714\\
3.32908322708068	-127337578136.645\\
3.32918322958074	-127298044048.781\\
3.3292832320808	-127258509960.917\\
3.32938323458086	-127218975873.053\\
3.32948323708093	-127179441785.189\\
3.32958323958099	-127139907697.325\\
3.32968324208105	-127100946567.256\\
3.32978324458111	-127061412479.392\\
3.32988324708118	-127021878391.528\\
3.32998324958124	-126982344303.664\\
3.3300832520813	-126942810215.8\\
3.33018325458136	-126903849085.731\\
3.33028325708143	-126864314997.867\\
3.33038325958149	-126824780910.003\\
3.33048326208155	-126785246822.139\\
3.33058326458161	-126746285692.07\\
3.33068326708168	-126706751604.206\\
3.33078326958174	-126667217516.342\\
3.3308832720818	-126628256386.273\\
3.33098327458186	-126588722298.409\\
3.33108327708193	-126549188210.545\\
3.33118327958199	-126510227080.476\\
3.33128328208205	-126470692992.612\\
3.33138328458211	-126431158904.748\\
3.33148328708218	-126392197774.679\\
3.33158328958224	-126352663686.815\\
3.3316832920823	-126313129598.951\\
3.33178329458236	-126274168468.882\\
3.33188329708243	-126234634381.018\\
3.33198329958249	-126195673250.949\\
3.33208330208255	-126156139163.085\\
3.33218330458261	-126117178033.016\\
3.33228330708268	-126077643945.152\\
3.33238330958274	-126038109857.288\\
3.3324833120828	-125999148727.219\\
3.33258331458286	-125959614639.355\\
3.33268331708293	-125920653509.286\\
3.33278331958299	-125881119421.422\\
3.33288332208305	-125842158291.353\\
3.33298332458311	-125802624203.489\\
3.33308332708318	-125763663073.421\\
3.33318332958324	-125724128985.557\\
3.3332833320833	-125685167855.488\\
3.33338333458336	-125645633767.624\\
3.33348333708343	-125606672637.555\\
3.33358333958349	-125567711507.486\\
3.33368334208355	-125528177419.622\\
3.33378334458361	-125489216289.553\\
3.33388334708368	-125449682201.689\\
3.33398334958374	-125410721071.62\\
3.3340833520838	-125371759941.551\\
3.33418335458386	-125332225853.687\\
3.33428335708393	-125293264723.618\\
3.33438335958399	-125253730635.754\\
3.33448336208405	-125214769505.685\\
3.33458336458411	-125175808375.616\\
3.33468336708418	-125136274287.752\\
3.33478336958424	-125097313157.683\\
3.3348833720843	-125058352027.615\\
3.33498337458436	-125019390897.546\\
3.33508337708443	-124979856809.682\\
3.33518337958449	-124940895679.613\\
3.33528338208455	-124901934549.544\\
3.33538338458461	-124862400461.68\\
3.33548338708468	-124823439331.611\\
3.33558338958474	-124784478201.542\\
3.3356833920848	-124745517071.473\\
3.33578339458486	-124706555941.404\\
3.33588339708493	-124667021853.54\\
3.33598339958499	-124628060723.471\\
3.33608340208505	-124589099593.402\\
3.33618340458511	-124550138463.333\\
3.33628340708518	-124511177333.265\\
3.33638340958524	-124471643245.401\\
3.3364834120853	-124432682115.332\\
3.33658341458536	-124393720985.263\\
3.33668341708543	-124354759855.194\\
3.33678341958549	-124315798725.125\\
3.33688342208555	-124276837595.056\\
3.33698342458561	-124237876464.987\\
3.33708342708568	-124198915334.918\\
3.33718342958574	-124159954204.849\\
3.3372834320858	-124120420116.985\\
3.33738343458586	-124081458986.916\\
3.33748343708593	-124042497856.848\\
3.33758343958599	-124003536726.779\\
3.33768344208605	-123964575596.71\\
3.33778344458611	-123925614466.641\\
3.33788344708618	-123886653336.572\\
3.33798344958624	-123847692206.503\\
3.3380834520863	-123808731076.434\\
3.33818345458636	-123769769946.365\\
3.33828345708643	-123730808816.296\\
3.33838345958649	-123691847686.228\\
3.33848346208655	-123652886556.159\\
3.33858346458661	-123613925426.09\\
3.33868346708668	-123575537253.816\\
3.33878346958674	-123536576123.747\\
3.3388834720868	-123497614993.678\\
3.33898347458686	-123458653863.609\\
3.33908347708693	-123419692733.54\\
3.33918347958699	-123380731603.471\\
3.33928348208705	-123341770473.403\\
3.33938348458711	-123302809343.334\\
3.33948348708718	-123263848213.265\\
3.33958348958724	-123225460040.991\\
3.3396834920873	-123186498910.922\\
3.33978349458736	-123147537780.853\\
3.33988349708743	-123108576650.784\\
3.33998349958749	-123069615520.715\\
3.34008350208755	-123031227348.442\\
3.34018350458761	-122992266218.373\\
3.34028350708768	-122953305088.304\\
3.34038350958774	-122914343958.235\\
3.3404835120878	-122875382828.166\\
3.34058351458786	-122836994655.892\\
3.34068351708793	-122798033525.823\\
3.34078351958799	-122759072395.755\\
3.34088352208805	-122720111265.686\\
3.34098352458811	-122681723093.412\\
3.34108352708818	-122642761963.343\\
3.34118352958824	-122603800833.274\\
3.3412835320883	-122565412661\\
3.34138353458836	-122526451530.931\\
3.34148353708843	-122487490400.863\\
3.34158353958849	-122449102228.589\\
3.34168354208855	-122410141098.52\\
3.34178354458861	-122371179968.451\\
3.34188354708868	-122332791796.177\\
3.34198354958874	-122293830666.108\\
3.3420835520888	-122255442493.835\\
3.34218355458886	-122216481363.766\\
3.34228355708893	-122178093191.492\\
3.34238355958899	-122139132061.423\\
3.34248356208905	-122100170931.354\\
3.34258356458911	-122061782759.08\\
3.34268356708918	-122022821629.011\\
3.34278356958924	-121984433456.738\\
3.3428835720893	-121945472326.669\\
3.34298357458936	-121907084154.395\\
3.34308357708943	-121868123024.326\\
3.34318357958949	-121829734852.052\\
3.34328358208955	-121790773721.983\\
3.34338358458961	-121752385549.71\\
3.34348358708968	-121713424419.641\\
3.34358358958974	-121675036247.367\\
3.3436835920898	-121636075117.298\\
3.34378359458986	-121597686945.024\\
3.34388359708993	-121559298772.751\\
3.34398359958999	-121520337642.682\\
3.34408360209005	-121481949470.408\\
3.34418360459011	-121442988340.339\\
3.34428360709018	-121404600168.065\\
3.34438360959024	-121366211995.792\\
3.3444836120903	-121327250865.723\\
3.34458361459036	-121288862693.449\\
3.34468361709043	-121250474521.175\\
3.34478361959049	-121211513391.106\\
3.34488362209055	-121173125218.832\\
3.34498362459061	-121134737046.559\\
3.34508362709068	-121095775916.49\\
3.34518362959074	-121057387744.216\\
3.3452836320908	-121018999571.942\\
3.34538363459086	-120980611399.668\\
3.34548363709093	-120941650269.6\\
3.34558363959099	-120903262097.326\\
3.34568364209105	-120864873925.052\\
3.34578364459111	-120826485752.778\\
3.34588364709118	-120787524622.709\\
3.34598364959124	-120749136450.436\\
3.3460836520913	-120710748278.162\\
3.34618365459136	-120672360105.888\\
3.34628365709143	-120633971933.614\\
3.34638365959149	-120595010803.545\\
3.34648366209155	-120556622631.272\\
3.34658366459161	-120518234458.998\\
3.34668366709168	-120479846286.724\\
3.34678366959174	-120441458114.45\\
3.3468836720918	-120403069942.177\\
3.34698367459186	-120364681769.903\\
3.34708367709193	-120325720639.834\\
3.34718367959199	-120287332467.56\\
3.34728368209205	-120248944295.286\\
3.34738368459211	-120210556123.013\\
3.34748368709218	-120172167950.739\\
3.34758368959224	-120133779778.465\\
3.3476836920923	-120095391606.191\\
3.34778369459236	-120057003433.918\\
3.34788369709243	-120018615261.644\\
3.34798369959249	-119980227089.37\\
3.34808370209255	-119941838917.096\\
3.34818370459261	-119903450744.823\\
3.34828370709268	-119865062572.549\\
3.34838370959274	-119826674400.275\\
3.3484837120928	-119788286228.001\\
3.34858371459286	-119749898055.727\\
3.34868371709293	-119711509883.454\\
3.34878371959299	-119673121711.18\\
3.34888372209305	-119634733538.906\\
3.34898372459311	-119596345366.632\\
3.34908372709318	-119558530152.154\\
3.34918372959324	-119520141979.88\\
3.3492837320933	-119481753807.606\\
3.34938373459336	-119443365635.332\\
3.34948373709343	-119404977463.059\\
3.34958373959349	-119366589290.785\\
3.34968374209355	-119328201118.511\\
3.34978374459361	-119290385904.033\\
3.34988374709368	-119251997731.759\\
3.34998374959374	-119213609559.485\\
3.3500837520938	-119175221387.211\\
3.35018375459386	-119136833214.937\\
3.35028375709393	-119099018000.459\\
3.35038375959399	-119060629828.185\\
3.35048376209405	-119022241655.911\\
3.35058376459411	-118983853483.638\\
3.35068376709418	-118946038269.159\\
3.35078376959424	-118907650096.885\\
3.3508837720943	-118869261924.611\\
3.35098377459436	-118830873752.338\\
3.35108377709443	-118793058537.859\\
3.35118377959449	-118754670365.585\\
3.35128378209455	-118716282193.311\\
3.35138378459461	-118678466978.833\\
3.35148378709468	-118640078806.559\\
3.35158378959474	-118601690634.285\\
3.3516837920948	-118563875419.807\\
3.35178379459486	-118525487247.533\\
3.35188379709493	-118487099075.259\\
3.35198379959499	-118449283860.78\\
3.35208380209505	-118410895688.507\\
3.35218380459511	-118373080474.028\\
3.35228380709518	-118334692301.754\\
3.35238380959524	-118296304129.481\\
3.3524838120953	-118258488915.002\\
3.35258381459537	-118220100742.728\\
3.35268381709543	-118182285528.25\\
3.35278381959549	-118143897355.976\\
3.35288382209555	-118106082141.497\\
3.35298382459561	-118067693969.223\\
3.35308382709568	-118029878754.745\\
3.35318382959574	-117991490582.471\\
3.3532838320958	-117953675367.992\\
3.35338383459586	-117915287195.719\\
3.35348383709593	-117877471981.24\\
3.35358383959599	-117839083808.966\\
3.35368384209605	-117801268594.488\\
3.35378384459611	-117763453380.009\\
3.35388384709618	-117725065207.735\\
3.35398384959624	-117687249993.256\\
3.3540838520963	-117648861820.983\\
3.35418385459636	-117611046606.504\\
3.35428385709643	-117573231392.025\\
3.35438385959649	-117534843219.752\\
3.35448386209655	-117497028005.273\\
3.35458386459662	-117459212790.794\\
3.35468386709668	-117420824618.521\\
3.35478386959674	-117383009404.042\\
3.3548838720968	-117345194189.563\\
3.35498387459686	-117306806017.29\\
3.35508387709693	-117268990802.811\\
3.35518387959699	-117231175588.332\\
3.35528388209705	-117192787416.059\\
3.35538388459711	-117154972201.58\\
3.35548388709718	-117117156987.101\\
3.35558388959724	-117079341772.623\\
3.3556838920973	-117040953600.349\\
3.35578389459736	-117003138385.87\\
3.35588389709743	-116965323171.392\\
3.35598389959749	-116927507956.913\\
3.35608390209755	-116889692742.434\\
3.35618390459761	-116851304570.161\\
3.35628390709768	-116813489355.682\\
3.35638390959774	-116775674141.203\\
3.3564839120978	-116737858926.725\\
3.35658391459787	-116700043712.246\\
3.35668391709793	-116662228497.767\\
3.35678391959799	-116624413283.289\\
3.35688392209805	-116586025111.015\\
3.35698392459812	-116548209896.536\\
3.35708392709818	-116510394682.058\\
3.35718392959824	-116472579467.579\\
3.3572839320983	-116434764253.101\\
3.35738393459836	-116396949038.622\\
3.35748393709843	-116359133824.143\\
3.35758393959849	-116321318609.665\\
3.35768394209855	-116283503395.186\\
3.35778394459861	-116245688180.707\\
3.35788394709868	-116207872966.229\\
3.35798394959874	-116170057751.75\\
3.3580839520988	-116132242537.271\\
3.35818395459886	-116094427322.793\\
3.35828395709893	-116056612108.314\\
3.35838395959899	-116018796893.836\\
3.35848396209905	-115980981679.357\\
3.35858396459912	-115943166464.878\\
3.35868396709918	-115905351250.4\\
3.35878396959924	-115867536035.921\\
3.3588839720993	-115829720821.442\\
3.35898397459937	-115791905606.964\\
3.35908397709943	-115754663350.28\\
3.35918397959949	-115716848135.802\\
3.35928398209955	-115679032921.323\\
3.35938398459961	-115641217706.844\\
3.35948398709968	-115603402492.366\\
3.35958398959974	-115565587277.887\\
3.3596839920998	-115527772063.408\\
3.35978399459986	-115490529806.725\\
3.35988399709993	-115452714592.246\\
3.35998399959999	-115414899377.768\\
3.36008400210005	-115377084163.289\\
3.36018400460011	-115339268948.81\\
3.36028400710018	-115302026692.127\\
3.36038400960024	-115264211477.648\\
3.3604840121003	-115226396263.17\\
3.36058401460037	-115188581048.691\\
3.36068401710043	-115151338792.007\\
3.36078401960049	-115113523577.529\\
3.36088402210055	-115075708363.05\\
3.36098402460062	-115038466106.367\\
3.36108402710068	-115000650891.888\\
3.36118402960074	-114962835677.409\\
3.3612840321008	-114925593420.726\\
3.36138403460086	-114887778206.247\\
3.36148403710093	-114849962991.769\\
3.36158403960099	-114812720735.085\\
3.36168404210105	-114774905520.607\\
3.36178404460111	-114737090306.128\\
3.36188404710118	-114699848049.444\\
3.36198404960124	-114662032834.966\\
3.3620840521013	-114624790578.282\\
3.36218405460136	-114586975363.804\\
3.36228405710143	-114549733107.12\\
3.36238405960149	-114511917892.641\\
3.36248406210155	-114474675635.958\\
3.36258406460162	-114436860421.479\\
3.36268406710168	-114399045207.001\\
3.36278406960174	-114361802950.317\\
3.3628840721018	-114323987735.839\\
3.36298407460187	-114286745479.155\\
3.36308407710193	-114249503222.472\\
3.36318407960199	-114211688007.993\\
3.36328408210205	-114174445751.309\\
3.36338408460212	-114136630536.831\\
3.36348408710218	-114099388280.147\\
3.36358408960224	-114061573065.669\\
3.3636840921023	-114024330808.985\\
3.36378409460236	-113986515594.507\\
3.36388409710243	-113949273337.823\\
3.36398409960249	-113912031081.139\\
3.36408410210255	-113874215866.661\\
3.36418410460261	-113836973609.977\\
3.36428410710268	-113799731353.294\\
3.36438410960274	-113761916138.815\\
3.3644841121028	-113724673882.132\\
3.36458411460287	-113687431625.448\\
3.36468411710293	-113649616410.97\\
3.36478411960299	-113612374154.286\\
3.36488412210305	-113575131897.603\\
3.36498412460312	-113537316683.124\\
3.36508412710318	-113500074426.44\\
3.36518412960324	-113462832169.757\\
3.3652841321033	-113425589913.073\\
3.36538413460337	-113387774698.595\\
3.36548413710343	-113350532441.911\\
3.36558413960349	-113313290185.228\\
3.36568414210355	-113276047928.544\\
3.36578414460361	-113238805671.861\\
3.36588414710368	-113200990457.382\\
3.36598414960374	-113163748200.699\\
3.3660841521038	-113126505944.015\\
3.36618415460386	-113089263687.332\\
3.36628415710393	-113052021430.648\\
3.36638415960399	-113014779173.965\\
3.36648416210405	-112976963959.486\\
3.36658416460412	-112939721702.802\\
3.36668416710418	-112902479446.119\\
3.36678416960424	-112865237189.435\\
3.3668841721043	-112827994932.752\\
3.36698417460437	-112790752676.068\\
3.36708417710443	-112753510419.385\\
3.36718417960449	-112716268162.701\\
3.36728418210455	-112679025906.018\\
3.36738418460462	-112641783649.334\\
3.36748418710468	-112604541392.651\\
3.36758418960474	-112567299135.967\\
3.3676841921048	-112530056879.284\\
3.36778419460487	-112492814622.6\\
3.36788419710493	-112455572365.917\\
3.36798419960499	-112418330109.233\\
3.36808420210505	-112381087852.55\\
3.36818420460511	-112343845595.866\\
3.36828420710518	-112306603339.183\\
3.36838420960524	-112269361082.499\\
3.3684842121053	-112232118825.816\\
3.36858421460537	-112194876569.132\\
3.36868421710543	-112157634312.449\\
3.36878421960549	-112120392055.765\\
3.36888422210555	-112083722756.877\\
3.36898422460562	-112046480500.194\\
3.36908422710568	-112009238243.51\\
3.36918422960574	-111971995986.827\\
3.3692842321058	-111934753730.143\\
3.36938423460587	-111897511473.46\\
3.36948423710593	-111860842174.571\\
3.36958423960599	-111823599917.888\\
3.36968424210605	-111786357661.204\\
3.36978424460612	-111749115404.521\\
3.36988424710618	-111711873147.837\\
3.36998424960624	-111675203848.949\\
3.3700842521063	-111637961592.265\\
3.37018425460636	-111600719335.582\\
3.37028425710643	-111563477078.898\\
3.37038425960649	-111526807780.01\\
3.37048426210655	-111489565523.326\\
3.37058426460662	-111452323266.643\\
3.37068426710668	-111415653967.754\\
3.37078426960674	-111378411711.071\\
3.3708842721068	-111341169454.387\\
3.37098427460687	-111304500155.499\\
3.37108427710693	-111267257898.816\\
3.37118427960699	-111230015642.132\\
3.37128428210705	-111193346343.244\\
3.37138428460712	-111156104086.56\\
3.37148428710718	-111118861829.877\\
3.37158428960724	-111082192530.988\\
3.3716842921073	-111044950274.305\\
3.37178429460737	-111008280975.416\\
3.37188429710743	-110971038718.733\\
3.37198429960749	-110934369419.845\\
3.37208430210755	-110897127163.161\\
3.37218430460761	-110860457864.273\\
3.37228430710768	-110823215607.589\\
3.37238430960774	-110786546308.701\\
3.3724843121078	-110749304052.017\\
3.37258431460787	-110712634753.129\\
3.37268431710793	-110675392496.445\\
3.37278431960799	-110638723197.557\\
3.37288432210805	-110601480940.874\\
3.37298432460812	-110564811641.985\\
3.37308432710818	-110527569385.302\\
3.37318432960824	-110490900086.413\\
3.3732843321083	-110453657829.73\\
3.37338433460837	-110416988530.841\\
3.37348433710843	-110380319231.953\\
3.37358433960849	-110343076975.27\\
3.37368434210855	-110306407676.381\\
3.37378434460862	-110269165419.698\\
3.37388434710868	-110232496120.809\\
3.37398434960874	-110195826821.921\\
3.3740843521088	-110158584565.237\\
3.37418435460887	-110121915266.349\\
3.37428435710893	-110085245967.461\\
3.37438435960899	-110048576668.572\\
3.37448436210905	-110011334411.889\\
3.37458436460912	-109974665113\\
3.37468436710918	-109937995814.112\\
3.37478436960924	-109900753557.429\\
3.3748843721093	-109864084258.54\\
3.37498437460937	-109827414959.652\\
3.37508437710943	-109790745660.763\\
3.37518437960949	-109753503404.08\\
3.37528438210955	-109716834105.192\\
3.37538438460962	-109680164806.303\\
3.37548438710968	-109643495507.415\\
3.37558438960974	-109606826208.526\\
3.3756843921098	-109570156909.638\\
3.37578439460987	-109532914652.955\\
3.37588439710993	-109496245354.066\\
3.37598439960999	-109459576055.178\\
3.37608440211005	-109422906756.289\\
3.37618440461012	-109386237457.401\\
3.37628440711018	-109349568158.513\\
3.37638440961024	-109312898859.624\\
3.3764844121103	-109276229560.736\\
3.37658441461037	-109239560261.848\\
3.37668441711043	-109202318005.164\\
3.37678441961049	-109165648706.276\\
3.37688442211055	-109128979407.387\\
3.37698442461062	-109092310108.499\\
3.37708442711068	-109055640809.611\\
3.37718442961074	-109018971510.722\\
3.3772844321108	-108982302211.834\\
3.37738443461087	-108945632912.946\\
3.37748443711093	-108908963614.057\\
3.37758443961099	-108872294315.169\\
3.37768444211105	-108835625016.28\\
3.37778444461112	-108799528675.187\\
3.37788444711118	-108762859376.299\\
3.37798444961124	-108726190077.41\\
3.3780844521113	-108689520778.522\\
3.37818445461137	-108652851479.634\\
3.37828445711143	-108616182180.745\\
3.37838445961149	-108579512881.857\\
3.37848446211155	-108542843582.969\\
3.37858446461162	-108506174284.08\\
3.37868446711168	-108470077942.987\\
3.37878446961174	-108433408644.099\\
3.3788844721118	-108396739345.21\\
3.37898447461187	-108360070046.322\\
3.37908447711193	-108323400747.433\\
3.37918447961199	-108286731448.545\\
3.37928448211205	-108250635107.452\\
3.37938448461212	-108213965808.563\\
3.37948448711218	-108177296509.675\\
3.37958448961224	-108140627210.787\\
3.3796844921123	-108104530869.693\\
3.37978449461237	-108067861570.805\\
3.37988449711243	-108031192271.917\\
3.37998449961249	-107994522973.028\\
3.38008450211255	-107958426631.935\\
3.38018450461262	-107921757333.047\\
3.38028450711268	-107885088034.158\\
3.38038450961274	-107848991693.065\\
3.3804845121128	-107812322394.177\\
3.38058451461287	-107775653095.288\\
3.38068451711293	-107739556754.195\\
3.38078451961299	-107702887455.307\\
3.38088452211305	-107666791114.214\\
3.38098452461312	-107630121815.325\\
3.38108452711318	-107593452516.437\\
3.38118452961324	-107557356175.344\\
3.3812845321133	-107520686876.455\\
3.38138453461337	-107484590535.362\\
3.38148453711343	-107447921236.474\\
3.38158453961349	-107411824895.38\\
3.38168454211355	-107375155596.492\\
3.38178454461362	-107338486297.604\\
3.38188454711368	-107302389956.51\\
3.38198454961374	-107265720657.622\\
3.3820845521138	-107229624316.529\\
3.38218455461387	-107192955017.64\\
3.38228455711393	-107156858676.547\\
3.38238455961399	-107120762335.454\\
3.38248456211405	-107084093036.565\\
3.38258456461412	-107047996695.472\\
3.38268456711418	-107011327396.584\\
3.38278456961424	-106975231055.491\\
3.3828845721143	-106938561756.602\\
3.38298457461437	-106902465415.509\\
3.38308457711443	-106866369074.416\\
3.38318457961449	-106829699775.527\\
3.38328458211455	-106793603434.434\\
3.38338458461462	-106757507093.341\\
3.38348458711468	-106720837794.453\\
3.38358458961474	-106684741453.359\\
3.3836845921148	-106648645112.266\\
3.38378459461487	-106611975813.378\\
3.38388459711493	-106575879472.284\\
3.38398459961499	-106539783131.191\\
3.38408460211505	-106503113832.303\\
3.38418460461512	-106467017491.21\\
3.38428460711518	-106430921150.116\\
3.38438460961524	-106394824809.023\\
3.3844846121153	-106358155510.135\\
3.38458461461537	-106322059169.041\\
3.38468461711543	-106285962827.948\\
3.38478461961549	-106249866486.855\\
3.38488462211555	-106213770145.762\\
3.38498462461562	-106177100846.873\\
3.38508462711568	-106141004505.78\\
3.38518462961574	-106104908164.687\\
3.3852846321158	-106068811823.594\\
3.38538463461587	-106032715482.5\\
3.38548463711593	-105996619141.407\\
3.38558463961599	-105960522800.314\\
3.38568464211605	-105923853501.426\\
3.38578464461612	-105887757160.332\\
3.38588464711618	-105851660819.239\\
3.38598464961624	-105815564478.146\\
3.3860846521163	-105779468137.053\\
3.38618465461637	-105743371795.959\\
3.38628465711643	-105707275454.866\\
3.38638465961649	-105671179113.773\\
3.38648466211655	-105635082772.68\\
3.38658466461662	-105598986431.586\\
3.38668466711668	-105562890090.493\\
3.38678466961674	-105526793749.4\\
3.3868846721168	-105490697408.307\\
3.38698467461687	-105454601067.213\\
3.38708467711693	-105418504726.12\\
3.38718467961699	-105382408385.027\\
3.38728468211705	-105346312043.934\\
3.38738468461712	-105310215702.84\\
3.38748468711718	-105274119361.747\\
3.38758468961724	-105238023020.654\\
3.3876846921173	-105202499637.356\\
3.38778469461737	-105166403296.263\\
3.38788469711743	-105130306955.169\\
3.38798469961749	-105094210614.076\\
3.38808470211755	-105058114272.983\\
3.38818470461762	-105022017931.89\\
3.38828470711768	-104985921590.796\\
3.38838470961774	-104950398207.498\\
3.3884847121178	-104914301866.405\\
3.38858471461787	-104878205525.312\\
3.38868471711793	-104842109184.219\\
3.38878471961799	-104806012843.125\\
3.38888472211805	-104770489459.827\\
3.38898472461812	-104734393118.734\\
3.38908472711818	-104698296777.641\\
3.38918472961824	-104662200436.547\\
3.3892847321183	-104626677053.249\\
3.38938473461837	-104590580712.156\\
3.38948473711843	-104554484371.063\\
3.38958473961849	-104518960987.765\\
3.38968474211855	-104482864646.672\\
3.38978474461862	-104446768305.578\\
3.38988474711868	-104411244922.28\\
3.38998474961874	-104375148581.187\\
3.3900847521188	-104339052240.094\\
3.39018475461887	-104303528856.796\\
3.39028475711893	-104267432515.702\\
3.39038475961899	-104231336174.609\\
3.39048476211905	-104195812791.311\\
3.39058476461912	-104159716450.218\\
3.39068476711918	-104124193066.92\\
3.39078476961924	-104088096725.826\\
3.3908847721193	-104052573342.528\\
3.39098477461937	-104016477001.435\\
3.39108477711943	-103980953618.137\\
3.39118477961949	-103944857277.044\\
3.39128478211955	-103909333893.746\\
3.39138478461962	-103873237552.652\\
3.39148478711968	-103837714169.354\\
3.39158478961974	-103801617828.261\\
3.3916847921198	-103766094444.963\\
3.39178479461987	-103729998103.87\\
3.39188479711993	-103694474720.572\\
3.39198479961999	-103658378379.478\\
3.39208480212005	-103622854996.18\\
3.39218480462012	-103586758655.087\\
3.39228480712018	-103551235271.789\\
3.39238480962024	-103515711888.491\\
3.3924848121203	-103479615547.397\\
3.39258481462037	-103444092164.099\\
3.39268481712043	-103408568780.801\\
3.39278481962049	-103372472439.708\\
3.39288482212055	-103336949056.41\\
3.39298482462062	-103301425673.112\\
3.39308482712068	-103265329332.019\\
3.39318482962074	-103229805948.72\\
3.3932848321208	-103194282565.422\\
3.39338483462087	-103158186224.329\\
3.39348483712093	-103122662841.031\\
3.39358483962099	-103087139457.733\\
3.39368484212105	-103051616074.435\\
3.39378484462112	-103015519733.341\\
3.39388484712118	-102979996350.043\\
3.39398484962124	-102944472966.745\\
3.3940848521213	-102908949583.447\\
3.39418485462137	-102872853242.354\\
3.39428485712143	-102837329859.056\\
3.39438485962149	-102801806475.758\\
3.39448486212155	-102766283092.46\\
3.39458486462162	-102730759709.161\\
3.39468486712168	-102695236325.863\\
3.39478486962174	-102659712942.565\\
3.3948848721218	-102623616601.472\\
3.39498487462187	-102588093218.174\\
3.39508487712193	-102552569834.876\\
3.39518487962199	-102517046451.578\\
3.39528488212205	-102481523068.28\\
3.39538488462212	-102445999684.981\\
3.39548488712218	-102410476301.683\\
3.39558488962224	-102374952918.385\\
3.3956848921223	-102339429535.087\\
3.39578489462237	-102303906151.789\\
3.39588489712243	-102268382768.491\\
3.39598489962249	-102232859385.193\\
3.39608490212255	-102197336001.895\\
3.39618490462262	-102161812618.597\\
3.39628490712268	-102126289235.298\\
3.39638490962274	-102090765852\\
3.3964849121228	-102055242468.702\\
3.39658491462287	-102019719085.404\\
3.39668491712293	-101984195702.106\\
3.39678491962299	-101948672318.808\\
3.39688492212305	-101913148935.51\\
3.39698492462312	-101877625552.212\\
3.39708492712318	-101842102168.914\\
3.39718492962324	-101807151743.411\\
3.3972849321233	-101771628360.112\\
3.39738493462337	-101736104976.814\\
3.39748493712343	-101700581593.516\\
3.39758493962349	-101665058210.218\\
3.39768494212355	-101629534826.92\\
3.39778494462362	-101594584401.417\\
3.39788494712368	-101559061018.119\\
3.39798494962374	-101523537634.821\\
3.3980849521238	-101488014251.523\\
3.39818495462387	-101452490868.225\\
3.39828495712393	-101417540442.722\\
3.39838495962399	-101382017059.424\\
3.39848496212405	-101346493676.125\\
3.39858496462412	-101311543250.622\\
3.39868496712418	-101276019867.324\\
3.39878496962424	-101240496484.026\\
3.3988849721243	-101204973100.728\\
3.39898497462437	-101170022675.225\\
3.39908497712443	-101134499291.927\\
3.39918497962449	-101098975908.629\\
3.39928498212455	-101064025483.126\\
3.39938498462462	-101028502099.828\\
3.39948498712468	-100993551674.325\\
3.39958498962474	-100958028291.027\\
3.3996849921248	-100922504907.729\\
3.39978499462487	-100887554482.226\\
3.39988499712493	-100852031098.928\\
3.39998499962499	-100817080673.425\\
3.40008500212505	-100781557290.126\\
3.40018500462512	-100746033906.828\\
3.40028500712518	-100711083481.325\\
3.40038500962524	-100675560098.027\\
3.4004850121253	-100640609672.524\\
3.40058501462537	-100605086289.226\\
3.40068501712543	-100570135863.723\\
3.40078501962549	-100534612480.425\\
3.40088502212555	-100499662054.922\\
3.40098502462562	-100464138671.624\\
3.40108502712568	-100429188246.121\\
3.40118502962574	-100394237820.618\\
3.4012850321258	-100358714437.32\\
3.40138503462587	-100323764011.817\\
3.40148503712593	-100288240628.519\\
3.40158503962599	-100253290203.016\\
3.40168504212605	-100218339777.513\\
3.40178504462612	-100182816394.215\\
3.40188504712618	-100147865968.712\\
3.40198504962624	-100112342585.414\\
3.4020850521263	-100077392159.911\\
3.40218505462637	-100042441734.408\\
3.40228505712643	-100006918351.11\\
3.40238505962649	-99971967925.6066\\
3.40248506212655	-99937017500.1036\\
3.40258506462662	-99902067074.6006\\
3.40268506712668	-99866543691.3025\\
3.40278506962674	-99831593265.7995\\
3.4028850721268	-99796642840.2965\\
3.40298507462687	-99761692414.7935\\
3.40308507712693	-99726169031.4954\\
3.40318507962699	-99691218605.9925\\
3.40328508212705	-99656268180.4895\\
3.40338508462712	-99621317754.9865\\
3.40348508712718	-99586367329.4835\\
3.40358508962724	-99550843946.1854\\
3.4036850921273	-99515893520.6824\\
3.40378509462737	-99480943095.1794\\
3.40388509712743	-99445992669.6765\\
3.40398509962749	-99411042244.1735\\
3.40408510212755	-99376091818.6705\\
3.40418510462762	-99340568435.3724\\
3.40428510712768	-99305618009.8694\\
3.40438510962774	-99270667584.3664\\
3.4044851121278	-99235717158.8634\\
3.40458511462787	-99200766733.3605\\
3.40468511712793	-99165816307.8575\\
3.40478511962799	-99130865882.3545\\
3.40488512212805	-99095915456.8515\\
3.40498512462812	-99060965031.3486\\
3.40508512712818	-99026014605.8456\\
3.40518512962824	-98991064180.3426\\
3.4052851321283	-98956113754.8396\\
3.40538513462837	-98921163329.3366\\
3.40548513712843	-98886212903.8336\\
3.40558513962849	-98851262478.3307\\
3.40568514212855	-98816312052.8277\\
3.40578514462862	-98781361627.3247\\
3.40588514712868	-98746411201.8217\\
3.40598514962874	-98711460776.3188\\
3.4060851521288	-98676510350.8158\\
3.40618515462887	-98642132883.1079\\
3.40628515712893	-98607182457.6049\\
3.40638515962899	-98572232032.102\\
3.40648516212905	-98537281606.599\\
3.40658516462912	-98502331181.096\\
3.40668516712918	-98467380755.593\\
3.40678516962924	-98432430330.09\\
3.4068851721293	-98398052862.3822\\
3.40698517462937	-98363102436.8792\\
3.40708517712943	-98328152011.3762\\
3.40718517962949	-98293201585.8732\\
3.40728518212955	-98258251160.3703\\
3.40738518462962	-98223873692.6624\\
3.40748518712968	-98188923267.1594\\
3.40758518962974	-98153972841.6565\\
3.4076851921298	-98119022416.1535\\
3.40778519462987	-98084644948.4456\\
3.40788519712993	-98049694522.9426\\
3.40798519962999	-98014744097.4397\\
3.40808520213005	-97980366629.7318\\
3.40818520463012	-97945416204.2288\\
3.40828520713018	-97910465778.7259\\
3.40838520963024	-97876088311.018\\
3.4084852121303	-97841137885.515\\
3.40858521463037	-97806187460.0121\\
3.40868521713043	-97771809992.3042\\
3.40878521963049	-97736859566.8012\\
3.40888522213055	-97701909141.2982\\
3.40898522463062	-97667531673.5904\\
3.40908522713068	-97632581248.0874\\
3.40918522963074	-97598203780.3796\\
3.4092852321308	-97563253354.8766\\
3.40938523463087	-97528875887.1687\\
3.40948523713093	-97493925461.6658\\
3.40958523963099	-97459547993.9579\\
3.40968524213105	-97424597568.4549\\
3.40978524463112	-97390220100.7471\\
3.40988524713118	-97355269675.2441\\
3.40998524963124	-97320892207.5362\\
3.4100852521313	-97285941782.0333\\
3.41018525463137	-97251564314.3254\\
3.41028525713143	-97216613888.8224\\
3.41038525963149	-97182236421.1146\\
3.41048526213155	-97147285995.6116\\
3.41058526463162	-97112908527.9037\\
3.41068526713168	-97078531060.1959\\
3.41078526963174	-97043580634.6929\\
3.4108852721318	-97009203166.9851\\
3.41098527463187	-96974252741.4821\\
3.41108527713193	-96939875273.7742\\
3.41118527963199	-96905497806.0664\\
3.41128528213205	-96870547380.5634\\
3.41138528463212	-96836169912.8556\\
3.41148528713218	-96801792445.1477\\
3.41158528963224	-96766842019.6447\\
3.4116852921323	-96732464551.9369\\
3.41178529463237	-96698087084.229\\
3.41188529713243	-96663709616.5212\\
3.41198529963249	-96628759191.0182\\
3.41208530213255	-96594381723.3104\\
3.41218530463262	-96560004255.6025\\
3.41228530713268	-96525626787.8947\\
3.41238530963274	-96490676362.3917\\
3.4124853121328	-96456298894.6838\\
3.41258531463287	-96421921426.976\\
3.41268531713293	-96387543959.2681\\
3.41278531963299	-96353166491.5603\\
3.41288532213305	-96318216066.0573\\
3.41298532463312	-96283838598.3495\\
3.41308532713318	-96249461130.6416\\
3.41318532963324	-96215083662.9337\\
3.4132853321333	-96180706195.2259\\
3.41338533463337	-96146328727.5181\\
3.41348533713343	-96111951259.8102\\
3.41358533963349	-96077573792.1024\\
3.41368534213355	-96043196324.3945\\
3.41378534463362	-96008245898.8915\\
3.41388534713368	-95973868431.1837\\
3.41398534963374	-95939490963.4758\\
3.4140853521338	-95905113495.768\\
3.41418535463387	-95870736028.0601\\
3.41428535713393	-95836358560.3523\\
3.41438535963399	-95801981092.6444\\
3.41448536213405	-95767603624.9366\\
3.41458536463412	-95733226157.2287\\
3.41468536713418	-95698848689.5209\\
3.41478536963424	-95664471221.813\\
3.4148853721343	-95630093754.1052\\
3.41498537463437	-95596289244.1925\\
3.41508537713443	-95561911776.4846\\
3.41518537963449	-95527534308.7768\\
3.41528538213455	-95493156841.0689\\
3.41538538463462	-95458779373.3611\\
3.41548538713468	-95424401905.6532\\
3.41558538963474	-95390024437.9454\\
3.4156853921348	-95355646970.2375\\
3.41578539463487	-95321269502.5297\\
3.41588539713493	-95287464992.6169\\
3.41598539963499	-95253087524.9091\\
3.41608540213505	-95218710057.2012\\
3.41618540463512	-95184332589.4934\\
3.41628540713518	-95149955121.7856\\
3.41638540963524	-95116150611.8728\\
3.4164854121353	-95081773144.165\\
3.41658541463537	-95047395676.4571\\
3.41668541713543	-95013018208.7493\\
3.41678541963549	-94979213698.8366\\
3.41688542213555	-94944836231.1287\\
3.41698542463562	-94910458763.4209\\
3.41708542713568	-94876654253.5081\\
3.41718542963574	-94842276785.8003\\
3.4172854321358	-94807899318.0925\\
3.41738543463587	-94773521850.3846\\
3.41748543713593	-94739717340.4719\\
3.41758543963599	-94705339872.764\\
3.41768544213605	-94671535362.8513\\
3.41778544463612	-94637157895.1435\\
3.41788544713618	-94602780427.4356\\
3.41798544963624	-94568975917.5229\\
3.4180854521363	-94534598449.815\\
3.41818545463637	-94500793939.9023\\
3.41828545713643	-94466416472.1945\\
3.41838545963649	-94432039004.4866\\
3.41848546213655	-94398234494.5739\\
3.41858546463662	-94363857026.8661\\
3.41868546713668	-94330052516.9533\\
3.41878546963674	-94295675049.2455\\
3.4188854721368	-94261870539.3328\\
3.41898547463687	-94227493071.6249\\
3.41908547713693	-94193688561.7122\\
3.41918547963699	-94159311094.0044\\
3.41928548213705	-94125506584.0916\\
3.41938548463712	-94091129116.3838\\
3.41948548713718	-94057324606.4711\\
3.41958548963724	-94023520096.5583\\
3.4196854921373	-93989142628.8505\\
3.41978549463737	-93955338118.9378\\
3.41988549713743	-93920960651.2299\\
3.41998549963749	-93887156141.3172\\
3.42008550213755	-93853351631.4045\\
3.42018550463762	-93818974163.6967\\
3.42028550713768	-93785169653.7839\\
3.42038550963774	-93751365143.8712\\
3.4204855121378	-93716987676.1634\\
3.42058551463787	-93683183166.2506\\
3.42068551713793	-93649378656.3379\\
3.42078551963799	-93615001188.6301\\
3.42088552213805	-93581196678.7174\\
3.42098552463812	-93547392168.8046\\
3.42108552713818	-93513587658.8919\\
3.42118552963824	-93479210191.1841\\
3.4212855321383	-93445405681.2713\\
3.42138553463837	-93411601171.3586\\
3.42148553713843	-93377796661.4459\\
3.42158553963849	-93343419193.7381\\
3.42168554213855	-93309614683.8253\\
3.42178554463862	-93275810173.9126\\
3.42188554713868	-93242005663.9999\\
3.42198554963874	-93208201154.0872\\
3.4220855521388	-93174396644.1745\\
3.42218555463887	-93140019176.4666\\
3.42228555713893	-93106214666.5539\\
3.42238555963899	-93072410156.6412\\
3.42248556213905	-93038605646.7285\\
3.42258556463912	-93004801136.8158\\
3.42268556713918	-92970996626.903\\
3.42278556963924	-92937192116.9903\\
3.4228855721393	-92903387607.0776\\
3.42298557463937	-92869583097.1649\\
3.42308557713943	-92835778587.2522\\
3.42318557963949	-92801974077.3394\\
3.42328558213955	-92768169567.4267\\
3.42338558463962	-92734365057.514\\
3.42348558713968	-92700560547.6013\\
3.42358558963974	-92666756037.6886\\
3.4236855921398	-92632951527.7758\\
3.42378559463987	-92599147017.8631\\
3.42388559713993	-92565342507.9504\\
3.42398559963999	-92531537998.0377\\
3.42408560214005	-92497733488.125\\
3.42418560464012	-92463928978.2122\\
3.42428560714018	-92430124468.2995\\
3.42438560964024	-92396319958.3868\\
3.4244856121403	-92362515448.4741\\
3.42458561464037	-92328710938.5614\\
3.42468561714043	-92295479386.4438\\
3.42478561964049	-92261674876.5311\\
3.42488562214055	-92227870366.6183\\
3.42498562464062	-92194065856.7056\\
3.42508562714068	-92160261346.7929\\
3.42518562964074	-92126456836.8802\\
3.4252856321408	-92093225284.7626\\
3.42538563464087	-92059420774.8499\\
3.42548563714093	-92025616264.9372\\
3.42558563964099	-91991811755.0245\\
3.42568564214105	-91958580202.9069\\
3.42578564464112	-91924775692.9942\\
3.42588564714118	-91890971183.0814\\
3.42598564964124	-91857166673.1687\\
3.4260856521413	-91823935121.0511\\
3.42618565464137	-91790130611.1384\\
3.42628565714143	-91756326101.2257\\
3.42638565964149	-91723094549.1081\\
3.42648566214155	-91689290039.1954\\
3.42658566464162	-91655485529.2827\\
3.42668566714168	-91622253977.1651\\
3.42678566964174	-91588449467.2523\\
3.4268856721418	-91554644957.3396\\
3.42698567464187	-91521413405.222\\
3.42708567714193	-91487608895.3093\\
3.42718567964199	-91454377343.1917\\
3.42728568214205	-91420572833.279\\
3.42738568464212	-91387341281.1614\\
3.42748568714218	-91353536771.2487\\
3.42758568964224	-91319732261.336\\
3.4276856921423	-91286500709.2184\\
3.42778569464237	-91252696199.3057\\
3.42788569714243	-91219464647.1881\\
3.42798569964249	-91185660137.2754\\
3.42808570214255	-91152428585.1578\\
3.42818570464262	-91118624075.2451\\
3.42828570714268	-91085392523.1275\\
3.42838570964274	-91052160971.0099\\
3.4284857121428	-91018356461.0972\\
3.42858571464287	-90985124908.9796\\
3.42868571714293	-90951320399.0669\\
3.42878571964299	-90918088846.9493\\
3.42888572214305	-90884284337.0366\\
3.42898572464312	-90851052784.919\\
3.42908572714318	-90817821232.8014\\
3.42918572964324	-90784016722.8887\\
3.4292857321433	-90750785170.7711\\
3.42938573464337	-90717553618.6535\\
3.42948573714343	-90683749108.7408\\
3.42958573964349	-90650517556.6232\\
3.42968574214355	-90617286004.5056\\
3.42978574464362	-90583481494.5929\\
3.42988574714368	-90550249942.4753\\
3.42998574964374	-90517018390.3577\\
3.4300857521438	-90483786838.2401\\
3.43018575464387	-90449982328.3274\\
3.43028575714393	-90416750776.2098\\
3.43038575964399	-90383519224.0922\\
3.43048576214405	-90350287671.9746\\
3.43058576464412	-90316483162.0619\\
3.43068576714418	-90283251609.9443\\
3.43078576964424	-90250020057.8268\\
3.4308857721443	-90216788505.7092\\
3.43098577464437	-90183556953.5916\\
3.43108577714443	-90150325401.474\\
3.43118577964449	-90116520891.5613\\
3.43128578214455	-90083289339.4437\\
3.43138578464462	-90050057787.3261\\
3.43148578714468	-90016826235.2085\\
3.43158578964474	-89983594683.0909\\
3.4316857921448	-89950363130.9733\\
3.43178579464487	-89917131578.8557\\
3.43188579714493	-89883900026.7382\\
3.43198579964499	-89850668474.6206\\
3.43208580214505	-89817436922.503\\
3.43218580464512	-89784205370.3854\\
3.43228580714518	-89750973818.2678\\
3.43238580964524	-89717742266.1502\\
3.4324858121453	-89684510714.0326\\
3.43258581464537	-89651279161.915\\
3.43268581714543	-89618047609.7975\\
3.43278581964549	-89584816057.6799\\
3.43288582214555	-89551584505.5623\\
3.43298582464562	-89518352953.4447\\
3.43308582714568	-89485121401.3271\\
3.43318582964574	-89451889849.2095\\
3.4332858321458	-89418658297.0919\\
3.43338583464587	-89385426744.9743\\
3.43348583714593	-89352195192.8568\\
3.43358583964599	-89318963640.7392\\
3.43368584214605	-89286305046.4167\\
3.43378584464612	-89253073494.2991\\
3.43388584714618	-89219841942.1815\\
3.43398584964624	-89186610390.0639\\
3.4340858521463	-89153378837.9464\\
3.43418585464637	-89120147285.8288\\
3.43428585714643	-89087488691.5063\\
3.43438585964649	-89054257139.3887\\
3.43448586214655	-89021025587.2711\\
3.43458586464662	-88987794035.1535\\
3.43468586714668	-88955135440.8311\\
3.43478586964674	-88921903888.7135\\
3.4348858721468	-88888672336.5959\\
3.43498587464687	-88855440784.4783\\
3.43508587714693	-88822782190.1559\\
3.43518587964699	-88789550638.0383\\
3.43528588214705	-88756319085.9207\\
3.43538588464712	-88723660491.5982\\
3.43548588714718	-88690428939.4807\\
3.43558588964724	-88657197387.3631\\
3.4356858921473	-88624538793.0406\\
3.43578589464737	-88591307240.923\\
3.43588589714743	-88558075688.8054\\
3.43598589964749	-88525417094.483\\
3.43608590214755	-88492185542.3654\\
3.43618590464762	-88459526948.0429\\
3.43628590714768	-88426295395.9253\\
3.43638590964774	-88393636801.6029\\
3.4364859121478	-88360405249.4853\\
3.43658591464787	-88327173697.3677\\
3.43668591714793	-88294515103.0453\\
3.43678591964799	-88261283550.9277\\
3.43688592214805	-88228624956.6052\\
3.43698592464812	-88195393404.4876\\
3.43708592714818	-88162734810.1652\\
3.43718592964824	-88129503258.0476\\
3.4372859321483	-88096844663.7251\\
3.43738593464837	-88064186069.4027\\
3.43748593714843	-88030954517.2851\\
3.43758593964849	-87998295922.9626\\
3.43768594214855	-87965064370.845\\
3.43778594464862	-87932405776.5226\\
3.43788594714868	-87899174224.405\\
3.43798594964874	-87866515630.0825\\
3.4380859521488	-87833857035.7601\\
3.43818595464887	-87800625483.6425\\
3.43828595714893	-87767966889.32\\
3.43838595964899	-87735308294.9976\\
3.43848596214905	-87702076742.88\\
3.43858596464912	-87669418148.5575\\
3.43868596714918	-87636759554.2351\\
3.43878596964924	-87603528002.1175\\
3.4388859721493	-87570869407.795\\
3.43898597464937	-87538210813.4726\\
3.43908597714943	-87505552219.1501\\
3.43918597964949	-87472320667.0325\\
3.43928598214955	-87439662072.7101\\
3.43938598464962	-87407003478.3876\\
3.43948598714968	-87374344884.0652\\
3.43958598964974	-87341686289.7427\\
3.4396859921498	-87308454737.6251\\
3.43978599464987	-87275796143.3026\\
3.43988599714993	-87243137548.9802\\
3.43998599964999	-87210478954.6577\\
3.44008600215005	-87177820360.3353\\
3.44018600465012	-87145161766.0128\\
3.44028600715018	-87111930213.8952\\
3.44038600965024	-87079271619.5728\\
3.4404860121503	-87046613025.2503\\
3.44058601465037	-87013954430.9279\\
3.44068601715043	-86981295836.6054\\
3.44078601965049	-86948637242.2829\\
3.44088602215055	-86915978647.9605\\
3.44098602465062	-86883320053.638\\
3.44108602715068	-86850661459.3156\\
3.44118602965074	-86818002864.9931\\
3.4412860321508	-86785344270.6707\\
3.44138603465087	-86752685676.3482\\
3.44148603715093	-86720027082.0258\\
3.44158603965099	-86687368487.7033\\
3.44168604215105	-86654709893.3808\\
3.44178604465112	-86622051299.0584\\
3.44188604715118	-86589392704.7359\\
3.44198604965124	-86556734110.4135\\
3.4420860521513	-86524075516.091\\
3.44218605465137	-86491416921.7686\\
3.44228605715143	-86458758327.4461\\
3.44238605965149	-86426099733.1236\\
3.44248606215155	-86394014096.5963\\
3.44258606465162	-86361355502.2738\\
3.44268606715168	-86328696907.9514\\
3.44278606965174	-86296038313.6289\\
3.4428860721518	-86263379719.3065\\
3.44298607465187	-86230721124.984\\
3.44308607715193	-86198635488.4567\\
3.44318607965199	-86165976894.1342\\
3.44328608215205	-86133318299.8118\\
3.44338608465212	-86100659705.4893\\
3.44348608715218	-86068001111.1669\\
3.44358608965224	-86035915474.6395\\
3.4436860921523	-86003256880.3171\\
3.44378609465237	-85970598285.9946\\
3.44388609715243	-85937939691.6722\\
3.44398609965249	-85905854055.1449\\
3.44408610215255	-85873195460.8224\\
3.44418610465262	-85840536866.4999\\
3.44428610715268	-85808451229.9726\\
3.44438610965274	-85775792635.6501\\
3.4444861121528	-85743134041.3277\\
3.44458611465287	-85711048404.8004\\
3.44468611715293	-85678389810.4779\\
3.44478611965299	-85645731216.1555\\
3.44488612215305	-85613645579.6281\\
3.44498612465312	-85580986985.3057\\
3.44508612715318	-85548901348.7784\\
3.44518612965324	-85516242754.4559\\
3.4452861321533	-85484157117.9286\\
3.44538613465337	-85451498523.6061\\
3.44548613715343	-85418839929.2836\\
3.44558613965349	-85386754292.7563\\
3.44568614215355	-85354095698.4339\\
3.44578614465362	-85322010061.9065\\
3.44588614715368	-85289351467.5841\\
3.44598614965374	-85257265831.0568\\
3.4460861521538	-85224607236.7343\\
3.44618615465387	-85192521600.207\\
3.44628615715393	-85160435963.6796\\
3.44638615965399	-85127777369.3572\\
3.44648616215405	-85095691732.8299\\
3.44658616465412	-85063033138.5074\\
3.44668616715418	-85030947501.9801\\
3.44678616965424	-84998288907.6576\\
3.4468861721543	-84966203271.1303\\
3.44698617465437	-84934117634.603\\
3.44708617715443	-84901459040.2805\\
3.44718617965449	-84869373403.7532\\
3.44728618215455	-84837287767.2259\\
3.44738618465462	-84804629172.9034\\
3.44748618715468	-84772543536.3761\\
3.44758618965474	-84740457899.8488\\
3.4476861921548	-84707799305.5263\\
3.44778619465487	-84675713668.999\\
3.44788619715493	-84643628032.4716\\
3.44798619965499	-84611542395.9443\\
3.44808620215505	-84578883801.6219\\
3.44818620465512	-84546798165.0945\\
3.44828620715518	-84514712528.5672\\
3.44838620965524	-84482626892.0399\\
3.4484862121553	-84449968297.7174\\
3.44858621465537	-84417882661.1901\\
3.44868621715543	-84385797024.6628\\
3.44878621965549	-84353711388.1355\\
3.44888622215555	-84321625751.6081\\
3.44898622465562	-84289540115.0808\\
3.44908622715568	-84256881520.7583\\
3.44918622965574	-84224795884.231\\
3.4492862321558	-84192710247.7037\\
3.44938623465587	-84160624611.1764\\
3.44948623715593	-84128538974.649\\
3.44958623965599	-84096453338.1217\\
3.44968624215605	-84064367701.5944\\
3.44978624465612	-84032282065.0671\\
3.44988624715618	-84000196428.5397\\
3.44998624965624	-83968110792.0124\\
3.4500862521563	-83936025155.4851\\
3.45018625465637	-83903939518.9577\\
3.45028625715643	-83871853882.4304\\
3.45038625965649	-83839768245.9031\\
3.45048626215655	-83807682609.3758\\
3.45058626465662	-83775596972.8484\\
3.45068626715668	-83743511336.3211\\
3.45078626965674	-83711425699.7938\\
3.4508862721568	-83679340063.2665\\
3.45098627465687	-83647254426.7392\\
3.45108627715693	-83615168790.2118\\
3.45118627965699	-83583083153.6845\\
3.45128628215705	-83550997517.1572\\
3.45138628465712	-83518911880.6298\\
3.45148628715718	-83487399201.8976\\
3.45158628965724	-83455313565.3703\\
3.4516862921573	-83423227928.843\\
3.45178629465737	-83391142292.3157\\
3.45188629715743	-83359056655.7883\\
3.45198629965749	-83326971019.261\\
3.45208630215755	-83295458340.5288\\
3.45218630465762	-83263372704.0015\\
3.45228630715768	-83231287067.4742\\
3.45238630965774	-83199201430.9468\\
3.4524863121578	-83167688752.2146\\
3.45258631465787	-83135603115.6873\\
3.45268631715793	-83103517479.16\\
3.45278631965799	-83071431842.6327\\
3.45288632215805	-83039919163.9005\\
3.45298632465812	-83007833527.3732\\
3.45308632715818	-82975747890.8458\\
3.45318632965824	-82944235212.1136\\
3.4532863321583	-82912149575.5863\\
3.45338633465837	-82880063939.059\\
3.45348633715843	-82848551260.3268\\
3.45358633965849	-82816465623.7995\\
3.45368634215855	-82784379987.2721\\
3.45378634465862	-82752867308.5399\\
3.45388634715868	-82720781672.0126\\
3.45398634965874	-82689268993.2804\\
3.4540863521588	-82657183356.7531\\
3.45418635465887	-82625097720.2258\\
3.45428635715893	-82593585041.4936\\
3.45438635965899	-82561499404.9662\\
3.45448636215905	-82529986726.234\\
3.45458636465912	-82497901089.7067\\
3.45468636715918	-82466388410.9745\\
3.45478636965924	-82434302774.4472\\
3.4548863721593	-82402790095.715\\
3.45498637465937	-82370704459.1877\\
3.45508637715943	-82339191780.4555\\
3.45518637965949	-82307106143.9281\\
3.45528638215955	-82275593465.196\\
3.45538638465962	-82244080786.4638\\
3.45548638715968	-82211995149.9364\\
3.45558638965974	-82180482471.2042\\
3.4556863921598	-82148396834.6769\\
3.45578639465987	-82116884155.9447\\
3.45588639715993	-82085371477.2125\\
3.45598639965999	-82053285840.6852\\
3.45608640216005	-82021773161.953\\
3.45618640466012	-81990260483.2208\\
3.45628640716018	-81958174846.6935\\
3.45638640966024	-81926662167.9613\\
3.4564864121603	-81895149489.2291\\
3.45658641466037	-81863063852.7018\\
3.45668641716043	-81831551173.9696\\
3.45678641966049	-81800038495.2374\\
3.45688642216055	-81768525816.5052\\
3.45698642466062	-81736440179.9778\\
3.45708642716068	-81704927501.2457\\
3.45718642966074	-81673414822.5135\\
3.4572864321608	-81641902143.7813\\
3.45738643466087	-81609816507.2539\\
3.45748643716093	-81578303828.5217\\
3.45758643966099	-81546791149.7896\\
3.45768644216105	-81515278471.0574\\
3.45778644466112	-81483765792.3251\\
3.45788644716118	-81452253113.593\\
3.45798644966124	-81420167477.0656\\
3.4580864521613	-81388654798.3334\\
3.45818645466137	-81357142119.6012\\
3.45828645716143	-81325629440.869\\
3.45838645966149	-81294116762.1369\\
3.45848646216155	-81262604083.4047\\
3.45858646466162	-81231091404.6725\\
3.45868646716168	-81199578725.9403\\
3.45878646966174	-81168066047.2081\\
3.4588864721618	-81136553368.4759\\
3.45898647466187	-81105040689.7437\\
3.45908647716193	-81073528011.0115\\
3.45918647966199	-81042015332.2793\\
3.45928648216205	-81010502653.5471\\
3.45938648466212	-80978989974.8149\\
3.45948648716218	-80947477296.0827\\
3.45958648966224	-80915964617.3505\\
3.4596864921623	-80884451938.6183\\
3.45978649466237	-80852939259.8861\\
3.45988649716243	-80821426581.1539\\
3.45998649966249	-80789913902.4217\\
3.46008650216255	-80758401223.6895\\
3.46018650466262	-80726888544.9573\\
3.46028650716268	-80695375866.2251\\
3.46038650966274	-80664436145.2881\\
3.4604865121628	-80632923466.5559\\
3.46058651466287	-80601410787.8237\\
3.46068651716293	-80569898109.0915\\
3.46078651966299	-80538385430.3593\\
3.46088652216305	-80506872751.6271\\
3.46098652466312	-80475933030.69\\
3.46108652716318	-80444420351.9578\\
3.46118652966324	-80412907673.2256\\
3.4612865321633	-80381394994.4935\\
3.46138653466337	-80350455273.5564\\
3.46148653716343	-80318942594.8242\\
3.46158653966349	-80287429916.092\\
3.46168654216355	-80255917237.3598\\
3.46178654466362	-80224977516.4227\\
3.46188654716368	-80193464837.6905\\
3.46198654966374	-80161952158.9583\\
3.4620865521638	-80131012438.0213\\
3.46218655466387	-80099499759.2891\\
3.46228655716393	-80067987080.5569\\
3.46238655966399	-80037047359.6198\\
3.46248656216405	-80005534680.8876\\
3.46258656466412	-79974022002.1554\\
3.46268656716418	-79943082281.2184\\
3.46278656966424	-79911569602.4862\\
3.4628865721643	-79880629881.5491\\
3.46298657466437	-79849117202.8169\\
3.46308657716443	-79818177481.8799\\
3.46318657966449	-79786664803.1477\\
3.46328658216455	-79755152124.4155\\
3.46338658466462	-79724212403.4784\\
3.46348658716468	-79692699724.7462\\
3.46358658966474	-79661760003.8091\\
3.4636865921648	-79630247325.077\\
3.46378659466487	-79599307604.1399\\
3.46388659716493	-79568367883.2028\\
3.46398659966499	-79536855204.4706\\
3.46408660216505	-79505915483.5336\\
3.46418660466512	-79474402804.8014\\
3.46428660716518	-79443463083.8643\\
3.46438660966524	-79411950405.1321\\
3.4644866121653	-79381010684.195\\
3.46458661466537	-79350070963.258\\
3.46468661716543	-79318558284.5258\\
3.46478661966549	-79287618563.5887\\
3.46488662216555	-79256105884.8565\\
3.46498662466562	-79225166163.9194\\
3.46508662716568	-79194226442.9824\\
3.46518662966574	-79162713764.2502\\
3.4652866321658	-79131774043.3131\\
3.46538663466587	-79100834322.3761\\
3.46548663716593	-79069894601.439\\
3.46558663966599	-79038381922.7068\\
3.46568664216605	-79007442201.7697\\
3.46578664466612	-78976502480.8327\\
3.46588664716618	-78945562759.8956\\
3.46598664966624	-78914050081.1634\\
3.4660866521663	-78883110360.2263\\
3.46618665466637	-78852170639.2893\\
3.46628665716643	-78821230918.3522\\
3.46638665966649	-78789718239.62\\
3.46648666216655	-78758778518.683\\
3.46658666466662	-78727838797.7459\\
3.46668666716668	-78696899076.8088\\
3.46678666966674	-78665959355.8718\\
3.4668866721668	-78635019634.9347\\
3.46698667466687	-78604079913.9976\\
3.46708667716693	-78573140193.0606\\
3.46718667966699	-78541627514.3284\\
3.46728668216705	-78510687793.3913\\
3.46738668466712	-78479748072.4543\\
3.46748668716718	-78448808351.5172\\
3.46758668966724	-78417868630.5801\\
3.4676866921673	-78386928909.6431\\
3.46778669466737	-78355989188.706\\
3.46788669716743	-78325049467.7689\\
3.46798669966749	-78294109746.8319\\
3.46808670216755	-78263170025.8948\\
3.46818670466762	-78232230304.9577\\
3.46828670716768	-78201290584.0207\\
3.46838670966774	-78170350863.0836\\
3.4684867121678	-78139411142.1465\\
3.46858671466787	-78108471421.2095\\
3.46868671716793	-78077531700.2724\\
3.46878671966799	-78046591979.3353\\
3.46888672216805	-78016225216.1934\\
3.46898672466812	-77985285495.2563\\
3.46908672716818	-77954345774.3193\\
3.46918672966824	-77923406053.3822\\
3.4692867321683	-77892466332.4452\\
3.46938673466837	-77861526611.5081\\
3.46948673716843	-77830586890.571\\
3.46958673966849	-77800220127.4291\\
3.46968674216855	-77769280406.492\\
3.46978674466862	-77738340685.555\\
3.46988674716868	-77707400964.6179\\
3.46998674966874	-77676461243.6808\\
3.4700867521688	-77646094480.5389\\
3.47018675466887	-77615154759.6018\\
3.47028675716893	-77584215038.6648\\
3.47038675966899	-77553275317.7277\\
3.47048676216905	-77522908554.5858\\
3.47058676466912	-77491968833.6487\\
3.47068676716918	-77461029112.7116\\
3.47078676966924	-77430662349.5697\\
3.4708867721693	-77399722628.6326\\
3.47098677466937	-77368782907.6956\\
3.47108677716943	-77338416144.5536\\
3.47118677966949	-77307476423.6166\\
3.47128678216955	-77276536702.6795\\
3.47138678466962	-77246169939.5376\\
3.47148678716968	-77215230218.6005\\
3.47158678966974	-77184290497.6635\\
3.4716867921698	-77153923734.5215\\
3.47178679466987	-77122984013.5845\\
3.47188679716993	-77092617250.4425\\
3.47198679966999	-77061677529.5055\\
3.47208680217005	-77031310766.3635\\
3.47218680467012	-77000371045.4265\\
3.47228680717018	-76970004282.2845\\
3.47238680967024	-76939064561.3475\\
3.4724868121703	-76908697798.2055\\
3.47258681467037	-76877758077.2685\\
3.47268681717043	-76847391314.1265\\
3.47278681967049	-76816451593.1895\\
3.47288682217055	-76786084830.0475\\
3.47298682467062	-76755145109.1105\\
3.47308682717068	-76724778345.9685\\
3.47318682967074	-76693838625.0315\\
3.4732868321708	-76663471861.8895\\
3.47338683467087	-76633105098.7476\\
3.47348683717093	-76602165377.8105\\
3.47358683967099	-76571798614.6686\\
3.47368684217105	-76541431851.5267\\
3.47378684467112	-76510492130.5896\\
3.47388684717118	-76480125367.4477\\
3.47398684967124	-76449758604.3057\\
3.4740868521713	-76418818883.3687\\
3.47418685467137	-76388452120.2267\\
3.47428685717143	-76358085357.0848\\
3.47438685967149	-76327145636.1478\\
3.47448686217155	-76296778873.0058\\
3.47458686467162	-76266412109.8639\\
3.47468686717168	-76236045346.7219\\
3.47478686967174	-76205105625.7849\\
3.4748868721718	-76174738862.6429\\
3.47498687467187	-76144372099.501\\
3.47508687717193	-76114005336.3591\\
3.47518687967199	-76083638573.2171\\
3.47528688217205	-76052698852.2801\\
3.47538688467212	-76022332089.1382\\
3.47548688717218	-75991965325.9962\\
3.47558688967224	-75961598562.8543\\
3.4756868921723	-75931231799.7123\\
3.47578689467237	-75900865036.5704\\
3.47588689717243	-75870498273.4285\\
3.47598689967249	-75839558552.4914\\
3.47608690217255	-75809191789.3495\\
3.47618690467262	-75778825026.2076\\
3.47628690717268	-75748458263.0656\\
3.47638690967274	-75718091499.9237\\
3.4764869121728	-75687724736.7818\\
3.47658691467287	-75657357973.6398\\
3.47668691717293	-75626991210.4979\\
3.47678691967299	-75596624447.3559\\
3.47688692217305	-75566257684.214\\
3.47698692467312	-75535890921.0721\\
3.47708692717318	-75505524157.9301\\
3.47718692967324	-75475157394.7882\\
3.4772869321733	-75444790631.6463\\
3.47738693467337	-75414423868.5043\\
3.47748693717343	-75384057105.3624\\
3.47758693967349	-75353690342.2205\\
3.47768694217355	-75323896536.8737\\
3.47778694467362	-75293529773.7317\\
3.47788694717368	-75263163010.5898\\
3.47798694967374	-75232796247.4479\\
3.4780869521738	-75202429484.3059\\
3.47818695467387	-75172062721.164\\
3.47828695717393	-75141695958.0221\\
3.47838695967399	-75111902152.6753\\
3.47848696217405	-75081535389.5333\\
3.47858696467412	-75051168626.3914\\
3.47868696717418	-75020801863.2495\\
3.47878696967424	-74990435100.1075\\
3.4788869721743	-74960641294.7607\\
3.47898697467437	-74930274531.6188\\
3.47908697717443	-74899907768.4769\\
3.47918697967449	-74869541005.3349\\
3.47928698217455	-74839747199.9881\\
3.47938698467462	-74809380436.8462\\
3.47948698717468	-74779013673.7043\\
3.47958698967474	-74749219868.3575\\
3.4796869921748	-74718853105.2155\\
3.47978699467487	-74688486342.0736\\
3.47988699717493	-74658692536.7268\\
3.47998699967499	-74628325773.5849\\
3.48008700217505	-74597959010.4429\\
3.48018700467512	-74568165205.0961\\
3.48028700717518	-74537798441.9542\\
3.48038700967524	-74507431678.8123\\
3.4804870121753	-74477637873.4655\\
3.48058701467537	-74447271110.3235\\
3.48068701717543	-74417477304.9767\\
3.48078701967549	-74387110541.8348\\
3.48088702217555	-74357316736.488\\
3.48098702467562	-74326949973.346\\
3.48108702717568	-74297156167.9992\\
3.48118702967574	-74266789404.8573\\
3.4812870321758	-74236995599.5105\\
3.48138703467587	-74206628836.3686\\
3.48148703717593	-74176835031.0218\\
3.48158703967599	-74146468267.8798\\
3.48168704217605	-74116674462.533\\
3.48178704467612	-74086307699.3911\\
3.48188704717618	-74056513894.0443\\
3.48198704967624	-74026147130.9024\\
3.4820870521763	-73996353325.5556\\
3.48218705467637	-73966559520.2088\\
3.48228705717643	-73936192757.0668\\
3.48238705967649	-73906398951.72\\
3.48248706217655	-73876605146.3732\\
3.48258706467662	-73846238383.2313\\
3.48268706717668	-73816444577.8845\\
3.48278706967674	-73786650772.5377\\
3.4828870721768	-73756284009.3957\\
3.48298707467687	-73726490204.0489\\
3.48308707717693	-73696696398.7021\\
3.48318707967699	-73666329635.5602\\
3.48328708217705	-73636535830.2134\\
3.48338708467712	-73606742024.8666\\
3.48348708717718	-73576948219.5198\\
3.48358708967724	-73546581456.3779\\
3.4836870921773	-73516787651.0311\\
3.48378709467737	-73486993845.6842\\
3.48388709717743	-73457200040.3374\\
3.48398709967749	-73426833277.1955\\
3.48408710217755	-73397039471.8487\\
3.48418710467762	-73367245666.5019\\
3.48428710717768	-73337451861.1551\\
3.48438710967774	-73307658055.8083\\
3.4844871121778	-73277864250.4615\\
3.48458711467787	-73248070445.1147\\
3.48468711717793	-73217703681.9728\\
3.48478711967799	-73187909876.626\\
3.48488712217805	-73158116071.2792\\
3.48498712467812	-73128322265.9324\\
3.48508712717818	-73098528460.5856\\
3.48518712967824	-73068734655.2388\\
3.4852871321783	-73038940849.892\\
3.48538713467837	-73009147044.5452\\
3.48548713717843	-72979353239.1983\\
3.48558713967849	-72949559433.8515\\
3.48568714217855	-72919765628.5047\\
3.48578714467862	-72889971823.1579\\
3.48588714717868	-72860178017.8111\\
3.48598714967874	-72830384212.4643\\
3.4860871521788	-72800590407.1175\\
3.48618715467887	-72770796601.7707\\
3.48628715717893	-72741002796.4239\\
3.48638715967899	-72711208991.0771\\
3.48648716217905	-72681415185.7303\\
3.48658716467912	-72652194338.1786\\
3.48668716717918	-72622400532.8318\\
3.48678716967924	-72592606727.485\\
3.4868871721793	-72562812922.1382\\
3.48698717467937	-72533019116.7914\\
3.48708717717943	-72503225311.4446\\
3.48718717967949	-72473431506.0978\\
3.48728718217955	-72444210658.5462\\
3.48738718467962	-72414416853.1994\\
3.48748718717968	-72384623047.8526\\
3.48758718967974	-72354829242.5058\\
3.4876871921798	-72325035437.159\\
3.48778719467987	-72295814589.6073\\
3.48788719717993	-72266020784.2605\\
3.48798719967999	-72236226978.9137\\
3.48808720218005	-72206433173.5669\\
3.48818720468012	-72177212326.0152\\
3.48828720718018	-72147418520.6684\\
3.48838720968024	-72117624715.3216\\
3.4884872121803	-72088403867.7699\\
3.48858721468037	-72058610062.4231\\
3.48868721718043	-72028816257.0763\\
3.48878721968049	-71999595409.5246\\
3.48888722218055	-71969801604.1778\\
3.48898722468062	-71940007798.831\\
3.48908722718068	-71910786951.2794\\
3.48918722968074	-71880993145.9326\\
3.4892872321808	-71851772298.3809\\
3.48938723468087	-71821978493.0341\\
3.48948723718093	-71792757645.4824\\
3.48958723968099	-71762963840.1356\\
3.48968724218105	-71733170034.7888\\
3.48978724468112	-71703949187.2371\\
3.48988724718118	-71674155381.8903\\
3.48998724968124	-71644934534.3387\\
3.4900872521813	-71615140728.9919\\
3.49018725468137	-71585919881.4402\\
3.49028725718143	-71556126076.0934\\
3.49038725968149	-71526905228.5417\\
3.49048726218155	-71497684380.99\\
3.49058726468162	-71467890575.6432\\
3.49068726718168	-71438669728.0916\\
3.49078726968174	-71408875922.7448\\
3.4908872721818	-71379655075.1931\\
3.49098727468187	-71349861269.8463\\
3.49108727718193	-71320640422.2946\\
3.49118727968199	-71291419574.7429\\
3.49128728218205	-71261625769.3961\\
3.49138728468212	-71232404921.8445\\
3.49148728718218	-71203184074.2928\\
3.49158728968224	-71173390268.946\\
3.4916872921823	-71144169421.3943\\
3.49178729468237	-71114948573.8427\\
3.49188729718243	-71085154768.4958\\
3.49198729968249	-71055933920.9442\\
3.49208730218255	-71026713073.3925\\
3.49218730468262	-70997492225.8408\\
3.49228730718268	-70967698420.494\\
3.49238730968274	-70938477572.9424\\
3.4924873121828	-70909256725.3907\\
3.49258731468287	-70880035877.839\\
3.49268731718293	-70850242072.4922\\
3.49278731968299	-70821021224.9405\\
3.49288732218305	-70791800377.3889\\
3.49298732468312	-70762579529.8372\\
3.49308732718318	-70733358682.2855\\
3.49318732968324	-70704137834.7338\\
3.4932873321833	-70674916987.1822\\
3.49338733468337	-70645123181.8354\\
3.49348733718343	-70615902334.2837\\
3.49358733968349	-70586681486.732\\
3.49368734218355	-70557460639.1804\\
3.49378734468362	-70528239791.6287\\
3.49388734718368	-70499018944.077\\
3.49398734968374	-70469798096.5253\\
3.4940873521838	-70440577248.9737\\
3.49418735468387	-70411356401.422\\
3.49428735718393	-70382135553.8703\\
3.49438735968399	-70352914706.3186\\
3.49448736218405	-70323693858.767\\
3.49458736468412	-70294473011.2153\\
3.49468736718418	-70265252163.6636\\
3.49478736968424	-70236031316.112\\
3.4948873721843	-70206810468.5603\\
3.49498737468437	-70177589621.0086\\
3.49508737718443	-70148368773.457\\
3.49518737968449	-70119147925.9053\\
3.49528738218455	-70089927078.3536\\
3.49538738468462	-70060706230.8019\\
3.49548738718468	-70031485383.2503\\
3.49558738968474	-70002837493.4937\\
3.4956873921848	-69973616645.942\\
3.49578739468487	-69944395798.3904\\
3.49588739718493	-69915174950.8387\\
3.49598739968499	-69885954103.287\\
3.49608740218505	-69856733255.7354\\
3.49618740468512	-69828085365.9788\\
3.49628740718518	-69798864518.4272\\
3.49638740968524	-69769643670.8755\\
3.4964874121853	-69740422823.3238\\
3.49658741468537	-69711774933.5673\\
3.49668741718543	-69682554086.0156\\
3.49678741968549	-69653333238.4639\\
3.49688742218555	-69624112390.9122\\
3.49698742468562	-69595464501.1557\\
3.49708742718568	-69566243653.604\\
3.49718742968574	-69537022806.0524\\
3.4972874321858	-69508374916.2958\\
3.49738743468587	-69479154068.7442\\
3.49748743718593	-69449933221.1925\\
3.49758743968599	-69421285331.4359\\
3.49768744218605	-69392064483.8843\\
3.49778744468612	-69362843636.3326\\
3.49788744718618	-69334195746.576\\
3.49798744968624	-69304974899.0244\\
3.4980874521863	-69276327009.2678\\
3.49818745468637	-69247106161.7162\\
3.49828745718643	-69217885314.1645\\
3.49838745968649	-69189237424.408\\
3.49848746218655	-69160016576.8563\\
3.49858746468662	-69131368687.0997\\
3.49868746718668	-69102147839.5481\\
3.49878746968674	-69073499949.7915\\
3.4988874721868	-69044279102.2399\\
3.49898747468687	-69015631212.4833\\
3.49908747718693	-68986410364.9316\\
3.49918747968699	-68957762475.1751\\
3.49928748218705	-68928541627.6234\\
3.49938748468712	-68899893737.8669\\
3.49948748718718	-68871245848.1104\\
3.49958748968724	-68842025000.5587\\
3.4996874921873	-68813377110.8021\\
3.49978749468737	-68784156263.2505\\
3.49988749718743	-68755508373.4939\\
3.49998749968749	-68726860483.7374\\
3.50008750218755	-68697639636.1857\\
3.50018750468762	-68668991746.4292\\
3.50028750718768	-68639770898.8775\\
3.50038750968774	-68611123009.1209\\
3.5004875121878	-68582475119.3644\\
3.50058751468787	-68553827229.6079\\
3.50068751718793	-68524606382.0562\\
3.50078751968799	-68495958492.2997\\
3.50088752218805	-68467310602.5431\\
3.50098752468812	-68438089754.9914\\
3.50108752718818	-68409441865.2349\\
3.50118752968824	-68380793975.4784\\
3.5012875321883	-68352146085.7218\\
3.50138753468837	-68323498195.9653\\
3.50148753718843	-68294277348.4136\\
3.50158753968849	-68265629458.6571\\
3.50168754218855	-68236981568.9005\\
3.50178754468862	-68208333679.144\\
3.50188754718868	-68179685789.3874\\
3.50198754968874	-68151037899.6309\\
3.5020875521888	-68121817052.0792\\
3.50218755468887	-68093169162.3227\\
3.50228755718893	-68064521272.5661\\
3.50238755968899	-68035873382.8096\\
3.50248756218905	-68007225493.0531\\
3.50258756468912	-67978577603.2965\\
3.50268756718918	-67949929713.54\\
3.50278756968924	-67921281823.7834\\
3.5028875721893	-67892633934.0269\\
3.50298757468937	-67863986044.2704\\
3.50308757718943	-67835338154.5138\\
3.50318757968949	-67806690264.7573\\
3.50328758218955	-67778042375.0007\\
3.50338758468962	-67749394485.2442\\
3.50348758718968	-67720746595.4876\\
3.50358758968974	-67692098705.7311\\
3.5036875921898	-67663450815.9746\\
3.50378759468987	-67634802926.218\\
3.50388759718993	-67606155036.4615\\
3.50398759968999	-67577507146.7049\\
3.50408760219005	-67548859256.9484\\
3.50418760469012	-67520211367.1919\\
3.50428760719018	-67491563477.4353\\
3.50438760969024	-67462915587.6788\\
3.5044876121903	-67434840655.7174\\
3.50458761469037	-67406192765.9608\\
3.50468761719043	-67377544876.2043\\
3.50478761969049	-67348896986.4477\\
3.50488762219055	-67320249096.6912\\
3.50498762469062	-67291601206.9347\\
3.50508762719068	-67263526274.9733\\
3.50518762969074	-67234878385.2167\\
3.5052876321908	-67206230495.4602\\
3.50538763469087	-67177582605.7036\\
3.50548763719093	-67149507673.7422\\
3.50558763969099	-67120859783.9857\\
3.50568764219105	-67092211894.2291\\
3.50578764469112	-67063564004.4726\\
3.50588764719118	-67035489072.5112\\
3.50598764969124	-67006841182.7546\\
3.5060876521913	-66978193292.9981\\
3.50618765469137	-66950118361.0367\\
3.50628765719143	-66921470471.2802\\
3.50638765969149	-66892822581.5236\\
3.50648766219155	-66864747649.5622\\
3.50658766469162	-66836099759.8057\\
3.50668766719168	-66808024827.8443\\
3.50678766969174	-66779376938.0877\\
3.5068876721918	-66750729048.3312\\
3.50698767469187	-66722654116.3698\\
3.50708767719193	-66694006226.6132\\
3.50718767969199	-66665931294.6518\\
3.50728768219205	-66637283404.8953\\
3.50738768469212	-66609208472.9339\\
3.50748768719218	-66580560583.1773\\
3.50758768969224	-66552485651.2159\\
3.50768769219231	-66523837761.4594\\
3.50778769469237	-66495762829.498\\
3.50788769719243	-66467114939.7414\\
3.50798769969249	-66439040007.78\\
3.50808770219255	-66410392118.0235\\
3.50818770469262	-66382317186.062\\
3.50828770719268	-66353669296.3055\\
3.50838770969274	-66325594364.3441\\
3.5084877121928	-66297519432.3827\\
3.50858771469287	-66268871542.6261\\
3.50868771719293	-66240796610.6647\\
3.50878771969299	-66212148720.9082\\
3.50888772219305	-66184073788.9468\\
3.50898772469312	-66155998856.9854\\
3.50908772719318	-66127350967.2288\\
3.50918772969324	-66099276035.2674\\
3.5092877321933	-66071201103.306\\
3.50938773469337	-66042553213.5495\\
3.50948773719343	-66014478281.5881\\
3.50958773969349	-65986403349.6266\\
3.50968774219356	-65958328417.6652\\
3.50978774469362	-65929680527.9087\\
3.50988774719368	-65901605595.9473\\
3.50998774969374	-65873530663.9859\\
3.5100877521938	-65845455732.0245\\
3.51018775469387	-65816807842.2679\\
3.51028775719393	-65788732910.3065\\
3.51038775969399	-65760657978.3451\\
3.51048776219405	-65732583046.3837\\
3.51058776469412	-65704508114.4223\\
3.51068776719418	-65676433182.4609\\
3.51078776969424	-65647785292.7043\\
3.5108877721943	-65619710360.7429\\
3.51098777469437	-65591635428.7815\\
3.51108777719443	-65563560496.8201\\
3.51118777969449	-65535485564.8587\\
3.51128778219455	-65507410632.8973\\
3.51138778469462	-65479335700.9359\\
3.51148778719468	-65451260768.9745\\
3.51158778969474	-65423185837.0131\\
3.51168779219481	-65395110905.0516\\
3.51178779469487	-65367035973.0902\\
3.51188779719493	-65338961041.1288\\
3.51198779969499	-65310886109.1674\\
3.51208780219506	-65282811177.206\\
3.51218780469512	-65254736245.2446\\
3.51228780719518	-65226661313.2832\\
3.51238780969524	-65198586381.3218\\
3.5124878121953	-65170511449.3604\\
3.51258781469537	-65142436517.3989\\
3.51268781719543	-65114361585.4375\\
3.51278781969549	-65086286653.4761\\
3.51288782219555	-65058211721.5147\\
3.51298782469562	-65030136789.5533\\
3.51308782719568	-65002061857.5919\\
3.51318782969574	-64973986925.6305\\
3.5132878321958	-64945911993.6691\\
3.51338783469587	-64918410019.5028\\
3.51348783719593	-64890335087.5414\\
3.51358783969599	-64862260155.58\\
3.51368784219606	-64834185223.6186\\
3.51378784469612	-64806110291.6572\\
3.51388784719618	-64778608317.4909\\
3.51398784969624	-64750533385.5295\\
3.51408785219631	-64722458453.5681\\
3.51418785469637	-64694383521.6066\\
3.51428785719643	-64666881547.4404\\
3.51438785969649	-64638806615.479\\
3.51448786219655	-64610731683.5175\\
3.51458786469662	-64582656751.5561\\
3.51468786719668	-64555154777.3899\\
3.51478786969674	-64527079845.4284\\
3.5148878721968	-64499004913.467\\
3.51498787469687	-64471502939.3008\\
3.51508787719693	-64443428007.3393\\
3.51518787969699	-64415353075.3779\\
3.51528788219705	-64387851101.2117\\
3.51538788469712	-64359776169.2502\\
3.51548788719718	-64331701237.2888\\
3.51558788969724	-64304199263.1226\\
3.51568789219731	-64276124331.1611\\
3.51578789469737	-64248622356.9949\\
3.51588789719743	-64220547425.0335\\
3.51598789969749	-64193045450.8672\\
3.51608790219756	-64164970518.9058\\
3.51618790469762	-64137468544.7395\\
3.51628790719768	-64109393612.7781\\
3.51638790969774	-64081318680.8167\\
3.5164879121978	-64053816706.6504\\
3.51658791469787	-64026314732.4841\\
3.51668791719793	-63998239800.5227\\
3.51678791969799	-63970737826.3564\\
3.51688792219805	-63942662894.395\\
3.51698792469812	-63915160920.2287\\
3.51708792719818	-63887085988.2673\\
3.51718792969824	-63859584014.101\\
3.5172879321983	-63831509082.1396\\
3.51738793469837	-63804007107.9733\\
3.51748793719843	-63776505133.8071\\
3.51758793969849	-63748430201.8457\\
3.51768794219856	-63720928227.6794\\
3.51778794469862	-63693426253.5131\\
3.51788794719868	-63665351321.5517\\
3.51798794969874	-63637849347.3854\\
3.51808795219881	-63610347373.2191\\
3.51818795469887	-63582272441.2577\\
3.51828795719893	-63554770467.0914\\
3.51838795969899	-63527268492.9252\\
3.51848796219906	-63499193560.9637\\
3.51858796469912	-63471691586.7975\\
3.51868796719918	-63444189612.6312\\
3.51878796969924	-63416687638.4649\\
3.5188879721993	-63388612706.5035\\
3.51898797469937	-63361110732.3372\\
3.51908797719943	-63333608758.1709\\
3.51918797969949	-63306106784.0047\\
3.51928798219955	-63278604809.8384\\
3.51938798469962	-63251102835.6721\\
3.51948798719968	-63223027903.7107\\
3.51958798969974	-63195525929.5444\\
3.51968799219981	-63168023955.3781\\
3.51978799469987	-63140521981.2119\\
3.51988799719993	-63113020007.0456\\
3.51998799969999	-63085518032.8793\\
3.52008800220006	-63058016058.713\\
3.52018800470012	-63030514084.5467\\
3.52028800720018	-63003012110.3805\\
3.52038800970024	-62975510136.2142\\
3.52048801220031	-62947435204.2528\\
3.52058801470037	-62919933230.0865\\
3.52068801720043	-62892431255.9202\\
3.52078801970049	-62864929281.7539\\
3.52088802220055	-62837427307.5876\\
3.52098802470062	-62809925333.4214\\
3.52108802720068	-62782423359.2551\\
3.52118802970074	-62754921385.0888\\
3.5212880322008	-62727992368.7177\\
3.52138803470087	-62700490394.5514\\
3.52148803720093	-62672988420.3851\\
3.52158803970099	-62645486446.2188\\
3.52168804220106	-62617984472.0525\\
3.52178804470112	-62590482497.8863\\
3.52188804720118	-62562980523.72\\
3.52198804970124	-62535478549.5537\\
3.52208805220131	-62507976575.3874\\
3.52218805470137	-62480474601.2211\\
3.52228805720143	-62453545584.85\\
3.52238805970149	-62426043610.6837\\
3.52248806220156	-62398541636.5174\\
3.52258806470162	-62371039662.3512\\
3.52268806720168	-62343537688.1849\\
3.52278806970174	-62316608671.8137\\
3.5228880722018	-62289106697.6474\\
3.52298807470187	-62261604723.4812\\
3.52308807720193	-62234102749.3149\\
3.52318807970199	-62207173732.9437\\
3.52328808220205	-62179671758.7775\\
3.52338808470212	-62152169784.6112\\
3.52348808720218	-62125240768.24\\
3.52358808970224	-62097738794.0738\\
3.52368809220231	-62070236819.9075\\
3.52378809470237	-62042734845.7412\\
3.52388809720243	-62015805829.37\\
3.52398809970249	-61988303855.2038\\
3.52408810220256	-61961374838.8326\\
3.52418810470262	-61933872864.6663\\
3.52428810720268	-61906370890.5001\\
3.52438810970274	-61879441874.1289\\
3.52448811220281	-61851939899.9626\\
3.52458811470287	-61825010883.5915\\
3.52468811720293	-61797508909.4252\\
3.52478811970299	-61770006935.2589\\
3.52488812220306	-61743077918.8878\\
3.52498812470312	-61715575944.7215\\
3.52508812720318	-61688646928.3503\\
3.52518812970324	-61661144954.1841\\
3.5252881322033	-61634215937.8129\\
3.52538813470337	-61606713963.6466\\
3.52548813720343	-61579784947.2755\\
3.52558813970349	-61552282973.1092\\
3.52568814220356	-61525353956.7381\\
3.52578814470362	-61498424940.3669\\
3.52588814720368	-61470922966.2006\\
3.52598814970374	-61443993949.8295\\
3.52608815220381	-61416491975.6632\\
3.52618815470387	-61389562959.2921\\
3.52628815720393	-61362060985.1258\\
3.52638815970399	-61335131968.7546\\
3.52648816220406	-61308202952.3835\\
3.52658816470412	-61280700978.2172\\
3.52668816720418	-61253771961.846\\
3.52678816970424	-61226842945.4749\\
3.52688817220431	-61199340971.3086\\
3.52698817470437	-61172411954.9375\\
3.52708817720443	-61145482938.5663\\
3.52718817970449	-61118553922.1952\\
3.52728818220455	-61091051948.0289\\
3.52738818470462	-61064122931.6577\\
3.52748818720468	-61037193915.2866\\
3.52758818970474	-61010264898.9155\\
3.52768819220481	-60982762924.7492\\
3.52778819470487	-60955833908.378\\
3.52788819720493	-60928904892.0069\\
3.52798819970499	-60901975875.6357\\
3.52808820220506	-60875046859.2646\\
3.52818820470512	-60847544885.0983\\
3.52828820720518	-60820615868.7271\\
3.52838820970524	-60793686852.356\\
3.52848821220531	-60766757835.9848\\
3.52858821470537	-60739828819.6137\\
3.52868821720543	-60712899803.2426\\
3.52878821970549	-60685970786.8714\\
3.52888822220556	-60658468812.7051\\
3.52898822470562	-60631539796.334\\
3.52908822720568	-60604610779.9628\\
3.52918822970574	-60577681763.5917\\
3.52928823220581	-60550752747.2205\\
3.52938823470587	-60523823730.8494\\
3.52948823720593	-60496894714.4782\\
3.52958823970599	-60469965698.1071\\
3.52968824220606	-60443036681.7359\\
3.52978824470612	-60416107665.3648\\
3.52988824720618	-60389178648.9936\\
3.52998824970624	-60362249632.6225\\
3.53008825220631	-60335320616.2513\\
3.53018825470637	-60308391599.8802\\
3.53028825720643	-60281462583.509\\
3.53038825970649	-60255106524.933\\
3.53048826220656	-60228177508.5619\\
3.53058826470662	-60201248492.1907\\
3.53068826720668	-60174319475.8196\\
3.53078826970674	-60147390459.4484\\
3.53088827220681	-60120461443.0773\\
3.53098827470687	-60093532426.7061\\
3.53108827720693	-60066603410.335\\
3.53118827970699	-60040247351.759\\
3.53128828220706	-60013318335.3878\\
3.53138828470712	-59986389319.0167\\
3.53148828720718	-59959460302.6455\\
3.53158828970724	-59932531286.2744\\
3.53168829220731	-59906175227.6984\\
3.53178829470737	-59879246211.3272\\
3.53188829720743	-59852317194.9561\\
3.53198829970749	-59825388178.5849\\
3.53208830220756	-59799032120.0089\\
3.53218830470762	-59772103103.6377\\
3.53228830720768	-59745174087.2666\\
3.53238830970774	-59718818028.6906\\
3.53248831220781	-59691889012.3194\\
3.53258831470787	-59664959995.9483\\
3.53268831720793	-59638603937.3723\\
3.53278831970799	-59611674921.0011\\
3.53288832220806	-59584745904.63\\
3.53298832470812	-59558389846.0539\\
3.53308832720818	-59531460829.6828\\
3.53318832970824	-59504531813.3116\\
3.53328833220831	-59478175754.7356\\
3.53338833470837	-59451246738.3645\\
3.53348833720843	-59424890679.7885\\
3.53358833970849	-59397961663.4173\\
3.53368834220856	-59371605604.8413\\
3.53378834470862	-59344676588.4701\\
3.53388834720868	-59318320529.8941\\
3.53398834970874	-59291391513.523\\
3.53408835220881	-59265035454.947\\
3.53418835470887	-59238106438.5758\\
3.53428835720893	-59211750379.9998\\
3.53438835970899	-59184821363.6286\\
3.53448836220906	-59158465305.0526\\
3.53458836470912	-59131536288.6815\\
3.53468836720918	-59105180230.1055\\
3.53478836970924	-59078251213.7343\\
3.53488837220931	-59051895155.1583\\
3.53498837470937	-59025539096.5823\\
3.53508837720943	-58998610080.2111\\
3.53518837970949	-58972254021.6351\\
3.53528838220956	-58945325005.264\\
3.53538838470962	-58918968946.6879\\
3.53548838720968	-58892612888.1119\\
3.53558838970974	-58865683871.7408\\
3.53568839220981	-58839327813.1648\\
3.53578839470987	-58812971754.5887\\
3.53588839720993	-58786042738.2176\\
3.53598839970999	-58759686679.6416\\
3.53608840221006	-58733330621.0656\\
3.53618840471012	-58706974562.4895\\
3.53628840721018	-58680045546.1184\\
3.53638840971024	-58653689487.5424\\
3.53648841221031	-58627333428.9664\\
3.53658841471037	-58600977370.3903\\
3.53668841721043	-58574048354.0192\\
3.53678841971049	-58547692295.4432\\
3.53688842221056	-58521336236.8672\\
3.53698842471062	-58494980178.2911\\
3.53708842721068	-58468624119.7151\\
3.53718842971074	-58442268061.1391\\
3.53728843221081	-58415912002.5631\\
3.53738843471087	-58388982986.1919\\
3.53748843721093	-58362626927.6159\\
3.53758843971099	-58336270869.0399\\
3.53768844221106	-58309914810.4639\\
3.53778844471112	-58283558751.8879\\
3.53788844721118	-58257202693.3118\\
3.53798844971124	-58230846634.7358\\
3.53808845221131	-58204490576.1598\\
3.53818845471137	-58178134517.5838\\
3.53828845721143	-58151778459.0078\\
3.53838845971149	-58125422400.4318\\
3.53848846221156	-58099066341.8557\\
3.53858846471162	-58072710283.2797\\
3.53868846721168	-58046354224.7037\\
3.53878846971174	-58019998166.1277\\
3.53888847221181	-57993642107.5517\\
3.53898847471187	-57967286048.9756\\
3.53908847721193	-57940929990.3996\\
3.53918847971199	-57914573931.8236\\
3.53928848221206	-57888217873.2476\\
3.53938848471212	-57861861814.6716\\
3.53948848721218	-57835505756.0956\\
3.53958848971224	-57809149697.5195\\
3.53968849221231	-57783366596.7387\\
3.53978849471237	-57757010538.1626\\
3.53988849721243	-57730654479.5866\\
3.53998849971249	-57704298421.0106\\
3.54008850221256	-57677942362.4346\\
3.54018850471262	-57651586303.8586\\
3.54028850721268	-57625803203.0777\\
3.54038850971274	-57599447144.5017\\
3.54048851221281	-57573091085.9256\\
3.54058851471287	-57546735027.3496\\
3.54068851721293	-57520951926.5687\\
3.54078851971299	-57494595867.9927\\
3.54088852221306	-57468239809.4167\\
3.54098852471312	-57441883750.8407\\
3.54108852721318	-57416100650.0598\\
3.54118852971324	-57389744591.4838\\
3.54128853221331	-57363388532.9078\\
3.54138853471337	-57337605432.1269\\
3.54148853721343	-57311249373.5509\\
3.54158853971349	-57285065202.3134\\
3.54168854221356	-57258938326.8554\\
3.54178854471362	-57232754155.6179\\
3.54188854721368	-57206627280.16\\
3.54198854971374	-57180500404.702\\
3.54208855221381	-57154373529.244\\
3.54218855471387	-57128246653.7861\\
3.54228855721393	-57102177074.1076\\
3.54238855971399	-57076050198.6497\\
3.54248856221406	-57049980618.9712\\
3.54258856471412	-57023853743.5132\\
3.54268856721418	-56997784163.8348\\
3.54278856971424	-56971714584.1563\\
3.54288857221431	-56945645004.4779\\
3.54298857471437	-56919575424.7994\\
3.54308857721443	-56893505845.121\\
3.54318857971449	-56867436265.4425\\
3.54328858221456	-56841423981.5436\\
3.54338858471462	-56815354401.8651\\
3.54348858721468	-56789342117.9662\\
3.54358858971474	-56763329834.0672\\
3.54368859221481	-56737317550.1683\\
3.54378859471487	-56711305266.2694\\
3.54388859721493	-56685292982.3704\\
3.54398859971499	-56659280698.4715\\
3.54408860221506	-56633268414.5726\\
3.54418860471512	-56607313426.4531\\
3.54428860721518	-56581301142.5542\\
3.54438860971524	-56555346154.4348\\
3.54448861221531	-56529391166.3153\\
3.54458861471537	-56503378882.4164\\
3.54468861721543	-56477423894.297\\
3.54478861971549	-56451526201.9571\\
3.54488862221556	-56425571213.8376\\
3.54498862471562	-56399616225.7182\\
3.54508862721568	-56373718533.3783\\
3.54518862971574	-56347763545.2589\\
3.54528863221581	-56321865852.919\\
3.54538863471587	-56295968160.579\\
3.54548863721593	-56270013172.4596\\
3.54558863971599	-56244115480.1197\\
3.54568864221606	-56218275083.5593\\
3.54578864471612	-56192377391.2194\\
3.54588864721618	-56166479698.8795\\
3.54598864971624	-56140639302.3191\\
3.54608865221631	-56114741609.9792\\
3.54618865471637	-56088901213.4188\\
3.54628865721643	-56063060816.8584\\
3.54638865971649	-56037220420.298\\
3.54648866221656	-56011380023.7376\\
3.54658866471662	-55985539627.1772\\
3.54668866721668	-55959699230.6168\\
3.54678866971674	-55933858834.0564\\
3.54688867221681	-55908075733.2755\\
3.54698867471687	-55882235336.7151\\
3.54708867721693	-55856452235.9342\\
3.54718867971699	-55830669135.1533\\
3.54728868221706	-55804886034.3724\\
3.54738868471712	-55779102933.5915\\
3.54748868721718	-55753319832.8106\\
3.54758868971724	-55727536732.0297\\
3.54768869221731	-55701810927.0284\\
3.54778869471737	-55676027826.2475\\
3.54788869721743	-55650302021.2461\\
3.54798869971749	-55624576216.2447\\
3.54808870221756	-55598793115.4639\\
3.54818870471762	-55573067310.4625\\
3.54828870721768	-55547341505.4611\\
3.54838870971774	-55521672996.2392\\
3.54848871221781	-55495947191.2379\\
3.54858871471787	-55470221386.2365\\
3.54868871721793	-55444552877.0146\\
3.54878871971799	-55418827072.0133\\
3.54888872221806	-55393158562.7914\\
3.54898872471812	-55367490053.5695\\
3.54908872721818	-55341821544.3477\\
3.54918872971824	-55316153035.1258\\
3.54928873221831	-55290484525.904\\
3.54938873471837	-55264873312.4616\\
3.54948873721843	-55239204803.2397\\
3.54958873971849	-55213536294.0179\\
3.54968874221856	-55187925080.5755\\
3.54978874471862	-55162313867.1332\\
3.54988874721868	-55136702653.6908\\
3.54998874971874	-55111091440.2485\\
3.55008875221881	-55085480226.8061\\
3.55018875471887	-55059869013.3638\\
3.55028875721893	-55034257799.9214\\
3.55038875971899	-55008703882.2586\\
3.55048876221906	-54983092668.8163\\
3.55058876471912	-54957538751.1534\\
3.55068876721918	-54931984833.4906\\
3.55078876971924	-54906430915.8278\\
3.55088877221931	-54880876998.1649\\
3.55098877471937	-54855323080.5021\\
3.55108877721943	-54829769162.8393\\
3.55118877971949	-54804215245.1764\\
3.55128878221956	-54778718623.2931\\
3.55138878471962	-54753164705.6303\\
3.55148878721968	-54727668083.7469\\
3.55158878971974	-54702171461.8636\\
3.55168879221981	-54676674839.9803\\
3.55178879471987	-54651178218.097\\
3.55188879721993	-54625681596.2137\\
3.55198879971999	-54600184974.3303\\
3.55208880222006	-54574688352.447\\
3.55218880472012	-54549249026.3432\\
3.55228880722018	-54523752404.4599\\
3.55238880972024	-54498313078.3561\\
3.55248881222031	-54472873752.2523\\
3.55258881472037	-54447434426.1485\\
3.55268881722043	-54421995100.0447\\
3.55278881972049	-54396555773.9408\\
3.55288882222056	-54371116447.837\\
3.55298882472062	-54345677121.7332\\
3.55308882722068	-54320295091.4089\\
3.55318882972074	-54294913061.0846\\
3.55328883222081	-54269473734.9808\\
3.55338883472087	-54244091704.6565\\
3.55348883722093	-54218709674.3322\\
3.55358883972099	-54193327644.0079\\
3.55368884222106	-54167945613.6836\\
3.55378884472112	-54142563583.3594\\
3.55388884722118	-54117238848.8146\\
3.55398884972124	-54091856818.4903\\
3.55408885222131	-54066532083.9455\\
3.55418885472137	-54041207349.4007\\
3.55428885722143	-54015825319.0764\\
3.55438885972149	-53990500584.5316\\
3.55448886222156	-53965175849.9868\\
3.55458886472162	-53939908411.2216\\
3.55468886722168	-53914583676.6768\\
3.55478886972174	-53889258942.132\\
3.55488887222181	-53863991503.3667\\
3.55498887472187	-53838666768.822\\
3.55508887722193	-53813399330.0567\\
3.55518887972199	-53788131891.2914\\
3.55528888222206	-53762864452.5262\\
3.55538888472212	-53737597013.7609\\
3.55548888722218	-53712329574.9956\\
3.55558888972224	-53687062136.2303\\
3.55568889222231	-53661851993.2446\\
3.55578889472237	-53636584554.4793\\
3.55588889722243	-53611374411.4936\\
3.55598889972249	-53586164268.5078\\
3.55608890222256	-53560896829.7425\\
3.55618890472262	-53535686686.7568\\
3.55628890722268	-53510476543.771\\
3.55638890972274	-53485323696.5648\\
3.55648891222281	-53460113553.579\\
3.55658891472287	-53434903410.5933\\
3.55668891722293	-53409750563.387\\
3.55678891972299	-53384597716.1808\\
3.55688892222306	-53359387573.195\\
3.55698892472312	-53334234725.9888\\
3.55708892722318	-53309081878.7825\\
3.55718892972324	-53283929031.5763\\
3.55728893222331	-53258776184.3701\\
3.55738893472337	-53233680632.9433\\
3.55748893722343	-53208527785.7371\\
3.55758893972349	-53183432234.3104\\
3.55768894222356	-53158279387.1041\\
3.55778894472362	-53133183835.6774\\
3.55788894722368	-53108088284.2506\\
3.55798894972374	-53082992732.8239\\
3.55808895222381	-53057897181.3972\\
3.55818895472387	-53032801629.9705\\
3.55828895722393	-53007763374.3232\\
3.55838895972399	-52982667822.8965\\
3.55848896222406	-52957629567.2493\\
3.55858896472412	-52932534015.8226\\
3.55868896722418	-52907495760.1753\\
3.55878896972424	-52882457504.5281\\
3.55888897222431	-52857419248.8809\\
3.55898897472437	-52832380993.2337\\
3.55908897722443	-52807342737.5865\\
3.55918897972449	-52782361777.7188\\
3.55928898222456	-52757323522.0716\\
3.55938898472462	-52732342562.2039\\
3.55948898722468	-52707304306.5566\\
3.55958898972474	-52682323346.6889\\
3.55968899222481	-52657342386.8212\\
3.55978899472487	-52632361426.9535\\
3.55988899722493	-52607380467.0858\\
3.55998899972499	-52582399507.2181\\
3.56008900222506	-52557475843.1299\\
3.56018900472512	-52532494883.2622\\
3.56028900722518	-52507571219.174\\
3.56038900972524	-52482590259.3063\\
3.56048901222531	-52457666595.2181\\
3.56058901472537	-52432742931.13\\
3.56068901722543	-52407819267.0418\\
3.56078901972549	-52382895602.9536\\
3.56088902222556	-52358029234.6449\\
3.56098902472562	-52333105570.5567\\
3.56108902722568	-52308181906.4685\\
3.56118902972574	-52283315538.1598\\
3.56128903222581	-52258449169.8512\\
3.56138903472587	-52233525505.763\\
3.56148903722593	-52208659137.4543\\
3.56158903972599	-52183792769.1456\\
3.56168904222606	-52158983696.6164\\
3.56178904472612	-52134117328.3078\\
3.56188904722618	-52109250959.9991\\
3.56198904972624	-52084441887.4699\\
3.56208905222631	-52059575519.1612\\
3.56218905472637	-52034766446.6321\\
3.56228905722643	-52009957374.1029\\
3.56238905972649	-51985148301.5737\\
3.56248906222656	-51960339229.0446\\
3.56258906472662	-51935530156.5154\\
3.56268906722668	-51910721083.9863\\
3.56278906972674	-51885969307.2366\\
3.56288907222681	-51861160234.7074\\
3.56298907472687	-51836408457.9578\\
3.56308907722693	-51811599385.4286\\
3.56318907972699	-51786847608.679\\
3.56328908222706	-51762095831.9293\\
3.56338908472712	-51737344055.1797\\
3.56348908722718	-51712592278.43\\
3.56358908972724	-51687897797.4599\\
3.56368909222731	-51663146020.7102\\
3.56378909472737	-51638451539.7401\\
3.56388909722743	-51613699762.9904\\
3.56398909972749	-51589005282.0203\\
3.56408910222756	-51564310801.0502\\
3.56418910472762	-51539616320.08\\
3.56428910722768	-51514921839.1099\\
3.56438910972774	-51490227358.1397\\
3.56448911222781	-51465532877.1696\\
3.56458911472787	-51440895691.979\\
3.56468911722793	-51416201211.0088\\
3.56478911972799	-51391564025.8182\\
3.56488912222806	-51366869544.8481\\
3.56498912472812	-51342232359.6575\\
3.56508912722818	-51317595174.4668\\
3.56518912972824	-51292957989.2762\\
3.56528913222831	-51268378099.8651\\
3.56538913472837	-51243740914.6745\\
3.56548913722843	-51219103729.4838\\
3.56558913972849	-51194523840.0727\\
3.56568914222856	-51169886654.8821\\
3.56578914472862	-51145306765.471\\
3.56588914722868	-51120726876.0599\\
3.56598914972874	-51096146986.6488\\
3.56608915222881	-51071567097.2376\\
3.56618915472887	-51046987207.8265\\
3.56628915722893	-51022464614.1949\\
3.56638915972899	-50997884724.7838\\
3.56648916222906	-50973362131.1522\\
3.56658916472912	-50948782241.7411\\
3.56668916722918	-50924259648.1095\\
3.56678916972924	-50899737054.4779\\
3.56688917222931	-50875214460.8463\\
3.56698917472937	-50850691867.2147\\
3.56708917722943	-50826169273.5831\\
3.56718917972949	-50801646679.9515\\
3.56728918222956	-50777181382.0994\\
3.56738918472962	-50752658788.4678\\
3.56748918722968	-50728193490.6157\\
3.56758918972974	-50703728192.7637\\
3.56768919222981	-50679262894.9116\\
3.56778919472987	-50654797597.0595\\
3.56788919722993	-50630332299.2074\\
3.56798919972999	-50605867001.3553\\
3.56808920223006	-50581401703.5032\\
3.56818920473012	-50556993701.4307\\
3.56828920723018	-50532528403.5786\\
3.56838920973024	-50508120401.506\\
3.56848921223031	-50483712399.4334\\
3.56858921473037	-50459304397.3609\\
3.56868921723043	-50434896395.2883\\
3.56878921973049	-50410488393.2157\\
3.56888922223056	-50386080391.1431\\
3.56898922473062	-50361672389.0706\\
3.56908922723068	-50337321682.7775\\
3.56918922973074	-50312913680.7049\\
3.56928923223081	-50288562974.4119\\
3.56938923473087	-50264212268.1188\\
3.56948923723093	-50239861561.8257\\
3.56958923973099	-50215510855.5327\\
3.56968924223106	-50191160149.2396\\
3.56978924473112	-50166809442.9466\\
3.56988924723118	-50142458736.6535\\
3.56998924973124	-50118165326.14\\
3.57008925223131	-50093814619.8469\\
3.57018925473137	-50069521209.3334\\
3.57028925723143	-50045227798.8198\\
3.57038925973149	-50020934388.3063\\
3.57048926223156	-49996640977.7927\\
3.57058926473162	-49972347567.2792\\
3.57068926723168	-49948054156.7656\\
3.57078926973174	-49923760746.2521\\
3.57088927223181	-49899524631.518\\
3.57098927473187	-49875231221.0045\\
3.57108927723193	-49850995106.2705\\
3.57118927973199	-49826758991.5364\\
3.57128928223206	-49802522876.8024\\
3.57138928473212	-49778286762.0684\\
3.57148928723218	-49754050647.3343\\
3.57158928973224	-49729814532.6003\\
3.57168929223231	-49705635713.6458\\
3.57178929473237	-49681399598.9117\\
3.57188929723243	-49657220779.9572\\
3.57198929973249	-49632984665.2232\\
3.57208930223256	-49608805846.2687\\
3.57218930473262	-49584627027.3141\\
3.57228930723268	-49560448208.3596\\
3.57238930973274	-49536269389.4051\\
3.57248931223281	-49512147866.2301\\
3.57258931473287	-49487969047.2756\\
3.57268931723293	-49463790228.321\\
3.57278931973299	-49439668705.146\\
3.57288932223306	-49415547181.971\\
3.57298932473312	-49391368363.0165\\
3.57308932723318	-49367246839.8415\\
3.57318932973324	-49343125316.6665\\
3.57328933223331	-49319061089.271\\
3.57338933473337	-49294939566.096\\
3.57348933723343	-49270818042.921\\
3.57358933973349	-49246753815.5255\\
3.57368934223356	-49222632292.3505\\
3.57378934473362	-49198568064.955\\
3.57388934723368	-49174503837.5595\\
3.57398934973374	-49150439610.164\\
3.57408935223381	-49126375382.7685\\
3.57418935473387	-49102311155.373\\
3.57428935723393	-49078246927.9775\\
3.57438935973399	-49054239996.3615\\
3.57448936223406	-49030175768.966\\
3.57458936473412	-49006168837.3501\\
3.57468936723418	-48982104609.9546\\
3.57478936973424	-48958097678.3386\\
3.57488937223431	-48934090746.7226\\
3.57498937473437	-48910083815.1066\\
3.57508937723443	-48886076883.4906\\
3.57518937973449	-48862127247.6542\\
3.57528938223456	-48838120316.0382\\
3.57538938473462	-48814113384.4222\\
3.57548938723468	-48790163748.5857\\
3.57558938973474	-48766214112.7493\\
3.57568939223481	-48742264476.9128\\
3.57578939473487	-48718314841.0763\\
3.57588939723493	-48694365205.2399\\
3.57598939973499	-48670415569.4034\\
3.57608940223506	-48646465933.5669\\
3.57618940473512	-48622516297.7305\\
3.57628940723518	-48598623957.6735\\
3.57638940973524	-48574731617.6165\\
3.57648941223531	-48550781981.7801\\
3.57658941473537	-48526889641.7231\\
3.57668941723543	-48502997301.6662\\
3.57678941973549	-48479104961.6092\\
3.57688942223556	-48455212621.5523\\
3.57698942473562	-48431377577.2748\\
3.57708942723568	-48407485237.2179\\
3.57718942973574	-48383592897.1609\\
3.57728943223581	-48359757852.8835\\
3.57738943473587	-48335922808.606\\
3.57748943723593	-48312087764.3286\\
3.57758943973599	-48288252720.0511\\
3.57768944223606	-48264417675.7737\\
3.57778944473612	-48240582631.4963\\
3.57788944723618	-48216747587.2188\\
3.57798944973624	-48192912542.9414\\
3.57808945223631	-48169134794.4434\\
3.57818945473637	-48145357045.9455\\
3.57828945723643	-48121522001.6681\\
3.57838945973649	-48097744253.1701\\
3.57848946223656	-48073966504.6722\\
3.57858946473662	-48050188756.1743\\
3.57868946723668	-48026411007.6764\\
3.57878946973674	-48002690554.9579\\
3.57888947223681	-47978912806.46\\
3.57898947473687	-47955192353.7416\\
3.57908947723693	-47931414605.2437\\
3.57918947973699	-47907694152.5252\\
3.57928948223706	-47883973699.8068\\
3.57938948473712	-47860253247.0884\\
3.57948948723718	-47836532794.37\\
3.57958948973724	-47812812341.6516\\
3.57968949223731	-47789091888.9332\\
3.57978949473737	-47765428731.9943\\
3.57988949723743	-47741708279.2758\\
3.57998949973749	-47718045122.3369\\
3.58008950223756	-47694381965.398\\
3.58018950473762	-47670661512.6796\\
3.58028950723768	-47646998355.7407\\
3.58038950973774	-47623335198.8018\\
3.58048951223781	-47599729337.6424\\
3.58058951473787	-47576066180.7035\\
3.58068951723793	-47552403023.7646\\
3.58078951973799	-47528797162.6052\\
3.58088952223806	-47505134005.6663\\
3.58098952473812	-47481528144.5069\\
3.58108952723818	-47457922283.3475\\
3.58118952973824	-47434316422.1882\\
3.58128953223831	-47410710561.0288\\
3.58138953473837	-47387104699.8694\\
3.58148953723843	-47363556134.4895\\
3.58158953973849	-47339950273.3301\\
3.58168954223856	-47316401707.9502\\
3.58178954473862	-47292795846.7908\\
3.58188954723868	-47269247281.411\\
3.58198954973874	-47245698716.0311\\
3.58208955223881	-47222150150.6512\\
3.58218955473887	-47198601585.2713\\
3.58228955723893	-47175053019.8915\\
3.58238955973899	-47151504454.5116\\
3.58248956223906	-47128013184.9112\\
3.58258956473912	-47104464619.5313\\
3.58268956723918	-47080973349.931\\
3.58278956973924	-47057482080.3306\\
3.58288957223931	-47033933514.9507\\
3.58298957473937	-47010442245.3504\\
3.58308957723943	-46987008271.5295\\
3.58318957973949	-46963517001.9292\\
3.58328958223956	-46940025732.3288\\
3.58338958473962	-46916534462.7284\\
3.58348958723968	-46893100488.9076\\
3.58358958973974	-46869666515.0867\\
3.58368959223981	-46846175245.4864\\
3.58378959473987	-46822741271.6655\\
3.58388959723993	-46799307297.8447\\
3.58398959973999	-46775873324.0238\\
3.58408960224006	-46752439350.203\\
3.58418960474012	-46729062672.1616\\
3.58428960724018	-46705628698.3408\\
3.58438960974024	-46682252020.2994\\
3.58448961224031	-46658818046.4786\\
3.58458961474037	-46635441368.4373\\
3.58468961724043	-46612064690.3959\\
3.58478961974049	-46588688012.3546\\
3.58488962224056	-46565311334.3132\\
3.58498962474062	-46541934656.2719\\
3.58508962724068	-46518615274.0101\\
3.58518962974074	-46495238595.9687\\
3.58528963224081	-46471861917.9274\\
3.58538963474087	-46448542535.6656\\
3.58548963724093	-46425223153.4038\\
3.58558963974099	-46401903771.1419\\
3.58568964224106	-46378584388.8801\\
3.58578964474112	-46355265006.6183\\
3.58588964724118	-46331945624.3565\\
3.58598964974124	-46308626242.0946\\
3.58608965224131	-46285364155.6123\\
3.58618965474137	-46262044773.3505\\
3.58628965724143	-46238782686.8682\\
3.58638965974149	-46215520600.3859\\
3.58648966224156	-46192201218.124\\
3.58658966474162	-46168939131.6417\\
3.58668966724168	-46145734340.9389\\
3.58678966974174	-46122472254.4566\\
3.58688967224181	-46099210167.9743\\
3.58698967474187	-46075948081.492\\
3.58708967724193	-46052743290.7892\\
3.58718967974199	-46029538500.0864\\
3.58728968224206	-46006276413.6041\\
3.58738968474212	-45983071622.9013\\
3.58748968724218	-45959866832.1985\\
3.58758968974224	-45936662041.4957\\
3.58768969224231	-45913457250.7929\\
3.58778969474237	-45890309755.8696\\
3.58788969724243	-45867104965.1668\\
3.58798969974249	-45843957470.2435\\
3.58808970224256	-45820752679.5407\\
3.58818970474262	-45797605184.6174\\
3.58828970724268	-45774457689.6942\\
3.58838970974274	-45751310194.7709\\
3.58848971224281	-45728162699.8476\\
3.58858971474287	-45705015204.9243\\
3.58868971724293	-45681925005.7805\\
3.58878971974299	-45658777510.8573\\
3.58888972224306	-45635630015.934\\
3.58898972474312	-45612539816.7902\\
3.58908972724318	-45589449617.6464\\
3.58918972974324	-45566359418.5026\\
3.58928973224331	-45543269219.3589\\
3.58938973474337	-45520179020.2151\\
3.58948973724343	-45497088821.0713\\
3.58958973974349	-45473998621.9276\\
3.58968974224356	-45450965718.5633\\
3.58978974474362	-45427875519.4195\\
3.58988974724368	-45404842616.0553\\
3.58998974974374	-45381809712.691\\
3.59008975224381	-45358719513.5472\\
3.59018975474387	-45335686610.183\\
3.59028975724393	-45312711002.5982\\
3.59038975974399	-45289678099.234\\
3.59048976224406	-45266645195.8697\\
3.59058976474412	-45243612292.5055\\
3.59068976724418	-45220636684.9207\\
3.59078976974424	-45197661077.336\\
3.59088977224431	-45174628173.9717\\
3.59098977474437	-45151652566.387\\
3.59108977724443	-45128676958.8022\\
3.59118977974449	-45105701351.2175\\
3.59128978224456	-45082725743.6327\\
3.59138978474462	-45059807431.8275\\
3.59148978724468	-45036831824.2427\\
3.59158978974474	-45013913512.4375\\
3.59168979224481	-44990937904.8528\\
3.59178979474487	-44968019593.0475\\
3.59188979724493	-44945101281.2423\\
3.59198979974499	-44922182969.4371\\
3.59208980224506	-44899264657.6318\\
3.59218980474512	-44876346345.8266\\
3.59228980724518	-44853428034.0214\\
3.59238980974524	-44830567017.9956\\
3.59248981224531	-44807648706.1904\\
3.59258981474537	-44784787690.1647\\
3.59268981724543	-44761926674.139\\
3.59278981974549	-44739065658.1133\\
3.59288982224556	-44716204642.0875\\
3.59298982474562	-44693343626.0618\\
3.59308982724568	-44670482610.0361\\
3.59318982974574	-44647621594.0104\\
3.59328983224581	-44624817873.7642\\
3.59338983474587	-44601956857.7384\\
3.59348983724593	-44579153137.4922\\
3.59358983974599	-44556292121.4665\\
3.59368984224606	-44533488401.2203\\
3.59378984474612	-44510684680.9741\\
3.59388984724618	-44487880960.7279\\
3.59398984974624	-44465134536.2612\\
3.59408985224631	-44442330816.015\\
3.59418985474637	-44419527095.7688\\
3.59428985724643	-44396780671.3021\\
3.59438985974649	-44373976951.0559\\
3.59448986224656	-44351230526.5892\\
3.59458986474662	-44328484102.1225\\
3.59468986724668	-44305737677.6558\\
3.59478986974674	-44282991253.1891\\
3.59488987224681	-44260244828.7224\\
3.59498987474687	-44237498404.2557\\
3.59508987724693	-44214809275.5686\\
3.59518987974699	-44192062851.1019\\
3.59528988224706	-44169373722.4147\\
3.59538988474712	-44146684593.7275\\
3.59548988724718	-44123995465.0403\\
3.59558988974724	-44101306336.3531\\
3.59568989224731	-44078617207.666\\
3.59578989474737	-44055928078.9788\\
3.59588989724743	-44033238950.2916\\
3.59598989974749	-44010549821.6044\\
3.59608990224756	-43987917988.6967\\
3.59618990474762	-43965286155.7891\\
3.59628990724768	-43942597027.1019\\
3.59638990974774	-43919965194.1942\\
3.59648991224781	-43897333361.2866\\
3.59658991474787	-43874701528.3789\\
3.59668991724793	-43852069695.4712\\
3.59678991974799	-43829495158.3431\\
3.59688992224806	-43806863325.4354\\
3.59698992474812	-43784288788.3073\\
3.59708992724818	-43761656955.3996\\
3.59718992974824	-43739082418.2714\\
3.59728993224831	-43716507881.1433\\
3.59738993474837	-43693933344.0151\\
3.59748993724843	-43671358806.887\\
3.59758993974849	-43648784269.7588\\
3.59768994224856	-43626209732.6307\\
3.59778994474862	-43603692491.282\\
3.59788994724868	-43581117954.1539\\
3.59798994974874	-43558600712.8052\\
3.59808995224881	-43536026175.6771\\
3.59818995474887	-43513508934.3284\\
3.59828995724893	-43490991692.9798\\
3.59838995974899	-43468474451.6311\\
3.59848996224906	-43445957210.2825\\
3.59858996474912	-43423497264.7134\\
3.59868996724918	-43400980023.3647\\
3.59878996974924	-43378462782.0161\\
3.59888997224931	-43356002836.447\\
3.59898997474937	-43333542890.8778\\
3.59908997724943	-43311082945.3087\\
3.59918997974949	-43288622999.7396\\
3.59928998224956	-43266163054.1704\\
3.59938998474962	-43243703108.6013\\
3.59948998724968	-43221243163.0322\\
3.59958998974974	-43198783217.4631\\
3.59968999224981	-43176380567.6734\\
3.59978999474987	-43153920622.1043\\
3.59988999724993	-43131517972.3147\\
3.59998999974999	-43109115322.5251\\
3.60009000225006	-43086712672.7355\\
};
\addplot [color=mycolor3,solid]
  table[row sep=crcr]{%
3.60009000225006	-43086712672.7355\\
3.60019000475012	-43064310022.9459\\
3.60029000725018	-43041907373.1562\\
3.60039000975024	-43019504723.3666\\
3.60049001225031	-42997159369.3565\\
3.60059001475037	-42974756719.5669\\
3.60069001725043	-42952411365.5568\\
3.60079001975049	-42930066011.5467\\
3.60089002225056	-42907663361.7571\\
3.60099002475062	-42885318007.747\\
3.60109002725068	-42862972653.7369\\
3.60119002975074	-42840627299.7268\\
3.60129003225081	-42818339241.4962\\
3.60139003475087	-42795993887.4861\\
3.60149003725093	-42773648533.476\\
3.60159003975099	-42751360475.2454\\
3.60169004225106	-42729072417.0148\\
3.60179004475112	-42706784358.7842\\
3.60189004725118	-42684439004.7741\\
3.60199004975124	-42662208242.323\\
3.60209005225131	-42639920184.0925\\
3.60219005475137	-42617632125.8619\\
3.60229005725143	-42595344067.6313\\
3.60239005975149	-42573113305.1802\\
3.60249006225156	-42550825246.9496\\
3.60259006475162	-42528594484.4985\\
3.60269006725168	-42506363722.0475\\
3.60279006975174	-42484132959.5964\\
3.60289007225181	-42461902197.1453\\
3.60299007475187	-42439671434.6942\\
3.60309007725193	-42417440672.2432\\
3.60319007975199	-42395209909.7921\\
3.60329008225206	-42373036443.1205\\
3.60339008475212	-42350805680.6694\\
3.60349008725218	-42328632213.9979\\
3.60359008975224	-42306458747.3263\\
3.60369009225231	-42284285280.6548\\
3.60379009475237	-42262111813.9832\\
3.60389009725243	-42239938347.3116\\
3.60399009975249	-42217764880.6401\\
3.60409010225256	-42195591413.9685\\
3.60419010475262	-42173475243.0765\\
3.60429010725268	-42151359072.1844\\
3.60439010975274	-42129185605.5128\\
3.60449011225281	-42107069434.6208\\
3.60459011475287	-42084953263.7287\\
3.60469011725293	-42062837092.8367\\
3.60479011975299	-42040720921.9446\\
3.60489012225306	-42018604751.0526\\
3.60499012475312	-41996545875.9401\\
3.60509012725318	-41974429705.048\\
3.60519012975324	-41952313534.156\\
3.60529013225331	-41930254659.0434\\
3.60539013475337	-41908195783.9309\\
3.60549013725343	-41886136908.8183\\
3.60559013975349	-41864078033.7058\\
3.60569014225356	-41842019158.5933\\
3.60579014475362	-41819960283.4807\\
3.60589014725368	-41797901408.3682\\
3.60599014975374	-41775899829.0352\\
3.60609015225381	-41753840953.9226\\
3.60619015475387	-41731839374.5896\\
3.60629015725393	-41709837795.2566\\
3.60639015975399	-41687836215.9236\\
3.60649016225406	-41665834636.5905\\
3.60659016475412	-41643833057.2575\\
3.60669016725418	-41621831477.9245\\
3.60679016975424	-41599829898.5915\\
3.60689017225431	-41577885615.038\\
3.60699017475437	-41555884035.7049\\
3.60709017725443	-41533939752.1514\\
3.60719017975449	-41511995468.5979\\
3.60729018225456	-41489993889.2649\\
3.60739018475462	-41468049605.7114\\
3.60749018725468	-41446162617.9374\\
3.60759018975474	-41424218334.3839\\
3.60769019225481	-41402274050.8304\\
3.60779019475487	-41380329767.2769\\
3.60789019725493	-41358442779.5029\\
3.60799019975499	-41336555791.7289\\
3.60809020225506	-41314611508.1753\\
3.60819020475512	-41292724520.4014\\
3.60829020725518	-41270837532.6274\\
3.60839020975524	-41248950544.8534\\
3.60849021225531	-41227063557.0794\\
3.60859021475537	-41205233865.0849\\
3.60869021725543	-41183346877.3109\\
3.60879021975549	-41161517185.3164\\
3.60889022225556	-41139630197.5424\\
3.60899022475562	-41117800505.5479\\
3.60909022725568	-41095970813.5534\\
3.60919022975574	-41074141121.5589\\
3.60929023225581	-41052311429.5645\\
3.60939023475587	-41030481737.57\\
3.60949023725593	-41008652045.5755\\
3.60959023975599	-40986879649.3605\\
3.60969024225606	-40965049957.366\\
3.60979024475612	-40943277561.1511\\
3.60989024725618	-40921447869.1566\\
3.60999024975624	-40899675472.9416\\
3.61009025225631	-40877903076.7266\\
3.61019025475637	-40856130680.5117\\
3.61029025725643	-40834358284.2967\\
3.61039025975649	-40812643183.8612\\
3.61049026225656	-40790870787.6463\\
3.61059026475662	-40769155687.2108\\
3.61069026725668	-40747383290.9958\\
3.61079026975674	-40725668190.5604\\
3.61089027225681	-40703953090.1249\\
3.61099027475687	-40682237989.6895\\
3.61109027725693	-40660522889.254\\
3.61119027975699	-40638807788.8185\\
3.61129028225706	-40617092688.3831\\
3.61139028475712	-40595377587.9476\\
3.61149028725718	-40573719783.2917\\
3.61159028975724	-40552004682.8562\\
3.61169029225731	-40530346878.2003\\
3.61179029475737	-40508689073.5443\\
3.61189029725743	-40487031268.8884\\
3.61199029975749	-40465373464.2324\\
3.61209030225756	-40443715659.5765\\
3.61219030475762	-40422057854.9206\\
3.61229030725768	-40400457346.0441\\
3.61239030975774	-40378799541.3882\\
3.61249031225781	-40357199032.5117\\
3.61259031475787	-40335541227.8558\\
3.61269031725793	-40313940718.9794\\
3.61279031975799	-40292340210.1029\\
3.61289032225806	-40270739701.2265\\
3.61299032475812	-40249139192.3501\\
3.61309032725818	-40227538683.4736\\
3.61319032975824	-40205995470.3767\\
3.61329033225831	-40184394961.5003\\
3.61339033475837	-40162851748.4034\\
3.61349033725843	-40141251239.5269\\
3.61359033975849	-40119708026.43\\
3.61369034225856	-40098164813.3331\\
3.61379034475862	-40076621600.2362\\
3.61389034725868	-40055078387.1393\\
3.61399034975874	-40033535174.0423\\
3.61409035225881	-40012049256.7249\\
3.61419035475887	-39990506043.628\\
3.61429035725893	-39969020126.3106\\
3.61439035975899	-39947476913.2137\\
3.61449036225906	-39925990995.8963\\
3.61459036475912	-39904505078.5789\\
3.61469036725918	-39883019161.2615\\
3.61479036975924	-39861533243.9441\\
3.61489037225931	-39840047326.6267\\
3.61499037475937	-39818618705.0888\\
3.61509037725943	-39797132787.7714\\
3.61519037975949	-39775704166.2335\\
3.61529038225956	-39754218248.9161\\
3.61539038475962	-39732789627.3782\\
3.61549038725968	-39711361005.8403\\
3.61559038975974	-39689932384.3024\\
3.61569039225981	-39668503762.7645\\
3.61579039475987	-39647075141.2266\\
3.61589039725993	-39625646519.6887\\
3.61599039975999	-39604275193.9303\\
3.61609040226006	-39582846572.3924\\
3.61619040476012	-39561475246.6341\\
3.61629040726018	-39540103920.8757\\
3.61639040976024	-39518675299.3378\\
3.61649041226031	-39497303973.5794\\
3.61659041476037	-39475932647.821\\
3.61669041726043	-39454618617.8422\\
3.61679041976049	-39433247292.0838\\
3.61689042226056	-39411875966.3254\\
3.61699042476062	-39390561936.3465\\
3.61709042726068	-39369190610.5882\\
3.61719042976074	-39347876580.6093\\
3.61729043226081	-39326562550.6304\\
3.61739043476087	-39305248520.6516\\
3.61749043726093	-39283934490.6727\\
3.61759043976099	-39262620460.6938\\
3.61769044226106	-39241306430.715\\
3.61779044476112	-39220049696.5156\\
3.61789044726118	-39198735666.5367\\
3.61799044976124	-39177478932.3374\\
3.61809045226131	-39156164902.3585\\
3.61819045476137	-39134908168.1592\\
3.61829045726143	-39113651433.9598\\
3.61839045976149	-39092394699.7605\\
3.61849046226156	-39071137965.5611\\
3.61859046476162	-39049938527.1413\\
3.61869046726168	-39028681792.9419\\
3.61879046976174	-39007425058.7426\\
3.61889047226181	-38986225620.3227\\
3.61899047476187	-38965026181.9029\\
3.61909047726193	-38943769447.7035\\
3.61919047976199	-38922570009.2837\\
3.61929048226206	-38901370570.8638\\
3.61939048476212	-38880228428.2235\\
3.61949048726218	-38859028989.8037\\
3.61959048976224	-38837829551.3838\\
3.61969049226231	-38816687408.7435\\
3.61979049476237	-38795487970.3237\\
3.61989049726243	-38774345827.6833\\
3.61999049976249	-38753203685.043\\
3.62009050226256	-38732004246.6232\\
3.62019050476262	-38710862103.9828\\
3.62029050726268	-38689777257.122\\
3.62039050976274	-38668635114.4817\\
3.62049051226281	-38647492971.8414\\
3.62059051476287	-38626408124.9806\\
3.62069051726293	-38605265982.3402\\
3.62079051976299	-38584181135.4794\\
3.62089052226306	-38563038992.8391\\
3.62099052476312	-38541954145.9783\\
3.62109052726318	-38520869299.1175\\
3.62119052976324	-38499784452.2566\\
3.62129053226331	-38478756901.1753\\
3.62139053476337	-38457672054.3145\\
3.62149053726343	-38436587207.4537\\
3.62159053976349	-38415559656.3724\\
3.62169054226356	-38394474809.5116\\
3.62179054476362	-38373447258.4303\\
3.62189054726368	-38352419707.349\\
3.62199054976374	-38331392156.2677\\
3.62209055226381	-38310364605.1864\\
3.62219055476387	-38289337054.1051\\
3.62229055726393	-38268366798.8033\\
3.62239055976399	-38247339247.722\\
3.62249056226406	-38226311696.6407\\
3.62259056476412	-38205341441.3389\\
3.62269056726418	-38184371186.0371\\
3.62279056976424	-38163400930.7353\\
3.62289057226431	-38142430675.4335\\
3.62299057476437	-38121460420.1318\\
3.62309057726443	-38100490164.83\\
3.62319057976449	-38079519909.5282\\
3.62329058226456	-38058549654.2264\\
3.62339058476462	-38037636694.7041\\
3.62349058726468	-38016666439.4023\\
3.62359058976474	-37995753479.8801\\
3.62369059226481	-37974840520.3578\\
3.62379059476487	-37953927560.8355\\
3.62389059726493	-37933014601.3132\\
3.62399059976499	-37912101641.791\\
3.62409060226506	-37891188682.2687\\
3.62419060476512	-37870333018.5259\\
3.62429060726518	-37849420059.0036\\
3.62439060976524	-37828564395.2609\\
3.62449061226531	-37807651435.7386\\
3.62459061476537	-37786795771.9958\\
3.62469061726543	-37765940108.2531\\
3.62479061976549	-37745084444.5103\\
3.62489062226556	-37724228780.7676\\
3.62499062476562	-37703373117.0248\\
3.62509062726568	-37682574749.0615\\
3.62519062976574	-37661719085.3188\\
3.62529063226581	-37640920717.3555\\
3.62539063476587	-37620065053.6128\\
3.62549063726593	-37599266685.6495\\
3.62559063976599	-37578468317.6863\\
3.62569064226606	-37557669949.723\\
3.62579064476612	-37536871581.7598\\
3.62589064726618	-37516073213.7965\\
3.62599064976624	-37495332141.6128\\
3.62609065226631	-37474533773.6495\\
3.62619065476637	-37453792701.4658\\
3.62629065726643	-37432994333.5026\\
3.62639065976649	-37412253261.3188\\
3.62649066226656	-37391512189.1351\\
3.62659066476662	-37370771116.9514\\
3.62669066726668	-37350030044.7676\\
3.62679066976674	-37329288972.5839\\
3.62689067226681	-37308547900.4001\\
3.62699067476687	-37287864123.9959\\
3.62709067726693	-37267123051.8122\\
3.62719067976699	-37246439275.408\\
3.62729068226706	-37225755499.0037\\
3.62739068476712	-37205071722.5995\\
3.62749068726718	-37184387946.1953\\
3.62759068976724	-37163704169.7911\\
3.62769069226731	-37143020393.3868\\
3.62779069476737	-37122336616.9826\\
3.62789069726743	-37101652840.5784\\
3.62799069976749	-37081026359.9537\\
3.62809070226756	-37060342583.5495\\
3.62819070476762	-37039716102.9248\\
3.62829070726768	-37019089622.3001\\
3.62839070976774	-36998463141.6753\\
3.62849071226781	-36977836661.0506\\
3.62859071476787	-36957210180.4259\\
3.62869071726793	-36936583699.8012\\
3.62879071976799	-36916014514.956\\
3.62889072226806	-36895388034.3313\\
3.62899072476812	-36874818849.4861\\
3.62909072726818	-36854192368.8614\\
3.62919072976824	-36833623184.0162\\
3.62929073226831	-36813053999.171\\
3.62939073476837	-36792484814.3258\\
3.62949073726843	-36771915629.4806\\
3.62959073976849	-36751403740.4149\\
3.62969074226856	-36730834555.5697\\
3.62979074476862	-36710265370.7245\\
3.62989074726868	-36689753481.6589\\
3.62999074976874	-36669241592.5932\\
3.63009075226881	-36648672407.748\\
3.63019075476887	-36628160518.6823\\
3.63029075726893	-36607648629.6166\\
3.63039075976899	-36587136740.5509\\
3.63049076226906	-36566624851.4852\\
3.63059076476912	-36546170258.1991\\
3.63069076726918	-36525658369.1334\\
3.63079076976924	-36505203775.8472\\
3.63089077226931	-36484691886.7815\\
3.63099077476937	-36464237293.4954\\
3.63109077726943	-36443782700.2092\\
3.63119077976949	-36423328106.923\\
3.63129078226956	-36402873513.6369\\
3.63139078476962	-36382418920.3507\\
3.63149078726968	-36361964327.0645\\
3.63159078976974	-36341567029.5579\\
3.63169079226981	-36321112436.2717\\
3.63179079476987	-36300715138.765\\
3.63189079726993	-36280317841.2584\\
3.63199079976999	-36259920543.7517\\
3.63209080227006	-36239465950.4655\\
3.63219080477012	-36219125948.7384\\
3.63229080727018	-36198728651.2317\\
3.63239080977024	-36178331353.7251\\
3.63249081227031	-36157934056.2184\\
3.63259081477037	-36137594054.4913\\
3.63269081727043	-36117196756.9846\\
3.63279081977049	-36096856755.2575\\
3.63289082227056	-36076516753.5303\\
3.63299082477062	-36056176751.8032\\
3.63309082727068	-36035836750.076\\
3.63319082977074	-36015496748.3489\\
3.63329083227081	-35995156746.6218\\
3.63339083477087	-35974874040.6741\\
3.63349083727093	-35954534038.947\\
3.63359083977099	-35934251332.9994\\
3.63369084227106	-35913911331.2722\\
3.63379084477112	-35893628625.3246\\
3.63389084727118	-35873345919.3769\\
3.63399084977124	-35853063213.4293\\
3.63409085227131	-35832780507.4817\\
3.63419085477137	-35812497801.534\\
3.63429085727143	-35792272391.3659\\
3.63439085977149	-35771989685.4183\\
3.63449086227156	-35751764275.2502\\
3.63459086477162	-35731481569.3026\\
3.63469086727168	-35711256159.1344\\
3.63479086977174	-35691030748.9663\\
3.63489087227181	-35670805338.7982\\
3.63499087477187	-35650579928.6301\\
3.63509087727193	-35630354518.462\\
3.63519087977199	-35610129108.2938\\
3.63529088227206	-35589960993.9052\\
3.63539088477212	-35569735583.7371\\
3.63549088727218	-35549567469.3485\\
3.63559088977224	-35529399354.9599\\
3.63569089227231	-35509173944.7918\\
3.63579089477237	-35489005830.4032\\
3.63589089727243	-35468837716.0146\\
3.63599089977249	-35448726897.4055\\
3.63609090227256	-35428558783.0169\\
3.63619090477262	-35408390668.6283\\
3.63629090727268	-35388279850.0192\\
3.63639090977274	-35368111735.6306\\
3.63649091227281	-35348000917.0215\\
3.63659091477287	-35327890098.4124\\
3.63669091727293	-35307779279.8033\\
3.63679091977299	-35287668461.1942\\
3.63689092227306	-35267557642.5851\\
3.63699092477312	-35247446823.976\\
3.63709092727318	-35227393301.1465\\
3.63719092977324	-35207282482.5374\\
3.63729093227331	-35187228959.7078\\
3.63739093477337	-35167118141.0987\\
3.63749093727343	-35147064618.2691\\
3.63759093977349	-35127011095.4395\\
3.63769094227356	-35106957572.61\\
3.63779094477362	-35086904049.7804\\
3.63789094727368	-35066850526.9508\\
3.63799094977374	-35046854299.9007\\
3.63809095227381	-35026800777.0712\\
3.63819095477387	-35006804550.0211\\
3.63829095727393	-34986751027.1915\\
3.63839095977399	-34966754800.1414\\
3.63849096227406	-34946758573.0914\\
3.63859096477412	-34926762346.0413\\
3.63869096727418	-34906766118.9912\\
3.63879096977424	-34886769891.9412\\
3.63889097227431	-34866773664.8911\\
3.63899097477437	-34846834733.6206\\
3.63909097727443	-34826838506.5705\\
3.63919097977449	-34806899575.2999\\
3.63929098227456	-34786960644.0294\\
3.63939098477462	-34766964416.9793\\
3.63949098727468	-34747025485.7088\\
3.63959098977474	-34727086554.4382\\
3.63969099227481	-34707204918.9472\\
3.63979099477487	-34687265987.6766\\
3.63989099727493	-34667327056.4061\\
3.63999099977499	-34647445420.915\\
3.64009100227506	-34627506489.6445\\
3.64019100477512	-34607624854.1534\\
3.64029100727518	-34587743218.6624\\
3.64039100977524	-34567861583.1714\\
3.64049101227531	-34547979947.6803\\
3.64059101477537	-34528098312.1893\\
3.64069101727543	-34508216676.6982\\
3.64079101977549	-34488335041.2072\\
3.64089102227556	-34468510701.4957\\
3.64099102477562	-34448629066.0046\\
3.64109102727568	-34428804726.2931\\
3.64119102977574	-34408980386.5816\\
3.64129103227581	-34389156046.8701\\
3.64139103477587	-34369331707.1585\\
3.64149103727593	-34349507367.447\\
3.64159103977599	-34329683027.7355\\
3.64169104227606	-34309858688.024\\
3.64179104477612	-34290091644.0919\\
3.64189104727618	-34270267304.3804\\
3.64199104977624	-34250500260.4484\\
3.64209105227631	-34230733216.5164\\
3.64219105477637	-34210966172.5844\\
3.64229105727643	-34191141832.8728\\
3.64239105977649	-34171432084.7203\\
3.64249106227656	-34151665040.7883\\
3.64259106477662	-34131897996.8563\\
3.64269106727668	-34112130952.9243\\
3.64279106977674	-34092421204.7718\\
3.64289107227681	-34072654160.8398\\
3.64299107477687	-34052944412.6873\\
3.64309107727693	-34033234664.5348\\
3.64319107977699	-34013524916.3823\\
3.64329108227706	-33993815168.2298\\
3.64339108477712	-33974105420.0773\\
3.64349108727718	-33954395671.9248\\
3.64359108977724	-33934743219.5518\\
3.64369109227731	-33915033471.3993\\
3.64379109477737	-33895381019.0263\\
3.64389109727743	-33875671270.8738\\
3.64399109977749	-33856018818.5008\\
3.64409110227756	-33836366366.1278\\
3.64419110477762	-33816713913.7549\\
3.64429110727768	-33797061461.3819\\
3.64439110977774	-33777409009.0089\\
3.64449111227781	-33757813852.4154\\
3.64459111477787	-33738161400.0424\\
3.64469111727793	-33718566243.4489\\
3.64479111977799	-33698913791.076\\
3.64489112227806	-33679318634.4825\\
3.64499112477812	-33659723477.889\\
3.64509112727818	-33640128321.2955\\
3.64519112977824	-33620533164.7021\\
3.64529113227831	-33600938008.1086\\
3.64539113477837	-33581400147.2946\\
3.64549113727843	-33561804990.7012\\
3.64559113977849	-33542209834.1077\\
3.64569114227856	-33522671973.2937\\
3.64579114477862	-33503134112.4798\\
3.64589114727868	-33483596251.6658\\
3.64599114977874	-33464058390.8518\\
3.64609115227881	-33444520530.0379\\
3.64619115477887	-33424982669.2239\\
3.64629115727893	-33405444808.41\\
3.64639115977899	-33385906947.596\\
3.64649116227906	-33366426382.5615\\
3.64659116477912	-33346888521.7476\\
3.64669116727918	-33327407956.7131\\
3.64679116977924	-33307927391.6787\\
3.64689117227931	-33288446826.6442\\
3.64699117477937	-33268966261.6098\\
3.64709117727943	-33249485696.5753\\
3.64719117977949	-33230005131.5409\\
3.64729118227956	-33210581862.286\\
3.64739118477962	-33191101297.2515\\
3.64749118727968	-33171678027.9966\\
3.64759118977974	-33152197462.9621\\
3.64769119227981	-33132774193.7072\\
3.64779119477987	-33113350924.4523\\
3.64789119727993	-33093927655.1973\\
3.64799119977999	-33074504385.9424\\
3.64809120228006	-33055081116.6875\\
3.64819120478012	-33035657847.4325\\
3.64829120728018	-33016291873.9571\\
3.64839120978024	-32996868604.7022\\
3.64849121228031	-32977502631.2267\\
3.64859121478037	-32958136657.7513\\
3.64869121728043	-32938770684.2759\\
3.64879121978049	-32919347415.021\\
3.64889122228056	-32900038737.3251\\
3.64899122478062	-32880672763.8496\\
3.64909122728068	-32861306790.3742\\
3.64919122978074	-32841940816.8988\\
3.64929123228081	-32822632139.2029\\
3.64939123478087	-32803266165.7275\\
3.64949123728093	-32783957488.0315\\
3.64959123978099	-32764648810.3356\\
3.64969124228106	-32745340132.6397\\
3.64979124478112	-32726031454.9438\\
3.64989124728118	-32706722777.2479\\
3.64999124978124	-32687414099.552\\
3.65009125228131	-32668162717.6356\\
3.65019125478137	-32648854039.9397\\
3.65029125728143	-32629602658.0233\\
3.65039125978149	-32610293980.3274\\
3.65049126228156	-32591042598.411\\
3.65059126478162	-32571791216.4946\\
3.65069126728168	-32552539834.5782\\
3.65079126978174	-32533288452.6618\\
3.65089127228181	-32514037070.7454\\
3.65099127478187	-32494785688.829\\
3.65109127728193	-32475591602.6921\\
3.65119127978199	-32456340220.7757\\
3.65129128228206	-32437146134.6389\\
3.65139128478212	-32417952048.502\\
3.65149128728218	-32398757962.3651\\
3.65159128978224	-32379506580.4487\\
3.65169129228231	-32360312494.3118\\
3.65179129478237	-32341175703.9544\\
3.65189129728243	-32321981617.8176\\
3.65199129978249	-32302787531.6807\\
3.65209130228256	-32283650741.3233\\
3.65219130478262	-32264456655.1864\\
3.65229130728268	-32245319864.8291\\
3.65239130978274	-32226183074.4717\\
3.65249131228281	-32207046284.1143\\
3.65259131478287	-32187909493.757\\
3.65269131728293	-32168772703.3996\\
3.65279131978299	-32149635913.0422\\
3.65289132228306	-32130499122.6848\\
3.65299132478312	-32111419628.107\\
3.65309132728318	-32092282837.7496\\
3.65319132978324	-32073203343.1718\\
3.65329133228331	-32054123848.5939\\
3.65339133478337	-32035044354.016\\
3.65349133728343	-32015964859.4382\\
3.65359133978349	-31996885364.8603\\
3.65369134228356	-31977805870.2825\\
3.65379134478362	-31958726375.7046\\
3.65389134728368	-31939704176.9063\\
3.65399134978374	-31920624682.3284\\
3.65409135228381	-31901602483.5301\\
3.65419135478387	-31882580284.7317\\
3.65429135728393	-31863500790.1539\\
3.65439135978399	-31844478591.3555\\
3.65449136228406	-31825456392.5572\\
3.65459136478412	-31806491489.5384\\
3.65469136728418	-31787469290.74\\
3.65479136978424	-31768447091.9417\\
3.65489137228431	-31749482188.9228\\
3.65499137478437	-31730459990.1245\\
3.65509137728443	-31711495087.1057\\
3.65519137978449	-31692530184.0868\\
3.65529138228456	-31673565281.068\\
3.65539138478462	-31654600378.0492\\
3.65549138728468	-31635635475.0304\\
3.65559138978474	-31616670572.0115\\
3.65569139228481	-31597705668.9927\\
3.65579139478487	-31578798061.7534\\
3.65589139728493	-31559833158.7345\\
3.65599139978499	-31540925551.4952\\
3.65609140228506	-31522017944.2559\\
3.65619140478512	-31503053041.2371\\
3.65629140728518	-31484145433.9978\\
3.65639140978524	-31465295122.538\\
3.65649141228531	-31446387515.2986\\
3.65659141478537	-31427479908.0593\\
3.65669141728543	-31408572300.82\\
3.65679141978549	-31389721989.3602\\
3.65689142228556	-31370814382.1209\\
3.65699142478562	-31351964070.6611\\
3.65709142728568	-31333113759.2013\\
3.65719142978574	-31314263447.7415\\
3.65729143228581	-31295413136.2817\\
3.65739143478587	-31276562824.8219\\
3.65749143728593	-31257712513.3621\\
3.65759143978599	-31238919497.6818\\
3.65769144228606	-31220069186.222\\
3.65779144478612	-31201276170.5417\\
3.65789144728618	-31182425859.0819\\
3.65799144978624	-31163632843.4016\\
3.65809145228631	-31144839827.7213\\
3.65819145478637	-31126046812.041\\
3.65829145728643	-31107253796.3607\\
3.65839145978649	-31088460780.6804\\
3.65849146228656	-31069725060.7796\\
3.65859146478662	-31050932045.0993\\
3.65869146728668	-31032196325.1986\\
3.65879146978674	-31013403309.5183\\
3.65889147228681	-30994667589.6175\\
3.65899147478687	-30975931869.7167\\
3.65909147728693	-30957196149.8159\\
3.65919147978699	-30938460429.9152\\
3.65929148228706	-30919724710.0144\\
3.65939148478712	-30900988990.1136\\
3.65949148728718	-30882310565.9923\\
3.65959148978724	-30863574846.0916\\
3.65969149228731	-30844896421.9703\\
3.65979149478737	-30826160702.0695\\
3.65989149728743	-30807482277.9483\\
3.65999149978749	-30788803853.827\\
3.66009150228756	-30770125429.7057\\
3.66019150478762	-30751447005.5845\\
3.66029150728768	-30732825877.2427\\
3.66039150978774	-30714147453.1214\\
3.66049151228781	-30695469029.0002\\
3.66059151478787	-30676847900.6584\\
3.66069151728793	-30658226772.3167\\
3.66079151978799	-30639548348.1954\\
3.66089152228806	-30620927219.8537\\
3.66099152478812	-30602306091.5119\\
3.66109152728818	-30583684963.1702\\
3.66119152978824	-30565063834.8284\\
3.66129153228831	-30546500002.2662\\
3.66139153478837	-30527878873.9244\\
3.66149153728843	-30509315041.3622\\
3.66159153978849	-30490693913.0204\\
3.66169154228856	-30472130080.4582\\
3.66179154478862	-30453566247.896\\
3.66189154728868	-30435002415.3337\\
3.66199154978874	-30416438582.7715\\
3.66209155228881	-30397874750.2092\\
3.66219155478887	-30379310917.647\\
3.66229155728893	-30360747085.0848\\
3.66239155978899	-30342240548.302\\
3.66249156228906	-30323734011.5193\\
3.66259156478912	-30305170178.9571\\
3.66269156728918	-30286663642.1743\\
3.66279156978924	-30268157105.3916\\
3.66289157228931	-30249650568.6089\\
3.66299157478937	-30231144031.8262\\
3.66309157728943	-30212637495.0434\\
3.66319157978949	-30194130958.2607\\
3.66329158228956	-30175681717.2575\\
3.66339158478962	-30157175180.4748\\
3.66349158728968	-30138725939.4716\\
3.66359158978974	-30120276698.4683\\
3.66369159228981	-30101827457.4651\\
3.66379159478987	-30083378216.4619\\
3.66389159728993	-30064928975.4587\\
3.66399159978999	-30046479734.4555\\
3.66409160229006	-30028030493.4523\\
3.66419160479012	-30009581252.4491\\
3.66429160729018	-29991189307.2254\\
3.66439160979024	-29972740066.2222\\
3.66449161229031	-29954348120.9985\\
3.66459161479037	-29935956175.7748\\
3.66469161729043	-29917564230.5511\\
3.66479161979049	-29899172285.3274\\
3.66489162229056	-29880780340.1037\\
3.66499162479062	-29862388394.88\\
3.66509162729068	-29844053745.4358\\
3.66519162979074	-29825661800.2121\\
3.66529163229081	-29807327150.7679\\
3.66539163479087	-29788935205.5442\\
3.66549163729093	-29770600556.1\\
3.66559163979099	-29752265906.6558\\
3.66569164229106	-29733931257.2116\\
3.66579164479112	-29715596607.7675\\
3.66589164729118	-29697261958.3233\\
3.66599164979124	-29678927308.8791\\
3.66609165229131	-29660649955.2144\\
3.66619165479137	-29642315305.7702\\
3.66629165729143	-29624037952.1055\\
3.66639165979149	-29605760598.4409\\
3.66649166229156	-29587425948.9967\\
3.66659166479162	-29569148595.332\\
3.66669166729168	-29550871241.6673\\
3.66679166979174	-29532593888.0027\\
3.66689167229181	-29514373830.1175\\
3.66699167479187	-29496096476.4528\\
3.66709167729193	-29477819122.7882\\
3.667191679792	-29459599064.903\\
3.66729168229206	-29441379007.0178\\
3.66739168479212	-29423101653.3532\\
3.66749168729218	-29404881595.468\\
3.66759168979224	-29386661537.5828\\
3.66769169229231	-29368441479.6977\\
3.66779169479237	-29350278717.592\\
3.66789169729243	-29332058659.7069\\
3.66799169979249	-29313838601.8217\\
3.66809170229256	-29295675839.7161\\
3.66819170479262	-29277455781.8309\\
3.66829170729268	-29259293019.7253\\
3.66839170979274	-29241130257.6196\\
3.66849171229281	-29222967495.514\\
3.66859171479287	-29204804733.4083\\
3.66869171729293	-29186641971.3027\\
3.66879171979299	-29168479209.197\\
3.66889172229306	-29150373742.8709\\
3.66899172479312	-29132210980.7652\\
3.66909172729318	-29114105514.4391\\
3.66919172979325	-29095942752.3335\\
3.66929173229331	-29077837286.0073\\
3.66939173479337	-29059731819.6812\\
3.66949173729343	-29041626353.3551\\
3.66959173979349	-29023520887.0289\\
3.66969174229356	-29005415420.7028\\
3.66979174479362	-28987367250.1562\\
3.66989174729368	-28969261783.83\\
3.66999174979374	-28951156317.5039\\
3.67009175229381	-28933108146.9573\\
3.67019175479387	-28915059976.4107\\
3.67029175729393	-28897011805.864\\
3.67039175979399	-28878963635.3174\\
3.67049176229406	-28860915464.7708\\
3.67059176479412	-28842867294.2242\\
3.67069176729418	-28824819123.6776\\
3.67079176979424	-28806770953.1309\\
3.67089177229431	-28788780078.3638\\
3.67099177479437	-28770731907.8172\\
3.67109177729443	-28752741033.0501\\
3.6711917797945	-28734750158.283\\
3.67129178229456	-28716759283.5159\\
3.67139178479462	-28698768408.7488\\
3.67149178729468	-28680777533.9817\\
3.67159178979474	-28662786659.2146\\
3.67169179229481	-28644795784.4475\\
3.67179179479487	-28626862205.4599\\
3.67189179729493	-28608871330.6928\\
3.67199179979499	-28590937751.7052\\
3.67209180229506	-28573004172.7176\\
3.67219180479512	-28555070593.73\\
3.67229180729518	-28537137014.7424\\
3.67239180979524	-28519203435.7548\\
3.67249181229531	-28501269856.7672\\
3.67259181479537	-28483336277.7796\\
3.67269181729543	-28465402698.792\\
3.67279181979549	-28447526415.5839\\
3.67289182229556	-28429592836.5963\\
3.67299182479562	-28411716553.3882\\
3.67309182729568	-28393840270.1802\\
3.67319182979575	-28375963986.9721\\
3.67329183229581	-28358087703.764\\
3.67339183479587	-28340211420.5559\\
3.67349183729593	-28322335137.3478\\
3.673591839796	-28304458854.1397\\
3.67369184229606	-28286639866.7112\\
3.67379184479612	-28268763583.5031\\
3.67389184729618	-28250944596.0745\\
3.67399184979624	-28233125608.646\\
3.67409185229631	-28215306621.2174\\
3.67419185479637	-28197487633.7888\\
3.67429185729643	-28179668646.3603\\
3.67439185979649	-28161849658.9317\\
3.67449186229656	-28144030671.5031\\
3.67459186479662	-28126211684.0745\\
3.67469186729668	-28108449992.4255\\
3.67479186979674	-28090688300.7764\\
3.67489187229681	-28072869313.3479\\
3.67499187479687	-28055107621.6988\\
3.67509187729693	-28037345930.0498\\
3.675191879797	-28019584238.4007\\
3.67529188229706	-28001822546.7516\\
3.67539188479712	-27984060855.1026\\
3.67549188729718	-27966356459.233\\
3.67559188979725	-27948594767.584\\
3.67569189229731	-27930890371.7145\\
3.67579189479737	-27913128680.0654\\
3.67589189729743	-27895424284.1959\\
3.67599189979749	-27877719888.3263\\
3.67609190229756	-27860015492.4568\\
3.67619190479762	-27842311096.5872\\
3.67629190729768	-27824606700.7177\\
3.67639190979774	-27806902304.8481\\
3.67649191229781	-27789255204.7581\\
3.67659191479787	-27771550808.8886\\
3.67669191729793	-27753903708.7985\\
3.67679191979799	-27736199312.929\\
3.67689192229806	-27718552212.839\\
3.67699192479812	-27700905112.7489\\
3.67709192729818	-27683258012.6589\\
3.67719192979825	-27665610912.5689\\
3.67729193229831	-27647963812.4788\\
3.67739193479837	-27630374008.1683\\
3.67749193729843	-27612726908.0783\\
3.6775919397985	-27595137103.7678\\
3.67769194229856	-27577490003.6778\\
3.67779194479862	-27559900199.3672\\
3.67789194729868	-27542310395.0567\\
3.67799194979874	-27524720590.7462\\
3.67809195229881	-27507130786.4357\\
3.67819195479887	-27489540982.1252\\
3.67829195729893	-27471951177.8147\\
3.67839195979899	-27454418669.2837\\
3.67849196229906	-27436828864.9731\\
3.67859196479912	-27419296356.4421\\
3.67869196729918	-27401706552.1316\\
3.67879196979924	-27384174043.6006\\
3.67889197229931	-27366641535.0696\\
3.67899197479937	-27349109026.5386\\
3.67909197729943	-27331576518.0076\\
3.6791919797995	-27314044009.4766\\
3.67929198229956	-27296568796.7251\\
3.67939198479962	-27279036288.1941\\
3.67949198729968	-27261561075.4426\\
3.67959198979975	-27244028566.9116\\
3.67969199229981	-27226553354.1601\\
3.67979199479987	-27209078141.4086\\
3.67989199729993	-27191602928.6571\\
3.6799919998	-27174127715.9057\\
3.68009200230006	-27156652503.1542\\
3.68019200480012	-27139177290.4027\\
3.68029200730018	-27121759373.4307\\
3.68039200980024	-27104284160.6792\\
3.68049201230031	-27086866243.7072\\
3.68059201480037	-27069391030.9557\\
3.68069201730043	-27051973113.9838\\
3.68079201980049	-27034555197.0118\\
3.68089202230056	-27017137280.0398\\
3.68099202480062	-26999719363.0678\\
3.68109202730068	-26982301446.0959\\
3.68119202980075	-26964940824.9034\\
3.68129203230081	-26947522907.9314\\
3.68139203480087	-26930162286.739\\
3.68149203730093	-26912744369.767\\
3.681592039801	-26895383748.5745\\
3.68169204230106	-26878023127.382\\
3.68179204480112	-26860662506.1896\\
3.68189204730118	-26843301884.9971\\
3.68199204980125	-26825941263.8047\\
3.68209205230131	-26808580642.6122\\
3.68219205480137	-26791277317.1992\\
3.68229205730143	-26773916696.0068\\
3.68239205980149	-26756613370.5938\\
3.68249206230156	-26739252749.4014\\
3.68259206480162	-26721949423.9884\\
3.68269206730168	-26704646098.5755\\
3.68279206980174	-26687342773.1625\\
3.68289207230181	-26670039447.7496\\
3.68299207480187	-26652736122.3366\\
3.68309207730193	-26635490092.7032\\
3.683192079802	-26618186767.2902\\
3.68329208230206	-26600940737.6568\\
3.68339208480212	-26583637412.2438\\
3.68349208730218	-26566391382.6104\\
3.68359208980225	-26549145352.977\\
3.68369209230231	-26531899323.3435\\
3.68379209480237	-26514653293.7101\\
3.68389209730243	-26497407264.0766\\
3.6839920998025	-26480161234.4432\\
3.68409210230256	-26462915204.8098\\
3.68419210480262	-26445726470.9558\\
3.68429210730268	-26428480441.3224\\
3.68439210980275	-26411291707.4685\\
3.68449211230281	-26394102973.6146\\
3.68459211480287	-26376914239.7606\\
3.68469211730293	-26359725505.9067\\
3.68479211980299	-26342536772.0528\\
3.68489212230306	-26325348038.1989\\
3.68499212480312	-26308159304.3449\\
3.68509212730318	-26291027866.2705\\
3.68519212980325	-26273839132.4166\\
3.68529213230331	-26256707694.3422\\
3.68539213480337	-26239518960.4883\\
3.68549213730343	-26222387522.4138\\
3.6855921398035	-26205256084.3394\\
3.68569214230356	-26188124646.265\\
3.68579214480362	-26170993208.1906\\
3.68589214730368	-26153919065.8957\\
3.68599214980375	-26136787627.8213\\
3.68609215230381	-26119656189.7469\\
3.68619215480387	-26102582047.452\\
3.68629215730393	-26085507905.1571\\
3.686392159804	-26068376467.0827\\
3.68649216230406	-26051302324.7878\\
3.68659216480412	-26034228182.4929\\
3.68669216730418	-26017154040.198\\
3.68679216980424	-26000079897.9031\\
3.68689217230431	-25983063051.3877\\
3.68699217480437	-25965988909.0928\\
3.68709217730443	-25948972062.5774\\
3.6871921798045	-25931897920.2825\\
3.68729218230456	-25914881073.7671\\
3.68739218480462	-25897864227.2518\\
3.68749218730468	-25880847380.7364\\
3.68759218980475	-25863830534.221\\
3.68769219230481	-25846813687.7056\\
3.68779219480487	-25829796841.1902\\
3.68789219730493	-25812779994.6748\\
3.687992199805	-25795820443.9389\\
3.68809220230506	-25778803597.4236\\
3.68819220480512	-25761844046.6877\\
3.68829220730518	-25744884495.9518\\
3.68839220980525	-25727867649.4364\\
3.68849221230531	-25710908098.7006\\
3.68859221480537	-25693948547.9647\\
3.68869221730543	-25677046293.0083\\
3.68879221980549	-25660086742.2725\\
3.68889222230556	-25643127191.5366\\
3.68899222480562	-25626224936.5802\\
3.68909222730568	-25609265385.8444\\
3.68919222980575	-25592363130.888\\
3.68929223230581	-25575460875.9316\\
3.68939223480587	-25558558620.9753\\
3.68949223730593	-25541656366.0189\\
3.689592239806	-25524754111.0626\\
3.68969224230606	-25507851856.1062\\
3.68979224480612	-25490949601.1498\\
3.68989224730618	-25474104641.973\\
3.68999224980625	-25457202387.0166\\
3.69009225230631	-25440357427.8398\\
3.69019225480637	-25423455172.8834\\
3.69029225730643	-25406610213.7066\\
3.6903922598065	-25389765254.5297\\
3.69049226230656	-25372920295.3529\\
3.69059226480662	-25356075336.176\\
3.69069226730668	-25339287672.7787\\
3.69079226980675	-25322442713.6019\\
3.69089227230681	-25305597754.425\\
3.69099227480687	-25288810091.0277\\
3.69109227730693	-25272022427.6304\\
3.691192279807	-25255177468.4535\\
3.69129228230706	-25238389805.0562\\
3.69139228480712	-25221602141.6588\\
3.69149228730718	-25204814478.2615\\
3.69159228980725	-25188084110.6437\\
3.69169229230731	-25171296447.2464\\
3.69179229480737	-25154508783.849\\
3.69189229730743	-25137778416.2312\\
3.6919922998075	-25120990752.8339\\
3.69209230230756	-25104260385.216\\
3.69219230480762	-25087530017.5982\\
3.69229230730768	-25070799649.9804\\
3.69239230980775	-25054069282.3626\\
3.69249231230781	-25037338914.7448\\
3.69259231480787	-25020608547.1269\\
3.69269231730793	-25003878179.5091\\
3.692792319808	-24987205107.6708\\
3.69289232230806	-24970474740.053\\
3.69299232480812	-24953801668.2147\\
3.69309232730818	-24937128596.3764\\
3.69319232980825	-24920455524.5381\\
3.69329233230831	-24903782452.6998\\
3.69339233480837	-24887109380.8615\\
3.69349233730843	-24870436309.0232\\
3.6935923398085	-24853763237.1848\\
3.69369234230856	-24837090165.3465\\
3.69379234480862	-24820474389.2878\\
3.69389234730868	-24803801317.4494\\
3.69399234980875	-24787185541.3906\\
3.69409235230881	-24770569765.3319\\
3.69419235480887	-24753953989.2731\\
3.69429235730893	-24737338213.2143\\
3.694392359809	-24720722437.1555\\
3.69449236230906	-24704106661.0967\\
3.69459236480912	-24687490885.0379\\
3.69469236730918	-24670932404.7586\\
3.69479236980925	-24654316628.6998\\
3.69489237230931	-24637758148.4205\\
3.69499237480937	-24621199668.1413\\
3.69509237730943	-24604583892.0825\\
3.6951923798095	-24588025411.8032\\
3.69529238230956	-24571466931.5239\\
3.69539238480962	-24554965747.0241\\
3.69549238730968	-24538407266.7448\\
3.69559238980975	-24521848786.4656\\
3.69569239230981	-24505347601.9658\\
3.69579239480987	-24488789121.6865\\
3.69589239730993	-24472287937.1867\\
3.69599239981	-24455729456.9075\\
3.69609240231006	-24439228272.4077\\
3.69619240481012	-24422727087.9079\\
3.69629240731018	-24406225903.4082\\
3.69639240981025	-24389782014.6879\\
3.69649241231031	-24373280830.1881\\
3.69659241481037	-24356779645.6884\\
3.69669241731043	-24340335756.9681\\
3.6967924198105	-24323834572.4684\\
3.69689242231056	-24307390683.7481\\
3.69699242481062	-24290946795.0278\\
3.69709242731068	-24274502906.3076\\
3.69719242981075	-24258059017.5873\\
3.69729243231081	-24241615128.8671\\
3.69739243481087	-24225171240.1468\\
3.69749243731093	-24208727351.4266\\
3.697592439811	-24192340758.4858\\
3.69769244231106	-24175896869.7656\\
3.69779244481112	-24159510276.8248\\
3.69789244731118	-24143123683.8841\\
3.69799244981125	-24126679795.1638\\
3.69809245231131	-24110293202.2231\\
3.69819245481137	-24093906609.2824\\
3.69829245731143	-24077520016.3416\\
3.6983924598115	-24061190719.1804\\
3.69849246231156	-24044804126.2396\\
3.69859246481162	-24028474829.0784\\
3.69869246731168	-24012088236.1377\\
3.69879246981175	-23995758938.9764\\
3.69889247231181	-23979429641.8152\\
3.69899247481187	-23963043048.8745\\
3.69909247731193	-23946713751.7132\\
3.699192479812	-23930384454.552\\
3.69929248231206	-23914112453.1703\\
3.69939248481212	-23897783156.0091\\
3.69949248731218	-23881453858.8478\\
3.69959248981225	-23865181857.4661\\
3.69969249231231	-23848852560.3049\\
3.69979249481237	-23832580558.9232\\
3.69989249731243	-23816308557.5415\\
3.6999924998125	-23800036556.1598\\
3.70009250231256	-23783764554.778\\
3.70019250481262	-23767492553.3963\\
3.70029250731268	-23751220552.0146\\
3.70039250981275	-23734948550.6329\\
3.70049251231281	-23718733845.0307\\
3.70059251481287	-23702461843.649\\
3.70069251731293	-23686247138.0468\\
3.700792519813	-23670032432.4446\\
3.70089252231306	-23653817726.8424\\
3.70099252481312	-23637545725.4607\\
3.70109252731318	-23621331019.8584\\
3.70119252981325	-23605173610.0358\\
3.70129253231331	-23588958904.4336\\
3.70139253481337	-23572744198.8314\\
3.70149253731343	-23556586789.0087\\
3.7015925398135	-23540372083.4065\\
3.70169254231356	-23524214673.5838\\
3.70179254481362	-23508057263.7611\\
3.70189254731368	-23491899853.9384\\
3.70199254981375	-23475742444.1157\\
3.70209255231381	-23459585034.293\\
3.70219255481387	-23443427624.4703\\
3.70229255731393	-23427270214.6476\\
3.702392559814	-23411112804.825\\
3.70249256231406	-23395012690.7818\\
3.70259256481412	-23378855280.9591\\
3.70269256731418	-23362755166.9159\\
3.70279256981425	-23346655052.8727\\
3.70289257231431	-23330554938.8296\\
3.70299257481437	-23314454824.7864\\
3.70309257731443	-23298354710.7432\\
3.7031925798145	-23282254596.7\\
3.70329258231456	-23266211778.4364\\
3.70339258481462	-23250111664.3932\\
3.70349258731468	-23234011550.35\\
3.70359258981475	-23217968732.0863\\
3.70369259231481	-23201925913.8227\\
3.70379259481487	-23185883095.559\\
3.70389259731493	-23169840277.2954\\
3.703992599815	-23153797459.0317\\
3.70409260231506	-23137754640.768\\
3.70419260481512	-23121711822.5044\\
3.70429260731518	-23105669004.2407\\
3.70439260981525	-23089683481.7566\\
3.70449261231531	-23073640663.4929\\
3.70459261481537	-23057655141.0087\\
3.70469261731543	-23041669618.5246\\
3.7047926198155	-23025684096.0404\\
3.70489262231556	-23009698573.5563\\
3.70499262481562	-22993713051.0721\\
3.70509262731568	-22977727528.588\\
3.70519262981575	-22961742006.1038\\
3.70529263231581	-22945756483.6197\\
3.70539263481587	-22929828256.9151\\
3.70549263731593	-22913900030.2104\\
3.705592639816	-22897914507.7263\\
3.70569264231606	-22881986281.0216\\
3.70579264481612	-22866058054.317\\
3.70589264731618	-22850129827.6124\\
3.70599264981625	-22834201600.9077\\
3.70609265231631	-22818273374.2031\\
3.70619265481637	-22802345147.4985\\
3.70629265731643	-22786474216.5733\\
3.7063926598165	-22770545989.8687\\
3.70649266231656	-22754675058.9436\\
3.70659266481662	-22738804128.0184\\
3.70669266731668	-22722933197.0933\\
3.70679266981675	-22707004970.3887\\
3.70689267231681	-22691191335.2431\\
3.70699267481687	-22675320404.3179\\
3.70709267731693	-22659449473.3928\\
3.707192679817	-22643578542.4677\\
3.70729268231706	-22627764907.3221\\
3.70739268481712	-22611893976.397\\
3.70749268731718	-22596080341.2514\\
3.70759268981725	-22580266706.1057\\
3.70769269231731	-22564395775.1806\\
3.70779269481737	-22548582140.035\\
3.70789269731743	-22532768504.8894\\
3.7079926998175	-22517012165.5233\\
3.70809270231756	-22501198530.3777\\
3.70819270481762	-22485384895.2321\\
3.70829270731768	-22469628555.866\\
3.70839270981775	-22453814920.7204\\
3.70849271231781	-22438058581.3543\\
3.70859271481787	-22422302241.9882\\
3.70869271731793	-22406545902.6221\\
3.708792719818	-22390789563.256\\
3.70889272231806	-22375033223.8899\\
3.70899272481812	-22359276884.5238\\
3.70909272731818	-22343520545.1577\\
3.70919272981825	-22327821501.5711\\
3.70929273231831	-22312065162.205\\
3.70939273481837	-22296366118.6184\\
3.70949273731843	-22280609779.2523\\
3.7095927398185	-22264910735.6657\\
3.70969274231856	-22249211692.0792\\
3.70979274481862	-22233512648.4926\\
3.70989274731868	-22217813604.906\\
3.70999274981875	-22202114561.3194\\
3.71009275231881	-22186472813.5123\\
3.71019275481887	-22170773769.9257\\
3.71029275731893	-22155132022.1187\\
3.710392759819	-22139432978.5321\\
3.71049276231906	-22123791230.725\\
3.71059276481912	-22108149482.9179\\
3.71069276731918	-22092507735.1109\\
3.71079276981925	-22076865987.3038\\
3.71089277231931	-22061224239.4967\\
3.71099277481937	-22045582491.6897\\
3.71109277731943	-22029998039.6621\\
3.7111927798195	-22014356291.855\\
3.71129278231956	-21998714544.048\\
3.71139278481962	-21983130092.0204\\
3.71149278731968	-21967545639.9928\\
3.71159278981975	-21951961187.9653\\
3.71169279231981	-21936376735.9377\\
3.71179279481987	-21920792283.9102\\
3.71189279731993	-21905207831.8826\\
3.71199279982	-21889623379.855\\
3.71209280232006	-21874096223.607\\
3.71219280482012	-21858511771.5794\\
3.71229280732018	-21842984615.3314\\
3.71239280982025	-21827400163.3038\\
3.71249281232031	-21811873007.0558\\
3.71259281482037	-21796345850.8078\\
3.71269281732043	-21780818694.5597\\
3.7127928198205	-21765291538.3117\\
3.71289282232056	-21749764382.0636\\
3.71299282482062	-21734294521.5951\\
3.71309282732068	-21718767365.347\\
3.71319282982075	-21703240209.099\\
3.71329283232081	-21687770348.6305\\
3.71339283482087	-21672300488.1619\\
3.71349283732093	-21656830627.6934\\
3.713592839821	-21641303471.4454\\
3.71369284232106	-21625833610.9768\\
3.71379284482112	-21610421046.2878\\
3.71389284732118	-21594951185.8193\\
3.71399284982125	-21579481325.3507\\
3.71409285232131	-21564068760.6617\\
3.71419285482137	-21548598900.1932\\
3.71429285732143	-21533186335.5042\\
3.7143928598215	-21517716475.0356\\
3.71449286232156	-21502303910.3466\\
3.71459286482162	-21486891345.6576\\
3.71469286732168	-21471478780.9686\\
3.71479286982175	-21456066216.2796\\
3.71489287232181	-21440710947.37\\
3.71499287482187	-21425298382.681\\
3.71509287732193	-21409885817.992\\
3.715192879822	-21394530549.0825\\
3.71529288232206	-21379175280.173\\
3.71539288482212	-21363762715.484\\
3.71549288732218	-21348407446.5745\\
3.71559288982225	-21333052177.665\\
3.71569289232231	-21317696908.7555\\
3.71579289482237	-21302341639.846\\
3.71589289732243	-21287043666.716\\
3.7159928998225	-21271688397.8065\\
3.71609290232256	-21256390424.6765\\
3.71619290482262	-21241035155.767\\
3.71629290732268	-21225737182.637\\
3.71639290982275	-21210439209.507\\
3.71649291232281	-21195083940.5975\\
3.71659291482287	-21179785967.4675\\
3.71669291732293	-21164487994.3375\\
3.716792919823	-21149247316.987\\
3.71689292232306	-21133949343.857\\
3.71699292482312	-21118651370.727\\
3.71709292732318	-21103410693.3765\\
3.71719292982325	-21088112720.2465\\
3.71729293232331	-21072872042.8961\\
3.71739293482337	-21057631365.5456\\
3.71749293732343	-21042390688.1951\\
3.7175929398235	-21027150010.8446\\
3.71769294232356	-21011909333.4941\\
3.71779294482362	-20996668656.1437\\
3.71789294732368	-20981427978.7932\\
3.71799294982375	-20966244597.2222\\
3.71809295232381	-20951003919.8717\\
3.71819295482387	-20935820538.3008\\
3.71829295732393	-20920637156.7298\\
3.718392959824	-20905453775.1588\\
3.71849296232406	-20890213097.8084\\
3.71859296482412	-20875029716.2374\\
3.71869296732418	-20859903630.4459\\
3.71879296982425	-20844720248.875\\
3.71889297232431	-20829536867.304\\
3.71899297482437	-20814410781.5125\\
3.71909297732443	-20799227399.9416\\
3.7191929798245	-20784101314.1501\\
3.71929298232456	-20768975228.3587\\
3.71939298482462	-20753791846.7877\\
3.71949298732468	-20738665760.9963\\
3.71959298982475	-20723539675.2048\\
3.71969299232481	-20708413589.4133\\
3.71979299482487	-20693344799.4014\\
3.71989299732493	-20678218713.61\\
3.719992999825	-20663149923.598\\
3.72009300232506	-20648023837.8066\\
3.72019300482512	-20632955047.7946\\
3.72029300732518	-20617886257.7827\\
3.72039300982525	-20602760171.9912\\
3.72049301232531	-20587691381.9793\\
3.72059301482537	-20572622591.9673\\
3.72069301732543	-20557611097.7349\\
3.7207930198255	-20542542307.723\\
3.72089302232556	-20527473517.711\\
3.72099302482562	-20512462023.4786\\
3.72109302732568	-20497393233.4667\\
3.72119302982575	-20482381739.2342\\
3.72129303232581	-20467370245.0018\\
3.72139303482587	-20452358750.7694\\
3.72149303732593	-20437347256.537\\
3.721593039826	-20422335762.3045\\
3.72169304232606	-20407324268.0721\\
3.72179304482612	-20392312773.8397\\
3.72189304732618	-20377358575.3868\\
3.72199304982625	-20362347081.1543\\
3.72209305232631	-20347392882.7014\\
3.72219305482637	-20332381388.469\\
3.72229305732643	-20317427190.0161\\
3.7223930598265	-20302472991.5632\\
3.72249306232656	-20287518793.1102\\
3.72259306482662	-20272564594.6573\\
3.72269306732668	-20257610396.2044\\
3.72279306982675	-20242713493.531\\
3.72289307232681	-20227759295.0781\\
3.72299307482687	-20212862392.4047\\
3.72309307732693	-20197908193.9518\\
3.723193079827	-20183011291.2784\\
3.72329308232706	-20168114388.605\\
3.72339308482712	-20153217485.9316\\
3.72349308732718	-20138320583.2582\\
3.72359308982725	-20123423680.5848\\
3.72369309232731	-20108526777.9114\\
3.72379309482737	-20093629875.238\\
3.72389309732743	-20078790268.3441\\
3.7239930998275	-20063893365.6707\\
3.72409310232756	-20049053758.7768\\
3.72419310482762	-20034214151.8829\\
3.72429310732768	-20019374544.989\\
3.72439310982775	-20004477642.3156\\
3.72449311232781	-19989638035.4217\\
3.72459311482787	-19974855724.3074\\
3.72469311732793	-19960016117.4135\\
3.724793119828	-19945176510.5196\\
3.72489312232806	-19930394199.4052\\
3.72499312482812	-19915554592.5113\\
3.72509312732818	-19900772281.3969\\
3.72519312982825	-19885989970.2826\\
3.72529313232831	-19871150363.3887\\
3.72539313482837	-19856368052.2743\\
3.72549313732843	-19841585741.1599\\
3.7255931398285	-19826860725.8251\\
3.72569314232856	-19812078414.7107\\
3.72579314482862	-19797296103.5963\\
3.72589314732868	-19782571088.2614\\
3.72599314982875	-19767788777.1471\\
3.72609315232881	-19753063761.8122\\
3.72619315482887	-19738338746.4773\\
3.72629315732893	-19723613731.1425\\
3.726393159829	-19708888715.8076\\
3.72649316232906	-19694163700.4728\\
3.72659316482912	-19679438685.1379\\
3.72669316732918	-19664713669.803\\
3.72679316982925	-19649988654.4682\\
3.72689317232931	-19635320934.9128\\
3.72699317482937	-19620595919.578\\
3.72709317732943	-19605928200.0226\\
3.7271931798295	-19591260480.4673\\
3.72729318232956	-19576592760.9119\\
3.72739318482962	-19561925041.3566\\
3.72749318732968	-19547257321.8012\\
3.72759318982975	-19532589602.2459\\
3.72769319232981	-19517921882.6905\\
3.72779319482987	-19503311458.9147\\
3.72789319732993	-19488643739.3593\\
3.72799319983	-19474033315.5835\\
3.72809320233006	-19459422891.8077\\
3.72819320483012	-19444755172.2523\\
3.72829320733018	-19430144748.4765\\
3.72839320983025	-19415534324.7006\\
3.72849321233031	-19400923900.9248\\
3.72859321483037	-19386370772.9285\\
3.72869321733043	-19371760349.1526\\
3.7287932198305	-19357149925.3768\\
3.72889322233056	-19342596797.3805\\
3.72899322483062	-19327986373.6047\\
3.72909322733068	-19313433245.6083\\
3.72919322983075	-19298880117.612\\
3.72929323233081	-19284326989.6157\\
3.72939323483087	-19269773861.6194\\
3.72949323733093	-19255220733.623\\
3.729593239831	-19240667605.6267\\
3.72969324233106	-19226114477.6304\\
3.72979324483112	-19211618645.4136\\
3.72989324733118	-19197065517.4173\\
3.72999324983125	-19182569685.2004\\
3.73009325233131	-19168073852.9836\\
3.73019325483137	-19153578020.7668\\
3.73029325733143	-19139024892.7705\\
3.7303932598315	-19124529060.5537\\
3.73049326233156	-19110090524.1164\\
3.73059326483162	-19095594691.8996\\
3.73069326733168	-19081098859.6828\\
3.73079326983175	-19066660323.2455\\
3.73089327233181	-19052164491.0287\\
3.73099327483187	-19037725954.5914\\
3.73109327733193	-19023230122.3746\\
3.731193279832	-19008791585.9373\\
3.73129328233206	-18994353049.5\\
3.73139328483212	-18979914513.0627\\
3.73149328733218	-18965475976.6254\\
3.73159328983225	-18951094735.9676\\
3.73169329233231	-18936656199.5303\\
3.73179329483237	-18922217663.093\\
3.73189329733243	-18907836422.4352\\
3.7319932998325	-18893455181.7774\\
3.73209330233256	-18879016645.3401\\
3.73219330483262	-18864635404.6824\\
3.73229330733268	-18850254164.0246\\
3.73239330983275	-18835872923.3668\\
3.73249331233281	-18821491682.709\\
3.73259331483287	-18807167737.8307\\
3.73269331733293	-18792786497.173\\
3.732793319833	-18778405256.5152\\
3.73289332233306	-18764081311.6369\\
3.73299332483312	-18749757366.7586\\
3.73309332733318	-18735376126.1008\\
3.73319332983325	-18721052181.2226\\
3.73329333233331	-18706728236.3443\\
3.73339333483337	-18692404291.466\\
3.73349333733343	-18678080346.5878\\
3.7335933398335	-18663813697.489\\
3.73369334233356	-18649489752.6107\\
3.73379334483362	-18635223103.512\\
3.73389334733368	-18620899158.6337\\
3.73399334983375	-18606632509.5349\\
3.73409335233381	-18592365860.4362\\
3.73419335483387	-18578041915.5579\\
3.73429335733393	-18563775266.4592\\
3.734393359834	-18549508617.3604\\
3.73449336233406	-18535299264.0412\\
3.73459336483412	-18521032614.9424\\
3.73469336733418	-18506765965.8436\\
3.73479336983425	-18492556612.5244\\
3.73489337233431	-18478289963.4256\\
3.73499337483437	-18464080610.1064\\
3.73509337733443	-18449871256.7872\\
3.7351933798345	-18435661903.4679\\
3.73529338233456	-18421452550.1487\\
3.73539338483462	-18407243196.8294\\
3.73549338733468	-18393033843.5102\\
3.73559338983475	-18378824490.1909\\
3.73569339233481	-18364672432.6512\\
3.73579339483487	-18350463079.332\\
3.73589339733493	-18336311021.7922\\
3.735993399835	-18322101668.473\\
3.73609340233506	-18307949610.9332\\
3.73619340483512	-18293797553.3935\\
3.73629340733518	-18279645495.8538\\
3.73639340983525	-18265493438.3141\\
3.73649341233531	-18251341380.7743\\
3.73659341483537	-18237189323.2346\\
3.73669341733543	-18223094561.4744\\
3.7367934198355	-18208942503.9346\\
3.73689342233556	-18194847742.1744\\
3.73699342483562	-18180752980.4142\\
3.73709342733568	-18166600922.8745\\
3.73719342983575	-18152506161.1143\\
3.73729343233581	-18138411399.354\\
3.73739343483587	-18124316637.5938\\
3.73749343733593	-18110279171.6131\\
3.737593439836	-18096184409.8529\\
3.73769344233606	-18082089648.0927\\
3.73779344483612	-18068052182.112\\
3.73789344733618	-18053957420.3518\\
3.73799344983625	-18039919954.371\\
3.73809345233631	-18025882488.3903\\
3.73819345483637	-18011845022.4096\\
3.73829345733643	-17997807556.4289\\
3.7383934598365	-17983770090.4482\\
3.73849346233656	-17969732624.4675\\
3.73859346483662	-17955695158.4868\\
3.73869346733668	-17941714988.2856\\
3.73879346983675	-17927677522.3049\\
3.73889347233681	-17913697352.1037\\
3.73899347483687	-17899717181.9025\\
3.73909347733693	-17885679715.9218\\
3.739193479837	-17871699545.7206\\
3.73929348233706	-17857719375.5194\\
3.73939348483712	-17843739205.3183\\
3.73949348733718	-17829816330.8966\\
3.73959348983725	-17815836160.6954\\
3.73969349233731	-17801855990.4942\\
3.73979349483737	-17787933116.0725\\
3.73989349733743	-17774010241.6508\\
3.7399934998375	-17760030071.4496\\
3.74009350233756	-17746107197.028\\
3.74019350483762	-17732184322.6063\\
3.74029350733768	-17718261448.1846\\
3.74039350983775	-17704338573.7629\\
3.74049351233781	-17690415699.3412\\
3.74059351483787	-17676550120.6991\\
3.74069351733793	-17662627246.2774\\
3.740793519838	-17648761667.6352\\
3.74089352233806	-17634838793.2136\\
3.74099352483812	-17620973214.5714\\
3.74109352733818	-17607107635.9292\\
3.74119352983825	-17593242057.2871\\
3.74129353233831	-17579376478.6449\\
3.74139353483837	-17565510900.0027\\
3.74149353733843	-17551645321.3606\\
3.7415935398385	-17537779742.7184\\
3.74169354233856	-17523971459.8557\\
3.74179354483862	-17510105881.2136\\
3.74189354733868	-17496297598.3509\\
3.74199354983875	-17482489315.4883\\
3.74209355233881	-17468623736.8461\\
3.74219355483887	-17454815453.9835\\
3.74229355733893	-17441007171.1208\\
3.742393559839	-17427256184.0377\\
3.74249356233906	-17413447901.175\\
3.74259356483912	-17399639618.3124\\
3.74269356733918	-17385831335.4497\\
3.74279356983925	-17372080348.3666\\
3.74289357233931	-17358329361.2834\\
3.74299357483937	-17344521078.4208\\
3.74309357733943	-17330770091.3376\\
3.7431935798395	-17317019104.2545\\
3.74329358233956	-17303268117.1713\\
3.74339358483962	-17289517130.0882\\
3.74349358733968	-17275823438.7846\\
3.74359358983975	-17262072451.7014\\
3.74369359233981	-17248321464.6183\\
3.74379359483987	-17234627773.3147\\
3.74389359733993	-17220876786.2315\\
3.74399359984	-17207183094.9279\\
3.74409360234006	-17193489403.6243\\
3.74419360484012	-17179795712.3207\\
3.74429360734018	-17166102021.017\\
3.74439360984025	-17152408329.7134\\
3.74449361234031	-17138714638.4098\\
3.74459361484037	-17125078242.8857\\
3.74469361734043	-17111384551.582\\
3.7447936198405	-17097748156.0579\\
3.74489362234056	-17084054464.7543\\
3.74499362484062	-17070418069.2302\\
3.74509362734068	-17056781673.7061\\
3.74519362984075	-17043145278.182\\
3.74529363234081	-17029508882.6578\\
3.74539363484087	-17015872487.1337\\
3.74549363734093	-17002236091.6096\\
3.745593639841	-16988599696.0855\\
3.74569364234106	-16975020596.3409\\
3.74579364484112	-16961384200.8168\\
3.74589364734118	-16947805101.0722\\
3.74599364984125	-16934226001.3276\\
3.74609365234131	-16920646901.583\\
3.74619365484137	-16907067801.8384\\
3.74629365734143	-16893488702.0938\\
3.7463936598415	-16879909602.3492\\
3.74649366234156	-16866330502.6046\\
3.74659366484162	-16852751402.86\\
3.74669366734168	-16839229598.8949\\
3.74679366984175	-16825650499.1503\\
3.74689367234181	-16812128695.1852\\
3.74699367484187	-16798606891.2201\\
3.74709367734193	-16785027791.4755\\
3.747193679842	-16771505987.5104\\
3.74729368234206	-16757984183.5453\\
3.74739368484212	-16744519675.3598\\
3.74749368734218	-16730997871.3947\\
3.74759368984225	-16717476067.4296\\
3.74769369234231	-16704011559.244\\
3.74779369484237	-16690489755.2789\\
3.74789369734243	-16677025247.0934\\
3.7479936998425	-16663503443.1283\\
3.74809370234256	-16650038934.9427\\
3.74819370484262	-16636574426.7571\\
3.74829370734268	-16623109918.5715\\
3.74839370984275	-16609645410.386\\
3.74849371234281	-16596238197.9799\\
3.74859371484287	-16582773689.7943\\
3.74869371734293	-16569309181.6088\\
3.748793719843	-16555901969.2027\\
3.74889372234306	-16542494756.7966\\
3.74899372484312	-16529030248.6111\\
3.74909372734318	-16515623036.205\\
3.74919372984325	-16502215823.7989\\
3.74929373234331	-16488808611.3929\\
3.74939373484337	-16475401398.9868\\
3.74949373734343	-16462051482.3603\\
3.7495937398435	-16448644269.9542\\
3.74969374234356	-16435237057.5482\\
3.74979374484362	-16421887140.9216\\
3.74989374734368	-16408479928.5155\\
3.74999374984375	-16395130011.889\\
3.75009375234381	-16381780095.2624\\
3.75019375484387	-16368430178.6359\\
3.75029375734393	-16355080262.0093\\
3.750393759844	-16341730345.3828\\
3.75049376234406	-16328380428.7563\\
3.75059376484412	-16315087807.9092\\
3.75069376734418	-16301737891.2827\\
3.75079376984425	-16288445270.4356\\
3.75089377234431	-16275095353.8091\\
3.75099377484437	-16261802732.9621\\
3.75109377734443	-16248510112.115\\
3.7511937798445	-16235217491.268\\
3.75129378234456	-16221924870.4209\\
3.75139378484462	-16208632249.5739\\
3.75149378734468	-16195339628.7269\\
3.75159378984475	-16182047007.8798\\
3.75169379234481	-16168811682.8123\\
3.75179379484487	-16155519061.9653\\
3.75189379734493	-16142283736.8978\\
3.751993799845	-16129048411.8302\\
3.75209380234506	-16115813086.7627\\
3.75219380484512	-16102577761.6952\\
3.75229380734518	-16089342436.6277\\
3.75239380984525	-16076107111.5602\\
3.75249381234531	-16062871786.4926\\
3.75259381484537	-16049636461.4251\\
3.75269381734543	-16036458432.1371\\
3.7527938198455	-16023223107.0696\\
3.75289382234556	-16010045077.7816\\
3.75299382484562	-15996867048.4936\\
3.75309382734568	-15983631723.426\\
3.75319382984575	-15970453694.138\\
3.75329383234581	-15957275664.85\\
3.75339383484587	-15944097635.562\\
3.75349383734593	-15930976902.0535\\
3.753593839846	-15917798872.7655\\
3.75369384234606	-15904620843.4775\\
3.75379384484612	-15891500109.969\\
3.75389384734618	-15878379376.4605\\
3.75399384984625	-15865201347.1725\\
3.75409385234631	-15852080613.664\\
3.75419385484637	-15838959880.1555\\
3.75429385734643	-15825839146.647\\
3.7543938598465	-15812718413.1385\\
3.75449386234656	-15799597679.63\\
3.75459386484662	-15786534241.901\\
3.75469386734668	-15773413508.3925\\
3.75479386984675	-15760350070.6636\\
3.75489387234681	-15747229337.1551\\
3.75499387484687	-15734165899.4261\\
3.75509387734693	-15721102461.6971\\
3.755193879847	-15708039023.9681\\
3.75529388234706	-15694975586.2391\\
3.75539388484712	-15681912148.5101\\
3.75549388734718	-15668848710.7812\\
3.75559388984725	-15655785273.0522\\
3.75569389234731	-15642779131.1027\\
3.75579389484737	-15629715693.3737\\
3.75589389734743	-15616709551.4243\\
3.7559938998475	-15603703409.4748\\
3.75609390234756	-15590639971.7458\\
3.75619390484762	-15577633829.7963\\
3.75629390734768	-15564627687.8469\\
3.75639390984775	-15551678841.6769\\
3.75649391234781	-15538672699.7274\\
3.75659391484787	-15525666557.778\\
3.75669391734793	-15512660415.8285\\
3.756793919848	-15499711569.6585\\
3.75689392234806	-15486762723.4886\\
3.75699392484812	-15473756581.5391\\
3.75709392734818	-15460807735.3692\\
3.75719392984825	-15447858889.1992\\
3.75729393234831	-15434910043.0292\\
3.75739393484837	-15421961196.8593\\
3.75749393734843	-15409012350.6893\\
3.7575939398485	-15396120800.2989\\
3.75769394234856	-15383171954.1289\\
3.75779394484862	-15370223107.959\\
3.75789394734868	-15357331557.5685\\
3.75799394984875	-15344440007.1781\\
3.75809395234881	-15331548456.7876\\
3.75819395484887	-15318599610.6177\\
3.75829395734893	-15305708060.2272\\
3.758393959849	-15292873805.6163\\
3.75849396234906	-15279982255.2259\\
3.75859396484912	-15267090704.8354\\
3.75869396734918	-15254199154.445\\
3.75879396984925	-15241364899.8341\\
3.75889397234931	-15228530645.2231\\
3.75899397484937	-15215639094.8327\\
3.75909397734943	-15202804840.2218\\
3.7591939798495	-15189970585.6108\\
3.75929398234956	-15177136330.9999\\
3.75939398484962	-15164302076.389\\
3.75949398734968	-15151467821.778\\
3.75959398984975	-15138633567.1671\\
3.75969399234981	-15125856608.3357\\
3.75979399484987	-15113022353.7248\\
3.75989399734993	-15100245394.8933\\
3.75999399985	-15087468436.0619\\
3.76009400235006	-15074634181.451\\
3.76019400485012	-15061857222.6196\\
3.76029400735018	-15049080263.7882\\
3.76039400985025	-15036303304.9567\\
3.76049401235031	-15023526346.1253\\
3.76059401485037	-15010806683.0734\\
3.76069401735043	-14998029724.242\\
3.7607940198505	-14985310061.1901\\
3.76089402235056	-14972533102.3587\\
3.76099402485062	-14959813439.3068\\
3.76109402735068	-14947093776.2549\\
3.76119402985075	-14934374113.203\\
3.76129403235081	-14921597154.3715\\
3.76139403485087	-14908934787.0992\\
3.76149403735093	-14896215124.0472\\
3.761594039851	-14883495460.9953\\
3.76169404235106	-14870775797.9434\\
3.76179404485112	-14858113430.671\\
3.76189404735118	-14845393767.6191\\
3.76199404985125	-14832731400.3468\\
3.76209405235131	-14820069033.0744\\
3.76219405485137	-14807406665.802\\
3.76229405735143	-14794744298.5296\\
3.7623940598515	-14782081931.2572\\
3.76249406235156	-14769419563.9848\\
3.76259406485162	-14756757196.7124\\
3.76269406735168	-14744094829.44\\
3.76279406985175	-14731489757.9471\\
3.76289407235181	-14718884686.4543\\
3.76299407485187	-14706222319.1819\\
3.76309407735193	-14693617247.689\\
3.763194079852	-14681012176.1961\\
3.76329408235206	-14668407104.7032\\
3.76339408485212	-14655802033.2104\\
3.76349408735218	-14643196961.7175\\
3.76359408985225	-14630591890.2246\\
3.76369409235231	-14618044114.5112\\
3.76379409485237	-14605439043.0184\\
3.76389409735243	-14592891267.305\\
3.7639940998525	-14580286195.8121\\
3.76409410235256	-14567738420.0987\\
3.76419410485262	-14555190644.3854\\
3.76429410735268	-14542642868.672\\
3.76439410985275	-14530095092.9587\\
3.76449411235281	-14517547317.2453\\
3.76459411485287	-14504999541.5319\\
3.76469411735293	-14492451765.8186\\
3.764794119853	-14479961285.8847\\
3.76489412235306	-14467413510.1713\\
3.76499412485312	-14454923030.2375\\
3.76509412735318	-14442432550.3036\\
3.76519412985325	-14429942070.3698\\
3.76529413235331	-14417451590.4359\\
3.76539413485337	-14404961110.5021\\
3.76549413735343	-14392470630.5682\\
3.7655941398535	-14379980150.6344\\
3.76569414235356	-14367489670.7005\\
3.76579414485362	-14355056486.5462\\
3.76589414735368	-14342566006.6123\\
3.76599414985375	-14330132822.458\\
3.76609415235381	-14317642342.5241\\
3.76619415485387	-14305209158.3698\\
3.76629415735393	-14292775974.2155\\
3.766394159854	-14280342790.0611\\
3.76649416235406	-14267909605.9068\\
3.76659416485412	-14255533717.532\\
3.76669416735418	-14243100533.3776\\
3.76679416985425	-14230667349.2233\\
3.76689417235431	-14218291460.8485\\
3.76699417485437	-14205858276.6941\\
3.76709417735443	-14193482388.3193\\
3.7671941798545	-14181106499.9445\\
3.76729418235456	-14168730611.5696\\
3.76739418485462	-14156354723.1948\\
3.76749418735468	-14143978834.82\\
3.76759418985475	-14131602946.4452\\
3.76769419235481	-14119227058.0703\\
3.76779419485487	-14106908465.475\\
3.76789419735493	-14094532577.1002\\
3.767994199855	-14082213984.5049\\
3.76809420235506	-14069895391.9096\\
3.76819420485512	-14057519503.5347\\
3.76829420735518	-14045200910.9394\\
3.76839420985525	-14032882318.3441\\
3.76849421235531	-14020563725.7488\\
3.76859421485537	-14008302428.933\\
3.76869421735543	-13995983836.3377\\
3.7687942198555	-13983665243.7424\\
3.76889422235556	-13971403946.9266\\
3.76899422485562	-13959085354.3313\\
3.76909422735568	-13946824057.5155\\
3.76919422985575	-13934562760.6997\\
3.76929423235581	-13922301463.8839\\
3.76939423485587	-13910040167.0681\\
3.76949423735593	-13897778870.2523\\
3.769594239856	-13885517573.4365\\
3.76969424235606	-13873256276.6207\\
3.76979424485612	-13860994979.8049\\
3.76989424735618	-13848790978.7686\\
3.76999424985625	-13836529681.9528\\
3.77009425235631	-13824325680.9165\\
3.77019425485637	-13812121679.8802\\
3.77029425735643	-13799917678.8439\\
3.7703942598565	-13787713677.8076\\
3.77049426235656	-13775509676.7714\\
3.77059426485662	-13763305675.7351\\
3.77069426735668	-13751101674.6988\\
3.77079426985675	-13738954969.442\\
3.77089427235681	-13726750968.4057\\
3.77099427485687	-13714604263.149\\
3.77109427735693	-13702400262.1127\\
3.771194279857	-13690253556.8559\\
3.77129428235706	-13678106851.5991\\
3.77139428485712	-13665960146.3423\\
3.77149428735718	-13653813441.0856\\
3.77159428985725	-13641666735.8288\\
3.77169429235731	-13629520030.572\\
3.77179429485737	-13617373325.3153\\
3.77189429735743	-13605283915.838\\
3.7719942998575	-13593137210.5812\\
3.77209430235756	-13581047801.104\\
3.77219430485762	-13568958391.6267\\
3.77229430735768	-13556868982.1494\\
3.77239430985775	-13544722276.8927\\
3.77249431235781	-13532690163.1949\\
3.77259431485787	-13520600753.7177\\
3.77269431735793	-13508511344.2404\\
3.772794319858	-13496421934.7631\\
3.77289432235806	-13484389821.0654\\
3.77299432485812	-13472300411.5881\\
3.77309432735818	-13460268297.8904\\
3.77319432985825	-13448178888.4131\\
3.77329433235831	-13436146774.7154\\
3.77339433485837	-13424114661.0176\\
3.77349433735843	-13412082547.3199\\
3.7735943398585	-13400050433.6221\\
3.77369434235856	-13388018319.9244\\
3.77379434485862	-13376043502.0061\\
3.77389434735868	-13364011388.3084\\
3.77399434985875	-13352036570.3902\\
3.77409435235881	-13340004456.6924\\
3.77419435485887	-13328029638.7742\\
3.77429435735893	-13316054820.8559\\
3.774394359859	-13304080002.9377\\
3.77449436235906	-13292105185.0195\\
3.77459436485912	-13280130367.1012\\
3.77469436735918	-13268155549.183\\
3.77479436985925	-13256180731.2648\\
3.77489437235931	-13244263209.1261\\
3.77499437485937	-13232288391.2078\\
3.77509437735943	-13220370869.0691\\
3.7751943798595	-13208396051.1509\\
3.77529438235956	-13196478529.0121\\
3.77539438485962	-13184561006.8734\\
3.77549438735968	-13172643484.7347\\
3.77559438985975	-13160725962.596\\
3.77569439235981	-13148808440.4573\\
3.77579439485987	-13136890918.3185\\
3.77589439735993	-13125030691.9593\\
3.77599439986	-13113113169.8206\\
3.77609440236006	-13101252943.4614\\
3.77619440486012	-13089335421.3227\\
3.77629440736018	-13077475194.9635\\
3.77639440986025	-13065614968.6043\\
3.77649441236031	-13053754742.2451\\
3.77659441486037	-13041894515.8859\\
3.77669441736043	-13030034289.5266\\
3.7767944198605	-13018174063.1674\\
3.77689442236056	-13006371132.5877\\
3.77699442486062	-12994510906.2285\\
3.77709442736068	-12982707975.6488\\
3.77719442986075	-12970847749.2896\\
3.77729443236081	-12959044818.7099\\
3.77739443486087	-12947241888.1302\\
3.77749443736093	-12935438957.5505\\
3.777594439861	-12923636026.9708\\
3.77769444236106	-12911833096.3912\\
3.77779444486112	-12900030165.8115\\
3.77789444736118	-12888284531.0113\\
3.77799444986125	-12876481600.4316\\
3.77809445236131	-12864678669.8519\\
3.77819445486137	-12852933035.0517\\
3.77829445736143	-12841187400.2515\\
3.7783944598615	-12829441765.4513\\
3.77849446236156	-12817696130.6512\\
3.77859446486162	-12805950495.851\\
3.77869446736168	-12794204861.0508\\
3.77879446986175	-12782459226.2506\\
3.77889447236181	-12770713591.4504\\
3.77899447486187	-12759025252.4298\\
3.77909447736193	-12747279617.6296\\
3.779194479862	-12735591278.6089\\
3.77929448236206	-12723845643.8087\\
3.77939448486212	-12712157304.7881\\
3.77949448736218	-12700468965.7674\\
3.77959448986225	-12688780626.7467\\
3.77969449236231	-12677092287.7261\\
3.77979449486237	-12665403948.7054\\
3.77989449736243	-12653772905.4642\\
3.7799944998625	-12642084566.4436\\
3.78009450236256	-12630396227.4229\\
3.78019450486262	-12618765184.1817\\
3.78029450736268	-12607134140.9406\\
3.78039450986275	-12595445801.9199\\
3.78049451236281	-12583814758.6788\\
3.78059451486287	-12572183715.4376\\
3.78069451736293	-12560552672.1964\\
3.780794519863	-12548921628.9553\\
3.78089452236306	-12537347881.4936\\
3.78099452486312	-12525716838.2525\\
3.78109452736318	-12514085795.0113\\
3.78119452986325	-12502512047.5497\\
3.78129453236331	-12490938300.0881\\
3.78139453486337	-12479307256.8469\\
3.78149453736343	-12467733509.3853\\
3.7815945398635	-12456159761.9236\\
3.78169454236356	-12444586014.462\\
3.78179454486362	-12433012267.0003\\
3.78189454736368	-12421495815.3182\\
3.78199454986375	-12409922067.8566\\
3.78209455236381	-12398348320.3949\\
3.78219455486387	-12386831868.7128\\
3.78229455736393	-12375258121.2511\\
3.782394559864	-12363741669.569\\
3.78249456236406	-12352225217.8869\\
3.78259456486412	-12340708766.2048\\
3.78269456736418	-12329192314.5226\\
3.78279456986425	-12317675862.8405\\
3.78289457236431	-12306159411.1584\\
3.78299457486437	-12294700255.2557\\
3.78309457736443	-12283183803.5736\\
3.7831945798645	-12271667351.8915\\
3.78329458236456	-12260208195.9889\\
3.78339458486462	-12248749040.0863\\
3.78349458736468	-12237289884.1836\\
3.78359458986475	-12225773432.5015\\
3.78369459236481	-12214314276.5989\\
3.78379459486487	-12202912416.4758\\
3.78389459736493	-12191453260.5732\\
3.783994599865	-12179994104.6706\\
3.78409460236506	-12168534948.7679\\
3.78419460486512	-12157133088.6448\\
3.78429460736518	-12145673932.7422\\
3.78439460986525	-12134272072.6191\\
3.78449461236531	-12122870212.496\\
3.78459461486537	-12111468352.3729\\
3.78469461736543	-12100066492.2498\\
3.7847946198655	-12088664632.1267\\
3.78489462236556	-12077262772.0036\\
3.78499462486562	-12065860911.8805\\
3.78509462736568	-12054516347.5369\\
3.78519462986575	-12043114487.4138\\
3.78529463236581	-12031769923.0702\\
3.78539463486587	-12020368062.9471\\
3.78549463736593	-12009023498.6035\\
3.785594639866	-11997678934.2599\\
3.78569464236606	-11986334369.9163\\
3.78579464486612	-11974989805.5727\\
3.78589464736618	-11963645241.2292\\
3.78599464986625	-11952300676.8856\\
3.78609465236631	-11940956112.542\\
3.78619465486637	-11929668843.9779\\
3.78629465736643	-11918324279.6343\\
3.7863946598665	-11907037011.0702\\
3.78649466236656	-11895749742.5062\\
3.78659466486662	-11884462473.9421\\
3.78669466736668	-11873117909.5985\\
3.78679466986675	-11861830641.0344\\
3.78689467236681	-11850600668.2498\\
3.78699467486687	-11839313399.6858\\
3.78709467736693	-11828026131.1217\\
3.787194679867	-11816738862.5576\\
3.78729468236706	-11805508889.773\\
3.78739468486712	-11794278916.9885\\
3.78749468736718	-11782991648.4244\\
3.78759468986725	-11771761675.6398\\
3.78769469236731	-11760531702.8553\\
3.78779469486737	-11749301730.0707\\
3.78789469736743	-11738071757.2861\\
3.7879946998675	-11726841784.5016\\
3.78809470236756	-11715611811.717\\
3.78819470486762	-11704439134.712\\
3.78829470736768	-11693209161.9274\\
3.78839470986775	-11682036484.9224\\
3.78849471236781	-11670863807.9173\\
3.78859471486787	-11659633835.1327\\
3.78869471736793	-11648461158.1277\\
3.788794719868	-11637288481.1226\\
3.78889472236806	-11626115804.1176\\
3.78899472486812	-11614943127.1125\\
3.78909472736818	-11603827745.887\\
3.78919472986825	-11592655068.8819\\
3.78929473236831	-11581482391.8769\\
3.78939473486837	-11570367010.6514\\
3.78949473736843	-11559251629.4258\\
3.7895947398685	-11548078952.4208\\
3.78969474236856	-11536963571.1952\\
3.78979474486862	-11525848189.9697\\
3.78989474736868	-11514732808.7442\\
3.78999474986875	-11503617427.5186\\
3.79009475236881	-11492502046.2931\\
3.79019475486887	-11481443960.8471\\
3.79029475736893	-11470328579.6215\\
3.790394759869	-11459270494.1755\\
3.79049476236906	-11448155112.95\\
3.79059476486912	-11437097027.5039\\
3.79069476736918	-11426038942.0579\\
3.79079476986925	-11414980856.6119\\
3.79089477236931	-11403922771.1659\\
3.79099477486937	-11392864685.7198\\
3.79109477736943	-11381806600.2738\\
3.7911947798695	-11370748514.8278\\
3.79129478236956	-11359747725.1613\\
3.79139478486962	-11348689639.7152\\
3.79149478736968	-11337688850.0487\\
3.79159478986975	-11326630764.6027\\
3.79169479236981	-11315629974.9362\\
3.79179479486987	-11304629185.2697\\
3.79189479736993	-11293628395.6032\\
3.79199479987	-11282627605.9367\\
3.79209480237006	-11271626816.2701\\
3.79219480487012	-11260626026.6036\\
3.79229480737018	-11249682532.7166\\
3.79239480987025	-11238681743.0501\\
3.79249481237031	-11227738249.1631\\
3.79259481487037	-11216737459.4966\\
3.79269481737043	-11205793965.6096\\
3.7927948198705	-11194850471.7226\\
3.79289482237056	-11183906977.8356\\
3.79299482487062	-11172963483.9486\\
3.79309482737068	-11162019990.0616\\
3.79319482987075	-11151076496.1746\\
3.79329483237081	-11140190298.0671\\
3.79339483487087	-11129246804.1801\\
3.79349483737093	-11118360606.0727\\
3.793594839871	-11107417112.1857\\
3.79369484237106	-11096530914.0782\\
3.79379484487112	-11085644715.9707\\
3.79389484737118	-11074758517.8632\\
3.79399484987125	-11063872319.7557\\
3.79409485237131	-11052986121.6482\\
3.79419485487137	-11042099923.5407\\
3.79429485737143	-11031213725.4333\\
3.7943948598715	-11020384823.1053\\
3.79449486237156	-11009498624.9978\\
3.79459486487162	-10998669722.6698\\
3.79469486737168	-10987783524.5623\\
3.79479486987175	-10976954622.2344\\
3.79489487237181	-10966125719.9064\\
3.79499487487187	-10955296817.5784\\
3.79509487737193	-10944467915.2504\\
3.795194879872	-10933639012.9225\\
3.79529488237206	-10922810110.5945\\
3.79539488487212	-10912038504.046\\
3.79549488737218	-10901209601.7181\\
3.79559488987225	-10890437995.1696\\
3.79569489237231	-10879609092.8416\\
3.79579489487237	-10868837486.2932\\
3.79589489737243	-10858065879.7447\\
3.7959948998725	-10847294273.1963\\
3.79609490237256	-10836522666.6478\\
3.79619490487262	-10825751060.0993\\
3.79629490737268	-10814979453.5509\\
3.79639490987275	-10804265142.7819\\
3.79649491237281	-10793493536.2335\\
3.79659491487287	-10782779225.4645\\
3.79669491737293	-10772007618.9161\\
3.796794919873	-10761293308.1471\\
3.79689492237306	-10750578997.3782\\
3.79699492487312	-10739864686.6092\\
3.79709492737318	-10729150375.8403\\
3.79719492987325	-10718436065.0713\\
3.79729493237331	-10707721754.3024\\
3.79739493487337	-10697007443.5334\\
3.79749493737343	-10686350428.544\\
3.7975949398735	-10675636117.7751\\
3.79769494237356	-10664979102.7856\\
3.79779494487362	-10654264792.0167\\
3.79789494737368	-10643607777.0273\\
3.79799494987375	-10632950762.0378\\
3.79809495237381	-10622293747.0484\\
3.79819495487387	-10611636732.059\\
3.79829495737393	-10600979717.0695\\
3.798394959874	-10590322702.0801\\
3.79849496237406	-10579722982.8702\\
3.79859496487412	-10569065967.8807\\
3.79869496737418	-10558466248.6708\\
3.79879496987425	-10547809233.6814\\
3.79889497237431	-10537209514.4715\\
3.79899497487437	-10526609795.2615\\
3.79909497737443	-10516010076.0516\\
3.7991949798745	-10505410356.8417\\
3.79929498237456	-10494810637.6318\\
3.79939498487462	-10484210918.4219\\
3.79949498737468	-10473668494.9914\\
3.79959498987475	-10463068775.7815\\
3.79969499237481	-10452526352.3511\\
3.79979499487487	-10441926633.1412\\
3.79989499737493	-10431384209.7108\\
3.799994999875	-10420841786.2804\\
3.80009500237506	-10410299362.85\\
3.80019500487512	-10399756939.4196\\
3.80029500737518	-10389214515.9892\\
3.80039500987525	-10378672092.5588\\
3.80049501237531	-10368129669.1284\\
3.80059501487537	-10357644541.4775\\
3.80069501737543	-10347102118.0471\\
3.8007950198755	-10336616990.3962\\
3.80089502237556	-10326074566.9657\\
3.80099502487562	-10315589439.3149\\
3.80109502737568	-10305104311.664\\
3.80119502987575	-10294619184.0131\\
3.80129503237581	-10284134056.3622\\
3.80139503487587	-10273648928.7113\\
3.80149503737593	-10263163801.0604\\
3.801595039876	-10252735969.189\\
3.80169504237606	-10242250841.5381\\
3.80179504487612	-10231823009.6667\\
3.80189504737618	-10221337882.0158\\
3.80199504987625	-10210910050.1445\\
3.80209505237631	-10200482218.2731\\
3.80219505487637	-10190054386.4017\\
3.80229505737643	-10179626554.5303\\
3.8023950598765	-10169198722.6589\\
3.80249506237656	-10158770890.7875\\
3.80259506487662	-10148343058.9162\\
3.80269506737668	-10137972522.8243\\
3.80279506987675	-10127544690.9529\\
3.80289507237681	-10117174154.861\\
3.80299507487687	-10106803618.7692\\
3.80309507737693	-10096375786.8978\\
3.803195079877	-10086005250.8059\\
3.80329508237706	-10075634714.7141\\
3.80339508487712	-10065264178.6222\\
3.80349508737718	-10054950938.3098\\
3.80359508987725	-10044580402.218\\
3.80369509237731	-10034209866.1261\\
3.80379509487737	-10023896625.8138\\
3.80389509737743	-10013526089.7219\\
3.8039950998775	-10003212849.4095\\
3.80409510237756	-9992899609.09718\\
3.80419510487762	-9982529073.00531\\
3.80429510737768	-9972215832.69295\\
3.80439510987775	-9961902592.3806\\
3.80449511237781	-9951646647.84776\\
3.80459511487787	-9941333407.5354\\
3.80469511737793	-9931020167.22305\\
3.804795119878	-9920764222.6902\\
3.80489512237806	-9910450982.37785\\
3.80499512487812	-9900195037.84501\\
3.80509512737818	-9889881797.53265\\
3.80519512987825	-9879625852.99981\\
3.80529513237831	-9869369908.46697\\
3.80539513487837	-9859113963.93413\\
3.80549513737843	-9848858019.40129\\
3.8055951398785	-9838602074.86844\\
3.80569514237856	-9828403426.11512\\
3.80579514487862	-9818147481.58227\\
3.80589514737868	-9807948832.82895\\
3.80599514987875	-9797692888.2961\\
3.80609515237881	-9787494239.54277\\
3.80619515487887	-9777295590.78945\\
3.80629515737893	-9767096942.03612\\
3.806395159879	-9756840997.50327\\
3.80649516237906	-9746699644.52946\\
3.80659516487912	-9736500995.77613\\
3.80669516737918	-9726302347.0228\\
3.80679516987925	-9716103698.26947\\
3.80689517237931	-9705962345.29566\\
3.80699517487937	-9695763696.54233\\
3.80709517737943	-9685622343.56851\\
3.8071951798795	-9675480990.5947\\
3.80729518237956	-9665339637.62088\\
3.80739518487962	-9655140988.86755\\
3.80749518737968	-9644999635.89374\\
3.80759518987975	-9634915578.69944\\
3.80769519237981	-9624774225.72562\\
3.80779519487987	-9614632872.75181\\
3.80789519737993	-9604548815.5575\\
3.80799519988	-9594407462.58369\\
3.80809520238006	-9584323405.38939\\
3.80819520488012	-9574182052.41557\\
3.80829520738018	-9564097995.22127\\
3.80839520988025	-9554013938.02696\\
3.80849521238031	-9543929880.83266\\
3.80859521488037	-9533845823.63836\\
3.80869521738043	-9523761766.44406\\
3.8087952198805	-9513677709.24975\\
3.80889522238056	-9503650947.83496\\
3.80899522488062	-9493566890.64066\\
3.80909522738068	-9483540129.22587\\
3.80919522988075	-9473513367.81108\\
3.80929523238081	-9463429310.61678\\
3.80939523488087	-9453402549.20199\\
3.80949523738093	-9443375787.7872\\
3.809595239881	-9433349026.37241\\
3.80969524238106	-9423322264.95762\\
3.80979524488112	-9413295503.54283\\
3.80989524738118	-9403326037.90756\\
3.80999524988125	-9393299276.49277\\
3.81009525238131	-9383329810.85749\\
3.81019525488137	-9373303049.4427\\
3.81029525738143	-9363333583.80743\\
3.8103952598815	-9353364118.17215\\
3.81049526238156	-9343394652.53687\\
3.81059526488162	-9333425186.9016\\
3.81069526738168	-9323455721.26632\\
3.81079526988175	-9313486255.63104\\
3.81089527238181	-9303516789.99577\\
3.81099527488187	-9293604620.14001\\
3.81109527738193	-9283635154.50473\\
3.811195279882	-9273722984.64897\\
3.81129528238206	-9263753519.01369\\
3.81139528488212	-9253841349.15793\\
3.81149528738218	-9243929179.30216\\
3.81159528988225	-9234017009.4464\\
3.81169529238231	-9224104839.59064\\
3.81179529488237	-9214192669.73487\\
3.81189529738243	-9204280499.87911\\
3.8119952998825	-9194425625.80286\\
3.81209530238256	-9184513455.9471\\
3.81219530488262	-9174658581.87085\\
3.81229530738268	-9164746412.01508\\
3.81239530988275	-9154891537.93883\\
3.81249531238281	-9145036663.86258\\
3.81259531488287	-9135181789.78633\\
3.81269531738293	-9125326915.71008\\
3.812795319883	-9115472041.63383\\
3.81289532238306	-9105617167.55758\\
3.81299532488312	-9095762293.48133\\
3.81309532738318	-9085964715.1846\\
3.81319532988325	-9076109841.10835\\
3.81329533238331	-9066312262.81161\\
3.81339533488337	-9056514684.51487\\
3.81349533738343	-9046659810.43862\\
3.8135953398835	-9036862232.14188\\
3.81369534238356	-9027064653.84515\\
3.81379534488362	-9017267075.54841\\
3.81389534738368	-9007469497.25167\\
3.81399534988375	-8997729214.73445\\
3.81409535238381	-8987931636.43771\\
3.81419535488387	-8978134058.14097\\
3.81429535738393	-8968393775.62375\\
3.814395359884	-8958653493.10653\\
3.81449536238406	-8948855914.80979\\
3.81459536488412	-8939115632.29256\\
3.81469536738418	-8929375349.77534\\
3.81479536988425	-8919635067.25812\\
3.81489537238431	-8909894784.74089\\
3.81499537488437	-8900154502.22367\\
3.81509537738443	-8890471515.48596\\
3.8151953798845	-8880731232.96873\\
3.81529538238456	-8871048246.23102\\
3.81539538488462	-8861307963.7138\\
3.81549538738468	-8851624976.97609\\
3.81559538988475	-8841941990.23838\\
3.81569539238481	-8832259003.50067\\
3.81579539488487	-8822518720.98344\\
3.81589539738493	-8812893030.02524\\
3.815995399885	-8803210043.28753\\
3.81609540238506	-8793527056.54982\\
3.81619540488512	-8783844069.81211\\
3.81629540738518	-8774218378.85391\\
3.81639540988525	-8764535392.1162\\
3.81649541238531	-8754909701.158\\
3.81659541488537	-8745284010.19981\\
3.81669541738543	-8735658319.24161\\
3.8167954198855	-8725975332.5039\\
3.81689542238556	-8716349641.5457\\
3.81699542488562	-8706781246.36702\\
3.81709542738568	-8697155555.40882\\
3.81719542988575	-8687529864.45062\\
3.81729543238581	-8677961469.27194\\
3.81739543488587	-8668335778.31374\\
3.81749543738593	-8658767383.13505\\
3.817595439886	-8649141692.17686\\
3.81769544238606	-8639573296.99817\\
3.81779544488612	-8630004901.81949\\
3.81789544738618	-8620436506.6408\\
3.81799544988625	-8610868111.46212\\
3.81809545238631	-8601299716.28343\\
3.81819545488637	-8591731321.10475\\
3.81829545738643	-8582220221.70557\\
3.8183954598865	-8572651826.52689\\
3.81849546238656	-8563140727.12772\\
3.81859546488662	-8553572331.94903\\
3.81869546738668	-8544061232.54986\\
3.81879546988675	-8534550133.15069\\
3.81889547238681	-8525039033.75152\\
3.81899547488687	-8515527934.35235\\
3.81909547738693	-8506016834.95318\\
3.819195479887	-8496505735.554\\
3.81929548238706	-8487051931.93435\\
3.81939548488712	-8477540832.53517\\
3.81949548738718	-8468087028.91551\\
3.81959548988725	-8458575929.51634\\
3.81969549238731	-8449122125.89668\\
3.81979549488737	-8439668322.27703\\
3.81989549738743	-8430214518.65737\\
3.8199954998875	-8420760715.03771\\
3.82009550238756	-8411306911.41805\\
3.82019550488762	-8401853107.79839\\
3.82029550738768	-8392399304.17873\\
3.82039550988775	-8382945500.55907\\
3.82049551238781	-8373548992.71893\\
3.82059551488787	-8364095189.09927\\
3.82069551738793	-8354698681.25912\\
3.820795519888	-8345302173.41898\\
3.82089552238806	-8335905665.57883\\
3.82099552488812	-8326509157.73869\\
3.82109552738818	-8317112649.89854\\
3.82119552988825	-8307716142.0584\\
3.82129553238831	-8298319634.21825\\
3.82139553488837	-8288923126.37811\\
3.82149553738843	-8279583914.31747\\
3.8215955398885	-8270187406.47733\\
3.82169554238856	-8260848194.4167\\
3.82179554488862	-8251451686.57655\\
3.82189554738868	-8242112474.51592\\
3.82199554988875	-8232773262.45529\\
3.82209555238881	-8223434050.39465\\
3.82219555488887	-8214094838.33402\\
3.82229555738893	-8204755626.27339\\
3.822395559889	-8195473709.99227\\
3.82249556238906	-8186134497.93164\\
3.82259556488912	-8176795285.871\\
3.82269556738918	-8167513369.58988\\
3.82279556988925	-8158231453.30877\\
3.82289557238931	-8148892241.24813\\
3.82299557488937	-8139610324.96701\\
3.82309557738943	-8130328408.68589\\
3.8231955798895	-8121046492.40478\\
3.82329558238956	-8111764576.12366\\
3.82339558488962	-8102482659.84254\\
3.82349558738968	-8093258039.34093\\
3.82359558988975	-8083976123.05981\\
3.82369559238981	-8074751502.5582\\
3.82379559488987	-8065469586.27709\\
3.82389559738993	-8056244965.77548\\
3.82399559989	-8047020345.27387\\
3.82409560239006	-8037795724.77227\\
3.82419560489012	-8028571104.27066\\
3.82429560739018	-8019346483.76905\\
3.82439560989025	-8010121863.26745\\
3.82449561239031	-8000897242.76584\\
3.82459561489037	-7991729918.04375\\
3.82469561739043	-7982505297.54214\\
3.8247956198905	-7973337972.82005\\
3.82489562239056	-7964113352.31844\\
3.82499562489062	-7954946027.59635\\
3.82509562739068	-7945778702.87426\\
3.82519562989075	-7936611378.15216\\
3.82529563239081	-7927444053.43007\\
3.82539563489087	-7918276728.70798\\
3.82549563739093	-7909109403.98588\\
3.825595639891	-7899942079.26379\\
3.82569564239106	-7890832050.32121\\
3.82579564489112	-7881664725.59912\\
3.82589564739118	-7872554696.65654\\
3.82599564989125	-7863387371.93444\\
3.82609565239131	-7854277342.99186\\
3.82619565489137	-7845167314.04928\\
3.82629565739143	-7836057285.1067\\
3.8263956598915	-7826947256.16412\\
3.82649566239156	-7817837227.22154\\
3.82659566489162	-7808784494.05848\\
3.82669566739168	-7799674465.1159\\
3.82679566989175	-7790564436.17332\\
3.82689567239181	-7781511703.01025\\
3.82699567489187	-7772458969.84718\\
3.82709567739193	-7763348940.9046\\
3.827195679892	-7754296207.74154\\
3.82729568239206	-7745243474.57847\\
3.82739568489212	-7736190741.4154\\
3.82749568739218	-7727138008.25233\\
3.82759568989225	-7718085275.08927\\
3.82769569239231	-7709089837.70571\\
3.82779569489237	-7700037104.54265\\
3.82789569739243	-7691041667.15909\\
3.8279956998925	-7681988933.99603\\
3.82809570239256	-7672993496.61247\\
3.82819570489262	-7663998059.22892\\
3.82829570739268	-7654945326.06585\\
3.82839570989275	-7645949888.6823\\
3.82849571239281	-7636954451.29874\\
3.82859571489287	-7628016309.6947\\
3.82869571739294	-7619020872.31115\\
3.828795719893	-7610025434.92759\\
3.82889572239306	-7601087293.32355\\
3.82899572489312	-7592091855.94\\
3.82909572739318	-7583153714.33596\\
3.82919572989325	-7574215572.73192\\
3.82929573239331	-7565220135.34836\\
3.82939573489337	-7556281993.74432\\
3.82949573739343	-7547343852.14028\\
3.8295957398935	-7538405710.53624\\
3.82969574239356	-7529524864.71171\\
3.82979574489362	-7520586723.10767\\
3.82989574739368	-7511648581.50363\\
3.82999574989375	-7502767735.6791\\
3.83009575239381	-7493829594.07506\\
3.83019575489387	-7484948748.25054\\
3.83029575739393	-7476067902.42601\\
3.830395759894	-7467187056.60148\\
3.83049576239406	-7458306210.77695\\
3.83059576489412	-7449425364.95242\\
3.83069576739419	-7440544519.1279\\
3.83079576989425	-7431663673.30337\\
3.83089577239431	-7422782827.47884\\
3.83099577489437	-7413959277.43383\\
3.83109577739443	-7405078431.6093\\
3.8311957798945	-7396254881.56428\\
3.83129578239456	-7387431331.51927\\
3.83139578489462	-7378550485.69474\\
3.83149578739468	-7369726935.64973\\
3.83159578989475	-7360903385.60471\\
3.83169579239481	-7352079835.5597\\
3.83179579489487	-7343313581.2942\\
3.83189579739493	-7334490031.24918\\
3.831995799895	-7325666481.20417\\
3.83209580239506	-7316900226.93867\\
3.83219580489512	-7308076676.89365\\
3.83229580739518	-7299310422.62815\\
3.83239580989525	-7290544168.36265\\
3.83249581239531	-7281777914.09715\\
3.83259581489537	-7272954364.05213\\
3.83269581739544	-7264245405.56614\\
3.8327958198955	-7255479151.30064\\
3.83289582239556	-7246712897.03514\\
3.83299582489562	-7237946642.76964\\
3.83309582739568	-7229237684.28365\\
3.83319582989575	-7220471430.01815\\
3.83329583239581	-7211762471.53216\\
3.83339583489587	-7202996217.26666\\
3.83349583739593	-7194287258.78067\\
3.833595839896	-7185578300.29468\\
3.83369584239606	-7176869341.80869\\
3.83379584489612	-7168160383.3227\\
3.83389584739618	-7159451424.83671\\
3.83399584989625	-7150799762.13024\\
3.83409585239631	-7142090803.64425\\
3.83419585489637	-7133381845.15826\\
3.83429585739643	-7124730182.45179\\
3.8343958598965	-7116078519.74531\\
3.83449586239656	-7107369561.25932\\
3.83459586489662	-7098717898.55285\\
3.83469586739669	-7090066235.84637\\
3.83479586989675	-7081414573.1399\\
3.83489587239681	-7072762910.43342\\
3.83499587489687	-7064111247.72695\\
3.83509587739694	-7055516880.79998\\
3.835195879897	-7046865218.09351\\
3.83529588239706	-7038270851.16655\\
3.83539588489712	-7029619188.46007\\
3.83549588739718	-7021024821.53311\\
3.83559588989725	-7012430454.60615\\
3.83569589239731	-7003836087.67918\\
3.83579589489737	-6995184424.97271\\
3.83589589739743	-6986647353.82526\\
3.8359958998975	-6978052986.8983\\
3.83609590239756	-6969458619.97133\\
3.83619590489762	-6960864253.04437\\
3.83629590739768	-6952327181.89692\\
3.83639590989775	-6943732814.96996\\
3.83649591239781	-6935195743.82251\\
3.83659591489787	-6926658672.67506\\
3.83669591739794	-6918064305.7481\\
3.836795919898	-6909527234.60065\\
3.83689592239806	-6900990163.4532\\
3.83699592489812	-6892453092.30575\\
3.83709592739819	-6883973316.93781\\
3.83719592989825	-6875436245.79037\\
3.83729593239831	-6866899174.64292\\
3.83739593489837	-6858419399.27498\\
3.83749593739843	-6849882328.12753\\
3.8375959398985	-6841402552.75959\\
3.83769594239856	-6832922777.39166\\
3.83779594489862	-6824443002.02372\\
3.83789594739868	-6815963226.65579\\
3.83799594989875	-6807483451.28785\\
3.83809595239881	-6799003675.91991\\
3.83819595489887	-6790523900.55198\\
3.83829595739893	-6782044125.18404\\
3.838395959899	-6773621645.59562\\
3.83849596239906	-6765141870.22768\\
3.83859596489912	-6756719390.63926\\
3.83869596739919	-6748296911.05084\\
3.83879596989925	-6739817135.6829\\
3.83889597239931	-6731394656.09448\\
3.83899597489937	-6722972176.50605\\
3.83909597739944	-6714549696.91763\\
3.8391959798995	-6706127217.32921\\
3.83929598239956	-6697762033.5203\\
3.83939598489962	-6689339553.93187\\
3.83949598739969	-6680974370.12296\\
3.83959598989975	-6672551890.53454\\
3.83969599239981	-6664186706.72563\\
3.83979599489987	-6655764227.13721\\
3.83989599739993	-6647399043.3283\\
3.8399959999	-6639033859.51939\\
3.84009600240006	-6630668675.71048\\
3.84019600490012	-6622303491.90157\\
3.84029600740018	-6613995603.87217\\
3.84039600990025	-6605630420.06326\\
3.84049601240031	-6597265236.25435\\
3.84059601490037	-6588957348.22495\\
3.84069601740044	-6580592164.41604\\
3.8407960199005	-6572284276.38665\\
3.84089602240056	-6563976388.35725\\
3.84099602490062	-6555668500.32785\\
3.84109602740069	-6547360612.29846\\
3.84119602990075	-6539052724.26906\\
3.84129603240081	-6530744836.23966\\
3.84139603490087	-6522436948.21027\\
3.84149603740094	-6514129060.18087\\
3.841596039901	-6505878467.93098\\
3.84169604240106	-6497570579.90159\\
3.84179604490112	-6489319987.6517\\
3.84189604740118	-6481069395.40182\\
3.84199604990125	-6472761507.37242\\
3.84209605240131	-6464510915.12254\\
3.84219605490137	-6456260322.87266\\
3.84229605740143	-6448009730.62277\\
3.8423960599015	-6439816434.1524\\
3.84249606240156	-6431565841.90252\\
3.84259606490162	-6423315249.65263\\
3.84269606740169	-6415121953.18226\\
3.84279606990175	-6406871360.93238\\
3.84289607240181	-6398678064.46201\\
3.84299607490187	-6390484767.99164\\
3.84309607740194	-6382291471.52127\\
3.843196079902	-6374098175.0509\\
3.84329608240206	-6365904878.58052\\
3.84339608490212	-6357711582.11015\\
3.84349608740219	-6349518285.63978\\
3.84359608990225	-6341324989.16941\\
3.84369609240231	-6333188988.47855\\
3.84379609490237	-6324995692.00818\\
3.84389609740243	-6316859691.31733\\
3.8439960999025	-6308666394.84696\\
3.84409610240256	-6300530394.1561\\
3.84419610490262	-6292394393.46524\\
3.84429610740268	-6284258392.77438\\
3.84439610990275	-6276122392.08352\\
3.84449611240281	-6267986391.39267\\
3.84459611490287	-6259907686.48132\\
3.84469611740294	-6251771685.79046\\
3.844796119903	-6243635685.09961\\
3.84489612240306	-6235556980.18826\\
3.84499612490312	-6227420979.49741\\
3.84509612740319	-6219342274.58606\\
3.84519612990325	-6211263569.67472\\
3.84529613240331	-6203184864.76337\\
3.84539613490337	-6195106159.85203\\
3.84549613740344	-6187027454.94068\\
3.8455961399035	-6178948750.02934\\
3.84569614240356	-6170870045.11799\\
3.84579614490362	-6162848635.98616\\
3.84589614740369	-6154769931.07482\\
3.84599614990375	-6146748521.94298\\
3.84609615240381	-6138727112.81115\\
3.84619615490387	-6130648407.89981\\
3.84629615740393	-6122626998.76798\\
3.846396159904	-6114605589.63615\\
3.84649616240406	-6106584180.50431\\
3.84659616490412	-6098562771.37248\\
3.84669616740419	-6090598658.02016\\
3.84679616990425	-6082577248.88833\\
3.84689617240431	-6074555839.7565\\
3.84699617490437	-6066591726.40418\\
3.84709617740444	-6058570317.27235\\
3.8471961799045	-6050606203.92003\\
3.84729618240456	-6042642090.56771\\
3.84739618490462	-6034677977.2154\\
3.84749618740469	-6026713863.86308\\
3.84759618990475	-6018749750.51076\\
3.84769619240481	-6010785637.15844\\
3.84779619490487	-6002821523.80612\\
3.84789619740494	-5994914706.23332\\
3.847996199905	-5986950592.881\\
3.84809620240506	-5979043775.30819\\
3.84819620490512	-5971079661.95587\\
3.84829620740518	-5963172844.38307\\
3.84839620990525	-5955266026.81026\\
3.84849621240531	-5947359209.23746\\
3.84859621490537	-5939452391.66465\\
3.84869621740544	-5931545574.09185\\
3.8487962199055	-5923638756.51904\\
3.84889622240556	-5915731938.94624\\
3.84899622490562	-5907882417.15294\\
3.84909622740569	-5899975599.58014\\
3.84919622990575	-5892126077.78685\\
3.84929623240581	-5884219260.21404\\
3.84939623490587	-5876369738.42075\\
3.84949623740594	-5868520216.62746\\
3.849596239906	-5860670694.83416\\
3.84969624240606	-5852821173.04087\\
3.84979624490612	-5844971651.24758\\
3.84989624740619	-5837122129.45429\\
3.84999624990625	-5829329903.44051\\
3.85009625240631	-5821480381.64722\\
3.85019625490637	-5813630859.85392\\
3.85029625740643	-5805838633.84015\\
3.8503962599065	-5798046407.82637\\
3.85049626240656	-5790254181.81259\\
3.85059626490662	-5782404660.01929\\
3.85069626740669	-5774612434.00552\\
3.85079626990675	-5766820207.99174\\
3.85089627240681	-5759085277.75747\\
3.85099627490687	-5751293051.74369\\
3.85109627740694	-5743500825.72991\\
3.851196279907	-5735765895.49565\\
3.85129628240706	-5727985128.63777\\
3.85139628490712	-5720227280.0917\\
3.85149628740719	-5712475161.12358\\
3.85159628990725	-5704734501.31136\\
3.85169629240731	-5696993841.49914\\
3.85179629490737	-5689258911.26488\\
3.85189629740744	-5681529710.60856\\
3.8519962999075	-5673806239.5302\\
3.85209630240756	-5666082768.45183\\
3.85219630490762	-5658370756.52937\\
3.85229630740769	-5650664474.18486\\
3.85239630990775	-5642958191.84035\\
3.85249631240781	-5635263368.65175\\
3.85259631490787	-5627574275.04109\\
3.85269631740794	-5619885181.43044\\
3.852796319908	-5612201817.39773\\
3.85289632240806	-5604529912.52093\\
3.85299632490812	-5596858007.64413\\
3.85309632740819	-5589191832.34528\\
3.85319632990825	-5581531386.62438\\
3.85329633240831	-5573876670.48143\\
3.85339633490837	-5566227683.91643\\
3.85349633740844	-5558584426.92939\\
3.8535963399085	-5550946899.5203\\
3.85369634240856	-5543315101.68915\\
3.85379634490862	-5535689033.43596\\
3.85389634740869	-5528062965.18277\\
3.85399634990875	-5520448356.08548\\
3.85409635240881	-5512839476.56614\\
3.85419635490887	-5505230597.04681\\
3.85429635740894	-5497627447.10542\\
3.854396359909	-5490035756.31994\\
3.85449636240906	-5482444065.53445\\
3.85459636490912	-5474858104.32692\\
3.85469636740919	-5467277872.69734\\
3.85479636990925	-5459709100.22366\\
3.85489637240931	-5452140327.74998\\
3.85499637490937	-5444577284.85426\\
3.85509637740944	-5437014241.95853\\
3.8551963799095	-5429462658.21871\\
3.85529638240956	-5421916804.05683\\
3.85539638490962	-5414376679.47291\\
3.85549638740969	-5406836554.88899\\
3.85559638990975	-5399307889.46097\\
3.85569639240981	-5391779224.03295\\
3.85579639490987	-5384262017.76084\\
3.85589639740994	-5376744811.48872\\
3.85599639991	-5369233334.79455\\
3.85609640241006	-5361733317.25629\\
3.85619640491012	-5354233299.71803\\
3.85629640741019	-5346739011.75772\\
3.85639640991025	-5339250453.37536\\
3.85649641241031	-5331767624.57095\\
3.85659641491037	-5324290525.34449\\
3.85669641741044	-5316819155.69599\\
3.8567964199105	-5309347786.04748\\
3.85689642241056	-5301887875.55488\\
3.85699642491062	-5294433694.64023\\
3.85709642741069	-5286979513.72557\\
3.85719642991075	-5279536791.96682\\
3.85729643241081	-5272094070.20808\\
3.85739643491087	-5264662807.60523\\
3.85749643741094	-5257231545.00238\\
3.857596439911	-5249806011.97749\\
3.85769644241106	-5242386208.53054\\
3.85779644491112	-5234972134.66155\\
3.85789644741119	-5227563790.37051\\
3.85799644991125	-5220161175.65742\\
3.85809645241131	-5212764290.52228\\
3.85819645491137	-5205373134.96509\\
3.85829645741144	-5197987708.98585\\
3.8583964599115	-5190608012.58457\\
3.85849646241156	-5183228316.18328\\
3.85859646491162	-5175860078.9379\\
3.85869646741169	-5168491841.69252\\
3.85879646991175	-5161135063.60304\\
3.85889647241181	-5153778285.51356\\
3.85899647491187	-5146427237.00203\\
3.85909647741194	-5139081918.06845\\
3.859196479912	-5131748058.29078\\
3.85929648241206	-5124414198.51311\\
3.85939648491212	-5117086068.31338\\
3.85949648741219	-5109763667.69161\\
3.85959648991225	-5102446996.64779\\
3.85969649241231	-5095130325.60397\\
3.85979649491237	-5087825113.71605\\
3.85989649741244	-5080525631.40608\\
3.8599964999125	-5073226149.09612\\
3.86009650241256	-5065938125.94205\\
3.86019650491262	-5058650102.78799\\
3.86029650741269	-5051373538.78983\\
3.86039650991275	-5044096974.79167\\
3.86049651241281	-5036826140.37146\\
3.86059651491287	-5029566765.10715\\
3.86069651741294	-5022307389.84284\\
3.860796519913	-5015053744.15649\\
3.86089652241306	-5007805828.04808\\
3.86099652491312	-5000563641.51763\\
3.86109652741319	-4993327184.56512\\
3.86119652991325	-4986090727.61262\\
3.86129653241331	-4978865729.81602\\
3.86139653491337	-4971646461.59737\\
3.86149653741344	-4964427193.37873\\
3.8615965399135	-4957219384.31598\\
3.86169654241356	-4950011575.25323\\
3.86179654491362	-4942815225.34639\\
3.86189654741369	-4935618875.43955\\
3.86199654991375	-4928428255.11066\\
3.86209655241381	-4921243364.35972\\
3.86219655491387	-4914064203.18673\\
3.86229655741394	-4906890771.59169\\
3.862396559914	-4899723069.5746\\
3.86249656241406	-4892561097.13547\\
3.86259656491412	-4885404854.27428\\
3.86269656741419	-4878254340.99105\\
3.86279656991425	-4871109557.28577\\
3.86289657241431	-4863964773.58049\\
3.86299657491437	-4856831449.03111\\
3.86309657741444	-4849698124.48173\\
3.8631965799145	-4842576259.08825\\
3.86329658241456	-4835454393.69478\\
3.86339658491462	-4828338257.87925\\
3.86349658741469	-4821227851.64168\\
3.86359658991475	-4814123174.98206\\
3.86369659241481	-4807024227.90039\\
3.86379659491487	-4799931010.39667\\
3.86389659741494	-4792843522.4709\\
3.863996599915	-4785761764.12308\\
3.86409660241506	-4778685735.35322\\
3.86419660491512	-4771615436.1613\\
3.86429660741519	-4764545136.96939\\
3.86439660991525	-4757486296.93338\\
3.86449661241531	-4750427456.89736\\
3.86459661491537	-4743380076.01725\\
3.86469661741544	-4736332695.13715\\
3.8647966199155	-4729291043.83499\\
3.86489662241556	-4722260851.68873\\
3.86499662491562	-4715230659.54248\\
3.86509662741569	-4708206196.97417\\
3.86519662991575	-4701187463.98382\\
3.86529663241581	-4694174460.57142\\
3.86539663491587	-4687167186.73697\\
3.86549663741594	-4680159912.90252\\
3.865596639916	-4673164098.22397\\
3.86569664241606	-4666174013.12338\\
3.86579664491612	-4659183928.02278\\
3.86589664741619	-4652205302.07809\\
3.86599664991625	-4645226676.13339\\
3.86609665241631	-4638259509.3446\\
3.86619665491637	-4631292342.55581\\
3.86629665741644	-4624330905.34497\\
3.8663966599165	-4617375197.71208\\
3.86649666241656	-4610425219.65715\\
3.86659666491662	-4603480971.18016\\
3.86669666741669	-4596542452.28113\\
3.86679666991675	-4589609662.96004\\
3.86689667241681	-4582682603.21691\\
3.86699667491687	-4575761273.05173\\
3.86709667741694	-4568839942.88655\\
3.867196679917	-4561930071.87727\\
3.86729668241706	-4555020200.868\\
3.86739668491712	-4548121789.01462\\
3.86749668741719	-4541223377.16125\\
3.86759668991725	-4534336424.46377\\
3.86769669241731	-4527449471.7663\\
3.86779669491737	-4520568248.64678\\
3.86789669741744	-4513692755.10521\\
3.8679966999175	-4506822991.14159\\
3.86809670241756	-4499958956.75592\\
3.86819670491762	-4493100651.94821\\
3.86829670741769	-4486248076.71844\\
3.86839670991775	-4479395501.48868\\
3.86849671241781	-4472554385.41482\\
3.86859671491787	-4465718998.91891\\
3.86869671741794	-4458883612.423\\
3.868796719918	-4452053955.50504\\
3.86889672241806	-4445235757.74298\\
3.86899672491812	-4438417559.98092\\
3.86909672741819	-4431605091.79682\\
3.86919672991825	-4424798353.19066\\
3.86929673241831	-4418003073.74041\\
3.86939673491837	-4411207794.29016\\
3.86949673741844	-4404412514.83991\\
3.8695967399185	-4397628694.54556\\
3.86969674241856	-4390850603.82916\\
3.86979674491862	-4384078242.69071\\
3.86989674741869	-4377305881.55227\\
3.86999674991875	-4370544979.56973\\
3.87009675241881	-4363784077.58718\\
3.87019675491887	-4357034634.76054\\
3.87029675741894	-4350285191.9339\\
3.870396759919	-4343541478.68521\\
3.87049676241906	-4336809224.59242\\
3.87059676491912	-4330076970.49963\\
3.87069676741919	-4323350445.9848\\
3.87079676991925	-4316629651.04791\\
3.87089677241931	-4309914585.68898\\
3.87099677491937	-4303205249.908\\
3.87109677741944	-4296495914.12702\\
3.8711967799195	-4289798037.50194\\
3.87129678241956	-4283105890.45481\\
3.87139678491962	-4276413743.40768\\
3.87149678741969	-4269733055.51646\\
3.87159678991975	-4263052367.62523\\
3.87169679241981	-4256377409.31196\\
3.87179679491987	-4249713910.15459\\
3.87189679741994	-4243050410.99721\\
3.87199679992	-4236392641.41779\\
3.87209680242006	-4229740601.41633\\
3.87219680492012	-4223094290.99281\\
3.87229680742019	-4216453710.14724\\
3.87239680992025	-4209813129.30167\\
3.87249681242031	-4203184007.61201\\
3.87259681492037	-4196560615.5003\\
3.87269681742044	-4189937223.38859\\
3.8727968199205	-4183325290.43278\\
3.87289682242056	-4176713357.47697\\
3.87299682492062	-4170112883.67706\\
3.87309682742069	-4163512409.87715\\
3.87319682992075	-4156917665.6552\\
3.87329683242081	-4150328651.01119\\
3.87339683492087	-4143745365.94514\\
3.87349683742094	-4137167810.45704\\
3.873596839921	-4130595984.54689\\
3.87369684242106	-4124029888.21469\\
3.87379684492112	-4117469521.46044\\
3.87389684742119	-4110914884.28414\\
3.87399684992125	-4104360247.10785\\
3.87409685242131	-4097817069.08745\\
3.87419685492137	-4091273891.06706\\
3.87429685742144	-4084742172.20257\\
3.8743968599215	-4078210453.33808\\
3.87449686242156	-4071684464.05154\\
3.87459686492162	-4065164204.34295\\
3.87469686742169	-4058649674.21231\\
3.87479686992175	-4052140873.65962\\
3.87489687242181	-4045637802.68489\\
3.87499687492187	-4039140461.28811\\
3.87509687742194	-4032648849.46927\\
3.875196879922	-4026162967.22839\\
3.87529688242206	-4019677084.98751\\
3.87539688492212	-4013202661.90253\\
3.87549688742219	-4006733968.39551\\
3.87559688992225	-4000265274.88848\\
3.87569689242231	-3993802310.9594\\
3.87579689492237	-3987350806.18623\\
3.87589689742244	-3980899301.41306\\
3.8759968999225	-3974453526.21784\\
3.87609690242256	-3968013480.60056\\
3.87619690492262	-3961579164.56125\\
3.87629690742269	-3955150578.09988\\
3.87639690992275	-3948727721.21646\\
3.87649691242281	-3942304864.33305\\
3.87659691492287	-3935893466.60553\\
3.87669691742294	-3929487798.45597\\
3.876796919923	-3923082130.30641\\
3.87689692242306	-3916687921.31275\\
3.87699692492312	-3910293712.31909\\
3.87709692742319	-3903905232.90338\\
3.87719692992325	-3897522483.06562\\
3.87729693242331	-3891151192.38377\\
3.87739693492337	-3884779901.70191\\
3.87749693742344	-3878414340.59801\\
3.8775969399235	-3872048779.4941\\
3.87769694242356	-3865694677.5461\\
3.87779694492362	-3859346305.17605\\
3.87789694742369	-3853003662.38396\\
3.87799694992375	-3846661019.59186\\
3.87809695242381	-3840329835.95566\\
3.87819695492387	-3833998652.31947\\
3.87829695742394	-3827678927.83917\\
3.878396959924	-3821359203.35888\\
3.87849696242406	-3815045208.45654\\
3.87859696492412	-3808736943.13215\\
3.87869696742419	-3802434407.38571\\
3.87879696992425	-3796137601.21722\\
3.87889697242431	-3789846524.62668\\
3.87899697492437	-3783561177.6141\\
3.87909697742444	-3777281560.17947\\
3.8791969799245	-3771007672.32278\\
3.87929698242456	-3764733784.4661\\
3.87939698492462	-3758471355.76532\\
3.87949698742469	-3752208927.06454\\
3.87959698992475	-3745957957.51966\\
3.87969699242481	-3739706987.97479\\
3.87979699492487	-3733461748.00786\\
3.87989699742494	-3727222237.61889\\
3.879996999925	-3720988456.80786\\
3.88009700242506	-3714760405.57479\\
3.88019700492512	-3708538083.91967\\
3.88029700742519	-3702321491.8425\\
3.88039700992525	-3696110629.34328\\
3.88049701242531	-3689905496.42201\\
3.88059701492537	-3683700363.50075\\
3.88069701742544	-3677506689.73538\\
3.8807970199255	-3671313015.97002\\
3.88089702242556	-3665130801.36056\\
3.88099702492562	-3658948586.7511\\
3.88109702742569	-3652772101.71959\\
3.88119702992575	-3646601346.26603\\
3.88129703242581	-3640436320.39042\\
3.88139703492587	-3634277024.09276\\
3.88149703742594	-3628123457.37306\\
3.881597039926	-3621975620.2313\\
3.88169704242606	-3615833512.6675\\
3.88179704492612	-3609691405.1037\\
3.88189704742619	-3603560756.6958\\
3.88199704992625	-3597435837.86585\\
3.88209705242631	-3591310919.0359\\
3.88219705492637	-3585191729.78391\\
3.88229705742644	-3579083999.68781\\
3.8823970599265	-3572976269.59172\\
3.88249706242656	-3566874269.07357\\
3.88259706492662	-3560777998.13338\\
3.88269706742669	-3554687456.77114\\
3.88279706992675	-3548602644.98685\\
3.88289707242681	-3542523562.78051\\
3.88299707492687	-3536444480.57417\\
3.88309707742694	-3530376857.52374\\
3.883197079927	-3524314964.05126\\
3.88329708242706	-3518253070.57877\\
3.88339708492712	-3512202636.26219\\
3.88349708742719	-3506152201.94561\\
3.88359708992725	-3500107497.20698\\
3.88369709242731	-3494074251.62425\\
3.88379709492737	-3488041006.04152\\
3.88389709742744	-3482013490.03675\\
3.8839970999275	-3475991703.60992\\
3.88409710242756	-3469975646.76105\\
3.88419710492762	-3463959589.91217\\
3.88429710742769	-3457954992.2192\\
3.88439710992775	-3451956124.10418\\
3.88449711242781	-3445957255.98916\\
3.88459711492787	-3439969847.03005\\
3.88469711742794	-3433982438.07093\\
3.884797119928	-3428000758.68976\\
3.88489712242806	-3422030538.4645\\
3.88499712492812	-3416060318.23924\\
3.88509712742819	-3410095827.59193\\
3.88519712992825	-3404137066.52257\\
3.88529713242831	-3398184035.03116\\
3.88539713492837	-3392236733.1177\\
3.88549713742844	-3386295160.78219\\
3.8855971399285	-3380353588.44668\\
3.88569714242856	-3374423475.26708\\
3.88579714492862	-3368493362.08748\\
3.88589714742869	-3362574708.06378\\
3.88599714992875	-3356656054.04007\\
3.88609715242881	-3350748859.17228\\
3.88619715492887	-3344841664.30448\\
3.88629715742894	-3338940199.01463\\
3.886397159929	-3333044463.30273\\
3.88649716242906	-3327154457.16879\\
3.88659716492912	-3321270180.61279\\
3.88669716742919	-3315391633.63475\\
3.88679716992925	-3309513086.65671\\
3.88689717242931	-3303645998.83457\\
3.88699717492937	-3297784640.59038\\
3.88709717742944	-3291923282.34619\\
3.8871971799295	-3286073383.25791\\
3.88729718242956	-3280223484.16962\\
3.88739718492962	-3274379314.65929\\
3.88749718742969	-3268540874.7269\\
3.88759718992975	-3262708164.37247\\
3.88769719242981	-3256886913.17394\\
3.88779719492987	-3251059932.39746\\
3.88789719742994	-3245244410.77689\\
3.88799719993	-3239434618.73426\\
3.88809720243006	-3233630556.26958\\
3.88819720493012	-3227826493.80491\\
3.88829720743019	-3222033890.49614\\
3.88839720993025	-3216241287.18736\\
3.88849721243031	-3210460143.03449\\
3.88859721493037	-3204678998.88162\\
3.88869721743044	-3198903584.3067\\
3.8887972199305	-3193133899.30974\\
3.88889722243056	-3187375673.46867\\
3.88899722493062	-3181617447.62761\\
3.88909722743069	-3175859221.78654\\
3.88919722993075	-3170112455.10138\\
3.88929723243081	-3164371417.99417\\
3.88939723493087	-3158636110.46491\\
3.88949723743094	-3152900802.93565\\
3.889597239931	-3147176954.56229\\
3.88969724243106	-3141453106.18894\\
3.88979724493112	-3135734987.39353\\
3.88989724743119	-3130028327.75403\\
3.88999724993125	-3124321668.11453\\
3.89009725243131	-3118620738.05297\\
3.89019725493137	-3112925537.56937\\
3.89029725743144	-3107236066.66372\\
3.8903972599315	-3101552325.33603\\
3.89049726243156	-3095874313.58628\\
3.89059726493162	-3090196301.83653\\
3.89069726743169	-3084529749.24269\\
3.89079726993175	-3078863196.64885\\
3.89089727243181	-3073208103.2109\\
3.89099727493187	-3067553009.77296\\
3.89109727743194	-3061903645.91297\\
3.891197279932	-3056265741.20889\\
3.89129728243206	-3050627836.5048\\
3.89139728493212	-3044995661.37866\\
3.89149728743219	-3039369215.83048\\
3.89159728993225	-3033748499.86024\\
3.89169729243231	-3028127783.89001\\
3.89179729493237	-3022518527.07568\\
3.89189729743244	-3016914999.8393\\
3.8919972999325	-3011311472.60292\\
3.89209730243256	-3005719404.52244\\
3.89219730493262	-3000127336.44197\\
3.89229730743269	-2994540997.93944\\
3.89239730993275	-2988966118.59282\\
3.89249731243281	-2983391239.2462\\
3.89259731493287	-2977822089.47752\\
3.89269731743294	-2972258669.2868\\
3.892797319933	-2966700978.67404\\
3.89289732243306	-2961143288.06127\\
3.89299732493312	-2955597056.6044\\
3.89309732743319	-2950056554.72549\\
3.89319732993325	-2944516052.84657\\
3.89329733243331	-2938987010.12356\\
3.89339733493337	-2933457967.40055\\
3.89349733743344	-2927940383.83344\\
3.8935973399335	-2922422800.26633\\
3.89369734243356	-2916910946.27717\\
3.89379734493362	-2911404821.86596\\
3.89389734743369	-2905904427.0327\\
3.89399734993375	-2900409761.7774\\
3.89409735243381	-2894920826.10005\\
3.89419735493387	-2889437620.00064\\
3.89429735743394	-2883954413.90124\\
3.894397359934	-2878482666.95774\\
3.89449736243406	-2873010920.01424\\
3.89459736493412	-2867550632.22665\\
3.89469736743419	-2862090344.43905\\
3.89479736993425	-2856635786.2294\\
3.89489737243431	-2851186957.59771\\
3.89499737493437	-2845743858.54397\\
3.89509737743444	-2840306489.06818\\
3.8951973799345	-2834874849.17034\\
3.89529738243456	-2829448938.85045\\
3.89539738493462	-2824028758.10851\\
3.89549738743469	-2818614306.94452\\
3.89559738993475	-2813199855.78054\\
3.89569739243481	-2807796863.77245\\
3.89579739493487	-2802393871.76437\\
3.89589739743494	-2796996609.33424\\
3.895997399935	-2791610806.06001\\
3.89609740243506	-2786225002.78578\\
3.89619740493512	-2780844929.0895\\
3.89629740743519	-2775470584.97117\\
3.89639740993525	-2770101970.4308\\
3.89649741243531	-2764739085.46837\\
3.89659741493537	-2759376200.50595\\
3.89669741743544	-2754024774.69943\\
3.8967974199355	-2748679078.47085\\
3.89689742243556	-2743333382.24228\\
3.89699742493562	-2737999145.16962\\
3.89709742743569	-2732664908.09695\\
3.89719742993575	-2727336400.60223\\
3.89729743243581	-2722013622.68547\\
3.89739743493587	-2716696574.34665\\
3.89749743743594	-2711385255.58579\\
3.897597439936	-2706079666.40288\\
3.89769744243606	-2700779806.79792\\
3.89779744493612	-2695485676.77091\\
3.89789744743619	-2690197276.32185\\
3.89799744993625	-2684908875.87279\\
3.89809745243631	-2679631934.57964\\
3.89819745493637	-2674354993.28648\\
3.89829745743644	-2669083781.57128\\
3.8983974599365	-2663824029.01198\\
3.89849746243656	-2658564276.45268\\
3.89859746493662	-2653310253.47133\\
3.89869746743669	-2648061960.06793\\
3.89879746993675	-2642819396.24248\\
3.89889747243681	-2637576832.41704\\
3.89899747493687	-2632345727.74749\\
3.89909747743694	-2627120352.6559\\
3.899197479937	-2621894977.56431\\
3.89929748243706	-2616681061.62862\\
3.89939748493712	-2611467145.69293\\
3.89949748743719	-2606264688.91314\\
3.89959748993725	-2601062232.13335\\
3.89969749243731	-2595865504.93151\\
3.89979749493737	-2590674507.30763\\
3.89989749743744	-2585489239.26169\\
3.8999974999375	-2580309700.79371\\
3.90009750243756	-2575135891.90368\\
3.90019750493762	-2569962083.01365\\
3.90029750743769	-2564799733.27952\\
3.90039750993775	-2559643113.12334\\
3.90049751243781	-2554486492.96716\\
3.90059751493787	-2549341331.96689\\
3.90069751743794	-2544196170.96662\\
3.900797519938	-2539056739.54429\\
3.90089752243806	-2533923037.69992\\
3.90099752493812	-2528795065.4335\\
3.90109752743819	-2523672822.74503\\
3.90119752993825	-2518556309.63451\\
3.90129753243831	-2513445526.10194\\
3.90139753493837	-2508340472.14733\\
3.90149753743844	-2503235418.19271\\
3.9015975399385	-2498141823.394\\
3.90169754243856	-2493048228.59529\\
3.90179754493862	-2487966092.95248\\
3.90189754743869	-2482883957.30967\\
3.90199754993875	-2477807551.24481\\
3.90209755243881	-2472736874.7579\\
3.90219755493887	-2467671927.84894\\
3.90229755743894	-2462612710.51794\\
3.902397559939	-2457559222.76488\\
3.90249756243906	-2452511464.58978\\
3.90259756493912	-2447463706.41468\\
3.90269756743919	-2442427407.39548\\
3.90279756993925	-2437391108.37628\\
3.90289757243931	-2432366268.51298\\
3.90299757493937	-2427341428.64968\\
3.90309757743944	-2422322318.36434\\
3.9031975799395	-2417314667.23489\\
3.90329758243956	-2412307016.10545\\
3.90339758493962	-2407305094.55396\\
3.90349758743969	-2402308902.58042\\
3.90359758993975	-2397312710.60688\\
3.90369759243981	-2392327977.78924\\
3.90379759493987	-2387348974.54955\\
3.90389759743994	-2382369971.30987\\
3.90399759994	-2377402427.22608\\
3.90409760244006	-2372434883.1423\\
3.90419760494012	-2367478798.21442\\
3.90429760744019	-2362522713.28653\\
3.90439760994025	-2357572357.9366\\
3.90449761244031	-2352627732.16462\\
3.90459761494037	-2347688835.9706\\
3.90469761744044	-2342755669.35452\\
3.9047976199405	-2337828232.3164\\
3.90489762244056	-2332900795.27827\\
3.90499762494062	-2327984817.39605\\
3.90509762744069	-2323074569.09178\\
3.90519762994075	-2318164320.78751\\
3.90529763244081	-2313259802.06119\\
3.90539763494087	-2308366742.49077\\
3.90549763744094	-2303473682.92035\\
3.905597639941	-2298586352.92789\\
3.90569764244106	-2293704752.51337\\
3.90579764494112	-2288828881.67681\\
3.90589764744119	-2283958740.4182\\
3.90599764994125	-2279088599.15958\\
3.90609765244131	-2274229917.05687\\
3.90619765494137	-2269376964.53212\\
3.90629765744144	-2264524012.00736\\
3.9063976599415	-2259682518.6385\\
3.90649766244156	-2254841025.26965\\
3.90659766494162	-2250005261.47874\\
3.90669766744169	-2245175227.26579\\
3.90679766994175	-2240350922.63079\\
3.90689767244181	-2235532347.57374\\
3.90699767494187	-2230719502.09464\\
3.90709767744194	-2225912386.19349\\
3.907197679942	-2221110999.8703\\
3.90729768244206	-2216309613.5471\\
3.90739768494212	-2211519686.37981\\
3.90749768744219	-2206729759.21251\\
3.90759768994225	-2201951291.20112\\
3.90769769244231	-2197172823.18973\\
3.90779769494237	-2192400084.75629\\
3.90789769744244	-2187633075.9008\\
3.9079976999425	-2182871796.62326\\
3.90809770244256	-2178116246.92368\\
3.90819770494262	-2173366426.80204\\
3.90829770744269	-2168622336.25836\\
3.90839770994275	-2163878245.71468\\
3.90849771244281	-2159145614.3269\\
3.90859771494287	-2154418712.51707\\
3.90869771744294	-2149691810.70724\\
3.908797719943	-2144970638.47536\\
3.90889772244306	-2140255195.82143\\
3.90899772494312	-2135551212.32341\\
3.90909772744319	-2130847228.82539\\
3.90919772994325	-2126148974.90531\\
3.90929773244331	-2121456450.56319\\
3.90939773494337	-2116763926.22107\\
3.90949773744344	-2112082861.03485\\
3.9095977399435	-2107407525.42658\\
3.90969774244356	-2102732189.81832\\
3.90979774494362	-2098068313.36595\\
3.90989774744369	-2093404436.91359\\
3.90999774994375	-2088746290.03917\\
3.91009775244381	-2084093872.74271\\
3.91019775494387	-2079452914.60215\\
3.91029775744394	-2074811956.46159\\
3.910397759944	-2070170998.32103\\
3.91049776244406	-2065541499.33637\\
3.91059776494412	-2060917729.92967\\
3.91069776744419	-2056299690.10091\\
3.91079776994425	-2051681650.27216\\
3.91089777244431	-2047075069.59931\\
3.91099777494437	-2042468488.92646\\
3.91109777744444	-2037867637.83156\\
3.9111977799445	-2033278245.89256\\
3.91129778244456	-2028688853.95356\\
3.91139778494462	-2024105191.59251\\
3.91149778744469	-2019527258.80942\\
3.91159778994475	-2014949326.02632\\
3.91169779244481	-2010382852.39913\\
3.91179779494487	-2005822108.34989\\
3.91189779744494	-2001267093.8786\\
3.911997799945	-1996712079.40731\\
3.91209780244506	-1992162794.51397\\
3.91219780494512	-1987624968.77653\\
3.91229780744519	-1983087143.0391\\
3.91239780994525	-1978555046.87961\\
3.91249781244531	-1974028680.29808\\
3.91259781494537	-1969508043.2945\\
3.91269781744544	-1964993135.86887\\
3.9127978199455	-1960483958.02119\\
3.91289782244556	-1955980509.75146\\
3.91299782494562	-1951477061.48173\\
3.91309782744569	-1946985072.3679\\
3.91319782994575	-1942493083.25408\\
3.91329783244581	-1938006823.7182\\
3.91339783494587	-1933532023.33823\\
3.91349783744594	-1929057222.95826\\
3.913597839946	-1924588152.15624\\
3.91369784244606	-1920124810.93217\\
3.91379784494612	-1915667199.28605\\
3.91389784744619	-1911215317.21789\\
3.91399784994625	-1906763435.14972\\
3.91409785244631	-1902323012.23746\\
3.91419785494637	-1897888318.90314\\
3.91429785744644	-1893453625.56883\\
3.9143978599465	-1889024661.81247\\
3.91449786244656	-1884607157.21201\\
3.91459786494662	-1880189652.61155\\
3.91469786744669	-1875777877.58905\\
3.91479786994675	-1871371832.14449\\
3.91489787244681	-1866971516.27788\\
3.91499787494687	-1862576929.98923\\
3.91509787744694	-1858182343.70058\\
3.915197879947	-1853799216.56783\\
3.91529788244706	-1849421819.01303\\
3.91539788494712	-1845044421.45823\\
3.91549788744719	-1840678483.05933\\
3.91559788994725	-1836312544.66043\\
3.91569789244731	-1831952335.83949\\
3.91579789494737	-1827597856.59649\\
3.91589789744744	-1823249106.93145\\
3.9159978999475	-1818906086.84436\\
3.91609790244756	-1814568796.33522\\
3.91619790494762	-1810237235.40403\\
3.91629790744769	-1805905674.47284\\
3.91639790994775	-1801585572.69756\\
3.91649791244781	-1797265470.92227\\
3.91659791494787	-1792956828.30289\\
3.91669791744794	-1788648185.6835\\
3.916797919948	-1784345272.64207\\
3.91689792244806	-1780048089.17859\\
3.91699792494812	-1775756635.29306\\
3.91709792744819	-1771470910.98548\\
3.91719792994825	-1767190916.25585\\
3.91729793244831	-1762916651.10418\\
3.91739793494837	-1758648115.53045\\
3.91749793744844	-1754379579.95673\\
3.9175979399485	-1750122503.5389\\
3.91769794244856	-1745865427.12108\\
3.91779794494862	-1741619809.85916\\
3.91789794744869	-1737374192.59724\\
3.91799794994875	-1733134304.91328\\
3.91809795244881	-1728900146.80726\\
3.91819795494887	-1724671718.27919\\
3.91829795744894	-1720449019.32908\\
3.918397959949	-1716232049.95692\\
3.91849796244906	-1712015080.58475\\
3.91859796494912	-1707809570.36849\\
3.91869796744919	-1703604060.15223\\
3.91879796994925	-1699410009.09188\\
3.91889797244931	-1695215958.03152\\
3.91899797494937	-1691027636.54911\\
3.91909797744944	-1686845044.64466\\
3.9191979799495	-1682668182.31815\\
3.91929798244956	-1678497049.5696\\
3.91939798494962	-1674331646.399\\
3.91949798744969	-1670171972.80635\\
3.91959798994975	-1666018028.79165\\
3.91969799244981	-1661864084.77695\\
3.91979799494987	-1657721599.91816\\
3.91989799744994	-1653579115.05936\\
3.91999799995	-1649448089.35647\\
3.92009800245006	-1645317063.65357\\
3.92019800495012	-1641191767.52863\\
3.92029800745019	-1637072200.98164\\
3.92039800995025	-1632958364.0126\\
3.92049801245031	-1628850256.62151\\
3.92059801495037	-1624747878.80838\\
3.92069801745044	-1620645500.99524\\
3.9207980199505	-1616554582.33801\\
3.92089802245056	-1612463663.68077\\
3.92099802495062	-1608384204.17944\\
3.92109802745069	-1604304744.67811\\
3.92119802995075	-1600231014.75473\\
3.92129803245081	-1596163014.4093\\
3.92139803495087	-1592106473.21977\\
3.92149803745094	-1588044202.4523\\
3.921598039951	-1583993390.84072\\
3.92169804245106	-1579948308.8071\\
3.92179804495112	-1575908956.35143\\
3.92189804745119	-1571869603.89575\\
3.92199804995125	-1567841710.59598\\
3.92209805245131	-1563813817.29622\\
3.92219805495137	-1559797383.15235\\
3.92229805745144	-1555780949.00848\\
3.9223980599515	-1551770244.44257\\
3.92249806245156	-1547765269.4546\\
3.92259806495162	-1543766024.04459\\
3.92269806745169	-1539772508.21253\\
3.92279806995175	-1535784721.95842\\
3.92289807245181	-1531796935.7043\\
3.92299807495187	-1527820608.6061\\
3.92309807745194	-1523844281.50789\\
3.923198079952	-1519879413.56558\\
3.92329808245206	-1515914545.62328\\
3.92339808495212	-1511955407.25892\\
3.92349808745219	-1508001998.47252\\
3.92359808995225	-1504054319.26407\\
3.92369809245231	-1500112369.63357\\
3.92379809495237	-1496176149.58102\\
3.92389809745244	-1492245659.10642\\
3.9239980999525	-1488320898.20978\\
3.92409810245256	-1484396137.31313\\
3.92419810495262	-1480482835.57239\\
3.92429810745269	-1476569533.83164\\
3.92439810995275	-1472661961.66885\\
3.92449811245281	-1468765848.66196\\
3.92459811495287	-1464869735.65507\\
3.92469811745294	-1460979352.22613\\
3.924798119953	-1457094698.37515\\
3.92489812245306	-1453210044.52416\\
3.92499812495312	-1449336849.82908\\
3.92509812745319	-1445469384.71194\\
3.92519812995325	-1441601919.59481\\
3.92529813245331	-1437745913.63358\\
3.92539813495337	-1433889907.67235\\
3.92549813745344	-1430045360.86702\\
3.9255981399535	-1426200814.06169\\
3.92569814245356	-1422361996.83432\\
3.92579814495362	-1418528909.18489\\
3.92589814745369	-1414701551.11342\\
3.92599814995375	-1410879922.6199\\
3.92609815245381	-1407058294.12637\\
3.92619815495387	-1403248124.78875\\
3.92629815745394	-1399437955.45113\\
3.926398159954	-1395639245.26942\\
3.92649816245406	-1391840535.0877\\
3.92659816495412	-1388047554.48393\\
3.92669816745419	-1384266033.03607\\
3.92679816995425	-1380484511.58821\\
3.92689817245431	-1376708719.71829\\
3.92699817495437	-1372938657.42633\\
3.92709817745444	-1369168595.13437\\
3.9271981799545	-1365409991.99831\\
3.92729818245456	-1361657118.44021\\
3.92739818495462	-1357904244.8821\\
3.92749818745469	-1354162830.4799\\
3.92759818995475	-1350421416.07769\\
3.92769819245481	-1346685731.25344\\
3.92779819495487	-1342955776.00714\\
3.92789819745494	-1339231550.33879\\
3.927998199955	-1335513054.24839\\
3.92809820245506	-1331800287.73594\\
3.92819820495512	-1328093250.80144\\
3.92829820745519	-1324391943.4449\\
3.92839820995525	-1320690636.08835\\
3.92849821245531	-1317000787.88771\\
3.92859821495537	-1313310939.68707\\
3.92869821745544	-1309632550.64233\\
3.9287982199555	-1305954161.59759\\
3.92889822245556	-1302281502.1308\\
3.92899822495562	-1298614572.24196\\
3.92909822745569	-1294953371.93108\\
3.92919822995575	-1291297901.19814\\
3.92929823245581	-1287642430.46521\\
3.92939823495587	-1283998418.88817\\
3.92949823745594	-1280360136.88909\\
3.929598239956	-1276721854.89001\\
3.92969824245606	-1273095032.04684\\
3.92979824495612	-1269468209.20366\\
3.92989824745619	-1265847115.93843\\
3.92999824995625	-1262231752.25115\\
3.93009825245631	-1258622118.14183\\
3.93019825495637	-1255018213.61046\\
3.93029825745644	-1251420038.65704\\
3.9303982599565	-1247827593.28157\\
3.93049826245656	-1244235147.9061\\
3.93059826495662	-1240654161.68653\\
3.93069826745669	-1237073175.46696\\
3.93079826995675	-1233503648.4033\\
3.93089827245681	-1229934121.33963\\
3.93099827495687	-1226370323.85392\\
3.93109827745694	-1222812255.94615\\
3.931198279957	-1219259917.61634\\
3.93129828245706	-1215713308.86448\\
3.93139828495712	-1212172429.69057\\
3.93149828745719	-1208631550.51667\\
3.93159828995725	-1205102130.49866\\
3.93169829245731	-1201572710.48065\\
3.93179829495737	-1198054749.61855\\
3.93189829745744	-1194536788.75645\\
3.9319982999575	-1191024557.4723\\
3.93209830245756	-1187523785.34405\\
3.93219830495762	-1184023013.2158\\
3.93229830745769	-1180527970.6655\\
3.93239830995775	-1177032928.1152\\
3.93249831245781	-1173549344.72081\\
3.93259831495787	-1170071490.90436\\
3.93269831745794	-1166593637.08792\\
3.932798319958	-1163127242.42738\\
3.93289832245806	-1159660847.76683\\
3.93299832495812	-1156200182.68424\\
3.93309832745819	-1152750976.75756\\
3.93319832995825	-1149301770.83087\\
3.93329833245831	-1145858294.48213\\
3.93339833495837	-1142420547.71135\\
3.93349833745844	-1138982800.94056\\
3.9335983399585	-1135556513.32568\\
3.93369834245856	-1132135955.28875\\
3.93379834495862	-1128715397.25182\\
3.93389834745869	-1125306298.37079\\
3.93399834995875	-1121897199.48976\\
3.93409835245881	-1118493830.18669\\
3.93419835495887	-1115096190.46156\\
3.93429835745894	-1111704280.31438\\
3.934398359959	-1108318099.74516\\
3.93449836245906	-1104937648.75389\\
3.93459836495912	-1101562927.34057\\
3.93469836745919	-1098193935.5052\\
3.93479836995925	-1094824943.66983\\
3.93489837245931	-1091467410.99036\\
3.93499837495937	-1088109878.3109\\
3.93509837745944	-1084763804.78733\\
3.9351983799595	-1081417731.26377\\
3.93529838245956	-1078077387.31816\\
3.93539838495962	-1074742772.9505\\
3.93549838745969	-1071413888.16079\\
3.93559838995975	-1068090732.94903\\
3.93569839245981	-1064767577.73727\\
3.93579839495987	-1061455881.68141\\
3.93589839745994	-1058144185.62556\\
3.93599839996	-1054843948.7256\\
3.93609840246006	-1051543711.82565\\
3.93619840496012	-1048254934.0816\\
3.93629840746019	-1044966156.33755\\
3.93639840996025	-1041683108.17145\\
3.93649841246031	-1038405789.5833\\
3.93659841496037	-1035134200.5731\\
3.93669841746044	-1031862611.5629\\
3.9367984199605	-1028602481.70861\\
3.93689842246056	-1025348081.43227\\
3.93699842496062	-1022093681.15592\\
3.93709842746069	-1018845010.45753\\
3.93719842996075	-1015607798.91504\\
3.93729843246081	-1012370587.37255\\
3.93739843496087	-1009139105.40802\\
3.93749843746094	-1005913353.02143\\
3.937598439961	-1002693330.21279\\
3.93769844246106	-999479036.982111\\
3.93779844496112	-996270473.329378\\
3.93789844746119	-993061909.676645\\
3.93799844996125	-989864805.179815\\
3.93809845246131	-986667700.682985\\
3.93819845496137	-983482055.342058\\
3.93829845746144	-980296410.001131\\
3.9383984599615	-977116494.238155\\
3.93849846246156	-973942308.05313\\
3.93859846496162	-970773851.446056\\
3.93869846746169	-967611124.416934\\
3.93879846996175	-964454126.965763\\
3.93889847246181	-961297129.514593\\
3.93899847496187	-958151591.219324\\
3.93909847746194	-955006052.924056\\
3.939198479962	-951871973.78469\\
3.93929848246206	-948737894.645325\\
3.93939848496212	-945609545.083911\\
3.93949848746219	-942486925.100448\\
3.93959848996225	-939370034.694936\\
3.93969849246231	-936258873.867376\\
3.93979849496237	-933153442.617767\\
3.93989849746244	-930053740.946109\\
3.9399984999625	-926954039.274451\\
3.94009850246256	-923865796.758696\\
3.94019850496262	-920777554.242941\\
3.94029850746269	-917700770.883088\\
3.94039850996275	-914623987.523236\\
3.94049851246281	-911552933.741335\\
3.94059851496287	-908487609.537385\\
3.94069851746294	-905428014.911386\\
3.940798519963	-902374149.863339\\
3.94089852246306	-899326014.393243\\
3.94099852496312	-896277878.923147\\
3.94109852746319	-893241202.608953\\
3.94119852996325	-890204526.29476\\
3.94129853246331	-887179309.136469\\
3.94139853496337	-884154091.978179\\
3.94149853746344	-881134604.397839\\
3.9415985399635	-878120846.395451\\
3.94169854246356	-875112817.971014\\
3.94179854496362	-872110519.124529\\
3.94189854746369	-869113949.855994\\
3.94199854996375	-866123110.165412\\
3.94209855246381	-863132270.474829\\
3.94219855496387	-860152889.940148\\
3.94229855746394	-857173509.405468\\
3.942398559964	-854205588.02669\\
3.94249856246406	-851237666.647913\\
3.94259856496412	-848275474.847086\\
3.94269856746419	-845319012.624211\\
3.94279856996425	-842368279.979288\\
3.94289857246431	-839423276.912315\\
3.94299857496437	-836484003.423294\\
3.94309857746444	-833544729.934273\\
3.9431985799645	-830616915.601154\\
3.94329858246456	-827689101.268036\\
3.94339858496462	-824772746.09082\\
3.94349858746469	-821856390.913604\\
3.94359858996475	-818945765.31434\\
3.94369859246481	-816040869.293026\\
3.94379859496487	-813141702.849664\\
3.94389859746494	-810248265.984254\\
3.943998599965	-807360558.696794\\
3.94409860246506	-804472851.409335\\
3.94419860496512	-801596603.277778\\
3.94429860746519	-798720355.146222\\
3.94439860996525	-795855566.170567\\
3.94449861246531	-792990777.194913\\
3.94459861496537	-790131717.79721\\
3.94469861746544	-787278387.977459\\
3.9447986199655	-784430787.735659\\
3.94489862246556	-781588917.07181\\
3.94499862496562	-778752775.985912\\
3.94509862746569	-775922364.477966\\
3.94519862996575	-773091952.97002\\
3.94529863246581	-770273000.617976\\
3.94539863496587	-767454048.265932\\
3.94549863746594	-764646555.069791\\
3.945598639966	-761839061.87365\\
3.94569864246606	-759037298.255461\\
3.94579864496612	-756241264.215222\\
3.94589864746619	-753450959.752935\\
3.94599864996625	-750666384.868599\\
3.94609865246631	-747887539.562215\\
3.94619865496637	-745108694.25583\\
3.94629865746644	-742341308.105348\\
3.9463986599665	-739573921.954867\\
3.94649866246656	-736817994.960287\\
3.94659866496662	-734062067.965708\\
3.94669866746669	-731311870.54908\\
3.94679866996675	-728567402.710404\\
3.94689867246681	-725828664.449678\\
3.94699867496687	-723095655.766904\\
3.94709867746694	-720368376.662081\\
3.947198679967	-717641097.557259\\
3.94729868246706	-714925277.608339\\
3.94739868496712	-712209457.659418\\
3.94749868746719	-709505096.866401\\
3.94759868996725	-706800736.073384\\
3.94769869246731	-704102104.858317\\
3.94779869496737	-701409203.221202\\
3.94789869746744	-698722031.162039\\
3.9479986999675	-696040588.680827\\
3.94809870246756	-693364875.777566\\
3.94819870496762	-690694892.452256\\
3.94829870746769	-688024909.126946\\
3.94839870996775	-685366384.957539\\
3.94849871246781	-682707860.788132\\
3.94859871496787	-680055066.196677\\
3.94869871746794	-677413730.761124\\
3.948798719968	-674772395.32557\\
3.94889872246806	-672136789.467969\\
3.94899872496812	-669506913.188318\\
3.94909872746819	-666882766.486619\\
3.94919872996825	-664258619.78492\\
3.94929873246831	-661645932.239123\\
3.94939873496837	-659033244.693327\\
3.94949873746844	-656432016.303433\\
3.9495987399685	-653830787.913539\\
3.94969874246856	-651235289.101596\\
3.94979874496862	-648651249.445556\\
3.94989874746869	-646067209.789516\\
3.94999874996875	-643488899.711428\\
3.95009875246881	-640910589.633339\\
3.95019875496887	-638343738.711153\\
3.95029875746894	-635782617.366918\\
3.950398759969	-633221496.022683\\
3.95049876246906	-630671833.834351\\
3.95059876496912	-628122171.646019\\
3.95069876746919	-625578239.035638\\
3.95079876996925	-623045765.58116\\
3.95089877246931	-620513292.126682\\
3.95099877496937	-617986548.250155\\
3.95109877746944	-615459804.373628\\
3.9511987799695	-612944519.653003\\
3.95129878246956	-610434964.51033\\
3.95139878496962	-607925409.367657\\
3.95149878746969	-605427313.380887\\
3.95159878996975	-602929217.394117\\
3.95169879246981	-600442580.563249\\
3.95179879496987	-597955943.732381\\
3.95189879746994	-595475036.479465\\
3.95199879997	-592999858.804499\\
3.95209880247006	-590530410.707486\\
3.95219880497012	-588060962.610472\\
3.95229880747019	-585602973.669361\\
3.95239880997025	-583150714.306201\\
3.95249881247031	-580698454.943041\\
3.95259881497037	-578257654.735783\\
3.95269881747044	-575816854.528526\\
3.9527988199705	-573381783.89922\\
3.95289882247056	-570953588.763456\\
3.95299882497062	-568529977.290052\\
3.95309882747069	-566111522.436805\\
3.95319882997075	-563698224.203714\\
3.95329883247081	-561290082.590779\\
3.95339883497087	-558887097.598001\\
3.95349883747094	-556489269.225378\\
3.953598839971	-554096597.472912\\
3.95369884247106	-551709082.340602\\
3.95379884497112	-549326723.828448\\
3.95389884747119	-546950094.894245\\
3.95399884997125	-544578049.622403\\
3.95409885247131	-542211160.970718\\
3.95419885497137	-539849428.939189\\
3.95429885747144	-537492853.527816\\
3.9543988599715	-535141434.736599\\
3.95449886247156	-532795745.523333\\
3.95459886497162	-530454639.972429\\
3.95469886747169	-528118691.04168\\
3.95479886997175	-525787898.731088\\
3.95489887247181	-523462835.998447\\
3.95499887497187	-521142356.928167\\
3.95509887747194	-518827034.478044\\
3.955198879972	-516517441.605871\\
3.95529888247206	-514212432.39606\\
3.95539888497212	-511912579.806405\\
3.95549888747219	-509618456.794701\\
3.95559888997225	-507328917.445358\\
3.95569889247231	-505044534.716172\\
3.95579889497237	-502765881.564936\\
3.95589889747244	-500491812.076062\\
3.9559988999725	-498223472.165139\\
3.95609890247256	-495959715.916577\\
3.95619890497262	-493701116.288172\\
3.95629890747269	-491448246.237717\\
3.95639890997275	-489199959.849624\\
3.95649891247281	-486957403.039482\\
3.95659891497287	-484719429.891701\\
3.95669891747294	-482487186.321871\\
3.956798919973	-480259526.414402\\
3.95689892247306	-478037596.084885\\
3.95699892497312	-475820249.417729\\
3.95709892747319	-473608632.328524\\
3.95719892997325	-471401598.90168\\
3.95729893247331	-469200295.052787\\
3.95739893497337	-467004147.824051\\
3.95749893747344	-464812584.257675\\
3.9575989399735	-462626750.269251\\
3.95769894247356	-460445499.943188\\
3.95779894497362	-458269979.195077\\
3.95789894747369	-456099615.067121\\
3.95799894997375	-453933834.601527\\
3.95809895247381	-451773783.713883\\
3.95819895497387	-449618316.488601\\
3.95829895747394	-447468578.84127\\
3.958398959974	-445323997.814096\\
3.95849896247406	-443184000.449282\\
3.95859896497412	-441049732.66242\\
3.95869896747419	-438920621.495714\\
3.95879896997425	-436796093.991368\\
3.95889897247431	-434677296.064975\\
3.95899897497437	-432563654.758737\\
3.95909897747444	-430455170.072656\\
3.9591989799745	-428351269.048935\\
3.95929898247456	-426253097.603166\\
3.95939898497462	-424160082.777553\\
3.95949898747469	-422071651.614301\\
3.95959898997475	-419988950.029001\\
3.95969899247481	-417911405.063857\\
3.95979899497487	-415839016.718868\\
3.95989899747494	-413771212.036241\\
3.959998999975	-411709136.931565\\
3.96009900247506	-409652218.447046\\
3.96019900497512	-407600456.582682\\
3.96029900747519	-405553851.338475\\
3.96039900997525	-403511829.756629\\
3.96049901247531	-401475537.752734\\
3.96059901497537	-399444402.368995\\
3.96069901747544	-397418423.605412\\
3.9607990199755	-395397601.461986\\
3.96089902247556	-393381362.980921\\
3.96099902497562	-391370854.077807\\
3.96109902747569	-389365501.794849\\
3.96119902997575	-387365306.132047\\
3.96129903247581	-385370267.089401\\
3.96139903497587	-383380384.666912\\
3.96149903747594	-381395085.906784\\
3.961599039976	-379415516.724607\\
3.96169904247606	-377441104.162586\\
3.96179904497612	-375471848.220721\\
3.96189904747619	-373507748.899013\\
3.96199904997625	-371548806.197461\\
3.96209905247631	-369595020.116064\\
3.96219905497637	-367645817.697029\\
3.96229905747644	-365702344.855946\\
3.9623990599765	-363764028.635018\\
3.96249906247656	-361830869.034247\\
3.96259906497662	-359902866.053631\\
3.96269906747669	-357980019.693172\\
3.96279906997675	-356062329.95287\\
3.96289907247681	-354149796.832723\\
3.96299907497687	-352241847.374937\\
3.96309907747694	-350339627.495103\\
3.963199079977	-348442564.235425\\
3.96329908247706	-346550657.595903\\
3.96339908497712	-344663907.576537\\
3.96349908747719	-342782314.177327\\
3.96359908997725	-340905877.398274\\
3.96369909247731	-339034597.239377\\
3.96379909497737	-337168473.700636\\
3.96389909747744	-335306933.824256\\
3.9639990999775	-333451123.525827\\
3.96409910247756	-331600469.847554\\
3.96419910497762	-329754972.789438\\
3.96429910747769	-327914632.351478\\
3.96439910997775	-326079448.533674\\
3.96449911247781	-324249421.336026\\
3.96459911497787	-322424550.758534\\
3.96469911747794	-320604836.801199\\
3.964799119978	-318790279.464019\\
3.96489912247806	-316980305.789201\\
3.96499912497812	-315176061.692334\\
3.96509912747819	-313376974.215623\\
3.96519912997825	-311583043.359069\\
3.96529913247831	-309794269.12267\\
3.96539913497837	-308010651.506428\\
3.96549913747844	-306232190.510342\\
3.9655991399785	-304458886.134412\\
3.96569914247856	-302690738.378638\\
3.96579914497862	-300927174.285226\\
3.96589914747869	-299169339.769764\\
3.96599914997875	-297416661.874459\\
3.96609915247881	-295669140.59931\\
3.96619915497887	-293926775.944317\\
3.96629915747894	-292189567.909481\\
3.966399159979	-290457516.4948\\
3.96649916247906	-288730621.700276\\
3.96659916497912	-287008310.568112\\
3.96669916747919	-285291729.013901\\
3.96679916997925	-283580304.079845\\
3.96689917247931	-281874035.765945\\
3.96699917497937	-280172924.072202\\
3.96709917747944	-278476968.998615\\
3.9671991799795	-276786170.545183\\
3.96729918247956	-275099955.754113\\
3.96739918497962	-273419470.540995\\
3.96749918747969	-271744141.948032\\
3.96759918997975	-270073969.975226\\
3.96769919247981	-268408954.622576\\
3.96779919497987	-266749095.890082\\
3.96789919747994	-265093820.819949\\
3.96799919998	-263444275.327767\\
3.96809920248006	-261799886.455742\\
3.96819920498012	-260160654.203872\\
3.96829920748019	-258526578.572159\\
3.96839920998025	-256897086.602807\\
3.96849921248031	-255273324.211406\\
3.96859921498037	-253654718.440162\\
3.96869921748044	-252041269.289073\\
3.9687992199805	-250432976.758141\\
3.96889922248056	-248829267.88957\\
3.96899922498062	-247231288.59895\\
3.96909922748069	-245638465.928487\\
3.96919922998075	-244050799.878179\\
3.96929923248081	-242467717.490233\\
3.96939923498087	-240890364.680237\\
3.96949923748094	-239318168.490398\\
3.969599239981	-237751128.920716\\
3.96969924248106	-236188673.013394\\
3.96979924498112	-234631946.684023\\
3.96989924748119	-233080376.974809\\
3.96999924998125	-231533390.927956\\
3.97009925248131	-229992134.459054\\
3.97019925498137	-228456034.610308\\
3.97029925748144	-226925091.381719\\
3.9703992599815	-225398731.81549\\
3.97049926248156	-223878101.827213\\
3.97059926498162	-222362628.459092\\
3.97069926748169	-220851738.753332\\
3.97079926998175	-219346578.625523\\
3.97089927248181	-217846002.160076\\
3.97099927498187	-216351155.272579\\
3.97109927748194	-214861465.005239\\
3.971199279982	-213376358.40026\\
3.97129928248206	-211896981.373232\\
3.97139928498212	-210422760.966361\\
3.97149928748219	-208953124.22185\\
3.97159928998225	-207489217.055291\\
3.97169929248231	-206029893.551093\\
3.97179929498237	-204576299.624846\\
3.97189929748244	-203127289.36096\\
3.9719992999825	-201684008.675025\\
3.97209930248256	-200245884.609247\\
3.97219930498262	-198812344.20583\\
3.97229930748269	-197384533.380364\\
3.97239930998275	-195961306.217259\\
3.97249931248281	-194543808.632105\\
3.97259931498287	-193130894.709312\\
3.97269931748294	-191723710.364471\\
3.972799319983	-190321109.681991\\
3.97289932248306	-188924238.577462\\
3.97299932498312	-187531951.135294\\
3.97309932748319	-186144820.313282\\
3.97319932998325	-184763419.069222\\
3.97329933248331	-183386601.487523\\
3.97339933498337	-182015513.483775\\
3.97349933748344	-180649009.142388\\
3.9735993399835	-179288234.378952\\
3.97369934248356	-177932043.277877\\
3.97379934498362	-176581008.796959\\
3.97389934748369	-175235703.893992\\
3.97399934998375	-173894982.653385\\
3.97409935248381	-172559418.032935\\
3.97419935498387	-171229582.990437\\
3.97429935748394	-169904331.610299\\
3.974399359984	-168584236.850318\\
3.97449936248406	-167269871.668288\\
3.97459936498412	-165960090.148619\\
3.97469936748419	-164655465.249106\\
3.97479936998425	-163356569.927544\\
3.97489937248431	-162062258.268344\\
3.97499937498437	-160773103.229299\\
3.97509937748444	-159489104.810411\\
3.9751993799845	-158210835.969474\\
3.97529938248456	-156937150.790898\\
3.97539938498462	-155668622.232479\\
3.97549938748469	-154405250.294215\\
3.97559938998475	-153147034.976108\\
3.97569939248481	-151893976.278157\\
3.97579939498487	-150646647.158157\\
3.97589939748494	-149403901.700518\\
3.975999399985	-148166312.863036\\
3.97609940248506	-146933880.645709\\
3.97619940498512	-145706605.048539\\
3.97629940748519	-144484486.071525\\
3.97639940998525	-143267523.714667\\
3.97649941248531	-142055717.977966\\
3.97659941498537	-140849068.86142\\
3.97669941748544	-139647576.365031\\
3.9767994199855	-138451240.488798\\
3.97689942248556	-137260061.232721\\
3.97699942498562	-136074038.5968\\
3.97709942748569	-134893172.581035\\
3.97719942998575	-133717463.185427\\
3.97729943248581	-132546910.409974\\
3.97739943498587	-131381514.254678\\
3.97749943748594	-130221274.719538\\
3.977599439986	-129066191.804555\\
3.97769944248606	-127916265.509727\\
3.97779944498612	-126771495.835056\\
3.97789944748619	-125631882.78054\\
3.97799944998625	-124497426.346181\\
3.97809945248631	-123368126.531979\\
3.97819945498637	-122243410.380137\\
3.97829945748644	-121124423.806246\\
3.9783994599865	-120010593.852512\\
3.97849946248656	-118901920.518934\\
3.97859946498662	-117798403.805512\\
3.97869946748669	-116699470.754451\\
3.97879946998675	-115606267.281341\\
3.97889947248681	-114518220.428388\\
3.97899947498687	-113435330.195591\\
3.97909947748694	-112357023.625154\\
3.979199479987	-111284446.63267\\
3.97929948248706	-110217026.260341\\
3.97939948498712	-109154189.550373\\
3.97949948748719	-108097082.418357\\
3.97959948998725	-107045131.906497\\
3.97969949248731	-105997765.056997\\
3.97979949498737	-104956127.78545\\
3.97989949748744	-103919074.176263\\
3.9799994999875	-102887750.145027\\
3.98009950248756	-101861009.776153\\
3.98019950498762	-100839998.98523\\
3.98029950748769	-99823571.8566677\\
3.98039950998775	-98812874.3060569\\
3.98049951248781	-97806760.4178072\\
3.98059951498787	-96806376.1075088\\
3.98069951748794	-95810575.4595714\\
3.980799519988	-94820504.3895853\\
3.98089952248806	-93835016.9819603\\
3.98099952498812	-92855259.1522866\\
3.98109952748819	-91880084.9849739\\
3.98119952998825	-90910067.4378175\\
3.98129953248831	-89945779.4686123\\
3.98139953498837	-88986075.1617682\\
3.98149953748844	-88031527.4750802\\
3.9815995399885	-87082709.3663436\\
3.98169954248856	-86138474.919968\\
3.98179954498862	-85199397.0937485\\
3.98189954748869	-84266048.8454804\\
3.98199954998875	-83337284.2595734\\
3.98209955248881	-82413676.2938225\\
3.98219955498887	-81495224.9482278\\
3.98229955748894	-80581930.2227892\\
3.982399559989	-79674365.075302\\
3.98249956248906	-78771383.5901758\\
3.98259956498912	-77873558.7252058\\
3.98269956748919	-76980890.480392\\
3.98279956998925	-76093378.8557344\\
3.98289957248931	-75211023.8512329\\
3.98299957498937	-74333825.4668876\\
3.98309957748944	-73461783.7026985\\
3.9831995799895	-72594898.5586656\\
3.98329958248956	-71733170.0347888\\
3.98339958498962	-70876598.1310682\\
3.98349958748969	-70025182.8475038\\
3.98359958998975	-69178924.1840956\\
3.98369959248981	-68337822.1408436\\
3.98379959498987	-67501876.7177477\\
3.98389959748994	-66671087.914808\\
3.98399959999	-65845455.7320245\\
3.98409960249006	-65024980.1693971\\
3.98419960499012	-64209661.226926\\
3.98429960749019	-63398925.9468159\\
3.98439960999025	-62593920.244657\\
3.98449961249031	-61794071.1626544\\
3.98459961499037	-60999378.700808\\
3.98469961749044	-60209842.8591177\\
3.9847996199905	-59424890.6797885\\
3.98489962249056	-58645668.0784105\\
3.98499962499062	-57871602.0971888\\
3.98509962749069	-57102406.2572257\\
3.98519962999075	-56338481.6289777\\
3.98529963249081	-55579656.3251065\\
3.98539963499087	-54825987.6413914\\
3.98549963749094	-54077475.5778325\\
3.985599639991	-53334120.1344298\\
3.98569964249106	-52595921.3111832\\
3.98579964499112	-51862879.1080928\\
3.98589964749119	-51134936.2293791\\
3.98599964999125	-50412149.9708216\\
3.98609965249131	-49694520.3324202\\
3.98619965499137	-48982047.3141751\\
3.98629965749144	-48274673.6203065\\
3.9863996599915	-47572513.8423737\\
3.98649966249156	-46875453.3888176\\
3.98659966499162	-46183549.5554176\\
3.98669966749169	-45496745.0463943\\
3.98679966999175	-44815154.4533066\\
3.98689967249181	-44138663.1845957\\
3.98699967499187	-43467385.8318204\\
3.98709967749194	-42801207.8034218\\
3.987199679992	-42140129.0993998\\
3.98729968249206	-41484264.3113136\\
3.98739968499212	-40833498.847604\\
3.98749968749219	-40187947.2998301\\
3.98759968999225	-39547495.0764329\\
3.98769969249231	-38912142.1774123\\
3.98779969499237	-38282003.1943274\\
3.98789969749244	-37656963.5356192\\
3.9879996999925	-37037080.4970672\\
3.98809970249256	-36422354.0786713\\
3.98819970499262	-35812784.2804316\\
3.98829970749269	-35208371.1023481\\
3.98839970999275	-34609057.2486413\\
3.98849971249281	-34014900.0150906\\
3.98859971499287	-33425899.4016961\\
3.98869971749294	-32842055.4084578\\
3.988799719993	-32263368.0353757\\
3.98889972249306	-31689779.9866702\\
3.98899972499312	-31121348.5581209\\
3.98909972749319	-30558073.7497278\\
3.98919972999325	-29999955.5614909\\
3.98929973249331	-29446936.6976306\\
3.98939973499337	-28899131.749706\\
3.98949973749344	-28356426.1261581\\
3.9895997399935	-27818877.1227664\\
3.98969974249356	-27286427.4437513\\
3.98979974499362	-26759191.6806719\\
3.98989974749369	-26237055.2419692\\
3.98999974999375	-25720075.4234227\\
3.99009975249381	-25208252.2250323\\
3.99019975499388	-24701585.6467981\\
3.99029975749394	-24200018.3929406\\
3.990399759994	-23703607.7592392\\
3.99049976249406	-23212353.7456941\\
3.99059976499412	-22726256.3523051\\
3.99069976749419	-22245315.5790723\\
3.99079976999425	-21769474.1302161\\
3.99089977249431	-21298789.3015161\\
3.99099977499437	-20833261.0929723\\
3.99109977749444	-20372889.5045847\\
3.9911997799945	-19917617.2405738\\
3.99129978249456	-19467558.8924985\\
3.99139978499462	-19022599.8687999\\
3.99149978749469	-18582797.4652575\\
3.99159978999475	-18148094.3860917\\
3.99169979249481	-17718605.2228617\\
3.99179979499487	-17294215.3840083\\
3.99189979749494	-16874982.1653111\\
3.991999799995	-16460905.56677\\
3.99209980249506	-16051928.2926056\\
3.99219980499513	-15648164.9343769\\
3.99229980749519	-15249500.9005249\\
3.99239980999525	-14855993.4868291\\
3.99249981249531	-14467585.3975099\\
3.99259981499537	-14084391.2241264\\
3.99269981749544	-13706296.3751196\\
3.9927998199955	-13333358.1462689\\
3.99289982249556	-12965576.5375744\\
3.99299982499562	-12602951.5490361\\
3.99309982749569	-12245425.8848745\\
3.99319982999575	-11893056.840869\\
3.99329983249581	-11545844.4170198\\
3.99339983499587	-11203788.6133267\\
3.99349983749594	-10866832.1340102\\
3.993599839996	-10535089.5706295\\
3.99369984249606	-10208446.3316254\\
3.99379984499612	-9886959.71277749\\
3.99389984749619	-9570572.41830624\\
3.99399984999625	-9259399.03977069\\
3.99409985249631	-8953324.98561181\\
3.99419985499638	-8652407.5516091\\
3.99429985749644	-8356646.73776257\\
3.9943998599965	-8065985.2482927\\
3.99449986249656	-7780480.37897901\\
3.99459986499663	-7500132.1298215\\
3.99469986749669	-7224940.50082017\\
3.99479986999675	-6954905.49197501\\
3.99489987249681	-6689969.80750652\\
3.99499987499687	-6430248.03897372\\
3.99509987749694	-6175625.59481758\\
3.995199879997	-5926102.4750381\\
3.99529988249706	-5681781.81203842\\
3.99539988499712	-5442589.12130515\\
3.99549988749719	-5208535.86199421\\
3.99559988999725	-4979633.4932615\\
3.99569989249731	-4755870.55595111\\
3.99579989499737	-4537252.77964099\\
3.99589989749744	-4323780.16433115\\
3.9959998999975	-4115446.98044363\\
3.99609990249756	-3912258.95755638\\
3.99619990499763	-3714216.09566942\\
3.99629990749769	-3521318.39478272\\
3.99639990999775	-3333560.12531835\\
3.99649991249781	-3150947.01685425\\
3.99659991499788	-2973479.06939043\\
3.99669991749794	-2801156.28292689\\
3.996799919998	-2633972.92788566\\
3.99689992249806	-2471934.73384472\\
3.99699992499812	-2315041.70080404\\
3.99709992749819	-2163293.82876364\\
3.99719992999825	-2016685.38814557\\
3.99729993249831	-1875222.10852777\\
3.99739993499837	-1738903.98991024\\
3.99749993749844	-1607725.30271504\\
3.9975999399985	-1481697.50609807\\
3.99769994249856	-1360809.14090341\\
3.99779994499862	-1245060.20713108\\
3.99789994749869	-1134462.16393698\\
3.99799994999875	-1029003.5521652\\
3.99809995249881	-928690.101393697\\
3.99819995499888	-833516.082044516\\
3.99829995749894	-743492.953273561\\
3.998399959999	-658609.25592493\\
3.99849996249906	-578870.719576573\\
3.99859996499913	-504272.187608335\\
3.99869996749919	-434819.962555962\\
3.99879996999925	-370510.606672683\\
3.99889997249931	-311345.265874089\\
3.99899997499937	-257323.367202385\\
3.99909997749944	-208444.910657569\\
3.9991999799995	-164710.469197438\\
3.99929998249956	-126118.896906402\\
3.99939998499962	-92671.3397000496\\
3.99949998749969	-64367.2246205869\\
3.99959998999975	-41206.7235553522\\
3.99969999249981	-23189.7219127864\\
3.99979999499987	-10316.276988669\\
3.99989999749994	-2586.40024215595\\
4	-0.0711132297004573\\
};
\addlegendentry{Energy Diff};

\end{axis}
\end{tikzpicture}%
	\caption{A zoom in on the error when stepping backwards in time.}
	\label{fig:backwardDataError}
\end{figure}
\fi


\end{document}
